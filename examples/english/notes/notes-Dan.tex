\CommentedBook {Dan}

\Note  1:1-6:28 {} {\it The Narratives.\/} This first section of the book highlights both God's absolute control
over the kingdoms of this world  
and the sincere devotion that Daniel and his friends offered to 
God. Daniel wanted his readers to learn that although God's people are sometimes persecuted, kings
and kingdoms rise and fall according to God's purpose. Daniel also taught that God would greatly
bless those who paid attention to him as God's faithful  
spokesman. This material divides into six separate narratives: the 
vindication of Daniel and his friends (<1:1-21>), Nebuchadnezzar's 
dream (<2:1-49>), deliverance from the furnace (<3:1-30>), Nebuchadnezzar's second dream (<4:1-37>),
judgment on Belshazzar (<5:1-31>)  and Daniel's deliverance from the den of lions (<6:1-28>).

\Note  1:1-21 {} {\it Vindication of Daniel and His Friends.\/} The prophet set 
the context of his book by narrating his (and his companions') personal history of
captivity. training, faithfulness and service to King  
Nebuchadnezzar.

\ww {In the third year of the rule of Jehoiakim} % BBE
    {In the third year of the reign of Jehoiakim} % Jubilee2000
    {In the third year of the reign of King Jehoiakim} % NETfree
    {In the third year of the reign of Jehoiakim king} % UKJV
    {In the third year of the reign of Jehoiakim} % RNKJV
    {In the third year of the reign of Jehoiakim} % Webster
\Note 1:1 {In the third year of the reign of Jehoiakim} In 605 B.C., the 
same year Nebuchadnezzar defeated an Assyrian-Egyptian coalition at Carchemish and initiated
Babylon's rise to international  
power. Subsequent to victory at Carchemish Nebuchadnezzar advanced against Jehoiakim (<2Ki 24:1-2>;
<2Ch 36:5-7>) and took Daniel and a number of other Judahites captive.
This was the first of three invasions of Judah by Nebuchadnezzar.
The second was in 597 B.C. (<2Ki 24:10-14>) and the third in 587 B.C. (<2Ki 25:1-24>). The 
apparent discrepancy between <Dan 1:1> and <Jer 25:1> and 
<46:2> (where Jeremiah placed Nebuchadnezzar's attack against Jehoiakim during Jehoiakim's fourth
rather than third year) may be  explained by the difference between the Babylonian and Jewish 
systems of chronology. Under the Babylonian system, which Daniel apparently used, the first year of
a king's reign was viewed as an  \"accession year," and the reign itself was counted as beginning on 
the first of the month of Nisan in the following year. Nebuchadnezzar king of
Babylon. Nebuchadnezzar led the Babylonians to  
victory at Carchemish in 605 B.C. as crown prince and commander 
of the army. Shortly after this victory he assumed the Babylonian 
throne upon the death of his father, Nabopolassar (626-605 B.C.). 
Nebuchadnezzar's reign (605-562 B.C.) forms the historical background for much in the books of
Jeremiah, Ezekiel and Daniel.


\ww {And the Lord gave into his hands} % BBE
    {And the Lord gave} % Jubilee2000
    {Now the Lord delivered} % NETfree
    {And the Lord gave} % UKJV
%    {And  יהוה gave} % RNKJV
    {And      הוהי       gave}={And {\HebrewFont   הוהי}  gave} % RNKJV
    {And the Lord gave} % Webster
\Note 1:2 {And the Lord delivered} Israel's defeat by the Babyloniansis 
not to be explained simply by analysis of the military and politial
conditions of the time. God was sovereignly at work in the affairs of 
the nations. He used the Babylonians to judge his own people' 
breaking their covenant obligations (<2Ki 17:15>,<18-20>; <21:12-15> 
<24:3-4>).

\ww {he took them away} % BBE
    {which he carried} % Jubilee2000
    {He brought them} % NETfree
    {which he carried} % UKJV
    {which he carried} % RNKJV
    {which he carried} % Webster
\Note 1:2 {These he carried off} Refers to the plunder of vessels 
from the temple, not to the deportation of captives. 

\ww {the store-house of his god} % BBE
    {the treasure house of his god} % Jubilee2000
    {the treasury of his god} % NETfree
    {the treasure house of his god} % UKJV
    {the treasure house of his elohim} % RNKJV
    {the treasure-house of his god} % Webster
\Note 1:2 {the temple of his god} 
Marduk was the chief god of the Babylonian pantheon (cf. Jer 50:2).

\ww {the writing and language of the Chaldaeans} % BBE
    {the letters and speech of the Chaldeans} % Jubilee2000
    {the literature and language of the Babylonians} % NETfree
    {the learning and the tongue of the Chaldeans} % UKJV
    {the learning and the tongue of the Chaldeans} % RNKJV
    {the learning and the language of the Chaldeans} % Webster
\Note 1:4 {the language and literature of the Babylonians} Babylonian literature was written in
cuneiform and primarily on clay tablets. Thousands of these tablets have been discovered. Study of 
this literature would have introduced Daniel and his friends to the 
polytheistic worldview of the Babylonians, which prominently featured magic, sorcery and astrology. 


\ww {from the king's table} % BBE
    {daily provision of the king's food} % Jubilee2000
    {from his royal delicacies} % NETfree
    {daily provision of the king's food} % UKJV
    {daily provision of the king's meat} % RNKJV
    {daily provision of the king's food} % Webster
\Note 1:5 {from the king's table} Jehoiachin later received the same 
provision under the rule of the Babylonian king Evil-Merodach(<2Ki 25:27-30>).

\Note 1:6 {Daniel, Hananiah, Mishael, and Azariah} Characteristic Hebrew names. Two of them contain
the Hebrew component el,  meaning \"God," and two the component yah, a shortened form of 
 \"Yahweh" ( \"the LORD"). Daniel means  \"My judge is God," Hananiah  \"Yahweh is gracious,"
Mishael  \"Who is what God is?" and Azariah  \"Yahweh has helped."

\Note 1:7 {Belteshazzar}={Belteshazzar ... Shadrach ...   Meshach ...  Abednego}
The meanings of these names are disputed. Suggestions for Belteshazzar:
 \"Bel [another name for Marduk, the chief Babylonian 
god] protect his life" or \"Lady, protect the king." Shadrach: \"I am 
very fearful (of God)" or \"The command of Aku [the Sumerian 
moon god]." Meshach: \"I am of little account" or \"Who is what 
Aku is?" Abednego: \"Servant of the shining one."


\ww {he would not make himself unclean} % BBE
    {he would not defile himself} % Jubilee2000
    {he would not defile himself} % NETfree
    {he would not defile himself} % RNKJV
    {he would not defile himself} % UKJV
    {he would not defile himself} % Webster
\Note 1:8 {resolved not to defile himself} The reason for Daniel's conclusion that the king's food
would defile him and his friends is not  
given. Perhaps eating it involved violation of the dietary laws of the 
Mosaic legislation (<Lev 11:1-47>), which prohibited eating pork or 
meat from which blood had not been drained (<Lev 17:10-14>). It 
may also have involved partaking of food that had been offered to 
Babylonian idols.


\ww {their faces seemed fairer and they were fatter in flesh} % BBE
    {their countenances appeared fairer and fatter in flesh} % Jubilee2000
    {their appearance was better and their bodies were healthier} % NETfree
    {their countenances appeared fairer and fatter in flesh} % UKJV
    {their countenances appeared fairer and fatter in flesh} % RNKJV
    {their countenances appeared fairer and fatter in flesh} % Webster
\Note 1:15 {they looked healthier and better nourished} 
God blessed Daniel and his friends for their obedience to the Lord and their refusal to
compromise their faith in a heathen environment (<Deut 8:3>; <Matt 4:4>).


\ww {God gave them knowledge and made them expert in all book-learning and wisdom} % BBE
    {God gave them knowledge and intelligence in all letters and science} % Jubilee2000
    {God endowed them with knowledge and skill in all sorts of literature and wisdom} % NETfree
    {God gave them knowledge and skill in all learning and wisdom} % UKJV
    {Elohim gave them knowledge and skill in all learning and wisdom} % RNKJV
    {God gave them knowledge and skill in all learning and wisdom} % Webster
\Note 1:17 {God gave knowledge and understanding of all kinds of literature and learning} God's
blessing was not limited to physical 
well-being but included outstanding success in intellectual development during their three years of
Babylonian education. visions and dreams of all kinds. With a view to what follows in the book
(chs. <"2"_2:1>; <"4-5"_4:1>) Daniel was distinguished from his companions by his ability to interpret dreams and
visions, much as Joseph had been set apart by the same in the court of Pharaoh (<Ge 40:8>;
<41:16>).



\ww {at the end of the time fixed by the king} % BBE
    {at the end of the days after which the king had said} % Jubilee2000
    {When the time appointed by the king arrived} % NETfree
    {at the end of the days that the king had said} % RNKJV
    {at the end of the days that the king had said} % UKJV
    {at the end of the days that the king had said} % Webster
\Note 1:18 {At the end of the time set by the king} After the three years mentioned in <"verse" 5>.



\ww {wonder-workers and users of secret arts} % BBE
    {magicians}={magicians and astrologers} % Jubilee2000
    {magicians and astrologers} % NETfree
    {magicians and astrologers} % RNKJV
    {magicians and astrologers} % UKJV
    {magicians}={magicians and astrologers} % Webster
\Note 1:20 {the magicians and enchanters} The term here translated \"magician" is also used in
<Ge 41:8> and <24> and <Ex 7:11>. The term translated \"enchanters" occurs only here and in \<2:2> and is
sometimes rendered \"conjurer" or \"soothsayer." Daniel and his friends demonstrated superior insight
on the matters about which they were questioned.

\ww {on till the first year of King Cyrus} % BBE
    {unto the first year of king Cyrus} % Jubilee2000
    {until the first year of Cyrus the king} % NETfree
    {unto the first year of king Cyrus} % RNKJV
    {unto the first year of king Cyrus} % UKJV
    {to the first year of king Cyrus} % Webster
\Note 1:21 {until the first year of King Cyrus} Babylon
fell to Cyrus in 539 B.C., 66 years after Daniel had been taken captive to Babylon. Daniel lived
through the entire period of the Babylonian captivity. Cyrus issued a decree in the first year of
his reign that permitted the Israelites to return from captivity and to take with them the vessels
from the temple that had been seized by Nebuchadnezzar (<Ezr 1:7-11>). The statement does not signify
that Daniel died in the first year of Cyrus's reign (\<10:1>).

\Note 2:1-49 {} {\it Nebuchadnezzar's First Dream.\/} While in the service of Nebuchadnezzar Daniel interpreted the
king's dream, revealing that Daniel was greatly blessed by God and that God was moving  
history toward the establishment of his kingdom.

\ww {In the second year} % BBE
    {in the second year} % Jubilee2000
    {In the second year} % NETfree
    {in the second year} % UKJV
    {in the second year} % RNKJV
    {in the second year} % Webster
\Note 2:1 {In the second year of his reign} There is no contradiction 
between this statement and the completion of the three-year period of training for Daniel and his
friends mentioned in \<1:5> and  <18-20> if one understands that the first year of training was
considered Nebuchadnezzar's \"accession year," while the second and  
third years would correspond with the \"first" and \"second" years 
of Nebuchadnezzar's reign. It was during Nebuchadnezzar's second year, according to the Babylonian
system of accession-year  dating, that the dream occurred (see <"note on" 1:1>n).

\ww {his spirit was troubled and his sleep went from him} % BBE
    {his spirit was troubled, and his sleep fled from him} % Jubilee2000
    {His mind was disturbed and he suffered from insomnia} % NETfree
    {his spirit was troubled, and his sleep brake from him} % UKJV
    {his spirit was troubled, and his sleep brake from him} % RNKJV
    {his spirit was troubled, and his sleep broke from him} % Webster
\Note 2:1 {his mind was troubled and he could not sleep} It was widely believed in the 
ancient Near East that the gods spoke to human beings in dreams. 
Nebuchadnezzar's agitation is understandable because the dream 
had implications for the future of his kingdom. When a dream 
could not be remembered, it was believed to be a sign that the deity was angry with the person
involved.

\ww {the wonder-workers, and the users of secret arts} % BBE
    {magicians, astrologers} % Jubilee2000
    {magicians,   astrologers} % NETfree
    {the magicians, and the astrologers} % UKJV
    {the magicians, and the astrologers} % RNKJV
    {the magicians, and the astrologers} % Webster
\Note 2:2 {magicians, enchanters} See <"note on" 1:20>n.



\ww {those who made use of evil powers} % BBE
    {enchanters} % Jubilee2000
    {sorcerers} % NETfree
    {sorcerers} % UKJV
    {sorcerers} % RNKJV
    {sorcerers} % Webster
\Note 2:2 {sorcerers} Practitioners of divination through means
such as witchcraft. Their activities were prohibited by God (<Ex 22:18>; <Dt 18:10>; <Isa 47:9>,
<12>; <Jer 27:9>).

\ww {Chaldaeans} % BBE
    {Chaldeans} % Jubilee2000
    {wise men} % NETfree
    {Chaldeans} % UKJV
    {Chaldeans} % RNKJV
    {Chaldeans} % Webster
\Note 2:2 {astrologers} This term translates the Hebrew term for 
Chaldeans; it is probably used here as a designation for a class of 
soothsayers concerned with astrology rather than as a designation 
for an ethnic group. See <1:4>; <3:8>; <5:30>; <9:1>
and NIV text notes.

\ww {Aramaean} % BBE
    {Syriack} % Jubilee2000
    {Aramaic} % NETfree
    {Syriack} % UKJV
    {Syriack} % RNKJV
    {Syriac} % Webster
\Note 2:4 {Aramaic} From here until the end of chapter 7 the text is written in Aramaic rather than in
Hebrew (<Ezr 4:8-6:18> was also written in Aramaic). It is not clear why the two languages were used. 
but Aramaic may have been used for the sections containing 
prophecies that would have been of more interest to non-Jews.

\ww {if you do not make clear to me the dream and the sense of it} % BBE
    {if ye will not make known unto me the dream with its interpretation} % Jubilee2000
    {If you do not inform me of both the dream and its interpretation} % NETfree
    {if all of you will not make known unto me the dream, with the interpretation thereof} % UKJV
    {if ye will not make known unto me the dream, with the interpretation thereof} % RNKJV
    {if ye will not make known to me the dream, with the interpretation of it} % Webster
\Note 2:5 {If you do not tell me what my dream was and interpret it. }
Nebuchadnezzar formulated a plan for testing his advisors. If they 
could not relate the dream back to him he would have no confidence in their interpretation (see <"v." 9>).


\ww {there is no other who is able to make it clear to the king, but the gods} % BBE
    {there is no one that can show it before the king except the angels}={there is no one that can show it before the king except the angels of God} % Jubilee2000
    {no one exists who can disclose it to the king, except for the gods} % NETfree
    {there is no other that can show it before the king, except the gods} % UKJV
    {none other that can shew it before the king, except the elohim} % RNKJV
    {there is no other that can show it before the king, except the gods} % Webster
\Note 2:11 {No one can reveal it to the king except the gods} The 
wise men were forced to confess that they were unable to do what 
the king asked. They claimed that only the gods have such power 
and that they do not reveal such things to men. See <Exod 8:18-19>. 

\ww {make a request for the mercy of the God of heaven in the question of this secret} % BBE
    {to petition mercies of the God of heaven concerning this mystery} % Jubilee2000
    {mercy from the God of heaven concerning this mystery} % NETfree
    {mercies of the God of heaven concerning this secret} % UKJV
    {mercies of the Elohim of heaven concerning this secret} % RNKJV
    {mercies of  the God of heaven concerning this secret} % Webster
\Note 2:18 {plead for mercy from the God of heaven concerning this 
mystery} Daniel also realized that human wisdom was insufficient 
to meet the king's demand (see <"note on" 2:11>n). Daniel addressed 
God as the ruler of the stars to which the heathen astrologers 
looked for guidance.

\ww {secret} % BBE
    {mystery} % Jubilee2000
    {mystery} % NETfree
    {secret} % UKJV
    {secret} % RNKJV
    {secret} % Webster
\Note2:19 {mystery} Here the word denotes an enigma that can be interpreted only by God's revelation. The
term was later used by Daniel as a reference to God's hidden purpose at work in history (<4:9>).


\ww {by him kings are taken away and kings are lifted up} % BBE
    {he removes kings and sets up kings} % Jubilee2000
    {deposing some}={deposing some kings and establishing others} % NETfree
    {he removes kings, and sets up kings} % UKJV
    {he removeth kings, and setteth up kings} % RNKJV
    {he removeth kings, and setteth up kings} % Webster
\Note 2:21 {he sets up kings and deposes them} Daniel alluded to  the 
content of the dream. See BC 36.



\ww {He is the unveiler of deep and secret things} % BBE
    {He reveals that which is deep and hidden} % Jubilee2000
    {he reveals deep and hidden things} % NETfree
    {He reveals the deep and secret things} % UKJV
    {He revealeth the deep and secret things} % RNKJV
    {He revealeth the deep and secret things} % Webster
\Note 2:22 {He reveals deep and hidden things} See <"note on" 2:11>n.

\ww {I give you praise and worship, O God of my fathers} % BBE
    {Unto thee, O God of my fathers, do I confess and give thee praise} % Jubilee2000
    {God of my fathers, I acknowledge and glorify you} % NETfree
    {I thank you, and praise you, O you God of my fathers} % UKJV
    {I thank thee, and praise thee, O thou Elohim of my fathers} % RNKJV
    {I thank thee, and praise thee, O thou God of my fathers} % Webster
\Note 2:23 {I thank and praise you, O God of my fathers} Daniel was
deeply grateful for God's mercy in responding to his prayer. The 
divine revelation he received was in stark contrast to the silence of
the false deities of the heathen soothsayers. Only God knows all
things and is sovereign over all creation. God chose to exalt Daniel 
by imparting to him special knowledge.

\ww {I will make clear to him the sense of the dream} % BBE
    {I will show unto the king the interpretation} % Jubilee2000
    {I will disclose the interpretation to him!} % NETfree  %Dokumentace? Půjde to, aby ! zrušil tečku?
    {I will show unto the king the interpretation} % UKJV
    {I will shew unto the king the interpretation} % RNKJV
    {I will show to the king the interpretation} % Webster
\Note 2:24 {I will interpret his dream for him} Daniel spoke here only 
of the interpretation of the dream. The text assumes that he already 
knew the content.


\ww {there is a God in heaven, the unveiler of secrets} % BBE
    {there is}={there is a God in the heavens who reveals the mysteries} % Jubilee2000
    {there is a God in heaven who reveals mysteries} % NETfree
    {there is a God in heaven that reveals secrets} % UKJV
    {there is an Elohim in heaven that revealeth secrets} % RNKJV
    {there is a God in heaven that revealeth secrets} % Webster
\Note 2:28 {there is a God in heaven who reveals mysteries} As Joseph had done in Egypt
(<Ge 10:8>; <41:16>), %Gn, not Ex
Daniel attributed his  
knowledge of the dream and its interpretation to divine revelation. 
God showed himself superior in his ability to reveal secrets and 
mysteries. in days to come. Literally, \"in the after part of the days." 
This expression can mean \"in the end times" or \"in the last days,"
which is the time of restoration after the exile (see <Dt 4:30>). The 
phrase may also simply refer to the general future (<Ge 49:1>; <Dt 4:30>; 
<31:29>). The Septuagint (the Greek translation of the OT) interprets 
ithere as \"in the last days," although it is difficult to determine Daniel's intended usage. The
Greek expression is used five times in the  
New Testament, two with reference to the age begun at Pentecost 
(<Ac 2:17>; <Heb 1:2>) and three with regard to the end of the age preceding the second advent of Christ
(<2Ti 3:1>; <Jas 5:3>; <2Pe 3:3>).
%%%%%%%%%%%%%%%%%%%%%%% CHECK and correct!!!!

\ww {head}={head ... gold, ... breast and arms ... silver, ... middle and sides ... brass, legs ... iron, ... feet in part of iron and in part of potter's earth} % BBE
    {head}={head ... gold, ... breast and arms ... silver, ... belly and thighs... brass, legs ... iron, ... feet  part of iron and part of baked clay} % Jubilee2000
    {head}={head ... gold, ... breast and arms ... silver, ... belly and thighs... brass, legs ... iron, ... feet  part of iron and part of clay} % NETfree
    {head}={head ... gold, ... breast and arms ... silver, ... belly and thighs... brass, legs ... iron, ... feet  part of iron and part of clay} % UKJV
    {head}={head ... gold, ... breast and arms ... silver, ... belly and thighs... brass, legs ... iron, ... feet  part of iron and part of clay} % RNKJV
    {head}={head ... gold, ... breast and arms ... silver, ... belly and thighs... brass, legs ... iron, ... feet  part of iron and part of clay} % Webster
\Note 2:32-33 {The head ... gold, its chest and arms ... silver, its 
belly and thighs ... bronze, its legs ... iron, its feet partly ... 
iron and partly ... clay} Moving from the head to the feet of the 
image, there is a decrease in both the value and weight of the materials but a general increase in
its strength. The image was clearly  too heavy with fragile feet.


\ww {stone}={stone ...not by hands} % BBE
    {stone}={stone ...not with hands} % Jubilee2000
    {stone}={not by human hands} % NETfree
    {stone}={stone ...without hands} % UKJV
    {stone}={stone ...without hands} % RNKJV
    {stone}={stone ...without hands} % Webster 
\Note 2:34 {rock ... not by human hands} Unlike the kingdoms represented by the statue, this rock would be
formed by God himself. In  the Old Testament a rock is often associated with kingship; here it 
is linked to the kingdom itself (see <1Co 10:4> and its <"note"_1Co 10:4>).
It is likely that Daniel had in mind the
Messiah, the great son of David, who  
would establish God's kingdom over all of the earth—including the 
Gentile nations (\<"v." 35>)—after the restoration from exile. See theological article "The Kingdom of
God" at <"Matthew 4"_Matt 4:1>. It struck the  
statue on its feet of iron and clay. Some interpreters view the 
mixture of iron and clay in the feet of the image as representing a
second phase of the fourth kingdom—as distinguished from the 
legs, which were made of solid iron (cf. \<"vv." 41-43>).



\ww {you are the head of gold}={you are the head of gold ... another kingdom ... third kingdom ... fourth kingdom} % BBE
    {Thou}={Thou art this head of gold ... another kingdom ... third kingdom ... fourth kingdom} % Jubilee2000
    {You are the head of gold}={You are the head of gold ... another kingdom ... third kingdom ... fourth kingdom} % NETfree
    {You are this head of gold}={You are this head of gold ... another kingdom ... third kingdom ... fourth kingdom} % UKJV
    {Thou art this head of gold}={Thou art this head of gold ... another kingdom ... third kingdom ... fourth kingdom} % RNKJV 
    {this head of gold}={Thou art this head of gold ... another kingdom ... third kingdom ... fourth kingdom} % Webster
\Note 2:38-40 {You are that head of gold ... another kingdom ... a third kingdom ... Finally ... a fourth kingdom} 
The four kingdoms represent the Babylonian,
Medo-Persian, Greek and Roman Empires. The climax of the dream occurs in the time of the fourth 
kingdom (see \"Introduction" and chart \"Visions in Daniel," at Daniel 2).

\ww {they will not be united} % BBE
    {they shall not cleave one to another} % Jubilee2000
    {without adhering to one another} % NETfree
    {they shall not cleave one to another} % UKJV
    {they shall not cleave one to another} % RNKJV
    {they shall not cleave one to another} % Webster
\Note 2:43 {the people will be a mixture and will not remain united}
The fourth kingdom would constitute a composite of peoples who 
would not adhere together well. Efforts to combine the diverse elements of the kingdom would not
succeed.



\ww {in the days of those kings} % BBE
    {in the days of those kings} % Jubilee2000
    {In the days of those kings} % NETfree
    {in the days of these kings} % UKJV
    {in the days of these kings} % RNKJV
    {in the days of these kings} % Webster
\Note 2:44 {In the time of those kings} Some interpreters surmise that 
 \"those kings" refers to the succeeding kings of the fourth kingdom. 
It seems best, however, to understand them as referring to the succession of the rulers of the four
kingdoms previously mentioned in  
this chapter. the God of heaven will set up a kingdom that will 
never be destroyed. Like other prophets Daniel spoke of the 
kingdom of God that would be established after the exile as permanent (e.g., <Isa 9:7>; <Joel 2:26-27>;
<Am 9:15>). The New Testament  
explains that the kingdom began with the first coming of Jesus and 
will reach its consummation at Christ's glorious return. See theological article \"The Kingdom of
God" at <"Matthew 4"_Matt 4:1>. 

\ww {Then King Nebuchadnezzar, falling down on his face} % BBE
    {Then the king Nebuchadnezzar fell upon his face} % Jubilee2000
    {Then King Nebuchadnezzar bowed down with his face to the ground} % NETfree
    {Then the king Nebuchadnezzar fell upon his face} % RNKJV
    {Then the king Nebuchadnezzar fell upon his face} % UKJV
    {Then the king Nebuchadnezzar fell upon his face} % Webster
\Note 2:46 {Then King Nebuchadnezzar fell prostrate before Daniel} In a remarkable reversal of roles Daniel
was exalted to a position of great honor by virtue of the Lord's intervention on his
behalf. Nebuchadnezzar's reaction anticipated the coming kingdom  of God.

\ww {your God is a God of gods} % BBE
    {the God}={the God that is your God is a God of gods} % Jubilee2000
    {your God is a God of gods} % NETfree
    {your God is a God of gods} % UKJV
    {your Elohim is an Elohim of elohim} % RNKJV
    {God of gods}={your God is a God of gods} % Webster
\Note 2:47 {your God is the God of gods} Nebuchadnezzar's statement 
does not signify that he recognized Israel's God as the only true
God, but he did perceive him to be superior to the deities of the 
Babylonian pantheon. 

\ww {and a Lord of kings} % BBE
    {and the Lord of the kings} % Jubilee2000
    {and Lord of kings} % NETfree
    {and a Lord of kings} % UKJV
    {and a master of kings} % RNKJV
    {and a Lord of kings} % Webster
\Note 2:47 {and the Lord of kings}
Nebuchadnezzar 
declared that Israel's God was supreme also over human rulers and 
their kingdoms. This is a unifying theme of Daniel <1:1-6:28>.


\ww {ruler over all the land of Babylon} % BBE
    {governor over the whole province of Babylon} % Jubilee2000
    {authority over the entire province of Babylon} % NETfree
    {ruler over the whole province of Babylon} % RNKJV
    {ruler over the whole province of Babylon} % UKJV
    {ruler over the whole province of Babylon} % Webster
\Note 2:48 {ruler over the entire province of Babylon} The Babylonian Empire was divided into
provinces. Daniel was appointed the  
ruler (cf. <3:2>) of the province in which the capital city was located. 
For accounts of similar ascents to political power by Jews in foreign 
lands, see <Ge 41:37-44> (Joseph)
and <Est 8:1-2> (Mordecai). 
Daniel's friends were similarly exalted as his assistants (\<"v." 49>). The 
divine approval of Daniel is another dominant theme in this portion of the book. Although prominent
in Babylon, he never compromised his faith: he was a reliable prophet of God.

\Note 3:1-30 {} {\it Deliverance From the Furnace.\/} Daniel recounted God's 
miraculous deliverance of his friends from the fiery furnace to instruct his readers that God's
people must admire Daniel's companions and be faithful to God alone. He also illustrated that God 
would eventually frustrate even the mightiest kings who tempt his 
people to abandon their God to worship another.

\ww {an image} % BBE
    {a statue} % Jubilee2000
    {statue} % NETfree
    {an image} % RNKJV
    {an image} % UKJV
    {an image} % Webster
\Note 3:1 {an image} Opinions differ as to whether this extraordinary 
image was of Nebuchadnezzar himself or of a Babylonian deity or 
whether it was merely an obelisk. From what is known of Babylonian religious tradition, it seems
likely that the image was either of  
Bel or of Nabu, Nebuchadnezzar's patron deity. Prostration before 
the image of this deity would also indicate submission to Nebuchadnezzar, the deity's representative
(cf. <2:46>).

\ww {of gold} % BBE
    {of gold} % Jubilee2000
    {golden} % NETfree
    {of gold} % RNKJV
    {of gold} % UKJV
    {of gold} % Webster
\Note 3:1 {gold} Probably  
gold overlay, the fabrication of the image being much like that described in <Isa 40:19>, <41:7> and
<Jer 10:3-9>.


\ww {sixty cubits high and six cubits wide} % BBE
    {height}={height was sixty cubits and its breadth six cubits} % Jubilee2000
    {ninety feet tall and nine feet wide} % NETfree
    {height was threescore cubits, and the breadth thereof six cubits} % RNKJV
    {height was threescore cubits, and the breadth thereof six cubits} % UKJV
    {hight}={hight was sixty cubits, and the breadth of it six cubits} % Webster
\Note 3:1 {ninety feet high and nine feet wide} The proportions are the reason some have 
concluded that the image was an obelisk rather than a human form 
(the proportions of the human body are six to one). However, the 
image may have stood on a piedestal or had a stylized shape. 


\ww {the valley of Dura} % BBE
    {the plain of Dura} % Jubilee2000
    {the plain of Dura} % NETfree
    {the plain of Dura} % RNKJV
    {the plain of Dura} % UKJV
    {the plain of Dura} % Webster
\Note 3:1 {the plain of Dura} Its location is uncertain. It is usually associated with 
Tolul Dura, located about six miles south of Babylon.


\ww {the captains, the chiefs, the rulers, the wise men, the keepers of public money, the judges, the overseers} % BBE
    {the great}={the great ones, the assistants and captains, the judges, the treasurers, those of the council, presidents} % Jubilee2000
    {the satraps, prefects, governors, counselors, treasurers, judges, magistrates} % NETfree
    {the princes, the governors, and the captains, the judges, the treasurers, the counsellors, the sheriffs} % UKJV
    {the princes, the governors, and the captains, the judges, the treasurers, the counsellers, the sheriffs} % RNKJV
    {the princes, the governors, and the captains, the judges, the treasurers, the counselors, the sherifs} % Webster
\Note 3:2 {satraps ... officials} The precise responsibilities of these seven different types of officials
are not known. Five of the seven  terms seem to be of Persian origin, perhaps indicating that Daniel 
did not complete the writing of this account until after the begin. 
ning of Persian rule in 539 B.C.

\Note 3:4-6 {}  See WLC 130.

\ww {horn, pipe, harp, trigon, psaltery, bagpipe} % BBE
    {cornet, flute, harp, sackbut, psaltery, dulcimer} % Jubilee2000
    {horn, flute, zither, trigon, harp, pipes} % NETfree
    {cornet, flute, harp, sackbut, psaltery, dulcimer} % RNKJV
    {cornet, flute, harp, sackbut, psaltery, dulcimer} % UKJV
    {cornet, flute, harp, sackbut, psaltery, dulcimer} % Webster
\Note 3:5 {horn ... pipes} Three of the six terms used for different types 
of musical instruments account for the only Greek loanwords 
( \"zither," \"harp" and \"pipes") in Daniel. This is not surprising, since 
the exchange of musicians and their instruments at royal courts 
has a long history. The presence of these Greek terms does not 
therefore constitute compelling evidence that this account was 
written after the conquests of Alexander the Great.

\ww {burning and flaming fire} % BBE
    {burning fiery furnace} % Jubilee2000
    {furnace of blazing fire} % NETfree
    {burning fiery furnace} % UKJV
    {burning fiery furnace} % RNKJV
    {burning fiery furnace} % Webster
\Note 3:6 {a blazing furnace} Furnaces, or kilns, were widely used in 
Babylon for the firing of bricks (<Ge 11:3>). It was not unusual to use 
such furnaces for execution by burning (<Jer 29:22>; see Herodotus. 
1.86; 4.69; see also 2 Maccabees 7).


\ww {Chaldaeans} % BBE
    {Chaldeans} % Jubilee2000
    {Chaldeans} % NETfree
    {Chaldeans} % UKJV
    {Chaldeans} % RNKJV
    {Chaldeans} % Webster
\Note 3:8 {Chaldeans} See NIV text note and <"note on" 2:2>. The term 
 \"Chaldeans" as used here is best understood as indicating nationality rather than function. The
informants looked down on the  
Jews simply because they were Jews (<"v." 12>; <Est 3:5>. The privileged position of Shadrach, Meshach and
Abednego (\<2:49>) heightened the Chaldeans' hostility toward them (\<"v." 12>). 

\ww {Shadrach, Meshach, and Abed-nego} % BBE
    {Shadrach, Meshach, and Abednego} % Jubilee2000
    {Shadrach, Meshach, and Abednego} % NETfree
    {Shadrach, Meshach, and Abednego} % UKJV
    {Shadrach, Meshach, and Abed-nego} % RNKJV
    {Shadrach, Meshach, and Abed-nego} % Webster
\Note 3:12 {Shadrach, Meshach and Abed-nego} See <"note on" 1:7>n. Daniel was either not present or exempted from
demonstrating his loyalty because of his high position (<2:48>).


\ww {what god is there who will be able to take you out of my hands?} % BBE
    {who}={who is that god that shall deliver you out of my hands?} % Jubilee2000
    {who is that god who can rescue you from my power?} % NETfree
    {who is that God that shall deliver you out of my hands?} % UKJV
    {who is that Elohim that shall deliver you out of my hands?} % RNKJV
    {who}={who is that God that shall deliver you out of my hands?} % Webster
\Note 3:15 {Then what god will be able to rescue you from my 
hand?} From Nebuchadnezzar's polytheistic, heathen perspective 
there was no god capable of such deliverance. Unwittingly Nebuchadnezzar challenged the power of the
God of Israel. 
%Do dokumentace

\ww {our God, whose servants we are, is able}={our God, whose servants we are, is able ... we will not be the servants of your gods} % BBE
    {our God whom we serve is able}={our God whom we serve is able ... we will not worship thy god} % Jubilee2000
    {our God whom we are serving exists, he is able}={our God whom we are serving exists, he is able ... we don’t serve your gods} % NETfree
    {our God whom we serve is able}={our God whom we serve is able ... we will not serve your gods} % UKJV
    {our Elohim whom we serve is able}={our Elohim whom we serve is able ... we will not serve thy elohim} % RNKJV
    {our God whom we serve is able}={our God whom we serve is able ... we will not serve thy gods} % Webster
\Note 3:17-18 {the God we serve is able ... we will not serve your gods} The men did not assert that God
always protects his people from physical harm (<Isa 43:1-2>). Although he may opt to do so, and
certainly is able, the central idea is that God's people should remain obedient to their Lord no
matter what the circumstances because he is far more trustworthy than any human ruler and more
powerful than any force on Earth. Thus the first six chapters of Daniel exalt the prophet and his
friends as men who were unflinchingly faithful to God throughout their ordeals. See WLC 109.

\ww {a son of the gods} % BBE
    {Son of God} % Jubilee2000
    {that of a god} % NETfree
    {Son of God} % UKJV
    {Son of Elohim} % RNKJV
    {son of God} % Webster
\Note 3:25 {a son of the gods} In the ancient world the expression \"son of the gods" could refer to various types
of heavenly beings. Here it meant \"angel" (<"v." 28>). No explanation is given for why Nebuchadnezzar
recognized the fourth person in the furnace as a heavenly being (see <"note on v." 28>n). Perhaps the
miraculous presence of the fourth person was in itself sufficient reason for this conclusion.

\ww {the Most High God} % BBE
    {the most high God} % Jubilee2000
    {the most high God} % NETfree
    {the most high God} % UKJV
    {the most high Elohim} % RNKJV
    {the most high God} % Webster
\Note 3:26 {the Most High God} A title for God's universal authority. As in verse <29> ( \"no other god can save in
this way") and in <2:47>, this confession on the lips of a pagan was not an acknowledgment that
Daniel's Lord alone was God but rather that Daniel's God was supreme above other deities
(<4:2>, <17>, <34>). On the lips of an Israelite the same confession implied monotheism (<4:24-32>;
<5:18>, <21>; <7:18-27>).  

\Note 3:27 {}  See WCF 5.3.

\Note 3:28 {angel} The angel may be identified with \"the angel of the 
LORD," who may have represented an appearance of Christ prior 
to his incarnation (cf. <6:22>; see notes on <Ge 16:7> and <Ex 3:2>). God 
promised his presence when Israel walked through fire (<Isa 43:1-3>).

\ww {no other god} % BBE
    {no other god} % Jubilee2000
    {no other god} % NETfree
    {no other God} % UKJV
    {no other Elohim} % RNKJV
    {no other god} % Webster
\Note 3:29 {no other god} See <"note on verse" 26>. See also WCF 20.4.

\ww {the king gave} % BBE
    {the king promoted} % Jubilee2000
    {Nebuchadnezzar promoted} % NETfree
    {the king promoted} % UKJV
    {the king promoted} % RNKJV
    {the king promoted} % Webster
\Note 3:30 {the king promoted} As this narrative makes clear, their 
prominence resulted from their faithfulness to God, not from compromise with the Babylonians. 

\Note  4:1-37 {} {\it Nebuchadnezzar's Second Dream.\/} The prophet narrated 
the story of the king's second dream and its interpretation. Once 
again Daniel was exalted and Nebuchadnezzar humbled before 
God.

\ww {Nebuchadnezzar the king} % BBE
    {King Nebuchadnezzar} % Jubilee2000
    {King Nebuchadnezzar} % NETfree
    {Nebuchadnezzar the king} % UKJV
    {Nebuchadnezzar the king} % RNKJV    
    {Nebuchadnezzar the king} % Webster
\Note 4:1 {King Nebuchadnezzar} This is the book's final incident associated with Nebuchadnezzar. It too is
placed late in the king's 43-year reign, at a time when his building projects were completed 
and his power was at its height (cf. vv. <4>, <30>). At that time Nebuchadnezzar ruled over the most
powerful kingdom on Earth, but he was no match for the God of Israel.

\ww {the Most High God} % BBE
    {the high God} % Jubilee2000
    {the most high God} % NETfree
    {the high God} % UKJV
    {the high Elohim} % RNKJV    
    {the high God} % Webster
\Note 4:2 {the Most High God} See <"notes on" 2:47>n and <3:26>n and <28>n.
%%%%% Do dokumentace jako příklad

\Note 4:3 {How great}={How great are his signs!} Nebuchadnezzar's confession in this verse 
and in <"verses" 34-35> communicates one of the central themes of 
the book of Daniel; namely, the absolute sovereignty of the God of 
Israel over the kingdoms of the earth and their rulers.

\Note 4:6-7 {}  See <"notes on" 1:20>n and <2:2>n. 

\Note 4:8 {Belteshazzar} See <"note on" 1:7>n.

\ww {} % BBE
    {} % Jubilee2000
    {} % NETfree
    {} % UKJV
    {} % RNKJV    
    {} % Webster
\Note 4:9 {spirit of the holy gods} Although he spoke in pagan terms 
Nebuchadnezzar stated an important truth. The presence of God's 
Spirit in an individual has remarkable effects. Here his ability to 
give extraordinary insight into God's mystery, such as was later given to Paul and the church
(<1Co 2:6-16>), is in view.

\ww {spirit of the holy gods} % BBE
    {spirit of the holy God} % Jubilee2000
    {spirit of the holy gods} % NETfree
    {spirit of the holy gods} % UKJV
    {spirit of the holy elohim} % RNKJV    
    {spirit of the holy gods} % Webster
\Note 4:9 {no mystery is too difficult for you} See <2:47> and <"note on" 2:19>n.

\ww {there was a tree} % BBE
    {I saw a tree} % Jubilee2000
    {there was a tree} % NETfree
    {behold a tree} % UKJV
    {behold a tree} % RNKJV    
    {behold a tree} % Webster
\Note 4:10 {before me stood a tree} See <Eze 31> for an extensive ' description of a nation (Assyria), %Egypt?
using the imagery of a tree. Similar imagery is found in <Ps 1:3>; <37:35>; <52:8>; <92:12>; <Jer
11:16-17> and <17:8>  (see also <Mt 13:32>).

\ww {up to heaven} % BBE
    {unto heaven} % Jubilee2000
    {Its top reached far into the sky} % NETfree
    {reached unto heaven} % UKJV
    {reached unto heaven} % RNKJV    
    {reached to heaven} % Webster
\Note 4:11 {its top touched the sky} The term \"sky" may also be
translated \"heaven," a key term in this chapter. The tree represented Nebuchadnezzar's kingdom
reaching from Earth to heaven (<"vv." 11>, <20>, <22>) and protecting birds, which defy the separation of
the two spheres (<"vv." 12>, <21>). In truth the king was not only subject to the judgment of heaven for
his pride (<"vv." 13>, <23>, <31>) but also dependent on the God of heaven for his existence
(<"vv." 15>, <22>, <25>, <33>) and sanity (<"v." 34>).  


\ww {watcher} % BBE
    {watchman} % Jubilee2000
    {holy sentinel} % NETfree
    {watcher} % UKJV
    {watcher} % RNKJV    
    {watcher} % Webster
\Note 4:13 {messenger} Although Nebuchadnezzar continued speaking in terms of his pagan religion, he
acknowledged that he saw a holy, heavenly being in his vision. This common ancient Near Eastern
belief fits well with the Biblical truth that God involves himself in Earth's affairs through
revelations by angels.

\ww {let him} % BBE
    {let him} % Jubilee2000
    {let it live} % NETfree
    {let his portion be} % UKJV
    {let his portion be} % RNKJV    
    {let it be} % Webster
\Note 4:15 {Let him} From Hebrew pronoun \"he" it becomes clear that the dream concerned a human being and
not just a tree. See <"note on verse" 22>n.

\ww {the heart of a beast be given to him} % BBE
    {let a beast's heart be given unto him} % Jubilee2000
    {let an animal's mind be given to him} % NETfree
    {let a beast's heart be given unto him} % UKJV
    {let a beast's heart be given unto him} % RNKJV    
    {let a beast's heart be given to him} % Webster
\Note 4:16 {let him be given the mind of an animal} Nebuchadnezzar
may have suffered from a recognized mental illness called lycanthropy which comes from the Greek
words lukos ( \"wolf") and anthropos ( \"man") in which a person is deluded into behaving like a wolf
or some other animal (<"v." 33>; see also <"note on" 4:33>n).

\ww {seven times} % BBE
    {seven times} % Jubilee2000
    {seven periods of time} % NETfree
    {seven times} % UKJV
    {seven times} % RNKJV    
    {seven times} % Webster
\Note 4:16 {till seven times pass by for him} Seven periods
of an unspecified duration (cf. <"vv." 23>, <25>). Most interpreters conclude that \"time" represents a
period of one year. Verse <33> suggests that the period was longer than a day, week or month.

\ww {It is you, O King} % BBE
    {thou, O king}={it is thou, O king} % Jubilee2000
    {it is you, O king} % NETfree
    {It is you, O king} % UKJV
    {It is thou, O king} % RNKJV    
    {thou, O king}={It is thou, O king} % Webster
\Note 4:22 {you, O king, are that tree!} With this statement—much like  that of Nathan to David (<2Sa 12:7>)—a
direct application was made to Nebuchadnezzar.
%%%Jestli to pujde, tak do dokumentace


\ww {That they will send you out from among men, to be with the beasts of the field} % BBE
    {they shall drive thee from among men, and thy dwelling shall be with the beasts of the field} % Jubilee2000
    {You will be driven from human society, and you will live with the wild animals} % NETfree
    {they shall drive you from men, and your dwelling shall be with the beasts of the field} % UKJV
    {they shall drive thee from men, and thy dwelling shall be with the beasts of the field} % RNKJV    
    {they shall drive thee from men, and thy dwelling shall be with the beasts of the field} % Webster
\Note 4:25 {You will be driven away from people and will live with 
the wild animals.} In words more specific than those in <"verse" 15> 
Daniel indicated the form of mental illness that God would bring 
upon the mighty Nebuchadnezzar. Similar symptoms occasionally 
afflicted King George III of England (1738-1820) and Otto of 
Bavaria (1848-1916). See <"note on" 4:16>n.


\ww {till you are certain that the Most High is ruler in the kingdom of men} % BBE
    {until thou shalt understand that the most High takes rule over the kingdom of men} % Jubilee2000
    {before you understand that the Most High is ruler over human kingdoms} % NETfree
    {till you know that the most High rules in the kingdom of men} % UKJV
    {till thou know that the most High ruleth in the kingdom of men} % RNKJV    
    {till thou shalt know that the most High ruleth in the kingdom of men} % Webster
\Note 4:25 {until you acknowledge that the Most High is sovereign over the kingdoms of men}
The purpose of Nebuchadnezzar's humiliation was to compel him 
to recognize God's sovereignty. See WCF 2.2.

\ww {your kingdom will be safe for you} % BBE
    {thy kingdom shall remain sure unto thee} % Jubilee2000
    { your kingdom will be restored to you} % NETfree
    {your kingdom shall be sure unto you} % UKJV
    {thy kingdom shall be sure unto thee} % RNKJV    
    {thy kingdom shall be sure to thee} % Webster
\Note 4:26 {your kingdom will be restored to you} Nebuchadnezzar 
was assured that, in spite of the severity and length of his illness, 
he would regain the throne subsequent to his acknowledgment of 
God's sovereignty. Heaven rules. For the first time in Scripture 
 \"heaven" is used as a substitute name for God (cf. <"v." 37>). Compare 
<Matt 5:3> with <Lk 6:20>.

\Note 4:30 {}  See WLC 105.

\ww {had grass for his food like the oxen} % BBE
    {ate grass as the oxen} % Jubilee2000
    {ate grass like oxen} % NETfree
    {did eat grass as oxen} % UKJV
    {did eat grass as oxen} % RNKJV    
    {ate grass as oxen} % Webster
\Note 4:33 {ate grass like cattle} Because Nebuchadnezzar exhibited 
traits characteristic of oxen, the form of his mental illness is sometimes termed boanthropy. See
<"note on" 4:16>n. 

\Note 4:34-35,37 {}  Although Nebuchadnezzar confessed God's sovereignty in no uncertain terms, he never
explicitly affirmed the God of Israel as the only supreme Creator of the universe. See WCF 2.2:
5.1.

%\ww {King of heaven} % BBE
%    {King of heaven} % Jubilee2000
%    {} % NETfree
%    {King of heaven} % UKJV
%    {} % RNKJV    
%    {} % Webster
\Note 4:37 {King of heaven} This unique term brings together the theme 
of the chapter: the rule of God from heaven (see <4:26> and its <"note"_4:26>n). 

\Note 5:1-31 {} {\it Judgment on Belshazzar.\/} Daniel turned next to an account of God's judgment against
Belshazzar. In this narrative the  king is condemned for his impudent disregard for the holiness of
Israel's God and of his temple. 

\ww {Belshazzar the king} % BBE
    {Belshazzar the king} % Jubilee2000
    {King Belshazzar} % NETfree
    {Belshazzar the king} % UKJV
    {Belshazzar the king} % RNKJV    
    {Belshazzar the king} % Webster
\Note 5:1 {King Belshazzar} Belshazzar means \"Bel, protect the king." It 
is not to be confused with Belteshazzar, the Babylonian name given to Daniel (see <"note on"
1:7>n).
From Babylonian sources we know  that Nabonidus, Nebuchadnezzar's son-in-law, was the last king of 
Babylon. Belshazzar, the eldest son of Nabonidus, was made co-regent with his father and placed in
charge of affairs in Babylon while Nabonidus spent extensive periods of time at Tema in Arabia. The 
events of this chapter took place in 539 B.C., the year of Babylon's 
fall to the Persians and of the edict releasing Israelites from captivity. 42 years after the death
of Nebuchadnezzar in 563 B.C.

\ww {great feast} % BBE
    {great banquet} % Jubilee2000
    {great banque} % NETfree
    {great feast} % UKJV
    {great feast} % RNKJV    
    {great feast} % Webster
\Note 5:1 {banquet} The banquet scene juxtaposes 
the splendor of the event and  the divine judgment that would soon be meted out (cf.
<Ge 40:20-22>; <Mk 6:21-28>).

\ww {Belshazzar, while he was overcome with wine} % BBE
    {Belshazzar, under the influence of the wine} % Jubilee2000
    {While under the influence of the wine} % NETfree
    {Belshazzar, while he tasted the wine} % UKJV
    {Belshazzar, whiles he tasted the wine} % RNKJV    
    {Belshazzar, while he tasted the wine} % Webster
\Note 5:2 {While Belshazzar was drinking} Under the influence of alcohol  Belshazzar committed a sacrilegious act. Even from a heathen standpoint the holy things of
other religions were to be held  in reverence.

\ww {gold and silver vessels}={gold and silver vessels ... from the Temple in Jerusalem} % BBE
    {vessels of gold and of silver}={vessels of gold and of silver ... from the Temple in Jerusalem} % Jubilee2000
    {gold and silver vessels}={gold and silver vessels ... from the temple in Jerusalem} % NETfree
    {golden and silver vessels}={golden and silver vessels ... taken out of the temple which was in Jerusalem} % UKJV
    {golden and silver vessels}={golden and silver vessels ... taken out of the temple which was in Jerusalem} % RNKJV    
    {golden and silver vessels}={golden and silver vessels ... taken out of the temple which was in Jerusalem} % Webster
\Note 5:2 {the gold and silver goblets ... from the temple in Jerusalem}
See <"note on" 1:2>n.

\Note 5:2 {his father} %See NIV text note.
Nebuchadnezzar is called the father of
Belshazzar here and in <"verses" 11>, <13> and <18>, and in <"verse" 22> Belshazzar is called the
 \"son" of Nebuchadnezzar. Although we know that 
Belshazzar was the immediate son of Nabonidus, not Nebuchadnezzar, the terms father and 
son were often used in the ancient world in the broader sense of 
 \"ancestor" or \"predecessor" and \"descendant" or \"successor." respectively. It is likely that
Belshazzar was the grandson of Nebuchadnezzar through his mother, Nitocris.


\ww {gave praise to the gods} % BBE
    {praised the gods} % Jubilee2000
    {praised the gods} % NETfree
    {praised the gods} % UKJV
    {praised the elohim} % RNKJV    
    {praised the gods} % Webster
\Note 5:4 {praised the gods} The temple vessels were defiled not only by being put to profane use but also
by being used to honor the false deities of Babylon.

\ww {the users of secret arts, the Chaldaeans, and the readers of signs} % BBE
    {the magicians, the Chaldeans, and the fortune-tellers} % Jubilee2000
    {the astrologers, wise men, and diviners} % NETfree
    {the astrologers, the Chaldeans, and the soothsayers} % UKJV
    {the astrologers, the Chaldeans, and the soothsayers} % RNKJV    
    {the astrologers, the Chaldeans, and the sooth-sayers} % Webster
\Note 5:7 {the enchanters, astrologers and diviners} See <"notes on" 1:20>n and <2:2>n (cf. <2:27>;
<4:7>).

\ww {Whoever is able to make out this writing, and make clear to me the sense of it} % BBE
    {Whoever shall read this writing and show me its interpretation} % Jubilee2000
    {anyone who could read this inscription and disclose its interpretation} % NETfree
    {Whosoever shall read this writing, and show me the interpretation thereof} % UKJV
    {Whosoever shall read this writing, and shew me the interpretation thereof} % RNKJV    
    {Whoever shall read this writing, and show me the interpretation of it} % Webster
\Note 5:7 {Whoever reads this writing and tells me what it means}
Once again the king demanded a double requirement: to declare the portent and then to
interpret it (cf. <2:3>).

\ww {a ruler of high authority in the kingdom} % BBE
    {the third ruler in the kingdom} % Jubilee2000
    {third ruler in the kingdom} % NETfree
    {the third ruler in the kingdom} % UKJV
    {the third ruler in the kingdom} % RNKJV    
    {the third ruler in the kingdom} % Webster
\Note 5:7 {third highest ruler in the kingdom} Next in power under Nabonidus and his
co-regent Belshazzar (see <"note on" 5:1>n).

\ww {they were not able to make out the writing or give the sense of it to the king} % BBE
    {they could not read the writing, nor make known to the king its interpretation} % Jubilee2000
    {they were unable to read the writing or to make known its interpretation to the king} % NETfree
    {they could not read the writing, nor make known to the king the interpretation thereof} % UKJV
    {they could not read the writing, nor make known to the king the interpretation thereof} % RNKJV    
    {they could not read the writing, nor make known to the king the interpretation of it} % Webster
\Note 5:8 {they could not read the writing or tell the king what it meant}
   See <2:2-13> and <4:7>; see also <Ge 41:8>.

%\ww {} % BBE
%    {} % Jubilee2000
%    {} % NETfree
%    {} % UKJV
%    {} % RNKJV    
%    {} % Webster
\Note 5:10 {the queen} %See NIV text note.
It is unlikely that she was a consort of Belshazzar since these women were already present at the banquet
(vv. 2-3). She may have been the widow of Nebuchadnezzar, but it is more likely that she was
Nitocris, the wife of Nabonidus, daughter of Nebuchadnezzar and mother of Belshazzar.

\ww {the spirit of the holy gods} % BBE
    {the spirit of the holy God} % Jubilee2000
    {a spirit of the holy gods} % NETfree
    {the spirit of the holy gods} % UKJV
    {the spirit of the holy elohim} % RNKJV    
    {the spirit of the holy gods} % Webster
\Note 5:11 {the spirit of the holy gods} See <4:8>. It is not surprising that the queen mother was
more familiar with the events of Daniel's time than was Belshazzar. It is likely that Daniel was in his 80s by  
539 B.C. He had been a young man when taken to Babylon 66 years 
earlier in 605 B.C. (see <"note on" 1:1>n).

\ww {were seen to be in him} % BBE
    {was found to have} % Jubilee2000
    {Thus there was found in this man Daniel} % NETfree
    {were found in the same Daniel} % UKJV
    {were found in the same Daniel} % RNKJV    
    {were found in the same Daniel} % Webster
\Note 5:12 {This man ... was found to have} This divine enablement 
can be described theologically as the presence of God's Spirit in an 
individual or as a person possessing a remarkable spirit. 


\ww {Belteshazzar} % BBE
    {Beltechazzar} % Jubilee2000
    {Belteshazzar} % NETfree
    {Belteshazzar} % UKJV
    {Belteshazzar} % RNKJV    
    {Belteshazzar} % Webster
\Note 5:12 {Belteshazzar} See <"note on" 1:7>n.

\ww {a ruler of high authority in the kingdom} % BBE
    {the third ruler in the kingdom} % Jubilee2000
    {third ruler in the kingdom} % NETfree
    {the third ruler in the kingdom} % UKJV
    {the third ruler in the kingdom} % RNKJV    
    {the third ruler in the kingdom} % Webster
\Note 5:16 {third highest ruler in the kingdom} See <"note on verse" 7>n.

\ww {Keep your offerings} % BBE
    {Let thy gifts be for thyself} % Jubilee2000
    {Keep your gifts} % NETfree
    {Let your gifts be to yourself} % UKJV
    {Let thy gifts be to thyself} % RNKJV    
    {Let thy gifts be to thyself} % Webster
\Note 5:17 {You may keep your gifts} Some think that Daniel rejected 
Belshazzar's offer of reward not only because he did not seek such 
honors but also because of his consciousness that it was only by 
God's mercy that he had been able to respond to the king's request; he did not want to use his
God-given role as a means of personal profit (<Ge 14:23>). Yet he had accepted such rewards before 
(<2:48>) and did so again later (<"v." 29>). Perhaps he was avoiding any 
pressure to modify the ominous message (<Nu 22:18>; <Mic 3:5>, <11>). 

\ww {the Most High God gave} % BBE
    {the most high God} % Jubilee2000
    {the most high God bestowed} % NETfree
    {most high God gave} % UKJV
    {the most high Elohim gave} % RNKJV    
    {the most high God gave} % Webster
\Note 5:18 {the Most High God gave} See <2:37> and <4:36>.

\ww {Nebuchadnezzar, your father} % BBE
    {Nebuchadnezzar thy father} % Jubilee2000
    {your father Nebuchadnezzar} % NETfree
    {Nebuchadnezzar your father} % UKJV
    {Nebuchadnezzar thy father} % RNKJV    
    {Nebuchadnezzar thy father} % Webster
\Note 5:18 {your father Nebuchadnezzar} See <"note on verse" 2>n.

\Note 5:20-21 {}  See <4:31-33>.

\ww {the Most High is ruler} % BBE
    {the most high God takes rule} % Jubilee2000
    {the most high God rules} % NETfree
    {the most high God ruled} % UKJV
    {the most high Elohim ruled} % RNKJV    
    {the most high God ruleth} % Webster
\Note 5:21 {Most High God is sovereign} This statement summarizes 
the book's theology (see \"Introduction: Purpose and Distinctives").

\ww {you, his son} % BBE
    {thou his son} % Jubilee2000
    {you, his son} % NETfree
    {you his son} % UKJV
    {thou his son} % RNKJV    
    {thou his son} % Webster
\Note 5:22 {But you his son} See <"note on verse" 2>n.

\ww {though you had knowledge of all this} % BBE
    {though thou knewest all this} % Jubilee2000
    {although you knew all this} % NETfree
    {though you knew all this} % UKJV
    {though thou knewest all this} % RNKJV    
    {though thou knewest all this} % Webster
\Note 5:22 {though you knew all this} Because the king was without excuse—even more so than his 
father—the time of mercy had passed (see <1Ti 1:13>). See WLC 151. 

\Note 5:23 {}  See WLC 105.

\ww {Then} % BBE
    {Then} % Jubilee2000
    {Therefore} % NETfree
    {Then} % UKJV
    {Then} % RNKJV    
    {Then} % Webster
\Note 5:24 {Therefore} The writing on the wall was God's answer to the 
arrogant challenge presented by Belshazzar's pride and defiance of 
the God who had demonstrated his existence and sovereignty in 
the time of Nebuchadnezzar.

\ww {Mene, tekel, peres} % BBE
    {MENE, MENE, TEKEL, UPHARSIN} % Jubilee2000
    {MENE, MENE, TEQEL, and PHARSIN} % NETfree
    {MENE, MENE, TEKEL, UPHARSIN} % UKJV
    {MENE, MENE, TEKEL, UPHARSIN} % RNKJV    
    {MENE, MENE, TEKEL, UPHARSIN} % Webster
\Note 5:25 {MENE, MENE, TEKEL, PARSIN} Literally, \"numbered, numbered, 
weighed, divided" or \"mina [a unit of weight], mina, shekel, half 
shekel."

\ww {Mene} % BBE
    {MENE} % Jubilee2000
    {mene} % NETfree
    {MENE} % UKJV
    {MENE} % RNKJV    
    {MENE} % Webster
\Note 5:26 {Mene} %See NIV text note.
The original script for this word 
could be understood as either a verb or a noun. Daniel read it as a 
verb meaning \"numbered" or \"counted" and interpreted it as signifying that the days and years of Belshazzar's reign had been determined by God and were about to end.

\ww {Tekel} % BBE
    {TEKEL} % Jubilee2000
    {teqel} % NETfree
    {TEKEL} % UKJV
    {TEKEL} % RNKJV    
    {TEKEL} % Webster
\Note 5:27 {Tekel} %See NIV text note.
This word could also be understood as either a verb or a noun. Daniel
read it as a verb meaning  
 \"weighed" and interpreted it as signifying that Belshazzar failed to 
measure up to God's standards of righteousness.

\ww {Peres} % BBE
    {PERES} % Jubilee2000
    {peres} % NETfree
    {PERES} % UKJV
    {PERES} % RNKJV    
    {PERES} % Webster
\Note 5:28 {Peres} %See NIV text note.
Daniel construed this word as a 
verb meaning \"divided" and interpreted it to signify that Belshazzar's kingdom would be taken from
him and given to the Medes  
and Persians. If, as is likely, those present at the banquet understood the three terms as nouns
that simply indicated various monetary weights (mene, a weight equivalent to 60 Babylonian shekels: 
tekel, the shekel: peres, a half shekel), then it is not surprising that 
they failed to comprehend the significance of the inscription. 
Medes and Persians. See \"Introduction: Purpose and Distinctives." See also BC 36.

\ww {by the order of Belshazzar} % BBE
    {Belshazzar commanded} % Jubilee2000
    {on Belshazzar's orders} % NETfree
    {commanded Belshazzar} % UKJV
    {commanded Belshazza} % RNKJV    
    {commanded Belshazza} % Webster
\Note 5:29 {Belshazzar's command} Like Nebuchadnezzar Belshazzar 
honored Daniel (<2:48>), but unlike Nebuchadnezzar he did not honor Daniel's God (<2:46-47>). The honor
that Daniel and his companions had repeatedly received because of their faithfulness to God 
had established Daniel's credibility as a prophet. He was not a 
compromiser; he was faithful to God. Therefore his later prophecies ("chs. 7-12"_7:1>) could be fully
trusted. 

\ww {Belshazzar}={Belshazzar ... was put to death}  % BBE
    {Belshazzar}={Belshazzar ...was slain}  % Jubilee2000
    {Belshazzar}={Belshazzar ... was killed}  % NETfree
    {Belshazzar}={was Belshazzar ... slain}  % UKJV
    {Belshazzar}={was Belshazzar ... slain}  % RNKJV    
    {Belshazzar}={was Belshazzar ... slain}  % Webster
\Note 5:30 {Belshazzar ... was slain} Ancient Near Eastern texts and 
the Greek historians Herodotus and Xenophon record that Babylon was taken in a surprise attack by
the Persians while the Babylonians were engaged in reveling and dancing.

\ww {Darius the Mede} % BBE
    {Darius the Median} % Jubilee2000
    {Darius the Mede} % NETfree
    {Darius the Median} % UKJV
    {Darius the Median} % RNKJV    
    {Darius the Median} % Webster
\Note 5:31 {Darius the Mede} It has long been alleged that this and other references to \"Darius the Mede"
in the book of Daniel (<6:1>, <6>,  <9>, <25>, <28>; <9:1>; <11:1>) are historical errors. See <"note on" 6:1>n.

\Note 5:32 {}  See WCF 19.2.

\Note 6:1-28 {} {\it Deliverance From the Den of Lions.\/} The prophet recounted his treatment under Darius the Mede,
who succeeded Belshazzar. During his reign Daniel was thrown into a lions' den, and only 
through faith did he emerge unscathed.

%\ww {} % BBE
%    {} % Jubilee2000
%    {} % NETfree
%    {} % UKJV
%    {} % RNKJV    
%    {} % Webster
\Note 6:1 {Darius} See <"note on" 5:31>n. While it is true that Darius the Mede 
is not referred to in extant historical sources outside the Scripture 
and that there was no interval between Belshazzar/Nabonidus (see 
<"note on" 5:1>N) and the accession of Cyrus of Persia, this does not necessarily mean that the book of
Daniel is in error. Most likely \"Darius  the Mede" was a throne name for Cyrus, the founder of the Persian 
Empire (see <"note on v." 28>N). It is also possible, but not as likely, that it 
was a designation for Gubaru, a general who defected from Nebuchadnezzar to Cyrus, led the Persian
conquest of Babylon and was  made governor by Cyrus over the territories the Persians had taken from the Babylonians.

\ww {special spirit} % BBE
    {over abundance}={over abundance of the Spirit} % Jubilee2000
    {extraordinary spirit} % NETfree
    {excellent spirit} % UKJV
    {excellent spirit} % RNKJV    
    {excellent spirit} % Webster
\Note 6:3 {his exceptional qualities} See <1:17>; <4:8> and <5:12>.

\ww {the law of his God} % BBE
    {the law of his God} % Jubilee2000
    {the law of his God} % NETfree
    {the law of his God} % UKJV
    {the law of his Elohim} % RNKJV    
    {the law of his God} % Webster
\Note 6:5 {the law of his God} Daniel's adversaries affirmed not only his 
moral integrity but also the visible nature of his piety and commitment to the God of Israel. Thus
the book's major theme of Daniel's  holiness and reliability is affirmed once again.
 
\ww {All}={All ... have made a common decision} % BBE
    {All}={All ... have agreed in common accord}  % Jubilee2000
    {To all}={To all ... it seemed like a good idea}  % NETfree
    {All}={All ... have consulted together}  % UKJV
    {All}={All ... have consulted together}  % RNKJV    
    {All}={All ... have consulted together}  % Webster
\Note 6:7 {have all agreed} The false implication was that Daniel had 
concurred with the proposal. These officials were hypocritical in 
their seeming devotion to Darius. Their scheme was an attempt to 
manipulate him into securing their own designs. 


\ww {whoever makes any request to any god or man but you}={whoever makes any request to any god ... but you} % BBE
    {whoever shall ask a petition of any God}={whoever shall ask a petition of any God ... except of thee}  % Jubilee2000
    {anyone who prays to any god}={anyone who prays to any god ... other than you}  % NETfree
    {whosoever shall ask a petition of any God}={whosoever shall ask a petition of any God ... save of you}  % UKJV
    {whosoever shall ask a petition of any Elohim}={whosoever shall ask a petition of any Elohim ... }  % RNKJV    
    {whoever shall ask a petition of any God}={whoever shall ask a petition of any God ... except of thee}  % Webster
\Note 6:7 {who prays to any god or man ... except to you} The proposal would have seemed 
to Darius to be more political than religious and would have served 
to consolidate his authority over newly conquered territories.

\ww {the law of the Medes and Persians} % BBE
    {the law of Media and of Persia} % Jubilee2000
    {the law of the Medes and Persians} % NETfree
    {the law of the Medes and Persians} % UKJV
    {the law of the Medes and Persians} % RNKJV    
    {the law of the Medes and Persians} % Webster
\Note 6:8 {the laws of the Medes and Persians, which cannot be repealed} See <Est 1:19> and <8:8>. The
irrevocable nature of Persian  law is also attested in extra-Biblical writings. The effect of the
decree was to create a conflict for Daniel between allegiance to the  Lord and obedience to human government.

\ww {opening in the direction of Jerusalem} % BBE
    {open toward Jerusalem} % Jubilee2000
    {opened toward Jerusalem} % NETfree
    {windows being open in his chamber toward Jerusalem} % UKJV
    {windows being open in his chamber toward Jerusalem} % RNKJV    
    {windows being open in his chamber towards Jerusalem} % Webster
\Note 6:10 {opened toward Jerusalem} See <1Ki 8:44> and <48>, as well as <Ps 5:7> and <138:2>.

\ww {three times a day} % BBE
    {three times a day} % Jubilee2000
    {Three times daily} % NETfree
    {three times a day} % UKJV
    {three times a day} % RNKJV    
    {three times a day} % Webster
\Note 6:10 {Three times a day} See <Ps 55:17-18>.

\ww {down on his knees} % BBE
    {knelt} % Jubilee2000
    {he was kneeling} % NETfree
    {kneeled upon his knees} % UKJV
    {kneeled upon his knees} % RNKJV    
    {kneeled upon his knees} % Webster
\Note 6:10 {down on his knees} Standing may have been a regular posture in 
prayer (<1Ch 23:30>; <"Ne 9"_Ne 9:1>). While kneeling in prostration marked a lowering of oneself, appropriate in
circumstances of particular  solemnity (<1Ki 8:54>; <Ezr 9:5>; see also <Ps 95:6>; <Lk 22:41>; <Ac 7:60>; 
<9:40>).

\ww {as he had done before} % BBE
    {as he was used to doing before} % Jubilee2000
    {as he had been accustomed to do previously} % NETfree
    {as he did in old times} % UKJV
    {as he did aforetime} % RNKJV    
    {as he did before} % Webster
\Note 6:10 {just as he had done before} Evidently Daniel's prayer habits were public knowledge, a mark of
his genuine piety. 

\ww {one of the prisoners of Judah} % BBE
    {of the sons of the captivity of the Jews} % Jubilee2000
    {one of the captives from Judah} % NETfree
    {of the children of the captivity of Judah} % UKJV
    {of the children of the captivity of Judah} % RNKJV    
    {of the children of the captivity of Judah} % Webster
\Note 6:13 {who is one of the exiles from Judah} This ethnic identification of Daniel is perhaps indicative
of prejudice toward the Jews on the part of the other officials (cf. <3:8>). That Daniel's ethnic
identity was widely known reveals that he had not compromised his  
heritage in favor of success in captivity—an important lesson to 
the readers.

\ww {his heart was fixed on keeping Daniel safe} % BBE
    {heart on Daniel to deliver him}={he set his heart on Daniel to deliver him} % Jubilee2000
    {began thinking about how he might rescue Daniel} % NETfree
    {set his heart on Daniel to deliver him} % UKJV
    {set his heart on Daniel to deliver him} % RNKJV    
    {heart on Daniel to deliver him}={set his heart on Daniel to deliver him} % Webster
\Note 6:14 {he was determined to rescue Daniel} Darius immediately 
perceived that he had been victimized by the intrigue of his own 
officials in order to trap Daniel. His appreciation for Daniel remained unshaken.

\ww {Your God}={Your God ... will keep you safe} % BBE
    {Thy God}={Thy God ... may he deliver thee} % Jubilee2000
    {Your God}={Your God ... will rescue you} % NETfree
    {Your God}={Your God ... will deliver you} % UKJV
    {Thy Elohim}={Thy Elohim ... will deliver thee} % RNKJV    
    {Thy God}={Thy God ... he will deliver thee} % Webster
\Note 6:16 {May your God ... rescue you!} Against his own will Darius 
was forced to comply with the decree. Nevertheless, he hoped 
that Daniel's God would intervene on behalf of his faithful servant. 

\ww {it was stamped with the king's stamp and with the stamp of the lords} % BBE
    {sealed it with his own signet}={sealed it with his own signet ring and with the signet ring of his princes} % Jubilee2000
    {sealed it with his signet ring and with those of his nobles} % NETfree
    {sealed it with his own signet, and with the signet of his lords} % UKJV
    {sealed it with his own signet, and with the signet of his masters} % RNKJV    
    {sealed it with his own signet, and with the signet of his lords} % Webster
\Note 6:17 {sealed it with his own signet ring and with the rings of 
his nobles} Signet rings and cylinder seals were commonly used by 
the Assyrians, Babylonians and Persians. The ring or cylinder was 
rolled across impressionable clay to leave the personal mark of the 
owner of the seal. Breaking open whatever was sealed in this way 
would be a violation of the law.

\ww {My God has sent his angel} % BBE
    {My God has sent his angel} % Jubilee2000
    {My God sent his angel} % NETfree
    {My God has sent his angel} % UKJV
    {My Elohim hath sent his angel} % RNKJV    
    {My God hath sent his angel} % Webster
\Note 6:22 {My God sent his angel} Likely the angel of the Lord (see <"note on" 3:28>n).

\ww {gave orders for them to take Daniel up out of the hole} % BBE
    {commanded that they should take Daniel up out of the den} % Jubilee2000
    {gave an order to haul Daniel up from the den} % NETfree
    {commanded that they should take Daniel up out of the den} % UKJV
    {commanded that they should take Daniel up out of the den} % RNKJV    
    {commanded that they should take Daniel out of the den} % Webster
\Note 6:23 {gave orders to lift Daniel out of the den} Darius could do 
this without violating the initial decree, since its demands had already been fulfilled.

\Note 6:26-27 {}  See <2:47>; <3:17-18> and <28-29>; <4:2-3> and <28-37> and <5:18-29>.
As in the previous narratives the
Lord revealed himself to be greater than human rulers or kingdoms, for his sovereignty extends over
nature and history. But this decree went far beyond those earlier confessions in acknowledging God
as a living, enduring and saving deity, whose kingdom is eternal and secure.

\ww {order} % BBE
    {statute} % Jubilee2000
    {edict} % NETfree
    {decree} % UKJV
    {decree} % RNKJV    
    {decree} % Webster
\Note 6:26 {decree} Darius's
decree does not imply that he actually converted from pagan polytheism to faith in Daniel's God
alone any more than did Cyrus's proclamation that God had instructed him to send the Jews home (<Ezr
1:3-4>; <Isa 44:28>; <45:4>).

\ww {Daniel did well} % BBE
    {Daniel was prospered} % Jubilee2000
    {Daniel prospered} % NETfree
    {Daniel prospered} % UKJV
    {Daniel prospered} % RNKJV    
    {Daniel prospered} % Webster
\Note 6:28 {Daniel prospered} The major theme of God's blessing toward Daniel
appears again. Daniel remained faithful, refusing to compromise. For this reason he rose in
prominence under both Babylonian and Persian kings. This fact exalted Daniel as a faithful Israelite
whose prophecies could be trusted. the reign of Darius and the reign of Cyrus.
%See NIV text note.
The wording may be understood in two ways: (1) Daniel prospered under the rule of Gubaru (see
<"note on v." 1>n) as well as under Cyrus; or (2) Daniel prospered under the reign of Darius, even in the
reign of Cyrus. In the latter case, Darius the Mede and Cyrus are understood to be two names for the
same ruler (see <"note on v." 1>n).

\Note 7:1-12:13 {} {\it The Visions.\/} In these chapters Daniel turned from
historical narrative to reports of visions. These visions depend on the two main themes set forth
in the first six chapters of the book: Israel's God was in control of all nations, and Daniel could
be trusted as God's uncompromising prophet. These chapters prepared an exiled Israel for the long
delay of the restoration and the trials to come under the control of foreign powers. They also
encouraged the people of God not to give up hope that God's kingdom would come at the end of these
trials. Daniel touched on four main topics: the four beasts (<7:1-28>), the ram and the goat (<8:1-27>),
the \"seventy weeks" (<9:1-27>) and the future of God's people (<10:1-12:13>).


\Note 7:1-28 {} {\it Vision of the Four Beasts.\/} Daniel reported his dream of four beasts. The dream traces
the history of foreign kingdoms oppressing Israel until their earthly dominion was given to the \"one
like a son of man" and to the saints.

\ww {the first year of Belshazzar} % BBE
    {the first year of Belshazzar} % Jubilee2000
    {first year of King Belshazzar} % NETfree
    {the first year of Belshazzar} % UKJV
    {the first year of Belshazzar} % RNKJV    
    {the first year of Belshazzar} % Webster
\Note 7:1 {the first year of Belshazzar} See <"note on" 5:1>n. It is not known whether
Belshazzar's coregency with Nabonidus began at the same time as the accession of Nabonidus (556
B.C.) or a few years later. In any case, the events of this chapter (and <"ch. 8"_8:1>) are lobe placed
chronologically between those of chapters 4 and 5.
%%%%%%%%%%%%%%???? lobe

%\ww {} % BBE
%    {} % Jubilee2000
%    {} % NETfree
%    {} % UKJV
%    {} % RNKJV    
%    {} % Webster
\Note 7:2 {great sea} Whether or not this is a reference
to the Mediterranean Sea is immaterial. What is clear is that the sea is symbolical the chaotic
restlessness that characterized the sinful nations oppressing Israel. See the interpretation given
in <"verse" 17> and in <Isa 17:12-13> and <57:20>.

\ww {four great beasts} % BBE
    {four great beasts} % Jubilee2000
    {four large beasts} % NETfree
    {four great beasts} % UKJV
    {four great beasts} % RNKJV    
    {four great beasts} % Webster
\Note 7:3 {Four great beasts} These four beasts represent
four kingdoms (<"vv." 17>, <23>). It is clear that there is a close correspondence between the four
kingdoms of Nebuchadnezzar's vision of the image in chapter 2 and those symbolized by the beasts in
this chapter. For identification of the four kingdoms, see chart, \"Visions in Daniel," at Da 2.

\Note 7:4 {a lion}={a lion ... eagle's wings} The lion with eagle's wings is an
appropriate symbol for the Babylonian Empire (cf. <Jer 50:44>; <Eze 17:3>). Winged lions were common
Babylonian art forms often placed at the entrances of important public buildings.

\ww {wings were pulled off}={wings were pulled off ... a man's heart was given to it} % BBE
    {wings were plucked off}={wings were plucked off ... a man’s heart was given to it} % Jubilee2000
    {wings were pulled off}={wings were pulled off ... a human mind was given to it} % NETfree
    {the wings thereof were plucked}={the wings thereof were plucked ...a man's heart was given to it} % UKJV
    {the wings thereof were plucked}={the wings thereof were plucked ... a man's heart was given to it} % RNKJV    
    {wings were plucked}={wings were plucked ... a man's heart was given to it} % Webster
\Note 7:4 {wings were torn off ... and the heart of man was given to it}
Perhaps this is a reference to Nebuchadnezzar's
humiliation and later restoration following a seven-year period of insanity (<4:1-37>).

\ww {another beast, like a bear} % BBE
    {the second beast}={the second beast, like unto a bear} % Jubilee2000
    {a second beast}={a second beast ... like a bear} % NETfree
    {another beast}={another beast ... like to a bear} % UKJV
    {another beast}={another beast ... like to a bear} % RNKJV    
    {another beast}={another beast ... like a bear} % Webster
\Note 7:5 {a second beast ... like a bear ... raised up on one of ib sides, and it had three ribs in
its mouth}
The Medo-Persian kingdom is symbolized by a beast with a voracious appetite. The raised side may
represent the superior status of Persia, and the three ribs likely point to Persia's conquests over
Lydia (546 B.C.), Babylon (539 B.C.) and Egypt (525 B.C.). See <"note on" 8:3>n.

\ww {another beast, like a leopard} % BBE
    {another, like a tiger} % Jubilee2000
    {another beast like a leopard} % NETfree
    {another, like a leopard} % UKJV
    {another, like a leopard} % RNKJV    
    {another, like a leopard} % Webster
\Note 7:6 {another beast ... like a leopard ... it had four wings... four heads}
The Greek Empire is symbolized by a leopard,
which is known for its speed. Alexander the Great (356-323 B.C.) conquered the Persian Empire with
great rapidity. He encountered the Persians in three major battles: (1) At the Granicus River (334
B.C.)he gained entry into Asia Minor. (2) At Issus (333 B.C.). he was enabled to occupy Syria,
Canaan and Egypt. (3) At Arbela (331 B.C.) he destroyed the last Persian army and pushed onward
toward India. See also <8:5-8>. Shortly after his premature death at age thirty-three the empire he had
established divided into four parts: Macedonia under Cassander, Thrace and Asia Minor under
Lysimachus, Syria under Seleucus and Egypt under Ptolemy.  


\ww {a fourth beast}={a fourth beast ... causing fear and very troubling, full of power and very strong} % BBE
    {the fourth beast}={the fourth beast ... dreadful and terrible, and exceedingly strong} % Jubilee2000
    {a fourth beast}={a fourth beast ... dreadful, terrible, and very strong} % NETfree
    {a fourth beast}={a fourth beast ... dreadful and terrible, and strong exceedingly} % UKJV
    {a fourth beast}={a fourth beast ... dreadful and terrible, and strong exceedingly} % RNKJV    
    {a fourth beast}={a fourth beast ... dreadful and terrible, and strong exceedingly} % Webster
\Note 7:7 {a fourth beast—terrifying and frightening and very powerful}
History has revealed that this
unidentified beast represents  Rome, the kingdom that ultimately assimilated the various parts of 
the divided Greek kingdom.

%\ww {} % BBE
%    {} % Jubilee2000
%    {} % NETfree
%    {} % UKJV
%    {} % RNKJV    
%    {} % Webster
\Note 7:7 {it had ten horns} The ten horns symbolize ten kings or kingdoms arising
from the Roman kingdom (<"v."  24>). It is not clear whether these horns are successive or
contemporaneous. Some suggest that they represent a second phase of  
thefourth kingdom, \"a revived Roman Empire" of the last days, but 
there is no evidence of such a distinction.

\ww {a little one, before which three of the first horns were pulled up by the roots} % BBE
    {another little horn, before whom three of the first horns were plucked up}={another little horn, before whom three of the first horns were plucked up by the roots} % Jubilee2000
    {a small one - came up between them, and three of the former horns were torn out by the roots} % NETfree
    {another little horn, before whom there were three of the first horns plucked up by the roots} % UKJV
    {another little horn, before whom there were three of the first horns plucked up by the roots} % RNKJV    
    {another little horn, before whom there were three of the first horns plucked up by the roots:} % Webster
\Note 7:8 {another horn, a little one ... three of the first horns were 
uprooted before it} The ten horns are prior in time to the \"little horn" which uproots three of them. Here is another phase of the 
fourth kingdom. Many interpreters have suggested that the little 
horn symbolizes the rise of the antichrist (<2Th 2:3-4>, <8>). If so, this is 
the first Scriptural reference to the antichrist.

\ww {eyes like a man's eyes in this horn, and a mouth saying great things} % BBE
    {eyes like the eyes of man, and a mouth speaking grand things} % Jubilee2000
    {eyes resembling human eyes and a mouth speaking arrogant things} % NETfree
    {eyes like the eyes of man, and a mouth speaking great things} % UKJV
    {eyes like the eyes of man, and a mouth speaking great thing} % RNKJV    
    {eyes like the eyes of a man, and a mouth speaking great things} % Webster
\Note 7:8 {eyes like the eyes of a man and a mouth that spoke boastfully} The imagery suggests 
that this horn represents an individual rather than a kingdom. 

\ww {a very old man took his seat} % BBE
    {an Elder of great age did sit} % Jubilee2000
    {the Ancient of Days took his seat} % NETfree
    {the Ancient of days did sit} % UKJV
    {the Ancient of days did sit} % RNKJV    
    {the Ancient of days did sit} % Webster
\Note 7:9 {the Ancient of Days took his seat} The title \"Ancient of Days" occurs in the Bible only in this chapter (<"vv." 13>, <22>). A similar 
expression appears in Ugaritic texts to designate the great God El. 
It is clearly used as a designation for God, who is sitting to judge, 
and it implies that God is eternal or that he has ruled from ancient 
times.

\ww {clothing}={clothing ...hair } % BBE
    {garment}={garment ...hair } % Jubilee2000
    {attire}={attire ...hair } % NETfree
    {garment}={garment ... hair} % UKJV
    {garment}={garment ... hair} % RNKJV    
    {garment}={garment ... hair} % Webster
\Note 7:9 {clothing ... hair} Although God appeared in magnificent 
glory to Daniel, he still revealed himself in a recognizably human 
form so that Daniel could grasp what he saw.

\ww {his sea}={his seat ... its wheels} % BBE
    {his throne}={his throne ... his wheels} % Jubilee2000
    {His throne}={His throne ... its wheels} % NETfree
    {his throne}={his throne ... his wheels} % UKJV
    {his throne}={his throne ... his wheels} % RNKJV    
    {his throne}={his throne ... his wheels} % Webster
\Note 7:9 {His throne ... and  its wheels} The depiction of God's throne resembles that of Ezekiel's
vision (<Eze 1:15-28>). As in  other parts of the ancient world the  
divine throne is depicted as having wheels, like a mobile chariot throne used most notably in
battle. Similar motifs lie behind the  pillar of fire that led Israel during the Exodus (<Ex 13:21-22>).

\ww {the books were open} % BBE
    {the books were opened} % Jubilee2000
    {the books were opened} % NETfree
    {the books were opened} % UKJV
    {the books were opened} % RNKJV    
    {the books were opened} % Webster
\Note 7:10 {the books were opened} See <12:1> (see also <Ex 32:32>; <Ps 149:9>; <Isa 4:3>; <65:6>;
<Mal 3:16>; <Lk 10:20>; <Rev 5:1-5>; <6:12-16>; <20:12>). See BC 37.

%\ww {} % BBE
%    {} % Jubilee2000
%    {} % NETfree
%%    {} % UKJV
%    {} % RNKJV    
%    {} % Webster
\Note 7:11-12 {}  A contrast is drawn between the complete
destruction of the fourth kingdom and the measure of continuance granted the preceding kingdoms as
their people and customs were absorbed into the succeeding kingdoms.

\ww {one like a man} % BBE
    {like a Son of man} % Jubilee2000
    {one like a son of man} % NETfree
    {one like the Son of man} % UKJV
    {one like the Son of man} % RNKJV    
    {the son of man}={one like the son of man} % Webster
\Note 7:13 {one like a son of man} The
term \"son of man" may mean simply \"a man." The Hebrew equivalent is used for Daniel in <8:17> and is
used many times of Daniel's contemporary Ezekiel (e.g., <Eze 2:1>, <3>, <6>).
In contrast to the beasts who
misruled the earth, this \"one" will preside over creation as God had intended before the fall; he
will have dominion over the beasts (<Ge 1:26-28>; <Ps 8>). Daniel may have been the earliest witness to
this special use of \"son of man." Later Jewish apocalyptic literature written between the Old and
New Testaments draws upon this passage and speaks of the \"son of man" as a supernatural human being
who brings the power of heaven to Earth. Daniel saw someone like a man; i.e., someone who was to
be compared with a man yet was somehow qualitatively different (<"v." 14>). The expression \"son of man"
is used 69 times in the Synoptic Gospels and 12 times in John's Gospel to refer to Christ. It is in
fact the most common title Jesus used of himself.

\ww {coming with the clouds of heaven} % BBE
    {in the clouds of heaven} % Jubilee2000
    { with the clouds of the sky} % NETfree
    {came with the clouds of heaven} % UKJV
    {with the clouds of heaven} % RNKJV    
    {with the clouds of heaven} % Webster
\Note 7:13 {coming with the clouds of heaven} Elsewhere in
the Old Testament only God is said to appear on clouds (<Ps 104:3>; <Isa 19:1>). The One like a man
originates in heaven and comes by divine initiative. He is the same as the rock cut out of the
mountain, but not by human hands (<2:45>; see <"note on" 7:14>n).

\ww {to him was given authority} % BBE
    {he gave him dominion} % Jubilee2000
    {To him was given ruling authority} % NETfree
    {there was given him dominion} % UKJV
    {there was given him dominion} % RNKJV    
    {there was given him dominion} % Webster
\Note 7:14 {He was given authority} God gives
him vice-regency over all the nations. He fulfills the symbolic rule of the rock cut out of a
mountain (<2:44-45>).


\ww {all peoples}={all peoples ... were his servants ... his kingdom is one which will not come to destruction} % BBE
    {all the peoples}={all the peoples ... served him ... his kingdom such that it shall never be corrupted} % Jubilee2000
    {All peoples}={All peoples ... were serving him ... His kingdom will not be destroyed} % NETfree
    {all people}={all people ... should serve him ... his kingdom that which shall not be destroyed} % UKJV
    {all people}={all people ... should serve him ... his kingdom that which shall not be destroyed} % RNKJV    
    {all people}={all people ... should serve him ... his kingdom that which shall not be destroyed} % Webster
\Note 7:14 {all peoples ... worshiped him ... his kingdom is one that will never be
destroyed} The \"son of man"  whom Daniel envisioned was none other than the great son of David, the
Messiah. Isaiah also spoke of his kingdom as never ending  (<Isa 9:7>).
Jesus clearly confirmed this Messianic connection by an 
allusion to this passage. For this he was accused by the religious 
leaders of his day of blasphemy (<Mt 26:64-65>; <Mk 14:62-64>). In 
serving him, people serve God.

\ww {spirit was pained}={spirit was pained ...were troubling} % BBE
    {spirit was troubled}={spirit was troubled ... astonished} % Jubilee2000
    {spirit was distressed}={spirit was distressed ...were alarming } % NETfree
    {was grieved in my spirit}={was grieved in my spirit ...troubled me } % UKJV
    {was grieved in my spirit}={was grieved in my spirit ... troubled me} % RNKJV    
    {was grieved in my spirit}={was grieved in my spirit ... troubled me} % Webster
\Note 7:15 {troubled in spirit ... was deeply troubled} Daniel was 
horrified by what he saw and asked an angel to elucidate the vision.

\ww {the saints of the Most High} % BBE
    {they} % Jubilee2000
    {The holy ones of the Most High} % NETfree
    {the saints of the most High} % UKJV
    {the saints of the most High} % RNKJV    
    {the saints of the most High} % Webster
\Note 7:18 {the saints of the Most High} See <"verses" 21-22>, <25> and <27>. 
Not angels but true believers who will share responsibility in the 
administration of the kingdom (<1Co 6:1-11>; <2Ti 2:12>; <Rev 22:5>).

\ww {will take the kingdom} % BBE
    {shall take the kingdom}={they shall take the kingdom} % Jubilee2000
    {will receive the kingdom} % NETfree
    {shall take the kingdom} % UKJV
    {shall take the kingdom} % RNKJV    
    {shall take the kingdom} % Webster
\Note 7:18 {will receive the kingdom} There is close identification between 
the \"son of man" as King (<"vv." 13-14>) and the \"saints of the Most 
High" as those who participate in his kingdom (see \"\x/took the kingdom/" at <"v." 22>; see also
<"v." 27>).

\ww {for ever} % BBE
    {until the age} % Jubilee2000
    {forever} % NETfree
    {for ever} % UKJV
    {for ever} % RNKJV    
    {for ever} % Webster
\Note 7:18 {forever} See <6:26>, <7:14>  and <"their"_6:26>n <"notes"_7:14>n.

\ww {that horn made war on the saints and overcame them} % BBE
    {this horn made war against the saints and overcame them} % Jubilee2000
    {that horn began to wage war against the holy ones and was defeating them} % NETfree
    {the same horn made war with the saints, and prevailed against them} % UKJV
    {the same horn made war with the saints, and prevailed against them} % RNKJV    
    {the same horn made war with the saints, and prevailed against them} % Webster
\Note 7:21 {this horn was waging war against the saints and defeating them} Daniel recounted additional
information about the hostility of the little horn (<"v." 8>) toward the people of God (cf. <Rev 13:7>).

\ww {Till he came, who was very old} % BBE
    {until such time as the Elder of great age came} % Jubilee2000
    {until the Ancient of Days arrived} % NETfree
    {Until the Ancient of days came} % UKJV
    {Until the Ancient of days came} % RNKJV    
    {Until the Ancient of days came} % Webster
\Note 7:22 {until the Ancient of Days came} Although the little horn 
(<"v." 8>) would prevail for a time against God's people, in the end he 
would fall under the judgment of God (cf. <Zec 14:1-4>; <Rev 13:7-17>; 
<19:20>).


\ww {took the kingdom} % BBE
    {possessed the Kingdom} % Jubilee2000
    {take possession of the kingdom} % NETfree
    {possessed the kingdom} % UKJV
    {possessed the kingdom} % RNKJV    
    {possessed the kingdom} % Webster
\Note 7:22 {possessed the kingdom} God's intervention in history will 
lead to what the New Testament calls \"the kingdom of God" (see 
theological article \"The Kingdom of God" at <Mt 4>).

%\ww {} % BBE
%    {} % Jubilee2000
%    {} % NETfree
%    {} % UKJV
%    {} % RNKJV    
%    {} % Webster
\Note 7:24 {three kings} A few of the ten, but an indefinite number.

\ww {he will say words against the Most High} % BBE
    {he shall speak}={he shall speak great words against the most High} % Jubilee2000
    {He will speak words against the Most High} % NETfree
    {he shall speak great words against the most High} % UKJV
    {he shall speak great words against the most High} % RNKJV    
    {words against the most High}={he shall speak great words against the most High} % Webster
\Note 7:25 {He will speak against the Most High} More details are given of the activities of the little
horn (<"v." 8>) as a ruler who opposes   God.


\ww {put an end to the saints} % BBE
    {break down the saints} % Jubilee2000
    {harass the holy ones} % NETfree
    {wear out the saints} % UKJV
    {wear out the saints} % RNKJV    
    {wear out the saints} % Webster
\Note 7:25 {oppress his saints} He will persecute God's people.

\ww {for a time and times and half a time} % BBE
    {time and times and the half}={until a time and times and the half or dividing of a time} % Jubilee2000
    {For a time, times, and half a time} % NETfree
    {until a time and times and the dividing of time} % UKJV
    {until a time and times and the dividing of time} % RNKJV    
    {until a time and times and the dividing of time} % Webster
\Note 7:25 {for a  time, times and half a time} The word \"time" is the same word
used in <4:16> and <4:23> and, as there (see <"note on" 4:16>n), may be understood as representing a period
of one year (cf. <Rev 12:14>). It is  
best understood as symbolic of a period of time that will be shortened when God suddenly
intervenes.

\ww {the judge} % BBE
    {the Judge} % Jubilee2000
    {the court} % NETfree
    {the judgment} % UKJV
    {the judgment} % RNKJV    
    {the judgment} % Webster
\Note 7:26 {the court} The court of heaven (see <"v." 10>).

\ww {will be given to the people of the saints} % BBE
    {be given to the holy people} % Jubilee2000
    {will be delivered to the people of the holy ones} % NETfree
    {shall be given to the people of the saints} % UKJV
    {shall be given to the people of the saints} % RNKJV    
    {shall be given to the people of the saints} % Webster
\Note 7:27 {will be handed over to the saints} After God's people face 
the trials of oppressive kingdoms they will rule over all forever. See 
<"note on" 7:18>n. 

\ww {greatly troubled by my thoughts, and the colour went from my face} % BBE
    {very troubled in my thoughts, and my countenance changed in me} % Jubilee2000
    {troubled me greatly, and the color drained from my face} % NETfree
    {much troubled me, and my countenance changed in me} % UKJV
    {much troubled me, and my countenance changed in me} % RNKJV    
    {much troubled me, and my countenance changed in me} % Webster
\Note 7:28 {deeply troubled ... turned pale} Thoughts of Israel falling 
under repeated and prolonged oppression from foreign powers still 
troubled Daniel, even though the ultimate outcome would be divine intervention resulting in victory
for God's people. See also <"v." 15> and its <"note"_7:15>n.

\ww {kept the thing in my heart} % BBE
    {kept the word in my heart} % Jubilee2000
    {kept the matter to myself} % NETfree
    {kept the matter in my heart} % UKJV
    {kept the matter in my heart} % RNKJV    
    {kept the matter in my heart} % Webster
\Note 7:28 {kept the  matter to myself}
Daniel mentioned this to inform his readers 
that he did not delight in the prospect of such a future for God's 
people. Despite his authority in the Gentile courts of Babylon and 
Persia, no one could rightly accuse him of betraying his loyalty to 
God's people. He spoke of these future events with regret.

\Note 8:1-12:13 {}  Daniel resumed use of the Hebrew language in the 
book's last five chapters. He had written <2:4-7:28> in Aramaic (see 
<"note on" 2:4>n).

\Note 8:1-27 {} {\it Vision of the Ram and the Goat.}\/ The prophet recorded a 
vision concerning the treatment of God's people under the Medo-Persians and Greeks.

\ww {In the third year of the rule of Belshazzar the king} % BBE
    {In the third year of the reign of king Belshazzar} % Jubilee2000
    {In the third year of King Belshazzar's reign} % NETfree
    {In the third year of the reign of king Belshazzar} % UKJV
    {In the third year of the reign of king Belshazzar} % RNKJV    
    {In the third year of the reign of king Belshazzar} % Webster
\Note 8:1 {In the third year of King Belshazzar's reign} That is, two 
years after Daniel's dream in chapter 7 (see <"note on" 7:1>n).

\ww {when I saw it, I was} % BBE
    {I saw it, that I}={I saw it, that I was} % Jubilee2000
    {I saw myself} % NETfree
    {I saw, that I was} % UKJV
    {I saw, that I was} % RNKJV    
    {I saw, that I}={I saw, that I was} % Webster
\Note 8:2 {I saw myself} Daniel experienced a visionary journey like that 
of Ezekiel (<Eze 3:10-15>).

\ww {the strong town Shushan, which is in the country of Elam} % BBE
    {Shushan}={Shushan, which is the head of the kingdom in the province of Persia} % Jubilee2000
    {Susa the citadel, which is located in the province of Elam} % NETfree
    {Shushan in the palace, which is in the province of Elam} % UKJV
    {Shushan in the palace, which is in the province of Elam} % RNKJV    
    {in the province of Elam}={Shushan in the palace, which is in the province of Elam} % Webster
\Note 8:2 {the citadel of Susa in the province of Elam}
In Daniel's time \x/Shushan/ was the capital of Elam, about 230 
miles east of Babylon. It is unclear whether Elam was then independent or aligned with either
Babylon or Media. Later, however,  as one of three royal cities, \x/Shushan/ became the diplomatic and
administrative capital of the Persian Empire (cf. <Est 1:2>; <Ne 1:1>).

\ww {water-door of the Ulai} % BBE
    {river of Ulai} % Jubilee2000
    {Ulai Canal} % NETfree
    {river of Ulai} % UKJV
    {river of Ulai} % RNKJV    
    {river Ulai} % Webster
\Note 8:2 {Ulai  Canal} This canal near \x/Shushan/ connected two rivers that flowed into 
the Persian Gulf.


\ww {a male sheep with two horns} % BBE
    {a ram was standing before the river, which had two horns} % Jubilee2000
    {a ram with two horns} % NETfree
    {a ram which had two horns} % UKJV
    {a ram which had two horns} % RNKJV    
    {a ram which had}={a ram which had two horns} % Webster
\Note 8:3 {a ram with two horns} Verse 20 identifies the ram and its horns as a symbol for the kings of the
Medo-Persian Empire. One of the horns was longer than the other but grew up later. Medo-Persian
history clarifies the symbolism here. The Medes became strong and independent of Assyria after 631
B.C. The Persians began as an insignificant segment of the Median kingdom but eventually rose to
control it when Cyrus (reigned 559-530 B.C.) of Anshan (in Elam) brought Media under his control
(550 B.C.). Cyrus  added to his list of titles \"King of the Medes." Thus both horns 
were long but the one representing Persia longer because it was 
superior in might, and later in growing because it came to power 
after the other.

\ww {pushing to the west and to the north and to the south} % BBE
    {smote with the horns to the west, to the north, and to the south} % Jubilee2000
    {butting westward, northward, and southward} % NETfree
    {pushing westward, and northward, and southward} % UKJV
    {pushing westward, and northward, and southward} % RNKJV    
    {pushing westward, and northward, and southward} % Webster
\Note 8:4 {he charged toward the west and the north and the south}
Cyrus initially took Asia Minor; afterward, both northern and 
southern Mesopotamia. Subsequent rulers extended Medo-Persian control far to the East.

\ww {made himself great} % BBE
    {made himself great} % Jubilee2000
    {acted arrogantly} % NETfree
    {became great} % UKJV
    {became great} % RNKJV    
    {became great} % Webster
\Note 8:4 {became great} The Persian Empire became larger and more powerful than any previous empire
in ancient Near Eastern history.

\ww {the he-goat had a great horn between his eyes} % BBE
    {the goat}={the goat had a notable horn between his eyes} % Jubilee2000
    {This goat had a conspicuous horn between its eyes} % NETfree
    {the goat had a notable horn between his eyes} % UKJV
    {the goat had a notable horn between his eyes} % RNKJV    
    {a notable horn between his eyes}={the goat had a notable horn between his eyes} % Webster
\Note 8:5 {a goat with a prominent horn between his eyes came from the west}
<"Verse" 21> identifies the goat as Greece and the 
large horn between his eyes as its first king. The symbolism is a 
clear depiction of the rise of the Greek Empire under the leadership of Alexander the Great (356-323 B.C.).

\ww {over the face of all the earth without touching the earth} % BBE
    {upon the face of the whole earth and did not touch the earth} % Jubilee2000
    {over the surface of all the land without touching the ground} % NETfree
    {on the face of the whole earth, and touched not the ground} % UKJV
    {on the face of the whole earth, and touched not the ground} % RNKJV    
    {on the face of the whole earth, and touched not the ground} % Webster
\Note 8:5 {crossing the whole  earth without touching the ground} This depicts the amazing 
rapidity of Alexander's conquests (see <"note on" 7:6>n). In only three 
years he was able to defeat the powerful Persian Empire.

\ww {the he-goat became very great} % BBE
    {the he goat made himself very great} % Jubilee2000
    {The male goat acted even more arrogantly} % NETfree
    {the he goat waxed very great} % UKJV
    {the he goat waxed very great} % RNKJV    
    {the he-goat became very great} % Webster
\Note 8:8 {The goat became very great} Alexander's empire quickly 
exceeded the Persian Empire in size. By 327 B.C. Alexander had 
moved eastward into what is today Afghanistan and then on to the 
Indus Valley.


\ww {when he was strong, the great horn was broken} % BBE
    {when he was at his greatest strength, that great horn was broken} % Jubilee2000
    {no sooner had the large horn become strong than it was broken} % NETfree
    {when he was strong, the great horn was broken} % UKJV
    {when he was strong, the great horn was broken} % RNKJV    
    {when he was strong, the great horn was broken} % Webster
\Note 8:8 {but at the height of his power his large horn was broken off}
When his own troops refused to advance farther eastward Alexander returned to Babylon,
where he died at the age of thirty-two, most probably of typhoid fever.

\ww {in its place came up four other horns} % BBE
    {in its place came up another four marvellous ones} % Jubilee2000
    {there arose four conspicuous horns in its place} % NETfree
    {for it came up four notable ones} % UKJV
    {for it came up four notable ones} % RNKJV    
    {in its stead came up four notable ones} % Webster
\Note 8:8 {in its place four prominent horns grew up} <"Verse" 22>
indicates that these horns symbolize four kingdoms that 
emerged from Alexander's empire but were inferior in strength to 
its original domain. Historical records indicate that after a time of 
internal struggle four of Alexander's generals were able to secure 
portions of the former Greek Empire as their own kingdoms. See 
<"note on" 7:6>n.

\ww {another horn, a little one} % BBE
    {a little horn} % Jubilee2000
    {a small horn} % NETfree
    {a little horn} % UKJV
    {a little horn} % RNKJV    
    {a little horn} % Webster
\Note 8:9 {another horn, which started small} <"Verse" 23> indicates that 
this horn symbolizes a wicked ruler who would arise in one of the 
four Greek kingdoms after an extended interval of time ( \"in the latter part of their reign"). The
descriptions of the actions of this ruler (<"vv." 9-14>, <23-25>) identify 
him as Antiochus IV Epiphanes, the ruler of the Seleucid kingdom from 175 to 164 B.C. This horn is
not to be identified with the \"little horn" of <7:8>, which would arise during the Roman rather than
the Greek period.

\ww {to the beautiful land} % BBE
    {toward the desirable}={toward the desirable land} % Jubilee2000
    {toward the beautiful land} % NETfree
    {toward the pleasant land} % UKJV
    {toward the pleasant land} % RNKJV    
    {towards the pleasant}={toward the pleasant land} % Webster
\Note 8:9 {toward the Beautiful Land} Daniel showed his love for the promised land by this
expression.

\ww {the army of heaven} % BBE
    {the host of heaven} % Jubilee2000
    {the army of heaven} % NETfree
    {the host of heaven} % UKJV
    {the host of heaven} % RNKJV    
    {the host of heaven} % Webster
\Note 8:10 {the host of the heavens} Or the stars (cf. <Jer 33:22>), symbolizing the people of
God (cf. <12:3>; <Ge 12:3>; <15:5>; <Ex 12:41>) and/or a  
heavenly army (<Isa 14:13>; also see 2 Maccabees 9:10). Antiochus's coins picture a star above his head. Epiphanes 
means \"God manifest." The attack against the people of God amounted to an attack against heaven itself.

\ww {pulling down some of the army, even of the stars, to the earth and crushing them under its feet} % BBE
    {cast down}={cast down part of the host and of the stars to the ground and trod them under} % Jubilee2000
    {brought about the fall of some of the army and some of the stars to the ground, where it trampled them} % NETfree
    {cast down some of the host and of the stars to the ground, and stamped upon them} % UKJV
    {cast down some of the host and of the stars to the ground, and stamped upon them} % RNKJV    
    {of the host and of the stars to the ground, and stamped upon them}={cast down some of the host and of the stars to the ground, and stamped upon them} % Webster
\Note 8:10 {threw some of the starry host down to the earth and trampled on them}
This is a  symbolic depiction of the severe persecution of God's people under 
Antiochus IV Epiphanes, who attempted to abolish Israel's traditional worship and way of life (see
 \"Introduction: Purpose and Distinctives": cf. <11:21-35>; 1 Maccabees 1:10-64). 



\ww {as great as the lord of the army} % BBE
    {the prince of the host did he magnify himself}={against the prince of the host did he magnify himself} % Jubilee2000
    {acted arrogantly against the Prince of the army} % NETfree
    {magnified himself even to the prince of the host} % UKJV
    {magnified himself even to the prince of the host} % RNKJV    
    {even to the prince of the host}={magnified himself even to the prince of the host} % Webster
\Note 8:11 {as great as the Prince of the host} The \"Prince" is to be understood as God. the Lord of hosts. See <"verse" 25>, where the designation is \"Prince of
princes." Antiochus IV took the name Epiphanes ( \"God manifest") and viewed himself as the incarnate manifestation of Zeus (the chief god of the Greek pantheon).

\ww {the regular burned offering was taken away} % BBE
    {the daily}={the daily sacrifice was taken away} % Jubilee2000
    {the daily sacrifice was removed} % NETfree
    {the daily sacrifice was taken away} % UKJV
    {the daily sacrifice was taken away} % RNKJV    
    {the daily}={the daily sacrifice was taken away} % Webster
\Note 8:11 {took away the daily sacrifice from him} See <"verses" 12-13> and <11:31>. 
Antiochus IV ordered the cessation of all ceremonial observances related to the worship of the Lord at the Jerusalem temple and in 
the cities of Judah. the place of his sanctuary was brought low. Antiochus IV not only entered the Most Holy Place and plundered 
the silver and gold vessels, but he also erected an altar to Zeus on top of the altar of the Lord in the temple court and offered swine 
upon it (see <"note on" 11:31>n).

\ww {against} % BBE
    {the host was given over by reason of the prevarication upon the daily}={the host was given over by reason of the prevarication upon the daily sacrifice} % Jubilee2000
    {The army was given over, along with the daily sacrifice} % NETfree
    {an host was given him against the daily sacrifice} % UKJV
    {an host was given him against the daily sacrifice} % RNKJV    
    {against the daily}={an host was given him against the daily sacrifice} % Webster
\Note 8:12 {the host of the saints and the daily sacrifice were given over to it} God's people were subjected to the power of the horn 
that started small (<"v." 9>), Antiochus IV. This entailed the cessation of regular temple observances.

\ww {things went well for it} % BBE
    {prospered} % Jubilee2000
    {enjoyed success} % NETfree
    {prospered} % UKJV
    {prospered} % RNKJV    
    {prospered} % Webster
\Note 8:12 {It prospered in everything it did}
The vision depicts the apparent success of the wicked acts of Antiochus IV (the horn that started
small). That success included the destruction of copies of the Hebrew Scripture (cf. 1 Maccabees 
1:56-57).

\ww {For two thousand, three hundred evenings and mornings} % BBE
    {Unto two thousand and three hundred}={Unto two thousand and three hundred days of evening and morning} % Jubilee2000
    {To 2,300 evenings and mornings} % NETfree
    {Unto two thousand and three hundred days} % UKJV
    {Unto two thousand and three hundred days} % RNKJV    
    {Until two thousand and three hundred days} % Webster
\Note 8:14 {It will take 2,300 evenings and mornings} The phrase 
 \"evenings and mornings" occurs in the Old Testament only here 
and in <"verse" 26>. Some understand it as a reference to the evening 
and morning sacrifices (cf. <Ex 29:38-42>). On that basis it would represent 1,150 days. Others view
it as simply an expression for 2,300  days. Since the beginning of the persecutions of Antiochus IV
could be linked with any one of a number of incidents beginning early as 171 B.C., it is difficult to determine which understanding of  
the phrase is to be preferred. The number 23 may be symbolic of a fixed period, as in apocalyptic literature outside the Bible.

\ww {the holy place will be made clean} % BBE
    {then shall the sanctuary be justified} % Jubilee2000
    {hen the sanctuary will be put right again} % NETfree
    {then shall the sanctuary be cleansed} % UKJV
    {then shall the sanctuary be cleansed} % RNKJV    
    {then shall the sanctuary be cleansed} % Webster
\Note 8:14 {the sanctuary will be reconsecrated} The temple was cleansed 
and rededicated under the leadership of Judas Maccabeus on December 25, 165 B.C. (see <"note on" 11:34>n;
cf. <Zec 9:13-17>). 

\Note 8:16 {Gabriel} This angel is mentioned four times in Scripture 
(<9:21>; <Lk 1:11>, <19>, <26>). The name denotes one who is strong in the 
Lord (Gabriel means \"strength of God") because of a relationship 
with him.

\Note 8:17 {son of man} See <"note on" 7:13>n. The \"strong man of God" (see \<"note on v." 16>n), the angel Gabriel, was speaking to this exalted mortal.


\ww {the vision has to do with the time of the end} % BBE
    {at the time appointed}={at the time appointed by God the vision shall be fulfilled} % Jubilee2000
    {the vision pertains to the time of the end} % NETfree
    {at the time of the end shall be the vision} % UKJV
    {at the time of the end shall be the vision} % RNKJV    
    {at the time of the end}={at the time of the end shall be the vision} % Webster
\Note 8:17 {the vision concerns the time of the end} See also <"verse" 19> 
( \"the appointed time of the end"). This expression does not necessarily have to do with the absolute
end of history. It occurs in <11:27>  and <35> in contexts that probably refer to the end of the persecutions under Antiochus IV.

\ww {what is to come in the later time of the wrath} % BBE
    {is to come in the last end of the wrath}={that which is to come in the last end of the wrath} % Jubilee2000
    {what will happen in the latter time of wrath} % NETfree
    {what shall be in the last end of the indignation} % UKJV
    {what shall be in the last end of the indignation} % RNKJV    
    {what shall be in the last end of the indignation} % Webster
\Note 8:19 {what will happen later in the time of wrath} The \"timed wrath" may here refer to the time of God's judgment on his people 
Israel during the period of their subjection to the Babylonians, Persians and Greeks.

\ww {The sheep} % BBE
    {The ram} % Jubilee2000
    {The ram} % NETfree
    {The ram} % UKJV
    {The ram} % RNKJV    
    {The ram} % Webster
\Note 8:20 {ram} See <"notes on verses" 3>n--<4>n.
%dopkumentace, musíte rozumět TeXu


\ww {he-goat}={he-goat ... horn} % BBE
    {he goat}={he goat ...horn } % Jubilee2000
    {male goat}={male goat ...  horn} % NETfree
    {rough goat}={rough goat ...  horn} % UKJV
    {rough goat}={rough goat ...  horn} % RNKJV    
    {rough goat}={rough goat ...  horn} % Webster
\Note 8:21 {goat ... horn} See <"notes on verses" 5>n and <8>n.

%\ww {xfour} % BBE
%    {the law of the Medes and Persians} % Jubilee2000
%    {the law of the Medes and Persians} % NETfree
%    {the law of the Medes and Persians} % UKJV
%    {the law of the Medes and Persians} % RNKJV    
%    {the law of the Medes and Persians} % Webster
\Note 8:22 {four} See <"note on verse" 8>n.

%\ww {} % BBE
%    {} % Jubilee2000
%    {} % NETfree
%    {} % UKJV
%    {} % RNKJV    
%    {} % Webster
\Note 8:23-25 {}  See <"notes on verses" 9-14>n. Some interpreters have found a picture of the antichrist in the descriptions of the horn of this 
chapter (<"v." 8>) by viewing Antiochus IV as a type of any powerful opponent of God's people in the future.

\ww {numbers} % BBE
    {many} % Jubilee2000
    {many} % NETfree
    {many} % UKJV
    {many} % RNKJV    
    {many} % Webster
\Note 8:25 {mighty ... many} The faithful Jews, as well as \"the mighty" or \"the strong ones" of the <"verse" 24>.

\ww {prince of princes} % BBE
    {Prince of princes} % Jubilee2000
    {Prince of princes} % NETfree
    {Prince of princes} % UKJV
    {Prince of princes} % RNKJV    
    {Prince of princes} % Webster
\Note 8:25 {Prince of princes} A reference to God. 


\ww {he will be broken, though not by men's hands} % BBE
    {without hand he shall be broken} % Jubilee2000
    {he will be broken apart - but not by human agency} % NETfree
    {he shall be broken without hand} % UKJV
    {he shall be broken without hand} % RNKJV    
    {he shall be broken without hand} % Webster
\Note 8:25 {he will be destroyed, but not by human power} Antiochus IV was not assassinated, nor did he die in battle. 
His death in 164 B.C. resulted from a physical or nervous disorder. For variant accounts of his death see 1 Maccabees 6:1-16 and 2 Maccabees 9:1-28.


\ww {keep the vision secret} % BBE
    {shut thou up the vision} % Jubilee2000
    {seal up the vision} % NETfree
    {shut you up the vision} % UKJV
    {shut thou up the vision} % RNKJV    
    {shut thou up the vision} % Webster
\Note 8:26 {seal up the vision} A \"seal" was used either to authenticate 
or certify something or to close up or secure something for confidentiality or safekeeping. The
second sense seems most fitting in  this context (see <"note on" 6:17>n).


\ww {for it has to do with the far-off future} % BBE
    {for many days}={for it shall be for many days} % Jubilee2000
    {for it refers to a time many days from now} % NETfree
    {for it shall be for many days} % UKJV
    {for it shall be for many days} % RNKJV    
    {for many days}={for it shall be for many days} % Webster
\Note 8:26 {for it concerns the distant future} Literally, \"[the vision]
pertains to many days." The conquests  of Alexander (333-323 B.C.) occurred nearly two centuries after 
Daniel's vision (c. 550 B.C.), while Antiochus IV was active about a century and a half after Alexander (171-164 B.C.).


%%


%\ww {} % BBE
%    {} % Jubilee2000
%    {} % NETfree
%    {} % UKJV
%    {} % RNKJV    
%    {} % Webster
\Note 9:1-27 {} {\it Vision of the Seventy Weeks.}\/ Daniel recorded an account of a revelation he received concerning Jeremiah's prophecy about 
the 70 years of Jerusalem's desolation. The vision followed Daniel's prayer in which he confessed the justice of Jerusalem's desolation 
and sought the favor of God for the restoration of the city and the temple. 
This vision revealed that the time of Judah's exile was extended because the people of God had not yet repented of the sins  that had brought exile upon them.

\ww {the first year of Darius, the son of Ahasuerus} % BBE
    {the first year of Darius the son of Ahasuerus} % Jubilee2000
    {In the first year of Darius son of Ahasuerus} % NETfree
    {In the first year of Darius the son of Ahasuerus} % UKJV
    {In the first year of Darius the son of Ahasuerus} % RNKJV    
    {In the first year of Darius the son of Ahasuerus} % Webster
\Note 9:1 {the first year of Darius son of Xerxes.} See <"notes on" 5:30-31>n 
and <6:1>n. The term \"Xerxes" (not the same person mentioned in <Est 1:1>) may be a royal title rather than a personal name. The first year 
of Darius's reign was 539 B.C.

\ww {saw clearly from the books}={saw clearly from the books ... the word of the Lord to the prophet Jeremiah ... waste of Jerusalem was to be complete, that is, seventy years} % BBE
    {saw diligently in the books}={saw diligently in the books ... the LORD spoke unto Jeremiah the prophet ... the desolation of Jerusalem in seventy years} % Jubilee2000
    {understand from the sacred books}={understand from the sacred books ... the word of the LORD disclosed to the prophet Jeremiah, the years for the fulfilling of the desolation of Jerusalem were seventy in number}  % NETfree
    {understood by books}={understood by books ... the word of the LORD came to Jeremiah the prophet ... accomplish seventy years in the desolations of Jerusalem}  % UKJV
    {understood by books}={understood by books ... came to Jeremiah the prophet  ... accomplish seventy years in the desolations of Jerusalem}  % RNKJV    
    {understood by books}={understood by books ... came to Jeremiah the prophet ... accomplish seventy years in the desolations of Jerusalem}  % Webster
\Note 9:2 {understood from the Scriptures ... given to Jeremiah ... that the desolation of Jerusalem would last seventy years} See <Jer 25:11-12> and <29:10>. Daniel was concerned because the 70 years of exile had nearly come to an end but the Israelites were not ready to return to the land. Interpreters differ on the dates of the beginning and ending of the 70-year period and on whether it is to be understood as a round number, suggesting a human lifetime,
or an exact time period. Some date the period from 586 B.C. (the destruction of Jerusalem by Nebuchadnezzar) to 515 B.C., when the  restoration of the temple was completed under Zerubbabel (<Ezr 6:13-18>; <Zec 4:9>). %Not 1:12
Others date the beginning of the period to the  year of Daniel's own captivity (604 B.C.: see <"note on" 1:1>n). Daniel was 
also undoubtedly aware that Isaiah had prophesied Israel's release 
from exile under the Persian ruler Cyrus (<Isa 44:28>; <45:1-13>). As Daniel apparently did here, the writer of Chronicles cited Cyrus's release of the exiles as having taken place in 539 B.C. as the fulfillment  of Jeremiah's prophecy (<2Ch 36:21>). In the literature of the ancient Near East 70 years was a standard time period during which a god would punish his people for disloyalty. This period could be lengthened or shortened by the reactions of the people. For this reason it is not surprising that there would be some flexibility in the ways different Biblical writers applied the number to Israel's history.

%\ww {} % BBE
%    {} % Jubilee2000
%    {} % NETfree
%    {} % UKJV
%    {} % RNKJV    
%    {} % Webster
\Note 9:4-19 {}  Daniel's prayer is rooted in a covenantal understanding of the Lord's relation to his people (blessing for obedience and cursing for disobedience; see especially <"vv." 5>, <7>, <11-12>, <14>; <Lev 26:14-45>;  <Dt 28:15-68>; <30:1-5>). For a similar prayer see <Ne 9>. The prayer contains four parts: (1) worship (<"v." 4>); (2) a confession of sin  (<"vv." 5-11a>); (3) recognition of the justice of God in his judgment on sin 
(<"vv." 11b-14>); %je to možné?
and (4) a plea for God's mercy based on concern for his name, kingdom and will (<"vv." 15-19>). The prayer is grounded  
in God's promises (<"v." 2>), was voiced in a spirit of contrition and humility (<"v." 3>) and provides a model for appropriate elements of effective prayer. 

\ww {Gabriel, whom I had seen in the vision at first} % BBE
    {Gabriel, whom I had seen in the vision at the beginning} % Jubilee2000
    {the man Gabriel, whom I had seen previously in a vision} % NETfree
    {the man Gabriel, whom I had seen in the vision at the beginning} % UKJV
    {the man Gabriel, whom I had seen in the vision at the beginning} % RNKJV    
    {the man Gabriel, whom I had seen in the vision at the beginning} % Webster
\Note 9:21 {Gabriel, the man I had seen in the earlier vision} See <"note on" 8:16>n.

\Note 9:24 {Seventy weeks} The "seventy 'sevens'" (lit., \"seventy 
weeks") represent 490 years (see <"note on" 9:24-27>n). The 70 years of 
exile (v. 2) are multiplied seven times in accordance with the pattern of covenantal curses
(<Lev 26:14>, <21>, <24>, <28>). God extended the  exile because of Israel's continuing sinfulness. Just as the 70 years 
of exile predicted by Jeremiah may have followed a standard formula (see <"note on v." 2>n), the period 
of 490 years probably represented a standard formula as well. For instance, the intertestamental,
non-canonical book Jubilees structures the whole of history  
into periods of 490 years. It is likely, therefore, that Daniel had in 
mind not a precise calculation of years but broadly defined segments of time. This extension of time
was not absolute; it could be  lengthened if the people continued to rebel or shortened if they 
repented.

\ww {have been fixed}={have been fixed ... to} % BBE
    {are determined}={are determined ... to}  % Jubilee2000
    {have been determined}={have been determined ... to}  % NETfree
    {are determined}={are determined ... to} % UKJV
    {are determined}={are determined ... to} % RNKJV    
    {are determined}={are determined ... to} % Webster
\Note 9:24 {are decreed ... to} Six things were to be
accomplished during the period of \"seventy `sevens.'\thinspace"
As with all Old Testament prophecies about the restoration from exile in the latter days, these six items 
are fulfilled in the work of Christ in bringing the kingdom of God (see theological articles \"The Kingdom of God" at <Mt 4> and \"The Plan of the Ages" at Heb 7). The New Testament teaches that the kingdom was inaugurated in the first coming of Christ, continues now and will reach its consummation at Christ's return. Therefore, some aspects of these predictions are more closely related to 
Christ's first coming, others to his second coming and still others are fulfilled by both his first and second comings. 

%\ww {} % BBE
%    {} % Jubilee2000
%    {} % NETfree
%    {} % UKJV
%    {} % RNKJV    
%    {} % Webster
\Note 9:25-27 {}  The \"seventy weeks" of years are divided into three subunits of 49 years (seven \"sevens";
<"v." 25>), 434 years (sixty-two  \"sevens"; <"v." 26>) and seven years (one \"seven"; <"v." 27>).
Interpreters  differ over whether these subunits are to be viewed as a continuous sequence or as subunits
separated by time intervals. Many attempts have been made to understand this chronology as precise numbers of years, but all attempts fall short of completeness due to the fact that these numbers were intended as round figures of representative periods of time. Although Daniel's calculations are not to be taken as precise. the basic pattern of his prediction may be discerned without falling into speculation. The order to rebuild Jerusalem (<"v." 25>) was followed by seven \"'sevens" or 49 years (<"v." 25>), at which time the rebuilding of Jerusalem was completed (see Ezra and Nehemiah). This was followed by sixty-two \"sevens" or 434 years 
(<"v."  25>), at which time the Messiah was cut off (<"v." 26>: see <"note"_26>n). The single \"seven" was fulfilled during or near the time of 
Christ's earthly ministry (<"v." 27>).

\ww {a prince, on whom the holy oil has been put} % BBE
    {Anointed} % Jubilee2000
    {anointed one, a prince} % NETfree
    {Messiah the Prince} % UKJV
    {Messiah the Prince} % RNKJV    
    {Messiah the Prince} % Webster
\Note 9:25 {Anointed One, the ruler} Two interpretations of this figure are possible: (1) He is the Messiah, the Christ. (2) He is a king 
whom God has anointed as his instrument in accomplishing his will (cf. <Isa 45:1>). While most interpreters take the anointed one and 
the ruler in verse <25> to be the same person, there is some disagreement as to whether or not this
figure is identical to the person or persons referred to as \"anointed one" and \"ruler" in verse 
<26>. In verse <26> the ruler appears to act against God. If the same ruler is intended in both verses, he is most likely not to be equated 
with the Messiah.

\ww {one on whom the holy oil has been put will be cut off} % BBE
    {the Anointed}={the Anointed One shall be killed} % Jubilee2000
    {an anointed one will be cut off} % NETfree
    {shall Messiah be cut off} % UKJV
    {shall Messiah be cut off} % RNKJV    
    {shall Messiah be cut off} % Webster
\Note 9:26 {the Anointed One will be cut off} This is either a reference to the crucifixion of Christ or to judgment that God would bring against a king who had overstepped his bounds as God's instrument of judgment (see <"note on v." 25>n).

\ww {the town and the holy place will be made waste together with a prince} % BBE
    {the ruling people that shall come shall destroy the city and the sanctuary} % Jubilee2000
    {As for the city and the sanctuary, the people of the coming prince will destroy them} % NETfree
    {the people of the prince that shall come shall destroy the city and the sanctuary} % UKJV
    {the people of the prince that shall come shall destroy the city and the sanctuary} % RNKJV    
    {the people of the prince that shall come shall destroy the city and the sanctuary} % Webster
\Note 9:26 {The people of the ruler  who will come will destroy the city and the sanctuary} A reference either to the Greek Antiochus IV Epiphanes as a precursor  
to the Roman general Titus (see \"Introduction: Purpose and Distinctives") or directly to Titus and/or his armies, who destroyed  Jerusalem in A.D. 70. 

\ww {a strong order will be sent out against the great number for one week} % BBE
    {In one week (they are now seventy) he shall confirm the covenant by many} % Jubilee2000
    {He will confirm a covenant with many for one week} % NETfree
    {he shall confirm the covenant with many for one week} % UKJV
    {he shall confirm the covenant with many for one week} % RNKJV    
    {he shall confirm the covenant with many for one week} % Webster
\Note 9:27 {He will confirm a covenant with many for one \"seven."} 
The most likely antecedent of \"he" is \"the Anointed One" or the ruler" (<"v." 26>). It is popular to interpret this statement as descriptive 
of an agreement that the antichrist will establish with Jewish people who have re-gathered in the land of Israel during the \"tribulation" period, but this outlook is less likely. In the middle of the `seven' he will put an end to sacrifice and offering. This may be a reference to the termination of the Old Testament sacrificial system by the atoning death of Christ. It is also possible that it refers to the desecration of the temple by Antiochus IV Epiphanes 
or Titus (see note on v. 26). Some interpreters take the less likely view that this is a reference to the antichrist's prohibition of
 \"sacrifice and offering" (perhaps standing for religious practice in general) by the re-gathered Jewish people after three and a half years  
(<Rev 11:2>; <12:6>, <14>) of the \"tribulation" period. And on a wing of 
the temple he will set up an abomination that causes desolation. Daniel most likely described the
destruction of the temple  under either Antiochus IV Epiphanes or Titus (see <"note on v." 26>n 
and \"Introduction: Purpose and Distinctives"), rather than actions of a future antichrist. Phrases similar to \"an abomination that 
causes desolation" occur in <8:13>, <11:31> and <12:11> (<"see"_8:13>n <"their"_11:31>n <"notes"_12:11>n),  
as well as in 1 Maccabees 1:54. Daniel <8:13> and 1 Maccabees 1:54 refer to the activities of Antiochus IV. Daniel  
used the same language to describe one who would defile the temple in the time near that of the
Messiah. Jesus alluded to this  abomination in <Matt 24:15> and <Mk 13:14>. 

\Note 10:1-12:13 {} {\it Vision of the Future of God's People.}\/ 
The prophet turned his attention to a final, lengthy vision that focused on the reign of Antiochus IV Epiphanes (see \"Introduction: Purpose and 
Distinctives") and looked beyond that reign as well. This material divides into four main sections: the angel's announcement to Daniel (<10:1-11:1>), events from Daniel until Antiochus IV Epiphanes  (<11:2-20>), the reign of Antiochus IV Epiphanes (<11:21-12:3>) and a final message to Daniel (<12:4-13>).

\Note 10:1-11:1 {} {\it The Angel's Message to Daniel.}\/ 
Daniel was prepared by an angelic being to receive a revelation pertaining to \"a time yet to come" (<10:14>).

\ww {In the third year of Cyrus, king of Persia} % BBE
    {In the third year of Cyrus king of Persia} % Jubilee2000
    {In the third year of King Cyrus of Persia} % NETfree
    {In the third year of Cyrus king of Persia} % UKJV
    {In the third year of Cyrus king of Persia} % RNKJV    
    {In the third year of Cyrus king of Persia} % Webster
\Note 10:1 {In the third year of Cyrus king of Persia} In 537 B.C. See <"notes on" 1:21>n, <5:30>n, <6:1>n and <9:1>n. The repatriated exiles were at this 
time back in the land to rebuild the temple (<Ezr 1:1-4>; <3:8>), but they would soon have to give up the rebuilding (<Ezr 4:24>).

\ww {grief} % BBE
    {was mourning} % Jubilee2000
    {was mourning} % NETfree
    {was mourning} % UKJV
    {was mourning} % RNKJV    
    {was mourning} % Webster
\Note 10:2 {mourned} Daniel probably mourned because of the state of Jerusalem (<Ne 1:4>; <Isa 61:3-4>; <64:8-12>; <66:10>).

\ww {a man clothed in a linen robe} % BBE
    {a man clothed in linens} % Jubilee2000
    {man clothed in linen} % NETfree
    {man clothed in linen} % UKJV
    {man clothed in linen} % RNKJV    
    {man clothed in linen} % Webster
\Note 10:5 {man clothed in linen} Verses <5>--<6> give a detailed description of an angel, perhaps Gabriel (<9:21>) or the one who spoke to  Gabriel (<8:16>). 
His appearance was similar to that of the glory of the Lord (<Eze 1:26-28>; <Rev 1:12-16>). For other references to angels see <Jdg 13:6>, <Eze 9:2-3>; <10:2> and <Lk 24:4>.

\ww {great shaking} % BBE
    {great fear} % Jubilee2000
    {overcome with fright} % NETfree
    {great quaking} % UKJV
    {great quaking} % RNKJV    
    {great quaking} % Webster
\Note 10:7 {terror overwhelmed} See <Isa 6:5> and <Lk 5:8>.

\ww {your words have come to his ears: and I have come because of your words} % BBE
    {thy words were heard, and I am come because of thy words} % Jubilee2000
    {your words were heard. I have come in response to your words} % NETfree
    {your words were heard, and I am come for your words} % UKJV
    {thy words were heard, and I am come for thy words} % RNKJV    
    {thy words were heard, and I am come for thy words} % Webster
\Note 10:12 {your words were heard, and I have come in response to them}
The vision and revelation that Daniel received came as a direct response to his prayers.

\ww {But the angel of the kingdom of Persia put himself against me} % BBE
    {But the prince of the kingdom of Persia} % Jubilee2000
    {However, the prince of the kingdom of Persia} % NETfree
    {But the prince of the kingdom of Persia} % UKJV
    {But the prince of the kingdom of Persia} % RNKJV    
    {But the prince of the kingdom of Persia} % Webster
\Note 10:13 {But the prince of the Persian kingdom resisted me} 
In the context it is apparent that this prince refers to an evil, but powerful, spiritual being (cf. <Job 1:6-12>; <Ps 82>; <Isa 24:21>; <Lk 11:14-26>)   
assigned by Satan to activity pertaining to Persian rule. 
Similarly, the archangel Michael is called \"the great prince who protects" Israel (<12:1>). 
The host of heaven are said to fight for Israel elsewhere in the Old Testament (<Jdg 5:20>; <2Ki 6:15-18>; <Ps 103:20-21>).


\ww {Michael, one of the chief angels, came to my help} % BBE
    {Michael, one of the chief princes, came to help me} % Jubilee2000
    {Michael, one of the leading princes, came to help me} % NETfree
    {Michael, one of the chief princes, came to help me} % UKJV
    {Michael, one of the chief princes, came to help me} % RNKJV    
    {Michael, one of the chief princes, came to help me} % Webster
\Note 10:13  {Then Michael, one of the chief princes, came to help me} 
Michael is depicted as the commander of the holy angels in <Jude 9> and <Rev 12:7>. 
Here a glimpse is given into the spiritual battles waged in the heavenly realms that affect events on Earth (cf. <Eph 6:12>; <Rev 12:7-9>).




\renum Dan 10:20 = Jubilee2000 10:21-21
\renum Dan 10:20 = NETfree 10:21-21
\renum Dan 10:20 = UKJV 10:21-21
\renum Dan 10:20 = RNKJV 10:21-21
\renum Dan 10:20 = Webster 10:21-21
\ww {true writings} % BBE
    {the scripture of truth} % Jubilee2000
    {a dependable book} % NETfree
    {the scripture of truth} % UKJV
    {the scripture of truth} % RNKJV    
    {the scripture of truth} % Webster
\Note 10:20 {the Book of Truth} A metaphor for God's knowledge and control over all of history.



\renum Dan 10:21 = Jubilee2000 10:20-20
\renum Dan 10:21 = NETfree 10:20-20
\renum Dan 10:21 = UKJV 10:20-20
\renum Dan 10:21 = RNKJV 10:20-20
\renum Dan 10:21 = Webster 10:20-20
\ww {I am going back to make war with the angel of Persia} % BBE
    {I must return to fight with the prince of the Persians} % Jubilee2000
    {I am about to return to engage in battle with the prince of Persia} % NETfree
    {will I return to fight with the prince of Persia} % UKJV
    {will I return to fight with the prince of Persia} % RNKJV    
    {will I return to fight with the prince of Persia} % Webster
\Note 10:21 {I will return to fight against the prince of Persia} See <"note on verse" 13>n.


\renum Dan 10:21 = Jubilee2000 10:20-20
\renum Dan 10:21 = NETfree 10:20-20
\renum Dan 10:21 = UKJV 10:20-20
\renum Dan 10:21 = RNKJV 10:20-20
\renum Dan 10:21 = Webster 10:20-20
\ww {the angel of Greece} % BBE
    {the prince of Grecia} % Jubilee2000
    {the prince of Greece} % NETfree
    {the prince of Grecia} % UKJV
    {the prince of Grecia} % RNKJV    
    {the prince of Grecia} % Webster
\Note 10:21 {the prince of Greece} This is a fallen angel or demonic power assigned by Satan to participate in the affairs of 
the Greek kingdom (see <"note on v." 13>n; see <Jn 14:30>; <Eph 6:12>). Although both Persia and Greece would
conquer God's people, Daniel was to understand that their power would be limited by the power of God, whose purposes always prevail.


\renum Dan 10:21 = Jubilee2000 10:21-21
\renum Dan 10:21 = NETfree 10:21-21
\renum Dan 10:21 = UKJV 10:21-21
\renum Dan 10:21 = RNKJV 10:21-21
\renum Dan 10:21 = Webster 10:21-21
\ww {no one}={no one ... but Michael} % BBE
    {no one}={no one ... but Michael} % Jubilee2000
    {no one}={no one ... except Michael} % NETfree
    {none}={none ... but Michael} % UKJV
    {none}={none ... but Michael} % RNKJV    
    {none}={none ... but Michael} % Webster
\Note 10:21 {No one ... except Michael} Michael's 
interest in protecting Israel (see <"note on v." 13>n; cf. <12:1>) corresponded with that of the messenger,
who was directly concerned about  God's purposes.

%\ww {} % BBE
%    {} % Jubilee2000
%    {} % NETfree
%    {} % UKJV
%    {} % RNKJV    
%    {} % Webster
\Note 11:1 {in the first year of Darius the Mede} Earlier the angel who 
was speaking to Daniel had given assistance to Michael (see <"note on" 10:13>n), perhaps in connection with the Persian decree to permit 
the Jews to return to their homeland.

\Note 11:2-20 {} {\it From Daniel Until Antiochus IV Epiphanes.}\/
The revelation given to Daniel in 11:2-20 concerned ancient Near Eastern  history from the time of Daniel until the time of Antiochus IV Epiphanes. 
The prophet's vision was unusually detailed, describing intricate interconnections among events far beyond that normally given to an Israelite prophet. 
Such details drew the attention of early readers of this book and demonstrated Daniel's reliability.

\ww {three kings to come in Persia} % BBE
    {there shall yet be three kings in Persia} % Jubilee2000
    {Three more kings will arise for Persia} % NETfree
    {three kings in Persia} % UKJV
    {three kings in Persia} % RNKJV    
    {three kings in Persia} % Webster
\Note 11:2 {Three more kings will appear in Persia} Cambyses 
(529-523 B.C.), Pseudo-Smerdis or Gaumata (523-2 B.C.) and Darius I (522-486 B.C.). 

\ww {the fourth} % BBE
    {the fourth} % Jubilee2000
    {a fourth} % NETfree
    {the fourth} % UKJV
    {the fourth} % RNKJV    
    {the fourth} % Webster
\Note 11:2 {a fourth} Xerxes I (485-464 B.C.). 

\ww {wealth} % BBE
    {riches} % Jubilee2000
    {unusually rich} % NETfree
    {his riches} % UKJV
    {richer} % RNKJV    
    {his riches} % Webster
\Note 11:2 {his wealth}  See <Est 1:4>. 

\ww {he will put his forces in motion against all the kingdoms of Greece} % BBE
    {he shall stir up all against the realm of Grecia} % Jubilee2000
    {he will stir up everyone against the kingdom of Greece} % NETfree
    {he shall stir up all against the realm of Grecia} % UKJV
    {he shall stir up all against the realm of Grecia} % RNKJV    
    {he shall stir up all against the realm of Grecia} % Webster
\Note 11:2 {he will stir up everyone against the kingdom of Greece} Xerxes waged a number of campaigns against Greece, beginning in 480 B.C. 

\ww {a strong king will come to power} % BBE
    {a valiant king shall stand up} % Jubilee2000
    {a powerful king will arise} % NETfree
    {a mighty king shall stand up} % UKJV
    {a mighty king shall stand up} % RNKJV    
    {a mighty king shall stand up} % Webster
\Note 11:3 {a mighty king will appear} Alexander the Great (336-323 B.C. ). See <"notes on" 7:6>n and <8:5>n and <8>n.

\ww {his kingdom will be broken}={his kingdom will be broken ... to the four winds of heaven} % BBE
    {his kingdom shall be broken}={his kingdom shall be broken ... divided by the four winds of heaven} % Jubilee2000
    {his kingdom will be broken}={his kingdom will be broken up ... toward the four winds of the sky} % NETfree
    {his kingdom shall be broken}={his kingdom shall be broken ... toward the four winds of heaven} % UKJV
    {his kingdom shall be broken}={his kingdom shall be broken ... toward the four winds of heaven} % RNKJV    
    {his kingdom shall be broken}={his kingdom shall be broken ... toward the four winds of heaven} % Webster
\Note 11:4 {his empire will be broken up ... toward the four winds of heaven} See <"notes on" 7:6>n and <8:8>n.

%\ww {} % BBE
%    {} % Jubilee2000
%    {} % NETfree
%    {} % UKJV
%    {} % RNKJV    
%    {} % Webster
\Note 11:5 {the king of the south} Ptolemy I Soter (323-285 B.C.).

\ww {one of his captains will be stronger than he} % BBE
    {of his principalities, shall make himself strong} % Jubilee2000
    {one of his subordinates will grow strong} % NETfree
    {one of his princes; and he shall be strong above him} % UKJV
    {one of his princes; and he shall be strong above him} % RNKJV    
    {of his princes; and he shall be strong above him}={one of his princes; and he shall be strong above him} % Webster
\Note 11:5 {one of his commanders will become even stronger}
Seleucus I Nicator (311-280 B.C.). Seleucus broke with Ptolemy, became king of 
Babylon and controlled territories from the Indus River in the east, to Syria in the west.

%\ww {} % BBE
%    {} % Jubilee2000
%    {} % NETfree
%    {} % UKJV
%    {} % RNKJV    
%    {} % Webster
\Note 11:6-20 {}  Verses 6-20 contain detailed predictions of relations between the king of the North (the Seleucid kingdom) and the king 
of the South (the Ptolemaic kingdom). This section may be divided into three parts: (1) events concerning Laodice and Berenice (<"vv." 6-9>), (2) the career of Antiochus III (<"vv." 10-19>) and (3) the reign of Seleucus IV (<"v." 20>).

\ww {the daughter of the king of the south} % BBE
    {the king's daughter of the south} % Jubilee2000
    {the daughter of the king of the south} % NETfree
    {the king's daughter of the south} % UKJV
    {the king's daughter of the south} % RNKJV    
    {the king's daughter of the south} % Webster
\Note 11:6 {The daughter of the king of the South} Berenice, the  
daughter of Ptolemy ll Philadelphus (285-246 B.C.).

\Note 11:6 {to make an agreement}
Refers to a marriage alliance (c. 250 B.C.) between Antiochus II Theos (261-246 B.C.) of Syria and Ptolemy II of Egypt.

\ww {she will not keep the strength of her arm; and his offspring will not keep their place} % BBE
    {she shall not retain the power of the arm; neither shall he stand, nor his arm} % Jubilee2000
    {she will not retain her power, nor will he continue in his strength} % NETfree
    {she shall not retain the power of the arm; neither shall he stand, nor his arm} % UKJV
    {she shall not retain the power of the arm; neither shall he stand, nor his arm} % RNKJV    
    {she shall not retain the power of the arm; neither shall he stand, nor his arm} % Webster
\Note 11:6 {she  will not retain her power, and he and his power will not last} 
Laodice, the former wife of Antiochus, instigated a conspiracy that resulted in the poisoning deaths of Berenice, Antiochus II and their infant son.

\ww {a branch from her roots} % BBE
    {the new shoot from her roots} % Jubilee2000
    {one from her family line} % NETfree
    {out of a branch of her roots} % UKJV
    {out of a branch of her roots} % RNKJV    
    {out of a branch of her roots} % Webster
\Note 11:7 {One from her family line will arise} Ptolemy III Euergetes (246-221 B.C.), the brother of Berenice (see <"note on v." 6>n).

\ww {forcing his way into the strong place of the king of the north} % BBE
    {shall enter into the fortress of the king of the north} % Jubilee2000
    {will enter the stronghold of the king of the north} % NETfree
    {shall enter into the fortress of the king of the north} % UKJV
    {shall enter into the fortress of the king of the north} % RNKJV    
    {shall enter into the fortress of the king of the north} % Webster
\Note 11:7 {He will attack the forces of the king of the North} Ptolemy III attacked the Seleucid kingdom, had Laodice (see <"note on v." 6>n) put to death 
and returned to Egypt with considerable booty.

\ww {he will come into the kingdom of the king of the south, but he will go back to his land} % BBE
    {shall the king of the south enter into the kingdom and return to his own land} % Jubilee2000
    {the king of the north will advance against the empire of the king of the south, but will withdraw to his own land} % NETfree
    {the king of the south shall come into his kingdom, and shall return into his own land} % UKJV
    {the king of the south shall come into his kingdom, and shall return into his own land} % RNKJV    
    {kingdom, and shall return into his own land}={the king of the south shall come into his kingdom, and shall return into his own land} % Webster
\Note 11:9 {the king of the North will invade the realm of the king of the South} This refers to the unsuccessful campaign of Seleucus II Callinicus (246-226 B.C.), the son of Laodice, against the  Ptolemaic kingdom in 240 B.C.

\ww {his son} % BBE
    {the sons of that one} % Jubilee2000
    {His sons} % NETfree
    {his sons} % UKJV
    {his sons} % RNKJV    
    {his sons} % Webster
\Note 11:10 {His sons} Seleucus III Ceraunus (226-223 B.C.) and Antiochus III the Great (223-187 B.C.). 

\ww {will make war, and will get together an army of great forces} % BBE
    {shall be stirred up and shall assemble a multitude of great armies} % Jubilee2000
    {will wage war, mustering a large army} % NETfree
    {shall be stirred up, and shall assemble a multitude of great forces} % UKJV
    {shall be stirred up, and shall assemble a multitude of great forces} % RNKJV    
    {shall be stirred up, and shall assemble a multitude of great forces} % Webster
\Note 11:10 {will prepare for war and assemble a great army} Antiochus III fought with the Ptolemies from 222-187 B.C. and for a time gained control of Canaan, as well as western Syria.

\ww {his strong place} % BBE
    {his fortress} % Jubilee2000
    {the enemy's fortress} % NETfree
    {his fortress} % UKJV
    {his fortress} % RNKJV    
    {his fortress} % Webster
\Note 11:10 {his fortress} This probably refers to Raphia, a Ptolemaic fortress in southern Canaan. A major battle was fought  there in 217 B.C.

\Note 11:11 {the king of the south} Ptolemy IV Philopator (221-203 B.C.). 

\ww {make war on him, on this same king of the north} % BBE
    {with the king of the north}={fight with him even with the king of the north} % Jubilee2000
    {fight against the king of the north} % NETfree
    {fight with him, even with the king of the north} % UKJV
    {fight with him, even with the king of the north} % RNKJV    
    {fight with him}={fight with him, even with the king of the north} % Webster
\Note 11:11 {fight against the king of the North} Antiochus III. He suffered great losses (over 14,000 men) at the battle of Raphia in 217 B.C. 

\ww {the king of the north will get together an army greater than the first} % BBE
    {the king of the north shall put another multitude greater than the former} % Jubilee2000
    {the king of the north will again muster an army, one larger than before} % NETfree
    {shall set forth a multitude greater than the former} % UKJV
    {shall set forth a multitude greater than the former} % RNKJV    
    {shall set forth a multitude greater than the former} % Webster
\Note 11:13 {the king of the North will muster another army}
In alliance with Philip V of Macedon, he raised an even larger army to 
invade the Ptolemaic kingdom. Ptolemy IV died in mysterious circumstances and was succeeded by
Ptolemy V Epiphanes (203-181  B.C.), his four-year-old son.

\ww {the king of the north will come, and put up earthworks and take a well-armed town} % BBE
    {he king of the north shall come and cast up a mount and shall take the strong cities} % Jubilee2000
    {the king of the north will advance and will build siege mounds and capture a well-fortified city} % NETfree
    {the king of the north shall come, and cast up a mount, and take the most fenced cities} % UKJV
    {the king of the north shall come, and cast up a mount, and take the most fenced cities} % RNKJV    
    {the king of the north shall come, and cast up a mount, and take the most fortified cities} % Webster
\Note 11:15 {the king of the North will come and build up siege ramps and will capture a fortified city} 
This refers to the victory of Antiochus III at Sidon over the Egyptian general Scopas in 198 B.C. 
It marked the end of Ptolemaic rule in the area only much later referred to as Palestine.

\ww {the beautiful land} % BBE
    {the glorious land} % Jubilee2000
    {the beautiful land} % NETfree
    {the glorious land} % UKJV
    {the glorious land} % RNKJV    
    {the glorious land} % Webster
\Note 11:16 {the Beautiful Land} The promised land (see <"vv." 41>, <45>; <8:9>).

%\ww {} % BBE
%    {} % Jubilee2000
%    {} % NETfree
%    {} % UKJV
%    {} % RNKJV    
%    {} % Webster
%\Note 11:17-19 {}  See WSC 105.

\ww {he will make an agreement with him; and he will give him the daughter} % BBE
    {shall do upright things with him, and he shall give him a daughter}  % Jubilee2000
    {he will form alliances. He will give the king of the south a daughter in marriage}  % NETfree
    {upright ones with him; thus shall he do: and he shall give him the daughter}  % UKJV
    {upright ones with him; thus shall he do: and he shall give him the daughter}  % RNKJV    
    {upright ones with him; thus shall he do: and he shall give him the daughter}  % Webster
\Note 11:17 {He ... will make an alliance with the king of the South. And he will give him a daughter in marriage} 
Cleopatra, the daughter of Antiochus III, was given in marriage to the boy king Ptolemy V. 

\ww {this will not take place} % BBE
    {she shall not stand} % Jubilee2000
    {it will not turn out to his advantage} % NETfree
    {she shall not stand on his side} % UKJV
    {she shall not stand on his side} % RNKJV    
    {she shall not stand}={she shall not stand on his side} % Webster
\Note 11:17 {his plans will not succeed or help him} 
Cleopatra aligned herself with the Egyptians rather than with her father. 
She sought Roman help against the attempt of Antiochus III to take coastal cities in Asia Minor controlled by the Egyptians.

\ww {a chief, by his destruction, will put an end to the shame offered by him} % BBE
    {a prince shall cause him to cease his affront} % Jubilee2000
    {a commander will bring his shameful conduct to a halt} % NETfree
    {a prince for his own behalf shall cause the reproach offered by him to cease} % UKJV
    {a prince for his own behalf shall cause the reproach offered by him to cease} % RNKJV    
    {a prince for his own behalf shall cause the reproach offered by him to cease} % Webster
\Note 11:18 {a commander will put an end to his insolence} The Roman general Lucius Cornelius Scipio
defeated Antiochus III in several battles and forced him to cede Asia Minor to Roman control (the Peace of Apamea; 188 B.C.).
At this time the second son of Antiochus III, later to be known as Antiochus IV Epiphanes, was taken  hostage to Rome.

\ww {his place will be taken by one} % BBE
    {Then shall succeed in his throne} % Jubilee2000
    {after him one} % NETfree
    {Then shall stand up in his estate} % UKJV
    {Then shall stand up in his estate} % RNKJV    
    {Then shall stand up in his estate} % Webster
\Note 11:20 {His successor} Seleucus IV Philopator (187-175 B.C.), the elder son of Antiochus III. 

\ww {a man with the glory of a king to get wealth together} % BBE
    {a taker of taxes} % Jubilee2000
    {an exactor} % NETfree
    {a raiser of taxes} % UKJV
    {a raiser of taxes} % RNKJV    
    {a raiser of taxes} % Webster
\Note 11:20 {a tax collector} Heliodorus (see 2 Maccabees 3:7-40). 

\Note 11:21-12:3 {} {\it The Rule of Antiochus IV Epiphanes.}\/ Daniel turned 
to the most important character in the history outlined thus far: the great Antiochus IV, who persecuted the Jews and defiled the temple.
The prophet concentrated on his accession and character (<11:21-24>), his career (<11:25-31>), the conditions of God's people 
during his reign (<11:32-35>), a summary of his religious attitudes (<11:36-39>), his heart's ambition (<11:40-45>) and a description of his defeat (<12:1-3>).

\ww {a low person, to whom the honour of the kingdom had not been given} % BBE
    {shall succeed in his place, to whom they shall not give the honour of the Kingdom}={a vile person shall succeed in his place, to whom they shall not give the honour of the Kingdom} % Jubilee2000
    {a despicable person to whom the royal honor has not been rightfully conferred} % NETfree
    {a vile person, to whom they shall not give the honour of the kingdom} % UKJV
    {a vile person, to whom they shall not give the honour of the kingdom} % RNKJV    
    {a vile person, to whom they shall not give the honor of the kingdom} % Webster
\Note 11:21 {a contemptible person ... not ... given the honor of royalty} This is the infamous Antiochus IV Epiphanes (175-164 B.C.), 
brother of Seleucus IV but not his legitimate successor, since Seleucus IV had a son, Demetrius Soter, also known as Demetrius I. 
See <"verses" 23-24> and <"notes on" 8:9-14>n.

\ww {the ruler of the agreement will have the same fate} % BBE
    {shall be broken; yea, also the prince of the covenant} % Jubilee2000
    {a covenant leader will be destroyed} % NETfree
    {shall be broken; yea, also the prince of the covenant} % UKJV
    {shall be broken; yea, also the prince of the covenant} % RNKJV    
    {shall be broken; yes, also the prince of the covenant} % Webster
\Note 11:22 {a prince of the covenant will be destroyed} Perhaps this is a reference to the assassination of the high priest Onias III by the 
supporters of Antiochus IV (175-163 B.C.) in Jerusalem in 171 B.C. (cf. 2 Maccabees 4:32-43 [an Apocryphal book]).

\Note 11:25 {the king of the south} Ptolemy VI Philometor (181-146 B.C.), son of Ptolemy V and Cleopatra and nephew of Antiochus (see <"note on v." 17>n).

\ww {he will be forced to give way} % BBE
    {he shall not prevail} % Jubilee2000
    {he will not be able to prevail} % NETfree
    {he shall not stand} % UKJV
    {he shall not stand} % RNKJV    
    {he shall not stand} % Webster
\Note 11:25 {he will not be able to stand} Antiochus IV defeated Ptolemy VI at Pelusium, located on the border of Egypt (cf.  1 Maccabees 1:16-19).

\ww {he will go back to his land with great wealth; and his heart will be against the holy agreement} % BBE
    {he shall return into his land with great riches; and his heart}={he shall return into his land with great riches; and his heart shall be against the holy covenant} % Jubilee2000
    {the king of the north will return to his own land with much property. His mind will be set against the holy covenant} % NETfree
    {shall he return into his land with great riches; and his heart shall be against the holy covenant} % UKJV
    {shall he return into his land with great riches; and his heart shall be against the holy covenant} % RNKJV    
    {shall he return into his land with great riches}={shall he return into his land with great riches; and his heart shall be against the holy covenant} % Webster
\Note 11:28 {The king of the North will return ... but his heart will be set against the holy covenant.} 
As a result of intrigues in Jerusalem against his supporters, Antiochus IV plundered the temple on his return from Egypt to Antioch in Syria (cf. 1 Maccabees  1:20-28).

\ww {he will come back and come into the south} % BBE
    {he shall turn toward the south} % Jubilee2000
    {he will again invade the south} % NETfree
    {he shall return, and come toward the south} % UKJV
    {he shall return, and come toward the south} % RNKJV    
    {he shall return, and come towards the south} % Webster
\Note 11:29 {he will invade the South again} Antiochus IV invaded  Egypt again in 168 B.C.

\ww {those who go out from the west will come against him} % BBE
    {the ships of Chittim shall come against him} % Jubilee2000
    {The ships of Kittim will come against him} % NETfree
    {the ships of Chittim shall come against him} % UKJV
    {the ships of Chittim shall come against him} % RNKJV    
    {the ships of Chittim shall come against him} % Webster
\Note 11:30 {Ships of the western coastlands will oppose him} Roman armies under Gaius Popilius Laenas
forced Antiochus IV to retreat from Egypt.

\ww {full of wrath against the holy agreement} % BBE
    {indignation against the holy covenant} % Jubilee2000
    {indignation against the holy covenant} % NETfree
    {indignation against the holy covenant} % UKJV
    {indignation against the holy covenant} % RNKJV    
    {indignation against the holy covenant} % Webster
\Note 11:30 {vent his fury against the holy covenant} Antiochus determined to exterminate Jewish religion.

\ww {take away the regular burned offering}={take away the regular burned offering ... an unclean thing causing fear} % BBE
    {take away the daily}={take away the daily sacrifice ... abomination that makes desolate} % Jubilee2000
    {stopping the daily sacrifice}={stopping the daily sacrifice ... the abomination that causes desolation} % NETfree
    {take away the daily sacrifice}={take away the daily sacrifice ... abomination that makes desolate} % UKJV
    {take away the daily sacrifice}={take away the daily sacrifice ... abomination that maketh desolate} % RNKJV    
    {take away the daily}={take away the daily sacrifice ... abomination that maketh desolate} % Webster
\Note 11:31 {abolish the daily sacrifice ... set up the abomination 
that causes desolation} The desecration of the temple in December 168 B.C. by Antiochus IV (cf. 1
Maccabees 1:54,59; 2 Maccabees 6:2 [Apocryphal books]; see <"notes on" 8:11>n; <9:27>n; <12:11>n).

\ww {the people who have knowledge of their God will be strong} % BBE
    {the people that do know their God shall be strong} % Jubilee2000
    {the people who are loyal to their God will act valiantly} % NETfree
    {the people that do know their God shall be strong} % UKJV
    {the people that do know their Elohim shall be strong} % RNKJV    
    {the people that know their God shall be strong} % Webster
\Note 11:32 {the people who know their God will firmly resist him} 
Refers to those who opposed Antiochus IV and remained faithful to the Lord even unto death (1
Maccabees 1:61-63 [an Apocryphal book]).

\ww {they will have a little help} % BBE
    {they shall be helped with a little help} % Jubilee2000
    {they will be granted some help} % NETfree
    {they shall be helped with a little help} % UKJV
    {they shall be helped with a little help} % RNKJV    
    {they shall be assisted with a little help} % Webster
\Note 11:34 {they will receive a little help} Possibly this is a reference 
to Mattathias, an elderly priest, and his five sons (John, Simon, Judas, Eleazar and Jonathan), who waged a guerrilla war against Antiochus IV. 
Mattathias died in 166 B.C. His sons carried on the struggle and became known as the Maccabees. Victory was achieved under Judas Maccabeus in December 165 B.C., when the temple was cleansed and the daily sacrifices restored (cf. 1 Maccabees 4:36-39).

\ww {the time of the end}={the time of the end ... the fixed time} % BBE
    {the time of the end}={the time of the end ... time appointed} % Jubilee2000
    {the time of the end}={the time of the end ... the appointed time} % NETfree
    {the time of the end}={the time of the end ... a time appointed} % UKJV
    {the time of the end}={the time of the end ... a time appointed} % RNKJV    
    {the time of the end}={the time of the end ... a time appointed} % Webster
\Note 11:35 {the time of the end ... the appointed time} See <"note on" 8:17>n.

\Note 11:36-12:3 {} {At his proudest moment this king will be destroyed right at Mount Zion in the heart of the Holy Land} (vv. 44-45).
His  defeat in <12:1-3> is described in terms of the absolute end of history. Because these prophesies have
not found a historical fulfillment, it is difficult to discern how literal or metaphorical they are, 
and our interpretation must be guarded. Certain details in <11:36-12:3> cannot be easily harmonized
with the time of Antiochus IV.  For this reason many evangelical interpreters understand these 
verses to be descriptive of the antichrist who will persecute God's people just prior to the second advent of Christ (cf. <12:1-3>). Yet 
this understanding requires the assumption of an extended time interval between the events depicted in <11:21-35> and those in 
<11:36-12:3>, which the text does not communicate. It is possible that these prophesied events were averted, altered or delayed (see 
 \"Introduction to the Prophetic Books").

    \Note 11:36-39 {} {\it This king \"will do as he pleases ... [and] magnify himself above every god".}\/ (v. <36>). He
will \"show no regard for the gods  of his fathers" (v. <37>) but will \"greatly honor those who acknowledge him" (v. <39>).
%predelat na \vdef



\ww {till the wrath is complete} % BBE
    {until the indignation is accomplished} % Jubilee2000
    {until the time of wrath is completed} % NETfree
    {till the indignation be accomplished} % UKJV
    {till the indignation be accomplished} % RNKJV    
    {till the indignation be accomplished} % Webster
\Note 11:36 {until the time of wrath is completed} Just as in <8:17> and <11:35>, the time of persecution is subject to God's control. 


\ww {at the time of the end} % BBE
    {at the end of the time} % Jubilee2000
    {At the time of the end} % NETfree
    {at the time of the end} % UKJV
    {at the time of the end} % RNKJV    
    {at the time of the end} % Webster
\Note  11:40 {At the time of the end} See <"note on" 8:17>n. 

\ww {beautiful land} % BBE
    {glorious land} % Jubilee2000
    {beautiful land} % NETfree
    {glorious land} % UKJV
    {glorious land} % RNKJV    
    {glorious land} % Webster
\Note 11:41 {Beautiful Land} Canaan (see <"vv." 16>, <45>; <8:9>).

\ww {he will come to his end with no helper} % BBE
    {he shall come to his end, and shall have no one to help him} % Jubilee2000
    {he will come to his end, with no one to help him} % NETfree
    {he shall come to his end, and none shall help him} % UKJV
    {he shall come to his end, and none shall help him} % RNKJV    
    {he shall come to his end, and none shall help him} % Webster
\Note 11:45 {Yet he will come to his end, and no one will help him} See <Joel 3> (see also <Zec 14:1-4>;
 <2Th 2:8>; <Rev 16:13-16>; <19:11-21>).

\ww {at that time} % BBE
    {at that time} % Jubilee2000
    {At that time} % NETfree
    {at that time} % UKJV
    {at that time} % RNKJV    
    {at that time} % Webster
\Note 12:1 {At that time} Michael, the angelic protector of Israel, will
 not permit God's people to be persecuted forever. He will judge those who oppress his
 people. Michael, the great prince who protects your people. See <"note on" 10:13>n.

\ww {a time of trouble} % BBE
    {a time of trouble} % Jubilee2000
    {a time of distress} % NETfree
    {a time of trouble} % UKJV
    {a time of trouble} % RNKJV    
    {a time of trouble} % Webster
\Note 12:1 {a time of distress} 
See <Matt 24:21> and <Mk 13:19>, where Jesus drew upon these  prophecies about Antiochus  IV to describe the time of the Roman siege against Jerusalem in
 A.D. 70. 
 
 \ww {your people will be kept safe} % BBE
    {thy people shall escape} % Jubilee2000
    {escape}={your own people ... will escape} % NETfree
    {your people shall be delivered} % UKJV
    {thy people shall be delivered} % RNKJV    
    {thy people shall be delivered} % Webster
\Note 12:1 {your people ... will be  delivered} 
 This deliverance is not necessarily from martyrdom (\<"v." 2>) but from the power of Satan  (cf. <Mt 6:13>; <2Ti 4:18>). 
As such the verse assures God's people that he will deliver them from Satan's temptation to apostatize during the time of distress.
 
\ww {will come out of their sleep, some to eternal life and some to eternal shame} % BBE
    {everlasting confusion}={shall be awakened, some for eternal life, and some for shame and everlasting confusion} % Jubilee2000
    {will awake - some to everlasting life, and others to shame and everlasting abhorrence} % NETfree
    {shall awake, some to everlasting life, and some to shame and everlasting contempt} % UKJV
    {shall awake, some to everlasting life, and some to shame and everlasting contempt} % RNKJV    
    {shall awake, some to everlasting life, and some to shame}={shall awake, some to everlasting life, and some to shame and everlasting contempt} % Webster
\Note 12:2 {will awake ... to everlasting life ... and everlasting 
contempt} This is a prediction of the bodily resurrection of the 
godly and ungodly prior to a final judgment (<Mt 25:46>; <Jn 5:28-29>). 

\Note 12:4-13 {} {\it A Final Message to Daniel.}\/ The book concludes by setting out a future course of events and by promising Daniel rest in the eternal state.

\ww {let the words be kept secret} % BBE
    {shut up the words} % Jubilee2000
    {close up these words} % NETfree
    {shut up the words} % UKJV
    {shut up the words} % RNKJV    
    {shut up the words} % Webster
\Note 12:4 {seal the words of the scroll} The act of sealing was understood as giving something a mark of
authentication (see <"note on"  8:26>n).

\ww {a time, times, and a half} % BBE
    {time, times, and a half}={a time, times, and a half} % Jubilee2000
    {a time, times, and half a time} % NETfree
    {a time, times, and an half} % UKJV
    {a time, times, and an half} % RNKJV    
    {a time, times, and a half} % Webster
\Note 12:7 {a time, times and half a time} See <"note on" 7:25>n.

\ww {the sense of them was not clear to me} % BBE
    {I did not understand} % Jubilee2000
    {I did not understand} % NETfree
    {I understood not} % UKJV
    {I understood not} % RNKJV    
    {I understood not} % Webster
\Note 12:8 {I did not understand} Daniel did not comprehend the angel's response (<"v." 7>) to his initial inquiry (<"v." 6>), so he rephrased the  question.

\ww {the regular burned offering is taken away, and an unclean thing causing fear is put up} % BBE
    {the daily}={the daily sacrifice is taken away until the abomination of desolation} % Jubilee2000
    {the daily sacrifice is removed and the abomination that causes desolation is set in place} % NETfree
    {the daily sacrifice shall be taken away, and the abomination that makes desolate set up} % UKJV
    {the daily sacrifice shall be taken away, and the abomination that maketh desolate set up} % RNKJV    
    {shall be taken away, and the abomination that maketh desolate set up}={the daily sacrifice shall be taken away, and the abomination that maketh desolate set up} % Webster
\Note 12:11 {The daily sacrifice is abolished and the abomination that causes desolation is set up} 
See <"note on" 9:27>. The simile activity of Antiochus IV prefigured this activity of the Roman Titus  in A.D. 70. 

\ww {thousand, three hundred and thirty-five days}={1,290 days ... 1,335 days}  % BBE
    {one thousand three hundred and thirty-five days}={1,290 days ... 1,335 days}  % Jubilee2000
    {1,335 days}={1,290 days ... 1,335 days}  % NETfree
    {thousand three hundred and five and thirty days}={1,290 days ... 1,335 days}  % UKJV
    {thousand three hundred and five and thirty days}={1,290 days ... 1,335 days}  % RNKJV    
    {thousand three hundred and five and thirty days}={1,290 days ... 1,335 days} % Webster
\Note 12:12 {1,290 days ... 1,335 days} The angel clarified his previous answer (<"v." 7>; see <"note on
v." 8>n). The significance of these time  frames is obscure.
%not 6




\putBot 2:1 {Daniel's Visions of the Four Kingdoms} [danielovyvize] () {
\medskip
\hrule
\vskip1cm
   \Heros \cond \setfontsize{at 10pt}\rm
   \picw=0.9\hsize
   \inspic{Nabuko-English-blank-crop.pdf}
%\medskip
%\hrule
%\vskip1cm
\bigskip
   \puttext 15mm 130mm {\bf Identification of the Four Kingdoms}
%\medskip
%\hrule
%%\vskip1cm
%\bigskip
   \puttext 110mm 130mm {\bf Chronology of the Empires}
  \puttext 13mm 120mm {Vision in chap. 2}
  \puttext 45mm 120mm {Vision in chap. 7}
  \puttext 67mm 120mm {Vision in chap. 8}
  \puttext 91mm 120mm {Identification}
   \puttext 134mm 113mm {600}
   \puttext 134mm 99.5mm {500}
   \puttext 134mm 84.5mm {400}
   \puttext 134mm 70mm {300}
   \puttext 134mm 55mm {200}
   \puttext 145mm 50.5mm {167 B.C.}
   \puttext 135mm 47mm {Maccabees and}
   \puttext 140mm 43.5mm { Hasmoneans}
   \puttext 134mm 41mm {100}
   \puttext 134mm 26mm {\bf 0}
   \puttext 134mm 11mm {100}
   \puttext 145mm 16mm {70 A.D.}
   \puttext 140mm 12.5mm {Fall of Jerusalem}
\bf  
   \puttext 142mm 117.5mm {626 B.C.}
   \puttext 142mm 104.5mm {539 B.C.}
   \puttext 142mm 74mm {330 B.C.}
   \puttext 142mm 36mm {63 B.C.}
\puttext 15mm 100mm {Head of gold}   
 \puttext 48mm 100mm {Lion}
  \puttext 91mm 100mm {Babylon (<2:37-38>)}
\puttext 12mm 75mm {\x/Breast and arms/}   
\puttext 17mm 71mm {of siver}   
 \puttext 48mm 75mm {Bear}   
  \puttext 70mm 75mm {\x/Male sheep/}   
    \puttext 91mm 75mm {Medo-Persia (<8:20>)}   
\puttext 12mm 46mm {\x/Middle and sides/}   
\puttext 20mm 42mm {of \x/brass/}   
 \puttext 48mm 44mm {\x/Leopard/}   
  \puttext 70mm 44mm {\x/He-goat/}   
   \puttext 91mm 44mm {\x/Greece/ (<8:21>)}   
\puttext 20mm 28mm {Legs of}
\puttext 22mm 24mm {iron}
 \puttext 46mm 28mm {Terrifying and}
 \puttext 48mm 24mm {frightening}
 \puttext 52mm 20mm {beast}
  \puttext 91mm 26mm {Rome} 
\puttext 15mm 2mm {\x/Feet of iron/}
\puttext 15mm -2mm {\x/and potter's earth/}
}

\putArticle 5:20 {Who was Darius the Mede?} [6] ()




\putCite 1:8 {%Druhořadých věcí nedosáhneme tím, že je prohlásíme za prvořadé. 
%Druhořadých věcí dosáhneme pouze tehdy, budou-li na prvním místě věci první. 
You can’t get second things by putting them first. 
You get second things only by putting first things first.
\quotedby{C. S. Lewis}}

\putCite 3:30 {%Odvaha není jen jednou ze ctností, ale je formou každé ctnosti v bodě zkoušky, tedy v bodě nejvypjatější skutečnosti.
Courage is not simply one of the virtues but the form of every virtue at the testing point, which means at the point of highest reality. 
\quotedby{C.S.Lewis}}

\putCite 5:4 {%Moc korumpuje, a absolutní moc korumpuje absolutně.
Power tends to corrupt; absolute power corrupts absolutely
\quotedby {Lord Acton}} 


\putCite 6:11 {%Život křesťana je život na území okupovaném nepřáteli. 
Enemy-occupied territory---that is what this world is. Christianity is the story of how the rightful king has landed, you might say landed in disguise, and is calling us to take part in a great campaign of sabotage.
\quotedby{C.S.Lewis}}









\endinput

\variants 6 {BBE} {Jubilee2000} {NETfree} {UKJV} {RNKJV} {Webster} 

\ww {} % BBE
    {} % Jubilee2000
    {} % NETfree
    {} % UKJV
    {} % RNKJV    
    {} % Webster

\ww {}={} % BBE
    {}={} % Jubilee2000
    {}={} % NETfree
    {}={} % UKJV
    {}={} % RNKJV
    {}={} % Webster









4:3 How great ... Nebuchadnezzar's confession in this verse 
and in verses 34-35 communicates one of the central themes of 
the book of Daniel; namely, the absolute sovereignty of the God of 
Israel over the kingdoms of the earth and their rulers.

4:6-7 See notes on 1:20 and 2:2. 

4:8 Belteshazzar. See note on 1:7.

4:9 spirit of the holy gods. Although he spoke in pagan terms 
Nebuchadnezzar stated an important truth. The presence of God's 
Spirit in an individual has remarkable effects. Here his ability to 
give extraordinary insight into God's mystery, such as was later given to Paul and the church (1Co
2:6-16), is in view. no mystery is too difficult for you. See 2:47 and note on 2:19.

4:10-12 before me stood a tree. See Ezekiel 31 for an extensive 
description of a nation (Assyria), using the imagery of a tree. Similar imagery is found in Psalms
1:3, 37:35, 52:8, 92:12, Jeremiah  11:16-17 and 17:8 (see also Mt 13:32). its top touched the sky. 
The term \"sky" may also be translated \"heaven," a key term in this 
chapter. The tree represented Nebuchadnezzar's kingdom reaching from Earth to heaven (vv. 11,20,22)
and protecting birds, which  defy the separation of the two spheres (vv. 12,21). In truth the king 
was not only subject to the judgment of heaven for his pride (w. 
13,23,31) but also dependent on the God of heaven for his existence (vv. 15,22,25,33) and sanity
(v. 34). 

