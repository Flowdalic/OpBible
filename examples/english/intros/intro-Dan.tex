
\def\nadpis#1{\smallskip\noindent{\bf#1}\par\nobreak}
\def\podbarvi#1#2#3{\setbox0=\hbox{#3}\leavevmode
{\localcolor\rlap{#1\strut\vrule width\wd0}#2\box0}}
%\nadpis{Přehled}
%
%\nadpis{Autor:} Daniel
%
%\nadpis{Záměr:}
%
%\begitems
%* Připravit babylonského krále \x/Nabuchodonozor/a na přijetí špatných zpráv kapitoly 4. budováním důvěry ve spolehlivost a pravdivost Danielových proroctví a všemohoucnost jeho Boha v předchozích kapitolách. 
%* Ujistit Izraelity (zajatce i první navrátilce do Země), že Bůh panuje nad dějinami a že jeho prorok Daniel říkal pravdu, když mluvil o prodloužené době útisku před závěrečnou fází Božího království. 
%* Připravit generace vzdálené budoucnosti na pronásledování, které je bude čekat v době Antiocha IV. Epifana.
%* Připravit věřící v ještě vzdálenější budoucnosti na příchod Mesiáše v době čtvrtého království.
%\enditems 
\nadpis{Overview}

\nadpis{Author:} Daniel

\nadpis{Purpose:}

\begitems
* To prepare the king of Babylon    Nebuchadnezzar %  \x/Nabuchodonosor/a 
to receive the bad news of chapter 4 by building confidence in the reliability and truthfulness of Daniel's prophecies and the omnipotence of his God in the preceding chapters. 
* Reassure the Israelites (captives and first returnees to the Land) that God is in control of history and that His prophet Daniel was telling the truth when he spoke of a prolonged period of oppression before the final phase of God's kingdom. 
* To prepare the generations of the distant future for the persecution that will await them in the time of Antiochus IV Epiphanes.
* To prepare believers in the even more distant future for the coming of the Messiah in the time of the fourth kingdom.

\enditems 
%\doImage {Středozem}[stredozem](){stredozemne.jpg}
\insertBot {The empires of Daniel's distant visions}[mapa](){\inspic{fertile-crescent-crop.pdf}%
  \Heros \cond \setfontsize{at 9pt}\rm
  \vskip-1mm
\putstext 2mm 108mm {\vtop{\hsize6.5cm \baselineskip9pt \noindent
  \leftskip=3pt \rightskip=3pt % miry byly bohuzel vzaty v nespravnem kontextu 
  Soon after Alexander' death his kingdom was divided among his four generals, so-called {\it Diadochi.\/} They did not make a deal peacefully, however, they fought for power fierslessly.
  As a result, Egypt was ruled by Ptolemaios I; Syria and Mesopotamia by Seleucos, Thracia amd Asia Minor by Lysimachus and Macedonia by Cassander.
    After 277 B.C. Only 3 hellenistic kingdoms remained: in Egypt, Syria, and Macedonia. They lasted until 63 B.C. when the Greek empire was conquered by Rome.
    }}
\putstext 55mm 20mm {\vtop{\hsize8.4cm \baselineskip10pt \noindent The 
  \leftskip=3pt \rightskip=3pt % miry byly bohuzel vzaty v nespravnem kontextu 
Holy Land was under the rule of the Ptolemaic dynasty (i.e. Egypt) from 323 to 198 BC; then under the rule of the Seleucids (i.e. Syria) from 198 to 142 BC. Their internecine strife is foreshadowed by <Dan 11:2-45>.  
The Hasmonean dynasty kept Judea independent of the Seleucid empire from 140 until 63 B.C., when Rome conquered the Greek empire and installed Herod the Great as king of Judea.}}   %(<Matt 2:1>).}}
%  \LMfonts \sans \setfontsize{at9pt}\rm
  \Heros \cond \setfontsize{at9pt}\rm 
  \puttext 145mm 29mm {<"Acts 2:9"_Acts 2:9>}
  \puttext 145mm 32.5mm {<"Ezek 32:16"_Ezek 32:16>}
  \puttext 145mm 36mm {<"Jer 25:25"_Jer 25:25>}
  \puttext 145mm 39.5mm {<"Jr 49:35"_Jr 49:35>}
  \puttext 145mm 43mm {<"Isa 22:6"_Isa 22:6>}
  \puttext 144mm 46.5mm {<"Isa 11:11"_Isa 11:11>}
  \puttext 142mm 50mm {<"Gen 10:22"_Gen 10:22>}
  \puttext 134mm 58mm {<"Dan 8:2"_Dan 8:2>}
  \puttext 134mm 55mm {<"Esth 1:2"_Esth 1:2>}
  \puttext 60mm 36mm {<"Dan 1:1"_Dan 1:1>}
  \puttext 55mm 53mm {<"Dan 11:5 (``king of the south'')"_Dan 11:5>}
  \puttext 57mm 66mm {<"Dan 11:6 (``king of the north'')"_Dan 11:6>}
  \puttext 88mm 45mm {<"Dan 3:1"_Dan 3:1>}
  \puttext 95mm 41mm {<"Dan 1:2"_Dan 1:2>}
  \puttext 95mm 37.5mm {<"Gen 11:2"_Gen 11:2>}
  \puttext 95mm 34mm {<"Gen 14:1"_Gen 14:1>}
  \puttext 106.5mm 34mm {<",9"_Gen 14:9>}
  \puttext 95mm 30.5mm {<"Josh 7:21"_Josh 7:21>}
  \puttext 95mm 27mm {<"Isa 11:11"_Isa 11:11>}
  \puttext 95mm 23.5mm {<"Zech 5:11"_Zech 5:11>}
  \puttext 115mm 42mm {<"Gen 11:31"_Gen 11:31>}
  \Heros \setfontsize{at 9pt}\rm
  \puttext 48mm 55mm {\c[-40/\kern1pt]{Jerusalem}}
  \town 47.5mm 53mm
  \puttext 110mm 42mm{Ur}
  \town 111mm 45mm
  \puttext 122mm 55mm{\x/Shushan/}
  \town 121.5mm 53mm
  \puttext 88mm 48mm{Tolul Dura}
  \town 100.75mm 50mm
  \puttext 92mm 55mm{Babylon}
  \town 101.5mm 53mm
  \Heros \setfontsize{at 13pt}\rm
  \puttext 62mm 70mm {\c[10/\kern7pt\pdfrotate{-1}]{THE SELEUCIDS}}
  \puttext 2mm 35mm  {\c[0/\kern3pt\pdfrotate{2.5}]{THE PTOLEMIES}}
  \Heros \setfontsize{at 11pt}\rm
  \puttext 130mm 50mm {\c[-40/\kern4pt\pdfrotate{-1}]{ELAM}}
  \puttext 103mm 55mm {\c[-25/\kern4pt\pdfrotate{1}]{SINEAR}}
  \puttext 50mm 30mm {A R A B I C}
  \puttext 55mm 25mm {D E S E R T}
  \puttext 126mm 35mm {\c[-50/\kern1pt\pdfrotate{1}]{{\bi Persian ~ ~ ~ gulf}}}  
  \puttext 42mm 24mm {\c[-60/\kern1pt\pdfrotate{0}]{{\bi Red ~ sea}}}  
  \puttext 5mm 62mm{{\bi Mediterranean sea}}
  \puttext 62mm 111mm{{\bi Black sea}}
  \puttext 130mm 100mm{{\bi Caspian}}
  \puttext 132mm 90mm{{\bi sea}}
  \puttext 70mm 80mm{\setfontsize{at 5pt}\c[-20/\kern1pt\pdfrotate{-3}]{\it Eufrates}}
  \puttext 92mm 84mm{{\setfontsize{at 5pt}\c[-40/\kern1pt\pdfrotate{-3}]{\it Tigris}}}
      \puttext 28mm 24mm{{\setfontsize{at 5pt}\c[-40/\kern1pt\pdfrotate{-3}]{\it Nile}}}
      \puttext 2mm 5mm{{\Heros \setfontsize{at 7pt}\it Satellite Bible Atlas,\/ \rm W.Schlegel}}
  \puttext 2mm 2mm{\Heros \setfontsize{at 7pt}\rm Used by permission.}
  }
%\leftline{<3:1> \hskip 20mm <8:2>; <Esth 1:2>}

\nadpis{Date:} Shortly after 539 B.C.

\nadpis{Key truths:}

\begitems
* Daniel and his friends were faithful to God even in exile.

* Daniel can be trusted to tell the truth because he never compromised his faith, even under pressure from his slavers.

* God is the absolute ruler of all history.

* Israel's slavery is prolonged until a total of four kingdoms (of which Babylon is the first) succeed in dominion over her because God's people have not turned away from their sins. 

* Although there is much suffering in Israel's future, God's Anointed, Christ, will one day come to bring salvation.



\enditems
%\vglue-3cm \leftline{\hskip 8cm<3:1> \hskip 20mm <8:2>; <Est 1:2>}



\nadpis{Author}
The authorship of Daniel is a subject of protracted debate among interpreters.
Many scholars date the book's composition between 170 and 165 B.C., during the reign of Antiochus IV Epiphanes, long after the lifetime of the prophet Daniel (the so-called {Maccabean} dating, cf. the article \"Who was Darius the Mede?"). 
This date, however, is contradicted by the book itself, which indicates that Daniel is its principal author (<9:2>; <10:2>) and that it was written shortly after the conquest of Babylon by Cyros %\x/Cyrus/ 
in 539 B.C. Furthermore, Christ himself explicitly links the book to the prophet Daniel (<Matt 24:15>).

\nadpis{Time and place of origin}
 
The dispute over the dating of the book of Daniel involves three basic issues:

\begitems\style n
* the nature of the prophecy,
* the alleged historical errors in Daniel, and 
* the linguistic features of the Hebrew and Aramaic in the book.
\enditems

Generally speaking, Israel's prophets were primarily concerned with religious and social circumstances affecting themselves and their peers. When the prophets predicted the future, it usually concerned near future events.
For this reason, some interpreters are of the opinion that Daniel's vision concerning \uv{king of the north} and \uv{king of the south} (\<11:2-12:3>) is too detailed to have been written by Daniel, who lived some 200--300 years before the events depicted in the prophecy.

However, this position denies the supernatural nature of the prophecy, as is the case with the occasional practices of other prophets (e.g., <1Kgs 13:2>; <Isa 44:28>). Although the passage <Dan 11:2-12:3> is unusual, it is certainly not impossible that Daniel knew such details; after all, it was to him that God revealed secrets as to no one else (cf. e.g. <2:19-23>).  

%\renum Da 5:31 = CSP 6:1-1
%\renum Da 5:31 = CEP 6:1-1
%\renum Da 5:31 = B21 6:1-1
%\renum Da 5:31 = SNC 6:1-1 

%\renum Da 6:1 = CSP 6:2-29
%\renum Da 6:1 = CEP 6:2-29
%\renum Da 6:1 = B21 6:2-29
%\renum Da 6:1 = SNC 6:2-29

Some advocates of late dating argue historical inaccuracies attributed to the book by Daniel.
They question Belshazzar's relationship to Nebuchadnezzar % \x/Nabuchodonosor/ 
(see <"note" 5:2>n), as well as the identity of Darius the Mede (see <"note on" 6:1>n). 

In addition, they identify the four kingdoms foretold by Daniel (chs. 2; 7) as Babylon, Medea, Persia, and Greece (including the Seleucids and Ptolemies). However, this identification is problematic because there is no historical evidence for an independent Mede kingdom in the interval between the kingdoms of Babylon and Persia.
The Persian king Cyros (550--530 BC) conquered Medea in 549 BC and Babylon in 539 BC (see notes <5:1>n and <5:31>n).

Advocates of the early dating of the book understand the four kingdoms sequence to predict Babylon, Medo-Persia, Greece, and Rome. 
This view is supported by the allusion to the \uv{Medes and Persians} in <5:28>, which shows that the author considered both nations to be parts of one kingdom.





%\Citehere 3 (\kern-2mm) {
%   Člověk přišel na svět proto, aby tady byl, pracoval a žil. Jen moudrý se snaží náš svět postrčit dál, posunout výš. A jen vůl mu v~tom brání.
%   \quotedby {Jan Werich}
%}

Supporters of the late date argue that several terms borrowed from Greek to refer to musical instruments occur in the text (see <"note" 3:5>n), as do late Hebrew and Aramaic terms (see <"note" 2:4>n).
None of these arguments, however, is convincing.
There is abundant evidence of contact between the Greeks and the peoples of the Near East before the time of Alexander the Great. These are quite sufficient to explain the existence of a minimal number of words taken from Greek before Alexander's conquest. 

The original names of musical instruments commonly accompany their bearers without a corresponding equivalent in the local language; compare today's Czech untranslated terminology associated with musical instruments:  \uv{gibson}, \uv{jumbo}, \uv{stratocaster}, \uv{telecaster}, \uv{Les Paul}, \uv{stage piano}, \uv{hohner}, \uv{humbucker}, \uv{single-coil}, etc.
On the contrary: Proponents of Maccabean dating have trouble explaining the complete absence of terms adopted from Greek, {{\it outside\/}} musical terminology. If the book had been written under Greek rule, commercial, military, political, administrative, etc. terminology would have been rife with Greek terms. But there is nothing of the sort in the book.

The Aramaic and Hebrew of the book of Daniel can be dated anywhere between the late sixth and early second centuries B.C. In other words, the linguistic evidence does not give much weight to either aspect: neither late nor early dating.

The argument for a second-century B.C. date is at odds with the biblical claim regarding the date and authorship of the book of Daniel, and the late dating does not demonstrate the late dating convincingly enough.   A date shortly after 539 B.C. (see <1:21>) best fits the nature of the prophecy, the historical dates, and the language of the text.

 \nadpis{Purpose and Distinctiveness}
 
Daniel contains two different types of material.
 In the first six chapters there are six historical narratives; in the second half (chapters 7--12) there are four visions, almost exclusively predictive. Among the six narratives of the first half, chapter 2 stands out because it also contains a prediction. 

An examination of the content of the historical narratives shows that they are independent wholes, pieced together with a purpose.
The narrative offers neither a history of Israel under Babylonian or Persian rule nor a biographical account of Daniel and his friends. It has two main emphases.

On the one hand, the stories show how God's absolute sovereignty extends into the affairs of all nations 
(<2:47>; <3:17-18>; <4:28-37>; <5:18-31> <6:25-28>).
Jerusalem was in ruins, God's people in captivity, and wicked rulers seemed to triumph, but God remains sovereign.
According to his unwavering will, he enters among the kingdoms of this world to establish a universal kingdom of which there will never be an end.

Although all nations have believed that deities are territorial, that they have power only over the territory where their people dwell (and because they want to rule the whole world, their people must conquer other territories for themselves and establish their religion there), the experience of the Israelites in captivity shows that their Lord is not limited in any way, not even territorially; He is Lord over the whole earth, including the deities of other nations. And that he does not abandon his people wherever they go. Sometime around that time, the idea began to emerge that since the sacred is not a place on earth where God dwells apart from other places, then time will be sacred. And the holidays on the calendar began to take on importance. 


\Citehere 3 (\kern-2mm) {
%Lidé nikdy nepáchají zlo tak důsledně a vášnivě,
%jako když to dělají z náboženského přesvědčení.
%Men never do evil so completely and cheerfully
%as when they do it from religious convictions.
One never does evil so fully and gaily, 
as when one does it through a false principle of conscience.
\quotedby {Blaise Pascal}
}
The visions of chapters 7--12 contain predictions of future times during which the truth of the narrative will become more important to God's people.
Although the Israelites suffered under the rule of both the Babylonians and the Persians, they did not suffer any widespread and systematic attack on their faith. This did not occur until Antiochus IV Epiphanes, ruler over the Seleucid empire between 175--164 B.C., sought to eradicate the religion of the Jews and force them to conform to Greek religious practices. Many Jews obeyed him, but others resisted and suffered adversity. 
One of the main reasons for writing the book of Daniel is to prepare God's people for the time of Antiochus IV Epiphanes and to encourage perseverance in those who would live through the coming times of persecution.


The book also looks beyond the time of Antiochus IV Epiphanes to the coming of Christ who will one day destroy all human empires and establish His eternal kingdom of righteousness and peace. All of these events are in view in the prophecies of Daniel.
The book has served as a powerful encouragement to God's people suffering oppression and continues to be an inspiration to persecuted believers today. 

\nadpis{Christ in Daniel}


Daniel's focus on the restoration of Israel after the exile turns the attention to Jesus quite directly.
Like some other prophets, Daniel predicted a glorious future for God's people, the fulfillment of which the New Testament 
is linked to the first and second coming of Christ, as well as to the whole history of the Church.

While much controversy surrounds the details of the fulfillment of Daniel's visions, the basic structure of Daniel's visions leaves no one in doubt that Christ is the fulfillment of the prophet's hopes.
This is most clearly seen in the way Jesus refers to Himself as the \uv{Son of Man} (e.g., <Matt 9:6>; <10:23>; <12:8>).
Daniel used the term in the sense of God's exalted Davidic king, representing God on earth.
Jesus, the Messiah, is the ultimate Davidic King; only he fulfills the predictions of the Son of Man in Daniel's visions (see \<"notes on" 7:13>n and \<7:14>n; see the theological article 
<"Kingdom of God" Mt 4>a). 

In addition, Daniel learned in chapter 9 that Jeremiah's prediction of 70 years of exile would be extended 
to \uv{seventy weeks} years (\<9:24>), or about 490 years.
This prediction reaches its initial fulfillment at Christ's first coming. The delay corresponds to the series of four foreign empires that will oppress God's people (\<2:1-49>) and to the rock that became \uv{a great mountain that filled the whole earth} (\<2:35>), which Daniel refers to as \uv{a kingdom that will not be destroyed} (\<2:44>). 
This is the kingdom of Christ, which was inaugurated by His first coming, continues and grows to this day, and will reach its consummation at Christ's glorious return (see the theological articles <"The Kingdom of God"  Mt 4> and <"The Plan of the Ages"  Heb 7>.)

Daniel foresaw other, even more concrete events that have come to the fore again in the New Testament.
For example, Jesus refers to Daniel's prediction of \uv{exact abomination} (see \<"note on" 9:27>n; \<11:31>n; \<12:11>n),
which originally pointed to the desecration of the temple by Antiochus IV Epiphanes of Greece (see Introduction: Intent and Peculiarities) as a foreshadowing of the destruction of the temple by the Roman general Titus in 70 CE (see <"notes on" Mt 24:15>n and <Mk 13:14>n).

Most Christians associate this typology with Antichrist, whose spirit is already at work in the world (see <"notes on" 1Jn 2:18>n) and will appear in fullness, apparently as a specific person, near Christ's return (see <"notes on" 2Te 2:3>n).

%\endinput
%Chtělo by to možná něco jako \ww, ale aby bylo možné jich psát několik do stejného odstavce, byť s jinými frázemi. Nebudou se vyhledávat v textu, ale musejí přepínat mezi verzemi.
%Viz např. předposlení odstavec výše, začínající slovem Prodloužení: Slovní spojení v \uv{uvozovkách} by chtěla variovat podle překladů. 
%asi \vdef

%Pak za takovýmto Úvodem bude ještě muset následovat Osnova, tu zatím nemám, vydá možná na půl stránky.

\Outline

\rightnote{The stories of Daniel and his friends illustrate both their loyalty to God and his supremacy over all nations.}


\begitems
* Narration (\<1:1-6:28>)

  \begitems
  * The loyalty of Daniel and his friends (<1:1-21>)
  * (<2:1-49>)
  * Deliverance from the fiery furnace (<3:1-30>)
  *  Nebuchadnezzar's second dream (<4:1-37>)
  * The Judgment of Balsazar (<5:1-31>)
  * Deliverance from the lion's den (<6:1-28>)
  \enditems

\rightnote{Daniel's visions of the future of God's people, looking back to the long after the end of the exile.
          God revealed to Daniel that the four great kingdoms
          would dominate and persecute Israel. At the time of the fourth of these, God will set up His kingdom, of which there will be no end.}
* Vision (<7:1-12:13>)

  \begitems
  * Vision of the Four beasts (<7:1-28>)
  * The vision of the ram and the goat (<8:1-27>).
  * Vision of the seventy weeks (<9:1-27>)
  * Vision of the future of God's people \nl (<10:1-12:13>)
    \begitems
    * The angel's message to Daniel \nl (<10:1-11:1>)
    * From Daniel to Antiochus IV Epiphanes \nl (<11:21-12:3>)
    * Final message to Daniel \nl (<12:4-13>)
    \enditems
  \enditems
\enditems

