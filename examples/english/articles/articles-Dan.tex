
\CommentedBook{Dan}

\Article [6] % Kdo byl Darius Médský?

%%\Cite A %Dáme to do někam textu místo Wericha, ať to můžu ukazovat
%%{Daniel 3:16\break
%%   Šadrach, Méšak a Abed-nego odpověděi králi:
%%   \uv{Nebúkadnesare, nám není třeba dávat ti \hbox{odpověď}.}
%%}
%{Druhořadých věcí nedosáhneme tím, že je prohlásíme za prvořadé. Druhořadých věcí dosáhneme pouze tehdy, budou-li na prvním místě věci první.
%\quotedby {C. S. Lewis}
%}

\Cite  A
{%Ani v literatuře a umění nebude nikdo, kdo si zakládá na originalitě, nikdy originální: zatímco když se jenom snažíte říkat pravdu (a nezáleží vám na tom, kolikrát už byla řečena), stanete se v devíti případech z deseti originálními, ani si toho nevšimnete. 
Even in literature and art, no man who bothers about originality will ever be original: whereas if you simply try to tell the truth (without caring twopence how often it has been told before) you will, nine times out of ten, become original without ever having noticed it.
\quotedby {C. S. Lewis}
}


%\swapCites

%Poprvé je zmíněn v \<5:31>. Někteří (hlavně kritičtí, tj. liberální) teologové, zastánci pozdního (tzv. \uv{makabejského}) datování vzniku knihy Daniel (podle nich kolem roku 165 př.Kr.),  tvrdí, že
%(1) Darius Médský 
%\insertCite A\left \
%nikdy neexistoval, protože v jiných starověkých dokumentech není zmíněn; (2) jméno Darius použil neznámý makabejský autor, špatně obeznámený s perskou
%historií, a zaměnil ho s legendárním \x/Dariem/ I. (255--484) Perským (nikoliv Médským); (3) autor se chybně domníval, že Babylón dobyla Médea, nikoliv Persie, a že pod vedením tohoto
%legendárního \uv{\x/Daria/} Médové vládli světovému impériu několik let, než padlo do rukou Peršanům.
%
%Díky tomu mohou advokáti makabejského (pozdního) data tvrdit, že čtyři království z \x/Nabuchodonozor/ova snu (<Da 2>) jsou (1) babylónské; (2) médské; (3) perské a (4) řecké, což
%jim poskytuje výhodu omezení horizontu Danielových proroctví ne dále než do roku 165 př.Kr. (Pokud by kniha vznikla až v této době, všechna údajná \uv{proroctví}  by se dala vysvětlit zpětným pohledem na historické události {\it poté,\/} co nastaly. Problém s tradiční identifikací čtvrtého království coby Říma pro liberály spočívá v tom, že takový pohled předpokládá skutečné pravé prediktivní proroctví, což racionalistický vyšší kriticismus zásadně nepřipouští.)
%Udržitelnost hypotézy makabejského data proto závisí na výše uvedeném vysvětlení \uv{\x/Daria/ Médského} (protože podle tohoto vysvětlení existuje médské království před perským).
%Proto je tato postava velmi důležitá; její identifikace má závažné teologické důsledky.

%Peršan Darius I., syn Hystapesův, však nemůže být ztotožněn  s \x/Dariem/ Médským hned z několika důvodů:

He is first mentioned in \<5:31>. Some (mainly critical, i.e., liberal) theologians, advocating a late (so-called \uv{Maccabean}) dating of the book of Daniel (according to them around 165 B.C.), claim that
(1) Darius Medes 
\insertCite A\left\
never existed because he is not mentioned in other ancient documents; (2) the name Darius was used by an unknown Maccabean author, poorly acquainted with Persian
history, and confused it with the legendary Darius I (255--484) of Persia (not the Medes); (3) the author erroneously assumed that Babylon was conquered by Medea, not Persia, and that under the leadership of this
legendary \uv{Darius} the Medes ruled the world empire for several years before it fell into the hands of the Persians.

As a result, advocates of the Maccabean (late) date can claim that the four kingdoms of Nebuchadnezzar's dream (<Dan 2>) are (1) Babylonian; (2) Median; (3) Persian; and (4) Greek, which
gives them the advantage of limiting the horizon of Daniel's prophecies to no further than 165 B.C. (If the book was written at this time, all of the alleged \uv{prophecies} could be explained in retrospect by looking back at the historical events {\it after they occurred.\/} The problem with the traditional identification of the fourth kingdom as Rome for liberals is that such a view presupposes actual true predictive prophecy, which rationalist higher criticism fundamentally does not allow.)
\insertCite A\right

The tenability of the Maccabean date hypothesis therefore depends on the above explanation of \uv{Darius the Mede} (since according to this explanation the Median kingdom pre-dates the Persian kingdom).
Therefore, this figure is very important; its identification has serious theological implications.

However, the Persian Darius I, son of Hystapes, cannot be identified with Darius the Mede for several reasons:




\begitems \style n
%* Darius I. byl rodem Peršan, bratranec krále \x/Cýra/; nebyl to v žádném případě Méd.
%\insertCite A\right
%* Darius I. byl mladík kolem dvaceti let, když zavraždil podvodníka Gaumatu (který se vydával za \x/Cýr/ova syna Smerdise) v roce 522 př.Kr.  Nemohlo mu být 62 (\<5:31>). 
%* Darius I. nebyl králem Babylóna před \x/Cýr/em, jak tvrdí liberální teorie. Samostatným vládcem se stal až sedm let po \x/Cýr/ově smrti (srv. <Ezd 4:5>).
%* Taková zmatenost ohledně národnosti a časové posloupnosti \x/Daria/ a \x/Cýra/ byla v helenistickém světě druhého století př.Kr. absolutně nemyslitelná.
%Studenti museli číst Xenofóna,  Hérodota a další řecké historiky pátého a čtvrtého století př.Kr. Od Xenofóna a Hérodota máme informace o \x/Cýr/ovi a \x/Dariov/i.
%Jakýkoliv řecky píšící autor, který by umístil \x/Daria/ před \x/Cýra/, by ukončil svou spisovatelskou kariéru výsměchem veřejnosti; nikdo by ho už nikdy nebral vážně.  
* Darius I was a Persian by birth, a cousin of king Cyrus; he was by no means a Mede.
%\insertCite A\right
* Darius I was a young man of about twenty when he murdered the impostor Gaumata (who claimed to be Cyrus's son Smerdis) in 522 BC.  He could not have been 62 (\<5:31>). 
* Darius I was not king of Babylon before Cyrus as liberal theories claim. He did not become an independent ruler until seven years after the death of Cyrus (cf. <Ezd 4:5>).
* Such confusion about the nationality and chronology of Darius and Cyrus was absolutely unthinkable in the Hellenistic world of the second century BC.
Students must have read Xenophon, Herodotus and other Greek historians of the fifth and fourth centuries B.C. From Xenophon and Herodotus we have information about Cyrus and Darius.
Any Greek writer who placed Darius before Cyrus would have ended his writing career in public ridicule; he would never be taken seriously again.  
\enditems

%Darius Perský (<Ezd 4:5>) a Darius Médský (<Da 5:31>) tedy nemají spolu nic do činění; zmatek je pouze na straně zastánců teorie pozdního data, nikoliv na straně autora knihy Daniel.
%
%Nicméně je pravda, že archeologie dosud neobjevila žádnou zmínku o \uv{\x/Dariov/i Médském} z doby, kdy žil, mimo Bibli.
%(Až do devatenáctého století totéž platilo o Balsazarovi, místokráli, zastupujícím svého otce Nabonida. Kritičtí teologové, zastávající makabejské datování, tvrdili, že
%Balsazar je další fiktivní postava v Danielovi, dokud nebyly objeveny babylónské tabulky z jeho doby,  potvrzující, že Balsazar sloužil jako mladší král v posledních letech vlády svého otce Nabonida. Srv. <"pozn." 5:1>n).
%
%Přesto \x/Daria/ Médského dokážeme identifikovat.
%V knize Daniel je několik náznaků, že Darius nebyl svrchovaným králem, ale že byl dočasně dosazen na trůn nějakou vyšší autoritou.
%Ve verši \<9:1> čteme, že \uv{byl učiněn  králem}. Je zde použit pasivní kořen {\it hofal\/} u slovesného tvaru \uv{homlak} (\Homlak) namísto běžného \uv{malak} (\Malak\ \uv{stal se králem}), používaného v kontextu získání trůnu dobytím nebo dědictvím (např. <1Sa 13:1>).
%Podobně ve verši \<5:31> čteme, že Darius \uv{\x/ujal království/} (\uv{qabbel} \Qabbel), jako kdyby mu bylo svěřeno vyšší autoritou.
%
%Samotné jméno Darius (staropersky {\it Da-ri-ya-(h)u-(ú-)iš\/} \Dariahuuish, hebr. \Dariawush) je zřejmě příbuzné s {\it dara,\/} které se v avestánštině (mrtvý severovýchodní staroíránský jazyk) objevuje jako
%výraz pro \uv{krále}. Podobně jako označení {\it augustus\/} mezi Římany, mohlo být i přízvisko {\it dārayawush\/} (\uv{královský}) zvláštním čestným titulem, který
%mohl sloužit i jako vlastní jméno, podobně jako české příjmení \uv{Král}.

Thus, Darius Persian (<Ezra 4:5>) and DariusMedian (<Dan 5:31>) have nothing to do with each other; the confusion is only on the part of the late date theorists, not on the part of the author of Daniel.

However, it is true that archaeology has not yet discovered any mention of {\it Darius the Medes} from the time he lived, outside the Bible.
(Until the nineteenth century, the same was true of Balsazar, the viceroy representing his father Nabonidus. Critical theologians, advocating Maccabean dating, have argued that
Balsazar was another fictional character in Daniel until Babylonian tablets from his time were discovered confirming that Balsazar served as a junior king in the last years of his father Nabonidus' reign. Cf. <"note on" 5:1>n).

Nevertheless, we can identify Darius the Mede.
There are several indications in the book of Daniel that Darius was not a sovereign king, but was temporarily placed on the throne by some higher authority.
In verse \<9:1>, we read that \uv{was made king}. The passive root {{\it hofal\/}} is used here for the verb form \uv{homlak} (\Homlak) instead of the common \uv{malak} (\Malak \ \ \uv{became king}) used in the context of gaining the throne by conquest or inheritance (e.g., <1Sam 13:1>).
Similarly, in verse \<5:31> we read that Darius \uv{\x/took the kingdom/} (\uv{qabbel} \Qabbel), as if it had been conferred on him by a higher authority.

The very name Darius (Old Persian {\it Da-ri-ya-(h)u-(ú-)ish/} \Dariahuuish, Heb. \Dariawush) is probably related to {\it dara,\/} which appears in Avestan (a dead northeastern ancient Iranian language) as an expression for a king.
Like the appellation {{it augustus}} among the Romans, the surname {{\it dārayawush\/}} (\uv{royal}) may have been a special honorary title that
could also serve as a proper name, like the English surname \uv{King}.

%Zdá se tedy, že záhy po porážce Babylóna médo-perskými vojsky si \x/Cýr/ovu osobní přítomnost vynutila jiná fronta jeho rozpínajícího se impéria. Jevilo se mu tedy jako účelné
%svěřit království Gubarovi-\x/Dariov/i i s titulem Král Babylóna, aby panoval přibližně rok, než se \x/Cýros/  osobně vrátí ke své korunovační slavnosti v Mardukově chrámu.
%Po tomto roce vlády v roli místokrále zůstal Darius správcem Babylóna, ale koruna byla odevzdána jeho nadřízenému vládci \x/Cýr/ovi (který ji později předal svém nejstaršímu synovi Kambýsésovi, srv. \<"pozn." 11:2>n, při korunovaci králem Babylóna).
%
%Tento scénář podporuje text knihy tím, že Daniel nikde neuvádí žádný pozdější rok \x/Dariov/y vlády než \uv{první} (<9:1>), což indikuje její velmi krátké trvání.
%I kdyby to mělo znamenat, že jeden rok patřila vláda Médům (víme, že tomu tak  nebylo; patřila  Peršanu \x/Cýr/ovi), jednoletá říše by sotva mohla uhájit svou legitimní
%pozici coby království číslo dvě v řadě impérií výrazně trvanlivějších: babylónské vydrželo 73 let, perské 208 let, řecké by v roce 165 př.Kr. mělo za sebou 167 své existence. 
%
So it seems that soon after the defeat of Babylon by the Medo-Persian armies, Cyrus personal presence was forced by another front of his expanding empire. It seemed expedient to him, therefore.
to entrust the kingdom to Gubar-Darius with the title of King of Babylon, to rule for about a year before Cyrus personally returned for his coronation ceremony in the temple of Marduk.
After this year's reign as viceroy, Darius remained as governor of Babylon, but the crown was handed over to his superior ruler Cyrus (who later passed it on to his eldest son Cambyses, cf. \<"note on" 11:2>n, at his coronation as king of Babylon).

This scenario is supported by the text of the book in that Daniel nowhere mentions any later year of Darius's reign than \uv{first} (<9:1>), indicating its very short duration.
Even if this were to mean that the one-year reign belonged to the Medes (we know it did not; it belonged to Persian Cyrus), a one-year empire could hardly have defended its legitimate
position as the number two kingdom in a series of empires of considerably greater durability: the Babylonian lasted 73 years, the Persian 208 years, the Greek would have had 167 years of existence by 165 BC. 

%Kromě toho slovní hříčka Danielovy interpretace nápisu na zdi v <5:28>, která spojuje dva významy stejného kořene P--R--S (\PRS): {\it p$^e$r\char238 
%sat\/} (\Persat\ \uv{rozděleno}) a
%{\it pārās} (\LaMedeUParas\ \uv{dáno Médům a Peršanům}), zároveň ujišťuje, že autor knihy psal v přesvědčení, že království číslo jedna (babylónské) přechází pod
%nadvládu Peršanů již spojených s Médy a tím se stává královstvím číslo dvě. Kniha Daniel neponechává žádný prostor pro kritickou spekulaci o dřívějším médském království, které
%údajně mohl mít autor knihy na mysli.

%Čtvrté království je pak Řím, jediné, které dokázalo pokořit Řecko (<2:40>), a během jehož existence vzniklo Božím zásahem království věčné, jemuž nebude konce (<2:44>) -- církev.
%(Srv. graf \ref[danielovyvize] na str. \pgref[danielovyvize]).
In addition, a pun on Daniel's interpretation of the inscription on the wall in <5:28>, which combines two meanings of the same root P--R--S (\PRS): {\it p$^e$r\char238 sat/} (\Persat \ \ \uv{divided}) and {\it pārās} {\LaMedeUParas} (\uv{given to the Medes and Persians}), while assuring that the author of the book wrote in the belief that kingdom number one (Babylonian) would pass under
to the rule of the Persians already allied with the Medes and thus becomes kingdom number two. The book of Daniel leaves no room for critical speculation about the earlier Median kingdom, which
the author of the book may have had in mind.

The fourth kingdom, then, is Rome, the only one that has been able to subdue Greece (<2:40>), and during whose existence the eternal kingdom of which there will be no end (<2:44>) -- the church -- has come into being by divine intervention.
(Cf. chart \ref[danielovyvize] on p. \pgref[danielovyvize]).

\endinput


