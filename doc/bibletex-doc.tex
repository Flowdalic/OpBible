\cslang
\fontfam[schola]
\def\TODO{{\Red TODO}}
\verbchar`
\everytt{\typosize[9/]\catcode`<=13 }
\everyintt{\catcode`<=13 \Magenta}
\catcode`<=13 \def<#1>{{$\langle$\it#1$\rangle$}}

\tit OP-Bible -- zpracování komentovaných biblí \TeX/em

OP-Bible je sada nástrojů (maker a dalšího podpůrného softwaru) na
zpracování biblických textů. Hlavním cílem je umožnit navázat na z internetu
stažené úplné texty Bible a propojit je s doplněnými poznámkami podle
požadovaných kritérií popsaných níže. Výsledkem po zpracování \TeX/em je
komentovaná Bible, tj. Bible vytištěná společně s poznámkami. PDF soubor je
navíc bohatě prolinkovaný hypertextovými odkazy.

Úplný text Bible je na internetu k dispozici v různých překladech a české
překlady jsou nejméně čtyři. Je možné si stáhnout třeba jen jeden překlad
(dále mu říkáme \uv{základní text}) a doplňovat jej poznámkami, které
zapisujeme do přidružených souborů tak, že není nutná žádná editace či jiný
zásah do základního textu. Je též možné poznámky psát tak, že jsou víceméně
univerzální například pro kterýkoli z českých překladů, takže pozměněním jen
výchozí konfigurace v hlavním souboru zpracování \TeX/ načte jiný základní
text se stejnými poznámkami, které se navíc automaticky modifikují podle
vyjadřovacích zvyklostí toho kterého překladu.

Pro zpracování \TeX/em je tedy třeba mít odpovídající verzi základního textu
(předzpracovanou způsobem, popsaným v sekci~\ref[zakladni]) a dále soubory s
poznámkami (jejich syntaxe je uvedena v sekcích \ref[note] a \ref[glosy]).
\TeX/ přečte hlavní soubor se základními pokyny, pak načte data všech
potřebných souborů a vytvoří PDF knihy -- komentované Bible. V ní je jednak
(typicky větším písmem) základní text a (typicky menším písmem, například
pod čarou a třeba ve dvou sloupcích) poznámky k textu. Je respektován
základní požadavek, že k jednotlivým frázím základního textu jsou poznámky
umístěny na stejné stránce jako samotné fráze základního textu. O rozmístění
objektů na stránkách se \TeX/ stará víceméně automaticky podle nastavených
typografických pravidel, které se dají modifikovat dle potřeby. Požadavky na
typografickou úpravu nebo další specifické požadavky, které se dají
algoritmicky popsat, jsou implementovány makry \TeX/u a \TeX/ je během
zpracování přečte a automaticky provede.

Veškerá data, která \TeX/ zpracovává (konfigurace, základní text, texty
poznámek, maker) mají textový formát, tj. dají se prohlížet a případně
upravovat běžným textovým editorem.


\sec[zakladni] Formát základního textu

Předpokládá se, že základní text Bible je uložen v souborech s příponou `.txs`
(text source). Každá z~66~knih Bible je uložena ve svém `.txs` souboru.
Názvy `.txs` souborů je možné deklarovat pomocí příkazu `\def\txsfile` v
hlavním souboru (viz~sekci~\ref[hlavni]). Název `.txs` souboru by měl obsahovat
jednak značku použitého překladu Bible a dále zkratku knihy. Například
`CzeBKR-Da.txs` obsahuje základní text překladu Bible Kralické, knihy
Daniel.

Každý řádek `.txs` souboru obsahuje jeden verš Bible uvozený 
`#<číslo-kapitoly>:<číslo-verše>`. Verše musejí být uvedeny ve správném
pořadí. Například začátek souboru `CzeBKR-Da.txs` vypadá takto (části textu
jsou v ukázce vynechány):

\begtt
#1:1 Léta třetího kralování Joakima krále Judského, ... a oblehl jej.
#1:2 I vydal Pán v ruku jeho Joakima ... do domu pokladu boha svého.
...
\endtt

Základní texty Bible lze získat například z modulů Swordu
\url{https://www.crosswire.org/sword/modules/ModDisp.jsp?modType=Bibles}.
Jednotlivé `.txs` soubory je pak možné vygenerovat následujícím postupem.

ZIP stažený z uvedené www stránky (tzv modul) někam rozbalte. Je třeba
mít dále v počítači instalován balíček `libsword-dev` a dále program
`mod2tex`, který je součástí OP-Bible. Pomocí `installmgr -l` zjistíte
seznam stažených modulů. Máte-li nastaven aktuální adresář v místě, kam jste
rozbalili ZIPy a vznikl tam adresář `modules`, pak jsou takové moduly
nalezeny. Moduly obsahují texty v binárním formátu, my je potřebujeme
převíst do textového formátu. K tomu stačí napsat do příkazového řádku: 
\begtt 
mod2tex modul > soubor
\endtt 
kde `modul` je název modulu. Ve výsledném souboru
máte kompletní základní text daného překladu (modulu).
Například po
\begtt
mod2tex CzeBKR > CzeBKR.out
\endtt
je v souboru `CzeBKR.out` kompletní překlad Bible Kralické.
Ten nyní můžete rozdělit do `.txs` souborů příkazem
\begtt
maketxs CzeBKR.out
\endtt
Uvedený příkaz kromě 66 `.txs` souborů vytvoří soubor `CzeBKR-books.tex`, ve
kterém jsou tituly a zkratky jednotlivých knih, tedy je tam:
\begtt
\BookTitle Gen Gen {Genesis}
\BookTitle Exod Exod {Exodus}
\BookTitle Lev Lev {Leviticus}
\BookTitle Num Num {Numbers}
\BookTitle Deut Deut {Deuteronomy}
\BookTitle Josh Josh {Joshua}
\BookTitle Judg Judg {Judges}
...
\endtt
Tyto tituly nejsou v češtině, protože je zdroj Sword neobsahuje. Je tedy
třeba tento soubor manuálně upravit a vložit místo anglických české názvy.
Zkratky knih jsou uvedeny dvakrát stejně. První ponechte a druhou zkratku 
můžete pozměnit podle zvyklostí českého jazyka, viz též sekci~\ref[hlavni].

Jednotlivé `.txs` soubory můžete vložit do
separátního adresáře projektu, například s názvem `sources/`. 
Odtud si je \TeX/ bude číst, pokud mu jejich umístění oznámíte v konfiguraci
v hlavním souboru, viz sekci~\ref[hlavni]. Se soubory nadále nebudete
potřebovat nijak manipulovat, není potřebné je editovat ani doplňovat. V
adresáři `sources/` tedy můžete mít \uv{datový sklad} všech základních textů
Bible pro všechny použité překlady najednou.


\sec[note] Formát poznámek s vazbou na fráze v základním textu

Hlavním smyslem nástroje OP-Bible je vytvořit PDF ze základního textu nejen
se samotným textem Bible, ale s propojenými poznámkami. K tomu slouží
mimo jiné soubory `notes-*.tex` (například `notes-Ge.tex` pro knihu Genesis), ve
kterých jsou zapsány poznámky k jednotlivým frázím základního textu dle
následujícího smluveného formátu. Jednotlivá poznámka má tvar:

\begtt
\Note <číslo-kapitoly>:<číslo-knihy> {<fráze>} <text-poznámky> 
<prázdný řádek>
\endtt
Například:
\begtt
\Note 1:2 {nádobí} Odkaz na nádobí z vypleněného chrámu, nikoliv na deportaci zajatců.

\endtt
Příklad je ze souboru `notes-Da.tex`, tedy ze souboru s poznámkami ke knize
Daniel. Konkrétně se poznámka váže na kapitolu první, verš druhý a na frázi
\uv{nádobí}. Tato fráze musí být bezpodmínečně ve stanoveném verši
základního textu obsažena. Pak ji \TeX/ propojí s odpovídajícím místem v
základním textu, tj. zajistí, že výskyt fráze i samotné poznámky je na
společné stránce. 

Povšimněme si, že fráze se váže na překlad Bible Kralické, zatímco v
překladu ekumenickém je použit ve stejném verši termín \uv{nádob}. Jak
zařídit, aby se stejná poznámka automaticky mírně modifikovala podle
použitého překladu, je popsáno v sekci~\ref[ruzne-preklady].

Pokud <fráze> není ve stanoveném verši základního textu doslovně obsažena,
\TeX/ během zpracování oznámí do logu a a na terminál o tom varování a
poznámku zařadí k danému verši, jakoby <fráze> byla na začátku verše..

Vytištěná poznámka (v závislosti na typografickém nastavení) obsahuje
například zopakování čísla kapitoly a verše, pak následuje komentovaná fráze
(například tučně) a za ní je vlastní text poznámky.

Někdy je třeba v základním textu vyhledat mírně jinou frázi, než jakou pak
chceme zapsat do vytištěné poznámky (např. v základním textu je slovo v
jiném pádě nebo je to mírně jinak formulovaná fráze). Pak je možné těsně za
`{<fráze>}` napsat rovnítko následované `{<fráze-k-tisku>}`, tedy

\begtt
\Note <číslo-kapitoly>:<číslo-knihy> {<fráze>}={<fráze k tisku>} <text-poznámky> 
<prázdný řádek>
\endtt
V takovém případě se v základním textu hledá <fráze>, ale ve vlastní
poznámce se vytiskne jako heslo poznámky <fráze-k-tisku>. Například:
\begtt
\Note 1:2 {vydal Pán}={Pán vydal} Porážka Izraele Babylónem není vysvětlitelná
   jen pouhou vojenskou a politickou analýzou oné doby. Bůh vždy jednal svrchovaně 
   v záležitostech národů. Babylóňany použil jako nástroj potrestání svého vlastního
   lidu za porušení smluvních závazků.

\endtt
V tomto příkladě se hledá ve verši druhém základního textu \uv{vydal Pán} 
ale v poznámce je vytištěno heslo \uv{Pán vydal}.

Povšimněte si, že tato poznámka se rovněž váže na verš druhý kapitoly první
knihy Daniel (protože je v souboru `notes-Da.tex`). Je tedy možné mít více
poznámek k různým frázím, které se váží na stejný biblický verš. 

Jednotlivé poznámky ve zdrojovém souboru poznámek jsou odděleny prázdnými
řádky. To je nutné, jinak by \TeX/ při jejich čtení nepoznal, kde končí text
poznámky. Také se tím zvyšuje přehlednost zdrojového souboru. Další řádky
poznámky můžete (ale nemusíte) odsadit. Obecná pravidla pro zápis
zdrojových textů pro \TeX/ nejsou předmětem tohoto manuálu, najdete je v
různých učebnicích k~\TeX/u.

Pokud se poznámka váže k celému verši (tj. bez specifikované fráze), pište
`{}`, tedy prázdnou frázi k vyhledání. Například:
\begtt
\Note 1:1-21 {}={Uchování rituální čistoty}  Prorok uvozuje kontext své knihy vyprávěním
   osobní historie (své i svých přátel) zajetí, vzdělání, věrnosti Bohu a služby 
   králi Nabuchodonozorovi.
\endtt
Tato ukázka navíc demonstruje, že je také možné uvést celý rozsah čísel veršů.
Je-li <fráze> pro vyhledání prázdná (jako v tomto příkladě), pak se poznámka
umístí na stejnou stranu, kde je začátek prvního verše z uvedeného rozsahu.
Je-li <fráze> neprázdná, musí se vyskytovat v prvním verši z uvedeného
rozsahu veršů. Rozsah veršů bude stejně vytištěn v úvodu poznámky.
Symbol pro rozsah „`-`“ je jediný znak \uv{mínus} běžně dosažitelný na
klávesnici. Nesmí to být žádný speciální znak vypadající třeba podobně jako
vodorovná čárka.



\sec[glosy] Formát dalších vložených poznámek (glosy, články)

\TODO...

\sec[fmt] Údaje určující formátování základního textu

Základní text stažený z internetu a generovaný programem `mod2tex` neobsahuje
žádné formátovací ani doplňkové údaje, jakými jsou například názvy kapitol
nebo místa, kde se má ukončit odstavec či přejít z formátování v bloku do
formátování na střed řádků a zpět.

Potože do základního textu nechceme nijak zasahovat, je třeba tyto doplňkové
údaje deklarovat k příslušným veršům pomocí speciálních příkazů `\fmtadd`,
`\fmtpre` a `\fmtins`. Tyto příkazy jsou typicky v souborech `fmt-*.tex`,
například `fmt-BKR-Da.tex`. Je vhodné udržovat tyto formátovací soubory
závislé jednak na knize (Daniel v uvedené příkladu), ale také na použitém
překladu (Bible Kralická v příkladu). Je sice možné začít s jedním souborem
pro každou knihu a soubory pro ostatní překlady pořídit jako kopie výchozích,
ale nakonec bude asi potřebné pro rozličné varianty základního textu
formátovací pokyny mírně modifikovat v souladu s použitým překladem.

Syntaxe použití uvedených příkazů je následující:
\begtt
\FormatedBook{<zkratka-knihy>}
\fmtpre{<číslo-kapitoly>:<číslo-verše>}{<fmt-pokyn>}
\fmtadd{<číslo-kapitoly>:<číslo-verše>}{<fmt-pokyn>}
\fmtins{<číslo-kapitoly>:<číslo-verše>}{<fráze>}{<fmt-pokyn>}
\endtt
Na začátku souboru nebo skupiny
příkazů `\fmt*` je třeba nejprve deklarovat pomocí `\FormatedBook`, které
biblické knihy se formátování týká, protože v prvním parametru příkazů `\fmt*` je už
jen uvedeno číslo kapitoly a číslo verše. 

<fmt-pokyn> je \uv{vzkaz}, který je předán \TeX/u k formátování.
Například `\endgraf` značí konec odstavce. `\begcenter` otvírá pasáž s
centrovaným textem a ta musí být uzavřena pomocí `\endcenter`. Nebo 
`\chaptit{<text>}` vloží <text> jako titul kapitoly.

Příkaz `\fmtpre` vkládá <fmt-pokyn>
na začátek uvedeného verše (ještě před případně vytištěné číslo verše horním
indexem). Příkaz `\fmtadd` vkládá <fmt-pokyn> na konec uvedeného verše.
Konečně `\fmtins` vkládá <fmt-pokyn> dovnitř verše za první výskyt stanovené
<fráze>, která ve verši musí doslovně existovat. Jinak \TeX/ vypíše varování
a <fmt-pokyn> nezařadí vůbec.

Jak může vypadat použití příkazů `\fmt*` je vidět například v souboru
`fmt-BKR-Da.tex`.


\sec[ruzne-preklady] Různé (ale podobné) verze základního textu

Existuje například více českých překladů Bible, které používají různé zvyklosti pro
vyjádření některých frází. Máme tedy různé varianty základního textu a
chceme k nim připojit společný poznámkový aparát, ve kterém se budou
označené fráze automaticky měnit podle zvoleného překladu. K tomu slouží příkazy
popsané v této sekci.

Nejsou-li varianty deklarovány příkazem `\vdef` (viz níže), pak
příkaz `\x/<fráze>/` použitý v textu vypíše <fráze>. Je ovšem možné
deklarovat varianty překladu pomocí 

\begtt
\variants <počet-variant> {<varianta-A>} {<varianta-B>} {<varianta-C>} {<varianta-D>}
\endtt
(zde `<varianta-D>` nemusí být poslední, počet parametrů je určen hodnotou
`<počet-variant>`). Tyto parametry jsou značkami pro konkrétní překlad, např.
`CzeBKR` pro překlad podle Bible Kralické nebo `CzeCEP` pro ekumenický překlad.

Následně je možné definovat varianty frází pro jednotlivé varianty překladu:
\begtt
\vdef {<fráze-A>} {<fráze-B>} {<fráze-C>} {<fráze-D>}
\endtt
Zde musí být stejný počet parametrů pro fráze jako je `<počet-variant>` uvedený v
předchozím příkazu `\variants`.
Pokud je nyní v hlavním souboru například uvedeno:
\begtt
\def\tmark {<varianta-C>}
\endtt
pak `\x/<fráze-A>/` se promění na <fráze-C>.

Je-li ovšem v hlavním souboru
definováno `\def\tmark {<varianta-A>}`, pak `\x/<fráze-A>/` vypíše
`<fráze-A>`, jako kdyby žádné varianty překladu nebyly deklarovány.

Ilustrujme si toto na příkladu. Předpokládejme, že máme soubor `vars.tex`,
který je čten z hlavního souboru pomocí `\input` a soubor obsahuje:

\begtt
\variants 3 {CzeBKR}         {CzeCEP}       {CzeCSP}
\vdef       {Baltazar}       {Beltšasar}    {Beltšasar}
\vdef       {Nabuchodonozor} {Nebúkadnesar} {Nebúkadnesar}
\vdef       {sedm let}       {sedm let}     {sedm časů}
\endtt

Je-li v hlavním souboru `\def\tmark{CzeBKR}`, pak
\begtt
\x/Baltazar/        vypíše Baltazar
\x/Nabuchodonozor/  vypíše Nabuchodonozor
\x/sedm let/        vypíše sedm let
\x/cokoli/          vypíše cokoli
\endtt

Je-li ale v hlavním souboru `\def\tmark{CzeCEP}`, pak
\begtt
\x/Baltazar/        vypíše Beltšasar
\x/Nabuchodonozor/  vypíše Nebúkadnesar
\x/sedm let/        vypíše sedm let
\x/cokoli/          vypíše cokoli a na terminálu bude varování,
                                  že fráze /cokoli/ nemá deklarován překlad.
\endtt
Například poznámka v knize Daniel uvedená
v~sekci~\ref[note] by mohla znít:
\begtt
\Note 1:1-21 {}={Uchování rituální čistoty}  Prorok uvozuje kontext své knihy vyprávěním
   osobní historie (své i svých přátel) zajetí, vzdělání, věrnosti Bohu a služby 
   králi \x/Nabuchodonozor/ovi.
\endtt
a tato poznámka při `\def\tmark{CzeCEP}` na svém konci zní
\uv{\dots a služby králi Nebúkadnesarovi}, zatímco při
`\def\tmark{CzeBKR}` zní \uv{\dots a služby králi Nabuchodonozorovi}.

Kromě `\vdef`, což deklaruje varianty překladu platné pro celou Bibli je
možné použít `\wdef`, což deklaruje varianty překladu jen pro konkrétní verš
konkrétní knihy a jen pro spárování poznámek `\Note` s výchozím textem.
Syntaxe je následující:
\begtt
\CommentedBook {<zkratka-knihy>}
\wdef <číslo-kapitoly>:<číslo-verše> {<fráze-A>} {<fráze-B>} {<fráze-C>}
\endtt
kde počet parametrů pro fráze se musí shodovat s hodnotou `<počet-variant>` v
příkazu `\variants`.

Pro každý příkaz `\Note`:
\begtt
\Note <číslo-kapitoly>:<číslo-verše> {<fráze>} <text-poznámky>
\endtt
se před vyhledáním `<fráze>` ve stanoveném verši tato `<fráze>`
automaticky promění podle aktuální varianty `\tmark` překladu následujícím způsobem:
\begitems
* Je-li `<fráze>` deklarována jako `<fráze-A>` pomocí `\wdef` pro konkrétní
  verš, použije se místo ní odpovídající variantní fráze, jinak
* je-li `<fráze>` deklarována jako `<fráze-A>` pomocí `\vdef`,
  použije se místo ní odpovídající variantní fráze bez ohledu na knihu a číslo verše,
* jinak se použije pro vyhledání v textu přímo `<fráze>`.
\enditems

Variantní fráze deklarované pomocí `\wdef` mají také rozšířenou syntaxi: může
těsně následovat rovnítko a `{<fráze-pro-tisk>}`. Takže například na poznámku
`\Note` uvedenou sekci~\ref[note] může navazovat deklarace `\wdef` takto:
\begtt
\CommentedBook {Da}
\wdef 1:2 {vydal Pán} {Hospodin mu vydal}={Hospodin vydal} {Hospodin vydal}
\endtt
Tento příklad funguje následovně: Při zpracování první varianty základního textu
(překlad CzeBKR v~našem příkladě) se vyhledává \uv{vydal Pán} a tiskne se \uv{Pán vydal}
(odlišná fráze pro tisk je přímo v parametru `\Note`). Při
zpracování druhé varianty základního textu se vyhledá \uv{Hospodin mu vydal}
a vytiskne se v poznámce \uv{Hospodin vydal} a konečně při zpracování třetí
varianty se vyhledá \uv{Hospodin vydal} a vytiskne se též 
\uv{Hospodin vydal}, protože třetí varianta nemá těsně následované rovnítko
s odlišnou frází pro tisk.  

Je-li `<fráze-pro-tisk>` deklarována jednak v parametru `\Note` a jednak
pomocí `\wdef`, má přednost údaj z `\wdef`.

Některé varianty překladu mají jiné číslování veršů. V takovém případě je
možné použít příkaz `\renum` takto:

\begtt
\renum <zkratka-knihy> <výchozí-č-kap>:<výchozí-č-verše> = <překlad> <č-kap>:<č-od>-<č-do>
\endtt
kde `<překladu>` je značka pro konkrétní překlad.
Místo `<výchozí-č-kap>:<výchozí-č-verše>` se použije při `\def\tmark{<překlad>}`
ve skutečnosti `<č-kap>:<č-od>`.
Takové přečíslování se netýká jen
tohoto verše, ale celého úseku veršů vymezeného rozsahem `<č-od>-<č-do>`.

Například:
\begtt
\renum Da 4:16 = CzeCEP 4:13-34
\endtt
znamená, že `\Note 4:16 {sedm let} Sedm období neurčené ...` je poznámka,
která ve výchozím překladu <varianta-A> (CzeBKR v našem předchozím příkladě)
se váže na text `sedm let`, který je umístěn skutečně ve verši Da~4:16.
Když ale je `\def\tmark{CzeCEP}`, pak uvedená poznámka `\Note 4:16`
se váže na verš Da~4:13 tohoto překladu, ve kterém je vyhledána fráze
`sedm let`. A k přečíslování dojde i pro následující verše, takže
4:16 dle BKR je 4:13 dle CEP, 4:17 dle BKR je 4:14 dle CEP, atd., až
4:37 dle BKR je 4:34 dle CEP.

Má-li se přečíslování týkat jediného verše, je třeba uvést shodné
`<č-od>` i `<č-do>`, tedy například
\begtt
\renum Da 4:16 = CzeCEP 4:13-13
\endtt

\sec[odkazy] Metody vytváření hyperlinkových odkazů

Odkazy na verše se píší
mezi znaky \code{<} a \code{>}. Text mezi těmito znaky se tiskne jako
aktivní odkaz na konkrétní verš, ale musí mít předepsaný formát, aby
\TeX/ poznal, na jaký verš má propojit hypertextový odkaz.
Specifikace odkazu má tyto možnosti:

\begitems
* "<slovo>" <údaj> ... vytiskne se aktivní <slovo> <údaj>
* <údaj> ... vytiskne se aktivní <údaj>
\enditems
%
`<údaj>` může být zapsán jako
`<kniha> <kapitola>:<verš>`. Zde `<kniha>` je zkratka knihy, <kapitola>
je číslo kapitoly a <verš> může obsahovat číslo jediného verše
nebo úsek veršů ve tvaru `<od>-<do>`. Je-li uveden úsek veršů, pak link
odkazuje na první verš úseku.

Údaj nemusí obsahovat `<kniha>`, pak link odkazuje na verš aktuální knihy.
Neobsahuje-li údaj `<kniha>`, pak nemusí obsahovat ani `<kapitola>:`, link
odkazuje na verš aktuální knihy a kapitoly.

Za uzavíracím znakem \code{>} může těsně následovat písmeno `n`, které určí, že
link neodkazuje na verš, ale na poznámku k verši.

Příklady:

\begtt \catcode`<=12
<13> vytiskne se 13 jako odkaz na verš 13 aktuální knihy a kapitoly
<"verš" 13> vytiskne se "verš~13" jako odkaz na verš 13
<"konec verše" Gn 2:11> vytiskne se "konec verše~Gn~2:11" a odkazuje na Gn 2:11.
<"ponznámku" 13>n vytiskne se "poznámku~13" a odkazuje na poznámku k verši 13.
\endtt


\sec[hlavni] Hlavní soubor s údaji o všech dalších souborech

Hlavní soubor je soubor, který se \TeX/u předloží jako první. Například je
uveden na příkazvém řádku pro spuštění \TeX/u. V něm jsou informace, jaké
další soubory si má \TeX/ přečíst. Nakonec \TeX/ vytvoří soubor PDF stejného
názvu, jako je název hlavního souboru.

Hlavní soubor pro použití OP-Bible může vypadat takto:

\begtt
\load[op-bible]  % macros OP-Bible

\def\tmark     {CzeBKR}            % Variant of translation
\input {sources/\tmark-books.tex}  % Book titles and marks

\def\txsfile   {sources/\tmark-\bmark.txs}     % Localization of .txs files
\def\fmtfile   {formats/fmt-\tmark-\bmark.tex} % Localisation of fmts
\def\notesfile {notes/notes-\bmark.tex}        % Localisation of notes

\def\printedbooks {%
   Gn Ex Lv Nu Dt Joz Sd Rt 1S 2S 1Kr 2Kr 1CPa 2Pa Ezd Neh
   Est Jb Ž Př Kaz Pís Iz Jr Pl Ez Da Oz Jl Am Abd Jon Mi
   Na Abk Sf Ag Za Mal 
   Mt Mk L J Sk Ř 1K 2K Ga Ef Fp Ko 1Te 2Te 1Tm 2Tm 
   Tt Fm Žd Jk 1Pt 2Pt 1J 2J 3J Ju Zj
}
\processbooks % Generates books declared in \printedbooks
\bye
\endtt

Pomocí `\load[op-bible]` načte \TeX/ makra OP-Bible.

`\tmark` je označení překladu, které se používá v názvech dále čtených
souborů. Je nutno je definovat pomocí `\def`. V ukázce je označení pro
český překlad podle Bible Kralické.

`\input {sources/\tmark-books.tex}` přečte soubor, kde jsou pomocí
`\BookTitle` deklarovány potřebné vazby mezi `<b-značkou>`, `<a-značkou>` a
titulem knihy. Formát příkazu `\BookTilte` je následující:

\begtt
\BookTitle <b-značka> <a-značka> {<titul knihy>}
\endtt
Začátek takto čteného souboru může vypadat třeba takto:

\begtt
\BookTitle Gen  Gn  {První Mojžíšova (Genesis)}
\BookTitle Exod Ex  {Druhá Mojžíšova (Exodus)}
\BookTitle Lev  Lv  {Třetí Mojžíšova (Levicitus)}
\BookTitle Num  Nu  {Čtvrtá Mojžíšova (Numeri)}
\BookTitle Deut Dt  {Pátá Mojžíšova (Deuteronomium)}
\BookTitle Josh Joz {Jozue}
\BookTitle Judg Sd  {Soudců}
...
\endtt
V prvním sloupci za `BookTitle` je `<b-značka>`, která je dále používána v
názvech souborů jako `\bmark`. Ve druhém sloupci je `<a-značka>`, tedy
aktuální značka knihy, která je používána v konkrétním překladu a bude
použita v textu knihy například při sestavování obsahu či v odkazech. V ukázce
máme český překlad, tj. `<a-značka>` knihy Soudců je Sd, zatímco `<b-značka>`
je Judg. Pro názvy souborů se doporučuje ponechat anglické značky, 
abychom nemuseli mít v názvech souborů např. české znaky, takže číst se bude
soubor v jehož názvu je Judg, protože taková je zde <b-značka>, zatímco v
textu Bible bude tato kniha Soudců označena jako Sd.

Soubor s 66 údaji `\BookTitle` je připraven po extrakci základních textů z
Swordu pomocí `mod2tex` a `maketxs`. Je možné je použít, ovšem `<a-značky>`
a tituly knih je potřeba manuálně upravit podle zvyklostí daného překladu,
jak je předvedeno v ukázce výše. 

Do tohoto souboru je možné přidat další informace o jednotlivých knihách
pomocí příkazů `\BookException`, `\BookPre`, `\BookPost`, viz ...\TODO

Pokud chcete mít v názvech souborů <a-značky>, je to možné, ale doporučuje
se kromě toho použít následující sadu výjimek, které odstraňují diakritiku
z~`\amark` v názvech souborů, ale ponechávají ji v textech Bible:

\begtt
\BookException Ž   {\def\amark{Z}}
\BookException Př  {\def\amark{Pr}}
\BookException Pís {\def\amark{Pis}}
\BookException Ř   {\def\amark{R}}
\BookException Žd  {\def\amark{Zd}}
\endtt

Je také možné mít některé typy souborů pojmenované podle <b-značky> a jiné
podle <a-značky>, například:

\begtt
\def\txsfile   {sources/\tmark-\bmark.txs}        % Localization of .txs files
\def\fmtfile   {formats/fmt-\tmark-\amark.tex}    % Localisation of fmts
\def\notesfile {notes/notes-\amark.tex}           % Localisation of notes
\endtt

`\txsfile` deklaruje umístění základních textů Bible, tedy `.txs` souborů,
viz sekci~\ref[zakladni]. V umístění může být specifikován adresář (v ukázce
je `sources/`) a názvy souborů mohou obsahovat `\tmark`, což se nahradí
definovaným označením překladu a dále by měl obsahovat `\bmark`, což se
nahradí `<b-značkou>` knihy. 

`\fmtfile` deklaruje umístění souborů s formátovacími pokyny. Jejich smysl
je vyložen v sekci~\ref[fmt].

`\notesfile` deklaruje umístění souborů s texty poznámek. Jejich formát je
popsán v sekci~\ref[note].

`\printedbooks` obsahuje `<a-značky>` těch knih, které chcete \TeX/em zpracovat.
Pokud děláte třeba jen testovací tisky, můžete zpracovat jen některé
biblické knihy a mít tedy alternativní definici, například
`\def\printedbooks{Da}`.

`\processbooks` spustí cyklus přes všechny biblické knihy podle jejich `<a-značek>`
uložených v~`\printedbooks`. Pro každou takovou knihu \TeX/ nejprve
načte soubor s poznámkami (umístěný podle `\notesfile`), dále přečte soubor
s formátovacími údaji (umístěný podle `fmtfile`) a data z~těchto souborů si
uloží přechodně do paměti. Konečně \TeX/ přečte základní
text umístěný v `\txsfile` a vytvoří sazbu tohoto textu společně s
poznámkami podle stanovených formátovacích pokynů. 

Před provedením `\processbook` je možné pomocí `\input` načít další makra,
která nějakým způsobem ovlivní sazbu výsledné Bible. 

\sec Způsob zpracování \TeX/em

...

\bye
