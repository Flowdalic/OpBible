\load [vlna]
\csquotes \cslang
\margins/1 a4 (20,,,)mm
\fixmnotes\right
\fontfam[schola]
\def\TODO{{\Red TODO}}
\everytt{\typosize[9/]\catcode`<=13 }
\everyintt{\catcode`<=13 }
\def\`{\bgroup \_setverb \toindex}
\def\toindex#1`{\egroup{\tt\Magenta #1}\ea\iindex\ea{\ignoreit#1}}
\def\_printii #1&{\noindent{\tt\bslash#1}\quad}
\def\pozor{\par\noindent {\bf Na co si dát pozor:} }
\def\doporucujeme{\par\noindent {\bf Doporučujeme:} }
\verbchar`
\catcode`<=13 \def<#1>{{$\langle$\it#1$\rangle$}}
\def\OpBible/{OpBible}

\hyperlinks\Blue\Blue

{\nopagenumbers

\tit \OpBible/

\centerline{\typosize [16/]\bf automatická sazba studijních Biblí \TeX/em}
\bigskip
\centerline{\bi Verze 1.0}
\bigskip
\centerline{\setfontsize{at15pt}\it  Petr Olšák, Šimon Konečný}
\bigskip\bigskip
\centerline{\it Revidováno \the\day. \the\month. \the\year}
\bigskip
\centerline{https://www.overleaf.com/project/62e3832c89a3bfe389d592ba}

\vfill
{\it
\rightline{Nestačí, abys byl mým služebníkem,}
\rightline{který má pozvednout Jákobovy kmeny}
\rightline{a přivést zpátky ty z Izraele, kdo byli ušetřeni;}
\rightline{dal jsem tě za světlo pronárodům,}
\rightline{abys byl spása má do končin země.}
\smallskip
\rightline{Iz 49:6} %CEP
}\eject


Kolekce maker \OpBible/ může být volně kopírována, šířena a používána v souladu s GNU
General Public Licence (Verze 2, 1991).

Můžete si přenést části tohoto softwaru do svých vlastních maker a/nebo učinit je součástí jiných
balíčků, avšak balíček, jakkoliv se lišící od původní distribuce, se nesmí jmenovat \OpBible/.

Adaptace specifických implementací (např. fontů) jsou považovány za doplňující soubory a
jejich začlenění neporušuje licenci.


\vfill\eject

\nonum\sec[preface] Předmluva


Verš z Iz 49 na titulní straně je inspirující: Bylo by málo napsat a vysázet \TeX em jenom jeden komentář k Bibli coby něčí celoživotní dílo. \OpBible/ nabízí možnost napsat komentující poznámky k
jakékoliv Bibli komukoliv kdekoliv; jakmile je poznámkový aparát dokončen, studijní Bible je vysázena. 

Dlouholeté vize Šimona Konečného o použití \TeX/u k automatické sazbě studijních Biblí došly svého naplnění spoluprací navázanou s Petrem Olšákem.  Vycházejíce z realistického předpokladu, že
ti, kdo budou OP\discretionary{-}{-}{-}Bibli nejvíce využívat, budou zároveň těmi, kdo mají s \TeX/em nejméně zkušeností, snažili jsme se učinit obsluhu co možná nejjednodušší.

Nicméně nemá smysl předstírat, že sazba studijní Bible je něco triviálního.  
Prohlásit, že  je to programátorský \uv{oříšek}, je těžce podsazené. 
Je to spíše kokosová palma, obsypaná zralými plody.

Náš záměr umožnit vytvoření jednoho komentářového aparátu pro více biblických překladů současně byl
navíc komplikován skutečností, že různé verze Bible mají v některých místech různá číslování veršů, jinak rozlámané odstavce, 
jinou strukturu poetických pasáží, jiné rozmístění nadpisů, doplněných překladateli; že 
některé překlady mají zcela jiné názvy knih (a tudíž i jejich zkratky),
že  v názvech souborů nelze mít diakritiku, 
ale v odkazech na knihy ano a bezpočtem dalších podobných specialitek.

Všechny problémy, na které jsme narazili, se podařilo překonat a všechno funguje k naší spokojenosti.  Výsledná kolekce nástrojů je značně komplikovaná,
pročež použití jejich definic je až nedůtklivě choulostivé na prohřešky proti syntaxi \TeX/u.

Proto \OpBible/ obsahuje i nástroje k snadnějšímu vyhledání míst, kde zůstaly překlepy, zapomenuté závorky apod.; a v této dokumentaci navíc
tam, kde jsme to uznali za vhodné, připojujeme  odstavečky nadepsané {\bf Na co si dát pozor} a {\bf Doporučujeme}.


%Doplníte něco sem?

\vfill 

\rightline{Petr Olšák}
\rightline{Šimon Konečný}
\eject








%\sec Co je to \OpBible/?
%
%\noindent
%\OpBible/ je sada nástrojů (maker a dalšího podpůrného softwaru) na zpracování biblických textů. Hlavním cílem je propojit text Bible s komentujícími poznámkami.
%%podle požadovaných kritérií popsaných níže.
%Výsledkem po zpracování \TeX/em je studijní Bible, tj. Bible vytištěná společně s poznámkami, o jejíž sazbu se postará \TeX\ automaticky.
%%PDF soubor je navíc bohatě prolinkovaný hypertextovými odkazy.

%\medskip
 %%\centerline{\picw=15cm \inspic daniel.png }
%\medskip



\notoc\nonum\sec Obsah
\maketoc
\vfil\break



%\nonum\notoc
\sec Čím je \OpBible/ specifická?

Mezi hlavní přednosti \OpBible/ oproti tradiční sazbě patří:

\begitems
* {\bf Cena:} \begitems
  * Software samotný je zdarma jako public domain pod licencí Open Source (viz Předmluva). 
  * Mnohem významnější však je cena,
  kterou ušetříte za sazeče, kterého byste jinak museli platit několik let, než studijní Bibli    vysází ručně.
        \enditems
* {\bf Čas:} Sazba celé studijní Bible netrvá řádově roky, nýbrž minuty. Opatřit  úplný text Bible poznámkovým
  aparátem, který rozláme stránky tak, aby poznámky skončily na stejných stranách s
  verši, které komentují, trvá na systému s  čtyřjádrovým procesorem Ryzen 3 kolem
  deseti minut. Pracujete-li na konkrétní knize a zpracováváte jen tu jednu, bývá to otázka sekund.
* {\bf Flexibilita:} Výsledkem není jedna jediná verze Bible, do které již není možné provést žádné
  další  zásahy. Pokud se rozhodnete některé poznámky upravit, doplnit, připsat další apod.,
  máte tuto novou revidovanou verzi okamžitě vysázenou a připravenou k tisku.
* {\bf Variabilita:} Výsledkem  nemusí  být
  jen jedna Bible. Poznámky lze psát způsobem, který umožňuje vysázet se stejným poznámkovým aparátem tolik Biblí, kolik v daném jazyce existuje různých překladů. Vyjadřuje-li se Vaše poznámka k babylonskému králi Nabuchodonozorovi,  poznámka uvede jméno Nabuchodonozor,
  pokud sázíte studijní Bibli  kralickou, 
  ale pokud přepnete na Ekumenický překlad, poznámka změní jméno krále na
  Nebúkadnesar. 
  V češtině máme nejméně 6 kompletních  překladů Bible 
  (z toho nejméně 4 snadno stažitelné);  svůj komentář můžete
   napsat tak, že výsledkem bude 6 různých studijních Biblí naráz.
  Podrobněji viz kapitola~\ref[preklady].
* {\bf Interaktivní výstupní formát}.
  Výsledkem zpracování je PDF soubor, který je bohatě propojen aktivními odkazy. 
  Ty jsou nedocenitelným pomocníkem pro proof-reading (korektury) před odevzdáním do tisku (viz níže).
  % Pokud jej tedy nechcete tisknout, můžete se chopit myši a procházet text jak libo za pomocí těchto odkazů.
  % nakonec by se měly tisknout;  možná by stálo za špás zmínit (nebo i zahrnout do balíčku?) ten způsob, jak výsledek opatřit ořezovými značkami
* {\bf Preciznost:} \begitems
  * Průměrná studijní Bible obsahuje kolem 20.000 poznámek a v nich kolem 80.000
  odkazů na biblické verše a jiné poznámky. Poznámky píšou lidé; je nevyhnutelně zákonité, že
  budou obsahovat chyby z přehlédnutí či překlepu. Prakticky není v lidských silách je všechny
  najít a opravit. Je to nicméně v silách stroje. \OpBible/ vypisuje varovná hlášení,
  pokud například poznámka odkazuje na neexistující poznámku nebo verš.
  * Kontrolu správnosti odkazů na verše, a poznámky, které sice existují (takže nevyvolají chybové hlášení),
    ale z nějakého důvodu to nebyly ty správné,
    umožňuje skutečnost, že všechny odkazy jsou  aktivní linky, které při zastavení kurzoru zobrazí bezprostřední kontext
   odkazovaného místa (pokud PDF prohlížíte například správnou verzí programu Evince). A  při kliknutí tam rovnou skočí. 
  * Fráze, kterou poznámka komentuje, se zvýrazní odlišným řezem fontu (např. tučně).
    Zároveň se však i vyhledá v biblickém textu příslušného verše a stránky se budou zalamovat
   tak, aby poznámka byla na stejné stránce se svou frází, nikoliv jen podle čísla verše. Může
   se stát, že dvě poznámky ke stejnému verši komentují dvě různá slova  onoho verše a že
   stránkový zlom bude vycházet mezi těmito slovy. V takovém případě  budou poznámky následovat
   své fráze.
   * Biblické texty se připraví (například stažením z internetu) do zvláštních souborů a není třeba do nich vůbec nic přidávat. Všechny poznámky k nim a další pokyny k sazbě se zapisují do oddělených souborů. Pak to \TeX/ spojí dohromady.
    \enditems
\enditems

{\bf Na co si dát pozor:} Ačkoliv záměr tohoto programu je šířit Boží Slovo a dobrou zprávu v něm obsaženou, \OpBible/ sama {\it neodpouští nic!\/} (Co byste taky čekali, je to
software. Pro odpuštění jděte k Bohu.) Jak se dočtete dále, \OpBible/ načte do paměti vždy jednu celou biblickou knihu (tzv. základní text) a pak teprve začne chroustat Vaše
poznámky a skládat je dohromady na stránky s příslušnými verši. Když se Vám stane, že se dopustíte chyby v \TeX ovské syntaxi (např. zapomenete uzavírací složenou závorku apod.), \TeX\ chybu uvidí někde uprostřed této přiřazovací procedury a tomu taky bude odpovídat nepřehledné chybové hlášení, ve kterém je vysoce pravděpodobné, že se nezorientujete.

\OpBible/ s~výskytem podobných situací počítá a nabízí užitečné nástroje, jak Vám pomoct z podobných brynd, ale raději buďte pozorní a důslední. Neprohloupíte, když s první napsanou
poznámkou projedete všechny překlady, které plánujete používat, abyste se přesvědčili, že všechno funguje, jak má.  Pak stačí pracovat jenom s jedním z nich, ale je radno
spouštět \TeX\ po dokončení každé nové poznámky, ať podle jeho protestů snadno najdete tu, ve které máte chybu.

{\bf Doporučujeme:} Používejte operační systém Linux. \OpBible/, coby soubor maker pro \TeX, poběží na kterémkoliv systému se stejnými výsledky, ale instalace na Linuxu má několik
nepřehlédnutelných výhod. Mezi ty nejvýznamnější je nutno zařadit PDF prohlížeč Evince, jehož novější (píše se rok 2023) verze dokáže zobrazit bezprostřední kontext hyperlinkového
odkazu při pouhém zastavení kurzoru bez nutnosti na něj klikat. To je nedocenitelná pomůcka pro kontrolu správnosti odkazů na biblické pasáže nebo poznámky.
Jiné systémy tuto vymoženost (zatím) nenabízejí. Hyperlinky v nich sice skočí na příslušné místo po kliknutí, ale už netrefí zpátky, odkud byl vyslán signál ke skoku, což je únavné. V této dokumentaci předpokládáme instalaci na Linuxu.

A taky samozřejmě doporučujeme -- nebo spíše i považujeme za nezbytnou podmínku -- alespoň  základní gramotnost v programu \TeX.
Jestli s \TeX/em nemáte vůbec žádné zkušenosti, zkuste začít třeba 
\ulink[http://petr.olsak.net/tpp.html]{tady}. 
Čas, investovaný do tohoto preliminárního vzdělání, nebude promarněný; čím více budete \TeX/u rozumět, tím lépe (a radostněji) budete psát své poznámky k Bibli.

\sec[Instalace] Co potřebujeme k provozu \OpBible/

Je třeba mít nějaký textový editor, který nezanechává v textu žádné skryté formátovací údaje (určený k editování například programů). Záleží na uživateli, jaký editor mu vyhovuje.
Za ideální lze považovat takové, které podle přípony zdrojového souboru poznají programovací jazyk a barví slova podle jeho syntaxe. Např. Vi, Emacs, ... 
\ulink[https://cs.wikipedia.org/wiki/Seznam_textových_editorů]{vyberte si}, který by Vám vyhovoval. 


Dále je třeba mít \TeX/ovou distribuci s programem \LuaTeX/ a s výchozími makry \OpTeX/ a konečně je třeba mít nějaký prohlížeč výstupu, tedy PDF souborů (doporučujeme novou verzi Evince).

Na jakém operačním systému to budete provozovat, není podstatné, doporučujeme ale 
\ulink[https://www.alza.cz/linux-distribuce?kampan=adwalz_alza_bee_gen_clanky_alza-clanek-linuxove-distribuce-pro-rok-2020-32590&ppcbee-adtext-variant=pruvodce-vyberem-co-je-clanek&gclid=Cj0KCQiAg_KbBhDLARIsANx7wAzZRawTUBcz3NLtfQDTvAFL_VGEPwg1jhZI4Quv_5HxIie5fVYunykaAoG_EALw_wcB]{Linux}.

Je také možné pracovat v online režimu bez nutnosti cokoli instalovat, viz sekci~\ref[OverLeaf].

\secc[TeXlive] \TeX/ová distribuce na lokálním počítači

Doporučujeme nejnovější \ulink[https://tug.org/texlive/]{\TeX/live}.
Distribuci lze instalovat na libvolném operačím systému a to přímo z webu. Je též součástí běžných linuxových distribucí. 

\TeX/live obsahuje program \LuaTeX/, který bude zpracovávat Vaše vstupní soubory a vyrábět výstupní PDF. Také obsahuje výchozí sadu maker \ulink[http://petr.olsak.net/optex/]{\OpTeX/}, kterou makra \OpBible/ potřebují a rozšiřují ji o možnosti popsané v tomto manuálu. Výchozí sada maker (často nazývaná také formát) vymezuje způsob značkování dokumentů a řídí formátování. \TeX/ové distribuce obsahují i jiné výchozí sady maker. Nejznámější je zřejmě \LaTeX/, ale ten \OpBible/ nepotřebuje.

%\secc[Linux] Linux

%Na internetu najdete velké množství tzv. \"distribucí" Linuxu, které se liší například výchozím uživatelským prostředím (window manager), koncepcí doplňování a obnovování balíčků s různým softwarem atd. \TODO.
%Snad jen nějaké linky, popř. doporučení těch uživatelsky nejpřívětivějších... Možná zmínku o VirtualBoxu...


%\secc[LuaTeX] \LuaTeX

%To je program, který bude zpracovávat Vaše vstupní soubory a vyrábět výstupní PDF. Je součástí \"\TeX/ové distribuce" \ulink[https://tug.org/texlive/]{\TeX/live}. Tu lze instalovat přímo z webu, ale je též přímo součástí Linuxových distribucí.

%\secc[OPTeX] OpTeX

%Je sada maker (pokynů pro \LuaTeX/), která řídí zpracování dokumentů a nabízejí uživateli pravidla, jak označkovat vstupní soubory. Jednou z dalších a asi nejrozšířenějších výchozích sad maker pro \TeX/ je \LaTeX/, ale ten při provozování \OpBible/ nepotřebujeme. \ulink[http://petr.olsak.net/optex/]{\OpTeX/} je součástí distribuce \TeX/live, takže jej skoro jistě budete mít po instalaci nejnovější verze \TeX/live.

%\secc[OPBible]  \OpBible/

%Je rozšiřující sada maker navazující na \OpTeX/ a specializující se na úkol vytvořit komentované Bible v souladu s tímto návodem.

\secc[OverLeaf] Varianta: vzdálený přístup na serveru OverLeaf.com

Overleaf je webové prostředí pro sdílenou přípravu \TeX/ových dokumentů třeba i více uživateli. Nemusíte mít \TeX/live isntalován a může jej provozoval online prostřednictvím \ulink[https://www.overleaf.com]{Overleaf}. Je k dispozici i \TODO: \ulink[??]{výchozí projekt} obsahující soubory pro zpracování pomocí \OpBible/. Jako uživatelé Overleafu si jej můžete zkopírovat (klonovat) do svého projektu a tam pokračovat v činnosti. Práce s \TeX/ovou distribucí přímo v počítači je ale výrazně rychlejší a pohodlnější.

%\secc[evince] Evince

%`flatpak` : flatpak run org.gnome.Evince

%Úplný text Bible je na internetu k dispozici v různých překladech a českých
%překladů je nejméně šest. Je možné si stáhnout třeba jen jeden překlad


%(dále mu říkáme \uv{základní text}) a doplňovat jej poznámkami, které
%zapisujeme do přidružených souborů tak, že není nutná žádná editace či jiný
%zásah do základního textu. Je též možné poznámky psát tak, že jsou víceméně
%univerzální například pro kterýkoli z českých překladů, takže pozměněním jen
%výchozí konfigurace v hlavním souboru zpracování \TeX/ načte jiný základní
%text se stejnými poznámkami, které se navíc automaticky modifikují podle
%vyjadřovacích zvyklostí toho kterého překladu.

%Pro zpracování \TeX/em je tedy třeba mít odpovídající verzi základního textu
%(předzpracovanou způsobem, popsaným v sekci~\ref[txs]) a dále soubory s
%poznámkami (jejich syntaxe je uvedena v sekci \ref[note]).
%\TeX/ přečte hlavní soubor se základními pokyny, pak načte data všech
%potřebných souborů a vytvoří PDF soubor knihy -- komentované Bible. V ní je jednak
%(typicky větším písmem) základní text a (typicky menším písmem, například
%pod čarou a třeba ve dvou sloupcích) poznámky k textu. 

%\medskip
%\centerline{\picw=15cm \inspic daniel.png }
%\medskip

%Je respektován
%základní požadavek, že k jednotlivým frázím základního textu jsou poznámky
%umístěny na stejné stránce jako samotné fráze základního textu. O rozmístění
%objektů na stránkách se \TeX/ stará víceméně automaticky podle nastavených
%typografických pravidel, které se dají modifikovat dle potřeby. Požadavky na
%typografickou úpravu nebo další specifické požadavky, které se dají
%algoritmicky popsat, jsou implementovány makry \TeX/u a \TeX/ je během
%zpracování přečte a automaticky provede.
%
%Veškerá data, která \TeX/ zpracovává (konfigurace, základní text, texty
%poznámek, maker) mají textový formát, tj. dají se prohlížet a případně
%upravovat běžným textovým editorem.

%\notoc\nonum\sec Obsah
%\maketoc
%\vfil\break
}

\sec Spuštění \TeX/u

%Balíček \OpBible/ předpokládá použítí \TeX/ového engine \LuaTeX/ se sadou výchozích maker \OpTeX/, viz \url{http://petr.olsak.net/optex}. 

Máte-li nainstalovanou \TeX/ovou distribuci na svém počítači (například \TeX/live
2022) %; nemáte-li, viz kapitola \ref~[instalace])
a máte-li v systému k dipozici příkazový řádek, pak \TeX/ spustíte pomocí:

\begtt
optex main.tex
\endtt
kde `main.tex` je název hlavního souboru (může se jmenovat i jinak). 
Příkaz `optex` spustí \LuaTeX/ s makry \OpTeX. Výsledkem zpracování je soubor 
`main.pdf` a zpráva o zpracování je v `log` souboru `main.log`.

Můžete vyzkoušet, jestli Vám to funguje (zatím bez maker \OpBible/). Vytvořte si jednoduchý soubor `main.tex` v textovém editoru s tímto obsahem:

\begtt
\fontfam[lm]
Ahoj světe!
\bye
\endtt
a spusťte příkaz `optex main.tex`. Výsledný PDF soubor `main.pdf` si prohlédněte PDF prohlížečem.

%Je možné též využít nástroj \ulink[https://www.overleaf.com]{Overleaf} s
%\TeX/ovou distribucí provozovanou na webovém %serveru. Můžete vyjít z výchozího
%projektu \ulink[??]{\OpBible/ na Oveleaf} a zkopírovat si jej do svého
%projektu. Práce s~\TeX/ovou distribucí přímo v počítači je ale výrazně
%rychlejší a pohodlnější. 


\sec Struktura souborů zpracovávaných \TeX/em

\secc[hlavni] Hlavní soubor s údaji o všech dalších souborech

Hlavní soubor je soubor, který se \TeX/u předloží jako první. Například je
uveden na příkazovém řádku pro spuštění \TeX/u. V něm jsou informace, jaké
další soubory si má \TeX/ přečíst. Nakonec \TeX/ vytvoří soubor PDF stejného
názvu, jako je název hlavního souboru.

Hlavní soubor pro použití \OpBible/ (například `main.tex`) může vypadat takto:

\begtt
\load[op-bible]  % makra OpBible
\cslang          % inicializace českých vzorů dělení slov

% Varianty překladu:
\def\tmark     {BKR}    % Bible kralická
%\def\tmark     {PSP}    % Pavlíkův studijní překlad
%\def\tmark     {CSP}    % Český studijní překlad
%\def\tmark     {CEP}    % Český ekumenický překlad
%\def\tmark     {B21}    % Bible pro 21. století
%\def\tmark     {SNC}    % Slovo na cestu

\input {Cze-vars.tex}   % Deklarace frází pro různé varianty překladu
\BookTitle Gen  Gn {První Mojžíšova (Genesis)}
\BookTitle Exod Ex {Druhá Mojžíšova (Exodus)}
\BookTitle Lev  Lv {Třetí Mojžíšova (Leviticus)}
\BookTitle Num  Nu {Čtvrtá Možíšova (Numeri)}
\BookTitle Deut Dt {Pátá Mojžíšova (Deuteronomium)}
\BookTitle Josh Joz {Jozue}
\BookTitle Judg Sd {Soudců}
\BookTitle Ruth Rt {Rút}
\BookTitle 1Sam 1S {První Samuelova}
\BookTitle 2Sam 2S {Druhá Samuelova}
\BookTitle 1Kgs 1Kr {První Královská}
\BookTitle 2Kgs 2Kr {Druhá Královská}
\BookTitle 1Chr 1CPa {První Paralipomenon (1. Letopisů)}
\BookTitle 2Chr 2Pa {Druhá Paralipomenon (2. Letopisů)}
\BookTitle Ezra Ezd {Ezdráš}
\BookTitle Neh  Neh {Nehemjáš}
\BookTitle Esth Est {Ester}
\BookTitle Job  Jb {Jób}
\BookTitle Ps   Ž {Žalmy}
\BookTitle Prov Př {Přísloví}
\BookTitle Eccl Kaz {Kazatel}
\BookTitle Song Pís {Píseň písní}
\BookTitle Isa  Iz {Izajáš}
\BookTitle Jer  Jr {Jeremjáš}
\BookTitle Lam  Pl {Pláč}
\BookTitle Ezek Ez {Ezechiel}
\BookTitle Dan  Da {Daniel}
\BookTitle Hos  Oz {Ozeáš}
\BookTitle Joel Jl {Jóel}
\BookTitle Amos Am {Ámos}
\BookTitle Obad Abd {Abdijáš}
\BookTitle Jonah Jon {Jonáš}
\BookTitle Mic  Mi {Micheáš}
\BookTitle Nah  Na {Nahum}
\BookTitle Hab  Abk {Abakuk}
\BookTitle Zeph Sf {Sofonjáš}
\BookTitle Hag  Ag {Ageus}
\BookTitle Zech Za {Zacharjáš}
\BookTitle Mal  Mal {Malachiáš}
\BookTitle Matt Mt {Matouš}
\BookTitle Mark Mk {Marek}
\BookTitle Luke L {Lukáš}
\BookTitle John J {Jan}
\BookTitle Acts Sk {Skutky apoštolské}
\BookTitle Rom  Ř {Římanům}
\BookTitle 1Cor 1K {První list Korintským}
\BookTitle 2Cor 2K {Druhý list Korintským}
\BookTitle Gal  Ga {Galatským}
\BookTitle Eph  Ef {Efezským}
\BookTitle Phil Fp {Filipským}
\BookTitle Col  Ko {Koloským}
\BookTitle 1Thess 1Te {První list Tesalonickým}
\BookTitle 2Thess 2Te {Druhý list Tesalonickým}
\BookTitle 1Tim 1Tm {První list Timoteovi}
\BookTitle 2Tim 2Tm {Druhý list Timoteovi}
\BookTitle Titus Tt {Titovi}
\BookTitle Phlm  Fm {Filemonovi}
\BookTitle Heb   Žd {Židům}
\BookTitle Jas   Jk {List Jakubův}
\BookTitle 1Pet  1Pt {První list Petrův}
\BookTitle 2Pet  2Pt {Druhý list Petrův}
\BookTitle 1John 1J {První list Janův}
\BookTitle 2John 2J {Druhý list Janův}
\BookTitle 3John 3J {Třetí list Janův}
\BookTitle Jude  Ju {List Judův}
\BookTitle Rev   Zj {Zjevení Janovo}     

\BookException Ž   {\def\amark{Z}}
\BookException Př  {\def\amark{Pr}}
\BookException Pís {\def\amark{Pis}}
\BookException Ř   {\def\amark{R}}
\BookException Žd  {\def\amark{Zd}}

  % Tituly knih a jejich značky \amark

\def\txsfile     {sources/Cze\tmark-\amark.txs}     % Umístění txs souborů
\def\fmtfile     {formats/fmt-Cze\tmark-\amark.tex} % Umístění fmt souborů
\def\notesfile   {notes/notes-\amark.tex}           % Umístění souborů s poznámkami
\def\introfile   {others/intro-\amark.tex}          % Umístění soubotů s úvody ke knihám
\def\articlefile {others/articles-\amark.tex}       % Umístění soubor s články

\def\printedbooks {%
   Gn Ex Lv Nu Dt Joz Sd Rt 1Sa 2Sa 1Kr 2Kr 1Pa 2Pa Ezd Neh
   Est Jb Ž Př Kaz Pís Iz Jr Pl Ez Da Oz Jl Am Abd Jon Mi
   Na Abk Sf Ag Za Mal 
   Mt Mk Lk Jn Sk Ř 1K 2K Ga Ef Fp Ko 1Te 2Te 1Tm 2Tm 
   Tt Fm Žd Jk 1Pt 2Pt 1Jn 2Jn 3Jn Ju Zj
}
\processbooks % Generuje dokument s knihami deklarovanými v \printedbooks
\bye
\endtt

Pojďme se nyní podívat, co který z těch řádků dělá a které budou vyžadovat úpravy z Vaší strany pro konkrétní  potřeby Vašeho projektu.

Pomocí \`\load[op-bible]` načte \TeX/ makra balíčku \OpBible/. To je ten nejdůležitější program, který se stará o sazbu.%, čili tady nemusíte (naopak: nesmíte) nic měnit.

Příkaz \`\cslang` nastaví české vzory dělení slov, předpokládá se tedy český
text. Přitom `cs` je ISO zkratka jazyka, můžete použít i jiné jazyky:
`\enlang` pro angličtinu, `\delang` pro němčinu, `\eslang` pro španělštinu
atd. Všechny tyto jazykové možnosti jsou uvedeny \ulink[http://petr.olsak.net/ftp/olsak/optex/optex-doc.pdf]{v dokumentaci k \OpTeX/u}. %Taky ponechte nedotčeno.

Příkaz `\def\tmark {<značka>}` definuje makro \`\tmark` jako značku
použitého překladu (můžete si pamatovat Translation mark). Značky všech dostupných překladů jsou uvedeny v souboru
`Cze-vars.tex`. Jednu z nich je třeba vybrat jako označení zrovna zpracovávaného
překladu. Např. `BKR` je značka pro překlad Bible kralické.
V ukázce jsou uvedeny všechny obvyklé možnosti pro definici překladové značky v případě českých Biblí. Jen jedna možnost (ta skutečně vybraná) nemá před sebou znak uvozující komentář `%`. %Ne raději  takto než \% ?

%Tady už se očekává Váš zásah do souboru. 
Pokud prohlížíte  právě rozpracovanou knihu v překladu `BKR`, ponechte soubor `main.tex` ve výše uvedené podobě.
Až budete chtít přepnout do řekněme ekumenického překladu, znakem procenta zakomentujete (tj. zneviditelníte pro \TeX) řádek s Biblí kralickou, ale zviditelníte (odkomentujete)
řádek s ekumenickým překladem. Pak úsek s definicemi  `\tmark` bude vypadat takto:

\begtt
% Variants of translation:
%\def\tmark     {BKR}    % Bible kralická
%\def\tmark     {PSP}    % Pavlíkův studijní překlad
%\def\tmark     {CSP}    % Český studijní překlad
\def\tmark     {CEP}    % Český ekumenický překlad
%\def\tmark     {B21}    % Bible pro 21. století
%\def\tmark     {SNC}    % Slovo na cestu
\endtt

Vždy však musí být jeden z překladů aktivní, čili `\tmark` musí být definována. Kdybyste zapomněli napsat procento před řádek, který chcete zrušit, svět se nezboří;
bude platit poslední definice, kterou \TeX\ načte a kterou předefinuje případné předchozí.

\`\input {Cze-vars.tex}` přečte konfiguraci o variantách překladu
ze souboru `Cze-vars.tex`. Soubor je v ukázce nazván `Cze-vars.tex`, protože
obsahuje informace o českých překladech. Podrobněji viz sekci~\ref[vars]. Nesahat na tento řádek, byť asi budete editovat soubor `Cze-vars.tex`, tímto řádkem volaný.

\`\BookTitle Gen  Gn {První Mojžíšova (Genesis)}
\BookTitle Exod Ex {Druhá Mojžíšova (Exodus)}
\BookTitle Lev  Lv {Třetí Mojžíšova (Leviticus)}
\BookTitle Num  Nu {Čtvrtá Možíšova (Numeri)}
\BookTitle Deut Dt {Pátá Mojžíšova (Deuteronomium)}
\BookTitle Josh Joz {Jozue}
\BookTitle Judg Sd {Soudců}
\BookTitle Ruth Rt {Rút}
\BookTitle 1Sam 1S {První Samuelova}
\BookTitle 2Sam 2S {Druhá Samuelova}
\BookTitle 1Kgs 1Kr {První Královská}
\BookTitle 2Kgs 2Kr {Druhá Královská}
\BookTitle 1Chr 1CPa {První Paralipomenon (1. Letopisů)}
\BookTitle 2Chr 2Pa {Druhá Paralipomenon (2. Letopisů)}
\BookTitle Ezra Ezd {Ezdráš}
\BookTitle Neh  Neh {Nehemjáš}
\BookTitle Esth Est {Ester}
\BookTitle Job  Jb {Jób}
\BookTitle Ps   Ž {Žalmy}
\BookTitle Prov Př {Přísloví}
\BookTitle Eccl Kaz {Kazatel}
\BookTitle Song Pís {Píseň písní}
\BookTitle Isa  Iz {Izajáš}
\BookTitle Jer  Jr {Jeremjáš}
\BookTitle Lam  Pl {Pláč}
\BookTitle Ezek Ez {Ezechiel}
\BookTitle Dan  Da {Daniel}
\BookTitle Hos  Oz {Ozeáš}
\BookTitle Joel Jl {Jóel}
\BookTitle Amos Am {Ámos}
\BookTitle Obad Abd {Abdijáš}
\BookTitle Jonah Jon {Jonáš}
\BookTitle Mic  Mi {Micheáš}
\BookTitle Nah  Na {Nahum}
\BookTitle Hab  Abk {Abakuk}
\BookTitle Zeph Sf {Sofonjáš}
\BookTitle Hag  Ag {Ageus}
\BookTitle Zech Za {Zacharjáš}
\BookTitle Mal  Mal {Malachiáš}
\BookTitle Matt Mt {Matouš}
\BookTitle Mark Mk {Marek}
\BookTitle Luke L {Lukáš}
\BookTitle John J {Jan}
\BookTitle Acts Sk {Skutky apoštolské}
\BookTitle Rom  Ř {Římanům}
\BookTitle 1Cor 1K {První list Korintským}
\BookTitle 2Cor 2K {Druhý list Korintským}
\BookTitle Gal  Ga {Galatským}
\BookTitle Eph  Ef {Efezským}
\BookTitle Phil Fp {Filipským}
\BookTitle Col  Ko {Koloským}
\BookTitle 1Thess 1Te {První list Tesalonickým}
\BookTitle 2Thess 2Te {Druhý list Tesalonickým}
\BookTitle 1Tim 1Tm {První list Timoteovi}
\BookTitle 2Tim 2Tm {Druhý list Timoteovi}
\BookTitle Titus Tt {Titovi}
\BookTitle Phlm  Fm {Filemonovi}
\BookTitle Heb   Žd {Židům}
\BookTitle Jas   Jk {List Jakubův}
\BookTitle 1Pet  1Pt {První list Petrův}
\BookTitle 2Pet  2Pt {Druhý list Petrův}
\BookTitle 1John 1J {První list Janův}
\BookTitle 2John 2J {Druhý list Janův}
\BookTitle 3John 3J {Třetí list Janův}
\BookTitle Jude  Ju {List Judův}
\BookTitle Rev   Zj {Zjevení Janovo}     

\BookException Ž   {\def\amark{Z}}
\BookException Př  {\def\amark{Pr}}
\BookException Pís {\def\amark{Pis}}
\BookException Ř   {\def\amark{R}}
\BookException Žd  {\def\amark{Zd}}

` přečte informace o značkách (zkratkách) knih Bible a jsou jim
přiřazeny názvy knih. Podrobněji o těchto údajích pojednává sekce~\ref[books]. 
%Tento řádek (podobně jako ten předchozí) ponechte beze změny. 
%ale můžete změnit podobu zkratek biblických knih, které budete používat.
%Pokud je chcete změnit, otevřete soubor `Cze-books.tex` a uvidíte několik sloupců:
%První z nich obsahuje řídicí sekvenci `\BookTitle`.
%Pak následují dva sloupce zkratek, z nichž první patří odkazům, které budete používat 
%(a objeví se i v názvech některých souborů)
%a druhý má spíše technický význam, odkazuje na názvy dalších souborů. Změnit můžete jen ten první z nich; a změna, kterou provedete,
%se musí projevit i v dolní části souboru za definicemi `\BookException`, popřípadě `\nochapbooks`. Tyto dvě skupiny musejí obsahovat přesně stejné zkratky jako první sloupec
%zkratek za `\BookTitle`. Tytéž zkratky musejí být parametrem definice `\printedbooks`
%v hlavním souboru `main.tex` (viz výše i níže).   

Makrem \`\txsfile` (definovaným pomocí příkazu `\def`) je určeno umístění `.txs`
souborů v adresářové struktuře. Pro každou knihu Bible musí být připraven
jeden `.txs` soubor obsahující základní text pro příslušnou knihu. Názvy
souborů se liší podle značky knihy a máme-li více překladů, obsahuje název souboru
také značku překladu. V makru `\txsfile` lze
použít \`\tmark` jako značku překladu a \`\amark` nebo \`\bmark` jako značku
knihy. O značkách knih se píše v sekci~\ref[books], o formátu `.txs` souborů
se píše v sekci~\ref[txs]. V ukázce je vymezeno, že `.txs` soubory jsou
umístěny v adresáři `sources/` a mají názvy
`Cze<značka-překladu>-<značka-knihy>.txs`, například tedy `CzeBKR-Gn.tex`.

Zmíněné české překlady jsou již připraveny k použití, nemusíte je vytvářet sami.
Pokud byste potřebovali nějaký jiný existující překlad, musíte ho dostat do formátu, použitelném pro OpBibli, tady do podoby, 
v jaké jsou soubory `*.txs` v adresáři `sources/`.
S přípravou jednotlivých `.txs` knih z existujícího zdroje  Vám pomůže skript `maketxs`, viz~\ref[txs].
Pokud tvoříte zbrusu nový překlad a plánujete ho pro použití s OpBiblí, asi neuškodí ho rovnou psát po jednotlivých souborech pro každou knihu
přímo v požadovaném formátu, viz taktéž~\ref[txs].

Makrem \`\fmtfile` je vymezeno umístění souborů upřesňující formátování
základního textu. Každá kniha Bible každého použitého překladu má svůj formátovací soubor.
To je něco, co nemůže být společné všem překladům (na rozdíl od poznámek),
protože odstavce a doplněné nadpisy má každý překlad své vlastní.
Naším záměrem byla tzv. non-destruktivní editace, jinými slovy, formátování biblického textu
bez zásahu do něho.
O těchto souborech se píše v sekci~\ref[fmt].

Makrem \`\notesfile` je vymezeno umístění souborů s poznámkami.
Do nich budete psát svůj komentář. 
Každá kniha
Bible má svůj soubor s poznámkami. Poznámky odkazují na místo v
základním textu a úkolem \TeX/u je vytvořit stránky se základním textem a
poznámkami k němu. Podrobněji o způsobu psaní souborů s poznámkami se píše v
sekci~\ref[note]. Všimněte si, že soubory s poznámkami jsou společné všem
překladům, tj. nejsou zde zvlášť soubory odlišené pomocí \`\tmark`. Pravidla
psaní poznámek totiž umožní obsáhnout vyjadřovací možnosti různých překladů
společným zápisem, ovšem v jednom jazyku (například češtině). Chcete-li psát
poznámky pro zcela jiný jazyk, je třeba založit nový projekt (nejlépe v 
jiném adresáři) s jinými `.txs`
soubory, jinými formátovacími a poznámkovými soubory.

Makrem \`\introfile` se specifikují soubory, kde jsou napsány úvody k
jednotlivým knihám. Ke každé knize je možné vytvořit samostatný soubor s 
úvodem.

Makrem \`\articlefile` se určují názvy souborů kde, jsou teologické články,
rozmístěné různě po biblických knihách. %\TODO: dopsat...

Makro \`\printedbooks` obsahuje značky těch knih, které chcete \TeX/em 
zpracovat.
V ukázce je požadavek na zpracování celé Bible, tedy všech 66~knih 
protestantského kánonu.
Pokud děláte třeba jen testovací tisky, můžete zpracovat jen některé
biblické knihy a mít v hlavním souboru tedy alternativní definici, například
`\def\printedbooks{Da}`. Stačí ji uvést za definici pro celou Bibli, protože
pozdější definice stejného makra má přednost.


%Zkratky knih můžete přizpůsobit svým preferencím, ale učiňte tak před 



Příkaz \`\processbooks` spustí zpracování všech knih specifikovaných v makru
\`\printedbooks`.  Pro každou knihu \TeX/ přečte příslušný základní text z
`.txs` souboru, formátuje jej pomocí údajů z příslušného formátovacího
souboru a připojí k nim poznámky z příslušného poznámkového souboru.
Tady už nepotřebujete nic měnit.

Příkaz \`\bye` ukončí činnost \TeX/u.
Cokoliv napíšete za toto rozloučení s \TeX em, bude ignorováno.



Do hlavního souboru před \`\processbooks` můžete přidávat také vlastní makra a nastavení, která ovlivní kompletně celou sazbu.

Je možné tam například dát deklaraci \`\ChapterPre``{<kód>}` anebo \`\ChapterPost``{<kód>}`. Tyto kódy se pak provedou před a za každou kapitolou.
%Např. \readplan -- musíme dodělat


\secc[vars] Soubor deklarující varianty překladu

Pracujeme-li s jedinou variantou překladu, není nutné tento soubor vytvářet
a používat. Pak stačí odstranit pokyn k jeho čtení z hlavního souboru.

V ukázce v sekci~\ref[hlavni] je čten soubor `Cze-vars.tex`, který
by měl obsahovat deklaraci značek variant překladů pomocí \`\variants`:
\begtt
\variants <počet-variant> {<značka>} {<značka>} ... {<značka>}
\endtt
kde `<počet-variant>` je počet variant překladů (do společného jazyka,
například češtiny) a pak jsou vyjmenovány všechny značky variant překladů.
Například
\begtt
\variants 6 {BKR} {PSP} {CSP} {CEP} {B21} {SNC}
\endtt
deklaruje zkratky pro 6 variant překladů: BKR: Bible kralická, PSP:~Pavlíkův
studijní překlad, CSP:~Český studijní překlad, CEP: Český ekumenický
překlad, B21: Bible pro 21. století, SNC: Slovo na cestu.

Takto definované varianty překladu musejí odpovídat definicím `\tmark` v hlavním souboru `main.tex` včetně velkých či malých písmen.

Dobře zvažte předem, jaký počet překladů budete chtít (měnit to později bude značně obtížné, byť přeci jen možné)
a hlavně pořadí překladů: Jak to bude uvedeno  v definici `\variants`, tak to bude platit pro celý projekt.
Do všech poznámek, komentujících nějakou frázi, která zní v různých překladech
různě,  budete její různá znění uvádět právě v tomto pořadí.

Pokud víte, že se nějaká fráze či slovo bude vyskytovat častěji, než jen v jediné poznámce, 
můžete ji nadefinovat rovnou v tomto souboru `Cze-vars.tex` pomocí příkazu \`\vdef`.
Frází uvedených za \`\vdef` musí být přesně tolik, jako je
`<počet-variant>`, jsou uzavřeny do složených závorek a korespondují s~jednotlivými variantami překladu ve stejném pořadí, jak jsou varianty 
uvedeny v deklarci \`\variants`. Například:
\begtt
\vdef {Joakim} {Jehójákím} {Jójákím} {Jójakím} {Joakim} {Jójakím}
\endtt
deklaruje, že jméno Judského krále Joakima je přepisováno jinak v různých
variantách překladu. Například v CEP nebo v SNC to je Jójakím, Pavlík jej
nazývá Jehójákímem atd. Až budeme psát poznámky týkající se tohoto krále do
poznámkového souboru, napíšeme do textu poznámky jen `\x/Joakim/` (první varianta překladu v definici `\variants`)
a toto se promění v odpovídající frázi používanou v právě zpracovávaném
překladu, který jsme deklarovali v~hlavním souboru pomocí `\def\tmark{...}`.

Takže po změně `\def\tmark{...}` v~hlavním souboru se všechny výskyty
`\x/Joakim/` v textech poznámek mohou začít automaticky chovat jinak
a přizpůsobí se frazeologii dané varianty překladu.
Pak  lze taková slova třeba i skloňovat: Zápis `\x/Joakim/ovi` dá v poznámce pod Českým ekumenickým překladem podobu `Jójakímovi`. 
Podrobněji o tom pojednává sekce~\ref[preklady].

Příkaz \`\variants` deklarující zkratky variant překladu je v souboru pro
varianty jediný, zatímco příkazů \`\vdef` definujících variantní fráze může být
v souboru více, protože pochopitelně existuje více frází, které se v různých
variantách překladu různě přepisují, není to jen král Joakim.

Celé pasáže textu mohou být zpracovány rozdílným způsobem v závislosti na nastavené variantě překladu. K tomu slouží příkaz větvení \`\switch`. Podrobněji se o něm píše v sekci~\ref[switch]. Například názvy jednotlivých překladů (které se pak použijí v záhlaví stránek) je možné deklarovat pomocí `\def\bibname` rozdílně pro různé překlady:
\begtt
\switch {BKR}{\def\bibname{Bible kralická}}%
        {B21}{\def\bibname{Bible pro 21. století}}%
        {CEP}{\def\bibname{Český ekumenický překlad}}%
        {CSP}{\def\bibname{Český studijní překlad}}%
        {PSP}{\def\bibname{Pavlíkův studijní překlad}}%
        {SNC}{\def\bibname{Slovo na cestu}}%
\endtt
Taková deklarace je součástí již připraveného souboru `Cze-vars.tex`.

\secc[books] Soubor s názvy knih

V hlavním souboru je pokyn na čtení souboru s názvy knih, například
`\BookTitle Gen  Gn {První Mojžíšova (Genesis)}
\BookTitle Exod Ex {Druhá Mojžíšova (Exodus)}
\BookTitle Lev  Lv {Třetí Mojžíšova (Leviticus)}
\BookTitle Num  Nu {Čtvrtá Možíšova (Numeri)}
\BookTitle Deut Dt {Pátá Mojžíšova (Deuteronomium)}
\BookTitle Josh Joz {Jozue}
\BookTitle Judg Sd {Soudců}
\BookTitle Ruth Rt {Rút}
\BookTitle 1Sam 1S {První Samuelova}
\BookTitle 2Sam 2S {Druhá Samuelova}
\BookTitle 1Kgs 1Kr {První Královská}
\BookTitle 2Kgs 2Kr {Druhá Královská}
\BookTitle 1Chr 1CPa {První Paralipomenon (1. Letopisů)}
\BookTitle 2Chr 2Pa {Druhá Paralipomenon (2. Letopisů)}
\BookTitle Ezra Ezd {Ezdráš}
\BookTitle Neh  Neh {Nehemjáš}
\BookTitle Esth Est {Ester}
\BookTitle Job  Jb {Jób}
\BookTitle Ps   Ž {Žalmy}
\BookTitle Prov Př {Přísloví}
\BookTitle Eccl Kaz {Kazatel}
\BookTitle Song Pís {Píseň písní}
\BookTitle Isa  Iz {Izajáš}
\BookTitle Jer  Jr {Jeremjáš}
\BookTitle Lam  Pl {Pláč}
\BookTitle Ezek Ez {Ezechiel}
\BookTitle Dan  Da {Daniel}
\BookTitle Hos  Oz {Ozeáš}
\BookTitle Joel Jl {Jóel}
\BookTitle Amos Am {Ámos}
\BookTitle Obad Abd {Abdijáš}
\BookTitle Jonah Jon {Jonáš}
\BookTitle Mic  Mi {Micheáš}
\BookTitle Nah  Na {Nahum}
\BookTitle Hab  Abk {Abakuk}
\BookTitle Zeph Sf {Sofonjáš}
\BookTitle Hag  Ag {Ageus}
\BookTitle Zech Za {Zacharjáš}
\BookTitle Mal  Mal {Malachiáš}
\BookTitle Matt Mt {Matouš}
\BookTitle Mark Mk {Marek}
\BookTitle Luke L {Lukáš}
\BookTitle John J {Jan}
\BookTitle Acts Sk {Skutky apoštolské}
\BookTitle Rom  Ř {Římanům}
\BookTitle 1Cor 1K {První list Korintským}
\BookTitle 2Cor 2K {Druhý list Korintským}
\BookTitle Gal  Ga {Galatským}
\BookTitle Eph  Ef {Efezským}
\BookTitle Phil Fp {Filipským}
\BookTitle Col  Ko {Koloským}
\BookTitle 1Thess 1Te {První list Tesalonickým}
\BookTitle 2Thess 2Te {Druhý list Tesalonickým}
\BookTitle 1Tim 1Tm {První list Timoteovi}
\BookTitle 2Tim 2Tm {Druhý list Timoteovi}
\BookTitle Titus Tt {Titovi}
\BookTitle Phlm  Fm {Filemonovi}
\BookTitle Heb   Žd {Židům}
\BookTitle Jas   Jk {List Jakubův}
\BookTitle 1Pet  1Pt {První list Petrův}
\BookTitle 2Pet  2Pt {Druhý list Petrův}
\BookTitle 1John 1J {První list Janův}
\BookTitle 2John 2J {Druhý list Janův}
\BookTitle 3John 3J {Třetí list Janův}
\BookTitle Jude  Ju {List Judův}
\BookTitle Rev   Zj {Zjevení Janovo}     

\BookException Ž   {\def\amark{Z}}
\BookException Př  {\def\amark{Pr}}
\BookException Pís {\def\amark{Pis}}
\BookException Ř   {\def\amark{R}}
\BookException Žd  {\def\amark{Zd}}

`. Tento soubor musí obsahovat příkazy
\`\BookTitle` ve formátu:
\begtt
\BookTitle <a-značka> <b-značka> {<nezkrácený titul knihy>}
\endtt
Mezi značkami a titulem knihy musí být aspoň jedna mezera.
Začátek takto čteného souboru může vypadat třeba takto:

\begtt
\BookTitle  Gn   Gen   {První Mojžíšova (Genesis)}
\BookTitle  Ex   Exod  {Druhá Mojžíšova (Exodus)}
\BookTitle  Lv   Lev   {Třetí Mojžíšova (Levicitus)}
\BookTitle  Nu   Num   {Čtvrtá Mojžíšova (Numeri)}
\BookTitle  Dt   Deut  {Pátá Mojžíšova (Deuteronomium)}
\BookTitle  Joz  Josh  {Jozue}
\BookTitle  Sd   Judg  {Soudců}
...
\endtt
V prvním sloupci za \`\BookTitle` jsou `<a-značky>`, které jsou dále
používány v textu poznámek a jsou pomocí nich tvořeny odkazy na různá místa
v Bibli.

Ve druhém sloupci jsou `<b-značky>`, které mohou být stejné jako
`<a-značky>`, ale mohou se také lišit. Je třeba možné, že názvy `.txs`
souborů vznikly exportem z nějakého softwaru a značky knih jsou tam jiné, než
potřebujeme použít v textu našich poznámek. Pak je možné v hlavním souboru
deklarovat umístění `.txs` souborů s využitím \`\bmark` místo \`\amark`, tedy
\begtt
\def\txsfile   {sources/Cze\tmark-\bmark.txs}
\endtt
a mít soubory `CzeBKR-Gen.txs`, `CzeBKR-Exod.txs`, zatímco v textu používáme
značky `Gn`, `Ex`, atd., nikoliv `Gen`, `Exod`.

Makro \`\amark` obsahuje `<a-značku>` právě zpracovávané knihy a dále makro
\`\bmark` zahrnuje `<b-značku>` právě zpracovávané knihy.

Poznamenejme, že makro \`\printedbooks` (v souboru `main.tex`) se značkami všech knih, které chceme
zpracovat (viz sekci~\ref[hlavni]), obsahuje `<a-značky>`, nikoli `<b-značky>`.

Ve třetím parametru za \`\BookTitle` jsou názvy knih ve složených závorkách.

Soubor s 66 údaji \`\BookTitle` %(protestantského kánonu) 
je vygenerován automaticky po extrakci
základních textů z Swordu pomocí `mod2tex` a `maketxs` (viz
sekci~\ref[txs]). Je možné jej použít, ovšem `<a-značky>` a tituly knih je
potřeba asi manuálně upravit podle zvyklostí daného překladu, jak je předvedeno
v ukázce výše.

Do souboru s názvy knih je možné přidat další informace o jednotlivých knihách
pomocí příkazů \`\BookException`, \`\BookPre`, \`\BookPost`. Mají tuto syntaxi:
\begtt
\BookException <a-mark> {<text-výjimky>}
\BookPre       <a-mark> {<text-před-knihou>}
\BookPost      <a-mark> {<text-po-knize>}
\endtt
Přitom <text-výjimky> se vloží před čtením všech souborů knihy vymezené značkou <a-mark>
uvnitř cyklu pro čtení všech knich příkazem `\processbooks`, dále
<text-před-knihou> se vloží po načtení souborů knihy, ale před
zpracováním prvního verše. Konečně <text-po-knize> se vloží za posledním
vešem knihy.

Příklad použítí \`\BookException`:
pokud chcete mít v názvech souborů `<a-značky>` a ty obsahují diakritická
znaménka, je možné upravit hodnotu makra \`\amark` tak, aby neobsahovalo
diakritiku takto:
\begtt
\BookException Ž   {\def\amark{Z}}
\BookException Př  {\def\amark{Pr}}
\BookException Pís {\def\amark{Pis}}
\BookException Ř   {\def\amark{R}}
\BookException Žd  {\def\amark{Zd}}
\endtt
Pak třeba soubory `notes-\amark.tex` jsou ve skutečnosti pojmenovány `notes-Z.tex`,
`notes-Pr.tex` atd. Přitom v textu poznámek dále normálně používáte značky
knih Ž, Př, Pís atd.

V Bibli je pět knih, které mají jen jednu kapitolu (Abdijáš, Filemon, 2 a 3 list Janův a Judův).
Vzhledem k tomu, že odkazy do nich se nepíší s číslem kapitoly (`Fm 1:4`), nýbrž jen číslem verše (`Fm 4`), 
musíme \TeX\ naučit, které to jsou, aby u nich číslo kapitoly
neočekával, ale přitom aby věděl, že je to číslo 1,~které se nikam nepíše.
Při odkazování na takové knihy se pak použije jiný způsob interpretace odkazu, viz sekci~\ref[vudaj].
Dosáhneme toho definováním makra \`\nochapbooks`, které musí obsahovat <a-značky> těchto knih:
`\def\nochapbooks {Abd Fm 2Jn 3Jn Ju}`, samozřejmě totožné s těmi, které jsou již uvedeny v definicích `\BookTitle` (v souboru `Cze-books.tex`).

\secc[txs] Formát základního textu, tzv. `.txs` soubory

Předpokládá se, že základní text Bible je uložen v souborech s příponou `.txs`
(text source). Každá z~66~knih Bible je uložena ve svém `.txs` souboru.
Názvy `.txs` souborů a jejich umístění musejí odpovídat deklaraci `\def\txsfile`
v~hlavním souboru `main.tex` (viz~sekci~\ref[hlavni]).

Každý řádek `.txs` souboru obsahuje jeden verš Bible uvozený 
`#<číslo-kapitoly>:<číslo-verše>`. Verše musejí být uvedeny ve správném
pořadí. Například začátek souboru `CzeBKR-Da.txs` vypadá takto (části textu
jsou v ukázce vynechány):

\begtt
#1:1 Léta třetího kralování Joakima krále Judského, ... a oblehl jej.
#1:2 I vydal Pán v ruku jeho Joakima ... do domu pokladu boha svého.
...
\endtt

Základní texty Bible lze získat například z modulů Swordu
\url{https://www.crosswire.org/sword/modules/ModDisp.jsp?modType=Bibles}.
Jednotlivé `.txs` soubory je pak možné vygenerovat následujícím postupem (na Linuxu):

ZIP stažený z uvedené www stránky (tzv. modul) někam rozbalte. Je třeba
mít dále v počítači instalován balíček `libsword-dev` a dále program
`mod2tex`, který je součástí \OpBible/. Pomocí `installmgr -l` zjistíte
seznam stažených modulů. Máte-li nastaven aktuální adresář v místě, kam jste
rozbalili ZIPy, a vznikl tam adresář `modules`, pak jsou takové moduly
nalezeny. Moduly obsahují texty v binárním formátu, my je potřebujeme
převést do textového formátu. K tomu stačí napsat do příkazového řádku: 
\begtt 
mod2tex modul > soubor
\endtt 
kde `modul` je název modulu. Ve výsledném souboru
máte kompletní základní text daného překladu (modulu).
Například po
\begtt
mod2tex CzeBKR > CzeBKR.out
\endtt
je v souboru `CzeBKR.out` kompletní překlad Bible kralické.
Ten nyní můžete rozdělit do `.txs` souborů příkazem
\begtt
maketxs CzeBKR.out
\endtt
Uvedený příkaz kromě 66 `.txs` souborů vytvoří soubor `CzeBKR-books.tex`, ve
kterém jsou tituly a zkratky jednotlivých knih, tedy je tam:
\begtt
\BookTitle Gen Gen {Genesis}
\BookTitle Exod Exod {Exodus}
\BookTitle Lev Lev {Leviticus}
\BookTitle Num Num {Numbers}
\BookTitle Deut Deut {Deuteronomy}
\BookTitle Josh Josh {Joshua}
\BookTitle Judg Judg {Judges}
...
\endtt
Tyto tituly nejsou v češtině, protože je zdroj Sword neobsahuje. Je tedy
třeba tento soubor manuálně upravit a vložit místo anglických české názvy.
Zkratky knih jsou uvedeny dvakrát stejně. První pozměňte podle zvyklostí
používání zkratek v českých Biblích a druhou zkratku můžete ponechat.
Pak deklarace 
\begtt
\def\txsfile  {sources/Cze\tmark-\bmark.txs}
\endtt
v hlavním souboru způsobí, že se vytvořené `.txs` soubory budou hledat 
v~adresáři `sources/` a předpokládají se jejich názvy
`CzeBKR-Gen.txs`, `CzeBKR-Exod.txs`, `CzeBKR-Lev.txs`, atd.

Pokud by měl nastat ten vzácný případ, že byste skládali Bibli z několika různých zdrojů, řekněme Starý Zákon byste chtěli v překladu Dr. Jana Hejčla a Nový Zákon v překladu Františka Žilky, budete muset trochu žonglovat s názvy souborů, aby výsledná definice `\tmark` byla stejná pro celou Bibli. Oba `*.out` soubory (např. `HEJCL.out` a `ZILKA.out`)\fnote{Jejich moduly nejsou na Swordu, `*.out` soubory by musely vzniknout stažením např. z `https://obohu.cz` a nějakým šikovným skriptíkem (nebo ručně?) převedením do požadované podoby.} by bylo nutno spojit v jeden a ten pak pojmenovat třeba `CzeHecjlZilka.out` a pak teprve provést 
`maketxs CzeHejclZilka.out`.  Pak lze mít v hlavním souboru `main.tex` definici    `\def\tmark{HejclZilka}` a 
výsledná studijní Bible bude mít Hejčlův Starý Zákon a Žilkův Nový.








Se soubory `.txs` nadále nebudete
potřebovat nijak manipulovat, není potřebné je editovat ani doplňovat. V
adresáři `sources/` tedy můžete mít \uv{datový sklad} všech základních textů
Bible pro všechny použité varianty překlady najednou. V případě šesti variant
překladu tam tedy máte $6\times66=396$ souborů.

Máte-li v `.txs` souborech něco, co chcete nakonec formátovat jinak, je možné
v hlavním souboru použít \`\cnvtext``{<co>}{<jak>}`. \TeX/ v každém verši z
`.txs` souboru vyhledá všechny výskyty `<co>` a nahradí je `<jak>`.
Například máte-li v `.txs` souborech úseky textu v hranatých závorkách, tedy
`[něco takového]` a chcete je tisknout kuzívou, do hlavního souboru napište:
\begtt
\cnvtext{[}{\bgroup\it}  \cnvtext{]}{\/\egroup}
\endtt

Může se stát, že `.txs` soubor nepoužívá správné typografické uvozovky (tj. v
češtině „...“), ale místo nich tam jsou programátorské uvozovky {\tt"}...{\tt"}. 
Bez zásahu do `.txs` souboru je možné toto opravit přidáním pokynu do
hlavního souboru:
\begtt
\quotationmarks{„}{“}
\endtt
Pak se budou automaticky nahrazovat programátorské uvozovky v `.txs` souboru českými
typografickými. Podobně můžete v hlavním souboru deklarovat nahrazování
anglickými uvozovkami či jinými, například deklarací
`\quotationmarks{`“`}{`”`}`.
Programátorská uvozovka je pak implicitně
nahrazena otevírací typografickou (první parametr v deklaraci), ale pokud za
ní následuje mezera, konec verše, konec odstavce, tečka nebo čárka, je
nahrazena zavírací typografickou uvozovkou.

\secc[fmt] Údaje určující formátování základního textu v souborech `fmt-*.tex`

Základní text v `.txs` souborech neobsahuje
žádné formátovací ani doplňkové údaje, jakými jsou například názvy kapitol
nebo místa, kde se má ukončit odstavec či přejít z formátování v bloku do
formátování na střed řádků a zpět.

%\mnote{\code{\\fmtadd} \code{\\fmtpre} \code{\\fmtins}}
Protože do základního textu nechceme nijak zasahovat,\fnote{Řekněte si, že to jsou  svatá Písma, tudíž \uv{nedotknutelný} text. Jediné, co Vás oprávní do základního textu zasáhnout, je ta málo pravděpodobná možnost, že byste ve zdroji ze Swordu (nebo odkud jste ho získali) objevili chybu vůči tištěné verzi. Tu pak bude skutečně lepší odstranit na místě.} je třeba tyto doplňkové
údaje deklarovat k příslušným veršům pomocí speciálních příkazů \`\fmtadd`,
\`\fmtpre` a \`\fmtins`. Tyto příkazy jsou typicky v souborech `fmt-*-*.tex`,
například `fmt-CzeBKR-Da.tex`. Je vhodné udržovat tyto formátovací soubory
závislé jednak na knize (Daniel v uvedeném příkladu), ale také na použitém
překladu (Bible kralická v příkladu). Je sice možné začít s jedním souborem
pro každou knihu a soubory pro ostatní překlady pořídit jako kopie výchozích,
ale nakonec bude asi potřebné pro rozličné varianty základního textu
formátovací pokyny mírně modifikovat v souladu s použitým překladem.

Syntaxe použití uvedených příkazů je následující:
\begtt
\fmtpre{<číslo-kapitoly>:<číslo-verše>}{<fmt-pokyn>}
\fmtadd{<číslo-kapitoly>:<číslo-verše>}{<fmt-pokyn>}
\fmtins{<číslo-kapitoly>:<číslo-verše>}{<fráze>}{<fmt-pokyn>}
\endtt
kde <fmt-pokyn> je \uv{vzkaz}, který je předán \TeX/u k formátování.
%\mnote{\code{\\endgraf} \code{\\begcenter} \code{\\endcenter}}
Například \`\endgraf` značí konec odstavce. \`\begcenter` otvírá pasáž s
centrovaným textem a ta musí být (někde později) uzavřena pomocí \`\endcenter`. Nebo 
\`\chaptit{<text>}` vloží <text> jako titul kapitoly, kdežto
\`\schaptit{<text>}` vloží nadpis perikopy jinam, než před první verš kapitoly a udělá nad a pod nadpisem patřičné místo.

%\mnote{\code{\\fmtadd} \code{\\fmtpre} \code{\\fmtins}}
Příkaz \`\fmtpre` vkládá <fmt-pokyn>
na začátek uvedeného verše (ještě před případně vytištěné číslo verše horním
indexem). Příkaz \`\fmtadd` vkládá <fmt-pokyn> na konec uvedeného verše.
Konečně \`\fmtins` vkládá <fmt-pokyn> dovnitř verše za první výskyt stanovené
<fráze>, která ve verši musí doslovně existovat. Jinak \TeX/ vypíše varování
a <fmt-pokyn> nezařadí vůbec.

Jak může vypadat použití příkazů `\fmt*` je vidět například v souboru
`fmt-BKR-Da.tex`.

Příkaz \`\fmtfont``{<číslo-kapitoly>:<číslo-verše>}{<fráze>}{<font>}`
slouží k označení vybrané fráze zvoleným fontem.
Například `\fmtfont{1:26}{lidé}{\em}` v knize Genesis způsobí, že slovo
\"lidé" bude vytištěno kurzívou, protože `\em` je inteligentní přepínač do kurzívy (automaticky doplňuje za slovo italickou korekci `\/`, o kterou se nemusíte starat, což byste museli, kdybyste do kurzívy přepínali obyčejným `\it`).
Namísto `\em` je možno použít jakýkoli jiný přepínač fontu.

Mimo `\begcenter` a `\endcenter` je také možné použít řízené odsazování pomocí
\`\ind``<číslo>` (jako indent). V~místě vložení se ukončí řádek a další řádek začíná
odsazen o `<číslo>` odstavcových zarážek. Vkládání takových `\ind` příkazů
prostřednictvím `\fmtins` nebo `\fmtpre` může být ale docela pracné a
nepřehledné, přitom Bible se zhusta hemží poetickými pasážemi, které vyžadují spousty
různě odsazených řádků. K tomu účelu může sloužit příkaz \`\fmtpoetry`, jehož
použití nejprve ukážeme na příkladě pro překlad CEP knihy Jeremiáš 23:9--15:
%(srovnejte si to se skutečnou sazbou příslušných veršů této knihy):
\medskip
\picw=.5\hsize
\centerline{\inspic{images/CEP-Jr-7-12.png} }
\vskip-2pt  % je to ze dvou stran, tak aby to vypadalo trochu plynule
\centerline{\inspic{images/CEP-Jr-12-14.png} }

Poetická část této ukázky byla vysázena takto:
\begtt
\fmtpre{23:9}{\endgraf}
\fmtpoetry{23:9}{prorocích:// nitru,/// chvějí,// člověk,/// vínem,///}
\fmtpoetry{23:10}{// cizoložníků,/// kletbou/// vyschly;/// zlem///}
\fmtpoetry{23:11}{// rouhají,/// činy,///}
\fmtpoetry{23:12}{// kluzká,/// temnoty// padnou;/// zlo,// trestu,///}
\fmtpoetry{23:13}{// nepatřičnost:/// Baalově///}
\fmtpoetry{23:14}{// věc:/// klamání.// zlovolníků,/// činů.// Sodoma///}
\fmtins{23:9}{prorocích:}{\bigskip}
\fmtpre{23:13}{\medskip}
\fmtpre{23:14}{\medskip}
\fmtpre{23:15}{\bigskip}
\endtt

\`\fmtpoetry``{<číslo-kapitoly>:<číslo-verše>}{<formátovací-údaje>}` stanoví
formátování konkrét%\-článku provedete pomocí \`\insertCite` `<písmeno>\left` a současně na jiném místě v textu článku
ně uvedeného verše. Ve <formátovacích-údajích> se
vyskytují slova z konců řádků následovaná jedním nebo více lomítky. Počet
lomítek udává, o kolik odstavcových zarážek bude odsazen další řádek za
uvedeným slovem. <formátovací-údaje> musejí nutně skončit jedním nebo více
lomítky a mohou (ale nemusí) začínat jedním nebo více lomítky, Pokud
začínají, je i začátek verše odsazen o~příslušný počet odstavcových zarážek,
ale číslo verše je vysazeno vystrčené mírně doleva do prostoru odstavcové
zarážky.

Chceme-li při použití \`\fmtpoetry` navíc vložit vertikální mezery, je to
možné, ale až po příkazu `\fmtpoetry`, jak je vidět i v ukázce, kde je
vložena mezera polovičního řádku \`\medskip` a mezera celého řádku
\`\bigskip`. Platí totiž pravidlo, že pokud na stejné místo pomocí `\fmtins`
nebo `\fmtpre` vložíte více formátovacích pokynů, ty se nakonec provedou v
opačném pořadí, než byly do `.fmt` souboru vloženy, Takže  například u verše 23:13
výše uvedené ukázky se nejprve provede na začátku verše `\medskip` a
pak se následující řádek odsadí o dvě odstavcové zarážky podle `\fmtpoetry` (`\fmtpoetry{23:13}{// nepatřičnost:...}`).

\secc Poznámky v souborech `notes-*.tex`

Tyto soubory (umístěné podle deklarace `\notesfile` v hlavním souboru)
obsahují mimo jiné poznámky k jednotlivým veršům nebo částem veršů. K psaní poznámek
slouží zejména příkaz \`\Note`. Protože toto je hlavní důvod, proč byl
vytvořen balíček \OpBible/, je tomu věnována celá sekce~\ref[note].

\sec Poznámky a další objekty s vazbou na základní text

\secc[note] Poznámky vázané na fráze základního textu: příkaz `\Note`

Hlavním smyslem nástroje \OpBible/ je vytvořit PDF ze základního textu nejen
se samotným textem Bible, ale s propojenými poznámkami. K tomu slouží
mimo jiné soubory `notes-*.tex` (například `notes-Gn.tex` pro knihu Genesis), ve
kterých jsou zapsány poznámky k jednotlivým frázím základního textu dle
následujícího smluveného formátu. Jednotlivá poznámka 
je uvozena příkazem \`\Note` a má tvar:

%\mnote{\code{\\Note}}
\begtt
\Note <číslo-kapitoly>:<číslo-verše> {<fráze>} <text-poznámky>
<prázdný řádek>
\endtt
Například:
\begtt
\Note 1:2 {nádobí} Odkaz na nádobí z vypleněného chrámu, nikoliv na deportaci zajatců.

\endtt
Příklad je ze souboru `notes-Da.tex`, tedy ze souboru s poznámkami ke knize
Daniel. Konkrétně se poznámka váže na kapitolu první, verš druhý a na frázi
\uv{nádobí}. Tato fráze musí být bezpodmínečně ve stanoveném verši
základního textu obsažena. Pak ji \TeX/ propojí s odpovídajícím místem 
v~základním textu, tj. zajistí, že výskyt fráze i samotné poznámky je na
společné stránce. Jinými slovy, stránky se nelámou podle čísla verše a jemu odpovídajícího čísla poznámky; lámou se podle komentované fráze.
Když stránkový zlom vyjde uvnitř verše, v němž komentujeme více než jednu frázi, poznámky následují své fráze, nejen začátek verše.
V ukázkové knize Daniel je to vidět v překladech CEP a B21 na verši 9:27 -- jedna fráze na jedné straně, druhá na druhé, a stejně tak i poznámky k~nim. 

Povšimněme si, že fráze se váže na překlad Bible kralické, zatímco v
překladu ekumenickém je použit ve stejném verši termín \uv{nádob}. Jak
zařídit, aby se stejná poznámka automaticky mírně modifikovala podle
použitého překladu, je popsáno v sekci~\ref[ww].

Pokud <fráze> není ve stanoveném verši základního textu doslovně obsažena,
\TeX/ během zpracování oznámí do logu a na terminál o tom varování a
poznámku zařadí k danému verši, jakoby <fráze> byla na začátku verše.

Vytištěná poznámka (v závislosti na typografickém nastavení) obsahuje
například zopakování čísla kapitoly a verše, pak následuje komentovaná fráze
(například tučně) a za ní je vlastní text poznámky.

Někdy je třeba v základním textu vyhledat mírně jinou frázi, než jakou pak
chceme zapsat do vytištěné poznámky (např. v základním textu je slovo v
jiném pádě nebo je to mírně jinak formulovaná fráze). Pak je možné těsně za
`{<fráze>}` napsat rovnítko následované `{<fráze-k-tisku>}`, tedy

\begtt
\Note <číslo-kapitoly>:<číslo-verše> {<fráze>}={<fráze k tisku>} <text-poznámky>
<prázdný řádek>
\endtt
V takovém případě se v základním textu hledá <fráze>, ale ve vlastní
poznámce se vytiskne jako heslo poznámky <fráze-k-tisku>. Například:
\begtt
\Note 1:2 {vydal Pán}={Pán vydal} Porážka Izraele Babylónem není vysvětlitelná
   jen pouhou vojenskou a politickou analýzou oné doby. Bůh vždy jednal svrchovaně 
   v záležitostech národů. Babylóňany použil jako nástroj potrestání svého vlastního
   lidu za porušení smluvních závazků.

\endtt
V tomto příkladě se hledá ve verši druhém základního textu \uv{vydal Pán} 
ale v poznámce je vytištěno heslo \uv{Pán vydal}.

Povšimněte si, že tato poznámka se rovněž váže na verš druhý kapitoly první
knihy Daniel (protože je v souboru `notes-Da.tex`). Je tedy možné mít více
poznámek k různým frázím, které se váží na stejný biblický verš. 

Jednotlivé poznámky ve zdrojovém souboru poznámek jsou odděleny prázdnými
řádky. To je nutné, jinak by \TeX/ při jejich čtení nepoznal, kde končí text
poznámky. Také se tím zvyšuje přehlednost zdrojového souboru. Další řádky
poznámky můžete (ale nemusíte) odsadit. 
%Obecná pravidla pro zápis zdrojových textů pro \TeX/ nejsou předmětem tohoto manuálu, najdete je v různých učebnicích k~\TeX/u.


Pokud se poznámka váže k celému verši (tj. bez specifikované fráze), pište
`{}`, tedy prázdnou frázi k vyhledání. Například:
\begtt
\Note 1:1-21 {}={Uchování rituální čistoty}  Prorok uvozuje kontext své knihy
   vyprávěním osobní historie (své i svých přátel) zajetí, vzdělání, věrnosti Bohu
   a služby králi Nabuchodonozorovi.
\endtt
Tato ukázka navíc demonstruje, že je také možné uvést celý rozsah čísel veršů.
Je-li <fráze> pro vyhledání prázdná (jako v tomto příkladě), pak se poznámka
umístí na stejnou stranu, kde je začátek prvního verše z~uvedeného rozsahu.
Je-li <fráze> neprázdná, musí se vyskytovat v prvním verši z~uvedeného
rozsahu veršů. Rozsah veršů bude stejně vytištěn v úvodu poznámky.
Symbol pro rozsah „`-`“ je jediný znak \uv{mínus} běžně dosažitelný na
klávesnici. Má kód~45. Nesmí to být žádný speciální znak vypadající třeba jen podobně jako
vodorovná čárka.

Pořadí vytištěných poznámek k jednomu verši odpovídá pořadí frází, na které
se vážou, v základním textu. Na pořadí, ve kterém jsou zapsány ve zdrojovém
souboru to tedy nemá vliv. Poznámky, které se vážou pomocí prázdného parametru `{}` k celému
verši, jsou řazeny jako první a je-li jich více, pak jsou řazeny podle pořadí
ve zdrojovém souboru.

\secc[clanky] Příkazy pro vložení dalších objektů

Do souboru `notes-*.tex` je možné kromě příkazů \`\Note` zapisovat pokyny pro vložení dalších objektů s vazbou na konkrétní verš v základním textu. Takový objekt se umístí\fnote
{Podle implicitní definice vzhledu jednotlivých stran, to je možné předefinovat.}
% Ale my to nemáme předefinováno, ne? Nemůžeme tady uvést příklad, což je škoda. Nebo to nějak jde?
na spodní část strany pod dva sloupce poznámek nebo citáty se umístí na
horní část strany. Vazbu na místo textu vymezíme pomocí údaje 
`<číslo-kapitoly>:<číslo-verše>` podobně jako v případě poznámek `\Note`. Objekt se umístí na stejnou stranu, kde začíná příslušný verš.
Pokud se na stránku pod verš objekt nevejde, umístí se na spodní část
následující stránky.
%% Asi by to chtělo nějaké příklady

Obrázky vložíte pomocí \`\putImage` (sekce~\ref[putimage]), články pomocí \`\putArticle` (sekce~\ref[putarticle]), citace pomocí \`\putCute` (sekce~\ref[topcite]), obrázky přes dvě stránky v otevřené knize pomocí \`\putSpanImage` nebo \`\putStanText` (sekce~\ref[spanimage]).

Pořadí, v jakém do souboru `notes-*.tex` příkazy na vožení těchto objektů napíšete, nemá na konečný vzhled po vytištění žádný vliv. Můžete mít všechny objekty jednoho typu soustředěny na jednom místě zdrojového souboru, ale taky je můžete mít umístěny mezi poznámkami `\Notes` typicky podle čísla verše, ke kterému jsou objekty vázány.




\secc[putimage] Vložení obrázků

Obrázky s vazbou na konkrétní verš umístíme příkazem \`\putImage`:

%\mnote{\code{\\putImage}}
\begtt
\putImage <číslo-kapitoly>:<číslo-verše> {<titulek>} [<lejblík>] (<parametry>) {<soubor>}
\endtt
Například:

\begtt
\putImage 2:1 {Danielovy vize} [danielovyvize] () {Nabuco.pdf}
\endtt

\begitems
* `<číslo-kapitoly>:<číslo-verše>` označuje místo, ke kterému se obrázek umístí.
* `<titulek>` je titulek (popisek) k obrázku. V implicitním typografickém
  nastavení se vytiskne do proužku, který odděluje horní část strany od
  samotného obrázku.
* `<lejblík>` udává lejblík k obrázku pro případné reference. Můžete pak na
  něj odkazovat pomocí `...viz obrázek` \`\ref``[<lejblík>]` na straně \`\pg`, což
  vytiskne v případě naší ukázky ...viz obrázek {\Blue Danielovy vize} na straně~123.
  Přitom text v tomto odkazu bude propojen hyperlinkem s místem, kde se
  vyskytuje obrázek. Když nechcete `<lejblík>` použít, pište `[]`
  (tj.~jakoby prázdný lejblík).
* `<parametry>` určují případný další pokyn pro formátování obrázku. Mohou
  chybět, jak naznačuje ukázka.
  Implicitně se obrázek roztáhne na celou šířku stránky. Pokud ho chcete mít
  třeba menší, pište do místa pro `<parametry>` třeba `\picw=9cm` což
  způsobí, že obrázek bude mít šířku 9\,cm a bude umístěn na střed.
* `<soubor>` je úplné jméno souboru s obrázkem. Může mít příponu `pdf` (pro
  vektorové obrázky) nebo `png` či `jpg` pro bitmapové obrázky.
  Soubory s obrázky musejí být uloženy v adresáři `images/`. Jsou-li
  umístěny jinde, je třeba nastavit pro toto umístění parametr \`\picdir`,
  například `\picdir={bibleImages/}`. Toto je rozumné nastavit v hlavním
  souboru, viz sekci~\ref[hlavni].
\enditems

\secc[putarticle] Vložení článků

Článek s textem (typicky delším než poznámka) lze umístit podobně jako
obrázky, tj. při implicitním typografickém nastavení na spodní stranu
stránky, na níž začíná verš specifikovaný pomocí
`<číslo-kapitoly>:<číslo-verše>`. Nevejde-li se tam, vloží se na spodní část
následující strany. Nevejde-li se na jedinou stranu, vloží se jeho další
úsek na spodní část následující strany (atd. až do chvíle, kdy je celý
článek do textu vložen).

Text článků ke každé knize Bible zapisujte do souboru nazvaném a 
uloženém podle makra \`\articlefile`, například
v souboru `articles-Gn.tex` jsou všechny články pro knihu Genesis. Jak obsah
tohoto souboru vypadá, popíšeme za chvíli.

V souboru `notes-*.tex` je třeba uvést požadavek na umístění článku
pomocí \`\putArticle`:

%\mnote{\code{\\putArticle}}
\begtt
\putArticle <číslo-kapitoly>:<číslo-verše> {<titulek>} [<číslo-článku>] (<parametry>)
\endtt
Například:
\begtt
\putArticle 6:1 {Kdo byl Darius Médský?} [6] ()
\endtt

\begitems
* `<číslo-kapitoly>:<číslo-verše>` označuje místo, ke kterému se obrázek umístí.
* `<titulek>` je titulek ke článku. Vytiskne se podobně jako v případě
  obrázku.
* `<číslo-článku>` je nějaké číselné označení článku, které musí být v rámci
  jedné knihy jednoznačné. Typicky se články vážou na kapitoly, pak by
  `<číslo-článku>` mělo znamenat číslo odpovídající kapitoly. Podle tohoto
  čísla se článek do sazby neumisťuje (k tomu slouží údaj
  `<číslo-kapitoly>:<číslo-verše>`), nicméně je toto číslo použité v
  odkazech. Můžete například psát \code{...viz <"článek" Da 6>a} a vytiskne se
  ...viz {\Blue článek~Da~6}. Formát a vlastnosti takových odkazů
  popisuje sekce~\ref[odkazy]. Navíc se podle `<čísla-článku>` vyhledá
  vlastní text článku v příslušném souboru `atricles-*.tex`.
* `<parametry>` jsou parametry přesněji určující formátování článku.
  \TODO...
\enditems

Text článků ke vložení musí být uveden v souboru `articles-*.tex`.
Tento soubor musí obsahovat řádek uvozený \`\Article` takto:
%\mnote{\code{\\Article}}
\begtt
\Article [<číslo-článku>]
\endtt
Za ním následuje text článku. Pak může pokračovat další
`\Article [<číslo-článku>]` následovaný další textem atd. V jednom souboru
jsou tak shomážděny všechny články k dané knize.

Pokud zadáte požadavek \`\putArticle` a příslušný soubor `article-*.tex`
nebo příslušný řádek v něm \`\Article` `[<číslo-článku>]` neexistuje, zpracování \TeX em
skončí s chybou.


\secc[topcite] Vložení citací do horní části stránky

%\mnote{\code{\\putCite}}
Příkaz \`\putCite` `<číslo-kapitoly>:<číslo-verše> {<text>}` vloží při
implicitním typografickém nastavení <text> ve formě citátu na začátek
stránky, na které se nachází `<číslo-kapitoly>:<číslo-verše>`.

Samotný <text> může obsahovat na konci \`\quotedby {<autor>}`. V takovém
případě je vytištěn <autor> na novém řádku (při citátu umístěném na levé
stránce) nebo za předpokladu, že je dost místa, je na posledním řádku posunutý více
vpravo (při citátu umístěném na levé stránce).

Pro řízený přechod na nový řádek můžete v <textu> požít příkaz \`\nl` (new line).


\secc[cite] Vložení citací do okraje v textu článku

Text článků je v implicitním nastavení sazby formátován do dvou sloupců.
Vnější sloupec je možné přerušit a do vzniklého místa vsunout např. citát či
jinou myšlenku. Tento text bude vyčuhovat do vnějšího okraje.

Uvnitř článku (tedy těsně po příkazu \`\Article` `[<číslo-článku>]` je
třeba vložit deklaraci citovaného textu pomocí:
\`\Cite` `<písmeno> {<text>}`. %\mnote{\code{\\Cite}}
Zde <písmeno> je typicky `A`. Pokud ale chcete do jednoho článku vložit více
citátů, je třeba je odlišit dalšími písmeny, tedy `B`, `C`, atd. a všechny
citáty napište na začátek článku.

%\mnote{\code{\\insertCite}}
Příkazem \`\Cite` je citát pouze deklarován. Jeho skutečné vložení do textu
článku provedete pomocí \`\insertCite` `<písmeno>\left` a současně na jiném místě v textu článku
použijte \`\insertCite` `<písmeno>\right`.
Je-li článek na levé stránce dvoustrany, citát se umístí
jen podle `\insertCite <písmeno>\left`, jinými slovy `\insertCite <písmeno>\right` je
ignorováno. Je-li článek na pravé stránce otevřené dvoustrany, umístění se
řídí jen podle `\insertCite <písmeno>\right`. Hned pod řádkem, ve kterém je
příslušné `\insertCite` uvedeno, se citát skutečně objeví. Řádek samotný se
kvůli tomu nerozdělí (jinak řečeno, odstavec se kvůli vložení citátu neukončí).

Důvod, proč je nutné uvést dvě lokace \`\insertCite` je následující. Dopředu
nevíme, na které stránce (liché nebo sudé) se článek s vloženým citátem
objeví. Protože citát má zasahovat do vnějšího okraje, je třeba jej na levé
stránce umístit do prvního sloupce, na pravé stránce do druhého sloupce,
tedy někam poněkud jinam. Lokace `\insertCite <písmeno>\left` by tedy měla
odpovídat nějakému řádku v prvním sloupci a `\insertCite <písmeno>\right`
jinému řádku v druhém sloupci článku.

Je rozumné si odladit (tj. dopředu vyzkoušet), jak vypadá umístění citátu
pro obě varianty, které mohou nastat (vlevo/vpravo). Pokud se chcete
podívat, jak to dopadne při variantě, která neodpovídá správné poloze
stránky, můžete použít na začátku textu článku příkaz
\`\swapCites`, který jejich pozice vzájemně prohodí. %\mnote{\code{\\swapCites}}
Toto by ale nemělo zůstat zapnuté pro definitivní tisk, proto aktivované
`\swapCites` způsobí varování na terminál a do log souboru.


\sec[preklady] Různé (ale podobné) verze základního textu

Úvod do této problematiky je v sekci~\ref[vars]. Zde je k jednotlivým možnostem podrobná dokumentace.


\secc[cmdx] Deklarace variant překladu a použití pomocí příkazu `\x`

Nejsou-li varianty deklarovány příkazem \`\vdef`, pak
příkaz \`\x``/<fráze>/` použitý v textu vypíše <fráze>. Je ovšem možné
deklarovat varianty překladu. Počet variant je nutné stanovit příkazem `\variants`
(viz sekci~\ref[vars]). To se provede jedinkrát v celém dokumentu. Pak mohou následovat příkazy `\vdef` vždy s tolika parametry, kolik variant bylo specifikováno příkazem `\variants`. Například 

\begtt \typosize[8.5/11]
\variants 6 {BKR}      {PSP}         {CSP}       {CEP}       {B21}         {SNC}
\vdef       {Baltazar} {Béltešaccar} {Beltšasar} {Beltšasar} {Baltazar}    {Beltšasar}
\vdef       {Darius}   {Dárjáveš}    {Dareios}   {Darjaveš}  {Darjaveš}    {Darjaveš}
\vdef       {Daria}    {Dárjáveše}   {Dareia}    {Darjaveše} {Darjaveše}   {Darjaveše}
\vdef       {Daniel}   {Dánijjél}    {Daniel}    {Daniel}    {Daniel}      {Daniel} 
\vdef       {sedm let} {sedm dob}    {sedm časů} {sedm let}  {sedm období} {po dobu sedmi let}
\endtt

Pokud je nyní v hlavním souboru deklarováno `\def\tmark {<varianta>}`,
pak se `\x``/<fráze>/` promění na <frázi> specifikované <varianty>.
Přitom parametr <fráze> příkazu \`\x` musí být shodný s~první frází uvedenou
v příkazu \`\vdef`. Je-li v v našem případě v hlavním souboru `\def\tmark{BKR}`, pak
\begtt
\x/Baltazar/        vypíše Baltazar
\x/Darius/          vypíše Darius
\x/sedm let/        vypíše sedm let
\x/cokoli/          vypíše cokoli
\endtt

Je-li ale v hlavním souboru `\def\tmark{CEP}`, pak
\begtt
\x/Baltazar/        vypíše Beltšasar
\x/Darius/          vypíše Darjaveš
\x/sedm let/        vypíše sedm let
\x/cokoli/          vypíše cokoli a na terminálu bude varování,
                                  že fráze /cokoli/ nemá deklarován překlad.
\endtt
Například poznámka v souboru `notes-Da.tex` (pro knihu Daniel) by mohla znít:
\begtt \catcode`\<=12
\Note 5:31 {Darius}={\x/Darius/ Médský} Některé školy tvrdí, že tento a další 
   (<6:1>, <6:6>, <6:9>, <6:25>, <6:28>; <9:1>; <11:1>)     
   odkazy na \x/Daria/ Médského v knize \x/Daniel/ jsou historické omyly.
\endtt
a tato poznámka se při `\def\tmark{CEP}` vypíše ve znění:

{\medskip \leftskip=\parindent \noindent
{\bf 5:31 Darjavéš Médský} \ Některé školy tvrdí, že tento a další 
(6:1, 6:6, 6:9, 6:25, 6:28; 9:1; 11:1)     
odkazy na Darjavéše Médského v knize Daniel jsou historické omyly.
\medskip}

Vybrané parametry příkazu \`\vdef` mohou být prázdné (zapsané jako~`{}`),
což značí nedefinovanou frázi pro daný jazyk.
Bude-li potřeba takovou frázi použít pomocí \`\x``/.../`, vypíše se varování.
Dále parametr může obsahovat jediný znak `"`, což znamená, že se
použije stejná fráze, jako v předchozím parametru. Takže naše
výše uvedená ukázka může také vypadat takto:
\begtt \typosize[8.5/11]
\vdef       {Baltazar} {Béltešaccar} {Beltšasar} {"}         {Baltazar}    {Beltšasar}
\vdef       {Darius}   {Dárjáveš}    {Dareios}   {Darjaveš}  {"}           {"}
\vdef       {Daria}    {Dárjáveše}   {Dareia}    {Darjaveše} {"}           {"}
\vdef       {Daniel}   {Dánijjél}    {Daniel}    {"}         {"}           {"} 
\vdef       {sedm let} {sedm dob}    {sedm časů} {sedm let}  {sedm období} {po dobu sedmi let}
\endtt

\pozor Uvozovky na místě fráze nelze použít v definicích `\ww` (viz~kapitola~\ref[ww]), které  předcházejí `\Note`. Fungují pouze v definici `\vdef` v souboru s variantami, platnými pro
celou Bibli. 


\secc[ww] Variantní deklarace frází pro spárování poznámek s textem

Před příkazem \`\Note` může bezprostředně předcházet deklarace vyhledávaného
slova podle variant překladu pomocí \`\ww` (to je zkratka za watchword).
Příkaz \`\ww` má tolik parametrů, kolik je variant překladu deklarovaných
příkazem \`\variants` a tyto parametry mohou být jednoduché (ve formátu
`{<hledaná-fráze>}`) nebo složené (ve formátu
`{<hledaná-fráze>}={<co-vytisknout>}`). Bezprostředně následující `\Note`
pak bude ignorovat svůj parametr pro hledanou frázi a použije parametr z \`\ww`
odpovídající zpracovávané variantě jazyka. Například:

\begtt
\ww {liternímu umění a jazyku}  %BKR
    {písmu a jazyku Kasdím} %PSP
    {chaldejskému písemnictví a jazyku}  %CSP
    {kaldejskému písemnictví a jazyku} %CEP
    {babylonskému písemnictví a jazyku}  %B21
    {kaldejský jazyk}={kaldejský jazyk ... s národním písemnictvím}  %SNC
\Note 1:4 {} Babylónská literatura byla ...
\endtt

Zde uvedená poznámka `Note 1:4` vyhledá ve verši 1:4 text \"liternímu umění a jazyku",
když je zpracovávaná varianta překladu BKR, a vyhledá text
\"kaldejskému písmenictví a jazyku" při zpracování varianty překladu CEP.
Ukázka předpokládá, že bylo pomocí příkazu `\variants` deklarováno šest
variant překladu ve stanoveném pořadí. Můžete též stanovit rozdílnou frázi
pro vyhledání a pro vytištění v poznámce, jak ukazuje následující příklad:

\begtt
\ww {Baltazar}  %BKR
    {Béltešaccar}={Béltešaccar ... Šadrach ... Méšach ...i Avéd-negó} %PSP
    {Beltšasar}={Beltšasar ... Šadrak ... Méšak ...i Abed-nego}  %CSP
    {Beltšasar}={Beltšasar, Chananjášovi Šadrak, Míšaelovi Méšak a Azarjášovi Abed-nego}
    {Baltazar}={Baltazar, Chananiášovi Šadrach, Mišaelovi Mešach a Azariášovi Abednego}
    {Beltšasarem}={Beltšasarem, Chananjáše Šadrakem, Míšaele Méšakem a Azarjáše Abed-negem}
\Note 1:7  {} Přesný význam těchto jmen je předmětem diskusí...
\endtt

Fráze pro hledání a nahrazení se použijí přesně tak, jak jsou napsány v
parametrech příkazu \`\ww`. To neplatí pro poznámky \`\Note`, které nemají
před sebou \`\ww`. Pak při použití

\begtt
\Note <verš> {<hledaná-fráze>} <text> <prázdný řádek>
\endtt
nebo
\begtt
\Note <verš> {<hledaná-fráze>}={<co-vytisknout>} <text> <prázdný řádek>
\endtt
se <hledaná-fráze> nejprve transformuje podle dat z `\vdef`. Teprve pokud
tato data pro hledanou frázi neexistují, použije se <hledaná-fráze> tak,
jak je.

\doporucujeme Pro přehlednost se vyplatí mít každý překlad na novém řádku, a za komentujícím procentem jeho označení, abychom bez tápání věděli, kam co patří.
Poslední řádek souboru s poznámkami ke konkrétní knize, který \TeX\ načte, by měl obsahovat jediný příkaz`\endinput`.
Cokoliv bude následovat pod tímto pokynem na dalších řádcích, \TeX\ už neuvidí. (Ale nezaměňujte ho s `\end` nebo `\bye`, aby v tomto místě neskočil svůj běh, ale pokračoval čtením
dalších souborů.)

Pod příkazem `\endinput` tedy můžeme mít připraveno několik řádků např. v této podobě:

\begtt
\ww {}={} % BKR
    {}={} % PSP
    {}={} % CSP
    {}={} % CEP
    {}={} % B21
    {}={} % SNC
\Note 1:1 {}
\endtt

a tyto řádky pak stačí zkopírovat na místo nové poznámky, upravit číslo kapitoly a verše za `\Note`, popřípadě umazat `={}`, kde nejsou zapotřebí. Neztratíme přehled,  kam
patří které fráze, ať už je vypisujeme ručně nebo kopírujeme z biblického programu či  on-line zdroje.




\secc[switch] Větvení zpracování textu podle variant překladu

%\mnote{\code{\\switch}}
Pomocí příkazu \`\switch` můžete větvit zpracování vstupního textu v
závislosti na nastavené hodnotě parametru \`\tmark`, tedy v závislosti na
aktuálně zpracovávané variantě jazyka. Příkaz má tuto syntaxi:
\begtt
\switch {<seznam variant>} {<co provést>}%
        {<seznam variant>} {<co provést>}% ... atd
        {<seznam variant>} {<co provést>}
\endtt
Dvojic `{<seznam variant>}{<co provést>}` může být uvedeno libovolně mnoho.
Za každou dvojicí `{<seznam variant>}{<co provést>}` (s výjimkou poslední
dvojice) musí bezprostředně a bez mezer následovat další taková dvojice,
proto je nutné při přechodu na
další řádek napsat za uzavírací závorku procento, které zakryje mezeru z
konce řádku. Mezery na začátku dalšího řádku nevadí.
Procento za dvojicí můžete číst jako \uv{pokračuje další dvojice}.

`<seznam variant>` je
jedna překladová varianta nebo seznam překladových variant oddělených
čárkou a bez mezer. \TeX/ pak pracuje takto: vyskytuje-li
se v `<seznam variant>` varianta definovaná v parametru `\tmark`, provede se
následující `<co provést>`. Nevyskytuje-li se tam taková varianta,
následující `<co provést>` se přeskočí. Příklad:
\begtt
\switch {BKR} {Balsazar}%
        {PSP} {Bélšaccar}%
        {CSP,CEP,B21,SNC} {Belšasar}
\endtt
Příklad ukazuje, jak vytisknout slovo Balsazar v závislosti na
zpracovávané variantě jazyka.

Jakmile \TeX/ najde shodu a provede `<co provést>`, pak případné následující
údaje v rámci stejného příkazu `\switch` přeskočí. Dále platí pravidlo, že
pokud je `<seznam variant>` prázdný, provede se `<co provést>` vždy, pokud
tedy není přeskočeno dle předchozího pravidla. Takže prázný `<seznam variant>` na
konci dvojic parametrů příkazu `\switch` se vyhodnotí jako \uv{ostatní
případy}. Příklad uvedený výše se tedy dá zapsat také takto:
\begtt
\switch {BKR} {Balsazar}%
        {PSP} {Bélšaccar}%
        {}    {Belšasar}
\endtt

Příkaz `\switch` se dá použít nejen na jednotlivé fráze uvnitř poznámek
`\Note`, ale třeba také na celé úseky vstupního textu obsahující třeba i
několik poznámek `\Note`, několik definic atd.

Příkaz `\switch` se nedá použít v parametrech jiných maker. Tam funguje jen
příkaz `\x/<fráze>/`.



\secc[renum] Přečíslování veršů podle variant překladu

%\mnote{\code{\\renum}}
Některé varianty překladu mají jiné číslování veršů. V takovém případě je
možné použít příkaz \`\renum` takto:

\begtt
\renum <zkratka-knihy> <výchozí-č-kap>:<výchozí-č-verše> = <překlad> <č-kap>:<č-od>-<č-do>
\endtt
kde `<překlad>` je značka pro konkrétní překlad.
Místo `<výchozí-č-kap>:<výchozí-č-verše>` se použije při `\def\tmark{<překlad>}`
ve skutečnosti `<č-kap>:<č-od>`.
Takové přečíslování se netýká jen
tohoto verše, ale celého úseku veršů vymezeného rozsahem `<č-od>-<č-do>`.

Například:
\begtt
\renum Da 6:1 = CSP  6:2-29
\renum Da 6:1 = CEP  6:2-29
\renum Da 6:1 = B21  6:2-29
\renum Da 6:1 = SNC  6:2-29
\endtt

Pavlíkův studijní překlad má v na přelomu 5. a 6. kapitoly číslování shodné s Biblí kralickou, takže není nutno ho přečíslovat.
Ostatní překlady uvádějí obsah verše BKR 6:1 (`Líbilo se pak` `Dariovi`...)
pod číslem 6:2 a celá kapitola 6 je o jedno číslo posunutá vůči BKR. BKR a PSP 6:28 je v CSP, CEP, B21 a SNC 6:29.
Poznámky číslujeme podle překladu, uvedeného jako první parametr definice `\variants` v souboru `Cze-vars.tex`, tedy v naší ukázce `BKR`, a to včetně odkazů, např. Viz pozn. 5:31.
Přečíslované překlady změní číslo poznámky podle skutečného čísla verše, který poznámka komentuje,  včetně odkazu, který se vytiskne jako  Viz pozn. 6:1.

Má-li se přečíslování týkat jediného verše, je třeba uvést shodné `<č-od>` i `<č-do>`, jako např. 
\begtt
\renum Da 5:31 = CSP 6:1-1
\renum Da 5:31 = CEP 6:1-1
\renum Da 5:31 = B21 6:1-1
\renum Da 5:31 = SNC 6:1-1
\endtt
Po takové deklaraci se nakonec poznámka 
\begtt \catcode`\<=12
\Note 5:31 {Darius}={\x/Darius/ Médský} Některé školy tvrdí, že tento a další 
   (<6:1>, <6:6>, <6:9>, <6:25>, <6:28>; <9:1>; <11:1>)     
   odkazy na \x/Daria/ Médského v knize \x/Daniel/ jsou historické omyly.
\endtt
vytiskne takto:

{\medskip \leftskip=\parindent \noindent
{\bf 6:1 Darjavéš Médský} \ Některé školy tvrdí, že tento a další 
(6:2, 6:7, 6:10, 6:26, 6:29; 9:1; 11:1)     
odkazy na Darjavéše Médského v knize Daniel jsou historické omyly.
\medskip}

Lze si poradit i s takovými lahůdkami, jako je posun čísla o pouhou část verše.
BKR Da 2:28 \"jest Bůh na nebi, kterýž zjevuje tajné věci" je v SNC \"Bůh, který je v nebesích, odhaluje tajemství." pod číslem 2:27;
před poznámkou k této frázi tedy uvedeme `\renum Da 2:28 = SNC 2:27-27`. 
Ale pak potřebujeme poznámku i k frázi \"v potomních dnech" z téhož verše, avšak -- světe, div se -- fráze \"posledních dnů" už
je v SNC ve verši 2:28! Nepotřebujeme spekulovat, proč na nás překladatelé SNC zkoušejí takové triky, že půl verše pošlou pod jiné číslo, než všichni ostatní;
postačí nám, že víme, jak si s tím poradit: 
Před tuto novou poznámku  napíšeme `\renum Da 2:28 = SNC 2:28-28` a všechno funguje, jak má: Kde se číslování rozchází, tam se přečísluje; kde se
shoduje, zůstane shodné.



\sec[odkazy] Metody vytváření hyperlinkových odkazů

Odkaz je úsek textu, podle kterého čtenář i po vytištění pozná, na jaké jiné místo
v textu (nebo v~internetu) se může podívat. Obsahuje tedy typicky číselný údaj o
stránce nebo o čísle kapitoly, sekce atd. Navíc, pokud čtenář pracuje s PDF
prohlížečem, pak tento úsek textu je {\em aktivní}, tj. po najetí myším
kurzorem nad tento text je možné klikout a PDF prohlížeč přejde na stanovené
místo v dokumentu (nebo do internetu).

Bible je neměnným způsobem
strukturovaný text. Obsahuje (v protestantském kánonu) 66 knih s ustálenými značkami pro tyto knihy,
každá kniha má své kapitoly číslované od jedné a každá kapitola má verše
číslované od jedné. Není tedy potřeba nechat \TeX/, aby generoval tato čísla
automaticky (jako to dělá při sazbě třeba odborného textu členícího se na
kapitoly a sekce) a není tedy vesměs potřeba používat lejblíky ve zdrojovém
dokumentu (které \TeX/ při zpracování přiřadí k vygenerovaným číslům) a odkazovat pomocí
těchto lejblíků, jak je popsáno v sekci 1.4.3 v dokumentaci \OpTeX/u. Je
daleko účelnější odkazovat přímo do konkrétního místa v Bibli, které má už
po mnohá staletí pevné označení knihy, čísla kapitoly a čísla verše.
(Canterburský arcibiskup Stephen Langton  počátkem 13. století, když vyučoval na univerzitě v Paříži  [a ještě nebyl arcibiskupem]
rozdělil Bibli na kapitoly. 
V polovině 16. století pak francouzský knihtiskař Robert Estienne rozdělil Nový Zákon na verše a k tomu připojil Starý Zákon,
který  židovští písmáci o staletí dříve byli rozdělili na verše, avšak rafinovaně nikoliv na kapitoly.
Od roku 1553, kdy Estienne vydal první francouzskou Bibli takto očíslovanou, tento systém používáme dodnes.)


Odkazy píšeme na konkrétní místo v Bibli mezi \code{<} a \code{>}. Text mezi
těmito znaky se vytiskne tak, jak je napsán (až na výjimky posané níže). 
\TeX/ ale musí umět odkaz správně interpretovat, aby jej mohl
udělat aktivní s možným proklikem na správné místo v Bibli. K tomu slouží
poměrně značné množství pravidel, kterým se věnuje tato sekce.

\secc[uudaj] Základní pravidlo s úplnými údaji

Odkaz mezi znaky \code{<} a \code{>} má tvar `"<text>" <údaj>` nebo jen
`<údaj>`. Přitom úplný `<údaj>` má tvar
`<kniha> <kapitola>:<verš>`. Zde `<kniha>` je zkratka knihy (za ní musí být
mezera), `<kapitola>` je číslo kapitoly a `<verš>` je číslo verše.
Příklad:

\begtt \catcode`<=12
... viz také verš <Jr 8:13>
... viz také <"verš" Jr 8:13>
\endtt
V prvním případě se vytiskne ... viz také verš {\Blue Jr 8:13} a ve druhém
... viz také {\Blue verš Jr 8:13}. Aktivní (umožňující proklik) bude v obou případech
oblast zde vyznačena modře a proklik vede na Jr 8:13.

\secc Specifikátor odkazu

Za koncovým znakem odkazu `>` může těsně následovat specifikátor odkazu, což
je jedno z písmen:
\begitems
* `n` ... odkazuje se na poznámku (note),
* `g` ... odkazuje se na glosu (která přijde ke slovu, vznikne-li někdy další verze \OpBible/ s dvousloupcovou sazbou základního textu),
* `a` ... odkazuje se na článek (article),
* `i` ... odkazuje se na úvod (introduction).
\enditems
Specifikátor odkazu se nevytiskne, je to jen interní informace, kam má
směřovat aktivní odkaz (tj.~proklik myší). Nenásleduje-li za ukončujícím
znakem údaje `>` žádný z uvedených specifikátorů, jedná se o odkaz na verš
(to bude asi nejčastější případ).
Příklad odkazu na poznámku:

\begtt \catcode`\<=12
... viz <"poznámku k" Jr 7:4>n, kde je uvedeno více informací.
\endtt
vytiskne ... viz {\Blue poznámku k Jr 7:4}, kde je uvedeno více informací.
Proklik vede na první poznámku k~verši Jr~7:4, nikoli k verši samotnému.

V případě odkazu na článek (specifikátor `a`), má úplný údaj formát
`<kniha> <kapitola>`, tj. chybí informace o verši, protože články můžeme chápat jako úvody
ke kapitolám. V případě odkazu na úvod ke knize (specifikátor `i`) má úplný
údaj formát `<kniha>` (chybí informace o kapitole i verši), protože to
jsou úvody ke knihám. V~ostatních případech má úplný údaj formát, jak bylo
řečeno v sekci~\ref[uudaj] s výjimkou popsanou v~sekci~\ref[vudaj]

\secc[vudaj] Výjimka pro formát úplného údaje u některých knih

Knihy Abd, Fm, 2Jn, 3Jn, a Ju nejsou členěny na kapitoly. V odkazu na verš, poznámku
k verši nebo glosu do některé z těchto knih tedy chybí informace o kapitole
a formát úplného odkazu vypadá takto: `<kniha> <verš>`. Aby \TeX/ věděl, že
má uplatnit tuto výjimku, je třeba seznam zkratek těchto knih definovat v
makru \`\nochapbooks`. Například v souboru `Cze-books.tex` je řečeno
\begtt
\def\nochapbooks{Abd Fm 2Jn 3Jn Ju}
\endtt
Protože pro různé jazyky existují různé zkrakty knih, je třeba toto makro
definovat v závislosti na použitém jazyku.

\secc[nudaj] Neúplný údaj

Někdy dokáže umístění verše čtenář stanovit z kontextu, takže `<údaj>` nemusí být úplný. V~{\em neúplném údaji} může chybět `<kniha>` nebo
`<kniha> <kapitola>:` nebo všechno. Například:

\begtt \catcode`<=12
... analogii vidíme též ve <"verších" Jr 8:13>, <9:7> a <11:3>
... viz verše <Jr 8:13>, <15>, <17>
... viz všechny <"verše" Jr 8:13>--<22>,
... (srv. <Jr 8:13> a <"jeho poznámku">n).
\endtt
Vytiskne se:
... analogii vidíme též ve {\Blue verších Jr 8:13}, {\Blue 9:7} a {\Blue 11:3}
... viz verše {\Blue Jr 8:13}, {\Blue 15}, {\Blue 17}
... viz všechny {\Blue verše Jr 8:13}--{\Blue 22}.
... (srv. {\Blue Jr 8:13} a {\Blue jeho poznámku}).

Údaje `9:7`, `11:3`, `15`, `17`, `22` a poslední prázdný údaj
v těchto příkladech jsou neúplné. Čtenář ví, že se týkají
knihy Jeremjáš a že v případě, kdy není uvedena ani kapitola, se týkají kapitoly 8 knihy
Jeremjáš. V posledním příkladě s prázdným údajem čtenář ví, že se jedná o poznámku
k~Jr~8:13. Ví to i \TeX/ a přidělí ke všem těmto neúplným odkazům správné interní
odkazy (podle kterých se provede proklik), protože neúplný údaj přebírá
nespecifikované informace z předchozího údaje. Toto pravidlo se vztahuje
lokálně na jeden objekt textu: poznámku, článek, úvod atd. Je-li hned první údaj v
některém objektu textu neúplný, nespecifikované informace se nahradí
zkratkou aktuálně zpracovávané knihy, případně číslem aktuální kapitoly,
případně verše.

Je-li neúplný údaj uvozen znakem `\`, přebírá nespecifikované informace z
aktuálně zpracovávané knihy, případně kapitoly a případně verše bez ohledu, jaký údaj
předchází. Například:
\begtt \catcode`<=12
\Note 4.5 {} Myšlenka se opakuje v <Jr 4:5>. Ale také v \<8:3> najdete o tom zmínku.
\endtt
Zde odkaz {\Blue 8:3} vede na verš 3 kapitoly 8 aktuální knihy. Kdyby tam
znak `\` nebyl, pak by tento odkaz vedl na 8:3 knihy Jeremjáš.

Neúplný údaj se vytiskne tak, jak je, jako neúplný. Shora uvedená pravidla na jeho
doplnění se provedou jen interně, aby aktivní odkaz správně fungoval po
kliknutí myší.

Existuje ještě jeden odlišný formát neúplného údaje: chybí číslo verše. To
se pozná tak, že je přítomná zkratka knihy a v údaji chybí dvojtečka a
nejedná se o knihu ze seznamu výjimek `\nochapbooks`. Tedy údaj má tvar
`<kniha> <kapitola>` (oddělené mezerou). \TeX/ v takovém případě vytvoří
interní odkaz na první verš dané kapitoly. Příklady:
\begtt \catcode`<=12
... viz <"kapitolu" Da 7>
... Josefův příběh (<Gn 39-41>)
\endtt
Ve druhém příkladě je uveden rozsah kapitol a \TeX/ vytvoří interní odkaz na
první verš kapitoly 39. Srovnejte též se sekcí~\ref[rozsahversu].

\secc[rozsahversu] Formát pro rozsah veršů a pro úsek ve verši

V každém údaji je možné místo `<verš>` nebo místo `<kapitola>:<verš>`
psát rozsah veršů ve vormátu
`<od>-<do>`. \TeX/ vytvoří interní link jen na první verš rozsahu a
spojovník (znak `-`, ASCII 45) uvedený v rozsahu promění v pomlčku. Příklady:

\begtt \catcode`<=12
<Jr 8:3-7>,
<Jr 8:3-9:5>,
<3-7>,
<8:3-7>.
\endtt
Třeba první odkaz z tohoto příkladu se vytiskne jako {\Blue Jr 8:3--7}
a nabízí proklik jen na Jr~8:3.

Někdy je potřeba odkazovat na úsek verše, nikoli na celý verš. To se dělá
připojením písmene bezprostředně za číslo verše. Například
\begtt \catcode`<=12
... viz <Da 9:11b>
\endtt
Takové písmeno můžete připojit k úplnému i neúplnému odkazu. Pro potřebu
hyperlinku se tato písmena ignorují, ale vytisknou se. Tedy uvedený příklad
vytiskne ... viz {\Blue Da~9:11b}, ale proklik vede na Da~9:11.

\secc Zatajení údaje

Když nepostačuje žádné z pravidel pro vytvoření interního linku z odkazu
uvedené výše, můžete vše, co se má v odkazu
vytisknout, uzavřít do `"..."` a následný údaj, podle kterého se má vytvořit interní
odkaz, zatajit. K tomu stačí těsně za zavírací uvozovku `"` připojit znak
podtržítko `_` následovaný údajem. Tento údaj se netiskne. Například
\begtt \catcode`<=12
<"První kniha Samuelova"_1Sa 1:1>
\endtt
vytiskne pouze text {\Blue První kniha Samuelova}, který interně odkazuje na první
verš této knihy.

\secc[Renum] Přečíslování odkazu

Pokud odkaz směřuje na verš, který má v konkrétním jazykovém překladu odlišné
číslování specifikované makrem `\renum`, pak do odkazu zapište údaj podle
výchozího číslování a \TeX/ si jej sám přepočítá podle dat uvedených v
`\renum`. Vytiskne přepočítaný údaj a použije jej pro interní odkaz.
Srv.~\ref[renum]. 

\secc Redukce údajů

Možná budete chtít do závorek pro odkaz psát úplné údaje a požadovat jejich automatickou 
redukci na neúplné, odkazují-li na aktuální knihu.
Neodkazují-li na aktuální knihu, údaj zůstane úplný. Toho můžete docílit
přidáním \`\re` před otevírací závorku údaje (pravidlo případné redukce 
se uplatní jen na tento jediný údaj) nebo pomocí příkazu \`\reduceref`. Když jej
použijete v poznámce (nebo v \TeX/ové skupině), pravidlo redukce se uplatní 
pro všechny následující úplné údaje této poznámky (této \TeX/ové skupiny). 
Když jej použijete v deklaraci sazby,
pravidlo případné redukce se uplatní na všechny úplné údaje v dokumentu. 
Příklad:

\begtt \catcode`<=12
\re<"verš" Da 7:3>
\endtt
se vytiskne jako {\Blue verš Da~7:3}, je-li tento odkaz uveden mimo knihu
Daniel. Když je ale tento odkaz uveden v knize Daniel, vytiskne se
jen {\Blue verš~7:3}, který interně odkazuje na Da~7:3. 

Pravidlo redukce odkazů nastavené pomocí \`\reduceref` lze vypnout 
příkazem \`\noreduceref`. Od něj až do konce poznámky (\TeX/ové skupiny)
se odkazy chovají, jakoby `\reduceref` nebylo zapnuté.

\secc[knihajinak] Značka pro knihu se může vytisknout jinak

Je-li značka pro knihu deklarovaná v různých překladových variantách různě pomocí
`\vdef`, pak v~odkazech používejte výhradně údaj <kniha> podle první
varianty. Je-li ale pomocí `\tmark` nastavena nějaká alternativní překladová
varianta, nakonec se odkaz vytiskne podle údaje `\vdef` této překladové varianty.
Vnitřně se ale odkazy propojují podle základní varianty. Tuto vlastnost
využijete, pokud deklarujete pomocí `\vdef` třeba pro značky `1Pa` a `2Pa` (Paralipomenon)
alternativní text `1Let` a `2Let` (Letopisů) při variantě B21 (Bible pro 21. století). Když pak
napíšete odkaz \code{<"viz" 1Pa 2:3>}, dostanete {\Blue viz 1Pa~2:3} v obvyklých
překladových variantách, ale vytiskne se {\Blue viz 1Let~2:3}, použijete-li
překladovou variantu~B21.

\secc[fudaj] Nepodařené odkazy, tj. odkazy na neexistující místo

Je-li uveden odkaz na neexistující verš nebo neexistující poznámku, pak
nastanou dvě možnosti. Je-li to odkaz na knihu, která se záměrně netiskne
(protože například pracujeme s testovacím výtiskem jen vybraných knih Bible,
viz též \`\printedbooks` v sekci~\ref[hlavni]), pak odkaz je sice obarven,
jakoby byl aktivní, ale aktivní není a v logu při zpracování \TeX/em se
neobjeví žádné varování. Pokud ale odkaz směřuje na neexistující verš nebo
poznámku tištěné části Bible, pak odkaz je aktivní, proklik směřuje na
poslední stránku PDF souboru a v logu se objeví o tom varování, že odkaz
není správný. Při prvním zpracování \TeX/em jsou ale neošetřeny všechny
odkazy a v logu je velké množství varování o nesprávných odkazech.
Teprve při následujícím zpracování \TeX/em jsou odkazy směřující do
existujícího místa správně prolinkovány.

\secc Trasování odkazů

Implicitně je zapnuto vypisování podrobného trasování odkazů v log souboru.
Vypnete to pomocí \`\notracinglinks` a zpětně zapnete pomocí \`\tracinglinks`.
Navíc pomocí \`\tracingouterlinks` vypnete potlačení varování o odkazech na
neexistující knihy, což umožní v logu najít případné neexistující odkazy z
důvodu překlepu ve značce knihy.

\secc Odkazy na kapitoly a na knihy

Chcete-li odkázat na knihu s proklikem na začátek této knihy, pište
\`\cref``[<kniha>]`, například `\cref[Gn]` vytiskne {\Blue Gn} s proklikem na
začátek knihy Genesis. Podobně `\cref[Gn 2]` vytiskne {\Blue Gn~2} s~proklikem
na začátek kapitoly Den odpočinutí. Chcete-li vytisknout něco jiného, pak za
údajem `[...]` musí těsně následovat `{<text>}`, kde `<text>` je text, který
se má vytisknout a stát se aktivním odkazem. Takže třeba
`\cref[Gn 2]{Den opdočinutí}` vytiskne {\Blue Den odpočinutí} s proklikem na
odpovídající kapitolu.

\secc Odkazy na stránky

Ve svém textu můžete umístit netištěný cíl odkazu na stránku pomocí \`\pglabel``[<lejblík>]` a
pak můžete odkazovat na stránku pomocí \`\pgref``[<lejblík>]`. V místě
`\pgref[<lejblík>]` se vytiskne číslo strany, které je možné prokliknout.
Podobně jako v případě `\cref` se dá pomocí příkazu `\pgref[<lejblík>]{<text>}`
vytisknout jiný `<text>` než číslo strany, který proklikne na místo, kde
je umístěný `\pglabel[<lejblík>]`.



\sec[mapy] Mapy, obrázky a jejich legendy

\secc[mapa-varianty] Varianty překladu

Podobně jako lze měnit znění hledané fráze v poznámce, aby odpovídalo skutečné podobě aktuálního základního textu, lze připravit 
popis mapy (a podobných grafických objektů, zobrazujících výrazy, které se v různých překladech liší) tak, aby vždy odpovídal aktuální verzi Bible.

Za předpokladu, že máme k dispozici slepou mapu, kterou chceme popsat svou legendou, se jakýkoliv text umisťuje přes obrázek např. takto:
\begtt
\puttext 5mm 62mm{{\bi Středozemní moře}}
\endtt

První rozměr (`5mm`) určuje horizontální posun na ose x, druhý pak vertikální na ose y, kdy souřadnice `0mm 0mm` má levý dolní roh obrázku.





\medskip
\line{\picw=150pt \inspic{images/BKR-Da-intro-map.png}  \hss\inspic{images/CEP-Da-intro-map.png}  \hss\inspic{images/B21-Da-intro-map.png} } 
\medskip
\leftline{Mapa z Úvodu k Danielovi:  BKR \hss CEP \hss B21}
\medskip
Například si můžete všimnout, jak se v ukázce výše mění nejen fráze \uv{král půlnoční} a \uv{král polední} (BKR) na \uv{král Severu} a \uv{král Jihu} (CEP) a \uv{jižní král}
a \uv{severní král} (B21), což je běžný aktivní link jako v poznámkách, ale také samotný popis mapy , např. 
jméno města Susan (kousek nad Perským zálivem) z kraličtiny na Šúšan v CEP a na Súsy podle B21. Podobně jméno oblasti Sinear
metamorfuje na Šineár a na Babylonii. Postarají se o to definice `\vdef` v souboru `Cze-vars.tex`. 

Obrázek musí být načten před zahájením popisu, a všechno, co na něm má být, musí být uvnitř definice \`\insertBot`. 
Výše uvedená ukázka byla zapsána jako

\begtt
\insertBot {Říše Danielových vzdálených vizí}[mapa](){
 \inspic{fertile-crescent-crop.pdf}% slepá mapa 
  \Heros \cond \setfontsize{at 9pt}\rm %font
  \vskip-1mm 
\putstext 2mm 108mm {\vtop{\hsize6.5cm %šířka boxu \vtop
                     \baselineskip10pt %řádková rozteč uvnitř boxu
                     \noindent         %šetříme místem, nepotřebujeme odsazovat
\leftskip=3pt \rightskip=3pt %o kolik bude poloprůhledný stín přesahovat text 
  Brzy po smrti Alexandra Velikého ... až do porážky Římem  v r. 63 př. Kr.}%konec\vtop
  }%konec \putstext
  \LMfonts \sans \setfontsize{at9pt}\rm
  \puttext 145mm 29mm {<"Sk 2:9"_Sk 2:9>}
  \puttext 145mm 32.5mm {<"Ez 32:16"_Ez 32:16>}
.
.
.
  \puttext 2mm 5mm{{\Heros \setfontsize{at 7pt}\it Satellite Bible Atlas,\/ \rm W.Schlegel}}
  \puttext 2mm 2mm{\Heros \setfontsize{at 7pt}\rm Použito se svolením.}
  }%konec \insertBot
\endtt

{\bf Na co si dát pozor:} Uvnitř `\insertBot` se nesmí vyskytnout prázdný řádek.



\secc[town] Makro `\town` pro symbol města na mapě


Města Jeruzalém, Babylon, Tolul Dura, Susan a Ur jsou na mapě vidět jako drobné kroužky s červeným středem a černým obvodem.
Vlastnosti tohoto kroužku lze nastavit makrem \`\townparams`, jehož implicitní hodnoty jsou tyto:

\begtt
\def\townparams{
   \hhkern=.8pt     % poloměr koužku
   \lwidth=.5pt     % tloušťka obrysové čáry
   \fcolor=\Red     % barva kroužku
   \lcolor=\Black   % barva obrysové čáry
}
\endtt

Samotné makro \`\town` umisťuje tuto značku souřadnicemi, podobně jako makro `\puttext`, ale bez dalšího textu, tedy např.

\begtt
  \town 101.5mm 53mm %město Babylon
\endtt

\secc [sklon] Nakloněný text

V ukázce máme jméno města Jeruzalém vytištěno šikmo, aby nekolidovalo s \uv{králem Jihu} a bylo dobře vidět.
Toho lze dosíci přidáním dalších parametrů k definici `\puttext`:

\begtt
\puttext 48mm 55mm {\c[-40/\kern1pt\pdfrotate{0}]{Jeruzalém}}
\endtt

O naklonění nápisu se postaralo číslo `-40`. Další údaje viz~\ref[maptitles].



\secc[maptitles] Nápisy podél křivky

Některé nápisy na mapě vyžadují \uv{prohnout} podle terénu, hlavně názvy rozlehlých oblastí, v našem případě Ptolemaiovci a Seleukovci, či Perský záliv, nebo třeba také drobnější
jména řek (Eufrat, Tigris, Nil). 
Budeme měnit doplňující parametry k `\puttext`, jak už jsme je viděli při umisťování Jeruzaléma do mapy v~\ref[sklon].

Příkaz `\kern` určí rozteč mezi písmeny; číslice v definici `\pdfrotate` pak sílu zakřivení. Záporná hodnota prohne nápis konkávně (jako duhu), kladná konvexně (jako mísu).
Dynastie Alexandrových potomků jsou na naší mapě zapsány takto:

\begtt
\puttext 62mm 70mm {\c[10/\kern7pt\pdfrotate{-1}]{SELEUKOVCI}}
\puttext 2mm 37mm {\c[0/\kern4pt\pdfrotate{2.5}]{PTOLEMAIOVCI}}
\endtt

Kdybychom potřebovali nápis, který by vlnil do tvaru písmene S (byl by současně konkávní i konvexní), musíme ho seskládat 
ze dvou či více příkazů `\puttext`, slepených k sobě, aby vypadaly jako jeden souvislý text. 




\secc[bkgrnd] Částečně průhledné pozadí souvislého textu


V ukázce \ref[mapa-varianty] byl dvojí souvislý text v mapě původně umístěn do světlejších obdélníků, připravených grafickým editorem (obrázek vlevo). Textem pak bylo nutno se do nich
trefit; pokud to nevycházelo, musel se předělat obrázek a zkoušet se trefit znova.

\medskip
\centerline{\picw=150pt\inspic images/fertile-crescent-crop-old.pdf \hss\inspic{images/Da-map-shadowed-text.png} }
\smallskip

Šlo to, ale bylo to únavně pracné a časově náročné.
\OpBible/ nabízí nápaditější řešení: Když místo `\puttext` (s příslušnými parametry, viz~\ref[maptitles]) napíšeme `\putstext` (Put-Shadowed-Text), pak jakkoliv dlouhý text, který napíšeme, bude automaticky podložen částečně průhledným bílým pozadím; obrázek slepé mapy nepotřebuje 
žádné předpřipravené plochy. Před prvním použitím `\putstext` lze nastavit úroveň průhlednosti bílého stínu úpravou implicitní hodnoty `\def\shadowparameter{.1}`
(obrázek vpravo). `\def\shadowparameter{1}` znamená solidní neprůhledné bílé pozadí; menší číslo znamená větší průhlednost.
Poté se však tato hodnota uloží do tzv. page-resources výstupního PDF a na všech dalších stránkách se používá stejná, nelze ji tedy měnit a mít na různých místech jednoho dokumentu
různou. Kdyby snad měla vzniknout nečekaně nadměrná poptávka po možnosti úroveň průhlednosti za jízdy měnit, může se to stát podnětem pro implementaci v případné další verzi \OpBible/.
V tuto chvíli se nám nezdálo nutné komplikovat makra zakládáním dalších a dalších page-resources, pročež uživatel nechť se spokojí s možností nastavit průhlednost stínu pod texty
na mapách  pro celou Bibli jednotně.  



%\medskip
%\centerline{\picw=150pt   \hss\inspic{images/Da-map-shadowed-text.png}  } 
%\medskip

\sec Nástroje na zařazení časových os

\secc[spanimage] Obrázek či text přes dvě strany

Pokud chceme vložit obrázek nebo text přes dvě stránky v otevřené dvoustraně, můžeme
použít
\begitems
* \`\insertSpanImage`: vloží připravený PDF obrázek, lze použít v úvodu knihy,
* \`\insertSpanText`: vloží text (například časovou osu), lze použít v úvodu knihy,
* \`\putSpanImage`: vloží připravený PDF obrázek, kotví se vzhledem k číslu
   kapitoly a verše, lze použít v souboru s poznámkami,
* \`\putSpanText`: jako `\putSpanImage`, ale místo obrázku vkládá text.
\enditems
Příkazy `\insertSpanImage` a `\insertSpanText` umístí obrázek či text 
do spodní části dvou stránek podle následujícího pravidla. Předpokládejme, že
příkaz samotný je spuštěn v okamžiku, kdy \TeX/ vytváří aktuální stránku s
číslem $c$. Pak
\begitems
* je-li $c$ sudé a obrázek či text se na výšku do aktuální strany vejde,
  bude umístěn na stranách $c$ a $c+1$,
* je-li $c$ sudé a obrázek či text se na výšku do aktuální strany nevejde,
  bude umístěn na stranách $c+2$ a $c+3$,
* je-li $c$ liché, obrázek či text se umístí na stránky $c+1$, $c+2$.
\enditems
Tím je zajištěno, že obrázek či text bude vždy viditelný na dvojstraně
otevřené knížky.

Příkazy `\putSpanImage` či `\putSpanText` pracují podle stejného pravidla,
jen číslo $c$ v tomto případě odpovídá číslu strany, na které se objeví
začátek verše specifikovaného v parametru příkazu.

Uvedené příkazy mají tyto parametry:

\`\insertSpanImage` `{Titul} [<lejblík>] (<parametry>) {<jméno souboru>}`
Zde se <Titul> použije v záhlaví obrázku a <lejblík> je možné stanovit,
chcete-li se na obrázek odvolávat (a tím vytvořit aktivní odkaz) pomocí `\ref`.
Nechcete-li použít lejblík, nechte parametr prázdný: `[]`.
Dále <parametry> mohou upřesnit způsob umístění obrázku, typicky je tento
údaj prázdný: `()`. Konečně <jméno souboru> je jméno souboru s obrázkem
včetně přípony. Typicky se jedná o PDF soubor, tj. má příponu `.pdf`.

\`\insertSpanText` `{Titul} [<lejblík>] (<parametry>) {<text>}`
Parametry jsou stejné jako má `\insertSpanImage`, jen poslední parametr se liší.
Je v něm text, který se má vytisknout přes dvě strany. Typicky tam může být
použita sada příkazů `\timeline`, `\timelinewidth`, `\arrowtext`, `\tlput`,
`\tline`, `\tlines` pro vytvoření časové osy, viz sekci~\ref[timeline].

\`\putSpanImage` `<kapitola>:<verš> {Titul} [<lejblík>] (<parametry>) {<jméno souboru>}`
Navíc je zde <kapitola>:<verš> specifikující verš, jehož začátek je na straně $c$
a obrázek se umístí podle pravidla výše. Ostatní parametry jsou stejné jako
u `\insertSpanImage`.

\`\putSpanText` `<kapitola>:<verš> {Titul} [<lejblík>] (<parametry>) {<text>}`
se chová stejně, jako `\insertSpanText`, ale navíc parametr <kapitola>:<verš>
má stejný význam jako v příkaze `\putSpanImage`.


\secc[timeline] Příkazy k vytvoření časové osy

Tyto příkazy je možné použít v parametru <text> v příkazech \`\insertSpanText` nebo \`\putSpanText`

Nejprve je třeba stanovit počet roků (či jiných jednotek) na celou šířku osy. 
V těchto jednotkách se budou zadávat všechny ostatní údaje. V dalším textu
budeme těmto jednotkám říkat roky.
Nastavení parametrů časové osy provedeme pomocí \`\timeline` a \`\timelinewidth`:
\begtt \catcode`\<=13
   \timeline <počet roků>
   \timelinewidth <šířka osy>
\endtt
Například po
\begtt
   \timeline 500
   \timelinewidth 25cm
\endtt
bude osa široká 25\,cm a v této šířce bude zahrnuto 500 let, takže jeden rok
bude představovat šířku 0,5 milimetru. Ovšem je obvyklejší zadat šířku osy
jako nějaký díl celkové šířky strany (či dvojstranného obrázku, použije-li
se osa v příkaze \`\insertSpanText` nebo \`\putSpanText`). Například takto:
`\timelinewidth=.95\hsize`.

Časovou osu sestavujeme po řádcích. Příkazy pro texty či úsečky, které
jsou ve stejném řádku, píšeme pod sebe v libovolném pořadí. K posunu na další
řádek slouží příkaz \`\vskip` `<rozměr>`, přitom rozměr můžete udávat v
násobcích výšky řádku při použití jednotky \`\l`. tedy třeba `\vskip 1.5\l`
znamená posun o jeden a půl řádku dolů.

Text vložíme pomocí \`\tlput` `<příznak> <pozice> <praporek> (<nastavení>) {<text>}`.
Parametr <příznak> může být `a`, chceme-li, aby text byl nad místem aktuální
polohy sazby, nebo `b`, chceme-li mít text pod místem aktuální polohy sazby.
Když je <příznak> `a`, víceřádkový text se roztahuje směrem nahoru a při `b` směrem dolu.
Parametr <pozice> je místo na časové ose (v rocích), ke kterému má být text připojen. 
Od tohoto místa bude text přetékat doprava, je-li <praporek> specifikován
pokynem \`\rlap`, bude přetékat doleva, je-li <praporek> specifikován pokynem
\`\llap`. Konečně, je-li <praporek> prázdný, bude mít text centrované řádky a
<pozice> pak odpovídá tomuto středu řádků. Parametr `<nastavení>` může být
prázdný, tedy `()`, nebo obsahuje nastavení fontu, barvy fontu atd. pro
následující <text>. A <text> obsahuje text k vytištění. V případě
víceřádkového textu oddělujte jednotlivé řádky příkazem \`\cr`. Příklad:
\begtt
   \tlput b 25   (\it)      {Abraham\cr má 100 let}
\endtt 
Text má dva řádky, jejich společný střed je pod bodem 25 časové osy.

\`\tline` `<od>..<do>` vytvoří vodorovnou úsečku začínající v místě <od> a
končící v místě <do>. Údaje jsou v rocích.

\`\tlines` `<w1>|<w2>|<w3>|` (a případně další) vloží do osy krátké svislé čárky.
Počet parametrů `<w>|` může být libovolný, každý z nich reprezentuje jednu
svislou čárku a čísla mezi nimi udávají vzdálenost mezi sousedními čárkami v
rocích. Například
\begtt
\tline 0..100
\tlines 20|20|20|20|
\endtt
vytvoří stoletou vodorovnou linku a na ní svislé linky označujících 20,
40, 60 a 80 roků.

\`\arrowtext` `<od>..<do> (<nastavení>) {<text>}`
vytiskne vodorovnou úsečku z místa <od> do místa <do> (údaje jsou v rocích)
a prostředek úsečky je přerušen a je v tomto místě napsán <text>. Na
okrajích úsečky jsou šipky směřující směrem ven s úsečky.


\sec Varianty formátování stránek

V implicitní podobě je nastavena jednosloupcová sazba pro hlavní text Bible
a pro úvody ke knihám a pro anotace. Dvousloupcová sazba je nastavena pro
poznámky k veršům.

Implicitně jsou čísla kapitol velká a ve vnějším okraji. Ve vnějším okraji jsou též
zvětšené značky uvozovek připojené k citacím. Příkazem \`\normalchapnumbers`
dojde ke změně tohoto nastavení: čísla kapitol pak jsou vsazena do levého
horního rohu prvního odstavce a zvětšené značky uvozovek jsou zrušeny.
Vnější okraj je pak zmenšen, protože v něm už není žádný tiskový materiál.  

Implicitně zaujímá každá poznámka \`\Note` nový odstavec ve dvousloupcové
sazbě. Pomocí `\mergednotes` je možné zajistit, že všechny poznámky vztažené
k jedinému verši jsou vždy spojeny do společného odstavce. Ale je to za cenu
toho, že se nezaručí, aby výskyt frází v hlavním textu byl na stejné straně
jako fráze v poznámce. Celý odstavec spojené poznámky se propojí pouze se
začátkem verše, tj. začátek verše a začátek poznámky k verši jsou na stejné
straně, samotné fráze se neprohledávají.

Další varianty formátování stránek jsou zatím jen v plánu a v této verzi \OpBible/ nejsou
implementovány.


\sec Možnosti hledání chyb

Může se stát, že uděláte nějaký překlep v souborech `notes-*`, `intro-*`,
`articles-*`. Kdybyste soubor rovnou zařadili do celého zpracování Bible,
TeX ohlásí chybu ve zcela jiném okamžiku, než když četl příslušný soubor a
zpětné dohledávání takové chyby je velmi obtížné, navíc se typicky chyba
projeví v jiné fázi zpracování a tedy je oznámena často velmi
nesrozumitelným a zavádějícím způsobem.

Je ovšem možné nově napsané soubory nejprve zpracovat přímo, ale bez
propojení s hlavním textem a v takovém módu jsou chyby hlášeny přímočařeji.
Přímé hledání chyb provedete příkazem \`\checksyntax` <seznam souborů> `{}`
Zde seznam souborů jsou názvy kontrolovaných souborů bez přípony `.tex`,
takže například můžete mít na konci hlavního souboru:

\begtt
\checksyntax intro-Dan articles-Dan notes-Dan {}
\endtt

provede přímou kontrolu vyjmenovaných souborů. Na výstupu je text těchto
souborů bez důrazu na jakékoli formátování. Syntaktické chyby v souborech se
ale projeví přímočaře.

Poznámka: použití \`\checksyntax` deaktivuje příkaz \`\processbooks`, který v
takovém případě neprovede nic, takže základní text Bible se vůbec nenačte.



%Tahle sekce je asi zbytečnám tak ji mažu. Předpokládám, že dodáme jako součást distribuce něco jako "blank project" se soubory, které nebudou obsahovat teologický materiál (leda jako samply), ale příklady použitelných definic, třeba s komentářem, co dělají.
%
%\sec[examples] Příklady
%
%\secc[sample-main] Příklad hlavního souboru `main.tex`
%
%
%\begtt
%\load[op-bible]  % macros OpBible
%\cslang
%\def\shadowparameter{0.075}
%%\checksyntax notes-Da intro-Da articles-Da fmt-BKR-Da {} 
%%\normalchapnumbers  
%%\mergednotes        
%%\let\notecolor=\relax %  disables \Red search phrases %
%
%%\input hebrew  % Hebrew phrases are declared here
%\input greek   % Greek  -"-
%
%% Variants of translation:
%\def\tmark     {BKR}    % Bible kralická
%%\def\tmark     {PSP}    % Pavlíkův studijní překlad
%%\def\tmark     {CSP}    % Český studijní překlad
%%\def\tmark     {CEP}    % Český ekumenický překlad
%%%\def\tmark     {B21}    % Bible pro 21. století
%%\def\tmark     {SNC}    % Slovo na cestu
%
%\input {Cze-vars.tex}   % Variants of translation
%\BookTitle Gen  Gn {První Mojžíšova (Genesis)}
\BookTitle Exod Ex {Druhá Mojžíšova (Exodus)}
\BookTitle Lev  Lv {Třetí Mojžíšova (Leviticus)}
\BookTitle Num  Nu {Čtvrtá Možíšova (Numeri)}
\BookTitle Deut Dt {Pátá Mojžíšova (Deuteronomium)}
\BookTitle Josh Joz {Jozue}
\BookTitle Judg Sd {Soudců}
\BookTitle Ruth Rt {Rút}
\BookTitle 1Sam 1S {První Samuelova}
\BookTitle 2Sam 2S {Druhá Samuelova}
\BookTitle 1Kgs 1Kr {První Královská}
\BookTitle 2Kgs 2Kr {Druhá Královská}
\BookTitle 1Chr 1CPa {První Paralipomenon (1. Letopisů)}
\BookTitle 2Chr 2Pa {Druhá Paralipomenon (2. Letopisů)}
\BookTitle Ezra Ezd {Ezdráš}
\BookTitle Neh  Neh {Nehemjáš}
\BookTitle Esth Est {Ester}
\BookTitle Job  Jb {Jób}
\BookTitle Ps   Ž {Žalmy}
\BookTitle Prov Př {Přísloví}
\BookTitle Eccl Kaz {Kazatel}
\BookTitle Song Pís {Píseň písní}
\BookTitle Isa  Iz {Izajáš}
\BookTitle Jer  Jr {Jeremjáš}
\BookTitle Lam  Pl {Pláč}
\BookTitle Ezek Ez {Ezechiel}
\BookTitle Dan  Da {Daniel}
\BookTitle Hos  Oz {Ozeáš}
\BookTitle Joel Jl {Jóel}
\BookTitle Amos Am {Ámos}
\BookTitle Obad Abd {Abdijáš}
\BookTitle Jonah Jon {Jonáš}
\BookTitle Mic  Mi {Micheáš}
\BookTitle Nah  Na {Nahum}
\BookTitle Hab  Abk {Abakuk}
\BookTitle Zeph Sf {Sofonjáš}
\BookTitle Hag  Ag {Ageus}
\BookTitle Zech Za {Zacharjáš}
\BookTitle Mal  Mal {Malachiáš}
\BookTitle Matt Mt {Matouš}
\BookTitle Mark Mk {Marek}
\BookTitle Luke L {Lukáš}
\BookTitle John J {Jan}
\BookTitle Acts Sk {Skutky apoštolské}
\BookTitle Rom  Ř {Římanům}
\BookTitle 1Cor 1K {První list Korintským}
\BookTitle 2Cor 2K {Druhý list Korintským}
\BookTitle Gal  Ga {Galatským}
\BookTitle Eph  Ef {Efezským}
\BookTitle Phil Fp {Filipským}
\BookTitle Col  Ko {Koloským}
\BookTitle 1Thess 1Te {První list Tesalonickým}
\BookTitle 2Thess 2Te {Druhý list Tesalonickým}
\BookTitle 1Tim 1Tm {První list Timoteovi}
\BookTitle 2Tim 2Tm {Druhý list Timoteovi}
\BookTitle Titus Tt {Titovi}
\BookTitle Phlm  Fm {Filemonovi}
\BookTitle Heb   Žd {Židům}
\BookTitle Jas   Jk {List Jakubův}
\BookTitle 1Pet  1Pt {První list Petrův}
\BookTitle 2Pet  2Pt {Druhý list Petrův}
\BookTitle 1John 1J {První list Janův}
\BookTitle 2John 2J {Druhý list Janův}
\BookTitle 3John 3J {Třetí list Janův}
\BookTitle Jude  Ju {List Judův}
\BookTitle Rev   Zj {Zjevení Janovo}     

\BookException Ž   {\def\amark{Z}}
\BookException Př  {\def\amark{Pr}}
\BookException Pís {\def\amark{Pis}}
\BookException Ř   {\def\amark{R}}
\BookException Žd  {\def\amark{Zd}}

  % Book titles and marks
%
%\def\txsfile     {sources/Cze\tmark-\amark.txs}     % Location of .txs files
%\def\fmtfile     {formats/fmt-Cze\tmark-\amark.tex} % Location of fmts
%\def\notesfile   {notes/notes-\amark.tex}           % Location of notes
%\def\introfile   {others/intro-\amark.tex}          % Location of introductions
%\def\articlefile {others/articles-\amark.tex}       % Location of articles%
%
%\def\printedbooks {%
%   Gn Ex Lv Nu Dt Joz Sd Rt 1Sa 2Sa 1Kr 2Kr 1Pa 2Pa Ezd Neh
%   Est Jb Ž Př Kaz Pís Iz Jr Pl Ez Da Oz Jl Am Abd Jon Mi
%   Na Abk Sf Ag Za Mal 
%   Mt Mk Lk Jn Sk Ř 1K 2K Ga Ef Fp Ko 1Te 2Te 1Tm 2Tm 
%   Tt Fm Žd Jk 1Pt 2Pt 1Jn 2Jn 3Jn Ju Zj
%}%
%
%\def\printedbooks{Da} 
%\processbooks % Generates books declared in \printedbooks
%\bye
%\endtt%
%
%\secc[sample-notes] Příklad souboru s poznámkami
%
%\begtt
%\Note 1:0 {} 
%\switch {BKR}{Vzat jsa ke dvoru královskému Daniel s tovaryši svými, 
%             8. věrnost Bohu zachoval, 
%             17. jsa zvláštními dary od Boha obdařen, pro ně u téhož dvora vzácný učiněn.}% 
%        {PSP}{}% 
%        {CSP}{Daniel oddělen}% 
%        {CEP}{Bůh chrání věrné judské jinochy na \x/Nabuchodonozor/ově dvoře 
%              a dá jim moudrost nad jiné.}% 
%        {B21}{}%  
%        {SNC}{}%
%
%
%\putArticle 1:18 {Výkupné (\<1:18-19>)} [1] () %1Pt
%
%\ww {mrtvým}  %BKR
%    {mrtvým}  %PSP
%    {mrtvým}  %CSP
%    {mrtvým}  %CEP
%    {mrtví}  %B21
%    {zemřelým}  %SNC
%    {} {} {} {}
%\Note  4:6 {mrtví} Teologický univerzalizmus (nauka, že všichni budou spaseni) spoléhá 
%na \uv{druhou šanci} po smrti. 
%Takovýto stav je výstižně ilustrován vyučujícím, který by zadal úkol k vypracování, 
%a přitom nestanovil termín odevzdání. 
%Kdyby pro rozhodnutí požádat Krista o odpuštění skutečně bylo k dispozici nekonečno času, 
%neučinil by je nikdo.
%
%\putBot 2:1 {Danielovy vize} [danielovyvize] () {
%   \Heros \cond \setfontsize{at 10pt}\rm
%   \inspic{Nabuco-crop.pdf}
%   \puttext 5mm 112mm {\bf Čtyři království}
%   \puttext 110mm 112mm {\bf Chronologie impérií}
%  \puttext 13mm 107mm {Vize kap. 2}
%  \puttext 45mm 107mm {Vize kap. 7}
%  \puttext 67mm 107mm {Vize kap. 8}
%  \puttext 90mm 107mm {Identifikace}
%   \puttext 139mm 104mm {600}
%   \puttext 139mm 90.5mm {500}
%   \puttext 139mm 76.5mm {400}
%   \puttext 139mm 62.5mm {300}
%   \puttext 139mm 49mm {200}
%   \puttext 139mm 45mm {167 př. Kr.}
%   \puttext 139mm 41mm {Makabejci}
%   \puttext 137mm 37mm {a Hasmoneovci}
%   \puttext 139mm 32mm {100}
%   \puttext 139mm 21mm {\bf 0}
%   \puttext 139mm 2mm {100}
%   \puttext 139mm 11mm {70 po Kr.}
%   \puttext 139mm 7mm {Pád Jeruzaléma}
%\bf  
%   \puttext 144mm 108.5mm {626 př. Kr.}
%   \puttext 144mm 95.5mm {539 př. Kr.}
%   \puttext 144mm 67mm {330 př. Kr.}
%   \puttext 144mm 30mm {63 př. Kr.}
%\puttext 15mm 91mm {\x/Hlava zlatá/}   
% \puttext 48mm 91mm {Lev}
%  \puttext 90mm 91mm {\x/Babylon/ (<2:37-38>)}
%\puttext 15mm 66mm {\x/Prsy a ramena/}   
%\puttext 17mm 62mm {\x/z stříbra/}   
% \puttext 46mm 66mm {\x/Nedvěd/}   
%  \puttext 70mm 66mm {\x/Skopec/}   
%    \puttext 90mm 66mm {Médo-Persie (<8:20>)}   
%\puttext 15mm 46mm {\x/Břicho i bedra/}   
%\puttext 20mm 42mm {\x/z mědi/}   
% \puttext 46mm 44mm {\x/Pard/}   
%  \puttext 70mm 44mm {Kozel}   
%   \puttext 90mm 44mm {\x/Řecko/ (<8:21>)}   
%\puttext 15mm 30mm {\x/Hnátové/}
%\puttext 15mm 26mm {\x/z železa/}
% \puttext 45mm 30mm {\x/Hrozná/}
% \puttext 46mm 26mm {\x/šelma/}
%  \puttext 90mm 30mm {Řím} 
%\puttext 10mm 2mm {\x/Nohy z částky z železa/}
%\puttext 15mm -2mm {\x/a z částky z hliny/}
%}
%
%\putArticle 5:20 {Kdo byl \x/Darius/ Médský?} [6] ()
%
%\putCite 1:8 {Druhořadých věcí nedosáhneme tím, že je prohlásíme za prvořadé. 
%Druhořadých věcí dosáhneme pouze tehdy, budou-li na prvním místě věci první. 
%\quotedby{C. S. Lewis}}
%
%\putCite 3:30 {Odvaha není jen jednou ze ctností, ale je formou každé ctnosti v bodě zkoušky, 
%tedy v bodě nejvypjatější skutečnosti.
%\quotedby{C.S.Lewis}}
%
%\putCite 5:4 {Moc korumpuje, a absolutní moc korumpuje absolutně. \quotedby {Lord Acton}} 
%
%\putCite 6:11 {Život křesťana je život na území okupovaném nepřáteli. \quotedby{C.S.Lewis}}
%
%\endinput
%
%\ww {}  %BKR
%    {}  %PSP
%    {}  %CSP
%    {}  %CEP
%    {}  %B21
%    {}  %SNC
%\Note 1:1 {}     
%
%\endtt
%
%\secc[sample-articless] Příklad souboru s články%
%
%\begtt
%\Cite  A
%{Ani v literatuře a umění nebude nikdo, kdo si zakládá na originalitě, nikdy originální: 
%zatímco když se jenom snažíte říkat pravdu (a nezáleží vám na tom, kolikrát už byla řečena), 
%stanete se v devíti případech z deseti originálními, ani si toho nevšimnete. 
%\quotedby {C. S. Lewis}
%}
%%\swapCites
%
%\Article [1] %Výkupné 1Pt
%
%Představme si s scénu jako z akčního thrileru: Vaši šestnáctiletou dceru unesou teroristé a
%požadují po vás výkupné. %
%
%Prodáte všechno, co máte, zadlužíte se na\-do\-smr\-ti, abyste dali potřebnou sumu dohromady,
%pak s penězi v kufru přijdete na smluvené místo uprostřed mostu, položíte kufr a odejdete na 
%určený konec mostu, jak vám bylo nařízeno. Ze tmy se vynoří dodávka, zastaví uprostřed mostu, 
%vystoupí vaše dcera, vy si oddychnete, že je naživu a snad i v pořádku, ona vezme kufr a hodí ho 
%do auta, pak se otočí směrem k vám, plácne se levou dlaní do pravého bicepsu a zároveň vztyčí 
%pravou pěst hřbetem ruky směrem k vám a přitom vykřikne: \uv{Díky, troubo!} 
%A naskočí do dodávky a zmizí.
%
%\insertCite A\left \
%Takové plivnutí do tváře lásce, která pro nás všechno obětovala, děláme pokaždé, když dáme %přednost hříchu před svatostí (\<"v." 15-16>).         
%%\endtt
%%Article, quotation apod.
%
%\secc[sample-fmt] Příklad souboru `fmt`
%
%\begtt %CEP Jr
%\fmtpoetry{23:28}{pravdivě.//// obilí?“,///}
%\fmtpoetry{23:29}{// oheň,/// Hospodinův,///}
%\fmtins{23:28}{pravdivě.}{\medskip}
%\fmtpre{23:30}{\medskip}
%
%\fmtfont{23:28}{pravdivě}{\em}
%\endtt
%
%\secc[sample-intro] Příklad souboru pro Úvod
%
\sec[suma] Shrnutí uživatelských definic

%Sem bych dal všechno, co uživatel může potřebovat nastavit. Kdyby se to mohlo vejít na dvě strany,  mohli bychom doporučit to vytisknout jako cheat-sheet.

Tím se myslí `\amark`, `\tmark` atd? Nestačí rejsřík, který je připojen?


\sec Rejstřík

\def\_sortinglang{en}
\begmulti 3
\typosize[9/]
\makeindex
\endmulti

\bye




% to co je nize by melo byt asi ve verenem Overleaf projektu s prislusnymi 00-README v jednotlivych adresarich.

\secc[summa-main] Hlavní soubor `main.tex`

\begtt
\load[op-bible]  % macros OpBible
\cslang

%\checksyntax notes-Da intro-Da articles-Da fmt-BKR-Da {} % Error tracer; when active, \processbooks is disabled (does nothing)
%\normalchapnumbers  % smaller chapter number (just over first 2 lines); no big quote marks next to the quotations
%\mergednotes        % more than one note to a single verse make one paragraph but at the expanse of ignoring the search phrases
%\let\notecolor=\relax %  disables \Red search phrases 
\def\shadowparameter{.1}%text background transparency; {1}=solid white
%\def\townparams{
%   \hhkern=.8pt     % poloměr koužku
%   \lwidth=.5pt     % tloušťka obrysové čáry
%   \fcolor=\Red     % barva kroužku
%   \lcolor=\Black   % barva obrysové čáry
%} 


%\input hebrew  % Hebrew phrases are declared here
%\input greek   % Greek  -"-

% Variants of translation:
\def\tmark     {BKR}    % Bible kralická
%\def\tmark     {PSP}    % Pavlíkův studijní překlad
%\def\tmark     {CSP}    % Český studijní překlad
%\def\tmark     {CEP}    % Český ekumenický překlad
%\def\tmark     {B21}    % Bible pro 21. století
%\def\tmark     {SNC}    % Slovo na cestu

\input {Cze-vars.tex}   % Variants of translation
\BookTitle Gen  Gn {První Mojžíšova (Genesis)}
\BookTitle Exod Ex {Druhá Mojžíšova (Exodus)}
\BookTitle Lev  Lv {Třetí Mojžíšova (Leviticus)}
\BookTitle Num  Nu {Čtvrtá Možíšova (Numeri)}
\BookTitle Deut Dt {Pátá Mojžíšova (Deuteronomium)}
\BookTitle Josh Joz {Jozue}
\BookTitle Judg Sd {Soudců}
\BookTitle Ruth Rt {Rút}
\BookTitle 1Sam 1S {První Samuelova}
\BookTitle 2Sam 2S {Druhá Samuelova}
\BookTitle 1Kgs 1Kr {První Královská}
\BookTitle 2Kgs 2Kr {Druhá Královská}
\BookTitle 1Chr 1CPa {První Paralipomenon (1. Letopisů)}
\BookTitle 2Chr 2Pa {Druhá Paralipomenon (2. Letopisů)}
\BookTitle Ezra Ezd {Ezdráš}
\BookTitle Neh  Neh {Nehemjáš}
\BookTitle Esth Est {Ester}
\BookTitle Job  Jb {Jób}
\BookTitle Ps   Ž {Žalmy}
\BookTitle Prov Př {Přísloví}
\BookTitle Eccl Kaz {Kazatel}
\BookTitle Song Pís {Píseň písní}
\BookTitle Isa  Iz {Izajáš}
\BookTitle Jer  Jr {Jeremjáš}
\BookTitle Lam  Pl {Pláč}
\BookTitle Ezek Ez {Ezechiel}
\BookTitle Dan  Da {Daniel}
\BookTitle Hos  Oz {Ozeáš}
\BookTitle Joel Jl {Jóel}
\BookTitle Amos Am {Ámos}
\BookTitle Obad Abd {Abdijáš}
\BookTitle Jonah Jon {Jonáš}
\BookTitle Mic  Mi {Micheáš}
\BookTitle Nah  Na {Nahum}
\BookTitle Hab  Abk {Abakuk}
\BookTitle Zeph Sf {Sofonjáš}
\BookTitle Hag  Ag {Ageus}
\BookTitle Zech Za {Zacharjáš}
\BookTitle Mal  Mal {Malachiáš}
\BookTitle Matt Mt {Matouš}
\BookTitle Mark Mk {Marek}
\BookTitle Luke L {Lukáš}
\BookTitle John J {Jan}
\BookTitle Acts Sk {Skutky apoštolské}
\BookTitle Rom  Ř {Římanům}
\BookTitle 1Cor 1K {První list Korintským}
\BookTitle 2Cor 2K {Druhý list Korintským}
\BookTitle Gal  Ga {Galatským}
\BookTitle Eph  Ef {Efezským}
\BookTitle Phil Fp {Filipským}
\BookTitle Col  Ko {Koloským}
\BookTitle 1Thess 1Te {První list Tesalonickým}
\BookTitle 2Thess 2Te {Druhý list Tesalonickým}
\BookTitle 1Tim 1Tm {První list Timoteovi}
\BookTitle 2Tim 2Tm {Druhý list Timoteovi}
\BookTitle Titus Tt {Titovi}
\BookTitle Phlm  Fm {Filemonovi}
\BookTitle Heb   Žd {Židům}
\BookTitle Jas   Jk {List Jakubův}
\BookTitle 1Pet  1Pt {První list Petrův}
\BookTitle 2Pet  2Pt {Druhý list Petrův}
\BookTitle 1John 1J {První list Janův}
\BookTitle 2John 2J {Druhý list Janův}
\BookTitle 3John 3J {Třetí list Janův}
\BookTitle Jude  Ju {List Judův}
\BookTitle Rev   Zj {Zjevení Janovo}     

\BookException Ž   {\def\amark{Z}}
\BookException Př  {\def\amark{Pr}}
\BookException Pís {\def\amark{Pis}}
\BookException Ř   {\def\amark{R}}
\BookException Žd  {\def\amark{Zd}}

  % Book titles and marks

\def\txsfile     {sources/Cze\tmark-\bmark.txs}     % Location of .txs files
\def\fmtfile     {formats/fmt-Cze\tmark-\amark.tex} % Location of fmts
\def\notesfile   {notes/notes-\amark.tex}           % Location of notes
\def\introfile   {others/intro-\amark.tex}          % Location of introductions
\def\articlefile {others/articles-\amark.tex}       % Location of articles

\def\printedbooks {%
   Gn Ex Lv Nu Dt Joz Sd Rt 1Sa 2Sa 1Kr 2Kr 1Pa 2Pa Ezd Neh
   Est Jb Ž Př Kaz Pís Iz Jr Pl Ez Da Oz Jl Am Abd Jon Mi
   Na Abk Sf Ag Za Mal 
   Mt Mk Lk Jn Sk Ř 1K 2K Ga Ef Fp Ko 1Te 2Te 1Tm 2Tm 
   Tt Fm Žd Jk 1Pt 2Pt 1Jn 2Jn 3Jn Ju Zj
}

\def\printedbooks{Da} 
\processbooks % Generates books declared in \printedbooks
\bye
\endtt

 

\secc[summa-intro] Soubor `intro-*.tex`

\secc[summa-notes] Soubor `notes-*.tex`

\begtt
%\ww {}%vyhledaná fráze jednotlivých překladů
%      ={}vytištěná fráze % BKR
%    {}={} % PSP
%    {}={} % CSP
%    {}={} % CEP
%    {}={} % B21
%    {}={} % SNC
\Note  1:1 {} %fráze společná všem překladům, např. moudrost, jinak prázdná skupina{}
            %={}%fráze, která není v Bibli, ale tiskne se, např.  Vize čtyř \x/šelem/
        Text poznámky.
\endtt




\secc[summa-article] Soubor `article-*.tex`

\secc[summa-quote] Příkaz `\putCite`

\secc[summa-mapy] Popis map

\secc[summa-fmt] Formátovací soubory `fmt-*-*.tex`

\secc[summa-vars] Soubor překladových variant `Cze-vars.tex`






\vfill\eject

\sec Rejstřík

\def\_sortinglang{en}
\begmulti 3
\typosize[9/]
\makeindex
\endmulti

\bye
