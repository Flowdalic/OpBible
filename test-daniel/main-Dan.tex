\load[op-bible]  % macros OP-Bible

\input hebrew  % Hebrew phrases are declared here
\cslang

\def\tmark       {BKR}                        % Variant of translation
\def\txsfile     {./Cze\tmark-\bmark.txs}     % Localization of .txs files
\def\fmtfile     {./fmt-Cze\tmark-\amark.tex} % Localisation of fmts
\def\notesfile   {./notes-\amark.tex}         % Localisation of notes
\def\introfile   {./intro-\amark.tex}         % Localisation of introductions
\def\articlefile {./articles-\amark.tex}      % Localisation of articles

\let\notecolor=\relax

\switch {BKR}{\def\bibname{Bible kralická}}%
        {B21}{\def\bibname{Bible pro 21. století}}%
        {CEP}{\def\bibname{Český Ekumenický překlad}}%
        {CSP}{\def\bibname{Český Studijní překlad}}%
        {PSP}{\def\bibname{Pavlíkův studijní překlad}}%
        {SNC}{\def\bibname{Slovo na cestu}}

%\loggingall


\variants 6 {CzeBKR} {CzePSP} {CzeCSP} {CzeCEP} {CzeB21} {CzeSNC}

\vdef {Joakim} {Jehójákím} {Jójákím} {Jójakím} {Joakim} {Jójakím}
\vdef {Daniel} {} {Daniel} {Daniel} {} {}
\vdef {Chananiáš} {} {} {Chananjáš} {} {}
\vdef {Mizael} {} {} {Míšael} {} {}  % Mizach je totéž?
\vdef {Mizach} {} {} {Míšak} {} {}
\vdef {Azariáš} {} {} {Azarjáš} {} {}
\vdef {Sidrach} {} {} {Šadrak} {} {}
\vdef {Mesak} {} {} {Méšak} {} {}
\vdef {Abedneg} {} {} {Abed-neg} {} {} % Adbenág je totéž?
\vdef {Cýr} {Kóreš} {Kýr} {Kýr} {Kýr} {Kýr}
\vdef {Mardocheus} {} {} {Mardocheus} {} {}  % ??
\vdef {Dura} {} {} {Dúra} {} {} 
\vdef {Balsazar} {} {} {Belšasar} {} {}
\vdef {Darius} {} {} {Darius} {} {} % ??
\vdef {Izaiáš} {Isajá} {Izajáš} {Izajáš} {Izaiáš} {Izajáš}

\vdef  {Nabuchodonozor král Babylonský}     
       {Nevúchadneccar, král Bávelu}
       {babylonský král Nebúkadnesar}
       {Nebúkadnesar, babylónský král}
       {babylonský král Nabukadnezar}                  
       {babylónský král Nebúkdnesar}                        

\vdef  {Jeremiáš} %BKR
        {Jeremjáš} %PSP
        {Jeremjáš} %CSP
        {Jeremjáš} %CEP
        {Jeremiáš} %B21
        {Jeremjáš} %SNC

\vdef {Ezechiel} %BKR
       {Ezekiél} %PSP
       {Ezechiel} %CSP
       {Ezechiel} %CEP
       {Ezechiel} %B21
       {Ezechiel} %SNC


\vdef {Zorobábel} %BKR
       {Zerubbável} %PSP
       {Zerubábel} %CSP
       {Zerubábel} %CEP
       {Zerubábel} %B21
       {Zerubábel} %SNC

\vdef {Ezdráš} %BKR
       {Ezrá} %PSP
       {Ezdráš} %CSP
       {Ezdráš} %CEP
       {Ezdráš} %B21
       {Ezdráš} %SNC

\vdef {Nehemiáš} %BKR
       {Nechemjá} %PSP
       {Nehemjáš} %CSP
       {Nehemjáš} %CEP
       {Nehemiáš} %B21
       {Nwehemjáš} %SNC





\vdef {Baltazar}       {}  {Beltšasar}    {Beltšasar}    {} {}
\vdef {Nabuchodonozor} {}  {Nebúkadnesar} {Nebúkadnesar} {} {}
\vdef {sedm let}       {}  {sedm časů}    {sedm let}     {} {}
\vdef {léto}           {}  {čas}          {léto}         {} {}
\vdef {Balsazar}  {Bélšaccar} {Belšasar}   {Belšasar}  {Belšasar}  {Belšasar}

\vdef {Darius}  {Dárjáveš}   {Dareios} {Darjaveš}   {Darjaveš}   {Darjaveš}
\vdef {Daria}   {Dárjáveše}  {Dareia}  {Darjaveše}  {Darjaveše}  {Darjaveše}
\vdef {Dariov}  {Dárjávešov} {Dareiov} {Darjavešov} {Darjavešov} {Darjavešov}

\vdef {Asver} {Achašvéróš} {Achašvéróš} {Achašveroš} {Ahasver} {Achašvéroš}

\vdef {Gabriel} {Gavríél} {Gabriel} {Gabriel} {Gabriel} {}

\vdef {zvířata} {} {} {zvířata} {} {} % ??
\vdef {zvířat}  {} {} {zvířat}  {} {} % ??
\vdef {lidu svatých Nejvyššího}  {} {} {lidu svatých Nejvyššího}  {} {} % ??
\vdef {oškubána}  {} {} {oškubána}  {} {} % ??
\vdef {lidské srdce}  {} {} {lidské srdce}  {} {} % ??
\vdef {pard}  {} {} {levhard}  {} {}

\vdef {Susan} {Šúšán} {Šúšan} {Šúšan} {Súsy} {Šúšan} % nominativ, akuzativ, vokativ
%\vdef {Susanu} {Šúšánu} {Šúšanu} {Šúšanu} {Sús} {Šúšanu} % genitiv, 
%\vdef {Susanu} {Šúšánu} {Šúšanu} {Šúšanu} {Súsách} {Šúšan} % ablativ
%\vdef {Susanu} {Šúšánu} {Šúšanu} {Šúšanu} {Súsám} {Šúšanu}  % dativ
% to musí být lokální \wdef, jinak to nemá řešení. Est 1:1 má B21 v Súsách, CSP v Šúšanu, BKR v Susan

\vdef {Elam}  {} {} {Elam}  {} {} % ??

\CommentedBook {Da}
\wdef 1:2  {vydal Pán} {} {Hospodin vydal} {Hospodin mu vydal}={Hospodin vydal} {} {}
\wdef 1:2  {nádobí} {} {nádob} {nádob}={nádoby} {} {}
\wdef 1:2  {do domu boha svého} {} {} {z Božího domu} {} {}
\wdef 1:4  {liternímu umění a jazyku} {} {} {kaldejskému písemnictví}={kaldejské písemnictví} {} {}
\wdef 1:5  {z stolu královského} {} {} {z královských lahůdek}={královské lahůdky} {} {}
\wdef 1:8  {nepoškvrňoval} {} {} {neposkvrní} {} {}
\wdef 1:9  {milost a lásku u správce} {} {} {slitování}={milosrdenství a slitování u velitele dvořanů} {} {}
\wdef 1:14 {uposlechl} {} {} {vyslyšel} {} {}
\wdef 1:15 {tváře jejich byly krásnější} {} {} {jejich vzhled je lepší} {} {}
\wdef 1:17 {vidění a snům} {} {} {viděním a snům} {} {}
\wdef 1:18 {dokonali dnové} {} {} {Po uplynutí doby} {} {}
\wdef 1:20 {mudrce a hvězdáře} {} {} {věštce a zaklínače} {} {}
\wdef 1:21 {léta prvního Cýra krále} {} {} {prvního roku vlády krále Kýra} {} {}
\wdef 2:1  {Léta pak druhého} {} {} {Ve druhém roce} {} {}
\wdef 2:1  {ze sna protrhl} {} {} {nemohl spát} {} {}
\wdef 2:2  {mudrce} {} {} {čaroděje} {} {}
\wdef 2:4  {Syrsky} {} {} {aramejsky} {} {}
\wdef 2:5  {Neoznámíte-li mi snu} {} {} {Jestliže mi neoznámíte sen} {} {}
\wdef 2:11 {kromě bohů} {} {} {mimo bohy} {} {}
\wdef 2:18 {Bohu nebeskému} {} {} {Boha nebes} {} {}
\wdef 2:19 {věc tajná} {} {} {tajemství} {} {}
\wdef 2:21 {ssazuje krále, i ustanovuje krále} {} {} {krále sesazuje, krále ustanovuje} {} {}
\wdef 2:23 {oslavuji a chválím} {} {} {chci vzdávat čest a chválu} {} {}
\wdef 2:24 {výklad ten oznámím} {} {} {sdělím králi výklad} {} {}
\wdef 2:28 {jest Bůh na nebi, kterýž zjevuje tajné věci} {} {} {je Bůh v nebesích, který odhaluje tajemství} {} {}
\wdef 2:28 {v potomních dnech} {} {} {v posledních dnech} {} {}
\wdef 2:38 {hlava zlatá} {} {} {zlatá hlava} {} {}
\wdef 2:47 {Bůh bohů a Pán králů} {} {} {Bohem bohů a Pán králů} {} {}
\wdef 2:48 {krajinou Babylonskou} {} {} {babylónskou krajinou} {} {}
\wdef 2:49 {býval v bráně královské} {} {} {zůstal na královském dvoře} {} {}
\wdef 3:1  {obraz} {} {} {sochu} {} {}
\wdef 3:1  {zlatý} {} {} {zlatou} {} {}
\wdef 3:5  {trouby} {} {} {rohu} {} {}
\wdef 3:8  {Kaldejští} {} {} {hvězdopravci} {} {}
\wdef 3:12 {Sidrach, Mizael a Abednego} {} {} {Šadrak, Méšak a Abed-nego} {} {}
\wdef 3:15 {který jest ten Bůh} {} {} {kdo je ten Bůh} {} {}
\wdef 3:17 {vytrhne nás} {} {} {vysvobodí nás} {} {}
\wdef 3:30 {zvelebil} {} {} {král zařídil} {} {}
%\wdef 3:31 {Nabuchodonozor král} {} {} {Král Nebúkadnesar} {} {}
%\renum Da 4:8 = CzeCEP 4:5-34 
%\wdef 4:9  {nic} {} {} {žádné tajemství ti nedělá potíže} {} {}
%\wdef 4:26 {království tvé tobě zůstane} {} {} {tvé království se ti opět dostane} {} {}
%\wdef 4:26 {nebesa} {} {} {Nebesa} {} {}
%\wdef 4:33 {bylinu jako vůl jedl} {} {} {pojídal rostliny jako dobytek} {} {}
%\wdef 4:37 {krále nebeského} {} {} {Krále nebes} {} {}
% Upraveno v souboru notes-Da.tex a odpovídajícím způsobem i v CzeBKR-Dan.txs 
\wdef 5:7  {hvězdáři} {} {} {hvězdopravce a planetáře} {} {}
\wdef 5:22 {ačkolis} {} {} {ačkoli jsi} {} {}
\wdef 5:23 {proti Pánu nebes} {} {} {Pána nebes} {} {}
\wdef 5:24 {Protož} {} {} {Proto} {} {}
\wdef 5:25 {Mene} {} {} {Mené} {} {}
\wdef 5:25 {ufarsin} {} {} {ú-parsín} {} {}
\wdef 5:26 {Mene} {} {} {Mené} {} {}
\wdef 5:28 {Médským} {} {} {Médům} {} {}
\wdef 5:29 {Balsazarova} {} {} {Belšasar} {} {}
\renum Da 5:31 = CzeCEP 6:1-1
\wdef 6:1  {Dariovi} {} {} {Darjaveš} {} {}
\renum Da 6:1 = CzeCEP 6:2-29
\wdef 6:3  {duch znamenitější} {} {} {mimořádný duch} {} {}
\wdef 6:7  {všickni} {} {} {Všichni} {} {}
\wdef 6:7  {Kdož by} {} {} {kdo by se} {} {}
\wdef 6:8  {práva} {} {} {zákona} {} {}
\wdef 6:10 {když se dověděl} {} {} {Když se Daniel dověděl} {} {}
\wdef 6:10 {třikrát} {} {} {Třikrát} {} {}
\wdef 6:13 {synů Judských} {} {} {judských přesídlenců} {} {}
\wdef 6:14 {zarmoutil} {} {} {byl velmi znechucen} {} {}
\wdef 6:16 {Bůh tvůj} {} {} {tvůj Bůh} {} {}
\wdef 6:23 {vytáhnouti} {} {} {vytáhli} {} {}
\wdef 6:24 {rozkázal} {} {} {poručil} {} {}
\wdef 6:26 {nařízení} {} {} {rozkaz} {} {}
\wdef 6:28 {šťastně} {} {} {dobře dařilo} {} {}
\wdef 7:3  {čtyři šelmy} {} {} {čtyři veliká zvířata} {} {}
\wdef 7:4  {lvu} {} {} {lev} {} {}
\wdef 7:5  {nedvědu} {} {} {medvědu} {} {}
\wdef 7:6  {pardovi} {} {} {levhart} {} {}
\wdef 7:7  {šelma čtvrtá} {} {} {čtvrté zvíře} {} {}
\wdef 7:8  {roh poslední} {} {} {další malý roh} {} {}
\wdef 7:8  {oči podobné očím lidským} {} {} {oči lidské} {} {}
\wdef 7:9  {Starý dnů} {} {} {Věkovitý} {} {}
\wdef 7:9  {roucho} {} {} {oblek} {} {}
\wdef 7:9  {trůn} {} {} {stolec} {} {}
\wdef 7:13 {Synu člověka} {} {} {Syn člověka} {} {}
\wdef 7:13 {s oblaky} {} {} {s nebeskými oblaky} {} {}
\wdef 7:14 {panství} {} {} {království} {} {}
\wdef 7:14 {všickni lidé} {} {} {všichni lidé} {} {}
\wdef 7:15 {zhrozil} {} {} {naplnila hrůzou} {} {}
\wdef 7:18 {království svatých} {} {} {království se ujmou svatí} {} {}
\wdef 7:18 {výsostí} {} {} {Nejvyššího} {} {}
\wdef 7:22 {Starý dnů} {} {} {Věkovitý} {} {}
\wdef 7:28 {v srdci} {} {} {ve svém srdci} {} {}
\wdef 8:1  {Léta třetího kralování Balsazara} {} {} {V třetím roce kralování krále Belšasara} {} {}
\wdef 8:3  {vyšší} {} {} {větší} {} {}
\wdef 8:4  {trkal} {} {} {trkat} {} {}
\wdef 8:4  {veliké} {} {} {Dělal, co se mu zlíbilo} {} {}
\wdef 8:8  {velikým} {} {} {se velice vzmohl} {} {}
\wdef 8:8  {čtyři místo něho, na čtyři strany světa} {} {} {čtyř nebeských větrů} {} {}
\wdef 8:9  {k zemi Judské} {} {} {k nádherné zemi} {} {}
\wdef 8:10 {některé} {} {} {část toho zástupu} {} {}
\wdef 8:10 {svrhl} {} {} {srazil} {} {}
\wdef 8:11 {knížeti} {} {} {veliteli} {} {}
\wdef 8:11 {zastavena} {} {} {zrušil} {} {}
\wdef 8:12 {vojsko to vydáno v převrácenost proti ustavičné oběti} {} {} {Zástup byl sveden ke vzpouře} {} {}
\wdef 8:12 {šťastně mu se dařilo} {} {} {dařilo se mu} {} {}
\wdef 8:17 {času} {} {} {doby konce} {} {}
\wdef 8:20 {Skopec} {} {} {beran} {} {}
\wdef 8:21 {Kozel} {} {} {kozel} {} {}
\wdef 8:25 {knížeti} {} {} {Veliteli velitelů} {} {}
\wdef 9:26 {Mesiáš} {} {} {pomazaný} {} {}
\wdef 10:12 {přiložil srdce své, abys rozuměl} {} {} {kdy ses rozhodl porozumět} {} {}
\wdef 10:13 {kníže království Perského} {} {} {ochránce perského království} {} {}
\wdef 10:13 {jedenmecítma dnů} {} {} {po jednadvacet dní} {} {}
\wdef 11:31 {obět} {} {} {oběť} {} {}
\wdef 11:34 {malou pomoc} {} {} {trochu pomoci} {} {}


         % Language variants

\BookTitle Gen  Gn {První Mojžíšova (Genesis)}
\BookTitle Exod Ex {Druhá Mojžíšova (Exodus)}
\BookTitle Lev  Lv {Třetí Mojžíšova (Leviticus)}
\BookTitle Num  Nu {Čtvrtá Možíšova (Numeri)}
\BookTitle Deut Dt {Pátá Mojžíšova (Deuteronomium)}
\BookTitle Josh Joz {Jozue}
\BookTitle Judg Sd {Soudců}
\BookTitle Ruth Rt {Rút}
\BookTitle 1Sam 1S {První Samuelova}
\BookTitle 2Sam 2S {Druhá Samuelova}
\BookTitle 1Kgs 1Kr {První Královská}
\BookTitle 2Kgs 2Kr {Druhá Královská}
\BookTitle 1Chr 1CPa {První Paralipomenon (1. Letopisů)}
\BookTitle 2Chr 2Pa {Druhá Paralipomenon (2. Letopisů)}
\BookTitle Ezra Ezd {Ezdráš}
\BookTitle Neh  Neh {Nehemjáš}
\BookTitle Esth Est {Ester}
\BookTitle Job  Jb {Jób}
\BookTitle Ps   Ž {Žalmy}
\BookTitle Prov Př {Přísloví}
\BookTitle Eccl Kaz {Kazatel}
\BookTitle Song Pís {Píseň písní}
\BookTitle Isa  Iz {Izajáš}
\BookTitle Jer  Jr {Jeremjáš}
\BookTitle Lam  Pl {Pláč}
\BookTitle Ezek Ez {Ezechiel}
\BookTitle Dan  Da {Daniel}
\BookTitle Hos  Oz {Ozeáš}
\BookTitle Joel Jl {Jóel}
\BookTitle Amos Am {Ámos}
\BookTitle Obad Abd {Abdijáš}
\BookTitle Jonah Jon {Jonáš}
\BookTitle Mic  Mi {Micheáš}
\BookTitle Nah  Na {Nahum}
\BookTitle Hab  Abk {Abakuk}
\BookTitle Zeph Sf {Sofonjáš}
\BookTitle Hag  Ag {Ageus}
\BookTitle Zech Za {Zacharjáš}
\BookTitle Mal  Mal {Malachiáš}
\BookTitle Matt Mt {Matouš}
\BookTitle Mark Mk {Marek}
\BookTitle Luke L {Lukáš}
\BookTitle John J {Jan}
\BookTitle Acts Sk {Skutky apoštolské}
\BookTitle Rom  Ř {Římanům}
\BookTitle 1Cor 1K {První list Korintským}
\BookTitle 2Cor 2K {Druhý list Korintským}
\BookTitle Gal  Ga {Galatským}
\BookTitle Eph  Ef {Efezským}
\BookTitle Phil Fp {Filipským}
\BookTitle Col  Ko {Koloským}
\BookTitle 1Thess 1Te {První list Tesalonickým}
\BookTitle 2Thess 2Te {Druhý list Tesalonickým}
\BookTitle 1Tim 1Tm {První list Timoteovi}
\BookTitle 2Tim 2Tm {Druhý list Timoteovi}
\BookTitle Titus Tt {Titovi}
\BookTitle Phlm  Fm {Filemonovi}
\BookTitle Heb   Žd {Židům}
\BookTitle Jas   Jk {List Jakubův}
\BookTitle 1Pet  1Pt {První list Petrův}
\BookTitle 2Pet  2Pt {Druhý list Petrův}
\BookTitle 1John 1J {První list Janův}
\BookTitle 2John 2J {Druhý list Janův}
\BookTitle 3John 3J {Třetí list Janův}
\BookTitle Jude  Ju {List Judův}
\BookTitle Rev   Zj {Zjevení Janovo}     

\BookException Ž   {\def\amark{Z}}
\BookException Př  {\def\amark{Pr}}
\BookException Pís {\def\amark{Pis}}
\BookException Ř   {\def\amark{R}}
\BookException Žd  {\def\amark{Zd}}

  % Czech book titles

\def\printedbooks {%
   Gn Ex Lv Nu Dt Joz Sd Rt 1S 2S 1Kr 2Kr 1CPa 2Pa Ezd Neh
   Est Jb Ž Př Kaz Pís Iz Jr Pl Ez Da Oz Jl Am Abd Jon Mi
   Na Abk Sf Ag Za Mal
   Mt Mk L J Sk Ř 1K 2K Ga Ef Fp Ko 1Te 2Te 1Tm 2Tm
   Tt Fm Žd Jk 1Pt 2Pt 1J 2J 3J Ju Zj
}
\def\printedbooks {Da}

%\checksyntax articles-Da notes-Da intro-Da {}

\processbooks % Generates books declared in \printedbooks

\bye
