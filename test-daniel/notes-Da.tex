\CommentedBook {Da}
%          BKR       PSP                CSP          CEP            B21        SNC
%\wdef 5:31 {Darius}  {Dárjáveš}    {Dareios}~6:1 {Darjaveš}~6:1 {Darjaveš}~6:1  {Darjaveš}~6:1 ;

\Note 1:1-6:28 {}={Vyprávění}  První část knihy vyzdvihuje jednak Boží absolutní svrchovanost nad královstvími tohoto světa, jedna upřímnou odevzdanou věrnost Bohu, kterou Daniel a jeho přátelé projevovali.
Daniel chtěl svým čtenářům vštípit přesvědčení, že přestože Boží lid někdy trpívá pronásledování, králové a království povstávají a hroutí se podle Božích záměrů. Daniel také učí, že Bůh se hojně odmění těm, kdo jemu, Danielovi, věnují pozornost coby Božímu mluvčímu. Tento materiál je rozdělen do šesti navzájem nezávislých vyprávění, každé v jedné kapitole a každé obsahuhjící nějaký zázrak:
Uchování rituální čistoty Danielem a jeho přáteli (<1:1-21>); \x/Nabuchodonozor/ův sen (<2:1-49>); záchrana z ohnivé pece (<3:1-30>); \x/Nabuchodonozor/ův druhý sen (<4:1-37>); soud nad \x/Baltazar/em (<5:1-31>) a záchrana Daniela ze lví jámy (<6:1-28>).

\Note 1:1-21 {}={Uchování rituální čistoty}  Prorok uvozuje kontext své knihy vyprávěním osobní historie (své i svých přátel) zajetí, vzdělání, věrnosti Bohu a služby králi \x/Nabuchodonozor/ovi.

\Note 1:1 {\x/Joakim/}  Tj. 605 př.Kr.; téhož roku, kdy \x/Nabuchodonozor/ porazil ;   asyrsko-egyptskou koalici v bitvě u Karchemiše (<Jr 46:2>) a zahájil tak mezinárodní vzestup Babylónu k moci. Po bitvě u Karchemiše \x/Nabuchodonozor/ zaútočil na \x/Joakim/a (<2Kr 24:1-2>; <2Pa 36:5-7>) a zajal Daniela a jeho přátele. 
To byla první ze tří \x/Nabuchodonozor/ových  invazí do Judska. Druhá nastala roku 597 př.Kr. (<2Kr 24:10-14>) a třetí 587 př.Kr. (<2Kr 25:1-24>). Zdánlivou diskrepanci mezi <Da 1:1> a <Jr 25:1> a <Jr 46:2> (kde \x/Jeremiáš/ umisťuje \x/Nabuchodonozor/ův útok na \x/Joakim/a do \x/Joakim/ova čtvrtého roku místo třetího) lze objasnit rozdílem mezi babylónským a židovským systémem chronologie. V babylónském systému, který používá Daniel, byl první rok vlády panovníka považován za \uv{korunovační rok} a vláda samotná se počítala až od prvního dne měsíce Nisan následujícího roku.

\Note 1:1 {Nabuchodonozor král Babylonský}  \x/Nabuchodonozor/ přivedl Babylóňany k vítězství u Karchemiše v roce 605 př.Kr. coby korunní princ a velitel armády. Krátce po tomto vítězství se ujal babylónského trůnu po smrti svého otce Nabopolasara (626--605 př.Kr.). \x/Nabuchodonozor/ova vláda (605--562 př.Kr.) tvoří většinu historického pozadí  biblických knih \x/Jeremiáš/, \x/Ezechiel/ a \x/Daniel/.     

\Note 1:2 {vydal Pán}={Hospodin vydal} Porážka Izraele Babylónem není vysvětlitelná jen pouhou vojenskou a politickou analýzou oné doby. Bůh vždy jednal svrchovaně v záležitostech národů. Babylóňany použil jako nástroj potrestání svého vlastního lidu za porušení smluvních závazků (<2Kr 17:15>, <2Kr 17:18-20>; <2Kr 21:12-15>; <2Kr 24:3-4>).

\Note 1:2 {nádobí} Odkaz na nádobí z vypleněného chrámu, nikoliv zajatců. Deportace proběhly ve třech vlnách: První roku 605 př.Kr., kdy mezi odvlečenými byl i Daniel; druhá roku 597 př.Kr., zahrnovala i Ezechiela. Třetí deportace, při které Babylóňané zničili Jeruzalém a chrám, nastala roku 586 př.Kr.

\Note 1:2 {do domu boha svého}  Hlavní božstvo babylónského panteonu byl Marduk  (srv. <Jr 50:2>).

\Note 1:4 {liternímu umění a jazyku} Babylónská literatura byla psána komplikovaným sumérským a akkadským slabičným klínovým písmem primárně na hliněných tabulkách. Těchto dochovaných tabulek existují tisíce. Studium této literatury seznámilo Daniela a jeho přátele s polyteistickým světonázorem Babylóňanů, plného kouzlení, čarodějnictví a astrologie.  Mluvený jazyk pro běžné použití byla nicméně aramejština (srv. <2:4>), psaná snadno osvojitelným abecedním systémem.

\Note 1:5 {z stolu královského} Později se \x/Joakim/ovi dostalo stejného zaopatření 
(<2Kr 25:27-30>).

\Note 1:6 {\x/Daniel/, \x/Chananiáš/, \x/Mizael/ a \x/Azariáš/} Charakteristická  hebrejská jména. Dvě z nich obsahují prvek EL, znamenající \uv{Bůh}, a dvě JAH, což je zkratka osobního Božího jména, které překládáme jako \uv{Hospodin}.  \x/Daniel/ znamená \uv{Můj soudce je Bůh}, \x/Chananiáš/ \uv{Hospodin je milostivý}, \x/Mizael/ \uv{Kdo je jako Bůh?} a \x/Azariáš/ \uv{Hospodin mi pomáhá}.

\Note 1:7  {Baltazar}={\x/Baltazar/ ... \x/Sidrach/ ... \x/Mesak/ ...  \x/Abedneg/o}
     Přesný význam těchto jmen je předmětem diskusí. Převažují tyto názory: 
     \x/Baltazar/: {\em Bel} [jiné jméno pro Marduka, hlavního boha babylónského panteonu]
     {\em chraň jeho život} nebo {\em Paní, ochraňuj krále}; 
     \x/Sidrach/: {\em Velice se bojím (Boha)} nebo {\em Přikázání Aku} [sumérského měsíčního boha];
     \x/Mesak/:   {\em Jsem bezvýznamný} nebo {\em Kdo je to, co Aku?};
     \x/Abedneg/o: {\em Služebník zářícího.}
     
\Note 1:8 {nepoškvrňoval}={neposkvrní} Důvod, pro který byl Daniel přesvědčen, že by ho králův pokrm poskvrnil, není uveden. Pravděpodobně jídlo znamenalo porušení dietních předpisů Mojžíšova zákona  (<Lv 11:1-47>), zakazujících konzumaci vepřového nebo masa nezbaveného krve (<Lv 17:10-14>). Také mohlo zahrnovat pokrmy, obětované babylónský modlám. 

\Note 1:9 {milost a lásku u správce} Danielův osud v mnohém připomíná Josefův příběh (<Gn 39-41>).

\Note 1:12 {deset} Často se symbolickým významem dokonalosti nebo plného počtu. \dopsat % Abrahamova přímluva za sodomu, poznámka nebo článek

\Note 1:14 {uposlechl} Daniel neslíbil, že se v případě zchátralejšího zevnějšku přizpůsobí a poslechnou nařízení v rozporu s Božím Zákonem. Je možné (a ve světle dalších kapitol i docela pravděpodobné), že už tehdy byli rozhodnuti neposlechnout bezbožného vladaře a raději zemřít, než zkompromitovat víru.

\Note 1:15 {tváře jejich byly krásnější} Bůh Danielovi a jeho přátelům požehnal pro jejich věrnost Božímu Slovu (<Dt 8:3>; <Mt 4:4>). Nepřál jim smrt, která by je nejspíše čekala, kdyby byl správce s výsledkem nespokojen, a oni by přesto na svém odmítání rituálně nečisté stravy trvali
(srv. <"pozn." 14>). 

\Note 1:17 {moudrost} Danielova moudrost se stala příslovečnou ještě za jeho života; Ezechiel říká králi Týru ironicky, že je moudřejší nad Daniela (<Ez 28:3>). 

\Note 1:17 {vidění a snům}={vidění a sny} Daniel převyšoval i své přátele schopností interpretovat sny, pro kterou byl vyvýšen nade všechny ostatní, podobně jako kdysi Josef u faraonova dvora (<Gn 40:8>; <Gn 41:16>). 

\Note 1:18 {dokonali dnové}={po uplynutí doby} Po třech letech, zmíněných ve  <"v." 5>.

\Note 1:20 {desetkrát} viz (<"pozn." 12>n).

\Note 1:20 {mudrce a hvězdáře} První výraz se vyskytuje také v <Gn 41:8, 24> a <Ex 7:11>; druhý se objevuje pouze zde a v <2:2>. 
Srv. (<"pozn." Gn 41:8>n). \dopsat %Hebrew

\Note 1:21 {léta prvního Cýra krále} Tj. \x/Cýr/ovy vlády nad Babylónem, tedy 539 př.Kr. Daniel v roce 537 př.Kr. dosud žil (<10:1>); dožil se návratu Judejců ze zajetí do Země.

\Note 2:1-49 {}={\it\x/Nabuchodonozor/ův první sen\/} Daniel ve službách krále (mimo jiné) interpretoval jeho sny, což odhaluje nejen, že  byl (\x/Daniel/) zahrnut Božím požehnáním, ale také že Bůh směřoval historii  k nastolení svého království.

\Note 2:1 {Léta pak druhého} První rok výcviku Daniela a jeho přátel byl
zároveň \x/Nabuchodonozor/ův \uv{korunovační rok}. Druhý a třetí rok výcviku byl první a druhý rok
\x/Nabuchodonozor/ovy vlády. Srv <"pozn." 1:1>n).

\Note 2:1 {} Mezi tímto tvrzením a dokončením tříletého výcviku mladíků (<1:5>) není žádná diskrepance: 
První  \x/Nabuchodonozor/ův rok byl Babylóňany považován za \uv{korunovační rok} a druhý a třetí rok Danielovy výuky za první a druhý  \x/Nabuchodonozor/ovy vlády (viz <"pozn." 1:1>n).

\Note 2:1 {ze sna protrhl} Na Starém Blízkém Východě bylo běžné mít za to, že božstva promlouvají k lidem také skrze některé (hlavně velmi živé) sny. \x/Nabuchodonozor/ovo zneklidnění je pochopitelné; poselství shůry mělo implikace pro budoucnost království. Zapomenutí snu bylo považováno za hněv božstva vůči adresátovi.

\Note 2:2 {mudrce} viz <"pozn." 1:20>n. 

\Note 2:4 {Syrsky} Odsud až po konec kapitoly 7 je text psán aramejsky a ne hebrejsky, podobně jako např.  <Ezd 4:8-6:18>. Aramejština byl oficiální úřední jazyk; nelze také vyloučit, že tyto pasáže jsou v aramejštině toho důvodu,  že obsahují proroctví, která mohla pohany zajímat více než Židy.

\Note 2:5 {Neoznámíte-li mi snu} \x/Nabuchodonozor/ zkouší své poradce. Nebudou-li schopni připomenout mu jeho sen, nemá proč důvěřovat ani jejich potencionálnímu výkladu (srv. <9>).

\Note 2:11 {kromě bohů} Mudrcům nezbylo než přiznat, faraonův požadavek je nad jejich síly. S odkazem na nadpřirozený zdroj měli pravdu, kterou potvrzuje i \x/Daniel/ (<19>; <27>).

\Note 2:18 {Bohu nebeskému} \x/Daniel/ se modlí rovnou k Bohu, který vládne hvězdám, na rozdíl od astrologů, kteří se doptávají jenom hvězd samotných, protože nikoho vyššího neznají.

\Note 2:19 {věc tajná} Skryté věci jsou u Hospodina (<Dt 29:28>). Jsou {\it transcendentní\/} --- nepoznatelné, pokud nejsou zjeveny. Později je tentýž výraz použit pro označení Božích skrytých záměrů s dějinami světa (<4:6>). 

\Note 2:21 {ssazuje krále, i ustanovuje krále}={Krále sesazuje,krále nastoluje}. Narážka na obsah snu, líčící sérii povstávajících a zanikajících impérií, se ještě za \x/Nabuchodonozor/ova života stane jeho vlastní palčivou zkušeností (<4:28-30>). 

\Note 2:22 {skryté} Viz <"pozn." 2:11>. 

\Note 2:23 {oslavuji a chválím} \x/Daniel/ vyjadřuje hlubokou vděčnost za Boží milost vyslyšením modlitby. Boží zjevení je v příkrém kontrastu s mlčením falešných božstev pohanských hadačů. Jenom Bůh zná všechno a vládne všemu. \x/Daniel/a vyvýšil mimořádným poznáním, jež mu svěřil.

\Note 2:24 {výklad ten oznámím}={výklad ... přednesu} \x/Daniel/ zde mluví pouze o výkladu, čímž implikuje, že obsah snu je mu již znám.

\Note 2:28 {jest Bůh na nebi, kterýž zjevuje tajné věci}={Bůh, zjevující tajemství} Podobně jako Josef v Egyptě (<Gn 40:8>; <Gn 41:16>) si ani \x/Daniel/ nepřisvojuje poznání a výklad snu, nýbrž připisuje je Bohu.

\Note 2:28 {v potomních dnech}={v budoucích dnech} Dosl. \uv{v posledních dnech}, čemuž sz lidé         rozuměli jako době obnovy národa po návratu z exilu (viz <Dt 4:30>). Tatáž fráze může označovat 
    jakoukoliv budoucnost  (<Gn 49:1>).  V NZ je použita celkem pětkrát, z čehož dvakrát odkazuje na věk, započatý o Letnicích (<Sk 2:17>; <Žd 1:2>), a třikrát na závěr dějin před Kristovým druhým příchodem (<2Tm 3:1>; <Jk 5:3> a <2Pt 3:3>). 

\Note 2:32-33 {}={Nejtěžší nahoře, nejkřehčí dole} Směrem odshora dolů materiálům ubývá nejen na            váze, ale i na hodnotě. Obraz vystihuje osud všech impérií,  nejen těchto čtyř nejbližších: 
    Každá instituce, stát, říše, civilizace se nakonec zhroutí vlastní vahou, když  nezadržitelná byrokracie přeroste svůj původní účel služby lidem a sama tyje na jejich úkor.

\Note 2:34 {kámen}={bez dotyku lidské ruky} Na rozdíl od ostatních království, budovaných lidmi,       toto vystaví Bůh sám. Zde  je kámen asociován s Kristovým královstvím 
   (srv. <1Kor 10:4> a <"pozn." 1Kor 10:4>n).

\Note 2:38-40 {hlava zlatá}={zlatá hlava... čtvrté království} Čtyři království reprezentují po sobě jdoucí impéria babylónské, médo-perské, řecké a římské. Bohem založené nepomíjitelné Kristovo království bude inaugurováno v časech římské světovlády (viz též Úvod a graf \uv{Danielovy vize} v <Da 7> na str.~\ref[Danielovy vize]).

\Note 2:44 {}={za dnů oněch králů} Někteří interpreti tvrdí, že se jedná o linii králů posledního            království; daleko pravděpodobnější však je rozumět tomuto verši jako označujícímu panovníky         království, zmíněných ve verších <2:38-40>.

\Note 2:44 {}={Nezničitelné království} Podobně jako i jiní proroci, \x/Daniel/ hovoří o království, které Bůh založí po návratu z exilu jako permanentní (srv. např. <Iz 9:7>; <Jl 2:26-27> či 
<Am 9:15>). NZ učí, že toto království začalo Kristovým prvním příchodem a dosáhne svého dovršení jeho slavným návratem (viz článek <"Království Boží u" Mt 4> na str. \ref[Království Boží]). 

\Note 2:46 {}={král padl na tvář} Pozoruhodná výměna rolí anticipuje příchod Božího království,         vysoce převyšujícího i nejmocnější lidská impéria.

\Note 2:47 {Bůh bohů a Pán králů}={Bůh bohů a Pán králů} \x/Nabuchodonozor/ vyznává Hospodinovu svrchovanost nejen nad bezmocnými pohanskými božstvy, ale i nad králi, jako je on sám. To je téma, sjednocující prvních šest kapitol knihy Daniel.

\Note 2:48 {krajinou Babylonskou}={Babylonskou provincií} Babylónské impérium bylo rozděleno na provincie;  \x/Daniel/ byl ustanoven vládcem nad provincií s hlavním městem. 
Podobný vzestup Židů k moci v cizích zemích lze vidět v <Gn 41:37-44> (Josef) a <Est 8:1-2> (\x/Mardocheus/). \x/Daniel/ovi přátelé ho na jeho přímluvu u krále v této pozici nahradili (<49>).

\Note 2:49 {býval v bráně královské}={zůstával na královském dvoře} Pravděpodobně důvod, proč se ho netýkal trest za odmítnutí poklonit se \x/Nabuchodonozor/ově modle (<3:20>).

\Note 3:1 {obraz}={sochu} Názory badatelů, zda se jedná o mimořádnou podobiznu  \x/Nabuchodonozor/a samotného, či zda to bylo zobrazení nějakého babylónského božstva, nebo pouhý obelisk, se liší. Z toho, co je známo o babylónských náboženských tradicích, je pravděpodobné, že obraz zpodobňoval Béla nebo Nabu, některé z těchto \x/Nabuchodonozor/ových ochraňujících božstev. Padnout na tvář před božstvem znamenalo zároveň vyjádření podřízenosti \x/Nabuchodonozor/ovi, který božstvo reprezentoval na zemi (srv. <2:46> a <"pozn." 2:46>n). 

\Note 3:1 {zlatý}={ze zlata}  Pravděpodobně pozlacený; zhotovení sochy se podobalo popisu v 
<Iz 40:19>; <Iz 41:7> a  <Jr 10:3-9>.

\Note 3:1 {výška}={60x6} Rozměry jsou důvodem, proč někteří vyvozují, že obraz byl spíše obelisk, než podobizna člověka (jehož proporce jsou 6:1). Socha však mohla stát na piedestalu, nebo mít stylizovaný tvar.

\Note 3:1 {\x/Dura/} Přesné umístění není známo. Obvykle bývá spojeno s Tolul Dura asi 
10 km jižně od Babylóna.

\Note 3:2 {}={satrapy ... hodnostáře} Přesné rozsahy pravomocí těchto  různých druhů úředníků nejsou známy. Pět ze sedmi termínů vypadá na perský původ, což by naznačovalo, že \x/Daniel/ tento zápis dokončil až po dobytí Babylóna Peršany  roku 539 př.Kr.

\Note 3:5 {trouby}={rohu ... dud} Tři z vyjmenovaných šesti hudebních nástrojů  nesou jména, převzatá z řečtiny (citera, harfa a dudy). To vede některé interprety k názoru, že kniha byla sepsána až po dobytí Persie Alexandrem Velikým. 
To je však závěr, který nevyplývá nutně z premisy (tzv. argument non-sequitur); mezi hudebníky je běžné používat mezinárodní termíny k označení hudebních nástrojů. Pojmy jako  \uv{gibsonka}, \uv{jumbo} \uv{strato/tele-caster}, \uv{Les Paul} apod. jsou jednoznačně srozumitelné jak češtině (v rámci muzikantského slangu), tak i v mezinárodním kontextu; a přesto z jejich zdomácnělé přítomnosti v jazyce nelze vyvodit závěr, že se v něm objevily až po zhroucení železné opony v roce 1989, kdy se západním vlivům otevřela volná cesta. 

\Note 3:6 {ohnivé}={ohnivá pec} Pece byly v Babylóně běžně používány k vypalování cihel  (<"srv." Gn 11:3>). Nebylo neobvyklé používat je jako nástroj popravy upálením zaživa (<"viz" Jr 29:22> nebo též 2Mak 7).  

\Note 3:8 {Kaldejští}={mágové} <"Viz pozn." 2:2>n.

\Note 3:8 {}={Chaldejci} Výraz, indikující národnost, nikoliv funkci. Chaldejci shlíželi spatra na Židy z rasově-etnických antisemitských předsudků (<"srv." 12>; <Est 3:5-6>). 
Privilegovaná pozice \x/Sidrach/a, \x/Mizael/a a  \x/Abedneg/a znásobila nevraživost Chaldejců vůči nim (<12>).

\Note 3:12 {Sidrach}={\x/Sidrach/, \x/Mizach/ a \x/Abedneg/o} \dopsat %\x//(<"">)

\Note 3:15 {který jest ten Bůh} \x/Nabuchodonozor/ nevědomky vyzval Hospodina, Boha Izraele, ke změření sil; z jeho polyteistické, pohanské perspektivy žádný bůh ničeho podobného není schopen.

\Note 3:17  {vytrhne nás}={vysvobodí} Věrní služebníci ani na vteřinu nepochybují o Boží                        svrchovanosti, přestože jsou si vědomi, že ve své všemohoucnosti je \uv{všeho schopen};              nesázejí na automaticky zaručenou ochranu za všech okolností. 

\Note 3:18 {}={i kdyby ne} \x/Daniel/ovi přátelé počítají s reálnou možností, že věrnost Bohu je bude stát život. To je však nezviklá v jejich rozhodnutí zůstat věrni. Věrnost poddaných přináší Králi slávu (<29-33>), o to větší, když je to věrnost tváří v tvář smrti.

\Note 3:25 {}={syna bohů} Fráze, použitelná pro různé druhy nebeských bytostí; zde je míněn \uv{anděl} <"v." 28>.

\Note 3:26 {služebníci Boha nejvyššího} Titul pro Boží univerzální autoritu. Podobně jako ve
        <"verši" 29> a <2:47> neznamená toto vyznání ze rtů polyteistického pohana, 
        že \x/Daniel/ův Bůh je jediný živý, nýbrž že je nadřazený ostatním božstvům (<4:31-34>).
        Z úst věrného Izraelity totéž vyznání implikuje monoteismus (<5:18,21>; <7:18-21>).

\Note 3:28 {anděla} Anděl může být \uv{anděl Hospodinův}, jenž může znamenat Kristovo preinkarnační zjevení (<"srv." 6:22>; viz též <"poznámky k" Gn 16:7>n a <Ex 3:2>n). Bůh přislíbil svou přítomnost, budou\discretionary{-}{-}{-}li jeho lidé nuceni projít ohněm (<Iz 43:1-3>). 

\Note 3:29 {není}={není jiného Boha} Viz <"pozn." 26>n.

\Note 3:30 {zvelebil} Příběh si dává záležet na tom, aby bylo jasné, že \x/Daniel/ovi přátelé dosáhli vyvýšení svou věrností  Bohu, nikoliv kompromisem s Babylóňany.

\wdef 3:31  
    {Nabuchodonozor král}   %BKR
    {Nevúkadneccar, král}   %PSP
    {Král Nebúkadnesar}   %CSP
    {Král Nebúkadnesar}   %CEP
    {Král Nabukadnezar}   %B21
    {krále Nebúkadnesara}   %SNC
\Note 3:31 {Nabuchodonozor král} Poslední incident v knize, spojený s \x/Nabuchodonozor/em. Je umístěn do pozdního období jeho třiačtyřicetileté vlády, kdy dokončil své stavební projekty (<4:27>) a byl na svém vrcholu, neomezený vládce nejmocnější říše na světě. 

\wdef 3:32  
    {Bůh nejvyšší}   %BKR
    {Bůh, Nejvyšší}   %PSP
    {Bůh nejvyšší}   %CSP
    {Bůh nejvyšší}   %CEP
    {Nejvyěšší Bůh}   %B21
    {nejvyšší Bůh}   %SNC
\Note 3:32 {Bůh nejvyšší} Viz <"poz." 2:47>n;  <3:26>n a <3:28>.

\wdef 3:33  
    {jak veliká}   %BKR
    {jak velmi jsou veliká}   %PSP
    {Jak veliká}   %CSP
    {Jak veliká}   %CEP
    {Jak nesmírné}   %B21
    {Jak mimořádné}   %SNC
\Note 3:33 {jak veliká}...  \x/Nabuchodonozor/ovo vyznání opakovaně připomíná hlavní téma knihy, absolutní svrchovanost Boží ("srv." <4:33-34> a <"pozn." 7:1-12:13>n).

\Note 4:3-4 {} Viz <"pozn." 1:20>n a <2:2>n.

\Note 4:5 {Baltazar} <"Viz pozn." 1:7>.

\wdef 4:6  
    {duch bohů svatých}   %BKR
    {duch svatých bohů}   %PSP
    {duch svatých bohů}   %CSP
    {duch svatých bohů}   %CEP
    {duch svatých bohů}   %B21
    {ducha svatého Boha}   %SNC
\Note 4:6 {duch bohů svatých} Ačkoliv \x/Nabuchodonozor/ vyjadřuje svůj respekt k \x/Daniel/ovi pohanskou terminologií, dotýká se pravdy: Přítomnost Ducha Svatého má na člověka nepřehlédnutelný účinek; v \x/Daniel/ově případě je to vhled do Božích neproniknutelných tajemství, dar, který o mnoho později obdržel i apoštol Pavel a církev  (<1Kor 2:6-16>).

\wdef 4:6
    {nic tajného není tobě nesnadného}   %BKR
    {žádné tajemství tě netísní}   %PSP
    {žádné tajemství ti nedělá potíže}   %CSP
    {žádné tajemství ti nedělá potíže}   %CEP
    {ti žádné tajemství není těžké}   %B21
    {umíš vyložit každé tajemství}   %SNC
\Note 4:6 {nic tajného není tobě nesnadného}  <"Viz" 2:47> a <"pozn." 2:19>n.

\Note 4:7 {strom} Ezechiel 31 také líčí království metaforou vysokého stromu. Podobné obrazy vidíme např. v <Ž 1:3>; <Ž 37:35>; <Ž 52:10>; <Ž 92:13-14>; <Jr 11:16-17>; <Jr 17:8> nebo <Mt 13:32>. 

\Note 4:10 {svatý}  \x/Nabuchodonozor/ připouští, že viděl svatou nebeskou bytost. Víra v podobné bytosti byla na Starém Blízkém Východě běžná a koreluje s biblickým přesvědčením, že Bůh zasahuje do pozemských záležitostí, mnohdy skrze své služebníky anděly.

\wdef 4:13  
    {Srdce jeho od lidského ať jest rozdílné}   %BKR
    {Jeho srdce nechť se z lidského změní}   %PSP
    {Jeho srdce z lidského ať se promění}   %CSP
    {Jeho srdce ať je jiné, než je srdce lidské}   %CEP
    {Lidský rozum ať jej opustí}   %B21
    {se nebude chovat jako člověk}   %SNC
\Note 4:13 {Srdce jeho od lidského ať jest rozdílné}
    Je zřejmé, že se jedná o člověka, nikoliv o strom. Srv. <"pozn." 4:19>.

\wdef 4:13  
    {srdce zvířecí nechť jest dáno jemu}   %BKR
    {nechť je mu dáno srdce zvířete}   %PSP
    {srdce budiž mu dáno zvířecí}   %CSP
    {ať je mu dáno srdce zvířecí}   %CEP
    {dostane rozum zvířecí}   %B21
    {bude žít jako zvíře}   %SNC
\Note 4:13 {srdce zvířecí nechť jest dáno jemu} 
     \x/Nabuchodonozor/  byl postižen mentální poruchou zvanou lykantropie 
     (z řeckého {\em lukos} -- vlk a {\em anthropos} -- člověk), 
     při níž  se člověk chová jako vlk nebo i jiné zvíře (<30>; viz též <"pozn."30>n. 

\wdef 4:13  
    {sedm let}   %BKR
    {sedm dob}   %PSP
    {sedm časů}   %CSP
    {sedm let}   %CEP
    {sedm období}   %B21
    {Po dobu sedmi let}   %SNC
\wdef 4:13  %Není v textu, dořešit
    {léto}   %BKR
    {doba}   %PSP
    {čas}   %CSP
    {léto}   %CEP
    {období}   %B21
    {léto}   %SNC
\Note 4:13 {sedm let} 
     Sedm období neurčené délky (srv. vv. <20> a <22>). Většina interpretů se shoduje na závěru, že {\em \x/léto/} znamená jeden rok. Verš <30> naznačuje, že doba byla delší než den, týden či měsíc.

\wdef 4:19  
    {Ty jsi ten}   %BKR
    {to jsi ty}   %PSP
    {to jsi ty}   %CSP
    {jsi ty}   %CEP
    {ten strom jsi ty}   %B21
    {ten strom jsi ty}   %SNC
\Note 4:19 {Ty jsi ten} Pointa vyprávění, podobná Nátanově napomenutí Davida (<2Sa 14:7>), znamená přímou aplikaci pro \x/Nabuchodonozor/a.

%\wdef 4:22  % upravit později, až bude jak
%    {}   %BKR
%    {}   %PSP
%    {}   %CSP
%    {}   %CEP
%    {}   %B21
%    {}   %SNC
\Note 4:22 {}={zaženou ... s polní zvěří ... spásat porost}  \x/Daniel/ opakuje po nebeském poslu (<13>) popis mentální poruchy, kterou Hospodin postihne nejmocnějšího muže světa. Podobnými symptomy občas trpěl panovník Království Velké Británie a Irska Jiří III. nebo Ota I. Bavorský. Viz <"pozn." 13>n.

\wdef 4:22
    {dokudž bys nepoznal}   %BKR
    {než budeš moci poznat}   %PSP
    {dokud nepoznáš}   %CSP
    {dokud nepoznáš} %CEP
    {než poznáš}   %B21
    {poznáš}   %SNC
\Note 4:22 {dokudž bys nepoznal} Záměrem \x/Nabuchodonozor/ova ponížení bylo přimět ho uznat Boží svrchovanost. 

\wdef 4:23
    {království tvé tobě zůstane}   %BKR
    {bude tvé království zajištěno}   %PSP
    {tvé království se ti dostane}   %CSP
    {tvé království se ti opět dostane}   %CEP
    {své království znovu dostaneš}   %B21
    {své království dostaneš zpět}   %SNC
\Note 4:23 {království tvé tobě zůstane} \x/Daniel/ ujišťuje \x/Nabuchodonozor/a, že poté, co uzná Boží svrchovanost, ani izolace dlouhodobou duševní poruchou ho nepřipraví o království.

\wdef 4:23
    {nebesa}   %BKR
    {}   %PSP
    {}   %CSP
    {Nebesa}   %CEP
    {}   %B21
    {}   %SNC
\Note 4:23 {nebesa}  První výskyt v Písmu, kdy výraz \uv{nebesa} je použit jako synonymum pro Boha. Srv. např.  <Mt 5:3> s <Lk 6:20>.

\wdef 4:30
    {bylinu jako vůl jedl}   %BKR
    {jedl býlí jako skot}   %PSP
    {porost jako dobytek spásal}   %CSP
    {pojídal rostliny jako dobytek}   %CEP
    {jedl trávu jako býk}   %B21
    {jedl byliny jako dobytek}   %SNC
\Note 4:30 {bylinu jako vůl jedl}
     Vzhledem k tomu, že  se \x/Nabuchodonozor/ projevoval rysy charakteristickými pro býložravce, je jeho mentální porucha někdy nazývána {\em boantropií.} Viz <"pozn." 13>.

\wdef 4:34  
    {krále nebeského}   %BKR
    {Krále nebes}   %PSP
    {Krále nebes}   %CSP
    {Krále nebes}   %CEP
    {Krále nebes}   %B21
    {Krále, který je na nebi}   %SNC
\Note 4:34 {krále nebeského} Toto jedinečné slovní spojení shrnuje téma kapitoly: vládu Boha z nebes (srv. <23> a <"pozn." 23>n). 


\Note 5:1 {Balsazar}
     \x/Balsazar/ znamená \uv{{\em Bel} -- ochraňuj krále!} 
     Nezaměnit se jménem \x/Baltazar/, které v Babylóně dostal Daniel (viz <"pozn." 1:7>n). 
     Nabonidus, \x/Nabuchodonozor/ův zeť, byl posledním vládcem Babylónu. 
     \x/Balsazar/, nejstarší Nabonidův syn, byl ustanoven spoluvládcem společně se svým otcem.
     Byl mu svěřen Babylón, zatímco Nabonidus trávil mnoho času v Arábii.
     Události kapitoly 5 se odehrály v roce 539 př.Kr. (42 let po \x/Nabuchodonozor/ově
     smrti v roce 563 př.Kr., kdy Babylón padl do rukou Peršanů a kdy byl vydán
     edikt, propouštějící Izraelity z otroctví. 
     
     
\Note 5:2 {}={Když okusil víno}  \x/Balsazar/ se dopustil svatokrádežného zločinu i z hlediska pohanství, které   posvátné předměty jiných náboženství chová v úctě.

\Note 5:2 {}={nádoby ... z jeruzalémského chrámu} Viz  <"pozn." 1:2>. 
     
\Note 5:2 {otec}  \x/Nabuchodonozor/ je nazýván otcem \x/Balsazar/a zde a ve verších 
<5:11>, <5:13> a <5:18>; a ve verši <5:22> je \x/Balsazar/ nazván \x/Nabuchodonozor/ovým synem.
Ačkoli víme, že \x/Balsazar/ byl přímým synem Nabonida  (srv. <"pozn." 5:1>n), v antickém světě bylo běžné používat pojmu \uv{otec} a \uv{syn} v širším smyslu předka či předchůdce a potomka či následníka. Je pravděpodobné, že \x/Balsazar/ byl \x/Nabuchodonozor/ovým vnukem přes svou matku Nitocris.
     
\Note 5:4 {}={chválili bohy} Nádoby z Hospodinova chrámu byly znesvěceny nejen profánním použitím, ale také účastí na oslavě babylónských falešných božstev. 
     
\Note 5:7 {hvězdáři}={hvězdáři, Kaldejci a hadači} Viz  <"pozn." 1:20>n a  <2:2>n (srv. <2:27>; <4:7>). 
     
\Note 5:7 {}={Kdokoli přečte a vyloží} Opět je požadavek dvojí: (1) přednést obsah znamení a (2) podat jeho výklad  (<"srv." 2:2>).      
     
\Note 5:7 {}={třetím v království} Další v hierarchii po Nabonidovi a jeho spoluvládci \x/Balsazar/ovi  (viz <"pozn." 5:1>n).     
     
\Note 5:8 {}={nedokázali přečíst ani sdělit výklad} Viz  <2:2-13> a  <4:6>; srv. též <Gn 41:8>.   
 
\Note 5:10 {}={královna} Je vysoce nepravděpodobné, že by to byla \x/Balsazar/ova manželka; ty všechny již byly na banketu přítomny  (<2-3>). Mohla to být vdova po \x/Nabuchodonozor/ovi, ale pravděpodobněji to byla  Nitocris, manželka Nabonidova, dcera \x/Nabuchodonozor/a a \x/Balsazar/ova matka.
     
\Note 5:11 {}={duch svatých bohů}   <"Viz" 4:8>. 
Nepřekvapuje, že s událostmi \x/Daniel/ova života byla králova matka obeznámena lépe než  \x/Balsazar/ sám. \x/Daniel/ovi mohlo tou dobou (v roce 539 př.Kr.) být přes 80 let. O 66 let dříve (v roce 605 př.Kr.) byl odvlečen do Babylóna jako mladík (<1:4>).     
     
%\Note 5:12 {}={neobyčejný duch ... jasnozřivost} \x// (<"">)     
     
\Note 5:12 {Baltazar} Viz <"pozn." 1:7>n.     
     
\Note 5:16 {}={třetí v království} Viz  <"pozn." 7>n.     
     
\Note 5:17 {}={někomu jinému} Někteří interpreti jsou toho názoru, že \x/Daniel/ odmítl pocty a odměny nejen ze skromnosti, že o bohatství a moc nestál, ale i s vědomím toho, že schopnost odpovědět králi má jen díky Boží milosti a nechtěl Bohem svěřenou roli zneužít k osobnímu prospěchu (srv. <Gn 14:23>). Nicméně jindy podobné dary bez problémů přijal (<2:48>; <5:29>). Možná, že se chtěl vyhnout jakémukoliv tlaku upravit poselství nevítané zvěsti; možná, že mu bylo jasné, že stejnak není o co stát, když království tuto noc vezme svůj konec.
     
\Note 5:18 {Bůh nejvyšší}   Viz <2:37> a  <4:33>.     
     
\Note 5:18 {}={tvému otci \x/Nabuchodonozor/ovi}   Viz <"pozn." 2>n.
     
\Note 5:20-21 {} Viz  <4:31-33>.     
     
\Note 5:21 {}={nad lidským královstvím panuje Bůh nejvyšší} Toto tvrzení shrnuje teologii celé knihy (Viz Úvod: Záměr a zvláštnosti).     
     
\Note 5:22 {ačkolis}={přestožes o tom všem věděl} \x/Balsazar/ doplatil na to, že nepoužíval rozum  (<Mt 7:24-27>). Byl bez výmluvy ještě více, než \x/Nabuchodonozor/, a proto jeho čas milosti vypršel  (<"viz" 1Tm 1:13>).     
\dopsat %článek o racionalitě: být racionální neznamená nevěřit v Boha a být materialista; být racionální znamená umět vyvodit logické závěry z daných premis. Víra v Boha je axiom, z něhož všechno vyplývá.
     
\Note 5:23 {proti Pánu nebes}  Viz <"pozn." 4:34>n.     
     
\Note 5:24 {Protož}={proto} Nápis na zdi byla Boží odpověď na \x/Balsazar/ovu arogantní pýchu a zpupnost před Bohem, který svou svrchovanost demonstroval o generaci dříve na \x/Nabuchodonozor/ovi  (<4:31-34>).     
     
\Note 5:25 {Mene}={mene mene tekel ufarsin} Dosl. \uv{sečteno, sečteno, zváženo a rozděleno} nebo \uv{mína [váhová, tedy i měnová jednotka], mína, šekel a půlšekel}. 

\Note 5:25 {ufarsin} Aramejsky \uv{a parsin}.     

\Note 5:26 {Mene} V původní aramejštině lze tomuto výrazu rozumět jako slovesu nebo jako podstatnému jménu. \x/Daniel/ jej přečetl jako sloveso \uv{sečteno} a vyložil jako dny \x/Balsazar/ovy vlády, které Bůh přivedl ke konci.      
     
\Note 5:27 {Tekel} Rovněž sloveso nebo podstatné jméno. \x/Daniel/ slovo čte  jako sloveso \uv{zváženo} a interpretuje je ve smyslu \x/Balsazar/ovy nedostatečnosti před Bohem  (srv. <"pozn." Lk 3:17>n).     
     
\Note 5:28 {Peres} \x/Daniel/ intepretuje jako sloveso \uv{rozděleno} ve významu království, které bude \x/Balsazar/ovi odebráno a a předání Médům a Prešanům. Pokud hosté na banketu interpretovali tyto tři výrazy jako podstatná jména ve smyslu měnových hodnot ({\it mene\/} je ekvivalent 60 babylonských šekelů, {\it tekel ufarsin\/} lze chápat jako {\it šekel a půl\/}),  je pochopitelné, že jim nápis nedával žádný smysl (srv. <8>).   
     
\Note 5:28 {Médským}={Médům a Peršanům} Viz Úvod: Záměr a zvláštnosti.
     
\Note 5:29 {Balsazarova}={poručil} \x/Balsazar/, podobně jako jeho otec \x/Nabuchodonozor/, uctil \x/Daniel/a. Na rozdíl od \x/Nabuchodonozor/a však neuctil \x/Daniel/ova Boha. Čest, které se \x/Daniel/ovi a jeho přátelům opakovaně dostalo pro jejich věrnost Bohu,  zdůrazňuje důvěryhodnost \x/Daniel/a coby proroka. Nikdy svou víru nezkompromitoval; vždy zůstal věrný Bohu, byť to bylo  tváří v tvář smrti. Proto lze spolehnout i na jeho pozdější proroctví (kapitoly 7--12). 
     
\Note 5:30 {zabit} Texty Starého Blízkého Východu i řečtí historikové Hérodotos a Xenofón zaznamenávají, že Babylóňané byli zaskočeni překvapivým útokem Peršanů, zatímco se veselili a tančili.
     
\Note 5:31 {Darius}={\x/Darius/ Médský} Některé školy tvrdí, že tento a další 
(<6:1>, <6:6>, <6:9>, <6:25>, <6:28>; <9:1>; <11:1>)     
odkazy na \x/Daria/ Médského v knize \x/Daniel/ jsou historické omyly. Viz <"pozn." 6:1>n. 

\Note 6:1 {Dariovi}={\x/Darius/} Viz <"pozn." 5:31>n.      
I když je pravda, že \x/Darius/ Médský není zmíněn v dochovaných historických zdrojích mimo Bibli a že mezi \x/Balsazar/em/Nabonidem (viz <"pozn." 5:1>n) a nástupem Kýra Perského není žádný časový interval, neznamená to nutně, že kniha Daniel chybuje. \uv{\x/Darius/ Médský} je s největší pravděpodobností trůnní jméno Kýra, zakladatele Perského impéria (viz <"pozn." 28>). Je také možné, byť méně pravděpodobné, že nositelem tohoto označení byl Gubaru, gerenál, který přeběhl od  \x/Nabuchodonozor/a ke Kýrovi, vedl perské dobytí Babylóna a kterého  Kýros učinil vládcem nad územím, které Persie zabrala Babylónu. 

\Note 6:3 {duch znamenitější}={neobyčejný duch} Viz  <1:17>;  <4:6> a <5:12>.

\Note 6:5 {zákon}={Zákon jeho Boha}  \x/Daniel/ovi protivníci potvrzují nejen jeho morální integritu, ale také viditelnou a všeobecně známou zbožnost a odevzdanost Bohu Izraele. Tím je znovu připomenuto téma knihy -- Danielova svatost a důvěryhodnost.

\Note 6:7 {všickni}={všichni} Falešná implikace, že \x/Daniel/ s návrhem souhlasil. Tito úředníci se vůči Dareiovi chovali jako pokrytci; manipulovali s ním, aby dosáhli svých cílů.   

\Note 6:7 {Kdož by}={kdož by se ... modlil k jakémukoli bohu ... kromě tebe} Návrh Dareiovi mohl připadat spíše politický než náboženský; jeho zřejmý záměr je upevnit vladařovu autoritu nad nedávno dobytými územími. 

\Note 6:8 {práva}={nezměnitelné} Perský legislativní systém je první v historii, který vladařovu absolutní moc nějakým způsobem omezuje; v tomto případě nezvratností. Panovník si musel pečlivě rozmyslet, jaký zákon vydá, aby se nemohl obrátit proti němu, protože nebylo možné ho vzít zpátky.
Později král doplatil na svou důvěru svým podřízeným (<14>).
Viz též <Est 1:19>;  <Est 8:8> a <"pozn." Est 7:7>.

\Note 6:10 {když se dověděl}  \x/Daniel/ ani na okamžik nezaváhal a nedal se zviklat ve své věrnosti Bohu, přestože věděl, že ho to může stát život (srv. <"pozn." 5:29>n). 

\Note 6:10 {Jeruzalému}={směrem k Jeruzalému} Viz <1Kr 8:44>, <1Kr 8:48>; srv. <Ž 5:8> a <Ž 138:2>.       

\Note 6:10 {třikrát}={třikrát denně} Viz <Ž 55:18>.     

\Note 6:10 {klekal}={poklekal} Modlitba vestoje byla běžná (<1Pa 23:30>; <Neh 9:1>).      
                               Modlitba vkleče vyjadřuje poníženost, vhodnou zejména za okolností mimořádné vážnosti 
                               (<1Kr 8:54>; <Ezd 9:5>; viz též <Ž 95:6>; <Lk 22:41>; <Sk 7:60>; <Sk 9:40>).     
                               
\Note 6:10 {činíval}={tak činil i dříve} \x/Daniel/ova zbožnost byla veřejně známá; proto se nepřátelům hodila jako vítaná záminka intriky proti němu  (<5>).     

\Note 6:13 {synů Judských}={ze zajatých synů judských} Identifikace \x/Daniel/a etnickým původem prozrazuje antisemitské předsudky úředníků. Všeobecná známost etnické identity ukazuje, že \x/Daniel/ nezkompromitoval své dědictví ve prospěch prosperity v cizí zemi. To byla důležitá lekce pro původní čtenáře. 

\Note 6:14 {zarmoutil} \x/Dariov/i okamžitě došlo, že se stal obětí manipulativní  intriky svých  úředníků, ale byl bezmocný s tím něco udělat, protože zákon médský a perský je nezrušitelný (viz <"pozn." 6:8>).     

\Note 6:16 {Bůh tvůj}={kéž tě vysvobodí} \x/Darius/ byl nucen proto své vůli vynést rozsudek, k němuž byl zmanipulován. Jedinou naději na \x/Daniel/ovu záchranu vidí v Bohu, jehož \x/Daniel/ uctívá, o jehož všemohoucnosti neměl pochyb. % (<"">)     

\Note 6:17 {zapečetil} Mezi Asyřany, Babylóňany i Peršany byly běžné pečetní prsteny a válcové pečetě pro použití s hlínou nebo voskem. Rozlomit králem označenou pečeť znamenalo porušení zákona.

\Note 6:19 {}={nespal, postil se} \x/Darius/ byl na vrcholu zoufalství. Mohl v tom sehrát roli respekt vůči \x/Daniel/ovu Bohu, nejen obava, že přijde o nejschopnějšího úředníka (<2>).     

\Note 6:22 {anděla}  Dost možná \uv{anděl Hospodinův}  (viz <"pozn." 3:28>).     

\Note 6:23 {vytáhnouti}={vytáhnout} Vytažením z jámy \x/Darius/ neporušil zákon; ten byl naplněn o den dříve, když tam \x/Daniel/a uvrhli. 

\Note 6:24 {rozkázal} Historik Josephus Flavius líčí tuto epizodu o jeden detail podrobněji. Podle něho úředníci, když viděli, že jim nevyšel cíl sprovodit \x/Daniel/a ze světa, protestovali stížností, že lvy musel někdo předem nakrmit. Král dal tedy lvy před jejich zraky nakrmit masem a teprve pak je tam rozkázal naházet. Pro didaktický záměr autora knihy \x/Daniel/ nebyla tato okolnost důležitá a nestála mu za zmínku; podstatné pro něho bylo zachycení principu, že proti Bohu Izraele (jehož v této generaci reprezentuje \x/Daniel/) nelze bojovat. Srv. <Př 26:27>;  <Ex 14:25-28>; <Ezd 6:6-12>;  <Est 7:9> apod. 

\Note 6:26 {nařízení} 
     Dariův dekret neimplikuje automaticky, že Darius konvertoval od svého
     pohanského polyteismu k víře v Danielova Boha, o nic více, než Kýrova proklamace, že Bůh mu dal pokyn poslat Židy domů (<Ezd 1:3-4>, <Iz 44:28>, <Iz 45:4>).
     \x/Daniel/ova věrnost navzdory ohrožení života přinesla Bohu slávu po celém \x/Dariov/ě království, tedy po celém známém světě.  

\Note 6:28 {šťastně}={dařilo} Opakovaný výskyt jednoho z hlavních témat první poloviny knihy  (srv. <"pozn." 1:1-6:28>n a <3:30>n).     

\Note 6:28 {království}={za králování Dareia i za kralování Kýra} Lze též  číst jako {\em Dareia, to jest za kralování Kýra}     Verši je možno rozumět dvěma způsoby: (1) Daniel prosperoval pod vládou Gubaru  (viz <"pozn." 1>n) stejně jako pod vládou Kýra; anebo (2) \x/Daniel/ prosperoval pod vládou \x/Daria/, čili pod vládou Kýra. Ve druhém případě jsou  \x/Darius/ Médský a \x/Cýr/os Perský dvě jména jednoho a téhož panovníka  (viz <"pozn." 1>n).



\Note 7:1-12:13 {}={Danielovy vize} 
     V těchto kapitolách Daniel opouští historické vyprávění
     a zaznamenává své vize. Tyto vize navazují na předchozích šest kapitol dvěma hlavními tématy: 
     1)~Hospodin, Bůh Izraele, je svrchovaný Pán nade všemi národy a 
     2])~Daniel, nekompromisní Boží prorok, je spolehlivě důvěryhodný. Tyto kapitoly připravují exulanty na dlouhé čekání na plné znovuobnovení Izraele, jakož i na zkoušky a utrpení pod
        nadvládou cizích mocností. Jsou také Božímu lidu povzbuzením, aby se nevzdával naděje,
        že Boží království jednou přijde učinit všemu trápení konec. Daniel se dotýká čtyř
        hlavních témat: 1)~čtyři \x/zvířata/ (<7:1-28>),  
                        2)~beran a kozel (<8:1-27>),
                        3)~\uv{sedmdesát týdnů} (<9:1-27>) a 
                        4)~budoucnost Božího lidu (<10:1>--<12:13>).   

\Note 7:1-28 {}={Vize čtyř \x/zvířat/}
     Danielův sen o čtyřech šelmách zachycuje historii střídání cizích
     království, které Izrael utiskovaly, až do doby, kdy jejich pozemská vláda byla dána 
     \uv{\x/lidu svatých Nejvyššího/}

\Note 7:3 {moře} 
     Není zřejmé, zda je míněno nějaké konkrétní moře (snad Středozemní?). Nicméně
     lze mít za to, že moře symbolizuje chaotický neklid, charakteristický pro hříšné národy,
     okupující Izrael. Viz interpretaci ve  <17> a v <Iz 17:12-13> a <Iz 57:20>. 

\Note 7:3 {čtyři šelmy}
    Čtyři \x/zvířata/ reprezentují čtyři království (<17> a <23>).
    Spojitost  s \x/Nabuchodonozor/ovou vizí sochy v kapitole 2 je zřejmá. Pro jejich identifikaci viz náčrt      {\it Danielovy vize\/} na str.~\pgref[danielovavize]. 

\Note 7:4 {lvu}={lev ... orličí křídla} 
     Lev s orlími křídly symbolizuje Babylón (srv.~<Jr 50:44>, <Ez 17:3>).
     Okřídlení lvi byli běžné babylónské artefakty, často umisťované u vchodů významných veřejných budov.      {\bf \x/oškubána/ ... \x/lidské srdce/} Snad odkaz na \x/Nabuchodonozor/ovu
     proměnu a navrácení do lidské společnosti po sedmiletém ponížení nepříčetností
     (<4:31-34>).

\Note 7:5 {nedvědu}={medvěd ... k jedné straně ... tři žebra} 
     Médo-perské království je symbolizováno šelmou s nenasytnou žravostí. Vztyčená
     strana může reprezentovat nadřazenou pozici Persie. Tři žebra pravděpodobně znamenají
     vítězství Persie nad Lydií (546 př.Kr.), Babylónem (539 př.Kr.) a Egyptem (525 př.Kr.).
     Viz <"poznámku" 8:3>n.

\Note 7:6 {pardovi}={\x/pard/ ... čtyři ptačí křídla ... čtyřhlavé}
     Řecko je symbolizováno \x/pard/em, proslulým svou rychlostí.
     Alexandr Veliký (356--323 př.Kr.) dobyl Persii velmi rapidně.
     Střetl se s Peršany ve třech velkých bitvách:
     1) bitvou u řeky Gráníkos (334 př.Kr.) získal vstup do Malé Asie; 
     2) bitva u Issu (333 př.Kr.) mu umožnila okupovat Sýrii, Kenaán a Egypt; 
     3) v~bitvě u~Gaugamél porazil perskou armádu definitivně a otevřel si cestu do Indie.
        Viz též <Da 8:5-8>. Krátce po jeho předčasné smrti (ve věku 33 let) se říše, kterou
        vytvořil, rozpadla na čtyři části: v Makedonii vládl Kassandros, v Thrákii a Malé Asii
        Lýsimachos, v Sýrii Seleukos a v Egyptě Ptolemaios.

\Note 7:7 {šelma čtvrtá}
     Historie nás učí, že tato neidentifikované zvíře je Řím --- impérium, které postupně
     asimilovalo různé části rozděleného řeckého království. 
     {\bf deset rohů} Znamenají deset římských králů (viz v. <24>). Není zřejmé, zda následují po      sobě nebo vládnou současně. Pro spekulativní domněnku, že mají znamenat druhou fázi čtvrtého    království, \uv{oživenou říši římskou} posledních dnů, však jednoznačně přesvědčivý textový      důkaz neexistuje. 

\Note 7:8 {roh poslední}={roh poslední ... tři z dřívějších rohů byly před ním vyvráceny}
     Deset rohů časově předchází \uv{malému,} který vyvrátí tři z~nich. Je to další fáze
     čtvrtého království. Mnozí mají za to, že malý roh symbolizuje vzestup antikrista 
     <2Te 2:3-4>. Pokud je tomu tak, pak je toto první zmínka o antikristu v Písmu.  

\Note 7:8 {oči podobné očím lidským}{oči podobné očím lidským ... a ústa}
     Metafora naznačuje, že roh reprezentuje spíše člověka než království. 
     
\Note 7:9 {Starý dnů} Jediný výskyt v Bibli je v této kapitole (srv. <13>, <22>). Podobný výraz se objevuje v ugaritských textech k označení velkého Boha  {\em El}. Zde je použit jako označení pro Boha, který zasedl k soudu, a implikuje, že je věčný a  panuje od pradávna.

\Note 7:9 {roucho}={roucho ... vlasy} Ačkoliv se Bůh Danielovi zjevil v neskutečné slávě, přesto to bylo v podobě rozpoznatelné jako lidská.

\Note 7:9 {trůn}={trůn ... kola} Vyobrazení Božího trůnu koreluje s vizí proroka Ezechiele (<Ez 1:15-28>).
      Nebeský trůn je zobrazen s koly (podobně jako v památkách jiných národů z oné doby) --- jako královský válečný vůz. Podobný motiv se skrývá za ohnivým sloupem, který vedl Izrael běhemhttps://www.overleaf.com/project/61384698a6c2d94bfabb15d3 Exodu (<Ex 13:21-22>).

\Note 7:10 {knihy}={otevřeny knihy} Viz  <12:1> (viz též <Ex 32:32>; <Ž 149:9>; <Iz 4:3>; <Iz 65:6>;  <Mal 3:16>; <Lk 10:20>; <Zj 5:1-5>; <Zj 6:12-16> a <Zj 20:12>).

\Note 7:11-12 {}  Určitá lhůta je ponechána předchozím královstvím, jejichž obyvatelé i se svými zvyklostmi byli absorbovány následujícími říšemi. V kontrastu s tím je vyjádřen důraz na totální zkázu čtvrtého království. Srv. <"pozn." 3>

\Note 7:13 {Synu člověka} Tento termín může znamenat jednoduše \uv{člověk}. Hebrejský
   ekvivalent tohoto aramejského výrazu je v <Da 8:17>  použit pro Daniele, stejně jako pro jeho současníka    Ezechiele v <Ez 2:1>, <Ez 2:3>, <Ez 2:6>. 
   Daniel je jedním z prvních (ne\discretionary{-}{-}{-}li
   vůbec první), kdo používá toto spojení.  Pozdější židovská mezizákonní apokalyptická
   literatura navazuje na tuto pasáž a vykresluje \uv{syna člověka} jako nadpřirozenou bytost,
   přinášející nebeskou moc na Zem. Daniel viděl {\em podobného Synu člověka}, tedy někoho srovnatelného s člověkem, a přesto výrazně odlišného (<14>). V evangeliích  je výraz \uv{Syn člověka} používán ve vztahu ke Kristu (69 výskytů v synoptických evangeliích a 12 v Janově). Ježíš sám sebe nejčastěji označoval právě tímto titulem.

\Note 7:13 {s oblaky} Jinde ve SZ je to jedině Hospodin, o němž je řečeno, že se objevuje na oblacích  (<Ž 104:3>; <Iz 19:1>). Podobný Synu člověka pochází z nebes, sestupuje z Božího rozhodnutí; je totožný se skálou, vytrženou z hory, avšak nikoliv lidskou rukou  (<2:45>; viz <"pozn." 14>n).

\Note 7:14 {panství}={Byla mu dána vláda} Bůh mu svěřuje \uv{místodržitelství} nade všemi národy. Tím je naplněna role skály, vytržené z hory (<2:44-45>).

\Note 7:14 {všickni lidé}={všichni lidé ... nebude zničeno} \uv{Syn člověka}, kterého \x/Daniel/ viděl, je veliký syn Davidův, Mesiáš. \x/Izaiáš/ také líčí jeho království jako nikdy nekončící  (<Iz 9:7>). Ježíš toto mesiánské spojení potvrzuje narážkou na právě tuto pasáž. Z toho důvodu byl náboženskými vůdci své doby nařčen z rouhání (<Mt 26:64-65>; <Mk 14:62-64>). Kdo slouží jemu, slouží Bohu.

\Note 7:15 {zhrozil}={děsilo mne} \x/Daniel/ byl zděšen tím, co viděl (<28>), přesto se dožaduje vysvětlení nesrozumitelných jevů (<16,19>). Je to od něho projev odvahy: touhy znát pravdu navzdory tušení, že odhalená záhada bude ještě děsivější (<21>). Milovat pravdu, ať je jakákoliv; hledat ji, ať je kdekoliv, je známka vysokého stupně zralosti. Zavírat před ní oči jen proto, že je nepříjemná, je dětinské, nedospělé, a v důsledku vede k paralyzující neschopnosti se na těžké časy připravit. 

\Note 7:18 {království svatých}={svatí}  Viz <21-22>, <25> a <27>. Nebudou to andělé, nýbrž věrní věřící, komu bude svěřeno království (srv. <1Kor 6:1-11>; <2Tm 2:12>; <Zj 22:5>). 

\Note 7:18 {výsostí}={Nejvyššího} Mezi \uv{Synem člověka} coby Králem (<13-14>) a \uv{svatými Nejvyššího} coby těmi, kdo budou mít podíl na jeho království, je úzká spojitost (<22>, <27>).

\Note 7:18 {na věky}  Viz <6:26>;  <7:14> a <"jeho pozn." 7:14>n.

\Note 7:21 {válku}={válku proti svatým ... přemáhá} Než se Boží věrní ujmou věčné vlády nad světem  (<18>, <27> <Mt 5:5>), budou vystaveni pronásledování, které prověří jejich víru  (<Zj 13:7-10>; <Zj 14:12>).

\Note 7:22 {Starý dnů} Přestože malý roh  (<8>) bude mít nějaký čas k rouhání (<25>) a pronásledování svatých (<21>), nakonec podlehne Božímu soudu (<Za 14:1-4>; <Zj 19:2>).

\Note 7:22 {království} Boží zásahy do historie směřují k tomu, co NZ nazývá \uv{královstvím Božím}  (viz čl. <"Království Boží" Mt 4>a).

\Note 7:25 {času}={času a časů a půl času} Slovo pro \uv{čas} je stejné jako v <4:13> a <4:20>, kde může znamenat jeden rok (viz <"pozn." 4:13> a <Zj 12:14>). Nejjistější zřejmě bude považovat toto časové určení za dobu, kterou  Boží intervence zkrátí ve prospěch svého lidu (<Mt 24:22>).
\dopsat hebr.

\Note 7:26 {soud} Nebeský soud (viz <10>). Nebesa jsou stvořené místo, Boží soudní síň, nikoliv Boží \uv{bydliště} (<1Kr 8:27>). 

\Note 7:27 {lidu svatých} Viz  <"pozn." 18>n.

 \Note 7:28 {Daniel}={vyděšen} Myšlenky o Izraeli, který nepřestal hřešit ani v zajetí a vykoledoval si tak sedminásobný trest pod vládou cizích království   (srv. <"pozn." 9:11>),
\x/Daniel/a deptaly (<8:27>), přestože věděl, že ultimátní závěr dějin vyústí ve vítěznou vládu Božích věrných  (srv. <"pozn." 18>). 

 \Note 7:28 {v srdci} \x/Daniel/ zmiňuje tento detail, aby zdůraznil, že si nelibuje v takové představě apokalyptických vizí.  Ani s autoritou nad pohany u Babylonského i Perského královského dvora nemohl být osočen ze zrady věrnosti Božímu lidu. O budoucích událostech hovořil zkroušeně a s lítostí. Nemnul si ruce nad ztrestanými nevěrnými ani nad zahynuvšími nepřáteli. Tímto postojem srdce je příkladem všem.
 
\Note 8:1-12:23 {} \x/Daniel/ se v poledních 5 kapitolách vrací k hebrejštině. Text <2:4-7:28>
je v Aramejštině (viz <"pozn." 2:4>n).
   
\Note 8:1 {Léta třetího kralování Balsazara}
     Viz <"pozn." 5:1>n. Není jednoznačně jisté, zda \x/Balsazar/ova
     spoluvláda s Nabonidem začala zároveň s Nabonidovým nástupem (556 př.Kr.) nebo o něco
     později. Ať tak či onak, události páté a osmé kapitoly je nutno chronologicky umístit
     mezi události kapitol 4 a 5. 

\Note 8:1 {vidění}={ukázalo se vidění} \x/Daniel/ zakouší podobnou \uv{cestovní} vizi jako prorok \x/Ezechiel/ (<Ez 3:12-15>).

\Note 8:2 {Susan} V \x/Daniel/ově době byl \x/Susan/ hlavním městem území jménem \x/Elam/,  
asi 370km východně od Babylóna. Není zřejmé, zda tehdy byla  tato země  nezávislá nebo spojena s Babylonem či Médeou. Později se však stal diplomatickým a administrativním hlavním městem Perského impéria (srv. <Est 1:2>; <Neh 1:1>).
\dopsat wdef Susan a Elam

%\note 8:2 {}={} Kanál, spojující řeky Dez a Karcheh, obtékající  \x/Susan/, dnešní Shush, každá z jedné strany, a ústící do Perského zálivu. 
%to je špatně Karkheh ústí do

\Note 8:3 {dva rohy} Verš <20> identifikuje berana se dvěma rohy jako krále Médského a Perského království.

\Note 8:3 {vyšší}={vyšší ... později} Symbolismus pomáhá objasnit Médo-Perská historie. Médové se stali mocnými a nezávislými na Asýrii po r. 631 př.Kr. Peršané začali jako nevýznamná část médského království, ale chopili se nadvlády, když si \x/Cýr/os (559--530 př.Kr.) z Anšanu (v \x/Elam/u) podmanil Médeu. Tehdy připojil ke svým titulům i \uv{Král Médů}. Proto oba rohy jsou vysoké, ale vyšší reprezentuje mocnější Persii, a vyrostl později, protože Persie se dostala k moci později než Médea.

\Note 8:4 {trkal}={na západ, na sever a na jih} \x/Cýr/os nejprve dobyl Malou Asii, poté severní i jižní Mezopotámii. Panovníci, následující po něm, rozšířili Médo-Perskou vládu daleko na Východ. 

\Note 8:4 {veliké}={stal se velikým} Perské impérium se stalo rozlehlejším a mocnějším než kterákoliv předchozí říše v dějinách Blízkého Východu.

\Note 8:5 {kozel}  <"Verš" 21> identifikuje kozla jako Řecko a roh mezi očima jako jeho prvního krále. Symbolismus výstižně zachycuje rozmach Řeckého impéria pod vedením Alexandra Velikého (356--323 př.Kr.). 

\Note 8:5 {}={sotva se země dotýkal} Obraz postihuje rychlost, s jakou Alexandr dobýval cílová území   (viz <"pozn." 7:6>n). Mocnou Persii (viz <"pozn."4>n) dokázal dobýt během pouhých tří let.

\Note 8:8 {velikým}={nesmírně velikým} Alexandrovo impérium brzy překonalo perskou říši i rozlohou. V roce 327 př.Kr. pokrývalo území dnešního Afghánistánu a později sahalo až do údolí Indus.

\Note 8:8 {roh}={roh se zlomil}  Když Alexandrova armáda odmítla pokračovat dále na východ, vrátil se do Babylóna, kde zemřel ve věku třiatřiceti let. 

\Note 8:8 {čtyři místo něho, na čtyři strany světa}  <"Verš" 22> napovídá, že tyto rohy symbolizují čtyři království, na které se rozdělí Alexandrova říše. Historické záznamy dokládají, že po určité době vnitřních tahanic  se čtyři Alexandrovi generálové podělili o bývalé řecké impérium. Viz <"pozn." 7:6>n.

\Note 8:9 {roh}={nepatrný roh}  <"Verš" 23> naznačuje, že rok symbolizuje zlého krále, který povstane z jednoho ze čtyř řeckých království po delším čase (\uv{koncem jejich kralování}). 
Líčení skutků tohoto krále (<"vv." 9-14>, <23-25>) identifikuje jako Antiocha IV. Epifana z dynastie
Seleukovců, vládnoucího mezi 175 a 164 př.Kr. Nezaměnit s malým rohem ze <7:8>, který povstane až za římského období, po řeckém. 

\Note 8:9 {k zemi Judské} Dosl. k Nádherné (Slavné, Ctnostné, Ozdobné) --- rozumí se (implicitně) Zemi Zaslíbené. Antiochos IV. Epifanés ve snaze helenizovat svá území zakázal židovské náboženství včetně chrámových obřadů, čtení Tóry, zachovávání Sabatu i obřízku; Židé byli dokonce nuceni účastnit se helénským modloslužeb. K aktivnímu odporu se odhodlali po hrdinském činu starého kněze Matatiáše, který neuposlechl rozkaz obětovat modle a namísto toho zabil královského úředníka i Žida, ochotného oběť vykonat. (To nám mj. připomíná, že pronásledování bezbožnou vládou si lidé vykoledují ani ne tak tím, co by dělali, jako spíše tím, co udělat odmítnou.)
Tím odstartoval vzpouru Makabejských, vylíčenou v deuterokanonické knize 2~Makabejská a v jejím románovém zpracování Howarda Fasta {\em Moji stateční bratři.}
Svátek Chanuka Židé dodnes slaví na památku vítězství tohoto povstání, jemuž se roku 164 př.Kr. skutečně podařilo osvobodit Jeruzalém ode všech helenizátorů a znovu posvětit Chrám.
\dopsat reference a přehled svátků, sladit s Úvodem

\Note 8:10 {některé}={z hvězd} Hvězdy symbolizují Boží lid; srv. <12:3>; <Jr 33:22>; <Gn 12:3>;  <Gn 15:5> a/nebo nebeskou armádu andělů %(<Ex 12:41>)
(<Iz 14:13>; viz též <2Mak 9:10>). % 2 Makabejská, sz apokryf
Antiochovy mince ho zobrazují s hvězdou nad hlavou. Útok na lid Boží znamená útok proti nebi.

\Note 8:10 {svrhl}={strhl na zem ... a pošlapal je} Symbolické zachycení krutého pronásledování Božího lidu za Antiocha IV. Epifana. Viz <"pozn." 8:9>n a Úvod: Záměr a zvláštnosti. Srv. <11:21-35>; <1Mak 1:10-64>. %1 Makabejská, sz apokryf 

\Note 8:11 {knížeti} Zde označení pro Boha, Pána zástupů (hebr. \uv{armád}\dopsat hebr.; srv <Iz 6:3>). Ve verši <25> je formulace \uv{kníže knížat}. Antiochus IV. si osvojil přídomek Epifanés (\uv{Bůh zjevený}) a považoval sám sebe za manifestaci Dia, hlavního boha řeckého panteonu. 

\Note 8:11 {zastavena}={zrušil každodenní oběť} Viz (<"vv." 12-13>; <11:31> a <"pozn." 8:9>.)

\Note 8:11 {svatyně}={zpustošil svatyni} Antiochos IV. nejen vstoupil do Nejsvětější Svatyně jeruzalémského chrámu a vydrancoval z ní zlaté a stříbrné nádoby, ale dokonce na Hospodinově oltáři na chrámovém nádvoří vztyčil oltář Diův a obětoval na něm prase (viz <"pozn." 11:31>).

\Note 8:12 {vojsko to vydáno v převrácenost proti ustavičné oběti} Boží lid je vydán v moc rohu, který začal jako malý (<"v." 9>), tedy Antiocha IV. Zákaz chrámové bohoslužby lze od bezbožného vladaře očekávat.

\Note 8:12 {šťastně mu se dařilo} Úspěch rohu, který začal jako malý (<"v." 9>, Antiocha IV.)
obnášel i zničení kopií hebrejských Písem  (<1Mak 1:56-57>). 

\Note 8:14 {}={2300 večerů a jiter} Fráze \uv{večerů a jiter} se v celém SZ objevuje pouze zde a ve <"verši" 26>. Lze ji chápat jako odkaz na večerní a ranní oběti (srv. <Ex 29:38-42>); v takovém případě by to znamenalo dobu 1150 dní. Jiné pohledy považují frázi za prosté označení 2300 dní. Vzhledem k tomu, že počátek pronásledování Antiochem IV. lze spojit s jakýmkoliv počtem incidentů již od 171 př.Kr., je těžké jednoznačně rozhodnout, ke kterému chápání fráze se přiklonit.
Číslo 23 může symbolizovat pevně stanovené období, podobně jak je tomu v mimobiblické apokalyptické literatuře.

\Note 8:14 {}={pak bude svatyně očištěna}  Chrám byl vyčištěn a znovu zasvěcen pod vedením Judy Makabejského 25. prosince 165 př.Kr. (viz <"pozn." 11:34>n; srv,. <Za 9:13-17>).

\Note 8:16 {Gabrieli} Tento anděl je v Písmu zmíněn jménem čtyřikrát (<9:21>; <Lk 1:11,19,26>). Jméno znamená \uv{Bůh je moje síla}.

\Note 8:17 {synu}={synu člověka} Viz <"pozn." 7:13>n. \uv{Silný Boží} (viz <"pozn." 16>) promlouvá k vznešenému smrtelníkovi. 

\Note 8:17 {času}={vidění času konce} Viz též <"v." 19>. Vazba nemusí nutně znamenat úplný závěr historie; Ve verších <11:27> a <11:35> je podobné spojení umístěno v kontextu, který pravděpodobně odkazuje ke konci pronásledování pod Antiochem IV.

\Note 8:19 {hněvu}={koncem hněvu} \uv{Čas hněvu} může znamenat období Božího hněvu, namířeného proti   Izraeli, vykonaného nadvládou Babylóňanů, Peršanů a Řeků. 

\Note 8:20 {Skopec} Viz <"pozn." 3>n a <4>n.

\Note 8:21 {Kozel}={kozel ... roh} Viz <"pozn." 5>n a <6>n.

\Note 8:22 {čtyři} Viz  <"pozn." 8>.

\Note 8:23-25 {} Viz  <"pozn." 9-14>. Někteří vykladači spatřují v popisu rohu v této kapitole (<"v." 8>)
obraz Antikrista v podobě Antiocha IV. coby typu mocného oponenta Božího lidu v budoucnosti.  
\dopsat typologii někam do článku

\Note 8:25 {mnohé} Tj. věrné Židy.

\Note 8:25 {knížeti}={proti knížeti knížat} Odkaz na Boha.

\Note 8:25 {}={bez přispění lidské ruky bude zničen} Antiochus IV. nebyl zavražděn ani nepadl v bitvě. Jeho smrt v roce 164 př.Kr. nastala jako důsledek nervové choroby. Záznamy jeho smrti jsou v <1Mak 6:1-16> a <2Mak 9:1-28>.

\wdef 8:26 {zavři to vidění} %BKR
    {to vidění ukryj}  %PSP
    {to vidění uzavři}  %CSP
    {podrž to vidění v tajnosti}  %CEP
    {Ty ho však zachovej v tajnosti}  %B21
    {Ty je však zapečeť}  %SNC
\Note 8:26 {zavři to vidění} Pečeť byla používána k jednomu ze dvou účelů: (1) Buďto jako certifikát autentičnosti (\uv{Toto je opravdu královo nařízení!}) anebo (2) k zabezpečení  důvěrného dokumentu. Tento druhý význam nejlépe odpovídá kontextu (srv. <"pozn." 6:18>n).


%     BKR                       PSP                         CSP                                 CEP                                         B21                                    SNC
\wdef 8:26 {nebo jest mnohých dnů}    {neboť je na mnohé dni}      {je totiž na velmi dlouho}      {neboť se uskuteční za mnoho dnů}      {protože se týká vzdálené budoucnosti}      {protože se týká vzdálené budoucnosti}    
\Note 8:26 {nebo jest mnohých dnů} Dobyvačné tažení Alexandra Velikého (333--323 př.Kr.) začalo téměř dvě století po \x/Daniel/ových vizích (cca. 500 př.Kr.); Antiochus IV. vládl století a půl po Alexandrovi (171--164 př.Kr.). 

%\wdef {} %BKR
%    {}   %PSP
%    {}   %CSP
%    {}   %CEP
%    {}   %B21
%    {}   %SNC
\Note 9:1 {Daria} Viz <"pozn." 5:31>n a <6:1>n. První rok \x/Dariov/y vlády byl 539 př. Kr.

%\wdef {Asverošova} {Achašvéróšova} {Achašvéróšova} {Achašveróšova} {Ahasverova} {Achašvéroše}
%\Note 9:1 {Asverošova} Nezaměnit s králem z <Est 1:1>.  

\Note 9:1 {Asver} Nezaměnit s králem z <Est 1:1>.  Hebrejská odvozenina původního perského Xerxes (\uv{král všech mužů} nebo \uv{hrdina mezi králi}). Mohl to být královský titul, a ne vlastní jméno. 

\Note 9:2 {Jeremiáš}={Jeremiáš ... sedmdesátého léta} Viz <Jr 25:11> a <Jr 29:10>.
        Sedmdesát let lze považovat buďto za zaokrouhlený věk jednoho lidského života, anebo za přesné časové určení. Někteří toto období datují od 586 př.Kr. (zničení jeruzalémského Chrámu \x/Nabuchodonozor/em) do 515 př.Kr. kdy byl dokončena rekonstrukce Chrámu 
        \x/Zorobábel/em (srv. <"pozn." 26>n). Jiní považují za začátek 70 let první rok \x/Daniel/ova zajetí (604 př.Kr., srv. <"pozn." 1:1>n). 
        Není pochyb, že \x/Daniel/ byl obeznámen s \x/Izaiáš/ovým proroctvím o propuštění Izraele z otroctví pohanským vladařem jménem  \x/Cýr/os (<Iz 44:28>; <Iz 45:1-13>).
        Viz čl. <"\x/Cýr/os je můj pastýř" Iz 44>a.
        Daniel považuje \x/Cýr/ovo propuštění za rok naplnění \x/Jeremiáš/ova proroctví, podobně jako autor kniha Paralipomenon \dopsat [větvení pro Letopisů]
        (<2Pa 36:22>). V literatuře starého Blízkého Východu bylo 70 let standardní dobou trestu božstva nad neposlušným lidem. Tento čas mohl být prodloužen nebo zkrácen podle reakce lidí (viz <Jr 18:7-10>; viz též <"Úvod k prorockým knihám">i). Proto  určitá flexibilita  v aplikaci číslice 70 různými biblickými autory není překvapivá.
        
        
\Note 9:3 {modlitbou}={modlitbou, v postu, žíni a popelu} \x/Daniel/ byl zděšen, protože věděl, že Izrael byl 70 let v zajetí za trest pro své hříchy, ale ani po 70 letech se od svých hříchů neodvrátil. Srv. <"pozn." 11>n.

\Note 9:4-19 {} \x/Daniel/ova modlitba vyrůstá z chápání vztahu s Bohem jako smluvního (požehnání za poslušnost a prokletí za neposlušnost; viz zejména <"vv." 5,7,11-12>; <Lv 26:14-45>; <Dt 28:15-68>; <Dt 30:1-5>). Podobnou modlitbu předkládá Boží lid v <Neh 9:6-38>. Modlitbu tvoří čtyři části: (1) uctívání (<"v." 4>);  (2) doznání hříchů (<"vv." 5-11a>); (3) uznání Boží spravedlnosti a zaslouženosti trestu (<"vv." 11b-14>) a (4) prosba o Boží slitování, opřená o Jeho jméno, království a vůli (<"vv." 15-19>). Modlitba stojí na Božích zaslíbeních (<"v." 2>), je pronesena zkroušeným duchem (<"v." 3>) a ukazuje tak vzor náležitých prvků účinné modlitby.

\Note 9:11 {Mojžíše} \x/Daniel/ovi bylo jasné, že když si Izrael nevzal ponaučení ze sedmdesátiletého otroctví, čeká ho sedminásobný trest, jak píše Mojžíš (<Lv 26:18, 21, 24, 28>). Národ bude sloužit cizincům 490 dalších let. Historie nás učí, že tato lhůta byla Boží milostí o něco zkrácena (srv. <Jr 18:8>).
\dopsat


\wdef 9:24 {Sedmdesáte téhodnů} {sedmdesát sedmic} {sedmdesát sedmiletí} {Sedmdesát týdnů let} {Sedmdesát týdnů} {Sedmdesát týdnů}
\Note 9:21 {Gabriel} Viz <"pozn." 8:16>n. 

%\wdef 9:24 {Sedmdesáte téhodnů} {sedmdesát sedmic} {sedmdesát sedmiletí} {Sedmdesát týdnů let} {Sedmdesát týdnů} {Sedmdesát týdnů}
\Note 9:24 {Sedmdesáte téhodnů} Sedmdesát \uv{týdnů} let je 490 let (70$\times$7).

\vdef %9:25  
    {téhodnů sedm}   %BKR
    {sedm sedmic}   %PSP
    {sedm sedmiletí}   %CSP
    {sedm týdnů}   %CEP
    {sedmero týdnů}   %B21
    {sedm a potom šedesát dva týdnů}   %SNC
\vdef %9:26  
    {šedesáti a dvou}   %BKR
    {šedesáti a dvou sedmicích}   %PSP
    {šedesáti a dvou sedmiletích}   %CSP
    {šedesáti dvou týdnů}   %CEP
    {dvaašedesát týdnů}   %B21
    {šedesáti dvou týdnech}   %SNC
\vdef %9:27  
    {v téhodni posledním}   %BKR
    {jednu sedmici}   %PSP
    {jednoho sedmiletí}   %CSP
    {v jednom týdnu}   %CEP
    {v týdnu posledním}   %B21
    {v posledním týdnu}   %SNC
% Nefunguje, nepřepíná, ani když \Note má jen 9:25 nebo 9:26. Musím přepsat poznámku, aby nebyla závislá na překladech
%když to není v {...}, \x/.../ nešlape
\Note 9:25-27 {} \uv{Sedmdesát týdnů} let je rozděleno do tří menších celků o délce: 49 let (\uv{\x/téhodnů sedm/}, <"v." 25>); 434 let (\uv{\x/šedesáti a dvou/}, <"v." 26>) a 7 let (\uv{\x/v téhodni posledním/}, <"v." 27>).
%\Note 9:25-27 {} \uv{Sedmdesát týdnů} let je rozděleno do tří menších celků o délce: 49 let (\uv{sedm týdnů}, <"v." 25>); 434 let (\uv{šedesát dva týdnů}, <"v." 26>) a 7 let (\uv{jeden (poslední) týden}, <"v." 27>). 
Vykladači se rozcházejí v otázce, zda tyto části tvoří navazující sekvenci, nebo zda jsou odděleny nějakými časovými intervaly. Pokusy srovnat chronologii příliš přesně selhávají, protože počty let byly zamýšleny jako zaokrouhlená čísla, reprezentující období. Přestože \x/Daniel/ovy kalkulace nemohou být brány jako precizní, základní vzorec je zřetelný bez nebezpečí spekulací:
Po pokynu obnovit Jeruzalém (<"v." 25>) následuje \uv{sedm týdnů}, tedy 49 let, během nichž byla rekonstrukce Jeruzaléma dokončena (viz knihy \x/Ezdráš/ a \x/Nehemiáš/). Poté následovalo \uv{šedesát dva týdnů}, tedy 434 let (<"v." 25>), během nichž bude zabit Mesiáš (<"v." 26>; viz <"pozn." 26>n).  Poslední \uv{jeden týden} (<"v." 27>) bude naplněn v blízkosti doby Kristovy pozemské služby. Pokud by zkáza verše <27> měla znamenat destrukci Chrámu Římany v roce 70 po Kr., pak součet \uv{týdnů} nevychází bez mezer mezi nimi. To by se dalo přičíst Boží milosti, která původní lhůtu zkrátila (viz <"pozn." 11>n).

\Note 9:26 {Mesiáš}={Mesiáš ... svatyni} Podle tohoto proroctví bude Mesiáš zabit před zničením jeruzalémského chrámu.
Druhý chrám, který po návratu z babylónského zajetí obnovil \x/Zorobábel/ 
(<Ezd 5:2>; <Ezd 6:15>; <Za 1:12>; <Za 4:9>) a který v roce 20 př. Kr. ještě rozšířil Herodes ve snaze zavděčit se Židům, byl srovnán se zemí Římany v roce 70 po Kr. a od té doby nebyl nikdy obnoven. Na jeho místě dnes stojí mešita; není žádá naděje, že by mohl být znovu zbudován, aby před jeho dalším zničením mohl být zabit Mesiáš. Příchod Mesiáše musíme hledat mezi 20 př.Kr. a 70 po Kr.







\Note 10:12 {přiložil srdce své, abys rozuměl} Danielova moudrost nebyla náhodná; usiloval o ni celým srdcem (srv. <Jr 29:16>). 

\Note 10:13 {kníže království Perského} Teritoriální démon, okupující Persii. V anglických Biblích nese označení ``Prince of Persia.'' Populární videohra a film téhož jména jsou typickou ukázkou zpohanštění kultury, kdysi křesťanské; lze je považovat za oslavu tohoto prastarého démona.

\Note 10:13 {jedenmecítma dnů}={Dvacet jedna dnů} Když se modlíme, uvádíme do pohybu souvislosti v neviditelném světě, o jakých většinou nemáme ani nejmenší tušení.





\Note 11:31 {obět}={odejmou oběť ... postaví ohavnost zpuštění} 
Znesvěcení Chrámu v prosinci 168 př.Kr. Antiochem IV. (srv. <1Mak 1:54, 59>; <2Mak 6:2>); viz <"pozn." 8:11>n; <9:27>n; <12:11>n.

\Note 11:34 {malou pomoc}  Snad odkaz na starého kněze Matatiáše a jeho pět synů (Jan, Šimon, Juda, Eleazar a Jonatan), kteří vedli partyzánskou válku proti Antiochovi IV. Matatiáš zemřel v roce 166 př.Kr. Jeho synové pokračovali v boji    a prosluli jako \uv{Makabejští}. Pod vedením Judy Makabejského dosáhlí vítězství v roce 165 př.Kr., kdy chrám byl vyčištěn a denní oběti obnoveny (srv. <1Mak 4:36-39>).

\endinput

\wdef 9:1  
    {}   %BKR
    {}   %PSP
    {}   %CSP
    {}   %CEP
    {}   %B21
    {}   %SNC
\Note 9:1 {}={}  \x// (<"">) (<"">) 




