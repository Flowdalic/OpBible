\CommentedBook {Da}

%\font\hebrew=[SILEOT] at10pt
%\input hebrajka

%\hyphenation{A-zar-já-šo-vi} %asi není třeba, to byly u 1:7 špatně napsané poznámky ve \ww

%\fontfam [Shlomo.ttf]  
% \dagesh\waw\resh\shwa\mem\kamac\shin


\Note 1:1-6:28 {}={Vyprávění}  První část knihy vyzdvihuje jednak Boží absolutní svrchovanost nad královstvími tohoto světa, jedna upřímnou odevzdanou věrnost Bohu, kterou Daniel a jeho přátelé projevovali. 
Daniel chtěl svým čtenářům vštípit přesvědčení, že přestože Boží lid někdy trpívá pronásledování, králové a království povstávají a hroutí se podle Božích záměrů. Daniel také učí, že Bůh se hojně odmění těm, kdo jemu, Danielovi, věnují pozornost coby Božímu mluvčímu. Tento materiál je rozdělen do šesti navzájem nezávislých vyprávění, každé v jedné kapitole a každé obsahující nějaký zázrak:
Uchování rituální čistoty Danielem a jeho přáteli (<1:1-21>); \x/Nabuchodonozor/ův sen (<2:1-49>); záchrana z ohnivé pece (<3:1-30>); \x/Nabuchodonozor/ův druhý sen (<4:1-37>); soud nad \x/Baltazar/em (<5:1-31>) a záchrana Daniela ze lví jámy (<6:1-28>).

\Note 1:1-21 {}={Uchování rituální čistoty}  Prorok uvozuje kontext své knihy vyprávěním osobní historie (své i svých přátel) zajetí, vzdělání, věrnosti Bohu a služby králi \x/Nabuchodonozor/ovi.

\ww {Joakima}  %BKR
    {Jehójákíma} %PSP
    {Jójakíma}  %CSP
    {Jójakíma} %CEP
    {Joakima}  %B21 
    {Jójakíma}  %SNC
\Note 1:1 {Joakim}  Tj. 605 př.Kr.; téhož roku, kdy \x/Nabuchodonozor/ porazil  asyrsko-egyptskou koalici v bitvě u Karchemiše (<Jr 46:2>) a zahájil tak mezinárodní vzestup Babylónu k moci. Po bitvě u Karchemiše \x/Nabuchodonozor/ zaútočil na \x/Joakim/a (<2Kr 24:1-2>; <2Pa 36:5-7>) a zajal Daniela a jeho přátele. 
To byla první ze tří \x/Nabuchodonozor/ových  invazí do Judska. Druhá nastala roku 597 př.Kr. (<2Kr 24:10-14>) a třetí 587 př.Kr. (<2Kr 25:1-24>). Zdánlivou diskrepanci mezi <Da 1:1> a <Jr 25:1> a <Jr 46:2> (kde \x/Jeremiáš/ umisťuje \x/Nabuchodonozor/ův útok na \x/Joakim/a do \x/Joakim/ova čtvrtého roku místo třetího) lze objasnit rozdílem mezi babylónským a židovským systémem chronologie. V babylónském systému, který používá Daniel, byl první rok vlády panovníka považován za \uv{korunovační rok} a vláda samotná se počítala až od prvního dne měsíce Nisan následujícího roku.

\ww {Nabuchodonozor král Babylonský}  %BKR
    {Nevúchadneccar, král Bávelu} %PSP
    {babylonský král Nebúkadnesar}  %CSP
    {Nebúkadnesar, babylónský král} %CEP
    {babylonský král Nabukadnezar}  %B21 
    {babylónský král Nebúkadnesar}  %SNC
\Note 1:1 {Nabuchodonozor král Babylonský}  \x/Nabuchodonozor/ přivedl Babylóňany k vítězství u Karchemiše v roce 605 př.Kr. coby korunní princ a velitel armády. Krátce po tomto vítězství se ujal babylónského trůnu po smrti svého otce Nabopolasara (626--605 př.Kr.). \x/Nabuchodonozor/ova vláda (605--562 př.Kr.) tvoří většinu historického pozadí  biblických knih \x/Jeremiáš/, \x/Ezechiel/ a \x/Daniel/.     

\ww {vydal Pán}  %BKR
    {Pán vydal} %PSP
    {Hospodin}={Hospodin ... vydal}  %CSP
    {Hospodin mu vydal}={Hospodin ... vydal} %CEP
    {Hospodin mu vydal}={Hospodin ... vydal}  %B21 
    {se stal jeho vazalem}  %SNC
\Note 1:2 {} Porážka Izraele Babylónem není vysvětlitelná jen pouhou vojenskou a politickou analýzou oné doby. Bůh vždy jednal svrchovaně v záležitostech národů. Babylóňany použil jako nástroj potrestání svého vlastního lidu za porušení smluvních závazků (<2Kr 17:15>; <2Kr 17:18-20>; <2Kr 21:12-15>; <2Kr 24:3-4>).

\ww {nádobí}  %BKR
    {zařízení} %PSP
    {předměty}  %CSP
    {nádob} %CEP
    {vybavení}  %B21 
    {nádob}  %SNC
\Note 1:2 {} Odkaz na nádobí z vypleněného chrámu, nikoliv zajatců. Deportace proběhly ve třech vlnách: První roku 605 př.Kr., kdy mezi odvlečenými byl i Daniel; druhá roku 597 př.Kr., zahrnovala i Ezechiela. Třetí deportace, při které Babylóňané zničili Jeruzalém a chrám, nastala roku 586 př.Kr.

\ww {do domu boha svého}  %BKR
    {v dům svých bohů} %PSP
    {do domu svého boha}  %CSP
    {do domu svého božstva} %CEP
    {do chrámu svého boha}  %B21 
    {do klenotnice svého božstva}  %SNC
\Note 1:2 {do domu boha svého}  Hlavní božstvo babylónského panteonu byl Marduk  (srv. <Jr 50:2>).

\ww {z semene královského}  %BKR
    {ze semene království} %PSP
    {z královského potomstva}  %CSP
    {z královského potomstva} %CEP
    {z královského rodu}  %B21 
    {ze šlechtických a královských rodů}  %SNC
\Note 1:3 {} \x/Daniel/ a jeho přátelé patří mezi potomky Davidovy královské linie (<"v." 6>), které \x/Nabuchodonozor/ odvlekl při první deportaci roku 605 př.Kr. (srv. <"pozn." 2>n). 

\ww {liternímu umění a jazyku}  %BKR
    {písmu a jazyku Kasdím} %PSP
    {chaldejskému písemnictví a jazyku}  %CSP
    {kaldejskému písemnictví a jazyku} %CEP
    {babylonskému písemnictví a jazyku}  %B21 
    {kaldejský jazyk}={kaldejský jazyk ... s národním písemnictvím}  %SNC
\Note 1:4 {liternímu umění a jazyku} Babylónská literatura byla psána komplikovaným sumérským a akkadským slabičným klínovým písmem primárně na hliněných tabulkách. Těchto dochovaných tabulek existují tisíce. Studium této literatury seznámilo Daniela a jeho přátele s polyteistickým světonázorem Babylóňanů, plného kouzlení, čarodějnictví a astrologie.  Mluvený jazyk pro běžné použití byla nicméně aramejština (srv. <2:4>), psaná snadno osvojitelným abecedním systémem.

\ww {z stolu královského}  %BKR
    {z králova pokrmu} %PSP
    {z královského přídělu}  %CSP
    {z královských lahůdek} %CEP
    {z královského stolu}  %B21 
    {z jeho stolu} %SNC
\Note 1:5 {z stolu královského} Později se \x/Joakim/ovi dostalo stejného zaopatření (<2Kr 25:27-30>).

\ww {Daniel}={\x/Daniel/, \x/Chananiáš/, \x/Mizael/ a \x/Azariáš/}  %BKR
    {Dánijjél, Chananjá, Míšáél a Azarjá} %PSP
    {Daniel, Chananjáš, Míšael a Azarjáš}  %CSP
    {Daniel, Chananjáš, Míšael a Azarjáš} %CEP
    {Daniel, Chananiáš, Mišael a Azariáš}  %B21 
    {Daniel, Chananjáš, Míšael a Azarjáš}  %SNC
\Note 1:6 {Daniel}={\x/Daniel/, \x/Chananiáš/, \x/Mizael/ a \x/Azariáš/} Charakteristická  hebrejská jména. Dvě z nich obsahují prvek EL, znamenající \uv{Bůh}, a dvě JAH, což je zkratka osobního Božího jména, které překládáme jako \uv{Hospodin}.  \x/Daniel/ znamená \uv{Můj soudce je Bůh}, \x/Chananiáš/ \uv{Hospodin je milostivý}, \x/Mizael/ \uv{Kdo je jako Bůh?} a \x/Azariáš/ \uv{Hospodin mi pomáhá}.

\ww {Baltazar}={\x/Baltazar/ ... \x/Sidrach/ ... \x/Mizach/ ... \x/Abdenágo/}  %BKR
    {Béltešaccar}={Béltešaccar ... Šadrach ... Méšach ...i Avéd-negó} %PSP
    {Beltšasar}={Beltšasar ... Šadrak ... Méšak ...i Abed-nego}  %CSP
    {Beltšasar}={Beltšasar ... Šadrak ... Méšak ... Abed-nego} %CEP
    {Baltazar}={Baltazar ... Šadrach ... Mešach ... Abednego}  %B21 
    {Beltšasarem}={Beltšasarem ... Šadrakem ... Méšakem ... Abed-negem}  %SNC
\Note 1:7  {Baltazar}={\x/Baltazar/ ... \x/Sidrach/ ... \x/Mizach/ ... \x/Abdenágo/}
     Přesný význam těchto jmen je předmětem diskusí. Převažují tyto názory: 
     \x/Baltazar/: {\em Bel} [jiné jméno pro Marduka, hlavního boha babylónského panteonu]
     {\em chraň jeho život} nebo {\em Paní, ochraňuj krále}; 
     \x/Sidrach/: {\em Velice se bojím (Boha)} nebo {\em Přikázání Aku} [sumérského měsíčního boha];
     \x/Mizach/:   {\em Jsem bezvýznamný} nebo {\em Kdo je to, co Aku?};
     \x/Abdenágo/: {\em Služebník zářícího.}
     
\ww {se nepoškvrňoval}  %BKR
    {se nebude poskvrňovat} %PSP
    {se nebude poskvrňovat}  %CSP
    {se neposkvrní} %CEP
    {se neposkvrní}  %B21 
    {nemůže ustoupit}  %SNC
\Note 1:8 {se nepoškvrňoval} Důvod, pro který byl Daniel přesvědčen, že by ho králův pokrm poskvrnil, není uveden. Pravděpodobně jídlo znamenalo porušení dietních předpisů Mojžíšova zákona  (<Lv 11:1-47>), zakazujících konzumaci vepřového nebo masa nezbaveného krve (<Lv 17:10-14>). Také mohlo zahrnovat pokrmy, obětované babylónský modlám. 

\ww {milost a lásku u správce}  %BKR
    {před tváří vedoucího komorníků k laskavosti a k slitování} %PSP
    {u vrchního komorníka milosrdenství a slitování}  %CSP
    {u velitele dvořanů milosrdenství a slitování} %CEP
    {u vrchního dvořana daroval přízeň a náklonnost}={u vrchního dvořana ... přízeň a náklonnost}  %B21 
    {porozumění a byl ochoten vyjít mu vstříc}={porozumění ... vyjít mu vstříc}  %SNC
\Note 1:9 {milost a lásku u správce} Danielův osud v mnohém připomíná Josefův příběh (<Gn 39-41>).

\Note 1:12 {deset} Často se symbolickým významem dokonalosti nebo plného počtu. \dopsat  Abrahamova přímluva za Sodomu, poznámka nebo článek.

\ww {uposlechl}  %BKR
    {vyhověl} %PSP
    {poslechl}  %CSP
    {vyslyšel} %CEP
    {vyhověl}  %B21 
    {přistoupil}  %SNC
\Note 1:14 {uposlechl} Daniel neslíbil, že se v případě zchátralejšího zevnějšku přizpůsobí a poslechnou nařízení v rozporu s Božím Zákonem. Je možné (a ve světle dalších kapitol i docela pravděpodobné), že už tehdy byli rozhodnuti neposlechnout bezbožného vladaře a raději zemřít, než zkompromitovat víru.

\ww {tváře jejich byly krásnější}  %BKR
    {jejich vzezření byla lepší} %PSP
    {vypadali dobře}  %CSP
    {jejich vzhled je lepší} %CEP
    {ypadali zdravěji}  %B21 
    {vypadají lépe}  %SNC
\Note 1:15 {tváře jejich byly krásnější} Bůh Danielovi a jeho přátelům požehnal pro jejich věrnost Božímu Slovu (<Dt 8:3>; <Mt 4:4>). Nepřál jim smrt, která by je nejspíše čekala, kdyby byl správce s výsledkem nespokojen, a oni by přesto na svém odmítání rituálně nečisté stravy trvali
(srv. \<"pozn." 14>n). 

\Note 1:17 {moudrost} Danielova moudrost se stala příslovečnou ještě za jeho života; Ezechiel říká králi Týru ironicky, že je moudřejší nad Daniela (<Ez 28:3>). 

\ww {vidění a snům}  %BKR
    {vidění a snů} %PSP
    {vidění a sny}  %CSP
    {viděním a snům} %CEP
    {viděním a snům}  %B21 
    {vidění a sny}  %SNC
\Note 1:17 {vidění a snům} Daniel převyšoval i své přátele schopností interpretovat sny, pro kterou byl vyvýšen nade všechny ostatní, podobně jako kdysi Josef u faraonova dvora (<Gn 40:8>; <Gn 41:16>). 

\ww {když se dokonali dnové}  %BKR
    {po uplynutí dní} %PSP
    {Po uplynutí doby}  %CSP
    {Po uplynutí doby} %CEP
    {stanovená lhůta}  %B21 
    {Když uplynula doba}  %SNC
\Note 1:18 {když se dokonali dnové} Po třech letech, zmíněných ve  <"v." 5>.

\ww {desetkrát}  %BKR
    {o deset pídí} %PSP
    {desetkrát}  %CSP
    {desetkrát} %CEP
    {desetkrát}  %B21 
    {mnohonásobně}  %SNC
\Note 1:20 {desetkrát} viz (<"pozn." 12>n).

\ww {mudrce a hvězdáře}  %BKR
    {písmaři a zaříkávači} %PSP
    {kouzelníky a věštce}  %CSP
    {věštce a zaklínače} %CEP
    {věštce a kouzelníky}  %B21 
    {poradce a astrology}  %SNC
\Note 1:20 {mudrce a hvězdáře} První výraz se vyskytuje také v <Gn 41:8>, <24> a <Ex 7:11>; druhý se objevuje pouze zde a v \<2:2>. Srv. (<"pozn." Gn 41:8>n). \dopsat %Hebrew 

\ww {léta prvního Cýra krále}  %BKR
    {první rok krále Kóreše} %PSP
    {prvého roku krále Kýra}  %CSP
    {prvního roku vlády krále Kýra} %CEP
    {prvního roku krále Kýra}  %B21 
    {prvního roku krále Kýra}  %SNC
\Note 1:21 {léta prvního Cýra krále} Tj. \x/Cýr/ovy vlády nad Babylónem, tedy 539 př.Kr. Daniel v roce 537 př.Kr. dosud žil (<10:1>); dožil se návratu Judejců ze zajetí do Země.

\Note 2:1-49 {}={\it\x/Nabuchodonozor/ův první sen\/} Daniel ve službách krále (mimo jiné) interpretoval jeho sny, což odhaluje nejen, že  byl (\x/Daniel/) zahrnut Božím požehnáním, ale také že Bůh směřoval historii  k nastolení svého království.

\ww {Léta pak druhého}  %BKR
    {v druhém roce} %PSP
    {Ve druhém roce}  %CSP
    {Ve druhém roce} %CEP
    {Ve druhém roce}  %B21 
    {Ve druhém roce}  %SNC
\Note 2:1 {Léta pak druhého} První rok výcviku Daniela a jeho přátel byl
zároveň \x/Nabuchodonozor/ův \uv{korunovační rok}. Druhý a třetí rok výcviku byl první a druhý rok
\x/Nabuchodonozor/ovy vlády.  %Srv. <"pozn." 1:1>n).
Mezi tímto tvrzením a dokončením tříletého výcviku mladíků (<1:5>) není žádná diskrepance: 
První  \x/Nabuchodonozor/ův rok byl Babylóňany považován za \uv{korunovační rok} a druhý a třetí rok Danielovy výuky za první a druhý  \x/Nabuchodonozor/ovy vlády (viz <"pozn." 1:1>n).

\ww {ze sna protrhl}  %BKR
    {se jeho duch rozrušil} %PSP
    {V duchu se rozrušil}  %CSP
    {Rozrušil se} %CEP
    {rozrušil}  %B21 
    {vyděsil}  %SNC
\Note 2:1 {ze sna protrhl} Na Starém Blízkém Východě bylo běžné mít za to, že božstva promlouvají k lidem také skrze některé (hlavně velmi živé) sny. \x/Nabuchodonozor/ovo zneklidnění je pochopitelné; poselství shůry mělo implikace pro budoucnost království. Zapomenutí snu bylo považováno za hněv božstva vůči adresátovi.


\ww {mudrce, a hvězdáře i kouzedlníky a Kaldejské}  %BKR
    {písmaře a zaříkávače a čarodějníky a Kasdím} %PSP
    {kouzelníky a věštce i čaroděje a chaldejce}  %CSP
    {věštce, zaklínače, čaroděje a hvězdopravce} %CEP
    {věštce, kouzelníky, čaroděje a mágy}  %B21 
    {astrology, jasnovidce a členy kněžské třídy}  %SNC
\Note 2:2 {mudrce, a hvězdáře i kouzedlníky a Kaldejské} Viz <"pozn." 1:20>n. 

\ww {Syrsky}  %BKR
    {arámsky} %PSP
    {aramejsky}  %CSP
    {aramejsky} %CEP
    {aramejsky}  %B21 
    {aramejsky}  %SNC
\Note 2:4 {Syrsky} Odsud až po konec kapitoly 7 je text psán aramejsky a ne hebrejsky, podobně jako např.  <Ezd 4:8-6:18>. Aramejština byl oficiální úřední jazyk; nelze také vyloučit, že tyto pasáže jsou v aramejštině toho důvodu,  že obsahují proroctví, která mohla pohany zajímat více než Židy.

\ww {Neoznámíte-li mi snu}  %BKR
    {Jestliže mi nedáte na vědomí ten sen} %PSP
    {Jestliže mi ten sen ani jeho výklad nesdělíte}  %CSP
    {Jestliže mi neoznámíte sen} %CEP
    {Pokud mi nesdělíte onen sen}  %B21 
    {Buď mi ho oznámíte}  %SNC
\Note 2:5 {Neoznámíte-li mi snu} \x/Nabuchodonozor/ zkouší své poradce. Nebudou-li schopni připomenout mu jeho sen, nemá proč důvěřovat ani jejich potencionálnímu výkladu (srv. <"v." 9>).


\ww {kromě bohů}  %BKR
    {leč bohové} %PSP
    {kromě bohů}  %CSP
    {mimo bohy} %CEP
    {kromě bohů}  %B21 
    {jen bohové}  %SNC
\Note 2:11 {kromě bohů} Mudrcům nezbylo než přiznat, faraonův požadavek je nad jejich síly. S odkazem na nadpřirozený zdroj měli pravdu, kterou potvrzuje i \x/Daniel/ (<19>; <27>).


\ww {Bohu nebeskému}  %BKR
    {Boha nebes} %PSP
    {Boha nebes}  %CSP
    {Boha nebes} %CEP
    {Boha nebes}  %B21 
    {Boha}  %SNC
\Note 2:18 {Bohu nebeskému} \x/Daniel/ se modlí rovnou k Bohu, který vládne hvězdám, na rozdíl od astrologů, kteří se doptávají jenom hvězd samotných, protože nikoho vyššího neznají.


\ww {věc tajná}  %BKR
    {tajemství} %PSP
    {tajemství}  %CSP
    {tajemství} %CEP
    {tajemství}  %B21 
    {vidění}  %SNC
\Note 2:19 {věc tajná} Skryté věci jsou u Hospodina (<Dt 29:29>). Jsou {\em transcendentní} --- nepoznatelné, pokud nejsou zjeveny. Později je tentýž výraz použit pro označení Božích skrytých záměrů s dějinami světa (\<4:6>). 
%%%%% Příklad do dokumentace. Zkontrolovat Dt 29:29, někde je to 29:28

\ww {ssazuje krále, i ustanovuje krále}  %BKR
    {odstraňuje krále a nastoluje krále} %PSP
    {Krále sesazuje, krále nastoluje}  %CSP
    {krále sesazuje, krále ustanovuje} %CEP
    {sesazuje krále a jiné nastolí}  %B21 
    {Bez jeho vůle není žádný vládce ustanoven ani sesazen}  %SNC
\Note 2:21 {ssazuje krále, i ustanovuje krále} Narážka na obsah snu, líčící sérii povstávajících a zanikajících impérií, a který se ještě za \x/Nabuchodonozor/ova života stane jeho vlastní palčivou zkušeností (<4:28-30>). 

\ww {věci hluboké a skryté}  %BKR
    {věci nevyzpytatelné a skryté} %PSP
    {hlubiny a věci ukryté}  %CSP
    {hlubiny a skryté věci} %CEP
    {věci skryté v hlubinách}  %B21 
    {tajemství}={tajemství ... skryté věci}  %SNC
\Note 2:22 {věci hluboké a skryté} Viz <"pozn." 2:11>n. 

\ww {těť oslavuji a chválím}  %BKR
    {děkuji a vychvaluji tě} %PSP
    {chválím a uctívám}  %CSP
    {chci vzdávat čest a chválu} %CEP
    {Chválím tě a oslavuji}  %B21 
    {Tobě děkuji, Bože mých otců, a tebe chválím}={Tobě děkuji ... a tebe chválím}  %SNC
\Note 2:23 {těť oslavuji a chválím} \x/Daniel/ vyjadřuje hlubokou vděčnost za Boží milost vyslyšením modlitby. Boží zjevení je v příkrém kontrastu s mlčením falešných božstev pohanských hadačů. Jenom Bůh zná všechno a vládne všemu. \x/Daniel/a vyvýšil mimořádným poznáním, jež mu svěřil.

\ww {výklad ten oznámím}  %BKR
    {mohu ten výklad králi vyjevit}  %PSP
    {výklad králi přednesu}  %CSP
    {sdělím králi výklad} %CEP
    {já mu ten sen vyložím}  %B21 
    {á mu všechno oznámím}  %SNC
\Note 2:24 {výklad ten oznámím} \x/Daniel/ zde mluví pouze o výkladu, čímž implikuje, že obsah snu  již zná.


%%%%%%%%%%%%%%%%%%%%%%%%%%%%%%%%%%%
%%%%%%%%%%%%%%%%%%%%%%%%%%%%%%%%%%%
\renum Da 2:28 = CzeSNC 2:27-27
%zkouším to tady, protože SNC má zrovna tuhle část verše 28 jako součást verše 27, ale má i 28. verš, kde je jeho zbytek.
%Snad to nebude dělat bugr někde jinde
%A musím se v Hradci podívat do výtisku, jestli je to tak, jak tvrdí bibleserver
% Jo je, SNC 2:27, ale jenom část verše, zbytek = 28 jako ostatní
%Uvést do dokumentace jako příklad, že \renum si s tímporadí,ale musí se zapsat dvakrát ke stejnému verši jako při těchhle 2 poznámkách k verši 2:28
%%%%%%%%%%%%%%%%%%%%%%%%%%%%%%%%%%%
%%%%%%%%%%%%%%%%%%%%%%%%%%%%%%%%%%%
\ww {jest Bůh na nebi, kterýž zjevuje tajné věci}  %BKR
    {v nebesích je Bůh, zjevující tajemství}  %PSP
    {jest Bůh na nebesích, zjevující tajemství}  %CSP
    {je Bůh v nebesích, který odhaluje tajemství}  %CEP
    {Na nebi je však Bůh, který zjevuje tajemství}  %B21 
    {jen Bůh, který je v nebesích, odhaluje tajemství}  %SNC
\Note 2:28 {jest Bůh na nebi, kterýž zjevuje tajné věci} Podobně jako Josef v Egyptě (<Gn 40:8>; <Gn 41:16>) si ani \x/Daniel/ nepřisvojuje poznání a výklad snu, nýbrž připisuje je Bohu.


%Tak tohle funguje, uvést do dokumentace jako příklad. Půl verše má jiné číslo a jde to.
\renum Da 2:28 = CzeSNC 2:28-28
\ww {v potomních dnech}  %BKR
    {v posledku dní} %PSP
    {v budoucích dnech}  %CSP
    {v posledních dnech} %CEP
    {v posledních dnech}  %B21 
    {posledních dnů}  %SNC
\Note 2:28 {v potomních dnech} Dosl. \uv{v posledních dnech}, čemuž sz lidé rozuměli jako době obnovy národa po návratu z exilu (viz <Dt 4:30>). Tatáž fráze může označovat 
    jakoukoliv budoucnost  (<Gn 49:1>).  V NZ je použita celkem pětkrát, z čehož dvakrát odkazuje na věk, započatý o Letnicích (<Sk 2:17>; <Žd 1:2>), a třikrát na závěr dějin před Kristovým druhým příchodem (<2Tm 3:1>; <Jk 5:3> a <2Pt 3:3>). 

\Note 2:32-33 {}={Nejtěžší nahoře, nejkřehčí dole} Směrem odshora dolů materiálům ubývá nejen na váze, ale i na hodnotě. Obraz vystihuje osud všech impérií,  nejen těchto čtyř nejbližších: 
    Každá instituce, stát, říše, civilizace se nakonec zhroutí vlastní vahou, když  nezadržitelná byrokracie přeroste svůj původní účel služby lidem a sama tyje na jejich úkor.

\ww {kámen, kterýž nebýval v rukou}  %BKR
    {kámen, jenž nebyl v rukou} %PSP
    {bez rukou nějaký kámen}  %CSP
    {se bez zásahu rukou utrhl kámen} %CEP
    {se bez dotyku lidské ruky vylomil kámen}  %B21 
    {bez lidského přispění -- přiletěl kámen}  %SNC
\Note 2:34 {kámen, kterýž nebýval v rukou} Na rozdíl od ostatních království, budovaných lidmi, toto vystaví Bůh sám. Zde  je kámen asociován s Kristovým královstvím    (srv. <1Kor 10:4> a <"jeho pozn.">n).
%%%%% Co s tím? Jak napsat "a jeho poznámku", aby to byl link, ale už neopakoval čísla? Šlo by, aby nebyl uveden ani verš?

\renum Da 2:38 = CzeSNC 2:37-37
\ww {hlava zlatá}={hlava zlatá ... království pak čtvrté}  %BKR
    {hlava ze zlata}={hlava ze zlata ... čtvrté království} %PSP
    {zlatá hlava}={zlatá hlava ... čtvrté království}  %CSP
    {zlatá hlava}={zlatá hlava ... čtvrté království} %CEP
    {zlatá hlava}={zlatá hlava ... čtvrté království}  %B21 
    {Zlatá hlava}={Zlatá hlava ... Říše čtvrtá}  %SNC
\Note 2:38-40 {hlava zlatá}={hlava zlatá ... království pak čtvrté} Čtyři království reprezentují po sobě jdoucí impéria babylónské, médo-perské, řecké a římské. Bohem založené nepomíjitelné Kristovo království bude inaugurováno v časech římské světovlády (viz též Úvod a graf \uv{Danielovy vize} v <Da 2> na str.~\pgref[danielovyvize]).

\ww {Za dnů pak těch králů}={za dnů ... těch králů}  %BKR
    {v dnech oněch králů} %PSP
    {Za dnů oněch králů}  %CSP
    {Ve dnech těch králů} %CEP
    {Za dnů těch králů}  %B21 
    {V době těchto státních útvarů}  %SNC
\Note 2:44 {Za dnů pak těch králů}={za dnů ... těch králů} Někteří interpreti tvrdí, že se jedná o linii králů posledního  království; daleko pravděpodobnější však je rozumět tomuto verši jako označujícímu panovníky království, zmíněných ve verších <2:38-40>.

\ww {na věky nebude zkaženo}  %BKR
    {na věky nebude moci být vyvráceno} %PSP
    {nebude zničeno navěky}  %CSP
    {nebude zničeno navěky} %CEP
    {se nikdy nezhroutí}  %B21 
    {nikdy nebude zničeno }  %SNC
\Note 2:44 {na věky nebude zkaženo} Podobně jako i jiní proroci, \x/Daniel/ hovoří o království, které Bůh založí po návratu z exilu jako permanentní (srv. např. <Iz 9:7>; <Jl 2:26-27> či <Am 9:15>). NZ učí, že toto království začalo Kristovým prvním příchodem a dosáhne svého dovršení jeho slavným návratem (viz článek <"Království Boží u" Mt 4> na str. \ref[Království Boží]). %To budu musel prepsat

\ww {padl na tvář}  %BKR
    {padl na svůj obličej} %PSP
    {padl na tvář}  %CSP
    {padl na tvář} %CEP
    {padl před Danielem na tvář}  %B21 
    {hluboce se Danielovi poklonil}  %SNC
\Note 2:46 {padl na tvář} Pozoruhodná výměna rolí anticipuje příchod Božího království, vysoce převyšujícího i nejmocnější lidská impéria.

\ww {Bůh bohů a Pán králů}  %BKR
    {Bůh bohů a Pán králů} %PSP
    {bůh bohů a pán králů}  %CSP
    {Bohem bohů a Pán králů}={Bůh bohů a Pán králů} %CEP
    {Bůh bohů, Pán králů}  %B21 
    {Bůh nade všemi bohy}={Bůh nade všemi bohy, ... vládne nade všemi králi}  %SNC
\Note 2:47 {Bůh bohů a Pán králů} \x/Nabuchodonozor/ vyznává Hospodinovu svrchovanost nejen nad bezmocnými pohanskými božstvy, ale i nad králi, jako je on sám. To je téma, sjednocující prvních šest kapitol knihy Daniel.

\ww {krajinou Babylonskou}  %BKR
    {krajem Bávelu} %PSP
    {babylonskou provincií}  %CSP
    {babylónskou krajinou} %CEP
    {provincie Babylon}  %B21 
    {babylónské krajiny}  %SNC
\Note 2:48 {krajinou Babylonskou} Babylónské impérium bylo rozděleno na provincie;  \x/Daniel/ byl ustanoven vládcem nad provincií s hlavním městem. Podobný vzestup Židů k moci v cizích zemích lze vidět v <Gn 41:37-44> (Josef) a <Est 8:1-2> (\x/Mardocheus/). \x/Daniel/ovi přátelé ho na jeho přímluvu u krále v této pozici nahradili (\<49>).

\ww {býval v bráně královské}  %BKR
    {byl v bráně králově} %PSP
    {zůstal na královském dvoře}  %CSP
    {zůstal na královském dvoře} %CEP
    {zůstával na královském dvoře}  %B21 
    {zůstal na královském dvoře}  %SNC
\Note 2:49 {býval v bráně královské} Pravděpodobně důvod, proč se ho netýkal trest za odmítnutí poklonit se \x/Nabuchodonozor/ově modle (<3:20>).

\ww {zlatý}  %BKR
    {ze zlata} %PSP
    {zlatou}  %CSP
    {zlatou} %CEP
    {zlatou}  %B21 
    {zlatou}  %SNC
\Note 3:1 {zlatý}  Pravděpodobně pozlacený; zhotovení sochy se podobalo popisu v <Iz 40:19>; <Iz 41:7> a  <Jr 10:3-9>.

\ww {obraz}  %BKR
    {sochu} %PSP
    {sochu}  %CSP
    {sochu} %CEP
    {sochu}  %B21 
    {sochu}  %SNC
\Note 3:1 {obraz} Názory badatelů, zda se jedná o mimořádnou podobiznu  \x/Nabuchodonozor/a samotného, či zda to bylo zobrazení nějakého babylónského božstva, nebo pouhý obelisk, se liší. Z toho, co je známo o babylónských náboženských tradicích, je pravděpodobné, že obraz zpodobňoval Béla nebo Nabu, některé z těchto \x/Nabuchodonozor/ových ochraňujících božstev. Padnout na tvář před božstvem znamenalo zároveň vyjádření podřízenosti \x/Nabuchodonozor/ovi, který božstvo reprezentoval na zemi (srv. <2:46> a <"jeho pozn.">n). 
%%%%%% verš a jeho poznámku.

\ww {výška}={60x6}  %BKR
    {výška}={60x6} %PSP
    {výška}={60x6}  %CSP
    {výška}={60x6} %CEP
    {vysoká}={60x6}  %B21 
    {vysoká}={30x3}  %SNC
\Note 3:1 {výška}={60x6} Rozměry jsou důvodem, proč někteří vyvozují, že obraz byl spíše obelisk, než podobizna člověka (jehož proporce jsou 6:1, nikoliv 10:1). Socha však mohla stát na piedestalu, nebo mít stylizovaný tvar.

\ww {Dura}  %BKR
    {Dúrá} %PSP
    {Dúra}  %CSP
    {Dúra} %CEP
    {Dura}  %B21 
    {Dúra}  %SNC
\Note 3:1 {Dura} Přesné umístění není známo. Obvykle bývá spojeno s Tolul Dura asi 10 km jižně od Babylóna.

\ww {knížata}={knížata ... úředníky}  %BKR
    {vladařů}={vladařů ... rádců} %PSP
    {satrapy}={satrapy ... komisaře}  %CSP
    {satrapy}={satrapy ... úředníky} %CEP
    {satrapy}={satrapy ... hodnostáře}  %B21 
    {hodnostáři}={hodnostáři ... pokladníci}  %SNC
\Note 3:2 {knížata}={knížata ... úředníky} Přesné rozsahy pravomocí těchto  různých druhů úředníků nejsou známy. Pět ze sedmi termínů vypadá na perský původ, což by naznačovalo, že \x/Daniel/ tento zápis dokončil až po dobytí Babylóna Peršany  roku 539 př.Kr.

\ww {trouby}={trouby ... muziky}  %BKR
    {rohu}={rohu ... dud} %PSP
    {roh}={roh ... orchestr}  %CSP
    {rohu}={rohu ... dud} %CEP
    {roh}={roh ... buben}  %B21 
    {fanfáry}  %SNC
\Note 3:5 {trouby}={trouby ... muziky} Tři z vyjmenovaných  hudebních nástrojů  nesou jména, převzatá z řečtiny (citera, harfa a dudy). To vede některé interprety k názoru, že kniha byla sepsána až po dobytí Persie Alexandrem Velikým. 
To je však závěr, který nevyplývá nutně z premisy (tzv. argument non-sequitur); mezi hudebníky je běžné používat mezinárodní termíny k označení hudebních nástrojů. Pojmy jako  \uv{gibsonka}, \uv{jumbo} \uv{strato/tele-caster}, \uv{Les Paul} apod. jsou jednoznačně srozumitelné jak češtině (v rámci muzikantského slangu), tak i v mezinárodním kontextu; a přesto z jejich zdomácnělé přítomnosti v jazyce nelze vyvodit závěr, že se v něm objevily až po zhroucení železné opony v roce 1989, kdy se západním vlivům otevřela volná cesta. 
Kromě toho se názvy řeckých hudebníků i nástrojů vyskytují i v asyrských inskripcích dávno před \x/Nabuchodonozor/em. 


\ww {peci ohnivé}={ohnivé pece} %BKR
    {ohnivé pece} %PSP
    {ohnivé pece} %CSP
    {ohnivé pece} %CEP
    {ohnivé pece} %B21
    {rozpálené pece} %SNC
\Note 3:6 {ohnivé pece} Pece byly v Babylóně běžně používány k vypalování cihel  (<"srv." Gn 11:3>). Nebylo neobvyklé používat je jako nástroj popravy upálením zaživa (<"viz" Jr 29:22> nebo též <2Mak 7>).  

\ww {Kaldejští}  %BKR
    {Kasdejci} %PSP
    {Chaldejci}  %CSP
    {hvězdopravci} %CEP
    {mágové}  %B21 
    {astrologové}  %SNC
\Note 3:8 {Kaldejští} Tj. mágové, astrologové <"Viz pozn." 2:2>n.  %\Note 3:8 
 Zde však výraz \uv{Chaldejci} indikuje národnost, nikoliv funkci. Chaldejci shlíželi spatra na Židy z rasově-etnických antisemitských předsudků (<"srv." 12>; <Est 3:5-6>). 
 Privilegovaná pozice \x/Daniel/ových přátel znásobila nevraživost Chaldejců vůči nim (\<12>).

\renum Da 3:12 = CzeSNC 3:13-13
\Note 3:12 {Sidrach}={\x/Sidrach/, \x/Mizach/ a \x/Abdenágo/} Viz <"pozn." 1:7>n a <2:49>n. 

\ww {který jest ten Bůh}  %BKR
    {kdo je onen Bůh} %PSP
    {kdo je bůh}  %CSP
    {kdo je ten Bůh} %CEP
    {který bůh}  %B21 
    {Chtěl bych vidět toho boha}  %SNC
\Note 3:15 {který jest ten Bůh} \x/Nabuchodonozor/ nevědomky vyzval Hospodina, Boha Izraele, ke změření sil; z jeho polyteistické, pohanské perspektivy žádný bůh ničeho podobného není schopen.


\ww {vytrhne nás}  %BKR
    {umí nás vysvobodit} %PSP
    {nás vysvobodí}  %CSP
    {vysvobodí nás} %CEP
    {zachrání nás}  %B21 
    {udělá to}  %SNC
\Note 3:17  {vytrhne nás} Věrní služebníci ani na vteřinu nepochybují o Boží svrchovanosti, přestože jsou si vědomi, že ve své všemohoucnosti je \uv{všeho schopen}; nesázejí na automaticky zaručenou ochranu za všech okolností. 

\ww {Buď že nevytrhne}  %BKR
    {pakli ne} %PSP
    {i kdyby ne}  %CSP
    {i kdyby ne} %CEP
    {i kdyby ne}  %B21 
    {nebude-li chtít}  %SNC
\Note 3:18 {Buď že nevytrhne} \x/Daniel/ovi přátelé počítají s reálnou možností, že věrnost Bohu je bude stát život. To je však nezviklá v jejich rozhodnutí zůstat věrni. Věrnost poddaných přináší Králi slávu (<29-33>), o to větší, když je to věrnost tváří v tvář smrti.

\ww {synu Božímu}  %BKR
    {syna Božího} %PSP
    {syna bohů}  %CSP
    {božímu synu} %CEP
    {boží syn}  %B21 
    {Boží syn}  %SNC
\Note 3:25 {synu Božímu} Fráze, použitelná pro různé druhy nebeských bytostí; zde je míněn \uv{anděl} <"v." 28>.

\ww {služebníci Boha nejvyššího}  %BKR
    {nevolníci Boha, Nejvyššího} %PSP
    {otroci boha Nejvyššího}  %CSP
    {služebníci Boha nejvyššího} %CEP
    {služebníci Nejvyššího Boha}  %B21 
    {kteří sloužíte nejvyššímu Bohu}  %SNC
\Note 3:26 {služebníci Boha nejvyššího} Titul pro Boží univerzální autoritu. Podobně jako ve
        <"verši" 29> a <2:47> neznamená toto vyznání ze rtů polyteistického pohana, 
        že \x/Daniel/ův Bůh je jediný živý, nýbrž že je nadřazený ostatním božstvům (<4:31-34>).
        Z úst věrného Izraelity totéž vyznání implikuje monoteismus (<5:18>, <21>; <7:18-21>).

\ww {anděla}  %BKR
    {anděla} %PSP
    {anděla}  %CSP
    {anděla} %CEP
    {anděla}  %B21 
    {posel}  %SNC
\Note 3:28 {anděla} Anděl může být \uv{anděl Hospodinův}, jenž může znamenat Kristovo preinkarnační zjevení (<"srv." 6:22>; viz též <"poznámky k" Gn 16:7>n a <Ex 3:2>n). Bůh přislíbil svou přítomnost, budou-li jeho lidé nuceni projít ohněm (<Iz 43:1-3>). 

\ww {není Boha jiného}  %BKR
    {není jiného Boha} %PSP
    {není jiného boha}  %CSP
    {není jiného Boha} %CEP
    {Žádný jiný Bůh}  %B21 
    {žádný jiný bůh}  %SNC
\Note 3:29 {není Boha jiného}={není jiného Boha} Viz <"pozn." 26>n.

\ww {zvelebil}  %BKR
    {zařídil}={zařídil, aby se ... šťastně vedlo} %PSP
    {pomohl ku zdaru}  %CSP
    {zařídil}={zařídil, aby se ... dobře dařilo} %CEP
    {postaral}={se ... postaral, aby se ... dobře dařilo}  %B21 
    {těšili králově přízni}  %SNC
\Note 3:30 {zvelebil} Příběh si dává záležet na tom, aby bylo jasné, že \x/Daniel/ovi přátelé dosáhli vyvýšení svou věrností  Bohu, nikoliv kompromisem s Babylóňany. 

\ww {Nabuchodonozor král}   %BKR
    {Nevúchadneccar, král}   %PSP
    {Král Nebúkadnesar}   %CSP
    {Král Nebúkadnesar}   %CEP
    {Král Nabukadnezar}   %B21
    {krále Nebúkadnesara}   %SNC
\Note 3:31 {Nabuchodonozor král} Poslední incident v knize, spojený s \x/Nabuchodonozor/em. Je umístěn do pozdního období jeho třiačtyřicetileté vlády, kdy dokončil své stavební projekty (<4:27>) a byl na svém vrcholu, neomezený vládce nejmocnější říše na světě. 

\ww {Bůh nejvyšší}   %BKR
    {Bůh, Nejvyšší}   %PSP
    {Bůh Nejvyšší}   %CSP
    {Bůh nejvyšší}   %CEP
    {Nejvyšší Bůh}   %B21
    {nejvyšší Bůh}   %SNC
\Note 3:32 {Bůh nejvyšší} Viz <"poz." 2:47>n;  <3:26>n a <3:28>.

\ww {jak veliká}   %BKR
    {jak velmi jsou veliká}   %PSP
    {Jak veliká}   %CSP
    {Jak veliká}   %CEP
    {Jak nesmírné}   %B21
    {Jak mimořádné}   %SNC
\Note 3:33 {jak veliká}...  \x/Nabuchodonozor/ovo vyznání opakovaně připomíná hlavní téma knihy, absolutní svrchovanost Boží ("srv." <4:33-34> a <"pozn." 7:1-12:13>n).

\Note 4:3-4 {} Viz <"pozn." 1:20>n a <2:2>n.

\Note 4:5 {Baltazar} <"Viz pozn." 1:7>.

\ww {duch bohů svatých}   %BKR
    {duch svatých bohů}   %PSP
    {duch svatých bohů}   %CSP
    {duch svatých bohů}   %CEP
    {duch svatých bohů}   %B21
    {duch svatého Boha}   %SNC
\Note 4:6 {duch bohů svatých} Ačkoliv \x/Nabuchodonozor/ vyjadřuje svůj respekt k \x/Daniel/ovi pohanskou terminologií, dotýká se pravdy: Přítomnost Ducha Svatého má na člověka nepřehlédnutelný účinek; v \x/Daniel/ově případě je to vhled do Božích neproniknutelných tajemství, dar, který o mnoho později obdržel i apoštol Pavel a církev  (<1Kor 2:6-16>).

\ww {nic tajného není tobě nesnadného}   %BKR
    {žádné tajemství tě netísní}   %PSP
    {žádné tajemství ti nedělá potíže}   %CSP
    {žádné tajemství ti nedělá potíže}   %CEP
    {ti žádné tajemství není těžké}   %B21
    {umíš vyložit každé tajemství}   %SNC
\Note 4:6 {nic tajného není tobě nesnadného}  <"Viz" 2:47> a <"pozn." 2:19>n.

\Note 4:7 {strom} Ezechiel 31 také líčí království metaforou vysokého stromu. Podobné obrazy vidíme např. v <Ž 1:3>; <Ž 37:35>; <Ž 52:10>; <Ž 92:13-14>; <Jr 11:16-17>; <Jr 17:8> nebo <Mt 13:32>. 

\Note 4:10 {svatý}  \x/Nabuchodonozor/ připouští, že viděl svatou nebeskou bytost. Víra v podobné bytosti byla na Starém Blízkém Východě běžná a koreluje s biblickým přesvědčením, že Bůh zasahuje do pozemských záležitostí, mnohdy skrze své služebníky anděly.

\ww {Srdce jeho od lidského ať jest rozdílné}   %BKR
    {Jeho srdce nechť se z lidského změní}   %PSP
    {Jeho srdce z lidského ať se promění}   %CSP
    {Jeho srdce ať je jiné, než je srdce lidské}   %CEP
    {Lidský rozum ať jej opustí}   %B21
    {se nebude chovat jako člověk}   %SNC
\Note 4:13 {Srdce jeho od lidského ať jest rozdílné}
    Je zřejmé, že se jedná o člověka, nikoliv o strom. Srv. <"pozn." 4:19>n.

\ww {srdce zvířecí nechť jest dáno jemu}   %BKR
    {nechť je mu dáno srdce zvířete}   %PSP
    {srdce budiž mu dáno zvířecí}   %CSP
    {ať je mu dáno srdce zvířecí}   %CEP
    {dostane rozum zvířecí}   %B21
    {bude žít jako zvíře}   %SNC
\Note 4:13 {srdce zvířecí nechť jest dáno jemu} 
     \x/Nabuchodonozor/  byl postižen mentální poruchou zvanou lykantropie 
     (z řeckého {\em lukos} -- vlk a {\em anthropos} -- člověk), 
     při níž  se člověk chová jako vlk nebo i jiné zvíře (<30>; viz též <"pozn."30>n. 

\ww {sedm let}   %BKR
    {sedm dob}   %PSP
    {sedm časů}   %CSP
    {sedm let}   %CEP
    {sedm období}   %B21
    {Po dobu sedmi let}   %SNC
%\ww   %Není v textu, dořešit
%    {léto}   %BKR
%    {doba}   %PSP
%    {čas}   %CSP
%    {léto}   %CEP
%    {období}   %B21
%    {léto}   %SNC
\Note 4:13 {sedm let} 
     Sedm období neurčené délky (srv. vv. <20> a <22>). Většina interpretů se shoduje na závěru, že {\em \x/léto/} znamená jeden rok. Verš <30> naznačuje, že doba byla delší než den, týden či měsíc.

\ww {Ty jsi ten}   %BKR
    {to jsi ty}   %PSP
    {to jsi ty}   %CSP
    {jsi ty}   %CEP
    {ten strom jsi, králi, ty}   %B21
    {ten strom jsi ty}   %SNC
\Note 4:19 {Ty jsi ten} Pointa vyprávění, podobná Nátanově napomenutí Davida (<2Sa 14:7>), znamená přímou aplikaci pro \x/Nabuchodonozor/a.

%\ww  % upravit později, až bude jak
%    {}   %BKR
%    {}   %PSP
%    {}   %CSP
%    {}   %CEP
%    {}   %B21
%    {}   %SNC
\Note 4:22 {}={zaženou ... s polní zvěří ... spásat porost}  \x/Daniel/ opakuje po nebeském poslu (<13>) popis mentální poruchy, kterou Hospodin postihne nejmocnějšího muže světa. Podobnými symptomy občas trpěl panovník Království Velké Británie a Irska Jiří III. nebo Ota I. Bavorský. Viz <"pozn." 13>n.

\ww {dokudž bys nepoznal}   %BKR
    {než budeš moci poznat}   %PSP
    {dokud nepoznáš}   %CSP
    {dokud nepoznáš} %CEP
    {než poznáš}   %B21
    {poznáš}   %SNC
\Note 4:22 {dokudž bys nepoznal} Záměrem \x/Nabuchodonozor/ova ponížení bylo přimět ho uznat Boží svrchovanost. 

\ww {království tvé tobě zůstane}   %BKR
    {bude tvé kralování zajištěno}   %PSP
    {tvé království se ti dostane}   %CSP
    {tvé království se ti opět dostane}   %CEP
    {své království znovu dostaneš}   %B21
    {své království dostaneš zpět}   %SNC
\Note 4:23 {království tvé tobě zůstane} \x/Daniel/ ujišťuje \x/Nabuchodonozor/a, že poté, co uzná Boží svrchovanost, ani izolace dlouhodobou duševní poruchou ho nepřipraví o království.

\ww {nebesa}   %BKR
    {nebesa}   %PSP
    {nebesa}   %CSP
    {Nebesa}   %CEP
    {Nebes}   %B21
    {nebeský Bůh}   %SNC
\Note 4:23 {nebesa}  První výskyt v Písmu, kdy výraz \uv{nebesa} je použit jako synonymum pro Boha. Srv. např.  <Mt 5:3> s <Lk 6:20>.

\ww {bylinu jako vůl jedl}   %BKR
    {jedl býlí jako skot}   %PSP
    {porost jako dobytek spásal}   %CSP
    {pojídal rostliny jako dobytek}   %CEP
    {jedl trávu jako býk}   %B21
    {jedl byliny jako dobytek}   %SNC
\Note 4:30 {bylinu jako vůl jedl}
     Vzhledem k tomu, že  se \x/Nabuchodonozor/ projevoval rysy charakteristickými pro býložravce, je jeho mentální porucha někdy nazývána {\em boantropií.} Viz <"pozn." 13>.

\ww {krále nebeského}   %BKR
    {Krále nebes}   %PSP
    {Krále nebes}   %CSP
    {Krále nebes}   %CEP
    {Krále nebes}   %B21
    {Krále, který je na nebi}   %SNC
\Note 4:34 {krále nebeského} Toto jedinečné slovní spojení shrnuje téma kapitoly: vládu Boha z nebes (srv. <23> a <"pozn." 23>n). 


\Note 5:1 {Balsazar}
     \x/Balsazar/ znamená \uv{{\em Bel} -- ochraňuj krále!} 
     Nezaměnit se jménem \x/Baltazar/, které v Babylóně dostal Daniel (viz <"pozn." 1:7>n). 
     Nabonidus, \x/Nabuchodonozor/ův zeť, byl posledním vládcem Babylónu. 
     \x/Balsazar/, nejstarší Nabonidův syn, byl ustanoven spoluvládcem společně se svým otcem.
     Byl mu svěřen Babylón, zatímco Nabonidus trávil mnoho času v Arábii.
     Události kapitoly 5 se odehrály v roce 539 př.Kr. (42 let po \x/Nabuchodonozor/ově
     smrti v roce 563 př.Kr., kdy Babylón padl do rukou Peršanů a kdy byl vydán
     edikt, propouštějící Izraelity z otroctví. 
     
     
\Note 5:2 {}={Když okusil víno}  \x/Balsazar/ se dopustil svatokrádežného zločinu i z hlediska pohanství, které   posvátné předměty jiných náboženství chová v úctě.

\Note 5:2 {}={nádoby ... z jeruzalémského chrámu} Viz  <"pozn." 1:2>n. 
     
\ww {otec}  %BKR
    {otec} %PSP
    {otec}  %CSP
    {otec} %CEP
    {otec}  %B21 
    {děd}  %SNC
\Note 5:2 {otec}  \x/Nabuchodonozor/ je nazýván otcem \x/Balsazar/a zde a ve verších 
<5:11>, <5:13> a <5:18>; a ve verši <5:22> je \x/Balsazar/ nazván \x/Nabuchodonozor/ovým synem.
Ačkoli víme, že \x/Balsazar/ byl přímým synem Nabonida  (srv. <"pozn." 5:1>n), v antickém světě bylo běžné používat pojmu \uv{otec} a \uv{syn} v širším smyslu předka či předchůdce a potomka či následníka. Je pravděpodobné, že \x/Balsazar/ byl \x/Nabuchodonozor/ovým vnukem přes svou matku Nitocris.
     
\Note 5:4 {}={chválili bohy} Nádoby z Hospodinova chrámu byly znesvěceny nejen profánním použitím, ale také účastí na oslavě babylónských falešných božstev. 
     
\ww {hvězdáři}   %BKR
    {zaříkávačů}   %PSP
    {věštce}   %CSP
    {zaklínače}   %CEP
    {kouzelníky}   %B21
    {zaklínače}   %SNC
\Note 5:7 {hvězdáři}={hvězdáři, Kaldejci a hadači} Viz  <"pozn." 1:20>n a  <2:2>n (srv. <2:27>; <4:7>). 
     
\Note 5:7 {}={Kdokoli přečte a vyloží} Opět je požadavek dvojí: (1) přednést obsah znamení a (2) podat jeho výklad  (<"srv." 2:2>).      
     
\Note 5:7 {}={třetím v království} Další v hierarchii po Nabonidovi a jeho spoluvládci \x/Balsazar/ovi  (viz <"pozn." 5:1>n).     
     
\Note 5:8 {}={nedokázali přečíst ani sdělit výklad} Viz  <2:2-13> a  <4:6>; srv. též <Gn 41:8>.   
 
\Note 5:10 {}={královna} Je vysoce nepravděpodobné, že by to byla \x/Balsazar/ova manželka; ty všechny již byly na banketu přítomny  (<2-3>). Mohla to být vdova po \x/Nabuchodonozor/ovi, ale pravděpodobněji to byla  Nitocris, manželka Nabonidova, dcera \x/Nabuchodonozor/a a \x/Balsazar/ova matka.
     
\Note 5:11 {}={duch svatých bohů}   <"Viz" 4:8>. 
Nepřekvapuje, že s událostmi \x/Daniel/ova života byla králova matka obeznámena lépe než  \x/Balsazar/ sám. \x/Daniel/ovi mohlo tou dobou (v roce 539 př.Kr.) být přes 80 let. O 66 let dříve (v roce 605 př.Kr.) byl odvlečen do Babylóna jako mladík (<1:4>).     
     
%\Note 5:12 {}={neobyčejný duch ... jasnozřivost} \x// (<"">)     
     
\Note 5:12 {Baltazar} Viz <"pozn." 1:7>n.     
     
\Note 5:16 {}={třetí v království} Viz  <"pozn." 7>n.     
     
\Note 5:17 {}={někomu jinému} Někteří interpreti jsou toho názoru, že \x/Daniel/ odmítl pocty a odměny nejen ze skromnosti, že o bohatství a moc nestál, ale i s vědomím toho, že schopnost odpovědět králi má jen díky Boží milosti a nechtěl Bohem svěřenou roli zneužít k osobnímu prospěchu (srv. <Gn 14:23>). Nicméně jindy podobné dary bez problémů přijal (<2:48>; <5:29>). Možná, že se chtěl vyhnout jakémukoliv tlaku upravit poselství nevítané zvěsti; možná, že mu bylo jasné, že stejnak není o co stát, když království tuto noc vezme svůj konec.
     
\ww {Bůh nejvyšší}   %BKR
    {Bůh, Nejvyšší}   %PSP
    {Bůh Nejvyšší}   %CSP
    {Bůh nejvyšší}   %CEP
    {Nejvyšší Bůh}   %B21
    {nejvyšší Bůh}   %SNC
\Note 5:18 {Bůh nejvyšší}   Viz <2:37> a  <4:33>.     
     
\Note 5:18 {}={tvému otci \x/Nabuchodonozor/ovi}   Viz <"pozn." 2>n.
     
\Note 5:20-21 {} Viz  <4:31-33>.     
     
\Note 5:21 {}={nad lidským královstvím panuje Bůh nejvyšší} Toto tvrzení shrnuje teologii celé knihy (Viz Úvod: Záměr a zvláštnosti).     
     
\ww {ačkolis o tom všem věděl}   %BKR
    {třebaže jsi toto vše věděl}   %PSP
    {přestožes o tom všem věděl}   %CSP
    {ačkoli jsi o tom všem věděl}   %CEP
    {přestože jsi to všechno věděl}   %B21
    {Tys o tom všem věděl, a přece ses nenaučil pokoře}   %SNC
\Note 5:22 {ačkolis}={přestožes o tom všem věděl} \x/Balsazar/ doplatil na to, že nepoužíval rozum  (<Mt 7:24-27>). Byl bez výmluvy ještě více, než \x/Nabuchodonozor/, a proto jeho čas milosti vypršel  (<"viz" 1Tm 1:13>).     
\dopsat %článek o racionalitě: být racionální neznamená nevěřit v Boha a být materialista; být racionální znamená umět vyvodit logické závěry z daných premis. Víra v Boha je axiom, z něhož všechno vyplývá.
     
     
\ww {proti Pánu nebes}  %BKR
    {proti Pánu nebes} %PSP
    {proti pánu nebes}  %CSP
    {nad Pána nebes} %CEP
    {proti Pánu nebes}  %B21 
    {nad Boha, který je Pánem nebes}  %SNC
\Note 5:23 {proti Pánu nebes}  Viz <"pozn." 4:34>n.     

\ww {Protož}   %BKR
    {Vtom}   %PSP
    {Nato}   %CSP
    {Proto}   %CEP
    {proto}   %B21
    {Proto}   %SNC
\Note 5:24 {Protož}={proto} Nápis na zdi byla Boží odpověď na \x/Balsazar/ovu arogantní pýchu a zpupnost před Bohem, který svou svrchovanost demonstroval o generaci dříve na \x/Nabuchodonozor/ovi  (<4:31-34>).     
     
\ww {Mene}={Mene, mene, tekel, ufarsin}   %BKR
    {Mené}={Mené, mené, tekél úfarsín}   %PSP
    {MeNé}={MeNé, MeNé, TeKéL, uFaRSín}   %CSP
    {Mené}={Mené, mené, tekel ú-parsín}   %CEP
    {MENE}={MENE MENE TEKEL UFARSIN}   %B21
    {Mené}={Mené, mené, tekel uparsin}   %SNC
\Note 5:25 {Mene}={mene mene tekel ufarsin} Dosl. \uv{sečteno, sečteno, zváženo a rozděleno} nebo \uv{mína [váhová, tedy i měnová jednotka], mína, šekel a půlšekel}. 

\ww {ufarsin}   %BKR
    {úfarsín}   %PSP
    {uFaRSín}   %CSP
    {ú-parsín}   %CEP
    {UFARSIN}   %B21
    {uparsin}   %SNC
\Note 5:25 {ufarsin} Aramejsky \uv{a parsin}.     


\ww {Mene}   %BKR
    {Mené}  %PSP
    {MeNé}   %CSP
    {Mené}  %CEP
    {Mene}  %B21
    {Mené}  %SNC
\Note 5:26 {Mene} V původní aramejštině lze tomuto výrazu rozumět jako slovesu nebo jako podstatnému jménu. \x/Daniel/ jej přečetl jako sloveso \uv{sečteno} a vyložil jako dny \x/Balsazar/ovy vlády, které Bůh přivedl ke konci.      

\ww {tekel}   %BKR
    {tekél}   %PSP
    {TeKéL}   %CSP
    {Tekel}   %CEP
    {Tekel}   %B21
    {Tekel}   %SNC
\Note 5:27 {Tekel} Rovněž sloveso nebo podstatné jméno. \x/Daniel/ slovo čte  jako sloveso \uv{zváženo} a interpretuje je ve smyslu \x/Balsazar/ovy nedostatečnosti před Bohem  (srv. <"pozn." Lk 3:17>n).     
     
\ww {Peres}   %BKR
    {perés}   %PSP
    {PeRéS}   %CSP
    {Peres}  %CEP
    {Peres}  %B21
    {Uparsin}   %SNC
\Note 5:28 {Peres} \x/Daniel/ intepretuje jako sloveso \uv{rozděleno} ve významu království, které bude \x/Balsazar/ovi odebráno a a předání Médům a Prešanům. Pokud hosté na banketu interpretovali tyto tři výrazy jako podstatná jména ve smyslu měnových hodnot ({\it mene\/} je ekvivalent 60 babylonských šekelů, {\it tekel ufarsin\/} lze chápat jako {\it šekel a půl\/}),  je pochopitelné, že jim nápis nedával žádný smysl (srv. <8>).   

\ww {Médským a Perským}   %BKR
    {Mádajovi a Persii}   %PSP
    {Médům a Peršanům}   %CSP
    {Médům a Peršanům}  %CEP
    {Médům a Peršanům}  %B21
    {Médům a Peršanům}   %SNC
\Note 5:28 {Médským}={Médům a Peršanům} Viz Úvod: Záměr a zvláštnosti.
     
     
%\ww {Balsazarova}   %BKR
%    {Bélšaccar}   %PSP
%    {Belšasar}   %CSP
%    {Belšasar}  %CEP
%    {Belšasar}  %B21
%    {Belšasar}   %SNC
\Note 5:29 {Balsazar} \x/Balsazar/, podobně jako jeho otec \x/Nabuchodonozor/, uctil \x/Daniel/a. Na rozdíl od \x/Nabuchodonozor/a však neuctil \x/Daniel/ova Boha. Čest, které se \x/Daniel/ovi a jeho přátelům opakovaně dostalo pro jejich věrnost Bohu,  zdůrazňuje důvěryhodnost \x/Daniel/a coby proroka. Nikdy svou víru nezkompromitoval; vždy zůstal věrný Bohu, byť to bylo  tváří v tvář smrti. Proto lze spolehnout i na jeho pozdější proroctví (kapitoly 7--12). 
     
\Note 5:30 {zabit} Texty Starého Blízkého Východu i řečtí historikové Hérodotos a Xenofón zaznamenávají, že Babylóňané byli zaskočeni překvapivým útokem Peršanů, zatímco se veselili a tančili.

\renum Da 5:31 = CzeCSP 6:1-1
\renum Da 5:31 = CzeCEP 6:1-1
\renum Da 5:31 = CzeB21 6:1-1
\renum Da 5:31 = CzeSNC 6:1-1 
% To jsem  schválně zvědav, jestli to mám správně

\Note 5:31 {Darius}={\x/Darius/ Médský} Některé školy tvrdí, že tento a další 
(<6:1>, <6:6>, <6:9>, <6:25>, <6:28>; <9:1>; <11:1>)     
odkazy na \x/Daria/ Médského v knize \x/Daniel/ jsou historické omyly. Viz <"pozn." 6:1>n. 

\renum Da 6:1 = CzeCSP 6:2-29
\renum Da 6:1 = CzeCEP 6:2-29
\renum Da 6:1 = CzeB21 6:2-29
\renum Da 6:1 = CzeSNC 6:2-29

\ww {Dariovi}   %BKR
    {Dárjávešovi}   %PSP
    {Dareiovi}   %CSP
    {Darjaveš}  %CEP
    {Darjaveš}  %B21
    {Darjaveš}   %SNC
\Note 6:1 {Dariovi} Viz <"pozn." 5:31>n.      
I když je pravda, že \x/Darius/ Médský není zmíněn v dochovaných historických zdrojích mimo Bibli a že mezi \x/Balsazar/em/Nabonidem (viz <"pozn." 5:1>n) a nástupem Kýra Perského není žádný časový interval, neznamená to nutně, že kniha Daniel chybuje. Někteří jsou toho názoru, že \uv{\x/Darius/ Médský} je trůnní jméno Kýra, zakladatele Perského impéria (viz <"pozn." 28>n). 
Je však také možné, že nositelem tohoto označení byl Gubaru, gerenál, který přeběhl od  \x/Nabuchodonozor/a ke Kýrovi, vedl perské dobytí Babylóna a kterého  Kýros učinil vládcem nad územím, které Persie zabrala Babylónu. 
Srv. čl. <"Kdo je \x/Darius/ Médský?" 6>a na str.\pg.


%%%
\ww {duch znamenitější}   %BKR
    {skvělý duch}   %PSP
    {neobyčejný duch}   %CSP6:4
    {mimořádný duch}  %CEP 6:4
    {vzácným duchem}  %B21 6:4
    {mimořádně schopný}   %SNC 6:4
\Note 6:3 {duch znamenitější}={neobyčejný duch} Viz  <1:17>;  <4:6> a <5:12>.



\ww {hledali příčiny proti Danielovi}  %BKR
    {záminky vůči Dánijjélovi} %PSP
    {snažili proti Danielovi nalézt}  %CSP
    {proti Danielovi záminku} %CEP
    {nějakou chybu}  %B21 
    {hledali nějakou záminku}  %SNC
\Note 6:4 {} Vypadá to, že vysocí úředníci nebyli poctiví, a proto měli strach z \x/Daniel/ova povýšení (<"v." 3>); bylo jim jasné, že  \x/Daniel/ jejich podvody odhalí a nebude tolerovat, ale nahlásí je králi. Nebyla to jen závist, co je motivovalo k zinscenovanému procesu; potřebovali se ho zbavit, aby sami přežili.

%\renum Da 6:5 = CzeCSP 6:6-6
\ww {zákona Boha jeho}  %BKR
    {zákona jeho Boha} %PSP
    {v zákoně jeho boha}  %CSP
    {zákona jeho Boha} %CEP
    {}  %B21 
    {}  %SNC
\Note 6:5 {zákon}={Zákon jeho Boha}  \x/Daniel/ovi protivníci potvrzují nejen jeho morální integritu, ale také viditelnou a všeobecně známou zbožnost a odevzdanost Bohu Izraele. Tím je znovu připomenuto téma knihy -- Danielova svatost a důvěryhodnost.

\ww {všickni}  %BKR
    {Všichni}  %PSP
    {Všichni}  %CSP
    {Všichni}  %CEP 
    {Všichni}  %B21 
    {Všichni}  %SNC 
\Note 6:7 {všickni} Falešná implikace, že \x/Daniel/ s návrhem souhlasil. Tito úředníci se vůči Dareiovi chovali jako pokrytci; manipulovali s ním, aby dosáhli svých cílů.   

\ww {Kdož by}={kdož by se ... modlil k jakémukoli bohu ... kromě tebe}  %BKR
    {kdo}={kdo ... bude cokoli vyprošovat od kteréhokoli boha ... leč od tebe}  %PSP
    {kdo by}={kdo by ... prosil u jakéhokoliv boha ... jiného než u tebe}  %CSP
    {kdo by}={kdo by se ... obracel v modlitbě na kteréhokoli boha ... kromě na tebe}  %CEP 
    {kdo by}={kdo by se ... modlil k jakémukoli bohu ... kromě tebe}  %B21 
    {který by zakazoval}={který by zakazoval obracet se ... v modlitbě nebo prosbě na kteréhokoliv boha ... krom na tebe}  %SNC 
\Note 6:7 {Kdož by}={kdož by se ... modlil k jakémukoli bohu ... kromě tebe} Návrh Dareiovi mohl připadat spíše politický než náboženský; jeho zřejmý záměr je upevnit vladařovu autoritu nad nedávno dobytými územími. 

\ww {práva}={práva ... kteréž jest neproměnitelné}  %BKR
    {zápis, aby se nedal změnit}  %PSP
    {nepomíjivého zákona}  %CSP
    {nezrušitelného zákona}  %CEP 
    {nepomíjivého zákona}  %B21 
    {nezměnitelného médskoperského zákona}  %SNC 
\Note 6:8 {práva} Perský legislativní systém je první v historii, který vladařovu absolutní moc nějakým způsobem omezuje; v tomto případě nezvratností. Panovník si musel pečlivě rozmyslet, jaký zákon vydá, aby se nemohl obrátit proti němu, protože nebylo možné ho vzít zpátky.
Později král doplatil na svou důvěru svým podřízeným (<14>).
Viz též <Est 1:19>;  <Est 8:8> a <"pozn." Est 7:7>n.

\ww {když se dověděl}  %BKR
    {jakmile se dověděl}  %PSP
    {Když se Daniel dověděl}  %CSP
    {Když se Daniel dověděl}  %CEP 
    {Když se Daniel dozvěděl}  %B21 
    {Když se Daniel o tom dozvěděl}  %SNC 
\Note 6:10 {když se dověděl}  \x/Daniel/ ani na okamžik nezaváhal a nedal se zviklat ve své věrnosti Bohu, přestože věděl, že ho to může stát život (srv. <"pozn." 5:29>n). 

\ww {proti Jeruzalému}  %BKR
    {naproti Jerúsalému}  %PSP
    {směrem k Jeruzalému}  %CSP
    {směrem k Jeruzalému}  %CEP 
    {směrem k Jeruzalému}  %B21 
    {k Jeruzalému}  %SNC 
\Note 6:10 {proti Jeruzalému} Viz <1Kr 8:44>, <1Kr 8:48>; srv. <Ž 5:8> a <Ž 138:2>.       

\ww {třikrát za den}  %BKR
    {třikrát za den} %PSP
    {třikrát denně}  %CSP
    {Třikrát za den} %CEP
    {Třikrát denně}  %B21 
    {třikrát denně}  %SNC
\Note 6:10 {třikrát}={třikrát denně} Viz <Ž 55:18>.     

 
\ww {klekal}  %BKR
    {klekal} %PSP
    {poklekal}  %CSP
    {klekal} %CEP
    {poklekal}  %B21 
    {na kolenou}  %SNC
\Note 6:10 {klekal}={poklekal} Modlitba vestoje byla běžná (<1Pa 23:30>; <Neh 9:1>).      
                               Modlitba vkleče vyjadřuje poníženost, vhodnou zejména za okolností mimořádné vážnosti 
                               (<1Kr 8:54>; <Ezd 9:5>; viz též <Ž 95:6>; <Lk 22:41>; <Sk 7:60>; <Sk 9:40>).  
                               
\ww {činíval}  %BKR
    {činit}  %PSP
    {jako vždy}  %CSP
    {činíval}  %CEP 
    {jako předtím}  %B21 
    {jako dříve}  %SNC 
\Note 6:10 {činíval} \x/Daniel/ova zbožnost byla veřejně známá; proto se nepřátelům hodila jako vítaná záminka intriky proti němu  (<5>).     

\ww {synů Judských}  %BKR
    {z vyhnanců Júdy}  %PSP
    {ze synů judských}  %CSP
    {z judských přesídlenců}  %CEP 
    {židovský vyhnanec}  %B21 
    {judský zajatec}  %SNC 
\Note 6:13 {synů Judských} Identifikace \x/Daniel/a etnickým původem prozrazuje antisemitské předsudky úředníků. Všeobecná známost etnické identity ukazuje, že \x/Daniel/ nezkompromitoval své dědictví ve prospěch prosperity v cizí zemi. To byla důležitá lekce pro původní čtenáře. 

\ww {zarmoutil}  %BKR
    {rozmrzel}  %PSP
    {znechucen}  %CSP
    {znechucen}  %CEP 
    {zarmoutil}  %B21 
    {znechucen}  %SNC 
\Note 6:14 {zarmoutil} \x/Dariov/i okamžitě došlo, že se stal obětí manipulativní  intriky svých  úředníků, ale byl bezmocný s tím něco udělat, protože zákon médský a perský je nezrušitelný (viz <"pozn." 6:8>n).     

\ww {Bůh tvůj}  %BKR
    {Tvůj Bůh}  %PSP
    {Tvůj bůh}  %CSP
    {tvůj Bůh}  %CEP 
    {tvůj Bůh}  %B21 
    {Bůh}  %SNC 
\Note 6:16 {Bůh tvůj}={kéž tě vysvobodí} \x/Darius/ byl nucen proto své vůli vynést rozsudek, k němuž byl zmanipulován. Jedinou naději na \x/Daniel/ovu záchranu vidí v Bohu, jehož \x/Daniel/ uctívá, o jehož všemohoucnosti neměl pochyb. % (<"">)     

\ww {zapečetil}  %BKR
    {zapečetil} %PSP
    {zapečetil}  %CSP
    {zapečetil} %CEP
    {zapečetil}  %B21 
    {Pečeť}  %SNC
\Note 6:17 {zapečetil} Mezi Asyřany, Babylóňany i Peršany byly běžné pečetní prsteny a válcové pečetě pro použití s hlínou nebo voskem. Rozlomit králem označenou pečeť znamenalo porušení zákona.

\Note 6:19 {}={nespal, postil se} \x/Darius/ byl na vrcholu zoufalství. Mohl v tom sehrát roli respekt vůči \x/Daniel/ovu Bohu, nejen obava, že přijde o nejschopnějšího úředníka (<2>).     

\Note 6:22 {anděla}  Dost možná \uv{anděl Hospodinův}  (viz <"pozn." 3:28>n).

\ww {vytáhnouti}  %BKR
    {vytáhnout}  %PSP
    {vytáhnout}  %CSP
    {vytáhli}  %CEP 
    {vytáhnou}  %B21 
    {vytáhli}  %SNC     
\Note 6:23 {vytáhnouti} Vytažením z jámy \x/Darius/ neporušil zákon; ten byl naplněn o den dříve, když tam \x/Daniel/a uvrhli. 

\ww {rozkázal}  %BKR
    {přikázal}  %PSP
    {rozkázal}  %CSP
    {poručil}  %CEP 
    {rozkaz}  %B21 
    {nařídil}  %SNC 
\Note 6:24 {rozkázal} Historik Josephus Flavius líčí tuto epizodu o jeden detail podrobněji. Podle něho úředníci, když viděli, že jim nevyšel cíl sprovodit \x/Daniel/a ze světa, protestovali stížností, že lvy musel někdo předem nakrmit. Král dal tedy lvy před jejich zraky nakrmit masem a teprve pak je tam rozkázal naházet. Pro didaktický záměr autora knihy \x/Daniel/ nebyla tato okolnost důležitá a nestála mu za zmínku; podstatné pro něho bylo zachycení principu, že proti Bohu Izraele (jehož v této generaci reprezentuje \x/Daniel/) nelze bojovat. Srv. <Př 26:27>;  <Ex 14:25-28>; <Ezd 6:6-12>;  <Est 7:9> apod. 

\ww {nařízení}  %BKR
    {nařízení} %PSP
    {nařízení}  %CSP
    {rozkaz} %CEP
    {nařizuji}  %B21 
    {Nařizuji}  %SNC
\Note 6:26 {nařízení} 
     \x/Dariův/ dekret neimplikuje automaticky, že \x/Darius/ konvertoval od svého
     pohanského polyteismu k víře v Danielova Boha, o nic více, než \x/Cýr/ova proklamace, že Bůh mu dal pokyn poslat Židy domů (<Ezd 1:3-4>, <Iz 44:28>, <Iz 45:4>).
     \x/Daniel/ova věrnost navzdory ohrožení života přinesla Bohu slávu po celém \x/Dariov/ě království, tedy po celém známém světě.  

\ww {šťastně}  %BKR
    {dobře}  %PSP
    {dobře}  %CSP
    {dobře}  %CEP 
    {dobře}  %B21 
    {dobře}  %SNC 
\Note 6:28 {šťastně} Opakovaný výskyt jednoho z hlavních témat první poloviny knihy  (srv. <"pozn." 1:1-6:28>n a <3:30>n).     

\ww {království}  %BKR
    {kralování}  %PSP
    {kralování}  %CSP
    {království}  %CEP 
    {za vlády}  %B21 
    {za panování}  %SNC 
\Note 6:28 {království} Lze též  číst jako {\em Dareia, to jest za kralování Kýra}     Verši je možno rozumět dvěma způsoby: (1) Daniel prosperoval pod vládou Gubaru  (viz <"pozn." 1>n) stejně jako pod vládou \x/Cýr/a; anebo (2) \x/Daniel/ prosperoval pod vládou \x/Daria/, čili pod vládou Kýra. Ve druhém případě jsou  \x/Darius/ Médský a \x/Cýr/os Perský dvě jména jednoho a téhož panovníka  (viz <"pozn." 1>n).



\Note 7:1-12:13 {}={Danielovy vize} 
     V těchto kapitolách Daniel opouští historické vyprávění
     a zaznamenává své vize. Tyto vize navazují na předchozích šest kapitol dvěma hlavními tématy: 
     1)~Hospodin, Bůh Izraele, je svrchovaný Pán nade všemi národy a 
     2])~Daniel, nekompromisní Boží prorok, je spolehlivě důvěryhodný. Tyto kapitoly připravují exulanty na dlouhé čekání na plné znovuobnovení Izraele, jakož i na zkoušky a utrpení pod
        nadvládou cizích mocností. Jsou také Božímu lidu povzbuzením, aby se nevzdával naděje,
        že Boží království jednou přijde učinit všemu trápení konec. Daniel se dotýká čtyř
        hlavních témat: 1)~čtyři \x/šelmy/ (<7:1-28>),  
                        2)~beran a kozel (<8:1-27>),
                        3)~\uv{sedmdesát týdnů} (<9:1-27>) a 
                        4)~budoucnost Božího lidu (<10:1>--<12:13>).   


\Note 7:1-28 {}={Vize čtyř \x/šelem/}
     Danielův sen o čtyřech šelmách zachycuje historii střídání cizích
     království, které Izrael utiskovaly, až do doby, kdy jejich pozemská vláda byla dána \uv{\x/svatým výsostí/}

\Note 7:3 {moře} 
     Není zřejmé, zda je míněno nějaké konkrétní moře (snad Středozemní?). Nicméně
     lze mít za to, že moře symbolizuje chaotický neklid, charakteristický pro hříšné národy,
     okupující Izrael. Viz interpretaci ve  <17> a v <Iz 17:12-13> a <Iz 57:20>. 



\ww {čtyři šelmy veliké}   %BKR
   {čtyři převeliká zvířata}   %PSP
   {čtyři veliké šelmy}   %CSP
   {čtyři veliká zvířata}   %CEP
   {čtyři obrovské šelmy}   %B21
   {čtyři dravé šelmy}   %SNC
\Note 7:3 {čtyři šelmy}
    Čtyři \x/šelmy/ reprezentují čtyři království (<17> a <23>).
    Spojitost  s \x/Nabuchodonozor/ovou vizí sochy v kapitole 2 je zřejmá. Pro jejich identifikaci viz náčrt      {\it Danielovy vize\/} na str.~\pgref[danielovyvize]. 

\ww {lvu}={lvu ... orličí křídla}  %BKR
   {lev}={lev ... křídla orla}   %PSP
   {lev}={lev ... orlí křídla}   %CSP
   {lev}={lev ... orlí křídla}   %CEP
   {lev}={lev ... orlí křídla}   %B21
   {lev}={lev ... orlí křídla}   %SNC
\Note 7:4 {lvu}={lev ... orličí křídla} 
     Lev s orlími křídly symbolizuje Babylón (srv.~<Jr 50:44>, <Ez 17:3>).
     Okřídlení lvi byli běžné babylónské artefakty, často umisťované u vchodů významných veřejných budov.      
     
\ww {vytrhána}={vytrhána ... srdce lidské}   %BKR
   {vyrvána}={vyrvána ... srdce člověka}   %PSP
   {utržena}={lidské srdce}   %CSP
   {oškubána}={oškubána ... lidské srdce}   %CEP
   {vylomena}={vylomena ... lidské srdce}   %B21
   {vytržena}={vytržena ... lidské srdce}   %SNC
\Note 7:4 {} Snad odkaz na \x/Nabuchodonozor/ovu
     proměnu a navrácení do lidské společnosti po sedmiletém ponížení nepříčetností
     (\<4:31-34>).

\ww {nedvědu}={nedvědu ... panství jedno vyzdvihla ... tři žebra}    %BKR
   {medvědu}={medvědu ... na jedné straně ... tři žebra}   %PSP
   {medvědu}={medvědu ... k jedné straně ... tři žebra}   %CSP
   {medvědu}={medvědu ... k jedné straně ... tři žebra}   %CEP
   {medvědu}={medvědu ... na jedné straně ... vztyčená ... tři žebra}   %B21
   {medvědu}={medvědu ... na zadních ...  tři žebra}   %SNC
\Note 7:5 {nedvědu}={nedvědu ... panství jedno vyzdvihla ... tři žebra} 
     Médo-perské království je symbolizováno šelmou s nenasytnou žravostí. Vztyčená
     strana může reprezentovat nadřazenou pozici Persie. Tři žebra pravděpodobně znamenají
     vítězství Persie nad Lydií (546 př.Kr.), Babylónem (539 př.Kr.) a Egyptem (525 př.Kr.).
     Viz <"poznámku" 8:3>n.

\ww {pardovi}={pardovi ... čtyři ptačí křídla ... čtyřhlavé}  %BKR
   {levhart}={levhart ... čtyři křídla ptactva ...  čtyři hlavy}   %PSP
   {levhart}={levhart ... čtyři ptačí křídla ... čtyři hlavy}    %CSP
   {levhart}={levhart ... čtyři ptačí křídla ... čtyřhlavé}    %CEP
   {pardálu}={pardálu ... čtyři ptačí křídla ... čtyři hlavy}    %B21
   {levhartovi}={levhartovi ... čtyři ptačí křídla ... čtyři hlavy}    %SNC
\Note 7:6 {pardovi}={pardovi ... čtyři ptačí křídla ... čtyřhlavé}
     Řecko je symbolizováno \x/pard/em, proslulým svou rychlostí.
     Alexandr Veliký (356--323 př.Kr.) dobyl Persii velmi rapidně.
     Střetl se s Peršany ve třech velkých bitvách:
     1) bitvou u řeky Gráníkos (334 př.Kr.) získal vstup do Malé Asie; 
     2) bitva u Issu (333 př.Kr.) mu umožnila okupovat Sýrii, Kenaán a Egypt; 
     3) v~bitvě u~Gaugamél porazil perskou armádu definitivně a otevřel si cestu do Indie.
        Viz též \<8:5-8>. Krátce po jeho předčasné smrti (ve věku 32 let, zřejmě na malárii, kterou se nakazil v Egyptě) se říše, kterou
        vytvořil, rozpadla na čtyři části.  Protože nezanechal dědice, kterému by připadla, rozebrali si ji jeho generálové: v Makedonii vládl Kassandros, v Thrákii a Malé Asii Lýsimachos, v Sýrii Seleukos a v Egyptě Ptolemaios, nikoli však v míru a pokoji, nýbrž v řevnivosti a touze vládnout všemu jako Alexandr.

\ww {šelma čtvrtá}   %BKR
   {čtvrté zvíře}   %PSP
   {čtvrtá šelma}   %CSP
   {čtvrté zvíře}   %CEP
   {čtvrtou šelmu}   %B21
   {čtvrtou šelmu}   %SNC
\Note 7:7 {šelma čtvrtá}
     Historie nás učí, že toto neidentifikované zvíře je Řím --- impérium, které postupně
     asimilovalo různé části rozděleného řeckého království. 
     {\bf deset rohů} Znamenají deset římských králů (viz v. <24>). Není zřejmé, zda následují po      sobě nebo vládnou současně. Pro spekulativní domněnku, že mají znamenat druhou fázi čtvrtého    království, \uv{oživenou říši římskou} posledních dnů, však jednoznačně přesvědčivý textový      důkaz neexistuje. 

\ww {roh poslední malý}   %BKR
   {další, malý}   %PSP
   {další roh, nepatrný}   %CSP
   {další malý roh}   %CEP
   {jiný, malý roh}   %B21
   {roh další}   %SNC
\Note 7:8 {roh poslední}={roh poslední ... tři z dřívějších rohů byly před ním vyvráceny}
     Deset rohů časově předchází \uv{malému,} který vyvrátí tři z~nich. Je to další fáze
     čtvrtého království. Mnozí mají za to, že malý roh symbolizuje vzestup Antikrista 
     <2Te 2:3-4>. Pokud je tomu tak, pak je toto první zmínka o Antikristu v Písmu.  

\ww {oči podobné očím lidským}   %BKR
   {oči jako oči člověka}   %PSP
   {oči jako oči lidské}   %CSP
   {oči jako oči lidské}   %CEP
   {akoby lidské oči}   %B21
   {lidské oči}   %SNC
\Note 7:8 {oči podobné očím lidským}{oči podobné očím lidským ... a ústa}
     Metafora naznačuje, že roh reprezentuje spíše člověka než království. 
     
\ww {Starý dnů}   %BKR
   {Pradávný dní}   %PSP
   {Věkovitý}   %CSP
   {Věkovitý}   %CEP
   {Odvěký}   %B21
   {Věčný}   %SNC
\Note 7:9 {Starý dnů} Jediný výskyt v Bibli je v této kapitole (srv. <13>, <22>). Podobný výraz se objevuje v ugaritských textech k označení velkého Boha  {\em El}. Zde je použit jako označení pro Boha, který zasedl k soudu, a implikuje, že je věčný a  panuje od pradávna.

\ww {roucho}={roucho ... vlasy}   %BKR
   {oblečení}={oblečení ... vlas}   %PSP
   {Roucho}={Roucho ... vlasy}    %CSP
   {oblek}={oblek ... vlasy}    %CEP
   {Roucho}={Roucho ... vlasy}    %B21
   {oděv}={oděv ... vlasy}    %SNC
\Note 7:9 {roucho}={roucho ... vlasy} Ačkoliv se Bůh Danielovi zjevil v neskutečné slávě, přesto to bylo v podobě rozpoznatelné jako lidská.

\ww {trůn}={trůn ... kola} %BKR
    {trůn}={trůn ... kola} %PSP
     {trůn}={trůn ... kola}  %CSP
     {stolec}={stolec ... kola} %CEP
     {trůn}={trůn ... kola}  %B21 
     {trůn}={trůn ... kolech}  %SNC
\Note 7:9 {trůn}={trůn ... kola} Vyobrazení Božího trůnu koreluje s vizí proroka Ezechiele (<Ez 1:15-28>).
      Nebeský trůn je zobrazen s koly (podobně jako v památkách jiných národů z oné doby) --- jako královský válečný vůz. Podobný motiv se skrývá za ohnivým sloupem, který vedl Izrael během Exodu (<Ex 13:21-22>).

\ww {knihy}  %BKR
    {knihy} %PSP
    {knihy}  %CSP
    {knihy} %CEP
    {knihy}  %B21 
    {soudní spisy}  %SNC
\Note 7:10 {knihy}={otevřeny knihy} Viz  <12:1> (viz též <Ex 32:32>; <Ž 149:9>; <Iz 4:3>; <Iz 65:6>;  <Mal 3:16>; <Lk 10:20>; <Zj 5:1-5>; <Zj 6:12-16> a <Zj 20:12>).

\Note 7:11-12 {}  Určitá lhůta je ponechána předchozím královstvím, jejichž obyvatelé i se svými zvyklostmi byli absorbováni následujícími říšemi. V kontrastu s tím je vyjádřen důraz na totální zkázu čtvrtého království. Srv. <"pozn." 3>n

\ww {Synu člověka}   %BKR
   {syn člověka}   %PSP
   {synu člověka}   %CSP
   {Syn člověka}   %CEP
   {synu člověka}   %B21
   {Syn člověka}   %SNC
\Note 7:13 {Synu člověka} Tento termín může znamenat jednoduše \uv{člověk}. Hebrejský
   ekvivalent tohoto aramejského výrazu je v <Da 8:17>  použit pro Daniele, stejně jako pro jeho současníka    Ezechiele v <Ez 2:1>, <3> a <6>. 
   Daniel je jedním z prvních (ne-li  vůbec první), kdo používá toto spojení.  Pozdější židovská mezizákonní apokalyptická
   literatura navazuje na tuto pasáž a vykresluje \uv{syna člověka} jako nadpřirozenou bytost,
   přinášející nebeskou moc na Zem. Daniel viděl {\em podobného Synu člověka}, tedy někoho srovnatelného s člověkem, a přesto výrazně odlišného (\<14>). V evangeliích  je výraz \uv{Syn člověka} používán ve vztahu ke Kristu (69 výskytů v synoptických evangeliích a 12 v Janově). Ježíš sám sebe nejčastěji označoval právě tímto titulem.

\ww {s oblaky nebeskými}  %BKR
    {s oblaky nebes} %PSP
    {s oblaky nebes}  %CSP
    {s nebeskými oblaky} %CEP
    {s nebeskými oblaky}  %B21 
    {na nebeských oblacích}  %SNC
\Note 7:13 {s oblaky} Jinde ve SZ je to jedině Hospodin, o němž je řečeno, že se objevuje na oblacích  (<Ž 104:3>; <Iz 19:1>). Podobný Synu člověka pochází z nebes, sestupuje z Božího rozhodnutí; je totožný se skálou, vytrženou z hory, avšak nikoliv lidskou rukou  (\<2:45>; viz \<"pozn." 14>n).

\ww {dáno jest jemu panství}   %BKR
   {byla mu dána vláda}   %PSP
   {Byla mu dána vláda}   %CSP
   {byla mu dána vladařská moc}   %CEP
   {Byla mu dána vláda}   %B21
   {Byla mu předána vláda}   %SNC
\Note 7:14 {panství}={Byla mu dána vláda} Bůh mu svěřuje \uv{místodržitelství} nade všemi národy. Tím je naplněna role skály, vytržené z hory (<2:44-45>).

\ww {všickni lidé}   %BKR
   {všechny národnosti}   %PSP
   {všichni lidé}   %CSP
   {všichni lidé}   %CEP
   {všichni lidé}   %B21
   {lidé všech národností}   %SNC
\Note 7:14 {všickni lidé}={všichni lidé ... nebude zničeno} \uv{Syn člověka}, kterého \x/Daniel/ viděl, je veliký syn Davidův, Mesiáš. \x/Izaiáš/ také líčí jeho království jako nikdy nekončící  (<Iz 9:7>). Ježíš toto mesiánské spojení potvrzuje narážkou na právě tuto pasáž. Z toho důvodu byl náboženskými vůdci své doby nařčen z rouhání (<Mt 26:64-65>; <Mk 14:62-64>). Kdo slouží jemu, slouží Bohu.

\ww {zhrozil se}   %BKR
   {byl zraněn}   %PSP
   {jsem se znepokojil}   %CSP
   {zmatený}   %CEP
   {rozrušilo}   %B21
   {vyděsilo}   %SNC
\Note 7:15 {zhrozil}={děsilo mne} \x/Daniel/ byl zděšen tím, co viděl (<28>), přesto se dožaduje vysvětlení nesrozumitelných jevů (<16>, <19>). Je to od něho projev odvahy: touhy znát pravdu navzdory tušení, že odhalená záhada bude ještě děsivější (<21>). Milovat pravdu, ať je jakákoliv; hledat ji, ať je kdekoliv, je známka vysokého stupně zralosti. Zavírat před ní oči jen proto, že je nepříjemná, je dětinské, nedospělé, a v důsledku vede k paralyzující neschopnosti se na těžké časy připravit. 

\ww {svatých}   %BKR
   {svatí}   %PSP
   {svatí}   %CSP
   {svatí}   %CEP
   {svatí}   %B21
   {svatí lidé}   %SNC
\Note 7:18 {království svatých}={svatí}  Viz <21-22>, <25> a <27>. Nebudou to andělé, nýbrž věrní věřící, komu bude svěřeno království (srv. <1Kor 6:1-11>; <2Tm 2:12>; <Zj 22:5>). 

\ww {výsostí}   %BKR
   {Nejvyššího}   %PSP
   {Nejvyššího}   %CSP
   {Nejvyššího}   %CEP
   {Nejvyššího}   %B21
   {Nejvyššímu}   %SNC
\Note 7:18 {výsostí}={Nejvyššího} Mezi \uv{Synem člověka} coby Králem (<13-14>) a \uv{svatými Nejvyššího} coby těmi, kdo budou mít podíl na jeho království, je úzká spojitost (<22>, <27>).

\ww {na věky}   %BKR
   {po věčnost}   %PSP
   {na věky}   %CSP
   {na věky}   %CEP
   {na věky}   %B21
   {navěky}   %SNC
\Note 7:18 {na věky}  Viz <6:26>;  <7:14> a <"jeho pozn.">n.

\ww {válku vedl s svatými, a přemáhal je}   %BKR
   {vedl boj se svatými a přemáhal je}   %PSP
   {vede válku se svatými a jak je přemáhá}   %CSP
   {vedl válku proti svatým a přemáhal je}   %CEP
   {vede válku proti svatým a přemáhá je}   %B21
   {pronásledoval svaté věřící a měl úspěch}   %SNC
\Note 7:21 {válku}={válku proti svatým ... přemáhá} Než se Boží věrní ujmou věčné vlády nad světem  (<18>, <27> <Mt 5:5>), budou vystaveni pronásledování, které prověří jejich víru  (<Zj 13:7-10>; <Zj 14:12>).

\ww {Starý dnů}   %BKR
   {Pradávný dní}   %PSP
   {Věkovitý}   %CSP
   {Věkovitý}   %CEP
   {Odvěký}   %B21
   {Věčný}   %SNC
\Note 7:22 {Starý dnů} Přestože malý roh  (<8>) bude mít nějaký čas k rouhání (<25>) a pronásledování svatých (<21>), nakonec podlehne Božímu soudu (<Za 14:1-4>; <Zj 19:2>).

\ww {království}   %BKR
   {kralování}   %PSP
   {království}   %CSP
   {království}   %CEP
   {království}   %B21
   {království}   %SNC
\Note 7:22 {království} Boží zásahy do historie směřují k tomu, co NZ nazývá \uv{královstvím Božím}  (viz čl. <"Království Boží" Mt 4>a).

\ww {času a časů, i do částky časů}   %BKR
   {dobu a doby a půl doby}   %PSP
   {času a časů a půl času}   %CSP
   {času a časů a poloviny času}   %CEP
   {čas a časy a půl času}   %B21
   {tři a půl let}   %SNC
\Note 7:25 {času}={času a časů a půl času} Slovo pro \uv{čas} je stejné jako v <4:13> a <4:20>, kde může znamenat jeden rok (viz <"pozn." 4:13>n a <Zj 12:14>n). Nejjistější zřejmě bude považovat toto časové určení za dobu, kterou  Boží intervence zkrátí ve prospěch svého lidu (<Mt 24:22>).
\dopsat hebr.

\Note 7:26 {soud} Nebeský soud (viz <10>). Nebesa jsou stvořené místo, Boží soudní síň, nikoliv Boží \uv{bydliště} (<1Kr 8:27>). 

\ww {lidu svatých}  %BKR
    {lidu svatých} %PSP
    {lidu svatých}  %CSP
    {lidu svatých} %CEP
    {svatému lidu}  %B21 
    {svatým lidem}  %SNC
\Note 7:27 {lidu svatých} Viz  <"pozn." 18>n.

\ww {velice zkormoutila}  %BKR
    {velmi zneklidňovaly} %PSP
    {zcela vyděšen}  %CSP
    {velice zhrozil} %CEP
    {zbledl hrůzou}  %B21 
    {Vyděsilo}  %SNC
\Note 7:28 {Daniel}={vyděšen} Myšlenky o Izraeli, který nepřestal hřešit ani v zajetí a vykoledoval si tak sedminásobný trest pod vládou cizích království   (srv. <"pozn." 9:11>n), \x/Daniel/a deptaly (<8:27>), přestože věděl, že ultimátní závěr dějin vyústí ve vítěznou vládu Božích věrných  (srv. \<"pozn." 18>n).

\ww {v srdci svém}   %BKR
   {ve svém srdci}   %PSP
   {v srdci}   %CSP
   {ve svém srdci}   %CEP
   {v srdci}   %B21
   {do srdce}   %SNC
 \Note 7:28 {v srdci} \x/Daniel/ zmiňuje tento detail, aby zdůraznil, že si nelibuje v takové představě apokalyptických vizí.  Ani s autoritou nad pohany u Babylonského i Perského královského dvora nemohl být osočen ze zrady věrnosti Božímu lidu. O budoucích událostech hovořil zkroušeně a s lítostí. Nemnul si ruce nad ztrestanými nevěrnými ani nad zahynuvšími nepřáteli. Tímto postojem srdce je příkladem všem.
 
\Note 8:1-12:23 {} \x/Daniel/ se v poledních 5 kapitolách vrací k hebrejštině. Text <2:4-7:28>
je v Aramejštině (viz <"pozn." 2:4>n).
   
\ww {Léta třetího kralování Balsazara}   %BKR
   {V třetím roce kralování Bélšaccara}   %PSP
   {Ve třetím roce kralování krále Belšasara}   %CSP
   {V třetím roce kralování krále Belšasara}   %CEP
   {Ve třetím roce vlády krále Belšasara}   %B21
   {Ve třetím roce panování krále Belšasara}   %SNC
\Note 8:1 {Léta třetího kralování Balsazara}
     Viz <"pozn." 5:1>n. Není jednoznačně jisté, zda \x/Balsazar/ova
     spoluvláda s Nabonidem začala zároveň s Nabonidovým nástupem (556 př.Kr.) nebo o něco
     později. Ať tak či onak, události páté a osmé kapitoly je nutno chronologicky umístit
     mezi události kapitol 4 a 5. 

\Note 8:1 {vidění}={ukázalo se vidění} \x/Daniel/ zakouší podobnou \uv{cestovní} vizi jako prorok \x/Ezechiel/ (<Ez 3:12-15>).

\ww {v Susan}   %BKR
   {v Šúšánu}   %PSP
   {v opevnění Šúšanu}   %CSP
   {na hradě Šúšanu}   %CEP
   {Súsách}   %B21
   {Šúšanu}   %SNC
\Note 8:2 {Susan} V \x/Daniel/ově době byl \x/Susan/ hlavním městem území jménem \x/Elam/,  
asi 370km východně od Babylóna. Není zřejmé, zda tehdy byla  tato země  nezávislá nebo spojena s Babylonem či Médeou. Později se však stal diplomatickým a administrativním hlavním městem Perského impéria (srv. <Est 1:2>; <Neh 1:1>).


%\note 8:2 {}={} Kanál, spojující řeky Dez a Karcheh, obtékající  \x/Susan/, dnešní Shush, každá z jedné strany, a ústící do Perského zálivu. 
%to je špatně Karkheh ústí do

\ww {dva rohy}  %BKR
    {dva rohy} %PSP
    {dva rohy}  %CSP
    {dva rohy} %CEP
    {dvěma rohy}  %B21 
    {dvěma rohy}  %SNC
\Note 8:3 {dva rohy} Verš <20> identifikuje berana se dvěma rohy jako krále Médského a Perského království.

\ww {vyšší}  %BKR
    {vyšší} %PSP
    {vyšší}  %CSP
    {větší} %CEP
    {delší}  %B21 
    {větším}  %SNC
\Note 8:3 {vyšší}={vyšší ... později} Symbolismus pomáhá objasnit Médo-Perská historie. Médové se stali mocnými a nezávislými na Asýrii po r. 631 př.Kr. Peršané začali jako nevýznamná část médského království, ale chopili se nadvlády, když si \x/Cýr/os (559--530 př.Kr.) z Anšanu (v \x/Elam/u) podmanil Médeu. Tehdy připojil ke svým titulům i \uv{Král Médů}. Proto oba rohy jsou vysoké, ale vyšší reprezentuje mocnější Persii, a vyrostl později, protože Persie se dostala k moci později než Médea.

\ww {trkal k západu, půlnoci a poledni}   %BKR
   {trkajícího k západu a k severu a k jihu}   %PSP
   {trkal na západ a na sever a na jih}   %CSP
   {trkat směrem k moři, na sever a na jih}   %CEP
   {trkat na západ, na sever i na jih}   %B21
   {trkat}={trkat ... na sever -- směrem k moři -- a na jih}   %SNC
\Note 8:4 {trkal}={na západ, na sever a na jih} \x/Cýr/os nejprve dobyl Malou Asii, poté severní i jižní Mezopotámii. Panovníci, následující po něm, rozšířili Médo-Perskou vládu daleko na Východ. 

\ww {veliké}   %BKR
   {stal se velikým}   %PSP
   {vyvýšil se}   %CSP
   {vzmohl se}   %CEP
   {stále mohutněl}   %B21
   {velice zmohutněl}   %SNC
\Note 8:4 {veliké}={stal se velikým} Perské impérium se stalo rozlehlejším a mocnějším než kterákoliv předchozí říše v dějinách Blízkého Východu.

\Note 8:5 {kozel}  <"Verš" 21> identifikuje kozla jako Řecko a roh mezi očima jako jeho prvního krále. Symbolismus výstižně zachycuje rozmach Řeckého impéria pod vedením Alexandra Velikého (356--323 př.Kr.). 

\Note 8:5 {}={sotva se země dotýkal} Obraz postihuje rychlost, s jakou Alexandr dobýval cílová území   (viz <"pozn." 7:6>n). Mocnou Persii (viz <"pozn."4>n) dokázal dobýt během pouhých tří let.

\ww {velikým}   %BKR
   {nesmírně velikým}   %PSP
   {se nadmíru vyvýšil}   %CSP
   {se velice vzmohl}   %CEP
   {mohutněl}   %B21
   {velice zmohutněl}   %SNC
\Note 8:8 {velikým}={nesmírně velikým} Alexandrovo impérium brzy překonalo perskou říši i rozlohou. V roce 327 př.Kr. pokrývalo území dnešního Afghánistánu a později sahalo až do údolí Indus.

\Note 8:8 {roh}={roh se zlomil}  Když Alexandrova armáda odmítla pokračovat dále na východ, vrátil se do Babylóna, kde zemřel ve věku třiatřiceti let. 

\ww {čtyři místo něho, na čtyři strany světa}   %BKR
   {místo něho vzrostly čtyři nápadné rohy ke čtyřem větrům nebes}   %PSP
   {místo něho vyrostly viditelné čtyři, a to na čtyři strany světa}   %CSP
   {místo něho vyrostly čtyři nápadné rohy do čtyř nebeských větrů}   %CEP
   {místo něj vyrostly do čtyř světových stran čtyři jiné velkolepé rohy}   %B21
   {na jeho místě vyrostly čtyři významné rohy -- každý na jinou stranu}   %SNC
\Note 8:8 {čtyři místo něho, na čtyři strany světa}  <"Verš" 22> napovídá, že tyto rohy symbolizují čtyři království, na které se rozdělí Alexandrova říše. Historické záznamy dokládají, že po určité době vnitřních tahanic  se čtyři Alexandrovi generálové podělili o bývalé řecké impérium. Viz <"pozn." 7:6>n.

\Note 8:9 {roh}={nepatrný roh}  <"Verš" 23> naznačuje, že rok symbolizuje zlého krále, který povstane z jednoho ze čtyř řeckých království po delším čase (\uv{koncem jejich kralování}). 
Líčení skutků tohoto krále (<"vv." 9-14>, <23-25>) identifikuje jako Antiocha IV. Epifana z dynastie
Seleukovců, vládnoucího mezi 175 a 164 př.Kr. Nezaměnit s malým rohem ze <7:8>, který povstane až za římského období, po řeckém. 

\ww {k zemi Judské}   %BKR
   {k okrase}   %PSP
   {k Ozdobě}   %CSP
   {k nádherné zemi}   %CEP
   {k Nádherné zemi}   %B21
   {k nádherné zemi}   %SNC
\Note 8:9 {k zemi Judské} Dosl. k Nádherné (Slavné, Ctnostné, Ozdobné) --- rozumí se (implicitně) Zemi Zaslíbené. Antiochos IV. Epifanés ve snaze helenizovat svá území zakázal židovské náboženství včetně chrámových obřadů, čtení Tóry, zachovávání Sabatu i obřízku; Židé byli dokonce nuceni účastnit se helénským modloslužeb. K aktivnímu odporu se odhodlali po hrdinském činu starého kněze Matatiáše, který neuposlechl rozkaz obětovat modle a namísto toho zabil královského úředníka i Žida, ochotného oběť vykonat. (To nám mj. připomíná, že pronásledování bezbožnou vládou si lidé vykoledují ani ne tak tím, co by dělali, jako spíše tím, co udělat odmítnou.)
Tím odstartoval vzpouru Makabejských, vylíčenou v deuterokanonické knize 2~Makabejská a v jejím románovém zpracování Howarda Fasta {\em Moji stateční bratři.}
Svátek Chanuka Židé dodnes slaví na památku vítězství tohoto povstání, jemuž se roku 164 př.Kr. skutečně podařilo osvobodit Jeruzalém ode všech helenizátorů a znovu posvětit chrám.
\dopsat reference a přehled svátků, sladit s Úvodem


\ww {svrhl}   %BKR
   {posrážel}   %PSP
   {shodil}   %CSP
   {srazil}   %CEP
   {strhl}   %B21
   {zmocnil se}   %SNC
\Note 8:10 {svrhl}={strhl na zem ... a pošlapal je} Symbolické zachycení krutého pronásledování Božího lidu za Antiocha IV. Epifana. Viz <"pozn." 8:9>n a Úvod: Záměr a zvláštnosti. Srv. <11:21-35>; <1Mak 1:10-64>. %1 Makabejská, sz apokryf 

\ww {z hvězd}   %BKR
    {z hvězd}   %PSP
    {z hvězd}   %CSP
    {hvězd}   %CEP
    {hvězdy}   %B21
    {některých jeho představitelů}   %SNC
\Note 8:10 {některé}={z hvězd} Hvězdy symbolizují Boží lid; srv. <12:3>; <Jr 33:22>; <Gn 12:3>;  <Gn 15:5> a/nebo nebeskou armádu andělů %(<Ex 12:41>)
(<Iz 14:13>; viz též <2Mak 9:10>). % 2 Makabejská, sz apokryf
Antiochovy mince ho zobrazují s hvězdou nad hlavou. Útok na lid Boží znamená útok proti nebi.


\ww {knížeti}   %BKR
   {Knížeti}   %PSP
   {veliteli}   %CSP
   {veliteli}   %CEP
   {Knížeti}   %B21
   {veliteli}   %SNC
\Note 8:11 {knížeti} Zde označení pro Boha, Pána zástupů (hebr. \uv{armád}\dopsat hebr.; srv <Iz 6:3>). Ve verši \<25> je formulace \uv{kníže knížat}. Antiochos IV. si osvojil přídomek Epifanés (\uv{Bůh zjevený}) a považoval sám sebe za manifestaci Dia, hlavního boha řeckého panteonu. 

\ww {zastavena}   %BKR
   {odňal}   %PSP
   {odebral}   %CSP
   {zrušil}   %CEP
   {zrušil}   %B21
   {zrušil}   %SNC
\Note 8:11 {zastavena}={zrušil každodenní oběť} Viz (<"vv." 12-13>; <11:31> a <"pozn." 8:9>n.)

\ww {svatyně}  %BKR
    {svatyně} %PSP
    {svatyně}  %CSP
    {svatyně} %CEP
    {svatyni}  %B21 
    {svatyni}  %SNC
\Note 8:11 {svatyně}={zpustošil svatyni} Antiochos IV. nejen vstoupil do Nejsvětější Svatyně jeruzalémského chrámu a vydrancoval z ní zlaté a stříbrné nádoby, ale dokonce na Hospodinově oltáři na chrámovém nádvoří vztyčil oltář Diův a obětoval na něm prase (viz <"pozn." 11:31>n).

\ww {vojsko to vydáno v převrácenost proti ustavičné oběti}   %BKR
   {proti ustavičné oběti bylo v bezbožnosti vypravováno vojsko}   %PSP
   {nad tou soustavnou bohoslužbou bylo věrolomně dáno vojsko}   %CSP
   {Zástup byl sveden ke vzpouře proti každodenní oběti}   %CEP
   {Vojsko i každodenní oběť mu při té vzpouře byly vydány do rukou}   %B21
   {Vojsko zlákal k odboji proti stálé službě}   %SNC
\Note 8:12 {vojsko to vydáno v převrácenost proti ustavičné oběti} Boží lid je vydán v moc rohu, který začal jako malý (<"v." 9>), tedy Antiocha IV. Zákaz chrámové bohoslužby lze od bezbožného vladaře očekávat.

\ww {šťastně mu se dařilo}   %BKR
   {měl úspěch}   %PSP
   {úspěšně se prosadilo}   %CSP
   {dařilo se mu, co činil}   %CEP
   {dařilo se mu vše, co podnikal}   %B21
   {cokoli činil, v tom měl úspěch}   %SNC
\Note 8:12 {šťastně mu se dařilo} Úspěch rohu, který začal jako malý (<"v." 9>, Antiocha IV.)
obnášel i zničení kopií hebrejských Písem  (<1Mak 1:56-57>). 

\Note 8:14 {}={2300 večerů a jiter} Fráze \uv{večerů a jiter} se v celém SZ objevuje pouze zde a ve <"verši" 26>. Lze ji chápat jako odkaz na večerní a ranní oběti (srv. <Ex 29:38-42>); v takovém případě by to znamenalo dobu 1150 dní. Jiné pohledy považují frázi za prosté označení 2300 dní. Vzhledem k tomu, že počátek pronásledování Antiochem IV. lze spojit s jakýmkoliv počtem incidentů již od 171 př.Kr., je těžké jednoznačně rozhodnout, ke kterému chápání fráze se přiklonit.
Číslo 23 může symbolizovat pevně stanovené období, podobně jak je tomu v mimobiblické apokalyptické literatuře.

\Note 8:14 {}={pak bude svatyně očištěna}  Chrám byl vyčištěn a znovu zasvěcen pod vedením Judy Makabejského 25. prosince 165 př.Kr. (viz <"pozn." 11:34>n; srv. <Za 9:13-17>).

\ww {Gabrieli}   %BKR
   {Gavríéli}   %PSP
   {Gabrieli}   %CSP
   {Gabrieli}   %CEP
   {Gabrieli}   %B21
   {Gabrieli}   %SNC
\Note 8:16 {Gabrieli} Tento anděl je v Písmu zmíněn jménem čtyřikrát (<9:21>; <Lk 1:11>, <19>, <26>). Jméno znamená \uv{Bůh je moje síla}.

\ww {synu člověčí}  %BKR
    {synu člověka} %PSP
    {lidský synu}  %CSP
    {lidský synu} %CEP
    {lidský synu}  %B21 
    {Danieli}  %SNC
\Note 8:17 {synu}={synu člověka} Viz <"pozn." 7:13>n. \uv{Silný Boží} (viz \<"pozn." 16>n) promlouvá k vznešenému smrtelníkovi.

\ww {v času uloženém}   %BKR
   {o čase konce}   %PSP
   {čas konce}   %CSP
   {doby konce}   %CEP
   {poslední čas}   %B21
   {času konce}   %SNC
\Note 8:17 {času}={vidění času konce} Viz též <"v." 19>. Vazba nemusí nutně znamenat úplný závěr historie; Ve verších <11:27> a <11:35> je podobné spojení umístěno v kontextu, který pravděpodobně odkazuje ke konci pronásledování pod Antiochem IV.

\ww {hněvu}   %BKR
   {rozhorlení}   %PSP
   {rozhořčení}   %CSP
   {hněvu}   %CEP
   {hněvu}   %B21
   {rozběsní}   %SNC
\Note 8:19 {hněvu}={koncem hněvu} \uv{Čas hněvu} může znamenat období Božího hněvu, namířeného proti   Izraeli, vykonaného nadvládou Babylóňanů, Peršanů a Řeků. 

\ww {Skopec}   %BKR
   {beran}   %PSP
   {beran}   %CSP
   {beran}   %CEP
   {beran}   %B21
   {beran}   %SNC
\Note 8:20 {Skopec} Viz <"pozn." 3>n a <4>n.

\ww {Kozel}={kozel ... roh}   %BKR
   {kozel}={kozel ... roh}   %PSP
   {kozel}={kozel ... roh}   %CSP
   {kozel}={kozel ... roh}   %CEP
   {kozel}={kozel ... roh}   %B21
   {kozel}={kozel ... roh}   %SNC
\Note 8:21 {Kozel}={kozel ... roh} Viz <"pozn." 5>n a <6>n.

\Note 8:22 {čtyři} Viz  <"pozn." 8>n.

\Note 8:23-25 {} Viz  <"pozn." 9-14>n. Někteří vykladači spatřují v popisu rohu v této kapitole (<"v." 8>)
obraz Antikrista v podobě Antiocha IV. coby typu mocného oponenta Božího lidu v budoucnosti.  
\dopsat typologii někam do článku

\Note 8:25 {mnohé} Tj. věrné Židy.

\ww {proti knížeti knížat}   %BKR
   {proti Knížeti knížat}   %PSP
   {proti Veliteli velitelů}   %CSP
   {proti Veliteli velitelů}   %CEP
   {proti Knížeti knížat}   %B21
   {nejvyššímu Veliteli}   %SNC
\Note 8:25 {knížeti}={proti knížeti knížat} Odkaz na Boha.

\Note 8:25 {}={bez přispění lidské ruky bude zničen} Antiochos IV. nebyl zavražděn ani nepadl v bitvě. Jeho smrt v roce 164 př.Kr. nastala jako důsledek nervové choroby. Záznamy jeho smrti jsou v <1Mak 6:1-16> a <2Mak 9:1-28>.

\ww {zavři to vidění} %BKR
    {to vidění ukryj}  %PSP
    {to vidění uzavři}  %CSP
    {podrž to vidění v tajnosti}  %CEP
    {Ty ho však zachovej v tajnosti}  %B21
    {Ty je však zapečeť}  %SNC
\Note 8:26 {zavři to vidění} Pečeť byla používána k jednomu ze dvou účelů: (1) Buďto jako certifikát autentičnosti (\uv{Toto je opravdu královo nařízení!}) anebo (2) k zabezpečení  důvěrného dokumentu. Tento druhý význam nejlépe odpovídá kontextu (srv. <"pozn." 6:18>n).


%     BKR                       PSP                         CSP                                 CEP                                         B21                                    SNC
\ww {nebo jest mnohých dnů}    
    {neboť je na mnohé dni}      
    {je totiž na velmi dlouho}      
    {neboť se uskuteční za mnoho dnů}      
    {neboť se týká vzdálené budoucnosti}      
    {protože se týká vzdálené budoucnosti}    
\Note 8:26 {nebo jest mnohých dnů} Dobyvačné tažení Alexandra Velikého (333--323 př.Kr.) začalo téměř dvě století po \x/Daniel/ových vizích (cca. 500 př.Kr.); Antiochos IV. vládl století a půl po Alexandrovi (171--164 př.Kr.). 

\ww {Daria} %BKR %nepotřebujem, je to \vdef
    {Dárjáveše}   %PSP
    {Dareia}   %CSP
    {Darjaveše}   %CEP
    {Darjaveše}   %B21
    {Darjaveš}   %SNC
\Note 9:1 {Daria} Viz <"pozn." 5:31>n a <6:1>n. První rok \x/Dariov/y vlády byl 539 př. Kr.

\ww {Asverova} {Achašvéróšova} {Achašvérošova} {Achašvérošova} {Ahasverova} {Achašvéroše}
\Note 9:1 {Asverova} Nezaměnit s králem z <Est 1:1>.  Hebrejská odvozenina původního perského Xerxes (\uv{král všech mužů} nebo \uv{hrdina mezi králi}). Mohl to být královský titul, a ne vlastní jméno. 

\ww {Jeremiášovi}   %BKR
   {Jeremjovi}   %PSP
   {Jeremjášovi}   %CSP
   {Jeremjášovi}   %CEP
   {Jeremiášovi}   %B21
   {Jeremjáše}   %SNC
\Note 9:2 {Jeremiášovi}={Jeremiáš ... sedmdesátého léta} Viz <Jr 25:11> a <Jr 29:10>.
        Sedmdesát let lze považovat buďto za zaokrouhlený věk jednoho lidského života, anebo za přesné časové určení. Někteří toto období datují od 586 př.Kr. (zničení jeruzalémského chrámu \x/Nabuchodonozor/em) do 515 př.Kr. kdy byl dokončena rekonstrukce chrámu 
        \x/Zorobábel/em (srv. <"pozn." 26>n). Jiní považují za začátek 70 let první rok \x/Daniel/ova zajetí (604 př.Kr., srv. <"pozn." 1:1>n). 
        Není pochyb, že \x/Daniel/ byl obeznámen s \x/Izaiáš/ovým proroctvím o propuštění Izraele z otroctví pohanským vladařem jménem  \x/Cýr/os (<Iz 44:28>; <Iz 45:1-13>).
        Viz čl. <"\x/Cýr/os je můj pastýř" Iz 44>a.
        Daniel považuje \x/Cýr/ovo propuštění za rok naplnění \x/Jeremiáš/ova proroctví, podobně jako autor kniha Paralipomenon \dopsat [větvení pro Letopisů]
        (<2Pa 36:22>). V literatuře starého Blízkého Východu bylo 70 let standardní dobou trestu božstva nad neposlušným lidem. Tento čas mohl být prodloužen nebo zkrácen podle reakce lidí (viz <Jr 18:7-10>; viz též <"Úvod k prorockým knihám">i). Proto  určitá flexibilita  v aplikaci číslice 70 různými biblickými autory není překvapivá.
        
        
\ww {modlitbou}  %BKR
    {modlitbou} %PSP
    {modlitbou}  %CSP
    {modlitbou} %CEP
    {v modlitbách}  %B21 
    {modlil}  %SNC
\Note 9:3 {modlitbou}={modlitbou, v postu, žíni a popelu} \x/Daniel/ byl zděšen, protože věděl, že Izrael byl 70 let v zajetí za trest pro své hříchy, ale ani po 70 letech se od svých hříchů neodvrátil. Srv. <"pozn." 11>n.

\Note 9:4-19 {} \x/Daniel/ova modlitba vyrůstá z chápání vztahu s Bohem jako smluvního (požehnání za poslušnost a prokletí za neposlušnost; viz zejména <"vv." 5>, <7>, <11-12>; <Lv 26:14-45>; <Dt 28:15-68>; <Dt 30:1-5>). Podobnou modlitbu předkládá Boží lid v <Neh 9:6-38>. Modlitbu tvoří čtyři části: (1) uctívání (\<"v." 4>);  (2) doznání hříchů (<"vv." 5-11a>); (3) uznání Boží spravedlnosti a zaslouženosti trestu (<"vv." 11b-14>) a (4) prosba o Boží slitování, opřená o Jeho jméno, království a vůli (<"vv." 15-19>). Modlitba stojí na Božích zaslíbeních (<"v." 2>), je pronesena zkroušeným duchem (<"v." 3>) a ukazuje tak vzor náležitých prvků účinné modlitby.

\Note 9:11 {Mojžíše} \x/Daniel/ovi bylo jasné, že když si Izrael nevzal ponaučení ze sedmdesátiletého otroctví, čeká ho sedminásobný trest, jak píše Mojžíš (<Lv 26:18>, <21>, <24>, <28>). Národ bude sloužit cizincům 490 dalších let. Historie nás učí, že tato lhůta byla Boží milostí o něco zkrácena (srv. <Jr 18:8>).
\dopsat


%\ww {Sedmdesáte téhodnů} {sedmdesát sedmic} {sedmdesát sedmiletí} {Sedmdesát týdnů let} {Sedmdesát týdnů} {Sedmdesát týdnů}
\Note 9:21 {Gabriel} Viz <"pozn." 8:16>n. 

\ww {Sedmdesáte téhodnů} {sedmdesát sedmic} {sedmdesát sedmiletí} {Sedmdesát týdnů let} {Sedmdesát týdnů} {Sedmdesát týdnů}
\Note 9:24 {Sedmdesáte téhodnů} Sedmdesát \uv{týdnů} let je 490 let (70$\times$7).

\vdef %9:25  
    {téhodnů sedm}   %BKR
    {sedm sedmic}   %PSP
    {sedm sedmiletí}   %CSP
    {sedm týdnů}   %CEP
    {sedmero týdnů}   %B21
    {sedm a potom šedesát dva týdnů}   %SNC
\vdef %9:26  
    {šedesáti a dvou}   %BKR
    {šedesáti a dvou sedmicích}   %PSP
    {šedesáti a dvou sedmiletích}   %CSP
    {šedesáti dvou týdnů}   %CEP
    {dvaašedesát týdnů}   %B21
    {šedesáti dvou týdnech}   %SNC
\vdef %9:27  
    {v téhodni posledním}   %BKR
    {jednu sedmici}   %PSP
    {jednoho sedmiletí}   %CSP
    {v jednom týdnu}   %CEP
    {v týdnu posledním}   %B21
    {v posledním týdnu}   %SNC
% Nefunguje, nepřepíná, ani když \Note má jen 9:25 nebo 9:26. Musím přepsat poznámku, aby nebyla závislá na překladech
%když to není v {...}, \x/.../ nešlape
\Note 9:25-27 {} \uv{Sedmdesát týdnů} let je rozděleno do tří menších celků o délce: 49 let (\uv{\x/téhodnů sedm/}, <"v." 25>); 434 let (\uv{\x/šedesáti a dvou/}, <"v." 26>) a 7 let (\uv{\x/v téhodni posledním/}, <"v." 27>).
%\Note 9:25-27 {} \uv{Sedmdesát týdnů} let je rozděleno do tří menších celků o délce: 49 let (\uv{sedm týdnů}, <"v." 25>); 434 let (\uv{šedesát dva týdnů}, <"v." 26>) a 7 let (\uv{jeden (poslední) týden}, <"v." 27>). 
Vykladači se rozcházejí v otázce, zda tyto části tvoří navazující sekvenci, nebo zda jsou odděleny nějakými časovými intervaly. Pokusy srovnat chronologii příliš přesně selhávají, protože počty let byly zamýšleny jako zaokrouhlená čísla, reprezentující období. Přestože \x/Daniel/ovy kalkulace nemohou být brány jako precizní, základní vzorec je zřetelný bez nebezpečí spekulací:
Po pokynu obnovit Jeruzalém (<"v." 25>) následuje \uv{sedm týdnů}, tedy 49 let, během nichž byla rekonstrukce Jeruzaléma dokončena (viz knihy \x/Ezdráš/ a \x/Nehemiáš/). Poté následovalo \uv{šedesát dva týdnů}, tedy 434 let (<"v." 25>), během nichž bude zabit Mesiáš (<"v." 26>; viz <"pozn." 26>n).  Poslední \uv{jeden týden} (<"v." 27>) bude naplněn v blízkosti doby Kristovy pozemské služby. Pokud by zkáza verše <27> měla znamenat destrukci chrámu Římany v roce 70 po Kr., pak součet \uv{týdnů} nevychází bez mezer mezi nimi. To by se dalo přičíst Boží milosti, která původní lhůtu zkrátila (viz <"pozn." 11>n).

\Note 9:26 {Mesiáš}={Mesiáš ... svatyni} Podle tohoto proroctví bude \x/Mesiáš/ zabit před zničením jeruzalémského chrámu. Druhý chrám, který po návratu z babylónského zajetí obnovil \x/Zorobábel/ 
(<Ezd 5:2>; <Ezd 6:15>; <Za 1:12>; <Za 4:9>) a který v roce 20 př. Kr. ještě rozšířil Herodes ve snaze zavděčit se Židům, byl srovnán se zemí Římany v roce 70 po Kr. a od té doby nebyl nikdy obnoven. Na jeho místě dnes stojí mešita; není žádá naděje, že by mohl být znovu zbudován, aby před jeho dalším zničením mohl být zabit \x/Mesiáš/. Příchod Mesiáše musíme hledat mezi 20 př.Kr. a 70 po Kr.

\ww {lid ten}   %BKR
    {lid panovníka, jenž přijde}   %PSP
    {lid jednoho vůdce, který přichází}   %CSP
    {lid vévody, který přijde}   %CEP
    {lid vůdce, který přichází}   %B21
    {Lid}   %SNC
\Note 9:26 {lid ten} Narážka buďto na řeckého Antiocha IV. Epifana jakožto předchůdce římského generála Tita (viz <"Úvod: Záměr a zvláštnosti.">i), anebo přímo na Tita a jeho amrádu, která zničila Jeruzalém v roce 70 po Kr.

\Note 9:27 {smlouvu} Nevyjádřený podmět \uv{on} je nejpravděpodobněji \x/Mesiáš/ anebo \uv{vůdce} z <"verše" 26>. Názor, že je zde řeč dohodě mezi Antikristem a židovským národem, shromážděným v Zemi v době \uv{velkého pronásledování}, je sice populární, leč méně pravděpodobný.

\ww {učiní konec oběti zápalné i oběti suché} {bude zastavovat oběti, i oběť daru} {ukončí obětní hod i přídavnou oběť} {zastaví obětní hod i oběť přídavnou} {obětem i darům konec učiní} {všechny oběti ztratí smysl}
\Note 9:27 {učiní konec oběti zápalné i oběti suché} Pravděpodobně ukončení sz obětního systému Kristovou zástupnou obětí jednou provždy a pro všechny (viz <Iz 53:10>; <Ř 12:1>).
Je však také možné, že se zde mluví o znesvěcení chrámu  Antiochem IV. Epifanem nebo Titem (viz <"pozn." 26>n.) Někteří zastávají méně pravděpodobný pohled, že je řeč o Antikristově zákazu náboženských praktik obecně shromážděnému Izraeli po třech a půl letech (<Zj 11:2>; <12:6>, <14>)  \uv{velkého pronásledování}.

\ww {poplénění hubící}   %BKR
    {k dovršení ohavností}   %PSP
    {na křídle ohavností bude pustošící}   %CSP
    {ohyzdné modly}   %CEP
    {Otřesná ohavnost}   %B21
    {ohavnými modlami}   %SNC
\Note 9:27 {} Patrně zkáza chrámu Antiochem IV. nebo Titem (viz <"pozn." 26>n a Úvod: Záměr a Zvláštnosti). Podobné fráze se vyskytují v <8:13>; <11:31> a <12:11> (viz poznámky k nim), stejně jako v <1Mak 1:54>. \<Da 8:3> a <1Mak 1:54> odkazují k Antiochovi IV. \x/Daniel/ používá stejný jazyk k popisu toho, kdo znesvětí chrám v blízkosti doby Mesiášovy. Ježíš na tuto ohavnost naráží v <Mt 24:15> a <Mk 13:14>.


%\ww {}   %BKR
%   {}   %PSP
%   {}   %CSP
%   {}   %CEP
%   {}   %B21
%   {}   %SNC

\Note 10:1-12:13 {}={\it Vize budoucnosti Božího lidu} Zde prorok obrací pozornost k  závěrečné, delší vizi, zaměřené na vládu Antiocha IV. Epifana (viz Úvod: Záměr a zvláštnosti) i na dobu po něm. Tento materiál se dělí na 4 hlavní části: (1) andělova zvěst \x/Daniel/ovi (<10:1-11:1>), (2) události od \x/Daniel/a po Antiocha IV. Epifana  (<11:2-20>), Vláda Antiocha IV. Epifana (<11:21-12:3>) a závěrečné poselství k \x/Daniel/ovi (<12:4-13>). 


\ww {Léta třetího Cýra krále Perského}   %BKR
    {V třetím roce Kóreše, krále Persie}   %PSP
    {Ve třetím roce perského krále Kýra}   %CSP
    {V třetím roce vlády Kýra, krále perského}   %CEP
    {Ve třetím roce perského krále Kýra}   %B21
    {Ve třetím roce panování perského krále Kýra}   %SNC
\Note 10:1 {Léta třetího Cýra krále Perského} V roce 537 př. Kr. Viz <"pozn." 1:21>n; <5:30>n; <6:1>n a <9:1>n. Tou dobou byli někteří Izraelité  zpátky v Zemi, kde budovali chrám (<Ezd 1:1-4>; <3:8>), avšak rekonstrukci budou nuceni brzy pozastavit (<Ezd 4:24>).

\ww {kvílil} {upadl v truchlivost} {truchlil} {truchlil} {držel}={držel ... smutek} {smutný}
\Note 10:2 {kvílil} Pravděpodobný důvod \x/Daniel/ova zármutku byl stav Jeruzaléma (<Neh 1:4>; <Iz 61:3-4>; <64:8-12>; <66:10>).

\ww        {muž}={muž ... oděný v roucho lněné} 
           {muž, oblečený plátěnými rouchy} 
           {muže, oděného lněným oděvem} 
           {muž oblečený ve lněném oděvu} 
           {muže oděného plátnem} 
           {muže oblečeného do lněného roucha}
\Note 10:5 {muž}={muž ... oděný v roucho lněné} 
    Verše <5> a <6> nabízejí detailní popis anděla, snad Gabriela (<9:2>) nebo toho, kdo ke Gabrielovi mluvil (<8:16>). Jeho zjev je podobný slávě Pána (<Ez 1:26-28>; <Zj 1:12-16>). Další zmínky o andělech jsou např.  <Sd 13:6>; <Ez 9:2-3>; <10:2> nebo <Lk 24:4>.
    
\ww {hrůza veliká}   %BKR
    {veliké zděšení} %PSP
    {veliké zděšení} %CSP
    {veliké zděšení} %CEP
    {taková hrůza}   %B21
    {hrůza}          %SNC
\Note 10:7 {hrůza veliká} Srv. <Iz 6:5> a <Lk 5:8>.

\ww {jsi přiložil srdce své, abys rozuměl}   %BKR
    {jsi své srdce odevzdal věnování pozornosti}   %PSP
    {kdys přiložil své srdce, abys pochopil}   %CSP
    {ses rozhodl porozumět}   %CEP
    {ses}={ses ... začal modlit za porozumění}   %B21
    {kdy jsi pochopil}={jsi pochopil ... a pokorně se začal modlit}   %SNC
\Note 10:12 {jsi přiložil srdce své, abys rozuměl} Danielova moudrost nebyla náhodná; usiloval o ni celým srdcem (srv. <Jr 29:16>). 

\ww {kníže království Perského}   %BKR
    {kníže království Persie}   %PSP
    {velitel perského království}   %CSP
    {ochránce perského království}   %CEP
    {kníže perského království}   %B21
    {zlý duch, který ovládá perského krále}   %SNC
\vdef   
    {kníže veliké, kterýž zastává synů lidu tvého}   %BKR
    {veliký kníže, jenž stojí při dětech tvého lidu}   %PSP
    {velký velitel, jenž zastává tvůj lid}   %CSP
    {velký ochránce, a bude stát při synech tvého lidu}   %CEP
    {veliký kníže a tvého lidu ochránce}   %B21
    {vznešený zastánce tvého lidu}   %SNC
    
\Note 10:13 {kníže království Perského} Teritoriální démon, Satanův anděl, okupující Persii. V anglických Biblích nese označení ``Prince of Persia.'' Populární videohra a film téhož jména jsou typickou ukázkou zpohanštění kultury, kdysi křesťanské; lze je považovat za oslavu tohoto prastarého démona.
Srv. <Jb 1:6-12>; <Ž 82>; <Iz 24:21>; <Lk 11:14-26>. Nebeské zástupy bojují proti nepříteli za Boží lid  i jinde v Písmu; viz např. <Sd 5:20>;  <2Kr 6:15-18>;  <Ž 103:20-21>. Archanděl Michael je ve  <"verši" 12:1> označen jako  \uv{\x/kníže veliké, kterýž zastává synů lidu tvého/}.


\Note 10:13 {Michal} Význam jména zní: \uv{Kdo je jako Bůh?} Tento archanděl je v <Ju 9> a ve <Zj 14:7>   vylíčen jako velitel nebeské armády andělů. 

\ww {jedenmecítma dnů}   %BKR
    {dvacet a jeden den} %PSP
    {dvacet jedna dní}   %CSP
    {jedenadvacet dní}   %CEP
    {Jedenadvacet dnů}   %B21
    {jedenadvacet dní}   %SNC
\Note 10:13 {jedenmecítma dnů}={Dvacet jedna dnů} Když se modlíme, uvádíme do pohybu souvislosti v neviditelném světě, o jakých většinou nemáme ani nejmenší tušení. Srv. <Ef 6:12> a <Zj 12:7-9>.

\ww {s knížetem Perským}   %BKR
    {s knížetem Persie}   %PSP
    {s velitelem Persie}   %CSP
    {s ochráncem Peršanů}   %CEP
    {s knížetem Persie}   %B21
    {se zlým duchem, který ovládá perského krále}   %SNC
\Note 10:20 {s knížetem Perským} Viz pozn. <13>n.

\ww {kníže Řecké}   %BKR
    {kníže Jávána}   %PSP
    {velitel Řeků}   %CSP
    {ochránce Řeků}   %CEP
    {kníže Řecka}   %B21
    {ten, který ovládá řeckého krále}   %SNC
\Note 10:20 {kníže Řecké} Teritoriální démon nad Řeckem (srv. <"pozn." 13>n; viz <Jn 14:30>; <Ef 6:12>). Přestože Persie i Řecko budou Božímu lidu vládnout (a často ho budou utiskovat, srv.<"pozn." 8:9>n),  \x/Daniel/ si díky svým vizím připomenul, že jejich moc bude omezí, ba usměrní Bůh, jehož záměry se nakonec vždy uskuteční.

\ww {v psání pravdomluvném}   %BKR
    {v zápise pravdy}   %PSP
    {v přípisu pravdy}   %CSP
    {ve spisu pravdy}   %CEP
    {v Knize pravdy}   %B21
    {v Knize pravdy}   %SNC
\Note 10:21 {v psání pravdomluvném}
Metafora pro Boží vševědoucnost a všemohoucnost. Srv. <Zj 20:12>.

%\ww {léta prvního Daria Médského}   %BKR
%    {v prvním roce Dárjáveše, Mádajovce}   %PSP
%    {V prvnímroce Dareia Médského}   %CSP
%    {V prvním roce vlády Darjaveše médského}   %CEP
%    {v prvním roce Darjaveše Médského}   %B21
%    {v prvním roce médského Darjaveše}   %SNC
%\Note 11:1 {léta prvního Daria Médského}={}  \x// (<"">) (<"">) 

\Note 11:2-20 {}={\em Od \x/Daniel/a po Antiocha IV. Epifana} 
    Zjevení, dané \x/Daniel/ovi v <11:2-20>, se zabývá blízkovýchodními dějinami ode dnů \x/Daniel/ových až po dobu Antiocha IV. Epifana. Prorokova vize je neobyčejně detailní; líčí  souvislosti mezi událostmi mnohem složitější, než jaké bývají běžně prorokům zjevovány. Tyto podrobnosti upoutávaly pozornost původních čtenářů a svědčily ve prospěch \x/Daniel/ovy důvěryhodnosti.

\Note 11:2 {tři králové} Kambýsés II. (Kýrův syn, 529--523 př.Kr.), Pseudo-Smerdis (vydával se za Kýrova syna Smerdise, poté, co ho zavraždil jeho bratr Kambýsés) neboli Gaumáta (523--522 př.Kr.) a \x/Darius/ I. (522-486 př.Kr.).

\ww {čtvrtý}  %BKR
    {čtvrtý} %PSP
    {čtvrtý}  %CSP
    {Čtvrtý} %CEP
    {čtvrtý}  %B21 
    {čtvrtý}  %SNC
\Note 11:2 {čtvrtý} Xerxés I. (485--464 př.Kr.), totožný s \x/Asver/em z <Est 1:1>.

\ww {zbohatne bohatstvím velikým}   %BKR
    {bude bohatnout}   %PSP
    {zbohatne}   %CSP
    {bohatství}   %CEP
    {mnohem bohatší}   %B21
    {bohatství}   %SNC
\Note 11:2 {zbohatne bohatstvím velikým} Viz <Est 1:4>.

\ww {vzbudí všecky proti království Řeckému}   %BKR
    {bude moci všechny pohnout proti království Jávána}   %PSP
    {podnítí všechny při království řeckém}   %CSP
    {proti řeckému království}   %CEP
    {podnítí všechny proti řecké říši}   %B21
    {v Řecku kdekoho popudí}   %SNC
\Note 11:2 {vzbudí všecky proti království Řeckému} Roku 480 př.Kr. vedl Xerxés proti Řecku sérii zprvu úspěšných výprav a námořních bitev. Později však byla část perského loďstva poražena v bitvě u Salamíny a roku 479 př.Kr. byla i perská pozemní armáda poražena v bitvě u Platají. Téhož roku  v bitvě u mysu Mykalé byla zahnána na ústup perská královská flotila.

\ww {král mocný}   %BKR
    {mocný král}   %PSP
    {chrabrý král}   %CSP
    {bohatýrský král}   %CEP
    {bojovný král}   %B21
    {schopný král}   %SNC
\Note 11:3 {král mocný} Alexandr Veliký (336--323 př.Kr.). Srv. <"pozn." 7:6>n; <8:5>n a  <8>n.  
%%%%% Tahle \Note se hodí jako příklad do dokumentace

\ww {na čtyři strany světa}   %BKR
    {čtyřem větrům nebes}   %PSP
    {čtyřem větrům nebes}   %CSP
    {podle čtyř nebeských větrů}   %CEP
    {do čtyř světových stran}   %B21
    {na čtyři části}   %SNC
\Note 11:4 {na čtyři strany světa}  Viz <"pozn." 7:6>n a <8:8>n. 

\ww {posilní se král polední}   %BKR
    {sílit bude král jihu}   %PSP
    {zesílí král jihu}   %CSP
    {se vzmůže král Jihu}   %CEP
    {Jižní král bude mocný}   %B21
    {Na jihu povstane silný král}   %SNC
\Note 11:5 {posilní se král polední} Ptolemaios I. Sótér (323--285 př.Kr.).  
   
\ww {jedno z knížat jeho, a mocnější bude nad něho}   %BKR
    {z jeho knížat, a bude sílit nad něho}   %PSP
    {jeden z jeho velitelů a ten zesílí nad něho}   %CSP
    {jeden z jeho velitelů se vzmůže víc než on}   %CEP
    {jeden z jeho velitelů se vzmůže ještě více}   %B21
    {jeden z jeho velitelů získá větší moc než on}   %SNC
\Note 11:5 {jedno z jeho knížat, a mocnější bude nad něho} 
    Seleukos I. Níkátór (311--280 př.Kr.) se oddělil od Ptolemaia, stal se králem Babylóna a ovládal území od Sýrie na západě po řeku Indus na východě.
   
\Note 11:6-20 {}={}  <"Verše" 6-20> obsahují detailní předpověď vztahů mezi  králem Severu (Seleukovská říše) a králem Jihu (Ptolemaiovské království). Tato sekce může být rozdělena do tří částí: (1) události v souvislosti s Laodikou a Berenikou ("vv." 6-9); (2) kariéra Antiocha III. (<"vv." 10-19>) a (3) vláda Seleuka IV. (<"v." 20>).


\ww {dcera krále poledního}   %BKR
    {dcera krále jihu}   %PSP
    {dcera krále jihu}   %CSP
    {dcera krále Jihu}   %CEP
    {Dcera jižního krále}   %B21
    {Dcera jižního krále}   %SNC
\Note 11:6 {dcera krále poledního}
    Berenika, dcera Ptolemaia II. Filadelfa (285--246 př.Kr.). 


\ww {aby učinila příměří}   %BKR
    {k uskutečnění společné dohody}   %PSP
    {aby se uskutečnilo právo}   %CSP
    {aby ujednala smír}   %CEP
    {na potvrzení dohody}   %B21
    {uzavřou spojenectví}   %SNC
\Note 11:6 {aby učinila příměří}
    Ve druhé syrské válce (někdy po roce 260 př. Kr.), vedené seleukovským vládcem Antiochem II. Theem, utrpěl Ptolemaios II. Filadelfos ztrátu egyptských držav na maloasijském pobřeží a nakonec souhlasil s mírem. Ten byl zpečetěn, když se Antiochos II. Theos kolem r. 250 př.Kr. oženil s jeho dcerou Berenikou.

\ww {vydána bude}   %BKR
    {vydávána bude}={vydána bude ona ... i ten, jenž ji zplodil, i ten, jenž ji posílil}   %PSP
    {vydána bude}={vydána bude ona ... a kdo ji zplodil, i ten, kdo ji ... podporova}   %CSP
    {bude}={bude ... vydána záhubě ... i s tím, který ji zplodil, a s tím, kdo jí byl oporou}   %CEP
    {bude}={bude ... i se svým otcem a ochráncem zrazena}   %B21
    {nejenom ji}={nejenom ji ... ale i jejího manžela a dokonce i otce potká smrt}   %SNC
\Note 11:6 {vydána bude}={vydána bude ona ... i syn její, i ten, kterýž ji posiloval} 
    Bývalá manželka Antiocha II. Laodika iniciovala spiknutí, jež vyústilo vraždou otravou Bereniky, Antiocha II. i jejich syna v kojeneckém věku.


\ww {povstane}   %BKR
    {povstane}={povstane z odnože jejích kořenů jeden}   %PSP
    {nastoupí}={nastoupí jeden z výhonků jejích kořenů}   %CSP
    {postoupí}={postoupí výhonek z jejích kořenů}   %CEP
    {nástupce}={nástupce z téhož rodu}   %B21
    {postaví}={se ... postaví jeden z její rodiny}   %SNC
\Note 11:7 {povstane}={povstane z výstřelku kořenů jejích}  Ptolemaios  III. Euergetés (246--221         př.Kr.), bratr zavražděné Bereniky (viz <"pozn." 6>n). Coby faraon (třetí z dynastie     Ptolemaiovců) podpořil pořízení překladu hebrejského SZ do řečtiny, známého jako Septuaginta (LXX). 


\ww {udeří na pevnost krále půlnočního}   %BKR
    {v pevnost krále severu}   %PSP
    {přitáhne do pevnosti krále severu}   %CSP
    {vejde do pevnosti krále Severu}   %CEP
    {Zaútočí na vojska severního krále}   %B21
    {zaútočí na pevnosti severního království}   %SNC
\Note 11:7 {udeří na pevnost krále půlnočního}  
    Ptolemaios III. napadl seleukovskou říši, obsadil Antiochii a nakonec dosáhl Babylónu. Dal popravit Laodiku (viz <"pozn." 6>n) a vrátil se do Egypta se značnou kořistí (srv. <"v." 8>).


\ww {přijde}={král půlnoční přijde ... král polední, a navrátí se do země své}   %BKR
    {přijde}={král severu přijde v království krále jihu, ale vrátí se na svou půdu}   %PSP
    {přijde}={král severu do království krále jihu, ale vrátí se do své země}   %CSP
    {vtrhne}={král Severu vtrhne do království krále Jihu, ale vrátí se do své země}   %CEP
    {Ten se sice pokusí říši jižního krále napadnout, ale stáhne se zpět na své vlastní území}  %B21
    {Ten se sice pokusí postavit proti Egyptu, ale vážně ho neohrozí}  %SNC
\Note 11:9 {přijde}={přijde ... král polední, a navrátí se do země své}  
    Narážka na neúspěšné tažení Seleuka II. Kallinika (246--226 př.Kr.), syna Laodiky (viz <"pozn." 6>n), proti ptolemaiovskému Egyptu.

\ww {synové}  %BKR
    {synové} %PSP
    {syn}  %CSP
    {synové} %CEP
    {Synové}  %B21 
    {Synové}  %SNC
\Note 11:10 {synové} Seleukos III. Keranus (226--223 př.Kr) a Antiochos III. Megás (223--187 př.Kr.).

\ww {válčiti budou, a seberou množství vojsk velikých}   %BKR
    {budou začínat válku a shromáždí spoustu mohutných sil}   %PSP
    {budou mobilizovat a shromáždí veliké množství vojska}   %CSP
    {budou válčit a shromáždí nesmírné množství vojska}   %CEP
    {povstanou k boji a shromáždí obrovské vojsko}   %B21
    {vybudují armádu}   %SNC
\Note 11:10 {válčiti budou, a seberou množství vojsk velikých}
    Antiochos III. válčil s Ptolemaiovci od 222--187 př.Kr. a na čas získal nadvládu nad \x/Kanán/em a západní Sýrií. 

\Note 11:10 {pevnost} Patrně Rafia, ptolemaiská pevnost v jižním \x/Kanán/u (na jihu dnešního pásma        Gazy). K hlavní bitvě zde došlo roku 217 př.Kr.

\ww {král polední}  %BKR
    {král jihu}      %PSP
    {král jihu}   %CSP
    {Král Jihu}   %CEP
    {Jižní král}   %B21
    {Toho}={jižní král}   %SNC
\Note 11:11 {král polední}  Ptolemaios IV. Filopatór (221--203 př.Kr.).
    
\ww {s králem půlnočním}   %BKR
    {s králem severu}   %PSP
    {s tím králem severu}   %CSP
    {s králem Severu}   %CEP
    {Severní král}   %B21
    {proti nim}   %SNC
\Note 11:11
    {s králem půlnočním}
     Antiochos III. Megás (223--187 př.Kr.; srv. <"pozn." 10>n). V bitvě u Rafie 217 př.Kr. utrpěl těžké ztráty (1400 mužů).
     
\ww {sšikuje množství větší než prvé}   %BKR
    {postaví spoustu mohutnější než ta první}   %PSP
    {Postaví}={Postaví ... množství větší než to předchozí}   %CSP
    {postaví ještě nesmírnější množství, než bylo první}   %CEP
    {znovu postaví obrovské vojsko, ještě větší než to předešlé}   %B21
    {sestaví ještě větší armádu než předtím}   %SNC
\Note 11:13 {sšikuje množství větší než prvé}
    Antiochos III. v alianci s Filipem V. Makedonským vybudoval ještě větší armádu k invazi na Ptolemaiovská území. Ptolemaios IV. Filopatór zemřel za neobjasněných okolností a nahradil ho jeho čtyřletý syn Ptolemaios V. Epifanés (203--181; srv. <"pozn." 17>n).

\ww {dobude měst hrazených}   %BKR
    {dobude dobře opevněného města}   %PSP
    {obsadí opevněné město}   %CSP
    {zmocní se opevněného města}   %CEP
    {přitáhne}={k opevněnému městu ... dobude je}   %B21
    {opevněné město dobude}   %SNC
\Note    11:15 {dobude měst hrazených}
    Vítězství Antiocha III. nad egyptským generálem Scopasem v bitvě u \x/Sidon/u v r. 198 př.Kr. To znamenalo konec vlády Ptolemaiovců v oblasti, které se teprve později začalo říkat Palestina.

\ww {v zemi Judské}   %BKR
    {v zemi okrasy}   %PSP
    {na Ozdobě země}   %CSP
    {v nádherné zemi}   %CEP
    {v Nádherné zemi}   %B21
    {v Krásné zemi}   %SNC
\Note 11:16 {v zemi Judské}
    Tj. v Zemi zaslíbené (viz <"vv." 41>, <45>; <8:9>).

\ww {dá jemu krásnou pannu}  %BKR
    {bude mu dávat dceru}   %PSP
    {dá mu svou dceru}  %CSP
    {dá mu jednu z dcer}  %CEP
    {mu dá svou dceru za ženu}   %B21
    {dá mu za ženu svoji dceru}   %SNC
\Note 11:17 {dá jemu krásnou pannu}
    Kleopatra, dcera Antiocha III. (nezaměnit s nejslavnější nositelkou tohoto jména, o pět generací mladší Kleopatrou VII. Egyptskou), byla roku 194 př.Kr. v Rafii (srv. <"pozn." 10>n <"a" 11>n) provdána za tehdy šestnáctiletého krále Ptolemaia V. 
    
\ww {ona nedostojí aniž bude držeti s ním}   %BKR
    {ona nebude při tom stát a nebude pro něho}   %PSP
    {To se mu ale nepodaří, to mu nevyjde}   %CSP
    {ona nebude stát při něm}   %CEP
    {To se mu však nepodaří a nesplní}   %B21
    {ona se postaví proti němu}   %SNC
\Note    11:17 {ona nedostojí aniž bude držeti s ním}
   Kleopatra se spojila s Egypťany proti svému otci. Ve snaze zmařit Antiochovy snahy o uchvácení pobřežních měst Malé Asie hledala pomoc u Římanů.  

\ww {vůdce přítrž učiní pohanění jeho}   %BKR
    {jistý velmož zarazí jeho posměch}   %PSP
    {Jeden vojevůdce mu však zatrhne jeho pohrdání}   %CSP
    {jeden konsul učiní přítrž jeho tupení}   %CEP
    {Jeden vojevůdce však jeho zpupnost ukončí}   %B21
    {Jeho dotěrnosti však učiní přítrž jeden velitel}   %SNC
\Note 11:18 {vůdce přítrž učiní pohanění jeho}  
    Římský generál Lucius Cornelius Scipio porazil Antiocha III. v několika bitvách a donutil ho postoupit Malou Asii římské nadvládě (Apamejský mír, 188 př.Kr.). V té době byl druhý syn Antiocha III., který bude později  znám jako Antiochos IV. Epifanés, odvlečen jako rukojmí do Říma.

\ww {na místo jeho}  %BKR
    {na jeho místě}   %PSP
    {na jeho místo}   %CSP
    {Na jeho místo}   %CEP
    {Jeho nástupce}   %B21
    {Po něm nastoupí}   %SNC
\Note    11:20 {na místo jeho}
    Antiocha III. vystřídal na trůnu jeho starší syn Seleukos IV. Filopatór (187--175 př.Kr.).

\ww {výběrčí}  %BKR
    {vymahač}={vymahače}   %PSP
    {vymahač}={vymahače}   %CSP
    {výběrčí}={výběrčímu}   %CEP
    {výběrčí}={výběrčího}   %B21
    {výběrčí}={výběrčího}   %SNC
\Note    11:20 {výběrčí}
    Heliodor (viz <2Mak 3:7-40>).
    
\Note    11:21-12:3 {}={Vláda Antiocha IV. Epifana}
        Zde se \x/Daniel/ soustředí na zachycení života Antiocha IV. Epifana, který pronásledoval Židy a znesvětil chrám. Ukazuje jeho nástup k moci a charakter (<21-24>);  kariéru
        (<25-31>); situaci Božího lidu pod jeho nadvládou (<32-35>); shrnutí jeho náboženského postoje  (<36-39>); jeho ambice  (<40-45>) a jeho porážku (<12:1-3>).

\ww {nevzácný}   %BKR
    {opovrženíhodný}  %PSP
    {pohrdaný}   %CSP
    {Opovrženíhodný}   %CEP
    {opovrženíhodný}   %B21
    {opovrženíhodná osoba}   %SNC
\Note    11:21 {nevzácný} 
        Neslavný Antiochos IV. Epifanés (175--164 př.Kr.), bratr Seleuka IV., nikoli však jeho legitimní nástupce, protože Seleukos IV. měl syna Demetria Sotéra, známého též jako Demetrius~I. Viz <"vv." 23-24> a <"pozn." 8:9-14>n.

\ww {vůdce}={i ten vůdce ...  smlouvu}   %BKR
    {tak i vládce smlouvy}   %PSP
    {ba i vůdce smlouvy}   %CSP
    {tak i vůdce smlouvy}   %CEP
    {včetně vůdce smlouvy}   %B21
    {i představitele Boží smlouvy}   %SNC
\Note    11:22 {vůdce}={i ten vůdce ... smlouvu} %doplnit \ww
        Snad odkaz na zavraždění velekněze Oniáše III. v Jeruzalémě podporovateli Antiocha IV. v roce 171 př.Kr. (srv. <2Mak 1:16-19>).
        
\ww {králi polednímu}   %BKR
    {král jihu}   %PSP
    {Král jihu}   %CSP
    {králi Jihu}   %CEP
    {jižnímu králi}   %B21
    {jižnímu králi}   %SNC
\Note    11:25 {králi polednímu}
    Tento jižní král je Ptolemaios VI. Filométor (181--146 př.Kr.),  syn Ptolemaia V. a Kleopatry (viz <"pozn." 17>n).

\ww {neostojí}   %BKR
    {nebude však moci obstát}   %PSP
    {neobstojí}   %CSP
    {neobstojí}   %CEP
    {bude poražen}   %B21
    {nezvítězí}  %SNC
\Note    11:25 {neostojí}
    Antiochos IV. porazil Ptolemaia VI v bitvě u Pelúsia (srv. <1Mak 1:16-19>).

\ww {navrátí se}={}   %BKR
    {vrátí se}={vrátí se ... ve svou zem a jeho srdce bude proti svaté smlouvě}   %PSP
    {se vrátí}={se vrátí do své země ... a jeho srdce bude jednat i proti svaté smlouvě}   %CSP
    {Navrátí se}={Navrátí se ... do své země ... ale jeho srdce bude proti svaté smlouvě}   %CEP
    {vydá zpět}={se ... vydá zpět do své země a cestou projeví své nepřátelství proti svaté smlouvě}   %B21
    {vracet}={vracet do své země ... se postaví proti svaté smlouvě}   %SNC
\Note 11:28 {navrátí se}={navrátí se do země své ... a srdce jeho bude proti smlouvě svaté}
    Odvetou za jeruzalémské intriky proti svým podporovatelům vyplenil Antiochos IV. chrám po cestě z Egypta zpět do  Antiochie v Sýrii (srv. <1Mak 1:20-28>). 

\ww {potáhne na poledne}   %BKR
    {bude zase přicházet na jih}   %PSP
    {e bude vracet a přijde na jih}   %CSP
    {opět potáhne proti Jihu}   %CEP
    {vytáhne znovu na jih}  %B21
    {opět potáhne proti jižnímu králi}   %SNC
\Note    11:29 {potáhne na poledne}
    Antiochos IV. napadl Egypt znovu v roce 168 př.Kr.
    
\ww {přijdou proti němu lodí Citejské}  %BKR
    {proti němu zasáhnou lodi z Kittím}   %PSP
    {Setkají se s ním totiž kitejské lodě}  %CSP
    {Přitáhne na něj loďstvo Kitejců}   %CEP
    {Postaví se mu totiž středomořské loďstvo}   %B21
    {Na lodích přijedou Kitejci}   %SNC
\Note    11:30 {přijdou proti němu lodí Citejské}
    Když po vpádu do Egypta Antiochos IV. obléhal Alexandrii,   Římané, znepokojeni jeho postupem, k němu vyslali poselstvo v čele s Gaiem Popiliem, které mělo dojednat mír mezi ním a egyptským králem Ptolemaiem VI. (srv. <"pozn." 25>n). Antiochos po přečtení usnesení senátu požádal Popilia, aby mohl dokument probrat se svými společníky. Gaius Popilius Laenas však místo toho nakreslil svou holí kolem krále kruh a zakázal mu z něj vystoupit, dokud na usnesení senátu neodpoví. Antiocha jeho jednání natolik vyvedlo z míry, že nakonec souhlasil se všemi římskými požadavky a uzavřel s Ptolemaiem VI. mír, s vojskem ustoupil z Egypta zpět do Sýrie a dokonce vyklidil i Kypr, který nedlouho předtím na Egypťanech dobyl.
    
\ww {zlobiti se bude proti smlouvě svaté}   %BKR
    {rozhorlí se proti svaté smlouvě}   %PSP
    {bude jednat proti svaté smlouvě}   %CSP
    {bude soptit a jednat proti svaté smlouvě}   %CEP
    {Svůj hněv proto obrátí proti svaté smlouvě}={Svůj hněv ... obrátí proti svaté smlouvě}   %B21
    {opět se postaví proti svaté smlouvě}  %SNC
\Note 11:30 {zlobiti se bude proti smlouvě svaté}
    Antiochos byl rozhodnut zcela vymýtit židovské náboženství.
    
    

\ww {odejmou také obět ustavičnou}   %BKR
   {odstraní ustavičnou oběť}   %PSP
   {Odstraní také soustavnou bohoslužbu}   %CSP
   {vymýtí každodenní oběť}   %CEP
   {zruší každodenní oběť}   %B21
   {Odstraní každodenní služby}   %SNC
\Note 11:31 {obět}={odejmou oběť ... postaví ohavnost zpuštění} 
Znesvěcení chrámu v prosinci 168 př.Kr. Antiochem IV. (srv. <1Mak 1:54, 59>; <2Mak 6:2>); viz \<"pozn." 8:11>n; <9:27>n; <12:11>n.


    
\ww {lid pak, kterýž zná Boha svého}   %BKR
    {lid znajících svého Boha}   %PSP
    {lid, který zná svého Boha}   %CSP
    {ti, kteří se znají ke svému Bohu}   %CEP
    {lidé znající svého Boha}   %B21
    {Ti však, co se ke svému Bohu přiznávali}   %SNC
\Note    11:32 {lid pak, kterýž zná Boha svého}
    Ti, kdo se Antiochovi IV. vzepřeli a zůstali Bohu věrni až k smrti (viz <1Mak 1:54>, <59>; <2Mak 6:2>). Věrnost znamená oběť (srv. 
    <"gl." Lv 16:15>g; <Kaz 1:9>).
    %navázat na poznámku (glosu?) ke slovu znát asi z Gn 3 


\ww {malou pomoc}   %BKR
   {trochou pomoci}   %PSP
   {trochu pomoci}   %CSP
   {trochu pomoci}   %CEP
   {nepatrné pomoci}   %B21
   {málo jim bude ulehčeno}   %SNC
\Note 11:34 {malou pomoc}  Snad odkaz na starého kněze Matatiáše a jeho pět synů (Jan, Šimon, Juda, Eleazar a Jonatan), kteří vedli partyzánskou válku proti Antiochovi IV. Matatiáš zemřel v roce 166 př.Kr. Jeho synové pokračovali v boji a prosluli jako \uv{Makabejští}. Pod vedením Judy Makabejského dosáhli vítězství v roce 165 př.Kr., kdy chrám byl vyčištěn a denní oběti obnoveny (srv. <1Mak 4:36-39>).

\ww {do času jistého}={do času jistého ... až do času uloženého}   %BKR
    {do času konce}={do času konce ... do určeného času}   %PSP
    {do času konce}={do času konce ... určenou dobou}   %CSP
    {pro dobu konce}={pro dobu konce ... do jistého času}   %CEP
    {do konce}={do konce ... v určený čas}   %B21
    {pro dobu konce}={pro dobu konce ... v určený čas}   %SNC
\Note 11:35 {do času jistého}={do času jistého ... do času uloženého}  
    Viz <"pozn." 8:17>n.
    

\Note 11:36-12:3 {}
    Král bude zničen na hoře Sion, přímo v srdci Svaté země (<"vv." 44-45>).  
    Jeho porážka je vylíčena v souvislosti s absolutním koncem dějin.
    Protože však tato proroctví (dosud)  nenalezla svá naplnění v historii, je těžké odhadnout, nakolik doslovná či metaforická interpretace je na místě; při  výkladu je zapotřebí obezřetnosti. 
    Určité detaily v <11:36-12:3> nelze sladit s dobou Antiocha IV.
    Z toho důvodu mnoho evangelikálů chápe tyto verše jako popis Antikrista, který bude pronásledovat Boží lid těsně před druhým Kristovým příchodem (srv. <12:1-3>). Tato interpretace však předpokládá  časový interval mezi událostmi, zachycenými  verši <11:21-35> a  <11:36-12:13> značně prodloužený, což  text nikde nenaznačuje.
    Nelze vyloučit, že naplnění těchto proroctví bylo odvráceno,  pozměněno či odloženo (srv. <"Úvod k prorokům">).
    
    

\ww {až do vykonání prchlivosti}   %BKR
    {než bude rozhorlení dokonán}   %PSP
    {dokud neskončí rozhořčení}   %CSP
    {dokud se nedovrší hrozný hněv}   %CEP
    {dokud se nezavrší čas hněvu}  %B21
    {dokud nebude dovršen hrozný hněv}   %SNC
\Note 11:36 {až do vykonání prchlivosti}  
    Podobně jako v <8:17> a <11:35>, i dobu pronásledování má Bůh pod svou kontrolou.


\ww {dokonání pak toho času}   %BKR
    {v době konce}   %PSP
    {v čase konce}   %CSP
    {V době konce}   %CEP
    {V posledním čase}   %B21
    {V době konce}   %SNC
\Note 11:40 {dokonání pak toho času}  
    Viz <"pozn." 8:17>n.

\ww {do země Judské}   %BKR
    {v zem okrasy}   %PSP
    {do Ozdoby země}   %CSP
    {do nádherné země}   %CEP
    {do Nádherné země}   %B21
    {do nádherné země}   %SNC
\Note 11:41 {do země Judské}
    Tj. do \x/Kanán/u;  Země zaslíbené (viz <"vv." 41>, <45>; <8:9> a <"pozn." 16>n).


\ww {když přijde k skonání svému, nebude míti žádného spomocníka}   %BKR
    {dojde k svému konci a pomoci mu nebude}   %PSP
    {dospěje až ke svému konci a nebude, kdo by mu pomohl}   %CSP
    {přijde jeho konec a nikdo mu nepomůže}   %CEP
    {potká svůj konec a nebude mu pomoci}   %B21
    {přijde jeho konec a nikdo mu nepomůže}   %SNC
\Note 11:45 {když přijde k skonání svému, nebude míti žádného spomocníka}  
    Vizl <Jl 3>;  <Za 14:1-4>; <2Te 2:8>;  <Zj 16:13-16> a <19:11-21>.
    
    

\ww {Toho času}   %BKR
    {V onen čas}   %PSP
    {v onom čase}   %CSP
    {V oné době}   %CEP
    {V tom čase}   %B21
    {V době konce}   %SNCa
\Note 12:1 {Toho času} \x/Michal/, andělský strážce Izraele, nedopustí pronásledování Božího lidu donekonečna. Ztrestá ty, kdo jeho lidem ubližují. 


\ww {Michal, kníže veliké}  %BKR
    {Mícháél, veliký kníže} %PSP
    {Michael, ten velký velitel}  %CSP
    {Míkael, velký ochránce} %CEP
    {Michael, veliký kníže}  %B21 
    {vznešeného zastánce}={vznešeného zastánce -- Michaela}  %SNC
\Note 12:1 {Michal, kníže veliké} Viz <"pozn." 10:13>n.

\ww {čas ssoužení}  %BKR
    {čas tísně} %PSP
    {čas soužení}  %CSP
    {doba soužení} %CEP
    {Čas soužení}  %B21 
    {doba soužení}  %SNC
\Note 12:1 {čas ssoužení} 
      Viz <Mt 24:21> a <Mk 13:19>, kde Ježíš navazuje na tato proroctví o Antiochovi IV. a používá je jako typ (viz čl. <"Typologie v Bibli" Ř 5>a na str.~\pg) římského obležení Jeruzaléma v r. 70 po Kr. 
      
\ww {vysvobozen bude lid tvůj}  %BKR
    {bude tvůj lid moci vyváznout} %PSP
    {unikne tvůj lid}  %CSP
    {bude vyproštěn tvůj lid} %CEP
    {bude tvůj lid zachráněn}  %B21 
    {obstojí}  %SNC
\Note 12:1 {vysvobozen bude lid tvůj} Vysvobození nemusí nutně znamenat záchranu před  mučednictvím (srv. <"v." 2>), ale vždy z moci Satanovy (srv. <Mt 6:13> a <2Tm 4:18>). Tyto verše ujišťují Boží lid o ochraně před Satanovým pokušení dopadnout od víry v době pronásledování. 

\ww {procítí}  %BKR
    {budou procitat} %PSP
    {procitnou}  %CSP
    {procitnou} %CEP
    {se probudí}  %B21 
    {vstanou z mrtvých}  %SNC
\Note 12:2 {procítí} Předpověď tělesného vzkříšení svatých i ztracených před Božím posledním soudem (<Mt 25:46>; <Jn 5:28-29>).

\ww {zapečeť}  %BKR
    {zapečeť} %PSP
    {zapečeť}  %CSP
    {zapečeť} %CEP
    {zapečeť}  %B21 
    {rozpečetěna}  %SNC
\Note 12:4 {zapečeť} Pečeť byla chápána jako známka autentičnosti (viz <"pozn." 8:26>n).

\ww {času}={času ... časích, i půl času}  %BKR
    {času}={času ... časům a polovině} %PSP
    {době, dobám a polovině}  %CSP
    {k času a časům a k polovině} %CEP
    {času a časech a polovině času}  %B21 
    {třech a půl časech}  %SNC
\Note 12:7 {času}={času ... časích, i půl času} Viz <"pozn." 7:25>n.


\ww {nerozuměl}  %BKR
    {nechápal} %PSP
    {nechápal}  %CSP
    {neporozuměl jsem} %CEP
    {nerozuměl}  %B21 
    {nerozuměl jsem}  %SNC
\Note 12:8 {nerozuměl} Andělova odpověď (<"v." 7>) na \x/Daniel/ovi otázku (<"v." 6>) mu připadala nesrozumitelná, a tak otázku preformuloval. 


\ww {ohavnost hubící}  %BKR
    {pustošící ohavnost} %PSP
    {pustošící ohavnosti}  %CSP
    {ohyzdná modla pustošitele} %CEP
    {otřesné ohavnosti}  %B21 
    {modlářský kult pustošitele}  %SNC
\Note 12:11-12 {ohavnost hubící} Viz <"pozn." 9:27>n. Zvrácenosti Antiocha IV. slouží jako předobraz (viz čl. <"Typologie v Bibli" Ř 5>a na str.~\pg) římského Tita v r. 70 po Kr. 
 
\ww {dnů}={1290 ... 1335 dnů}      %BKR
    {bude}={1290 ... 1335 dní}      %PSP
    {uplyne}={1290 ... 1335 dnů}    %CSP
    {uplyne}={1290 ... 1335 dní}    %CEP
    {uplyne}={1290 ... 1335 dnů}    %B21 
    {Tisíc}={1290 ... 1335 dní}     %SNC
\Note 12:11-12 {dnů}={1290 ... 1335 dnů} Ačkoliv anděl upřesnil svou odpověď (<"v." 7>; viz <"pozn." 8>n), pro nás zůstává význam těchto časových údajů obestřen tajemstvím (srv. <Dt 29:29>).


%\bNote 2:1 {Danielovy vize} \label[danielovyvize]\wlabel{}\picw=\hsize \inspic{images/Nabuco.pdf}

\putImage 2:1 {Danielovy vize} [danielovyvize] () {Nabuco.pdf}

\putArticle 6:1 {Kdo byl \x/Darius/ Médský?} [6] ()

\putCite 2:27
   {Smích a pláč jsou nejcennější majetek živého člověka.
   \quotedby {Jan Werich}
}
\putCite 3:1
   {Mnohdy jsem seděl u stolu pokrytého ubrusem, na kterém včerejší a předvčerejší omáčky nakreslily mapy neznámých světů, a přemýšlel, proč je tolik vedoucích a tak málo vědoucích?
   \quotedby {Jan Werich}
}
\putCite 3:28
   {Člověk přišel na svět proto, aby tady byl, pracoval a žil.\nl Jen moudrý se snaží náš svět postrčit dál, posunout výš.\nl A jen vůl mu v tom brání.
   \quotedby {Jan Werich}
}


\endinput

\ww {}  %BKR
    {} %PSP
    {}  %CSP
    {} %CEP
    {}  %B21 
    {}  %SNC
\ww {}={}  %BKR
    {}={}  %PSP
    {}={}   %CSP
    {}={}  %CEP
    {}={}   %B21 
    {}={}   %SNC
    
\Note 12:1 {}={}  \x// (<"">) (<"">) 

 >> Ptolemaios II Filadelfos incest s Arsinou II
 >> Ptolemaios III Euergetes rozvedl z Láodikou, oženil s Berenikou
 >> Ptolemaios IV? Filopator
 >> Ptolemaios V == Kleopatra III
 >> Ptolemaios VI
 >> Kleopatry III  
 >> Ptolemaios IX Soter  (Ptolemaios X jeho brácha) syn Kleopatry III + Ptolemaia VIII  
 >> Ptolemaios XII 
 >> Kleopatra VII
