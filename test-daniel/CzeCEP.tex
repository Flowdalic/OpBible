\source{CzeCEP}


\book{Daniel}{Da}
#1:1 Ve třetím roce kralování Jójakíma, krále judského, přitáhl Nebúkadnesar, babylónský král, k Jeruzalému a oblehl jej.
#1:2 Panovník Hospodin mu vydal do rukou judského krále Jójakíma a část nádob z Božího domu. Nebúkadnesar je dopravil do země Šineáru, do domu svého božstva; nádoby dal dopravit do klenotnice svého boha.
#1:3 Pak rozkázal král Ašpenazovi, vrchnímu nad dvořany, aby přivedl z Izraelců, a to z královského potomstva a ze šlechty,
#1:4 jinochy bez jakékoli vady, pěkného vzhledu, zběhlé ve veškeré moudrosti, kteří si osvojili poznání, rozumějí všemu vědění a jsou schopni stávat v královském paláci a naučit se kaldejskému písemnictví a jazyku.
#1:5 Král pro ně určil každodenní příděl z královských lahůdek a z vína, které pil při svých hodech, a dal je vychovávat po tři roky. Po jejich uplynutí měli stávat před králem.
#1:6 Z Judejců byli mezi nimi Daniel, Chananjáš, Míšael a Azarjáš.
#1:7 Velitel dvořanů jim změnil jména: Danielovi dal jméno Beltšasar, Chananjášovi Šadrak, Míšaelovi Méšak a Azarjášovi Abed-nego.
#1:8 Ale Daniel si předsevzal, že se neposkvrní královskými lahůdkami a vínem, které pil král při svých hodech. Požádal velitele dvořanů, aby se nemusel poskvrňovat.
#1:9 A Bůh dal Danielovi dojít u velitele dvořanů milosrdenství a slitování.
#1:10 Velitel dvořanů však Danielovi řekl: „Bojím se krále, svého pána, který vám určil pokrm a nápoj. Když uvidí, že jste v tváři přepadlejší než jinoši z vašich řad, připravíte mě u krále o hlavu.“
#1:11 Daniel tedy navrhl opatrovníkovi, kterého určil velitel dvořanů nad Danielem, Chananjášem, Míšaelem a Azarjášem:
#1:12 „Zkus to se svými služebníky po deset dní. Ať nám dávají k jídlu zeleninu a k pití vodu.
#1:13 Potom porovnáš vzhled náš a vzhled jinochů, kteří jedli královské lahůdky, a učiň se svými služebníky podle toho, co uvidíš.“
#1:14 Opatrovník je v té věci vyslyšel a zkusil to s nimi po deset dní.
#1:15 Po uplynutí deseti dnů se ukázalo, že jejich vzhled je lepší; byli statnější než ostatní jinoši, kteří jedli královské lahůdky.
#1:16 Opatrovník tedy odnášel jejich lahůdky a víno, které měli pít, a dával jim zeleninu.
#1:17 A Bůh dal těm čtyřem jinochům vědění a zběhlost ve veškerém písemnictví a moudrosti. Danielovi dal nadto porozumět všem viděním a snům.
#1:18 Po uplynutí doby, kdy podle králova nařízení měli být předvedeni, přivedl je velitel dvořanů před Nebúkadnesara.
#1:19 Král s nimi rozmlouval a žádný mezi nimi nebyl shledán takový jako Daniel, Chananjáš, Míšael a Azarjáš. Proto stávali před králem.
#1:20 Pokud šlo o porozumění moudrosti, které od nich král vyžadoval, shledal, že desetkrát předčí všechny věštce a zaklínače, kteří byli v celém jeho království.
#1:21 A Daniel tam zůstal až do prvního roku vlády krále Kýra. 
#2:1 Ve druhém roce svého kralování měl Nebúkadnesar sen. Rozrušil se a nemohl spát.
#2:2 Král tedy rozkázal zavolat věštce, zaklínače, čaroděje a hvězdopravce, aby mu pověděli, co se mu zdálo. I přišli a postavili se před králem.
#2:3 Král jim řekl: „Měl jsem sen a jsem rozrušen; chci ten sen znát.“
#2:4 Hvězdopravci promluvili ke králi aramejsky: „Králi, navěky buď živ! Pověz svým služebníkům ten sen a my ti sdělíme výklad.“
#2:5 Král hvězdopravcům odpověděl: „Mé slovo je příkaz: Jestliže mi neoznámíte sen a jeho výklad, budete rozsekáni na kusy a z vašich domů se stane hnojiště.
#2:6 Jestliže mi však sen a jeho výklad sdělíte, dostane se vám ode mne darů, odměny a velké pocty. Sdělte mi tedy sen a jeho výklad!“
#2:7 Odpověděli znovu: „Ať král poví svým služebníkům sen a my sdělíme výklad.“
#2:8 Král odpověděl: „Já vím jistě, že chcete získat čas, protože vidíte, že mé slovo je příkazem:
#2:9 Jestliže mi ten sen neoznámíte, čeká vás jediný rozsudek. Domluvili jste se, že mi budete říkat lživé a zlé věci, dokud se nezmění čas. A proto mi řekněte ten sen a poznám, že jste schopni sdělit mi i výklad.“
#2:10 Hvězdopravci králi odpověděli: „Není na zemi člověka, který by dovedl sdělit to, co král rozkázal. Nadto žádný velký a mocný král nežádal od žádného věštce, zaklínače nebo hvězdopravce takovou věc.
#2:11 Věc, kterou král žádá, je těžká a není nikoho jiného, kdo by ji králi sdělil, mimo bohy, kteří nepřebývají mezi smrtelníky.“
#2:12 Král se proto rozhněval, velice se rozzlobil a rozkázal všechny babylónské mudrce zahubit.
#2:13 Když byl vydán rozkaz a mudrci byli zabíjeni, hledali též Daniela a jeho druhy, aby byli zabiti.
#2:14 Tehdy se Daniel rozvážně a uvážlivě obrátil na Arjóka, velitele královy tělesné stráže, který vyšel zabíjet babylónské mudrce.
#2:15 Otázal se Arjóka, králova zmocněnce: „Proč je králův rozkaz tak přísný?“ Arjók Danielovi tu věc oznámil.
#2:16 Daniel vešel ke králi a prosil ho, aby mu dal určitou dobu, že králi ten výklad sdělí.
#2:17 Pak Daniel odešel do svého domu a oznámil tu věc svým druhům, Chananjášovi, Míšaelovi a Azarjášovi.
#2:18 Vyzval je, aby prosili Boha nebes o slitování ve věci toho tajemství, aby Daniel a jeho druhové nebyli zahubeni se zbytkem babylónských mudrců.
#2:19 I bylo to tajemství Danielovi zjeveno v nočním vidění. Daniel dobrořečil Bohu nebes.
#2:20 Promlouval takto: „Požehnáno buď jméno Boží od věků až na věky. Jeho moudrost i bohatýrská síla.
#2:21 On mění časy i doby, krále sesazuje, krále ustanovuje, dává moudrost moudrým, poznání těm, kdo mají rozum.
#2:22 Odhaluje hlubiny a skryté věci, poznává to, co je ve tmě, a světlo s ním bydlí.
#2:23 Tobě, Bože otců mých, chci vzdávat čest a chválu, neboť jsi mi dal moudrost a bohatýrskou sílu. Oznámils mi nyní, oč jsme tě prosili, oznámil jsi nám královu záležitost.“
#2:24 Daniel tedy vešel k Arjókovi, kterého král ustanovil, aby zahubil babylónské mudrce. Přišel a řekl mu toto: „Babylónské mudrce nehub! Uveď mě před krále, sdělím králi výklad.“
#2:25 Arjók neprodleně uvedl Daniela před krále a řekl mu: „Našel jsem muže z judských přesídlenců, který králi oznámí výklad.“
#2:26 Král oslovil Daniela, který měl jméno Beltšasar: „Jsi schopen oznámit mi sen, který jsem měl, a jeho výklad?“
#2:27 Daniel králi odpověděl: „Tajemství, na které se král ptá, nemohou králi sdělit ani mudrci ani zaklínači ani věštci ani planetáři.
#2:28 Ale je Bůh v nebesích, který odhaluje tajemství. On dal králi Nebúkadnesarovi poznat, co se stane v posledních dnech. Toto je sen, totiž vidění, která ti prošla hlavou na tvém lůžku:
#2:29 Tobě, králi vyvstaly na lůžku starosti o to, co se v budoucnu stane, a ten, který odhaluje tajemství, ti oznámil, co se stane.
#2:30 Já pak to nemám z moudrosti, které bych měl více než ostatní živé bytosti, ale to tajemství mi bylo odhaleno, aby výklad byl králi oznámen a abys poznal myšlení svého srdce.
#2:31 Ty jsi, králi, viděl jakousi velikou sochu. Byla to obrovská socha a její lesk byl mimořádný. Stála proti tobě a měla strašný vzhled.
#2:32 Hlava té sochy byla z ryzího zlata, její hruď a paže ze stříbra, břicho a boky z mědi,
#2:33 stehna ze železa, nohy dílem ze železa a dílem z hlíny.
#2:34 Viděl jsi, jak se bez zásahu rukou utrhl kámen a udeřil do železných a hliněných nohou sochy a rozdrtil je,
#2:35 a rázem bylo rozdrceno železo, hlína, měď, stříbro i zlato, a byly jako plevy na mlatě v letní době. Odnesl je vítr a nezbylo po nich ani stopy. A ten kámen, který do sochy udeřil, se stal obrovskou skálou a zaplnil celou zemi.
#2:36 Toto je sen. Též jeho výklad řekneme králi:
#2:37 Ty, králi, jsi král králů. Bůh nebes ti dal království, moc, sílu a slávu.
#2:38 A všechna místa, kde bydlí lidé, polní zvěř a nebeské ptactvo, dal ti do rukou a dal ti moc nad tím vším. Ty jsi ta zlatá hlava.
#2:39 Po tobě povstane další království, nižší než tvé, a pak další, třetí království, měděné, které bude mít moc nad celou zemí.
#2:40 Čtvrté království bude tvrdé jako železo, neboť železo drtí a drolí vše, a to království jako železo, které tříští všechno, bude drtit a tříštit.
#2:41 Že jsi viděl nohy a prsty dílem z hrnčířské hlíny a dílem ze železa, znamená, že království bude rozdělené a bude v něm něco z pevnosti železa, neboť jsi viděl železo smíšené s jílovitou hlínou.
#2:42 Prsty nohou dílem ze železa a dílem z hlíny znamenají, že království bude zčásti tvrdé a dílem křehké.
#2:43 Že jsi viděl železo smíšené s jílovitou hlínou, znamená, že se bude lidské pokolení mísit, avšak nepřilnou k sobě navzájem, jako se nesmísí železo s hlínou.
#2:44 Ve dnech těch králů dá Bůh nebes povstat království, které nebude zničeno navěky, a to království nebude předáno jinému lidu. Rozdrtí a učiní konec všem těm královstvím, avšak samo zůstane navěky,
#2:45 neboť jsi viděl, že se utrhl ze skály kámen bez zásahu rukou a rozdrtil železo, měď, hlínu, stříbro i zlato. Veliký Bůh dal králi poznat, co se v budoucnu stane. Sen je pravdivý a výklad spolehlivý.“
#2:46 Tu král Nebúkadnesar padl na tvář, poklonil se před Danielem a rozkázal, aby mu byla obětována oběť přídavná s vonnými dary.
#2:47 Král Daniela oslovil: „Vpravdě, váš Bůh je Bohem bohů a Pán králů, který odhaluje tajemství, neboť jsi mi dokázal toto tajemství odhalit.“
#2:48 Král pak Daniela povýšil, dal mu mnoho velikých darů i moc nad celou babylónskou krajinou a učinil ho nejvyšším správcem všech babylónských mudrců.
#2:49 Ale Daniel prosil krále, aby správou babylónské krajiny pověřil Šadraka, Méšaka a Abed-nega. Daniel sám zůstal na královském dvoře. 
#3:1 Král Nebúkadnesar dal zhotovit zlatou sochu, jejíž výška byla šedesát loket a šířka šest loket. Postavil ji na pláni Dúra v babylónské krajině.
#3:2 Král Nebúkadnesar poslal pro satrapy, zemské správce a místodržitele, poradce, správce pokladu, soudce, vysoké úředníky a všechny zmocněnce nad krajinami, aby přišli k posvěcení sochy, kterou král Nebúkadnesar postavil.
#3:3 Tehdy se shromáždili satrapové, zemští správcové a místodržitelé, poradci, správcové pokladu, soudcové, vysocí úředníci a všichni zmocněnci nad krajinami k posvěcení sochy, kterou král Nebúkadnesar postavil. Stáli proti soše, kterou postavil Nebúkadnesar.
#3:4 Hlasatel mocně volal: „Poroučí se vám, lidé různých národností a jazyků:
#3:5 Jakmile uslyšíte hlas rohu, flétny, citary, harfy, loutny, dud a rozmanitých strunných nástrojů, padnete a pokloníte se před zlatou sochou, kterou postavil král Nebúkadnesar.
#3:6 Kdo nepadne a nepokloní se, bude v tu hodinu vhozen do rozpálené ohnivé pece.“
#3:7 Proto v určenou dobu, jakmile všichni lidé uslyšeli hlas rohu, flétny, citary, harfy, loutny a rozmanitých strunných nástrojů, všichni lidé různých národností a jazyků padli a klaněli se před zlatou sochou, kterou král Nebúkadnesar postavil.
#3:8 V té době přišli muži hvězdopravci a udali Judejce.
#3:9 Hlásili králi Nebúkadnesarovi: „Králi, navěky buď živ!
#3:10 Ty jsi, králi, vydal rozkaz, aby každý člověk, až uslyší hlas rohu, flétny, citary, harfy, loutny a dud a rozmanitých strunných nástrojů, padl a poklonil se před zlatou sochou.
#3:11 Kdo nepadne a nepokloní se, má být vhozen do rozpálené ohnivé pece.
#3:12 Jsou zde muži Judejci, které jsi pověřil správou babylónské krajiny, Šadrak, Méšak a Abed-nego. Tito muži nedbají, králi, na tvůj rozkaz, tvé bohy neuctívají a před zlatou sochou, kterou jsi postavil, se neklanějí.“
#3:13 Tehdy Nebúkadnesar, rozlícen a rozhořčen, rozkázal přivést Šadraka, Méšaka a Abed-nega. Tito muži byli hned přivedeni před krále.
#3:14 Nebúkadnesar se jich otázal: „Je to tak, Šadraku, Méšaku a Abed-nego, že mé bohy neuctíváte a před zlatou sochou, kterou jsem postavil, jste se nepoklonili?
#3:15 Nuže, jste ochotni v čase, kdy uslyšíte hlas rohu, flétny, citary, harfy, loutny a dud a rozmanitých strunných nástrojů, padnout a poklonit se před sochou, kterou jsem udělal? Jestliže se nepokloníte, v tu hodinu budete vhozeni do rozpálené ohnivé pece. A kdo je ten Bůh, který by vás vysvobodil z mých rukou!“
#3:16 Šadrak, Méšak a Abed-nego odpověděli králi: „Nebúkadnesare, nám není třeba dávat ti odpověď.
#3:17 Jestliže náš Bůh, kterého my uctíváme, nás bude chtít vysvobodit z rozpálené ohnivé pece i z tvých rukou, králi, vysvobodí nás.
#3:18 Ale i kdyby ne, věz, králi, že tvé bohy uctívat nebudeme a před zlatou sochou, kterou jsi postavil, se nepokloníme.“
#3:19 Tu se Nebúkadnesar velice rozlítil a výraz jeho tváře se vůči Šadrakovi, Méšakovi a Abed-negovi změnil. Rozkázal vytopit pec sedmkrát víc, než se obvykle vytápěla.
#3:20 Mužům, statečným bohatýrům, kteří byli v jeho vojsku, rozkázal Šadraka, Méšaka a Abed-nega svázat a vhodit je do rozpálené ohnivé pece.
#3:21 Ti muži byli hned svázáni ve svých pláštích a suknicích i s čepicemi a celým oblečením a vhozeni do rozpálené ohnivé pece.
#3:22 Protože královo slovo bylo přísné a pec byla nadmíru vytopena, ony muže, kteří Šadraka, Méšaka a Abed-nega vynesli, usmrtil plamen ohně.
#3:23 A ti tři muži, Šadrak, Méšak a Abed-nego, padli svázaní do rozpálené ohnivé pece.
#3:24 Tu král Nebúkadnesar užasl a chvatně vstal. Otázal se královské rady: „Což jsme nevhodili do ohně tři svázané muže?“ Odpověděli králi: „Jistěže, králi.“
#3:25 Král zvolal: „Hle, vidím čtyři muže, jsou rozvázaní a procházejí se uprostřed ohně bez jakékoli úhony. Ten čtvrtý se svým vzhledem podobá božímu synu.“
#3:26 I přistoupil Nebúkadnesar k otvoru rozpálené ohnivé pece a zvolal: „Šadraku, Méšaku a Abed-nego, služebníci Boha nejvyššího, vyjděte a pojďte sem!“ Šadrak, Méšak a Abed-nego vyšli z ohně.
#3:27 Satrapové, zemští správci a místodržitelé a královská rada se shromáždili, aby viděli ty muže, nad jejichž těly neměl oheň moc; ani vlas jejich hlavy nebyl sežehnut, jejich pláště nedoznaly změny, ani nebyly cítit ohněm.
#3:28 Nebúkadnesar zvolal: „Požehnán buď Bůh Šadrakův, Méšakův a Abed-negův, který poslal svého anděla a vysvobodil své služebníky, kteří na něj spoléhali. Přestoupili královo slovo a vydali svá těla, aby nemuseli vzdát poctu a klanět se nějakému jinému bohu než bohu svému.
#3:29 Vydávám rozkaz: Kdokoli z lidí kterékoli národnosti a jazyka by řekl něco proti Bohu Šadrakovu, Méšakovu a Abed-negovu, ať je rozsekán na kusy a jeho dům ať je učiněn hnojištěm, neboť není jiného Boha, který by mohl vyprostit jako tento Bůh.“
#3:30 A král zařídil, aby se Šadrakovi, Méšakovi a Abed-negovi v babylónské krajině dobře dařilo.
#3:31 Král Nebúkadnesar všem lidem různých národností a jazyků, kteří bydlí na celé zemi: „Rozhojněn buď váš pokoj!
#3:32 Zalíbilo se mi sdělit vám, jaká znamení a jaké divy učinil na mně Bůh nejvyšší.
#3:33 Jak veliká jsou jeho znamení, jak mocné jsou jeho divy! Jeho království je království věčné, jeho vladařská moc po všechna pokolení. 
#4:1 Já, Nebúkadnesar, jsem spokojeně pobýval ve svém domě, pln svěžesti ve svém paláci.
#4:2 Viděl jsem sen a ten mě vystrašil. Představy ve snu na lůžku, vidění, která i prošla hlavou, mě naplnily hrůzou.
#4:3 Vydal jsem rozkaz, aby ke mně byli uvedeni všichni babylónští mudrci, aby mi sen vyložili.
#4:4 Přišli tedy věštci, zaklínači, hvězdopravci a planetáři. Vyprávěl jsem jim sen, ale jeho výklad mi nemohli oznámit.
#4:5 Konečně ke mně přišel Daniel, který má jméno Beltšasar podle jména mého boha; v něm je duch svatých bohů. Vyprávěl jsem mu sen:
#4:6 Beltšasare, nejvyšší z věštců, vím, že v tobě je duch svatých bohů a že žádné tajemství ti nedělá potíže. Pověz mi výklad vidění snu, který jsem viděl.
#4:7 Ve viděních, která mi prošla hlavou na mém lůžku jsem viděl: Hle, strom stál uprostřed země, jeho výška byla obrovská.
#4:8 Strom rostl a sílil, až jeho výška sahala k nebi. Bylo jej vidět od samého konce země.
#4:9 Měl nádherné listí a mnoho plodů, byla na něm potrava pro všechny. Polní zvěř pod ním nalézala stín, v jeho větvích bydleli nebeští ptáci a sytilo se z něho všechno tvorstvo.
#4:10 Ve viděních, která mi prošla hlavou na mém lůžku, jsem viděl: Hle, posel, a to svatý, sestupoval z nebe.
#4:11 Mocně volal a nařizoval toto: ‚Skácejte strom! Osekejte mu větve! Otrhejte mu listí! Rozházejte jeho plody! Ať uteče zvěř, která byla pod ním, i ptáci z jeho větví!
#4:12 Avšak pařez s kořeny ponechte v zemi, sevřený obručí z železa a bronzu, ve svěží zeleni pole; ať je skrápěn nebeskou rosou a se zvěří ať se dělí o rostliny země.
#4:13 Jeho srdce ať je jiné, než je srdce lidské, ať je mu dáno srdce zvířecí, dokud nad ním neuplyne sedm let.
#4:14 V rozhodnutí nebeských poslů je rozsudek, výpovědí svatých je věc uzavřená. Z toho živí poznají, že Nejvyšší má moc nad lidským královstvím a komu chce, je dává; může nad ním ustanovit i nejnižšího z lidí.‘
#4:15 Tento sen jsem viděl já, král Nebúkadnesar, a ty, Beltšasare, mi řekni jeho výklad. Žádný z mudrců mého království mi nemohl výklad oznámit. Ty však jsi toho schopen, neboť v tobě je duch svatých bohů.“
#4:16 Tu Daniel, který měl jméno Beltšasar, zůstal skoro hodinu strnulý a jeho myšlenky ho plnily hrůzou. Král mu řekl: „Beltšasare, snu ani výkladu se nehroz.“ Beltšasar odpověděl: „Můj pane, kéž by sen platil tvým nepřátelům a jeho výklad tvým protivníkům.
#4:17 Strom, který jsi viděl, který rostl a sílil, až jeho výška sahala k nebi a bylo ho vidět po celé zemi,
#4:18 který měl nádherné listí a mnoho plodů, na němž byla potrava pro všechny, pod nímž bydlela polní zvěř a v jehož větvích přebývali nebeští ptáci,
#4:19 jsi ty, králi, který jsi rostl a sílil. Tvá velikost rostla, až dosáhla k nebi, tvá vladařská moc až na konec země.
#4:20 Král viděl potom posla, a to svatého, jak sestupoval z nebe a nařizoval: ‚Skácejte strom a zničte jej, avšak pařez s kořeny ponechte v zemi sevřený obručí z železa a bronzu ve svěží zeleni pole, ať je skrápěn nebeskou rosou a ať má podíl s polní zvěří, dokud nad ním neuplyne sedm let.‘
#4:21 Toto je výklad, králi: Je to rozhodnutí Nejvyššího, které dopadlo na krále, mého pána.
#4:22 Vyženou tě pryč od lidí a budeš bydlet s polní zvěří. Za pokrm ti dají rostliny jako dobytku a nechají tě skrápět nebeskou rosou. Tak nad tebou uplyne sedm let, dokud nepoznáš, že Nejvyšší má moc nad lidským královstvím a že je dává, komu chce.
#4:23 A že bylo řečeno, aby byl pařez toho stromu i s kořeny ponechán, tvé království se ti opět dostane, až poznáš, že Nebesa mají moc.
#4:24 Kéž se ti, králi, zalíbí má rada: Překonej své hříchy spravedlností a svá provinění milostí k strádajícím; snad ti bude prodloužen klid.“
#4:25 To všechno dopadlo na krále Nebúkadnesara.
#4:26 Uplynulo dvanáct měsíců. Král se procházel po královském paláci v Babylóně
#4:27 a řekl: „Zdali není veliký tento Babylón, který jsem svou mocí a silou vybudoval jako královský dům ke slávě své důstojnosti?“
#4:28 Ještě to slovo bylo v ústech krále, když se snesl hlas z nebe: „Tobě je to řečeno, králi Nebúkadnesare: Tvé království od tebe odešlo.
#4:29 Vyženou tě pryč od lidí a budeš bydlit s polní zvěří. Dají ti za pokrm rostliny jako dobytku. Tak nad tebou uplyne sedm let, dokud nepoznáš, že Nejvyšší má moc nad lidským královstvím a že je dává, komu chce.“
#4:30 V tu hodinu se to slovo na Nebúkadnesarovi splnilo. Byl vyhnán pryč od lidí, pojídal rostliny jako dobytek, jeho tělo bylo skrápěno nebeskou rosou, až mu narostly vlasy jako peří orlům a nehty jako drápy ptákům.
#4:31 „Když uplynuly ty dny, pozdvihl jsem já Nebúkadnesar své oči k nebi a rozum se mi vrátil. Dobrořečil jsem Nejvyššímu a chválil jsem a velebil Věčně živého, neboť jeho vladařská moc je věčná, jeho království po všechna pokolení.
#4:32 Všichni obyvatelé země jsou považováni za nic. Podle své vůle nakládá s nebeským vojskem i s obyvateli země. Není, kdo by mohl zabraňovat jeho ruce a ptát se ho: ‚Co to děláš?‘
#4:33 Tou dobou se mi vrátil rozum a ke slávě mého království mi opět byla vrácena má důstojnost a lesk. Moje královská rada a hodnostáři mě vyhledali, opět jsem byl dosazen do svého království a byla mi přidána mimořádná velikost.
#4:34 Nyní tedy já Nebúkadnesar chválím, vyvyšuji a velebím Krále nebes. Všechno jeho dílo je pravda, jeho cesty právo. Ty, kteří si vedou pyšně, má moc ponížit.“ 
#5:1 Král Belšasar vystrojil velikou hostinu svým tisíci hodnostářům a před těmito tisíci pil víno.
#5:2 Při popíjení vína poručil Belšasar přinést zlaté a stříbrné nádoby, které odnesl Nebúkadnesar, jeho otec, z jeruzalémského chrámu, aby z nich pili král i jeho hodnostáři, jeho ženy i ženiny.
#5:3 Hned tedy přinesli zlaté nádoby odnesené z chrámu, to je z Božího domu v Jeruzalémě, a pili z nich král i jeho hodnostáři, jeho ženy i ženiny.
#5:4 Pili víno a chválili bohy zlaté a stříbrné, bronzové, železné, dřevěné a kamenné.
#5:5 V tu hodinu se ukázaly prsty lidské ruky a něco psaly na omítku zdi královského paláce naproti svícnu. Král viděl zápěstí ruky, která psala.
#5:6 Tu se barva králova obličeje změnila a myšlenky ho naplnily hrůzou, poklesl v kyčlích a kolena mu tloukla o sebe.
#5:7 Král mocně zvolal, aby přivedli zaklínače, hvězdopravce a planetáře. Babylónským mudrcům král řekl: „Kdokoli přečte toto písmo a sdělí mi výklad, bude oblečen do purpuru, na krk mu bude dán zlatý řetěz a bude mít v království moc jako třetí po mně.“
#5:8 Všichni královští mudrci tedy vstoupili, ale nebyli schopni písmo přečíst a oznámit králi výklad.
#5:9 Král Belšasar byl pln hrůzy a barva jeho obličeje se změnila. I hodnostáři byli zmateni.
#5:10 Po slovech krále a hodnostářů vešla do domu, kde hodovali, královna a řekla: „Králi, navěky buď živ! Nechť tě tvé myšlenky neplní hrůzou a barva tvého obličeje ať se nemění.
#5:11 Je v tvém království muž, v němž je duch svatých bohů. Za dnů tvého otce bylo shledáno, že je osvícený a zběhlý v moudrosti, která je jako moudrost bohů. Král Nebúkadnesar, tvůj otec, ho ustanovil nejvyšším z věštců, zaklínačů, hvězdopravců a planetářů, ano, tvůj otec, králi,
#5:12 neboť bylo shledáno, že Daniel, jemuž král dal jméno Beltšasar, má mimořádného ducha a poznání a že je zběhlý ve vykládání snů, řešení záhad a vysvětlování věcí nesnadných. Nechť je Daniel nyní zavolán a sdělí výklad.“
#5:13 Daniel byl hned přiveden ke králi. Král se Daniela otázal: „Ty jsi Daniel z judských přesídlenců, kterého přivedl král, můj otec, z Judska?
#5:14 Slyšel jsem o tobě, že je v tobě duch bohů a že bylo shledáno, že jsi osvícený a zběhlý v mimořádné moudrosti.
#5:15 Byli ke mně přivedeni mudrci, zaklínači, aby mi přečetli toto písmo a oznámili mi jeho výklad, ale nejsou schopni výklad té věci sdělit.
#5:16 O tobě jsem slyšel, že dokážeš podat výklad a vysvětlit nesnadné. Nyní tedy, dokážeš-li to písmo přečíst a výklad mi oznámit, budeš oblečen do purpuru, na krk ti bude dán zlatý řetěz a budeš mít v království moc jako třetí.“
#5:17 Daniel na to králi odpověděl: „Své dary si ponech a své odměny dej jinému. To písmo však králi přečtu a výklad mu oznámím.
#5:18 Slyš, králi! Bůh nejvyšší dal Nebúkadnesarovi, tvému otci, království a velikost, slávu a důstojnost.
#5:19 Pro velikost, kterou mu dal, se před ním třásli všichni lidé různých národností a jazyků a obávali se ho. Koho chtěl, zabil, koho chtěl, nechal žít, koho chtěl, povýšil, koho chtěl, ponížil.
#5:20 Když se jeho srdce povýšilo a jeho duch se stal náramně zpupný, byl svržen ze svého královského stolce a jeho sláva mu byla odňata.
#5:21 Byl vyhnán pryč od lidí, jeho srdce se stalo podobné zvířecímu, bydlil s divokými osly, za pokrm mu dávali rostliny jako dobytku a jeho tělo bylo skrápěno nebeskou rosou, dokud nepoznal, že Bůh nejvyšší má moc nad lidským královstvím a že nad ním ustanovuje, koho chce.
#5:22 Ani ty, jeho synu Belšasare, jsi neponížil své srdce, ačkoli jsi o tom všem věděl,
#5:23 ale povýšil ses nad Pána nebes. Přinesli před tebe nádoby z jeho domu a pil jsi z nich víno ty i tvoji hodnostáři, tvé ženy i ženiny, a chválil jsi bohy stříbrné a zlaté, bronzové, železné, dřevěné a kamenné, kteří nic nevidí, neslyší ani nevědí. Boha, v jehož rukou je tvůj dech a všechny tvé cesty, jsi však nevelebil.
#5:24 Proto bylo od něho posláno zápěstí ruky a napsáno toto písmo.
#5:25 Toto pak je písmo, které bylo napsáno: ‚Mené, mené, tekel ú-parsín‘.
#5:26 Toto je výklad těch slov: Mené - Bůh sečetl tvé kralování a ukončil je.
#5:27 Tekel - byl jsi zvážen na vahách a shledán lehký.
#5:28 Peres - tvé království bylo rozlomeno a dáno Médům a Peršanům.“
#5:29 Belšasar hned poručil, aby Daniela oblékli do purpuru, na krk mu dali zlatý řetěz a rozhlásili o něm, že má v království moc jako třetí.
#5:30 Ještě té noci byl kaldejský král Belšasar zabit. 
#6:1 Darjaveš médský se ujal království ve věku šedesáti dvou let.
#6:2 Darjavešovi se zalíbilo ustanovit nad královstvím sto dvacet satrapů, aby byli po celém království.
#6:3 Nad nimi byli tři říšští vládcové, z nichž jedním byl Daniel. Těm podávali satrapové hlášení, aby se tím král nemusel obtěžovat.
#6:4 Daniel pak vynikal nad říšské vládce i satrapy, neboť v něm byl mimořádný duch. Král ho zamýšlel ustanovit nad celým královstvím.
#6:5 Tu se říšští vládcové a satrapové snažili nalézt proti Danielovi záminku ohledně jeho správy království, ale žádnou záminku ani zlé jednání nalézt nemohli, neboť byl věrný. Žádnou nedbalost ani zlé jednání na něm neshledali.
#6:6 Proto si ti muži řekli: „Nenajdeme proti Danielovi žádnou záminku, ledaže bychom našli proti němu něco, co se týče zákona jeho Boha.“
#6:7 Říšští vládcové a satrapové se shlukli ke králi a naléhali na něj: „Králi Darjaveši, navěky buď živ!
#6:8 Všichni královští vládci, zemští správcové a satrapové, královská rada a místodržitelé se uradili, abys královským výnosem potvrdil zákaz: Každý, kdo by se v údobí třiceti dnů obracel v modlitbě na kteréhokoli boha nebo člověka kromě na tebe, králi, ať je vhozen do lví jámy.
#6:9 Nyní, králi, vydej zákaz a podepiš přípis, který by podle nezrušitelného zákona Médů a Peršanů nesměl být změněn.“
#6:10 Král Darjaveš tedy podepsal přípis a zákaz.
#6:11 Když se Daniel dověděl, že byl podepsán přípis, vešel do svého domu, kde měl v horní pokoji otevřená okna směrem k Jeruzalému. Třikrát za den klekal na kolena, modlil se a vzdával čest svému Bohu, jako to činíval dříve.
#6:12 Tu se ti muži shlukli a přistihli Daniela, jak se modlí a prosí svého Boha o milost.
#6:13 Hned šli ke králi a dovolávali se královského zákazu: „Zdali jsi nepodepsal zákaz, že každý člověk, který by se v údobí třiceti dnů modlil ke kterémukoli bohu nebo člověku kromě k tobě, králi, bude vhozen do lví jámy?“ Král odpověděl: „To slovo platí podle nezrušitelného zákona Médů a Peršanů.“
#6:14 Na to králi řekli: „Daniel z judských přesídlenců, králi, na tebe a zákaz, který si podepsal nedbá. Třikrát za den se modlí svou modlitbu.“
#6:15 Když král slyšel takovou řeč, byl velmi znechucen. Usilovně přemýšlel, jak by Daniela vysvobodil. Namáhal se až do západu slunce, jak by ho vyprostil.
#6:16 Ti muži se však shlukli ke králi a naléhali na něj: „Věz, králi, podle zákona Médů a Peršanů žádný zákaz ani výnos, který král vydá, nesmí být změněn.“
#6:17 Král tedy poručil, aby přivedli Daniela a vhodili ho do jámy, v níž byli lvi. Danielovi řekl: „Kéž tě tvůj Bůh, kterého stále uctíváš, vysvobodí.“
#6:18 Donesli jeden kámen a položili ho na otvor jámy. Král jej zapečetil pečetním prstenem svým a pečetními prsteny svých hodnostářů, aby se v Danielově záležitosti nedalo nic změnit.
#6:19 Pak se král odebral do svého paláce a ulehl, aniž co pojedl. Nedopřál si žádné obveselení a spánek se mu vyhýbal.
#6:20 Jak se začalo rozednívat, hned za úsvitu, král vstal a chvatně odešel k jámě, kde byli lvi.
#6:21 Když přišel k jámě, zarmouceným hlasem zavolal na Daniela. Řekl Danielovi: „Danieli, služebníku Boha živého, dokázal tě Bůh, kterého stále uctíváš, zachránit před lvy?“
#6:22 Tu Daniel promluvil ke králi: „Králi, navěky buď živ!
#6:23 Můj Bůh poslal a svého anděla a zavřel ústa lvům, takže mi neublížili. Vždyť jsem byl před ním shledán čistý a ani proti tobě, králi, jsem se ničeho zlého nedopustil.“
#6:24 Král tím byl velice potěšen a poručil, aby Daniela vytáhli z jámy. Daniel byl tedy z jámy vytažen a nebyla na něm shledána žádná úhona, protože věřil ve svého Boha.
#6:25 Král pak poručil, aby přivedli ty muže, kteří Daniela udali, a hodili je do lví jámy i s jejich syny a ženami. Ještě nedopadli na dno jámy, už se jich zmocnili lvi a rozdrtili jim všechny kosti.
#6:26 Tehdy král Darjaveš napsal všem lidem různých národností a jazyků, bydlícím po celé zemi: „Rozhojněn buď váš pokoj!
#6:27 Vydávám rozkaz, aby se v celé mé královské říši všichni třásli před Danielovým Bohem a obávali se ho, neboť on je Bůh živý a zůstává navěky, jeho království nebude zničeno a jeho vladařská moc bude až do konce.
#6:28 Vysvobozuje a vytrhuje, činí znamení a divy na nebi i na zemi. On vysvobodil Daniela ze lvích spárů.“
#6:29 Danielovi se pak dobře dařilo v království Darjavešově i v království Kýra perského. 
#7:1 V prvním roce vlády Belšasara, krále babylónského, viděl Daniel sen, vidění mu prošla hlavou na jeho lůžku. Hned tedy v hlavních rysech ten sen popsal.
#7:2 Daniel řekl: „Viděl jsem v nočním vidění, hle, čtyři nebeské větry rozbouřily Velké moře.
#7:3 A z moře vystoupila čtyři veliká zvířata, odlišná jedno od druhého.
#7:4 První bylo jako lev a mělo orlí křídla. Viděl jsem, že mu byla křídla oškubána, bylo pozvednuto od země a postaveno na nohy jako člověk a dáno lidské srdce.
#7:5 Hle, další zvíře, druhé, se podobalo medvědu. Bylo postaveno tváří k jedné straně. Mělo v tlamě mezi zuby tři žebra a bylo mu řečeno: ‚Vstaň a hojně se nažer masa!‘
#7:6 Potom jsem viděl, hle, další zvíře bylo jako levhart a mělo na hřbetě čtyři ptačí křídla. Bylo to zvíře čtyřhlavé a byla mu dána vladařská moc.
#7:7 Potom jsem v nočním vidění viděl, hle, čtvrté zvíře, strašné, příšerné a mimořádně mocné. Mělo veliké železné zuby, žralo a drtilo a zbytek rozšlapávalo svýma nohama. Bylo odlišné ode všech předešlých zvířat a mělo deset rohů.
#7:8 Prohlížel jsem rohy, a hle, vyrostl mezi nimi další malý roh a tři z dřívějších rohů byly před ním vyvráceny. Hle, na tom rohu byly oči jako oči lidské a ústa, která mluvila troufale.
#7:9 Viděl jsem, že byly postaveny stolce a že usedl Věkovitý. Jeho oblek byl bílý jako sníh, vlasy jeho hlavy jako čistá vlna, jeho stolec - plameny ohně, jeho kola - hořící oheň.
#7:10 Řeka ohnivá proudila a vycházela od něho, tisíce tisíců sloužily jemu a desetitíce desetitisíců stály před ním. Zasedl soud a byly otevřeny knihy.
#7:11 Tu jsem viděl, že pro ta troufalá slova, která roh mluvil, viděl jsem, že to zvíře bylo zabito, jeho tělo zničeno a dáno k spálení ohněm.
#7:12 Zbylým zvířatům odňali jejich vladařskou moc a byl jim ponechán život do určité doby a času.
#7:13 Viděl jsem v nočním vidění, hle, s nebeskými oblaky přicházel jakoby Syn člověka; došel až k Věkovitému, přivedli ho k němu.
#7:14 A byla mu dána vladařská moc, sláva a království, aby ho uctívali všichni lidé různých národností a jazyků. Jeho vladařská moc je věčná, která nepomine, a jeho království nebude zničeno.“
#7:15 Můj duch, můj, Danielův, byl uvnitř své schránky zmatený a vidění, která mi prošla hlavou, mě naplnila hrůzou.
#7:16 Přistoupil jsem k jednomu z těch, kteří tam stáli, a prosil jsem ho o hodnověrný výklad toho všeho. Řekl mi to a oznámil mi výklad té věci:
#7:17 „Ta čtyři veliká zvířata, to čtyři králové povstanou v zemi.
#7:18 Ale království se ujmou svatí Nejvyššího a budou mít království v držení až na věky, totiž až na věky věků.“
#7:19 Chtěl jsem mít jistotu o tom čtvrtém zvířeti, které bylo odlišné ode všech ostatních a bylo mimořádně strašné: mělo železné zuby, bronzové drápy, žralo a drtilo a zbytek rozšlapávalo svýma nohama,
#7:20 i o deseti rozích, které mělo na hlavě, a o dalším, který vyrostl a před nímž tři spadly, totiž o tom rohu, který měl oči a ústa mluvící troufale a jevil se větší než ostatní.
#7:21 Viděl jsem, že ten roh vedl válku proti svatým a přemáhal je,
#7:22 až přišel Věkovitý a soud byl předán svatým Nejvyššího; nadešla doba a království dostali do držení svatí.
#7:23 Řekl toto: „Čtvrté zvíře - na zemi bude čtvrté království, to se bude ode všech království lišit; pozře celou zemi, podupe ji a rozdrtí.
#7:24 A deset rohů - z toho království povstane deset králů. Po nich povstane jiný, ten se bude od předchozích lišit a sesadí tři krále.
#7:25 Bude mluvit proti Nejvyššímu a bude hubit svaté Nejvyššího. Bude se snažit změnit doby a zákon. Svatí budou vydáni do jeho rukou až do času a časů a poloviny času,
#7:26 avšak zasedne soud a vladařskou moc mu odejmou, a bude úplně vyhlazen a zahuben.
#7:27 Království, vladařská moc a velikost všech království pod celým nebem budou dány lidu svatých Nejvyššího. Jeho království bude království věčné a všechny vladařské moci ho budu uctívat a poslouchat.“
#7:28 Zde končí to slovo. Já, Daniel, jsem se velice zhrozil těch myšlenek, barva mé tváře se změnila, ale to slovo jsem uchoval ve svém srdci. 
#8:1 V třetím roce kralování krále Belšasara ukázalo se mně, Danielovi, vidění, po onom, které se mi ukázalo na počátku.
#8:2 Viděl jsem ve vidění - byl jsem ve vidění na hradě Šúšanu v élamské krajině - viděl jsem tedy ve vidění, že jsem u řeky Úlaje:
#8:3 Pozvedl jsem oči a spatřil jsem, hle, jeden beran stál před řekou a měl dva rohy. Ty rohy byly veliké, jeden však byl větší než druhý; větší vyrostl jako poslední.
#8:4 Viděl jsem berana trkat směrem k moři, na sever a na jih. Žádné zvíře před ním neobstálo a nikdo nic nevyprostil z jeho moci. Dělal, co se mu zlíbilo, a vzmohl se.
#8:5 Pozoroval jsem, a hle, ze západu přicházel kozel na celou zemi, ale země se nedotýkal. Ten kozel měl mezi očima nápadný roh.
#8:6 Přišel až k dvourohému beranovi, kterého jsem viděl stát nad řekou. Přiběhl k němu silně rozzuřen.
#8:7 Viděl jsem, že dostihl berana, rozlíceně do něho vrazil a zlomil mu oba rohy a beran neměl sílu mu odolat. Kozel ho povalil na zem a rozšlapal a nebyl nikdo, kdo by berana vyprostil z jeho moci.
#8:8 Kozel se velice vzmohl. Když byl na vrcholu moci, zlomil se ten velký roh a místo něho vyrostly čtyři nápadné rohy do čtyř nebeských větrů.
#8:9 Z jednoho z nich vyrazil jeden maličký roh, který se velmi vzmáhal na jih a na východ i k nádherné zemi.
#8:10 Vzmohl se tak, že sahal až k nebeskému zástupu, srazil na zem část toho zástupu, totiž hvězd, a rozšlapal je.
#8:11 Vypjal se až k veliteli toho zástupu, zrušil každodenní oběť a rozvrátil příbytek jeho svatyně.
#8:12 Zástup byl sveden ke vzpouře proti každodenní oběti. Pravdu srazil na zem a dařilo se mu, co činil.
#8:13 Slyšel jsem, jak jeden svatý mluví. Jiný svatý se toho mluvícího otázal: „Jak dlouho bude platit vidění o každodenní oběti a o vzpouře, která pustoší a dovoluje šlapat po svatyni i zástupu?“
#8:14 Řekl mi: „Až po dvou tisících a třech stech večerech a jitrech dojde svatyně spravedlnosti.“
#8:15 Když jsem já Daniel uviděl to vidění a snažil se mu porozumět, hle, postavil se naproti mně kdosi podobný muži
#8:16 a uslyšel jsem nad Úlajem lidský hlas, který takto volal: „Gabrieli, vysvětli mu to vidění.“
#8:17 Přišel tedy tam, kde jsem stál; zatímco přicházel, byl jsem ohromen a padl jsem tváří k zemi. Řekl mi: „Pochop, lidský synu, že to vidění se týká doby konce.“
#8:18 Když se mnou mluvil, ležel jsem v mrákotách tváří na zemi. Dotkl se mě a postavil mě na mé místo.
#8:19 Řekl : „Hle, sdělím ti, co se stane v posledním hrozném hněvu, neboť se to týká konce času.
#8:20 Dvourohý beran, kterého jsi viděl, jsou králové médští a perští.
#8:21 Chlupatý kozel je král řecký a veliký roh, který měl mezi očima, je první král.
#8:22 To, že se roh zlomil a místo něho vyvstaly čtyři, znamená, že vyvstanou čtyři království z toho pronároda, ale nebudou mít jeho sílu.
#8:23 Ke konci jejich kralování, až se naplní míra vzpurných, povstane král nestoudný, který bude rozumět hádankám.
#8:24 Bude zdatný svou silou, a nejen svou silou, bude přinášet neobyčejnou zkázu a jeho konání bude provázet zdar. Uvrhne do zkázy zdatné a lid svatých.
#8:25 Obezřetně a se zdarem bude jeho ruka lstivě jednat; ve svém srdci se bude vypínat, nerušeně uvrhne do zkázy mnohé. Postaví se proti Veliteli velitelů, avšak bude zlomen bez zásahu ruky.
#8:26 Vidění o večerech a jitrech, jak ti bylo pověděno, je pravdivé. Ty pak podrž to vidění v tajnosti, neboť se uskuteční za mnoho dnů.“
#8:27 Já Daniel jsem zůstal bez sebe a byl jsem těžce nemocen po řadu dní. Pak jsem vstal a konal své povolání u krále. Žasl jsem nad tím viděním, ale nikdo to nechápal. 
#9:1 V prvním roce vlády Darjaveše, syna Achašvérošova, který byl médského původu a byl králem nad královstvím kaldejským,
#9:2 v tom prvním roce jeho kralování jsem já Daniel porozuměl z knih počtu roků, o nichž se stalo slovo Hospodinovo proroku Jeremjášovi; vyplní se, že Jeruzalém bude po sedmdesát let v troskách.
#9:3 Obrátil jsem se k Panovníku Bohu, abych ho vyhledal modlitbou a prosbami o smilování v postu, žíněném rouchu a popelu.
#9:4 Modlil jsem se k Hospodinu, svému Bohu, a vyznával se mu slovy: „Ach, Panovníku, Bože veliký a hrozný, který dbáš na smlouvu a milosrdenství vůči těm, kteří tě milují a dodržují tvá přikázání!
#9:5 Zhřešili jsme a provinili se, jednali jsme svévolně, bouřili se a uchýlili od tvých přikázání a soudů.
#9:6 A neposlouchali jsme tvé služebníky proroky, kteří mluvili ve tvém jménu našim králům, našim velmožům, našim otcům a všemu lidu země.
#9:7 Na tvé straně, Panovníku, je spravedlnost, na nás je zjevná hanba až do tohoto dne; je na každém judském muži, na obyvatelích Jeruzaléma, na celém Izraeli, na blízkých i dalekých, ve všech zemích, do nichž jsi je rozehnal pro jejich zpronevěru, které se vůči tobě dopustili.
#9:8 Hospodine, na nás je zjevná hanba, na našich králích, na našich velmožích a na našich otcích, neboť jsme proti tobě zhřešili.
#9:9 Na Panovníku, našem Bohu, závisí slitování a odpuštění, neboť jsme se bouřili proti němu
#9:10 a neposlouchali jsme Hospodina, našeho Boha, a neřídili se jeho zákony, které nám vydával skrze své služebníky proroky.
#9:11 Celý Izrael přestoupil tvůj zákon a odchýlil se a neposlouchal tebe. Na nás je vylita kletba a zlořečení, jak je o tom psáno v zákoně Mojžíše, Božího služebníka, protože jsme proti tobě hřešili.
#9:12 Dodržel jsi své slovo, které jsi promluvil proti nám a proti našim soudcům, kteří nás soudili, že uvedeš na nás zlo tak veliké, že takové nebylo učiněno pod celým nebem; tak bylo učiněno v Jeruzalémě.
#9:13 Jak je psáno v zákoně Mojžíšově, přišlo na nás všechno to zlo. Neprosili jsme Hospodina, svého Boha, o shovívavost a neodvrátili se od svých nepravostí a nejednali prozíravě podle jeho pravdy.
#9:14 Proto Hospodin bděl nad tím zlem a uvedl je na nás, neboť Hospodin, náš Bůh, je spravedlivý ve všech svých činech, které koná, ale my jsme ho neposlouchali.
#9:15 Nyní, Panovníku, náš Bože, který jsi svůj lid vyvedl z egyptské země pevnou rukou, a tak sis učinil jméno, jaké máš až dodnes, zhřešili jsme, byli jsme svévolní.
#9:16 Panovníku, nechť se prosím podle tvé veškeré spravedlnosti odvrátí tvůj hněv a tvé rozhořčení od tvého města Jeruzaléma, tvé svaté hory, neboť pro naše hříchy a viny našich otců je Jeruzalém a tvůj lid tupen všemi, kteří jsou kolem nás.
#9:17 Nyní tedy, Bože náš, slyš modlitbu svého služebníka a jeho prosby o smilování a rozjasni tvář nad svou zpustošenou svatyní, kvůli sobě, Panovníku.
#9:18 Nakloň, můj Bože, své ucho a slyš, otevři své oči a viz, jak jsme zpustošeni my i město, které se nazývá tvým jménem. Vždyť ne pro své spravedlivé činy ti předkládáme své prosby o smilování, ale pro tvé velké slitování.
#9:19 Panovníku, slyš! Panovníku, odpusť! Panovníku, pozoruj a jednej! Neprodlévej! Kvůli sobě, můj Bože! Vždyť tvé město i tvůj lid se nazývá tvým jménem.“
#9:20 Ještě jsem rozmlouval a modlil se, vyznával hřích svůj i hřích Izraele, svého lidu, a předkládal Hospodinu, svému Bohu své prosby o smilování za svatou horu Boží,
#9:21 tedy ještě jsem rozmlouval v modlitbě, když Gabriel, ten muž, kterého jsem prve viděl ve vidění, spěšně přilétl a dotkl se mě v době večerního obětního daru.
#9:22 Poučil mě, když se mnou mluvil. Řekl: „Danieli, nyní jsem vyšel, abych ti posloužil k poučení.
#9:23 Prve při tvých prosbách o smilování vyšlo slovo a já jsem přišel, abych ti je oznámil, neboť jsi vzácný Bohu. Pochop to slovo a rozuměj vidění.
#9:24 Sedmdesát týdnů let je stanoveno tvému lidu a tvému svatému městu, než bude skoncováno s nevěrností, než budou zapečetěny hříchy, než dojde k zproštění viny, k uvedení věčné spravedlnosti, k zapečetění vidění a proroctví, k pomazání svatyně svatých.
#9:25 Věz a pochop! Od vyjití slova o navrácení a vybudování Jeruzaléma až k pomazanému vévodovi uplyne sedm týdnů. Za šedesát dva týdny bude opět vybudováno prostranství a příkop. Ale budou to svízelné doby.
#9:26 Po uplynutí šedesáti dvou týdnů bude pomazaný zahlazen a nebude již. Město a svatyni uvrhne do zkázy lid vévody, který přijde. Sám skončí v povodni, ale až do konce bude válka. Je rozhodnuto o pustošení.
#9:27 Vnutí svou smlouvu mnohým v jednom týdnu a v polovině toho týdne zastaví obětní hod i oběť přídavnou. Hle, pustošitel na křídlech ohyzdné modly, než se naplní čas a na pustošitele bude vylito rozhodnutí.“ 
#10:1 V třetím roce vlády Kýra, krále perského, bylo Danielovi, který byl pojmenován Beltšasar, zjeveno slovo. A to slovo je pravdivé; týká se velké strasti. Pochopil to slovo. Pochopení mu bylo dáno ve viděních.
#10:2 V těch dnech jsem já, Daniel, truchlil po celé tři týdny.
#10:3 Chutný chléb jsem nejedl, maso a víno jsem nevzal do úst, ani jsem se nepotíral mastí až do uplynutí celých tří týdnů.
#10:4 Dvacátého čtvrtého dne prvního měsíce jsem byl na břehu veliké řeky zvané Chidekel.
#10:5 Pozvedl jsem oči a spatřil jsem, hle, jakýsi muž oblečený ve lněném oděvu. Na bedrech měl pás z třpytivého zlata z Úfazu.
#10:6 Tělo měl jako chrysolit, tvář jako blesk, oči jako hořící pochodně, paže a nohy jako lesknoucí se bronz a zvuk jeho slov byl jako hrozný hluk.
#10:7 Já Daniel jsem to vidění viděl sám. Muži, kteří byli se mnou, žádné vidění neviděli, ale padlo na ně veliké zděšení a uprchli do úkrytu.
#10:8 Zůstal jsem sám a viděl jsem to veliké vidění, ale nezůstala ve mně síla. Velebnost mé tváře se změnila a byla zcela porušena; nezachoval jsem si sílu.
#10:9 Slyšel jsem zvuk jeho slov a jak jsem zvuk jeho slov uslyšel, přišly na mě mrákoty a padl jsem na tvář, totiž tváří na zem.
#10:10 A hle, dotkla se mě ruka a zatřásla mnou, takže jsem se pozvedl na kolena a dlaně rukou.
#10:11 Muž mi řekl: „Danieli, muži vzácný, pochop slova, která k tobě budu mluvit. Stůj na svém místě. Jsem poslán k tobě.“ Když se mnou mluvil to slovo, stál jsem a chvěl jsem se.
#10:12 Řekl mi: „Neboj se, Danieli, neboť od prvního dne, kdy ses rozhodl porozumět a pokořit se před svým Bohem, byla tvá slova vyslyšena a já jsem proto přišel.
#10:13 Avšak ochránce perského království stál proti mně po jednadvacet dní. Dokud mi nepřišel na pomoc Míkael, jeden z předních ochránců, zůstal jsem tam u perských králů.
#10:14 Přišel jsem, abych tě poučil o tom, co potká tvůj lid v posledních dnech, neboť vidění se týká těchto dnů.“
#10:15 Když ke mně promluvil tato slova, sklonil jsem tvář k zemi a oněměl jsem.
#10:16 A hle, kdosi podobný lidským synům se dotkl mých rtů. Otevřel jsem ústa a promluvil jsem k tomu, který stál proti mně: „Můj pane, pro to vidění mě sevřely křeče a nezachoval jsem si sílu.
#10:17 Jak by mohl služebník tohoto mého pána mluvit s tímto mým pánem? Od té doby není ve mně síla a nezůstal ve mně dech.“
#10:18 Opět se mě dotkl kdosi, kdo vypadal jako člověk, a dodal mi sílu.
#10:19 Řekl: „Neboj se, muži vzácný, pokoj tobě! Vzchop se, vzchop se!“ Když se mnou mluvil, nabyl jsem síly. Řekl jsem: „Ať můj pán mluví, neboť jsem již posílen.“
#10:20 Řekl: „Víš, proč jsem k tobě přišel? Nyní se opět vrátím, abych bojoval s ochráncem Peršanů. Odcházím, a hle, přichází ochránce Řeků.
#10:21 Zajisté, oznámím ti, co je zapsáno ve spisu pravdy. Není nikoho, kdo by mi dodával sílu v těch věcech, kromě vašeho ochránce Míkaela.“ 
#11:1 „V prvním roce vlády Darjaveše médského jsem stál při něm, abych mu dodával sílu a byl mu záštitou.
#11:2 Nyní ti tedy oznámím pravdu: Hle, v Persii povstanou ještě tři králové. Čtvrtý pak nabude většího bohatství než ostatní. Jakmile získá sílu ze svého bohatství, vyburcuje všechno proti řeckému království.
#11:3 Povstane však bohatýrský král, bude vládnout nad obrovskou říší a dělat, co se mu zlíbí.
#11:4 Ale až bude pevně stát, bude jeho království rozlomeno, rozděleno podle čtyř nebeských větrů, avšak ne jeho potomkům; nebude to už říše, jaké vládl on, neboť jeho království bude rozvráceno a dostane se jiným, nikoli jím.
#11:5 Tu se vzmůže král Jihu, ale jeden z jeho velitelů se vzmůže víc než on a bude vládnout. Bude vládnout nad nesmírnou říší.
#11:6 Po uplynutí několika let se spojí: dcera krále Jihu přijde ke králi Severu, aby ujednala smír; avšak paže si neuchová sílu, jeho paže neobstojí. Dcera bude v té době vydána záhubě i s těmi, kteří ji přivedli, i s tím, který ji zplodil, a s tím, kdo jí byl oporou.
#11:7 Avšak na jeho místo postoupí výhonek z jejích kořenů, přitáhne proti vojsku a vejde do pevnosti krále Severu a bude tam prosazovat svou sílu.
#11:8 I jejich bohy s jejich litými sochami a vzácnými nádobami, stříbro a zlato odveze do zajetí do Egypta; pak několik let nechá krále Severu na pokoji.
#11:9 Ten sice vtrhne do království krále Jihu, ale vrátí se do své země.
#11:10 Jeho synové však budou válčit a shromáždí nesmírné množství vojska. Jeden z nich přitáhne, přižene se jako povodeň a bude válčit dál až k jeho pevnosti.
#11:11 Král Jihu se rozhořčí, vytáhne a bude sním, s králem Severu, bojovat; ten sice postaví nesmírné množství vojska, ale to množství bude vydáno do rukou onoho.
#11:12 To množství bude odvedeno. Jeho srdce zpychne. Pobije desetitisíce, avšak moc si neudrží.
#11:13 Král Severu se vrátí a postaví ještě nesmírnější množství, než bylo první, a po určité době, po několika letech, přitáhne s velikým vojskem a obrovským vybavením.
#11:14 V té době se mnozí postaví proti králi Jihu a synové rozvratníků tvého lidu se pozdvihnou, aby se potvrdilo vidění, avšak klesnou.
#11:15 Král Severu opět přitáhne, nasype náspy a zmocní se opevněného města. Paže Jihu neobstojí, ani lid jeho vybraných sborů nebude mít sílu, aby obstál.
#11:16 Ten, který přitáhne proti němu, bude dělat, co se mu zlíbí, a nikdo před ním neobstojí. Zastaví se i v nádherné zemi a jeho ruka přinese zkázu.
#11:17 Pojme úmysl zmocnit se celého království, ujedná s ním smír a dá mu jednu z dcer, aby království strhla do zkázy. Ale ona nebude stát při něm.
#11:18 Pak obrátí tvář k ostrovům a mnohých se zmocní, avšak jeden konsul učiní přítrž jeho tupení; a nejen to: jeho tupení mu oplatí.
#11:19 Obrátí tedy tvář k pevnostem své země, ale klesne, padne a nenajdou ho.
#11:20 Na jeho místo povstane někdo, kdo dá projít výběrčímu vznešeným královstvím, ale po několika letech bude zlomen, a to ani hněvem ani bojem.
#11:21 Na jeho místo povstane Opovrženíhodný. Královská důstojnost mu nebude udělena, nýbrž nerušeně přitáhne a úskočně se zmocní království.
#11:22 Jako povodní budou před ním odplaveny a zlomeny paže protivníků, právě tak i vůdce smlouvy.
#11:23 S těmi, kteří se s ním spojí, bude jednat lstivě. Opovážlivě vytáhne s hrstkou pronároda.
#11:24 Nerušeně přitáhne do žirných krajin a učiní, co nečinili jeho otcové ani otcové jeho otců. Rozdělí mezi své lidi loupež, kořist a majetek a opevněným místům bude strojit úklady, až do času.
#11:25 Vyburcuje svou sílu i srdce proti králi Jihu, potáhne s velikým vojskem. Král Jihu s vojskem velice velkým a zdatným se s ním utká v boji, ale neobstojí, protože mu budou nastrojeny úklady.
#11:26 Ti totiž, kteří jídají jeho lahůdky, ho zlomí, jeho vojsko bude odplaveno, mnoho jich bude skoleno a padne.
#11:27 Srdce obou těch králů budou plná zloby a u jednoho stolu budou mluvit lež, ale nezdaří se to, neboť konec je ještě odložen do jistého času.
#11:28 Navrátí se tedy do své země s velikým jměním, ale jeho srdce bude proti svaté smlouvě. Podle toho bude jednat; pak se vrátí do své země.
#11:29 Po jistém čase opět potáhne proti Jihu, ale podruhé to nebude tak jako poprvé.
#11:30 Přitáhne na něj loďstvo Kitejců a bude zkrušen. Opět bude soptit a jednat proti svaté smlouvě. Obrátí se a přikloní k těm, kdo opustili svatou smlouvu.
#11:31 Jeho paže se napřáhnou a znesvětí svatyni i pevnost, vymýtí kadodenní oběť a dají tam ohyzdnou modlu pustošitele.
#11:32 Ty, kteří jednají svévolně vůči smlouvě, přivede úslisnostmi k rouhání. Avšak lid, ti, kteří se znají ke svému Bohu, zůstanou pevní a budou podle toho jednat.
#11:33 Prozíraví v lidu budou poučovat mnohé, ale budou po nějaký čas klesat pod mečem a plamenem, zajetím a loupeží.
#11:34 Když budou klesat, naleznou trochu pomoci, ale mnozí se k nim připojí úskočně.
#11:35 Někteří z prozíravých budou klesat, budou zkoušeni, tříbeni a běleni pro dobu konce, totiž do jistého času.
#11:36 Král bude dělat, co se mu zlíbí, bude se vypínat a činit větším nad každého boha a bude divně mluvit i proti Bohu bohů, co mu nepřísluší, a bude ho provázet zdar, dokud se nedovrší hrozný hněv, neboť rozhodnutí bude vykonáno.
#11:37 Nepřikloní se ani k bohům svých otců ani k Oblíbenci žen, nepřikloní se k žádnému bohu, neboť se bude činit větším nade všechny.
#11:38 Jen boha pevností bude ctít na jeho místě, zlatem, stříbrem, drahokamem a drahocennostmi bude ctít boha, kterého jeho otcové neznali.
#11:39 Cizího boha uvede do opevněných pevností; kdo ho uzná, toho zahrne slávou. Takovým svěří vládu nad mnohými a jako odměnu jim přidělí půdu.
#11:40 V době konce se s ním srazí král Jihu, ale král Severu proti němu zaútočí s vozbou a jízdou a obrovským loďstvem. Přitáhne proti zemím, zaplaví je a potáhne dál.
#11:41 Přitáhne i do nádherné země a mnozí klesnou. Z jeho rukou uniknou jen tito: Edóm, Moáb a přední z Amónovců.
#11:42 Vztáhne svou ruku na četné země; ani egyptská země nevyvázne.
#11:43 Získá vládu nad skrytými poklady zlata a stříbra a všemi egyptskými drahocennostmi. V jeho průvodu budou i Lúbijci a Kúšijci.
#11:44 Ale vyděsí ho zprávy z východu a ze severu. Vytáhne s velikým rozhořčením, aby mnohé zahladil a vyhubil jako klaté.
#11:45 Postaví své přepychové stany od moří k hoře svaté nádhery. Pak přijde jeho konec a nikdo mu nepomůže.“ 
#12:1 „V oné době povstane Míkael, velký ochránce, a bude stát při synech tvého lidu. Bude to doba soužení, jaké nebylo od vzniku národa až do této doby. V oné době bude vyproštěn tvůj lid, každý, kdo je zapsán v Knize.
#12:2 Mnozí z těch, kteří spí v prachu země, procitnou; jedni k životu věčnému, druzí k pohaně a věčné hrůze.
#12:3 Prozíraví budou zářit jako záře oblohy, a ti, kteří mnohým dopomáhají k spravedlnosti, jako hvězdy, navěky a navždy.
#12:4 A ty, Danieli, udržuj ta slova v tajnosti a zapečeť tuto knihu až do doby konce. Mnozí budou zmateně pobíhat, ale poznání se rozmnoží.“
#12:5 Já Daniel jsem viděl toto: Hle, povstali dva další muži, jeden na tomto břehu veletoku, druhý na onom břehu veletoku,
#12:6 a ten se otázal muže oblečeného ve lněném oděvu, který byl nad vodami veletoku: „Kdy nastane konec těch podivuhodných věcí?“
#12:7 I slyšel jsem muže oblečeného ve lněném oděvu, který byl nad vodami veletoku. Zvedl svou pravici i levici k nebi a přísahal při Živém navěky, že k času a časům a k polovině, až se dovrší roztříštění moci svatého lidu, dovrší se i všechno toto.
#12:8 Slyšel jsem, ale neporozuměl jsem. Řekl jsem: „Můj pane, jaké bude zakončení toho všeho?“
#12:9 Řekl: „Jdi, Danieli, tajuplná a zapečetěná budou ta slova až do doby konce.
#12:10 Mnozí se vytříbí, zbělí a budou vyzkoušeni. Svévolníci budou jednat svévolně; žádný svévolník se nepoučí, ale prozíraví se poučí.
#12:11 Od doby, kdy bude odstraněna každodenní oběť a vztyčena ohyzdná modla pustošitele, uplyne tisíc dvě stě devadesát dní.
#12:12 Blaze tomu, kdo se v důvěře dočká tisíce tří set třiceti pěti dnů.
#12:13 Ty vytrvej do konce. Pak odpočineš, ale na konci dnů povstaneš ke svému údělu.  

