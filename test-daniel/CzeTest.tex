\source{CzeBKR}

\book{Daniel}{Da}
#1:1 Léta třetího kralování Joakima krále Judského, přitáhl Nabuchodonozor král Babylonský k Jeruzalému, a oblehl jej.
#1:2 I vydal Pán v ruku jeho Joakima krále Judského, a něco nádobí domu Božího. Kterýž zavezl je do země Sinear, do domu boha svého, a nádobí to dal vnésti do domu pokladu boha svého.
#1:3 Rozkázal také král Ašpenazovi, správci dvořanů svých, aby přivedl z synů Izraelských, z semene královského a z knížat,
#1:4 Mládence, na nichž by nebylo žádné poškvrny, a krásného oblíčeje, a vtipné ke vší moudrosti, a schopné k umění i k nabývání jeho, a v kterýchž by byla síla, aby stávali na palácu královském, a učili se liternímu umění a jazyku Kaldejskému.
#1:5 I nařídil jim král odměřený pokrm na každý den z stolu královského, i vína, kteréž on sám pil, a aby je tak choval za tři léta, a po dokonání jich aby stávali před králem.
#1:6 Byli pak mezi nimi z synů Juda: Daniel, Chananiáš, Mizael a Azariáš.
#1:7 I dal jim správce dvořanů jména. Nazval Daniele Baltazarem, Chananiáše pak Sidrachem, a Mizaele Mizachem, a Azariáše Abdenágem.
#1:8 Ale Daniel uložil v srdci svém, aby se nepoškvrňoval pokrmem z stolu královského, a vínem, kteréž král pil. Pročež hledal toho u správce nad dvořany, aby se nemusil poškvrňovati.
#1:9 I způsobil Bůh Danielovi milost a lásku u správce nad dvořany.
#1:10 A řekl správce nad dvořany Danielovi: Já se bojím pána svého krále, kterýž vyměřil pokrm váš a nápoj váš, tak že uzřel-li by, že tváře vaše opadlejší jsou, nežli mládenců těch, kteříž podobně jako i vy chování býti mají, způsobíte mi to u krále, že přijdu o hrdlo.
#1:11 I řekl Daniel služebníku, kteréhož ustanovil správce dvořanů nad Danielem, Chananiášem, Mizaelem a Azariášem:
#1:12 Zkus, prosím, služebníků svých za deset dní, a nechť se nám vaření dává, kteréž bychom jedli, a voda, kterouž bychom pili.
#1:13 A potom nechť se spatří před tebou tváře naše a tváře mládenců, kteříž jídají pokrm z stolu královského, a jakž uhlédáš, učiň s služebníky svými.
#1:14 I uposlechl jich v té věci, a zkusil jich za deset dní.
#1:15 Po skonání pak desíti dnů spatříno jest, že tváře jejich byly krásnější, a byli tlustší na těle než všickni mládenci, kteříž jídali pokrm z stolu královského.
#1:16 Protož služebník brával ten vyměřený pokrm jejich, a víno nápoje jejich, a dával jim vaření.
#1:17 Mládence pak ty čtyři obdařil Bůh povědomostí a rozumností ve všelikém literním umění a moudrostí; nadto Danielovi dal, aby rozuměl všelikému vidění a snům.
#1:18 A když se dokonali dnové, po kterýchž rozkázal král, aby je přivedli, přivedl je správce dvořanů před Nabuchodonozora.
#1:19 I mluvil s nimi král. Ale není nalezen mezi všemi těmi, jako Daniel, Chananiáš, Mizael a Azariáš. I stávali před králem.
#1:20 A ve všelikém slovu moudrosti a rozumnosti, na kteréž se jich doptával král, nalezl je desetkrát zběhlejší nade všecky mudrce a hvězdáře, kteříž byli ve všem království jeho.
#1:21 I zůstával tu Daniel až do léta prvního Cýra krále. 
#2:1 Léta pak druhého kralování Nabuchodonozora měl Nabuchodonozor sen, a děsil se duch jeho, až se tudy i ze sna protrhl.
#2:2 I rozkázal král svolati mudrce, a hvězdáře i kouzedlníky a Kaldejské, aby oznámili králi sen jeho. Kteřížto přišli, a postavili se před králem.
#2:3 Tedy řekl jim král: Měl jsem sen, a předěsil se duch můj, tak že nevím, jaký to byl sen.
#2:4 I mluvili Kaldejští králi Syrsky: Králi, na věky buď živ. Pověz sen služebníkům svým, a oznámímeť výklad.
#2:5 Odpověděl král a řekl Kaldejským: Ten sen mi již z paměti vyšel. Neoznámíte-li mi snu i výkladu jeho, na kusy rozsekáni budete, a domové vaši v záchody obráceni budou.
#2:6 Pakli mi sen i výklad jeho oznámíte, daru, odplaty a slávy veliké důjdete ode mne. A protož sen i výklad jeho mi oznamte.
#2:7 Odpověděli po druhé a řekli: Nechať král sen poví služebníkům svým, a výklad jeho oznámíme.
#2:8 Odpověděl král a řekl: Jistotně rozumím tomu, že naschvál odtahujete, vidouce, že mi vyšel z paměti ten sen.
#2:9 Neoznámíte-li mi toho snu, jistý jest ten úsudek o vás. Nebo řeč lživou a chytrou smyslili jste sobě, abyste mluvili přede mnou, ažby se čas proměnil. A protož sen mi povězte, a zvím, budete-li mi moci i výklad jeho oznámiti.
#2:10 Odpověděli Kaldejští králi a řekli: Není člověka na zemi, kterýž by tu věc králi oznámiti mohl. Nadto žádný král, kníže neb potentát takové věci se nedoptával na žádném mudrci a hvězdáři aneb Kaldeovi.
#2:11 Nebo ta věc, na niž se král ptá, nesnadná jest, a není jiného, kdo by ji oznámiti mohl králi, kromě bohů, kteříž bydlení s lidmi nemají.
#2:12 Z té příčiný rozlítil se král a rozhněval velmi, a přikázal, aby zhubili všecky mudrce Babylonské.
#2:13 A když vyšel ortel, a mudrci mordováni byli, hledali i Daniele a tovaryšů jeho, aby zmordováni byli.
#2:14 Tedy Daniel odpověděl moudře a opatrně Ariochovi, hejtmanu nad žoldnéři královskými, kterýž vyšel, aby mordoval mudrce Babylonské.
#2:15 A odpovídaje, řekl Ariochovi, hejtmanu královskému: Proč ta výpověd náhle vyšla od krále? I oznámil tu věc Arioch Danielovi.
#2:16 Pročež Daniel všed, prosil krále, aby jemu prodlel času k oznámení výkladu toho králi.
#2:17 A odšed Daniel do domu svého, oznámil tu věc Chananiášovi, Mizaelovi a Azariášovi, tovaryšům svým,
#2:18 Aby se za milosrdenství modlili Bohu nebeskému příčinou té věci tajné, a nebyli zahubeni Daniel a tovaryši jeho s pozůstalými mudrci Babylonskými.
#2:19 I zjevena jest Danielovi u vidění nočním ta věc tajná. Pročež Daniel dobrořečil Bohu nebeskému.
#2:20 Mluvil pak Daniel a řekl: Buď jméno Boží požehnáno od věků až na věky, nebo moudrost a síla jeho jest.
#2:21 A on proměňuje časy i chvíle; ssazuje krále, i ustanovuje krále; dává moudrost moudrým a umění majícím rozum.
#2:22 On zjevuje věci hluboké a skryté; zná to, což jest v temnostech, a světlo s ním přebývá.
#2:23 Ó Bože otců mých, těť oslavuji a chválím, že jsi mne moudrostí a silou obdařil, ovšem nyní, že jsi mi oznámil to, čehož jsme žádali od tebe; nebo věc královu oznámil jsi nám.
#2:24 A protož Daniel všel k Ariochovi, kteréhož ustanovil král, aby zhubil mudrce Babylonské. A přišed, takto řekl jemu: Mudrců Babylonských nezahlazuj; uveď mne před krále a výklad ten oznámím.
#2:25 Tedy Arioch s chvátáním uvedl Daniele před krále, a takto řekl jemu: Našel jsem muže z zajatých synů Judských, kterýž výklad ten králi oznámí.
#2:26 Odpověděl král a řekl Danielovi, jehož jméno bylo Baltazar: Budeš-liž ty mi moci oznámiti sen, kterýž jsem viděl, i výklad jeho?
#2:27 Odpověděl Daniel králi a řekl: Té věci tajné, na niž se král doptává, nikoli nemohou mudrci, hvězdáři, věšťci a hadači králi oznámiti.
#2:28 Ale však jest Bůh na nebi, kterýž zjevuje tajné věci, a on ukázal králi Nabuchodonozorovi, co se díti bude v potomních dnech. Sen tvůj, a což jsi ty viděl na ložci svém, toto jest:
#2:29 Tobě, ó králi, na mysl přicházelo na ložci tvém, co bude potom, a ten, kterýž zjevuje tajné věci, ukázalť to, což budoucího jest.
#2:30 Strany pak mne, ne skrze moudrost, kteráž by při mně byla nade všecky lidi, ta věc tajná mně zjevena jest, ale skrze modlitbu, aby ten výklad králi oznámen byl, a ty myšlení srdce svého abys zvěděl.
#2:31 Ty králi, viděl jsi, a aj, obraz nějaký veliký, (obraz ten byl znamenitý a blesk jeho náramný), stál naproti tobě, kterýž na pohledění byl hrozný.
#2:32 Toho obrazu hlava byla z zlata výborného, prsy jeho a ramena jeho z stříbra, a břicho jeho i bedra jeho z mědi,
#2:33 Hnátové jeho z železa, nohy jeho z částky z železa a z částky z hliny.
#2:34 Hleděls na to, až se utrhl kámen, kterýž nebýval v rukou, a udeřil obraz ten v nohy jeho železné a hliněné, a potřel je.
#2:35 A tak potříno jest spolu železo, hlina, měď, stříbro i zlato, a bylo to všecko jako plevy z placu letního, a zanesl to vítr, tak že místa jejich není nalezeno. Kámen pak ten, kterýž udeřil v obraz, učiněn jest horou velikou, a naplnil všecku zemi.
#2:36 Toť jest ten sen. Výklad jeho také povíme králi:
#2:37 Ty králi, jsi král králů; nebo Bůh nebeský dal tobě království, moc a sílu i slávu.
#2:38 A všeliké místo, na němž přebývají synové lidští, zvěř polní i ptactvo nebeské dal v ruku tvou, a pánem tě ustavil nade všemi těmi věcmi. Ty jsi ta hlava zlatá.
#2:39 Ale po tobě povstane království jiné, nižší než tvé, a jiné království třetí měděné, kteréž panovati bude po vší zemi.
#2:40 Království pak čtvrté bude tvrdé jako železo. Nebo jakož železo drobí a zemdlévá všecko, tak, pravím, jako železo, kteréž potírá všecko, i ono potře a potříská všecko.
#2:41 Že jsi pak viděl nohy a prsty z částky z hliny hrnčířské a z částky z železa, království rozdílné znamená, v němž bude něco mocnosti železa, tak jakž jsi viděl železo smíšené s hlinou ničemnou.
#2:42 Ale prstové noh z částky z železa a z částky z hliny znamenají království z částky mocné a z částky ku potření snadné.
#2:43 A že jsi viděl železo smíšené s hlinou ničemnou, ukazuje, že se přízniti budou vespolek lidé, a však nebude se přídržeti jeden druhého, tak jako železo nedrží se z hlinou.
#2:44 Za dnů pak těch králů vzbudí Bůh nebeský království, kteréž na věky nebude zkaženo, a království to na žádného jiného nespadne, ale ono potře a konec učiní všechněm těm královstvím, samo pak státi bude na věky,
#2:45 Tak jakž jsi viděl, že se s hory utrhl kámen, kterýž nebýval v rukou, a potřel železo, měď, hlinu, stříbro a zlato. Bůh veliký oznámil králi, co býti má potom, a pravý jest sen ten i věrný výklad jeho.
#2:46 Tedy král Nabuchodonozor padl na tvář svou, a poklonil se Danielovi, a rozkázal, aby oběti a vůně libé obětovali jemu.
#2:47 A odpovídaje král Danielovi, řekl: V pravdě že Bůh váš jest Bůh bohů a Pán králů, kterýž zjevuje skryté věci, poněvadž jsi mohl vyjeviti tajnou věc tuto.
#2:48 Tedy král zvelebil Daniele, a dary veliké a mnohé dal jemu, a pánem ho učinil nade vší krajinou Babylonskou, a knížetem nad vývodami, a nade všemi mudrci Babylonskými.
#2:49 Daniel pak vyžádal na králi, aby představil krajině Babylonské Sidracha, Mizacha a Abdenágo. Ale Daniel býval v bráně královské. 
\endinput
#3:1 Potom Nabuchodonozor král udělav obraz zlatý, jehož výška byla šedesáti loket, šířka pak šesti loket, postavil jej na poli Dura v krajině Babylonské.
#3:2 I poslal Nabuchodonozor král, aby shromáždili knížata, vývody a vůdce, starší, správce nad poklady, v právích zběhlé, úředníky a všecky, kteříž panovali nad krajinami, aby přišli ku posvěcování obrazu, kterýž postavil Nabuchodonozor král.
#3:3 Tedy shromáždili se knížata, vývodové a vůdcové, starší, správcové nad poklady, v právích zběhlí, úředníci a všickni, kteříž panovali nad krajinami ku posvěcování obrazu toho, kterýž postavil Nabuchodonozor král, a stáli před obrazem, kterýž postavil Nabuchodonozor.
#3:4 Biřic pak volal ze vší síly: Vám se to praví lidem, národům a jazykům,
#3:5 Jakž uslyšíte zvuk trouby, píšťalky, citary, huslí, loutny, zpívání a všelijaké muziky, padněte a klanějte se obrazu zlatému, kterýž postavil Nabuchodonozor král.
#3:6 Kdož by pak nepadl a neklaněl se, té hodiny uvržen bude do prostřed peci ohnivé rozpálené.
#3:7 A protož hned, jakž uslyšeli všickni lidé zvuk trouby, píšťalky, citary, huslí, loutny a všelijaké muziky, padli všickni lidé, národové a jazykové, klanějíce se obrazu zlatému, kterýž postavil Nabuchodonozor král.
#3:8 A hned téhož času přistoupili muži Kaldejští, a s křikem žalovali na Židy,
#3:9 A mluvíce, řekli Nabuchodonozorovi králi: Králi, na věky buď živ.
#3:10 Ty králi, vynesls výpověd, aby každý člověk, kterýž by slyšel zvuk trouby, píšťalky, citary, huslí, loutny, zpívání a všelijaké muziky, padl a klaněl se obrazu zlatému,
#3:11 A kdož by nepadl a neklaněl se, aby uvržen byl do prostřed peci ohnivé rozpálené.
#3:12 Našli se pak někteří Židé, kteréž jsi představil krajině Babylonské, totiž Sidrach, Mizach a Abdenágo, kteřížto muži nedbali na tvé, ó králi, nařízení. Bohů tvých nectí, a obrazu zlatému, kterýž jsi postavil, se neklanějí.
#3:13 Tedy Naduchodonozor v hněvě a v prchlivosti rozkázal přivésti Sidracha, Mizacha a Abdenágo. I přivedeni jsou muži ti před krále.
#3:14 I mluvil Nabuchodonozor a řekl jim: Zoumyslně-li, Sidrachu, Mizachu a Abdenágo, bohů mých nectíte, a obrazu zlatému, kterýž jsem postavil, se neklaníte?
#3:15 Protož nyní, jste-liž hotovi, abyste hned, jakž uslyšíte zvuk trouby, píšťalky, citary, huslí, loutny, zpívání a všelijaké muziky, padli a klaněli se obrazu tomu, kterýž jsem učinil? Pakli se klaněti nebudete, té hodiny uvrženi budete do prostřed peci ohnivé rozpálené, a který jest ten Bůh, ješto by vás vytrhl z ruky mé?
#3:16 Odpověděli Sidrach, Mizach a Abdenágo, a řekli králi: My se nestaráme o to, ó Nabuchodonozoře, co bychom měli odpovědíti tobě.
#3:17 Nebo aj, buďto že Bůh, jehož my ctíme, (kterýž mocen jest vytrhnouti nás z peci ohnivé rozpálené, a tak z ruky tvé, ó králi), vytrhne nás.
#3:18 Buď že nevytrhne, známo buď tobě, ó králi, žeť bohů tvých ctíti a obrazu zlatému, kterýž jsi postavil, klaněti se nebudeme.
#3:19 Tedy Nabuchodonozor naplněn jsa prchlivostí, tak že oblíčej tváři jeho se proměnil proti Sidrachovi, Mizachovi a Abdenágovi, a odpovídaje, rozkázal rozpáliti pec sedmkrát více, než obyčej měli ji rozpalovati.
#3:20 A mužům silným, kteříž byli mezi rytíři jeho, rozkázal, aby svížíce Sidracha, Mizacha a Abdenágo, uvrhli do peci ohnivé rozpálené.
#3:21 Tedy svázali muže ty v pláštích jejich, v košilkách jejich, i v kloboucích jejich a v oděvu jejich, a uvrhli je do prostřed peci ohnivé rozpálené.
#3:22 Že pak rozkaz královský náhlý byl, a pec velmi rozpálená, z té příčiny muže ty, kteříž uvrhli Sidracha, Mizacha a Abdenágo, zadusil plamen ohně.
#3:23 Ale ti tři muži, Sidrach, Mizach a Abdenágo, padli do prostřed peci ohnivé rozpálené svázaní.
#3:24 Tedy Nabuchodonozor král zděsil se, a vstal s chvátáním, a promluviv, řekl hejtmanům svým: Zdaliž jsme neuvrhli tří mužů do prostřed peci svázaných? Odpověděli a řekli králi: Pravda jest, králi.
#3:25 On pak odpovídaje, řekl: Aj, vidím čtyři muže rozvázané, procházející se u prostřed ohně, a není žádného porušení při nich, a čtvrtý na pohledění podobný jest synu Božímu.
#3:26 A přistoupiv Nabuchodonozor k čelisti peci ohnivé rozpálené, mluvil a řekl: Sidrachu, Mizachu a Abdenágo, služebníci Boha nejvyššího, vyjděte a poďte sem. I vyšli Sidrach, Mizach a Abdenágo z prostředku ohně.
#3:27 Shromáždivše se pak knížata, vývodové a vůdcové a hejtmané královští, hleděli na ty muže, an žádné moci neměl oheň při tělích jejich, ani vlas hlavy jejich nepřiškvrkl, ani plášťové jejich se nezměnili, aniž co ohněm páchli.
#3:28 I mluvil Nabuchodonozor a řekl: Požehnaný Bůh jejich, totiž Sidrachův, Mizachův a Abdenágův, kterýž poslal anděla svého, a vytrhl služebníky své, kteříž doufali v něho, až i rozkazu královského neuposlechli, ale těla svá vydali, aby nesloužili a neklaněli se žádnému bohu, kromě Bohu svému.
#3:29 A protož toto já přikazuji, aby každý ze všelikého lidu, národu a jazyku, kdož by koli co rouhavého řekl proti Bohu Sidrachovu, Mizachovu a Abdenágovu, na kusy rozsekán byl, a dům jeho v záchod obrácen, proto že není Boha jiného, kterýž by mohl vytrhovati, jako tento.
#3:30 Tedy zvelebil zase král Sidracha, Mizacha a Abdenága v krajině Babylonské. 
#4:1 Nabuchodonozor král všechněm lidem, národům i jazykům, kteříž bydlí na vší zemi: Pokoj váš rozmnožen buď.
#4:2 Znamení a divy, kteréž učinil při mně Bůh nejvyšší, vidělo mi se za slušné, abych vypravoval.
#4:3 Znamení jeho jak veliká jsou, a divové jeho jak mocní jsou, království jeho království věčné, a panování jeho od národu do pronárodu.
#4:4 Já Nabuchodonozor, užívaje pokoje v domě svém, a kveta na palácu svém,
#4:5 Měl jsem sen, kterýž mne zhrozil, a myšlení na ložci svém, a to, což jsem viděl, znepokojilo mne.
#4:6 A protož vyšla ode mne ta výpověd, aby uvedeni byli přede mne mudrci Babylonští, kteříž by mi výklad toho snu oznámili.
#4:7 Tedy předstoupili mudrci, hvězdáři, Kaldejští a hadači. I pověděl jsem jim sen, a však výkladu jeho nemohli mi oznámiti.
#4:8 Až naposedy předstoupil přede mne Daniel, jehož jméno Baltazar, podlé jména boha mého, a v němž jest duch bohů svatých. Jemuž jsem oznámil sen:
#4:9 Baltazaře, kníže mudrců, já vím, že duch bohů svatých jest v tobě, a nic tajného není tobě nesnadného. Vidění snu mého, kterýž jsem měl, i výklad jeho oznam.
#4:10 U vidění pak, kteréž jsem viděl na ložci svém, viděl jsem, a aj, strom u prostřed země, jehož vysokost byla veliká.
#4:11 Veliký byl strom ten a mocný, a výsost jeho dosahovala až k nebi, a patrný byl až do končin vší země.
#4:12 Lístí jeho bylo pěkné, a ovoce jeho hojné, všechněm za pokrm, pod stínem pak jeho byla zvěř polní, a na ratolestech jeho bydlili ptáci nebeští, a z něho potravu měl všeliký živočich.
#4:13 Viděl jsem také u viděních svých na ložci svém, a aj, hlásný a svatý s nebe sstoupiv,
#4:14 Volal ze vší síly, a tak pravil: Podetněte strom ten, a osekejte ratolesti jeho, a otlucte lístí jeho, a rozmecte ovoce jeho; nechať se vzdálí zvěř od něho, a ptáci z ratolestí jeho.
#4:15 A však kmene kořenů jeho v zemi zanechejte, a v poutech železných a ocelivých na trávě polní, aby rosou nebeskou smáčín byl, a díl jeho s zvěří v bylině zemské.
#4:16 Srdce jeho od lidského ať jest rozdílné, a srdce zvířecí nechť jest dáno jemu, ažby sedm let vyplnilo se při něm.
#4:17 Usouzení hlásných a řeč žádosti svatých stane se, ažby k tomu přišlo, aby poznali lidé, že Nejvyšší panuje nad královstvím lidským, a že komuž chce, dává je, a toho, kterýž jest ponížený mezi lidmi, ustanovuje nad ním.
#4:18 Tento sen viděl jsem já Nabuchodonozor král, ty pak, Baltazaře, oznam výklad jeho. Nebo všickni mudrci v království mém nemohli mi výkladu oznámiti, ale ty můžeš, proto že duch bohů svatých jest v tobě.
#4:19 Tedy Daniel, jemuž jméno Baltazar, předěšený stál za jednu hodinu, a myšlení jeho strašila ho. A odpověděv král, řekl: Baltazaře, sen ani výklad jeho nechť nestraší tebe. Odpověděl Baltazar a řekl: Pane můj, sen tento přiď na ty, kteříž tě v nenávisti mají, a výklad jeho na tvé nepřátely.
#4:20 Strom ten, kterýž jsi viděl, veliký a mocný, jehož výsost dosahovala až k nebi, a kterýž patrný byl po vší zemi.
#4:21 A lístí jeho pěkné, a ovoce jeho hojné, a z něhož všickni pokrm měli, pod nímž byla zvěř polní, a na ratolestech jeho bydlili ptáci nebeští,
#4:22 Ty jsi ten, ó králi, kterýž jsi rozšířil se a zmocnil, a velikost tvá vzrostla a vznesla se až k nebi, a panování tvé až do konce země.
#4:23 Že pak viděl král hlásného a svatého sstupujícího s nebe, kterýž řekl: Podetněte strom ten a zkazte jej, a však kmene kořenů jeho v zemi zanechejte, a v okovách železných a ocelivých ať jest na trávě polní, aby rosou nebeskou smáčín byl, a s zvěří polní díl jeho, ažby sedm let vyplnilo se při něm:
#4:24 Tenť jest výklad, ó králi, a ortel Nejvyššího, kterýž vyšel na pána mého krále.
#4:25 Nebo zaženou tě lidé od sebe, a s zvěří polní bude bydlení tvé, a bylinu jako volům tobě jísti dávati budou, a rosou nebeskou smáčín budeš, až se vyplní sedm let při tobě, dokudž bys nepoznal, že panuje Nejvyšší nad královstvím lidským, a že komuž chce, dává je.
#4:26 Že pak řekli, aby zanechán byl kmen kořenů toho stromu, království tvé tobě zůstane, jakž jen poznáš, že nebesa panují.
#4:27 Protož ó králi, přijmi radu mou, a hříchy své spravedlností přetrhuj, a nepravosti své milostivostí k ssouženým, zdaby prodloužen byl pokoj tvůj.
#4:28 Všecko to přišlo na krále Nabuchodonozora.
#4:29 Nebo po dokonání dvanácti měsíců, procházeje se po palácu královském v Babyloně,
#4:30 Mluvil král a řekl: Zdaliž toto není ten Babylon veliký, kterýž jsem já vystavěl mocí síly své, aby byl stolicí království k ozdobě slávy mé?
#4:31 Ještě ta řeč byla v ústech krále, a aj, hlas s nebe přišel: Toběť se praví, Nabuchodonozoře králi, že království odešlo od tebe,
#4:32 Nýbrž tě lidé i z sebe vyvrhou, a s zvěří polní bydliti budeš. Bylinu jako volům tobě jísti dávati budou, ažby sedm let vyplnilo se při tobě, dokudž bys nepoznal, že panuje Nejvyšší nad královstvím lidským, a že komuž chce, dává je.
#4:33 V touž hodinu řeč ta naplnila se při Nabuchodonozorovi. Nebo z spolku lidí vyvržen byl, a bylinu jako vůl jedl, a rosou nebeskou tělo jeho smáčíno bylo, až na něm vlasy zrostly jako peří orličí, a nehty jeho jako pazoury ptačí.
#4:34 Při skonání pak těch dnů já Nabuchodonozor pozdvihl jsem očí svých k nebi, a rozum můj ke mně se zase navrátil. I dobrořečil jsem Nejvyššímu, a živého na věky chválil jsem a oslavoval; nebo panování jeho jest panování věčné, a království jeho od národu do pronárodu.
#4:35 A všickni obyvatelé země jako za nic počteni jsou, a podlé vůle své činí mezi vojskem nebeským i obyvateli země, aniž jest kdo, ješto by mu přes ruku dáti mohl, a říci jemu: Co to děláš?
#4:36 Téhož času rozum můj navrátil se ke mně, a k slávě království mého ozdoba má, i důstojnost má navrátila se ke mně; nadto i hejtmané moji a knížata má hledali mne, a zmocněn jsem v království svém, a velebnost větší jest mi přidána.
#4:37 Nyní tedy já Nabuchodonozor chválím, vyvyšuji a oslavuji krále nebeského, jehož všickni skutkové jsou pravda, a stezky jeho soud, a kterýž chodící v pýše může snižovati. 
#5:1 Balsazar král učinil hody veliké tisíci knížatům svým, a před nimi víno pil.
#5:2 A když pil víno Balsazar, rozkázal přinésti nádobí zlaté a stříbrné, kteréž vynesl Nabuchodonozor otec jeho z chrámu Jeruzalémského, aby z něho pili král i knížata jeho, ženy jeho i ženiny jeho.
#5:3 I přineseno jest nádobí zlaté, kteréž vynesli z chrámu domu Božího, kterýž byl v Jeruzalémě, a pili z něho král i knížata jeho, ženy jeho i ženiny jeho.
#5:4 Pili víno, a chválili bohy zlaté a stříbrné, měděné, železné, dřevěné a kamenné.
#5:5 V touž hodinu vyšli prstové ruky lidské, a psali naproti svícnu na stěně paláce královského, a král hleděl na částky ruky, kteráž psala.
#5:6 Tedy jasnost královská změnila se, a myšlení jeho zkormoutila ho, a pasové bedr jeho rozpásali se, i kolena jeho jedno o druhé se tlouklo.
#5:7 A zkřikl král ze vší síly, aby přivedeni byli hvězdáři, Kaldejští a hadači. I mluvil král a řekl mudrcům Babylonským: Kdokoli přečte psání toto, a výklad jeho mi oznámí, šarlatem odín bude, a řetěz zlatý na hrdlo jeho, a třetím v království po mně bude.
#5:8 I předstoupili všickni mudrci královští, ale nemohli písma toho čísti, ani výkladu oznámiti králi.
#5:9 Pročež král Balsazar velmi předěšen byl, a jasnost jeho změnila se na něm, ano i knížata jeho zkormouceni byli.
#5:10 Královna pak, příčinou té věci královské a knížat jeho, do domu těch hodů vešla, a promluvivši královna, řekla: Králi, na věky živ buď. Nechť tě neděsí myšlení tvá, a jasnost tvá nechť se nemění.
#5:11 Jest muž v království tvém, v němž jest duch bohů svatých, v kterémž za dnů otce tvého osvícení, rozumnost a moudrost, jako moudrost bohů, nalezena, jehož král Nabuchodonozor otec tvůj knížetem mudrců, hvězdářů, Kaldejských a hadačů ustanovil, otec tvůj, ó králi,
#5:12 Proto že duch znamenitý, i umění a rozumnost vykládání snů a oznámení pohádek, i rozvázání věcí nesnadných nalezeno při Danielovi, jemuž král jméno dal Baltazar. Nechať nyní zavolán jest Daniel, a oznámíť výklad ten.
#5:13 Tedy přiveden jest Daniel před krále. I mluvil král a řekl Danielovi: Ty-li jsi ten Daniel, jeden z synů zajatých Judských, kteréhož přivedl král otec můj z Judstva?
#5:14 Slyšel jsem zajisté o tobě, že duch bohů svatých jest v tobě, a osvícení i rozumnost a moudrost znamenitá nalezena jest v tobě.
#5:15 A nyní přivedeni jsou přede mne mudrci a hvězdáři, aby mi písmo toto přečtli, a výklad jeho oznámili, a však nemohli výkladu věci té oznámiti.
#5:16 Já pak slyšel jsem o tobě, že můžeš to, což jest nesrozumitelného, vykládati, a což nesnadného, rozvázati. Protož nyní, budeš-li moci písmo to přečísti, a výklad jeho mně oznámiti, v šarlat oblečen budeš, a řetěz zlatý na hrdlo tvé, a třetím v království po mně budeš.
#5:17 Tedy odpověděl Daniel a řekl před králem: Darové tvoji nechť zůstávají tobě, a odplatu svou dej jinému, a však písmo přečtu králi, a výklad oznámím jemu.
#5:18 Ty králi, slyš: Bůh nejvyšší královstvím a důstojností i slávou a okrasou obdařil Nabuchodonozora otce tvého,
#5:19 A pro důstojnost, kterouž ho obdařil, všickni lidé, národové a jazykové třásli a báli se před ním. Kohokoli chtěl, zabil, a kterékoli chtěl, bil, kteréž chtěl, povyšoval, a kteréž chtěl, ponižoval.
#5:20 Když se pak bylo pozdvihlo srdce jeho, a duch jeho zmocnil se v pýše, ssazen byl z stolice království svého, a slávu odjali od něho.
#5:21 Ano i z spolku synů lidských vyvržen byl, a srdce jeho zvířecímu podobné učiněno bylo, a s divokými osly bylo bydlení jeho. Bylinu jako volům dávali jemu jísti, a rosou nebeskou tělo jeho smáčíno bylo, dokudž nepoznal, že panuje Bůh nejvyšší nad královstvím lidským, a že kohož chce, ustanovuje nad ním.
#5:22 Ty také, synu jeho Balsazaře, neponížil jsi srdce svého, ačkolis o tom o všem věděl.
#5:23 Ale pozdvihls se proti Pánu nebes; nebo nádobí domu jeho přinesli před tebe, a ty i knížata tvá, ženy tvé i ženiny tvé pili jste víno z něho. Nadto bohy stříbrné a zlaté, měděné, železné, dřevěné a kamenné, kteříž nevidí, ani slyší, aniž co vědí, chválil jsi, Boha pak, v jehož ruce jest dýchání tvé i všecky cesty tvé, neoslavoval jsi.
#5:24 Protož nyní od něho poslána jest částka ruky této, a písmo to napsáno jest.
#5:25 A totoť jest písmo napsané: Mene, mene, tekel, ufarsin, totiž: Zčetl jsem, zčetl, zvážil a rozděluji.
#5:26 Tento pak jest výklad slov: Mene, zčetl Bůh království tvé, a k konci je přivedl.
#5:27 Tekel, zvážen jsi na váze, a nalezen jsi lehký.
#5:28 Peres, rozděleno jest království tvé, a dáno jest Médským a Perským.
#5:29 Tedy z rozkazu Balsazarova oblékli Daniele v šarlat, a řetěz zlatý dali na hrdlo jeho, a rozhlašovali o něm, že má býti pánem třetím v království.
#5:30 V touž noc zabit jest Balsazar král Kaldejský.
#5:31 Darius pak Médský ujal království v letech okolo šedesáti a dvou. 
#6:1 Líbilo se pak Dariovi, aby ustanovil nad královstvím úředníků sto a dvadceti, kteříž by byli po všem království.
#6:2 Nad těmi pak hejtmany tři, z nichžto Daniel přední byl, kterýmž by úředníci onino vydávali počet, aby se králi škoda nedála.
#6:3 Tedy Daniel převyšoval ty hejtmany a úředníky, proto že duch znamenitější v něm byl. Pročež král myslil ustanoviti jej nade vším královstvím.
#6:4 Tedy hejtmané a úředníci hledali příčiny proti Danielovi s strany království, a však žádné příčiny ani vady nemohli najíti; nebo věrný byl, aniž jaký omyl neb vada nalézala se při něm.
#6:5 Protož muži ti řekli: Nenajdeme proti Danielovi tomuto žádné příčiny, jediné leč bychom našli něco proti němu s strany zákona Boha jeho.
#6:6 Tedy hejtmané a úředníci ti shromáždivše se k králi, takto mluvili k němu: Darie králi, na věky buď živ.
#6:7 Uradili se všickni hejtmané království, vývodové, úředníci, správcové a vůdcové, abys ustanovil nařízení královské, a utvrdil zápověd: Kdož by koli vložil žádost na kteréhokoli boha neb člověka do třidcíti dnů, kromě na tebe, králi, aby uvržen byl do jámy lvové.
#6:8 Nyní tedy, ó králi, potvrď zápovědi této, a vydej mandát, kterýž by nemohl změněn býti podlé práva Médského a Perského, kteréž jest neproměnitelné.
#6:9 Pročež král Darius vydal mandát a zápověd.
#6:10 Daniel pak, když se dověděl, že jest vydán mandát, všel do domu svého, kdež otevřená byla okna v pokoji jeho proti Jeruzalému, a třikrát za den klekal na kolena svá, a modlíval se a vyznával se Bohu svému, tak jakož prvé to činíval.
#6:11 Tedy muži ti shromáždivše se a nalezše Daniele, an se modlí a pokorně prosí Boha svého,
#6:12 Tedy přistoupili a mluvili k králi o zápovědi královské: Zdaliž jsi nevydal mandátu, aby každý člověk, kdož by koli něčeho žádal od kterého boha neb člověka až do třidcíti dnů, kromě od tebe, králi, uvržen byl do jámy lvové? Odpověděv král, řekl: Pravéť jest slovo to, podlé práva Médského a Perského, kteréž jest neproměnitelné.
#6:13 Tedy odpovídajíce, řekli králi: Daniel ten, kterýž jest z zajatých synů Judských, nechtěl dbáti na tvé, ó králi, nařízení, ani na mandát tvůj, kterýž jsi vydal, ale třikrát za den modlívá se modlitbou svou.
#6:14 Tedy král, jakž uslyšel tu řeč, velmi se zarmoutil nad tím, a uložil král v mysli své vysvoboditi Daniele, a až do západu slunce usiloval ho vytrhnouti.
#6:15 Ale muži ti shromáždivše se k králi, mluvili jemu: Věz, králi, že jest takové právo u Médských a Perských, aby každá výpověd a nařízení, kteréž by král ustanovil, neproměnitelné bylo.
#6:16 I řekl král, aby přivedli Daniele, a uvrhli jej do jámy lvové. Mluvil pak král a řekl Danielovi: Bůh tvůj, kterémuž sloužíš ustavičně, on vysvobodí tebe.
#6:17 A přinesen jest kámen jeden, a položen na díru té jámy, a zapečetil ji král prstenem svým a prsteny knížat svých, aby nebyl změněn ortel při Danielovi.
#6:18 I odšel král na palác svůj, a šel ležeti, nic nejeda, a ničímž se obveseliti nedal, tak že i sen jeho vzdálen byl od něho.
#6:19 Tedy král hned ráno vstav na úsvitě, s chvátáním šel k jámě lvové.
#6:20 A jakž se přiblížil k jámě, hlasem žalostným zavolal na Daniele, a promluviv král, řekl Danielovi: Danieli, služebníče Boha živého, Bůh tvůj, kterémuž ty sloužíš ustavičně, mohl-liž tě vysvoboditi od lvů?
#6:21 Tedy Daniel mluvil s králem, řka: Králi, na věky buď živ.
#6:22 Bůh můj poslal anděla svého, kterýž zavřel ústa lvů, aby mi neuškodili; nebo před ním nevina nalezena jest při mně, nýbrž ani proti tobě, králi, nic zlého jsem neučinil.
#6:23 Tedy král velmi se z toho zradoval, a rozkázal Daniele vytáhnouti z jámy. I vytažen byl Daniel z jámy, a žádného úrazu není nalezeno na něm; nebo věřil v Boha svého.
#6:24 I rozkázal král, aby přivedeni byli muži ti, kteříž osočili Daniele, a uvrženi jsou do jámy lvové, oni i synové jejich i ženy jejich, a prvé než dopadli dna té jámy, zmocnili se jich lvové, a všecky kosti jejich zetřeli.
#6:25 Tedy Darius král napsal všechněm lidem, národům a jazykům, kteříž bydlili po vší zemi: Pokoj váš rozmnožen buď.
#6:26 Ode mne vyšlo nařízení toto, aby na všem panství království mého třásli a bálise před Bohem Danielovým; nebo on jestBůh živý a zůstávající na věky, a království jeho nebude zrušeno, ani panování jeho až do konce.
#6:27 Vysvobozuje a vytrhuje, a činí znamení a divy na nebi i na zemi, kterýž vysvobodil Daniele z moci lvů.
#6:28 Danielovi pak šťastně se vedlo v království Dariovu, a v království Cýra Perského. 
#7:1 Léta prvního Balsazara krále Babylonského Daniel měl sen a vidění svá na ložci svém, i napsal ten sen krátkými slovy.
#7:2 Mluvil Daniel a řekl? Viděl jsem u vidění svém v noci, a aj, čtyři větrové nebeští bojovali na moři velikém.
#7:3 A čtyři šelmy veliké vystupovaly z moře, jedna od druhé rozdílná.
#7:4 První podobná lvu, a křídla orličí měla. Hleděl jsem, až vytrhána byla křídla její, jimiž se vznášela od země, tak že na nohách jako člověk státi musila, a srdce lidské dáno jest jí.
#7:5 A aj, jiná šelma druhá podobná nedvědu, kteráž panství jedno vyzdvihla, a tři žebra v ústech jejích, mezi zuby jejími, a tak mluveno bylo k ní: Vstaň, nažer se hojně masa.
#7:6 Potom jsem viděl, a aj, jiná podobná pardovi, kteráž měla čtyři křídla ptačí na hřbetě svém, a čtyřhlavá byla šelma ta, jíž moc dána byla.
#7:7 Potom viděl jsem u viděních nočních, a aj, šelma čtvrtá strašlivá a hrozná a velmi silná, mající zuby železné veliké, kteráž zžírala a potírala, ostatek pak nohama svýma pošlapávala; a ta byla rozdílná ode všech šelm, kteréž byly před ní, a měla rohů deset.
#7:8 Pilně jsem šetřil těch rohů, a hle, roh poslední malý vyrostal mezi nimi, a tři z těch rohů prvních vyvráceni jsou před ním; a aj, oči podobné očím lidským v rohu tom, a ústa mluvící pyšně.
#7:9 Hleděl jsem, až trůnové ti svrženi byli, a Starý dnů posadil se, jehož roucho jako sníh bílé, a vlasové hlavy jeho jako vlna čistá, trůn jako jiskry ohně, kola jeho jako oheň hořící.
#7:10 Potok ohnivý tekl a vycházel od něho, tisícové tisíců sloužili jemu, a desetkrát tisíckrát sto tisíců stálo před ním; soud zasedl, a knihy otevříny byly.
#7:11 Patřil jsem tehdáž, hned jakž se začal zvuk té řeči pyšné, kterouž roh mluvil; patřil jsem, dokudž ta šelma nebyla zabita, a vyhlazeno tělo její, a dáno k spálení ohni.
#7:12 A i ostatkům šelm odjali panství; nebo dlouhost života jim odměřena byla až do času, a to uloženého času.
#7:13 Viděl jsem u vidění nočním, a aj, s oblaky nebeskými podobný Synu člověka přicházel; potom až k Starému dnů přišel, a před něj postaven byl.
#7:14 I dáno jest jemu panství a sláva i království, aby všickni lidé, národové a jazykové sloužili jemu; jehož panství jest panství věčné, kteréž nepomíjí, a království jeho, kteréž se neruší.
#7:15 I zhrozil se duch můj ve mně Danielovi u prostřed těla, a vidění má předěsila mne.
#7:16 Tedy přistoupil jsem k jednomu z přístojících, a ptal jsem se ho na jistotu vší té věci. I pověděl mi, a výklad řečí mi oznámil:
#7:17 Ty šelmy veliké, kteréž jsou čtyry, jsou čtyři králové, kteříž povstanou z země,
#7:18 A ujmou království svatých výsostí, kteříž obdržeti mají království až na věky, a až na věky věků.
#7:19 Tedy žádostiv jsem byl zprávy o šelmě čtvrté, kteráž rozdílná byla ode všech jiných, hrozná velmi; zubové její železní, a pazoury její ocelivé, kteráž zžírala, potírala, ostatek pak nohama svýma pošlapávala.
#7:20 Tolikéž o rozích desíti, kteříž byli na hlavě její, a o posledním, kterýž vyrostl, a před ním spadli tři; o tom rohu, pravím, kterýž měl oči a ústa mluvící pyšně, a byl na pohledění větší než jiní.
#7:21 Viděl jsem, an roh ten válku vedl s svatými, a přemáhal je,
#7:22 Až přišel Starý dnů, a oddán jest soud svatým výsostí, a čas přišel, aby to království svatí obdrželi.
#7:23 Řekl takto: Šelma čtvrtá znamená království čtvrté na zemi, kteréž rozdílné bude ode všech království, a zžíře všecku zemi, a zmlátí ji a potře ji.
#7:24 Rohů pak deset znamená, že z království toho deset králů povstane, a poslední povstane po nich, kterýž bude rozdílný od prvních, a poníží tří králů.
#7:25 A slova proti Nejvyššímu mluviti bude, a svaté výsostí potře; nadto pomýšleti bude, aby proměnil časy i práva, když vydáni budou v ruku jeho, až do času a časů, i do částky časů.
#7:26 V tom bude soud osazen, a panství jeho odejmou, vypléní a vyhladí je docela.
#7:27 Království pak i panství, a důstojnost královská pode vším nebem dána bude lidu svatých výsostí; jehož království bude království věčné, a všickni páni jemu sloužiti a jeho poslouchati budou.
#7:28 Až potud konec té řeči. Mne pak Daniele myšlení má velice zkormoutila, a krása má proměnila se při mně, slovo však toto v srdci svém zachoval jsem. 
#8:1 Léta třetího kralování Balsazara krále ukázalo mi se vidění, mně Danielovi po onom, kteréž se mi ukázalo na počátku.
#8:2 I viděl jsem u vidění, (tehdáž pak, když jsem viděl, byl jsem v Susan, na hradě, kterýž jest v krajině Elam), viděl jsem, pravím, byv u potoka Ulai.
#8:3 A pozdvih očí svých, viděl jsem, a aj, u toho potoka stál skopec jeden, kterýž měl dva rohy. A ti dva rohové byli vysocí, a však jeden vyšší než druhý, ale ten vyšší zrostl posléze.
#8:4 Viděl jsem skopce toho, an trkal k západu, půlnoci a poledni, jemuž žádná šelma odolati nemohla, aniž kdo co mohl vytrhnouti z moci jeho; pročež činil podlé vůle své, a to věci veliké.
#8:5 A když jsem to rozvažoval, aj, kozel přicházel od západu na svrchek vší země, a žádný se ho nedotýkal na zemi, a ten kozel měl roh znamenitý mezi očima svýma.
#8:6 A přišel až k tomu skopci majícímu dva rohy, kteréhož jsem byl viděl stojícího u potoka, a přiběhl k němu v prchlivosti síly své.
#8:7 Viděl jsem také, an dotřel na toho skopce, a rozlítiv se proti němu, udeřil jej, tak že zlámal oba rohy jeho, a nebylo síly v skopci k odpírání jemu. A poraziv ho na zemi, pošlapal jej, aniž byl, kdo by vytrhl skopce z moci jeho.
#8:8 Kozel pak velikým učiněn jest velmi. A když se ssilil, zlámal se roh ten veliký, i zrostli znamenití čtyři místo něho, na čtyři strany světa.
#8:9 Z těch pak jednoho vyšel roh jeden maličký, a zrostl velmi ku poledni a východu, a k zemi Judské.
#8:10 A zpjal se až k vojsku nebeskému, a svrhl na zemi některé z vojska toho i z hvězd, a pošlapal je.
#8:11 Anobrž až k vojska toho knížeti zpjal se, nebo od něho zastavena byla ustavičná obět, a zavržen příbytek svatyně Boží,
#8:12 Tak že vojsko to vydáno v převrácenost proti ustavičné oběti, a povrhlo pravdu na zemi, a což činilo, šťastně mu se dařilo.
#8:13 Tedy slyšel jsem jednoho svatého mluvícího, a řekl ten svatý tomu, kterýž tajné věci v počtu maje, mluví: Dokudž toto vidění o oběti ustavičné, a převrácenost na zpuštění přivodící trvati bude, a svaté služby vydávány budou i vojsko v pošlapání?
#8:14 A řekli mi: Až do dvou tisíc a tří set večerů a jiter, a přijdou k obnovení svému svaté služby.
#8:15 Stalo se pak, že když jsem já Daniel hleděl na to vidění, a ptal jsem se na rozum jeho, aj, postavil se podlé mne na pohledění jako muž.
#8:16 Slyšel jsem také hlas lidský mezi Ulaiem, kterýžto zavolav, řekl: Gabrieli, vylož tomuto vidění to.
#8:17 I přišel ke mně, kdež jsem stál, a když přišel, zhrozil jsem se, a padl jsem na tvář svou. I řekl ke mně: Pozoruj, synu člověčí; nebo v času uloženém vidění toto se naplní.
#8:18 Když pak on mluvil se mnou, usnul jsem tvrdě, leže tváří svou na zemi. I dotekl se mne, a postavil mne tu, kdež jsem byl stál,
#8:19 A řekl: Aj, já oznámím tobě to, což se díti bude až do vykonání hněvu toho; nebo v uloženém času konec bude.
#8:20 Skopec ten, kteréhož jsi viděl, an měl dva rohy, jsou králové Médský a Perský.
#8:21 Kozel pak ten chlupatý jest král Řecký, a roh ten veliký, kterýž jest mezi očima jeho, jest král první.
#8:22 Že pak zlámán jest, a povstali čtyři místo něho, čtvero království z svého národu povstane, ale ne s takovou silou.
#8:23 Při dokonání pak království jejich, když na vrch vzejdou nešlechetníci, povstane král nestydatý a chytrý.
#8:24 Jehož síla zmocní se, ačkoli ne jeho silou, tak že ku podivení hubiti bude, a šťastně se mu povede, až i vše vykoná; nebo hubiti bude silné i lid svatý.
#8:25 A obmyslností svou šťastně svede lest v předsevzetí svém, a v srdci svém zvelebí sebe, a v čas pokoje zhubí mnohé; nadto i proti knížeti knížat se postaví, a však bez rukou potřín bude.
#8:26 Vidění pak to večerní a jitřní, o němž povědíno, jest jistá pravda; pročež ty zavři to vidění, nebo jest mnohých dnů.
#8:27 Tedy já Daniel zchuravěl jsem, a nemocen jsem byl několik dnů. Potom povstav, konal jsem povinnost od krále poručenou, byv předěšen nad tím viděním, čehož však žádný na mně neseznal. 
#9:1 Léta prvního Daria syna Asverova z semene Médského, kteréhož učiněn jest králem v království Kaldejském,
#9:2 Léta prvního kralování jeho já Daniel porozuměl jsem z knih počtu let, o nichž se stalo slovo Hospodinovo k Jeremiášovi proroku, že se vyplní zpuštění Jeruzaléma sedmdesátého léta.
#9:3 A obrátil jsem tvář svou ku Pánu Bohu, hledaje ho modlitbou a pokornými prosbami, v postu, v žíni a popele.
#9:4 I modlil jsem se Hospodinu Bohu svému, a vyznávaje se, řekl jsem: Prosím, Pane Bože silný, veliký a všeliké cti hodný, ostříhající smlouvy a dobrotivosti k těm, kteříž tě milují, a ostříhají přikázaní tvých.
#9:5 Zhřešiliť jsme a převráceně jsme činili, bezbožnost jsme páchali, a protivili jsme se, a odvrátili od přikázaní tvých a soudů tvých.
#9:6 Aniž jsme poslouchali služebníků tvých proroků, kteříž mluvívali ve jménu tvém králům našim, knížatům našim a otcům našim, i všemu lidu země.
#9:7 Toběť, ó Pane, přísluší spravedlnost, nám pak zahanbení tváři, jakž se to děje nyní mužům Judským a obyvatelům Jeruzalémským, a všechněm Izraelským, blízkým i dalekým, ve všech zemích, kamž jsi je zahnal pro přestoupení jejich, jímž přestupovali proti tobě.
#9:8 Námť, ó Hospodine, sluší zahanbení tváři, králům našim, knížatům našim a otcům našim, neboť jsme zhřešili proti tobě,
#9:9 Pánu Bohu pak našemu milosrdenství a slitování, poněvadž jsme se protivili jemu,
#9:10 A neposlouchali jsme hlasu Hospodina Boha našeho, abychom chodili v naučeních jeho, kteréž předkládal před oči naše skrze služebníky své proroky.
#9:11 Nýbrž všickni Izraelští přestoupili zákon tvůj a odvrátili se, aby neposlouchali hlasu tvého; protož vylito jest na nás to prokletí a klatba, kteráž jest zapsána v zákoně Mojžíše služebníka Božího, nebo jsme zhřešili proti němu.
#9:12 Pročež splnil slovo své, kteréž mluvil proti nám a proti soudcům našim, kteříž nás soudili, a uvedl na nás toto zlé veliké, jehož se nestalo pode vším nebem, jakéž se stalo v Jeruzalémě.
#9:13 Tak jakž zapsáno jest v zákoně Mojžíšově, všecko to zlé přišlo na nás, a však ani tak jsme se nekořili před tváří hněvivou Hospodina Boha našeho, abychom se odvrátili od nepravostí svých, a šetřili pravdy jeho.
#9:14 Protož neobmeškal Hospodin s tím zlým, ale uvedl je na nás; nebo spravedlivý jest Hospodin Bůh náš ve všech skutcích svých, kteréž činí, jehož hlasu poslušni jsme nebyli.
#9:15 Nyní však, ó Pane Bože náš, kterýž jsi vyvedl lid svůj z země Egyptské rukou silnou, a způsobils sobě jméno, jakéž jest dnešního dne, zhřešili jsme, bezbožně jsme činili.
#9:16 Ó Pane, podlé vší tvé dobrotivosti nechť se, prosím, odvrátí hněv tvůj a prchlivost tvá od města tvého Jeruzaléma, hory svatosti tvé; nebo pro hříchy naše a pro nepravosti otců našich Jeruzalém a lid tvůj v pohanění jest u všech, kteříž jsou vůkol nás.
#9:17 Nyní tedy, ó Bože náš, vyslyš modlitbu služebníka svého, a pokorné prosby jeho, a zasvěť tvář svou nad svatyní svou spuštěnou, pro Pána.
#9:18 Nakloň, Bože můj, ucha svého a slyš, otevři oči své a viz zpuštění naše i města, kteréž jest nazváno od jména tvého; nebo ne pro nějaké naše spravedlnosti padajíce, pokorně prosíme tebe, ale pro milosrdenství tvá mnohá.
#9:19 Vyslyšiž, ó Pane, Pane, odpusť, Pane, pozoruj a učiň; neprodlévejž pro sebe samého, můj Bože, nebo od jména tvého nazváno jest město toto i lid tvůj.
#9:20 Ještě jsem mluvil a modlil se, a vyznával hřích svůj i hřích lidu svého Izraelského, a padna, pokorně jsem se modlil před tváří Hospodina Boha svého, za horu svatosti Boha svého,
#9:21 Ještě, pravím, mluvil jsem při modlitbě, a muž ten Gabriel, kteréhož jsem viděl v tom vidění na počátku, rychle přiletěv, dotekl se mne v čas oběti večerní.
#9:22 A slouže mi k srozumění, mluvil se mnou a řekl: Danieli, nyní jsem vyšel, abych tě naučil vyrozumívati tajemstvím.
#9:23 Při počátku pokorných proseb tvých vyšlo slovo, a já jsem přišel, aťbych je oznámil, nebo jsi velmi milý; pročež pozoruj slova toho, a rozuměj vidění tomu.
#9:24 Sedmdesáte téhodnů odečteno jest lidu tvému a městu svatému tvému k zabránění převrácenosti a k zapečetění hříchů, i vyčištění nepravosti a k přivedení spravedlnosti věčné, a k zapečetění vidění i proroctví, a ku pomazání Svatého svatých.
#9:25 Věziž tedy a rozuměj, že od vyjití výpovědi o navrácení a o vystavení Jeruzaléma až do Mesiáše vývody bude téhodnů sedm, potom téhodnů šedesáte dva, když již zase vzdělána bude ulice a příkopa, a ti časové budou přenesnadní.
#9:26 Po téhodnech pak těch šedesáti a dvou zabit bude Mesiáš, však jemu to nic neuškodí; nýbrž to město i tu svatyni zkazí, i lid ten svůj budoucí, tak že skončení jeho bude hrozné, ano i do vykonání boje bude boj stálý všelijak do vyplénění.
#9:27 Utvrdí však smlouvu mnohým v téhodni posledním, u prostřed pak toho téhodne učiní konec oběti zápalné i oběti suché; a skrze vojsko ohavné, až do posledního a uloženého poplénění hubící, na popléněné vylito bude zpuštění. 
#10:1 Léta třetího Cýra krále Perského, zjeveno bylo slovo Danielovi, kterýž sloul jménem Baltazar, a pravé bylo slovo to, i uložený čas dlouhý, a rozum toho slova i smysl zjeven jemu u vidění.
#10:2 V těch dnech já Daniel kvílil jsem za tři téhodny dnů.
#10:3 Pokrmu pochotného jsem nejedl, ani maso ani víno nevešlo do úst mých, aniž jsem se mastí mazal, až se vyplnili dnové tří téhodnů.
#10:4 Dne pak dvadcátého čtvrtého měsíce prvního, když jsem byl na břehu řeky veliké, to jest Hiddekel.
#10:5 Pozdvih očí svých, viděl jsem, a aj, muž jeden oděný v roucho lněné, a bedra jeho přepásaná byla zlatem ryzím z Ufaz.
#10:6 Tělo pak jeho jako tarsis, a oblíčej jeho na pohledění jako blesk, a oči jeho podobné pochodním hořícím, a ramena jeho i nohy jeho na pohledění jako měď vypulerovaná, a zvuk slov jeho podobný zvuku množství.
#10:7 Viděl jsem pak já Daniel sám vidění to, ale muži ti, kteříž se mnou byli, neviděli toho vidění, než hrůza veliká připadla na ně, až i utekli, aby se skryli.
#10:8 Pročež já zůstal jsem sám, a viděl jsem vidění to veliké, ale nezůstalo i ve mně síly, a krása má změnila se, a porušila na mně, aniž jsem mohl zadržeti síly.
#10:9 Tedy slyšel jsem zvuk slov jeho, a uslyšav zvuk slov jeho, usnul jsem tvrdě na tváři své, na tváři své na zemi.
#10:10 V tom aj, ruka dotkla se mne, a pozdvihla mne na kolena má a na dlaně rukou mých.
#10:11 I řekl mi: Danieli, muži velmi milý, pozoruj slov, kteráž já mluviti budu tobě, a stůj na místě svém, nebo nyní poslán jsem k tobě. A když promluvil ke mně slovo to, stál jsem, třesa se.
#10:12 Pročež řekl mi: Nebojž se, Danieli; nebo od prvního dne, jakž jsi přiložil srdce své, abys rozuměl, a trápil se před Bohem svým, vyslyšána jsou slova tvá, a já přišel jsem příčinou slov tvých.
#10:13 Ale kníže království Perského postavovalo se proti mně za jedenmecítma dnů, až aj, Michal, jeden přední z knížat, přišel mi na pomoc; protož jsem já zůstával tam při králích Perských.
#10:14 Již pak přišel jsem, aťbych oznámil, co potkati má lid tvůj v potomních dnech; nebo ještě vidění bude o těch dnech.
#10:15 A když mluvil ke mně ta slova, sklopiv tvář svou k zemi, oněměl jsem.
#10:16 A aj, jako podobnost člověka dotkla se rtů mých, a otevřev ústa svá, mluvil jsem a řekl tomu, kterýž stál naproti mně: Pane můj, příčinou toho vidění obrátili se bolesti mé na mne, a aniž jsem síly zadržeti mohl.
#10:17 Jakž tedy bude moci služebník Pána mého takový mluviti se Pánem mým takovým, poněvadž ve mně od toho času, ve mně, pravím, nezůstalo síly, ani dchnutí nepozůstalo ve mně?
#10:18 Pročež opět dotkl se mne na pohledění jako člověk, a posilnil mne.
#10:19 A řekl: Neboj se, muži velmi milý, pokoj tobě, posilň se, posilň se, pravím. Když pak on mluvil se mnou, posilněn jsa, řekl jsem: Nechť mluví Pán můj, nebo jsi mne posilnil.
#10:20 I řekl: Víš-liž, proč jsem přišel k tobě? Nebo již navrátím se, abych bojoval s knížetem Perským. A já odcházím, a aj, kníže Řecké přitáhne.
#10:21 Ale oznámímť to, což jest zapsáno v psání pravdomluvném; nebo ani jednoho není, ješto by sobě zmužile počínal se mnou v těch věcech, kromě Michala knížete vašeho. 
#11:1 A tak já léta prvního Daria Médského postavil jsem se, abych ho zmocňoval a posiloval.
#11:2 Již pak oznámímť pravdu: Aj, ještě tři králové kralovati budou v Perské zemi; potom čtvrtý zbohatne bohatstvím velikým nade všecky, a když se zmocní v bohatství svém, vzbudí všecky proti království Řeckému.
#11:3 I povstane král mocný, kterýž bude míti panství široké, a bude činiti podlé vůle své.
#11:4 Když se pak zmocní, potříno bude království jeho, a rozděleno bude na čtyři strany světa, však ne mezi potomky jeho, aniž bude panství jeho takové, jakéž bylo; nebo vykořeněno bude království jeho, a jiným mimo ně se dostane.
#11:5 Pročež posilní se král polední, ano i jedno z knížat jeho, a mocnější bude nad něho, a panovati bude; panství široké bude panství jeho.
#11:6 Po některých pak letech spřízní se; nebo dcera krále poledního dostane se za krále půlnočního, aby učinila příměří. Ale neobdrží síly ramene, aniž on ostojí s ramenem svým, ale vydána bude ona i ti, kteříž ji přivedou, i syn její, i ten, kterýž ji posilňoval v ty časy.
#11:7 Potom povstane z výstřelku kořenů jejích na místo jeho, kterýž přitáhne s vojskem svým, a udeří na pevnost krále půlnočního, a přičiní se, aby se jich zmocnil.
#11:8 Nadto i bohy jejich s knížaty jejich, s nádobami drahými jejich, stříbrem a zlatem v zajetí zavede do Egypta, a bude bezpečen za mnoho let před králem půlnočním.
#11:9 A tak přijde do království král polední, a navrátí se do země své.
#11:10 Ale synové onoho válčiti budou, a seberou množství vojsk velikých. A nenadále přijda, jako povodeň procházeti bude, a navracuje se, válkou dotírati bude až k jeho pevnostem.
#11:11 Pročež rozdrážděn jsa král polední, vytáhne, a bojovati bude s ním, s králem půlnočním, a sšikuje množství veliké, i bude vydáno množství to v ruku jeho.
#11:12 I pozdvihne se množství to, a povýší se srdce jeho, a ačkoli porazí na tisíce, a však se nezmocní.
#11:13 Potom navrátě se král půlnoční, sšikuje množství větší než prvé, a po dokonání času některých let, nenadále přijde s vojskem velikým a s dostatkem hojným.
#11:14 V těch časích mnozí se postaví proti králi polednímu, ale synové nešlechetní z lidu tvého zhubeni budou, a pro stvrzení vidění tohoto padnou.
#11:15 Nebo přitáhne král půlnoční, a vzdělaje náspy, dobude měst hrazených, tak že ramena poledního neostojí, ani lid vybraný jeho, aniž budou míti síly k odpírání.
#11:16 A přitáhna proti němu, bude činiti podlé vůle své, a nebude žádného, ješto by se postavil proti němu. Postaví se také v zemi Judské, kterouž docela zkazí rukou svou.
#11:17 Potom obrátí tvář svou, aby přitáhna s mocí všeho království svého, a ukazuje se jako by vše upřímě jednal, dovede něčeho. Nebo dá jemu krásnou pannu, aby ho zahubil skrze ni, ale ona nedostojí aniž bude držeti s ním.
#11:18 Zatím obrátí tvář svou k ostrovům, a dobude mnohých, ale vůdce přítrž učiní pohanění jeho, anobrž to hanění jeho na něj obrátí.
#11:19 Pročež obrátí tvář svou k pevnostem země své, ale klesne a padne, i zahyne.
#11:20 I povstane na místo jeho v slávě královské ten, kterýž rozešle výběrčí, ale ten po nemnohých dnech potřín bude, a to ne v hněvě, ani v boji.
#11:21 Na místě tohoto postaví se nevzácný, ačkoli nevloží na něj ozdoby královské, a však přijda pokojně, ujme království skrze úlisnost.
#11:22 A rameny jako povodní zachváceni budou před oblíčejem jeho mnozí, a potříni budou jako i ten vůdce, kterýž s ním smlouvu učinil.
#11:23 Nebo v tovaryšství s ním vejda, prokáže nad ním lest, a přijeda, zmocní se království s malým počtem lidu.
#11:24 Bezpečně také i do nejúrodnějších míst té krajiny vpadne, a činiti bude to, čehož nečinili otcové jeho, ani otcové otců jeho; loupež a kořisti a zboží jim rozdělí, ano i proti pevnostem chytrosti své vymýšleti bude, a to do času.
#11:25 Potom vzbudí sílu svou a srdce své proti králi polednímu s vojskem velikým, s nímž král polední vojensky se potýkati bude, s vojskem velikým a velmi silným, ale neostojí, proto že vymyslí proti němu chytrost.
#11:26 Nebo kteříž jídají pokrm, potrou jej, když vojsko onoho se rozvodní; i padnou, zbiti jsouce mnozí.
#11:27 Tehdáž obou těch králů srdce bude činiti zlé, a za jedním a týmž stolem lež mluviti budou, ale nepodaří se, proto že cíl uložený na jiný ještě čas odložen.
#11:28 A protož navrátí se do země své s zbožím velikým, a srdce jeho bude proti smlouvě svaté. Což učině, navrátí se do země své.
#11:29 V uložený čas navrátě se, potáhne na poledne, ale to nebude podobné prvnímu ani poslednímu.
#11:30 Nebo přijdou proti němu lodí Citejské, pročež bude jej to boleti, tak že opět zlobiti se bude proti smlouvě svaté. Což učině, navrátí se, a srozumění míti bude s těmi, kteříž opustili smlouvu svatou.
#11:31 A vojska veliká podlé něho státi budou, a poškvrní svatyně a pevnosti; odejmou také obět ustavičnou, a postaví ohavnost zpuštění,
#11:32 Tak aby ty, kteříž se bezbožně proti smlouvě chovati budou, v pokrytství posiloval úlisnostmi, lid pak, kterýž zná Boha svého, aby jímali. Což i učiní.
#11:33 Pročež vyučující lid, vyučující mnohé, padati budou od meče a ohně, zajetí a loupeže za mnohé dny.
#11:34 A když padati budou, malou pomoc míti budou; nebo připojí se k nim mnozí pochlebně.
#11:35 Z těch pak, kteříž jiné vyučují, padati budou, aby prubováni a čištěni a bíleni byli až do času jistého; neboť to ještě potrvá až do času uloženého.
#11:36 Král zajisté ten bude činiti podlé vůle své, a pozdvihne se a zvelebí nad každého boha, i proti Bohu nade všemi bohy nejsilnějšímu mluviti bude divné věci; a šťastně se mu povede až do vykonání prchlivosti, ažby se to, což uloženo jest, vykonalo.
#11:37 Ani k bohům otců svých se nenakloní, ani k milování žen, aniž k komu z bohů se nakloní, proto že se nade všecko velebiti bude.
#11:38 A na místě Boha nejsilnějšího ctíti bude boha, kteréhož neznali otcové jeho; ctíti bude zlatem a stříbrem, a kamením drahým a klénoty.
#11:39 A tak dovede toho, že pevnosti Nejsilnějšího budou boha cizího, a kteréž se mu viděti bude, poctí slávou, a způsobí, aby panovali nad mnohými, a zemi rozdělí místo mzdy.
#11:40 Při dokonání pak toho času trkati se s ním bude král polední, ale král půlnoční oboří se na něj s vozy a s jezdci a lodími mnohými, a přitáhna do zemí, jako povodeň projde.
#11:41 Potom přitáhne do země Judské, a mnohé země padnou. Tito pak ujdou ruky jeho, Idumejští a Moábští, a prvotiny synů Ammon.
#11:42 A když ruku svou vztáhne na země, ani země Egyptská nebude moci jeho zniknouti.
#11:43 Nebo opanuje poklady zlata a stříbra, a všecky klénoty Egyptské; Lubimští také a Mouřenínové za ním půjdou.
#11:44 V tom noviny od východu a od půlnoci přestraší jej; pročež vytáhne s prchlivostí velikou, aby hubil a mordoval mnohé.
#11:45 I rozbije stany paláce svého mezi mořemi, na hoře okrasy svatosti; a když přijde k skonání svému, nebude míti žádného spomocníka. 
#12:1 Toho času postaví se Michal, kníže veliké, kterýž zastává synů lidu tvého, a bude čas ssoužení, jakéhož nebylo, jakž jest národ, až do toho času; toho, pravím, času vysvobozen bude lid tvůj, kdožkoli nalezen bude zapsaný v knize.
#12:2 Tuť mnozí z těch, kteříž spí v prachu země, procítí, jedni k životu věčnému, druzí pak ku pohanění a ku potupě věčné.
#12:3 Ale ti, kteříž jiné vyučují, stkvítí se budou jako blesk oblohy, a kteříž k spravedlnosti přivozují mnohé, jako hvězdy na věčné věky.
#12:4 Ty pak Danieli, zavři slova tato, a zapečeť knihu tuto až do času jistého. Mnozíť budou pilně zpytovati, a rozmnoženo bude umění.
#12:5 Zatím viděl jsem já Daniel, a aj, jiní dva stáli, jeden z této strany břehu řeky, a druhý z druhé strany břehu též řeky.
#12:6 A řekl muži tomu oblečenému v roucho lněné, kterýž stál nad vodou té řeky: Když bude konec těm divným věcem?
#12:7 I slyšel jsem muže toho oblečeného v roucho lněné, kterýž stál nad vodou té řeky, an zdvihl pravici svou i levici svou k nebi, a přisáhl skrze Živého na věky, že po uloženém času, a uložených časích, i půl času, a když do cela rozptýlí násilí lidu svatého, dokonají se všecky tyto věci.
#12:8 A když jsem já slyše, nerozuměl, řekl jsem: Pane můj, jaký konec bude těch věcí?
#12:9 Tedy řekl: Odejdi, Danieli, nebo zavřína jsou a zapečetěna slova ta až do času jistého.
#12:10 Přečišťováni a bíleni a prubováni budou mnozí; bezbožní zajisté bezbožnost páchati budou, aniž co porozumějí kteří z nich, alemoudří porozumějí.
#12:11 Od toho pak času, v němž odjata bude obět ustavičná, a postavena ohavnost hubící, bude dnů tisíc, dvě stě a devadesát.
#12:12 Blahoslavený, kdož dočeká a přijde ke dnům tisíci, třem stům, třidcíti a pěti.
#12:13 Ty pak odejdi k místu svému, a odpočívati budeš, a zůstaneš v losu svém na skonání dnů.


