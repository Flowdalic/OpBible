\def\nadpis#1{\smallskip\noindent{\bf#1}\par\nobreak}


\nadpis{Přehled}

\nadpis{Autor:} Daniel

\nadpis{Záměr:}

\begitems
* Připravit babylonského krále \x/Nabuchodonozor/a na přijetí špatných zpráv kapitoly 4. budováním důvěry ve spolehlivost a pravdivost Danielových proroctví a všemohoucnost jeho Boha v předchozích kapitolách. 
* Ujistit Izraelity (zajatce i první navrátilce do Země), že Bůh panuje nad dějinami a že jeho prorok Daniel říkal pravdu, když mluvil o prodloužené době útisku před závěrečnou fází Božího království. 
* Připravit generace vzdálené budoucnosti na pronásledování, které je bude čekat v době Antiocha IV. Epifana.
* Připravit věřící v ještě vzdálenější budoucnosti na příchod Mesiáše v době čtvrtého království.
\enditems 

%\doImage {Středozem}[stredozem](){stredozemne.jpg}
\insertBot {Říše Danielových vzdálených vizí}[mapa](){
  \inspic{mapa-crop.pdf}
  \Heros \cond \setfontsize{at 9pt}\rm
\puttext -1.85mm 112.5mm {\vtop{\hsize7.1cm \baselineskip10pt \noindent
  Brzy po smrti Alexandra Velikého v roce 323 př. Kr. si jeho království rozdělili jeho čtyři generálové.
  %Diadochi, jak se jim říkalo, se nedohodli
  Nedohodli se však mírumilovně; navzájem nelítostně  válčili o moc. Výsledkem bylo, že Egypt připadl Ptolemaiovi I.,  Sýrie a  Mezopotámie Seleukovi, Thrácie a Malá Asie Lysimachovi a Makedonie Kassandrovi. 
  Po roce 277 zůstala jen tři helenistická království v Egyptě, Sýrii a Makedonii, která si podržela moc až do porážky Římem  v r. 63 př. Kr.}}
\puttext 54.15mm 19.5mm {\vtop{\hsize9.1cm \baselineskip10pt \noindent
    Svatá země byla pod nadvládou Ptolemaiovské dynastie (tzn. Egypta) od roku 323 do 198 př. Kr.; pak pod nadvládou Seleukovců (tedy Sýrie) od 198 do 142 př. Kr. Jejich vzájemné tahanicce předvídá <Da 11:2-45>.  Dynastie Hasmoneovců udržela Judeu nezávislou na seleukovské říši od roku 140  až do roku 63 př. Kr.,  kdy řecké impérium dobyl Řím a za krále Judeje dosadil Heroda Velikého (<Mt 2:1>).}}
  \LMfonts \sans \setfontsize{at9pt}\rm
  %\Heros \cond \setfontsize{at9pt}\rm 
  \puttext 145mm 29mm {<"Sk 2:9"_Sk 2:9>}
  \puttext 145mm 32.5mm {<"Ez 32:16"_Ez 32:16>}
  \puttext 145mm 36mm {<"Jr 25:25"_Jr 25:25>}
  \puttext 145mm 39.5mm {<"Jr 49:35"_Jr 49:35>}
  \puttext 145mm 43mm {<"Iz 22:6"_Iz 22:6>}
  \puttext 144mm 46.5mm {<"Iz 11:11"_Iz 11:11>}
  \puttext 142mm 50mm {<"Gn 10:22"_Gn 10:22>}
  \puttext 134mm 58mm {<"Da 8:2"_Da 8:2>}
  \puttext 134mm 55mm {<"Est 1:2"_Est 1:2>}
  \puttext 60mm 36mm {<"Da 1:1"_Da 1:1>}
  \puttext 55mm 52mm {<"Da 11:5 (\uv{\x/král polední/})"_Da 11:5>}
  \puttext 55mm 62mm {<"Da 11:6 (\uv{\x/král půlnoční/})"_Da 11:6>}
  \puttext 88mm 56mm {<"Da 3:1"_Da 3:1>}
  \puttext 95mm 41mm {<"Da 1:2"_Da 1:2>}
  \puttext 95mm 37.5mm {<"Gn 11:2"_Gn 11:2>}
  \puttext 95mm 34mm {<"Gn 14:1"_Gn 14:1>}
  \puttext 106.5mm 34mm {<",9"_Gn 14:9>}
  \puttext 95mm 30.5mm {<"Joz 7:21"_Joz 7:21>}
  \puttext 95mm 27mm {<"Iz 11:11"_Iz 11:11>}
  \puttext 95mm 23.5mm {<"Za 5:11"_Za 5:11>}
  \puttext 115mm 34mm {<"Gn 11:31"_Gn 11:31>}
  \Heros \setfontsize{at 9pt}\rm
  \puttext 51mm 52mm {\c[-40/\kern1pt\pdfrotate{0}]{Jeruzalém}}
  \puttext 116mm 37.5mm{Ur}
  \puttext 122mm 55mm{\x/Susan/}
  \puttext 88mm 53mm{Tolul Dura}
  \puttext 100mm 63mm{Babylon}
  \Heros \setfontsize{at 13pt}\rm
  \puttext 62mm 77mm {\c[10/\kern7pt\pdfrotate{-1}]{SELEUKOVCI}}
  \puttext 2mm 39mm {\c[0/\kern4pt\pdfrotate{2.5}]{PTOLEMAIOVCI}}
  \Heros \setfontsize{at 11pt}\rm
  \puttext 130mm 50mm {\c[-40/\kern4pt\pdfrotate{-1}]{\x/ELAM/}}
  \puttext 97mm 49.5mm {\c[-25/\kern4pt\pdfrotate{1}]{\x/SINEAR/}}
  \puttext 50mm 30mm{A R A B S K Á}
  \puttext 55mm 25mm{P O U Š Ť}
  }

%\leftline{<3:1> \hskip 20mm <8:2>; <Est %1:2>}

\nadpis{Datum:} Krátce po 539 př. Kr.

\nadpis{Klíčové pravdy:}

\begitems
* Daniel a jeho přátelé byli i v exilu Bohu věrni.

* Danielovi lze věřit, že říká pravdu, protože svou víru nikdy nezkompromitoval ani pod nátlakem svých otrokářů.

* Bůh je absolutní vládce nad veškerou historií.

* Otroctví Izraele je prodlouženo do doby, než se v nadvládě nad ním nevystřídají celkem čtyři  království (z nichž Babylón je první), protože Boží lid se neodvrátil od svých hříchů. 

* Přestože Izrael čeká v budoucnosti ještě spousta utrpení, Boží Pomazaný, Kristus, jednou přijde a přinese spásu.

\enditems
%\vglue-3cm \leftline{\hskip 8cm<3:1> \hskip 20mm <8:2>; <Est 1:2>}



\nadpis{Autor}

Autorství knihy \x/Daniel/ je mezi vykladači předmětem rozvleklých debat.
Mnoho badatelů datuje vznik knihy mezi 170 a 165 př. Kr., do doby vlády Antiocha IV. Epifana, dávno poté, co žil prorok Daniel (tzv. \uv{makabejské} datování, srv. čl.~<"Kdo byl \x/Darius/ Médský?"_6>a na str.~\pg). 
Toto datum je však v rozporu s knihou samotnou, která naznačuje, že \x/Daniel/ je její hlavní autor (<9:2>; <10:2>) a že byla sepsána krátce po dobytí Babylóna \x/Cýr/em v roce 539 př. Kr. Kromě toho  Kristus sám explicitně spojuje knihu s prorokem \x/Daniel/em (<Mt 24:15>).

\nadpis{Doba a místo vzniku}

Spor o datování knihy \x/Daniel/ zahrnuje tři základní problémy:

\begitems \style n
* povahu proroctví,
* údajné historické chyby v Danielovi a 
* jazykové rysy hebrejštiny a aramejštiny v knize.
\enditems

Obecně vzato, izraelští proroci se primárně zabývali náboženskými a společenskými okolnostmi, které se týkaly jich samých a jejich vrstevníků. Když proroci předvídali budoucnost, většinou se týkala blízkých událostí v dohlednu.
Z toho důvodu jsou někteří interpreti toho názoru, že \x/Daniel/ova vize ohledně \uv{krále Severu} a \uv{krále Jihu} (\<11:2-12:3>) je příliš podrobná na to, aby ji mohl sepsat \x/Daniel/, který žil nějakých 200--300 let před událostmi, zachycenými v proroctví.

Tento postoj však popírá nadpřirozený charakter proroctví, podobně jako v případě příležitostných praktik jiných proroků (např. <1Kr 13:2>; <Iz 44:28>). Přestože pasáž  \<Da 11:2-12:3> je neobvyklá, určitě není nemožné, aby \x/Daniel/ takové detaily znal; ostatně právě jemu Bůh zjevoval tajnosti jako nikomu jinému (srv. např. <2:19-23>).  




\renum Da 5:31 = CSP 6:1-1
\renum Da 5:31 = CEP 6:1-1
\renum Da 5:31 = B21 6:1-1
\renum Da 5:31 = SNC 6:1-1 

\renum Da 6:1 = CSP 6:2-29
\renum Da 6:1 = CEP 6:2-29
\renum Da 6:1 = B21 6:2-29
\renum Da 6:1 = SNC 6:2-29


Někteří zastánci pozdního datování argumentují historickými nepřesnostmi, které knize \x/Daniel/ přičítají.
Zpochybňují \x/Belšazar/ův vztah k \x/Nabuchodonozor/ovi (viz <"pozn." 5:2>n), stejně jako i identitu  \x/Daria/ Médského (viz <"pozn." 6:1>). 

Navíc identifikují čtyři království, předpovězená Danielem (kap. 2; 7), jako Babylón, Médeu, Persii a Řecko (včetně Seleucidů a Ptolemájů). Tato identifikace je však problematická, protože pro nezávislé Médské království v intervalu mezi královstvími Babylónským a Perským neexistuje historický důkaz.
Perský král \x/Cýros/ (550--530 př. Kr.) dobyl Médeu roku 549 př. Kr. a Babylón 539 př. Kr. (viz poznámky <5:1>n a <5:31>n).

Advokáti raného datování knihy chápou sekvenci čtyř království jako předpověď Babylóna, Médo-Persie, Řecka a Říma. 
Tento pohled podporuje narážka na \uv{Médy a Peršany} v <5:28>, která prokazuje, že autor považoval oba národy za součásti jednoho království.

%\Citehere 3 (\kern-2mm) {
%   Člověk přišel na svět proto, aby tady byl, pracoval a žil. Jen moudrý se snaží náš svět postrčit dál, posunout výš. A jen vůl mu v~tom brání.
%   \quotedby {Jan Werich}
%}
Podporovatelé pozdního data namítají, že se v textu vyskytuje několik termínů vypůjčených z řečtiny k označení hudebních nástrojů (viz <"pozn." 3:5>n), podobně jako i  pozdně hebrejské a aramejské výrazy (viz <"pozn." 2:4>n).
Žádný z těchto argumentů však není přesvědčivý.
Existuje nepřeberné množství důkazů o kontaktech mezi Řeky a národy Blízkého Východu před dobou Alexandra Velikého. Ty zcela postačují k vysvětlení existence minimálního počtu slov převzatých z řečtiny před Alexandrovým dobytím. 
Původní názvy hudebních nástrojů běžně provázejí své nositele bez odpovídajícího ekvivalentu v lokálním jazyce; srovnejme dnešní českou nepřekládanou terminologii, spojenou s hudebními nástroji:  \uv{gibsonka}, \uv{jumbo}, \uv{stratocaster}, \uv{telecaster}, \uv{Les Paul}, \uv{stage piano}, \uv{hohnerka}, \uv{humbucker}, \uv{single-coil} apod.
Naopak: Zastánci makabejského datování mají problém, jak vysvětlit absolutní absenci výrazů, přejatých z řečtiny, {\it mimo\/} hudební terminologii. Kdyby kniha vznikla až za řecké vlády, obchodní, vojenská, politická, administrativní apod. terminologie  by se hemžila řeckými pojmy. Nic takového však v knize není.

Aramejština a hebrejština knihy Daniel může být datována kdekoliv mezi pozdním šestým a raným druhým stoletím př. Kr. Jinými slovy, lingvistické důkazy nepřikládají příliš váhy žádnému z hledisek: ani pozdnímu, ani ranému datování.

Argument pro datum ve druhém století př. Kr. je v rozporu s biblickým tvrzením ohledně data a autorství knihy \x/Daniel/ a pozdní datování neprokazuje dostatečně přesvědčivě.   Datum krátce po 539 př. Kr. (viz <1:21>) nejlépe odpovídá povaze proroctví, historickým datům i jazykové stránce textu.


 \nadpis{Záměr a zvláštnosti}
 
 Daniel obsahuje dva různé druhy materiálu.
 V prvních šesti kapitolách je šest historických vyprávění; ve druhé polovině (kapitoly 7--12) jsou čtyři vize, téměř exkluzivně prediktivní. Mezi šesti příběhy první poloviny vyčnívá kapitola 2, protože také obsahuje předpověď. 

Zkoumání obsahu historických vyprávění ukazuje, že jsou to nezávislé celky, poskládané k sobě s určitým záměrem.
Vyprávění nenabízí ani historii Izraele pod babylónskou či perskou nadvládou, ani životopisné záznamy \x/Daniel/a a jeho přátel. Má dva hlavní důrazy.

Na jednu stranu příběhy ukazují, jak Boží absolutní svrchovanost zasahuje do záležitostí všech národů 
(<2:47>; <3:17-18>; <4:28-37>; <5:18-31> <6:25-28>).
Jeruzalém byl v troskách, Boží lid v zajetí a bezbožní vládcové se zdáli triumfovat, avšak Bůh zůstává svrchovaný.
Podle své neochvějné vůle vstupuje mezi království tohoto světa, aby založil univerzální království, jemuž nikdy nebude konce.

Tyto příběhy zachycují \x/Daniel/a a jeho přátele coby prominentní osoby v zemi svých otrokářů, ne však proto, že by zkompromitovali svou věrnost Bohu, ale naopak proto, že Boží požehnání je vyvýšilo.
To je ústřední motiv, protože dodává na důvěryhodnosti Danielovým proroctvím, zejména těm, která hovoří o prodlouženém utrpení Izraele. 


\Citehere 3 (\kern-2mm) {
Lidé nikdy nepáchají zlo tak důsledně a vášnivě,
jako když to dělají z náboženského přesvědčení.
\quotedby {Blaise Pascal}
}

Vize kapitol 7--12 obsahují predikce budoucích časů, během nichž pravdivost vyprávění nabude pro Boží lid na významu.
Ačkoliv Izraelité pod nadvládou Babylóňanů i Peršanů trpěli, nepostihl je žádný rozšířený a systematický útok na jejich víru. Ten nastal až s Antiochem IV. Epifanem, panovníkem nad Seleukovským impériem mezi roky 175--164 př. Kr, který usiloval vymýtit náboženství Židů a přinutit je přizpůsobit se řeckým náboženským praktikám. Mnoho Židů ho poslechlo, ale jiní se vzepřeli a trpěli protivenství. 
Jedním z hlavních důvodů pro sepsání knihy Daniel je připravit Boží lid na dobu Antiocha IV. Epifana a povzbudit k vytrvalosti ty, kteří budou žít v nadcházejících časech pronásledování.


Kniha také vzhlíží až za dobu Antiocha IV. Epifana ke Kristově příchodu, který jednou zničí všechna lidská impéria a nastolí své věčné království spravedlnosti a pokoje. Všechny tyto události mají \x/Danielov/a proroctví na zřeteli.
Kniha sloužila jako mocné povzbuzení pro Boží lid, trpící útiskem, a dodnes je pronásledovaným věřícím podnětnou inspirací. 

\nadpis{Kristus v \x/Daniel/ovi}

\x/Daniel/ova zaměřenost na obnovení Izraele po skončení vyhnanství obrací pozornost k Ježíšovi docela přímočaře.
Podobně jako i někteří jiní proroci \x/Daniel/ předpovídal Božímu lidu slavnou budoucnost, jejíž naplnění Nový Zákon 
spojuje s prvním a druhým příchodem Kristovým, stejně jako s celými dějinami církve.

Detaily naplnění \x/Danie/lových vizí sice obklopuje řada kontroverzí, avšak základní struktura \x/Daniel/ových vizí nenechává nikoho na pochybách, že naplněním prorokových nadějí je Kristus.
Nejzřetelněji je to vidět na způsobu, jakým se Ježíš označuje za \uv{Syna člověka} (např. <Mt 9:6>; <Mt 10:23>; <Mt 12:8>).
\x/Daniel/ používal tento pojem ve významu Bohem vyvýšeného davidovského krále, reprezentujícího Boha na zemi.
Ježíš, Mesiáš, je ultimátním davidovským Králem; jenom on naplňuje predikce o Synu člověka v \x/Daniel/ových vizích (viz \<"poznámky" 7:13>n a \<7:14>n; viz teologický článek 
<"Království Boží"  Mt 4>a). 

Kromě toho se \x/Daniel/ v 9. kapitole dozvěděl, že Jeremjášova predikce 70 let vyhnanství bude prodloužena 
na \uv{sedmdesát týdnů} let (\<9:24>), tedy asi 490 let.
Tato předpověď dochází počátku svého naplnění v Kristově prvním příchodu. Prodleva koresponduje se sérií čtyř cizích impérií, která budou Boží lid utlačovat (\<2:1-49>) a se skálou, která se stala \uv{horou velikou a naplnila celou zemi} (\<2:35>) a kterou \x/Daniel/ označuje jako \uv{království, jež nebude zničeno} (\<2:44>). 
To je království Kristovo, které bylo inaugurováno jeho prvním příchodem, dodnes pokračuje a roste, a svého dovršení dosáhne při Kristově slavném návratu (viz teologické články <"Království Boží" u Mt 4> a <"Plán věků" u Žd 7>.)

\x/Daniel/ předvídal i jiné, ještě konkrétnější události, které v Novém Zákoně znovu vstoupily do popředí.
Např. Ježíš  se odvolává na \x/Daniel/ovu predikci o \uv{otřesné ohavnosti} (viz \<pozn. 9:17>n; \<11:31>n; \<12:11>n),
která původně ukazovala na zneuctění chrámu řeckým Antiochem IV. Epifanem (viz Úvod: Záměr a zvláštnosti) coby předobraz zničení chrámu  římským generálem Titem v roce 70 po Kr. (viz <"poznámky k" Mt 24:15>n a <Mk 13:14>n).

Většina křesťanů spojuje tuto typologii s Antikristem, jehož duch již ve světě působí (viz <"pozn." 1Jn 2:18>n) a zjeví se v plnosti, zřejmě jako konkrétní osoba, v blízkosti Kristova návratu (viz <"pozn." 2Te 2:3>n).

%\endinput
%Chtělo by to možná něco jako \ww, ale aby bylo možné jich psát několik do stejného odstavce, byť s jinými frázemi. Nebudou se vyhledávat v textu, ale musejí přepínat mezi verzemi.
%Viz např. předposlení odstavec výše, začínající slovem Prodloužení: Slovní spojení v \uv{uvozovkách} by chtěla variovat podle překladů. 
%asi \vdef

%Pak za takovýmto Úvodem bude ještě muset následovat Osnova, tu zatím nemám, vydá možná na půl stránky.

\Outline

\begitems
\rightnote{Příběhy Daniela a jeho přátel ilustrují jednak jejich věrnost
           Bohu, jednak jeho nadřazenost nade všemi národy.}
* Vyprávění (\<1:1-6:28>)

  \begitems
  * Věrnost Daniela a jeho přátel (<1:1-21>)
  * \x/Nabuchodonozor/ův první sen (<2:1-49>)
  * Vysvobození z ohnivé pece (<3:1-30>)
  * \x/Nabuchodonozor/ův druhý sen (<4:1-37>)
  * Soud nad \x/Balsazar/em (<5:1-31>)
  * Vysvobození ze lví jámy (<6:1-28>)
  \enditems

\rightnote{Danielovy vize o budoucnosti Božího lidu, nahlížející až do časů
           dlouho po skončení vyhnanství.
          Bůh Danielovi zjevil, že čtyři velká království
          budou ovládat a pronásledovat Izrael. V době čtvrtého z nich Bůh nastolí své království, jemuž nebude konce.}
* Vize (<7:1-12:13>)

  \begitems
  * Vize  čtyř \x/šelem/ (<7:1-28>)
  * Vize \x/skopce/ a kozla (<8:1-27>).
  * Vize sedmdesáti týdnů (<9:1-27>)
  * Vize budoucnosti Božího lidu \nl (<10:1-12:13>)
    \begitems
    * Andělovo poselství k Danielovi \nl (<10:1-11:1>)
    * Od Daniela  po Antiocha IV. Epifana \nl (<11:21-12:3>)
    * Závěrečné sdělení Danielovi \nl (<12:4-13>)
    \enditems
  \enditems

\enditems
