%% Bible-TeX macros, see bibletex-doc.pdf

   \_doc
   Basic settings:
   \_cod

\load[vlna]
\load[mte]

\isfile{f-biblon.opm}\iftrue 
   \fontfam[biblon]
\else 
   \fontfam[lmfonts]
\fi

\chyph
\enablemte

\typosize[11/13]
\hyperlinks\Blue\Blue

\raggedbottom
\parindent=20pt

   \_doc
   Fonts:
   \_cod

\fontdef\bookfont{\setfontsize{at19.pt}\bf}
\fontdef\chapfont{\setfontsize{at13.pt}\bf}
\fontdef\markfont{\setfontsize{at7pt}\rm}

   \_doc
   <full-vref> is full reference to verse in the format
               <book-mark>/<chapter-num>:<verse-num>\nl
   `\alist:<full-vref>` expands to the list of replace action numbers applied 
   for given verse. The actions are referenced by its number. Each new action has newly
   allocated action-number.\nl
   `\action:<number>` expands to the replace action, we suppose that the
   corresponding verse in saved to `\tmpb`\nl
   `\newaction{<full-vref>}{<action-body>}` allocates new action.
   \_cod

\newcount\actionnum
\def\newaction#1#2{\incr\actionnum
   \unless\ifcsname alist:#1\endcsname \sxdef{alist:#1}{}\fi
   \sxdef{alist:#1}{\cs{alist:#1}\the\actionnum,}%
   \global\sdef{action:\the\actionnum}{#2}%
}

   \_doc
   `\replpre{<prefix>}{<text>}{<fail>}` replaces first occurence of <text>
   by `<prefix>{<text>}` in `\tmpb` macro. If the <text> is empty then
   `<prefix>{}` is inserted at the beginning of the `\tmpb`.\nl
   If <text> does not exists then <fail> is processed. The <fail>
   macro can use `\text` where <text> is saved.
   \_cod

\def\replpre#1#2#3{%
   \ifx^#2^\def\tmp{#1{}}\ea\ea\ea\def\ea\ea\ea\tmpb\ea\ea\ea{\ea\tmp\tmpb}%
   \else
      \def\replpredo##1#2##2\end{%
         \ifx^##2^\def\text{#2}#3% <fail>
         \else \replsave ##1#1{#2}##2\end \fi
      }%
      \def\replsave##1#2\end{\def\tmpb{##1}}%
      \ea\replpredo\tmpb#2\end
   \fi
}

   \_doc
   `\_wdef {<text>} <word-list> ;` binds <book-mark>/<text> to <word-list>. 
   If <text> (used in `\Note`) is bound to <word-list> then <text> is not
   searched directly in the given verse but words from <word-list> are used
   instead. First one is used.\nl
   The <word-list> is one or more pairs `{<word>}{<repl-word>}`. The
   `<word>` is used for searching in given verse but <repl-word> is actually
   printed in the note. You can use `{<word>}{}` which means that
   `<repl-word>` is the same as `<word>`.\nl
   Example:
   \begtt
   \CommentedBook{Da}
   \wdef {text} {wordA}{wordB} {wordC}{} ;
   \Note 1:13 {text} Note text.
   \endtt
   If given verse Da 1:13 includes wordA then it is replaced by `<prefix>{<wordA>}`
   and `{<wordB>}` is actualy printed in the note. Else: if the verse
   includes wordC then it is replaced by `<prefix>{WordC}` and wordC is
   actualy printed. Else: the warning is printed.\nl
   The `\wdef` creates binding described above which depends on current
   `\CommentedBook`, so you can use the same <text> in different books
   without any influence. If you apply `\wdef` to the same <text> in the
   same book secondly or more, then warning is printed and `\wdef` is
   ignored.
   \_cod

\def\wdef #1#2;{%
   \ifcsname wordlist:\the\CommentedBook/#1\endcsname
      \printwarn{\noexpand\wdef{#1} used secondly in book
                 "\the\CommentedBook", ignored}%
   \else
      \global\sdef{wordlist:\the\CommentedBook/#1}{#2}%
   \fi
}

   \_doc
   The <gen-vref> is generalized reference to the verse. It can be
   `<chapter-num>:<verse>` (the <book-mark> is appended from
   `\CommentedBook` token list) or `<chapter-num>:<verse-from>-<verse-to>`
   (only <verse-from> is used for generating <gen-vref>.\nl
   `\gentovref{<gen-vref>}` expands to <full-vref>. 
   \_cod

\newtoks\CommentedBook
\def\gentovref#1{\the\CommentedBook/\gentovrefA#1-\end}
\def\gentovrefA#1-#2\end{#1}

   \_doc
   The <word> given as a paremeter of the `\Note` macro (see bellow) is used as a word
   phrase which should be be searched in the given verse. This parameter <word>
   is transformed first by expansion of `\transformword{<word>}` to the <tword> variant and the
   <tword> is actually used for searching. The `\transformword{<word>}`
   expands to the variant of the <word> declared by `\wdef`. If not declared then
   it expands to the variant of the <word> declared by `\vdef`. If not declared then it
   expands to the <word> itself, i.e <tword> is equal to <word> in this case.
   \_cod

\def\transformword#1{%
   \ifcsname wdef:\the\CommentedBook/#1\endcsname \lastnamedcs
   \else \ifcsname vdef:#1\endcsname \lastnamedcs
   \else #1\fi\fi
}

   \_doc
   `\Note <gen-vref> <space> {<word>} <text> <empty-line>`
   transforms <word> to the <tword> (see above),
   saves <text> and activates replace-action of <tword> to
   `\doNote{<note-num>}{<tword>}` in given verse. 
   \nl
   `\Note` does following:
   \begitems
   * Allocates new <note-num>,
   * Transforms <gen-vref> to <full-vref> using `\gentovref`.
   * Transforms <word> to <tword> (to be searched and printed) by `\transformword`.
   * Reads <pword> (word to be printed in the note) if alternative syntax
     with `={<pword>}` is used. Else <pword> is equal to <tword>. 
   * Defines `\notetext:<note-num>` as <text>.
   * Defines `\noteref:<note-num>` as <full-vref>.
   * Defines `\notepre:<note-num>` as <gen-vref>.
   * Defines `\pword:<note-num>` as <pword>,
   * Does
   `\newaction{<full-vref>}{\replpre{\doNote{<note-num>}}{<tword>}{\notefail{<note-num>}}}`.
   \enditems
   The `\Note` macro has an alternative syntax 
   `\Note <gen-vref> <space> {<word>}={<pword>} <text> <empty-line>`.
   If <pword> is given then is is printed in the note instead <tword>.
   More preciselly: transformed <word> is used for searching
   (and it is kept in the verse unchanged) but <pword> is
   printed in the note.
   \_cod

\newcount\notenum
\outer\def\Note #1 #2{%
   \incr\notenum
   \sxdef{notepre:\the\notenum}{#1}%
   \edef\fullvref{\gentovref{#1}}%
   \edef\tword{\transformword{#2}}%
   \isnextchar={\NoteA}{\NoteA={}}%
}
\def\NoteA=#1 #2\par{%
   \global\sdef{notetext:\the\notenum}{#2}%
   \sxdef{noteref:\the\notenum}{\fullvref}%
   \ifx^#1^\sxdef{pword:\the\notenum}{\tword}\else \global\sdef{pword:\the\notenum}{#1}\fi
   \edef\tmp{%
      \noexpand\newaction{\fullvref}%
         {\noexpand\replpre{\noexpand\doNote{\the\notenum}}{\tword}{\noexpand\notefail{\the\notenum}}}}%
   \tmp
}
\def\notefail#1{%
   \printwarn{\csstring\\Note: \currverse: The text "\unexpanded\ea{\text}" not found}%
   \replpre{\doNote{#1}}{}{}% \Note is registered with the beginning of the verse
}
\def\printwarn#1{\wterm{WARNING (l.\the\inputlineno) #1}}

   \_doc
   When bible-text (from sword) is processed then book mark is saved to
   `\currbook` and each input line is separated to the
   <chapter-num>:<verse-num> and <verse-text>.
   The `\processverse <full-vref><space><verse-text>\end` is repeatedly processed.
   \_cod

\eoldef\processline#1{\processverse \currbook/#1\end}

   \_doc
   `\processverse <full-vref><space><verse-text>\end` does
   \begitems
   * defines `\currverse` as <full-vref>,
   * prepares `\currversenum`, `\currchapnum` from <full-vref>,
   * defines `\tmpb` as <verse-text>,
   * processes all actions from `\alist:<full-vref>`,
   * if `\currchapnum` changed, prints new chapter by `\printchap` 
   * prints verse from `\tmpb` using `\printverse`
   \enditems
   \_cod

\newcount\chapnum
\def\processverse #1 #2\end{%
   \edef\currverse{#1}%
   \preparechapverse #1
   \def\tmpb{#2}\def\tmpc{}%
   \ifcsname alist:#1\endcsname \ea\ea\ea\processactions \csname alist:#1\endcsname 0,\fi
   \ifnum\currchapnum=\chapnum \else \chapnum=\currchapnum\relax \printchap \fi
   \printverse
}
\def\processactions #1,{\ifnum #1=0
   \else \cs{action:#1}%
   \ea \processactions \fi
}
\def\preparechapverse #1/#2:#3 {\def\currchapnum{#2}\def\currversenum{#3}}

   \_doc
   `\printverse` prints verse from `\currversenum` and (possibly changed) `\tmpb`.
   It prints the single raised verse number first.\nl
   `\printchap` prints beginning od new chapter.
   `\printbeforefirst` is a macro which is executed just before first verse
   of the chapter, after all material from `\fmtpre` is executed. I.e after
   printing a chatper name (if declered by `\fmtpre`). 
   \_cod

\def\printverse{%
   \tmpc % material accumulated by \fmtpre
   \ifnum\currversenum=1 \printbeforefirst \fi
   \quitvmode \trymakedest{v:\currverse}%
   \raise5pt\hbox{\unless\ifnum\currversenum=1 \markfont\currversenum\fi}%
   \tmpb \space
}
\def\printchap{\bigskip}
%   \par\removelastskip
%   \medskip
%   {\chapfont\Red \the\chapnum}\par\nobreak\medskip
%}
\def\printbeforefirst{%
   \par\nobreak
   \vbox to0pt{\null\vskip-1ex
      \hbox to\parindent{\hss \chapfont\Red \the\chapnum\ \hss}\vss}\nobreak \vskip-2ex
   \noindent \hangindent=\parindent \hangafter=-2 \relax}

   \_doc
   `\fmtpre {<gen-vref>}{<what>}` adds <what> to `\tmpc`, i.e.\ at the
   beginning of the verse.\nl
   `\ftmadd {<gen-vref>}{<what>}` adds <what> to `\tmpb`, i.e.\ at the end
   of the verse.\nl
   `\fmtins {<gen-wref>}{<text>}{<what>}` inserts <what> after <text> in the
   verse. If <text> is not found th <what> is inserted like `\fmtpre` does it\nl
   All these commands allocate new action using `\newaction`. 
   \_cod

\let\FormatedBook=\CommentedBook
\def\fmtpre#1#2{\newaction{\gentovref{#1}}{\addto\tmpc{#2}}}
\def\fmtadd#1#2{\newaction{\gentovref{#1}}{\addto\tmpb{#2}}}
\def\fmtins#1#2#3{\newaction{\gentovref{#1}}{\replpre{\fmtafter{#3}}{#2}{\fmtfail{#3}}}}
\def\fmtafter#1#2{#2#1}
\def\fmtfail#1{\fmtwarn\addto\tmpc{#1}}
\def\fmtwarn{\printwarn{\string\fmtins: \currverse: The text "\unexpanded\ea{\text}" not found}}

   \_doc
   `\begcenter` starts the centering mode. It opens a group and does
   setting. User must use paired `\endcenter` in order to close this group.
   The `\centeringmode` status is checked by `\encenter` because curious error
   (about `#` character) should be occurr without this checking.
   \_cod

\newdimen\centermargin \centermargin=4em
\def\begcenter{\par \medskip
   \bgroup
   \def\centeringmode{y}
   \parindent=0pt
   \leftskip=\centermargin plus1fill
   \rightskip=\leftskip
}
\def\endcenter{\par
   \ifx\centeringmode\undefined
      \printwarn{\noexpand\endcenter ignored: no \noexpand\begcenter precedes}
   \else \egroup \medskip \fi}

%% Makra pro tvorbu linků specifikovaných jako <link>

   \_doc
   `\bref <link-spec>>` creates a ling given by <link-spec> and prints
   <link-spec>. See user manual for variants of the format of <link-spec>.
   We will set \code{\<} as active character with meaning `\bref`, so
   user can type \code{\<}<link-spec>\code{\>} for link specification.\nl
   Internal full link specification is <link-pre>:<book-mark>/<chapter-num>:<verse-num>.
   The macro `\bref` reads <link-spec> and creates internal full link
   specification saved in `\linkpe:\linkfspec` macros. Moreover, it creates
   macro `\linktext` wich expands to the text to be printed as active link.\nl
   \_cod

\def\bref {\futurelet\next\brefC}

   \_doc
   Macro `\brefC` scans `"` as a potential first character and
   saves <word> (from `"<word>"` specification) to `\linktext`.
   If there isn't `"` then `\linktext` is set as empty macro. 
   The next specification will be addet to the `\linktex` later.
   \_cod

\def\brefB {\futurelet\next\brefC}
\def\brefC {\ifx\next"\def\nextdo"##1"{\def\linktext{##1}\brefD}%
            \else \def\linktext{}\def\nextdo{\brefD}%
            \fi \nextdo
}

   \_doc
   If there si a space between `"<word>"` and the next `<link-spec>`, then
   it is saved as no-breakable space into `\linktext`. 
   \_cod

\def\brefD {\futurelet\next\brefE}
\def\brefE {\ea\ifx\space\next \addto\linktext{~}\def\nextdo{\afterassignment\brefF \let\next= }%
            \else \def\nextdo{\brefF}%
            \fi \nextdo
}

   \_doc
   Next <link-spec> is processed by macros `\brefF` ... `\brefJ`.
   \_cod

\def\brefF #1>{\brefG #1 >} 
\def\brefG #1 #2>{\ifx^#2^\edef\linkfspec{\currbook/}\afterfi{\brefH #1:>}%
                  \else \addto\linktext{#1~}\def\linkfspec{#1/}\afterfi{\brefGH#2:>}%
                  \fi
}
\def\brefGH #1 {\brefH #1}
\def\brefH #1:#2>{\ifx^#2^\ea\addto\ea\linkfspec\ea{\the\chapnum:}\afterfi{\brefI #1->}%
                  \else \addto\linktext{#1:}\addto\linkfspec{#1:}\afterfi{\brefHI #2->}%
                  \fi
}
\def\brefHI #1:{\brefI #1}

\def\brefI #1-#2>{\addto\linktext{#1}\addto\linkfspec{#1}%
                  \ifx^#2^\afterfi{\futurelet\next\brefK}%
                  \else \afterfi{\brefJ#2}\fi
}
\def\brefJ#1-{\addto\linktext{--#1}\futurelet\next\brefK}

\def\brefK{\def\linkpre{v}% default
   \ea\striptocomma\linkfspec,\end\linkfspec
   \ifx n\next \def\linkpre{n}\ea\brefKK\fi
   \brefL
}
\def\brefKK #1\brefL#2{\brefL}

\def\striptocomma #1,#2\end#3{\def#3{#1}}

   \_doc
   Macro `\brefL` creates the link `\linkpre:\linkfspec` with the text `\linktext`.
   \_cod

\def\brefL{\ensuredest \createlink}

   \_doc
   `\createlink` creates link only if it refeneces the place of printed book
   because we don't want to see many warnings about unreferenced links when
   we try to print only selected books. It creates link 
   `\linkpre:\linkfspec` with the text `\linktext`
   \_cod

\def\createlink{\ea\isprintedbook\linkfspec \iftrue
   \link[\linkpre:\linkfspec]{\Blue}{\linktext}%
   \else {\Blue\linktext}\fi
}
\def\isprintedbook #1/#2\iftrue{\ifcsname pbook!#1\endcsname}


\adef<{\bref}

   \_doc
   We don't create destinations for all verses, notes etc. but only for
   those which are referenced.
   Macro `\ensuredest` creates the item `\Xcreatedest` to .ref file
   and it is read in the second \TeX/ run.
   The `\trymakedest` macro is used ad the begining of each verse, note etc.
   Only referenced destinations are created.
   \_cod

\def\ensuredest{\openref \immediate\_wref\Xcreatedest{{\linkpre:\linkfspec}}}
\refdecl{
   \def\Xcreatedest#1{\sxdef{dest!#1}{}}
}
\def\trymakedest#1{\ifcsname dest!#1\endcsname \dest[#1]%
   \global \ea\let\csname dest!#1\endcsname \undefined \fi}


   \_doc
   The temporary macros are here. I plan to rewrite them.
   \_cod

%% Nasledujici makra jsou zatim provizorni a nedokumentovana.

\def\chaptit#1{\ifhmode \setbox0=\lastbox \par \nobreak\vskip-\baselineskip \fi
   \medskip{\chapfont\Red#1}\endgraf\nobreak\medskip}

\def\prevtmpb{}
\def\doNote#1#2{%
   \edef\tmpb{\cs{notepre:#1}}\replstring\tmpb{-}{--}%
   \fnote{%
      \ifx\prevtmpb\tmpb \else \tmpb \enskip \global\let\prevtmpb=\tmpb \fi
      \trymakedest{n:\currverse}%
      {\bf \expandafter \ifx \csname pword:#1\endcsname \empty \else \cs{pword:#1}. \fi}%
      \cs{notetext:#1}}{\Red#2}%
}
\def\_printfnotemark{}
\def\_textindent#1{\noindent}

\newcount \chapnum
\def\source#1{}
\def\BibleBook#1#2{\def\currbook{#2}%
   \bigskip {\bookfont #1}\par\nobreak\medskip \chapnum=0 }

\def\x/#1/{#1}

\def\dopsat{{\Red !!! DOPSAT !!! }}

\def\setvariant#1{}
\def\bibleinput#1 {\bgroup
   \catcode`##=13 \bgroup\lccode`~=`## \lowercase{\egroup\let~}=\processline 
   \input #1
   \egroup
}

\sxdef{pbook!Da}{} % zaradit do makra zpracovavajici seznam tistenych knih.

\endinput
\_endcode