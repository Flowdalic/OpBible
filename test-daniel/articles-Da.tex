\CommentedBook{Da}

\Article [6] % Kdo byl \x/Darius/ Médský?

%%\Cite A %Dáme to do někam textu místo Wericha, ať to můžu ukazovat
%%{Daniel 3:16\break
%%   Šadrach, Méšak a Abed-nego odpověděi králi:
%%   \uv{Nebúkadnesare, nám není třeba dávat ti \hbox{odpověď}.}
%%}
%{Druhořadých věcí nedosáhneme tím, že je prohlásíme za prvořadé. Druhořadých věcí dosáhneme pouze tehdy, budou-li na prvním místě věci první.
%\quotedby {C. S. Lewis}
%}

\Cite  A
{Ani v literatuře a umění nebude nikdo, kdo si zakládá na originalitě, nikdy originální: zatímco když se jenom snažíte říkat pravdu (a nezáleží vám na tom, kolikrát už byla řečena), stanete se v devíti případech z deseti originálními, ani si toho nevšimnete. 
\quotedby {C. S. Lewis}
}





%\swapCites

Poprvé je zmíněn v \<5:31>. Někteří (hlavně kritičtí, tj. liberální) teologové, zastánci pozdního (tzv. \uv{makabejského}) datování vzniku knihy \x/Daniel/ (podle nich kolem roku 165 př.Kr.),  tvrdí, že
(1) \x/Darius/ Médský 
\insertCite A\left \
nikdy neexistoval, protože v jiných starověkých dokumentech není zmíněn; (2) jméno \x/Darius/ použil neznámý makabejský autor, špatně obeznámený s perskou
historií, a zaměnil ho s legendárním \x/Dariem/ I. (255--484) Perským (nikoliv Médským); (3) autor se chybně domníval, že Babylón dobyla Médea, nikoliv Persie, a že pod vedením tohoto
legendárního \uv{\x/Daria/} Médové vládli světovému impériu několik let, než padlo do rukou Peršanům.

Díky tomu mohou advokáti makabejského (pozdního) data tvrdit, že čtyři království z \x/Nabuchodonozor/ova snu (<Da 2>) jsou (1) babylónské; (2) médské; (3) perské a (4) řecké, což
jim poskytuje výhodu omezení horizontu \x/Daniel/ových proroctví ne dále než do roku 165 př.Kr. (Pokud by kniha vznikla až v této době, všechna údajná \uv{proroctví}  by se dala vysvětlit zpětným pohledem na historické události {\it poté,\/} co nastaly. Problém s tradiční identifikací čtvrtého království coby Říma pro liberály spočívá v tom, že takový pohled předpokládá skutečné pravé prediktivní proroctví, což racionalistický vyšší kriticismus zásadně nepřipouští.)
Udržitelnost hypotézy makabejského data proto závisí na výše uvedeném vysvětlení \uv{\x/Daria/ Médského} (protože podle tohoto vysvětlení existuje médské království před perským).
Proto je tato postava velmi důležitá; její identifikace má závažné teologické důsledky.

Peršan \x/Darius/ I., syn Hystapesův, však nemůže být ztotožněn  s \x/Dariem/ Médským hned z několika důvodů:

\begitems \style n

* \x/Darius/ I. byl rodem Peršan, bratranec krále \x/Cýr/a; nebyl to v žádném případě Méd.
\insertCite A\right
* \x/Darius/ I. byl mladík kolem dvaceti let, když zavraždil podvodníka Gaumatu (který se vydával za \x/Cýr/ova syna Smerdise) v roce 522 př.Kr.  Nemohlo mu být 62 (\<5:31>). 
* \x/Darius/ I. nebyl králem Babylóna před \x/Cýr/em, jak tvrdí liberální teorie. Samostatným vládcem se stal až sedm let po \x/Cýr/ově smrti (srv. <Ezd 4:5>).
* Taková zmatenost ohledně národnosti a časové posloupnosti \x/Daria/ a \x/Cýr/a byla v helenistickém světě druhého století př.Kr. absolutně nemyslitelná.
Studenti museli číst Xenofóna,  Hérodota a další řecké historiky pátého a čtvrtého století př.Kr. Od Xenofóna a Hérodota máme informace o \x/Cýr/ovi a \x/Dariov/i.
Jakýkoliv řecky píšící autor, který by umístil \x/Daria/ před \x/Cýr/a, by ukončil svou spisovatelskou kariéru výsměchem veřejnosti; nikdo by ho už nikdy nebral vážně.  

\enditems

\x/Darius/ Perský (<Ezd 4:5>) a \x/Darius/ Médský (<Da 5:31>) tedy nemají spolu nic do činění; zmatek je pouze na straně zastánců teorie pozdního data, nikoliv na straně autora knihy \x/Daniel/.

Nicméně je pravda, že archeologie dosud neobjevila žádnou zmínku o \uv{\x/Dariov/i Médském} z doby, kdy žil, mimo Bibli.
(Až do devatenáctého století totéž platilo o \x/Balsazar/ovi, místokráli, zastupujícím svého otce Nabonida. Kritičtí teologové, zastávající makabejské datování, tvrdili, že
\x/Balsazar/ je další fiktivní postava v \x/Daniel/ovi, dokud nebyly objeveny babylónské tabulky z jeho doby,  potvrzující, že \x/Balsazar/ sloužil jako mladší král v posledních letech vlády svého otce Nabonida. Srv. <"pozn." 5:1>n).

Přesto \x/Daria/ Médského dokážeme identifikovat.
V knize \x/Daniel/ je několik náznaků, že \x/Darius/ nebyl svrchovaným králem, ale že byl dočasně dosazen na trůn nějakou vyšší autoritou.
Ve verši \<9:1> čteme, že \uv{byl učiněn  králem}. Je zde použit pasivní kořen {\it hofal\/} u slovesného tvaru \uv{homlak} (\Homlak) namísto běžného \uv{malak} (\Malak\ \uv{stal se králem}), používaného v kontextu získání trůnu dobytím nebo dědictvím (např. <1Sa 13:1>).
Podobně ve verši <5:31> čteme, že \x/Darius/ \uv{\x/ujal království/} (\uv{qabbel} \Qabbel), jako kdyby mu bylo svěřeno vyšší autoritou.

Samotné jméno \x/Darius/ (staropersky {\it Da-ri-ya-(h)u-(ú-)iš\/} \Dariahuuish, hebr. \Dariawush) je zřejmě příbuzné s {\it dara,\/} které se v avestánštině (mrtvý severovýchodní staroíránský jazyk) objevuje jako
výraz pro \uv{krále}. Podobně jako označení {\it augustus\/} mezi Římany, mohlo být i přízvisko {\it dārayawush\/} (\uv{královský}) zvláštním čestným titulem, který
mohl sloužit i jako vlastní jméno, podobně jako české příjmení \uv{Král}.

Zdá se tedy, že záhy po porážce Babylóna médo-perskými vojsky si \x/Cýr/ovu osobní přítomnost vynutila jiná fronta jeho rozpínajícího se impéria. Jevilo se mu tedy jako účelné
svěřit království Gubarovi-\x/Dariov/i i s titulem Král Babylóna, aby panoval přibližně rok, než se Kýros  osobně vrátí ke své korunovační slavnosti v Mardukově chrámu.
Po tomto roce vlády v roli místokrále zůstal \x/Darius/ správcem Babylóna, ale koruna byla odevzdána jeho nadřízenému vládci \x/Cýr/ovi (který ji později předal svém nejstaršímu synovi Kambýsésovi, srv. <"pozn." 11:2>n, při korunovaci králem Babylóna).

Tento scénář podporuje text knihy tím, že \x/Daniel/ nikde neuvádí žádný pozdější rok \x/Dariov/y vlády než \uv{první} (<9:1>), což indikuje její velmi krátké trvání.
I kdyby to mělo znamenat, že jeden rok patřila vláda Médům (víme, že tomu tak  nebylo; patřila  Peršanu \x/Cýr/ovi), jednoletá říše by sotva mohla uhájit svou legitimní
pozici coby království číslo dvě v řadě impérií výrazně trvanlivějších: babylónské vydrželo 73 let, perské 208 let, řecké by v roce 165 př.Kr. mělo za sebou 167 své existence. 

Kromě toho slovní hříčka \x/Daniel/ovy interpretace nápisu na zdi v <5:28>, která spojuje dva významy stejného kořene P--R--S (\PRS): {\it p$^e$r\char238 
sat\/} (\Persat\ \uv{rozděleno}) a
{\it pārās} (\LaMedeUParas\ \uv{dáno Médům a Peršanům}), zároveň ujišťuje, že autor knihy psal v přesvědčení, že království číslo jedna (babylónské) přechází pod
nadvládu Peršanů již spojených s Médy a tím se stává královstvím číslo dvě. Kniha \x/Daniel/ neponechává žádný prostor pro kritickou spekulaci o dřívějším médském království, které
údajně mohl mít autor knihy na mysli.

Čtvrté království je pak Řím, jediné, které dokázalo pokořit Řecko (<2:40>), a během jehož existence vzniklo Božím zásahem království věčné, jemuž nebude konce (<2:44>) -- církev.
(Srv. graf \ref[danielovyvize] na str. \pgref[danielovyvize]).

\endinput


