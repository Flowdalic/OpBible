\kniha{Matthew}
\zkratka{Matt}
<1:1 Generation; a record of the persons from whom, as a man, Jesus Christ descended. Records like this, and that in the third chapter of Luke, were carefully kept by the Jews, and showed that Jesus Christ was of the seed of Abraham, of the tribe of Judah, and of the family of David, according to the predictions of the prophets and the promises of God in the Old Testament; and thus they exhibit evidence that he is the true Messiah, the Saviour of men.>
<1:8 Jordan begat Ozias; between Joram and Ozias--the Uzziah of the Hebrew--three kings are omitted, namesly, Ahaziah, Joash, and Amaziah. See note of Mt 1:17.>
<1:11 Josias begat Jechonias; here Eliakim, son of Josiah and father of Jechonias, is omitted. See note on Mt 1:17.>
<1:16 Christ; the meaning of the word Christ is, Anointed. Persons who were set apart and consecrated to a public office under the Old Testament were, in many cases, anointed with oil, in token of their being endued by the Holy Spirit to fit them for their duties. So Christ having been appointed and consecrated of the Father to the office of Redeemer, is called in Hebrew, the original language of the Old Testament, the Messiah; in Greek, the original language of the New Testament, the Christ; and in English, the Anointed; all meaning the same thing: one set apart of God to the office of Redeemer, and divinely qualified for its fulfilment.>
<1:17 Fourteen generations; the equality of the numbers--fourteen generations thrice repeated--is made out by certain omissions. See notes on Mt 1:8,11. This squaring of numbers in the construction of genealogical tables seems to have been a common expedient for the assistance of the memory. In determining, however, the names to be omitted, the framers of these tables were doubtless guided by particular reasons. Thus some suppose that the three kings, Ahaziah, Joash, and Amaziah, were left out of the Jewish tables on account of their connection, through Athaliah, with the accursed house of Ahab.>
<1:18 On this wise; after this manner. Espoused to Joseph; engaged to be married to him. Before;before they were actually married. Of the Holy Ghost; the conception of Jesus Christ was miraculous, and effected by the power of God, according to his declaration, "A body hast thou prepared me."Heb 10:5. Though the fulfilment of the promises of God may be long delayed, in due time they will be accomplished. Implicit faith in God's word, and patient waiting for its fulfilment, are wise; for though heaven and earth pass away, his word will not pass away.>
<1:19 A public example; not willing to have her punished according to the law. De 22:21 Put her away privily, by writing a bill of divorcement, according to De 24:1.>
<1:20 Son; descendant of David. Kindness, conscientiousness, and a disposition to do right, with a calm, attentive consideration of the whole subject, in a case of difficulty, are a good preparation for learning the will of God concerning it.>
<1:21 Jesus; Jesus is the same name as Joshua, or, as it is written by some of the later Hebrew writers, Jeshua. It signifies the salvation of Jehovah. Save; deliver from the pollution, power, guilt, condemnation, and punishment of sin.>
<1:22 Fulfilled; the birth of Jesus was in fulfilment of a promise of God, by the prophet Isaiah, uttered more than seven hundred years before. Isa 7:14. God in the Old Testament spoke by his prophets, and what they then uttered was the testimony of God. So when they wrote what he directed them, it was the testimony of God; hence, their writings are called by the Holy Ghost, "the oracles of God." Ro 3:2 1Pe 1:2>
<1:23 Emmanuel; a proper title for Jesus Christ, because he was God as well as man, and dwelt among the sons of men.>
<1:24 To follow the directions of God is safe, useful, and blessed.>
<2:1 Of Judea; to distinguish it from another Bethlehem, in Galilee. Herod; this Herod was an Edomite. He had been proselyted to the Jewish religion, but was a very deceitful, wicked, and cruel man. Friends of the Saviour are sometimes found where we should least expect them.>
<2:2 Born King of the Jews; there was at this time, as we learn from heathen writers, a general expectation throughout the east, that one would be born in Judea who should possess universal dominion. Among the Gentiles, this expectation was probably founded on some imperfect acquaintance with the prophecies of the Old Testament. In the east; in their own country, which lay east of Judea. The nature of this star we have no means of determining. It is sufficient for us to know that God in some way made known to these Magi its meaning, and influenced them to take this journey, to find and pay their homage to the new-born King.>
<2:3 Troubled; Herod feared a rival, and his trouble caused the people to be troubled.>
<2:4 Chief priests; the principal ministers of religion among the Jews at that time. Scribes; writers and expounders of the divine law.>
<2:5 The prophet; Mic 5:2. Matthew does not quote the exact words of the prophet Micah, but the sense.>
<2:8 Deception and hypocrisy are often united with great cruelty, and end in misery.>
<2:9 Went before them; in their journey from Jerusalem to Bethlehem. Stood over where the young child was; so that they were guided by it to the exact spot.>
<2:13 Those who conscientiously follow the light which they have, will receive, in the use of proper means, all the light they need.>
<2:15 Out of Egypt have I called my Son; originally spoken by the prophet Ho 11:1; of the Israelitish nation as God's Son. But it was the appointment of God that in this, as in so many other things, the history of Christ's body the church should foreshadow his own personal history.>
<2:16 Had diligently inquired; Herod, supposing that the time of the appearance of the star, which he had accurately learned from the Magi, must agree with the age of Jesus, determined to destroy all the children in Bethlehem whose age could possibly come within that of the young child whose life he sought. The efforts of men to prevent the fulfilment of the word of God are unavailing.>
<2:17 Fulfilled; the scene in Judea was like that depicted by the prophet, Jer 31:15, so that his words most fitly describe it.>
<2:18 In Rama; north of Jerusalem, in the tribe of Benjamin, of which tribe Rachel was the mother. There is probably an allusion to Rachel's sepulchre, near to Bethlehem, where she is poetically represented as weeping for her slain children. Children as well as parents are exposed to sudden and unexpected death; therefore no present duty should be put off to a future time.>
<2:23 Nazareth; a place very much despised. Nazarene; one exceedingly despised, as the prophets foretold that Jesus Christ would be. Isa 53:2,3. The fulfilment of prophecy in the person of Christ proves him to be the true Messiah.>
<3:1 Baptist; the baptizer, a title given to John because he baptized. The wilderness of Judea; bordering on the Jordan and the Dead sea. It was a rough and thinly settled region, occupied chiefly as a place for pasturage.>
<3:2 Repent; repentance is a change of mind with regard to sin, especially as committed against God, which leads a person to hate, confess, and forsake it. Kingdom of heaven; the Messiah's reign as predicted by the prophets, or the sway of Christ's gospel and dispensation over the hearts, lives, and destinies of men, both in this world and in the next. This kingdom is spoken of in the Scriptures variously, in reference to its several aspects: first, in this world, as affecting the individual disciple in whose heart it is set up, as affecting the churches whom it gathers, and as influencing human society generally, even when not brought into the Christian church; and next, as extending from this world, through the judgment day, when it will be universally acknowledged, into the heavenly world, where it will reach its crowning glory. John the Baptist was its herald. Christ, after his resurrection and just before his ascension, declared his induction into it. Mt 28:18. The millennium and the judgment are stages in its continuous progress; and the consummation of the mediatorial kingdom is described. 1Co 15:24,28. Some texts in which the phrase is used refer mainly to one stage, and others to another, of its onward course. Men must hate and forsake their sins in order to be prepared for the kingdom of God. Pr 28:13.>
<3:3 Of the Lord; in the original it is, "Prepare ye the way of Jehovah." Isa 40:3. Christ was coming: "Make straight in the desert a highway for our God." Whenever the word Lord in the Old Testament is printed in capitals in our common English Bible, it is Jehovah in the original Hebrew; and the application by the Holy Spirit of what is said of Jehovah in the Old Testament to Jesus Christ in the New, is evidence that he is God. Mal 3:1; Joh 1:1>
<3:5 All Judea; people from all parts of the country.>
<3:6 Confessing their sins; the baptism of John was adapted to impress the minds of the people with a conviction of their pollution by sin, and of the necessity, through repentance, of spiritual cleansing by the Holy Ghost, in order to a right reception of the coming Saviour.>
<3:7 Pharisees; a sect among the Jews who were very strict in their outward forms of religion, but were inwardly corrupt, proud, and hypocritical. Sadducees; they denied a resurrection and the existence of angels and spirits, and generally were sceptical, and loose in their habits. Vipers; men who were malignant and bitter in their opposition to the character and will of Christ.>
<3:8 Fruits; show the reality of your repentance by forsaking your sins and obeying the commands of God. In order that repentance may be shown to be sincere, it must produce good works.>
<3:9 Think not; depend not on the piety of your ancestors, but become pious yourselves. Piety is not hereditary, and none can safely depend on the goodness of others; but in order to be saved, each one must become pious himself.>
<3:10 Hewn down; those who continue to neglect known duty will be destroyed.>
<3:11 He; Jesus Christ. Not worthy; though among all who were born of women none were greater in condition and honor than John. Mt 11:11, yet so much greater was Jesus Christ, even in his deepest humiliation, that John was not worthy to untie, or carry his shoe. Holy Ghost; by his Spirit he will purify all who believe in him, as gold is purified by the fire. The greatest and most honorable among men are so much less honorable than the Lord Jesus Christ, that they are not worthy to perform for him the most lowly service.>
<3:12 Wheat; the good. Chaff; the bad. He will make an endless separation between the righteous and the wicked. Mt 25:46. A knowledge of this should lead all to break off their sins by righteousness, and their iniquities by turning unto the Lord.>
<3:14 I have need to be baptized of thee; John, being a sinner, needed that spiritual renovation, the necessity and practicability of which were taught by baptism; but Jesus Christ being perfectly holy, did not need it. John therefore did not know why he should come to him to be baptized. But Christ showed him that under the circumstances in which they were placed, it was proper. Then John baptized him.>
<3:15 To fulfil all righteousness; all the requirements of God. Since Christ had taken upon himself the nature of sinful men, and put himself in their stead, it was proper that he should submit himself to every ordinance of God's appointment.>
<3:16 Lighting upon him; in token of his being endowed with the Holy Spirit for his work. Compare Joh 3:34. vs 16, 17. At the opening of the Saviour's ministry we have a manifestation of the Trinity: the Father, the Son, and the Holy Ghost, all cooperating in the great work of man's salvation.>
<3:17 A voice; the voice of God the father, acknowledging Christ as his beloved Son, and expressing his approbation of his character, office, and work.>
<4:1 The Spirit; the Holy Spirit. To be tempted of the devil; as were our first parents in Eden, and as are all their children. Christ must qualify himself for his office of Redeemer by successfully withstanding that temptation under which Adam and his children fell. See note on chap. Mt 3:15. At the same time he gave an example of the way to resist temptation, to baffle the tempter, and to overcome when tempted. God often leads his servants into great trials preparatory to the discharge of great and momentous duties.>
<4:3 Tempter; Satan, the adversary of God and man, who solicits to evil, and suggests motives to induce men to commit it.>
<4:4 It is written; De 8:3. By every word that proceedeth out of the mouth of God; every appointment of God for this purpose. Matthew does not here quote the exact words, but the sense.>
<4:5 Those who wrongly quote the Bible, and thus pervert its meaning, imitate the devil.>
<4:6 If thou be the Son of God, cast thyself down; Satan would have Jesus tempt God by a needless exposure of his life. It is written; Ps 91:11,12. The phrase, "in all thy ways," which is in the text quoted, meaning in the path of duty, Satan omitted, as if God would preserve a person from harm when out of the path of duty. This was a gross perversion of Scripture.>
<4:7 It is written; De 6:16. Thou shalt not tempt the Lord; try his power, truth, and faithfulness in opposition to his revealed will.>
<4:10 It is written; De 6:13.>
<4:11 If we steadfastly resist his temptations by refusing to comply with them, and follow the directions of Scripture, the tempter will flee from us, Jas 4:7, and we shall secure the assistance of good angels, who are sent forth to minister to them who are heirs of salvation. Heb 1:14.>
<4:12 Cast into prison; Lu 3:20. Galilee; the northen part of Palestine.>
<4:13 Capernaum; a town on the north-west shore of the sea of Galilee.>
<4:14 Fulfilled; Isa 9:1,2. Esaias in Greek is the same as Isaiah in Hebrew.>
<4:15 Galilee of the Gentiles; Galilee bordering on the gentile nations, who seem also to have been more or less intermixed with its inhabitants.>
<4:16 Saw great light; the light of Christ's presence and teaching. Compare Joh 8:12. Men who are without the gospel are in great darkness, but the reception of it will give them great light.>
<4:18 Sea of Galilee; called also the sea of Tiberias and the lake of Gennesareth; about thirteen miles long, and from six to nine miles wide: through it runs the Jordan. Those who are diligent in appropriate business are preparing for increased usefulness. From them Christ often selects his ministers; and he can so influence them, that they will forsake all and follow him.>
<4:19 Fishers of men; the means of taking them out of the kingdom of Satan, and bringing them into the kingdom of Christ.>
<4:23 Synagogues; the Jewish places of public worship. No diseases of body or soul are so complicated or stubborn that Jesus cannot heal them. All the diseased should therefore apply to him, that of his fulness they may receive according to their wants.>
<4:24 Syria; a country north and east of Palestine. Possessed with devils; devils at that time were permitted to have special influence over some men; and this gave Jesus Christ opportunity to show his controlling power over them, and his mercy in expelling them.>
<5:3 Poor in spirit; the humble, who feel their dependence on God in all things, temporal and spiritual, and look to him for the supply of every want; more especially those who feel their need, as sinners, of spiritual blessings, and look to Jesus Christ to grant them. Isa 66:2. Kingdom of heaven; the blessings of Messiah's reign in this world and the next. Chap Mt 3:2. True happiness does not consist in external condition, but in the state of the mind.>
<5:4 They that mourn; over their spiritual wants, and over sin as the guilty cause of them; who long for spiritual blessings, and come to Jesus Christ for them, according to his directions. Re 3:18. This beatitude includes also all the mourning to which God's children are subjected by the chastening through which God prepared them for the everlasting joy of heaven. Compare Heb 12:5-12.>
<5:5 The meek; those who are gentle and forgiving, submissive and teachable, patient under injuries, disposed not to render evil for evil, but to overcome evil with good. Inherit the earth; receive and enjoy every earthly and spiritual blessing that is for their best good here, and reign with Christ for ever hereafter. The sinful and lost condition of men need not hinder them from being truly and for ever blessed.>
<5:6 Hunger and thirst after righteousness; ardently desire to be and do right because it is right; trusting not in their own righteousness, but in the righteousness of Christ, which by the apostle is called the righteousness of God by faith of Jesus Christ which is unto all and upon all them that believe. Ro 3:22. Be filled; receive what they desire, and be satisfied. Ps 17:15.>
<5:7 The merciful; those who feel for the sufferings of others, and are disposed to relieve them. Mercy; from God. Compare Chap Mt 25:34-45. God requires us to exercise the compassion towards others which we need to have exercised towards us.>
<5:8 Pure in heart; freed from the dominion and pollution of sin. See God; have right views of him, and enjoy his presence here and hereafter.>
<5:9 Peacemakers; those who desire and seek to have all men at peace with God, with their own consciences, and with one another. Children of God; those who imitate him, and whom he will make heirs according to his promise. Ro 8:17.>
<5:10 Righteousness' sake; on account of their being and doing right. Great opposition to men is no certain evidence that they are wrong.>
<5:11 Falsely; when the evil which is said of you is false. For my sake; on account of your attachment and likeness to me.>
<5:12 So persecuted; Heb 11:35-38.>
<5:13 Salt of the earth; means of its preservation, by your holy doctrine, prayers, and example. Lost his savor; become worthless. Some think there is here an allusion to the fact that the salt in that country was mixed with earthy substances, which remained after it had lost its saltness, and were thrown like gravel upon the walks, and trodden down.>
<5:14 Light; that which shows things as they are, and gives to men right views of them.>
<5:16 Shine; let the goodness of your principles be seen in your conduct, that men may be led to honor God, the author of all good. Consistent Christian example is a means of leading men to honor God, and of greatly promoting their highest good.>
<5:17 Destroy the law; set aside either the principles or the moral precepts of the Old Testament. To fulfil; rightly to explain the nature and perfectly to enforce the precepts of the moral law, as well as perfectly to obey them in his own person, bear the curse which was prefigured in the ceremonial law, and thus fulfil the predictions of the prophets concerning the Messiah. Christ came not to make void the moral law as a rule of action, but to establish it, and give it practical efficacy over the hearts and lives of men, by leading them to love and obey it.>
<5:18 One jot; no part of the moral law or of the obligations to obey it shall be done away; nor shall any part of the ceremonial law, till its end is accomplished.>
<5:19 Least commandments; least, as compared with others. No precept of God's law may be set aside on the ground of its comparative unimportance; for the least disobedience to any command of God is highly offensive to him, while obedience in all things is his delight. The least; of the least repute as a teacher, because both by his example and his doctring he dishonors God's law. Great; worthy of honor as a teacher, because he honors the law by obeying it and teaching others to obey it.>
<5:20 Except your righteousness shall exceed--scribes and Pharisees; their righteousness was selfish, and consisted in externals; while the righteousness which God required is internal as well as external, and consists in conformity of heart and life to his revealed will.>
<5:21 By them of old time; rather, as the margin, "to them of old time;" and so below, verses Mt 5:27,33. The judgment; here the sentence of death from the lower court established by Moses in all the cities of Israel. De 16:18.>
<5:22 Angry with his brother; in his heart, to which God looks. Without a cause; not merely without an occasion, but rather, in an unreasonable degree, or with any mixture of malice. The judgment; the judgment of God. The Saviour's meaning is this: by the law of Moses literal murder is punished with death by common court; but in my kingdom anger in the heart will be regarded and treated as murder. Raca; vain fellow; blockhead. The Saviour puts a case where anger vents itself in railing. The council; the Sanhedrim, which was the highest Jewish court; but here it seems to represent the court of Christ, who will treat all railing accusations of one brother against another as offences of the gravest kind. Fool; vile wretch; the highest form of reproach in the mouth of a Jew. A disposition rightly to treat men is essential to acceptance with God.>
<5:23 Gift; religious offering. Altar; place where the offering was made. Aught; any cause of complaint.>
<5:25 Adversary; thy fellow-man who has just claims against thee. But the precept also looks beyond all human adversaries to God, with whom, under an example taken from earthly matters, it warns us to be reconciled while we are yet on the way to his judgment-seat. Opportunity to perform present duty should not be neglected, lest it be for ever lost.>
<5:28 In his heart; the laws of God extend to the thoughts; and men may violate them in their hearts without manifesting their feelings in outward conduct.>
<5:29 Offend thee; cause thee to sin. Profitable for thee; it is better to put away the causes of sin than to suffer its consequences. The avoidance of sin by self-denial, and if need by, by great sacrifices, will in the end be great gain.>
<5:31 Writing of divorcement; a certificate that their marriage relation was dissolved by his own act. See De 24:1.>
<5:32 Causeth her; exposeth her to commit adultery; because, according to the law of Christ's kingdom, her marriage to another man will be regarded as adultery.>
<5:33 Forswear thyself; commit perjury, or swear to that which is false. Perform unto the Lord; the Pharisees taught that religious oaths in which God's name was used were binding and should be filled, while they were less scrupulous about oaths by created things, and in common conversation. But our Lord taught that oaths of the latter kind proceed from evil, and should never be taken.>
<5:34 The practice of swearing in common conversation, or of swearing to a falsehood, shows great wickedness of heart.>
<5:36 Canst not make one hair white or black; thy head is a creature of God, over which thou hast no control; so that in swearing by it, thou swearest by him that made it and has it in his power.>
<5:37 Communication; conversation and discourse. Yea--nay; simple declarations, without profaneness of any kind.>
<5:39 Resist not evil; by rendering like for like. It is the spirit of kindness and forgiveness towards those who injure us which our Lord here inculcates. The forgiveness of injuries, and not the avenging of them, is an exhibition of true greatness and goodness.>
<5:40 Coat--cloak; the coat among the Jews was an inner garment, called a tunic, extending from the neck to the knee. Over this was a cloak or mantle, which was a large, loose garment, and when they travelled was girt tight round the body with a girdle Hence, "to gird up one's loins" implied readiness for labor or a journey. The girdle or sash answered also the purpose of purse for money. Let him have thy cloak; suffer losses, so far as duty will permit, rather than contend about them.>
<5:41 Compel thee to go a mile; the original word here rendered compel, denotes a compulsion by the public authorities and for public service. When thus called upon by rightful authority to travel or do public service, be ready to go farther or do even more than is required, rather than resist the government.>
<5:42 Give--turn not thou away; when the person who asks or would borrow is needy accommodate him, if consistently with duty you can do it.>
<5:44 Love your enemies; not their character or their conduct, but their souls. Pray for them and seek their good. Love to enemies, and a disposition to do them the greatest good which duty will permit, likens men to God.>
<5:46 If ye love them; them only. Publicans; tax-gatherers, who were considered as very wicked, and were often cruel and oppressive.>
<5:48 As your Father; imitate him in all his imitable perfections.>
<6:1 To be seen of them; to gain their applause. The character of external actions is determined by the feelings and motives.>
<6:2 They have their reward; they have it all in the applause of men, and receive no reward from God.>
<6:3 Thy left hand; let your good deeds be done without ostentation, and without seeking human praise.>
<6:6 Prayest; as an individual. Closet; a private room or retired place. Every one is bound to pray in secret. "Thou" enter into "thy" closet, and pray to "thy" Father.>
<6:7 Vain repetitions; words without meaning, or often repeated without corresponding thoughts and feelings. Heathen; persons not Jews, nor enlightened as to the character and will of God. Much speaking; many words, or words often and thoughtlessly repeated. 1Ki 18:26.>
<6:9 After this manner; this model, as to spirit, simplicity, and comprehensiveness. Our Father; Creator, Preserver, Guardian, and Friend. Hallowed by thy name; let all the manifestations of thyself be treated with reverence and love. Men are bound to unite with others in prayer, and when praying alone, to remember and pray for them "Pray ye," and say, "Our Father," a form suited to a number of persons. Acceptable prayer is the offering up of our desires for things agreeable to the will of God, in the name of Jesus Christ, with confession of sins, and thankful acknowledgment of his mercies.>
<6:10 Thy kingdom come; reign thou in all hearts, and lead them to do thy will on earth as it is done in heaven. Every person is bound to desire and daily to pray that God should reign in and over him and all people, as he reigns in heaven.>
<6:12 Debts; sins. Debtors; those who have trepassed against us.>
<6:13 Lead us not into temptation; keep us from being tempted, or if tempted, deliver us from the temptation, and from all evil. Thine is the kingdom; the reign, for the coming of which we pray, is thine; and the glory of its accomplishment will be thine for ever. Amen; so be it.>
<6:15 An unforgiving temper, if continued, will shut a man out of heaven, and shut him up in hell.>
<6:16 Disfigure their faces; by leaving their face unwashed, and their hair and beard undressed. In religious duties, all should be especially careful to avoid ostentation, and the seeking of the praises of men.>
<6:17 Anoint thy head; that is, dress and appear as usual.>
<6:19 Treasurers; those things which men most love, and which they regard as their chief good.>
<6:22 Single; healthy and clear, to discern objects aright.>
<6:23 Evil; diseased, and so not seeing things as they are. If therefore the light that is in thee be darkness; the Saviour now applies to the human mind the figure of the eye which he has just used. If the eye of thy soul be diseased, so that earthly treasurers appear to it better than heavenly, "how great is that darkness!">
<6:24 Serve; yield to or regard supremely two opposite objects, as are God and this world. Mammon; wealth, all earthly possessions. What a man regards supremely is his treasure, or his God. If it be any created thing, he has another god before Jehovah, an is, in this sense, an idolater.>
<6:25 Take no thought; no anxious thought, as the original word implies. More; more valuable. Meat; any kind of food for the support of the body. The argument is, that he who has given the greater gift, will not withhold the less. Anxiety about future support and comfort in this world is needless, hurtful, and wicked; for present obedience to God will insure all needed good.>
<6:26 Better; more valuable.>
<6:27 Unto his stature; better, to his age; that is, by all his anxiety prolong his life a moment beyond his appointed time. For measure, as applied to time, compare Ps 39:4.>
<6:30 Cast into the oven; cut down for fuel, and burnt. Little faith; little confidence in God.>
<6:32 Gentiles; those who know not God. Knoweth; he is acquainted with your wants, and in the proper use of means, without your anxiety, he will supply them.>
<6:33 See ye first; seek first an interest in the blessings of Christ's righteousness and reign. Chap Mt 3:2. All these things; all needed good he will bestow.>
<6:34 Shall take thought; the future will bring its supply.>
<7:1 Judge not; rashly, censoriously, or unjustly, the character or conduct of others. Harsh judgments will provoke retaliation.>
<7:2 Be judged--measured; you may expect to be treated as you treat others. Lu 6:37.>
<7:3 Men who are exceedingly blind to their own faults, are often exceedingly quicksighted to the faults of others.>
<7:5 Those who labor most successfully in advancing their own spiritual welfare, are the best fitted to be useful to others.>
<7:6 That which is holy; the holy flesh of the sacrifices. Trample them; as things to them valueless. Turn--and rend you; turn from the pearls in rage to attack the given, because he has offered them what they cannot eat. Dogs and swine represent selfish, quarrelsome, rapacious, and sensual men, whom it is often best to leave to themselves, lest our indiscreet labors be not only thrown away as regards them, but turn to our own injury. Scorners and scoffers should sometimes be let alone, lest, on being reproved, they become more injurious than they otherwise would be, to themselves and to others. Pr 9:7,8.>
<7:7 Ask; in every thing by prayer and supplication, with thanksgiving, let your requests be made known unto God. Seek; continue to ask of God the blessings which you need. Knock; at the door of his mercy and grace, with sincerity and earnestness, in the way of his appointment, and you shall be admitted to communion with him: in his light you will see light, and of his fulness receive according to all your wants. Men, in order to judge and act rightly with regard to their duty to themselves and their fellow-men, need wisdom and strength from above: they should therefore habitually ask them of God; and those who do this in dependence on Jesus Christ, may expect, for his sake, to receive them.>
<7:8 Every one; all who rightly ask, receive either what they ask or something better in its place.>
<7:11 Good gifts; things which are needed and truly beneficial. The readiness of a kind, affectionate parent to give necessary food to a famishing child, is but a faint emblem of the readiness of God to give all needed good to those who rightly ask him.>
<7:12 So; do to others as, under like circumstances, you ought to wish others to do to you. This is the law and the prophets; what is required in the Old Testament.>
<7:13 Strait gate; strait here means narrow and difficult, and represents the difficulty of entering on a religious life, or beginning heartily to obey God. Wide; easy to enter, requiring one only to follow his own depraved inclinations. The difficulties which stand in the way of beginning from the heart to obey God, need not and ought not to hinder any from doing it.>
<7:14 Few; that find or go in the way of life. This truth is contrary to what many teach. Therefore,>
<7:15 Beware; avoid false teachers. Sheep's clothing; appearing in the character of true teachers. Wolves; selfish, greedy of gain, and disposed to plunder. False teachers may, at first, appear very interesting; but they should be judged of, not by their appearance merely, but by the character and effects of their principles and conduct.>
<7:16 Fruits; the nature and effects of their doctrines and conduct.>
<7:21 Not every one; men are to be judged of, not by their words only, but by their principles and conduct. They must obey the revealed will of God, and to be accepted of him, must do it with the heart. The only sure test of true religion, is the doing of the known will of God.>
<7:23 Never knew you; as my disciples.>
<7:24 A wise man; one who selects good ends, and uses the right means to attain them.>
<7:25 The hopes of those who believe in Christ as the Lord their righteousness, and do his will, can never be disappointed.>
<7:27 Those hopes which are not founded on Jesus Christ, but upon human merit, or on the mercy of God without faith in Christ and obedience to him, will perish at the giving up of the ghost. Pr 11:7.>
<7:28 Ended; finished his sermon on the mount, as recorded in the last three chapters. Astonished; no wonder, for this is a most astonishing sermon. It fills up, in its explanations, the law of God to its divine fulness. It shows to men the way of excellence, usefulness, and happiness. It points out their dangers, and the way to escape them; their duties, and the way to perform them. It sets before them the motives best adapted to lead them to avoid the one and perform the other; and it does this with a brevity and clearness, a pertinency and fulness, a simplicity and directness, a beauty, comprehensiveness, and force which are truly divine.>
<7:29 Taught--as one having authority; the Pharisees quoted what the fathers had said; Christ spoke in his own right. He had authority over the winds and the waves, over diseases and devils, and over all creatures in heaven, earth, and hell. Such was his character, dominion, and work, that even in his deepest humiliation it was the duty, not only of men but of angels, to worship him. Heb 1:6.>
<8:1 Taught--as one having authority; the Pharisees quoted what the fathers had said; Christ spoke in his own right. He had authority over the winds and the waves, over diseases and devils, and over all creatures in heaven, earth, and hell. Such was his character, dominion, and work, that even in his deepest humiliation it was the duty, not only of men but of angels, to worship him. Heb 1:6.>
<8:2 Leper; leprosy was one of the most filthy, loathsome, and incurable of diseases. Thou canst; an expression of faith in his almighty power. Whenever Christ wills, our difficulties will be removed; and implicit confidence in him is a good preparation to receive his favor.>
<8:3 I will; in this Jesus showed that he is almighty, according to his declaration in Re 1:8.>
<8:4 Tell no man; either tell no man till thou hast shown thyself to the priest, that his judgment of the cure may not be influenced by any report of the miracle; or do not noise abroad the matter, a command often given by our Lord to those whom he had healed. Compare The gift, Mt 12:15-21. A testimony, Le 14:1-32; that he was really cured, and might safely be again admitted into society.>
<8:5 Centurion; a Roman officer who had command of a "century," consisting generally of about a hundred men.>
<8:8 Those who have the most exalted views of Jesus Christ, have humble and abasing views of themselves.>
<8:9 Under authority; to my superior officers, and therefore knowing how to render prompt obedience. Having soldiers under me; and therefore knowing how to receive prompt obedience. He means to say, Just as I obey and am obeyed, so thou hast only to command, and diseases will come and go at thy bidding.>
<8:10 So great faith; such strong confidence in the power of Christ to do whatever he pleased. In Isreal; among the Jews, whose spiritual advantages were much greater than those of any other people.>
<8:11 Many; from among the Gentiles and people less favored with light. Many, with small advantages, look to Christ and live; while others, whose advantages are much greater, reject him and perish.>
<8:12 Children of the kingdom; Jews favored with great privileges. Outer darkness; the darkness without the banqueting-hall, which is brightly illuminated. Compare Chap Mt 22:13. The banqueting-hall here represents the kingdom of heaven, and the outer darkness, hell.>
<8:14 Marriage is honorable in all, and is especially important in ministers of the gospel. A bishop who is the husband of one wife, is, in this respect, like the apostle Peter.>
<8:17 Took--bare; took them upon himself, and thus took them away from us. Such is the plain meaning of this passage, quoted from Isa 53:4. Bodily sickness is a part of the sorrow which sin has occasioned. By healing this, the Saviour shadowed forth the perfect redemption which he gives to our souls by taking our place, and being "wounded for our transgressions," and "bruised for our iniquities." Isa 53:5; 1Pe 2:24.>
<8:18 The other side; of the sea of Galilee.>
<8:19 Follow thee; in a special sense; become thy disciple and attendant. Persons sometimes express strong resolutions of becoming followers of Christ, without duly considering to what it will expose them, or what they must relinquish for his sake.>
<8:20 The Son of man; the Son of man in a preeminent sense. Had Jesus been a mere man, this title, which he commonly applied to himself, could have had no significancy. But now, being God, he described himself by it as "God manifested in the flesh," 1Ti 3:16. Hath not where; is destitute of a home and its comforts. He would have him understand that his followers must expect poverty and hardships, and be prepared to bear them. Poverty is no disgrace, unless brought upon men by their own fault. The poor resemble the Redeemer in their outward condition, more than the rich. He chooses for them, in this respect, that condition which, when on earth, he chose for himself.>
<8:22 Let the dead; the spiritually dead. Bury their dead; the literally dead. Let those who are impenitent sinners without spiritual life bury your father, and do you now what I command you. The omniscient Saviour saw that such a command was necessary to impress upon that disciple the supreme importance of his service, and the necessity of making every earthly feeling and interest subordinate to it. Compare his command to the rich young man in chap. Mt 19:21. Our obligations to Christ are greater than to father, mother, or any earthly friends; and we should not let our regard for them hinder us from promptly obeying him.>
<8:26 O ye of little faith; small confidence in my knowledge and power. Rebuked the winds; commanded them not to blow. For men to be fearful when following the directions of Christ, shows great want of confidence in him, and is both foolish and wicked.>
<8:28 The other side; the east side of the sea of Galilee. Gergesenes; the region in which was situated the city of Gergesa, and also that of Gadara, mentioned Mr 5:1. The tombs; these among the Jews were often excavations in hills and rocks, sometimes of great extent, with many apartments, which afforded shelter to those who had no better accommodations. The power and malice of unclean spirits is inconceivably great. It should be to us a matter of devout gratitude that they are made subject to the authority of Christ, and can harm none that put their trust in him.>
<8:29 Before the time; the day of judgment, the time appointed by God for their final torments. Even devils knew that God would fulfil his word, in punishing them at his own appointed time.>
<8:32 Go; by this permission the reality of the existence of unclean spirits, and their terrible power and malice, were manifested in a most striking way.>
<8:34 Besought him that he would depart; probably from fear lest his miraculous power should work them still greater worldly losses. Men who are not literally "possessed of devils," may still be influenced by evil spirits; and when so influenced, they are opposed to Christ, and wish him to depart from them. Covetousness leads men to act in the same way; and so debases them, that they prefer any thing by which they can make money, to the presence and glory of the Saviour.>
<9:1 His own city; Capernaum. Chap Mt 4:13; Mr 2:1.>
<9:2 Son; a title of condescension and kindness. Thy sins be forgiven thee; here, as everywhere in the holy Scriptures, disease is regarded as a fruit of sin. The forgiveness of the man's sins by the Saviour is a pledge that in due time his disease shall also be healed. Some think that this had been produced by special sinful indulgence. When men feel their need of Christ, and have living faith in him, they will let nothing hinder their application to him for help.>
<9:3 Blasphemeth; by usurping the prerogative of God to forgive sins.>
<9:4 Knowing their thoughts; by his divine omniscience, though they had not expressed them. Think ye evil; of me, as if I were a blasphemer in forgiving sins.>
<9:5 Easier; that is, one is as really the work of God as the other.>
<9:6 But that ye may know; by healing the sick of the palsy he manifests himself to be God, and therefore able to forgive sins.>
<9:8 Unto men; it was not a man that had done this divine work, but God manifest in the flesh. 1Ti 3:16.>
<9:9 Matthew; the writer of this gospel. Receipt of custom; the place where taxes were paid. Some abandoned men are called by the grace of Christ; and when he speaks to their hearts, they will immediately follow him.>
<9:10 The house; Matthew's house. Publicans and sinners; tax-gatherers and vicious persons.>
<9:12 Sick; sinners need the Saviour, as those that are sick need a physician. It was therefore proper that he should be with such, for the purpose of doing them good. And if any were really righteous, as the Pharisees imagined that they were, they did not need his presence as a Saviour. It is sometimes right to associate even with the openly vicious, for the purpose of doing them good.>
<9:13 Meaneth; Ho 6:6. Mercy; I am pleased with a merciful disposition, manifesting itself in doing good to the needs, more than with the most careful attention merely to external ceremonies. In these latter lay all the religion of the scribes and Pharisees. They scrupulously avoided the outward defilement of contact with publicans and sinners, while they had no compassion for their souls or bodies. No external observances will compensate for the want of a kind, compassionate disposition; and acts of mercy to the needy and to the guilty, from love to God and men, are peculiarly acceptable to him.>
<9:15 The children of the bride-chamber; the companions of the bridegroom during the marriage-feast. Jud 14:10,11. Then shall they fast; fasting is an expression of sorrow, not suitable for the marriage-feast while the bridegroom is still present. So Christ is the bridegroom of the church. While he was personally present with his disciples, it was not suitable that they should fast. After his removal from them, they would have trials that would make fasting proper.>
<9:16 New cloth; or, as the margin, raw or unwrought cloth, not yet dressed or fulled, and liable to shrink upon being wet. Taketh from the garment; namely, when by shrinking it tears itself from it. Very much of a person's usefulness depends upon the correctness of his judgment as to the time and manner of doing things, and upon his doing things which are not only right in themselves, but adapted to the circumstances in which he is placed, and to the character and condition of those whom he labors to benefit.>
<9:17 Into old bottles; bottles were then made, not of glass, but of the skins of animals. Of course, those that were old would be rotten; and new wine, if put into them, would, in the process of fermentation, burst them. By this and the preceding similitude our Lord teaches that the austerities of the old dispensation, under which John lived, cannot be profitably mixed in with the free spirit of the new. Compare chap. Mt 11:18,19.>
<9:18 Ruler; an elder to whom was committed the care of the synagogue. Even now dead; when he came to Jesus she was at the point of death. Before Jesus arrived at his house she was dead. Mr 5:23,35. Our highest comforts may be the occasion of our deepest sorrows; but application to Jesus, with unwavering confidence in him, will bring sure and all-sufficient relief.>
<9:20 Issue of blood; an unclean disease, according to the Mosaic law. Le 15:25. Hem; border or fringe.>
<9:22 Daughter; a term of tender kindness. Thy faith; the power of Christ was the cause, and her faith in him, leading her to take the proper measures, was the means of her being healed.>
<9:23 Minstrels; the persons hired to play on instruments of music at funerals. Making a noise; the noise of wailing, as was the custom.>
<9:24 Give place; retire: your services are not wanted. Not dead; that is, not permanently. Her death is but as a sleep from which she will be speedily awakened.>
<9:25 Went in; Mark tells us that he took with him five persons. Mr 5:37-40. These were all competent witnesses, as were the multitude without when they saw her, of the reality of the miracle.>
<9:27 Son of David; a phrase among the Jews for the Messiah, as descended from David.>
<9:28 Before he gives men the blessings which they need, he often tries the reality and strength of their faith, and leads them to manifest that they believe he is able to give what they seek; and thus, by the time and manner of bestowing his favors, he greatly increases their value.>
<9:33 The dumb spake; thus was the prophecy, Isa 35:5,6, fulfilled in Jesus; showing that he was the Christ. In Israel; in the land of Israel, or in the history of their nation.>
<9:34 Prince of the devils; they ascribed his beneficent miracles to the help of Satan, for the purpose of preventing the people from receiving him as the Messiah. No kindness can be so great and no mode of expressing it so wise and good, but that wicked men will sometimes find fault with it, and attribute it to the basest means and the vilest motives.>
<9:36 Sheep having no shepherd; not provided with proper guardians and teachers.>
<9:37 Plenteous; there are vast multitudes who need the gospel.>
<9:38 Lord of the harvest; the great, divine teacher. Send forth; prepare and incline many to go and preach the gospel.>
<10:1 Power; the casting out of unclean spirits is here distinguished as something distinct from the healing of any kind of disease. Christ has such infinite fulness, that he can communicate to his ministers and disciples all the gifts and qualifications which they need.>
<10:2 Apostles; messengers, persons who were sent. Simon; when Christ first saw him, he called him in Syro-Chaldaic, Cephas--in Greek, Petros--which means, a stone; signifying, that in his future life he would be a firm and steadfast supporter of the truth. James; this was he whom Herod slew. Ac 12:2.>
<10:3 Matthew; whom Christ called while sitting at the receipt of custom. Chap. Mt 9:9. James; he who wrote the epistle called by his name. Lebbeus; called also Judas. Lu 6:16.>
<10:4 Iscariot; the man of Charioth, to which town he belonged.>
<10:5 Gentiles; those who were not Jews. Samaritans; they occupied a country on the north of Judea, lying between Judea and Galilee, which formerly belonged to the tribe of Ephraim and the half tribe of Manasseh. After these tribes were carried captive by the king of Assyria, it was peopled to a great extent by heathen, and the religion of the Samaritans was a mixture of Judaism and paganism. 2Ki 17.24 He appoints to his disciples the place of their labors, and though it may not be the one which, if left to ourselves, we should choose, we must learn, whatsoever place or state he chooses for us, therewith to be content.>
<10:6 Lost sheep; expressive of their wandering and dangerous condition. The house of Israel; the Jews, descendants of Israel, and hence called Israelites. As the Jews were the covenant people of God, it was proper that the gospel should be first preached to them. The Christian dispensation, moreover, which knows no distinction between Jews and Gentiles, was not fully established till after our Lord's ascent and the pouring out of the Holy Spirit on the day of Pentecost.>
<10:7 Kingdom of heaven; see note to chap Mt 3:2.>
<10:8 Freely give; as you have received miraculous powers without paying for them, exert those powers without receiving compensation.>
<10:9 We should not delay present duty in order to be better prepared to perform it. When Christ commands; we should obey, trusting in him for what we need in order to obey him and to be accepted in it.>
<10:10 Scrip; a bag for provisions. Two coats; they were, in their first journeys among the Jews, not to provide either money or clothing, but to trust in Christ to supply them. Staves; if a man had a staff, he might take it; if he had shoes or sandals, he might wear them. Mr 6:8,9. But they were to go without delay, and not be detained to make further provision. Worthy of his meat; he deserves to be supported. So with you. This is a rule that holds good for all time.>
<10:11 Worthy; a man of reputation for piety and general worth, and who will be likely to receive your message. There abide till ye go thence; abide in one and the same house till ye leave that city. This would be equally conducive to their own comfort and the convenience of those who resorted to them.>
<10:12 Salute it; they were to use all the customary forms of politeness. Courtesy in ministers of the gospel and the manifestation of good-will to all, are required by Christ, and are essential to the highest comfort and usefulness of all who proclaim his truth.>
<10:13 Be worthy; if they receive your message, the blessings you desire shall come upon them. Not worthy; if they reject your message, blessings shall follow you, but not them.>
<10:14 Shake off; a strong expression of abhorrence of their sins, according to a custom among the Jews. Ac 13:51; 18:6.>
<10:15 More tolerable; their doom shall be less dreadful. They sinned against less light, and were less guilty than those who lived in the days of Christ.>
<10:16 As sheep; defenceless, unprotected by human power. Wolves; men disposed to assault and kill you. Serpents; emblems of wisdom. Doves; of innocence. Ministers of the gospel are bound to be wise as well as good; to exercise discretion as well as courage; not needlessly to exasperate even the worst of men, but meekly to instruct them.>
<10:17 Beware; be cautious, and not needlessly exasperate wicked men, nor expose yourselves to their wrath. Councils; the judicial tribunals of the Jews. No wisdom or goodness in the discharge of duty will secure the approbation of all, or prevent some from becoming open and bitter foes.>
<10:18 Testimony; of the truths of the gospel, which would turn against them should they reject it.>
<10:19 Take no thought; be not anxious.>
<10:21 Put to death; the unbelieving members of the family will cause the believing members to be put to death for their love to Christ.>
<10:22 Endureth; continues faithful to the end of life.>
<10:23 Be come; to deliver his people and take vengeance on his foes. The primary reference of these words is to Christ's providential coming to destroy the Jewish state and nation by the hand of the Romans. But this foreshadowed his final coming to take vengeance on all the wicked. When greatly opposed in one place, it is not always a mark of wisdom or goodness to stay there; nor is it any evidence of want of courage or fidelity sometimes to flee, even if, in order to do it, a person should, like Paul, be let down by a wall in a basket. 2Co 11:23.>
<10:24 Above his master; you must not expect that they will treat you better than they treat me.>
<10:26 No one in the path of duty should be disheartened on account of difficulties; for he will never meet with any which he will not, if he trust in Christ, be enabled either to overcome, or cheerfully and usefully to bear. Ac 5:41.>
<10:27 In darkness; privately. On the house-tops; publicly.>
<10:28 Him; God. Destroy--in hell; by making them miserable there for ever.>
<10:29 Your Father; he takes care of even the birds. Surely, then, he will take care of you.>
<10:32 Confess me; as his Saviour, and continue to obey me. Him will I confess; acknowledge and treat as my friend. Men's treatment of Christ in this world will determine his treatment of them in the world to come.>
<10:33 Deny me; desert my cause. I deny; deny to be my friend, and treat as my enemy.>
<10:34 A sword; the effect of my doctrine and teaching will be, not to unite those who confess and those who deny me, but to divide them, even though they belong to the same family. The publication of the gospel is the occasion of developing the human heart, and leading men to show whether they are for Christ or against him. Yet the persecutions and distresses which often follow, are never the proper effects of the gospel, but always the effect of men's opposition to it.>
<10:37  Father or mother--taketh not up his cross; a man must love Christ more than earthly friends, and follow Him notwithstanding all the trials to which it may expose him, or he cannot be His true disciple. Without making sacrifices, men cannot be disciples of Christ: but this should never hinder them from embracing, and steadfastly following him; for all the losses to which they may be called, even that of life itself, will be productive of their highest, their eternal good. Ro 8:18.>
<10:39 He that findeth his life shall lose it; though a man, by forsaking Christ, should preserve his life for a time, yet he would, by doing so, lose his soul. And though, by following Christ, he should lose his life, he would in this way save his soul.>
<10:40 Me--him that sent me; Christ and believers are so united, that what is done to them is considered as done to him; and he and the Father are so united, that what is done to one is done to the other. Men may at any time show kindness to Jesus Christ, by showing it, from love to him and his cause, to his disciples; and thus they may be continually enhancing their gracious and eternal reward.>
<10:41 In the name; on account of his being a prophet, from attachment to him and to his Lord. A prophet's--a righteous man's reward; he shall share in the spiritual blessings which God bestows on the prophet, or on the righteous man, whom he has thus aided.>
<10:42 Little ones; disciples, even the feeblest of them. In the name; because he is a disciple, from attachment to him and his Master; he shall receive the approbation and blessing of his Lord.>
<11:1 Little ones; disciples, even the feeblest of them. In the name; because he is a disciple, from attachment to him and his Master; he shall receive the approbation and blessing of his Lord.>
<11:2 In prison; Lu 3:19,20.>
<11:3 He that should come; the expected one--the Messiah. Though John had borne express testimony to Jesus, yet both he and his disciples may have been perplexed by erroneous ideas respecting the nature of His kingdom, and by their consequent failure to witness the fulfilment of their expectations concerning Him.>
<11:6 Not be offended in me; not dissatisfied with my character, conduct, and claims; but shall receive me as the Saviour, the Lamb of God that taketh away the sin of the world. Joh 1:29. Many were offended because Christ did not satisfy their carnal expectations concerning their long-promised Messiah. In these words the Saviour returned to John a virtual answer to his question, yet expressed in such a form that his enemies could take no advantage of it.>
<11:7 A reed shaken; an inconstant, unstable person.>
<11:8 Soft raiment; effeminate, delicate clothing. King's houses; the place for such persons is in the palaces of the great, not in the wilderness.>
<11:9 More than a prophet; more distinguised and honorable than any of the Old Testament prophets, because he was the forerunner of Christ, and stood in a nearer relation to him than any of them.>
<11:10 Written; Mal 3:1; Isa 40:3; Mt 3:3.>
<11:11 Greater; in dignity; more honorable in condition and employment. Least, as a prophet or teacher under the gospel dispensation. Is greater; his work would be one of higher dignity and privilege than John's, because he would stand in a still nearer relation to Christ and proclaim more fully the truths of his gospel. The work of a gospel minister is a most exalted employment; and he who faithfully performs it, is, in God's estimation, among the most honorable of the earth.>
<11:12 From the days of John the Baptist; from the days of his public appearance, when the kingdom of heaven, which before that had been something future, first began to come as something present. Violence--by force; men were strongly excited, and they pressed to hear and receive the gospel.>
<11:13 Prophesied until John; they prophesied of the kingdom of heaven as yet to come till John, when its coming began. See above, note on verse Mt 11:12.>
<11:14 Elias; the one who was foretold in the Old Testament under the name of Elijah, because he would resemble that prophet. Mal 4:5.>
<11:15 He that hath ears; let every one who can, hear and understand this concerning John and the coming of the kingdom of heaven. Our Lord commonly employs these words of some doctrine or statement that requires study and thought to comprehend it.>
<11:16 Children sitting in the markets; and imitating in their plays the transactions of life.>
<11:17 Piped unto you; played a lively tune, as at a wedding-feast. Mourned unto you; played a mournful tune, as at a funeral. Lamented; imitated the lamentations at funerals that accompanied the playing of the minstrels, chap Mt 9:23. These children are wayward, and will do nothing to please their mates.>
<11:18 Neither eating nor drinking; living very abstemiously. Though there is a great variety in the outward condition of ministers of the gospel, and in the mode of their communications with men, yet no condition and no manner of living or preaching will make the gospel universally acceptable, or lead any, without the grace of God, to embrace it.>
<11:19 Eating and drinking; living as did other people. They say, Behold a man gluttonous; they found fault with both, and rejected both, like fickle, capricious children, whom nothing could please. Wisdom is justified of her children; right and wise ways, like those which John and the Saviour pursued, will be approved by the spiritually wise and good.>
<11:20 Upbraid; rebuke and denounce judgments against them.>
<11:21 Chorazin--Bethsaida; cities in Galilee which he often visited, and in which he taught and wrought miracles. Tyre and Sidon; commercial cities in the western part of Palestine, on the Mediterranean sea. Sackcloth and ashes; the signs of sorrowing penitence. The evidence which, through the grace of God, would have convinced some who are now lost, had they enjoyed it, and might have led them to repentance, utterly fails to produce these effects upon others.>
<11:23 Exalted unto heaven; greatly distinguished by privileges. Brought down to hell; destroyed with an aggravated destruction. Remained; would not have been destroyed.>
<11:24 More tolerable; they will be punished less severely, because they have not sinned against and rejected so much light. The higher men are raised in privileges, the lower, if they continue to abuse then, will they sink in future woe.>
<11:25 Hid these things; not led them to perceive and embrace them, because there were the wisest and best reasons why he should not. Wise and prudent; in their own estimation, and so proud that they would not ask of God that wisdom which is from above. Babes; those who feel their dependence on God, and seek his aid. For all his dealings, however mysterious to men, God has the wisest and best of reasons. Those who love him will believe this, and rejoice in the conviction that he doeth all things well.>
<11:26 It seemed good; because it was good, right, and best.>
<11:27 All things are delivered; all things were by the Father committed to Christ as mediator. He is head over all things to his church, and the final judge of the living and the dead. Reveal him; as manifested in the person and work of the Saviour, and by his word and Spirit. None have right views of God, except those who learn his character from his Son. The course of wisdom therefore, for all who wish to know God, is to sit at the feet of Christ and learn from him.>
<11:28 Heavy-laden; burdened with sins or sorrows of any kind. Rest; relief, especially inward peace.>
<11:29 None need to be miserable. By submission to Jesus Christ, trust in him, and obedience to his commands, all may be happy in life, in death, and for ever.>
<12:1 Corn; in the Scriptures this word means grain of any kind, especially wheat and barley, which were common grains of Palestine. Pluck the ears; picked off the heads, and rubbed them in their hands to separate the kernels from the ear. Lu 6:1.>
<12:2 Not lawful; not right; a violation of the fourth commandment.>
<12:3 David; 1Sa 21:1-6. The necessity of the case justified him.>
<12:5 Read in the law; Nu 28:9,10. Profane the Sabbath; do what would have profaned it, had not the appropriate duties of the Sabbath required that labor. The Saviour refers to the killing and dressing of the animals for sacrifice, and other labors connected with the daily temple service. In Joh 7:23, he specifies circumcision as another work performed on the Sabbath. Blameless; without fault, because they did only what was proper on that day.>
<12:6 Greater than the temple; the argument is, that if in the service of the temple the priests might profane the Sabbath according to the outward letter, much more might his disciples in his service; for he was Lord both of the temple and the Sabbath. Neither the temple nor the Sabbath, nor any place or time or form of religious worship, should ever, in our affections, rival him who is Lord of all, or lead us in any respect to contravene his will with regard to them.>
<12:7 Mercy, and not sacrifice; see note on chap Mt 9:13. The fourth commandment always allowed men on the Sabbath-day to relieve the distressed, to feed the hungry, and to perform all those labors which public worship and the best discharge of the appropriate duties of holy time require.>
<12:8 Lord--of the Sabbath; he who made it, and to whose worship it is devoted. If it was right for David to appease his hunger, and for the priests to do what was needful for the worship of God in the temple, much more was it right for the disciples, in attending upon the Lord of the Sabbath and of the temple, to appease their hunger as they did on the Sabbath-day.>
<12:9 Imitators of Christ will on the Sabbath attend public worship, for the purpose of thus honoring God and benefiting their fellow-men.>
<12:11 Men often condemn in others, things which they without scruple allow in themselves.>
<12:16 Not make him known; he wished to be retired from public view, and do his works of love and mercy as the prophets had foretold that he would, Isa 42:2,3; and thus furnish new evidence that he was the Messiah.>
<12:18 Chosen; to be the Messiah. Show judgment; make known the truth to the Gentiles, and thus bring them into obedience to himself and become their Lord and Judge. Compare Isa 2:2-4; 11:10; 62:2; Mal 1:1, etc.>
<12:19 Not strive, nor cry; not come with outward show, as the Jews expected that he would. Human perfection, as exemplified in Christ, is compassionate, condescending, and kind; meek, lowly, and retiring. It does not unnecessarily awaken the opposition, or intrude upon the attention of others; while it is earnest and affectionate, active and perservering in doing good.>
<12:20 A bruised reed; an emblem of persons who are feeble, and crushed with difficulties. Not break; not oppress or trample them down. Smoking flax; the wick of the ancient lamps. Shall he not quench; Christ would not quench, but cherish the feeblest beginnings of true grace. Unto victory; till his truth and mercy become triumphant.>
<12:24 Beelzebub; this name, among the Jews was applied to the prince of unclean spirits. By applying it to Christ, they expressed the utmost contempt. The ascription to the devil of what is performed by the Holy Ghost, is a sin peculiarly offensive to God, and exceedingly dangerous to men.>
<12:26 Divided against himself; had their representation been true, Satan would have made war upon himself, which was absurd.>
<12:27 Children; disciples of the Pharisees, who laid claim to the power of casting out devils, and were reputed so to do. Shall be your judges; shall convict you of folly and wickedness, in ascribing to Satan in my case what you ascribe to God's help in their case.>
<12:28 Kingdom of God; the reign of the Messiah on earth. Is come unto you; has already come upon you.>
<12:29 How can one enter; our Lord now gives the true explanation of his casting out devils. Satan, as a strong man armed, has taken possession of this world and of the souls of men. But Christ is stronger than he. He casts him out of individual hearts at his will, and will finally cast him out of the world. See Lu 11:21,22; 10:18; Re 20:1-3.>
<12:30 There are in our world no neutrals; all men are either for Christ or against him.>
<12:31 Be forgiven unto men; if men repent of and forsake them, they are pardonable. Blasphemy against the Holy Ghost shall not be forgiven; it is unpardonable; it will never be repented of. The sin spoken of seems to have been that of malignantly ascribing to Satan what was known to be the work of the Spirit of God. Mr 3:30.>
<12:33 Tree; the heart. Fruit; conversation and conduct. Is known; there is a correspondence between men's feelings and their actions, as there is between a tree and its fruits: the one is known by the other. These words have a double reference. First, to Christ: let the Pharisees show that his works are evil, or admit that he is good. Secondly, to his adversaries: they are evil, and can neither do nor speak good things, as he says in the next verse: "O generation of vipers," etc. The difference in the character and conduct of men is according to the difference of their hearts; their chief concern, therefore, should be with their thoughts and feelings, not merely with their outward actions.>
<12:36 Idle; here a word wantonly and causelessly uttered, like the blasphemous charges brought by the Pharisees against our Lord.>
<12:37 Words; since they flow from the heart, and indicate its character, verse Mt 12:34. Justified; shown to be righteous. Condemned; shown to be wicked.>
<12:38 A sign; some miraculous appearance from heaven. Compare chap Mt 16:1; Mr 8:11; Joh 6:30. Like all cavillers, they profess not to be satisfied with the proofs he had given them of his divine mission: they must have signs according to their own dictation. Men who disbelieve and reject the truth, often profess to do it for want of evidence; while the evidence which God has furnished, and which is abundantly sufficient, they overlook or withstand.>
<12:39 No sign; no such sign as they desired. One would in due time be given, which would demonstrate his Messiahship; but it would not convince them.>
<12:40 In the whale's belly; in which he was a type of Christ's burial. Three days and three nights; that is, parts of three days and nights. The burial of Christ took place on Friday. That was reckoned, according to Jewish custom, as one day. Saturday, through the whole of which Christ was in the tomb, called the heart of the earth, was another day; and the Christian Sabbath, on the morning of which he rose from the dead, was the third day; or according to their mode of speaking, three days and three nights.>
<12:41 Men of Nineveh; Jon 3:5. Greater than Jonas; the Messiah, the Son of God.>
<12:42 Queen of the south; 1Ki 10:1-9. Uttermost parts; a very distant country. Greater than Solomon; though Solomon was the greatest of men as to wisdom, 1Ki 3:13, Jesus was greater than he, or any mere man.>
<12:43 Dry places; barren and desolate regions, here considered as the haunts of evil spirits.>
<12:44 I will return into my house; into the man in whom he had dwelt. Empty; not occupied by any other who would keep him out.>
<12:45 Seven; a large or full number. More wicked; some totally wicked spirits are more wicked than others. Worse than the first; if men do not grow better under the means of grace, and permit the Holy Spirit to take possession of their hearts, they will grow worse. This wicked generation; the primary reference of this awful parable is to the Jews of our Lord's day. Much culture had been bestowed by God upon their nation. Under the preaching of John they had recently given promising signs of repentance. But their hearts had remained, like an untenanted house, empty of God's presence and grace; and now the unclean spirit is returning, with seven more wicked spirits, to hurry them on to ruin temporal and eternal. The parable is fulfilled also in all nations and individuals who imitate the conduct of that "wicked generation." A man's heart, by withstanding conclusive evidence, is made harder, and his wickedness increased; so that his character by such a course grown constantly worse, and his last state will be worst of all.>
<12:48 Who is my mother? this question was designed to awaken attention, in order more usefully to communicate instruction.>
<12:50 Whosoever shall do the will of my Father--is my brother, and sister, and mother; my most intimate and endeared relatives and friends. These words contain a silent but powerful rebuke of the idolatrous honor paid by many to the mother of our Lord. No affection which ever did or can exist between earthly friends, equals in tenderness and strength that which subsists between Christ and those who do the will of his Father.>
<13:1 Seaside; the sea of Galilee.>
<13:2 Ship; a small vessel or fishing-boat>
<13:3 Parables; the parables of Christ were descriptions of natural things, for the purpose of illustrating spiritual things. The seven parables recorded in this chapter all relate to the kingdom of heaven among men. See note on Chap Mt 3:2. They are both illustrations of its nature and prophecies of its progress.>
<13:4 Wayside; where the ground was not ploughed, and the seed sown not covered. Careless hearers receive no benefit from the word of truth, though it be preached ever so faithfully.>
<13:5 Stony places; where the rocks were but slightly covered with earth. To be savingly benefited by the preaching of the gospel, it is not enough that persons admit its truths, that their feelings are excited, that they are greatly distressed on account of sin, or that they have a hope of salvation, and are exceedingly joyful. They must take Christ for their teacher and pattern; must trust in him for salvation; and whatever it may cost them, must perserve in obeying him to the end.>
<13:6 Because they had no root; the roots could not go down deep enough to obtain the moisture needful for their growth.>
<13:7 Thorns; parts of the field which had not been cleared. Choked; so shaded and exhausted in the ground as to prevent the grain from yielding increase. Supreme devotion to this world, whatever by a man's feelings and conduct in other respects, will prevent all saving efficacy of the gospel; and as long as it is continued, will exclude from the soul the love of God. 1Jo 2:15.>
<13:8 Good Ground; rich soil, and well prepared. Notice the gradation in respect to these four kinds of soil. In the first, the seed perishes without even springing up; in the second, it springs up, but withers away; in the third, it springs up and bears fruit, but not to perfection; in the fourth, it yields a harvest of perfect grain.>
<13:10 Why speakest thou--in parables? the question shows that this was the first time he had addressed the multitudes in this manner. Compare with this chapter the sermon on the mount, in which there are only similitudes intermingled with plain address.>
<13:11 You; his disciples, who loved him and desired to understand his teaching. The mysteries of the kingdom of heaven; the deep truths respecting the dispensation of the gospel, which had not before been revealed, or were revealed only in part, and which Christ opened plainly to his disciples. To them; to the multitudes without the circle of his disciples. Is not given; to know these mysteries. The hinderance to their receiving this knowledge is stated in verse Mt 13:13.>
<13:12 Hath; hath some knowledge of these mysteries. Shall be given; more knowledge. It is a practical knowledge of which the Saviour speaks, implying love towards him, and a desire to understand the truths which he taught. Hath not; hath not knowledge, because he hath neither love towards me nor desire to know my truth. Even that he hath; his present opportunities and privileges for knowing the truth. The Saviour here lays down a general principle of deep and solemn import, which all who hope to be saved would do well to ponder in their hearts. The way to have more light and grace is to make a diligent improvement of what is now granted to us.>
<13:13 Seeing, see not; have faculties and opportunities, but do not rightly use them; of course do not understand the truths which they do not desire to know. The ignorance, dulness, and prejudices which come from such a wrong state of heart, made it proper that the Saviour should veil his instructions in parables, which the careless and indifferent would neglect, but the earnest and humble would search into and understand.>
<13:14 In them is fulfilled; the language of Isaiah is a description of their case. Isa 6:9,10. Not perceive; not perceive the spiritual meaning of his words, because, as expressed in the next verse, they shut their eyes against the light. They were not converted, and not saved, as they might have been, had they loved the truth and desired to know it.>
<13:16 They see--they hear; with good effect. They loved the truth and desired to know it, and to them a knowledge of it was communicated.>
<13:17 Things which ye see--hear; things done by the Messiah, and truths taught by him.>
<13:18 The parable; understand the meaning of it. It represents four classes of hearers: the thoughtless, the fickle, the worldly, and the truly pious.>
<13:19 The word of the kingdom; the truths of the gospel. Understandeth if not; because he does not properly attend to it. This represents thoughtless, careless, and stupid hearers.>
<13:20 Anon; immediately; and as we are elsewhere taught, without either understanding or counting the cost of Christ's service. Compare Lu 14:25-33.>
<13:21 Root in himself; true Christian principle. Offended; discouraged, loses the interest which he once felt in the gospel, and turns back. This represents the fickle: persons of quick feelings, easily excited, and who for a time appear to be much engaged. But they are unstable, easily turned aside by difficulties, and so give up, and become more hardened than before.>
<13:22 Unfruitful; destitute of good works. He does not live a life of piety towards God, and of beneficence towards men. This represents the worldly-minded man, who is so occupied with the things of time, that he has no heart to attend to the salvation of his soul, or the souls of his fellow-men.>
<13:23 Beareth fruit; he receives the truth into the heart, and acts under its abiding influence. This represents the pious, the friends of God and men. They are all useful, but some more so than others. These truths, as to the various effects of the gospel, it was important that his disciples, who were to be preachers of it, should understand. They desired to understand them, and to them the understanding of them was given; while to his opposers, who did not wish to understand them, it was not given.>
<13:24 The kingdom of heaven is likened; the kingdom of heaven, here the visible church of Christ, is likened to a field in which the owner sows good seed, etc. Good seed; clean wheat, representing the truths of the gospel, and those who embrace them.>
<13:25 Tares; not our American tares, but a species of darnel bearing poisonous seeds, and having, before it comes to a head, a near resemblance to the stalks of wheat and barley. In places where Christ, by his ministers, communicates his truth, Satan and his agents will disseminate errors; and such is the state of the human heart, that they will, without cultivation, take root, spring up, and bring forth evil fruit. Men are therefore bound to take heed what they hear, as well as how they hear; for their adversary the devil goeth about, not only as a roaring lion, but also as an angel of light, seeking, in various ways, to destroy the souls of men.>
<13:26 Brought forth fruit; when the fruit began to grow. By their principles and conduct, the difference between those who embrace the gospel and those who embrace opposite errors, is seen.>
<13:28 Gather them up; by the process of weeding common in that country.>
<13:29 Root up also the wheat; on account of their resemblance and connection with each other. Men cannot in this world separate entirely the wicked from the righteous, or with certainty judge as to the characters of men. That must be left to the Searcher of hearts, and to the decisions of the day of judgment.>
<13:30 Harvest; the day of judgment. Reapers; the angels. Tares; the wicked. Wheat; the righteous. Ver Mt 13:49,50.>
<13:31 Another parable; this parable represents the progress which the gospel would make. From small beginnings it would increase, and its influence become extensive and powerful.>
<13:32 A tree; in that country the mustard grows much larger than it does in this.>
<13:33 Leaven; leaven is all-pervading and powerful. Though silent and hidden, it soon affects the whole mass. So would divine truth be, in its influence on individuals and on communities.>
<13:34 Without a parable spake he not; see note on ver Mt 13:13.>
<13:35 The prophet; Ps 78:2. The history of ancient Israel which the psalmist recounts was typical of the higher mysteries of Christ's kingdom, as the apostle Paul expressly teaches. 1Co 10:11.>
<13:37 Son of man; meaning himself, dispensing truth either personally or by his servants.>
<13:38 The field is the world; for by the appointment of Christ the good seed of the gospel is to be sown among all nations, so that the visible church shall be coextensive with the world. Children of the kingdom; children of God not in name alone, but in reality. Children of the wicked one; of Satan, though they be found among Christ's visible followers.>
<13:43 Righteous; the same as "the children of the kingdom," those who have believed and obeyed the gospel. Shine forth as the sun; be inexpressibly glorious in heaven. Ears to hear; let all who have ears hear and believe, and so act that they may escape the wailing of the wicked, and enjoy the glory of the righteous.>
<13:44 Buyeth that field; that, by obtaining possession of the field, he may obtain possession of the treasure in it. He who rightly estimates the value of his soul, will make its salvation his chief concern, and give up whatever prevents his obtaining it.>
<13:47 A net--cast into the sea; the sea is the world, and the net is the gospel with its ministers and ordinances. This parable has a close relation to that of the tares in the field. It shows the mixture of good and evil which will always exist in the visible church on earth. We should not be discouraged on account of the mixture of evil with good in God's church; for it has always been so, and will be so to the end of time.>
<13:48 It can be of no avail to any man to be a member of Christ's visible church, unless he have also the character of a Christian.>
<13:52 Every scribe; in allusion to the office of the Jewish scribes, which was to teach the law of Moses, Christ names those whom he calls to be teachers in the kingdom of heaven scribes. Instructed; trained and furnished as he should be. Ministers of the gospel should be always learning, not merely of men, but of God. They should also be habitually communicating, not merely what they learned years ago, but what they have lately learned, things new as well as old, that these truths may have in their own minds and the minds of others the freshness and beauty, the vigor and force of youth.>
<13:54 His own country; Nazareth. Chap Mt 2:23.>
<13:55 Carpenter's son; Joseph, his reputed father, was a carpenter.>
<13:56 These things; wisdom to teach in such an interesting and instructive manner, and power to work miracles.>
<13:57 Offended; at his humble birth and indigent circumstances. They were too proud to receive him as their teacher. In his own house; a man often has less influence with those among whom he spent his childhood than with others. To judge of persons by their wealth, or that of their relatives, or by any merely external distinctions, and not by their character and conduct, is evidence of a little mind, and of a proud heart.>
<13:58 Unbelief; as they rejected him, and disbelieved his Messiahship, notwithstanding all his miracles, he left them and departed to another place.>
<14:1 Herod the tetrarch; this was Herod Antipas, son of Herod the Great who slew the children at Bethlehem. Chap Mt 2:16. Tetrarch means the ruler of a fourth part, and was applied to him because he governed a part of his father's kingdom.>
<14:4 Not lawful; Herodias was the wife of Philip, Herod's brother, by whom she had a daughter named Salome. Herod had put away his own wife, the daughter of Aretas king of Arabia Petraea, and had taken Herodias, though her husband was still living.>
<14:5 Feared the multitude; he was afraid, should he put John to death, that they would rebel, and make him trouble; he therefore did not kill him, but put him in prison. Men are often disposed to commit crimes, from which they are restrained only by the fear of man, and other selfish considerations. This shows that their hearts are worse than their lives, and that they fear man more than God.>
<14:6 Seasons of feasting and revelry are seasons of great danger; and when attended with dancing and profaneness, render persons peculiarly liable to be overcome by temptation, and to fall under the power of the destroyer.>
<14:8 Instructed; her mother had told her what to ask. Charger; a large dish or platter. Continuance in known sin blunts, and finally obliterates the delicate perceptions, the tender sensibilities, and all the finer emotions of the human heart. It renders not only men, but women also, monsters of iniquity.>
<14:9 Sorry; he knew it was wrong, and was afraid it would make him trouble. Them which sat with him; he was more afraid of them than of God. No oath can lay a man under obligations to do wrong. It is a sin to take such an oath, and it is an additional sin to fulfil it. The wicked, while they often lay claim to great courage, and sometimes show what in some respects resembles it, are at heart great cowards. They are afraid even of being called cowards by those whose praise would be a blot; and to avoid it, they will commit murder, and expose themselves to the endless wrath of God.>
<14:10 Indulgence in one sin opens the way for and strongly tempts to the commission of others; and when men begin a course of iniquity, none but God knows where they will stop.>
<14:15 Evening; the Jews reckoned two evenings, one of which commenced about three o'clock in the afternoon, and is the one here referred to; the other commenced about six o'clock, and is referred to in verse Mt 14:23.>
<14:19 Blessed; he praised the Lord for that provision, and asked him to bless them in the reception of it. Those who labor to save the souls of men should, as they have opportunity and ability, supply the wants of their bodies; and while they help men to the bread which perishes, it may prepare them to receive that which endureth unto everlasting life.>
<14:20 Did all eat--were filled; besides the immediate act of mercy in feeding a vast multitude in the wilderness, this miracle was intended to have a deep symbolic meaning. By it Christ exhibited himself as "the bread of life." See the use which the Lord himself makes of it. Joh 6:27-58.>
<14:23 Habitual communion with God, and daily retirement for this purpose, is essential to holiness of character, and to great usefulness among men. It is also a safeguard against temptation, and a good preparation for the best discharge of duty.>
<14:25 Fourth watch; the Jews had four watches, or periods of the night. The first watch was from six to nine o'clock; the second, from nine to twelve; the third, from twelve to three; and the fourth, from three to six in the morning.>
<14:26 It is a spirit; they thought it was a spirit or ghost, supposing that for a man with a real body to walk on the water was impossible.>
<14:28 We must not be impatient, or needlessly expose ourselves to danger, even to be with Christ. If we do, he will show us that we lack faith; and that, had he not done better for us than we did for ourselves, we should have perished.>
<14:29 He walked; upheld by the divine power of Jesus Christ.>
<14:30 It is when our thoughts are turned away from Christ to the dangers around us, that we lose our courage.>
<14:31 Doubt; why didst thou doubt my power to continue to support thee?>
<14:32 The ship with the disciples in it, tossed all night by the waves, and detained by contrary winds, is an apt emblem of the church of Christ in the dark days of reproach and persecution. But the Saviour has his eye ever upon her, and when he comes to her help in the morning, her course will be calm and prosperous.>
<14:33 Son of God; this was a public acknowledgment of him as the Messiah.>
<14:34 Gennesaret; on the north-west side of the sea of Galilee.>
<15:2 Tradition; traditions were laws or precepts of men, which they said had been handed down by word of mouth from past generations, and many of which were afterwards written. They were often treated as of more authority than the laws of God. The scribes were the interpreters of these traditions, and could thus control the minds of the people. One of those traditions required the hands to be always washed before taking food. The object of this washing was to remove any ceremonial defilement that might have been unwittingly contracted in the intercourse of life. Our Saviour disregards it as a superstitious punctiliousness not required by the law of Moses. The Bible, as a rule of faith and practice, is perfect; and human traditions, however sanctioned, or by whomsoever taught, that add to it, take from it, or in any way pervert its meaning, are sources of error.>
<15:4 A reception of the Bible as the word of God, and a familiar acquaintance with its contents, is a great safeguard against false doctrines and vicious practices. Hence, the good of men, as well as the glory of God, requires its universal circulation among all classes of people.>
<15:5 It is a gift; that is, has been consecrated as a gift to the Lord. If children should announce to their parents that they had devoted to religious uses what might otherwise have been given to their support, the scribes said they were released from obligation to assist them, however much they might suffer. Thus, under pretext of religion, they nullified the law of God through their traditions.>
<15:6 Honor not; namely, by providing for them a comfortable support.>
<15:8 Draweth nigh; they pretended to honor God with words and outward observances, while their hearts and practices were opposed to him.>
<15:9 For doctrines; teaching as the commands of God what were merely the commands of men.>
<15:11 Not that; not food which goeth into the mouth, as the Pharisees pretended, but wickedness in the heart, coming out in false doctrines and wicked conduct, defileth a man.>
<15:13 Every plant; he means false teachers, such as these Pharisees, with their corrupt doctrines and practices. Shall be rooted up; God is continually rooting them out of his earthly church, as he did the Pharisees of old, by his providence cooperating with his word and Spirit; and in the world to come the separation shall be final and perfect.>
<15:14 Let them alone; regard not what they say, and have nothing to do with them.>
<15:15 When we do not clearly understand the Scriptures, we should ask God to teach us. And though he may see that a right use of our faculties would have removed our ignorance, yet, if we sincerely desire to know the truth, he will, in the proper use of means, instruct us, and make us wise to salvation.>
<15:16 Without understanding; common sense, if rightly exercised, would teach, that not food in the mouth, but sin in the heart defileth a man.>
<15:19 The teaching of God will lead a man to place less reliance upon external observances, and to look more to the state of his heart, in obedience to the command of Christ, "Make the tree good," in order that the fruit may be good.>
<15:21 Coasts of Tyre and Sidon; Zidonia, or Phoenicia, on the Mediterranean coast north of Palestine. Of this country Sidon was the earlier, and Tyre the later emporium.>
<15:22 Woman of Canaan; for the Zidonians were descended from Canaan, Ge 10:15. By Mr 7:26, she is also called a Greek, as being a Gentile in her religion; and a Syrophoenician, as belonging to the Syrian Phoenicia, as distinguished from the Libyan Phoenicia of Africa. The Evangelists dwell on her gentile descent, because this was made prominent in our Lord's answer to her.>
<15:23 Send her away; by granting her request.>
<15:24 I am not sent; the reference of our Lord is here to his personal ministry. See note on chap Mt 10:6. In the Bible, and in the bestowment of his blessings in providence, God makes much of "due time." Men, even good men, are often in great haste. They would do things, if they could, much sooner than God does them; but they would not do them so well.>
<15:26 Not meet; not suitable. Children's bread; that which was designed for the Jews, called children. Dogs; Gentiles, by the Jews called dogs. This he said to lead the woman to show her true character, which she soon did in a very striking manner.>
<15:27 Eat of the crumbs; as dogs, without robbing the children, eat the crumbs which fall from the table, so she thought she might receive this mercy without injury to any one; and she had the fullest confidence in his power thus to help her.>
<15:28 God often delays answering our requests, as a trail of our faith and humility. When these have been brought into exercise, a gracious answer will speedily come.>
<15:30 Maimed; such as had lost a limb, a hand, or foot. Restoring them, therefore, was an act of creative power. There is nothing men need which Jesus Christ cannot bestow. All should therefore wait upon him; and if not weary in doing his will, in due time they shall receive all needed good.>
<16:1 Pharisees--Sadducees; opposite sects among the Jews. Chap Mt 3:7. Tempting; trying him, in order to get something against him. Sign from heaven; some miracle in the skies besides those he had wrought upon the earth, and which they pretended would more clearly show his real character.>
<16:3 Hypocrites; pretending to one thing, while they sought another. Signs of the times; these had been numerous and decisive. They were far more convincing than many on which they daily acted with regard to this life. The sceptre had departed from Judah, and the lawgiver from between his feet; that is, the government of the country had departed from the tribe of Judah, and was then in the hands of the Romans, which Jacob, in blessing his sons, said should not be till Shiloh, or the Messiah, should come. Ge 49:10. John, the predicted messenger and forerunner of Christ, had come, chap Mt 3:3; Isa 40:3; Mal 3:1; 4:5; the Holy Ghost had descended from heaven visibly upon Jesus, and the Father had declared him to be his beloved Son, in whom he was well pleased. Chap Mt 3:16,17. He had wrought many incontestable miracles, and many predictions and promises of the Old Testament concerning the Messiah had been fulfilled in him, proving, most abundantly and conclusively, that he was the Christ. Yet they rejected all, and pretended that they wanted more evidence that he was the Messiah; while what they really wanted was, to put him to death, lest, as the Messiah, the people should believe in him. Evidence which fully satisfies men, and on which they readily act, with regard to this world, often fails to satisfy them in religion.>
<16:6 Leaven of the Pharisees; their doctrines, verse Mt 15:12, in which is included also their spirit of hypocrisy and vain-glory. Compare Lu 12:1>
<16:8 No displays of the power and love of Christ in times past, will of themselves lead his people rightly to trust in him for the future. In order to this, they must have his present teaching; and for this, as well as other things, they should pray "Give us this day our daily bread.">
<16:17 Bar-jona; son of Jonah; bar; being a Syriac word for son. Flesh and blood; man. In order rightly to apprehend divine truth, and suitably to regard it, men must be taught it, not merely by their fellow-men, but by their Father in heaven.>
<16:18 Thou art Peter; in the Greek, Petros, the same as Cephas--from the Aramaean, or Hebrew of our Lord's day--and meaning, rock. And upon this rock; in the Greek, petra, that is, rock. The less usual form, Petros, differs from petra in taking the masculine form, because it is given to a man as his epithet. The words "upon this rock I will build my church," have been differently interpreted among Protestants. First, "upon this rock," that is, upon thee, Peter, with allusion to the name "rock," which Christ had given him upon his first interview with him, Joh 1:42. According to this interpretation, Peter is called a rock only in a lower sense, as an eminent instrument to be employed by Christ in building up his church, just as he is afterwards said to receive the keys of the kingdom of heaven in a lower sense; for in the high sense, Christ alone is the rock on which the church is built, and he alone has the keys of the kingdom of heaven. Isa 28:16; 1Pe 2:6; 1Co 3:11; Eph 2:20; Re 1:18; 3:7. Secondly, "upon this rock," that is, upon the confession thou hast just made of me; or rather, upon the great truth contained in that confession, "Thou art the Christ, the Son of the living God". According to either of the above interpretations, Christ alone is the true foundation of the church. As it is written of him, "Behold, I lay in Zion for a foundation a stone, a tried stone, a precious cornerstone, a sure foundation; he that believeth shall not make haste," Isa 28:16. The apostle Peter says same, 1Pe 2:6. Paul also, in Eph 2:20, speaks of the church as "built upon the foundation of the apostles and prophets, Jesus Christ himself being the chief cornerstone." Thus, according to the prophet Isaiah and the apostles Peter and Paul, writing under the guidance of the Holy Spirit, the foundation of the church, and of the hopes of all true believers, is, not Peter, or Paul, or any creature, but "Jesus Christ, the same yesterday, today, and forever." Gates of hell; the counsels of the powers of evil. The gates of cities were anciently the places in which deliberations were held and plans formed.>
<16:19 The keys of the kingdom of heaven; keys are a symbol of power and authority. Bind--loose; the same gift is elsewhere bestowed on all the apostles and the disciples generally. Chap Mt 18:18. The words of this verse may be understood, first, of the authority which Christ bestowed upon the inspired teachers and guides of his primitive church to settle all questions respecting her. For eminent examples of the exercise of this power, see the decisions concerning gentile converts, Ac 11:1-18; 15:1-29. In this sense, the power ceased with inspiration. Secondly, the words may be understood of the common power conferred by Christ on his churches to regulate their own affairs, to administer discipline, and to admit to or exclude from their communion. In this sense this power continues in the visible church, and is valid so far as it is exercised in accordance with Christ's word. Apostles, in making known the will of God, and recording it in words which the Holy Ghost taught them, and faithful ministers in proclaiming it, allow or condemn on earth what God allows or condemns in heaven. Churches, when they act in accordance with his truth, bind or loose, that is, allow or disallow on earth what will br bound or loosed, allowed or disallowed, in heaven. Chap Mt 18:18.>
<16:20 Tell no man; the time had not come to proclaim him publicly as the Messiah. He must first die for the sins of men, according to the Scriptures, and rise again for their justification. 1Co 15:3,4. The publication of the whole truth would at some times be very improper. It would prevent much good, and occasion much evil. Very much of a person's usefulness on earth depends on his doing right things at the right time, and in the right way, as well as for the right end.>
<16:21 To show; that is, plainly. Before this, he had only given obscure intimations of his approaching death.>
<16:22 Rebuke him; this showed the self-sufficiency of Peter, his forwardness to express his opinion, and his liability to err. Shall not be; this was in direct opposition to what Christ had said should be, and what was essential to the salvation of men; showing that Peter was not infallible, but was often wrong. "As with a hammer of iron, Christ here crushes carnal prudence in Peter." We have infallible evidence of the fallibility of Peter, and that he was not, in character, conduct, or authority, above the other apostles.>
<16:23 Get thee behind me; a similar expression to what Christ had before used with regard to Satan, the great adversary of God and man. Thou savorest not; thinkest not. Peter did not coincide in his views with God, but with men in opposition to God. God caused this evidence to be placed upon a permanent record, that all might know that such as exalt Peter above his fellow-apostles, in this savor not the things that be of God, but those that be of men, and expose themselves to the rebuke of the Saviour, "Get thee behind me.">
<16:24 Come after me; follow my directions. Deny himself; abstain from all indulgences which stand in the way of duty. Take up his cross; resist the pleadings of carnal policy and appetite, and submit to whatever may be needful, in order to obey God. The life of disciples of Christ is one of self-denial. They must make sacrifices, and it is wise to do so, for it is the way to avoid the greatest loss and obtain the greatest gain.>
<16:25 Whosoever will save his life--lose his life; whosoever shall save his temporal life by renouncing the Saviour, shall lose his eternal life; and whosoever shall lose his temporal life by following the Saviour, shall secure his eternal life.>
<16:27 The Son of man shall come in the glory of his Father; the splendors of the godhead at the day of judgment, when those who have suffered for him on earth will reign with him in heaven.>
<16:28 Not taste of death; not die. Coming in his kingdom; coming to set up, extend, and render efficacious his reign over his people on earth, in preparation for their everlasting reign with him in heaven. There seems to be here a special reference to the awful manifestation of his presence and power in the destruction of Jerusalem and the Jewish state, by which was shadowed forth his final coming to judge the world.>
<17:1 Not taste of death; not die. Coming in his kingdom; coming to set up, extend, and render efficacious his reign over his people on earth, in preparation for their everlasting reign with him in heaven. There seems to be here a special reference to the awful manifestation of his presence and power in the destruction of Jerusalem and the Jewish state, by which was shadowed forth his final coming to judge the world.>
<17:2 Transfigured; changed in his appearance. His raiment was white as light; resplendent as lightning. The three apostles were here favored with a glimpse of the future glory of the Saviour and his true followers, well calculated to strengthen their faith in passing through the trying scenes that were before them, and through them to strengthen the faith of all his disciples amid "the sufferings of this present time." Ro 8:18; Php 3:21; 1Jo 3:2. The glory of the Saviour when on earth was veiled in him humanity; but on the Mount of Transfiguration it shone forth above the brightness of the sun; and as a full view of it, if continued, would unfit his people for their duties on earth, they must wait for this till they see him in heaven.>
<17:3 Moses and Elias; the representatives of "the law and the prophets" here appear in glory, but subordinate to the Saviour. Thus the unity of the old and new dispensation is set forth, and also the supreme dignity of "Christ the Son of the living God.">
<17:4 Tabernacles; a tabernacle was a temporary dwelling, covered usually with cloth or boughs of trees. Peter was so amazed and bewildered by the glory of the vision, that he knew not what he said. Mr 9:6; Lu 9:33. His plans were repeatedly in opposition to those of his Lord.>
<17:5 A voice; the voice of God the Father, as in chap. Mt 3:17, with the addition, "Hear ye him," attend to his instructions, and follow them. The delight of all parents in all children, from the foundation of the world to the end of time, if put together, would be infinitely less than the delight of the almighty Father in his beloved Son; and the way for men to glorify God is, to hearken to and honor the Son as they ought to honor the Father. Joh 5:23.>
<17:9 The vision; what they had seen and heard.>
<17:10 First come; come before the Messiah, since now he had appeared after him. They understood the prophecy, Mal 3:1-4, literally, of the Elias who had been translated to heaven; but our Saviour teaches them that it was fulfilled in the person of John the Baptist, who came "in the spirit and power of Elias." Lu 1:17. It is not enough for us to know the words of Scripture; we must understand their meaning, and make a right application of it. In order to this, we should seek assistance from those whose opportunities for understanding the meaning of Scripture and its right application have exceeded ours. We should also compare one portion of Scripture with another, and especially we should apply to Jesus Christ for the teaching of his Spirit.>
<17:11 Restore all things; set them in order, and bring them to a proper state for the coming of God as a Saviour.>
<17:12 Listed; desired or pleased to do.>
<17:15 Falleth into the fire; in the paroxysms of his disease, under the power of the evil spirit by which he was possessed. Ver Mt 17:18, and Mr 9:17,25,26. The family relation, while it gives us many of our choicest comforts, occasions also many of our deepest sorrows.>
<17:17 Suffer you; endure your perverseness and unbelief.>
<17:18 Rebuked the devil; bade him depart. Mr 9:25.>
<17:20 If ye have faith; the reference here is to that faith with which Christ, in the case of the apostles, connected the working of miracles. Nothing shall be impossible; no exercise of miraculous power, however great, that may be needful in the prosecution of your apostolic work. The spirit of this promise applies to all Christ's servants in all ages. No hinderance to their work can be so great that faith cannot overcome it. Had men higher views of Christ, greater confidence in him, and more entire devotion to his service, they might receive much more good themselves, and be instrumental of much greater good to others.>
<17:21 This kind; this kind of evil spirit. The words imply its great power and malice. Prayer and fasting; that faith which was necessary to work such a miracle, could not be obtained without much self-denial and prayer.>
<17:23 Exceeding sorry; because he was to be treated in that way, not knowing that his death was necessary for the salvation of men. That which here on earth occasions the people of God the greatest distresses, when they come to see the reasons for it and the benefits of it, they will see to have been overruled for the promotion of their highest and most enduring joys.>
<17:24 Tribute-money; paid yearly for the support of public worship and the service of the temple, amounting to half a shekel, or about twenty-five cents.>
<17:25 Prevented him; spoke before Peter had said any thing. When the Bible was translated into English, to prevent meant to go before. persons Strangers; persons not belonging to the family, not the children of the king who received the tribute.>
<17:26 Free; not expected to pay tribute. According to that rule, Christ, the Son of God, for the support of whose worship the money was paid would be free.>
<17:27 A piece of money; in the original, a stater, of the value of a shekel, or about fifty cents, which would pay the tribute for both Christ and Peter. Duty sometimes calls us not to insist upon all our just rights, but to take a different course for the purpose of preventing a wrong construction being put upon our actions, and of doing greater good.>
<18:1 Greatest; in that the kingdom on earth which they, in common with their countrymen, thought the Messiah would set up. Greatness in the view of men differs much from greatness in the sight of God. Men must give up seeking the one, in order to obtain the other. Among those who have right views of true greatness, there will never by any contention about it.>
<18:2 A little child; this he did to correct their false notions about his kingdom, to show them that it was spiritual, and that spiritual excellence, not outward splendor or authority, constituted greatness in it.>
<18:3 Converted; changed in their views and character. As little children; humble, docile, submissive, obedient. Many of the characteristics of little children afford important instruction to mankind.>
<18:5 In my name; from love to me, and because he belongs to me. Jesus Christ takes a deep interest in even the least and feeblest of his people, and views what is done to them as done to himself. Chap. Mt 25:40.>
<18:7 Woe unto the world; great evils will come on the world through the offences which men will commit. Needs be, such is the wickedness of men, that they will lead others to commit sin.>
<18:8 8, 9 Hand--foot--eye; these represent our strongest earthly desires and our dearest earthly possessions. These must all be denied and renounced rather than that we, by sinning ourselves, should be occasions of sin to others. Compare chap. Mt 5:29,30. However convenient or dear any thing may be, if it cause us to sin, it is better to do without it, than to have it and suffer the consequences.>
<18:10 These little ones; these disciples of mine, who are little ones in character. See note on ver Mt 18:3. The precept has reference to all Christ's humble followers, but is more especially applicable to those who are poor and obscure. Their angels; who are sent forth to minister to them. Heb 1:14. Always behold the face of my Father; how dear, then, must they be to God, and how great the peril of those who offend them. As angels, who always have access to God and enjoy intimate communion with him, are not ashamed to minister to the poorest and most humble of his people, no human being should be.>
<18:11 That which was lost; sinners. God and angels rejoice over their salvation. To illustrate this, he spoke a parable.>
<18:13 Rejoiceth more; this was natural. So with God. He rejoices in the salvation of the wandering and exposed. Of course, all should labor for this end. One way in which they could do it, he proceeded to point out in ver. Mt 18:15.>
<18:14 The seeking and saving of those who are lost, and the bringing of them to the fold of the Redeemer, gives joy to angels and to God.>
<18:15 Hear thee; if he is reclaimed. Gained thy brother; thou hast been instrumental in restoring him. The way to reclaim an offending brother is for some brother to go and converse with him alone. If this is not effectual, he is to take one or two more, and converse with him again. If that is not effectual, then it is their duty to communicate what has been done to the church. If, under their discipline, he will not reform, he is to be cut off.>
<18:16 Not hear; if he persevere in the wrong. Established; fully proved. De 19:15.>
<18:17 Neglect to hear them; if they cannot reclaim him. Heathen man; let him have no more connection with the church than you would allow to an open idolater.>
<18:18 Ye shall bind--loose; that is, the disciples who constitute the church spoken of in the preceding verse. This, which had been before said to Peter, chap Mt 16:19, is now said to the disciples generally, and it conferred as much power on them, as it did on him, and it promised as many blessings to them as it did to him. Whatever he or they should do in accordance with the directions of Christ, and in obedience to his will, would be ratified in heaven. This is true of the church and ministry of Christ in all ages. See note on chap. Mt 16:19.>
<18:19 Any thing; that is agreeable to his will.>
<18:20 In my name; under my authority, and for the purpose of doing any thing connected with the advancement of my kingdom. In the midst; I am present with them, to hear and bless them.>
<18:21 However numerous or aggravated are the offences of any brother, if he give evidence of penitence by confessing and forsaking his sins, all are bound to forgive him.>
<18:22 Seventy times seven; we are not to limit our forgiveness to any definite number of offences, but to forgive as often as we are injured.>
<18:23 The kingdom of heaven; the dealings of Christ with men in the end of the world and the day of judgment, which were to usher in the last and crowning stage of his mediatorial dispensation. See Mt 3:2; 1Co 15:24, etc.>
<18:24 Ten thousand talents; an immense sum, which he could never pay.>
<18:25 To be sold; it was customary then for creditors to sell debtors and their families for a sufficient length of time to pay their debts.>
<18:28 A hundred pence; a very small sum compared with ten thousand talents.>
<18:34 Tormentors; those who had authority to examine and extort confessions by torture.>
<18:35 Do also unto you; if we do not forgive others, God will not forgive us, but will punish us as we deserve. An unforgiving spirit is the spirit of perdition.>
<19:1 Galilee; this was the poorest part of Palestine. Hence, Galilean was a term of reproach. Judea; this lay to the south; and between it and Galilee was Samaria. Beyond; on the east side of the Jordan.>
<19:3 Tempting him; for the purpose of ensnaring him, in order to get him into difficulty. For every cause; whenever he chooses; as some of their teachers said that he might, and as they often did.>
<19:4 Have ye not read; Ge 1:27. In matters of religion, the appeal must be to the Bible; and an intimate acquaintance with it, and a cordial obedience to its laws, will give one a great advantage over his adversaries.>
<19:5 Marriage is an institution of God; honorable in all, ministers of the gospel as well as others; sacred in its obligations; and unless these obligations are violated by one of the parties, not to be dissolved till death.>
<19:6 One flesh; they are so united as to be no longer two, but one, each being a part of the other. Compare the apostle's words: "He that loveth his wife loveth himself." Eph 5:28. Of course they ought to be one in views, affections, and interests; and for a man to break such a union as this by putting away his wife for every cause, is wrong. Thus the question of the Pharisees was answered.>
<19:7 A writing of divorcement; De 24:1.>
<19:8 Suffered; he did not direct it, or suffer it in any such sense as to imply that God approved of it, or that it was right. It was a civil regulation of a civil government, suffered for a time on account of the wickedness of men, and in order to prevent the greater evils which that wickedness would otherwise have occasioned. It was a regulation as to the mode of putting away; not to justify that wrong practice, but to lessen, in some measure, its evils. Not so; from the beginning, and in all its stages, this putting away "for every cause" of one's wife was a violation of the will of God, as manifested in his works and his word. That God suffers the adoption, and for a time the continuance of practices, on account of the hardness of men's hearts, is no evidence of the moral rectitude of those practices. Nor is the giving of directions about them, and the adoption of regulations to lessen their evils while they continue, any evidence that God approves of them. The practices may still be a violation of what has been the will of God from the beginning, and obedience to him may require them to be done away.>
<19:9 I say unto you; I give you the right interpretation of the will of God in this matter. Fornication; here in the sense of adultery.>
<19:10 If the case of the man be so with his wife; if a man, to obey God, must live all his life with one wife, provided she lives and is faithful, whether he is pleased with her or not, then it is not good for a man to marry.>
<19:11 Cannot receive this saying; namely, that it is not good to marry. If all should, and act upon it, and not break any other command of God, the whole human race, when those now living are dead, would be extinct. Not to marry is contrary to the nature and wants of men, and to the will of God with regard to them. Ge 1:28. It is given; some individuals are capable of living with comfort and usefulness in an unmarried state, and may lawfully think it not best for them to marry; and some may be called for a time to perform special services, or meet special dangers, where they could not properly provide for a family. Such a case was noticed by Paul, 1Co 7.1, and applied to some who lived in his day, on account of the then present distresses. Individuals, in some peculiar circumstances, may find it expedient and useful to take a course which, were it not for those circumstances, would be both inexpedient and hurtful; and the great body of mankind may be required by the plainest dictates of God's word to take a different course.>
<19:12 So born; as to be unfit for marriage. Of men; by the wickedness of men, for their own selfish and ungodly purposes. Kingdom of heaven's sake; voluntarily abstaining from marriage in order to be more useful. Let him receive it; if a person is so situated as to be clearly an exception to what is applicable to ministers and men in general, and is disposed to live in an unmarried state because he believes that he can be more useful by so doing, let him so live; but let him cultivate the utmost purity of heart and life, and manifest it in all his conversation and conduct.>
<19:13 Little children; so small that Jesus took them up in his arms, put his hands on them, and blessed them. Mr 10.16. Luke calls them infants. Lu 18:15. Rebuked them; the disciples thought them too young to occupy the attention of Christ, or to be benefited by being brought to him.>
<19:14 Suffer little children; they have great need of me; they can be benefited by me; they are not beneath my notice, and I greatly delight in doing them good. Of such is the kingdom of heaven; see Mt 18:3, and note. Jesus Christ feels an interest in little children, and approves of their being brought to him in faith, love, and prayer. All parents should feel this, and thus bring them to the Saviour.>
<19:15 Laid his hands on them; in token of his blessing them.>
<19:17 Why callest thou me good? this question is asked because the young man addressed him simply as a human teacher, not as divine.>
<19:20 Have I kept; only in outward appearance, not in heart, as the sequel showed. A man may think he has always been good, and yet be entirely mistaken and totally unfit for heaven. He may sometimes feel uneasy, and be anxious to know what he must do to be saved; yet when told, he may not be willing to do it.>
<19:21 Be perfect; have a character that is "perfect and entire, wanting nothing." The Saviour, by this command, lays his hand immediately upon the faulty spot in his character, and points it out to him.>
<19:22 Sorrowful; by this he showed that he was not prepared for heaven. He loved his riches more than he loved his neighbor or God.>
<19:23 Hardly enter; it is with great difficulty that he can enter.>
<19:26 All things; God could make even a rich man humble, believing, and obedient, though men could not do it. It is possible for a rich man to go to heaven; but he will be much less likely to go there, than if he were not rich. Those, therefore, who are making it their chief object to be rich, are taking a course which tends for ever to destroy them.>
<19:28 In the regeneration; the reference here is not to the regeneration of the soul, for which the Greek commonly uses a different word, but to the time when God shall make all things new by bringing in the new heaven and new earth. 2Pe 3:13; Re 21:1,5. Sit upon twelve thrones; as assessors with Christ. Judging the twelve tribes of Israel; not authoritatively, for the final sentence belongs to Christ alone, but cooperating with him in his decisions. See note on 1Co 6:2. The world shall be acquitted or condemned according to the doctrines the appostles were inspired to preach.>
<19:29 A hundred-fold; shall receive vastly more real good in this world than all which he renounces for the sake of Christ, and in the world to come shall receive eternal life. For all the sacrifices which persons make from love to Christ and his cause, they will be graciously and abundantly rewarded, both in this world and in the world to come.>
<19:30 First--last; see Mt 20:16.>
<20:1 The kingdom of heaven; its comparative duties and rewards, as proclaimed in the judgment-day and inherited in heaven. See Mt 3:2>
<20:2 A penny; about fourteen cents. The proper compensation, at that time, for a day's labor.>
<20:3 Third hour; nine o'clock in the morning.>
<20:4 Till men begin to labor for Christ, they are, as to the great business of life, idle. They are doing nothing which will in the end promote their good.>
<20:5 Sixth and ninth hour; noon and three o'clock in the afternoon.>
<20:6 Eleventh hour; five o'clock in the afternoon.>
<20:13 No wrong; he gave them all that he agreed to, and all that justice and equity required.>
<20:15 Lawful; right, proper. What I will; what I see to be best. Evil; envious of others who receive as a free gift more than they can claim as their due. In the bestowment of his unmerited favors, Christ has a perfect right to do as he sees best. His doing this injures no one, and promotes the good of many. If any complain, they complain of infinite goodness under the guidance of infinite wisdom, and thus show that they are evil.>
<20:16 First; in privileges, and in their own estimation. Last; in the reception of the gifts of distinguishing grace. God distinguishes men by his favors as he sees best; not without wise and good reasons, but those reasons cannot always be seen by men. The above-mentioned truths were illustrated by the time of calling into his kingdom the Jews and the Gentiles, and his treatment of them; and they are continually illustrated in the dispensations of his providence and grace towards nations and families as well as individuals. Called; to enter the kingdom of heaven. Chosen; to enjoy its highest gracious benefits. Many who in this world are first in privileges, and are in many respects above others, will, in the world to come, be far below them.>
<20:18 Betrayed; Mt 26:49. Chief priests--scribes; ecclesiastical rulers. Mt 26:47.>
<20:19 Gentiles; the Romans, who were then the civil rulers of Judea, and who alone had authority to put an accused person to death.>
<20:20 Zebedee's children; James and John.>
<20:21 Sit--on thy right hand; be thy chief officers. In thy kingdom; thinking it would be an earthly kingdom. Parents often manifest much pride and ignorance in seeking worldly distinctions for their children, and Christ is wise and good in denying them those distinctions; for the attainment of them might for ever exclude them from true greatness and honor in his heavenly kingdom.>
<20:22 Ye know not; they did not understand the nature of his kingdom, and what they must do and suffer to be first in it. The cup--the baptism; the cup is that of suffering; Mt 26:39, etc. The baptism is his bloody death. The two together denote all the sufferings, inward and outward, through which our Lord's path to glory lay. We are able; in this they knew not what they said.>
<20:23 Ye shall drink; ye shall follow me in my sufferings. Not mine to give; except to those for whom it is prepared. Ambition in the disciples of Christ, and thirst for worldly glory, liken them to men of the world, not to Jesus Christ, and are evidence that disappointments and sufferings await them.>
<20:24 The ten; the other apostles were offended that some should seek to be placed above the rest.>
<20:26 No be so; in the kingdom of Christ: none of his ministers were to exercise civil power or authority over the rest. Great among you; to be truly great in his kingdom one must minister, as he has opportunity and ability, to the wants of others. Humility, and a disposition to serve others in the supply of their wants and the promotion of their highest good, are marks of true greatness in the kingdom of Christ, and especially among his ministers.>
<20:27 Chief; first in true spiritual worth. Your servant; most active in administering to the good of his fellow-disciples.>
<20:28 No to be ministered unto; his great object was, not to be assisted by others, but to assist them, and give his life for their redemption.>
<20:29 Jericho; a city about eight miles west of the Jordan, and twenty north-east of Jerusalem.>
<20:31 Rebuked them; commanded them to be silent. Cried the more; more loudly and earnestly, lest they should fail of the blessing. Christ kindly regards the sufferings of the distressed, and is pleased when they apply to him for help, believing that he is able and willing to bestow it. None who feel their need of his mercy, and desire to receive it, need fail of his grace.>
<21:1 Bethphage; a village on the south-east side of the mount of Olives, which was a hill about two miles east of Jerusalem, beyond the valley of Jehoshaphat. Through this valley ran the brook Cedron, or Kidron.>
<21:3 Say aught; say any thing against your taking them.>
<21:4 By the prophet; Zec 9:9. The prophecies of the Old Testament concerning the Messiah were all fulfilled in Jesus of Nazareth, thus proving with absolute certainty that he was the Christ.>
<21:5 Daughter of Zion; a poetic personification of Zion, which was that part of Jerusalem where David and the kings after him dwelt. It represents Jerusalem and its inhabitants. Behold, thy King cometh; this prophecy was universally understood of the Messiah; and thus Jesus openly claimed to be the one predicted by it. Sitting upon an ass; the common beast of the Israelitish rulers in ancient times, Jud 5:10; 10:4; and moreover a beast of peace, in contrast with the horse, which was specially employed in war. An ass, and a colt; Jesus rode upon the colt, Mr 11:7; Joh 12:14; the mother of the colt accompanying. Hence they are spoken of together by the evangelist.>
<21:8 Spread their garments; this was a royal honor, after the custom of the times.>
<21:9 Hosanna; Save now. An expression of joy, invoking blessings on him as the Messiah. Hosanna in the highest; let our hosannas on earth be responded to and ratified in the highest heavens.>
<21:12 Bought in the temple; the outer court of it, called the court of the Gentiles. Money-changers; those who exchanged the current coin of the day for the Israelitish half-shekel which was paid yearly for the support of the temple service. See note on Mt 17:24. For this they received a premium; and they were, moreover, often dishonest in their exactions. Sold doves; for the offerings in the temple. Le 14:22; Lu 2:24. Those who imitate Christ will manifest great zeal for God, and labor to remove all evils connected with his worship. The Bible will be their standard, and by it they will seek to regulate their own conduct and that of their fellow-men.>
<21:13 Written; Isa 56:7.>
<21:16 Read, Ps 8:2. The quotation was from the Septuagint, the Greek translation of the Old Testament, where the words "ordained strength" in the Hebrew, are translated "perfected praise." The conversion of children to the Saviour is foretold in the Scriptures. We ought therefore to seek and expect it, and when it takes place, to rejoice in it as a new evidence of the truth of the Bible and of the Messiahship of Jesus Christ.>
<21:17 Bethany; a village on the east side of the mount of Olives, adjoining Bethphage.>
<21:19 Let no fruit grow on thee hence-forward; the cursing of the fig-tree was a symbolical act designed to shadow forth the awful end of nations, communities, and individuals, that fail to bring forth fruit to God's praise. Compare Lu 13:6-9. In order to be accepted by Christ, it is not enough to have the leaves of an outward profession, or even the appearance of great fruitfulness; we must bear fruit. If we do not, we are exposed to his withering curse.>
<21:21 Have faith; the faith of miracles, which was given to the apostles, and by which they were enabled to work miracles in the name of Jesus. It shall be done; even the most difficult things, which are proper, shall be done. The spirit of this promise belongs to all God's servants in all ages. See note on Mt 17:20.>
<21:22 Believing; with the faith of miracles granted to them, and under the special teaching of the Holy Ghost, by which they could discern whether a thing was or was not according to the will of God. If they saw that it was not, they would not ask it, or believe that they should receive it. If they saw that it was, and did ask, believing, they would receive it, though it should be as difficult as the removal into the sea of mount Olivet, over which they were then passing.>
<21:23 These things; which he had been accomplishing at the temple and in other places. The authority of God, clearly and conclusively given, will not satisfy all, especially with regard to what they dislike. If a person have not their authority, or that of those with whom they associate, they will reject him, though he give unanswerable evidence of being sent of God.>
<21:25 The baptism of John; his ministration and teaching. Why did ye not then believe him? when he testified of me as the Messiah.>
<21:26 Many fear the people more than they fear God. It is often so with rulers: and truths which they reject, the common people receive gladly. The common people, the working people, should not be forbidden to read the word of God, and to think and judge for themselves; and when they learn what the truth of God is, they should not be hindered from obeying it.>
<21:28 Think ye; judge ye of what I am going to say in the following parable. The first; this represented the openly immoral and vicious, who publicly refused to obey God, such as publicans and harlots. Go work; this represented what God requires of men.>
<21:30 The second; this represented the scribes and Pharisees, who professed to obey God, and yet did not, but opposed him.>
<21:31 Openly wicked men are sometimes brought to repentance and salvation sooner than those who have been externally moral and professed a high regard for sacred things. Thus, those who appeared for a time to be first, are in reality last; and those who appeared to be last, are first.>
<21:32 Way of righteousness; the right way; the way of God's appointment. Believed him; though they had before refused to obey God, yet afterwards, under the preaching of John, they repented and obeyed him. Ye; the scribes and Pharisees repented not, and in what they had said condemned themselves.>
<21:33 Householder; this was designed to represent Jehovah. Husbandmen; the Jewish nation.>
<21:34 His servants; the prophets and teachers of religion among the Jews, who called upon them to render to God his due.>
<21:35 Beat one; this represented their treatment of his prophets.>
<21:37 His son; the Lord Jesus Christ, who was then addressing them, and whom they would kill.>
<21:39 Slew him; Mt 27:35.>
<21:40 The lord; the owner of all things. Those husbandmen; the Jews.>
<21:41 Other husbandmen; the blessings which they received and abused he would give to others. The greater the privileges of men, if they do not improve them, the greater will be their guilt; and without repentance and pardon, through faith in the Redeemer, the more awful will be their ruin.>
<21:42 In the scriptures; Ps 118:22,23. This was a prophecy of the truths and events represented in the parable, which were about to be fulfilled in the crucifixion of Christ, the casting off of the Jews, and the calling of the Gentiles into the church of God. The stone; the Lord Jesus Christ. The builders; the Jews, and especially their leaders, the scribes and Pharisees.>
<21:43 You; Jews. A nation; the Gentiles.>
<21:44 This stone; Jesus Christ. Whosoever should stumble at his lowly appearance, or the matter and manner of his teaching, would greatly suffer. It shall fall; those who against light should continue to oppose him, and on whom his righteous indignation should fall, would be destroyed with an awful destruction.>
<21:46 Sought to lay hands on him; for the purpose of destroying him, as the Scriptures, and as he in this parable, had foretold that they would.>
<22:1 Sought to lay hands on him; for the purpose of destroying him, as the Scriptures, and as he in this parable, had foretold that they would.>
<22:2 Kingdom of heaven; the Messiah's reign in the gospel dispensation, and the lot in the judgmentday, both of those who receive and those who reject it. See Mt 3:2. A marriage; a feast at the marriage of his son. This represented the blessings of the gospel. The provisions of God for the happiness of men are most abundant and free, and his invitations to them to come and receive according to their wants, are most urgent and sincere.>
<22:3 His servants; those ministers of his who were first sent to invite the Jews to embrace the Messiah. Would not come; representing their rejection of him and his salvation.>
<22:4 Other servants; other ministers, whom he afterwards sent.>
<22:6 Entreated them spitefully--slew them; this represents the treatment which the apostles and other ministers of the gospel received from the Jews.>
<22:7 Destroyed, those murderers; representing the destruction of Jerusalem, as foretold by our Lord. Lu 21:6-24. Though men have the power and the disposition, yet they have no right to reject the invitations of God, or to stay away from him and perish. It is a great dishonor to him, as well as a great wrong to themselves.>
<22:9 Highways; representing the offering of the gospel to the Gentiles and people of all descriptions.>
<22:11 Wedding-garment; it was customary for the man who made a wedding-feast, to provide wedding-garments for those whom he invited. If they would not come, or if they did come, but would not put on the wedding-garment, it was a great dishonor to the master of the feast. By this incident of the parable, our Lord shows that an outward acceptance of his gospel is not enough. We may join ourselves to the number of his visible followers; but if our souls have not the wedding-garment of faith, love, and holiness, we shall be cast out. A time is coming when God will examine into every man's character, and when those who have trusted to their own righteousness, without submission to or acceptance of the righteousness of Christ, will, with hypocrites and the openly vicious, be cast into outer darkness, where is weeping and gnashing of teeth.>
<22:12 Speechless; knowing that he was inexcusable.>
<22:13 Outer darkness; the darkness without the illuminated banqueting hall. The banqueting hall represents heaven with its joys; the outer darkness, hell with its anguish.>
<22:14 Called; invited to receive the blessings of the gospel. Chosen; by accepting its provisions to enjoy its benefits. None would accept the gracious invitations of the gospel, and be for ever blessed, if God had not from the beginning chosen them to salvation, through the sanctification of the Spirit and the belief of the truth. While those who reject Christ and are lost owe their destruction wholly to themselves, those who are saved are indebted for salvation to the riches of grace.>
<22:16 Herodians; they held that it was lawful to pay tribute to Cesar, or to the Romans, who had conquered and governed Judea. The Pharisees held that it was not, but was contrary to the divine law. De 17:15. Wicked men for evil purposes sometimes make great professions of respect to preachers of the gospel, and pretend to have an earnest desire to know the truth; yet when the truth is exhibited, as revealed by God, they reject it; thus showing that their real character was not that of sincere inquirers, but of objectors and hypocrites.>
<22:17 Tell us; settle the question so much disputed among us. Is it lawful to give tribute unto Cesar, or not? if he should say it was not lawful, they meant to accuse him to the civil authority as an enemy to the Romans; if he should say it was lawful, they meant to accuse him to the people as opposed to the law of God.>
<22:18 Perceived; he saw their hearts, their motives, as plainly as he did their faces.>
<22:19 Tribute-money; the Roman coin in which the civil taxes were paid.>
<22:21 Cesar's; this showed that as they were under his government, and enjoyed its protection, they ought to assist in supporting it; while, at the same time, they ought to obey God.>
<22:22 Marvelled; they were astonished at his wisdom in escaping their snare. Neither could accuse him, for he had maintained the rights of the government and the rights of God.>
<22:23 No resurrection; of the body after death. They denied any existence of the soul after death, and consequently any reunion of soul and body in a future resurrection.>
<22:24 Seed; children, who should be called after his brother, that no family in Israel might become extinct.>
<22:28 Whose wife; they thought the resurrection absurd, and that this difficulty would prove its absurdity. The difficulties which men bring forward, in order to show that doctrines revealed in the Bible are false, are often difficulties of their own making, and spring from ignorance of the Scriptures, not from these doctrines as God has revealed them.>
<22:29 The doctrines of the immortality of the soul and of the resurrection of the body were both taught in the Old Testament; as was also the obligation of men to love God with all the heart, and their neighbors as themselves. By Christ these truths were revealed with greater clearness, but the great requirements of both Testaments are substantially the same.>
<22:31 Toughing the resurrection; in proof from the Bible of the resurrection. Have ye not read, Ex 3:6,15. The manner in which God spoke of Abraham, Isaac, and Jacob, showed they were still alive, in opposition to the doctrine of the Sadducees. And as Jehovah was then the God of their living souls, he would in due time raise their bodies incorruptible and immortal.>
<22:33 Astonished; to witness his acquaintance with the scriptures, and the wisdom and justice of his application of them.>
<22:35 A lawyer; an expounder and teacher of the divine law.>
<22:36 In the law; the law of God.>
<22:37 Jesus said; De 6:5.>
<22:38 First; in importance, as it requires the duties we owe to God, which are the foundation of all true goodness.>
<22:39 The second; Le 19:18; requiring the duties we owe to men.>
<22:40 These two; they comprehend the substance of all that is required in the Old Testament scriptures.>
<22:42 The Bible, received as all given by inspiration of God, presents insuperable difficulties to those who reject the divinity of Christ, or consider him as possessing but one nature. They cannot rightly explain many things which, to those who embrace the truth, are perfectly plain.>
<22:43 In spirit; speaking under the guidance of the Holy Spirit. Ps 110:1.>
<22:44 Right hand; as sharing with me the throne of heaven. Thy footstool; till thou set thy feet upon thine enemies, as utterly vanquished.>
<22:45 How; how is Christ both David's Lord and son?>
<22:46 No man was able; the reason was, they did not rightly understand his character. If they had understood it, they could have answered. As man, he was David's son; and as God, he was his Lord.>
<23:1 No man was able; the reason was, they did not rightly understand his character. If they had understood it, they could have answered. As man, he was David's son; and as God, he was his Lord.>
<23:2 Sit in Moses' seat; they are the expounders and teachers of the law of Moses. Men may hold their first place as rulers and teachers in the visible church, and yet have no true religion, and they may show this by their conduct. But wicked examples, whoever may set them, should not be followed.>
<23:3 Observe and do; so far as they teach according to the laws of God; but beyond that, Do not ye after their works; do not imitate their example.>
<23:4 Heavy burdens; grievous and troublesome ceremonies and observances which they required. They rigidly expounded certain parts of the divine law as binding on the people, while they themselves, secretly or openly, claimed a release from them.>
<23:5 Phylacteries; slips of parchment worn about their persons, on which were written some divine precepts. The Pharisees made them broader than others, to intimate that they were more holy. For the same purpose they enlarged the borders or fringes which Moses had commanded them to wear on their garments. Nu 15:38.>
<23:6 Uppermost rooms; most honorable places at the table. The Jews of our Lord's day took their meals reclining on couches, which were arranged on three sides of a central table. In assigning the guests to their "rooms," or places, strict attention was paid to rank. Compare Lu 14:7-11.>
<23:7 Rabbi; master.>
<23:8 Brethren; equally children of God, and fellow-heirs of Christ; no one of you having authority to control the faith and practice of the rest.>
<23:9 Call no man your father; as having authority over your faith and practice. In matters of religion and conscience, ministers of the gospel cannot bow to mere human authority without giving to men what belongs only to God; and men who, on the ground of such authority, claim to be fathers and masters to their brethren, directing them what to believe and do, are antichrists, denying in practice the prerogatives of both the Father and the Son.>
<23:10 Masters; as leaders and controllers of Christ's ministers and people.>
<23:11 Servant; greatness in Christ's kingdom consists not in outward authority over others, but in the abundance of our labors and sacrifices for the welfare of our brethren. The greatest in the kingdom of Christ are those who most love him and their fellow-men, and are most ready to honor the one and do good to the other.>
<23:13 Shut up the kingdom of heaven; by your false interpretations of the law, and your opposition to me, its true expounder. Neither go in; they would not embrace Christ themselves, nor, if they could prevent it, would they suffer others to do it. The wickedness of the heart is so great, that it may lead men not only to reject Christ, but to make great efforts to induce others to reject him, and thus shut both themselves and others out of heaven.>
<23:14 Devour widows' houses; rob them of their estates. Therefore; on account of their hypocrisy. Greater damnation; more awful punishment.>
<23:15 Compass sea and land; make all sorts of efforts. Proselyte; convert to their religion. More the child of hell; more wicked.>
<23:16 Debtor; under obligation to keep his oath. Blind guides tamper with the conscience, make imaginary and futile distinctions between the guilt of different sins, passing over some lightly as if they were venial, or granting indulgences to commit them, and treating others no more wicked as deadly, while in all, self and sin are at the bottom; and those who lead, and those who follow, if they continue, will perish.>
<23:18 Guilty; if he does not fulfill his oath.>
<23:23 Pay tithe; devote a tenth part to the service of the temple. Mint and anise and cummin; herbs of small value. Weightier; more important. Judgment, mercy, and faith; justice to all, compassion to the needy, and piety towards God. To do justly, love mercy, and walk humbly with God, is a better evidence of true religion than all merely external observances; and scrupulous attention to little things, with neglect of great ones, is an indication that men are deceivers, or deceived.>
<23:24 Strain at a gnat; strain the liquid which you drink at the presence of a gnat in it, lest you should be made unclean by swallowing it. They reckoned the gnat among the unclean creeping things. Le 11:20,23 The reader will notice that the camel was also an unclean animal. The meaning therefore is, that they were very scrupulous about little things, while, without scruple, they committed great sins.>
<23:27 Whited sepulchres; sepulchres newly whitewashed, according to the custom of the country at certain periods.>
<23:29 Garnish; beautify; adorn; as if they had great regard for good men.>
<23:31 Witness--children of them which killed the prophets; by calling the murderers of the prophets "our fathers," they acknowledged themselves to be their literal children; and by imitating them in their deeds, they proved themselves their children in character.>
<23:32 Measure; the measure of their sins till wrath should come upon them.>
<23:34 Prophets; his apostles and other teachers of his religion. Ac 5:17,40; 7:59. The most awful denunciations of divine wrath against the wicked are perfectly consistent with the greatest kindness, the most tender compassions, and the most earnest desire that they should turn from their sins and live.>
<23:35 All the righteous blood shed upon the earth; they were about to murder the Son of God, and in so doing, to set as it were their seal and sanction to all the murders of good men before them. They would therefore be treated accordingly. Zacharias, son of Barachias; it is not certainly known to whom the Saviour refers. A probable opinion is, that Zechariah the son of Jehoiada is meant. See 2Ch 24:20-22. According to the arrangement of the Jewish canon, which puts the two books of Chronicles last, Abel is the first righteous man whose murder is recorded, and this Zechariah the last. Some think that Barachiah was another name borne by Jehoiada; others, that the reading should be Jehoiada instead of Barachias.>
<23:36 These things; the punishments due to their sins.>
<23:37 The reason why men are not saved is not that Christ is not able and willing to save them, nor that they are not under obligation to be saved, but that they will not come to him, or comply with the needful terms of salvation. Of course, if they perish, they will be their own destroyers, and the guilt will rest for ever on themselves.>
<23:38 Your house; their temple, which was soon after burned by the Romans, and remains desolate to this day.>
<23:39 Shall not see me hence forth; our Lord was now closing his personal ministry on earth. After his resurrection he showed himself not to all the people, but to chosen witnesses. Ac 10:41. They should never again enjoy his presence, till they were ready to receive him as their Messiah.>
<24:1 In the prophecy of this chapter, there is a double reference: first, to the destruction of the temple, and as connected with this the overthrow of the Jewish state and nation; secondly, to the end of the world. Both these events are included in the question of the disciples, ver 3, who seem to have connected them as inseparable from each other. The providential coming of the Son of man to destroy the city and temple, which was to be fulfilled before that generation had passed away, shadows forth, therefore, his more awful and majestic personal coming at "the end of the world." So far as the outward form of the prophecy is concerned, the first part is more occupied with the nearer event; the later, with the more distant. But it was not our Lord's purpose to reveal distinctly the separation of the two by a vast interval of time. The signs of the approaching catastrophy--wars, famine, pestilence, earthquakes, persecution, false prophets, etc.--were all fulfilled, as the history of these times shows, in respect to its nearer fulfilment in the destruction of Jerusalem. Another fulfilment remains for the last days. The darkening of the sun, moon, and stars, ver. 29, was fulfilled symbolically at the overthrow of the Jewish temple and city, this being a well-known emblem of revolutions and the fall of nations. See note on Isa 13:10, and the references. But it shall be literally fulfilled when heaven and earth shall pass away. The temple; this temple was built by the Jews after their return from the Babylonish captivity, and greatly enlarged and beautified by Herod.>
<24:2 Thrown down; expressive of the utter destruction which took place about forty years after.>
<24:3 It is a great privilege to be permitted to apply to Christ for instruction; for he can give us what we need, and in the best time and way.>
<24:5 False teachers abound in all ages, and seek in various ways to draw away disciples after them. We should not believe every spirit, but try the spirits whether they be of God. 1Jo 4:1. In order to do this, all should study the Scriptures, and compare what they hear with the word of God.>
<24:9 My name's sake; on account of their attachment to him. The hatred of men to Jesus Christ often shows itself in hatred to his people; and the manner in which men treat them, shows how, were he embodied and dwelling among men, they would treat him.>
<24:10 Offended; let to forsake him, and apostatize from his religion.>
<24:13 Endure; continue to obey Christ, notwithstanding all opposition. The sure and decisive test of friendship to Christ, in distinction from all counterfeits, is love to his character shown by perservering obedience to his commands.>
<24:14 All the world; all the countries then known.>
<24:15 The abomination of desolation; commonly understood of the eagles of the Roman standards, regarded as objects of idolatrous worship. Stand in the holy place; encamped about Jerusalem. Whoso readeth; Da 9:27; 12:11. Let him understand; that the destruction foretold by the prophet Daniel more than five hundred years before, is now about to be accomplished.>
<24:16 Flee into the mountains; to save themselves, and prevent their being taken by the Romans.>
<24:17 House-top; the tops of houses were then made flat, and persons often sat, walked, prayed, took their meals, and spent their nights upon them. To take any thing; but flee by the shortest way, and in the quickest manner.>
<24:19 Woe; on account of the increased difficulty of fleeing.>
<24:20 Winter--sabbath-day; because it would then be more difficult to escape. God's arrangements for the future are not so fixed that it is improper for us to pray that we may be favored in escaping from evils, and obtaining needed good. But while we pray, we must act; for in answering prayer, God encourages action, not idleness--the discharge of duty, not the neglect of it.>
<24:21 Then shall be great tribulation; great distress. It is stated that eleven hundred thousand were slain, and in the neighborhood two hundred and fifty thousand more. Ninety-seven thousand were sold into perpetual bondage, and multitudes perished by famine, pestilence, and cruel treatment.>
<24:22 Those days; days of distress. No flesh be saved; all the covenant people would perish. Elect's sake; those whom God had chosen to be his people. God orders the dispensations of providence and the manifestations of grace with special reference to his people; and in such a manner as shall secure their salvation.>
<24:24 If it were possible; this implied that it was not possible.>
<24:26 He is in the desert; that is, the Messiah is there.>
<24:27 So shall--the coming of the Son of man be; it shall be so public that all must see it.>
<24:28 Wheresoever the carcass is, there will the eagles be; wherever the Jews are, the Romans will be upon them, as eagles are upon their prey; the eagle being the ordinary standard of the Roman armies.>
<24:29 Shall the sun be darkened; on the twofold reference of these words, see the introductory note to the chapter. From this point onward the form of the prophecy has more immediate reference to Christ's final coming, yet not so as to exclude its earlier fulfilment. The language which Christ used to describe his coming in his providence to separate the righteous from the wicked at the destruction of Jerusalem, and the end of the Jewish commonwealth, was designed and strikingly adapted to carry our minds forward to his coming at the end of the world, when before him shall be gathered all nations, and he shall separate them one from another, as a shepherd divideth his sheep from the goats. Chap Mt 25:32.>
<24:30 The sign of the Son of man in heaven; the sign of his speedy coming. Shall see the Son of man; fulfilled in a lower figurative sense when Christ came providentially to destroy the Jewish city and nation: to be fulfilled in the highest sense at his final personal coming. The same is true of the gathering together of his elect, mentioned in the following verse.>
<24:34 This generation shall not pass--be fulfilled; that is, in the nearer event foretold. See the introductory note to the chapter.>
<24:36 That day; the day of "The coming of the Son of man," ver Mt 24:37; 2Ti 1:12,18. Christ did not tell them when it would be, and none but God know. He warned them to be prepared for it, and to be always ready.>
<24:37 Noe; the Greek method of spelling Noah. Ge 7:1.>
<24:40 Taken; and saved as a follower of Christ. Left; to perish through unbelief and rejection of him.>
<24:42 Ye know not; this was true with regard to the destruction of Jerusalem. It is also true with regard to each one's death; and it will be true with regard to the day of judgment. The day of our death, and of our being called to judgment, though known to God, is not revealed to us, that we may always be found in the path of duty, and thus, through grace, be prepared for those great events which are certain and near.>
<24:44 Ready; for the coming of your Lord, in whatever way.>
<24:45 Made ruler; given him the care of providing for his family.>
<24:51 Cut him asunder; the reference is to the punishment of cutting or sawing asunder. The meaning is, he shall punish him with awful severity.>
<25:1 Then; at the time when the Son of man shall come, as foretold in the preceding chapter. The object of this parable is to show that as we do not know when Christ will come, we should so live as to be always ready. Its highest reference is to his final personal coming; but this does not exclude lower references, as that of his particular coming to each individual at death. To meet the bridegroom; when he went, according to the custom of the age and country, to fetch home his bride by night. Men of very different characters here live together, make similar outward professions, and join in the same employments; but at death the difference between them will be manifest and great.>
<25:3 They that were foolish--took no oil with them; so that the flame of their lamps could not hold out. These correspond to those "sown on stony ground," who "endure but for a time." Mr 4:16,17.>
<25:4 Took oil; to feed their lamps. These agree with those "sown on good ground," who "bring forth fruit with patience." Mr 4:20, Lu 8:15.>
<25:8 Gone out; more literally, going out, for want of oil to recruit them. This shows the consternation of those who are not ready to meet Christ at his coming. It is not safe to rely on outward professions as evidence of piety, nor upon any thing which does not imply supreme love to God and real good-will to men.>
<25:9 Not so; no believer can furnish grace for his fellow-men. This comes only from God.>
<25:10 The bridegroom came; representing Christ as coming before the wicked are ready. The door was shut; all opportunity of preparing to meet Christ ceases at death.>
<25:12 I know you not; as my friends.>
<25:13 Watch therefore; this was the practical application which Christ made of the parable.>
<25:14 The kingdom of heaven; the object of this parable was to show, that as all our blessings come from God, we are accountable to him, and should so use them as to meet his approbation.>
<25:15 His several ability; representing the various gifts which God bestows on different individuals. All our blessings we receive from God, and to him we are justly accountable for the use of them. He does not give the same to all, and he requires only according to what a man hath. Of course, no one will be condemned for not having received more.>
<25:16 Traded with the same; showing the good improvement he made of what had been given him.>
<25:18 Hid his lord's money; where he would have no trouble in taking care of it, while he ought to have traded with it for his lord's profit. This shows that a failure to improve our opportunities to do good is a heinous sin which Christ will severely punish.>
<25:19 The lord of those servants; Jesus Christ.>
<25:20 Those who employ the blessings which God bestows in his service and to his glory, will have their blessings greatly increased.>
<25:21 Make thee ruler over many things; advance thee to a higher station in my service. The principle here stated is perpetually illustrated in Christ's dealings with his servants in this world, but will have its highest fulfilment at the resurrection of the just.>
<25:23 Good and faithful servant; he receives the same reward as the servant to whom five talents had been entrusted; because it is not the amount of talents given, but the faithful use of them, that Christ regards.>
<25:24 Not strewed; not scattered seed. This showed that he had no love to his lord, no disposition to honor him, or even to be just towards him.>
<25:26 Thou knewest; this may be taken as an argument out of the servant's own mouth: Didst thou know? Then thou oughtest, etc.>
<25:27 Exchangers; answering nearly to our brokers or bankers. They were persons who dealt in money. Usury; interest. Lawful and proper increase was the meaning of this word when our translation of the Bible was made, not unlawful interest, as it means now.>
<25:28 Take therefore; as he would not rightly use what he had, he could no longer have it. Those who refuse to employ what God gives them in his service and to his glory, will soon have all their blessings removed, and no more will be given to them for ever.>
<25:29 Every one that hath; a disposition to rightly use the blessings which God gives, is a preparation for more and greater blessings. That hath not; he that hath not this disposition, when God calls him to account, will be deprived of all blessings, and for his unfaithfulness will be punished as he deserves.>
<25:31 Come in his glory; at the day of judgment.>
<25:32 Separate them; according to their character.>
<25:33 Sheep; the righteous. Goats; the wicked.>
<25:34 The kingdom; of endless, heavenly glory.>
<25:37 When saw we thee; humility astonished at high honor put on poor services.>
<25:40 Have done it unto me; expressive of the intimate and endearing union of Christ and his people. Jesus Christ considers himself to be treated by men as they treat his known disciples. And from the manner in which we treat them, we may learn the manner in which we treat him.>
<25:45 Ye did not to me; men who neglect the wants of Christ's people, neglect him.>
<25:46 Everlasting; this is the same word in the original which in the next line is translated eternal, and means the same thing, endless duration. The day of judgment will be one of surpassing interest. The amazing contrast between the appearance of Christ, as he discoursed to his disciples on the mount of Olives, and his appearance when he comes in his glory and the glory of his Father, with his mighty angels--when before him are gathered all nations, and he separates them one from another, saying to one class, "Come, ye blessed of my Father, inherit the kingdom;" and to the other, "Depart, ye cursed, into everlasting fire," and they go away to meet him no more--will be such as we can now but faintly conceive. The clearness with which Christ makes known what will be the future state of the righteous and the wicked, and the liability of all to be at any time fixed in heaven or hell for eternity, should lead each one, without delay, to prepare to obtain the one and escape the other.>
<26:1 These sayings; his discourse to his disciples contained in the two preceding chapters, in answer to their inquiries. Mt 24:3.>
<26:2 After two days is the feast of the passover; this was a feast of the Jews, kept annually from the 15th to the 21st of their month Abib, answering in part to our April, in commomoration of God's passing over the houses of the Israelites, and not entering in to slay their first-born, when he slew the first-born of the Egyptians.Ex 12:1-30. The Son of man is betrayed to be crucified; as our Lord Jesus was the true sacrifice prefigured by the paschal lamb, it was proper that he, the great Antitype, should die at the passover, when the lamb that typified him was slain.>
<26:3 No advantages will of themselves overcome the opposition of the human heart to Jesus Christ; and rulers are often more wicked than the people, seeking by subtlety and deceit to effect that which they cannot or dare not openly accomplish.>
<26:6 In Bethany; this was six days before the passover. Joh 12:1. The evangelist therefore goes back a little in his narrative.>
<26:7 There came unto him a woman, Mary, the sister of Lazarus whom Jesus had raised from the dead Joh 12:3. Alabaster; a kind of stone carved into ornamental and costly boxes, or vases, especially for perfumes. Ointment; perfumes, liquid or more solid. Sat at meat; reclined on a couch, as the custom then was at meals, leaning on the left elbow.>
<26:8 When his disciples saw it--To what purpose is this waste? in Joh 12:4, Judas Iscariot is named as the author of this remark, which seems to have been seconded by some of Jesus' disciples. Compare Mr 14:4. The part taken by Judas in this transaction may have been the reason why it is related here in immediate connection with his treachery. That which is employed in expressions of love to Christ by his sincere and devoted followers, is often thought by others to be wasted and lost. But in the view of Christ it is well used, and he will see that it receives a gracious and an honorable reward.>
<26:10 She hath wrought a good work; she had expressed her sincere and ardent love to her Saviour.>
<26:12 For my burial; it was customary to anoint the body, or embalm it with spices and ointment, preparatory to burial. So he says this might be considered as preparatory to his burial.>
<26:13 A memorial of her; in remembrance of what she had done.>
<26:15 Thirty pieces of silver; that is, thirty shekels, the sum at which a servant's life was estimated. Ex 21:32. It was about fifteen dollars.>
<26:17 Unleavened bread; this was a term applied to the passover, because during that feast they used what was not leavened, or fermented. The passover; the paschal lamb slain on that occasion.>
<26:18 The city; Jerusalem. My time; his time to eat the passover and to close his life was near.>
<26:19 Made ready the passover; prepared the lamb and other things, according to the appointment of God. Ex 12:3-17.>
<26:24 Goeth; to death, the death of the cross. Written; Ps 22.1; 41:9; Isa 53:4-9; Da 9:26,27. Good for that man; on account of the endless woe which his sins would bring upon him.>
<26:25 Thou hast said; this was equivalent to saying, "Yes, it is you.">
<26:26 This is my body; the emblem, or representation of my body. As it is said of God, De 32:4. "He is the Rock;" not literally a rock, but in some respects like one--firm, stable, and unchanging. So, Ge 41:26, "The seven good kine are seven years;" not literally, but they represent seven years. So, Joh 15:5, "I am the vine, ye are the branches;" not literally, but represented or illustrated by the vine and its branches. So with the declaration, "This is my body." Christ did not design to teach his disciples that he was then breaking his own body, and that they were then eating it. His body was alive and unbroken: the disciples knew that what they ate was bread, not flesh. Besides, Matthew does not say that Jesus took his body and broke it, and said, Take, eat: but he took bread, and brake it: and it was bread: and "This is my body" means, it represents my body. The literal meaning of the words of the Bible is not always the true meaning. For instance Christ said, "Ye must be born again," Joh 3:7; meaning, not that a man must enter a second time into his mother's womb and be born, but that he must experience a change in his moral and religious character, called passing from death unto life. Joh 5:24. So, when he said, "Except ye eat the flesh of the Son of man, and drink his blood, ye have no life in you," Joh 6:53, the Jews, understanding it literally, said, "How can this man give us his flesh to eat?" Taking it literally, no wonder they thought it strange. He therefore let them know that he did not mean that they must literally eat his flesh; and that, should they do it, it would profit them nothing. The words that I speak unto you, saith he, are spirit and life. They have a spiritual, and not a literal, carnal meaning; they are designed to convey a knowledge of spiritual truths, the right understanding and due reception of which will promote the spiritual life of men.>
<26:29 Will not drink henceforth--drink it new with you; he means to say, "The time for my drinking with you the literal fruit of the vine is over; the kingdom of God is about to be established"-- this was by his propitiatory death, resurrection, assension, and the outpouring of his Spirit--"henceforth I will drink with you the spiritual new wine of the gospel"--new because it belongs to a new dispensation--"in my Father's kingdom." This was fulfilled to the apostles in a special sense, in the extraordinary measure of Christ's presence and Spirit bestowed upon them as his earthly instruments in establishing his kingdom. Compare Lu 22:29,30. It is fulfilled to all believers in a lower sense, in the communion they have with Christ in his earthly church; and it shall have a perfect fulfilment to all his true disciples in his heavenly kingdom.>
<26:31 Offended; made to stumble. The word is here, as elsewhere in the New Testament, applied to the life and conduct. How the disciples should be offended appears in the course of the narrative; they should be led to forsake him, and in one case to deny him. It is written; Zec 13:7. The Shepherd; Christ. The sheep; his disciples. Seasons of intimate and endearing communion with Christ may be followed by seasons of great and peculiar trial. It is a great blessing that Christ is acquainted beforehand with all our trials, and can in the best way provide for them; so that they shall in the end not injure us, but promote our benefit.>
<26:32 Into Galilee; Mt 28:7.>
<26:34 Deny me; Mt 26:74.>
<26:35 Though I should die; his meaning was, that he would not deny Christ even to save his own life. Persons may seem to be very courageous in time of safety, and be great cowards in time of danger. When a good man thinks highly of himself, and is confident in his own strength, he is preparing for a downfall; and if he does not lose his soul, he will have reason to be grateful.>
<26:36 Gethsemane; a garden, or retired spot, on the west side of the mount of Olives, not far from the brook Cedron.>
<26:37 Two sons of Zebedee; James and John. The same that were with him on the mount of transfiguration. Chap Mt 17:1,2. Very heavy; exceedingly depressed.>
<26:38 Even unto death; with death-like sorrow, inexpressible anguish.>
<26:39 This cup; the anguish he was enduring and was to endure.>
<26:40 Unto Peter; who had just before been so strong in his professions of fidelity to his Master.>
<26:41 That ye enter not into temptation; that is, so as to be overcome by it; for the hour of the Saviour's suffering was also to be that of the fiery trial of his disciples. Compare Lu 22:31. The spirit; "the inward man." This was animiated by sincere love towards the Saviour. The flesh; used here to denote all that weakness of fallen nature which makes it liable to be overcome by temptation.>
<26:42 The sufferings of Jesus Christ, when he bore our sins in his own body on the tree, and tasted death for every man, were inexpressibly great. The Father said, "Awake, O sword, against my Shepherd, and against the man that is my fellow: smite the Shepherd." Zec 13:7. His soul was made "an offering for sin," and as such it "pleased the Lord to bruise him" and "put him to grief." Isa 53:10.>
<26:45 Sleep on--take your rest; see note to Mr 14:41.>
<26:51 One of them; Peter.>
<26:53 Twelve legions; a Roman legion varied in number in different ages. In our Saviour's time it seems to have consisted of six thousand men. The idea is, vast multitudes.>
<26:54 Scriptures; Ps 22:1,18; Isa 53:3-10; Da 9:24-26; Zec 13:7. It must be; in order to fulfil the scriptures, and finish the work of redemption.>
<26:56 Forsook him; this was what Jesus meant by their being offended, and what he had foretold in verse Mt 26:31.>
<26:58 Afar off; on account of his fear of danger. See the end; the end of the trial, and its results.>
<26:60 Found none; they found none that agreed in their testimony, or that could make out an accusation that had even the appearance of weight.>
<26:61 Destroy the temple of God; this was false, both in the words used and in the meaning which they put upon them; he spoke of his body, and of rising again in three days. Joh 2:19-22.>
<26:63 I adjure thee; he thus put him under oath to answer truly.>
<26:64 Thou hast said; said the truth, I am the Christ. Sitting--coming; this was claiming divine power and honor.>
<26:65 Rent his clothes; as a token of great indignation. Blasphemy; in claiming to be the Messiah, the Son of God, and the judge of men.>
<26:66 Guilty of death; of a crime which deserves death. Le 24:16.>
<26:68 Prophesy; they had previously covered his face, Mr 14:65, and in derision of the omniscience which he claimed, they called upon him to designate the persons who smote him.>
<26:70 I know not; I do not understand what you mean. No past privileges or attainments can be relied on for future or even present support. God must give us day by day our daily supply of wisdom, goodness, and strength, or we shall have none. "Hold thou me up, and I shall be safe;" guide me, and I shall go in the right way.>
<26:73 Thy speech; thy peculiar manner of speaking. Betrayeth thee; showeth thee to be a Galilean.>
<26:74 Curse; imprecate God's curse if he uttered falsehood. Swear; declare with an oath.>
<26:75 The word of Jesus; verse Mt 26:34. Wept bitterly; on account of his sin, in denying his Lord. If a good man sin he will repent, mourn bitterly over his transgressions, and turn from them unto God. He has an advocate with the Father, Jesus Christ the righteous. One look from Him will lead the penitent disciple to loathe himself, and to admire, adore, and trust in the Redeemer. His life will be holy; his death safe, if not peaceful and triumphant; and his eternity blessed.>
<27:1 Put him to death; he had professed to be the Christ, and said that hereafter they should see him coming in the clouds of heaven. This they said was blasphemy, and for it they condemned him to die.>
<27:2 Pilate; the Roman governor. As the Jews had no authority to put any one to death, it was needful, in order to accomplish their design, that the Roman governor should condemn him. They brought him to Pilate for this purpose; and Christ had foretold that the Gentiles, to whom Pilate belonged, would put him to death. Mt 20:19.>
<27:3 Repented himself; he knew that Jesus was innocent, and yet he had been instrumental in delivering him to his murderers. He was therefore tormented by a conviction of his guilt. The evil of committing known sin is greater than sinners imagine; while the pleasure which they derive from it is less, and is nothing compared with the pain which it will give them. The commission of one sin not only opens the door and prepares the way for the commission of others, but leads to consequences which the sinner little imagines, and the contemplation of which fills him with horror. A deep conviction of sin, and the most pungent distress on account of it, full confession of guilt, and readiness to return the wages of iniquity, may all exist without true repentance, without any love to God and holiness, or any preparation for heaven.>
<27:4 What is that to us? the language of men hardened in wickedness, and determined to execute their murderous purpose. Tempters to iniquity are hard-hearted and cruel; they will get men into trouble, but will not relieve them. They are of their father the devil, and like him they will tempt men to sin, and then torment them on account of it.>
<27:5 There are two kinds of sorrow on account of sin. One is in view of its having been committed against God: this is godly sorrow, which worketh repentance unto salvation, and needeth not to be repented of. The other is sorrow on account of the evil to which sin exposes the sinner, and is wholly selfish. This is the sorrow of the world, which worketh death. 2Co 7:10. The sorrow of Peter exemplified the one, and that of Judas the other.>
<27:6 Treasury; the place where the offerings or contributions of the people were kept. Hypocrites and formalists are sometimes exceedingly scrupulous about little things, while they commit the greatest and most aggravated transgressions without hesitation, and without remorse. While practising the grossest deception, and even killing the friends of God, they profess to be doing him service.>
<27:7 Potter's field; a place where earthen-ware had been made. Strangers; those who came from other countries, and died at Jerusalem.>
<27:8 This day; the time when Matthew wrote his gospel, perhaps thirty years after the events here recorded.>
<27:9 Jeremy; Jeremiah. The event here referred to is not mentioned in Jeremiah, but in Zec 11:12,13. The name Jeremiah in several ancient translations and manuscripts is not mentioned, and it reads, that which was spoken by the prophet.>
<27:12 Slander and abuse, reviling and persecution may sometimes be best met by silence: and perfection of character requires such a control over one's own spirit as to be able not to speak, when it is manifest that no good will result from it.>
<27:18 Envy; the uneasiness they felt in witnessing his increasing influence among the people. The indulgence of envy is a great sin.>
<27:19 Nothing to do with that just man; do nothing against him, or towards condemning him.>
<27:24 All efforts of unrighteous magistrates to screen themselves from guilt in knowingly condemning the innocent or acquitting the guilty, will be fruitless. They may deceive themselves and their fellow-men, but they cannot deceive God. He will hold them responsible; and the measures they take to hide their guilt will only increase their wickedness, and aggravate their condemnation. It is a fearful thing to incur the guilt of blood. When unrighteously shed, it rises to heaven for vengeance. Pilate was stripped of his authority, and died in exile, it is said by his own hand. The state of the Jews for eighteen hundred years shows that the guilt of shedding the blood of Christ was awful, and that God is just.>
<27:25 His blood be on us; we will bear the blame of his crucifixion: if divine judgments come, let them come on us and our children.>
<27:26 Scourged Jesus; according to the custom of scourging one condemned to die before his execution.>
<27:28 Scarlet robe; in mockery of his claim to be the king of the Jews; for a purple or scarlet robe was the ensign of sovereignty.>
<27:29 A reed; in mockery, as a sceptre.>
<27:32 Cyrene; a city in the northern part of Africa. Bear his cross; at first the cross had been laid on Jesus, according to the custom of compelling malefactors to carry their own cross to the place of execution.>
<27:33 A place called Golgotha; without the city. Heb 13:12. Golgotha means a skull; but why the place bore this name is not agreed.>
<27:34 Vinegar--mingled with gall; Mark names "wine mingled with myrrh," an intensely bitter substance. Though it may have been intended as a stupefying draught, it still belonged to the process of a bloody, ignominious, and agonizing death. Thus was fulfilled the prophecy in Ps 69:21. In persecuting the Saviour, accusing and condemning him; in giving him vinegar and gall to drink, parting his garments among them and casting lots on his vesture, and never ceasing to afflict him till he gave up the ghost, all concerned were free and accountable; and though doing it with wicked hands and wicked hearts, they were fulfilling the predictions of God, and thus proving that Jesus was the Messiah.>
<27:35 Casting lots; to determine which soldier should have the garment. Ps 22:18.>
<27:36 Watched him; this was customary, to see that none should come and take down those who were crucified till it was ordered.>
<27:38 Two thieves; thus he was numbered with the transgressors, according to Isa 53:12.>
<27:40 Destroyed the temple; their false accusation, Mt 26:61; Joh 2:19-21.>
<27:42 King of Israel; the Messiah.>
<27:44 Cast the same in his teeth; they upbraided him in the same way.>
<27:45 Sixth hour; twelve o'clock noon. Ninth hour; three o'clock in the afternoon.>
<27:46 Eli, Eli, lama sabachthani; a phrase in the Chaldaic language, as then spoken in Judea, explained in the last part of the verse. Ps 22:1.>
<27:50 Yielded up the ghost; gave up his life. Jesus Christ retained his life and endured his agony till he had finished the work which God gave him to do. He suffered all that was needful in order to become the author of eternal salvation to all who believe on him. He then voluntarily gave up his life.>
<27:51 Veil of the temple; which separated the most holy place from the other parts of the temple. By this was signified that now the way into God's presence was opened by the blood of Christ. Heb 9:7,8; 10:19,20. The rending veil of the temple, the quaking earth, the bursting rocks, the opening graves, and the rising dead, all testified to the greatness of the event of the Saviour's death; and heaven and earth seemed to sympathize with their expiring Lord.>
<27:52 Which slept; who were dead. Arose; not now, but, as is added in the next verse, after his resurrection. These were the earnest of the general resurrection at the last day. The whole transaction was designed to show that through the resurrection of Christ his disciples shall also attain to a glorious resurrection.>
<27:53 The holy city; Jerusalem.>
<27:54 Centurion; the Roman officer in command on that occasion. This was the Son of God; as he had professed to be. The object of God in suffering the wickedness of men, is totally different from theirs in committing it. They mean it for evil, and it is evil, and they are evil in committing it; and without repentance and forgiveness, they will be for ever punished as evil-doers. But God means to overrule it, and he will overrule it for good. In all that he suffers to be done, as well as in all that he does himself, he is good. Joseph, Mordecai, Daniel and his three friends, and Christ and his disciples, are all striking illustrations of this great and momentous truth. Chap. Mt 28:18; Ge 50:20; Es 7:10; Da 3:20,30; 6:16,28.>
<27:57 The even; evening--some time after three o'clock. Sometimes there are friends of Christ among the rich, and in circumstances where we should least expect them. They may be associated with the wicked, and yet through grace be kept from being partakers of their sins. On the other hand there may be hypocrites, and the basest of men, connected with the people of God. It is not wise or safe to judge of bodies of men by certain individuals who belong to them--to approve or condemn the whole on account of the character and conduct of a part.>
<27:60 His own new tomb; thus Christ according to prophecy, was "with the rich in his death." Isa 53:9.>
<27:62 Day of the preparation; for keeping the passover. Men cannot by any efforts thwart the purposes of God. A man's heart deviseth his way, but the Lord directeth his steps. Pr 16:9. There are many devices in a man's heart; nevertheless the counsel of the Lord, that shall stand. Pr 19:21.>
<27:64 Until the third day; this shows what was then meant by three days, or three days and three nights, which Jesus said he should be in the heart of the earth, or the grave. Mt 12:40. The last error; that of taking him away, and then pretending that he was risen from the dead. This they said would be worse than his pretending to be the Messiah.>
<27:65 A watch; soldiers to watch his grave.>
<27:66 Sealing the stone; so that no one could open the sepulchre without breaking the seal. All the efforts of the Jews to show that Jesus Christ was guilty only tended more clearly to show, and more strikingly to illustrate, his innocence and their own guilt; and all their efforts after he was dead to prevent his resurrection, only tended more clearly to demonstrate that he had risen. So God taketh the wise in their own craftiness, and the counsel of the froward is carried headlong. Job 5:13.>
<28:1 In the end of the Sabbath; after the Sabbath. As it began to dawn; at break of day. The other Mary; Mary the wife of Cleophas, and mother of James the less, or younger, and Joses. The other James was the son of Zebedee, and brother of John.>
<28:2 There was; there had been before the arrival of the women.>
<28:4 No soldiers are so intrepid, but that a single angel can cause them to quake with fear, and become as dead men. He can even strike dead a hundred and eighty-five thousand in a night. Isa 37:36.>
<28:5 The angels of the Lord excel in strength, and whether for judgment or mercy, they do his commandments, hearkening unto the voice of his word. Ps 103:20. Safe then amidst all their trials are his people, to whom angels are ministering spirits, sent forth by him to minister to the heirs of salvation. Heb 1:14.>
<28:6 The Lord; of angels as well as men. The evidence is conclusive, that while Jesus Christ died for our sins according to the Scriptures, on the third day he rose again for our justification according to the Scriptures. Ro 4:25.>
<28:9 Held him by the feet; fell at his feet and embraced them. And worshipped him; Joh 5:23. 9, 17. While no holy man or angel ever suffered himself to be worshipped, Christ received divine worship, and never said any thing against it, nor has God the Father, or the Holy Spirit. On the contrary, it is the distinguishing trait of true believers, that they invoke his name, and serve the Lord Christ. Ac 9:14, Col 3:24. In doing this, they follow the direction, chapter Mt 4:10, "Worship the Lord thy God, and him only shalt thou serve;" and the direction, Heb 1:6, "Let all the angels of God worship him.">
<28:10 My brethren; his disciples. He still calls them brethren, though in the hour of his distress they had deserted him.>
<28:11 The watch; the soldiers who had been appointed to watch the sepulchre, and see that his disciples did not come and steal him away.>
<28:12 Taken counsel; in what way they could prevent the knowledge of his resurrection, and thus keep the people from receiving him as the Messiah.>
<28:13 The Jews did not deny the resurrection of Christ for want of evidence to prove it, nor did they hire the soldiers to tell a lie because they believed it or could substantiate it; but only to keep the people from knowing the truth. False teachers are afraid to trust the people with the means of knowledge, or to have them become acquainted with facts. They do not wish to have them examine, think, and judge for themselves, but to have them leave this to their teachers, who wish thus to rule over them.>
<28:14 Persuade him; not to punish them for sleeping on duty, which by the Roman law was death.>
<28:15 This saying; that the disciples came by night and stole him away. Until this day; the time when Matthew wrote this gospel.>
<28:17 Worshipped him; as the Son of God and the Saviour of men. Some doubted; whether his resurrection was real.>
<28:18 All power; power is here used in the sense of authority. Is given unto me; as mediator, God and man. As Christ has authority over all, and power to direct and govern all, they who put their trust in him will be for ever safe.>
<28:19 Teach; disciple all nations; proclaim to them the gospel, for the purpose of persuading them to become my disciples. The Father--the Son, and--the Holy Ghost; the one only living and true God. The making of all nations the disciples of Christ should be the great object of all. Some should labor for it in one way, and some in another, as the Lord shall call them. But all should strive together that the Scriptures may be translated into every tongue, and the gospel be preached to every creature.>
<28:20 I am with you; in this work, to guide, comfort, sanctify, and sustain you; to render you successful in awakening the attention of men, convincing them of sin, and turning them from darkness to light, and from the power of sin unto God. I will be with you and all who succeed you in preaching the gospel, to the end of time. Amen; so let it be and so it shall be. Amen. Christ, with his divine presence and aid, will be with his people in doing his will, to the end of time; and after having inclined and enabled them to serve him and their generation according to the will of God, will receive them to himself, that where he is they also may be, to behold his glory, the glory which he had with the Father before the world was. Joh 17:24.>
\kniha{Mark}
\zkratka{Mark}
<1:2 In the prophets; Isa 40:3; Mal 3:1; Mt 3:3, 11:10.>
<1:3 3-8. John the Baptist. Mt 3:1-12. For the reception of spiritual blessings, preparation is needful; and those things which tend to hinder men from feeling this, and making preparation, should be carefully avoided.>
<1:7 The more men receive of the illuminating and purifying influences of the Holy Spirit, the more humble will be their views of themselves, and the more exalted their views of the Redeemer.>
<1:9 9-11. Jesus baptized. Mt 3:13-17.>
<1:12 Driveth him; constraineth or inclineth him. The same word, in Mt 9:38, is translated "send forth.">
<1:13 No situation in this world is free from temptation; and in solitude men are often more exposed to it than in company. There may be solicitations to evil, and no inclination to comply with them; and thus men may be strongly tempted, and yet not commit sin. Resistance of temptation may increase their holiness, and better fit them for the duties of life. Jas 1:2,3; 1Pe 1:6,7.>
<1:14 Put in prison; Mt 14:3.>
<1:15 The time is fulfilled; the time for the coming of the Messiah, as predicted. Da 9:24-27.>
<1:16 16-20. Disciples of Christ called. Mt 4:18-22, Lu 5:1-11,27,28>
<1:18 Those whom Christ calls to preach the gospel, should forsake whatever would hinder them; and though they relinquish their prospects of temporal gain on earth, they may expect, if faithful, eternal gain in heaven.>
<1:22 As one that had authority; Mt 7:29.>
<1:24 To destroy us; he speaks in the name of himself and the other demons. The Holy One of God; the Messiah.>
<1:25 Hold thy peace; the demons everywhere recognized Jesus as the Messiah, but he uniformly commanded them to hold their peace. It was neither the time to proclaim his Messiahship, nor were they the proper heralds.>
<1:26 Torn him; convulsed him. Luke adds that he "hurt him not," Mr 4:35.>
<1:27 New doctrine; it was not merely the new revelations of truth that Jesus made which excited their astonishment, but also the new manifestations of divine power that accompanied it. With authority; in his own name, and with sovereign power: he commanded, and they obeyed.>
<1:30 Simon's wife's mother; Mt 8:14,15. Whatever may be the maladies of body or soul, of ourselves or our friends, there is encouragement to apply to Jesus Christ for relief, and no difficulties are so great that he cannot remove them.>
<1:34 They knew him; they knew that he was the Messiah, but he did not wish them to proclaim it. See note on verse Mr 1:25.>
<1:35 Early rising, for the purpose of engaging in secret prayer before entering on the duties of the day, is the dictate of true wisdom, and is highly conducive to health, excellence, usefulness, and enjoyment.>
<1:37 All men; this is a specimen of the manner in which the word all is sometimes used in the Bible, meaning, not literally every individual, but very many, as in verse Mr 1:5.>
<1:38 Therefore came I forth; that he might preach the gospel in various places.>
<1:40 If thou wilt; Mt 8:2-4. This was an acknowledgment of his divine power.>
<1:41 I will; this was the claiming and exercising of divine power.>
<1:43 Straitly; strictly.>
<1:44 Say nothing to any man; about the cure. See note on Mt 8:4. Show thyself to the priest; Le 14:2. This would show the priest that the cure was real, and give to him, as well as others, evidence that Jesus was the Messiah.>
<1:45 Blaze abroad; openly and publicly proclaim it. Could no more; this shows the manner in which could, could not, and other words denoting ability or inability, are sometimes used in the Bible--referring not to natural power, but to difficulties which stand in the way, and the disposition of a person to encounter and overcome them. It was said of Joseph's brethren, Ge 37:4, that they "could not speak peaceably unto him." This was for want of disposition, not of power.>
<2:1 Blaze abroad; openly and publicly proclaim it. Could no more; this shows the manner in which could, could not, and other words denoting ability or inability, are sometimes used in the Bible--referring not to natural power, but to difficulties which stand in the way, and the disposition of a person to encounter and overcome them. It was said of Joseph's brethren, Ge 37:4, that they "could not speak peaceably unto him." This was for want of disposition, not of power.>
<2:2 The word; the word of God, the truths of the gospel.>
<2:3 Borne of four; carried by four men. Mt 9:2-8.>
<2:4 Press; the crowd of people. Uncovered the roof; the roofs of the houses were then flat, and the sick man could be let down from them into the presence of Jesus.>
<2:5 Their faith; their confidence in his willingness and power to heal. Sickness is often the means of leading men to feel their need of divine help; and application to Christ, with strong confidence in him, is the way to obtain it.>
<2:8 Perceived in his spirit; by his knowledge of their hearts.>
<2:9 The manner in which Jesus Christ, when on earth, performed miracles, showed that he was able to forgive sins, and of course was truly divine.>
<2:10 Hath power; authority, right, and ability.>
<2:11 Thy bed; the small couch on which he lay.>
<2:12 On this fashion; they never before saw any one who could thus cure the palsy.>
<2:13 Seaside; the sea of Galilee. Mt 4:18. Hope of temporal blessings will often draw together multitudes of people; and when they are assembled, ministers of the gospel, if they have fit opportunity, should address them on the superior value of spiritual blessings, and point out the way to obtain them.>
<2:14 Levi; the same as Matthew, Mt 9:9. It was common among the Jews to have two or more names.>
<2:15 15-17. Christ eats with publicans. Mt 9:10-13.>
<2:16 Kind social dealings at proper times, with all sorts of persons, in order to do them good, is essential to the highest excellence and the greatest usefulness; and none are so exalted, that they ought to think it beneath them.>
<2:17 From all the occurrences of life we should endeavor to draw important instruction, and as we have opportunity, should communicate it for the benefit of others.>
<2:18 18-22. Disciples fasting. Mt 9:14-17.>
<2:23 23-25. Plucking the ears of corn. Mt 12:1-4.>
<2:26 Abiathar the high-priest; in the days of Abiathar, who was afterwards high-priest. It appears from 1Sa 21:1-6, that Ahimelech was high-priest when David ate the show-bread. But Abiathar his son shortly after succeeded him, and was high-priest when David was king.>
<2:27 The Sabbath was made for man; at the creation, Ge 2:2,3, for his benefit and happiness. Not man for the Sabbath; it is not, by superstitious observance, to be perverted to a denial of the just claims of mercy. The day is to be kept in such a manner as God has shown to be best suited to make men holy, and fit them for that rest which remains for his people. Heb 4:9. As the Sabbath was made for the whole human race, they have a right to its rest and privileges. This right does not come from men, but from God, and its exercise is essential to their present and future good. It should therefore be highly prized and faithfully used, according to his command. Ex 20:8.>
<2:28 Therefore; because the Sabbath was made for man. The argument is from the design of the Sabbath. Since it was made for man's good, the Son of man, who is God in human nature, who has come to redeem man, and who has all things pertaining to man's good in his own hands, must also be the Lord of the Sabbath. Let the reader compare this passage with Mt 12:1-8, and see how the argument continually rises. First, the Saviour justifies his disciples from an exceptional case, that of David when he was hungry; secondly, from the standing custom of profaning the Sabbath in its outward letter by the preparation of sacrifices, etc. Mt 12:5; thirdly, from the design of the Sabbath; finally, from his own character and office, as God come in human nature to redeem man. As Jesus Christ is Lord of the Sabbath, and the day belongs to him, he has a right to direct as to the time and manner of observing it. Those who devote it to worldly business, travelling, or amusement, or who spend it in idleness, are guilty of robbing the Saviour, and expose themselves to his curse.>
<3:1 1-5. The withered hand. Mt 12:9-13.>
<3:2 Hypocrites and persons who are guilty of great wickedness, are often disposed to find fault with and condemn the friends of God.>
<3:4 It is lawful; which was most proper: to do good, as Jesus contemplated, or to do evil, as the Pharisees intended? to save the man's life by removing his disease, or to leave him to die? They held their peace; no wonder, for they could not answer without condemning themselves.>
<3:5 With anger; holy indignation, just displeasure against their sins, and grief on account of them. Indignation at the sins of men is perfectly consistent with the deepest compassion for their souls; and no opposition or danger from the wicked should hinder us from doing them good, as we have opportunity.>
<3:7 The sea; the sea of Galilee. Judea; the southern and more thickly settled part of the country.>
<3:8 Jerusalem; the chief city. Idumea; that is, the land of Edom, which was south of Palestine, and was settled by the descendants of Esau. During the Babylonish captivity, they took the south part of Palestine as far as the city of Hebron. This part of the country was afterwards called Idumea, and it is to this that Mark refers. Beyond Jordan; the east side of that river. Tyre and Sidon; Mt 11:21.>
<3:11 Unclean spirits; the persons whom evil spirits possessed. Their prostration of themselves before Jesus, and their acknowledgment of him as the Son of God, are ascribed to the unclean spirits, because these acts were done under their impulse.>
<3:12 Not make him known; not proclaim him as the Messiah, because the proper time for this had not yet come, nor were they the proper heralds. See note on Mr 1:25.>
<3:13 13-19. The apostles chosen. Mt 10:1-4.>
<3:20 Not so much as eat bread; they had no time for their regular meals.>
<3:21 His friends; his relations. Lay hold on him; constrain him to retire from the multitude and take rest. Beside himself; deranged, because, in their view, he in his labors exceeded all reasonable bounds. That earnestness in the service of God, and that activity and perseverance in doing good which true religion inspires, appear to many to be indications of insanity, and awaken in them solicitude; while equal earnestness in the pursuit of worldly things awakens no such apprehensions, but is viewed with approbation.>
<3:22 22-27. Casting out devils by Beelzebub. Mt 12:24-30.>
<3:28 All sins shall be forgiven; their sins are pardonable. They may repent, and on repentance and faith in Christ, receive forgiveness.>
<3:29 Hath never forgiveness; he has done despite to the Holy Ghost, the author of all grace. He will never have grace to repent, believe on the Saviour, and receive pardon, but will die impenitent, and perish. See Mt 12:32.>
<3:30 He hath an unclean spirit; they said, he is possessed of the devil, and through Satanic influence works these miracles. Thus, by ascribing the work of God's Spirit, Mt 12:28, to Beelzebub, they blasphemed against the Holy Ghost.>
<3:31 31-35. Christ's brethren. Mt 12:46-50.>
<4:3 3-9. Parable of the sower. Mt 13:1-9. Natural objects were designed, and should be used, to illustrate and enforce spiritual truths; and the providences of God are a striking commentary on his word.>
<4:7 It is not enough to be excited under preaching, or in reading the Scriptures or the works of pious men, or to be much engaged in religion on the Sabbath. The influence of the Sabbath must be carried through the week. Men must be governed by the will of God in their business, as well as in their religious duties; and if need be, sacrifice property, ease, reputation, and even life itself, to honor him.>
<4:10 10-13. Speaking in parables. Mt 13:10-17. In the communication and reception of saving knowledge, human agency is needful; and would men be wise unto salvation, they must improve their opportunities to hear and understand divine truth.>
<4:11 Mystery of the kingdom; the deeper truths of the gospel, which had not before been revealed. Them that are without, Mt 3.2; without the circle of his disciples. These remained in ignorance through the hardness of their hearts, and their rejection of the light. This made it proper that the Saviour should instruct the multitude by parables, into the meaning of which the candid and teachable would inquire, and thus be made wise to salvation, while the careless and indifferent would neglect them.>
<4:12 Not perceive; because they do not desire to know the truth. Not understand; because they do not, in the right way, use proper means. Thus they are not converted or turned from their evil ways, and their sins are not forgiven.>
<4:13 Know ye not this parable? which is so plain and obvious. The words contain a gentle reproof for their dulness.>
<4:14 14-20. Parable of the sower explained. Mt 13:18-23.>
<4:21 Is a candle brought; spoken here of the candle of Christ's teachings, lighted in the souls of his disciples that they may let the light of their knowledge shine on others. Jesus Christ does not impart knowledge to men that they may keep it to themselves, but that they may impart it for the benefit of their fellow-men.>
<4:22 Nothing hid; a candle is not lighted to be hid, or to shine only on itself, but to give light to men. So Christ's instructions were explained to his disciples, not for their benefit merely, but to be by them communicated for the good of others.>
<4:23 Let him hear; let him who has opportunities improve them, not only for his own sake, but for the sake of his fellow-men.>
<4:24 With what measure--measured to you; the measure of sincere and earnest attention which you give to my instructions, will be the measure of knowledge which will be given back to you. If men do not improve their opportunities to obtain divine knowledge and prepare for the purity and bliss of heaven, these opportunities will soon cease, and they will be left in endless darkness and woe.>
<4:25 He that hath; hath such a desire for divine knowledge as rightly to improve his opportunities, shall increase it. Hath not--shall be taken; if he has no desire to improve his opportunities, they will be taken away, and their benefits be lost.>
<4:26 Kingdom of God; the reign of Christ in the hearts of men. Should cast seed, Mt 3.12; the seed is the good word of God sown in the heart, and made fruitful by God's grace. No one should be discouraged in efforts to do good, because he does not at once see the fruit of them. Let him go seasonably to rest at night, rise betimes in the morning, and spend each day in learning and doing the will of God, and God will make him useful.>
<4:27 Should sleep, and rise night and day; should sleep by night and rise by day. The seed does not come suddenly to maturity, but by a gradual process, while he who sowed it pursues his ordinary course of labor and rest.>
<4:28 Of herself; by the power which God gives, not man. First the blade; small shoot. Then the ear; the stalk and head. Full corn; the kernels full grown.>
<4:29 Putteth in the sickle; he gathereth the fruits of his labor. Men in this matter are workers together with God. One plants, another waters or cultivates, and God gives the increase. So with the rise and progress of religion in the soul. Men must hear, understand, believe, and obey it. The power which leads them to do this, is of God. To illustrate still further the progressive nature of his religion, its great increase from a small beginning, he spoke the parable of the mustard-seed.>
<4:30 30-32. Parable of the mustard-seed. Mt 13:31,32.>
<4:33 As they were able to hear it; as they were able to understand and profit by his instructions. Much evil may be done and much good be prevented by an untimely communication of truths which men will only misunderstand, pervert, and abuse. Time and manner demand attention, and call for wisdom as well as goodness, discretion as well as courage.>
<4:34 Without a parable spake he not; in his public instruction of the multitude. Expounded all things to his disciples; that they might in due time explain them to others.>
<4:35 The other side; of the sea of Galilee.>
<4:37 37-41. Christ stills the tempest. Mt 14:23-33.>
<4:40 No faith; why is it, after all you have seen and heard, that you have not such confidence in me as to prevent your fear? That course of Christ in his providence which sometimes leads his people to think that he cares less for them than they do for themselves is designed to show them their unbelief, and that what they want is confidence in him, to walk by faith, and not by sight; remembering that as the heavens are higher than the earth, so are his ways higher than their ways, and his thoughts than their thoughts.>
<5:1 The other side; the east side of the sea of Galilee.>
<5:2 A man; Matthew mentions two demoniacs. Mark mentions but one, and describes his case more fully; probably because it was the more remarkable. Evil spirits are active, and have great influence in the affairs of men. Men may be tempted to disbelieve this, yet all have reason to be sober and vigilant, and steadfastly to resist their adversary the devil, who goeth about as a roaring lion, seeking whom he may devour. 1Pe 5:8,9. 2-20. Legion of devils. Mt 8:28-34; Mt 26:53.>
<5:6 Worshipped him; bowed down before him in acknowledgment of his authority and power.>
<5:7 I adjure thee; this was said by the evil spirit, through the mouth of the man.>
<5:9 Legion; for the number of the Roman legion, see note on Mt 26:53. The word is here used simply in the sense of a multitude. We are many; the man speaks, under the influence of the evil spirits, in behalf of all of them. So in the following verse, and Mr 5:12.>
<5:12 All the devils; Luke says many devils were entered into him. Lu 8:30.>
<5:15 Sitting, and clothed, and in his right mind; this was evidence that the evil spirits had gone out of him, and that he was cured.>
<5:17 Him; Jesus. Men under the power of evil spirits oppose Jesus Christ, and wish him to depart from them. Covetousness leads men to treat him in the same way. Mt 8:34.>
<5:18 Prayed him; besought Jesus that he might accompany him.>
<5:20 Decapolis; or the land of the ten cities; a country lying east of the river Jordan, but including also Scythopolis and its territory on the western side. None should be afraid or ashamed to acknowledge their indebtedness to Jesus Christ, and at proper times, to make known what he hath done for them, that he may be honored, and that others may apply to him for help.>
<5:22 22-43. Jairus' daughter restored to life. Mt 9:18-26.>
<5:30 Virtue; healing power.>
<5:34 Thy faith hath made thee whole; this is a specimen of the manner in which the Bible speaks of the effect of means when rightly used; it is designed to encourage men thus to use them. Though pardon and salvation come to us through the Redeemer, and his work is the meritorious ground on which we receive them, yet the exercise, on our part, of faith in him, is the appointed means of obtaining them.>
<5:35 The Master; Jesus Christ.>
<5:36 Only believe; believe that I am able to restore her to life, and to do what I will.>
<5:39 Not dead, but sleepeth; her death, though real, is yet like sleep, in that she shall soon wake to life again.>
<5:40 Those who have no faith may scoff at the idea that Jesus Christ is almighty, and able to supply all the wants of his people; they may mock at a reliance on his constant and all-sufficient aid; but in due time his people will find that their most exalted expectations are more than realized.>
<5:41 Talitha-cumi; these were two words in Syro-Chaldaic, the language in which Christ spoke, meaning, Damsel, arise.>
<5:43 That no man should know it; that they should not publish the particulars of this cure, the time for greater manifestation of himself not having come.>
<6:1 1-6. His own country; Nazareth. Mt 13:54-58.>
<6:3 The carpenter; Jesus, before he began his ministry, seems to have wrought at the employment of a carpenter. Mt 13:55.>
<6:5 Could there do no mighty work; because of their unbelief, as is added by Matthew, Mt 13:58. This is an instance of the manner in which the words can and cannot are sometimes used in the Bible: he could not consistently, or with propriety; there do many mighty works. He healed a few, but not many; not because he had not power, but for other reasons. In order to understand correctly the meaning of words which speak of ability and inability, as used in the Bible, we must consider the subject about which they were spoken, the connection in which they are found, and the manner in which the speaker and writer used them.>
<6:7 7-11. The twelve apostles sent out. Mt 10:5-15; Lu 9:1-6.>
<6:8 Christ sends out his ministers under circumstances which are suited to teach them their dependence on him--that all their power to do good and accomplish the objects for which he employs them, comes from himself.>
<6:11 More tolerable for Sodom and Gomorrah; because the inhabitants of those cities did not sin against as great light as did those who rejected the apostles.>
<6:14 Herod; Herod Antipas, the son of Herod the Great. Heard of him; Jesus. 14-30. John the Baptist beheaded. Mt 14:1-12.>
<6:16 A guilty conscience awakened, forebodes dreadful evils; and transgressors never can enjoy permanent peace unless they repent, and believe with the heart on Him whose blood cleanseth from sin. 1Jo 1:7.>
<6:17 He had married her; Herod, as we learn from Josephus, had rejected his own wife to marry the wife of his brother Philip while he was still living.>
<6:20 Observed him; rather, as the margin, "kept him," namely, from the resentment of Herodias. Did many things; he did many things to which John urged him, but he would not put away his brother's wife. To show reverence towards God's ministers, and do many things gladly at their suggestion, avails nothing for the salvation of the soul while the sin which God's law forbides is cherished and persisted in.>
<6:21 When a convenient day was come; a day suitable for the purpose of Herodias, who was watching her opportunity to destroy John. The dancing of her daughter before Herod and his lords was probably a part of the plan suggested by her.>
<6:30 The apostles gathered themselves together unto Jesus; upon their return from their mission, described in verses Mr 6:7-13.>
<6:31 A desert place; a place less frequented, that they might be more retired. Occasional retirement from the tumult of the world is needful for all men, especially for ministers of the gospel. They need to commune much with their own hearts and with God, that by wisdom and strength derived from him in private, they may be better fitted for their public duties.>
<6:34 As sheep not having a shepherd; destitute of teachers who cared for their souls and were able to teach them the truth. 34-44. Five thousand fed, near the shore of the sea of Galilee. Mt 14:15-21.>
<6:45 45-52. Christ walking on the sea of Galilee. Mt 14:22-33.>
<6:52 Ministers of Christ, notwithstanding all the displays of his power and grace, have much remaining unbelief and hardness of heart. They need the constant influences of his Spirit, and should be watchful and prayerful, lest, after having preached to others, they themselves should be cast away.>
<6:53 Gennesaret; a small, fertile, and beautiful region on the west side of the sea of Galilee, which is thence called the lake of Gennesaret. Lu 5:1. 53-56. The sick healed. Mt 14:34-36.>
<6:55 It is not enough that we come to Christ ourselves; we should be active in bringing our fellow-men to him.>
<7:1 1-23. Traditions of the scribes and Pharisees. Mt 15:1-20.>
<7:4 Tables; the word in the original signifies couches, on which they were accustomed to recline at meals.>
<7:5 Formal and hypocritical teachers of religion are prone to add to the commands of God traditions and ceremonies of their own, and to be very anxious that men should observe them, while they neglect his appointments, and connive at, if they do not encourage, similar neglect in others.>
<7:6 Esaias; Isaiah. Isa 29:13-16.>
<7:8 Human additions to the word and worship of God tend to lessen the influence of divine institutions, and should be carefully avoided.>
<7:11 Corban--profited by me; what might have gone to thy maintenance is Corban, that is, consecrated as a religious gift to the service of the sanctuary.>
<7:14 To understand divine things, men must hearken diligently to the teachings of Christ in his works, his word, and his providence, and seek of him habitually the illuminating and purifying influences of his Spirit.>
<7:18 Men may take any kind of healthful food without spiritual defilement, whether human traditions allow it or not. If they acknowledge the goodness of God in giving it, and seek his blessing upon it, they may expect that it will promote their good.>
<7:19 Entereth not into his heart; does not reach or pollute the soul.>
<7:24 The various ways in which the different evangelists describe the same transaction, show that they did not copy one from the other. Each gives a true account, and relates those circumstances which impressed his own mind under the teaching of the Holy Ghost. 24-30. The Syrophenician woman. Mt 15:21-28.>
<7:26 A Greek; that is, a Gentile. Syrophenician; belonging to Syrophenicia, that is, the Syrian Phenicia, so called to distinguish it from the Libyan Phenicia, on the north coast of Africa.>
<7:27 Let the children; God's covenant children, that is, the Jews. First be filled; the gospel was first to be offered to the Jews, and to them our Lord's personal ministry on earth was chiefly restricted. See notes on Mr 7:29 and Mt 10:5,6.>
<7:29 For this saying--is gone out; though our Lord's mission was "to the lost sheep of the house of Israel," Mt 15:24, yet he always honored personal faith in himself wherever found. Mt 8:5-13.>
<7:35 The string of his tongue; more literally, the band of his tongue, meaning that which hindered its use.>
<8:1 1-9. Four thousand fed. Mt 15:32-39.>
<8:2 In following Christ, his people may, for a time, be destitute even of the necessaries of life. But he is never unmindful of their wants, and in due time he will supply them.>
<8:10 Dalmanutha; Matthew says he came into the coasts of Magdala. These two places were near together, so that either might be mentioned with equal propriety.>
<8:11 11-13. A sign sought. Mt 16:1-4.>
<8:12 Sighed deeply; on account of their wickedness. Pious men are grieved at the deceit and hypocrisy of the wicked; and earnestly desire and fervently pray that by forsaking their sins and turning to God, they may be prepared for heaven.>
<8:14 14-21. The leaven of the Pharisees. Mt 16:5-12; Pr 19:27.>
<8:15 The leaven of Herod; that is, of the Herodians, his partisans. Though the Pharisees and Herodians disagreed in their political opinions, they agreed in being actuated by corrupt worldly principles and a hypocritical spirit, which are here called their leaven.>
<8:17 Hardness of heart and blindness of mind are often found to a great extent in the disciples of Christ; and were it not for his continual intercession, and the rich blessings of his Spirit, they would fall away and perish.>
<8:23 Although Jesus Christ can bestow favors instantaneously, and without the use of means, he often sees it best to employ means, and to grant his favors gradually, that those who receive them may better understand his character, and more wisely improve the blessings which he gives.>
<8:24 As trees, walking; he saw men walking, but could distinguish them from trees only by their motion. He did not see them clearly.>
<8:25 Put his hands again--saw every man clearly; why the Saviour did not heal this man by an instantaneous act, as in so many other cases, we are not informed. One reason may have been, to shadow forth the gradual process by which, through his word and Spirit, he removes spiritual blindness from men's heart.>
<8:26 The town; Bethsaida, where so many of his mighty works were done. Mt 11:21.>
<8:27 Cesarea Philippi; a town in the north part of Galilee, and near mount Hermon. Philip the tetrarch greatly enlarged it, and called it Cesarea in honor of Tiberius Cesar. Philippi was added to distinguish it from another Cesarea which lay on the Mediterranean sea. Mt 16:13. 27-38. Christ foretells his death, and reproves Peter. Mt 16:13-28.>
<8:32 Openly; publicly and more plainly than he had done before.>
<8:33 Savorest not; thinkest not. Thy thoughts and those of God do not agree. Compare Isa 55:8,9. Those who think that some other course would be better than that which Christ takes, savor not the things which be of God, but those that be of men. This was often the case with Peter, and it showed, that notwithstanding all Christ had done for him, he was very liable to err.>
<8:34 Take up his cross; in allusion to the practice of compelling malefactors to bear their own cross to the place of execution. The meaning is, that he must make any sacrifice, submit to any self-denial, and encounter any difficulty which may be needful, in order to obey Christ's commands.>
<8:35 Save his life--lose it; the word life is here used in two senses: first, for the bodily life; secondly, for eternal life.>
<8:37 In exchange for his soul; as the price of its redemption. If his soul be lost, there is no price which he can pay to redeem it. It must be lost for ever.>
<8:38 Ashamed of me; ashamed to be my follower. When he cometh; at the day of judgment. If men would be owned of Christ in the day of judgment, they must be governed by his will, must not be afraid or ashamed to acknowledge him before men, and must perserveringly obey his commands.>
<9:1 The kingdom of God come with power; the gospel established, and rendered mightily efficacious to the salvation of men. Mt 3:2; 16:28. The assurance of the speedy triumph of the Redeemer is a source of great encouragement to his people, and prepares them for all needful labors, hardships, and sacrifices in his cause.>
<9:2 2-10. The transfiguration. Mt 17:1-9.>
<9:7 The great business of men is to hear the instructions of Christ, especially those which relate to his sufferings and death, and so to act as to influence as many as possible to believe on him, to the salvation of their souls.>
<9:10 What the rising from the dead should mean; though Christ had clearly foretold his resurrection from the dead, his disciples appear not to have understood, or not to have believed it.>
<9:11 11-13. Elias must first come; Mt 17:10-13.>
<9:12 Restoreth all things; the word "restore," used also in Mt 17:11, is taken from the Septuagint version of Mal 4:6: "Who [Elias] shall restore the heart of father to son," etc.: that is, bring them back to their former state of union in God's service. For the meaning of this prophecy, see note on Mal 4:6. And how it is written; the coming of Elias fulfils the prophecy concerning him, and also brings in its train the accomplishment of the sufferings predicted of the Son of man.>
<9:14 Questioning; disputing or debating with them. 14-29. The dumb spirit cast out. Mt 17:14-21.>
<9:15 Were greatly amazed; it has been supposed that a portion of the supernatural brightness of the Saviour's countenance on the mount of transfiguration yet remained.>
<9:17 Whatever calamities come upon children, it is the privilege and duty of parents to apply for them to the Saviour; and all their difficulties, however grievous or long-continued, he can remove.>
<9:23 There is often an important connection between the faith of parents and the blessings which Christ bestows on children; and never in this world will children fully know the benefits which their parents, through earnest application to, and strong faith in the Redeemer, have been instrumental in procuring for them.>
<9:24 Help thou; teach me to believe more fully thy willingness and power to help.>
<9:30 30-32. Christ foretells his death. Mt 17:22,23.>
<9:33 33-37. Who are greatest. Mt 18:1.>
<9:34 Held their peace; they were silent; ashamed, no doubt, as men always have reason to be when they contend which shall be the greatest. Desire of preeminence is a besetting sin even in ministers of the gospel. It is an evidence of wordly-mindedness which their Lord observes and highly disapproves, however unobserved by men.>
<9:38 In thy name; in professed reliance on thy power. He followeth not us; he did not with them attend on the Saviour. Those who think that a man cannot be useful because he does not follow them, and who are therefore disposed to hinder his doing good, differ greatly from Jesus Christ. And if Christ works by his servants in overcoming the power of evil, and exterminating wicked propensities and habits, even if forbidden by Christians, they should not, on this account, suspend their labors, or lessen their efforts for the good of men.>
<9:39 Forbid him not; the principle which the Saviour here lays down is one of wide application. When a man is laboring in Christ's cause with His manifest presence and blessing, forbid him not because he does not in all things agree with you, or is not of your party.>
<9:41 41, 42. Whosoever shall give you a cup of water to drink in my name--whosoever shall offend; the charge respecting the man that followed not with the disciples, naturally led the Saviour to speak of the great preciousness in God's sight of deeds of kindness and love towards his disciples, especially the lowly among them, and the great sin of offending them.>
<9:42 42-48. Warning against offences, or occasions of sin. Mt 18:6-9.>
<9:43 Offend thee; lead thee to commit an offence. The immediate reference here is to offences against Christ's little ones, whereby they are let into sin. The hand, the foot, and the eye represent men's strongest desires and the earthly objects dearest to them. Whatever sacrifices the doing of the will of God may require, it is wise cheerfully and promptly to make them; for the trouble it will occasion in this world is nothing to the misery which the neglect of it will occasion in the world to come.>
<9:48 Where their worm dieth not, and the fire is not quenched; language borrowed from Isa 66:24, where the carcasses of God's enemies are represented as devoured by worms that never die, and fire that is never quenched. This terrible imagery teaches that in hell the misery of the wicked will never end.>
<9:49 For every one shall be salted with fire--salted with salt; for the right understanding of this verse, the following particulars should be noted: first, the whole verse is better taken as a comparison, thus: For every one shall be salted with fire, as every sacrifice shall be salted with salt; secondly, the introductory word "for", as well as the terms used, shows that there is a reference backward to a salting with the fire of hell; thirdly, the words immediately following, "salt is good," "have salt in yourselves," make it clear that the present verse includes also the salting of God's Spirit. The meaning, then seems to be this: Allow yourselves to be salted with the fire of God's Spirit, [which includes the fire of affliction and severe self-denial,] or you will be salted with the fire of hell. In the former case men are living sacrifices, acceptable to God, seasoned with the salt of divine grace, as the Levitical sacrifices were seasoned with literal salt, Le 2:13; in the latter case, they are sacrifices to God's wrath. Men must, by the Holy Spirit, through trials, the discipline of Providence, and the word of truth, be purified from sin in this world, or remain under its power, and suffer its consequences for ever in the world to come. Heb 12:14; Re 22:10-15.>
<9:50 Wherewith will ye season it? the man from whose soul the salt of God's grace has perished, is fit only to be salted with the fire of his wrath. Compare Mt 5:13. Have salt in yourselves; secure the preserving influences of divine grace, that you may be kept henceforward from contests for superiority and from all evil, and live in harmony and peace.>
<10:1 1-12. Divorcement. Mt 19:1-12.>
<10:4 God sometimes suffers things to take place which are violations of his laws, and gives directions suited to lessen in some measure the evils of those violations, while men wickedly continue to indulge them. This, however, is not to be interpreted as if he approved of those violations, or did not require that they should be done away.>
<10:11 Shall put away; privately, without just cause, and without due form of law. Committeth adultery; if a man could not marry another, after he had unjustly put away his wife, without committing adultery, he could not do it before he had put her away. And as adultery was always forbidden, polygamy of course was forbidden.>
<10:12 She committeth adultery; as really as the husband did in the other case. Neither has a man right to have two wives, nor a woman two husbands.>
<10:13 Children from their earliest years need the blessing of Jesus Christ, and he is greatly pleased with those parents who feel this, and bring them to him, in prayer and faith, that they may receive it. 13-16. Christ blessing children. Mt 19:13-15.>
<10:14 Jesus saw it; saw that the disciples disapproved of children being brought to him for his blessing. Of such is the kingdom of God; both in this world and in heaven. Mt 3:2.>
<10:15 Receive the kingdom of God; submit to the guidance and government of Christ with the humble and docile spirit of a little child. Mt 3:2.>
<10:17 17-22. The rich young man. Mt 19:16-22.>
<10:21 Then Jesus--loved him; with that natural affection which good men feel towards amiable youth who are correct in their deportment, though destitute of true religion. One thing thou lackest; that one thing was supreme love to God. Persons may be amiable, kind, and moral in their deportment, and yet not be in heart truly pious. Such persons may at times feel anxious for their salvation, and yet not be willing to make the sacrifices and perform the duties which the gospel requires.>
<10:22 Grieved; that this great sacrifice was required of him. Thus his idolatrous love of wealth was immediately revealed.>
<10:23 23-31. Danger of riches. Mt 19:23-30.>
<10:24 Them that trust in riches; by these words the Saviour explains the difficulty that lies in the way of a rich man's salvation, which is the extreme danger that he will trust in his riches; a danger against which nothing but the abundant grace of God can guard him.>
<10:25 Great riches vastly increase the difficulties in the way of a man's salvation; and so long as the possessor trusts in them for happiness, his salvation is impossible.>
<10:26 Out of measure; greatly, exceedingly.>
<10:27 With God all things are possible; he can bring even a rich man to rennounce his dependence on riches, and to trust in the living God. As God is able to show rich men that their wealth belongs to him, and that their happiness here, as well as in the future world, requires them to devote it to his service, all should pray that God will lead them to do this, and thus honor him and promote their own good and that of their fellow-men.>
<10:29 For my sake, and the gospel's; from attachment to me and my cause. No one makes sacrifices or performs labors in obedience to Christ, and for the purpose of honoring him, without receiving great benefits in this world, and greater in the world to come.>
<10:30 A hundred-fold; blessings a hundred-fold greater than was the sacrifice he is called to make.>
<10:31 First; in the enjoyment of outward privileges and blessings. Last; in Christ's honor, because they have not improved these privileges. Mt 20:16.>
<10:32 Went before them; as their leader. The words indicate the firmness and alacrity with which he went to the sacrifice of himself on the cross. They were amazed; at the calmness and intrepidity with which he went up to Jerusalem, when he knew the rage and malice of his enemies. They were afraid; on account of the dangers to which they were exposed.>
<10:35 James and John; their mother, as appears from Matthew, spoke for them. 35-45. Request of Zebedee's sons. Mt 20:20-28.>
<10:37 Those who hope to be great in the kingdom of Christ by being exalter to worldly authority and power, will be sadly disappointed. Their seeking greatness by these means shows that they are governed by the spirit of this world, not by the spirit of Christ.>
<10:41 Much displeased; the apostles were displeased at the attempts of the sons of Zebedee to obtain superior rank; and Christ showed them, with much plainness, that it was not his will that such rivalry, or even such preeminence among them, should exist.>
<10:42 They which are accounted to rule; who have the title of rulers.>
<10:43 The way to be great in the kingdom of Christ is open to all; and all who take this way, and perseveringly pursue it, will obtain the prize.>
<10:46 46-52. Blind Bartimeus. Mt 20:29-34.>
<10:50 Casting away his garment; his outer garment, that he might more readily go to Jesus.>
<10:52 Thy faith hath made thee whole; the blessing which Christ granted is here ascribed to the means of obtaining it, as is often the case in the Bible. Matthew mentions two who were cured. Mark mentions but one. He may have been the more distinguished.>
<11:1 1-11. Christ rides into Jerusalem. Mt 21:1-17.>
<11:2 Christ has a right to all things, because he made all things, and by him all consist. Col 1:16,17. He can so influence the hearts of men that they will comply with his wishes, and cheerfully give up their possessions to any extent that he may require.>
<11:12 12-14. The fig-tree cursed. Mt 21:18-22.>
<11:13 Any thing; any of the earlier crop, for the fig-tree bears crops at different times. The time of figs; of the crop which this tree might have borne. The curse represented the fate of the barren church-member.>
<11:14 Even the vegetable creation is dependent upon Christ. There is not a plant or flower in the garden, not a tree by the wayside, in the orchard, the field, or the forest, but will wither away if not supported by him.>
<11:15 15-19. Traffickers driven from the temple. Mt 21:12-17.>
<11:16 Any vessel; any vessel used in or connected with their traffic.>
<11:22 Men who have strong and living faith in God, who pray for things agreeable to his will, and which he has promised to grant in answer to prayer, may confidently expect, in his time and way, to receive them.>
<11:23 This mountain; to remove a mountain was a common phrase for the most difficult thing. He shall have whatsoever he saith; the things which, in the name of Christ, under the guidance of his Spirit, and with the faith of miracles, he shall attempt, he shall accomplish; as when Peter said to the lame man, "In the name of Jesus Christ, rise up and walk;" and to Eneas, who had been confined to his bed eight years, "Eneas, Jesus Christ maketh thee whole." Ac 3:6; 9:34.>
<11:24 What things soever ye desire; in accordance with the will and promises of God, ye shall receive.>
<11:25 In order to pray acceptably, we must have a kind and forgiving disposition. If we do not forgive others, our heavenly Father will not forgive us. To inculcate this truth, and impress it upon our minds, God has made it our duty daily to pray, "Forgive us our debts, as we forgive our debtors." Mt 6:12-15. 25, 26. Forgiveness in prayer. Mt 6:12-15.>
<11:27 27-33. Christ's authority. Mt 21:23-27.>
<11:29 Pertinent and discriminating questions may lead opposers to see the truth more clearly and feel it more deeply than cogent arguments or long discussions. Friends of truth who, like Christ, are called to meet opposers, may wisely imitate him by asking them such appropriate questions as they cannot answer without acknowledging the truth. Then, whether they answer or not, truth will triumph.>
<12:1 1-12. Parable of the vineyeard. Mt 21:33-46. As God is the giver and owner of all our possessions, common honesty requires that they should be employed in his service.>
<12:9 A day of reckoning is coming, when Christ will call all men to account for the manner in which they have used the things which he intrusted to them, and will render to each according to his works.>
<12:10 This scripture; Ps 118:22.>
<12:13 13-17. Tribute to Cesar. Mt 22:15-22.>
<12:17 They marvelled at him; on account of his wisdom in so easily and completely avoiding their snares. As human government is an ordinance of God, and magistrates are his ministers to execute so much of his wrath against evil-doers as is needful to protect those who do well, it is his will that men who enjoy the benefits of government should pay for its support. Magistrates have a right to compensation for their services, and it is as really wicked to defraud the government as it is to defraud individuals.>
<12:18 18-27. Denial of the resurrection. Mt 22:23-33.>
<12:24 Wicked men often think that the difficulties which they suggest against revelation justify them in rejecting it. But a better acquaintance with the word of God would show them their folly, and the wisdom of those who receive and obey it.>
<12:26 In the bush; the burning bush. Ex 3:6.>
<12:27 Do greatly err; in denying the resurrection of the body, which, according to their views, implied also that the soul does not live after death.>
<12:28 First commandment of all; the greatest and most important.>
<12:29 One Lord; other nations worshipped many gods, but Jehovah, the God of Israel, was the one only living and true God.>
<12:31 Neighbor; fellow-man. There is none other commandment greater than these; rightly understood, they comprehend the substance of true religion.>
<12:33 Is more; more valuable then all merely external observances.>
<12:34 Discreetly; like one who had right views of religion. Not far from the kingdom of God; because he rightly apprehended its spiritual nature, and what the service of God required of him. Some persons are much nearer the kingdom of God than others. Correct views of his character and requirements, with a just estimate of internal rectitude and purity, compared with external observances, tend to prepare the mind for the reception of Christ, and for the devotion of heart and life to his service.>
<12:35 35-37. Christ David's son and Lord. Mt 22:41-46.>
<12:38 38-40. Warning against hypocrisy. Mt 23:1-12.>
<12:41 The treasury; the place for money to defray the expenses of the temple service. Jesus Christ is witness to what each one does for his cause. From the privilege and benefit of giving for the promotion of it, none, however poor, need be debarred. The value of their gifts in his estimation, so much on the amount as on the proportion which they give, and their motives in giving.>
<12:42 Two mites; a very small sum.>
<12:43 More in, than all they; more in proportion to her means--more for her, and more in God's estimation, than all they had given was for them.>
<13:1 Manner of stones; stones that were used in the building of the temple. These were immensely large. Josephus, the Jewish historian, who lived at that time, says some of them were twenty-five cubits long, eight thick, and twelve broad. The most firm and stable earthly structures are but temporary, and the most enduring earthly possessions come to an end. No one therefore should look to them as his chief good, or trust in them for happiness. 1-8. Destruction of the temple. Mt 24:1-8.>
<13:4 When all these things shall be fulfilled; the things which he had predicted.>
<13:9 9-23. Persecutions foretold. Mt 24:9-28.>
<13:11 Take no thought; avoid anxiety. Neither do ye premeditate; you need not prepare your defence beforehand. The Holy Ghost; will teach you what to say, and through you will speak the right things in the right way.>
<13:13 It is through much tribulation that Christians must enter the kingdom of God. But they should not be anxious. Let them be found at all times in the path of duty, and when trials come they may expect to be prepared for them. God will suffer no calamities to come upon them, except those which he will overrule for the advancement of his glory and their highest good.>
<13:14 Men must not only pray that God would help them, but they must make efforts to help themselves. God answers prayer in such a way as to encourage the performance of duty, not the neglect of it.>
<13:23 The minuteness and accuracy with which Jesus Christ foretold the events which preceded the destruction of Jerusalem, and the perfect fulfilment of his predictions with regard to it, are conclusive evidence that he is "the faithful and true Witness"--that when the word has gone out of his mouth, it standeth for ever. See Ps 33:11; Isa 46:9-11.>
<13:24 24-31. Christ's coming. Mt 24:29-35.>
<13:30 All these things; the things about which he had been speaking.>
<13:32 Neither the Son; it was said to Mary, "The Holy Ghost shall come upon thee, and the power of the Highest shall overshadow thee; therefore"--on account of his miraculous conception by the power of God--"that holy thing," or child "shall be called the Son of God." Lk 1:35. The Son, as born of the Virgin Mary, or as man, might be said, in truth, not to know many things which the Word, who "was in the beginning with God, and was God," did know. Joh 1:1-3. The day here spoken of was one of those things which the Son, as man, in the sense in which it is said, he "increased in wisdom," Lk 2:52, did not know; as man, he neither knew, nor was commissioned to make it known. Nothing but the event would reveal it.>
<13:33 Watch and pray; Mt 24:42-44.>
<13:35 The master of the house; who here represents Jesus Christ. As we know not the time of our death, and no man can reveal it to us, duty and interest require that we should so live as to be always ready. Then, whether we die suddenly, or after lingering illness, no sooner shall we be "absent from the body," than we shall be "present with the Lord," beholding his glory and rejoicing in the fulness of his love.>
<13:36 Sleeping; unprepared to meet him.>
<13:37 Watch; that when your Lord shall come, you may be ready.>
<14:1 When men's hearts are set upon doing mischief, their minds will be fruitful in resources to accomplish. 1-9. Christ's head anointed. Mt 26:1-13.>
<14:3 Love is fruitful in ways of expressing itself towards the object beloved. That which would be thought by others quite too expensive, and requiring too much self-denial, is performed, under the influence of true affection, with alacrity and delight.>
<14:4 Persons may be in the same society, and yet their hearts be going out towards totally different objects. Mary, in the fervor of love for the Saviour, was anointing him with very precious ointment, while Judas was saying, "Wherefore is this waste?" and preparing to go to his murderers with the question, "What will ye give me, and I will deliver him unto you?" Surely, "He will separate them one from another, as a shepherd divideth his sheep from the goats.">
<14:5 Three hundred pence; in the Greek, three hundred denarii. The denarius is commonly estimated at about fifteen cents. Three hundred denarii, then, would be about forty-five dollars.>
<14:8 She hath done what she could; showed her love in the best way in her power.>
<14:10 10, 11. Judas selleth his Master. Mt 26:14-16.>
<14:12 Killed the passover; killed the lamb that was slain on that occasion. 12-16. Passover prepared. Mt 26:17-19.>
<14:13 The city; Jerusalem.>
<14:14 Good man; the master of the house. Guest-chamber; a spare room for the use of visitors. It was customary at the time of the passover to keep such rooms ready furnished for the accommodation of strangers.>
<14:17 17-31. The last supper. Mt 26:20-35.>
<14:19 The thought of being instrumental in betraying Jesus Christ, and injuring his cause, is painful to his friends, and should lead them to earnest prayer and vigorous effort, that they may be kept from the commission of such dreadful sin.>
<14:23 The "fruit of the vine" is a proper element with which to celebrate the Lord's supper. It is a representation of his blood, which was shed for many for the remission of sins. All his friends should drink of it, in kind and grateful remembrance of him.>
<14:29 Great self-confidence is a disciple of Christ is the forerunner of a speedy downfall. "He that trusteth his own heart is a fool." Pr 28:26.>
<14:32 32-46. The agony in the garden. Mt 26:36-50.>
<14:35 The most earnest desires and fervent prayers for deliverance from evils are entirely consistent with perfect resignation to the will of God with regard to them.>
<14:36 Abba; a Syriac word, meaning, Father. This cup; the sufferings that were before him.>
<14:41 Sleep on now--it is enough; some take the first clause interrogatively: Do ye now sleep on in such circumstances? it is enough that ye have slept, etc. Others take the first clause ironically, and the second earnestly: Sleep on now, if ye can in such circumstances, etc. Others still take the first clause permissively, as much as to say, My season of prayer when I desired you to watch is ended: sleep on for the rest of the time before the betrayer comes. And then, after a pause, as he sees Judas coming, It is enough that you have slept, etc. The latter is perhaps the preferable view.>
<14:45 Master, Master; appearing to acknowledge him as his Lord, and to be rejoiced to see him.>
<14:49 The scriptures must be fulfilled; those scriptures which foretold that he would be taken and put to death.>
<14:50 All forsook him, and fled; all the disciples, lest they should be taken also. In times of great danger, our dependence cannot safely be placed on men; not even on good men. They cannot trust themselves. Their good resolutions may vanish, and their courage die. There is no safe dependence but on God.>
<14:51 A certain young man; who this young man was we have no means of knowing. He had perhaps been awakened by the tumult, hastily left his bed, cast a loose covering over him, and joined the crowd. Young men; the soldiers or servants. Laid hold on him; seized him, as if he were one of Christ's disciples.>
<14:55 The council; the sanhedrim, which was the highest Jewish tribunal. Found none; none that testified in such a manner as to answer their purpose.>
<14:58 Made with hands; this was not true. What he did say, and what he meant, is stated in Joh 2:19-21.>
<14:68 No one knows to what depths of iniquity a good man, when left to himself, will fall. He may deny his best friend, desert his greatest benefactor, and even testify and swear to a known lie. Were it not for the grace of God, he would never rise, but sink lower and lower in wickedness and woe for ever.>
<14:69 One of them; one of the disciples of Jesus.>
<14:70 Thy speech agreeth thereto; he spoke like a Galilean.>
<15:1 1-20. Christ before Pilate. Mt 27:1-31; Lk 23:1-25; Jn 18:28-19:16.>
<15:10 Envy; the uneasiness which they felt at his superior excellence and increasing influence.>
<15:11 When men who have the Bible and profess to be religious, prefer a robber and a murderer to the Prince of life, the Saviour of men, and wish the one to be set at liberty and the other crucified, they show that "the heart is deceitful above all things, and desperately wicked." Jer 17:9. No one, in view of such facts, need to marvel that men must be "born again," in order to inherit the kingdom of God.>
<15:16 Pretorium; the hall where the Roman governor, or praetor, held his court.>
<15:19 A reed; the reed, or staff, which, in derision, they had put into his hand as a sceptre. Worshipped him; prostrated themselves in derision, or bowed before him, as subjects do before their king.>
<15:21 21-39. The crucifixion. Mt 27:32-64.>
<15:23 Wine; Matthew says vinegar. It was probably wine which was soured, and might be called by either name. Mingled with myrrh; see note on Mt 27:34.>
<15:24 Men may be perfectly free, and accountable for their conduct, may commit great wickedness, and be ripening for ruin, and yet, in doing this, be fulfilling predictions which were uttered hundreds of years before; thus proving the truth of the Scriptures, and fulfilling the purposes of God. Ps 22:18.>
<15:25 Third hour; nine o'clock in the morning.>
<15:26 The King of the Jews; the crime for which the chief priests accused him before Pilate was, that he claimed to be king of the Jews, and thus was guilty of treason against the Roman government; though it was not for this, but for blasphemy in claiming to be the Son of God, that their council condemned him.>
<15:28 The scripture was fulfilled; Isa 53:12.>
<15:33 Sixth hour--until the ninth hour; from twelve o'clock, or noon, till three o'clock in the afternoon.>
<15:34 As Jesus Christ never committed sin, but was in all things a pattern of perfection, and yet voluntarily died a most shameful and agonizing death, he must have died a propitiation for the sins of men, "the just for the unjust," "that whosoever believeth in him should not perish, but have everlasting life." 1Pe 3:18; Joh 3:15.>
<15:37 This love of Jesus, as manifested in his humiliation and kindness, his sufferings and death for the sins of men, is great beyond all finite comprehension; and to be constrained by it to love and serve him, is guilt unspeakably great.>
<15:40 40-47. Christ's burial. Mt 27:55-61; Lk 23:50-56; Joh 10:31-42.>
<15:42 The preparation; the first day of the feast was called the day of preparation; the next, commencing at the setting of the sun, was the Sabbath; and it was a rule that the body of a malefactor should not remain on the cross over the Sabbath.>
<15:43 Waited for the kingdom of God; he believed that Jesus was the Messiah, and expected that he would soon set up his kingdom. Boldly; he was inspired by the Holy Spirit with courage, notwithstanding the Saviour was dead, thus to show his attachment to him. Jesus Christ sometimes has friends where we should least expect them. They may be associated with the wicked, and yet, through grace, be kept from being partakers of their sins. On the other hand, exceedingly wicked men may be connected with the people of God. In the council which condemned the Saviour was Joseph waiting for the kingdom of God; while among the apostles, Christ's chosen friends, was Judas who betrayed him. Lk 23:51.>
<15:44 The centurion; the officer who had charge of the crucifixion. Thus it was rendered certain that Jesus was truly dead.>
<15:46 A sepulchre which was hewn out of a rock; God so ordered things, that he was not buried in the graveyard of common malefactors, but where there could be the fullest evidence that on the third day he rose from the dead.>
<16:1 Anoint him; it was customary to anoint and embalm dead bodies in order to preserve them. 1-8. The resurrection. Mt 28:1.>
<16:3 When men love the Saviour and wish to honor him, they will often meet with difficulties. But if obstructed in one way, let them honor him in another; in due time apparently insurmountable obstacles may be in unexpected ways removed.>
<16:7 Though the friends of Christ may have deserted and even denied him, yet when they repent and turn to him he freely forgives them, and delights in removing their sorrows and promoting their joys.>
<16:11 Believed not; this shows, that notwithstanding all our Lord had said concerning his rising from the dead on the third day, his disciples did not expect it.>
<16:12 Another form; one different from that in which he had appeared before. Lk 24:13-31.>
<16:13 The residue; those disciples who remained at Jerusalem.>
<16:14 The eleven; the eleven apostles. Judas having hung himself. Upbraided them; sharply rebuked them for their unbelief, in not receiving the testimony of those who had seen him after his resurrection. To reject competent evidence in matters of religion, is a great sin. It shows unbelief and hardness of heart, which are exceedingly offensive to God. Without faith in what he has revealed, it is impossible to please him. Heb 11:6.>
<16:15 All the world; wherever men are found. Preach the gospel; proclaim the glad tidings of salvation through repentance of sin, and faith in Jesus Christ. Every creature; every human being who can hear and understand it. It is the will of Christ that the gospel should be preached to all men. By repenting of sin and believing in him all may obtain it; and if they do not, they will, by their neglect, be self-destroyers.>
<16:16 He that believeth; receives the testimony of God, and treats it, in his feelings and conduct, as true. Is baptized; expresses his belief in God's testimony concerning his Son, not in words only, but in actions, according to Christ's directions. Shall be saved; saved from the practice and consequences of sin; inclined and enabled to practise holiness, and to continue in it, till, through grace, he is prepared for, and raised to the eternal holiness and bliss of heaven. Believeth not; does not so credit the testimony of God, especially in regard to his Son, and the way of life through him, as to love and obey him. Shall be damned; shall be left in the love and practice of sin through time, and be miserable to eternity. Though salvation through Christ should be preached to all men, yet none, without believing on and obeying him, will be saved.>
<16:17 In my name; in reliance on my power, and making it known that it is I, and not they, who perform the miracles. Ac 3:6. Speak with new tongues; in languages which they had not before known. Ac 2:4-11.>
<16:18 Serpents; poisonous reptiles. They would be able, when needful, to handle them without injury. Ac 28:3-6. Deadly thing; mortal poison. Lay hands on the sick, Ac 9:17.>
<16:19 Received up into heaven; Ac 1:9. Right hand of God; a phrase denoting great exaltation and honor. As God has shown by the most conclusive evidence the truth of his gospel, those who continue to reject it are without excuse, and will perish with an awfully aggravated destruction.>
<16:20 The Lord working with them; by miracles, showing that they were sent of him; by removing obstacles, and giving them access to men: by accompanying the proclamation of his truth with his own power, and causing it to produce divine effects.>
\kniha{Luke}
\zkratka{Luke}
<1:1 Many have taken in hand to set forth; others wrote accounts of the times besides the four evangelists whose histories have come down to us, but these were the only men designated by God for the instruction of the world in all ages in respect to our Lord's life and teachings, and inspired by the Holy Ghost for the right accomplishment of this work. Among us; among the Christians then living. There are certain truths taught in the holy Scriptures which are most surely believed by all true Christians, and which are made the means of sanctifying their souls.>
<1:2 They; the persons who were eye-witnesses. From the beginning; the beginning of the things which they described. Ministers of the word; preachers of the gospel.>
<1:3 To me; Luke, the writer of this gospel. Having had perfect understanding; literally, having gone to the source, and accurately traced every thing from the first. Most excellent; a title of honor given to men in office. Ac 23:26 24:3 26:25. Theophilus; friend of God: supposed to be the name of a distinguished individual of Luke's acquaintance.>
<1:4 Those things; the things pertaining to Christ and the gospel.>
<1:5 Course of Abia; the priests were divided into twenty-four courses or classes. 1Ch 24:7-18. Each officiated a week, from one Sabbath to the next. The course of Abia, to which Zacharias belonged, was the eighth in order. 1Ch 24:10. Abia in Greek is the same as Abijah in Hebrew.>
<1:6 In order to be righteous in the sight of God, men must not only believe in Christ for salvation, but be disposed to observe all his commandments and ordinances, and to discharge with fidelity their private as well as public duties.>
<1:9 His lot was--temple of the Lord; more literally, he was chosen by lot to burn incense going into the temple of the Lord; that is, to go into the temple of the Lord to burn incense. The office of burning incense was esteemed the most honorable of all. It was assigned by lot for each day among the priests of the course, and no person could perform it more than once.>
<1:11 Angel of the Lord; it had been about four hundred years since God had sent the Jews a prophet, or made to them any direct revelation. Malachi was the last, and with him the Old Testament revelation closed. As the Messiah was about to appear, divine communications were again opened, and this angel was sent to announce his approach, the birth of his forerunner, and what he would do to "prepare the way of the Lord.">
<1:13 John; the meaning of this word is, the grace of the Lord, or Jehovah is gracious.>
<1:15 Shall drink neither wine nor strong drink; he was to be under the law of the Nazarites from his birth, like Samson. Jud 16:17; compared with Nu 6:1-6.>
<1:17 Go before him; before "the Lord their God." In the spirit and power of Elias; with the zeal and intrepidity of Elijah, as predicted by Malachi. Mal 4:5. Turn the hearts of the fathers to the children; see note on Mal 4:6. Prepared for the Lord; prepared to receive Christ at his coming.>
<1:18 The testimony of God is the highest and most conclusive of all evidence. The disbelief of it exposes men to his righteous displeasure, and deprives them of rich blessings which they might otherwise enjoy.>
<1:19 Gabriel; this is composed of two Hebrew words, which mean, God's strong one, or man of God; and is the name of the angel or messenger sent to Daniel to make known to him things concerning the Messiah. Da 8:16; Da 9:21.>
<1:22 Beckoned unto them; he showed them by signs that he had seen a vision.>
<1:24 Hid herself; lived in retirement and seclusion.>
<1:25 To take away my reproach; to have no children was considered among the Jews a reproach, while a family of children was accounted a great blessing. Le 26:9; 1Sa 1:6; Ps 113:9; 128:3.>
<1:28 Highly favored; in being chosen to be the mother of Jesus.>
<1:29 Troubled--case in her mind; perplexed at such a strange salutation, and wondered what it could mean.>
<1:31 Jesus; Jesus, in Greek, is the same as Joshua in Hebrew, and means, The salvation of Jehovah. As God was the immediate author of the human body and soul of Jesus Christ, and as in him the divine nature and the human nature were united, so that "the Word," who "was in the beginning with God," and "was God," "was made flesh and dwelt among us," it was on both accounts proper that he should be called "the Son of God," and also be declared to be "God manifest in the flesh." 1Ti 3:16.>
<1:32 The throne of his father David; David was, by God's appointment, the earthly head of his ancient church, and his throne typified the higher mediatorial throne of Christ, who was David's son according to the flesh.>
<1:33 The house of Jacob; that is, the church of God, which before Christ's coming consisted of "the house of Jacob" with the proselytes that joined themselves to it, but now includes all who by faith have become the children of Abraham. Ro 4:11-18; Ga 3:7-9.>
<1:35 That holy thing; the child whose conception was to be miraculously caused by the Holy Spirit.>
<1:43 Whence is this to me; why am I so favored as to be visited by the mother of my Lord? Distinguished favors of God to his people lead them to feel their unworthiness, and render them peculiarly humble and grateful.>
<1:45 Blessed is she that believed; Mary, a poor female, on the simple declaration of God, believed things much more strange than those which staggered the faith of Zacharias, the aged priest of the Lord.>
<1:46 The virgin Mary found no source of joy in herself. She ascribed all her blessings to the Lord, and rejoiced in him as God her Saviour, while she magnified his grace in so distinguishing her that "all generations should call her blessed.">
<1:47 As God is never in the Bible called the Saviour of angels or of holy beings, by calling him her Saviour, Mary acknowledged that she was a sinner, and needed his salvation; and if she needed salvation herself, she cannot save others.>
<1:48 Call me blessed; highly favored in having been the mother of Jesus. From these words some have inferred that it is proper to pray to Mary, and pay her divine honors. That this is an error is evident from the manner in which the same phrase in the original Greek is used in other parts of the Bible. They who "endure" afflictions with patience according to the will of God, as did Job, Jas 5:11; the "poor in spirit," "the meek," those who "hunger and thirst after righteousness," and "the pure in heart," Mt 5:2-11, are all called "blessed," the original word being the same. But neither any of these, nor Mary, are to be prayed to, or to receive divine honors.>
<1:51 Showed strength; in protecting his people and overcoming their foes.>
<1:55 As he spake to our fathers; Ge 12:1-3; 22:16-18. The coming of the Messiah, and the blessings which have followed and will follow, are a fulfilment of the promises made to Abraham and his spiritual seed, true Christians, who are of faith, and with Abraham heirs of the grace of life. Ge 12:3; 22:18; 26:4; 28:14; Ga 3:16,29.>
<1:58 Cousins; relations. Great mercy; in giving her a son.>
<1:59 Eighth day; after his birth; as was required in the law of Moses. Ge 21:4; Le 12:3. When parents receive their children as the gifts of God, and from their earliest years implore for them the blessings of his grace, they have reason to hope that God will renew their hearts, and so fill them with his Spirit as to fit them for usefulness on earth and for glory in heaven.>
<1:60 He shall be called John; probably her husband had informed her what they were to call him, verse, Lu 1:13.>
<1:62 Made signs; from this it would seem that he was deaf as well as dumb; otherwise it would not have been necessary to ask him this question by signs.>
<1:64 His mouth was opened immediately; the promise of God by Gabriel having been now fully accomplished, verse Lu 1:20.>
<1:66 Hand of the Lord; the gracious influence of his Spirit.>
<1:69 Horn of salvation; a mighty Saviour, the horn being an emblem of power. In the house; from the descendants.>
<1:70 Holy prophets; Ge 49:10; De 18:15; Isa 9:6,7; Isa 53:2-12. The more men are acquainted with the Bible, and the more observant they are of providence, the more they will see that one is the fulfilment of the other; and the more abundant and conclusive will appear the evidence that both have one Author, and are conspiring to the promotion of the same great end.>
<1:76 Thou, child; John. Prepare his ways; Isa 40:3; Mal 4:5; Mt 3:3.>
<1:78 Day-spring; the beginning of the glorious light of the gospel Isa 60:1-3.>
<1:79 In darkness; the darkness and desolation of sin. Ps 14:1-3; Isa 59:2-14; Ro 3:9-18. The way of peace; peace of conscience and peace with God. Pr 3:17.>
<1:80 The child; John. Waxed strong in spirit; increased in wisdom, power, and goodness. In the deserts; he lived in retirement in the wilderness of Judea. Mt 3:1. His showing unto Israel; showing himself to be the forerunner of the Messiah, and entering on his public ministry. Mt 3:1-3.>
<2:1 All the world; the words in the Greek may denote either all the Roman world, that is, the Roman empire, or Palestine and the neighboring countries. Should be taxed; literally, should be enrolled, that a census might be taken of the inhabitants in order to their taxation. This enrollment was a practical act of Roman sovereignty, and a most decisive proof that the sceptre had departed from Judah. Wicked men, in the prosecution of their selfish purposes, without intending and without knowing it, take such courses as fulfil the predictions and accomplish the benevolent purposes of God. Compare Isa 10:5-17.>
<2:2 When Cyrenius was governor of Syria; it is known with certainty that Cyrenius was appointed governor of Syria several years after our Saviour's birth, and that he then made an enrolment of the people. Upon the supposition that this was the enrolment here referred to, some have proposed to explain the words "was first made," to mean, was first carried out in its original design by the actual laying of a tax in accordance with the enrolment. But recent investigations have made it not improbable that Cyrenius was twice president of Syria, and that the enrolment connected with our Saviour's birth happened under his first presidence. This will explain why it is spoken of as then first made, because another enrollment followed.>
<2:3 Taxed; enrolled for taxation. His own city; the place where his ancestors lived.>
<2:13 Those manifestations which God makes of himself, especially in the person and work of his Son, are deeply interesting, not only to his people on earth, but also to the inhabitants of heaven.>
<2:14 On earth peace; as the result of the Saviour's advent. All who receive him have peace with God and the spirit of peace towards man; and the prevalence of his gospel will bring peace to the world. Good will toward men; kindness, compassion, and grace, manifested in the gift of a Saviour.>
<2:19 Pondered them; continued to think of them and study their meaning. The habit of treasuring up the sayings of the wise and good, especially those which are recorded in the Bible, and of observing the dispensations of Providence, is a source of rich instruction, and may be made a means of grace to ourselves and others.>
<2:22 Days of her purification; after the birth of a son, a mother among the Jews was required to remain at home, and was considered as unclean forty days. These were called the days of her purification. She was then required to offer for a burnt-offering a lamb, and for a sin-offering a turtle-dove, or a young pigeon. If she was too poor to bring a lamb, she was to bring two turtle-doves or young pigeons, and offer one for a burnt, and the other for a sin-offering; after which she was considered as clean. Le 12:2-8.>
<2:23 Holy to the Lord; consecrated to the Lord as his peculiar property. See Ex 13:12.>
<2:25 Consolation of Israel; the Messiah, from whom consolation comes. Aged persons who have long walked uprightly in piety towards God and good will towards men, often have, as they approach the close of life, remarkably clear and exalted views of the Saviour--views which disarm death of its terrors, and prepare them to ascribe "unto Him that loved us, and washed us from our sins in his own blood, and hath made us kings and priests unto God and his Father," "glory and dominion for ever and ever." Ge 49:1,10,22-26.>
<2:34 Fall; ruin by rejecting Christ. Rising again; salvation by believing in him. A sign; an object of peculiar derision. Isa 53:3; Ac 28:22.>
<2:35 A sword shall pierce through thy own soul; generally understood of the anguish which she would be called to endure as a witness of the Saviour's sufferings and death. The thoughts of many hearts may be revealed; by their treatment of the Saviour they will show the character of their hearts.>
<2:36 Aser; Asher. After the captivity, some of the remnant of the ten tribes were found united with the Jews.>
<2:37 Departed not; she was uniform in her daily attendance upon the services of the temple.>
<2:39 They returned into Galilee; in brief narratives like the present, intervening events are often passed over in silence. We know from Matthew that the wise men from the East found the Saviour at Bethlehem; that afterwards he was carried into Egypt; and after a sojourn there of some time, to Nazareth in Galilee, whence Mary had come with Joseph to Bethlehem before his birth.>
<2:42 Twelve years old; at this age it seems sons went with their parents to the passover.>
<2:43 Fulfilled the days; the eight days of the passover: one for preparation, and seven for the observance of the feast. Ex 12:15; Le 23:5,6.>
<2:44 Company; relatives and friends who travelled with them.>
<2:49 Wist; knew. About my Father's business; or, among my Father's matters; which was, in this case, studying his Father's law in his Father's house. Parents who regularaly and conscientiously take their children with them to the house of God, and train them in the way they should go, may expect that they will feel it to be a duty, and will esteem it a privilege, to engage early in the service of their heavenly Father. Pr 22:6.>
<2:50 Understood not the saying; about being occupied with his Father's business; especially, why he should call God his Father in so high and peculiar a sense.>
<2:51 Subject unto them; performed the appropriate duties of an affectionate and obedient child. These sayings; the sayings of Jesus, as well as those of the angel and of others concerning him. Those children who cheerfully obey their parents, in this respect resemble the holy child Jesus.>
<2:52 Increased in wisdom; this is spoken of Jesus as man. See Mt 24:36; Mr 13:32.>
<3:1 Tiberius Cesar; the Roman emperor who succeeded Augustus. Herod; Herod Antipas, son of Herod the Great. Tetrarch; literally, ruler of a fourth part. Iturea; a region of country east of the Jordan. Trachonitis; a country north of Iturea, towards Damascus. Abilene; this lay west of Damascus and north of Galilee.>
<3:2 High-priests; Annas had been high-priest, and was succeeded by his son-in-law Caiaphas. Both were still living and were called high-priests, though but one them officiated. Persons who spend the early part of life in retirement from the noise and bustle of the world, are often preparing for great usefulness. In due time, God calls them to public stations, and to the discharge of duties of extensive and lasting benefit to mankind.>
<3:3 The baptism of repentance; it implied the necessity of repentance in order to the remission of sins. The utter moral pollution of man by sin, and the necessity of spiritual cleansing by the Spirit of God, through repentance and faith in Jesus Christ, are fundamental truths taught under all dispensations; and without a deep conviction of these truths, men cannot be prepared to embrace the Redeemer and become partakers of his salvation.>
<3:4 The words of Esaias; Isa 40:3; Mt 3:3.>
<3:7 7-9. John's preaching. Mt 3:7-12.>
<3:8 Not to say--We have Abraham to our father; rely no more for salvation on your outward relation to Abraham. Of these stones--children unto Abraham; he who formed Adam out of the dust of the earth in his own image, can of these stones raise up holy men, who shall be Abraham's children, not by fleshly descent, but by having the character and doing the works of Abraham. Compare Joh 8:39; Ga 3:7. This is a clear intimation that the time has come when Abraham's seed shall no longer be reckoned by outward descent, but by character.>
<3:10 What shall we do? that is, in order to bring forth fruits worthy of repentance. Verse Lu 3:8. He enjoins upon each class of his hearers repentance, and the fruits of repentance appropriate to their condition in life. That repentance which is unto life, leads men to desire a knowledge of their duty for the purpose of performing it, to break off their sins, and to engage in doing good, as they have opportunity, to the bodies and souls of men.>
<3:13 Exact no more; collect no more than is required by the government.>
<3:15 In expectation; of the coming of the Messiah.>
<3:16 16-22. John's imprisonment--Christ baptized. Mt 14:1-13; 3:13-17>
<3:23 Thirty years; the age at which priests entered on their public duties. Nu 4:3,47. As was supposed; as was generally thought by those who did not know the history of his birth. The son of Heli; in Mt 1:16, it is said, "Jacob begat Joseph, the husband of Mary." Here Joseph is called "the son of Heli." Various ways have been proposed for reconciling the two genealogies of Matthew and Luke. One is, that Mary was the daughter of Heli; and on that account Joseph is called his son. Luke, it is then supposed, gives the genealogy of Mary, while Matthew gives that of Joseph.>
<4:1 1-13. Christ's fasting and temptation in the desert. Mt 4:1-11.>
<4:2 Temptations try human character. Though God often brings men into situations where temptations to sin are strong, he also gives them the means of resisting and overcoming them; and if they do overcome them, they will both honor him and benefit themselves.>
<4:14 In the power of the Spirit; under his powerful supports, and amid the displays of his influence.>
<4:16 To read; portions of the Old Testament were read in the synagogues each Sabbath. Followers of Christ, by imitating his example in habitually attending the public worship of God on the Sabbath, will find his promises to believers fulfilled in themselves; and that while worshipping in the way of his appointment on earth, they are preparing to worship him for ever in heaven.>
<4:17 Delivered unto him; by the minister or person who had the care of the sacred records. The place where it was written; Isa 61:1-3.>
<4:21 This scripture; the scripture which he had just read, and which he said was that day fulfilled, was written more than seven hundred years before, and strikingly described his character and work as the Messiah.>
<4:22 Gracious words; words of kindness and compassion which he uttered as he explained to them the spiritual meaning of the prophecy, and the salvation which he, as the Messiah, would grant to his people.>
<4:23 Heal thyself; this was a proverb, the meaning of which here was, What you are said to have done among strangers, do here among your acquaintance.>
<4:24 No prophet is accepted; those who have known him when a boy especially if in circumstances beneath their own, are less likely than strangers to receive and honor him.>
<4:25 I tell you; he told them, in illustration of what he had said and of the propriety of his conduct, of two cases recorded in their scriptures where miracles were wroght by the prophet Elijah and Elisha, not upon their fellow-countrymen, but upon foreigners: one, that of the widow of Sarepta, a gentile town between Tyre and Sidon; the other, that of Naaman the Syrian. 1Ki 17:9-24; 2Ki 5:14-17>
<4:27 Eliseus; this is the Greek manner of spelling the Hebrew word Elisha, as Elias is that of Elijah.>
<4:28 Heard these things; the things he had spoken in proof of the truth of what he had said, and in justification of his having wrought more miracles at Capernaum than at Nazareth. The most eminent Old Testament prophets, by the direction of God, had gone not only from their own town, but from their country, and wrought miracles among the heathen. He might justly do the same in the displays of his grace. Thus he showed that he claimed and exercised the right to bestow his unmerited favors upon such persons and places as he saw best; that they had no just claim to his wonderful works; and that his salvation was intended for the Gentile as well as the Jew. Filled with wrath; very angry at his teaching such doctrines. When God bestows more of his unmerited favors on some than he does on others, many are tempted to complain. But they should consider, that for all which he does, he has the wisest and best reasons. Wisdom, duty, and interest, therefore, require that we should acquiesce, and say, "Even so, Father; for so it seemed good in thy sight." It is an evidence of great depravity, when men complain that blessings are bestowed on others which they themselves reject.>
<4:29 Thrust him out; by force and violence. Down headlong; to destroy him.>
<4:33 33-44. The devil cast out--Peter's wife's mother and others healed. Mr 1:21-39>
<5:1 The common people are often more eager than their rulers to hear the truths of the gospel. These truths, plainly and kindly exhibited, meet their wants as sinners, and commend themselves to every man's conscience in the sight of God.>
<5:2 Two ships; fishing-boats.>
<5:5 Ministers of the gospel who have preached and labored long without apparent effect, should not be discouraged; but according to Christ's directions should continue to labor in humble dependence on him, and with believing expectations that in his own time and way he will give them success.>
<5:8 Depart from me; this was occasioned by the display of his divinity which Jesus had made, and Peter's conviction of his own unworthiness.>
<5:10 Catch men; by proclaiming to them the gospel, and thus bringing them from the service of Satan to the service of Christ.>
<5:12 12-15. The leper cleansed--the sick healed. Mt 8:1-4; 9:1-7>
<5:15 Information of the effects of Christ's power and grace upon some, is often instrumental in awakening the attention of others, and leading them to apply to him, and thus to become partakers of his salvation.>
<5:16 He withdrew himself into the wilderness, and prayed; it was his custom to do this, as the original implies, which might be rendered, He was in the habit of withdrawing himself, etc. The pure and sinless Saviour needed habitual communion with his heavenly Father to prepare him for the right discharge of the duties of his ministry. How much more do Christ's ministering servants, who are but sinful men, need such communion!>
<5:19 The tiling; Mr 2:4. Tiles were flat pieces of dried clay with which the house was covered.>
<5:23 The works of Jesus Christ, when on earth, showed that he had power to forgive sins and is truly divine.>
<5:27 27-32. Levi, or Matthew, called. Mt 9:9-13.>
<5:33 And they said unto him; in Mt 9:14, this question is put by the disciples of John; in Mr 2:18, by the disciples of John and of the Pharisees. Jesus gives a general answer to both. For the meaning of verses Lu 5:34-38, see notes on Mt 9:14-17. 33-39. Disciples fasting. Mt 9:14-17>
<5:39 When men are taught of Christ, and know by experience the preciousness of his salvation, they will never give up his religion for any other. Good as the advocates of other religions may think theirs to be, the friends of Christ know his to be better.>
<6:1 Second Sabbath after the first; the first was that which occurred on the second day of the feast of the Passover. The second Sabbath was the next, and was the first of the seven that were to precede the feast of Pentecost. Le 23:15-21. Works of needful mercy, and that attention to our bodily and mental wants which the appropriate duties of the Sabbath require, were always permitted by the fourth commandment, and are not forbidden under the gospel. Nu 28:9,10; Joh 7:22,23. 1-5. Plucking corn on the Sabbath. Mt 12:1-19; Mr 2:23-28.>
<6:6 6-11.Healing the withered hand. Mt 12:10-13; Mr 3:1-5>
<6:7 Men may make their scrupulous observance of the Sabbath, and their attention to other external duties of religion, a ground of self righteousness, and a cover under which they may indulge in great wickedness.>
<6:11 Great zeal for human traditions and the commandments of men may consist with enmity to God and deep malignity against those who obey him.>
<6:13 13-16. Apostles chosen. Mt 10:1.>
<6:19 Virtue; healing power. Doing good to the bodies of men often opens the way for benefiting their souls; and the one should be done for the sake of promoting the other.>
<6:20 Blessed be ye poor; for the meaning of these beatitudes and their opposite woes, ver Lu 6:20-26, see notes on the beatitudes in Mt 5:3-12. 20-49. See sermon on the mount. Mt 5:39-48; 7:1-27; 10:24; 12:35; 15:14>
<6:23 Patience under trials, especially when occasioned by those whom we have labored to benefit, and a disposition to do them good in all practicable ways notwithstanding their opposition, are peculiarly pleasing to God, and prepare the soul for the special enjoyment of his love.>
<6:24 You that are rich; rich in this world's goods, and trust in them for happiness.>
<6:25 You that are full; are satisfied with earthly enjoyments, and desire nothing better. Laugh; live in thoughtlessness and sinful mirth.>
<6:26 When all men shall speak well of you; on account of your conformity to this world in your teaching and conduct.>
<6:31 A frequent recognition of the manner in which we ought to wish that others should treat us, will help us to see the way in which we should treat them; and all hopes of heaven which do not lead us to strive habitually to do to others as we would that they should do to us, will fail at the giving up of the ghost. Job 11:20; Pr 10:28; Mt 25:40-46>
<6:40 The disciple is not above his master; this maxim was repeatedly used by our Lord, in different connections. Compare Mt 10:24, 25; Joh 13:16; 15:20. Here its obvious meaning is, that the disciple cannot be expected to go beyond his master in attainments. If the master be blind, the disciple must be blind also. That is perfect; fully instructed in the doctrine of his master.>
<7:1 Audience; hearing. 1-10. Healing the centurion's servant. Mt 8:5-13.>
<7:3 The condition of servants, when sick, is often very distressing. Having no relatives to care for them, it is the duty of their employers, as far as is practicable, to supply their wants; and when, under a deep conviction of their own unworthiness and insufficiency, any apply to Christ for help to those under their care, he delights to bestow the blessings which they need.>
<7:11 Nain; in Galilee, south-west from Capernaum.>
<7:12 No sorrows of a Christian mother, especially a widowed mother on the death of an only son, escape the tender and sympathizing notice of the Saviour. His bosom swells with pity; and when she thinks not of it, he is preparing to pour into her wounded spirit the balm of consolation and cause the desolate, sorrowing heart to sing for joy.>
<7:16 Visited his people; showed them mercy in sending one who could perform such miracles.>
<7:19 The dealings of Christ with his people are often exceedingly mysterious. He sometimes leaves them for a while to the most distressing calamities; and judging only from present appearances, they may be tempted to think that he has forgotten them. But at such times he calls them to consider his character and declarations; not to be offended at any thing which he either does or omits to do; but to feel that his ways are perfect, and that blessed for ever will be all those who put their trust in him. 19-35. Disciples of John sent to Jesus--Christ's testimony of John. Mt 11:2-19.>
<7:29 Justified God; by acknowledging John as a prophet sent by God, and approving of the counsel of God in sending him. Being baptized; the evangelist means to say that they now acted consistently with their former conduct in submitting themselves to his baptism.>
<7:30 Lawyers; these were the interpreters of the Jewish law, especially their traditionary law. They belonged to the sect of the Pharisees, and were one with them in spirit. Rejected the counsel of God; his counsel as shown in the mission of John. Against themselves; to their own hurt.>
<7:31 31-35. Children sitting in the market-place. Mt 11:16-19.>
<7:36 One of the Pharisees; his name was Simon.>
<7:37 A sinner; one who had been notoriously wicked.>
<7:38 Stood at his feet behind him; as, according to custom, he reclined at the table.>
<7:39 Spake within himself; he thought so, though he did not express it in words.>
<7:40 Jesus Christ is more pleased and honored by the affectionate offerings of penitent and grateful hearts, even of those who have been very great sinners, than by the most costly entertainments of the most distinguished self-righteous worldlings.>
<7:44 I entered into thy house; by invitation. Water for my feet; to provide water for washing the feet, was one of the rites of hospitality; to kiss an invited guest was another; and to anoint or rub the hair with olive oil, which imparted smoothness and fragrance, was another. But for some reason, Simon had omitted these. Yet this woman, whom he thought to be such a sinner that her presence must be polluting, had kissed his feet, washed them with tears, and anointed them with very precious ointment. If she had been as great a sinner as Simon supposed, yet her conduct showed that she was penitent, that her love was great, and she was accepted.>
<7:47 For she loved much; according to the parable, much love is the fruit of having been forgiven much. This woman shows much love, which should be to Simon a manifest proof that she had been forgiven much. Our Lord's words, then, may be thus paraphrased: Her sins, which are many, are forgiven; for, as thou seest, she hath loved much.>
<7:48 Thy sins are forgiven; Christ had power and authority, even in his deepest humiliation, to forgive the sins of men. Mt 9:6; Mr 2:10; Lu 5:24.>
<7:49 Who is this? a very pertinent question; and the true answer is, "God over all, blessed for ever." Ro 9:5.>
<7:50 Thy faith hath saved thee; faith, "which worketh by love," was the means of her salvation, as it will be of all who exercise it. Mr 16:16>
<8:1 Thy faith hath saved thee; faith, "which worketh by love," was the means of her salvation, as it will be of all who exercise it. Mr 16:16>
<8:2 Called Magdalene; from Magdala, a town south of Capernaum, midway on the western shore of the sea of Galilee. The gospel raises women from the deep degradation of being the slaves to the privilege and honor of being the companions and most valued friends of men; and often to be the most devoted, self-denying, and useful followers of the Lord Jesus.>
<8:4 4-15. Parable of the sower. Mt 13:1-23.>
<8:10 The manner in which the Saviour communicates instruction is suited to impart knowledge to those who desire it, who seek for it as men seek for silver, and search for it as they do for hidden treasures; while those who despise it, he leaves in ignorance, darkness, degradation, and death.>
<8:16 16-18. Candle under a bushel. Mr 4:21-25.>
<8:19 19-21. Christ's brethren. Mt 12:46-50.>
<8:21 None are so near to Jesus Christ as those who hear the word of God and do it. The union between him and them will live when all other ties are sundered, and will grow more intimate and delightful for ever.>
<8:22 22-25. The tempest on the sea of Gennesaret stilled. Mt 8:23-27.>
<8:25 Believers, notwithstanding their union to Christ and his deep interest in their welfare, may nevertheless be in great danger; and nothing will keep them from tormenting fears, but living and habitual faith in him.>
<8:26 26-39. The legion of devils. Mt 8:28-34.>
<8:37 Those who regard the possession and security of property more than the presence and favor of Christ, deprive themselves of inestimable blessings; and never, without a great change, can they be prepared for, or become partakers of, the bliss of heaven.>
<8:41 41-56. Jarius' daughter, and the woman with an issue of blood. Mt 9:18-38.>
<8:43 That faith in Christ which works by love leads those who have it to apply to him for what they need. And though their case, in the view of men, may be hopeless, in the Saviour they will find sure and all-sufficient aid.>
<8:45 Who touched me? said not for his own sake, but to draw out the woman from her privacy, and bring her to an open acknowledgment of him.>
<8:47 Came trembling; she had obtained healing from Jesus in a stealthy way; and for this she feared his rebuke.>
<9:1 The power and authority of ministers to preach the gospel and administer its ordinances come from Jesus Christ. On him they are dependent, and to him they should look for success in their work. 1-6. Twelve apostles sent out. Mt 10:1-42.>
<9:7 7-9. Herod desires to see Christ. Mt 14:1,2.>
<9:9 None are so high in authority or power as to be above the upbraidings of conscience; and none continue so long, or sink so low in wickedness, as permanently to stifle its voice. It may for a time appear to slumber, and then awake to whisper vengeance, or utter thundertones of wrath.>
<9:10 Went aside privately; he went by ship. Mt 14:13; Mr 6:32; Joh 6:1 Bethsaida; there were two places of this name. That best known was on the western side of the sea of Galilee. The other was on the northern side of the same lake; and to that the present passage refers.>
<9:13 The inexhaustible fulness and all-sufficience of the Saviour lay a permanent foundation for the peace and quietness of all who trust in him; and though destitute of resources in themselves, they may always find in him unfailing supplies.>
<9:18 18-21. Peter confesses Christ. Mt 16:13-20.>
<9:22 22-27. Christ foretells his death. Mt 16:21-28.>
<9:23 To deny one's self some things for the sake of obtaining others more valuable, instead of lessening, greatly increases enjoyment. This, to be a follower of Christ, a person must do daily, and thus be daily promoting his highest good.>
<9:28 28-36. The transfiguration of Christ. Mt 17:1-9; Mr 9:2-10.>
<9:31 His decease; literally, departure, meaning his death.>
<9:32 Were heavy with sleep; the transfiguration seems to have taken place in the night season, which will explain the statement of verse Lu 9:37, that they came down from the mount on the next day. While the Saviour was engaged in prayer, they slept; but they were awaked to behold his glory.>
<9:37 37-43. The lunatic healed. Mt 17:14-21.>
<9:38 When disease fastens on a child, and all human aid fails, the privilege of applying to Christ with the assurance of his ability to help, is a blessing which awakens the gratitude, and will for ever call forth the praises of every pious parent.>
<9:44 These sayings; the words of Jesus Christ, especially with regard to his death, which would shortly take place.>
<9:45 Understood not; they expected the Saviour would be a great worldly conqueror, and live for ever; and they did not understand how it could be that he would die.>
<9:46 46-50. Who should be greatest. Mt 18:1-6.>
<9:51 Received up; into heaven. Steadfastly set his face; resolutely determined to go. Jesus Christ, when the time had come, was no less intent upon dying at Jerusalem than the Jews were on putting him to death. Their object was to show that he was not the Messiah, and thus to prevent the people from receiving him. His object was to die for their sins and the sins of the world; to show, with absolute certainty, that he was the Messiah, and lead unnumbered millions to believe in him, experience his salvation, and eternally adore him. Hence, before the time had come, he would not commit himself to them, and after it had come, he would let nothing hinder him from doing it.>
<9:52 Sent messengers; in the original it is angels, which shows the manner in which this word angels is sometimes used in the Bible, meaning persons who are sent. Make ready; provide lodging and refreshment.>
<9:53 His face was; they saw that his purpose was to go to Jerusalem, and as they were great enemies to the Jews, to whom he was going, they would not entertain him.>
<9:54 When--James and John saw this; saw that the Samaritans would not entertain Christ. Elias; Elijah. 2Ki 1:10-12.>
<9:60 The claims of Jesus Christ to immediate and unreserved obedience are supreme; and no earthly connections or engagements can justify any in delaying to give him the homage of their hearts and the service of their lives.>
<9:62 Having put his hand to the plough, and looking back; the husbandman who puts his hands to the plough must keep them, and his eyes too, fixed upon it. If he looks back, as for example to converse with those behind him, his work will be poorly done. The Syrian plough, being light, required the weight of the ploughman's body on it to keep it in the furrow. If he looked off, it would start aside. So he who would be a worthy servant of Christ must give him his whole heart and his whole time. Fit for the kingdom of God; fit for the ministry of Christ's gospel in his kingdom. Such was plainly the original application of these words. But they apply with equal force to every kind of service which Christ requires. No man who is not ready to leave all when Christ calls, is prepared to serve him on earth, or enjoy him in heaven.>
<10:1 Other seventy; in addition to the twelve whom he had before appointed. Chap Lu 9:1,2. Into every city and place; in order to prepare the people for his coming. When Christ is about to visit a place in mercy, he, in his providence, often prepares the way for it; and the manner in which the manifestations of his will are treated, shows the character of his inhabitants, and the way in which they will receive him.>
<10:2 The harvest; the need and opportunity of preaching the gospel. Laborers; preachers. Mt 9:36-38.>
<10:3 Lambs among wolves; Mt 10:16. 3-16. Instruction to the seventy disciples. Mt 10:11-15,40; 11:20-24.>
<10:4 Purse--scrip; Mt 10:9,10. Salute no man; the mode of salutation then was more formal than now. He would not have them hindered by giving or receiving salutations, but would have them proceed directly to their work.>
<10:6 Son of peace be there; a man of a peaceful spirit, who will kindly receive you, and to whom you may give the blessing of God's peace.>
<10:9 The kingdom of God is come nigh unto you; the opportunity is given to embrace the Messiah and experience his salvation.>
<10:12 Those who reject the gospel reject the Saviour; and the greater their light, if they do not improve it, the greater will be their guilt and the more dreadful their condemnation. Chap Lu 12:47,48.>
<10:17 The devils are subject unto us through thy name; when, in reliance on thee, we command them to come out, they obey.>
<10:18 I beheld Satan--fall from heaven; the casting of Satan from heaven--probably with allusion to his original fall from heaven--means casting him out of his power over this world. Compare Re 12:7-9. This the Saviour saw from eternity in its beginning and completion. Every time that he encountered Satan, he overcame him; and the casting out of devils in his name was a sign and pledge that Satan, the prince of devils, shall, through the progress of His gospel, be finally cast out of all his power over this world.>
<10:19 To tread on serpents and scorpions; to tread on literal serpents and scorpions without harm, and to overcome wicked men, who are like serpents and scorpions in character. Nothing shall by any means hurt you; the chief reference of these words is to the spiritual victory which Christ gives his servants over all evil, of which the outward deliverances sometimes vouchsafed to them in this world are symbols and pledges. Compare Ro 8:28,37. Christ is able to give his ministers all the aid which they need for the discharge of their duties. In his name and strength they may commence their work, and go on from conquering to conquer, till every knee shall bow, and every tongue confess that he is Lord, to the glory of God the Father.>
<10:20 Your names are written in heaven; as heirs of eternal life.>
<10:21 Things hid from the wise and prudent. Mt 11:25-27.>
<10:23 Things seen by the disciples. Mt 13:16,17.>
<10:25 A certain lawyer; one whose business it was to study, explain, and teach the divine law. Tempted him; put his wisdom to the test.>
<10:26 The attention of those who inquire what they shall do to be saved, should be directed to the great fact, that by the works of the law they cannot be justified, and that the only way of salvation is through faith in Jesus Christ, who is "the end of the law for righteousness to every one that believeth.">
<10:27 27, 28. Love to God and man. Mt 22:37-40; Le 19:18; De 6:4,5.>
<10:29 Justify himself; maintain that he could not be condemned for having broken the divine law. Who is my neighbor? who is the person whom I am to love as myself?>
<10:30 Jericho; about twenty miles north-east of Jerusalem, and seven from the Jordan. Fell among thieves; more exactly, fell among robbers. The road from Jerusalem to Jericho led through a wilderness abounding in narrow and rocky passes, and was anciently, as now, infested with robbers.>
<10:31 By chance; without any design to help the Jew, or knowledge of his condition.>
<10:32 A Levite; the Levites assisted the priests in the services of the temple.>
<10:33 The love which the law of God requires, leads those who have it to do good, not merely to their friends or countrymen, but, as they have opportunity, to all, in imitation of Him who makes his sun to rise on the evil and on the good, and sends his rain on the just and on the unjust, and "who, though he was rich, for our sakes became poor, that we through his poverty might be rich.">
<10:35 Two pence; in value about twenty-eight cents, or the price of two days' labor. Mt 20:2. The host; the keeper of the inn.>
<10:36 Was neighbor; the ruler, who would have restricted the word neighbor to a very narrow circle of friends, is shown that all men whom he has the power of benefiting are his neighbors, and that he owes to all a debt of love and self-denial.>
<10:38 A certain village; Bethany. Mt 21:17.>
<10:40 Cumbered; busily occupied. Much serving; in providing entertainment for her guests. Though diligence in business and proper regard to family concerns are duties which should by no means be neglected, yet we may be so engrossed in them, and so troubled about them, as greatly to displease the Saviour, and injure ourselves. Our first regard should be for God, and our chief concern to learn and do his will. He will then so order his providence, that we never shall want any thing essential to our highest good.>
<10:41 Careful; anxious, perplexed. Many things; with regard to this world.>
<10:42 One thing is needful; needful especially, above all other things. That good part; the favor of God, through love and obedience to his commands. Shall not be taken away; Job 17:9; Joh 4:14; 10:27-30.>
<11:1 All who are wise will earnestly desire to be taught rightly to pray, and will ask Jesus Christ to instruct them. This is a blessing which he delights to give, and with it is connected, in his providence and by his grace, all needed good. "Ask and it shall be given you." Mt 7:7.>
<11:2 2-4. The Lord's prayer. Mt 6:9-13.>
<11:7 Shut; bolted, as the original word implies.>
<11:8 Because of his importunity he will rise and give him; the point of the parable is to show the power of importunity in prayer. If it prevails with selfish men, how much more with God, who loves his children, and takes pleasure in granting their requests.>
<11:9 Ask, and it shall be given; now comes the application of the parable. Be importunate in asking, seeking, and knocking at God's door, and you will be heard and answered. God often delays answering prayer, that he may try the faith and earnestness of the suppliant. 9-13. Asking of God in prayer. Mt 7:7-11.>
<11:12 Scorpion; a poisonous reptile, with eight legs, eight eyes, and a sting in its tail, which inflicts great pain.>
<11:13 Give the Holy Spirit; this is the gift of gifts, including in itself all needed good. God loves to have men pray for the greatest blessing he can bestow, the Holy Spirit, which he has promised to those who ask him. If any, therefore, do not receive it, and are not enlightened, sanctified, and saved, it is because they do not in faith and love ask for this blessing.>
<11:14 14-23. Casting out devils by Beelzebub. Mt 12:22-30.>
<11:24 24-26. Return of the unclean spirit. Mt 12:43-45.>
<11:26 Wicked men reject the revealed truth of God, not because there is not sufficient evidence that it is truth, but because they are wicked, and the truth condemns them. When men reject evidence which God gives, and seek such as he will not give, they grow more wicked, and their last state becomes worse than any which preceded it.>
<11:28 Yea, rather, blessed; the outward relation of Mary to Jesus as his mother was not so high a privilege, and did not confer such blessedness, as a believing and obedient spirt brings to the humblest of his disciples. How wrong, then, to exalt Mary to be an object of worship because of this outward relation. A disposition to hear the voice of God and obey it, is the greatest of blessings. Blessed as was the Virgin Mary on account of her being the mother of Jesus, more blessed, according to his decision, are all who believe on and obey him. Even Mary herself was more blessed as his believing and obedient disciple, than as his mother according to the flesh.>
<11:29 29-32. A sign sought. Mt 12:38-42; Mr 8:11,12.>
<11:33 33-36. Candle under a bushel. Mt 5:15; 6:22,23.>
<11:36 If thy whole body--full of light; a clear view of spiritual things is to the soul what sight is to the body. It enables us to see clearly and correctly all truth that relates to God and Christ. Thus the soul becomes like a well-lighted chamber, having no dark corner.>
<11:38 Washed; in the original, baptized. Great attention may be paid to outward forms and ceremonies, especially when men are taught to depend on them for salvation, and yet their hearts be abominably wicked. All such dependence is vain. To be accepted of God, men must give him their hearts, and must manifest this by obeying him and doing good, as they have opportunity, to their fellow-men.>
<11:39 And the Lord said; this discourse at the Pharisee's table has much in common with that recorded in Mt 23.1-39; but it was delivered on a different occasion. Make clean; cleanse by washing. Mt 23:25,26; Ravening; greedy violence.>
<11:40 Fools; they who mock God with outward forms merely, while inwardly full of impurity, are not only wicked, but most foolish. Made that which is without--that which is within; the argument is, that since God made the spirit as well as the outer man, he must require that also to be kept clean from pollution.>
<11:41 Give alms of such things as ye have; that is, according to our version, bestow your property, as you have means and opportunity, in deeds of love for Christ's sake, and you will be accepted of him. But we may better render, Give the things within as alms; the things, namely, within the cup and platter. Instead of spending your time in washing their outside, while "within they are full of extortion and excess," see that their contents are made clean by being devoted, in the fear of God, to men's good, and then the outside will be clean also. The cup and platter here represent covetous and rapacious men, whose souls are polluted by unrighteous gain. Let them begin by making their hearts clean, and then they need not be troubled about outward defilements.>
<11:43 Uppermost seats--and greetings; the most honorable places and public salutations. Mt 23:6,7.>
<11:44 As graves which appear not; so that men walking over them are polluted without knowing it. In Mt 23:27,28, there is a contrast between a fair outward appearance and inward abomination. Here the point is the carefulness of the Pharisees in hiding their wickedness.>
<11:45 Reproachest us; as being inwardly corrupt.>
<11:46 46-51. Sin and doom of the Pharisees. Mt 23:4,29-36.>
<11:49 The wisdom of God; as manifested in the words and works of Christ.>
<11:52 The key of knowledge; they prevented the people from obtaining the true knowledge of Jesus as the Messiah. They would not believe in him themselves, and they hindered others. To hinder men from obtaining that knowledge which God has revealed in his word, is a great sin, and one which exposes all who against light continue in it, to an awfully aggravated condemnation.>
<12:1 In the mean time; while he was delivering his discourse to the Pharisees. The leaven of the Pharisees; hypocrisy, the great sin of the Pharisees, which, like leaven, mingled itself with and corrupted all their religious services. Men should be especially careful to be at heart in all things honest, upright, and sincere, and to act from good motives; they should be more desirous of being right in the sight of God, than of appearing to be right in the sight of man.>
<12:2 For there is nothing covered; the Saviour shows the folly of hypocrisy from the consideration that every thing will at last be made known.>
<12:4 4-12 Be not afraid; a very common form of hypocrisy has always been dissimulation and the denial of Christ through fear of man. Compare Joh 12:42,43. The Saviour, therefore, next warns his disciples against this sin, because, first, men can do us no real harm, while God can destroy both soul and body in hell, verse Lu 12:4,5; because, secondly, God, who watches over the sparrows, will protect his faithful servants, verses Lu 12:8,9. He then warns his disciples against the blasphemy of the Holy Ghost, a sin in which the denial of Christ might end, verse Lu 12:10; and against anxiety in respect to their defence when brought before magistrates, verses Lu 12:11,12.>
<12:13 Speak to my brother; he wished to make use of the Saviour's authority and influence to increase his own estate, as some men now value religion simply from its worldly advantages.>
<12:15 Covetousness; over-anxiety and selfish greediness for earthly things. Consisteth not; neither the length, usefulness, and happiness of a man's life in this world, nor his eternal life hereafter, depend upon the amount of his earthly possessions.>
<12:16 In providing for happiness, men should act, not for time merely, but for eternity, that, at whatever moment they may be called from earth, they may go to, and not from, their treasures.>
<12:20 Thy soul shall be required; thou shalt die, and thy soul shall be required to go to judgment and give an account of its deeds while in the body. In trusting to riches for that happiness which can come only from God; in depending upon long life, when death may come this night; and in laying up treasure on earth, and not in heaven, men act the part of fools.>
<12:21 That layeth up treasure for himself; lives supremely for himself, not for God, which was the great sin charged upon this man.>
<12:22 Those who have that fear of God which leads them to avoid what displeases him, have no reason to fear any thing else. In him they may trust for whatever they need, and he has promised that, in the best way and time, he will supply them. 22-31. Taking thought for the morrow. Mt 6:25-34.>
<12:25 Add to his stature one cubit; see note on Mt 6:27.>
<12:32 The kingdom; of heavenly glory. Mt 3:2.>
<12:33 Give alms; use your wealth in doing good, and then you make it impossible that it should be lost; for the treasure which is given to the poor in Christ's name, is given to Christ, and he will lay it up for us in heaven. Bags which wax not old; heavenly purses to contain heavenly treasures. Let that which you regard as your chief good be in heaven. Your hearts will then be heavenly, and your treasure and blessedness be eternal.>
<12:35 Let your loins be girded; the girding up of the loins was a preparation for action. Be ready for duty. Your lights burning; be always watchful. 35-46. The faithful servant. Mt 24:42-51.>
<12:36 When he will return from the wedding; either his own wedding, in which he is the bridegroom, or the wedding of a friend. Weddings were attended in the night; and servants were accustomed to sit up and wait for their master's coming, that on his arrival they might immediately open the doors. So our Lord told his disciples to watch, and pointed out the blessedness of those who should do so. Mt 25:1-13.>
<12:37 Come forth and serve them; he will greatly honor and bless them.>
<12:38 Second watch; from nine in the evening to twelve. Third watch; from twelve to three in the morning.>
<12:39 Good man of the house; master of the house.>
<12:41 Unto us, or even to all? is it meant for us, as thine apostles, or for all men? Our Lord, in his answer, speaks of a steward set over his master's household, thus intimating that the parable has its highest reference to the ministers and rulers of his church; but shows at the close, verse Lu 12:48, that it applies to every one according to the measure of his knowledge and of the duties laid upon him.>
<12:49 To send fire; in the same sense in which he came to send a sword. Mt 10:34. Fire and sword are emblems of contention, distress, and ruin: not that this was the object of Christ's coming, or the tendency and proper effect of his gospel, but it would be the effect of the opposition which wicked men would make to it. What will I, if it be already kindled? did he regret the publication of the gospel, or would he desist from it on account of the contention it would occasion? No; he desired its publication, and that, as soon as practicable, it might be universal. Opposition to the best things often produces the greatest mischiefs. But no good thing, rightly done, is to be charged with any of the evils which opposition to it occasions.>
<12:50 A baptism; extreme suffering which he must pass through before the gospel could be fully published. Straitened; oppressed in spirit, in view of the sufferings which were before him.>
<12:51 Rather division; Christ came to send divisions in the same sense in which he came to send fire and sword. His gospel would not produce divisions, but men's opposition to it would. Ver Lu 12:19; Mt 10:34-36>
<12:54 Out of the west; from the Mediterranean sea, which lay west of Judea. Were men as quick to discern, and as wise to judge, in spiritual as they are in temporal things, and did they as earnestly and perserveringly pursue them, they might all, through grace, become rich for eternity. But while they know that to obtain temporal good they must be awake and active, must exercise judgment, lay plans, and diligently pursue them, they often hope to obtain eternal good without thought, plan, or effort.>
<12:55 The south wind; from the hot and sultry deserts of Arabia and Eqypt.>
<12:56 This time; the indications of the presence of the Messiah.>
<12:57 Even of yourselves; under the guidance of your own consciences enlightened by God's word. Judge--what is right; make a true judgment respecting the signs of the times and my claims to be the Messiah. Why not do this before you are summoned to God's judgment-seat to have him decide the question against you?>
<12:58 When thou goest with thine adversary; literally, For when thou goest with thine adversary; the word "for" connecting this verse immediately with the preceding. Under the figure of a man summoned by his adversary to appear before the magistrate, our Lord, in concluding this series of addresses, solemnly warns his hearers to be reconciled to God, who is both their adversary and their judge, while they are on the way to his judgment-seat, by acknowledging the claims of his Son Jesus Christ. Thus they can obtain pardon and eternal life; but if they refuse this, at God's bar the very last mite will be demanded of them; and as they will have nothing to pay, they must lie in the prison of despair for ever. Give diligence that thou mayest be delivered from him; by acknowledging of thyself his just claims, and satisfying them. This will be judging of one's self what is right.>
<13:2 The visible dealings of Providence with men in this world are no certain indications of their real character; but are suited to teach them the evil of sin, and the necessity of forsaking it, the certainty of death, and the wisdom as well as duty of being at all times prepared for it.>
<13:3 I tell you, Nay; sudden death is no evidence of peculiar wickedness; but death in any form is the effect of sin, and should remind us that we must repent of it, and be delivered from its power, or we shall perish.>
<13:4 Tower in Siloam; probably in the wall of Jerusalem, near the south-east corner, where was the pool of Siloam. See comment on Joh 9:7.>
<13:9 If those who enjoy the means of grace neglect them, and bring forth no fruits of holiness, God, in due time, will remove all such blessings from them, and leave them to endless barrenness and death.>
<13:11 A spirit of infirmity; a spirit that kept her bowed together; for her infirmity is ascribed to the power of Satan, verse Lu 13:16.>
<13:15 Hypocrite; he condemned Jesus for relieving on the Sabbath an infirm woman, who had suffered for eighteen years, when he would himself perform more labor for the relief of an animal from thirst for a single day.>
<13:16 A daughter of Abraham; a descendant of Abraham and possessing his faith. Whom Satan hath bound--loosed; the allusion is to the loosing of an animal from the stall, verse Lu 13:15. Satan has bound down this woman as an ox or ass is bound to the stall. Jesus Christ delights to bless those who habitually attend public worship. Though Satan may have bound them in chains of sin for many years, Christ is able and willing to deliver them. He often shows this on the Sabbath in the house of God.>
<13:18 18-21. Parables of the mustard-seed and leaven. Mt 18:31-33.>
<13:19 Divine grace in the heart may at first be small and feeble; men may hardly be able to perceive it; but by a proper use of the means, under the influences of the Holy Spirit, it will increase till its manifestations shall become visible to all.>
<13:24 Strive; in the original, agonize, make immediate and strenuous effort. Strait gate; difficult entrance of the way of life. Mt 7:13,14. Shall not be able; they do not seek in season, nor in a proper way. Thus the Saviour answers the question virtually, though not directly. The striving of men to enter the way of life, is the means by which God enables them to do it; while the neglect of this till death, renders it certain that they will never enter it, or take a step towards heaven.>
<13:25 The master--shut to the door; the reference is to the shutting of the door at a feast, after which none can be admitted. Compare Mt 25:10-12. The meaning is, that the day of grace is limited, and after it is closed, none who have continued to neglect it can obtain salvation. I know you not; he did not know them as his friends, because they had never been such.>
<13:26 We have eaten and drunk in thy presence; they rely on their outward relation to Christ; but he teaches them that this can be of no avail to those who have not kept his commandments.>
<13:29 Sit down in the kingdom of God; literally, recline in the kingdom of God, as at a joyous feast. The salvation of men does not depend upon either their outward position or the number or variety of their privileges, but upon the manner in which they improve them. Many who have had small advantages, and have moreover been despised and neglected, will be saved, because they have faithfully improved their opportunities; while others, who have had great advantages but neglected them, will be lost. Thus the contrast between men's standing in this world and in the world to come will be, in many cases, inconceivable great.>
<13:30 Last--first--first--last; these solemn words have a twofold fulfilment. First, in this world: the scribes and Pharisees stood first in God's kingdom as to their outward position and privileges; but by rejecting Christ, they made themselves last, while the publicans and sinners and the gentile nations, who they despised, by receiving him, became first; and so it has often been since. Secondly, in the world to come, where many that have stood high in reputation and outward privileges will be thrust down to hell, and many that have here been despised and persecuted will be exalted to glory everlasting.>
<13:32 That fox; sly, subtle, mischievous man. To-day and to-morrow; a short time. I shall be perfected; shall have completed my work.>
<13:33 I must walk; act openly for a few days, then go up to Jerusalem, and die. It cannot be; this is an instance of the manner in which the word cannot is sometimes used in the Bible, as describing what is not common, what is difficult, and will not take place. Out of Jerusalem; here the great council of the Jewish nation and the Roman governor held their courts; here criminals were tried; and here most of the prophets who had been murdered were put to death.>
<13:34 The persevering wickedness of sinners greatly grieved the Lord Jesus Christ. He would gladly have received them, and given them his salvation; but they refused to accept it, and thus became the guilty authors of their own destruction. 34, 35. Lamentation over Jerusalem. Mt 23:37-39.>
<14:1 On the sabbath-day; Jesus Christ was on a journey, and had no home. It was proper that he should take food where he was invited. He went to take such refreshment as his physical wants on that day required, and to do good to those who might be present. This affords no justification to visiting, or making dining-parties on the Sabbath. They watched him; to see if he would not do something for which they might accuse him.>
<14:3 Answering; he replied to their thoughts by the question which he put to them.>
<14:5 We should form the habit of drawing spiritual instruction from the common occurrences of life; and in our social intercourse, as well as in our religious efforts, should endeavor to do good to our fellow-men. 5, 6. Healing on the Sabbath. Mt 12:11,12.>
<14:7 He put forth a parable; showing the importance of humility.>
<14:8 Highest room; most honorable place at the table, where the principal personages reclined.>
<14:10 Have worship; receive honor.>
<14:11 Whosoever exalteth himself; is proud, and seeks to be honored above others. Shall be abased; by God. He that humbleth himself; who is humble, and shows it in his conduct. Shall be exalted; honored; raised to higher dignity and influence. Pr 16:18,19; Mt 5:3; 11:29; Mt 18:4; 23:12; Jas 4:6. This proverb is abundantly illustrated in God's dealings with men in this world, but will have its highest fulfilment in the world to come. The indulgence and display of pride indicate great wickedness of heart, and are sure precursors of coming abasement; while the cultivation and manifestation of humility are evidences of greatness, and harbingers of coming glory.>
<14:13 Call the poor; do good to the needy who cannot reward you.>
<14:14 The resurrection of the just; when God shall reward those who for his sake have done good, without the hope of any earthly recompense. Genuine benevolence will lead those who have it to do good for goodness' sake, rather than for any expected reward; and the less the recompense which it receives in this world, the greater may be its gracious reward in the world to come.>
<14:15 Eat bread in the kingdom of God; enjoy its blessings. See Mt 3:2.>
<14:16 A great supper; representing the rich and abundant provisions of the gospel.>
<14:17 To them that were bidden; to them that were regularly invited. These represent here the Jews, to whom the gospel was first offered, especially the scribes and Pharisees. Come, for all things are now ready; the invitation to those who hear the gospel to partake of its blessings. Jesus Christ has provided, and freely offers, the richest and most abundant blessings. All excuses which men make for not accepting them are vain and wicked.>
<14:18 To make excuse; showing the unwillingness of men to accept the offers of salvation. I must needs; literally, I have a necessity. This shows the manner in which necessity is sometimes used in the Bible to express a strong desire.>
<14:20 I cannot come; that is, he did not wish to come. He chose not to do it.>
<14:21 Angry; because those who were bidden slighted his invitation by neglecting his feast for totally inadequate reasons. Streets and lanes of the city; the dwelling-places of the poor and disabled, who here represent the publicans and sinners.>
<14:22 Yet there is room; however many may partake of the blessings of salvation, there are enough for all others; and all to whom the gospel is preached, are urged to partake of them.>
<14:23 The highways and hedges; lying without the city, by which is signified the calling of the Gentiles. Compel them; not by force, but by persuasion, by earnest, persevering entreaty.>
<14:24 None of those men--taste of my supper; a solemn announcement of the coming rejection of the Jews as a nation for their unbelief. But the words apply in all their force to the multitudes now in Christian lands who despise and neglect the gospel, while converts from among the heathen nations are multiplied. God is angry with men who will not accept of his salvation, and be for ever happy; and when for ever miserable, they will see that no part of the blame attaches to him, but that it all belongs to them.>
<14:25 Great multitudes with him; without any suitable apprehension of the self-denial which his service would require of them.>
<14:26 Hate not; if he be not willing for my sake to leave father and mother. Mt 10:37.>
<14:28 A tower; a high building, erected for observation and defence.>
<14:32 Ambassage; persons sent from one government to another, to represent the interests of their country. The point of this and the preceding comparison is, that they who undertake Christ's service should count the cost beforehand.>
<14:33 Forsaketh not all; all that stands in the way of duty--all that would hinder a man from doing the known will of God.>
<14:34 Salt is good; to season provisions, and perserve them from putrefaction. In the present connection, salt means divine grace manifested in a spirit of self-denial for Christ's sake. this brings salvation to its possessor and to others. If the salt have lost his savor; its saltness; if holy self-denial has given place to worldliness and self-indulgence. Seasoned? its saltness be restored.>
<14:35 For the land--the dunghill; for being sown to fertilize the soil, nor for being mingled with the dunghill. So a professed follower of Christ, who has lost His spirit, is of no value to the church here, and has no fitness for admission to heaven hereafter. Let all worldly-minded disciples hear this.>
<15:1 Publicans and sinners. Mt 9:10.>
<15:2 Murmured; found fault with him for associating with vicious persons, or permitting them to approach him. He therefore spoke three parables, showing that God receives and rejoices over sinners who return to him, however wicked they have been; and that it was highly proper that the Saviour of sinners should do the same. Murmuring when sinners come to Christ, and uneasiness at his reception of them, are evidences of a selfish, wicked spirit, which, without a great change, can never join in the employment or partake of the bliss of heaven.>
<15:3 He spake this parable; the three parables of this chapter contain each a vindication of the Saviour's conduct in receiving publicans and sinners. The point of them all is, that not what is safe, but what is lost, is the just occasion of labor in finding and joy upon recovery. We are not to infer from verse Lu 15:7 that there are any who were never lost and never need repentance. The Saviour simply reasons with the Pharisees upon common principles, as much as to say, If, as you think, you are safe within God's fold, why blame me for my solicitude to find and save the lost?>
<15:4 4-7. The lost sheep. Mt 18:11-14.>
<15:7 Joy shall be in heaven; as there is joy in heaven over the repentance of sinners, it was proper that Christ should associate with them, for the purpose of promoting their repentance. Ninety and nine just persons; there is more joy in heaven over one who repents and turns to God, than over many who have never sinned and need no repentance, or who, having sinned, think that they need none.>
<15:8 8-10. This parable is another illustration of the same truth.>
<15:10 As God, angels, and all holy beings rejoice at the repentance of sinners, all who repent, and all who are successful in leading others to repent, are increasing the happiness of heaven.>
<15:12 The younger; he represents openly wicked persons, such as the "publicans and sinners;" as the elder son does the Pharisees, "who trusted in themselves that they were righteous, and despised others." His living; literally, the living, that is, the estate in his hands. He paid over to the younger son his portion, but reserved in his own hands the elder son's portion. Wicked men wish to have their concerns in their own hands. They would rather choose and direct their course, than have God do it for them. This is setting up their wisdom and goodness above his, and will end in sad disappointment.>
<15:15 To feed swine; this was considered a very degrading employment, and to a Jew was especially odious. Le 11:7; De 14:8.>
<15:16 Husks; large pods growing on the carob-tree. They have a sweetish pulp, and small seeds like beans. Swine are fed on them, and poor people sometimes eat them.>
<15:17 Came to himself; came to have just views of things. Men must feel that they are lost, before they will be found; and unless they believe that away from God they will perish, they will never return to him. Nor, if they do believe this, will they ever return to him till they steadfastly resolve to do it.>
<15:18 Against heaven; against God as well as against his father.>
<15:20 Ran and fell on his neck; this represents the readiness with which God receives returning sinners. To be saved, men must not only resolve, but they must return to God; taking all the blame and shame of their departure to themselves, and ascribing righteousness to him, they must surrender all their interests for time and eternity to his care, guidance, and disposal.>
<15:22 When in humility and penitence men return to God, trusting in Jesus Christ for what they need he rejoices to receive them with exceeding great joy; and notwithstanding all their transgressions, he pardons them freely, and bestows upon them the blessings of his salvation.>
<15:23 Be merry; be joyful and happy; literally, eating, let us rejoice.>
<15:24 My son was dead--lost; he was dead to excellence and to happiness and dead as to being the means of either to his father's house. He was lost to goodness, to duty, and to heaven. Alive--found; he has returned with right feelings to his father and friends, and is a source of rich enjoyment to himself and them. Who, not lost to goodness, would not be partaker of their joy?>
<15:25 His elder son; he represents the scribes and Pharisees, who found fault with Jesus for receiving and kindly treating sinners who came to him. Music and dancing; expressions of joy.>
<15:30 This thy son; an expression of scorn and pride. He refuses to say. This my brother. Devoured thy living; squandered the property assigned to him.>
<15:31 Thou art ever with me; so that thou hast the full enjoyment of the portion of the estate reserved for thee. All that I have is thine; the younger son having received his portion of the estate, what remained would be now enjoyed by the other son, and fall to him when the father had done with it.>
<15:32 It was meet; suitable, proper. Had the elder son felt right, he would have thought so; and instead of murmuring, would have partaken of the joy. So with the scribes and Pharisees: had they felt right, instead of murmuring at Christ for receiving penitent sinners, they would have rejoiced with him and all the good on earth and in heaven, with exceeding joy.>
<16:1 There was a certain rich man; in this parable our Lord teaches the necessity of spiritual wisdom and forethought in providing for the world to come, by an example of worldly shrewdness. Its immediate reference is to the use which God requires us, as his stewards, to make of the property which he entrusts to us. But it includes all other gifts and opportunities of doing good. Steward; one intrusted with property, to be used according to the will of its owner.>
<16:2 Riches and all the blessings which men possess come from God and belong to him. With them men, as his stewards, are intrusted for a season. For the use of them they must give account, and they will be treated according to their works.>
<16:3 Said within himself; he thought. I cannot dig; work at any service labor.>
<16:6 Thy bill; thy writing. Write fifty; by allowing the debtor to alter his bill and diminish it one half. the steward hoped to gain his favor, and thus, in time of need, to secure his aid.>
<16:7 Fourscore; eighty.>
<16:8 The lord; the master of the steward. Commended; not his injustice, but his sagacity. Done wisely; acted shrewdly; manifested forethought and skill. Children of this world; those who seek earthly things as their chief good. Wiser than the chidren of light; more sagacious in the selection, and more skilful in the application of means to obtain temporal, than Christians are to obtain eternal good.>
<16:9 Of the mammon of unrighteousness; by the right use of it, as the original implies. Mammon is a Chaldee word signifying riches. It is here called the mammon of unrighteousness, as being with unrighteous men the great object of pursuit, and too commonly sought, moreover, by unrighteous means. That when ye fail; are discharged from your stewardship by death. They may receive you; that is, the friends whom you have made by bestowing your property in deeds of love and mercy. Our Saviour used the words, "they may receive me into their houses." They do not receive us by any right or authority of their own, for this belongs to Christ alone; but they welcome us to heaven and bear witness to our deeds of mercy, as being the evidence and fruit of that "faith which worketh by love." Compare the remarkable passage in Mt 25:34-46.>
<16:10 Faithful; as God's steward. In that which is least; our Saviour teaches that it is not the quantity committed to us that God will regard, but our fidelity in using it; and that our disposition is as thoroughly tried by a small as by a large amount of property or influence.>
<16:11 Unrighteous mammon; worldly things. True riches; heavenly treasures--satisfying, eternal good. To be happy hereafter, men must be honest towards God here. If they continue knowingly to rob him of what he gives them on earth, he will never bestow on them the riches of heaven.>
<16:12 Not been faithful; if not honest as stewards in what God committed to you for time, no one will give you heavenly riches for eternity.>
<16:13 God and mammon; Mt 6:24.>
<16:14 Derided him; because of the doctrine contained in the preceding discourse, in which he taught that all our wealth belongs to God, and that to obtain heaven, we must faithfully use it in his service, and that too with undivided love and devotion.>
<16:15 Justify yourselves; you pretend before men to be just and good, and are by them highly esteemed; but God, who sees your hearts, abhors and condemns your hypocrisy and worldliness.>
<16:16 Were until John; see notes on Mt 11:12,13. Every man; the Saviour alludes to the fact that the despised publicans and sinners are pressing into the kingdom of heaven, while the proud Pharisees reject it.>
<16:17 One tittle of the law to fail; he shows that the gospel--the kingdom of heaven which he has come to establish--does not relax the strictness of the divine law. Of this he gives, in the next verse, an instance.>
<16:18 Putteth away his wife, and married another; the Saviour here connects covetousness with licentiousness, both being sins of the Pharisees growing out of the common root of worldliness, and both excluding men from the kingdom of heaven.>
<16:19 Clothed in purple; an indication of great wealth. Fared sumptuously; lived in a luxurious and costly manner. A man's condition in this world is no certain criterion of his character. A wicked man may be rich and surrounded with all the comforts and luxuries of life, while a good man may be poor, afflicted, and helpless. He may want even that which is squandered by the wicked on their dogs. 19-31. To illustrate the folly, guilt, and ruin of being dishonest towards God and employing what he gives only in self-indulgence, our Saviour gave this account of the rich man and Lazarus.>
<16:20 Begger; literally, a poor man. Laid at his gate; there was then no public provision for the poor, and when disabled, they were often laid at the gates of the rich, that they might receive aid.>
<16:22 Abraham's bosom; a common expression among the Jews for the rest and bliss of heaven. Good men and bad must die. But their souls will live after death, in heaven or hell, according to their character. An impassable barrier will divide them. Those in heaven cannot help those in hell, and none from hell can ever ascend to heaven.>
<16:24 Father Abraham, have mercy on me; this shows that he was a Jew, or one of Abraham's descendants. This is the only instance mentioned in Scripture of any one praying to a departed spirit, and he gained nothing by it but an increase of torment. Prayer is an act of religious worship, and the command of Jehovah is, "Thou shalt worship the Lord thy God, and him only shalt thou serve." Mt 4:10.>
<16:25 Good things; wealth, honor, and pleasure. Evil things; poverty, contempt, and distress. The faculty of memory is a great blessing; but men may so conduct in this world that the exercise of it will for ever torment them in the world to come.>
<16:26 Neither can they pass; there can be no interchange of places between those in heaven and those in hell.>
<16:29 They have Moses and the prophets; the Old Testament scriptures. Great and momentous truths are revealed by God in the Bible. If men who have the Bible and the preaching of the gospel are not led, under the influence of the Spirit, to believe, no other means would be availing; but they will be left to pursue their chosen course of wickedness to the place of endless torment.>
<16:31 Neither will they be persuaded; persuaded to repent.>
<17:1 Impossible; such is the wickedness of men, that they will commit sin, and tempt others to sin. Mt 28:6,7; Mr 9:41,42. Men may be so wicked as to make it certain that they will commit great sins, and strongly tempt others to sin; and yet that certainty not diminish their responsibility or lessen their guilt.>
<17:3 3, 4. Forgiveness. Mt 18:15-22.>
<17:4 Increase our faith; see note on Mt 17:20.>
<17:5 To do their duty, all men need an increase of faith; and as Christ is the author and finisher of faith, all should habitually look to him for this inestimable gift. Heb 12:2.>
<17:6 Sycamine; the same as sycamore. Mt 17:20.>
<17:7 By and by; rather, immediately. The meaning is, he will not at once direct him to take his meal, but will have him wait till he has first served his master. 7-10. These verses inculcate the duty of obedience, patience, and humility; that after all the disciples had done or would do, their reward must be of grace, not of debt.>
<17:9 I trow not; think not.>
<17:10 No man ever did or ever can do for God more than He requires; and no mere man ever did his whole duty. Of course, no man can perform works of supererogation, that is, more than enough to save himself; he cannot do enough to insure his own salvation, nor can he ever be saved except through the grace of God in Jesus Christ.>
<17:12 Men that were lepers, which stood afar off; lepers were not allowed to dwell with or come near to persons in health. Le 13:46; Nu 5:2,3; Mt 8:2-4.>
<17:14 Show yourselves unto the priests; to obtain their testimony that they were really cured, and might be again admitted into society. Cleansed; healed.>
<17:18 This stranger; a foreigner, as were the Samaritans, and not a Jew. In this transaction the Lord saw foreshadowed the bringing in of the Gentiles to his church. However great or numerous the temporal favors God bestows upon men, few comparatively give him the glory--and those, only through the riches of divine grace.>
<17:19 Thy faith hath made thee whole; his confidence in Christ was the means, and the power of Christ the cause of his cure.>
<17:20 Kingdom of God; the reign of the Messiah. Mt 3:2. Not with observation; not with outward pomp and display, so that you can mark its progress, as you would that of an army, and say of it, "Lo here!" or, "Lo there!">
<17:21 Within you; the true reign of Christ is in the hearts of men, and it had already begun among them.>
<17:22 One of the days of the Son of man; he refers to the awful calamities about to come on the Jewish nation for their rejection of himself, when the unbelieving multitudes, who had rejected their true Messiah, would in vain wait and pray for the Messiah of their own imaginations; and even his disciples would desire the return of one of those blessed days when their Master was with them. At that time false Christs would appear, and they might be tempted to follow them; but he warned them not to do it.>
<17:23 See here; or, See there; to witness the works of these pretended Messiahs. Mt 24:23-27.>
<17:24 In his day; the day when Christ shall come to destroy his enemies, deliver his friends, and establish his kingdom. There is the same double reference here to Christ's providential coming to destroy the city and temple, and to his second personal coming, which has been noticed in the notes to Matthew, chap Mt 24.1-51. It will be like the lightning, which fills the heavens from one end to the other with its brightness.>
<17:25 Suffer many things; Mr 8:31.>
<17:26 26-31. Christ's coming. Mt 24:17,18, 37-39; Ge 19:23-25.>
<17:32 Lot's wife; she lost her life by disobeying God's command. Ge 19:17,26. So, if men do not follow Christ's directions, they will perish.>
<17:33 Seek to save his life; by disobeying the will of Christ. This declaration was fulfilled at the destruction of Jerusalem by the Romans, in respect to the temporal life of the Christians; and it will be fulfilled at the last day in its highest sense, in repsect to the eternal life of all believers, even though they may have been slain for Christ's sake. Mt 10:39.>
<17:34 Faith in Christ is the great characteristic of a saint, and the want of it, of a sinner. This makes a mighty difference in their character, condition, and prospects. Though they live in the same family, work in the same field, or sleep in the same bed, one, believing Christ, is led to follow his directions and be saved; the other, not believing him, neglects his directions, and is lost. 34-36. One shall be taken; Mt 24:40,41.>
<17:37 Where, Lord? where will such calamities come? Wheresoever the body is; wherever the unbelieving Jews are, there will their destroyers be upon them, as eagles upon their prey. Mt 24:28.>
<18:1 Always; habitually, perservingly. Not to faith; not to be discouraged, or cease to pray. No man fulfils his obligations to God or to himself who is not in the habit of daily prayer, and who is not sincere in his supplications for himself and his fellow-men.>
<18:3 Avenge me of mine adversary; by attending to my suit, and compelling him to do me justice.>
<18:6 Hear; attend and receive the instruction which this case affords.>
<18:7 Shall not God avenge his own elect; the argument is from the less to the greater. If importunity had such power with an unjust judge, who cared not for the poor widow's cause, how much more shall God, the just judge, who tenderly cares for his people, vindicate and deliver them from their foes? Cry day and night; pray daily, habitually. Though he bear long; though for a long time he delays to answer.>
<18:8 He will avenge them speedily; though the time may seem long to them, it is still short; for it is not delayed a moment beyond the proper hour. See 2Pe 3:8,9. Cometh; to avenge his elect. Shall he find faith; an intimation that God's delay will try the faith of even his true disciples.>
<18:9 A high opinion of our own goodness in comparison with that of others, and a disposition to exalt ourselves, are exceedingly offensive to God; while a deep conviction of our own unworthiness, hearty contrition for sin, and humble supplication for mercy, are his delight.>
<18:12 Tithes; a tenth part.>
<18:13 The publican, standing afar off; at the outer side of the temple, far from the holy place, as not worthy to approach it. Smote upon his breast; in token of his distress in view of his sins.>
<18:14 Justified; approved and accepted. Chap Lu 14:11.>
<18:15 Infants are not too young to need a Saviour; parents should feel this, and in faith implore for them his grace. He died to redeem them, and delights to have all parents seek for their children the blessings of his salvation. 15-30. Children brought to Christ; the rich ruler. Mt 19:13-30.>
<18:22 Christ, in his providence, tries the hearts of men, and gives them opportunity to see themselves, and to manifest to others whether they love him supremely. If they do and show this by giving up whatever is incompatible with his will, they greatly increase their present enjoyment, and secure eternal life.>
<18:31 All things that are written by the prophets; those who in the Old Testament had foretold his death. Da 9:25-27; Isa 53:3-10. 31-33. Christ foretells his death. Mt 20:17-19.>
<18:34 Understood none of these things; the things concerning his death, of which he had spoken. They supposed that the Messiah would be a great temporal prince, and reign for ever. Their wishes for this, and their expectations of it, were so strong that they did not believe or apprehend the meaning of what he had said. Mt 16:21-23; 20:17-19. Desire has great influence on belief. It is exceedingly difficult to make men correctly apprehend and cordially believe what they are opposed to; while they readily understand and easily believe what they wish to have true.>
<18:35 As he was come nigh unto Jericho; was near to, or in the vicinity of Jericho. Matthew and Mark say that it was as he departed from or went out of Jericho. Matthew also says that there were two blind men: Mark and Luke mention but one, probably because he was the most distinguished and best known. Mt 20:29-34; Mr 10:46-52.>
<19:3 The press; the great crowd of people.>
<19:4 That interest in Jesus Christ which leads men to wish to know more of him, and to put themselves in the way of obtaining this knowledge, may be, and often is, the means of their salvation. He is more ready than men suppose to meet them, and bestow upon them the blessings of his grace.>
<19:8 False accusation; charging a man, and taking from him more than he owed. Fourfold; four times as much. Ex 22:1; Nu 5:6,7. True repentance, wherever it exists, will dispose those who exercise it to "do justly, love mercy, and walk humbly with God." If they have wronged others, it will lead them to make full restitution, and as they have opportunity, to do good to such as they have injured, and to all.>
<19:9 He also is a son of Abraham; by outward descent, and has also the faith of Abraham.>
<19:10 That which was lost; Mt 18:11.>
<19:11 The kingdon of God should immediately appear; they thought that when Christ should enter Jerusalem he would proclaim himself king, deliver them from the Romans, and raise them to great earthly renown. To correct this idea, and give them right views of his kingdom, Christ spoke the following parable, which has some striking points of agreement with that recorded in Mt 25:14-30, but also differs from it in some essential particulars. There, different sums are intrusted to the servants, "to every man according to his several ability," and the same fidelity and success receive the same reward; the idea being, that God considers not the amount intrusted to his servants, but the use they make of it. Here, the point is, that God will bestow upon his servants rewards proportioned to their diligence in his service, and for this reason the difference in the amount of gifts is not made account of, but all receive the one pound.>
<19:12 A certain nobleman; this represented Jesus Christ. By his going to a far country is represented his ascending into heaven, which he must do before he would establish his kingdom on earth.>
<19:13 Ten pounds; the gifts and opportunities of doing good with which he intrusts men. Occupy; use in a right manner.>
<19:14 His citizens; the Jews. Joh 1:11.>
<19:20 A napkin; a towel or cloth.>
<19:21 An austere man; hard in his dealings, harsh, and cruel. Reapest that thou didst not sow; unreasonable, requiring too much, and taking what did not belong to him.>
<19:22 Out of thine own mouth; from thine own statement. Thou knewest that I should require a strict account of the use of what I gave; why didst thou not prepare to return to me what is justly my due?>
<19:23 The bank; a place of safe-keeping and profitable use. Usury; lawful interest, as the term then implied. Mt 25:27.>
<19:25 They said unto him; the servants that stood by, verse Lu 19:24. He hath ten pounds; they are amazed that their lord should give the slothful servant's pound to the man who has already the most money in his hands.>
<19:26 For I say unto you; the nobleman, without pausing in his discourse, goes on to state the principle upon which he proceeds in the distribution of his property to his servants. Every one which hath; he who, by proper diligence, has already increased the amount committed to him. Compare Mt 25:29. Those who in this world are disposed to use the blessings which God bestows to his glory, will hereafter receive more and greater blessings; while those who are not, will be deprived of what they now have, and be left destitute and wretched.>
<19:27 Slay them before me; an emblem of the ruin which would come on his persevering opposers.>
<19:28 28-40. Christ rides into Jerusalem. Mt 21:1-16.>
<19:40 The stones would immediately cry out; a proverbial expression, denoting the strong reasons there were for praising him, and the necessity that such priase should be offered to him. The character and works of Christ furnish the most abundant and powerful reasons for blessing and praising him; and neither he nor his friends are disposed to prevent any from doing this. It gives them joy, and they desire that all should join in it.>
<19:41 Wept over it; in view of its guilt, and the miseries which were coming upon it.>
<19:43 Thine enemies; the Romans. Compass thee round; this they did by digging a trench around Jerusalem. See Josephus, Jewish Wars, book 6.>
<19:44 Thy children; the inhabitants of the city. One stone upon another; Titus, their conqueror, caused the very ground to be ploughed up, in fulfilment of this declaration. The time of thy visitation; the time when mercy was offered, and they were entreated to accept it and be saved.>
<19:45 45, 46. Traffickers cast out. Mt 21:12,13.>
<19:47 He taught daily in the temple; for a number of days before his crucifixion.>
<20:1 Rulers in church and state are often very unsafe guides in the things of religion, and much more opposed to the truth and to those who preach it than are the common people. Hence the great importance of following the direction of Christ, Mt 23:8-10, and of searching the Scriptures, to see whether what rulers and teachers inculcate is according to them. If it is not, all should reject it. 1-8. Christ's authority. Mt 21:23-27.>
<20:9 Parables and similitudes drawn from earthly things, with which people are familiar, are wise and efficient means of inculcating divine truth; and that preaching which leads the hearers to think, and draw correct conclusions for themselves, will be likely to do them the most good. 9-18. The parable of the vineyard Mt 21:33-44.>
<20:19 Opposers of the gospel and of its faithful preachers are prone to represent their teaching as injurious to the state, and thus to attempt to set politicians and worldly men in active hostility against them. In this they would oftener succeed were it not that the truths of the gospel, plainly and kindly exhibited, commend themselves to the conscience, and thus through grace secure the common people in their favor. 19-38. Tribute to Cesar--the resurrection. Mt 22:15-33.>
<20:34 Children of this world; men in this world.>
<20:35 That world; the world of blessedness into which the righteous enter after the resurrection.>
<20:36 Equal unto the angels; in their immortality and bliss. The children of God--resurrection; like unto him after having been raised from the dead.>
<20:38 All live unto him; though dead as to us, they live to and with God; so that his covenant with them to be their God remains.>
<20:40 They; the Sadducees.>
<20:41 41-44. Christ David's son and Lord. Mt 22:41-46.>
<20:45 45-47. Scribes denounced. Mt 23:1-33.>
<20:46 Instead of always imitating the rich and great, and following those who are in high stations, it is sometimes necessary, in order to obey Christ, to reject their maxims, renounce their doctrines, and avoid their practices. This course may subject those who pursue it to many inconveniences, but it will honor the Saviour, benefit themselves, and promote the good of mankind.>
<21:1 1-4. The widow's two mites. Mr 12:41-44.>
<21:3 The value, in the sight of God, of what is given for his worship or for charitable purposes, depends not so much on the amount, as on the amount compared with the ability and the motives with which it is done.>
<21:5 Goodly stones; great and beautiful. Gifts; donations which had been received, and were suspended in different parts of the temple. 5-24. Destruction of Jerusalem. Mt 24:1-22. For the principles on which this prophecy is to be interpreted, and the double reference contained in it to the destruction of Jerusalem and the end of the world.>
<21:11 Fearful sights and great signs shall there be; these words had their primary fulfilment in the fearful appearances which were seen previous to the destruction of Jerusalem, and which are particularly described by Josephus in the sixth book of his history of the Jewish wars. A more awful fulfilment awaits them when "the end of the world" draws nigh.>
<21:13 Turn to you for a testimony; it shall give you the opportunity of bearing testimony to my gospel before kings and rulers. This will be a testimony in your behalf of your faithfulness, and against them, if they reject it. Compare Mt 24:14; Mr 13:9.>
<21:15 A mouth and wisdom; ability rightly to speak, a gift which no man or angel could bestow.>
<21:18 Not a hair--perish; you shall suffer no real harm, though you die for my sake. Compare Ro 8:28-39; 1Co 3:21-23; 1Pe 3:13. However great the wickedness of men, and however active their opposition to the people of God, they shall not in the end be able to injure them. His people should therefore be calm and quiet, as well as active and persevering in their labors for the promotion of his cause, trusting in him for whatever they need.>
<21:19 In your patience possess ye your souls; the word "possess" is here to be taken in the sense of gaining or saving. The whole verse might be rendered, By your endurance save ye your souls; the same as, "He that shall endure unto the end, the same shall be saved." Mt 24:13.>
<21:24 By the edge of the sword; eleven hundred thousand were slain. Led away captive; ninety-seven thousand were carried into captivity. Trodden down; desolated, oppressed. This has been done successively by the Romans, Saracens, Mamulukes, Franks, and by the Turks who continue to exercise dominion over and oppress it. The times of the Gentiles be fulfilled; the times during which God has determined that the Gentiles shall tread down Jerusalem.>
<21:25 The waves roaring; in the first application of the prophecy to the overthrow of Jerusalem, these words are to be understood figuratively of commotions among the nations. 25-33. Signs of Christ's coming--parable of the fig-tree. For the exposition of these verses and their twofold reference, see notes on Mt 24:29-35.>
<21:26 The powers of heaven; the same as the hosts of heaven; that is, the sun, moon, and stars.>
<21:34 Be overcharged with surfeiting; made heavy and careless by immoderate eating and drinking. That day; the day when Christ will come to save his people and take vengeance on his foes. These exhortations were applicable to the day of which he had been speaking, to the day of death, and to the day of judgment. Excessive eating and drinking tend not only to produce various bodily diseases, but to blind the mind, stupefy the conscience, and corrupt the heart. Christians should not indulge in these sins, which unfit them for the discharge of their duty, and prevent their being prepared for the coming of Christ. 34-36. Warning to be ready for Christ's coming. Mt 24:36-51.>
<21:36 Always; habitually. Escape all these things; the woes that would come on the wicked. Stand; stand with acceptance.>
<21:37 37, 38. These verses show the manner in which Jesus Christ spent the last days of his life, teaching the people the great truths of salvation. All who heard ought with the heart to have believed and been saved. So it is with all who hear the gospel now.>
<22:1 The feast of unleavened bread; the passover; so called, because during that feast they ate nothing that was leavened. Mt 26:2,17.>
<22:3 Then entered Satan into Judas; exerted over him peculiar influence, and excited him to betray his Lord.>
<22:4 Captains; those who commanded the guard that kept watch at the temple. 4-13. Judas selleth Christ; the passover prepared. Mt 26:14-19.>
<22:5 Money has a powerful influence over the human heart and the hope of obtaining even a small sum may awaken the most corrupt desires, and lead to the commission of the most horrid crimes. Well did he who knew what was in man say, "Take heed, and beware of covetousness.">
<22:7 The passover must be killed; the lamb which was eaten at the passover-festival must be killed, and preparations made for the feast.>
<22:14 The hour was come; the hour appointed for the eating of the paschal lamb. Mt 26:20.>
<22:15 With desire I have desired; a Hebrew form of expression signifying, I have greatly desired.>
<22:16 Until it be fulfilled in the kingdom of God; till the kingdom of God come by the offering of Christ, the true paschal lamb. Ver Lu 22:18.>
<22:17 The cup; not the cup mentioned verse Lu 22:20, but the cup that was used in celebrating the passover.>
<22:18 I will not drink; for the meaning of these words, see notes on Mt 26:29. What was drunk at the ordinance of the supper was not blood, but the fruit of the vine, the juice of the grape. This Christ used, and it is a fit emblem of his blood, which was shed for the remission of sins.>
<22:19 19, 20. The Lord's supper. Mt 26:26-30.>
<22:21 21-23. What is mentioned in these verses took place while they were eating the passover, and before the institution of the Lord's supper. The facts are not all mentioned in the order in which they took place. Joh 13:30; Mt 26:21-25; Mr 14:18-21.>
<22:24 The greatest; the most honorable--have the highest offices in that earthly kingdom which, notwithstanding all his instruction, they still thought he was about to establish. It seems to have been in connection with this strife about preeminence that our Lord rose from the table and washed his disciples' feet, thus setting them an example of condescending humility. Joh 13:4-17. Similiar contests are mentioned in Mt 18:1; 20:20-28.>
<22:25 Benefactors; a title of honor applied to earthly rulers, especially the kings of Syria and Egypt, and such as exercised dominion over others.>
<22:26 Not be so; in the kingdom of Christ, one of his ministers was not to have dominion over the others. They were all brethren, and he would be the greatest who should be the most humble, and most ready to perform all useful services.>
<22:28 Temptations; trials, labors, and sorrows.>
<22:29 I appoint unto you a kingdom, as my Father hath appointed unto me; that is, as the next verse shows, I make you sharers with me in the kingdom which I have received of my Father. But this is a spiritual kingdom, in which the most humble and active in serving their brethren are the greatest.>
<22:30 Eat and drink at my table; be admitted to near communion with me, as are the servants of a king who stand high in place and honor. Sit on thrones; to administer, under spiritual kingdom. In the preeminent place held by the apostles in the establishment of the Christian church, we find the earthly fulfilment of this promise. Another fulfilment is reserved for "the regeneration," when Christ shall make all things new. See note on Mt 19:28.>
<22:31 Sift you; all the apostles. As wheat; greatly agitate your minds with inward and outward trials, to induce you, if possible, to deny men, apostatize, and perish.>
<22:32 For thee; while he intercedes for all the apostles, he offers a special prayer for Peter. Fail not; not utterly fail, but when weakened revive again and become triumphant. Converted; turned from thy sin. Jesus Christ is acquainted with all the dangers of his people, and guards them, that when they fall, they may rise again, and be for ever victorious over all their foes.>
<22:33 33, 34. Peter's denial foretold. Mt 26:33-35.>
<22:35 When I sent you; Mt 10:9,10. Lacked ye any thing? did you suffer with want?>
<22:36 Let him sell his garment, and buy one; a figurative mode of warning the apostles that great difficulties and trials awaited them, which would require them to be like armed warriors, ready for the conflict.>
<22:37 This that is written; Isa 53:12.>
<22:38 It is enough; they needed the sword of the Spirit, the shield of faith, the breastplate of righteousness, and the helmet of salvation.>
<22:39 39-46. Agony in the garden. Mt 26:30-46; Mr 14:32-41.>
<22:43 Strengthening him; as a man he needed and received aid from on high.>
<22:47 47-71. Christ betrayed, led to the high-priest, and denied. Mt 26:47-75; Mr 14:43-72.>
<22:48 With a kiss; customarily a sign of affection, but now used by Judas to point out to the soldiers which was Jesus. Persons who, knowingly, for their own selfish ends, express towards Jesus Christ that which they do not feel, imitate Judas the traitor; and unless they repent and are forgiven, it will be true of them as it was of him, that it would have been better for them if they had not been born. Mt 26:24.>
<22:51 Suffer ye thus far; addressed to the guard who had seized him immediately upon the kiss of Judas. Mt 26:48; Mr 14:45,46. The probable meaning is, Allow me thus far; that is, give me liberty so far as to touch this man's ear. Some, however, suppose him to mean, Suffer the zeal of my disciples to have proceeded thus far in defending me.>
<22:60 Man, I know not; Matthew and Mark say that a maid in the second instance charged Peter with being one of Christ's disciples. But he here answers to a man. The maid may have stated it to those that were present, and a man then have charged Peter with it.>
<22:67 If I tell you; you do not ask for the sake of gaining information; your minds are fully made up to condemn me, and no statement of mine can influence your belief.>
<22:68 If I also ask you; propose questions to you to be answered, as the Saviour often did in reasoning with the Jewish rulers. For an example, see Mt 21:23-27. Nevertheless, as his time for suffering had now come, he proceeded solemnly to affirm his messiahship, Lu 22:69,70.>
<22:69 Sit on the right hand of the power of God; this they justly considered as his claiming to be the Messiah.>
<22:70 Ye say that I am; this meant, Yes, I am.>
<22:71 Have heard; have heard his claim to be the Son of God, which they said was blasphemy.>
<23:1 1-5. Christ brought before Pilate. Mt 27:1,2,11-18.>
<23:2 Perverting the nation; exciting the people to rebel against the Roman government. Give tribute; pay taxes. The accusation which the Jews brought against Jesus before Pilate was not that of blasphemy in claiming to be the Son of God, for which their council had condemned him. Mt 26:65; it was that of treason against the Roman government. His claiming to be the Son of God was no crime in a Roman court. They could not induce Pilate for this to put him to death. They therefore invented another accusation, and sought false witnesses to support it. So that while they condemned him for what was no crime, they sought to have Pilate condemn him for a crime which he never committed. Justly did he who knew what was in them, say, "Ye serpents, ye generation of vipers, how can ye escape the damnation of hell? Mt 23:33.>
<23:3 Thou sayest it; Jesus not only acknowledged himself to be the King of the Jews, but explained to Pilate that his kingdom was not of this world; so that he found no fault in him on this ground. Joh 18:36,37.>
<23:5 He stirreth up the people; causing tumults among them. All Jewry; the whole Jewish country.>
<23:7 He sent him to Herod; seeking thus to get rid of the case altogether. This was Herod Antipas, who put John to death. He was son to Herod the Great, under whose reign Christ was born, Mt 2:1; uncle to Herod Agrippa, by whom James was killed, and who was eaten of worms, Ac 12:2, 23; and great-uncle to that Agrippa who was almost persuaded by Paul to become a Christian. Ac 25.27.>
<23:9 He answered him nothing; the Saviour would not answer questions prompted by a vain curiosity, without any desire to know the truth.>
<23:11 Men of war; the soldiers of his guard. Set him at naught; treated him with contempt.>
<23:15 Is done unto him; rather, is done by him. The two examinations before Pilate and Herod had brought to light no action of Jesus worthy of death.>
<23:16 Chastise; scourge or whip him.>
<23:17 Of necessity; it was the custom, and the people would be displeased if he should not comply with it. Mt 27:15. This shows the manner in which the word necessity is sometimes used in the Bible.>
<23:18 18-25. Christ condemned by Pilate. Mt 27:20-26.>
<23:22 As civil government is an ordinance of God, established for the protection of the innocent, and the condemnation and punishment of the guilty, magistrates who, against evidence, acquit the guilty and condemn the innocent, are an abomination to the Lord. Injustice under the cover of law is aggravated wickedness. False witnesses and corrupt judges merit, and without repentance will receive a most awful condemnation.>
<23:26 Simon, a Cyrenian; Mt 27:32.>
<23:28 Weep for yourselves, and for your children; on account of the great calamities that were coming upon them.>
<23:29 Blessed are the barren; it were better not to have children than to have them exposed to such distresses.>
<23:30 To the mountains, Fall on us; to shelter us from the wrath of God. Compare Ho 10:8; Re 6:16. They feel that it is better to be crushed beneath the weight of the mountains, than to meet God in judgment, and bear the fierceness of his wrath.>
<23:31 If they do these things in a green tree; this was a proverb. A green tree represented one innocent and good; a dry tree, those who were wicked. If such distress as that of crucifixion would come upon one who was perfectly innocent, what would be the distress which, under the just indignation of God, would come upon those who were so exceedingly wicked as to murder his beloved Son?>
<23:32 Malefactors; evil-doers, thieves, robbers, etc. Mt 27:38.>
<23:33 33-49. The crucifixion. Mt 27:33-56.>
<23:34 Father, forgive them; this was a prayer for the pardon of his murderers. Jesus Christ, who was most compassionate and benevolent in his life, was also most kind and forgiving in his death. He desired the everlasting salvation even of his murderers, and for it he was willing to give up his life.>
<23:39 One of the malefactors; Matthew and Mark speak as if both the robbers had at first reviled him. One however repented, rebuked his companion, confessed the justice of their punishment, and looked to Christ for salvation.>
<23:40 True repentance leads a sinner to feel and acknowledge the justice of his condemnation, and at the same time to look to Jesus for salvation; expecting through him to receive it, that when absent from the body he may be present with the Lord. 2Co 5:8.>
<23:42 When thou comest into thy kingdom; more literally, When thou comest in the kingdom; that is, when thou comest in glory as the King Messiah. This will be when he shall "appear the second time without sin unto salvation" for all who have believed in him.>
<23:43 Paradise; the place and state of blessedness.>
<23:50 A counsellor; a member of the great Jewish council. 50-56. Christ's burial. Mt 27:57-61.>
<23:51 Deed of them; the doings of the council in condemning Christ. Waited for the kingdom of God; the reign of Jesus as the Messiah. Mt 3:2.>
<23:54 That day was the preparation; the preparation for the solemnities of the next day, which was the Sabbath.>
<24:1 1-12. The resurrection. Mt 28:1-10.>
<24:4 Two men; angels in the form of men. Matthew and Mark mention but one. This does not make their statements inconsistent with each other, but it shows that they did not write in concert, and that each gave his own independent account.>
<24:7 Events which the friends of Christ most dread, and against the occurrence of which they most earnestly strive, are often essential to their highest good. They should therefore never mourn as those who have no hope, at any thing which God does or suffers to be done, but should say with submission, "It is the Lord; let him do as seemeth him good.">
<24:13 Two of them; not of the eleven apostles, but of the other disciples. Threescore furlongs; ten ancient furlongs are equivalent to the modern geographical mile. The distance from Jerusalem to Emmaus was, then, about six geographical, or nearly seven English miles.>
<24:16 Their eyes were holden; kept from discovering who he was. Mr 16:12.>
<24:18 Cleopas; supposed to be the same as Alpheus, the father of James the less and Jude.>
<24:21 Redeemed Israel; they were thinking of an outward deliverance from bondage to the Romans, and the restoration of the kingdom to Israel. Compare Ac 1:6.>
<24:25 O fools; this is not the same word in the original as that which Christ condemns. Mt 5:22. That implied great wickedness; this, dulness, want of reflection and discernment, as manifested in not better understanding his teaching and the Scriptures. Slow of heart; indisposed and reluctant to believe that he must die. A more intimate acquaintance with the Scriptures, and a better understanding of them, would throw great light upon the providence of God; while it would support his people in trials, would nerve them for duty, and furnish them more thoroughly for every good work.>
<24:26 Ought not Christ; was it not needful, in order to fulfil the prophecies, that the Messiah should die? Did they not clearly foretell that he would be cut off, but not for himself? Da 9:26.>
<24:27 At Moses; the books which Moses wrote, the first books of the Old Testament. The things concerning himself; some of these things are mentioned in Ge 3:15; 49:10; Nu 21:8,9; De 18:15; Isa 9:6,7; Isa 53:1-12; Ps 16.1-11, 22.1-31, 110.1-7; Da 9:25-27; Mal 4:2-6.>
<24:28 Made as though he would have gone further; he kept on, giving no intimation that he would stop, till they entreated him to do so.>
<24:29 Constrained him; by their entreaties.>
<24:30 Sat at meat; reclined, as the custom was, at supper. Blessed it; asked the blessing of God, and gave thanks: in this he hath set us an example which we should follow.>
<24:31 Their eyes were opened; the difficulties in the way of their knowing him were removed. He vanished; disappeared. They saw him no more.>
<24:32 Our heart burn; glow with wonder and delight. Opened to us the scriptures; explained to us their meaning.>
<24:34 Simon; Simon Peter.>
<24:36 Peace be unto you; this was a form of salutation, and an expression of good-will.>
<24:38 Thoughts; doubts and suspicions as to who and what he was.>
<24:39 That it is I myself; that I have really arisen from the dead.>
<24:40 Showed them his hands and his feet; compare Joh 20:27.>
<24:41 Believed not for joy, and wondered; the news was so strange and joyful, that they did not believe it.>
<24:43 Took it, and did eat before them; this was further proof of the reality of his bodily presence with them.>
<24:44 Law of Moses--prophets--psalms; these comprehended the whole of the Old Testament; and were the names of the three portions into which it was divided by the Jews. The events of divine providence are a fulfilment of the divine word. In order to see this, and be rightly affected by it, men must have their understandings enlightened and their hearts purified by the Holy Spirit. They should therefore habitually seek his teaching; and while "careful for nothing," should "in every thing by prayer and supplication with thanksgiving make known their requests unto God." He will then "supply all their need, according to his riches in glory by Christ Jesus.">
<24:45 Understand the scriptures; those portions of them which foretold his death and resurrection, which they did not correctly understand before.>
<24:46 It behooved Christ to suffer, and to rise; otherwise he could not fulfil the prophecies concerning him, prove that he was the Messiah, or procure the salvation of men.>
<24:47 Beginning at Jerusalem; the dwelling-place of his murderers, who had enjoyed and hitherto resisted all the means of grace. The Saviour died and rose again, that repentance and forgiveness of sins might be preached in his name to all nations; and it is his revealed will that this should be done. All therefore, as they have opportunity and ability, should aid in this work, that the knowledge of his salvation may, through grace, be enjoyed by all people.>
<24:48 These things; his life, miracles, teaching, death, and resurrection on the third day according to the Scriptures, proving him, beyond the possibility of a mistake, to be the true Messiah, and that through repentance and faith in him, and in this way only, men can be saved.>
<24:49 The promise of my Father; his promise to give them the Holy Spirit, to enable them to work miracles in confirmation of the truth of their testimony, and to fit them for the duties of their office. Joe 2:28,29; Ac 2:16-21. Power from on high; the power which the Holy Spirit would give them to speak the various languages in which they would be called to preach, and do whatever might be needful to extend a knowledge of the gospel, and promote the salvation of all who should embrace it.>
<24:50 Bethany; a village on the side of mount Olivet, about two miles east of Jerusalem.>
<24:52 Worshipped him; as the Messiah, the Son of God, and the Saviour of men. As the disciples who had been most fully instructed by Christ, worshipped him as he ascended to heaven, and as the inhabitants of that world worship him, ascribing "blessing, and honor, and glory, and power to Him that sitteth on the throne, and to the Lamb for ever and ever;" all to whom he is made known should worship him also, that they may be fitted to join the employments and partake of the joys of heaven. Re 5:7-14.>
<24:53 Continually in the temple; they worshipped there daily till the feast of Pentecost, which took place in about ten days. Then the Holy Spirit descended upon them in cloven tongues, like as of fire, and the promise, verse Lu 24:49, was fulfilled. Ac 2:3.>
\kniha{John}
\zkratka{John}
<1:1 In the beginning; of the creation. He who was with God in the beginning of all created things, is before all things, and has existed from eternity. Compare chap Joh 17:5; Col 1:17. The Word; a term applied by John to the second person of the Godhead in his eternal existence. Just as he is called "the Life" and "the Light," because he has in himself life and light, and imparts them to his creatures at his pleasure; so he is called "the Word," because in him "are hid all the treasures of wisdom and knowledge," and by his word and Spirit he reveals them to men. See Re 19:13. With God; in holy union, fellowship, and communion. Was God; this is a direct assertion of his divine nature as existing from eternity. The Bible reveals that Jesus Christ is God, the maker of all things that ever were made. All who have the Bible are therefore bound to acknowledge him in this character, and to pay him divine honors.>
<1:3 All things were made--not any thing made; all things in the widest sense. Compare Col 1:16; Heb 1:2. And as God, he upholds all things which he has made. Heb 1:3.>
<1:4 In him was life; he is the author and sustainer of all created natural and spiritual life. The life was the light of men; the Word is the light and life of men absolutely; since every kind of light and life comes from him. But here the apostle means more especially, that he who is the spiritual life is also the spiritual light of fallen men, "dead in trespasses and sins;" because it is by making them alive to God, that he enlightens them; so that the two gifts of life and light always come together.>
<1:5 Shineth in darkness; sheds its rays among the spiritually ignorant, debased, and wretched. Comprehended it not; did not understand, and therefore rejected it. Compare chap Joh 8:19; 16:3; Mt 11:25-27; 1Co 2:8,14.>
<1:6 John; John the Baptist. Mt 3:1.>
<1:7 The Light; the Messiah. Through him might believe; through the testimony of John, might believe in Christ.>
<1:8 He was not that Light; John was "a burning and a shining light," chap Joh 5:35; but he received all his brightness from the eternal Word, who alone is the true Light, because he has light in himself, and is the only source of light to men.>
<1:9 Lighteth every man; the meaning is, that all true knowledge is from Christ. As Jesus Christ is the light of the world, all who reject him walk in spiritual darkness, while those who follow him have the light of life. God, who commanded the light to shine out of darkness, shines into their minds, and gives them the light of the knowledge of his glory in the face of Jesus Christ; and in his light they see correctly spiritual things. 2Co 4:6.>
<1:10 In the world; as its Creator, Instructor, and Redeemer. Knew him not; did not apprehend his true character.>
<1:11 His own; his own land Judea, and his own people the Jews. Received him not; they did not believe on him, but rejected and crucified him.>
<1:12 Power; right, privilege. Sons of God; spiritual children, adopted into his family, and conformed to his image.>
<1:13 Were born--of God; changed, by his Holy Spirit, from supreme love of self and sin to supreme love of God and holiness. No man inherits this holy character by nature, nor can it be given to him by man. It is the gift of God alone. The change in men called being "born again," by which they become children of God, is produced not by men, but by God, and to him will for ever be all the glory.>
<1:14 Made flesh; took upon him human nature--because a man. Thus the apostle teaches, in the most direct terms, that "the man Christ Jesus" is also the Word that was with God before the world was. Beheld his glory; Mt 17:1-9; Mr 9:2-10. Only begotten of the Father; possessed of his nature, and peculiarly the object of his infinite affection, dwelling in him, knowing him, and perfectly fitted to make him known. Grace and truth; favor to the guilty, knowledge of truth, and all needed good communicated to men.>
<1:15 He was before me; because he existed from eternity with the Father. Compare chap Joh 8:58.>
<1:16 All we; disciples of Christ. Grace for grace; the fuller grace of the gospel for the less grace of the law; or, as some understand the words, continually new and larger measures of grace--all needed variety and abundance of unmerited favors. Mt 13:12.>
<1:17 The law was given by Moses; a certain measure of grace accompanied the law of Moses, else no man could have been saved under it. Yet the proper office of the law itself was not grace, but rather restraint and conviction of sin. Ro 3:20; Ga 3:19. Grace and truth came by Jesus Christ; all the grace that belonged to the dispensation of the law came through Christ, while the gospel which he revealed is itself grace and truth in full measure.>
<1:18 Declared him; revealed him. The apostle here teaches that all the revelations men have received of God, whether under the law or the gospel, had Jesus Christ for their source.>
<1:21 Elias; Elijah. Mal 4:5; Mt 11:14; 16:14; I am not; not in their sense--not Elijah in person, though he was Elijah in the sense in which Malachi had predicted him. That prophet; Jeremiah, or some distinguished prophet who they expected would appear.>
<1:23 The voice; Mt 3:3.>
<1:26 With water; in distinction from the baptism of the Holy Ghost, which Christ shall bestow. See Mt 3:11.>
<1:28 Beyond Jordan; on the east side.>
<1:29 Lamb of God; him who is to be offered as an atoning sacrifice for the sins of men. Ex 12:3; 29:38-46; Isa 53:7; 1Co 5:7; 1Pe 1:18-21; Isa 53:4. Taketh away; expiates it and removes the penalty of it from man by bearing it in his own body on the tree. 1Pe 2:22-25.>
<1:31 I knew him not; see note to ver Joh 1:33. But that he should be made manifest; as much as to say, I knew not yet who this person was, but only that he was about to be manifested to Israel.>
<1:33 I knew him not; John may have had a personal acquaintance with Jesus, but he did not know him as the one that was to baptize with the Holy Ghost. He was not authorized to say of Jesus, This is the one of whom I spoke, till he had received from God the sign named in this verse. He that sent me; God. Which baptized with the Holy Ghost; whose it is to give both the ordinary and the miraculous influences of the Spirit.>
<1:34 Bare record; ver Joh 1:19-23; Mt 3:17.>
<1:39 The tenth hour; four in the afternoon.>
<1:41 Messias--Christ; the former word being Hebrew, the latter Greek, and both signifying the Anointed One. When men find the Saviour, and experience the blessedness of trusting in him, they desire that others, especially their friends, should become partakers of their joys.>
<1:42 Cephas; a Syriac word, meaning the same as the Greek word Peter, and the English word stone, or rock. Mt 16:18.>
<1:44 Bethsaida; Mt 11:21.>
<1:45 Moses--the prophets; Lu 24:44; Ge 3:15; 49:10; De 18:15-18; Isa 9:6,7; Isa 53:2-12; Jer 23:5,6.>
<1:46 Nazareth; Mt 2:23.>
<1:47 Israelite indeed; not merely a descendant of Israel, but one who has the true character of an Israelite--a holy and believing man. No guile; not hypocritical; open, honest, upright, sincere.>
<1:48 I saw thee; he saw him in a supernatural way. This manifestation of his supernatural knowledge led Nathanael to the confession in the following verse. In secret communion with God, though unseen by men, we are never unobserved by Jesus Christ. He witnesses all our feelings, and is acquainted with our whole character. By yielding to the evidences of his truth, and improving the light we have, we receive from him greater light, and prepare for brighter manifestation of his power and glory.>
<1:49 Rabbi; Master. Mt 23:7 Son of God; this was an acknowledgment of him as the Messiah.>
<1:50 Greater things; greater and more abundant evidences of his Messiahship.>
<1:51 Verily, verily; truly, truly. When thus repeated, it denotes the great importance and absolute certainty of what was declared. Hereafter; rather, henceforward. Heaven open--the angels of God ascending and descending; the allusion is to Jacob's vision of a ladder reaching to heaven, on which the angels of God were ascending and descending. The meaning is, henceforward ye shall see a constant communication established between heaven and the Son of man: the reference is not so much to particular appearances of angels, as to the full and constant communion which the Son of man has with God, and which he gives to each of his disciples in his measure.>
<2:1 The third day; after the events recorded in the last chapter. Cana of Galilee; a town west of the sea of Galilee, a few miles north of Nazareth; so called to distinguish it from Cana, near Sidon.>
<2:2 Marriage is an ordinance of divine appointment, and a means of great usefulness and happiness. At weddings, the presence and blessing of Jesus Christ should always be sought, and every thing conducted in such a manner as will honor him, and promote the benefit of all concerned.>
<2:3 They have no wine; meaning, their supply of wine has failed; for they had wine at the beginning, verse Joh 2:10. The words seem to contain a tacit request that Jesus would now supply the deficiency.>
<2:4 Woman, what have I to do with thee? an intimation that he cannot allow her or any other person to direct in regard to the exercise of his divine power. Mine hour; his time to furnish wine by a miracle, and thus show forth his glory.>
<2:6 Six water-pots of stone; it was the custom of the Jews to have such vessels for water, for purposes of purification. Firkins; the Greek word rendered firkins is supposed to denote the same measure as the Hebrew word bath, containing about 8 7/8 gallons.>
<2:8 Governor of the feast; the person who had the general superintendence.>
<2:10 Every man; this is a statement of what was usual on such occasions. Thus the governor of the feast testified to the purity and excellence of the wine miraculously furnished by the Saviour.>
<2:11 Manifested forth his glory; showed his divine power, and thus proved himself to be the Messiah. It is never said in the Scriptures, that any mere creature ever wrought miracles to show forth his own glory. This statement, which is here made with regard to the Son of God, is peculiar to him, and is adapted to lead men to pay him divine honors. Chap Joh 5:23.>
<2:12 Capernaum; Mt 4:13.>
<2:14 Found in the temple; this cleansing of the temple was early in our Lord's ministry, and must not be confounded with that recorded in Mt 21:12.>
<2:17 It was written; Ps 69:9. Hath eaten me up; I am full of consuming desire for the honor of thy house, and the purity of thy worship. The transactions recorded in the New Testament are in many cases a fulfilment of the declarations of the Old; and the more we become acquainted with each, and with the connection of one with the other, the more clear to our minds will be the evidence of the divine inspiration and value of both--that they have one Author, tend to form one character, and promote one great and benevolent end.>
<2:18 What sign showest thou; what miracle dost thou work in proof of thy authority to do these things?>
<2:19 This temple; meaning his body, verse Joh 2:21. I will raise it up; Jesus Christ had power to raise his own dead body to life, and actually did raise it on the third day after his death, according to his prediction. Mt 12:40; Ro 1:4.>
<2:20 Forty and six years; it was so many years since Herod the Great had commenced repairing, or, more properly, rebuilding the temple, sixteen years before the Saviour's birth. During this period additions, more or less, had been from time to time made to it. To this temple they wrongly applied the Saviour's words. The same misapplication of his words they made when they accused him before Pilate. Mt 26:61; Mr 14:58.>
<2:22 The scripture; those passages which foretold his death and resurrection. Ps 16:10,11; Ac 2:22-36.>
<2:24 Did not commit himself unto them; did not trust himself in their power.>
<2:25 He knew what was in man; he knew the hearts of men, and how, under circumstances, they would act. Of course he knew in all respects how to treat them. Jer 17:10; Re 2:23; Joh 1:1. The perfect knowledge of Jesus Christ gives him the fullest acquaintance with human character, shows him how in all respects to treat men while on earth, and fits him righteously and wisely to award to all the retributions of eternity.>
<3:1 A ruler; a member of the Sanhedrin or great Jewish council.>
<3:3 Except a man be born again; our Lord saw that Nicodemus had no true apprehension of the spiritual nature of the kingdom which he had come to establish, nor of the spiritual character required for admission to it. He therefore met his difficulty at the outset by teaching him that all men, be they Jews or Gentiles, must be made new in the inner man by a spiritual birth, before they can enter into his kingdom and enjoy its privileges. To be born again is to be made new men inwardly by a great change from supreme love of the Creator. See the kingdom of God; understand or enjoy its blessings. Mt 3:2.>
<3:4 How can a man; this question referred to natural birth, of earthly parents; the assertion of Christ referred to a spiritual change by the Holy Ghost.>
<3:5 Born of water and of the Spirit; purified by the Holy Spirit; of which baptism by water is a divinely appointed symbol. Enter into; become a member inwardly, and not merely in an outward way.>
<3:6 Born of the flesh is flesh--born of the Spirit is spirit; by the natural birth, fleshly children come from fleshly parents; by the spiritual birth, spiritual children come from the Holy Spirit. Flesh and spirit are here opposed to each other. The first denotes what is earthly and impure; the second, what is heavenly and holy. Compare Ro 8:1-9.>
<3:7 As all men are naturally destitute of the love of God, no one should think it strange that he must experience that change which Christ called being born again.>
<3:8 Listeth; pleaseth--where its author pleaseth to have it. So is every one; the Spirit's operation, like that of the wind, is directed by God, unseen, and known by its effects.>
<3:9 How; Nicodemus here asked a question which Jesus did not answer. He had revealed the fact, its nature, necessity, and author. The manner it was not necessary, and would not be useful, for Nicodemus to know. The fact might be believed, and all its benefits be secured, without knowing how it was accomplished. It is not necessary, in order to believe a fact and receive the benefit of it, that a man should understand the manner in which it is accomplished; and he should not let his ignorance of what God has not revealed hinder him from receiving and treating as true what he has revealed.>
<3:10 Master of Israel; teacher, which he was by virtue of his office as a ruler. These things; the things relating to the new birth, about which he had been speaking, and which were revealed in the Old Testament, which the Jewish rulers professed to teach. Ps 51:10; Eze 11:19; 36:26.>
<3:11 Do know--have seen; Christ's knowledge of truth was direct. He always spoke what he had seen with his Father, chap Joh 5:20; 8:38. What his followers, therefore, had heard and learned of him, they could declare with certain knowledge of its truth. As Jesus knew the truth of what he taught, all are bound to believe it, and to let it have its due influence over their hearts and lives.>
<3:12 Earthly things; things which take place on earth, such as being born of the Spirit, the evidences of which are obvious to the senses. Heavenly things; things less plain, relating to God, Christ, heaven, and eternity, the evidences of which are not addressed to sense, but to faith.>
<3:13 Hath ascended up to heaven; learned heavenly things by actual presence there, and come down from that world to reveal them. Son of man; Jesus Christ. Which is in heaven; whose proper dwelling-place is in heaven. He left heaven for a season only, to return thither again.>
<3:14 Moses lifted up the serpent; Nu 21:8,9 Be lifted up; on the cross a propitiation for the sins of men. 1Jo 2:2.>
<3:16 Gave his only begotten Son; the highest expression of his infinite compassion. Chap Joh 1:14.>
<3:17 Might be saved; on their compliance with the terms of salvation.>
<3:18 Not condemned; Ro 8:1.>
<3:19 The condemnation; cause of condemnation. Light is come; divine truth is revealed. Darkness; error and sin.>
<3:20 Reproved; shown to be evil, and as such condemned. The reason why men do not believe what Christ has taught is, that they love error, they do evil, and his truth on this account condemns them.>
<3:21 Doeth truth; acteth according to truth. Wrought in God; by the aid of his Spirit, and according to his will.>
<3:22 Baptized; not personally, but through is disciples, Joh 4:2.>
<3:25 About purifying; the question seems to have had reference to the relative worth of John's baptism as compared with that of Jesus.>
<3:26 All men come to him; many more came to Christ than to John. When sinners in great numbers come to the Saviour some men, if it lessen the number who follow them, are greatly grieved. But good men, with right views, rejoice in every accession to the number of Christ's followers. They are delighted to see him increase, though it cause them to decrease.>
<3:27 Except it be given him; every office in God's kingdom, and all success in doing good, is from God. He gives to all their place and work as he sees best. You should not be dissatisfied that a greater than I has come, for this is what I foretold.>
<3:29 Hath the bride is the bridegroom; in these words John teaches that Christ's relation to "the kingdom of heaven" is that of the bridegroom to the bride. The church is his own, and ought to render to him supreme honor. John, on the other hand, is only the bridegroom's friend. He was sent to foretell his coming, and now rejoices to be lost sight of in his greater glory.>
<3:30 He; Christ. Must increase; in influence and honor. It is a high spiritual attainment to be willing that others should excel us in usefulness and honor.>
<3:31 He that cometh; Christ. Is above all; in character and work, and ought to be honored above all. Is of the earth; as are John and all merely human teachers. Is earthly; inferior in character and teaching, and ought to hold an inferior place.>
<3:32 Seen and heard; in heaven with his Father. No man; few compared with the whole, and none uninfluenced by the Holy Spirit.>
<3:33 Hath set to his seal that God is true; by believing in Christ he acknowledged that what God hath said concerning him is true.>
<3:34 By measure; John and the apostles received the Holy Spirit only in a certain measure, but Christ without measure.>
<3:35 Hath given all things; pertaining to the salvation of men. Into his hand; as Mediator, that he might give eternal life to all who should believe in him. Compare chap Joh 17:2. As all things pertaining to the souls of men are in the hands of Christ and at his disposal, the eternal life of those who believe in him, and the eternal death of those who continue to reject him, are certain.>
<4:1 How the Pharisees had heard; he was aware that the knowledge of his growing popularity excited their envy and ill-will, which he wished for the present to avoid.>
<4:3 When good men are opposed, and their usefulness obstructed in one place, it is often wise for them to go to another; and the rejection of the gospel by some, proves the occasion of its being embraced by others; thus God, angels, and men are led to rejoice together. Lu 15:7-10.>
<4:4 Samaria; lying between Judea and Galilee.>
<4:5 Sychar; in the Old Testament it is called Shechem. Ge 33:18. It is now called Naplous, and lies a few miles south-east of the city of Samaria, between the mountains Ebal and Gerizim. Jacob gave to his son Joseph; see note to Ge 48:22.>
<4:8 Meat; according to the usage of the word in our version, meat is used for all kinds of food.>
<4:9 No dealings with the Samaritans; no friendly dealings. Mt 10:5.>
<4:10 The gift of God; the Saviour, and the blessings which he is ready to bestow. Living water; under the figure of living water, that is, the flowing water of a fountain in contrast with the stagnant water of a pool or cistern, he means the Holy Spirit, who purifies, refreshes, and invigorates the soul. Compare chap. Joh 7:37-39.>
<4:11 Whence; she supposed him to speak of literal living or running water, which was peculiarly fresh and good.>
<4:13 Earthly blessings, however numerous and great, cannot satisfy the desires of men. But the blessings of the Holy Spirit, which Christ gives to those who ask him, furnish and secure to them satisfying and eternal joys.>
<4:14 The water that I shall give him; the Holy Spirit shall dwell in him, and satisfy his soul. Shall never thirst; he does not mean that one draught shall satisfy him, but that he shall always have in his soul a well of living water, from which he shall continually drink and be satisfied.>
<4:18 Not thy husband; she was living with a man who was not her husband.>
<4:19 I perceive; this she did from his manner and his knowledge of her history.>
<4:20 This mountain; mount Gerizim, which lay at a little distance from Sychar. As she perceived him to be a prophet, she appealed to him to decide a disputed question between the Samaritans and the Jews about the proper place of public worship.>
<4:21 The hour cometh; the time is near. Neither in this mountain; not in this or that place merely, but in all places, may you worship him who is a Spirit, "in spirit and in truth.">
<4:22 We know what we worship; the Jews had both a divinely appointed system of worship, and a clear revelation from God of his nature and the service required by him. The Samaritans received only the five books of Moses, and their services on mount Gerizim were without the divine warrant, and mingled with superstitious observances. Salvation is of the Jews; from them the Messiah was to come.>
<4:23 Worship the Father; in heart, with sincere love and devotion, in every place where they may be. Men are prone to think too much of the place and mode of religious worship. God regards the state of the heart; and spiritual worship, in any place, is accepted of him.>
<4:26 He; the Messiah. This was the first time, so far as we are informed, that Jesus explicitly declared himself to be the Christ. It was, moreover, not in the presence of the Pharisees, who would have taken advantage of the assertion to misrepresent and persecute him; but among the Samaritans, who had no intercourse in religious matters with the Jews, and would make no bad use of the declaration.>
<4:34 As food to the hungry, and water to the thirsty, so is the doing of the will of God to those who love him; and whether engaged in sowing the spiritual seed of divine truth, or reaping the harvest, their employment is a source of the most elevated and refreshing joy.>
<4:35 Say not ye; that is, when ye have committed your seed to the ground. Four months; this was the usual period between seed-time and harvest. Look on the fields; the Samaritans, called by the woman, coming to hear the gospel; and men in every direction perishing for lack of knowledge.>
<4:36 He that soweth and he that reapeth; Jesus and others had sowed spiritual seed. The disciples might gather the fruit by being instrumental in the conversion of men; and thus Christ and his disciples, like sowers and reapers, rejoice together.>
<4:38 I sent you; spoken in anticipation of the mission of his apostles, when he should have finished his work on earth. Other men labored; spoken of all the preparation made by Christ and holy men before him for the work of the apostles. Ye are entered into their labors; they have prepared the way for your reaping the fruit.>
<4:41 Many more believed; this was the beginning of the harvest of which Jesus had spoken.>
<4:44 His own country; this is the same word which is used Mt 13:54;, and applied to Nazareth, where Jesus was brought up. As they did not receive him, he visited and wrought miracles in other parts of Galilee.>
<4:48 Except ye see signs; unless by miracles he showed that he was the Messiah, they would not believe on him.>
<4:50 The faith of parents and masters, Lu 7:1-10, and their prayers to Jesus Christ, are often the means of unspeakable blessings to their children and servants; and however distant they may be from their friends or from Christ, his power can reach them, and his grace supply their wants.>
<4:52 Seventh hour; one o'clock in the afternoon.>
<4:54 The second miracle; the second that he had wrought at Cana, the first having been the turning of water into wine. Chap Joh 2:6-11. In the interval he had wrought many miracles at Jerusalem, Joh 4:34.>
<5:1 The present chapter contains the first of those wonderful discourses of our Lord recorded by John, in which he sets forth his divine nature and office in his twofold relation to God and man. For the clearer understanding of his words, the reader should notice the following things: First, God is his Father and he is the Son of God in such a high and incommunicable sense, that he is equal with the Father in nature, verse Joh 5:18; enjoys his perfect love and knows all his counsels, verse Joh 5:20; performs all the works that his Father performs, verse Joh 5:19-21; has life in himself as the Father has it, and gives it to whom he will, verses Joh 5:21,26; claims equal honor with the Father, verse Joh 5:23; raises the dead and judges them, verses Joh 5:21,22,24-29. Secondly, as the Son of God on earth, he always acts in subordination to the will of the Father. He has not come of himself, but the Father has sent him, verse Joh 5:23, etc.; the Father has appointed to him the works that he shall do, verses Joh 5:20,22,26, etc.; he can do nothing of himself, but must always act in accordance with the Father's will, verses Joh 5:19,30; the Father who sent him bears witness to him by the works that he has given him to do, verse Joh 5:36; and by the scriptures of the Old Testament, verses Joh 5:39,45-47. Thirdly, it is as the Son of man--the Word made flesh--that he not only redeems, but judges men, verse Joh 5:27. Equality with the Father in nature, subordination to the Father in office, union with human nature in the work of redeeming and judging men, and in all these perfect union with the Father in counsel and will: these are the great doctrines that run through the present and similar following discourses.>
<5:2 Marked; this word, as shown by the italics, is not in the original. It probably should have been, as in the margin, gate. Ne 3:1,32; 12:39. Bethesda; "house of mercy." Many at that pool had been mercifully healed of their diseases.>
<5:8 Thy bed; which was a simple mat.>
<5:10 To carry thy bed; which they reckoned among the servile labor forbidden by the law. See Jer 17:21,22; Ne 13:15-20; where, however, the burdens borne were in the way of traffic and ordinary labor.>
<5:14 A worse thing; a worse evil than that from which Jesus had delivered him. All diseases are consequences of sin. Both the sufferings resulting from them, and the experience of relief, should therefore lead us to abhor and forsake it, that we may thus, through faith in the Redeemer, escape its endless consequences.>
<5:17 Worketh hitherto; worketh without intermission in upholding and quickening creation, ever since the day when he finished it. I work; he claimed to be the Son of God in such a sense that he had the power and right of working as God works. This they thought was blasphemy; and had he been only a man, it would have been. But as he was God as well as man, chap Joh 1:1, it was speaking and acting according to truth. The question was not whether Jesus possessed power to do those things, but it was whether he exercised his power agreeable to the will of the Father, or in opposition to it; and he answered them accordingly.>
<5:19 Of himself; in opposition to, or without the concurrence of the Father, which was the crime with which they charged him. He denied the charge, and asserted, that instead of opposition, as they contended, there was between him and the Father the most perfect agreement in plan, will, and operation. These also doeth the Son; there is oneness of operation.>
<5:20 Showeth him all things; makes him partaker of all his counsels, as well as acts with him in all his works. Greater works than these; the works referred to in the following verses.>
<5:21 The Son quickeneth; giveth life, natural and spiritual, to whom he will--thus doing the work of God, and showing that he is God. This is one of the greater things referred to. The other was the judging of all men at the last day, and awarding to them the retributions of eternity.>
<5:22 The Father judgeth no man; in the scheme of redemption, the Son was to be the final judge of men, the author of their resurrection from the dead, and of their eternal life in heaven. This was, "that all men should honor the Son, even as they honor the Father." He that thus honoreth not the Son, honoreth not the Father. The Pharisees, therefore, while they were pleading ostensibly for the honor of God, were in reality treating him as they treated Jesus Christ; and so it is with all men now.>
<5:23 As Jesus Christ the Son of God was appointed of the Father to be the dispenser of life to men--not only to heal the sick, but to raise the dead, and judge the world, "that all men should honor the Son, even as they honor the Father," those who do not thus honor him, but continue to neglect the object of his coming, will lose the benefits of his redemption.>
<5:24 Heareth my word; receiveth my instructions, and treateth them as true. Hath everlasting life; the beginning of that spiritual life which shall continue and increase for ever. From death unto life; from a state of sin and guilt to a state of holiness and bliss.>
<5:25 The dead shall hear--shall live; the dead here include both the spiritually and the naturally dead. Christ gave life to the souls of men, and also to their bodies, when he chose to do so. Jairus' daughter, the widow's son, and Lazarus were all by Jesus raised to life, and many who were dead in sin were quickened and made alive to God.>
<5:26 Given to the Son to have life in himself; here the Saviour brings to view both his oneness with the Father in nature, and his subordination to him in office. To have life in himself, with the power of giving life at will, is to be proper God. But the office of quickening whom he will he has received, as Mediator, from the Father, and exercises it in accordance with the Father's appointment.>
<5:27 Because he is the Son of man; it is the appointment of the Father that he who redeems and judges men should himself be the Son of man; that is, the Word made flesh. In this character God has appointed him to be Mediator, to open the way for, and give eternal life to all who should believe in him, and in pursuance of his work, to perform miracles, die, rise again, raise the dead, judge the world, and fix the condition of all for eternity.>
<5:28 The hour is coming--all that are in the graves shall hear his voice; he passes to the greatest and most astonishing manifestation which he is to make of the truth that he has life in himself; namely, the final resurrection by his word of the just and the unjust, and the decision of their destiny for eternity.>
<5:30 Of mine own self; in opposition to, and without the concurrence of the Father. I hear; from my Father. The idea is, that he dwells in the Father's bosom, and hears and knows all his counsels. Not mine own will; not to exalt myself, or build up a separate interest, but to honor the Father by doing his will.>
<5:31 Of myself; concerning myself, without any accompanying testimony from God. Not true; not to be received as valid.>
<5:32 Another; God, who testified of him by John the Baptist, by the descent upon him of the Holy Ghost, by miracles, and by a voice from heaven.>
<5:34 Not testimony from man; not from man only; yet Christ appealed to the testimony of John, as what ought to convince them.>
<5:36 The works which the Father hath given me; the whole course of his teachings and miracles.>
<5:37 The Father himself--hath borne witness of me; in addition to the testimony furnished by my works. He seems to refer to the testimony of the Father through the Scriptures, which he immediately afterwards urges. Some think that he also includes the voice from heaven upon his baptism. Neither heard his voice--seen his shape; the allusion is to the way in which holy men of old received revelations from God by voices and visions. The import of the Saviour's words is, The state of your hearts makes you unable to receive any testimony of the Father concerning me, outward or inward.>
<5:38 Ye have not his word; they did not receive the testimony of God, and they showed this by rejecting that of his Son.>
<5:39 The scriptures; the Old Testament, by following which they hoped for heaven; and yet those scriptures showed that he was the Messiah, and that they must believe in him, or perish.>
<5:40 Ye will not come to me; notwithstanding this evidence that he was the Messiah, they would not receive him.>
<5:41 I receive not honor; it was not his object to obtain human applause, but to honor God and save men.>
<5:42 Have not the love of God; this was the reason why they would not embrace him as the Messiah. Compare chap Joh 8:42. The reason why men do not receive the words of Christ and treat them as true, is, that they do not love God. As God manifest in the flesh, they do not love him, and choose not to have him to reign over them.>
<5:43 In my Father's name; by his appointment, and with conclusive evidence of being sent of him. In his own name; without being sent of God; actuated by a worldly spirit, and promising them temporal dominion and honor. Such were the false Christs who afterwards appeared, and whom the Jews followed to their destruction.>
<5:44 Receive honor; seek supremely human applause. Men cannot seek supremely human applause, and at the same time seek that honor which comes from God by believing on his Son. They should therefore, without hesitation and without delay, renounce the one, that they may secure the other.>
<5:45 Do not think that I; he did not come to condemn them, nor was there any occasion that he should do so. Moses; he had foretold, De 18:15-19, the coming of the Messiah, and the condemnation of those who should reject him. His writings therefore, which they professed to follow, condemned them.>
<6:1 Over the sea of Galilee; to its northern shore. See note to Lu 9:10.>
<6:2 2-14. Five thousand fed. Mt 14:13-21; Mr 6:32-44; Lu 9:10.>
<6:6 To prove him; try him whether he believed in the power of Christ to supply them. God in his providence does many things to prove his people--to lead them to show what is in their hearts, and thus prepare them to renounce dependence upon themselves, and put their trust in him.>
<6:12 However easy it is for God to supply all needed good, and however much he may give, he requires that no part of it be squandered, or suffered through negligence to be lost; but that all, by prudent care, should be saved for the benefit of those who need it.>
<6:14 That prophet; the Messiah. De 18:18.>
<6:15 Make him a king; a temporal sovereign, such as they expected their Messiah would be.>
<6:16 16-21. Christ walks on the sea. Mt 14:22-33; Mr 6:45-53.>
<6:17 Over the sea; to the west side.>
<6:19 Five and twenty or thirty furlongs; between three and four miles.>
<6:22 On the other side; on the north side. None other boat; there was no other when the disciples left.>
<6:23 There came other boats; from the west side of the sea, after the disciples left.>
<6:24 Took shipping; boats that came from Tiberias, which was on the west side of the sea.>
<6:26 Not because ye saw the miracles; not because ye were attracted to me by the revelation made in the miracles of my divine power and glory. Because ye did eat of the loaves; you seek from me only earthly good. Men often pay an external regard to Christ and his ordinances, not for the purpose of honoring him and obtaining spiritual blessings, but for the purpose of promoting their worldly interests, and accomplishing their selfish ends.>
<6:27 Labor not for; better, as the margin, work not for: bestow not your chief labor and anxiety upon. Meat which perisheth; temporal blessings. Meat which endureth; spiritual and eternal good. Sealed; authenticated as the true Messiah, the giver of eternal life.>
<6:28 Work the works of God; they have reference to the exhortation just given by the Saviour, "Labor--for that meat which endureth." The works of God; such as he required, and such as would secure the enduring good of which Christ spoke.>
<6:29 This is the work of God--believe on him whom he hath sent; Christ is the true bread from heaven. To believe on him is to receive this bread, and thus to do what God requires. The great work which God requires of a sinner, and that which is essential to salvation, is to believe on the Lord Jesus Christ.>
<6:30 What sign; sign from heaven in addition to what he had already shown in proof of the justness of his claims. Compare Mt 12:38; Mt 16:1. Like all cavillers, they demand other proof, and different from that which they have received.>
<6:31 Our fathers did eat manna; still thinking that Jesus was speaking of the bread that should nourish the body, they intimate that the miracle of the manna in the desert, Ex 16:13-18, was greater than that which he has wrought, and that they may reasonably ask of him a higher sign.>
<6:32 Moses gave you not that bread from heaven; it did not come from the heaven where God resides, but from the natural heaven, and was simply natural bread--"the meat which perisheth." The true bread from heaven; the true spiritual bread that comes from God's own presence and feeds the soul.>
<6:33 He which cometh down from heaven; rather, as the original may be rendered, that which cometh down from heaven; for the Jews, as appears from the following verse, did not yet understand him as speaking of himself.>
<6:34 Evermore give us this bread; not yet understanding its spiritual nature, but supposing it to be some miraculous kind of bread that should give life to the body.>
<6:35 I am the bread of life; the author, nourisher, and supporter of spiritual, eternal life. Having spoken of the bread from heaven, he now represents himself under the similitude of heavenly bread; and the eating of his flesh and drinking his blood, or spiritually believing on him, as essential to spiritual life. Never hunger--never thirst; never desire any higher or more satisfying good.>
<6:36 Believe not; of course they were still unsatisfied, and not partakers of the good of which he spoke.>
<6:37 Giveth me; Isa 53:10-12; Joh 17:2; Eph 1:3-12. Come to me; this means the same as he before meant by eating his flesh, or believing on him, and as he afterwards meant by drinking his blood.>
<6:40 I will raise him up; to everlasting life. He would thus do the will and accomplish the object of the Father.>
<6:44 Can come to me; trust in me as his Saviour. Draw him; by teaching him his need of a Saviour, and leading him to trust in him for salvation. The drawing of the Father mentioned in the New Testament, and which is needful to lead sinners to Christ, is the same as the teaching of the Father mentioned in the Old Testament. Isa 54:13; Mic 4:2. The reason why this drawing or teaching is needful is, men are so wicked that they never will come to Christ without it.>
<6:45 In the prophets; Isa 54:13. His doctrine about being drawn or taught of God was not new, but was the same which was taught in the Scriptures, and which they ought to have understood and believed.>
<6:46 Not that any man hath seen the Father; he guards them against the error of supposing that the Father teaches men by his personal visible presence. He teaches by his word, his Spirit, and his providence; leading men rightly to apprehend and cordially to obey his truth. He hath seen; the Saviour sets his immediate and full vision of the Father in contrast with the indirect knowledge which mere men have of him. His meaning is, that because he has seen the Father, he can teach men of the Father.>
<6:50 Not die; the eating of that bread will give eternal life to his soul, and in the end, a glorious immortality to his body also. Compare ver Joh 6:39,40.>
<6:51 My flesh, which I will give for the life of the world; an allusion, which could not be understood at the time by his hearers, to the gift of his flesh on the cross for the salvation of the world.>
<6:52 His flesh to eat; they meant literally. And the true answer to that question was, he would not in any way give them literally his flesh to eat. That was not his meaning. But by eating his flesh, he meant, believing on him as a Saviour, and thus receiving spiritual life and nourishment from him. Men often make objections to what they call the doctrines of Christ, when in fact their objections are not against his doctrines, rightly understood, but only against their own misconceptions of them; and the putting of a literal meaning upon such of his words as were designed to be figurative, and convey only a spiritual meaning, is absurd. Joh 7:34-36.>
<6:53 Eat the flesh--drink his blood; not literally, but spiritually, as the food and drink of the soul; thus, by a living union with him through faith, receiving from him forgiveness, sanctification, and eternal life. The Saviour has in mind the gift which he is about to make on the cross, of his flesh and blood for the life of the world. The view which he here gives of eating his flesh and drinking his blood, is the same that underlies the ordinance of the Lord's supper, afterwards instituted by him.>
<6:55 Meat indeed--drink indeed; I am the giver and sustainer of endless spiritual life.>
<6:56 Dwelleth in me; has a vital, saving union with me by faith, Joh 15:5; 1Co 6:17; resembling in some respects the union between me and my Father. Joh 17:21.>
<6:60 His disciples; disciples is here used in a general sense, for those who attended on his teaching.>
<6:62 What and if ye shall see; are you offended at the honor which I now claim as the bread of life? What will you say, then, when you see me receiving greater honor than this, by ascending to heaven, whence I came?>
<6:63 It is the Spirit that quickeneth; it was the spiritual, not the literal meaning of his words, which would profit them. The literal eating of his flesh would not benefit them. It was only the spiritual meaning, understood, believed, and obeyed, that would be the means of spiritual life to their souls. They are spirit--they are life; they contain spiritual food and life; for by receiving them, you receive me. The words of Christ have a spiritual meaning, and it is the right apprehension and cordial reception of this meaning which is life-giving to the soul. Men naturally do not receive this meaning, because such is their wickedness that they have not spiritual discernment. Hence the propriety of praying, "Open thou mine eyes, that I may behold wondrous things out of thy law; quicken thou me according to thy word." Ps 119:18,25; 1Co 2:14.>
<6:64 That believe not; this shows that by eating his flesh he meant believing on him.>
<6:65 Given unto him of my Father; the meaning of this is the same as his being drawn or taught of God, ver Joh 6:45.>
<6:68 Thou hast the words; teachest the way of eternal life.>
<6:70 A devil; under the control of Satan, and like him in character.>
<7:1 Jewry; Judea.>
<7:2 Feast of tabernacles; the Jews had three great annual feasts: the feast of the Passover, the feast of the Pentecost, and the feast of Tabernacles. De 16:1-15. This last was held from the fifteenth to the twenty-second day of their month Tizri, which included a part of September and October. This was a season of special thanksgiving for the ingathering of the harvest.>
<7:3 His brethren; or relatives. Depart hence; from this obscure and secluded region. Into Judea; where was the metropolis of the nation, and where men that sought to be known were accustomed to resort. That thy disciples also may see; thy disciples there.>
<7:4 That doeth any thing in secret; works in a secluded place, removed from the observation of men. And he himself seeketh; that is, while he himself seeketh. To be known openly; they upbraid him with acting inconsistently. He seeks to be known openly as the Son of God, and yet he keeps himself removed from public view. These things; the miracles which he wrought, in proof of his being the Messiah.>
<7:5 The teaching and example of the holiest man on earth will not, without the grace of God, lead even his relatives to believe in Christ and live.>
<7:6 My time; his time to go up to the feast, and manifest himself as the Messiah. Your time is always ready; you are in no danger of persecution, and can safely go at any time you may choose.>
<7:7 The world cannot; they have no occasion to do it.>
<7:10 In secret; privately, or in a retired manner.>
<7:12 Murmuring; private inquiry and contention about Jesus.>
<7:15 Letters; learning. Whence his learning? He has never been instructed by doctors of the law.>
<7:16 Not mine; I come from God, and I teach what he has committed to me to be taught.>
<7:17 Will do his will; here, as often elsewhere, the Saviour teaches that love and obedience towards the Father is the only true preparation for understanding and judging aright concerning the words of the Son whom he has sent. Of myself; of myself alone, without the direction of the Father. All men who have the Bible may know whether it is from God, and whether its doctrines are true. If they do not know, it is because they do not love and obey God, and it is wholly their own fault.>
<7:18 Of himself; this does not mean speaking about himself, or by his own power, but speaking independently of any commission or authority from God. A false teacher, not sent from God, would seek his own private ends. Christ was sent by the Father, sought his glory, and his testimony was true.>
<7:19 Why go ye about to kill me? under the pretence that I have broken the law of Moses respecting the Sabbath, chap Joh 5:8, while you yourselves are continually violating that law.>
<7:20 Has a devil; are influenced by an evil spirit.>
<7:21 Done one work; cured a man on the Sabbath. Chap Joh 5:8. Ye all marvel; because he had done such a work, as if it were a violation of holy time.>
<7:22 Moses therefore gave unto you circumcision; he enjoined it, though it did not originate with him. It was appointed of God, and was practised from the days of Abraham. Ye on the Sabbath-day circumcise; this required more labor on the Sabbath than he performed in healing a man; and as they justified the one, they ought not to condemn the other.>
<7:24 According to the appearance; according to the outward letter of an act. He still refers to the charge of breaking the Sabbath. Judged by the outward letter, the circumcising or healing of a man on the Sabbath might seem to be Sabbath-breaking. But judged according to truth, neither act was such.>
<7:27 We know this man; he parentage and place of birth. No man knoweth whence he is; the Jews knew that the Messiah should be a descendant of David, and born in Bethlehem. Mt 2:4-6. But they had an idea that, before his manifestation as king of Israel, he would hide himself, and then suddenly appear from an unknown quarter. Compare Mal 3:1; Mt 24:26.>
<7:28 Ye both know me--know whence I am; some understand the Saviour as upbraiding the Jews for their willful rejection of him, as much as to say, You know from my works who I am, and who has sent me. Compare chap Joh 3:2; 12:42. Others suppose that he concedes to them a merely earthly knowledge of himself, as if he had said, Ye do indeed know me as a man, and whence I am; and yet I have a higher origin, being sent from the Father. Whom ye know not; though they had the Scriptures, they had no true knowledge of God, and this was the reason why they did not ]know his Son. Compare chap Joh 8:19; 16:3; 17:3. An affirmation of the Bible may be true in one sense, and not true in another. In order, therefore, rightly to treat it, we must understand the sense in which a declaration is made, and in that sense, on the authority of God, must receive it.>
<7:30 His hour was not yet come; his time to be taken and slain.>
<7:34 Ye shall seek me; in the days of your distress you shall in vain seek the Messiah, who has been among you and been rejected by you. Where I am; that is, in God's presence; but to them the saying was unintelligible.>
<7:37 The last day; the closing day of the feast. On this day, water from the pool of Siloam was carried with great solemnity and poured out on the altar. Thirst; not literally, but spiritually--thirst in soul for satisfying enjoyment.>
<7:38 Rivers of living water; a living fountain shall be opened within him, whence shall flow streams refreshing his own soul and the souls of others. Compare chap Joh 4:14. No man is at liberty to interpret the words of Christ in a literal sense, when such interpretation is shown by the sense of men to be false.>
<7:39 Not yet given; not so fully and abundantly given as Christ foretold that he would be, and as he afterwards was.>
<7:40 The Prophet; the prophet who they thought would precede the Messiah. Chap Joh 1:21; Mt 16:14.>
<7:42 Of the seed of David--the town of Bethlehem; all this was fulfilled in Jesus of Nazareth; but from the neglect of careful inquiry, they remained ignorant of the fact.>
<7:49 This people; the common people, to whom the Pharisees imputed criminal ignorance of the Scriptures. Tyrannical teachers and rulers fear the elevation and influence of the common people. They wish to keep them in ignorance, and are often opposed even to their reading the Bible, and judging of its meaning. They would themselves do the reading and judging, as well as the governing. If others undertake to exercise their inalienable rights, they are filled with wrath, and ready to pronounce them accursed. But such curses will rebound on their authors. Ps 109:17.>
<7:50 One of them; one of the great council, or rulers of the Jewish nation. Chap Joh 3:2.>
<7:52 Of Galilee; this was an expression of contempt, as Galilee was a despised country. They knew that Nicodemus was not from Galilee, but they meant to reproach him for favoring a Galilean.>
<8:1 Mount of Olives; Mt 22:1.>
<8:5 Moses in the law; Le 20:10.>
<8:6 Tempting him; should he decide that she ought to be put to death, they would accuse him of assuming judicial authority; and should he decided the other way, they would accuse him of being opposed to Moses. On the ground; the words which follow in italics, not being in the original, might have been omitted.>
<8:7 Let him first; De 17:7. Men are sometimes very forward to accuse others, and seek to have them punished, when they are themselves guilty of equal, and perhaps greater crimes. Should conscience be awakened to do its office, and none but the innocent be suffered to accuse or condemn, the guilty would often go unpunished.>
<8:10 Saw none; none of her accusers. Condemned thee; passed upon thee a judicial sentence of condemnation.>
<8:11 Neither do I; I do not exercise this prerogative of the civil magistrate.>
<8:12 Light of life; that knowledge of God which is life to the soul. Chap Joh 1.4-9.>
<8:13 Of thyself; concerning thyself. You testify in your own case.>
<8:14 Though I bear record of myself--my record is true; he had before said, "If I bear witness of myself, my witness is not true," chap Joh 5:31; but there he was speaking of the validity of testimony according to the human rule of trying it. Here, on the contrary, he speaks of the quality of his own testimony from the high consciousness of his divine nature and mission. I know whence I came--whither I go; I know that I come from the Father and return again to him; therefore I know that my testimony of him is true. Ye cannot tell; more literally, Ye know not, as in the first part of the verse. As the Jews did not understand his divine nature and mission, they were not qualified to judge of his testimony.>
<8:15 After the flesh; according to outward appearances, under the power of prejudice, with selfish motives and worldly ends. No man; chap Joh 3:17; 12:47.>
<8:16 I am not alone; in my judgment. The Father is always united with me in it. Compare chap Joh 5:30.>
<8:17 The testimony of two men is true; De 17:6; 19:15. Compare Mt 18:16. He now returns to the human rule of trying testimony.>
<8:18 I am one that beareth witness-the Father that sent me beareth witness; his case is like that of an ambassador fully accredited by him who sends him. The testimony of such an ambassador is valid according to the human rule of judgment.>
<8:19 If ye had known me--known my Father also; the union between him and his Father was such, that to know the one was also to know the other. Compare chap Joh 14:9,10. We may have the Bible and all the means of grace, and yet be ignorant of the character of God, of Jesus Christ, and of the way of life through him. This is not because they are not plainly revealed, but because opposition to them blinds the mind, hardens the heart, and prevents the right apprehension of divine truth.>
<8:20 The treasury; the apartment in which was kept the money for the support of the temple service. His hour; chap Joh 7:30.>
<8:21 My way; chap Joh 7:33.>
<8:23 From beneath; earthly and sensual. From above; heavenly and divine.>
<8:24 I am he; the Messiah.>
<8:26 Many things; he might say much more, but he confined himself to those things which would be useful, and which he was commissioned of the Father to declare.>
<8:28 Lifted up the Son of man; upon the cross. Another of the obscure hints which the Saviour was in the habit of giving concerning the manner of his approaching death and its mighty results. The Jews raised him upon the cross to destroy him; but God made this the way of raising him to universal dominion. Compare chap Joh 12:32. Nothing of myself; nothing in opposition to, but all things in accordance with the appointment and will of the Father.>
<8:31 Continue in my word; continue to believe my declarations and obey my commands. Disciples indeed; true disciples. The only sure test of love to Christ is continued belief of his word and obedience to his commands.>
<8:32 Shall make you free; under the idea of freedom, our Lord includes two things: first, deliverance from the dominion of sin; secondly, the condition of sonship as contrasted with that of servants. Both of these he explains in the following verses.>
<8:33 Abraham's seed; and therefore not servants, but freemen. Were never in bondage; they probably refer, not to their national servitudes, which were notorious to all; but to the civil freedom secured to Abraham's seed by the law of Moses. According to this, a Hebrew could not be reduced to the condition of a bond-servant. Le 25:39-46.>
<8:35 Abideth ever; not permanently; he is liable at any time to be dismissed. Abideth ever; he has a permanent residence, and is heir to the estate. The son is here, first, the Son of God, who dwells with the Father, and is "heir of all things," Heb 1:2; secondly, every one whom the Son of God makes him a child of God, and a joint-heir with himself to the heavenly inheritance. Ro 8:17.>
<8:36 If the Son; the Son of God, who abides in his Father's house for ever, and to whom he has committed all power over it. Free indeed; for ye shall not only be delivered from the bondage of sin and its punishment, but made sons of God with and through Christ, and have an everlasting home with him in his Father's house.>
<8:37 Abraham's seed; literally they were, but not spiritually, not in the sense in which the promises to him were made. Ge 12:3; 18:18; Ge 22:18; Ga 3:14,16-18,29. My word hath no place; they would not receive his truth.>
<8:38 Seen with my Father; what is in accordance with the will of God the Father. Your father; verse Joh 8:44.>
<8:39 If ye were Abraham's children; if ye were like him in faith and practice.>
<8:41 We be not born of fornication; perceiving that Jesus uses the word father in a spiritual sense, they reply that they are no spurious race sprung from idolaters, but are the true children of God, since they and their fathers have worshipped him only. With the Hebrews, idolatry was spiritual fornication.>
<8:43 My speech; my manner of discourse and its true meaning. Cannot hear my word; that is, my doctrine. The reason why they could not was the perverse state of their hearts, as is taught in the next verse.>
<8:44 The lusts of your father; they would comply with his wishes in seeking to murder Christ. A murderer from the beginning; the first work of the devil on earth was to seduce our first parents into sin, whereby they and all their posterity were made subject to death. In this he was born a murderer and a liar. The father of it; he uttered the first lie in Eden. In opposing Christ and rejecting his truth, wicked men imitate the devil, and exert an influence which tends to destroy themselves and their fellow-men.>
<8:46 Convinceth me of sin; convicteth me of falsehood, or any thing wrong.>
<8:47 He that is of God; that loves him, and is like him in spirit.>
<8:48 A Samaritan; to a Jew one of the most disgraceful epithets that could be used. The Samaritans were greatly hated and despised as heretics and schismatics.>
<8:50 One that seeketh and judgeth; the Father would honor him, and condemn them.>
<8:51 Never see death; spiritual and eternal death; not perish in his sins.>
<8:54 If I honor myself; if what I say of my office is not sustained by God's accompanying testimony. My Father that honoreth me; he confirms what I say of myself.>
<8:56 Rejoiced to see my day; to hear of and obtain clear views of the coming of Christ.>
<8:58 I am; this denotes eternal self-existence. Ex 3:14; Joh 1:1,3; Col 1:17; Heb 1:6,8; Re 1:8. As Jesus Christ is truly and eternally divine, his kindness, compassion, and grace, in coming into the world, taking upon him human nature, and dying upon the cross, surpass all finite comprehension, and lay upon all to whom he is revealed unspeakable obligations to love and obey him.>
<8:59 Stones to cast at him; because he, being man, claimed to be also God. Ro 9:5.>
<9:1 Stones to cast at him; because he, being man, claimed to be also God. Ro 9:5.>
<9:3 Neither; neither his sin nor theirs was the cause of his blindness. That the works of God; the man was born blind, that Christ, by performing the divine work of healing him, might show himself to be God. God so orders things in his providence as best to display the true character of the Saviour; and men are sometimes left to suffer sore trials, that they may see his goodness, and magnify his power and grace in the removal.>
<9:4 I must work; the works of God--perform divine works. While it is day; while I live on earth. The night; death, which was to the Saviour the close of his earthly ministry.>
<9:5 I am the light of the world; the spiritual light of the world. This he says with reference to the natural light which he is about to restore to the blind man, and which was a striking symbol of the inward illumination that he gives to the souls of them that believe on him.>
<9:7 Pool of Siloam; this pool or reservoir was in the south-east part of Jerusalem, at the mouth of the narrow valley separating mount Zion from the hill Ophel. Its water comes from a subterranean channel, from a fountain higher up on the east side of Ophel. Lu 13:4. Sent; the meaning of the Hebrew word Siloam. Some think it was so called because its water was sent, that is, conducted to its place by the subterranean channel just named.>
<9:16 Keepeth not the Sabbath-day; he did not keep it as the Pharisees directed, but he did keep it according to the letter and spirit of the fourth commandment. A sinner; a transgressor of God's law.>
<9:21 He is of age; old enough to answer for himself.>
<9:22 Put out of the synagogue; excluded from the people and worship of God--excommunicated. Tyrants in church and state try by pains and penalties to prevent men from embracing the truth. They appeal not to reason and conscience, but to force. They labor to preclude inquiry, stifle private judgment, and in many cases prevent those under their control from receiving the instruction afforded by the providence and word of God.>
<9:24 The praise; the praise of healing him.>
<9:33 Do nothing; do no miracle.>
<9:34 Cast him out; of the synagogue.>
<9:35 Jesus Christ especially regards those who suffer for his sake, and will manifest himself to them in such a way as to lead them to worship and adore him. For the temporary enmity of men, they will be abundantly recompensed by the everlasting friendship of God.>
<9:39 For judgment; that those who feel their spiritual blindness, and apply to me for sight, may receive it; and that those who do not, but proudly imagine that they see enough already, and reject my aid, may sink in deeper darkness, and be more blind than ever. The effect upon men of Christ's teaching, is according to their treatment of it. This depends very much on their views of themselves, and of their need of his aid. If they feel that they are spiritually blind, and apply to him for sight, they will receive it; while others who view his help as needless, and think they see and know enough already, will remain in darkness, and their sin and consequent punishment be greater than if Christ had never come.>
<9:41 If ye were blind; had no opportunity or capacity for receiving spiritual light. Ye should have no sin; comparatively; your sin would be much less. We see; you pride yourselves on your knowledge of divine things, yet you reject me and God's revelation concerning me, and therefore remain ignorant, unpardoned, and in aggravated guilt.>
<10:1 The theme of this chapter is the character and office of the good shepherd of God's spiritual fold, of which Christ himself is the great example. The reader should study, in connection with it, Jer 23:1-6; and especially Eze 34.1-31. Entereth not by the door; the Saviour has in view men like the scribes and Pharisees, who usurp dominion over the fold of God, and rule the flock with cruelty and selfishness for their own private ends. The door is not yet directly Christ, for he too enters the fold by the door; but rather, in a more general sense, the Father, and his appointment. To enter the fold by the door, is to come in accordance with God's will, in respect not merely to outward order, but to spirit also. None are true disciples or ministers of Christ who do not believe in him and obey his commands. All others who enter the Christian church or ministry are false and selfish. Instead of aiding, they hinder the progress of his cause.>
<10:2 He that entereth in by the door; every true shepherd. Christ, the chief Shepherd, must not be excluded; for he is an example to all the under shepherds in this respect also, that he has entered the fold by the door. Compare chap Joh 8:42; 12:49.>
<10:3 The porter; the door-keeper. As he comes by God's authority, God's providence prepares the way for him, and God's Spirit sets his seal to his labors. By name; an allusion to the practice of eastern shepherds, who give names to their sheep.>
<10:4 Real Christians have spiritual discernment and relish of the great truths of the gospel. No instruction, however plausible or learned, which denies or omits the doctrine of Christ crucified, as a divine atoning Saviour, satisfies them, commends itself to their conscience, if enlightened, or meets their wants as sinners.>
<10:5 Strangers; false, irreligious teachers.>
<10:6 Understood not; the meaning of what he had been saying. He therefore proceeded to explain it, and in so doing he changed the figure somewhat, representing himself as the door.>
<10:7 I am the door; as the Father is the door to Christ, so he himself is the door to the under-shepherds and to all the sheep. Through him alone can men enter his church or the ministry which he has appointed.>
<10:8 Came before me; without entering the door. See note to verse Joh 10:1. The sheep did not hear them; the truly pious did not receive their false doctrines, or imitate their corrupt examples.>
<10:9 Find pasture; receive spiritual food, satisfying good.>
<10:10 The thief; one who takes the emoluments of the sacred office without performing its spiritual duties, and seeks his own aggrandizement, not the salvation of souls. Life; spiritual, eternal life.>
<10:11 I am the good Shepherd; in respect to the power of admission to God's fold, Christ has declared himself to be the door; in respect to his care over those within the fold, he now, by another change of the figure, calls himself "the good Shepherd"--the Shepherd of shepherds and of the flock, and the source of good to all.>
<10:12 A hireling; one whose great object in preaching is his own interest. The wolf; the enemy of God and his people.>
<10:14 I--know my sheep, and am known of mine; the knowledge of Christ and his people is mutual, and it is a knowledge of deep love and interest. The union between Christ and his people is intimate and unfailing. It resembles, in many respects, that between the Father and the Son. It is the fruit of the Spirit, and all the persons in the Godhead are engaged to perpetuate, increase, and render it eternal.>
<10:16 Not of this fold; those who as yet knew not God, especially from gentile nations. Isa 56:8.>
<10:18 No man taketh it; no man had power to take his life till he should voluntarily surrender himself to crucifixion and death. This commandment have I received; he was commissioned of God to die for the sins of men, and rise again for their justification. He had the power, disposition, and right to do these things.>
<10:20 Is mad; beside himself, through the influence of an evil spirit.>
<10:22 Feast of the dedication; this was a feast instituted by Judas Maccabaeus about one hundred and sixty-five years before Christ, in commemoration of the purification of the temple, and its renewed dedication to the worship of Jehovah, after it had been desecrated by idol-worship and the offering in it of swine's flesh, by Antiochus Epiphanes king of Syria. It began on the 25th day of their month Chisleu, or the 15th of our December, and continued eight days. Josephus, Ant. b. 12, chapter 11; 1 Maccabees 4:52-59; 2 Mac 10:1-8.>
<10:23 Solomon's porch; a portico on the east side of the temple.>
<10:25 The works; miracles. Bear witness; prove me to be the Messiah.>
<10:26 Not of my sheep; not my true followers. The reason why some who hear the gospel reject it and discard its fundamental truths, is, they have not the temper which the gospel inculcates, and do not perform the duties which it requires.>
<10:29 Is greater than all; see note to chap Joh 14:28.>
<10:30 I am my Father; the Jews rightly understood him to call God his Father, and himself the Son of God, in such a sense that he was equal with God. Compare chap. Joh 5:18. Are one; in nature, counsel, will and operation.>
<10:33 Makest thyself God; claimest to be divine, equal with the Father.>
<10:34 Your law; the Old Testament scriptures. Ps 82:6.>
<10:35 Unto whom the word of God came; who were appointed and commissioned to act as his agents in ruling and administering justice in his stead among men. Cannot be broken; cannot be set aside as speaking improperly when it calls magistrates gods on account of their office. The term is always used in such a connection as shows that they were but men.>
<10:36 Sanctified; set apart to the office of the Redeemer of lost men. Sent into the world; to do the work of the Messiah. Because I said, I am the Son of God; the argument is from the less to the greater: If mere men were called gods because the word of God came to them, how much more may he who is one with the Father, and whom the Father has set apart and sent into the world as the Saviour of men, call himself the Son of God. He goes on to show that his works justify him in taking to himself this title.>
<10:37 The works of my Father; divine works--the works of God. Believe me not; admit not my claim to be the Son of God. Jesus Christ, by the performance of divine works, proved himself to be divine, the Messiah, the Son of God, the Saviour of men. His claiming this character, therefore, instead of being blasphemy, as the Jews asserted, was acting in accordance with truth; and lovers of truth who embrace him in this character, know that it belongs to him. They pay him divine honors, not in derogation of, but to the glory of the Father. Php 2:10,11; Heb 1:6; Re 5:12,13.>
<10:38 Believe not me; that is, my declaration concerning myself. The Father is in me, and I in him; that we are one, as I declared to you, verse Joh 10:30.>
<10:39 They sought again to take him; because he still claimed to be the Messiah, the Son of God, truly divine--because he claimed to be what John, under the guidance of the Holy Ghost, at the beginning of this gospel declared him to be, God--in the language of Paul, "over all, God blessed for ever." Ro 9:5.>
<11:1 Bethany; on the side of the mount of Olives, about two miles from Jerusalem. Mt 21:17.>
<11:2 Mary; Mt 26:7; Mr 14:3.>
<11:4 This sickness is not unto death; not to a death from which he should not be quickly raised to life. Might be glorified; by raising Lazarus from the dead. The dispensations of Providence, as well as the instructions of the Bible, are designed to glorify the Son of God, by leading men to honor him as truly divine.>
<11:6 In the same place; Bethabara, chap Joh 1:28; 10:40; on the east side of the Jordan, about thirty miles north-east of Jerusalem.>
<11:9 Twelve hours in the day; an appointed season for me and all men to do the work assigned to us by God. Walk in the day--stumbleth not; if, in its proper season, a man does that to which God calls him, he is safe, because under God's protection. The man who makes it his object to learn the will of God, and to do it, however it may affect him and his condition in this world, is like one who travels in the daytime, when he can see objects distinctly, and treat them according to their character. But a man whose great object is himself, and who seeks supremely earthly things, is like one who travels in the night, without sun, moon, or stars. He is in darkness, and liable every moment to fall and perish.>
<11:10 If a man walk in the night, he stumbleth; if, through fear or selfishness, he neglect the work appointed him by God till the proper time is past, he can no longer perform it in safety.>
<11:16 Thomas--Didymus; these two words, one Hebrew, the other Greek, mean a twin. Die with him; with Jesus. Let us go with him, if it cost us our lives.>
<11:18 Fifteen furlongs; a little less than two miles. See note Lu 24:13.>
<11:23 Thy brother shall rise again; a declaration designedly so worded that it should have a double fulfilment, the present raising of Lazarus being a pledge of the second fulfilment at the general resurrection.>
<11:25 I am the resurrection; the author of the resurrection, and the giver of temporal and eternal life. Though he were dead; more exactly, though he die. The Saviour has in mind the case of those who have, like Lazarus, suffered natural death. Yet shall he live; his soul shall still live in blessed communion with God. To the believer, whose soul is made alive by union with God through Christ, the death of the body will be only a sleep, from which it shall be awakened at the resurrection, to a glorious immortality.>
<11:26 Liveth; yet enjoys natural life. Shall never die; the death of the soul. In this and the preceding verse Jesus designedly overlooks the death of the body, as if it were only a sleep in the grave for a season; his design being to direct the thoughts of Martha to himself, as the giver of a higher life than that which he is about to bestow upon her brother.>
<11:28 The Master; Mt 23:8,10.>
<11:33 Was troubled; greatly moved with sympathy and sorrow.>
<11:35 Jesus Christ tenderly and deeply sympathizes in human sorrow. He delights in soothing hearts that trust in him, and turning their temporary mourning into everlasting joy.>
<11:40 See the glory of God; as displayed in the mighty work he is about to perform. Compare verse Joh 11:4.>
<11:42 Because of the people; his object in thus speaking to the Father in the hearing of the people was, that they might have this additional evidence that he and his Father were one, and that he did every thing in accordance with his Father's will; that thus, in view of this new display of his life-giving power, they might be led to believe in him.>
<11:45 Believed on him; as the Messiah. The exhibitions of Christ in his word and his works, are treated by different persons in very different ways. Some are led to trust in him as their Saviour, and give him their hearts. Others bitterly oppose him, and do what they can to hinder the progress of his cause. Thus, to one his teaching by being received becomes a savor of life unto life, and to another, by being rejected, a savor of death unto death.>
<11:47 What do we? to prevent his increasing influence.>
<11:48 Believe on him; and receive him as the expected king of Israel. This, they profess to fear, will bring upon them the wrath of the Romans. Take away both our place and nation; by the murder of Jesus, they sought to avert this evil. But in this very way they brought it upon themselves. The means which sinners use to save themselves from coming evils, only hasten their approach, and make them more terrible.>
<11:49 Ye know nothing; nothing about the best way to prevent the people from embracing Jesus as the Messiah, and thus to preserve the nation from ruin. His idea was, that, innocent or guilty, it was best to kill him.>
<11:50 It is expedient; he thought it better that Jesus should be put to death, than that the nation should be ruined, as the rulers said it would be if the people should follow Christ.>
<11:51 He prophesied; though the above appears to have been his meaning, yet the Holy Spirit, through his words, expressed the momentous truth, that it was expedient that Jesus Christ, as the Saviour of lost men, should die, the just for the unjust--not for the Jewish nation only, but for all nations, propitiation for the sins of the world, that he might gather into heaven all who should believe and obey him.>
<11:53 From that day; adopting the counsel of Caiaphas, they sought to kill him.>
<11:54 The wilderness; the wilderness of Judea, that bordered on the Dead sea and the lower part of the Jordan valley.>
<11:55 To purify themselves; according to the requirement in Le 22:1-6.>
<12:1 Bethany; chap Joh 11:1.>
<12:2 Served; waited on the company. 2-8. Christ's feet anointed by Mary. Mt 26:6-13; Mr 14:3-9.>
<12:3 Spikenard; an aromatic plant, from which was made a precious ointment.>
<12:6 The bag; the purse which contained their money, and from which they assisted the poor. It is dangerous to be intrusted with public money, and those who are inclined to theft or fraud will often be placed in situations where they will be strongly tempted to commit it. The less our conduct is under the inspection of men, the more mindful we should be of the inspection of God, and the more careful to secure his approbation.>
<12:10 Innocence is, in this world, no certain security against suffering. The greater a person's influence for good, the greater may be his exposure, even from professed friends of God, to persecution and death.>
<12:11 By reason of him; his presence was a standing proof that Jesus, who had raised him from the dead, was the Messiah.>
<12:12 12-19. Christ rides into Jerusalem. Mt 21:1-16; Mr 11:1-11; Lu 19:29-44.>
<12:16 Was glorified; had ascended to heaven.>
<12:17 Bare record; they related what Jesus had done in raising Lazarus from the dead.>
<12:18 Also met him; went out to meet him and accompany him into the city.>
<12:19 Ye prevail nothing; nothing to stop his increasing influence among the people. All attempts to thwart the counsels of the Redeemer will be unavailing; and the efforts which men make to stop the progress of his cause, he will overrule for its advancement and prosperity.>
<12:20 Certain Greeks; that is, as the original word implies, Gentiles using the Greek language. The word should be distinguished from, "Grecians," Ac 6:1; Ac 9:29, who are Jews by birth or descent, using the Greek language.>
<12:23 The hour is come; the request of the Gentiles to see him he regards as a sign that the hour is at hand for the conversion of the gentile nations to himself. But this must be through his approaching suffering and death. The Son of man should be glorified; by his death, resurrection, and ascension to heaven.>
<12:24 It abideth alone; remains a single kernel; its death is essential to its future life and increase: so the death of Christ was essential to the future increase and prosperity of his kingdom. Without that he could not become the Saviour of either Jews or Gentiles.>
<12:25 Loveth his life; Mt 10:39; Lu 9:24. Hateth his life in this world; loveth it less than he does spiritual and eternal life. These words contain a solemn intimation that for Christ's disciples also, as well as for himself, the way to glory and eternal life is through suffering and self-denial.>
<12:26 Let him follow me; in the way of suffering, as well as of obedience. Where I am, there shall also my servant be; he must be with me first in suffering, and then he shall be with me also in glory. 2Ti 2:11,12.>
<12:27 This hour; the hour of suffering which was before him. For this cause; for the purpose of suffering, that men might be saved.>
<12:28 I have; in the attestations which he had borne to Christ the Messiah. And will; in the miracles at his death, resurrection, and ascension to glory.>
<12:30 For your sakes; that you might have this additional evidence that I am the Son of God, and always do that with which he is well pleased.>
<12:31 The judgment of this world; the time when this world, which is opposed to me and under the power of Satan, is to be conquered and subdued to myself. Prince of this world; Satan. Cast out; conquered, so that his power on earth shall thenceforward decline, till he shall be utterly subdued. The efforts of Satan and wicked men to procure the death of Christ were overruled for the promotion of his glory, the salvation of his people, and the ruin of all who continued to oppose him.>
<12:32 Lifted up; on the cross, as a sacrifice for the sins of men. Draw all men; chap Joh 3:14,15; 6:44.>
<12:34 Heard out of the law; learned from the Old Testament. Isa 9:7; Da 2:44; Da 7:14.>
<12:35 The light; the Messiah, the source of all true spiritual knowledge. Walk; while you have the means of knowledge improve them, lest they be taken away. While men have opportunities to obtain spiritual knowledge they should diligently improve them, lest their opportunities cease, and they be left to ignorance, darkness and woe.>
<12:36 Believe in the light; receive and obey my instructions, that you may be wise, and be instrumental in making others wise to salvation.>
<12:38 The saying of Esaias; Isa 53:1, fulfilled in their rejection of Christ.>
<12:39 They could not believe; it is said of Joseph's brethren, that they could not speak peaceably to him, Ge 37:4; and Christ said to the Jews, "How can ye believe, who receive honor one of another, and seek not the honor that cometh from God only?" Joh 5:44. The two things were incompatible. They must cease from the one in order to do the other. So here; as they would continue to love and cherish their sins, they could not, continuing this course, embrace the Messiah. The two things could not coexist. This was the reason why they should have renounced the one, and performed the other. But they would not do it. The prophecy of Isaiah showed that it was certain they would not. Isa 6:9,10.>
<12:40 He hath blinded their eyes; by presenting to them truths which they would reject, their rejection bringing them into greater darkness. In a similar sense it is said, that Jesus Christ came into the world to set members of families at variance, by leading some to trust in him, while others on this account opposed them and sought their death. Mt 10:21,35,36.>
<12:41 His glory; the glory of Christ, called by Isaiah the glory of Jehovah. Isa 6:1.>
<12:42 Many believed on him; were convinced that he was the Messiah. Did not confess him; did not openly declare their belief. The regarding of the praise of men more than the praise of God, while it may consist with a speculative conviction that Jesus is the Christ, is incompatible with that hearty obedience to him which is essential to salvation: men cannot at the same time continue them both, and should, without delay, through the grace of God, renounce the one and perform the other.>
<12:44 Not on me; not on me only, but also on my Father.>
<12:45 He that seeth me, seeth him that sent me; chap Joh 10:30,38.>
<12:47 I came not to judge; it was then his business to act not as judge, but as Saviour.>
<12:48 The word; the gospel, according to their treatment of which men will be judged at the last day.>
<12:49 Of myself; not of my own authority merely, but by the authority and appointment of the Father. Chap Joh 7:16-18.>
<12:50 His commandment; the message which he has commanded me to deliver to men. Is life everlasting; to them who believe and obey it.>
<13:1 Before the feast of the passover; our Lord ate the passover with his disciples on the evening of the Thursday before his crucifixion. Mt 26:17; Mr 14:12; Lu 22:7. From the statement of John, chap Joh 18:28; 19:14, some have inferred that, for reasons unknown to us--possibly from a difference in regard to the computation of time--a portion of the Jews, including the Jewish rulers, were in the habit of celebrating the passover one day later than the other portion. The love of Jesus Christ to his people is unchanging. They may therefore safely put their trust in him. Ro 8:37-39.>
<13:2 Supper being ended; rather, supper having come; for after he had washed his disciples' feet, he reclined again, verse Joh 13:12, and the supper went on, verse Joh 13:26.>
<13:3 Knowing that the Father had given all things into his hands, and that he was come from God, and went to God; as much as to say, he performed this act of condescending love with the full consciousness of his divine dignity and the heavenly glory which awaited him.>
<13:4 His garments; his mantle or outer garment.>
<13:7 Knowest not now; but thou shalt know; the meaning or object of what he did.>
<13:8 If I wash thee not; though the primary object of this washing, as explained by the Saviour himself, was to set his disciples an example of humility and love, he here uses it as a symbol of the spiritual cleansing which they must receive from him. Unless men are purified from the love and practice of sin by the Spirit of Christ, they have no interest in his salvation.>
<13:9 Not my feet only; if this washing be necessary to my having a part with thee, let it extend to my whole person.>
<13:10 He that is washed; that is, bathed, as the original implies, which here uses a different word from the preceding. The bathing represents "the washing of regeneration," which the apostles, with one exception, have already received. Save to wash his feet; which have been soiled in passing from the bath to his own home. This beautifully sets forth the daily cleansing which even regenerated men need from the defilement of daily life. Clean, but not all; washed in the bath or regeneration, with one exception.>
<13:12 Know ye; do you understand the meaning?>
<13:15 An example; of humility, condescension, and love. To inculcate the importance of these was his object in doing what was usually done by a servant.>
<13:17 These things; the truths which he had been teaching them. In imitating the example of Christ, especially his humility, condescension, and kindness--in believing his declarations, trusting in his merits, and obeying his commands, men may be supremely and eternally blessed.>
<13:18 The scripture; Ps 41:9; strikingly fulfilled in Judas.>
<13:19 Ye may believe; have new evidence that I am the Messiah, and continue to believe it.>
<13:20 Whomsoever I send; as my minister.>
<13:21 Christ is greatly grieved when any of his professed disciples so conduct as to injure themselves, dishonor him, and bring reproach on his cause.>
<13:23 Leaning on Jesus' bosom; the guests reclined on couches, each resting on his left elbow, with a pillow supporting his head, his face towards the table, and his feet towards the hinder part of the couch. As John lay next below Jesus, his head was in front of the Saviour's bosom; and in asking a question, he would naturally turn his head over and lean it upon his Master's breast. One of his disciples; John, the writer of this book.>
<13:26 A sop--dipped; a piece of food dipped in the sauce used on that occasion.>
<13:27 Entered into him; took full possession of him, and instigated him to carry out the purpose, already formed under his influence, of betraying his Master. Compare ver Joh 13:2.>
<13:31 Is the Son of man glorified; the hour of his conflict with Satan was to be that of His triumph over him and exaltation to heaven.>
<13:32 Glorify him; in his death, resurrection, and ascension, as the conqueror of death and hell.>
<13:33 As I said; chap Joh 7:34.>
<13:34 A new commandment; new as to its peculiar application to Christians, the clearness and power with which it was taught, and the motives with which it was enforced. One decisive evidence of love to Christ is love to his people. The manifestation of this, while it is among the brightest evidences of true religion, is also among the most powerful means of leading men to embrace it. Chap Joh 17:21.>
<13:36 Thou shalt follow me; through death upon the cross, to heaven. Chap Joh 21:18,19.>
<13:38 Net crow; Mt 26:74; Lu 22:60.>
<14:1 Be troubled; a season of great trial was just before them. But in passing through it, they must not lose their confidence in God or in him. Trust in God the Father, and in Jesus Christ his Son, is the great safeguard against troubles, and the all-sufficient support under them.>
<14:2 In my Father's house; in heaven. Many mansions; dwelling-places. I go to prepare a place for you; this going was begun by the Saviour's death--after which he never abode permanently with his disciples--and completed at his ascension. His death, resurrection, and ascension to heaven, were all parts of the one act of going to the Father to prepare a place for his followers.>
<14:3 Come again; the perfect fulfilment of this promise will be at Christ's second coming, when the bodies of believers, being raised in glory, will be reunited with their spirits, and they received by Christ to the everlasting mansions prepared for them in heaven. But it has also a previous blessed fulfilment to the spirit of each true Christian when he leaves this world. Lu 16:22; 23:43; 2Co 5:8; Re 14:13.>
<14:4 Wither I go; to the Father. The way ye know; to the Saviour the way was by death upon the cross, as he had often foretold his disciples. Compare note to verse Joh 14:3. To the disciples, the way was by faith in him, yet so that they too must follow him through death to the glory of heaven. Compare chap. Joh 13:36.>
<14:6 I am the way; to God's presence. The truth; the author and revealer of truth. The life; the author and giver of life, natural and spiritual. There is no way of access to the Father but through his Son Jesus Christ. Those, therefore, who willfully reject him, have no scriptural communion with God.>
<14:7 Known my Father; the reason of this is, their oneness; he being the brightness of the Father's glory, and the express image of his person. Chapter Joh 10:30; Heb 1:3. From henceforth ye know him; from this time onward begins your more perfect knowledge through me of the Father. It was in connection with the removal from the disciples of his personal presence, which was now just at hand, that the Comforter should be sent to teach them of Christ and the Father.>
<14:8 Show us the Father; he meant an outward showing.>
<14:10 Not of myself; not independently of, or in opposition to the Father, but from him; so that in me the Father himself speaks to you. He doeth the works; he gives them to me to do, chap Joh 5:36, and he is so present in me that my working is his working.>
<14:12 Greater; greater in number, extent, and influence, by Christ's power, and under the influence of the Spirit, which, after his ascension to heaven, he would give them. Mr 16:20; Ac 2:41. Faith in Jesus Christ is the means not only of justification and acceptance with God, but also of distinguished usefulness among men.>
<14:13 In my name; in dependence on, and for the purpose of honoring him.>
<14:14 Any thing; in accordance with his will, and which would be needful for the work to which he called them.>
<14:16 Another, Comforter; another than myself, one who shall make good to you the loss of my personal presence. This is the first time that the word "Comforter" is applied, in the Scriptures, to the Holy Ghost. The Greek word, which occurs only in writings of John, means both advocate--as it is rendered in 1Jo 2:1, where it is applied to Christ--and Comforter. The Holy Spirit is the Counsellor and Guide, as well as the Comforter of God's people.>
<14:17 The Spirit of truth; the Holy Spirit, who reveals the truth, and works in men to will and to do in obeying it. The world; men who seek earthly things as their chief good. Seeth him not; they have no spiritual view of him, do not feel their need of him, or seek his aid. In you; to enlighten their minds, purify their hearts, show them what they should do, and enable them to do it.>
<14:18 Comfortless; literally, orphans, bereft of my presence, as children of the presence of their father. Come to you; spiritually, through the Comforter.>
<14:19 Seeth me no more; that is, in my personal presence, the only way in which they are able to see me. Ye see me; spiritually. See below, ver Joh 14:21-23.>
<14:20 At that day; when the Holy Spirit should come, and they should enjoy not the bodily, but the spiritual presence of the Redeemer.>
<14:22 How is it; he supposed that Jesus spoke of his bodily presence. But after the Holy Ghost should come, he would remember and better understand the words of Christ.>
<14:23 Love to Jesus Christ will lead a man to obey his commands, and will secure to him the illuminating, purifying, and blissful presence of both the Father and the Son. They shall dwell with him and he with them, and his habitual communion be truly with the Father and his Son Jesus Christ. 1Jo 1:3.>
<14:26 All things; all things which should be needful to fit them for the duties of their office. Bring all things to your remembrance; in such a way that he should, at the same time, enlighten them as to their true meaning.>
<14:27 Peace I leave with you; as my parting gift. The allusion is to the Hebrew form of benediction, which is, "Peace be with you." My peace; that which resembles his own, and which he alone can give: "the peace of God, which passeth all understanding." Php 4:7. Not as the world giveth; their benedictions are empty and inefficacious; but mine are sincere and powerful. The consequences of thus dwelling and communing with the Father and the Son, are peace of conscience, joy in the Holy Ghost, and good hope, through grace, that when absent from the body, they shall be present with the Lord, beholding his glory and rejoicing in his love. Such a one, therefore, need not fear, though the earth be removed, and the mountains be carried into the midst of the sea, though the waters roar and are troubled, and the mountains shake with the swelling thereof; for he will be kept in perfect peace, his mind being stayed on God.>
<14:28 Greater than I; not in nature, but in condition. He is in a glorious and exalted state; I am in a humble and lowly condition, and if ye loved me, ye would rejoice in my going to him, for I shall then be in the same glorious and exalted state in which he is, and in which I was before creation. Chap Joh 17:5. From that state of glory he would send them the Holy Ghost, and accomplish all which he had promised. When Christ speaks of the Father as greater than himself, he refers not to his own nature, but to his office, condition, and work as Mediator; and it implies no inferiority in his original dignity, wisdom, power, and glory.>
<14:29 Have told you; of my departure from you by the death of the cross. Ye might believe; that is, more fully: might have your faith in me greatly strengthened by witnessing the fulfilment of my words.>
<14:30 Prince of this world; the devil. Nothing in me; no sin or weakness of which he can take advantage. The devil would not succeed in his object, but would only help to show that Jesus was the Messiah.>
<14:31 But that the world may know; fill out this clause thus: But [this conflict with the prince of this world is permitted] that the world may know, etc. In it they are to see an example of my obedience to the Father, even unto death.>
<15:1 The true vine; figuratively and spiritually. Husbandman; keeper of the vineyard.>
<15:2 Every branch in me; professed disciple. That beareth not fruit; does not live a holy life. Purgeth it; in the original, cleanseth it; that is, by pruning, which here represents all the discipline to which Christ subjects his disciples.>
<15:3 Now ye are clean; pruned, and thus spiritually purified and made fruitful. The word "clean" is used with reference to the word "purgeth" or cleanseth, in verse Joh 15:2.>
<15:4 Abide in me, and I in you; the union between Christ and his disciples is mutual. They abide in him by faith, love, and obedience. He abides in them through the Holy Spirit, as the source of their spiritual life, light, and strength. Compare Php 2:12,13.>
<15:5 The union of the branch to the vine by a vital communication is no more essential to its life and fruitfulness, than the union of souls to Christ, by receiving and trusting in him as the Saviour, is to their holiness and bliss.>
<15:6 Cast forth as a branch; cut off and cast away as a useless branch. This separation of the unfruitful branches takes place in a measure only in this world. It will be completed at the judgment-day. Compare the parable of the tares in the field. Mt 13:24-30,36-43.>
<15:7 It shall be done; your prayers offered according to the will of God, shall be answered.>
<15:9 Continue ye in my love; by continuing to obey my will.>
<15:11 That my joy might remain in you; that you might have, in union with me, that joy which I have in union with the Father. Might be full; namely, by having my joy abiding in you.>
<15:12 My commandment; chap Joh 13:34.>
<15:15 I call you not servants; do not treat you as servants in merely commanding you, but as friends, in communicating to you my plans, and the reasons of them.>
<15:16 Ye have not chosen me; they had not first chosen him, but he had chosen them; and their choice of him was the fruit of his choice of them. 1Jo 4:19. Ordained you; set you apart to the work to which I have called you. That your fruit should remain; that the precious results of your holy labor should remain to the world for all time, and to yourselves and the souls saved by your instrumentality throughout eternity. That whatsoever ye shall ask of the Father; to be connected, like the preceding clause, immediately with "I have ordained you." The appointment that they should go and bring forth fruit, and that their prayers to the Father should be answered, are both parts of one whole. The originating cause of the salvation of men is not their love to God, or their choosing him as their portion, but his love to them, and his choosing them to salvation through sanctification of the Spirit and belief of the truth. 2Th 2:13; Eph 1:4,5.>
<15:19 Of the world; governed by the principles and maxims of worldly men. Chosen you; to be my followers, and like me in character.>
<15:20 The servant--his lord; Mt 10:24,25.>
<15:21 For my name's sake; on account of your likeness and attachment to me. Know not him that sent me; and therefore have not known me nor you. Chap Joh 8:19.>
<15:22 Had not had sin; to such a degree; because they would not have sinned against such great light. Lu 12:48. No cloak; no covering or excuse.>
<15:23 Me--my Father; chap Joh 10:30; 14:7,9.>
<15:24 Such is the union between Christ and the Father, that as men treat the one, so they treat the other; and the greater the light which any have as to the character and will of God, the greater will be their guilt and condemnation, if they do not love and obey him.>
<15:25 Their law; Ps 35:19; 69:4; 109:3.>
<15:26 He shall testify of me; to the integrity and divinity of my character, and to the truth and meaning of my teachings.>
<15:27 Ye also; the apostles. From the beginning; the beginning of his public ministry. Mt 4:17-22; Ac 1:21,22.>
<16:1 Have I spoken unto you; forewarning you of the persecutions that shall come upon you. Offended; led to apostatize, or commit sin, to avoid suffering.>
<16:2 Put you out of the synagogues; chap Joh 9:22. It is not enough that a man follow the dictates of conscience. His mind must be enlightened as to the will of God; and when he understands what that will is, he must be disposed to do it, or his conscience will not be a safe guide.>
<16:4 I was with you; it was not then needful for them to know the trials that were coming upon them; but as he was to depart, it became needful that by looking to the Holy Spirit they might be prepared to meet them.>
<16:5 Whither goest thou? this very question had been asked before, chap Joh 13:36, but in a different sense from that intended by our Lord. There the inquiry had respect simply to the place whither. Here it refers to the nature of the place, which is the right hand of God, and the great good thus to be secured for the disciples who remain behind for a season.>
<16:7 Expedient; for the gift of the Spirit would be better for the church than the continued personal presence of Christ. Things which men exceedingly deprecate are often highly expedient; and God in accomplishing them consults not only his own glory, but their highest good, and the good of his kingdom.>
<16:8 Reprove; convince.>
<16:9 Of sin; especially the sin of rejecting the Saviour.>
<16:10 Of righteousness; that he was perfectly righteous; and that his work was accepted of God as a ground for the justification of sinners. This was proved by his resurrection from the dead, and his ascension to heaven.>
<16:11 Of judgment; that as Satan the god of this world was vanquished and condemned, so all his continued adherents will be, and have their part with him and his angels. Chap Joh 12:31; Mt 25:41.>
<16:12 Many things; with regard to the object and effects of his death, and the establishment and progress of his kingdom. Cannot bear them; they were not then prepared rightly to apprehend and properly to improve additional instruction.>
<16:13 Into all truth; all that would be needful to a full revelation of the gospel. Not speak of himself; not in opposition to, but in accordance with the Father and the Son. Chap Joh 5:19,30,31; 12:49,50. That shall he speak; he would communicate the will of the Father and the Son as far and as fast as the glory of God and the good of men would require.>
<16:14 Glorify me; the effect of his teaching would be to honor the Saviour. Shall receive of mine; or, shall take of mine--shall take of the things that pertain to my person and work, which is the same thing as taking of things that pertain to the Father, since the Father and the Son are one in nature and counsel, and the Father has committed all things to the Son. These words give us one of the decisive tests by which true teaching may be distinguished from that which is false. All true teaching agrees with the testimony of the Holy Spirit in glorifying Christ.>
<16:15 Are mine; Mt 11:27; 28:18.>
<16:16 A little while; the Saviour designedly puts these words into the form of a divine riddle, to be solved by the event. Ye shall not see me; because he would be removed from their presence by death. Ye shall see me; they should see him in his personal presence after his resurrection. This, however, should be only the earnest of a more glorious spiritual vision of him through the Comforter, after his ascension to heaven. Compare chap Joh 14:19. Because I do to the Father; it was by his death, resurrection, and ascension that Christ went to the Father, and these three events are here considered as constituting one whole.>
<16:18 We cannot tell what he saith; cannot understand his meaning.>
<16:20 Weep and lament; at his death. The world; wicked men. Turned into joy; by his resurrection and ascension, and the descent of the Holy Spirit. Men often weep at what will give them the greatest joy, and rejoice at that which will cause them the deepest sorrow.>
<16:21 Remembereth no more the anguish; the time of Zion's keenest anguish has always been the birth-time of her enlargement; and the time of the believer's deepest sorrow, the birth-time of his highest and holiest joys.>
<16:23 Ask me nothing; it would not be needful, as it then was, that they should make inquiries of him. The Holy Ghost would give them all needful instruction.>
<16:24 Asked nothing in my name; they had not been accustomed before this to pray in the name of Christ; but after this they would be, and for his sake God would bestow whatever they needed.>
<16:25 In proverbs; or parables--somewhat obscurely, and in such manner that they did not fully understand his meaning. The time cometh; after his resurrection, and the gift of the Holy Ghost. Plainly; he would more plainly instruct them by his Spirit, and they would more fully understand his truth.>
<16:26 I say not--that I will pray the Father; that is, I say not this simply, but something more. The Saviour does not mean to deny that he will intercede with the Father for his disciples; but rather to lead their minds beyond this truth, which he had frequently stated, to another: that the Father is one with him in loving them, so that his intercession for them must prevail.>
<16:27 The Father himself; of his own accord.>
<16:30 By this we believe; he had in the last few verses so fully met their difficulties about his meaning in verse Joh 16:16, and that without their stating them, that they were more than ever convinced of his omniscience and Messiahship. 30-32. Disciples of Christ may at some times possess and manifest strong confidence in him, and at others act as if they had none: were it not for his grace, all would utterly forsake him and perish.>
<16:32 To his own; notwithstanding the strong faith in him which they had expressed, they would soon desert him and return to their homes, or places of abode; and so far as human friends were concerned, he would be left alone.>
<16:33 In me; in living union with me. In the world; from the men and spirit of the world. Overcome; overcome all your enemies, and obtained for you eternal redemption from their power.>
<17:1 These words; the words contained in the preceding chapters. The hour; the time for his suffering and death. Glorify thy Son; by sustaining him in his coming trials, and showing that he is indeed the Messiah. May glorify thee; in making known thy salvation, and preparing multitudes for glory.>
<17:2 Power over all; for the salvation of his people. Mt 28:18; Joh 5:21; 6:37,40; 10:15,16.>
<17:3 This is life eternal; the right knowledge of God and Jesus Christ gives endless spiritual life to the soul. The knowledge of God and of Jesus Christ is as important to men as their eternal salvation. Hence, it is the duty of those who have this knowledge, to aid in imparting it to all people; and all laws, customs, and usages which tend to prevent this are wicked, and ought for ever to be done away.>
<17:4 Glorified thee on the earth; by doing in all things what he was commissioned of the Father to do. Finished the work; the work to which he was appointed.>
<17:5 Before the world was; Php 2:6.>
<17:6 I have manifested thy name; thy whole character and attributes; for these are all comprehended in God's name. The men; his disciples, especially the apostles.>
<17:7 All things--are of thee; they understood the truth, which the Saviour had so often maintained against his persecutors, that all his mighty works were wrought not merely of himself, but in accordance with the commission he had received from the Father. Chap Joh 5:19,30,36; 7:28; 8:28,54; 10:37,38; 12:49.>
<17:8 The words which thou gavest me; the instructions which he was commissioned to impart. The reception of the doctrines revealed, and the performance of the duties required by Jesus Christ, are evidences of our being given to him of the Father; they increase our knowledge of him and love to him as a Saviour, and awaken expectations which will not be disappointed, of dwelling with him for ever.>
<17:9 I pray for them; his disciples. Not for the world; the wicked. He did not at this time pray for his enemies, but for his friends.>
<17:10 I am glorified in them; on their side, by their reception of me as a Saviour, and committing themselves and all their interests to my guidance and disposal; on my side, by the manifestation which I make in them of my power and love, in sanctifying them, giving them the victory over the world and Satan, and bringing them to glory everlasting.>
<17:11 Through thine own name; see note to verse Joh 17:6.>
<17:12 I kept them in thy name; by the manifestation of thy character and will in my instructions and example. Son of perdition; Judas, whose ruin was foretold in the Scriptures. Ps 109:8; Ac 1:20. Those manifestations of God by which he makes known his character and will, the duty and blessedness of serving him, and the sin and misery of neglecting him, are means by which he keeps his people with his mighty power, through faith unto salvation. 1Pe 1:5.>
<17:13 My joy; see note to chap Joh 15:11. Fulfilled; made perfect and lasting.>
<17:14 There is a great difference between the spirit of the world and the spirit of Christ. One leads us to seek our chief good in earthly things, the other to seek it in learning and doing the will of God.>
<17:17 Through thy truth; by giving them right views of truth, and leading them to obey it. As divine truth is the great means of sanctification, the more clearly it is understood and the more faithfully it is obeyed, the more holy men will be, the more lovely will be their character, and the greater their usefulness and enjoyment.>
<17:19 I sanctify myself; consecrate and devote myself to my work, that they may be prepared and disposed to perform theirs.>
<17:20 These; apostles, or those who were then disciples. Which shall believe; all who should become his disciples and followers.>
<17:21 They all may be one; Christ brings all his true disciples into an inward living union with himself and the Father, and thus makes them all one with each other. That the world may believe that thou hast sent me; the manifestation in believers of this inward union of love and holiness, first with the Father and the Son, and then with one another, is to the world the most convincing proof of the truth of Christ's mission.>
<17:22 The glory; given him as a reward for his labors and sacrifices as Mediator. I have given them; by participation and promise, in order to their complete and perfect union to him and one another, that the world might see the excellence of his religion, and be led to embrace it.>
<17:23 Increasing union of views, affections, and efforts among the disciples of Christ, will furnish increasing evidence of the divine excellence of his religion, and lead increasing numbers to embrace it.>
<17:24 With me where I am; in heaven. The death of Christians is in answer to the prayers of Christ, and for the purpose of removing them to the perfect and everlasting enjoyment of his presence in heaven.>
<17:25 Not known; not known so as to love and obey him. These; his apostles and disciples.>
<17:26 Declared unto them thy name; made thee known to them. Will declare it; will more fully make thee known to them, to the increase of their love, union, and blessedness.>
<18:1 These words; the words of the wonderful prayer which he had just offered. Cedron; or Kidron, a torrent-bed which ran through the valley of Jehoshaphat, on the east side of Jerusalem, between the city and the mount of Olives. It has water only in the rainy season. 2Sa 15:23. By communing with God in prayer and supplication with thanksgiving, a man is prepared to go forth, in His name and strength, to any duties or trials to which he is called.>
<18:2 2-11. Jesus is betrayed. Mt 26:47-56.>
<18:4 The sufferings and death of Christ were all foreseen by him, and were perfectly voluntary. He carefully avoided committing himself to his enemies, or suffering them to take him before his time had come; but then he made no attempt to escape. He even went forth and met them, and committed himself to their disposal. He went as a lamb to the slaughter; as a sheep before her shearers is dumb, so he opened not his mouth. Isa 53:7,8; Ac 8:32,35.>
<18:6 Fell to the ground; prostrated themselves before him under the influence of his divine power. This showed that they were completely in his power; had he seen fit, he could easily have escaped their hands, or summoned legions of angels to his rescue.>
<18:8 These; his disciples. Such was the love of Christ to his disciples, and such his delight in their enjoyment, that he was more ready to save them from their enemies, than to save himself from the agonies of the cross.>
<18:9 Lost none; chap Joh 17:12.>
<18:11 The cup; the sufferings which my Father hath appointed for me, shall I not endure them?>
<18:13 13, 14. Christ before Caiaphas. Chap Joh 11:49-52; Mt 26:57.>
<18:14 The words of Scripture sometimes have a twofold meaning. They may express a meaning which the speaker had in his own mind; they may also express a more important meaning, intended by the Holy Spirit, though the speaker did not apprehend that meaning. Others may afterwards apprehend, and be more benefited by this meaning, than by that which the speaker had in his mind. Chap Joh 11:49-52.>
<18:15 Another disciple; supposed to be John, the writer of this gospel.>
<18:16 Brought in Peter; into the hall or court, which was a square space open above, around which the palace was built. The chamber in which the trial of Jesus was going on was open in front, so that Peter could witness the trial, and Jesus could see Peter. Mr 14:66; Lu 22:61.>
<18:19 Asked Jesus; questioned him, as if he and the Sanhedrin needed information on these points.>
<18:20 Openly; Mt 26:55.>
<18:21 Ask them which heard me; this was the legal and proper way of gaining evidence.>
<18:22 Men may violate law, commit injustice, and act the part of tyrants, out of professed regard to the ministers and institutions of religion; and while instigated by the devil and their own evil passions, they may think that they are serving God. Chap Joh 16:2.>
<18:25 25-27. Peter's denial. Mt 26:69-75.>
<18:28 Hall of judgment; the place where Pilate the Roman governor held his court. Lest they should be defiled; they would not enter this apartment of a Gentile, lest they should be unfitted to partake of the passover; so careful were they about contracting ceremonial pollution, while they were seeking to commit the most horrible crime. But that they might eat the passover; see not to chap Joh 13:1. Men may be very scrupulous as to the observance of outward ceremonies, yet very reckless as to the commission of enormous crimes. No outward profession or inward experience is evidence of piety, unless it lead men to hate known sin, and have respect to all God's commandments. Ps 66:18; Ps 119:6. 28-40. Christ before Pilate. Mt 27:1-25.>
<18:30 Malefactor; an evil-doer, violater of law.>
<18:31 Take ye him, and judge him; see note to chap Joh 19:6. Not lawful; the power to put one to death had been taken away from the Jews by the Roman government; and this was the reason why they sought to have Pilate condemn him.>
<18:32 The saying of Jesus; crucifixion was a Roman punishment, while the Jewish punishment for the crime of which they accused him was that of stoning. Le 24:11,15,16. By taking him to the Roman governor, they fulfilled his own prediction concerning the manner of his death. Mt 20:19.>
<18:33 Art thou the King of the Jews? this question was put in consequence of the Jews having charged him with pretending to be a king, which they said was rebellion against Caesar. Lu 23:2; Joh 19:12.>
<18:34 Of thyself; have you observed any thing treasonable in me; or do you ask this question in consequence of what others have told you?>
<18:35 Am I a Jew? as much as to say, Since I am not a Jew, I cannot be supposed to be acquainted with the controversy which thine own countrymen the Jews have with thee. I wish to hear from thine own mouth in respect to this charge of making thyself a king.>
<18:36 Not of this world; not an earthly kingdom. The kingdom of Jesus Christ is a spiritual kingdom. He reigns by truth and love over the hearts and lives of men. the friends of truth obey his voice, and find in doing it great reward. 2Co 4:2; Ps 19:11.>
<18:37 Art thou a king then? a king of any sort? Thou sayest; this was equivalent to replying, I am a king. He then proceeded to show what kind of a king he was; one who came into the world to make known the truth, and to govern men not by force, but by spiritual influence. Pilate saw that his claims were no crime against the Roman government, and hence said,>
<18:38 What is truth? by this question Pilate manifested both his ignorance of our Lord's meaning, and his indifference in respect to His doctrine. I find in him no fault; this the Holy Ghost caused to be written on an imperishable record, that it might stand an eternal monument of the falsehood of the Jews, and the perfect innocence of Jesus Christ.>
<18:39 Ye have a custom; Mt 27:15.>
<18:40 Not this man, but Barabbas; Pilate had asked them which they would have him set at liberty, Barabbas or Christ. Mt 27:17. Barabbas was guilty of sedition, the crime which they wrongfully charged upon Jesus; he was also a robber and a murderer. Yet they preferred him to the Son of God, the Saviour of the world. Mr 15:7; Lu 23:19,25.>
<19:1 Scourged him; apparently hoping by this punishment to satisfy the Jews. Compare Lu 23:22. 1-3. Christ scourged and mocked. Mt 27:26-31; Mr 15:15-20.>
<19:6 Take ye him, and crucify him; said in irony by Pilate, and so understood by the Jews. Compare chap Joh 18:31.>
<19:7 We have a law; Le 24:16. Pilate having pronounced him not guilty of sedition, of which they had accused him, they went back to the charge of blasphemy, on which he had been condemned by the council. Mt 26:63-66.>
<19:8 He was the more afraid; this was to Pilate a new charge. He saw that the Jews were resolved to kill him, though innocent; and as he claimed to be the Son of God, he was therefore more anxious to release him. His wife also had sent to him to have nothing to do with that just man, for she had suffered many things in a dream because of him. Mt 27:19. Men who do, or consent to others' doing what they know to be wrong, are always liable to great and distressing fears. Conscience condemns them; and though it sometimes sleeps, it may at any moment awake and fill them with terror.>
<19:9 Whence art thou? what is thine origin, human or divine? No answer; Jesus had given all needful information about himself, and he did not think proper to add to it.>
<19:11 Given thee from above; the power of the civil magistrate is from God, and to God he is accountable for the use of it. He that delivered me; Caiaphas, as high-priest, representing the Jewish council. The greater sin; guilty as Pilate was, the Jewish council was still more guilty. They had not only abused the power which God gave them, but were urging Pilate to abuse his; and they were doing it under greater light than he had. Judicial authority and power are gifts of God, for the use of which men are accountable to him; and injustice committed by magistrates under the cover of law, is among the most wicked of all transgressions.>
<19:12 Not Caesar's friend; Tiberius Caesar, then emperor of Rome. Finding that Pilate would not condemn Jesus for blasphemy, they returned to the former charge, of rebellion against the Roman government, and contended that if he released Jesus he was an enemy to Caesar. He might therefore be complained of to the emperor, who was a very jealous and cruel man; and he might thus lose his office, perhaps his life. This induced him to proceed with the trial.>
<19:13 Sat down in the judgment-seat; the place for passing sentence on criminals. He did this for the purpose of condemning Jesus. He loved his office more than his duty; and feared the loss of it more than the commission of judicial murder. Pavement--Gabbatha; an elevated place, paved with costly stones.>
<19:14 The preparation; the preparation for the next day, which was the Sabbath, and the great day of the Jewish Passover. Mr 15:42. About the sixth hour; not far from noon. Mark says the third hour, or nine o'clock in the morning. Mr 15:25. The whole proceedings took several hours, and the different evangelists refer to different periods of the transactions.>
<19:16 16-22. Christ crucified. Mt 27:32-35.>
<19:22 I have written; the meaning of this was, that what he had written he would not alter.>
<19:23 The actions of wicked men, as well as of good men, are a fulfilment of the Scriptures. Though they mean not so, neither do their hearts think so, yet they are evidences to the truth of God's declarations, and that in due time they will all be accomplished. Isa 10:7. 23-24. Christ's garments divided. Mt 27:35; Ps 22:18.>
<19:26 The disciple; John, chap Joh 13:23. The son; one who will henceforth treat thee as his mother, and supply thy wants.>
<19:27 The duty of filial affection, and of the most ready and conscientious discharge of relative duties, was taught by Jesus Christ amidst the agonies of the cross; and no one can imitate him who is not kind to his mother, and who does not, as he has ability and she has need, provide for the supply of her wants.>
<19:28 The scripture; Ps 69:21.>
<19:30 It is finished; his work of suffering for human salvation. Gave up the ghost; dismissed the soul from its connection with the body. Mt 27:50.>
<19:31 A high day; a great day--one of peculiar solemnity. Their legs might be broken; to hasten their death, so that they might be taken from the cross before the Sabbath.>
<19:35 He that saw it; John. Bare record; to a fact that established beyond doubt the real death of Jesus.>
<19:36 The scripture; Ex 12:46; Nu 9:12. A bone of him shall not be broken; these words, originally spoken of the paschal lamb, which was the type of Christ, were now fulfilled in the great antitype. The providences of God are so ordered as to be a fulfilment of his word; and both unite in proclaiming that his counsel shall stand, and that he will do all his pleasure. Isa 46:10.>
<19:37 Another scripture; Zec 12:10.>
<19:38 38-42. Christ's burial. Mt 27:57-61.>
<19:39 Nicodemus; chap Joh 3:1,2.>
<19:42 Nigh at hand; near the place where he was crucified.>
<20:1 Mary Magdalene; Mt 28:1. Believing woman, last at the cross and first at the tomb, has often manifested quenchless love for the Redeemer, and dauntless courage in his cause.>
<20:2 The other disciple; John.>
<20:8 He saw, and believed; that Jesus was risen from the dead. The fact that the grave-clothes had all been left in the tomb, arranged in orderly manner, convinced him that the body of Jesus had not been taken away by friends or foes.>
<20:9 For as yet they knew not; as much as to say, They needed this evidence of sight to convince them, because they did not yet understand the Scriptures. The scripture; Ps 2:7; 16.9,10; 110.1; Ac 2:25-36; Ac 13.33.>
<20:11 We are often distressed and weep at that which springs from and is the manifestation of infinite love, and which will best promote the glory of God and the everlasting good of men.>
<20:14 Christ is often much nearer to us, and much better acquainted with our condition, than we imagine; and he can easily so manifest himself that our weeping shall be turned to joy, and our mourning to praise.>
<20:15 Supposing him to be the gardener; the keeper of the garden in which the body of Jesus was buried. Mt 27:60.>
<20:16 After his resurrection, Christ first showed himself to Mary Magdalene, out of whom he had cast seven devils; not to Mary his mother. He would not by word or deed do any thing to countenance the superstitious reverence and idolatrous worship which has since been offered to the Virgin.>
<20:17 Touch me not; when Christ met the two women, Mt 28:9, they came and held him by the feet, and worshipped him. Mary might now be approaching for this purpose. But Jesus wished her without delay to go and tell his disciples that he was risen from the dead. She would have opportunity before his ascension for all proper expressions of her regard for him.>
<20:19 The same day at evening; the first day of the week, which was from that time, and has ever since been observed as the Lord's day, the Christian Sabbath. Verse Joh 20:26; Ac 20:7; 1Co 16:2; Re 1:10. From the resurrection of Christ to the present time, his people have assembled for public worship on the first day of the week, and while thus assembled he has often manifested himself to them as he does not to the world, and kindly spoken peace to the souls.>
<20:20 He showed unto them his hands; to convince them that he was certainly raised from the dead. Christ appealed to and admitted the correctness of the judgment of our senses. To these the evidences of his miracles and of his resurrection were addressed. By these it was known with perfect certainty, that his miracles were real, and his resurrection true; by these also it is known, with equal certainty, that the doctrine of transubstantiation is false.>
<20:21 Send I you; to proclaim the gospel, and make known the way of salvation.>
<20:22 Receive ye the Holy Ghost; this was to fit them for their work. Jesus Christ, by his Spirit, will furnish his ministers for the discharge of all the duties to which he calls them; and they may at all times with affectionate confidence look to him for all needed aid.>
<20:23 Remit--retained; the same power and authority are here conferred equally upon all the apostles, and no one is in any respect raised above another. The power conferred was, under the teaching of the Holy Ghost, to declare the way in which men can be pardoned, sanctified, and saved. Mt 16:19; 18:18.>
<20:24 Thomas; chap Joh 11:16.>
<20:25 Except I shall see; this shows how difficult it was to convince even the disciples that Jesus had risen from the dead.>
<20:26 After eight days; on the next Lord's day.>
<20:27 Reach hither thy finger; this showed that Jesus knew what Thomas had said.>
<20:28 My Lord and my God; this was addressed to Jesus Christ, and was commended by him as a just expression of true faith. Jesus Christ approves of being addressed by his people as their Lord and their God. The more they become acquainted with him, the deeper is their conviction that this is his true character, and the more do both affection and duty lead them thus to adore him. Chap Joh 5:23.>
<20:29 Blessed are they; they who like Thomas believe in Christ, and though they have not seen him, acknowledge him as their Lord and their God.>
<20:30 Signs; miracles in proof of his divinity, and his resurrection from the dead.>
<20:31 Ye might believe; exercise living faith in Christ, and through this faith be justified, sanctified, and saved. As the object of God in causing his truth to be written and printed was, that men might believe and be saved, all should be taught, and should be disposed to read it. It was given in this form to promote the salvation of men, and is often rendered effectual by the Holy Spirit for this purpose. It should therefore, without hinderance and without delay, be circulated among all people.>
<21:1 Sea of Tiberias; the same as the sea of Galilee and the lake of Gennesaret. Mt 4:18; 26:32; 28:10; Mr 14:28; Mr 16:7.>
<21:2 Of his disciples; Mt 4:21; 10:2-4.>
<21:3 Our wants are to be supplied through our own voluntary and active instrumentality. Due attention to worldly concerns is required, and is acceptable to Jesus Christ. When rightly employed in secular business, men are serving him as really as when employed in religious duties, and will equally meet his approbation.>
<21:5 Meat; the word in the original means something eaten with bread, as flesh or fish.>
<21:6 For success in our worldly business we are dependent upon God; and whenever we receive earthly blessings it is from him, who openeth his hand and supplieth the wants of every living thing. Ps 145:15,16.>
<21:7 That disciple; John. Was naked; without his outer garment, as was common with fishermen.>
<21:8 Two hundred cubits; about twenty rods.>
<21:9 A fire of coals--fish; which had been miraculously provided by the Lord.>
<21:12 Dine; the original word was applied to a meal taken in the early part of the day.>
<21:14 The third time; the third time that he appeared to the apostles when together. When ministers of Christ follow his directions, they will have abundant evidence of the truth of all his declarations; so that in addressing others on the great concerns of salvation, they may speak of what they know, and testify to what they have seen of the manifestations of his power and grace.>
<21:15 More than these? more than the other disciples, as Peter had intimated that he did. Mt 26:33; Joh 13:37. Lambs; the tender and feeble followers of Christ the great and good Shepherd. Chap Joh 10:11-16.>
<21:16 Sheep; more advanced Christians.>
<21:17 Peter was grieved; the thrice repeated question reminded him of his thrice repeated denial of his Lord. Feed; communicate spiritual instruction and take care of their souls. Ac 20:28. The most important of all qualifications for a minister of the gospel, and for the right discharge of all duties, is love to Jesus Christ; and those who possess and rightly manifest this, may expect to be rendered eminently useful to themselves and their fellow-men.>
<21:18 Girdedst thyself; he was at liberty to go and come at pleasure. Stretch forth thy hands; in crucifixion. Gird thee; bind thee for execution. The binding, though coming before crucifixion, is named after it.>
<21:19 Signifying; pointing out beforehand. Follow me; in an emphatic sense, Follow me not only as my disciple, but in my crucifixion also. No ardency of devotion to the cause of Christ, and no degree of usefulness will secure his servants from great trials, or even from violent death; but no trials will come upon them except under the direction of God, and such as will best prepare them to glorify and enjoy him.>
<21:20 Whom Jesus loved; chap Joh 13:23-26.>
<21:21 Lord, and what shall this man do? in what way shall he die?>
<21:22 That he tarry; that he stay on the earth and not die. Till I come; words designedly left indefinite. Our Lord's providential coming in the destruction of Jerusalem seems to have been referred to. What is that to thee? it is none of thy business, nothing about which you should concern yourself. Every man should mind his own proper business; while he should be careful not to neglect his appropriate concerns, he should be equally careful not to intermeddle with the concerns of others.>
<21:23 Should not die; a tradition from the days of the apostles, which was not true. If I will that he tarry till I come; it was the business of Christ to direct with regard to the length of John's life, and the manner of his death, and not the business of Peter. It would do him no good to be informed, and Christ would not encourage him in making useless inquiries.>
<21:24 This is the disciple; John, the writer of this gospel.>
<21:25 The world itself could not contain; this is a strong expression, designed to convey the idea that if all which Christ said and did were written, the books would be very many, much too numerous for the highest usefulness to men. Men may speak and write too much, even about religion. Their usefulness depends not so much on the amount of what is spoken or written, as on its fitness and propriety. "A word fitly spoken is like apples of gold in pictures of silver." Pr 25:11.>
\kniha{Acts}
\zkratka{Acts}
<1:1 The former treaties; the gospel of Luke. The book of Acts was written by Luke, and addressed to the same individual to whom he addressed his gospel. Lu 1:3. Began both to do and teach; the meaning is, that he has given Jesus' works and teachings from the beginning.>
<1:2 The day in which he was taken up; taken up to heaven. Lu 24:51.>
<1:3 His passion; his suffering, especially on the cross. As the resurrection of Christ was a demonstration of his Messiahship, and of the truth and justice of his claims, God has given to those who have the Bible infallible evidence of the absolute certainty of that great event.>
<1:4 The promise; the promise of the Holy Spirit. Heard of me; Joh 14:16,26; 15:26; 16:7-13.>
<1:5 John truly baptized with water; Mt 3:11; Mr 1:8; Lu 3:16; Joh 1:33.>
<1:6 Restore again the kingdom; the temporal kingdom, by delivering the Jews from the power of the Romans.>
<1:7 The times or the seasons; for the establishment of earthly kingdoms. The words contain a general rebuke of that curiosity which engages men in vain questions about the times and seasons of God's providential dealings with men. The more eagerly good men pry into what is not revealed, and what God did not design that they should know, the more likely they will be to misapprehend and neglect what he has revealed, and what deeply concerns both themselves and their fellow-men. It is therefore the dictate of wisdom always to remember, that secret things belong unto God, and things revealed to us and our children. De 29:29.>
<1:8 Ye shall receive power; he turns away their thoughts from these vain inquiries to the spiritual office and work for which they are about to be furnished from on high.>
<1:10 Two men; angels in the form of men. Lu 24:4; Joh 20:12.>
<1:11 So come; come to judgment. Mt 26:64; Re 1:7. The certainty that Jesus Christ will come to judgment should lead every man to continue in the faithful discharge of his appropriate duties, that when the Saviour shall appear he may be found ready, and lift up his head with joy, knowing that his eternal redemption has come.>
<1:12 A sabbath-day's journey; a little less than a mile.>
<1:14 Mary the mother of Jesus; this is the only time she is mentioned after the resurrection of Christ, and she was with other redeemed sinners supplicating God for mercy. The fact that Mary the mother of Jesus attended with others, when they met to implore divine mercy, is evidence that she, as well as they, needed mercy; that she was a sinner, and like other sinners, could be saved only through the rich grace of God in Jesus Christ.>
<1:16 This scripture; that quoted in verse Ac 1:20.>
<1:18 This man purchased; he was the occasion of purchasing, as it was purchased with the money which he received for betraying Christ. Falling headlong; he first hanged himself, Mt 27:5, and then fell as here mentioned.>
<1:20 The book of Psalms; Ps 69:25; 109:8. Bishopric; office. Men may be as free and accountable, as praise worthy or blame worthy in doing what the Bible declared before they were born that they would do, as if it had said nothing about them.>
<1:22 From the baptism of John; the time when Christ entered on the duties of his public ministry.>
<1:24 Which knowest the hearts of all; this is the peculiar prerogative of God. 1Ch 28:9; Ps 139:1,23; Jer 17:10; Joh 2:24,25; 21:17; Re 2:18,23; and yet this prayer was evidently addressed to Christ.>
<1:25 His own place; his appropriate abode; that for which he was fitted; the place of torment. Mt 25:46;26:24. In the future world every man will go to the place for which he is prepared. To be prepared for heaven, he must in this world be heavenly in temper and conduct. If he is not, his place in the next world will be in hell. Ps 9:17; Mt 5:30; 10:28; Lu 16:23; Re 20:14.>
<2:1 Pentecost; this signifies the fiftieth, and was the name of the feast which was celebrated on the fiftieth day from the second day of the Jewish passover.>
<2:3 Cloven tongues like as of fire; in the form of tongues divided, and in appearance like fire, which rested upon each of the apostles.>
<2:4 Filled with the Holy Ghost; received his miraculous gifts. Speak with other tongues; in various other languages, which they had not before known. When Christians are united in waiting upon God by prayer and supplication, they may expect in due time to receive abundantly of his blessings; and to be furnished by his Spirit for the various duties to which they are called.>
<2:5 Dwelling; sojourning during the feast of Pentecost. Every nation; a general expression for people of various countries, as mentioned in ver Ac 2:9-11.>
<2:6 The multitude--were confounded; astonished, and thrown into great perplexity.>
<2:8 How hear we; we Parthians, Medes, and Elamites hear each in his own tongue.>
<2:10 Proselytes; Gentiles who had been converted to the Jewish religion.>
<2:11 The wonderful works of God; with respect to his Son.>
<2:13 Opposers of the work of God show the weakness and wickedness of their cause, by the measures which they adopt, and the means they use to support it. When drunkenness shall teach men new languages, opposers of the gospel may be wise; till then they will be, in the sense of the Bible, fools.>
<2:14 The eleven; the eleven apostles.>
<2:15 The third hour; nine o'clock in the morning, too early for them to be affected with strong drink. This was also the hour of morning worship, and devout Jews were not accustomed to take food or drink till after that time.>
<2:16 This is that; a fulfilment of Joe 2:28-32. Peter does not quote the exact words of Joel, but the sense.>
<2:17 Last days; in the time of the Messiah; under the gospel dispensation. See visions--dream dreams; these are mentioned as among the ways in which God reveals his will to men. Mt 2:13.>
<2:18 Shall prophesy; proclaim the will of God, and make known future events. Ac 21:9-11.>
<2:19 I will show wonders; such as are described, or referred to, in Mt 24:29-42; Lu 21:25-36.>
<2:20 Sun--turned into darkness; the sun and moon were emblems of the civil government of the Jews, which should be overthrown and destroyed before that great and notable day, when the Lord should appear for the salvation of his friends and the destruction of his enemies. Compare notes to Matthew, chap Mt 24.1-51.>
<2:21 Call on the name of the Lord; Ro 10:12-14; 1Co 1:2; 2Ti 2:22. Prayer to God is as important to every individual as the salvation of his soul.>
<2:23 Him, being delivered; Mt 24.46; 26:53-56; Lu 22:22; Joh 18:37;19:11. Men may be very guilty in accomplishing what God has purposed. Ge 50:20; Isa 10:5-7; Ac 4:27,28.>
<2:24 It was not possible; that Jesus should continue in the grave, consistently with the fulfilment of God's determination to raise him up, as foretold by David. Ps 16:8-11.>
<2:25 Him; the Messiah. I foresaw the Lord; saw the Lord before my eyes, as an object of continual trust.>
<2:26 In hope; in hope of a resurrection, without corruption in the grave.>
<2:27 Hell; this word here does not mean, as it often does, the place of endless torment; but the place or state of the dead. Thy Holy One; Jesus Christ.>
<2:28 Full of joy; the joy of the Messiah, in view of his certain resurrection and ascension to heaven.>
<2:30 Sworn with an oath; 2Sa 7:12-29; Ps 89:3,4,35-37; 132:11; Lu 1:32,33. Of the fruit of his loins; of his descendants. To sit on his throne; rule over the people of God.>
<2:31 David in the Psalms often spoke of himself in language which applied also to Jesus Christ; the Holy Ghost thus spoke by him, and made known what should take place in future times. Ps 22:1-31; 110:1-7.>
<2:33 The promise; Joh 14:26; 15:26; 16:7,13-15. This; their power to speak in various languages.>
<2:34 Sit thou on my right hand; be exalted, and thine enemies all subdued. Ps 110:1; Mt 22:42-46.>
<2:36 House of Israel; the Jewish nation. Lord; Joh 17:2; 1Co 8:6; Eph 1:20-23.>
<2:37 Heard this; that they had crucified the Messiah, the Lord of glory. Pricked in their heart; convicted of sin and deeply distressed. What shall we do? to be saved from the guilt and punishment of sin. When the Holy Ghost accompanies the preaching of the gospel he convinces men of their sins, and leads them to ask, "What must we do to be saved?" And when they are instructed what to do, he inclines them to do it.>
<2:38 Repent; hate and forsake sin. Be baptized; in profession of their faith in Christ, and their consecration to his service. The gift of the Holy Ghost; the Holy Spirit, to enlighten their minds, purify their hearts, and fit them to know and do the will of God.>
<2:39 The promise; the promised influences of the Holy Spirit, and of pardon and salvation through repentance and faith in the Redeemer. Afar off; distant nations, Gentiles as well as Jews. Shall call; by the preaching of his gospel, and leading them to embrace it.>
<2:40 Save yourselves; by forsaking your sins and believing on the Messiah, deliver yourselves from the guilt and ruin of this perverse and wicked generation. Mt 11:16-19; 12:39; 16:4; 23:34-38. It is the duty of men to save themselves from eternal ruin by repenting of their sins, believing on Jesus Christ, and privately and publicly consecrating themselves to his service.>
<2:41 They that gladly received his word; they who believed what Peter had said, and were disposed to comply with his directions. Were added; added to the company of believers.>
<2:42 Continued steadfastly; in their attendance upon, reception of, and obedience to the teaching of the apostles; in Christian communion with one another, and united prayer and supplication for blessings on themselves and their fellow-men.>
<2:43 Fear came; on account of the great things which God had done and enabled the apostles to do.>
<2:44 All things common; so far as their mutual wants required. They did not establish a community of goods by any formal arrangement, but in the fulness of Christian love the rich sold their possessions and goods, that distribution might be made to such as needed aid.>
<2:46 They, continuing; unitedly to frequent the temple at the daily hours of prayer, and joyful partaking of bread at each other's houses, with sincere and upright hearts.>
<2:47 Favor with all; general favor. The church; the company of believers. Saved; from eternal ruin, through repentance of sin and faith in Jesus Christ. When Christians manifest that they are sincere and earnest in the cause of Christ; are united in affections and efforts; are joyful in the Lord, and strive to do good as they have opportunity to all, it may be expected that religion will prosper, and many be added to the church of such as shall be saved.>
<3:1 Ninth hour; three o'clock in the afternoon. The Jews had daily three hours of prayer, the third, sixth, and ninth, or at nine, twelve, and three o'clock. Da 6:10; Ps 55:17. Those who love God will love stated seasons for prayer, and will be disposed, as they have opportunity, daily to observe them. In prayer they commune with their greatest and best friend, and become more and more partakers of his excellence and joy.>
<3:2 Called Beautiful; this was a very splendid gate on the east side of the temple, near to Solomon's porch. Joh 10:23.>
<3:6 Such as I have; the power, namely, to cure him of his lameness. In the name of Jesus Christ; when the Saviour wrought miracles, the power existed in himself. He performed them in his Father's name only in the sense that the Father had sent him to do them. Joh 5:36; Joh 10:25. But when the power was wholly in Christ, and they obtained the exercise of it through faith in him. Compare verses Ac 3:12,16. All persons may be useful. If they cannot do good in one way, they may in another; and true religion will lead them to do it. They will make efforts for this purpose, and depend upon the power and grace of Christ for success.>
<3:12 Good men, when their efforts to be useful succeed, will not ascribe it to their own wisdom, power, or goodness, but to the grace and power of Christ, and they will give him the glory.>
<3:13 Glorified his Son; by showing in his resurrection and ascension that he was the Messiah, and that his claim to be divine was just.>
<3:14 Denied the Holy One; Ps 16:10; Ac 2:27; Mt 27:16-26.>
<3:16 His name; his power. Faith--hath given him this perfect soundness; faith was the means, Peter the instrument, and Christ the cause of the cure. Not only the manner in which Christ wrought miracles, but the manner in which the apostles wrought them and spoke of them, was adapted to lead men to view Christ as the author of those miracles, and to unite in honoring him as they honor the Father. Joh 5:23.>
<3:17 Through ignorance; they did not know when they crucified him, that he was the Messiah. They ought to have known it; and had they rightly improved their means of knowledge, they would have known it. But they hated him, and rejected the light, because their deeds were evil. Joh 15:24,25; 3:20; Lu 23:34; Ac 13:27; 1Co 2:8; 1Ti 1:13.>
<3:18 Christ should suffer; Ps 16:10,11; 22:15-18; 69:1-21; Isa 53:3-10; Da 9:26.>
<3:19 Be converted; turn from all your sins to the love and service of God. When the times of refreshing shall come; the connection of these words with the following verses shows that their primary reference must be to that great season of refreshing when Christ shall come again from heaven to judge his enemies, and give rest to his people. 2Th 1:7-10. Of this, the spiritual refreshments which he now gives through the outpouring of his Spirit are types and earnests. At that great day of refreshing, the sins of all who are found in Christ will be publicly blotted out, and they received with him to glory. By repentance and conversion, through the merits and grace of Christ, sin may be pardoned, and men delivered from its power and punishment. All therefore to whom he is made known, are bound thus to secure these in estimable blessings.>
<3:20 Which before was preached; proclaimed in the Old Testament scriptures, and afterwards more plainly made known by himself and his disciples. But another reading of the original is, "which was before ordained for you;" namely, to come as your Messiah, suffer, and be glorified. Compare chap Ac 2:23; 1Pe 1:20.>
<3:21 Times of restitution; when Christ shall appear in his glory, establish his kingdom as foretold in the Scriptures, and reward every man according to his works. Mt 25:31-46.>
<3:22 Moses truly said; De 18:15-19. Like unto me; Christ was like unto Moses in being appointed of God to make known his will, and being a divinely commissioned leader of his people.>
<3:23 Will not hear that Prophet; will not obey the Messiah.>
<3:24 Samuel; 2Sa 7:16,25,29.>
<3:25 Which God made with our fathers; Ge 12:3; 18:18; 22:18; Gal 3:16.>
<3:26 Unto you first; the Jews. Isa 59:20; Mt 10:5,6; Lu 24:47; Joh 1:11.>
<4:1 The captain; of the guard near the temple. Sadducees; Mt 22:23.>
<4:3 In hold; in prison, or under guard, for safe-keeping. Eventide; evening. When the gospel is faithfully preached, and multitudes embrace it, its opposers are greatly grieved. If they have power, they often attempt to stop its progress by force. But truth cannot be bound or imprisoned: and the imprisoning of those who proclaim it, is often the occasion of its wider extension and more abundant success.>
<4:4 The number; who had believed.>
<4:5 Rulers; members of the Sanhedrim, or great council of the Jewish nation, which consisted of about seventy persons, and had the general superintendence of public affairs.>
<4:6 Annas; he had been high-priest, and was father-in-law to Caiaphas, who was high-priest at that time. Of the kindred; relations.>
<4:7 They asked; by whose authority and power Peter and John had cured the lame man.>
<4:10 By the name of Jesus Christ of Nazareth; by his authority and power. The change produced in a man by the influences of the Holy Spirit is truly wonderful. He who followed Christ "afar off," and trembled at the voice even of a maid-servant, can now face undismayed the assembled dignitaries of the nation, and without faltering charge them with the commission of the most outrageous crime, the crucifixion of the Son of God, the Saviour of a lost world.>
<4:11 The stone; Ps 118:22; Isa 28:16; Mt 21:42.>
<4:12 Salvation; Ac 10:43; 1Ti 2:5,6.>
<4:13 Unlearned and ignorant; men in private life who had not been instructed in the schools, or by the doctors of the law. Took knowledge of them, that they had been with Jesus; recognized them as persons whom they had seen among the followers of Jesus.>
<4:14 Nothing against it; they could not deny the reality or greatness of the miracle, or the truth of what Peter had said.>
<4:16 Wicked men continue to oppose the cause of Christ without any good reason, and when they can with truth say nothing against it.>
<4:17 It; the knowledge of the miracle and its author.>
<4:19 Judge ye; God required them to speak; the council forbade them. Which ought they to obey? Human laws which require men to disobey God are of no obligation, and should not be obeyed.>
<4:20 We cannot; they could not obey their rulers, and yet do right; neither can any man, when rulers command what God forbids.>
<4:21 Because of the people; should they punish the apostles, they feared that the people would rise, and give them trouble.>
<4:23 Their own company; the company of believers.>
<4:24 In seasons of trial the friends of God unbosom themselves to him, and find him to be a very present and all-sufficient helper; able to do exceeding abundantly above all that they ask or think, so that they can add their testimony to that of ten thousand thousand, "Blessed are all they that put their trust in him." Ps 2:12.>
<4:25 The heathen rage; Ps 2:1,2.>
<4:27 Hast anointed; set apart, and consecrated to be the Saviour of men. Herod, and Pontius Pilate; Lu 23:1-12.>
<4:28 Determined before; Ac 2:23; 3:18. The enemies of God, in all their efforts to obstruct the progress of his cause, are doing only what he, for the wisest and best reasons, determined to suffer them to do, and what he will overrule for the highest good of his people. Ro 8:28.>
<4:30 Stretching forth thy hand; exerting thy power.>
<4:32 Had all things common; see note to chap Ac 2:44.>
<4:33 Great grace; much favor and assistance from God were granted them.>
<4:34 Neither was there any among them that lacked; that lacked a supply of their wants, though many were far from home, and had not with them the means of support.>
<4:35 As he had need; for the supply of his present necessities. Union to Christ by believing in him, produces union among his people. and leads them to delight in doing good, as they have opportunity, to all, and especially to those who are of the household of faith. Ga 6:10.>
<4:36 Cyprus; an island in the north-eastern part of the Mediterranean. Ac 13:4; 15:39.>
<5:2 His wife--being privy to it; secretly knowing and concurring in the design of keeping back a part of the price for which they sold the land, while they professed to bring the whole.>
<5:3 The Holy Ghost; who was present with the apostles, and under whose direction they acted. Lying is a great sin. Those who practise it are influenced by Satan, and imitate his example.>
<5:4 While it remained; before he sold it. In thine own power; at his disposal. It was optional with him to give it to the apostles or not, as he chose. There was no constraint or compulsion in this matter, but it was entirely voluntary. Unto God; by lying to the Holy Ghost, who was with the apostles, they lied unto God; for he was God.>
<5:5 Give up the ghost; instantly died.>
<5:6 Wound him up; in cloths, as was then the custom for burial.>
<5:8 Answered; said. So much; the sum that Ananias had brought, as if it were the whole price of the land.>
<5:9 Tempt the Spirit of the Lord; by trying to deceive him. Carry thee out; bury thee, as they did thy husband.>
<5:10 Sometimes known, deliberate wickedness is visited upon the sinner by immediate divine judgments; and no person, when he commits known iniquity, can be sure that it will not be so visited on him; and if it should not be, no impenitent transgressor will ultimately escape. Pr 11:21; 19:5.>
<5:13 Of the rest; of those outside of the Christian body. Durst no man join himself; that is, according to some, in a hypocritical way, like Ananias and Sapphira. But perhaps the meaning is, that the multitude without the church were so overwrought by the death of these two persons, that, for the present, no one dared openly to connect himself with the Christian body under the charge of the apostles. This may have been no permanent hinderance to the increase of the church, but only a salutary check promotive of its purity, while the work of conversion went on among the people. Magnified them; regarded and spoke of them with great respect.>
<5:14 The more; greater numbers in consequence received the gospel, being persuaded that it was from God. The Holy Spirit may make the destruction of some the occasion of the salvation of others. When it is seen that the wicked do not live out half their days, and that the wages of sin is death, many may be led to forsake sin and live. Ps 55:23; Ro 6:23.>
<5:15 Beds and couches; beds were used by the rich, and couches by the poor.>
<5:17 They that were with him; who agreed with him, especially the Sadducees, who denied the possibility of a resurrection. If Christ was indeed risen, as the apostles affirmed, it proved their doctrine false; and hence the bitterness of their opposition.>
<5:20 The words of this life; the way of eternal life through faith in Jesus Christ. Rulers sometimes forbid what God commands, and thus make it the duty of men to disobey them in order to obey him.>
<5:21 The senate of the children of Israel; men of age and influence, called elsewhere elders of the Jews, and the estate of the elders. Chap Ac 4:5; 22:5; 25:15.>
<5:24 The chief priests; these were the heads of the twenty-four courses into which the priests were divided. 1Ch 24.1-31; 2Ch 8:14; Lu 1:5. Whereunto this would grow; what would be the effects of it.>
<5:26 They; the officers. The people; those who favored the apostles.>
<5:28 This name; the name of Jesus. Bring this man's blood upon us; prove us guilty of murdering him. Men often shrink from taking the responsibility of their own actions, and are filled with indignation at the statement of the truth concerning them.>
<5:29 Obey God; he commanded them to preach; the rulers forbade it.>
<5:30 On a tree; the cross. Ga 3:13; 1Pe 2:24.>
<5:31 To give repentance; by sending down the Holy Spirit, convincing men of sin, and leading them to hate and forsake it.>
<5:32 These things; the resurrection of Christ, his ascension to heaven, and his giving repentance and pardon. The Holy Ghost; by his miraculous powers and his sanctifying effects.>
<5:33 Cut to the heart; enraged, filled with wrath. Chap Ac 7:54. When the exhibition of truth torments men, it is evident that they must experience a great change, or their torment will be eternal; for Jehovah is a God of truth, and the progress of his government will be developing and illustrating the truth for ever.>
<5:34 A doctor of the law; an interpreter and teacher of the divine law. Chap Ac 22:3.>
<5:36 Somebody; a person of eminence and distinction, a leader of the people.>
<5:37 Days of the taxing; the taxing of the Jews by the Roman government. Drew away much people; he contended that the taxing of Jews by a heathen government was unlawful, and many followed him.>
<5:38 This counsel or this work; the work in which the apostles were engaged.>
<5:40 They agreed; so far as not further to pursue measures for putting them to death, though they scourged them, and again commanded them not to preach.>
<5:41 Counted worthy; to be treated somewhat as Christ was, on account of their attachment to him and zeal in his cause. This was an evidence of their likeness and devotion to him. Like him, they despised the shame, and rejoiced in the prospect of coming glory. It is not in the power of the wicked to destroy the happiness of the righteous. They can destroy themselves, but cannot prevent a single individual who loves and trusts in Jesus Christ, from for ever shouting with the heavenly host, "Alleluia; for the Lord God omnipotent reigneth." Re 19:6.>
<6:1 Grecians; in the original Hellenists, that is, Jews, whether by descent or conversion to the Jewish religion, who used the Greek language. Daily ministration; daily distribution to the poor.>
<6:2 The twelve; the twelve apostles, Matthias having been elected after the death of Judas. Not reason; not reasonable or proper. It is not proper that ministers of the gospel should be drawn off from their appropriate work to attend to secular concerns. The more exclusively they are devoted to the preaching of the gospel and the discharge of religious duties, the more they will promote their own best interests, and those of their fellowmen. 1Ti 4:15.>
<6:3 Honest report; men of integrity and good reputation.>
<6:4 Prayer--ministry of the word; the appropriate duties of their office.>
<6:5 Multitude; the multitude of believers who were collected on the occasion. Proselyte of Antioch; a Gentile of that city, who had embraced the Jewish, and afterwards the Christian religion.>
<6:6 Laid their hands on them; in token of seeking for them the divine blessing, and consecrating them to their work.>
<6:7 The word of God increased; was preached, and embraced by greater numbers. When ministers of the gospel are wholly and earnestly devoted to their appropriate duties, and are assisted by brethren in the church who are pious, wise, able, and active in doing good, religion will generally prosper; and not only many of the common people, but of the educated, intelligent, and influential, may be expected to embrace it.>
<6:9 Libertines; libertines were properly persons, or the children of persons who had been enslaved, and were afterwards made free. In the present case Jewish libertines are meant, of whom there were great numbers, the descendants of those who had been carried as captives to Rome, and afterwards set free. The various classes of persons mentioned had each in Jerusalem a synagogue or place of worship. Mt 6:5. Disputing with Stephen; about the truth of what he declared.>
<6:10 Not able; he, being assisted by the Holy Ghost, was superior to them, and they were not able to answer his arguments. Mt 10:19,20; Lu 21:15. No array of numbers, learning, or talents, can fairly meet or refute the arguments which prove the Christian religion to be from God. Its truth is demonstrated by evidence which, if it be rightly apprehended, and the heart is sincere, will carry universal conviction. No one can reject it without showing that he is either ignorant or wicked.>
<6:11 Suborned men; got them to testify falsely.>
<6:12 Men who reject the Christian religion, and have power, are apt to oppose those who embrace it, especially if they are zealous and successful in its propagation. They sometimes contend that the interests of the state require this; and ecclesiastics, clothed with secular authority, and destitute of the spirit of Christ, and often among the most fierce and malignant of persecutors.>
<6:14 The customs; the Jewish ceremonies.>
<6:15 The face of an angel; benignant, calm, dignified, and resplendent.>
<7:1 So; as his accusers had said.>
<7:2 Mesopotamia; this word means between two rivers; it was the country which lay between the Tigris and the Euphrates. In this region was Ur of the Chaldees, where lived Terah, the father of Abraham. Charran; called in the Old Testament Haran. Ge 11:31. This was also in Mesopotamia. Opposers of religion who make inquiries about it, should be treated with courtesy and kindness. In answering their questions, we should endeavor to enlighten their minds with regard to Jesus Christ, that we may lead them to believe in him. 1Ti 2:25.>
<7:5 He; God. Promised that he would give it to him; Ge 12:7; 13:14-18.>
<7:6 Sojourn in a strange land; Egypt. Ge 15:13-16.>
<7:14 Threescore and fifteen; seventy-five. The number stated in Genesis as coming with Jacob into Egypt is sixty-six. Jacob, with Joseph and his two sons who were already there, and the five grandsons of Joseph mentioned in 1Ch 7:14-23, are supposed to make the seventy-five. See note to Ge 46:27.>
<7:16 Were carried; the fathers were carried: Joseph and others were carried and buried in Sychem, called in the Old Testament Shechem, which was near to Samaria, in a piece of ground bought by Jacob of the sons of Emmor, called in the Old Testament Hamor. Ge 33:18,19; Jos 24:32. Jacob was buried in the field of Machpelah, which was purchased by Abraham of the sons of Heth. Ge 23:4-20; 49:29,30; 50:13. In what way the name of Abraham became connected with the purchase at Sychem is not known.>
<7:17 The time of the promise; the time for its fulfilment Ge 12:7; Ge 15:14-16; 22:17. The people grew; Ex 1:7-9. God's promises and threatenings all have a set time for their accomplishment; and when that time approaches, his providences will be so ordered as to secure their exact and perfect fulfilment. Hab 2:3; Mt 24:35.>
<7:27 Persons who are most evidently in the wrong are generally the most forward and earnest in opposing those who would set them right; and the best endeavors to persuade men to live in peace are often met with insolence, reproach, and contempt.>
<7:30 An angel of the Lord; who was Jehovah himself. He is spoken of in Ex 23:20,21, as one in whom is God's name, and who has power to pardon sin. He is, therefore, with reason supposed to be the same as "the Word" that "was in the beginning with God," and "was God.">
<7:35 In obscurity and retirement God often prepares men for the discharge of great and momentous public duties; and when the proper time arrives, he so orders events that they cannot, without rebelling against him, refuse to leave their retirement and enter upon the responsibilities and toils of public life.>
<7:38 He; Moses. Ex 19:3-25.>
<7:42 Gave them up; Ps 81:12. It is written; Am 5:25-27. Have ye offered to me; that is, to me alone. The answer is, No; ye mingled with my worship that of your idols. See note to Am 5:26.>
<7:43 Moloch--Remphan; idol gods. Stephen does not quote the exact words of the prophet, but following the Septuagint, gives the sense, as in verses Ac 7:48,49, and in other places.>
<7:44 The tabernacle; a sacred tent, or movable structure, used before the erection of the temple for religious services, at the door of which God made communications to the people. Ex 29:42,43; 25:8,9,40; Heb 8:5.>
<7:45 Jesus; Joshua. Jesus in Greek is the same as Joshua in Hebrew. Jos 1:1,2; Heb 4:8; Jos 11:23.>
<7:46 Tabernacle; this is a different word from that which is translated tabernacle in verse Ac 7:44, and here means a permanent structure, or fixed habitation. 2Sa 7:2-7; 1Ch 22:7-19.>
<7:49 Men may so idolize a time, a place, or an outward form of worship, as entirely to unfit them to worship him who is a Spirit "in spirit and in truth," and lead them violently to oppose and bitterly to persecute those who maintain that none but spiritual worshippers can at any time or in any place or form be accepted of him.>
<7:51 Stiff-necked; unwilling to bow to the authority or regard the voice of God. Resist the Holy Ghost; by refusing to follow his directions. As your fathers did, so do ye; he comes now to the application of his narrative, in which he has made prominent the disobedience of the people to Moses, and their attachment to idolatry in the wilderness, verses Ac 7:35,39-43; as much as to say, As your fathers treated Moses, so do you treat that Prophet like unto Moses, of whom he prophesied. The manner in which the Holy Ghost and those who have been under his influence have in all ages been treated, shows a deep-rooted enmity in the human heart against God. Hence the necessity taught by Jesus Christ in Joh 3:3, and the duty inculcated in Job 22:21, Eze 18:31; Ac 3:19; 2Co 5:20.>
<7:52 The Just One; Jesus Christ. Ac 3:14; 2Ch 36:16; Mt 26:66; 27:20-26; Joh 19:12-18.>
<7:53 The disposition; the ministration. De 33:2; Ga 3:19; Heb 2:2.>
<7:54 Cut to the heart; exceedingly enraged.>
<7:57 Stopped their ears; as if unwilling to hear words which they affected to regard as blasphemous.>
<7:58 The witnesses laid down their clothes; the false witnesses mentioned in chap Ac 6:13. They, according to the law, were to cast the first stones; and they laid aside their upper garments, that they might do this with greater effect. Le 24:14-16; De 17:7. Saul; this is the first mention of him who was afterwards the great apostle of the Gentiles. Men may be very scrupulous in some respects in the observance of forms of law, while in others they grossly violate both its letter and its spirit; and forms of law, as well as professions of religion, may be prostituted to the commission of flagrant injustice and atrocious crimes.>
<7:59 Calling upon--and saying, Lord Jesus; the word God is not in the original, as its being printed in italics shows. The prayer was offered to Jesus Christ, and it was the custom of the Christians in the days of the apostles to pray to him. Lu 23:42,43; Ac 9:21; 22:16; 1Co 1:2. The Holy Spirit leads those who are under his influence to pray to Jesus Christ, and ask of him the richest blessings, not only for themselves and their friends, but also for their fellow-men. In thus honoring him as they honor the Father, they glorify him on earth, and prepare to dwell with him in heaven.>
<7:60 This sin; the sin of murdering him on account of his friendship to Christ. Mt 5:44; Lu 23:34. Fell asleep; peacefully died, and was received to glory by Jesus Christ, to whom he prayed.>
<8:1 Was consenting; concurred in putting Stephen to death.>
<8:3 Made havoc; furiously assaulted and laid waste. Hailing; or hauling, dragging by force.>
<8:4 The efforts of wicked men to stop the progress of the gospel are often overruled for its advancement; and yet their wickedness is as great, and without repentance their punishment will be as dreadful, as if their actions had not been overruled for good.>
<8:5 Philip; one of the seven first deacons. Chap Ac 6:5.>
<8:9 Sorcery; deceptive arts, pretending to foretell future events. Bewitched; amazed them, filled them with astonishment. It is the word which in verse Ac 8:13 is translated wondered.>
<8:10 The great power of God; endowed with supernatural power.>
<8:12 When the gospel is faithfully preached, and accompanied by the influences of the Holy Spirit, men of all classes embrace it. They may have followed artful deceivers, and been sunk in spiritual darkness and death; yet when they believe and follow Him who is the light of the world, they forsake their blind guides, and walk no longer in darkness, but have the light of life.>
<8:13 Believed; the miracles which he witnessed seem to have convinced him of the reality of the divine power that accompanied the gospel, though he had very false ideas respecting it, and soon showed that he did not love God nor his truth.>
<8:15 The Holy Ghost; his miraculous influences, so that they might work miracles.>
<8:19 This power; he supposed that, should he receive it, he might enrich or exalt himself.>
<8:20 Thy money perish; a strong expression of abhorrence of his selfishness, criminality, and danger.>
<8:21 In this matter; in the blessings of the gospel, and the work in which Peter and John were engaged.>
<8:22 Men may greatly displease God in their thoughts, as well as in their words and actions. Each one should therefore keep his heart with all diligence, and pray, "Cleanse thou me from secret faults," as well as, "Keep me back from presumptuous sins;" that not only the words of his mouth, but the meditations of his heart may be acceptable in the sight of God our strength and Redeemer.>
<8:23 The gall of bitterness--the bond of iniquity; in the most loathsome bondage to sin. Men whose great object is self-exaltation have very erroneous views of the nature of true religion, and of the character of those who possess it; when many profess religion, such men sometimes unite with them, but afterwards by their conduct show that their professions were false, and that they have no love to the gospel, and no part in its blessings.>
<8:24 None of these things; the punishments threatened.>
<8:25 Many villages; through which they passed on their way to Jerusalem.>
<8:26 Gaza; a city about sixty miles south-west of Jerusalem, towards Egypt. Which is desert; these words are supposed by some to refer to the city as having been recently laid waste. But they more probably point out the road which Philip was to take, as that one, of two or more, which ran through a desert region.>
<8:27 Ethiopia; a country south of Egypt. To worship; this showed that he was either a Jew or a proselyte to the Jewish religion.>
<8:29 The Spirit; the Holy Spirit.>
<8:32 The scripture which he read; Isa 53:7,8.>
<8:33 In his humiliation; this quotation is from the Septuagint, or Greek version of the Old Testament, which was probably the one that he used. Judgment; justice was denied him, and he was unrighteously put to death. Yet he lives in glory, and innumerable multitudes will eternally adore him. Who shall declare his generation? see note to Isa 53:8.>
<8:35 Preached unto him Jesus; showed him that it was Jesus of whom the prophet spoke, and pointed out the way of salvation through him. When persons wish to know the will of God for the purpose of doing it, and in order to this are in the habit of searching the Scriptures, God, in his providence, will enlighten them; and the knowledge of Christ which he communicates to a single individual may, in its influence, be felt through kingdoms and to future ages.>
<8:37 With all thy heart; if you are heartily convinced that Jesus is the Messiah, and trust in him for salvation. I believe; I receive him as my Saviour.>
<8:40 Azotus; a city called in the Old Testament Ashdod, about thirty miles north of Gaza. 1Sa 5:1. Cesarea; a city on the Mediterranean, about sixty miles north-west of Jerusalem.>
<9:1 High-priest; he was president of the Sanhedrim or great Jewish council, and signed letters in their name and with their authority. Men of great talents, superior religious advantages, and extensive learning, may be so opposed to Jesus Christ as to wish to destroy all, both men and women, who believe in him. Ac 22:4,5; 26:9-11.>
<9:2 Damascus; a city of Syria, about a hundred and twenty miles north-east of Jerusalem. To the synagogues; to the elders or rulers of the synagogues in Damascus, giving him full power to act in the matter, and requiring their cooperation. Any of this way; the way of the Lord--any of his disciples. Might bring them bound unto Jerusalem; to be tried by the Jewish council and punished. This would require the cooperation not only of the synagogues in Damascus, but of the civil authorities also, which was often granted upon solicitation.>
<9:4 Wicked men in persecuting Christians are persecuting Christ. But he can reveal himself to them in such a manner as to fill them with deep consternation, and lead them without reserve to give up themselves and all their interests to his guidance and disposal. Ac 22:6,7; 26:12-20.>
<9:5 The Lord; the Lord Jesus Christ. Ver Ac 9:17,27; 1Co 9:1; 15:8. Hard; painful and useless. Pricks; sharp irons or points at the end of a staff or goad with which they drove cattle. It was a proverbial expression, to denote that a person's efforts against others would only injure himself. In persecuting Christians, men injure not only others, but also themselves; and the further they proceed, the more they find that the way of transgressors is hard.>
<9:7 Stood speechless, hearing a voice, but seeing no man; by comparing the present account with chap Ac 22:9, we learn that Saul's companions heard a voice and saw a light; but that they neither understood what was said to Saul, nor saw the person of Jesus.>
<9:8 He saw no man; being blinded by the glory of that light. Chap Ac 22:11.>
<9:9 Three days; this meant till the third day, or one whole day and a part of two others. Mt 12:40; Mt 16:21.>
<9:11 Tarsus; the capital of Cilicia, a province of Asia Minor. He prayeth; this indicated the change he had experienced. When men offer from the heart supplications to God, he is ready to visit them in mercy; and to those who understand his character and ways, it is always encouraging to learn concerning any one, that he prays.>
<9:12 A vision; designed to prepare Saul for his interview with Ananias.>
<9:14 Call on thy name; the name of the Lord Jesus. Christians in the days of the apostles were distinguished by this. 1Co 1:2; 2Ti 2:22.>
<9:15 A chosen vessel--to bear my name; I have selected him to make me and my salvation known to Jews and Gentiles. Ga 1:15-17.>
<9:16 He must suffer; 2Co 11:23-28.>
<9:17 That appeared unto thee in the way; that Saul saw the person of the Lord Jesus is implied in what is said of his companions, that they saw no man; and is affirmed by Ananias here and in chap. Ac 22:14. Paul, moreover, mentions it as a necessary qualification of an apostle. 1Co 9:1.>
<9:20 The Son of God; the Messiah foretold by the prophets. No man knows, when he starts on a journey, what will take place before he returns. He may be called to pass through scenes and to discharge duties totally different from what he expected. A man's heart deviseth his way, but the Lord directeth his steps. Pr 16:9. Persecutors may become preachers, and those who went out to murder, return to save.>
<9:21 This name; the name of the Lord Jesus.>
<9:22 Very Christ; the true Messiah.>
<9:23 Many days; a part of this time he spent in Arabia, a country south and east of Judea. Ga 1:15-18. Wicked men are often disposed to use violence against those who differ from them in religion. They will favor a man who is openly hostile to Christ; but if he believes on him, they are ready to put him to death.>
<9:26 Assayed; attempted.>
<9:27 Barnabas; chap Ac 4:36.>
<9:29 Grecians; Jews and proselytes who spoke the Greek language. Chap Ac 6:1.>
<9:30 Cesarea; chap Ac 8:40. Tarsus; his native city. Verse Ac 9:11.>
<9:31 Judea and Galilee and Samaria; the three divisions into which Palestine, or the Holy Land, was divided. Edified; strengthened and advanced in knowledge and piety.>
<9:32 Lydda; a town in Judea a few miles south-east of Joppa.>
<9:34 Maketh thee whole; cureth thee. Peter was careful to show that this miracle was wrought not by his power, but by the power of Christ.>
<9:35 Saron; a fruitful region between Joppa and mount Carmel, called in the Old Testament Sharon. 1Ch 5:16; 27:29; Isa 33:9; 35:2; 65:10.>
<9:36 Joppa; a town on the Mediterranean about forty-five miles north-west of Jerusalem. Tabitha; a Syriac word, meaning the same as the Greek word Dorcas. Both were applied originally to the gazelle, an animal of great beauty.>
<9:39 Persons who are very benevolent and useful in life, will be affectionately remembered, and greatly lamented in death; and all their works performed from love to God and to men, will meet a gracious and abundant reward.>
<9:42 Believed in the Lord; the Lord Jesus, whom Peter preached, and by whose power he wrought this miracle.>
<10:1 Cesarea; chap Ac 8:40. Centurion; captain of a hundred men. He was a Gentile and uncircumcised, chap, Ac 11:3, but a sincere worshipper of God. Italian band; a band of Roman soldiers from Italy.>
<10:3 Ninth hour; three o'clock in the afternoon; one of the hours of daily prayer.>
<10:4 For a memorial; remembered, noticed with approbation. A beneficent disposition is greatly increased by the habit of daily prayer. Both united, and springing from love to God and to men, form an offering which, through grace, is peculiarly acceptable to God.>
<10:5 Joppa; chap Ac 9:36.>
<10:6 What thou oughtest to do; Cornelius had true faith in God according to the present measure of his light, and his prayers were answered in the reception of fuller light. Mt 13:12.>
<10:9 House-top; the tops of the houses were flat, and pious people often resorted to them for meditation and prayer. Sixth hour; twelve o'clock; with many, one of the stated hours of prayer. Ps 55:17; Da 6:10. When God is about to call his people to the discharge of special duties, he often in a special manner prepared them; and though at the time they may not see the reasons, or understand the meaning of his dealings, the subsequent dispensations of his providence may clearly reveal them.>
<10:10 Trance; a state in which he became insensible to external objects, and absorbed in what was presented to his mind.>
<10:11 Knit; fastened together. Let down to the earth; from heaven, to signify that the offer is made to the apostle by God himself.>
<10:12 All manner of four-footed beasts--and fowls of the air; clean and unclean alike.>
<10:13 Kill, and eat; he rightly understands the words to mean, kill and eat any one of the animals contained in the collection, without distinction of clean and unclean.>
<10:14 Common; not set apart as pure. The reference is, of course, to the Jewish distinction of clean and unclean animals.>
<10:15 Hath cleansed; declared no longer unclean for food. Under this symbol of the abolition of the Jewish distinction of clean and unclean animals, is signified the breaking down of "the middle wall of partition" between Jews and Gentiles, and the admission of the latter to common privileges with Israelites.>
<10:16 Thrice; three times, in order to make a deeper impression on Peter's mind, and prepare him to preach the gospel to Jews and Gentiles alike.>
<10:19 The Spirit; the Holy Spirit.>
<10:20 Doubting nothing; not doubting the lawfulness of going to Cornelius, though he was a Gentile.>
<10:22 Words; words of instruction.>
<10:24 Kinsmen; relatives.>
<10:25 Worshipped; prostrated himself before him.>
<10:26 A man; only a man. As Peter was only man, he would not receive any homage or respect, except what was proper for other men. Those who claim more, on account, as they say, of being his successors, manifest a spirit totally different from his.>
<10:28 Unlawful; viewed by the Jews as improper. God hath showed me; in the vision which he had seen.>
<10:29 Gainsaying; making objection.>
<10:33 When people are assembled with a real desire to hear from ministers of the gospel all which God has commanded them to preach, it is an evidence that he is about abundantly to bless them. While the minister is preaching, the Holy Ghost often so influences their minds as to lead them to glorify God.>
<10:34 No respecter of persons; he accepts and blesses all pious persons, whatever their nation or condition.>
<10:36 The word; the gospel or way of salvation through Jesus Christ.>
<10:37 That word--ye know; they had some general knowledge of the life and works of Christ.>
<10:38 Anointed Jesus; set him apart, and furnished him to be the Saviour of men.>
<10:39 A tree; the cross.>
<10:42 Quick; the living.>
<10:44 Fell on all them; endowed them with miraculous powers, and enabled them to speak in languages which they had never learned. Mt 3:11; chap Ac 2:2-4.>
<10:45 They of the circumcision; the brethren that accompanied him from Joppa, verse Ac 10:23.>
<10:48 In the name of the Lord, in public acknowledgment of their receiving him as their Saviour, and becoming his disciples.>
<11:1 The Gentiles; those to whom Peter preached at the house of Cornelius.>
<11:2 They--of the circumcision; the believing Jews. Contended with him; found fault with him for associating with, and preaching the gospel to Gentiles, called, in verse Ac 11:3, the uncircumcised. They had not yet apprehended the great truth communicated to Peter in the vision at Joppa, that the gospel of Christ knows no distinction between Jews and Gentiles. The church at Jerusalem claimed and exercised the right of private judgment in matters of religion. They had no idea of being satisfied with the conduct of Peter, unless he would give them good reasons for it; and he had no idea that they ought to be satisfied in any other way. He therefore candidly and fully stated the reasons which had satisfied his own mind, and these through the divine blessing, satisfied theirs. Hence, it is evident that he did not claim, and that they did not believe him to be pope, or to possess those prerogatives which have been assumed by his pretended successors.>
<11:12 Six brethren; the believing Jews who went with Peter from Cesarea to Joppa. Chap Ac 10:23,45.>
<11:14 All thy house; all his family.>
<11:16 The Lord; the Lord Jesus, chap Ac 1:5.>
<11:17 Withstand God; resist the clear indications of his will.>
<11:18 They held their peace; being convinced that Peter did right. Repentance unto life; that which, through the grace of God, secures eternal life. Though it is the duty of men to repent, that they may receive forgiveness of sins, yet all are so wicked that none will repent unless led to do it by the Holy Spirit. In this sense repentance, when exercised, is the gift of God.>
<11:19 Phenice; or Phoenicia, a province on the coast north of Palestine, the chief cities of which were Tyre and Sidon. Cyprus; an island in the north-east part of the Mediterranean sea. Antioch; a city of Syria, about three hundred miles north of Jerusalem, on the river Orontes, not far from the Mediterranean.>
<11:20 Cyrene; a province and city in the north part of Africa. Grecians; Jews speaking the Greek language. But another reading of the original is Greeks; that is, Gentile Greeks. This latter is to be preferred.>
<11:21 The hand of the Lord was with them; he accompanied their preaching with divine power. When the Lord accompanies the preaching of the gospel with his divine power, multitudes will believe it and turn from their sins. Hence, his presence and power should always be sought by preachers and all who desire the success of the gospel.>
<11:22 Sent forth Barnabas; for the purpose of assisting the brethren in preaching the gospel.>
<11:23 The grace of God; the great success God had given the preaching of the gospel in the conversion of sinners. With purpose of heart; that with a steady, heartfelt purpose they would continue to obey Christ.>
<11:24 Though it is the power of God which makes the gospel the means of salvation, yet goodness, faith, and fidelity are no less needful in ministers, than if they were to be the sole cause of their success.>
<11:25 Tarsus; Paul's native city. Chap Ac 9:11.>
<11:26 Assembled themselves; for public worship. With the church; the disciples of Christ.>
<11:27 Prophets; a class of inspired teachers in the primitive church. 1Co 12:28; 14.1-40; Eph 4:11,12. They unfolded the doctrines of the gospel under the illumination of the Holy Ghost, and sometimes foretold future events. Compare chap Ac 21:10,11.>
<11:28 Signified by the Spirit; made known, under the influence of the Holy Spirit. Dearth; famine. Claudius Caesar; he was the fifth Roman emperor, and reigned from A.D. 41 to 54. This famine took place as predicted, and is particularly noticed by Josephus, the Jewish historian. Antiq, chap. 2 sec 5.>
<11:29 An experimental reception of the gospel produces a benevolent disposition, a desire to do good to all, especially to the friends of Christ. The proper measure of contributions and efforts is the ability which God gives.>
<11:30 Elders; leading men in the churches.>
<12:1 Herod; Herod Agrippa, grandson of Herod the Great, mentioned in Mt 2:1. Vex; trouble, persecute.>
<12:2 James; one of the sons of Zebedee, Mt 4:21, called James the greater, or senior, to distinguish him from James the less, or younger, who was the son of Alpheus. Mt 10:3; Mr 15:40. No degree of piety or usefulness can always save Christians from persecution, or from sudden and violent death. Yet the wicked, while they thus seek to destroy the people of God, are often made instrumental in delivering them from all trouble, and putting them into immediate possession of the joys of heaven.>
<12:3 The days of unleavened bread; the passover. Ex 12:12-17; Lu 22:1.>
<12:4 Four quaternions; four companies of four soldiers each; making, in all, sixteen. One company guarded him three hours, and was then relieved by another. Easter; the passover. Easter is supposed to have been originally the name of a heathen feast, which occurred in the month of April. It was afterwards applied to the Jewish feast of the passover, which occurred about the same time. Tyndal, in his translation of the Bible into English in 1526, used this word instead of passover, and our English translators in 1611 retained it in their version. But there was no Christian feast called Easter in the days of Peter. And the word Pascha which is here translated Easter, means passover, and should have been so translated. Bring him forth; for trial and condemnation.>
<12:5 In all seasons of trial the people of God have an unfailing support. To him they may apply, with full assurance that he deeply sympathizes in their trials, and in the best time and way will grant them all needed aid.>
<12:6 Would have brought him; was about to bring him. Same night; the night before the day of his intended execution. Between two soldiers, bound with two chains; each wrist was chained, after the Roman manner, to the wrist of the adjacent soldier.>
<12:7 His chains fell; in a miraculous way.>
<12:8 Gird thyself; it was then customary to put a girdle around the body when about to walk. Sandals; these covered the soles of the feet, and were fastened by strings or straps. Thy garment; the mantle, or outer garment.>
<12:9 Wist not; knew not. That it was true; that it was a real event, as distinguished from a vision.>
<12:10 Ward; this word generally means a prison, but here it means the first and second guard, who seem to have been prevented in a supernatural way from seeing him. Iron gate; which led out of the prison to the city. Of his own accord; of itself, without human aid.>
<12:11 Was come to himself; became conscious that what had happened was a reality and no vision.>
<12:12 John--Mark; the writer of "The Gospel according to Mark," and the companion of Paul and Barnabas. Verse Ac 12:25.>
<12:15 Art mad; deranged, or bereft of reason. His angel; his guardian angel, who they thought had attended him, and come in his form, to make known something concerning him. It is sometimes difficult for Christians to believe the answers of their own prayers, though God has said that he is more ready to give blessings to those who ask him, than earthly parents are to give food to their children. Yet when he actually gives them, they are so speedy and abundant, that his people are astonished, and tempted to ascribe his mercies to almost any thing, rather than his gracious interposition in answer to their prayers.>
<12:17 Unto James; not James the son of Zebedee, who had been slain. Verse Ac 12:2. He is the same James that is mentioned in chap Ac 15:13; Ac 21:18; and is generally regarded to have been James the less, the son of Alpheus. In Galatians, Paul names among the apostles seen by him on his first visit to Jerusalem, "James the Lord's brother." Chap Ga 1:19. Afterwards he names, chapter Ga 2:9, "James, Cephas, and John," as "pillars" of the church in Jerusalem. Whether one and the same person is mentioned in these two passages, is a question about which learned men are not agreed. Went into another place; to avoid the rage of Herod.>
<12:19 The keepers; those to whose care Peter had been committed. Cesarea; chap Ac 8:40.>
<12:20 Tyre and Sidon; two cities of Phoenicia, on the Mediterranean, north of Cesarea. The king's chamberlain; the officer who had the care of his bedchamber. Was nourished; supplied with grain and other provisions.>
<12:22 Gave a shout; flattered him with boisterous applause, as if he were more than human. Noisy flattering applause of public speakers is adapted to injure them. It tends to feed their pride, lead them to forget their dependence on God, and prevent them from giving glory to him.>
<12:23 Gave not God the glory; he did not rebuke their impious flattery, but was glad to be called a god, and receive divine honors. Jehovah is a jealous God. Those who claim or consent to receive honors due only to him, or to assume any of his prerogatives, he views with peculiar abhorrence. Yet Jesus Christ received divine honors, and pronounces those blessed who bestow them. In him the Father is well pleased, and he commands all the angels in heaven to worship him. Of course he must be God. Joh 1:1; Heb 1:8; 1Joh 5:20.>
<12:24 Grew and multiplied; the gospel was more successful, and the number of believers greatly increased.>
<12:25 Barnabas and Saul returned from Jerusalem; they returned to Antioch. Their ministry; the service for which they were sent. Chap Ac 11:30.>
<13:1 Cyrene; chapter Ac 11:20. Herod; not the Herod spoken of in the preceding chapter, but his uncle, Herod Antipas, who is mentioned in Lu 3:1,19.>
<13:2 Ministered to the Lord; were engaged in divine worship. Separate; set apart. The work; the work of missionaries to the places afterwards mentioned. True ministers of Christ are prepared for, and called to their work by the Holy Ghost. This, however, does not supersede the necessity, or lessen the propriety, in order to their greatest usefulness, of their being set apart with prayer and the laying on of hands.>
<13:3 Laid their hands on them; the mode of setting them apart to their work.>
<13:4 Seleucia; a seaport at the mouth of the river Orontes, about fifteen miles from Antioch.>
<13:5 Salamis; a city in the south-east part of Cyprus. John to their minister; John whose surname was Mark, as their assistant.>
<13:6 Paphos; a city on the west side of Cyprus. Sorcerer; a magician, or fortune-teller; one who pretended to foretell future events. Bar-jesus; meaning, son of a man named Jesus, or Joshua.>
<13:7 Deputy; that is, the proconsul; the title borne by those governors of provinces that were appointed by the Roman senate. Prudent; intelligent, wise, candid.>
<13:8 Elymas; apparently an Arabic word meaning wise of learned, that is, in the arts of sorcery. Turn away the deputy; prevent his embracing the gospel. Teachers of falsehood and pretenders to superior power are always afraid of the faithful preaching of the gospel. So far as it is embraced and followed, their influence will be gone; they therefore misrepresent and oppose it, slander those who preach it, and in various ways seek to prevent men from receiving it.>
<13:9 Paul; his Hebrew name was Saul. This is the first time he is called Paul; but after this, he is always called by this name.>
<13:10 Child of the devil; like him in temper and conduct. Pervert; misrepresent, and turn men away from the truth.>
<13:11 The hand of the Lord; he will visit thee in judgment.>
<13:12 Doctrine of the Lord; not merely the doctrine, but the divine power accompanying it. Compare Mr 1:27.>
<13:13 Loosed; set sail, departed. Perga in Pamphylia; Pamphylia was a province in the south part of Asia Minor, and Perga was its capital. John departing from them; an act which Paul strongly disapproved. Chap Ac 15:38.>
<13:14 Antioch in Pisidia; so called, to distinguish it from Anitoch in Syria. Pisidia was a province north of Pamphylia, on the borders of which was Antioch.>
<13:15 Law and the prophets; portions of both which were read in the synagogue on the Sabbath.>
<13:16 Give audience; hearken. A correct and extensive knowledge of history, especially the history of the church, is of great importance to ministers of the gospel. To show what God has done in his providence, as well as what he has said in his word, is a means of communicating to men a knowledge of his character and will, and presenting the motives to love and obey him.>
<13:17 With a high arm; with great power, and in a wonderful manner.>
<13:18 Suffered he their manners; bore with their provocations and sustained their lives; or according to another reading, bore or fed them as a nurse a child, as it is in the margin.>
<13:19 Seven nations; Jos 3:10; Jos 11:8.>
<13:21 Cis; in Greek, is the same as Kish in Hebrew. 1Sa 9:1; 10:1.>
<13:22 Gave testimony; 1Sa 13:14; 16:1-12; Ps 89:20.>
<13:25 His course; course of service, his ministry. Not he; not the Messiah. Joh 1:20; Mt 3:11.>
<13:26 The word of this salvation; the offer of salvation through faith in Jesus Christ.>
<13:27 Because they knew him not; did not know him to be the Messiah. The voices of the prophets; they did not understand the true meaning of the prophecies. They have fulfilled them; by crucifying Christ, they did what the prophets had foretold. Men who do not understand the meaning of the Bible, and who malignantly persecute those who obey it, may nevertheless, in their opposition, be fulfilling its predictions, and thus adding to the evidences of its divine origin, and of the truth of its declarations.>
<13:28 No cause of death; no crime.>
<13:31 Them which came up; his apostles and others.>
<13:32 The promise which was made unto the fathers; of the Messiah and his salvation.>
<13:33 This day have I begotten thee; the resurrection of Christ from the dead was the great public manifestation of him as the Son of God; the act by which he was "declared to be the Son of God with power." Ro 1:4.>
<13:34 The sure mercies of David; the sure mercies promised to David, Isa 55:3; namely, that God would never remove his mercy from him, but that his throne should be established for ever. 2Sa 7:12-17. This promise is fulfilled in the resurrection of Jesus Christ the Son of David, and his exaltation to universal dominion.>
<13:35 Another psalm; Ps 16:10.>
<13:39 Could not be justified; Ro 3:20; Heb 9:8-28. By believing in Christ, men may be accepted and treated as righteous; but they are so wicked, that none will believe, unless God by his Spirit lead them to do it.>
<13:40 Which is spoken of in the prophets; Hab 1:5. The apostle follows the Greek version of the Seventy, which agrees for substance with the Hebrew.>
<13:41 A work; a work of desolating judgment; namely, the overthrow of the land by foreign enemies.>
<13:43 Religious proselytes; Gentiles who had embraced the Jewish religion. In the grace of God; in the profession and practice of the gospel.>
<13:46 Necessary; in order to obey the command and fulfill the appointment of God. Lu 24:47. Judge yourselves unworthy; show yourselves unfit longer to receive even the offer of salvation. We turn to the Gentiles; we devote ourselves to the work of preaching the gospel to the heathen.>
<13:47 Saying; Isa 49:6. Be for salvation; be a Saviour. Unto the ends of the earth; to all people.>
<13:48 Ordained to eternal life; Ro 8:28-30; 2Th 2:13; 1Pe 1:2. When any believe in Christ, it shows that they were from the beginning "chosen to salvation, through sanctification of the Spirit and belief of the truth;" the glory therefore of every thing good in them, and of all the good done or enjoyed by them, belongs to God.>
<13:50 Devout; devout in the observance of the Jewish law.>
<13:51 Shook off the dust; in token of abhorrence of their wickedness. Mr 6:11. Iconium; a city of Lycaonia, a province north-east of Pisidia.>
<14:1 In preaching, the manner, as well as the matter, is important; and it should be the earnest desire, the fervent prayer, and the diligent effort of every minister of the gospel, so to speak that multitudes of all classes shall, through the grace of God, be led to believe.>
<14:3 Gave testimony; proved the truth of what they taught, by enabling them in his name to work miracles.>
<14:4 When great numbers embrace the gospel, those who continue to reject it are often filled with wrath against those who preach it. The community is divided. A part join the friends, and a part the enemies of Christ, and great commotions follow. These results wicked men attribute to the gospel; but they spring from opposition to it and the opposers, not the faithful preachers of the gospel, are responsible for the evils thus occasioned.>
<14:6 Lystra and Derbe; cities of Lycaonia, a province of Asia Minor.>
<14:9 Faith to be healed; confidence in the power of Christ, by means of Paul, to heal him.>
<14:11 Speech of Lycaonia; the language of that province.>
<14:12 Jupiter; considered by the Greeks and Romans as the greatest of their gods. Mercurius; regarded as the god of eloquence.>
<14:13 Before their city; in front of their city was a temple dedicated to Jupiter. Garlands; wreaths of flowers, with which they decorated the victims to be offered in sacrifice.>
<14:14 Rent their clothes; in token of their abhorrence of such sacrifices.>
<14:15 Men of like passions; frail, sinful, dying men. These vanities; the worship of false and imaginary gods. Faithful ministers of the gospel will be careful to let none think of them as any thing more than men, of like infirmities with other men; and if they are instrumental of good, they will inculcate upon all that the glory belongs wholly to God.>
<14:16 All nations; the gentile world. Their own ways; ways of idolatry and wickedness, without a written revelation.>
<14:17 Witness; evidences of his existence, power, and goodness.>
<14:19 Persuaded the people; persuaded them that Paul and Barnabas were bad men, and deceivers.>
<14:22 Confirming the souls; instructing and establishing them in the faith and practice of the gospel.>
<14:23 Ordained them elders; set apart persons to take the care, oversight, and instruction of the churches. Churches need officers to take the oversight of them, to instruct them, and labor for their spiritual good; and it is proper that they should be set apart to this work, that the influences of the Holy Spirit should be sought for them, and they be particularly commended to his gracious guidance and care.>
<14:24 Pamphylia; on their way back towards Antioch in Syria, from which they went. Chap Ac 13:1.>
<14:25 Attalia; a seaport in Pamphylia from which they could sail to Antioch.>
<14:26 The work which they fulfilled; the missionary work to which they had been appointed, chap Ac 13:2,3.>
<14:27 The church; the disciples at Antioch. Opened the door of faith unto the Gentiles; prepared the way to preach to them the gospel, and led them to embrace it. Correct accounts of the manner in which God has delivered his people from trials, and crowned their labors with his blessing, are very useful. They lead Christians to put greater confidence in him, to pray more earnestly for spiritual blessings, and with increased fidelity use the means which are needful to obtain them.>
<15:1 Certain men; Jews who had become Christians. The brethren; the Christians at Antioch. The manner of Moses; as taught by him.>
<15:5 The Pharisees; though converted to Christianity, they still retained their attachment to many rites and ceremonies of the Jewish dispensation. The errors of men are not all at once removed by their conversion: they need further instruction, observation, and experience, as well as the continued teaching of the Holy Spirit. But if they are really born of God, and use proper means, he will enlighten them; and as they see their errors they will renounce them, and become more and more conformed in faith and practice to his revealed will.>
<15:7 By my mouth should hear the word of the gospel; Peter was the first who preached the gospel to the Gentiles, and thus opened the way for their admission into the Christian church. Ac 10:34-48.>
<15:8 Bare them witness; testified to their acceptance.>
<15:9 By faith; by believing in Christ.>
<15:10 Tempt ye God; by acting against the manifestations of his will. A yoke; the burdensome rites and ceremonies of the Jewish religion.>
<15:11 Even as they; Jews as well as Gentiles must be saved, not by works, or the observance of rites and ceremonies, but through grace, by faith in Jesus Christ.>
<15:13 James; see note to chap Ac 12:17.>
<15:14 Simeon; the Hebrew mode of spelling Simon, meaning Simon Peter. Verse Ac 15:7.>
<15:15 The prophets; they had foretold that the gospel should be preached to the Gentiles as well as Jews. Isa 2:2-4; 49:6; Am 9:11,12. The better Christians understand and obey the Bible, the more clearly they will see that they should receive and love one another as brethren, even as Christ receives and loves them.>
<15:16 I--will build again the tabernacle of David; here representing David's royal family, which, after the Babylonish captivity, fell into obscurity. It was rebuilt in the person of Christ, the son of David according to the flesh, and the true heir to David's throne. Lu 1:32,33. The apostle in this quotation follows the Greek translation of the Seventy.>
<15:17 Upon whom my name is called; who are called Jehovah's people.>
<15:18 Known unto God are all his works from the beginning of the world; as much as to say, The calling of the Gentiles into the church is in accordance with God's purpose from the beginning. Why then throw hinderances in their way?>
<15:19 Sentence; opinion, or judgment. That we trouble not them; by imposing upon them Jewish ceremonies. In the primitive church, Peter had no preeminence above James and the other apostles. He alone did not send out a letter to the brethren in Antioch, nor did James, or any, or all the apostles do it. They consulted with the elders and brethren, and sent out the epistle in the name of all united. This course seemed good not only to them, but also to the Holy Ghost. Verse Ac 15:28.>
<15:20 Pollutions of idols; from using meats offered in sacrifice to idols, or in any way conniving at idolatry. Fornication; a sin which was exceedingly common among the Gentiles, and against which they needed a special warning. Things strangled; animals killed by strangling without the shedding of their blood. The eating of blood was forbidden by the ancient law. Ge 9:4-6; Le 17:10-14. From the reading of the law in the synagogue on the Sabbath, verse Ac 15:21, this was known to the Jews; and should Christians use blood, it would unnecessarily prejudice the Jews against the gospel.>
<15:24 Troubled you with words, subverting your souls; compare the manner in which Paul speaks of these men in Ga 2:4. They had subverted their souls by inculcating error, and turning them from the truth. The law; the ritual law of Moses.>
<15:28 Seemed good to the Holy Ghost; the apostles, elders, and brethren, in their consultations on this subject, were under the influence of the Holy Spirit, and by him were led to a right result.>
<15:31 They rejoiced for the consolation; which the letter contained. A right understanding of the will of God and a disposition to follow it, give great joy to his people. It is a powerful means of extending his kingdom, increasing the number of his subjects, and hastening the time when all shall know him from the least to the greatest.>
<15:32 Prophets; chap Ac 11:27.>
<15:36 Every city; in which they had preached in their missionary tour. Chap Ac 13,14.>
<15:37 John--Mark; his mother was a sister of Barnabas. Col 4:10.>
<15:38 Who departed; chap Ac 13:13.>
<15:39 Cyprus; the native place of Barnabas. Chap Ac 4:36. Contentions among ministers and Christians, while they often show their weakness and wickedness, and are exceedingly injurious to the cause of Christ, are sometimes manifestly overruled to the furtherance of the gospel, and the wider and more rapid extension of his kingdom.>
<15:41 Confirming the churches; strengthening and establishing them in the faith and practice of the gospel.>
<16:1 Derbe and Lystra; chap Ac 14:6.>
<16:3 Circumcised him; this Paul did, not because it was needful to salvation, but to prevent the Jews from being prejudiced against Timothy as a preacher of the gospel. Christian wisdom will lead men to do, in some circumstances, what it will lead them to refuse to do under others. It is not enough that a thing is not forbidden: to be justified in doing it, we must also have reason to believe that it will be useful--that it is not only lawful, but also expedient. 1Co 6:12; 10:23.>
<16:4 The decrees; the decision to which the apostles and brethren at Jerusalem had come. Chap Ac 15:29.>
<16:6 Phrygia; this was the central and largest province in Asia Minor. Galatia; a province east of Phrygia. Asia; this word here, and in other places in the New Testament, refers to proconsular Asia, of which Ephesus was the capital.>
<16:7 Mysia; north-west of Phrygia. Assayed; attempted. Bithynia; a province east of Mysia, and north of Phrygia. The Spirit; the Holy Spirit.>
<16:8 Troas; a city near the site of ancient Troy, on the north part of the Aegean sea, which separates Asia Minor from Europe.>
<16:9 Macedonia; a country in the south-east of Europe.>
<16:10 We; from this it appears that Luke, the writer of this book, accompanied Paul. Compare chap Ac 20:5, etc. Assuredly gathering; being convinced.>
<16:11 Samothracia; an island in the north part of the Aegean sea. Neapolis; a seaport of Macedonia.>
<16:12 Philippi--a colony; inhabited by Roman citizens, and enjoying special privileges. This was the first introduction of the gospel into Europe.>
<16:14 Thyatira; a city of Lydia, a province in Asia Minor. Worshipped God; was a proselyte to the Jewish religion. Whose heart the Lord opened; inclined to believe what Paul preached. Whenever men believe on Jesus Christ, and are disposed to obey him, it is evidence that the Lord has renewed their hearts by his Spirit. To him, therefore, not to them, or those who have preached the gospel, belongs the glory.>
<16:16 To prayer; the place of prayer. Spirit of divination; an evil spirit, under the influence of which she professed to divine, that is to reveal things beyond the reach of human knowledge. Soothsaying; professing to foretell future events. Men are often more anxious to know their fortunes than their duty. They more earnestly seek, and more liberally pay for specious delusions and lying vanities, than for substantial realities and momentous truth.>
<16:18 Being grieved; at her debased and wretched condition, and the evil she might occasion. In the name of Jesus Christ; this showed that the author of this miracle was not Paul, but Christ.>
<16:19 Hope of their gains; the hope of making any more money in that wicked way. Under the influence of Satan, persons may pretend to foretell future events, and to exercise supernatural power; but when Christ delivers them from the snares of Satan, such pretensions will cease.>
<16:20 Many are greatly troubled when Satan is prevented from helping them to make money; and earnestly contend that the prosecution of their wicked employment is essential to the public good.>
<16:21 Customs--not lawful; a new religion contrary to the Roman law.>
<16:24 The inner prison; from which it would be most difficult to escape. Stocks; wooden frames in which their feet were fastened.>
<16:25 It is not in the power of the wicked to make the righteous unhappy, or prevent their rejoicing with exceeding joy.>
<16:26 Bands; chains, cords, or fetters by which they were confined.>
<16:27 Would have killed himself; to avoid the punishment of death, to which those were liable who allowed prisoners to escape. Compare chap Ac 12:19.>
<16:30 To be saved; from sin, and the wrath of a justly offended God. To a convicted sinner, the most important of all things is salvation from sin and hell. To obtain it, he must do something; and the business of a minister is, to show him what he must do, set before him the motives, and look to the Holy Ghost to lead him to do it.>
<16:31 Thy house; thy family.>
<16:35 Sergeants; the lictors who attended on the magistrates, and executed their commands.>
<16:37 Being Romans; having a right to the privileges of Roman citizens, whom it was unlawful thus to scourge and imprison. Let them come; this would be a public acknowledgment by the magistrates that they had done wrong. It is sometimes right and wise to claim the protection of civil law; to appeal from the judgment of inferior magistrates to higher tribunals, and let all know that the rights and privileges of citizens are not to be trampled on with impunity. Government is unfaithful to itself, to its subjects, and to God, if it fails to be a terror to evil-doers, and a praise and protection to those that do well.>
<16:39 Besought them; to overlook the injustice which had been done them, and to depart.>
<16:40 Comforted them; by recounting the goodness of God, and encouraging them to love and trust in him.>
<17:1 Amphipolis; the chief city of the first or eastern division of Macedonia. Thessalonica; the capital of the second division of Macedonia.>
<17:2 The scriptures; the Old Testament. The sermons of ministers should not be mere exhortations, or addresses to the passions and imaginations of men, but should contain sound argument and conclusive reasoning--not merely asserting, but proving the great doctrines and duties of the gospel, and enforcing them upon the consciences and hearts of men.>
<17:3 Must needs have suffered; the Scriptures foretold that the Messiah would suffer. His death was needful in order to their fulfilment, and to the salvation of men.>
<17:4 Consorted with Paul and Silas; joined them. Devout Greeks; gentile Greeks who had become acquainted with the true religion as revealed in the Old Testament, and were worshippers of Jehovah. The women also appear to have been of the same class. So also in verse Ac 17:12.>
<17:5 Moved with envy; at the success of Paul and Silas. Jason; a relation of Paul, at whose house he and Silas stayed. Ro 16:21. Rejecters of truth and violators of morality are apt to unite in opposing the gospel, and in doing this, the openly vicious of the basest sort can plead earnestly for the constitution and laws of the country, and express great fears lest these should be violated, and the public receive detriment.>
<17:7 Do contrary to the decrees of Caesar; are guilty of rebelling against him.>
<17:9 Taken security of Jason; exacted a bond of him, by which he was made responsible that no disturbance should be caused by the presence of Paul and Silas. In accordance with this they were immediately sent away.>
<17:10 Berea; a city of Macedonia south-west of Thessalonica.>
<17:11 More noble; more noble-minded--candid inquirers after truth. Whether those things were so; whether the things taught by Paul and Silas were true, in accordance with the Scriptures. Readiness of mind to hear the gospel, and daily to search the Scriptures for the purpose of understanding and obeying them, is evidence of true nobleness of spirit, and the means of increasing it, and of leading many to believe in Christ the Saviour of their souls.>
<17:13 The uneasiness which men feel when others receive the blessings which they themselves reject, is evidence of deep wickedness of heart. In refusing to enter the kingdom of heaven, and in trying to hinder others, they manifest the spirit of the great destroyer, and are hastening towards the place prepared for him and his angels.>
<17:15 Athens; the most distinguished city in Greece, and the seat of literature, philosophy, and the fine arts.>
<17:16 Men may be renowned for human learning, and greatly attached to rites and forms of devotion, and yet be ignorant of the only proper object and way of religious worship, and be sunk in deep degradation and wickedness.>
<17:17 Disputed; reasoned with them concerning the Messiahship of Jesus Christ, the worship of the one living and true God, and the retributions of eternity.>
<17:18 Epicureans; from Epicurus their founder, who taught that pleasure is the chief good. Stoics; from stoa, a Greek word meaning a porch; because, in a structure so named, Zeno the founder of this sect taught his doctrines. Among these the two following were prominent: that all things are fixed by necessity, and that the chief good of man lies in raising himself to a state of indifference to all earthly things.>
<17:19 Areopagus; that is, as the word means, Mars-hill; a rocky height in Athens, opposite the western end of the Acropolis, where the highest Athenian court was held.>
<17:22 Too superstitious; rather, very religiously inclined, very much disposed to honor the gods.>
<17:23 Passed by; passed through the city, and beheld the sacred places and objects. To the unknown God; or, as the inscription may also be rendered, To an unknown God. They not only worshipped all the gods that were known, but had an altar to one that was unknown. However this might have arisen, it showed the truth of what Paul said, that they were much inclined to honor the gods; and it furnished him with an admirable occasion of preaching to them the true God, who was to them "the unknown God.">
<17:24 Dwelleth not in temples, according to the heathen idea of a local and limited presence. He is not confined to any place, but fills all places.>
<17:26 Hath made of one blood; caused all men to spring from one family. Times--bounds; the countries where they should live, and the periods during which they should occupy them. As all the human race are the offspring, and are under the government of one common Father, they are all brethren of one family, and are bound to treat one another as such. Each has rights given him by his heavenly Father, of which no man or body of men can deprive him, without deep injustice against a brother, and flagrant rebellion against God.>
<17:27 Seek the Lord; the object of God in creation and providence is, that men may know, worship, and enjoy him. This they may do, for he is everywhere present, sustaining, upholding, and governing all things.>
<17:28 Your own poets; Aratus of Cilicia, a Greek poet who lived more than three hundred years before, and Cleanthes, who lived about the same time, and was successor of Zeno the founder of the Stoics, both expressed the sentiment referred to; and Aratus expressed it in the very words which Paul quotes.>
<17:29 We ought not to think; that Jehovah is like material objects of any kind.>
<17:30 Winked at; suffered, bore with, and did not give them a written expression of his condemnation of these sins, or come out in judgment against them. To repent; of their idolatry and wickedness, and turn to the worship and service of the one only living and true God.>
<17:31 That man; Jesus Christ. Ordained; appointed for this purpose. Assurance; conclusive evidence. A day of searching and righteous judgment is coming, when each and all must stand before an omniscient and almighty Judge, who will render to all according to their works. Ro 2:6-11; Re 14:11.>
<17:32 Some mocked; they considered a resurrection impossible and absurd.>
<17:34 Clave unto him; believed his doctrines, and followed him as their teacher. The Areopagite; a member of the Areopagus, or Athenian court.>
<18:1 Corinth; the capital of Achaia, a province of Greece.>
<18:2 Pontus; the north-eastern province of Asia Minor. Claudius; the Roman emperor, who reigned from A.D. 41 to 54.>
<18:3 Craft; trade, or occupation. They were tent-makers; it was the custom of the Jews to have their sons taught some mechanical art; that they might thus, in any emergency, be able to provide for themselves. A knowledge of useful business is important to all, especially to ministers who are called to preach the gospel and establish churches in heathen lands; and diligence in the discharge of appropriate duties is honorable, and highly conducive to usefulness and enjoyment.>
<18:4 Persuaded the Jews; that Jesus was the Christ.>
<18:5 Pressed in the spirit; distressed in view of the condition of the Jews, and powerfully moved to preach to them the gospel.>
<18:6 Opposed themselves; set themselves against him and his preaching. Shook his raiment; in token of deep abhorrence of their sins. Your blood; the guilt of your destruction. I am clean; free from blame.>
<18:7 One that worshipped God; a proselyte to the Jewish religion.>
<18:9 Ministers of the gospel are at times liable to great fear and depression of spirits. But God is able to encourage and protect them. He would have them remember, that among their opposers may be many who will be their warmest friends; and that by perseverance and fidelity in preaching the gospel, they may be instrumental in preparing them for an exceeding and eternal weight of glory.>
<18:10 Much people in this city; many who shall receive the gospel and be saved.>
<18:12 Gallio; the Roman governor of the province of Achaia, and brother of the philosopher Seneca. Made insurrection; assaulted and apprehended Paul.>
<18:13 Contrary to the law; the law of Moses.>
<18:14 Paul was now about to open his mouth; to speak in his own defence. Wrong or wicked lewdness; injustice, or any crime. Reason would that I should bear with you; it would be reasonable to hear the complaint and try the cause.>
<18:15 A question of words and names; merely about their religion.>
<18:17 Then all the Greeks; the Gentiles present, who sympathized with Gallio in his abhorrence of Jewish bigotry. Sosthenes; probably at that time a leader of the persecution against Paul. Cared for none of those things; troubled not himself about the treatment that Sosthenes received from the Greeks, being willing that he should suffer the indignation of the people.>
<18:18 Shorn his head; cut off his hair, as was the custom in fulfilment of a vow, or promise to the Lord. Cenchrea; the seaport of Corinth, about eight miles east of the city.>
<18:19 Ephesus; a city of Ionia in Asia Minor, about forty miles south of Smyrna.>
<18:21 This feast; the feast of the passover. A pious man with right views, in forming his plans, will not lose sight of his dependence on God, or forget that, if the Lord will, he shall live and do this or that; and he will honor Him by the suitable expression, at proper times, of this momentous truth.>
<18:22 Cesarea; a seaport on the western coast of Palestine. Gone up; to Jerusalem. Antioch; in Syria, the place from which he went out on his late tour.>
<18:23 Galatia and Phrygia; provinces which he had before visited. Strengthening all the disciples; confirming their faith in the gospel and encouraging them to persevere in holy living.>
<18:24 Alexandria; a city in the north part of Egypt, founded by Alexander the Great, about three hundred years before Christ.>
<18:25 This man was instructed in the way of the Lord; so far as it was revealed in the Old Testament, and made known by the preaching of John the Baptist.>
<18:26 Expounded unto him the way of God more perfectly; they showed him what had taken place since the preaching of John with regard to the Messiah, and pointed out more clearly the way of salvation through him. Great zeal and eloquence, united with love to Christ and the souls of men, are not enough to make a minister of the gospel most useful. He must also be acquainted with the word and the providence of God, and be able to show how one is a fulfilment and illustration of the other.>
<18:27 Achaia; a part of Greece, of which Corinth was the capital, where Aquila and Priscilla had lived. The brethren wrote; the Christians at Ephesus wrote letters, recommending Apollos to the Christians in Achaia. Helped them much; by his zeal, eloquence, and piety, he greatly assisted those who, through grace, had believed in Jesus Christ.>
<19:1 The upper coasts; Phrygia and Galatia, which lay east at a distance from the sea on higher ground than Ephesus.>
<19:2 Received the Holy Ghost; his miraculous powers. Not so much as heard; that the Holy Spirit had been given, in his special manifestations, as at the day of Pentecost. Ac 2:17,18.>
<19:6 The Holy Ghost came on them; with his miraculous power, as he had done on other disciples.>
<19:9 Divers were hardened; by resisting and rejecting the truth. That way; the way of salvation which Paul preached. The school; the room or place where Tyrannus taught. When the preaching of the gospel only hardens men, and leads them more bitterly to oppose the truth, it is sometimes best for ministers to leave them, and go where there is a greater prospect of success. The rejection of the gospel by some is thus made the occasion of salvation to others.>
<19:10 Asia; Ionia, or proconsular Asia, of which Ephesus was the capital.>
<19:13 Vagabond; wandering about from city to city in the exercise of their juggling arts. Exorcists; those who pretended, by repeating the name of God, and performing certain ceremonies, to heal diseases and expel evil spirits. We adjure you; charge you, by an oath in the name of Jesus, whom Paul preacheth, to come out.>
<19:15 Jesus I know, and Paul; their power to expel us we acknowledge, but not yours.>
<19:17 Magnified; he was increasingly honored. The efforts of wicked men to exalt themselves result in their deeper abasement; and attempts to dishonor Christ are overruled for the promotion of his glory, and the advancement of his cause.>
<19:18 Confessed; their former wicked deeds.>
<19:19 Curious arts; cunning, adroit practices of jugglery and deception. Books; those which taught the way to practise these arts. When men are truly converted to God and obey his will, they will forsake their former wicked practices, however popular or gainful; and all that property which cannot be made useful to mankind they will destroy.>
<19:21 Macedonia and Achaia; provinces in Greece which he had before visited. Chapter Ac 16:10; 17:15; 18:12.>
<19:22 Erastus; he had been chamberlain, or treasurer of Corinth. Ro 16:23. Asia; verse Ac 19:10.>
<19:23 That way; the way of salvation through Jesus Christ.>
<19:24 Shrines for Diana; Diana was a celebrated heathen goddess, to whom a splendid temple was erected at Ephesus. The shrines were small silver temples, in imitation of that of Diana, which were bought by multitudes, and kept as precious memorials. Craftsmen; the silversmiths who made these shrines.>
<19:25 Craft; trade.>
<19:26 Turned away much people; turned many from the worship of idols. The gospel, in proportion as it prevails, will render profitless all those trades and employments which demoralize and injure mankind. It will also render the conviction universal, that men have no right to prosecute or encourage any business which is not beneficial to their fellow-men.>
<19:31 Theatre; which was customarily used as a place for the public assembling of the citizens.>
<19:33 Putting him forward; for the purpose of giving him opportunity to address the assembly.>
<19:34 Cried out; to prevent Alexander from being heard. When, by the prevalence of truth and love, wicked men are prevented from making money by wrong-doing, they are often filled with wrath; they sometimes excite a tumult, and by senseless clamor try to drown the voice of reason, and to sweep away all that hinders them by the whirlwind of passion.>
<19:35 Town-clerk; a city magistrate, who presided in the senate, recorded the laws, and read in public what was to be communicated to the people. Appeased the people; quieted them, so that he could be heard. The image; the image of Diana which was kept in the temple, and which they said was given by Jupiter the chief of the gods.>
<19:36 Cannot be spoken against; cannot be denied; must be admitted as true.>
<19:37 Churches; temples.>
<19:38 A matter; just cause of complaint. Deptuies; proconsuls, who presided over the administration of justice. Implead one another; argue their cases in court. Sometimes men of standing and influence are found with rabble in a riot. Not unfrequently designing individuals, who live on the vices of the people, and whose pecuniary interests are affected by the increase of light, are the instigators of lawless outbreaks against liberty and life; such ought, for the safety of the public, to be visited with exemplary punishment.>
<19:39 Other matters; things which concern not individuals merely, but the public. Lawful assembly; one not got up by tumult, but regularly called.>
<19:40 Called in question for this day's uproar; accused to the Roman government and punished. Riotous assemblies were forbidden by Roman law, and the penalty for instigating them was death. The Ephesians therefore, especially Demetrius and his associates, were in no small danger; and when they came to consider, they were very willing to disperse, as is often the case with rioters under an efficient government.>
<20:4 Asia; Asia Minor, or proconsular Asia. Chap Ac 19:10.s The persons here mentioned were Christian brethren.>
<20:5 For us; this language shows that Luke, the author of this book, was one of the company. Troas; chap Ac 16:8.>
<20:6 Unleavened bread; the passover.>
<20:7 First day of the week; the Lord's day, or Christian Sabbath. To break bread; celebrate the Lord's supper. The first day of the week was observed by the apostles and those who were under the special guidance of the Holy Ghost, as the Lord's day or the Christian Sabbath, a day for hearing the gospel and public worship. It has been so observed by pious men and by the church of God in all subsequent ages, and it will continue to be so observed to the end of time.>
<20:13 Assos; a maritime city near Troas, on the Aegean sea.>
<20:14 Mitylene; the capital of the island of Lesbos.>
<20:15 Chios; an island in the Aegean sea, now called Scio. Samos; an island near the province of Lydia. Trogyllium; a town on the coast opposite Samos. Miletus; a maritime town south of Ephesus.>
<20:16 Sail by Ephesus; pass it without stopping.>
<20:17 The elders of the church; pastors and teachers. It has ever been the will of God that Christian churches should be supplied with pastors, who should preach to them the gospel, set them holy examples, superintend their religious concerns, and devote themselves to the promotion of their spiritual good. It has also ever been the will of God, that some of his ministers should be evangelists or missionaries; should visit and gather churches among the destitute in Christian and Pagan lands, and do what they can to preach the gospel to every creature.>
<20:20 Kept back nothing; refrained from teaching no doctrines or duties that would benefit the people.>
<20:22 Bound in the Spirit; influenced by the Holy Ghost and a strong conviction of duty.>
<20:26 Pure from the blood; am not guilty, through unfaithfulness, of the destruction of any. Ministers, to be free from the guilt of being accessory to the ruin of men, must declare to them, as far as they understand it, the divine will; and especially must they show them the author, nature, necessity, and evidences of repentance towards God and faith in our Lord Jesus Christ, and set before them the motives which God has revealed, and which are suited to lead them to exercise these spiritual graces; and in doing this, they must depend upon and earnestly seek the influences of the Holy Ghost.>
<20:28 Overseers; in the original, bishops, who in verse Ac 20:17 are called elders, and who had the oversight of the church. God; Jesus Christ.>
<20:29 Grievous wolves; teachers of false and destructive doctrines. The flock; the church of God.>
<20:31 Warn every one; of their danger, and point out the way to escape it.>
<20:34 Ministered unto my necessities; wrought for the supply of my wants.>
<20:35 Support the weak; supply the wants of the feeble and destitute. The words of the Lord Jesus; a saying of our Lord not recorded in the gospels, though the truth which it contains was taught abundantly by Jesus Christ on many occasions. More blessed to give than to receive; to be instrumental in communicating blessings, than in merely receiving them. If a minister of the gospel is not supported by others, it is no dishonor and no dereliction of duty for him to labor, working with his own hands, that he may supply his necessities; and in proportion as he resembles his Lord, he will find that the excellence, usefulness, and happiness of communicating blessings are greater than of merely receiving them. The soul is so made, that if it would be good, it must do good; and if it would be happy, it must delight in making others happy.>
<21:1 Them; the Ephesian elders at Miletus. Coos; a small island near the south-west corner of Asia Minor. Rhodes; an island not far from Coos. Patara; a seaport in the province of Lycia.>
<21:2 Phenicia; a country on the north-west border of Canaan, of which Tyre and Sidon were principal cities. Chap Ac 11:19.>
<21:3 Cyprus; an island in the north-east part of the Mediterranean sea. Chap Ac 13:4.>
<21:4 Said to Paul through the Spirit; the Holy Spirit made known to them the dangers to which Paul would be exposed, and led them to express to him their strong desire that he should not go up to Jerusalem. But he did not communicate by them to Paul, who himself went up to Jerusalem "bound in the Spirit," any authoritative command to desist from his purpose. It is delightful to Christians when travelling to find disciples of Christ in places which they visit, and to tarry with them, when convenient. The more they love Christ and are like him, the more they will love one another. But their stay together on earth must be short, and when called to part, it is a great privilege to commend one another to God, and beseech him, that if they meet not again on earth, they may meet, to part no more, in heaven.>
<21:7 Ptolemais; a city south of Tyre, and near mount Carmel.>
<21:8 Cesarea; a city still farther south, about sixty miles from Jerusalem. Evangelist; a preacher of the gospel and founder of churches, but not a settled pastor. The seven; Ac 6:1-9.>
<21:9 Did prophesy; see note to chap Ac 11:27.>
<21:10 Agabus; chap Ac 11:27,28.>
<21:11 Gentiles; the Romans, who then governed Judea.>
<21:12 We and they; Paul's companions and the disciples at Cesarea.>
<21:13 Affection for friends should not be so manifested or indulged as to unfit us or them for the performance of duty.>
<21:15 Carriages; not vehicles to carry them, but things which they carried, their baggage.>
<21:16 Mnason of Cyprus; formerly of Cyprus, but now living in Jerusalem, with whom Paul and his companions lodged.>
<21:18 James; see note to chapter Ac 12:17.>
<21:20 Zealous of the law; though they believed in Christ, they still observed the ceremonial law.>
<21:21 The Jews which are among the Gentiles; who lived in heathen countries. Moses; the ceremonial law which he enjoined. The customs; the religious practices of the Jews. To excite opposition to the gospel, its enemies often misrepresent it. They slander those who preach it, and in various ways show themselves to be followers of him who, from the beginning, abode not in the truth, but was a liar, and the father of it. Joh 8:44.>
<21:22 What is it; what is to be done?>
<21:24 Purify thyself with them; perform the ceremonies required for purification in the Mosaic law. Nu 6:1-20. Be at charges; by bearing the expense of the offerings made in such cases. Shave their heads; cut their hair, which during the continuance of their vow had been suffered to grow; and which, when cut, showed that they were released. Chap Ac 18:18. Are nothing; are not as they have been represented.>
<21:25 Touching the Gentiles which believe; chap Ac 15:19-29.>
<21:26 Entered into the temple; to make known to the priest when the vow would end, and thus prepare the way for the customary sacrifices to be offered. Some things are in themselves indifferent; if we do them we are not the better, and if we neglect them we are not the worse: if our doing them would tend to injure others, we should not do them; if it would tend to benefit others, we should not neglect them; but we should not do wrong to conciliate bad men, or for any other purpose.>
<21:27 Seven days; the time the vow was to last.>
<21:28 Greeks; Gentiles, who were considered as unclean, and were forbidden to enter that part of the temple appropriated to the Jews, lest they should defile it.>
<21:29 Trophimus; he had come with Paul from Ephesus. Ac 20:4. They supposed; they inferred this from having seen him in the city with Paul.>
<21:31 Chief captain; the Roman officer in command of the military forces at Jerusalem.>
<21:33 Bound with two chains; a fulfilment of the prophecy, Ac 21:11.>
<21:34 The castle; the fortress of Antonia, where the soldiers were stationed.>
<21:35 The stairs; which led from the temple to the castle.>
<21:36 The course which good men take to remove the prejudices and to conciliate the favor of the wicked, is sometimes the occasion of increased hostility, and more violent and deadly opposition.>
<21:38 That Egyptian; this Egyptian is mentioned by Josephus, Antiq B 20, chap 8, sec 6; and Jewish Wars, B 2, chap 13, sec 5.>
<21:39 Tarsus; Paul's native city. Chap Ac 9:11. No mean city; Augustus the Roman emperor made it a free city, which released the citizens from tribute, and permitted them to be governed by their own laws. Josephus says it was the metropolis, and most renowned city of Cilicia, Antiq B 1, chap 6, sec 6; and Xenophon calls it a great and flourishing city.>
<21:40 The Hebrew tongue; the dialect of the Hebrew then spoken in Palestine.>
<22:1 The best defence which a man can make of himself and his conduct often is to give a plain statement of the providences of God of the reasons which satisfied his own mind, and which, in his view, ought to satisfy the minds of others.>
<22:3 Tarsus; chap Ac 21:39. This city; Jerusalem. Gamaliel; a celebrated Jewish teacher.>
<22:4 This way; the christian cause.>
<22:5 High-priest; chap Ac 9:1,2. Estate of the elders; the Sanhedrin, or national council of the Jews.>
<22:6 6-13. The conversion of Paul. Chap Ac 9:3-18.>
<22:9 Heard not; see note to chap Ac 9:7.>
<22:14 Shouldest--see that Just One; Jesus Christ. Chap Ac 3:14; 7:52. See also note to chap Ac 9:17. Persecutors of Christ and his cause are sometimes made his cordial friends, and eminently instrumental in extending the triumphs of his kingdom, having been chosen to salvation, through sanctification of the Spirit, and belief of the truth. 2Th 2:13.>
<22:16 Wash away thy sins, calling on the name of the Lord; be cleansed by the Holy Spirit, through faith in Christ and obedience to his commands.>
<22:18 Him; the Lord Jesus Christ.>
<22:19 They know that I imprisoned; he thought that the knowledge which the men of Jerusalem had of his former treatment of the Christians would convince them of his sincerity, and dispose them to listen to his arguments. But the Saviour, who better knew their hearts, saw that it would not.>
<22:20 Stephen; chap Ac 7:58; 8:1.>
<22:21 Christ not only calls his ministers, but assigns them the places and conditions in which they are to labor. These may be very different from what they, if left to themselves, would select; yet he orders them in wisdom, and if his servants follow his directions, he will render them as useful as will best promote his glory, and the highest good of his kingdom.>
<22:22 Unto this word; that God would send him to the Gentiles. That the Gentiles should be admitted to equal privileges with the Jews in the Messiah's kingdom, was that doctrine which above all others gave offence to them.>
<22:23 Threw dust; expressive of their abhorrence.>
<22:24 Castle; chap Ac 21:34. Scourging; whipping--a Roman mode of torturing men, to make them confess their crimes. That he might know; his ignorance of the Hebrew tongue, used by both Paul and his enemies, probably prevented him from understanding the nature of the charge made against the apostle.>
<22:25 Thongs; cords or straps. A Roman; a Roman citizen. As civil government is an ordinance of God, it is sometimes proper, when assailed, to avail ourselves of its protection. True religion inculcates submission under trials, and the use of all suitable means to avoid and remove them.>
<22:28 This freedom; of Roman citizenship. This was sometimes conferred as a reward for distinguished services, sometimes inherited, and sometimes bought with money.>
<22:29 The chief captain--was afraid; to bind a Roman citizen, uncondemned, for the purpose of scourging him, was contrary to the Roman law, and exposed him who did it to severe punishment.>
<22:30 Have known the certainty; the crime of which he was accused. Them; the Jewish council.>
<23:1 Lived in all good conscience; done what I thought to be right.>
<23:2 Smite him on the mouth; by this he would indicate that Paul had uttered a falsehood.>
<23:3 Smite thee; visit in judgment. Whited wall; hypocrite.>
<23:5 Wist not; knew not. At this time the occupancy of the high-priesthood had become very irregular. If Ananias actually was then the high-priest, it is probable that Paul either did not know the fact, or at the moment did not advert to it. It is written; Ex 22:28.>
<23:6 I am a Pharisee; Paul agreed with the Pharisees in believing that the soul lives after the death of the body, and that there will be a resurrection--points on which they differed from the Sadducees. Resurrection of the dead; he contended that Jesus Christ had actually risen. This showed that He was the Messiah, that the doctrine of a resurrection was true, and that all men would be raised. 1Co 15:12-23.>
<23:7 There is no bond of permanent union among persecutors of Christians. They may, for selfish purposes, unite for a time in opposing the truth; but they are easily divided; and God, through their divisions, may deliver his people from their power.>
<23:8 Neither angel, nor spirit; no such thing as created spirits existing, separate from bodies. Confess both; receive both doctrines as true.>
<23:9 A spirit or an angel; chap Ac 22:6-8,17-21.>
<23:11 The Lord; the Lord Jesus. The Lord Jesus Christ is able, at any time, and in any place, to manifest himself to his people, to fill them with joy, and make their enemies the occasion of accomplishing what is most earnestly to be desired.>
<23:12 A curse; a solemn oath, imprecating divine vengeance on themselves should they eat or drink before they had killed Paul.>
<23:21 Looking for a promise from thee; that he would bring Paul down. Bigoted and hypocritical professors of religion, who have adopted wrong principles, and been corrupted by wicked practices, are often among the most deceitful, hardened, and cruel of mankind, and the most malignant opposers of divine truth.>
<23:23 Third hour of the night; nine o'clock.>
<23:24 Felix; he had been a slave of Antonia, the mother of Claudius the Roman emperor, but was freed and became governor of Judea.>
<23:27 An army; a band of soldiers. A Roman; a Roman citizen.>
<23:29 The persecution of peaceable citizens on account of their religion, of their reading the Bible, and judging of its meaning, is such an outrage as to be condemned even by heathen. When practised, it has ever been, and ever will be, a foul disgrace not only to the Christian, but to the civilized world.>
<23:31 Antipatris; a town about forty miles from Jerusalem, on the way to Cesarea.>
<23:35 Herod's judgment-hall, literally, Herod's praetorium; this was the palace built by Herod the Great at Cesarea, and now occupied by the Roman governor Felix.>
<24:1 Tertullus; a lawyer employed to plead against Paul.>
<24:5 Pestilent fellow; literally, a pest. Eloquence may be employed in propagating falsehood and promoting wickedness. It then becomes a world of iniquity, setting on fire the course of nature, being set on fire of hell. Jas 3:5-10.>
<24:6 Profane the temple; chap Ac 21:27-30.>
<24:7 Lysias; chap Ac 21:31-40; 23:26-30.>
<24:10 Christians rejoice to state facts and proclaim truth before those who are capable of judging. They are friends of free discussion, knowing that from it truth has nothing to fear. Though falsehood may be specious, and when set off with the trappings of oratory, may for a time prevail, truth stated in its native simplicity, and shining with its own brightness, will ultimately triumph.>
<24:14 The way; in the Acts of the Apostles the Christian religion is commonly called "the way," chap Ac 9:2; Ac 19:9. etc. Heresy; the original word signifies rather, sect, that is, schismatic party. So the Jews falsely and reproachfully called the Christians.>
<24:15 Have hope; of a resurrection. An abiding conviction of the certainty of a resurrection, and of a future retribution according to the deeds done in the body, tends powerfully to keep the conscience awake, and to lead men habitually to do right; while the disbelief of those truths removes a powerful restraint against doing wrong.>
<24:16 I exercise myself; habitually strive. A conscience void of offence; one that shall accuse me of no departure from duty towards God or man.>
<24:18 Purified; according to the ceremonies of the Mosaic law. Chap Ac 21:26-28.>
<24:20 Council; chap Ac 23:1-10.>
<24:22 That way; the Christian way. See note to verse Ac 24:14. Deferred; put off the further hearing of the case till Lysias should arrive. Whether Felix was sincere in this delay appears doubtful from his conduct as recorded in verses Ac 24:26,27.>
<24:24 Drusilla; she was daughter of the first Herod Agrippa; was married to Azizus king of Emesa, but afterwards left him, and became the wife of Felix.>
<24:25 Righteousness; doing right towards God and man. Temperance; the proper regulation of the appetites and passions. Judgment to come; the future general judgment, when all will receive from Christ according to their works. Chap Ac 17:31; Mt 25:31-46. Felix trembled; in view of his sins, and his prospects at the coming judgment. A convenient season; such a season does not seem ever to have arrived. When conviction of sin produces fearful forebodings of coming wrath, different persons take opposite courses. One inquires, "Lord, what wilt thou have me to do?" The Lord shows him, leads him to do it, and he is saved. Chap Ac 9:6. Another dismisses the subject, continues in known sin, and goes down to perdition. "To-day if ye will hear his voice, harden not your hearts." "He that being often reproved hardeneth his neck, shall suddenly be destroyed, and that without remedy." Heb 3:7,8; Pr 29:1.>
<24:26 Money--given; as a bribe, to induce him to release Paul. Worldly minded and covetous men may seek to make even the preaching of the gospel a means of adding to their unrighteous gain. Such place themselves in a position where there is little hope that even the plainest and most faithful exhibitions of the truth can profit them.>
<24:27 Came into Felix's room; succeeded him in office. Willing to show the Jews a pleasure; Felix knew that they had just grounds of accusing himself as governor to the Roman emperor, and he took this unrighteous way of conciliating their good will. Left Paul bound; detained him as a prisoner, though justice required his release. One unrighteous deed on the part of a ruler places him in the power of the wicked, who will compel him to buy their favor by further acts of injustice.>
<25:1 Came into Felix's room; succeeded him in office. Willing to show the Jews a pleasure; Felix knew that they had just grounds of accusing himself as governor to the Roman emperor, and he took this unrighteous way of conciliating their good will. Left Paul bound; detained him as a prisoner, though justice required his release. One unrighteous deed on the part of a ruler places him in the power of the wicked, who will compel him to buy their favor by further acts of injustice.>
<25:3 Desired favor; desired that Festus would favor them by sending for Paul, so that they might kill him.>
<25:5 Any wickedness; if he has committed any crime.>
<25:7 Persecutors of Christians, though high in ecclesiastical or political office, will often make statements which they cannot prove, which are not true, and the falsehood and malignity of which are so manifest, that they may be seen and condemned even by enlightened heathen.>
<25:10 Caesar's judgment-seat; the Roman tribunal, before which he then was. Of course there was no good reason why he should go to Jerusalem.>
<25:11 I appeal unto Caesar; I will go to Rome and be tried before the emperor. To this, as a Roman citizen, he had a right. When men destitute of the spirit of Christ, at the head of ecclesiastical affairs, deny the right of private judgment, and are disposed to persecute those who exercise it, good men sometimes have more to fear from them, than from the most absolute civil despot.>
<25:12 The council; his own council, with whom he was accustomed to advise in the administration of justice. The original Greek has a different word here from that applied to the Jewish council, or Sanhedrin.>
<25:13 Agrippa; Herod Agrippa, son of the Herod mentioned in chapter Ac 12:1, and great-grandson of Herod the Great, under whose reign Christ was born. Bernice; Agrippa's sister. She first married her uncle the king of Chalcis, and then Polemon king of Chilicia, whom she deserted to live with her brother Felix Agrippa. To salute Festus; to congratulate him upon his accession to office.>
<25:16 To condemn a man unheard, without his being informed of the nature, extent, and grounds of his accusation, or being permitted to meet and examine his accusers face to face, is the essence of tyranny; and must be condemned by the judgment and common-sense of the whole world.>
<25:19 Worldly politicians, high in office and clothed with great pomp and power, often think and speak very lightly of events into which angels desire to look, which fill heaven with rapture, and will be the theme of grateful and adoring praises from multitudes which no man can number, for ever and ever.>
<25:20 Doubted of such manner of questions; how such questions should be disposed of, or what course he should take with them.>
<25:21 Augustus; Augustus and Caesar were used as terms of office: each meant the Roman emperor. At that time, this emperor was Nero.>
<25:24 Have dealt with me; accused before me, and wished me to condemn.>
<25:26 No certain thing; no crime to allege, or accusation to specify. Somewhat; something definite.>
<25:27 Signify; point out, specify.>
<26:1 Signify; point out, specify.>
<26:2 It is pleasant to proclaim the gospel to intelligent hearers, especially such as are well acquainted with the Bible. True religion does not fear, but courts investigation. It accords with the word, and is proved to be true by the providence of God. The more it is examined in the light of the Scriptures and of facts, the deeper will be the conviction that it is from God; and the more intelligent men are, the more guilty they will be, if they do not embrace it.>
<26:3 Expert in all customs and questions--among the Jews; Agrippa was of Idumaean descent. But from the days of John Hyrcanus, the Idumaeans south of Palestine, to whom his family belonged, had adopted the Jewish religion, and were reckoned as Jews.>
<26:4 At the first among mine own nation at Jerusalem; though born in Tarsus, Paul was early sent to Jerusalem for his education. Chapter Ac 22:3.>
<26:6 The promise; of the Messiah.>
<26:7 Hope to come; hope to experience the fulfilment of the promise. Hope's sake; for having believed in Christ as the Messiah, and expecting a resurrection through him to endless life.>
<26:8 Incredible; not to be believed; absurd, or wanting evidence.>
<26:9 I verily thought with myself; in persecuting Christ, Paul was sincere in his error; but this did not make him guiltless, for his belief had its foundation in a wrong state of heart. Had he been humble, candid, and teachable, the evidence which Jesus gave of his Messiahship would have carried full conviction to his mind. Contrary to the name; in opposition to the teachings and to the followers of Christ. Conscience is not always a safe guide. It must be enlightened by the word and Spirit of God, and accompanied with a pious heart. All should feel this, and so acknowledge God, that he may direct their paths, guide them in judgment, and teach them his way--that way of pleasantness, that path of peace. Pr 3:6; Ps 25:9.>
<26:10 Authority; chap Ac 9:14; 22:4,5.>
<26:11 Compelled them; that is, used violent means to induce them. It does not follow that his efforts were successful. Blaspheme; the name of Jesus, by denying him to be the Messiah. Strange cities; cities in foreign lands.>
<26:12 Whereupon; while engaged in this persecution. Chap Ac 9:1-6.>
<26:17 From the people; that is of the Jews. Send thee; chap Ac 22:21.>
<26:18 Open their eyes; enlighten their minds. Darkness to light; the ignorance of heathenism to the saving knowledge of the gospel. Power of Satan; from his service to the service of God. Men naturally are ignorant of spiritual things, yield themselves the willing slaves of Satan, and walk in darkness. They need the gospel of Christ, accompanied by the enlightening and purifying influences of his Spirit, to turn them from supreme love of self and sin to supreme love of God and holiness. This gospel must be carried and preached to them, by men whom God raises up and sends forth for this purpose; and these should be aided by the prayers and contributions of all, till the gospel is preached to every creature. Mr 16:15.>
<26:20 Damascus; chap Ac 9:19-23. Meet for repentance; such as true repentance produces.>
<26:22 Small and great; all classes of people. None other things; except those which were foretold in the Old Testament. Isa 53:3-9; Ps 16:10; Ac 2:31; 13:35-37; Isa 9:1,2.>
<26:24 When a man treats the truths of the Bible as realities, and speaks and acts as if he expected to see their fulfilment, those who regard these truths as fables often think and speak of him as deranged. But the more fully a person obeys the word of God, and lives as if he expected its fulfilment, the greater evidence he gives of being in his right mind; and all men who, like the prodigal son, come to their right mind, will view and treat the Bible in the same way.>
<26:27 Believest thou the prophets? he appeals to Agrippa as a Jew, who professed to receive the scriptures of the Old Testament as the word of God.>
<26:28 Persuadest me; by the evidence exhibited that the prophecies of the Old Testament concerning the Messiah were fulfilled in Christ.>
<26:29 I would to God; I earnestly desire. As I am; real Christians. These bonds; the chains with which he was bound. However wickedly true Christians may be treated by others, they do not wish, so far as they are right, to render evil for evil; but in all suitable ways, to promote the highest temporal and eternal good even of their worst enemies.>
<26:32 Set at liberty; he is an innocent man, guilty of no crime.>
<27:1 We; Luke, Paul, and others. Italy; a country in the south part of Europe, between the Adriatic and Mediterranean seas.>
<27:2 Adramyttium; a seaport of Mysia in the north-western part of Asia Minor. It lay opposite to the isle of Lesbos. Aristarchus; chap Ac 19:29; 20:4.>
<27:3 Sidon; north of Cesarea, from which Paul sailed. Verse Ac 27:2; chap Ac 25:4,13,21. When a man's ways please the Lord, he can make not only his enemies, but strangers, and even heathen, not merely to be at peace with him, but to aid and assist him. Pr 16:7.>
<27:4 Under Cyprus; along its northern coast, between the island and the main land, to shield themselves from the violence of the wind. Winds were contrary; they were the westerly or north-westerly winds which prevail there at that season.>
<27:5 The sea of Cilicia and Pamphylia; the sea along the coast of those provinces of Asia Minor. Cilicia lay on the south coast of Asia Minor opposite Cyprus, and Pamphylia was the next province west. Lycia; next west of Pamphylia.>
<27:6 Alexandria; a city of Egypt.>
<27:7 Scarce; with difficulty. Cnidus; a town in the province of Caria next west of Lycia. It is in the south-western angle of Asia Minor, and has the isle of Rhodes opposite to it. Under Crete; near that island. Salmone; the eastern extremity of Crete.>
<27:8 Hardly passing it; coasting along it with difficulty. Fair Havens; on the southern side of Crete, about midway between its eastern and western extremities.>
<27:9 Much time was spent; on account of the contrary winds. The fast; connected with the great day of atonement. Le 16:29,30. This occurred about the twentieth of September, after which sailing was dangerous.>
<27:10 Much damage--our lives; these words of Paul seem to express not a revelation from God, but rather his own sound judgment. With regard to his own life, he had received from the Lord the assurance that he should see Rome, chap Ac 23:11; but he had not yet received any promise that the lives of all in the ship should be saved. see verses Ac 27:23,24.>
<27:12 Not commodious to winter in; being open to the wind and sea on the south. Phenice; a place in the south-west part of Crete. The majority are often in the wrong, and it is not always wise or safe to follow them. The great question should not be, on which side are the greatest numbers, but on which are truth and duty; and a truly pious man, in seasons of danger and difficulty, may say and do things which it would be unwise to attempt at other times.>
<27:13 The south wind blew softly; which would be favorable to their purpose, as the coast a few miles beyond the Fair Havens turn to the north of west. Close by; near the shore.>
<27:14 Euroclydon; these winds, now called Levanters, blow from nearly east-north-east.>
<27:15 Caught; suddenly met by the wind. Could not bear up; sail against it. Let her drive; before the wind.>
<27:16 Claudia; a small island a little south of west from the Fair Havens, at the distance of some forty or fifty miles. Melita, the next place where we find them, is a small island south of Sicily, not quite five hundred miles to the west of Clauda. To come by the boat; to secure it, by taking it on board. Verse Ac 27:17.>
<27:17 Taken up; taken into the ship, to prevent the boat being broken or lost. Used helps, undergirding; putting chains or ropes around the vessel, to strengthen it and keep it together. Quicksands; on the coast of Africa, south-west of them. Strake sail; these words do not seem to mean that they took in all sail, which would have left them drifting towards the quicksands at the mercy of the wind and waves; but rather, that they reduced their sail very low. This would enable them, while driven before the wind, to keep the ship's head in a measure towards the north-west, and thus avoid the African coast and the quicksands.>
<27:18 Lightened the ship; threw overboard some of her cargo.>
<27:19 Tackling; whatever belonged to the ship which could be spared.>
<27:20 Neither sun nor stars; the mariner's compass was not then known. When sailors could not see the heavenly bodies or the land, they did not know their course. God in his providence often shows men, especially those who traverse the ocean, that they are dependent on him; that all their efforts to deliver themselves are utterly insufficient, and that he must save them, or they must perish.>
<27:21 Long abstinence; from food, on account of the severity of the storm and the greatness of their danger.>
<27:24 God hath given thee all; for thy sake, and in answer to thy prayers, they shall be preserved.>
<27:27 Fourteenth night; after the commencement of the storm. Adria: in the wider sense, including not only the Adriatic gulf, but the Ionian sea south of it. Deemed; thought, judged.>
<27:28 Sounded; let down a lead and line to ascertain the depth of the water. Twenty fathoms; one hundred and twenty feet. Fifteen fathoms; ninety feet.>
<27:29 Four anchors; to hold the ship where she was. Stern; the hinder part of the ship. For the day; for daylight, that they might see where they were.>
<27:30 Shipmen; sailors. To flee; escape to the shore, and leave the others to take care of themselves or perish. Under color; under the pretence.>
<27:31 Paul; who saw what they intended. These; the sailors. Ye cannot be saved; their agency was necessary to manage the vessel. Though God had given Paul the lives of all in the ship, they were yet to be saved by the use of the appropriate means. A future event may be certain because God has determined and revealed it, and it may also be true that unless men use the proper means it will never take place. It was certain that all the two hundred and seventy-five who were with Paul in the ship would get to land, and it was also certain that unless the sailors should stay and manage the ship they would not get to land. So that the use of proper means is just as necessary to accomplish an event which is beforehand certain, as it would be if it were not certain, and its accomplishment depended solely on those means.>
<27:32 Cut off the ropes; to let the boat fall into the sea and float away, so that the sailors could not escape.>
<27:33 Taken nothing; no regular meals, or very little.>
<27:35 The goodness of God should be felt and acknowledged in all our blessings; and when about to partake of the bounties of Providence, we should thank him for them, and ask him to make them the means of our good.>
<27:38 Cast out the wheat; to lighten the ship, and get it as near the shore as possible.>
<27:39 Creek; bay, as the original word means. Shore; one where they could land.>
<27:40 Taken up; slipped or cut the ropes which fastened the anchors to the ship, so that the wind might drive her into the bay, now called St. Paul's bay. Loosed the rudder-bands; the rudder had been made fast during the storm. Now it was loosed, that they might again use it to steer the vessel.>
<27:41 Two seas met; and formed a sand-bar or bank, stretching out into the sea. Stuck-fast; in the sand, so that they could get no nearer to the shore.>
<27:42 To kill the prisoners; those whom they were taking to Rome for trial, lest the soldiers to whose care they had been committed should be punished for letting them go. Soldiers, accustomed to killing men, are apt to think little of the value and sacredness of human life. Fighting is adapted to harden men's hearts, and to nourish and strengthen those feelings which, if continued, will shut them for ever out of heaven.>
<27:43 The centurion; Julius. Verses Ac 27:1,3. Willing; wishing to save Paul. Thus was Paul made the means of again saving the prisoners from death. Verses Ac 27:24,31. It is often a great blessing to wicked men to have a Christian among them. For his sake they may be saved from death, and also in answer to his prayers, through the abounding grace of God, from endless perdition.>
<28:1 Melita; an island about sixty miles south of Sicily, now called Malta. It is a little north of west from the island of Clauda. See note to chap Ac 27:16.>
<28:2 The barbarous people; a term applied to the islanders as not speaking the Greek language. Compare Ro 1:14.>
<28:3 A viper; a poisonous serpent. Mt 3:7.>
<28:4 Vengeance suffereth not to live; that divine vengeance which even heathen persons believe to follow evil-doers. The sentiment that murderers deserve themselves to die, and that justice requires them to be put to death, is not only a dictate of revelation, but seems to be graven upon the hearts of all men.>
<28:6 Swollen; from the effect of poison.>
<28:11 Whose sign was Castor and Pollux; having on its prow painted or carved figures of Castor and Pollux, two heathen divinities, who were supposed to watch over sailors.>
<28:12 Syracuse; a city in the south-eastern part of Sicily. It lay on the way from Malta to Rome.>
<28:13 Fetched a compass; sailed in a winding course; either because they followed the irregularities of the coast, or because they were compelled to beat against a head wind. Rhegium; a city near the south-west extremity of Italy, in the present kingdom of Naples. The south wind blew; which was a favorable wind, as they were sailing north. Puteoli; north of Rhegium, towards Rome. It was about eight miles from the modern city of Naples.>
<28:15 The brethren; Christians at Rome. Appii-forum; a town about forty-three miles south of Rome. The Three Taverns; ten miles further towards Rome. Whom; the brethren from Rome, a part of whom met Paul at Appii-forum, and a part at The Three Taverns. The presence of Christian friends, especially in time of trouble, is delightful. Their countenance and support afford encouragement in duty, and the blessings which come through them should awaken new gratitude, and cause the offering of new thanksgiving to God.>
<28:16 With a soldier; to whom he is supposed to have been chained.>
<28:19 Against it; against Paul's being set at liberty. I was constrained; induced by a suitable regard to his safety, knowing that the Jews intended to kill him. Chap Ac 23:16; 25:11.>
<28:20 The hope of Israel; the Messiah. All proper efforts should be made to communicate to men correct information, and prevent their becoming so prejudiced as to hinder them from candidly hearing the truth and cordially embracing it.>
<28:22 This sect; Christians. The fact that some persons are very unpopular, and that many speak against them, is no certain evidence that they are wrong. This opposition may arise from the fact, that the prevalence of their doctrines and practices would interfere with the selfishness, pride, indolence, covetousness, and other vices of their opposers.>
<28:23 Expounded and testified; explained to them the meaning of the predictions of the Messiah, in the Old Testament, and showed that they were filled in Jesus Christ.>
<28:24 The same divine truths, presented by the same speaker, are treated by different men in a totally different manner. Some receive and treat them as truths; others reject, and treat them as errors. It is not enough, therefore, that men hear these truths, and the evidences which support them; they must also, by the Holy Spirit, be led to believe, or they will reject them. Hence ministers, while they preach to men, should also pray to God that his truth may be attended with his power, and be not only heard and understood, but also believed and obeyed, and thus be the means of eternal life.>
<28:25 Well spake the Holy Ghost; he spoke the truth. Esaias; Isa 6:9,10; Mt 13:14; Joh 12:39,40.>
<28:28 The salvation of God; the gospel, which makes known his salvation and the way to obtain it. Chap Ac 13:46.>
<28:29 Great reasoning; about what Paul had said to them.>
<28:30 Two whole years; during that time he was kept as a prisoner, preaching the gospel to such as visited him, and writing it as he had opportunity to others.>
<28:31 Preaching the kingdom of God; making known the gospel, and urging men to embrace it. We are very incompetent judges as to the time, place, and condition in which we may be most useful. If Paul, during the two years of his confinement as a prisoner at Rome, not only preached the gospel to all who came to him, but as has been supposed, also wrote the epistles to the Ephesians, Philippians, Colossians, to Timothy and Philemon, and to the Hebrews, he may thus already have done more good than he could have done by being at liberty, and preaching the gospel to all who would hear him during his whole life.>
\kniha{Romans}
\zkratka{Rom}
<1:1 A servant of Jesus Christ, called to be an apostle; he first places himself with the whole body of believers as "a servant of Jesus Christ," and then, in accordance with his usual custom, asserts his apostolic calling; for when he writes to a church he wishes to do so with the authority of an apostle--one specially chosen and sent out by Christ himself, to preach his gospel, work miracles, gather churches, and extend his kingdom among men. Separated; set apart by God for this work. Ga 1:15.>
<1:2 Which he had promised afore; he is careful to show at the outset that the gospel is no new religion, but the fulfilment of the promises made in the Old Testament to the fathers.>
<1:3 The seed of David; a descendant of David. According to the flesh; as to his human nature.>
<1:4 Declared--with power; powerfully, conclusively manifested to be The Son of God--according to the Spirit of holiness; as to his divine nature. The words, "according to the Spirit of holiness," stand in contrast with the words, "according to the flesh," and seem to denote the divine Spirit of Christ, which was from eternity, and became mysteriously united with "the man Christ Jesus." To this divine nature holiness is ascribed as an essential attribute of deity. By the resurrection from the dead; the resurrection of Christ was the crowning seal which God set to the claim of Jesus of Nazareth to be the Son of God in the high and incommunicable sense of having equality with God. Christ has a twofold nature, human and divine. He is both God and man. Of this, God has given abundant and conclusive evidence, which no man can reject without great guilt.>
<1:5 By whom; Jesus Christ. Grace and apostleship; the office of apostle, with that special grace which qualifies us to discharge its duties aright. Eph 3:8. In using the word "we," he joins himself with the other apostles. For obedience; that men of all nations might be led to obey Christ.>
<1:7 Grace--peace; in this apostolic prayer Jesus Christ is joined with the Father as the source from which grace and peace flow; which could not be, were he not equal with the Father in power and glory. Grace is the favor of God bestowed on men through Jesus Christ, and peace is its effect. Grace and peace, with all their blessings for this life and the future, come from the Father and the Son. For them men are indebted to both the Father and the Son; and to both should give all honor and glory. Re 5:13.>
<1:10 To come unto you; for the apostle had not yet been in Rome.>
<1:11 Spiritual gift; in the widest sense, including all that spiritual edification that comes from the Holy Spirit through the communication of the truth. Established; in the faith and practice of the gospel.>
<1:12 That I may be comforted together with you; lest he should seem arrogant in making himself a mere giver of spiritual good, he explains that he means the mutual edification of himself and the Roman Christians by their mutual intercourse. Christian intercourse is earnestly desired by Christian hearts, and is to those who are favored with it, a means of increasing excellence, usefulness, and enjoyment.>
<1:13 Let; hindered. Some fruit; be the means of good in Rome, as he had been in other places.>
<1:14 I am debtor; he was under obligation in consequence of what Christ had done for him. Greeks and Barbarians--wise and unwise; polished and rude, learned and ignorant. When Christ imparts to any one the blessings of his grace, it lays him under peculiar obligations to do good as he has opportunity; especially to promote the spiritual good of all his fellow-men.>
<1:16 It is the power of God; that through which he exerts his saving power on all who believe and obey it. Jew first; the gospel was first preached to the Jews, then to the Gentiles. As the gospel is the means by which God exerts on men his saving power, it should be preached to all people; and as neither the power, the love, nor the grace of God will ever save any who reject it, all who hear should without delay believe, that it may be the power of God to their salvation.>
<1:17 Therein is the righteousness of God revealed from faith to faith; in the original the words "from faith" are the same that are often elsewhere rendered "of faith," chap Ro 4:16; 10:6; Ga 3:7,9,12; and they may be so rendered here. This will give the following meaning: In it is revealed the righteousness of God; a righteousness which is of faith, and which is given to faith. The righteousness of God is here, as often elsewhere in Paul's writings, not God's personal righteousness, but the righteousness which he gives to sinners through their faith in Christ; in other words, it is his justifying grace, by which he freely pardons their sins, and accepts and treats them as righteous for Christ's sake. This righteousness is said to be "of faith," in contrast with that which is "of the law," chap Ro 10:5, such as the holy angels have, and such as the Jews vainly sought to obtain by observing the precepts of the Mosaic law. Chap Ro 10:3; Php 3:9. The apostle adds that this ighteousness which is "of faith" is also "to faith," since it must be received and appropriated by each one's personal faith. Shall live by faith; Hab 2:4. What the prophet says of faith, in the general sense of confidence in God and his word, the apostle rightly applies to faith in Christ; since all true faith is, in its essence, the same.>
<1:18 For the wrath of God is revealed; the word "for" connects this verse immediately with the preceding, as much as to say, There is need of such a righteousness as the gospel reveals, for the wrath of God is revealed from heaven, etc. He then proceeds to show, in the remainder of the chapter, how the Gentiles lie under this wrath; and in the following, how it rests on the Jews also; so that all men need to receive from God a righteousness which is not of law, but of faith. Ungodliness and unrighteousness; sins against God and men. Hold the truth in unrighteousness; prevent, by their wickedness, its proper effect.>
<1:19 That which may be known; the character of God as manifested in his works. God hath showed it; in creation and providence.>
<1:20 From the creation; ever since the creation. His eternal power and Godhead; his divinity, and worthiness of being loved, adored, and obeyed. Without excuse; having no reason for disobeying him. All to whom God has manifested himself in creation and providence, who do not worship him and are not thankful for the blessings which they receive, are without excuse, and have just reason to fear his awful displeasure.>
<1:21 Knew God; knew so much of him as to know that they ought to worship and serve him. Glorified him not; did not honor and obey him as God. Vain in their imaginations; senseless and wicked in their thoughts and reasonings about the proper object of worship. Foolish heart; their perverse, wicked mind. Was darkened; blinded as to the spiritual nature and perfections of God.>
<1:22 Professing themselves to be wise; pretending to great wisdom. Became fools; exhibited the greatest folly.>
<1:23 Changed; exchanged the one only living and true God for images of birds, beasts, and reptiles. The doing of what persons know to be wrong blinds their minds, hardens their hearts, and makes them more wicked than they were before. As a punishment for their sins, God often suffers them to commit other sins, and still others, until they bring upon themselves aggravated destruction.>
<1:24 God also; as a punishment for their sins in thus dishonoring him, abandoned them to the dominion of corrupt desires, appetites, and passions; and suffered them to commit the vilest abominations to their ruin.>
<1:25 Changed the truth of God into a lie; the true God for an idol, or false god. More than; instead of. Blessed for ever; worthy of eternal love and praise.>
<1:26 This cause; because of their wickedness in not worshipping him and in worshipping idols.>
<1:28 Reprobate mind; a mind abhorred of God, and upon which his curse rests. Not convenient; not fit or proper, a disgrace to human nature.>
<1:32 Knowing the judgment of God; their desert of his wrath. Worthy of death; justly exposed to it. Do the same; commit the crimes mentioned. Have pleasure; are pleased with others who commit them and encourage them in their crimes. The history of the world in all ages shows, that all means to overcome human depravity without the gospel of Christ, or to remove its evils without faith in him, will be unavailing. Philanthropists, therefore, and friends of external morality as well as of internal godliness should unite in making known Jesus Christ as soon as possible to every human being.>
<2:1 That judgest; that condemnest others on account of their sins. The apostle has in mind the Jews especially. Doest the same; committest similar sins. Men often practise what they condemn in others, without considering that in so doing they condemn themselves.>
<2:4 Or despisest thou; they who take occasion from God's long-suffering to go boldly on in sin, throw contempt upon his goodness. Not knowing; it is a willful and guilty ignorance, for it has its ground in forgetfulness of God. When the goodness, patience, and long-suffering of God encourage men in sin instead of leading them to forsake it, it is fearful evidence that they are ripening for ruin.>
<2:6 To every man according to his deeds; when the question is, What is the ground on which sinners, who have broken God's law, can be justified? Paul always answers, By faith, and not by the deeds of the law. Chap Ro 3:28; Ga 2:16, etc. But when the question is, What character will God accept? he answers with James, "Not the hearers of the law are just before God, but the doers of the law shall be justified:" verse Ro 2:13, compared with Jas 1:22-25; 2:14-26. True faith in Christ always makes men such "doers of the law." The faith that is without works is dead, Jas 2:26, and will be disowned by Christ at the last day. Mt 7:21-27.>
<2:8 Them that are contentious; who contend against the truth, rebel against God, and do what they know to be wrong.>
<2:9 The Jew first; especially to him, on account of his abuse of superior light and privileges.>
<2:10 The Jew first; on account of his wise improvement of his peculiar blessings. Mt 25:21; Lu 19:17. Great advantages, if rightly improved, will be the means of increasing future blessedness; if neglected and abused, or increasing future woe.>
<2:11 No respect of persons; God will not treat men according to their color, country, or outward condition, but according to their character and conduct.>
<2:12 Sinned without law; without a written law or revelation of duty. Perish without law; without being condemned for rejecting or sinning against a revelation which they never had. In the law; in possession of a written revelation. By the law; according to the revelation which they had.>
<2:13 Hearers of the law; those who have a written revelation of the will of God, and know their duty. Doers of the Law; those who do their duty.>
<2:14 Not the law; the written law. Things contained in the law; such things as the law requires. These; such as have not the light of revelation--the heathen. Are a law; they have a sense of moral right and wrong, arising from the moral nature or conscience which God has given them.>
<2:15 The work of the law; its effect in producing a conviction of duty, and of guilt in not doing it. Accusing--excusing; as they have done or not done what they thought to be right. No man will be condemned for want of light, or for violating a law which he never had; but for neglecting the light which God gave him, and doing what he knew to be wrong.>
<2:16 According to my gospel; the judgment of the last day will be by Jesus Christ, as is revealed in the gospel which Paul preached. This verse is connected in sense with the twelfth; what intervenes is a parenthesis.>
<2:17 A Jew; a friend of God--one of his peculiar people. Restest in the law; dependest upon the possession and external observance of it for salvation. Makest thy boast of God; of having him for thy God, while the Gentiles had other gods. Belonging outwardly to the true church and attending upon its ordinances, is no certain evidence of true religion, of the favor of God, or preparation for heaven.>
<2:18 Out of the law; out of the Scriptures.>
<2:21 Teachest thou not thyself? to practise what thou teachest.>
<2:22 Commit sacrilege; by profaning divine things, and taking to thyself what belongs to God.>
<2:24 Is blasphemed; spoken against and dishonored. Through you; on account of the wicked conduct of the Jews as God's professed people, as it was of old. Isa 52:5; Eze 36:23. Members of the church who live in immorality, greatly dishonor God, increase the wickedness of men, and prepare for an awfully aggravated destruction.>
<2:25 Circumcision; this was one of the rites in which they gloried; and it was useful if, by directing their thoughts to that inward purity and consecration to God which it signified, it led them more faithfully to obey God. If it did not, it did them no good; they would be treated no better than if they had not been circumcised. The observance of ordinances is useful if it leads men more faithfully to obey God; if it does not, it does them no good; and dependence upon such observances for salvation if continued, will ruin them.>
<2:26 The uncircumcision; those who have not been circumcised. Keep the righteousness of the law; do from the light of nature the things which the law requires, verse Ro 2:14.>
<2:27 Uncircumcision--by nature; that is, the Gentile, who remains as he was born, uncircumcised. Judge; condemn. By the letter; with a written revelation.>
<2:28 Is not a Jew; in the spiritual sense, that of being a friend of God. Outwardly; by outward descent from Abraham. Neither is that circumcision; in the spiritual sense; that which God approves and will reward as obedience to him.>
<2:29 Inwardly; in heart devoted to God, as was Abraham. Compare our Saviour's words: "If ye were Abraham's children, ye would do the works of Abraham." Joh 8:39. Of the heart, in the spirit; circumcision denoted the necessity of an inward, spiritual change, a real cutting off of sin, and the practice of holiness. This God will reward, not the mere outward observance. Not of men; men look on the outward appearance, and often expect reward for what is only external. Of God; he looks on the heart; and to be accepted of him, whatever men do they must do heartily as unto the Lord. His love must reign in their hearts, and his will govern their lives. Ordinances are designed to promote holiness of heart; if they do not, their object is not accomplished. However strict men may be in the outward observance, or however much praise they may receive for it from men, they are not approved and will not be accepted of God.>
<3:1 In the first part of this chapter the apostle meets various objections which might naturally arise in the mind of an unbelieving Jew to the doctrine which he has established in the preceding chapter, that the outward relation of the Jews to Abraham and their outward privileges cannot save them, but that God will deal with them, as with the Gentiles, according to their works. Verses Ro 3:1-8. He then returns to his great theme, that since Jews and Gentiles are alike under sin, they need alike righteousness of God which is of faith, not of works. What advantage; has the Jew above the Gentile, if both are sinners under condemnation, and neither can be justified or accepted of God on account of his works?>
<3:2 Oracles of God; the Scriptures, revealing salvation through a Saviour to come. As the Scriptures are the voice of God, making known his will and the way in which men can be accepted of him, those who possess them have blessings much greater than those who do not. Hence they should be given to all, and all should be taught to read and obey them.>
<3:3 What if some did not believe? had not faith in God, and as a consequence of their unbelief were unfaithful to God; for both these ideas are included in the original word. Shall their unbelief; their unbelief and unfaithfulness to God's covenant with them, but which, as the apostle has taught, they lost its benefits and brought upon themselves the wrath of God, "who will render to every man according to his deeds." Chap Ro 2:6. Make the faith of God without effect? annul God's faithfulness in fulfilling the terms of his covenant with Abraham and his seed? The unbelieving Jews thought that God's covenant with their fathers bound him to bestow upon them eternal life, irrespective of their own conduct, and that a failure to do this would be a violation of the divine faith. The apostle, having shown that circumcision and the other privileges of the covenant can profit only those who are faithful to its conditions, and that the unfaithful Jew will be condemned along with the Gentiles, rejects with horror the idea that this is an annulling of the divine faithfulness.>
<3:4 Let God be true; God is true, and all that deny it are false. This should always be admitted. As it is written; Ps 51:4. Justified--overcome; seen to be just and right when complained of, and in all that he does. The apostle, as often elsewhere, follows the rendering of the Seventy.>
<3:5 If our unrighteousness commend the righteousness of God; if our sins are made the occasion of showing the truth and justice of God, and are thus overruled for the display of his glory, is it not wrong for him to punish us? I speak as a man; as a short-sighted erring man might speak.>
<3:6 God forbid; certainly not: if it were, God would not be just or right in punishing any one; for the sins of all are in some way overruled for the display of divine perfection, and the advancement of divine glory. This, however, does not alter the evil nature and tendency of sin, nor lessen the guilt of him who commits it. The fact that God takes occasion with regard to the sins of men, to display his perfections, does not alter the evil nature of sin, or lessen the guilt or danger of those who commit it.>
<3:7 Through my lie; if when I am false, God shows his truth in punishing me as he has declared, and thus glorifies himself, why am I to blame? Because you were false; you felt wrong, and did wrong. When a man commits murder and is hung, the government is made a greater terror to evil-doers, and every man's life is rendered more safe; why is the murderer then to blame? Because he committed murder. He meant it unto evil, and it was evil. Though God, through his ordinance of civil government, punishes him and thus promotes the good of the community, that does not alter the nature of his crime, or the propriety of punishing him; the government had told him before that they would do it. Truth, therefore, as well as justice and the public good, required his execution.>
<3:8 And not rather; and why should we not rather say, if we carry out the false principle of the objector, Let us do evil, that good may come? as evil is overruled for good, why not commit it to accomplish that good? as some say is right, and affirm is taught in the Scriptures. Because it is wicked, and renders all who do it deserving of damnation. It is not the evil that does the good; but it is the counteracting and overruling of evil, and the treating of the evil-doer as he deserves, that does the good. Those who do evil for the purpose of accomplishing what they call good, or break the law of God professedly to honor him, will be justly condemned and awfully punished.>
<3:9 We; Jews. They; Gentiles. Better; in condition as to the way of justification. Can Jews be justified in any other way than Gentiles? In no wise; certainly not; because both are sinners, and if saved it must be not by their own works, but by believing in Christ. Thus the apostle returns to his great theme, that Jews as well as Gentiles need the righteousness of God which is by faith, as revealed in the gospel. All under sin; all in a state of guilt and condemnation as sinners. The quotations that follow are taken from various parts of the Old Testament.>
<3:10 As it is written; Ps 14:1-3; Ps 53:1-3.>
<3:11 None that understandeth; naturally aright the true character of God, or the blessedness of serving him. None that seeketh after God; as the chief good.>
<3:12 Out of the way; the way of truth, duty, and blessedness. Unprofitable; corrupt, worthless. Ho 10:1. There is none that doeth good; none naturally glorify God or do right.>
<3:13 Their throat is an open sepulchre; ready to swallow up and consume, as the grave did the body laid in it. Ps 5:9. The poison of asps; their words are destructive. Ps 140:3.>
<3:16 In their ways; they cause misery and ruin.>
<3:17 The way of peace; of holiness and blessedness to themselves and others.>
<3:18 No fear of God; none which leads them to love and obey him, or keeps them from breaking his laws. Ps 36:1. This is the account given of Jews who were blessed with the Scriptures and all the means of grace.>
<3:19 We know; are certain that this description given in the law, or the Bible, concerning men, applies to those who are under the law, who have the Bible. Of course it describes the natural character and state of Jews as well as Gentiles. May become guilty; shown or proved from their own conduct, and from the Bible, to be guilty and deserving of condemnation. The description of the natural character of man which God gives in the Bible, applies to all men. It is a description of the human race, and shows that all men are sinners, guilty, and justly condemned; and that if saved, it must be not on account of their works or worthiness, but on account of the works and worthiness of Christ.>
<3:20 By the deeds of the law; their own works in obedience to law. No flesh; no individual of the human race. Be justified; accepted of God or treated as righteous. The knowledge of sin; when compared with or tried by the law of God, men are shown to be sinners, shut up under righteous condemnation, without the possibility, on the ground of their own works, of ever being saved.>
<3:21 The righteousness of God; that which he has provided in and by his Son Jesus Christ, and which he freely gives to sinners upon condition of faith in Christ. See note to chap Ro 1:17. Without the law; which justifies men not on the ground that they have rendered to the law the obedience which it requires, but through faith in Christ. But it must be carefully remembered that this faith produces true obedience to God's law. See note to chap Ro 2:6. Is manifested; clearly revealed in the gospel. Being witnessed; having been referred to, foretold, and described in the Old Testament. Ge 3:15; 12:3; 15:6; De 18:15,19; Ps 51:14; 71:15,16; 85:10,13; 89:16; Ps 119:142; Isa 43:21; 45:5,24,25; 46:13; 51:5,7; 53:11; 54:17; 56:1; Isa 61:11; 62:1,2; Da 9:24; Ho 10:12; Hab 2:4; Mal 4:2. The way of salvation revealed in the Old Testament was the same which is revealed in the New. The revelation was not so clear and full, and it was addressed more to the outward senses: but in both, the salvation revealed is of grace, not of debt; obtained not by works, but by faith; and given not on account of human merits, but the merits of Christ.>
<3:22 By faith of Jesus Christ; the benefits of whose obedience and death are obtained not by human works or merit, but by receiving him as a Saviour, and trusting in him for salvation. No difference; between Jews and Gentiles, as to the way of salvation.>
<3:24 Freely by his grace; it is wholly of grace, not of debt, that men are saved.>
<3:25 Set forth; exhibited. Propitiation; propitiatory sacrifice. Through faith in his blood; that the propitiatory sacrifice of Christ may benefit a man, he must appropriate it to himself through faith in Christ's blood; for it was by the shedding of his blood that the propitiation was made. Declare his righteousness; show that he is righteous, in the forgiveness of sinners who believe on Christ. Sins that are past; committed in past times, and which God forbore to punish. The influence of Christ's atonement extends backward to the first believer, and forward to the end of time. From Abel to the trump of the archangel, all who are justified and saved receive this gift through the blood of Christ.>
<3:26 At this time; the time in which Paul lived, under the gospel dispensation. Just, and the justifier; that is, just while at the same time he is the justifier. These words set forth the only possible condition on which God can forgive sin. In doing so, he must be just to himself, his truth, his law, and the interests of his kingdom.>
<3:27 Where is boasting; in this way of saving sinners, what ground is there for them to be vain of their own merit or worthiness? None at all. It is excluded; it is not for their sakes, but for Christ's sake, that God pardons, accepts, and saves them. Not to them, but to him be all the glory. By what law? in what way is their boasting excluded? By their being saved through their own works? No; but by their being saved in God's way, by grace, through faith in Jesus Christ, called here the law of faith.>
<3:28 Therefore; in view of the whole subject and all the light that is thrown upon it. Without the deeds of the law; man's obedience to law is not the ground of his justification, but the merits of Christ.>
<3:29 Jews--Gentiles; he will be the God and Saviour of both--of all classes and all nations to whom Christ is made known, on the same condition--faith in his Son. Jehovah is the Creator, Preserver, and Benefactor of all, Jew and Gentile, high and low, rich and poor, bond and free. When they believe in Christ, he accepts them with equal readiness, adopts them into his family as his own children, and loves them with equal affection. He imprints on them his own blessed image, and as they treat each other so he regards them as treating him.>
<3:30 Circumcision--uncircumcision; Jews and Gentiles. By faith; by a righteousness which is of faith, not of law. Through faith; by means of their faith.>
<3:31 Make void the law; the law of God, as a rule of action, and sacredly binding on all who know it. Does the fact that God saves sinners through faith in Christ, lessen the sanctity and authority of his law as an expression of his will, or the obligations of men to obey it? By no means. We establish the law; show its excellence, its unchanging obligations, and lead men more earnestly, successfully, and perseveringly to strive to obey it. The way of saving sinners through the incarnation, obedience, suffering, death, resurrection, and intercession of Christ, and by faith in him, shows that the law of God is holy, just, and good; that the violation of it is unspeakably wicked; and that it cannot be violated with impunity; while the motives for obeying it in order to honor God, to show gratitude to the Redeemer, and become in heart and life like Him who was a living personification of its excellence, are greatly increased: such love and obedience are secured as never were, and never will be, secured among men in any other way. In perfectly obeying the divine law, Christ was a pattern of human perfection, which all who believe in him supremely desire and habitually strive to copy; saying from the heart, each for himself, "Such love, and meekness so divine, I would transcribe and make them mine. Be thou my pattern, make me bear More of thy gracious image here; Then God the Judge shall own my name Among the followers of the Lamb.">
<4:1 As pertaining to the flesh; in the way of the outward ordinances and works of law. These words should be connected with the following, hath found; that is, found as an advantage or cause of boasting. The answer, which the apostle omits, is, He hath found nothing. And this he proceeds to show.>
<4:2 2, 3. He hath whereof to glory; if his works are the meritorious ground of his justification, he is saved of debt, not of grace. He might glory in his works as the ground of his salvation, and take to himself the praise. But not before God; that is, but he has not before God any thing whereof to glory. It follows that he was not justified by works. And this agrees with the word of God. For what saith the scripture? see Ge 15:6. It; his belief. Was counted unto him for righteousness; was the ground of his being accepted as righteous.>
<4:4 That worketh; so as to be saved on the ground of his own merit. The reward; his salvation. Not of grace, but of debt; if, in obedience to law, a person is justified, his salvation is merited, not bestowed as a gratuitous favor.>
<4:5 That worketh not; who does not depend on his works for justification. The ungodly; sinners who believe in Christ. His faith; is the means of his justification and salvation, through the atonement and righteousness of Christ.>
<4:6 David; Ps 32:1,2. Imputeth righteousness; accepts and treats as righteous, though he is a sinner. Saints under the Old Testament were saved in the same way as saints under the New: not on account of their own works, but on account of Christ, and through faith in him.>
<4:7 Sins are covered; not punished, but forgiven.>
<4:8 Not impute sin; not charge it upon him, or inflict the suffering threatened against those who commit it.>
<4:9 This blessedness; the blessedness of having sin forgiven, being accepted of God, and rewarded as righteous. The circumcision; those only who are circumcised. Uncircumcision; upon those also who are not circumcised.>
<4:10 Not in circumcision; not after he was circumcised. In uncircumcision; before he was circumcised.>
<4:11 A seal of the righteousness of the faith; a token, or visible sign, that by means of the faith which he exercised before he was circumcised, he was justified and accepted with God. The father; the model or pattern as to the way of acceptance with God, for all who should believe, though not descendants of Abraham, and not circumcised: to encourage them to exercise such faith as he did, that they also might be justified, and through grace be delivered from the punishment of sin and rewarded with eternal bliss. It is dangerous to put the sign for the thing signified, or make the one a substitute for the other. Those who depend on the sign are destitute of the thing signified; and so long as they continue to do it will remain destitute. Glorying in the shadow, they lose the substance.>
<4:12 The father of circumcision; of his natural descendants who were circumcised, provided they exercised faith in Christ.>
<4:13 Heir of the world; Ge 12:2,3; 15:5,6; Ge 17:4-8; Ga 3:6-9,14,16-18,29. Not--through the law; not on the ground of obedience to the law, or through the merit of human works, but through the righteousness bestowed upon him by God through faith. Verse Ro 4:3. The way of salvation through faith in Christ is suited to all classes and conditions of men. None are so good that they can be saved in any other way; and none are so bad that they cannot be saved in this.>
<4:14 They which are of the law; those who seek justification by their own works. Be heirs; if they are by their own merits entitled to the blessings which God promised to Abraham. Faith is made void; is not needful. The promise; which God made to faith. Of none effect; useless. To connect this with the following verse, supply in thought, But the promise cannot be through the law; "because," etc.>
<4:15 Because the law worketh wrath; that is, this is its effect upon fallen sinful men. It lays God's authority upon their consciences, without furnishing the grace needful to enable them to overcome their corrupt passions. Instead of making them holy, therefore, and fit for heaven, it works wrath in two ways: first, by laying duty upon them which they do not perform, it becomes the occasion of provoking against them the divine wrath; secondly, in the same way it fills their minds with a sense of guilt and fearful apprehension of wrath to come. Where no law is, there is no transgression; were it possible that one should be absolutely without law, he could be guilty of no transgression; and the less clearly the divine law is revealed, the less does it operate to work wrath. Instead of saving those who have violated it, and yet seek to be justified by it, the law condemns them. As all men have violated it, none can be saved by it. If the promises were made only to those who should perfectly obey it, all would fail of the blessing. See note to chap Ro 5:20.>
<4:16 Therefore it is of faith; the promise of justification and salvation made to Abraham and his seed, that Jehovah would be a God to him and his seed, Ge 17:2-7, and referred to in Ga 3:29, was not made on condition of perfect obedience to law, or on the ground of human merit, but of grace through Jesus Christ to all who should believe. Sure to all the seed; that all who should in faith imitate Abraham, might obtain the blessing promised to him and his seed, of having Jehovah for their God and portion. Not to that only which is of the law; not to Jews only, but to Gentiles also--to all who believe. The father of us all; all of every nation who exercise faith in Christ.>
<4:17 As it is written; Ge 17:5. Before him; in his sight, and according to his promise. Quickeneth; giveth life to. Things which be not; which have not taken place. Though they may appear to men impossible, he speaks of them as if they were already accomplished, and thus shows their certainty.>
<4:18 Against hope; against all human expectation, or apparent possibility, In hope; that the things promised would certainly take place. The father; an illustrious pattern of faith, for the imitation of all who should believe.>
<4:19 Dead--deadness; as to what was promised, they being at a time of life when it would not be according to the ordinary course of nature.>
<4:20 He staggered not; he did not let his advanced age, or that of his wife, prevent him from believing that they should have a son and receive the blessings which God had promised. Giving glory to God; by the manifestation of strong faith in him. We should never doubt the truth of what God has declared, on account of any difficulties in the way of its fulfilment; but should expect its fulfilment as certainly as if there were no obstacles in its way. Isa 40:8; 46:10; Lu 21:23.>
<4:22 It; his unwavering confidence in God. Was imputed to him; as the means of his being accepted of God and graciously treated as righteous.>
<4:23 That it; that his faith was imputed to him for righteousness. What is written in the Scriptures was written for the instruction of men, not only of that age but of all ages. They are given by inspiration, and are all profitable for doctrine, reproof, correction, and instruction in righteousness. They should therefore be studied by all who have them; and should be sent to all the destitute, that they may be led to believe on Christ, and thus obtain eternal life.>
<4:24 It shall be imputed; if we possess and manifest faith similar to that of Abraham, our faith shall be imputed to us for righteousness, as his was to him. This account of Abraham was transmitted to us to induce us, by exercising similar faith, to become his spiritual seed, and heirs to the eternal blessings promised to him.>
<4:25 Was delivered for our offences; delivered to death on account of our sins. For our justification; in which is implied the resurrection of our bodies, and our admission, complete in soul and body, to the enjoyment of eternal life in heaven. Both the death of Christ and his resurrection were necessary to complete the work of our redemption. But the apostle naturally ascribes to the former the expiation of sin; to the latter, our introduction, through the justifying grace of God, to a new divine life in holy communion with him.>
<5:1 Have peace with God; are reconciled to him, and in a state of favor with him. Faith in Christ makes a great and blessed change in the state, character, condition, enjoyments, and prospects of men.>
<5:2 Into this grace; into this gracious state of peace and love. The glory of God; that glory which he has promised and will bestow upon his believing people.>
<5:3 Tribulations; trials--not because they are pleasant, but because they are useful. Patience; in the old sense of endurance--the quality of bearing suffering with calmness and unwavering fortitude.>
<5:4 Experience; also in the old Latin sense of trial, and then proof, tried integrity which comes from trial rightly endured, and is the object of God's approval. The same Greek word is used in Php 2:22, where our version renders it, "proof." Hope; the confident "hope of the glory of God," verse Ro 5:2.>
<5:5 Maketh not ashamed; it will not be disappointed--the glory hoped for will be realized. The love of God is shed abroad; the sweet sense of God's love towards us, which is always accompanied by the exercise of our love towards him. Both are caused in us by the Holy Ghost, and are a sure earnest of eternal life. Php 1:6. The love of God reigning in the heart is a sure evidence of having received the Holy Spirit, and under his influence, of being in a course of preparation for heaven.>
<5:6 Without strength; were wicked, lost, and destitute of resources to save ourselves, or provide for our own salvation. In due time; at the proper time in God's estimation--the right time. Died for the ungodly; in their stead, that they, by believing in him, might live for ever.>
<5:7 A righteous man; just, upright, and honest. A good man; not only just, but kind, compassionate, and governed by love to God and men.>
<5:8 God commendeth his love; shows it to be unspeakably greater, more disinterested, and abundant. Sinners; enemies to him, and deserving his displeasure.>
<5:9 Being now justified by his blood; the argument is from the less to the greater: If while we were yet enemies to God an expiation was made for our sins, much more, now that through that expiation we have been brought into an actual state of justification, shall we be saved from God's wrath.>
<5:10 We were reconciled; not personally and actually, for the apostle is speaking, as in verse Ro 5:8, of the expiatory death of Christ. He means, then, that a way of reconciliation was opened to us by death of Christ. Being reconciled; that is, personally and actually, through faith in Christ's expiatory death. We shall be saved by his life; both the death and resurrection of Christ are necessary to complete the work of our redemption. But here, as in chap Ro 4:25, he ascribes to his death the expiation of our sin, and to his life after his resurrection our actual introduction to a state of justification and eternal life. For Christ lives with all power in heaven and on earth to intercede for his saints and overrule all things for their good. Mt 28:18; Joh 14:19; Ro 8:28-39; Heb 7:25. A change in men from a state of enmity to God manifested by rebelling against him, to a state of love for him manifested by obeying him, is proof that they have passed from death unto life, and that they will be kept by the power of God through faith unto salvation. 1Pe 1:5.>
<5:11 Joy in God; greatly rejoice in his character and will; especially in the gift of his Son and the way of life through him. By whom; Christ. The atonement; reconciliation to God and the enjoyment of his favor.>
<5:12 Wherefore as by one man; that is, Adam. The apostle, in this verse, evidently begins a comparison between Adam and Christ, the same for substance as that contained in verses Ro 5:18,19. But before completing it, he pauses to throw in sundry remarks pertaining to it. Death by sin; as a consequence of sin. And so; as a consequence of sin, death passed upon all; all became subject to it. For that; because.>
<5:13 Until the law; before it was written, or communicated by Moses. Sin was in the world; men committed it, and suffered the consequences; God treated them as sinners. Sin is not imputed; it is not charged to men, or laid to their account; they are not held responsible and punished for it. When there is no law; because sin is a transgression of a wise and good law. It follows that there was such a law binding on men before the time of Moses, and before any written revelation of the will of God was made to men. There was a law given to Adam from the mouth of God, by the violation of which sin entered, and death by sin. There was a law, too, written upon the hearts of all men as moral beings. Chap Ro 2:14,15. Of course there could be, and there was, transgression--violation of law. This was proved by the fact that there was death as universal as after the giving of a written law by Moses.>
<5:14 Similitude; manner or likeness; namely, by violating a positive revealed law. Figure; in the original, type. Adam is the type of Christ, especially in the wide influence exerted by him on the human family. Of him that was to come; the Messiah. This was among the Jews a common mode of designating their expected Messiah. Compare Mt 11:3; Joh 6:14; Joh 11:27.>
<5:15 Not as the offence; having called Adam the type of Christ, it was natural that the apostle should show that there is not a likeness in all respects between Adam and Christ; or between the evil which comes through the one, and the good which comes through the other. In several respects there is a difference; some of which he proceeds to mention. Many be dead; there is a difference in the kind of extent of influence. That of Adam works death; that of Christ brings to all who receive him superabounding grace and life. Hath abounded; hath, to those who embrace Jesus Christ, gone beyond the mere removal of the evil which comes upon them through Adam; giving them good which is more safe, more abundant, more glorious than he or they ever lost, or could in any way, except through faith in him, ever have enjoyed. Joh 10:10. The evil which one offence of Adam brought on him and his posterity, shows in a wonderful manner the evil nature and destructive tendency of sin, and the great guilt and danger of committing it, and should lead all to hate and at once forsake it.>
<5:16 By one that sinned; Adam. There is a difference between the evil which came through Adam, and the good which comes through Christ, in another respect. By one; one offence, by which sin entered. The evil, expressed by the words judgment, death, and condemnation, came through and were made sure by one sin; but the grace of God in Jesus Christ pardons and triumphs over many sins, and bestows an exceeding and eternal weight of glory upon those who have committed numerous offences.>
<5:17 By one man's offence; that of Adam. By one; Adam. Much more; the reasons for saving believers in Christ appear much more numerous and strong than those for subjecting them to sin and death through Adam. As the latter has been done, they may be sure, from the character and word of God, that the former will in due time be accomplished.>
<5:18 Upon all men unto justification; the blessings provided by Christ are sufficient for all; they are offered to all to whom they are revealed; they should be accepted by all; and all who do accept them, as offered in the gospel, will be pardoned, justified, and saved.>
<5:19 As all who believe in Christ will be saved, all to whom he is made known are bound both by duty and interest to believe in him, and thus, through grace, prepare to live and rejoice with him for ever in heaven. If they do not, their destruction will be more dreadful than if they had never heard of him, or he had never come into the world.>
<5:20 The law entered; a written revelation of the will of God was given and embodied in the moral and ceremonial law of the Old Testament. That the offence might abound; as men, after the giving of the written law, had more commands and obligations which they knowingly violated, the number and guilt of their sins was greatly increased. Thus the law, through their opposition to it, and their voluntary disobedience of it, aggravated their condemnation; and was adapted to make them feel that if they were ever saved, it must be by grace, and thus prepare them to believe on Christ. Ga 3:24. Grace did much more abound; it triumphed over all obstacles, and saved those who believed in Christ, notwithstanding their greatly multiplied and aggravated transgressions.>
<5:21 Through righteousness; the righteousness which God gives through faith in Christ, who died for our sins according to the Scriptures, rose for our justification, and ever lives to make intercession for us. 1Co 15:3,4; Heb 7:25.>
<6:1 What shall we say; in view of the foregoing truths, and especially the fact that where sin abounded, grace did much more abound. Shall we continue to live in sin, that grace may the more abound?>
<6:2 God forbid; surely not; for that would be acting not only against the abounding, but against all operations of grace--against what is professed and is most earnestly desired by all true Christians. They have looked to Christ to be delivered not only from the punishment, but from the power of sin. For them, therefore, to continue in it that grace might be displayed in its forgiveness, would be not only wicked but absurd. It would be acting against the great object of their desires and efforts. We, that are dead to sin; that have, from a discovery of its evil and malignant nature, heartily renounced it and separated ourselves from it.>
<6:3 Were baptized into his death; were so united with him as to be the followers of him in his death by dying to sin as he did. See this idea more fully stated in verses Ro 6:10,11. True Christians will never make the fact that they are saved by grace and not by works, nor the fact that the greater and more numerous their sins the more abounding the grace which saves them, an occasion or excuse for continuing in sin.>
<6:4 We also should walk in newness of life; for our death with Christ to sin implies our resurrection with Christ to God, which is to us a new life of holiness. See on verses Ro 6:10,11.>
<6:5 Planted together; that is, as the original word implies, closely united, namely, with Christ. We shall be also; closely united with Christ. Our dying with Christ to sin, implies our rising with Christ to God. Verses Ro 6:10,11.>
<6:6 Our old man; our natural love of sin, and inclination to commit it. Is crucified with him; a repetition of the idea that we die with Christ to sin. The apostle uses the word crucified with reference to the manner of our Lord's death; perhaps also to intimate the lingering and painful nature of the process by which the old man dies, to give place to the new man. The body of sin; the same as "the law of sin which is in my members," chap Ro 7:23, which in the old man controls the body, making it a body of sin and death, chap Ro 7:24.>
<6:7 For he that is dead; that is, as the context shows, he that has died to sin. Compare verse Ro 6:18.>
<6:8 Dead with Christ; in the sense above explained--one with him in sympathy, desire, and effort as to the object of his death, the deliverance of his people from sin. We believe that we shall also live with him; be like him, through communications received from him, in living to God, even as the branch is like the vine. Joh 14:19; Joh 15:5; Heb 7:25.>
<6:10 He died unto sin; in reference to sin, the design of his death being to put away sin. He 9:26. By making expiation for sin he prepared the way for its forgiveness, and thus its removal from the souls of all that believe in him. In that he liveth; liveth in his new resurrection-life. He liveth unto God; his life is devoted to the glory of God in the furtherance of the work of redemption. Before his crucifixion, Christ lived unto God also. But that was a life of humiliation leading to the death of the cross, and may here be reckoned as a part of the process of his dying unto sin. His resurrection-life, on the contrary, is a life of exaltation, in which all power is given into his hands for the glory of the Father, in the over-throw of the kingdom of Satan and the establishment of the kingdom of God in this world.>
<6:11 Likewise reckon ye; be like Christ, in dying to sin and living to God. Dead indeed unto sin; dead in reference to sin, in the sense of putting it away from you, and having no more to do with it. Alive unto God; living a new life of holiness devoted to God's glory, in imitation of Christ's resurrection-life. Through Jesus Christ; by virtue of your union with him through faith. In this and the preceding verse, we have the key to the interpretation of the preceding comparison extended in various forms through verses Ro 6:4-9. Faith in Christ is the means not only of justification, but of sanctification; and produces a change not of state and condition only, but of character and conduct. It leads a person to live not unto himself, but unto Him who died for him and rose again.>
<6:12 Let not sin therefore reign; be not its slaves in being or doing wrong, but be the freemen and willing servants of Christ in being and doing right. In your mortal body; let not the mind be enslaved to, or polluted by the bodily propensities, appetites, or passions. Control and regulate them according to the will of God.>
<6:13 Neither yield ye your members; let not any of your faculties or powers by employed in the service or used as the instruments of sin. Yourselves; body and soul with all your powers employ in the service, and to the glory of God.>
<6:14 Over you; Christians, who have believed in Christ, and are justified by faith. Ye are not under the law; not under a legal dispensation, where perfect obedience to law, and freedom from all sin, are necessary to acceptance with God. The apostle had already shown that the law cannot deliver from either the guilt or the pollution of sin, but "worketh" wrath to all transgressors. Chap Ro 3:20; 4:15. But under grace; a gracious dispensation, under which men are justified, not by perfect obedience, but by faith in Christ, who died to redeem them from the curse of the law, being made a curse for them.>
<6:15 Shall we sin; if they should thus abuse the doctrine of salvation by grace, and take occasion from it to live in known sin, it would show that they loved sin, that they were its slaves; and continuing this course, would reap its wages, eternal death. Ro 8:13; Ga 6:7,8.>
<6:16 Every person daily chooses the service of self and sin, or of Christ and holiness. One leads to life, the other to death. Both, God sets before men, and invites them to choose life by taking the way which leads to it, and promises that if they do they shall live. De 30:19; Jos 24:15.>
<6:17 God be thanked; that they who were the servants of sin had forsaken it, and believed on the Lord Jesus Christ.>
<6:18 Made free; from the slavery of sin. Servants of righteousness; by believing and obeying Christ.>
<6:19 After the manner of men; as much as to say, In calling you the servants of righteousness, I do not mean that you are not truly free, but I use an illustration drawn from a relation with which you are familiar. Because of the infirmity of your flesh; your dulness, on account of your remaining carnality, in rightly apprehending divine truth. As ye have yielded your members; as they had heretofore employed them in the practice of sin, they should hereafter employ them in the practice of holiness. Familiar illustrations drawn by ministers from the common concerns of life with which their hearers are acquainted, are among the best modes of giving them clear conceptions of divine truth, and making a right impression upon their hearts.>
<6:20 When ye were the servants of sin; were wholly devoted to it. Free from righteousness; not in any way under its control--a most miserable freedom, as the apostle proceeds to show.>
<6:21 What fruit had ye; in that shameful, wicked course. Did it do you any good? The end of those things; their tendency, and the result to which when continued they lead. Is death; temporal, spiritual, eternal.>
<6:22 Free from sin; its condemning and reigning power. Servants to God; devoted to him. Fruit unto holiness; its results are increasing holiness, and of course increasing usefulness and happiness. Everlasting life; holiness, and happiness, which shall be perfect and eternal.>
<6:23 The wages of sin; its just desert. Is death; endless sinning and suffering. Eternal life; perfect, endless holiness and bliss. The future misery of the wicked is their just desert; and the future happiness of the righteous is the gracious gift of God, through the merits of Jesus Christ.>
<7:1 The apostle had shown, in chap Ro 4:15, that "the law worketh wrath," and is unable to give justification and salvation. He had further said, in chap Ro 6:14, that believers are not under law, but under grace. This latter idea he proceeds in the present chapter to unfold, in verses Ro 7:1-6; and while he vindicates the law as "holy, and just, and good," he yet shows the impossibility of gaining through it a victory over sin, in verses Ro 7:7-25. He then goes on to show, throughout the whole of the eighth chapter, the blessed state of those who are not under law, but under grace. The law; the Mosaic law, as he proceeds to illustrate.>
<7:2 Loosed from the law of her husband; from the law which, so long as he lived, bound him to her as her husband, and thus bound her to him.>
<7:3 She is free from that law; the release of her husband from it by death, is her release also.>
<7:4 Ye also are become dead to the law; in carrying out the comparison, the apostle necessarily changes its form somewhat. He could not well say that the law, which may be here regarded as their former husband, was dead. Instead of that, he says, Ye are become dead to the law; the essential idea being that the death of either party dissolves the relation existing between them. By the body of Christ; by his crucified body making expiation for your sins. Thus ye are released from the law as a means of justification before God, so that ye are no longer in this respect bound to it, any more than a woman is bound to her husband after he is dead. Thus the way is prepared that ye should be married to another, even Christ; in other words, should come into a state of justification by virtue of your union with Christ through faith. Deliverance from the law of God as a covenant of works, and from the necessity of obeying it as a ground of justification, is essential to the obeying of it as a rule of duty.>
<7:5 In the flesh; in their natural state, with no ground for justification except obedience to law, and under the necessity of perfectly obeying it or suffering its curse. Its strict requirements and its awful threatenings, instead of leading them to love and obey it, were the occasion, through their wickedness, of exciting against it greater hatred and more violent rebellion; thus, in the language of the Holy Ghost, "bringing forth fruit unto death.">
<7:6 We; Christians, who have seen that by the works of the law we cannot be justified, have given up dependence on obedience to it, and are trusting in the atonement and righteousness of Christ for salvation. Are delivered from the law; not as a just measure of obligation, but as a ground of justification, and from liability to suffer its curse. That being dead; the marginal reading, "being dead to that," is much to be preferred. It is a repetition of the idea that they are dead to the law, as in verse Ro 7:4. That we should serve in newness of spirit; serve God not in external form merely, or from slavish fear, but in spirit and in truth, from love to God and his laws.>
<7:7 Is the law sin? is the law answerable for sin because no one can be justified by it, and because it is made the occasion of increasing the wickedness of those who break it? By no means. Nay; on the contrary, I had not known sin; I had not understood my own exceeding sinfulness, had I not seen myself in the light of the law. By the law was the knowledge of sin: for instance, he had not known lust, the desire of forbidden objects, except the law had said, Thou shalt not covet; not desire what God forbids. As a correct view of the spirituality and extent of the divine law is essential to a right knowledge of one's sins, ministers of the gospel should faithfully preach it, and show its universal and perpetual obligation, that all may understand their true character, renounce dependence on their own works, and rely for salvation on the rich grace of God in Jesus Christ.>
<7:8 Sin; his sinful inclination led him to resist the commandment, and the more to indulge evil desires in opposition to its requirements. Resistance to its restraints increased his wickedness, and showed, beyond what he had before seen, his depravity of heart. Sin was dead; was in a slumbering state, not active and strong.>
<7:9 I was alive; in my own estimation, and thought I was blameless as touching the law. Php 3:6. The commandment; that which extends to all the thoughts and desires of the soul, and requires them to be holy, just, and good. Came; came to be apprehended in its spirituality and extent. Sin revived; rose to view in awful and aggravated increase of power and guilt. I died; as to all hope in myself from the law, or from my obedience to it. I saw that it condemned me, and that judged by it, I was lost.>
<7:10 Ordained to life; to give life to all who should perfectly obey it. Unto death; because I had broken it and fallen under its curse.>
<7:11 For sin; sin reigning in my soul. Taking occasion by the commandment; as Satan in Eden took occasion of the prohibition to eat the fruit of the tree of knowledge. Deceived me; as Satan did Eve, and thus seduced me into disobedience. The apostle has in view the blinding and deceiving nature of sinful passion. And by it slew me; thus turning the commandment into an instrument of my death, as verse Ro 7:10.>
<7:13 That which is good; the good law of God. Made death unto me? Was it the law which caused his ruin? By no means. It was his own wicked violations of it. Sin; this was the cause of his ruin. Working death in me by that which is good; by leading me to resist the law, to sin against greater light and stronger motives, and thus become more sinful: such are the effects of human depravity when left to act itself out under the influence of mere law. The fact that the more clearly men in their natural state see the purity and extent of the law of God, the more strenuously they resist it and thus increase their wickedness, shows most strikingly the hateful nature and desperate tendency of human depravity, and the utter fallace of all hope, from the influence of law merely, of ever removing or lessening it.>
<7:14 The law is spiritual; it requires perfect holiness of spirit; that men should love God with all their heart and soul and mind and strength; and that whatever they do, they should do all to the glory of God. But not even Paul, after his conversion, and after he had been preaching the gospel for years, did all this. So far as he fell short he was carnal, sinful, and needed the grace of God through Jesus Christ. I am carnal; fleshly and earthly in my affections, and thus sold under sin; under its power as a bond-servant. These words describe, first, the state of all unregenerate men; secondly, the condition of believers so far as "the law of the Spirit of life in Christ Jesus" has not made them free from it. In what follows to the end of the chapter, the apostle describes the painful conflict between the spiritual law of God and the carnal mind of man, in the soul of him who is earnestly seeking to render to the law a true inward obedience. What he says applies in a measure to the awakened and convicted sinner, who, without any clear apprehension of Christ's grace, is vainly seeking justification from the works of the law; but more fully to the warfare with sin in the heart of the true Christian; for he is spiritual only in part--not a willing, habitual devotee and slave of sin, but sold as a captive against his prevailing inclinations. He is not delighted or contented with his bondage. It is his grief and burden. He has tasted the beginning of liberty, and sighs and struggles for its completion.>
<7:15 That which I do; in violation of the law of God. I allow not; I do not love it, delight in it, or approve of it. What I would; to obey perfectly the law of God, that do I not. What I hate; to act in violation of it, or in any respect to fail of perfectly obeying it, that I do.>
<7:16 I consent unto the law; by disapproving and hating all violations of it, and condemning myself on account of them, I show that I approve the law as wise, holy, just, and good.>
<7:17 No more I--but sin; it is not my habitual inclination, my prevailing desire, to break the law. I do not love transgression, but abhor it; yet in many things I offend, and in all come short of perfect obedience, through the power of temptation and the strength of my own evil propensities, which are not yet entirely done away. Jas 3:2; 1Joh 1:8.>
<7:18 In me; by nature. In my flesh; my natural heart, as it is under the influence of law merely, without the grace of God. No good thing; nothing spiritually good; even now, under the influences of the Spirit and grace of God, much evil still remains. To will is present; I desire to be completely conformed to the will of God. But how to perform; to do that which would be perfect. I find not; I do not do it; on the contrary, I do as stated in verse Ro 7:15. Therefore it is true as stated in verse Ro 7:17.>
<7:21 A law; a constant tendency to evil, when I desire to be and do that only which is perfectly good.>
<7:22 I delight in the law of God; love it, and desire perfectly to obey it. After the inward man; inwardly, from the heart. I not merely approve of it in my conscience and judgment, but through the grace of God, I love it as the transcript of infinite perfection. Ps 1:2; Ps 119:24,77,174.>
<7:23 Another law; different from my prevailing inclination, my earnest desire. Warring against; opposing, hindering, and thwarting the full accomplishment of my wishes. The law of my mind; the desires of my heart, inspired by the Holy Ghost. Ga 5:17. Bringing me into captivity; a loathsome, hated bondage, which makes me abhor myself. Job 9:31; 42:6. Law of sin--in my members; propensities to evil which, notwithstanding all that grace has done, are not entirely removed. Christians of the greatest experience and highest attainments in the divine life, are not what they ought to be; not what they desire to be; not what they hope to be; not what God has promised that they shall be; and not what through grace, in fulfilment of his promise, they for ever will be--perfect even as their Father in heaven is perfect.>
<7:24 Wretched man; on account of remaining proneness to sin. Who shall deliver me; not the law, not my own efforts, or my abhorrence of myself on account of disobedience--not any expedients which ever have been or can be devised by creatures. Left to these merely, he who is filthy will remain filthy still. What then? Must I perish, or drag on for ever this body of death? No.>
<7:25 I thank God; for his unspeakable gift. 2Co 9:15; 1Pe 1:8. There is deliverance--complete, everlasting deliverance from all evil, and all propensity or liability to evil, through Jesus Christ our Lord; who, though he was rich, for our sakes became poor, that we through his poverty might be rich, being filled for ever with the fulness of God.>
<8:1 No condemnation; from God. Who walk; live habitually. Not after the flesh; not as corrupt nature prompts, seeking supremely earthly good and selfish gratification. After the Spirit; as the Holy Spirit directs, regarding principally God, Christ, heaven, and spiritual, eternal things.>
<8:2 The law of the Spirit of life; that new direction of the soul which is given by the Holy Ghost through faith in Christ. Made me free; delivered me from the condemning power of the divine law, and the reigning power of sin and death. This is that deliverance for which the apostle expresses his earnest longing in chap Ro 7:24. We are not to understand that it takes place instantly and perfectly, so that the experience described in the latter part of the seventh chapter wholly ceases, and that of the present chapter becomes absolute and uninterrupted. Rather do the two experiences run parallel with each other in a measure, yet so that the latter continually prevails more and more, as the Christian becomes more and more spiritual in his character.>
<8:3 What the law could not do; it could not deliver those who had broken it from condemnation and ruin. It could neither lead them to obey it, nor to repent of having broken it; nor could it lessen their disposition to violate it, notwithstanding its promises and threatenings. It could make no atonement for sin, and could not save from it. It was weak through the flesh; through man's depravity and transgression. In the likeness of sinful flesh; in human nature. For sin; to die on account of it, the just for the unjust. Condemned sin in the flesh; destroyed its power over those who trust in Christ, by showing, through his atonement, the evil nature of sin, the guilt of those who commit it, the excellence of the law which it violates, the anger of God against it; and so opening a way in which God could be just, and the justifier of all that believe in Jesus; and in addition to all this, procuring for them the grace of the Holy Spirit to help them to believe on Christ, and through their union with him, to render to the law a true spiritual obedience; thus delivering them from both the condemnation and the reigning power of sin, neither of which things the law could do.>
<8:4 That the righteousness of the law might be fulfilled in us; that we might be brought into that state of true righteousness which the law requires. The apostle, as the context shows, has immediate reference to deliverance from the reigning power of sin in the soul, but this can never be accomplished without deliverance also from its condemning power. The proper evidence of being interested in Christ, and entitled to the blessings of his salvation, is a disposition to regard the things which the Holy Spirit has revealed, and to follow his directions.>
<8:5 After the flesh; fleshly in their character. Do mind the things of the flesh; devote themselves to fleshly objects. Their outward conduct flows from their inward character, as a stream from a fountain. "The flesh" is to be taken here, as in Ga 5:19-21, and often elsewhere, in a wide sense. It includes all the earthly and corrupt passions, appetites, and desires which rule in the natural heart. They that are after the Spirit; they in whom the Holy Spirit habitually dwells, making them spiritual in their character. Mind--the things of the Spirit; yield themselves to the guidance of the Holy Spirit, and thus devote themselves supremely to the spiritual objects which he reveals.>
<8:6 To be carnally minded; the same as to mind the things of the flesh, as verse Ro 8:5. So, to be spiritually minded, is the same as to mind the things of the Spirit. Is death; spiritual death, and if continued will issue in eternal death. Is life and peace; spiritual life, manifesting itself in love, joy, peace, long-suffering, gentleness, goodness, faith, meekness, temperance, and all those dispositions and habits which promote the glory of God and the good of men.>
<8:7 Is enmity against God; opposed to his character and will. If it were not, it would love and obey him. Not subject to the law of God; it does not yield obedience, but acts in opposition to what he requires. Neither indeed can be; the two things are incompatible, contrary the one to the other; and the one should be renounced, that the other may be followed.>
<8:8 They that are in the flesh; in a fleshly state; the same as "they that are after the flesh," as verse Ro 8:4. Cannot please God; because a fleshly state is contrary to that spiritual state which God requires. For this reason they should no longer continue their wicked and dangerous course, but should turn from it and live. As those who seek their chief good in earthly things cannot please God, and so long as they continue this course show that they are at enmity with him, they ought without delay to renounce it, become his cordial friends, and show this by believing on his Son, and obeying his commands.>
<8:9 If so be that the Spirit of God dwell in you; the Holy Spirit, producing and leading you to manifest the fruits of the Spirit. Ga 5:22-24. The Spirit of Christ; the Holy Spirit, producing in him in some measure a likeness to Christ, and leading him earnestly to desire that it may be perfected. He is none of his; he is not interested in the blessings of His salvation.>
<8:10 If Christ be in you; by his Spirit, producing in you a likeness to himself. Because of sin; the body, whether of the believer or unbeliever, must indeed die and turn to dust, on account of sin. But the spirit is life; it is delivered from condemnation and is spiritually alive, having been raised with Christ, through the power of God, who raised him from the dead. Because of righteousness; because of the conformity of heart to the character and will of God, wrought in it by the Holy Ghost through faith in Christ.>
<8:11 The Spirit; the Holy Spirit. Dwell in you; producing in you the fruits of the Spirit. Ga 5:22,23. Quicken your mortal bodies; make them alive to God's service in the present life, so that from being "instruments of unrighteousness," they become "instruments of righteousness unto God;" and in the life to come, raise them up spiritual and immortal, to be united with the soul, that thus the whole man may be for ever with the Lord. 1Co 15:42-58; Php 3:20,21; 1Th 4:13-18.>
<8:12 Are debtors; owe obligation. Not to the flesh; because the minding of the flesh has never conferred upon us any real good, but always injury.>
<8:13 Ye shall die; the death not of the body merely, but of the soul--a death which shall last for ever. Body and soul shall die the second, an eternal death. Through the Spirit; through the influence and aid of the Holy Spirit, given to all who believe in Christ. Mortify the deeds of the body; resist, overcome, and cease to gratify sinful inclinations, and thus cause them to die. Ye shall live; a holy and blessed life--a life that shall outlive death, and be perfected in a world of life, light, and joy, where, as long as Christ lives, all who have his Spirit shall live with him, and be like him. Joh 6:57; Joh 14:19; 17:11,21,22,24. By believing in Christ persons may receive the Holy Spirit, under his influence become spiritually minded, overcome their propensities to evil, delight in the law of the Lord, and so act as through the grace of God to live for ever.>
<8:14 Are led by the Spirit; follow his guidance. This includes the idea of minding the things of the Spirit, and through his help, mortifying the deeds of the body. The sons of God; sonship here includes two things: first, likeness to God in character; secondly, heirship to the inheritance provided by God for his children.>
<8:15 Spirit of bondage; a servile, slavish spirit, governed by fear. Again to fear; that ye should be again in a state of fear, as ye were under the law. Spirit of adoption; the affectionate confidence of children, as contrasted with the servile fear of slaves. We cry, Abba, Father; Abba is the Chaldee word for father. The union of the two words gives emphasis to the endeared relation. Compare Mr 14:36; Ga 4:6.>
<8:16 The Spirit itself; the Holy Spirit. Beareth witness with our spirit; by working in us the dispositions, and leading us to form the habits and cherish the hopes of the children of God; and by his influence, enabling us to discern in ourselves these scriptural evidences of being born of God.>
<8:17 If children, then--joint-heirs with Christ; entitled to be, with Christ, partakers for ever of the blessings of Jehovah's kingdom. If so be that we suffer with him; deny ourselves for his sake, meet with calmness and bear with patience the trials which he brings upon us, and do whatever is needful to honor him and do good to men. That we may be also glorified together; Joh 17:24; 2Ti 2:11,12.>
<8:18 The sufferings of this present time; those which Christians endure in this world. Not worthy to be compared; are very small, as nothing in comparison. Glory--revealed in us; Eph 3:16-19; Col 3:4; 2Th 1:10; 1Jo 3:2. All sacrifices which men make to obey God, and all trials which they are called to endure, are light and momentary, compared with the blessings which he will bestow upon them in heaven.>
<8:19 The creature; the creation. In this and the three following verses the word rendered creature and creation is the same in the original Greek. It seems to denote the whole of this lower creation as brought under God's curse, and made subject to suffering and abuse in connection with the fall of man. Compare Ge 3:16-19. The manifestation of the sons of God; when they shall be seen and publicly acknowledge as his children, and take full possession of their inheritance as heirs of God, and joint-heirs with Christ.>
<8:20 Vanity; suffering and abuse. Not willingly; not by their own choice. Him who hath subjected the same; God, by his wise and righteous constitution of things. In hope; of deliverance from the evil endured.>
<8:21 The bondage of corruption; the bondage which consists in a corruptible state, with all the suffering belonging to such a state. The glorious liberty of the children of God; literally, "the liberty of the glory of the children of God;" that is, the liberty from corruption and suffering which the creation shall receive when the sons of God are manifested in glory. Verse Ro 8:19.>
<8:22 The whole creation; every part of creation which, without its own choice, has been subjected to evils, or is perverted and abused through the sins of men. Groaneth and travaileth in pain; experienceth and manifesteth deep distress under the evils which sin has caused. The sufferings which sin brings upon others, as well as on those who commit it, strikingly show its malignity, and should lead all to abhor and forsake it, and to look unto Jesus that they may be delivered from its power, and become the instruments of good to all who may feel their influence.>
<8:23 Not only they; the creatures which unwillingly suffer, or are perverted and abused through the sins of men. But ourselves also; true Christians, who are born of the Spirit, and have the foretastes of heaven. Groan within ourselves; under the evils which sin still occasions us. Waiting for the adoption; when body and soul, freed from all evils, shall be reunited, and be perfect in holiness and bliss.>
<8:24 Saved by hope; hope of future, everlasting glory; sustaining us in trials, animating us in duty, and leading us to persevere in keeping the commands of God. Hope that is seen, is not hope; what we have in possession, we do not hope for; we hope for future good, and persevere in the course which is needful to obtain it.>
<8:26 The Spirit; the Holy Spirit, who dwells in believers. Helpeth our infirmities; all the weaknesses that belong to us as fallen sinful beings, subject to suffering and death. In respect to these the Holy Spirit helps us by enabling us rightly to bear them, to overcome the evils they occasion, and in due time delivering us from them. Maketh intercession for us; by teaching us how to pray and what to pray for, and awakening in us those intense desires and fervent longings for spiritual blessings for ourselves and others, which cannot in any human language be fully uttered. All right and acceptable prayer is the fruit of the Holy Spirit, operating on the hearts of men, awakening pious emotions, and leading them to exercise such desires as are agreeable to the will of God. Though their feelings may not be uttered in words, he understands them, and glorifies himself in doing exceeding abundantly for all who pray in the Spirit, and watch thereunto with all perseverance. Eph 6:18.>
<8:27 He that searcheth the hearts; God. Knoweth what is the mind of the Spirit; he understands the feelings and desires caused by the Holy Ghost in the hearts of men, whether uttered in words or not; they are in accordance with his will, and he delights to answer them 1Jo 5:14. This is an unspeakable consolation and encouragement to the friends of God. And there is still another.>
<8:28 All things work together for good to them that love God; love to God distinguishes true Christians from all other men. He that loveth God is born of him, and all things shall work together for his good. The called; those who have been called by his grace out of the darkness and bondage of sin into the light and liberty of the children of God. As all things work together for good to those who love God, they are especially bound, in whatsoever state they are, therewith to be content; knowing that their trials, however great, will conspire to work out for them an exceeding and eternal weight of glory. 2Co 4:17.>
<8:29 Whom he did foreknow; as his people. He did not simply foreknow that they would be his people, but his foreknowledge of them as his people included the gracious purpose of bringing them into a state of salvation, as the apostle proceeds immediately to show. He also did predestinate to be conformed to the image of his Son; he determined to lead them by his Spirit to believe in Christ, and in this way to become like him, holy. First-born among many brethren; be their Prince, Leader, and Saviour, and have many who, as his brethren, should be joint-heirs with him to his kingdom of heavenly glory. Conformity in temper and conduct to the example of Christ, is the only sure evidence of being elected, and predestinated to eternal life.>
<8:30 Them he also called; called by his word and Spirit, taught them to feel their need of Christ, and led them to believe on him. Them he also justified; accepted as righteous. Them he also glorified; made heirs of eternal glory in heaven.>
<8:31 To these things; in view of the above-mentioned truths. If God be for us; if he is our friend, has led us to believe on his Son, and thus showed that he has determined to save us, and to cause all things to work for our good. Who can be against us? who can hinder our salvation, or on the whole do us any real harm?>
<8:32 Freely give us all things; as he has, self-moved, given us his only begotten Son to be our Saviour, and renewed our hearts by his Spirit, pardoned our sins, and justified us by his grace, who can doubt but he will give us all needed good? The death of Christ is conclusive evidence that nothing which will in the end promote their benefit, will be withheld from those who believe on him.>
<8:34 Who is he that condemneth? who can prevail against God, so as to destroy, or ultimately injure us? It is Christ; who from love, died in our stead when we were his enemies. But Paul cannot leave the subject here, since without the resurrection, ascension, and intercession of Christ, his death would be ineffectual to our redemption. He therefore passes on to these: yea rather, that is risen again, etc.>
<8:35 Who shall separate us from the love of Christ? that love which was stronger than death; which led him, when we were his enemies, to die for us, to give us his Holy Spirit, to reconcile and unite us to himself, and make us joint-heirs with him to all the blessings of his Father's kingdom. Who or what can separate us from such love? Shall tribulation; shall trials, or any thing which can come upon us?>
<8:36 As it is written; Ps 44:22. We are killed; constantly suffer, and are exposed to death.>
<8:37 More than conquerors; over all our trials; they shall not only fail to separate us from the love of Christ, or to diminish our love to him, or to do us any real harm, but they shall do us great good; showing us the evil of sin and the vanity of the world, quickening us in duty, and making us more holy and more happy. Him that loved us; Jesus Christ, the same yesterday, to-day, and for ever. All the blessings of believers come to them through Jesus Christ. They are given on his account, and are the purchase of his blood. They should therefore awaken in those who enjoy them, unfeigned and ever-increasing gratitude, and lead them to devote themselves, body and soul, for ever to his service.>
<8:39 The love of God--in Christ Jesus our Lord; "the love of God" here, like "the love of Christ," verse Ro 8:35, is his love towards us, which, however, always includes love on our part towards him.>
<9:1 In Christ; as one united to Christ, and devoted to his service. In the Holy Ghost; under his direction and influence.>
<9:2 Great heaviness and continual sorrow; on account of the deplorable condition and prospects of the unbelieving Jews, who constituted the mass of the nation, and whom, in the next verse, he calls his brethren and kinsmen according to the flesh, meaning his relatives and countrymen.>
<9:3 I could wish; were it proper, or would it benefit them. Accursed from Christ; subjected to the greatest calamities for his brethren, if by this means they could be saved. While true religion leads those who possess it earnestly to desire the salvation of all, it leads them especially to desire the salvation of their own countrymen, and most of all that of their relatives and friends, and to be willing to make any proper sacrifices for the sake of promoting it.>
<9:4 To whom pertaineth the adoption; who had been selected of God as his people, and on whom he had bestowed peculiar privileges. The glory; the visible emblem of the divine presence. The covenants; those made with their fathers. The giving of the law; Ex 20:1-17. The service of God; in his temple. The promises; those contained in the Old Testament.>
<9:5 Whose are the fathers; the patriarchs, priests, and prophets, a most illustrious ancestry. Of whom as concerning the flesh; from whom, as to his human nature, Christ descended. Who is over all, God blessed for ever; truly divine, Jehovah.>
<9:6 The word of God; his word of promise to Abraham and his seed. Hath taken none effect; hath failed of fulfilment. The great error of the unbelieving Jews was in thinking that the covenant of God with Abraham bound him to save all his children, without respect to their own personal faith and obedience. This error the apostle now proceeds to expose. Not all Israel; not all his natural descendants are, in spirit, true Israelites, to whom the promises of spiritual blessings were made. Ga 4:29.>
<9:7 Neither--are they all children; in the sense of being heirs to the promise. In Isaac shall thy seed be called; Ge 21:12. The meaning of these words is, that not all Abraham's children by natural descent shall be heirs of the promises made to him, but only those in the line of Isaac.>
<9:8 The children of the flesh; Abraham's children by mere natural descent. The children of God; in a special sense, as being heirs of the promises made to Abraham. The children of the promise; the children of Isaac, who was born in a supernatural way, according to God's promise, mentioned in the following verse. Are counted for the seed; the seed that inherits the promises made to Abraham. Thus the apostle shows that from the very beginning the blessings of the covenant did not flow in the line of mere natural descent to Abraham's children, but according to God's promise. Upon the same principle God acts under the gospel, reckoning as the true seed of Abraham those, and those only, who are born, not of blood, nor of the will of man, but of God, and who show this by believing in Christ. Joh 1:13; 6:29. The promises of the gospel are not made to any on account of their natural descent or their religious privileges, but on account of their union to Christ by believing on him.>
<9:9 The word of promise; Ge 18:10-14; 25:21-23.>
<9:10 And not only this; not only did God in this care reject a part of Abraham's seed by natural descent. By our father Isaac; to be connected immediately with verse Ro 9:12; in intermediate verse being thrown in to show that in rejecting Esau and taking Jacob, God was not influenced by any good or evil yet done by the children.>
<9:13 As it is written; Mal 1:2,3. The meaning of these words is, I chose Jacob and his seed to be heirs of the promises made to Abraham, and rejected Esau and his seed.>
<9:14 Is there unrighteousness with God? is it wrong for him to make such distinctions as he does among men? God forbid; certainly not: for every thing he does, he has the wisest and best reasons. Whatever God does is right; and however his dealings may appear to men, they should always feel that what he does is wise, holy, just, and good. In many things he calls men to walk by faith; and gives them opportunities to show thus whether they have or have not confidence in him.>
<9:15 He saith to Moses; Ex 33:19.>
<9:16 Of God that showeth mercy; the blessings which God bestows upon sinners originate wholly with himself. They are bestowed upon such persons, at such times, and in such ways and measures as he sees best, and are wholly of grace.>
<9:17 The scripture saith; Ex 9:16. Have I raised thee up; caused thee to stand; continued thee on earth a long time, notwithstanding all thy sins. My power in thee; my power to overcome all opposition, and by mighty signs and wonders, with a high hand and an outstretched arm, to deliver my people, according to my promise. That my name might be declared; that I might be made known as the one only living and true God, the omnipotent Jehovah, over all the earth.>
<9:18 Whom he will he hardeneth; as he did Pharaoh, by continuing him on earth notwithstanding his sins, and suffering him, under judgments and mercies, to act out his wickedness, and thus grow harder and more wicked than he was before.>
<9:19 Find fault; blame persons for doing wrong. Resisted his will; thwarted his counsels, by which "he hath mercy on whom he will have mercy, and whom he will he hardeneth," verse Ro 8:18. This is the old objection, that if God accomplishes all his purposes, he cannot blame men for their conduct. But it is a certain fact that he does govern them as free responsible beings, and hold them accountable for all their wickedness, although he may overrule it, as he did that of the Jewish council who instigated Pilate to crucify Christ, for the accomplishment of his own wise and good counsels.>
<9:20 Repliest against God; disputest against him, by finding fault with the principles upon which he governs the world.>
<9:21 Power; rightful power, as the original word implies. Of the same lump; the lump is here the mass of fallen sinful men, who can claim nothing at God's hand as a matter of right, and towards whom he may justly proceed, as he did towards Jacob and Esau, showing mercy to one, and withholding it from another.>
<9:22 What if God, willing; judged it best to manifest his wrath against transgressors of his laws and opposers of his government, and thus show his power to destroy his enemies and save his friends. Endured with much long-suffering; waited upon them a long time, as he did upon Pharaoh, while by their most unreasonable rebellion they grew harder; and thus the abuse of his forbearance became a savor of death unto death. 2Co 2:15,16. The vessels of wrath; men who perseveringly refused to obey God. Fitted to destruction; by their own wickedness. If God continues men in life and surrounds them with mercies, yet leaves them to pursue their own chosen way, they will grow more wicked, and become more hardened in sin; till, by rejecting his kind invitations, and abusing his providence and grace, they have fitted themselves for destruction.>
<9:23 The riches of his glory; the glorious perfections of his character, especially of his mercy and grace. The vessels of mercy; those whom he mercifully led to repent of their sins and believe on Christ.>
<9:24 Even us; believers in Christ. Called; effectually by his word and Spirit. Not of the Jews only; but of all nations.>
<9:25 Osee; the Greek form of the Hebrew word Hosea. Ho 2:23. Call them my people--not beloved; those who had been cast off as enemies, he would reclaim and gather as friends.>
<9:26 It shall come to pass; Ho 1:10. In the places where they had shown that they were not the people of God, there the change which grace would produce would be so manifest, that they would be acknowledged as his people. The salvation of any of the lost race of men originates in the love of God, and is accomplished by his power and grace, showing them their need of Christ, and inclining them to believe on him.>
<9:27 Esaias; Isaiah. Crieth; proclaimeth publicly. Isa 10:22,23. A remnant; a few only of them, compared with the whole, shall believe and be saved. This was repeatedly fulfilled in God's treatment of the nation before the coming of Christ; and now, in the apostle's day, it was having its great fulfilment in the fact that only a remnant of them received Christ as their Messiah, and were thus saved.>
<9:28 He will finish the work; the work of righteously destroying those who will not have him reign over them. Cut it short; accomplish it in a speedy and summary way. A short work; a work done with promptness and speed. The apostle quotes here, as often, from the Greek version of the Seventy.>
<9:29 Esaias said before; in an earlier passage. Isa 1:9. A seed; a remnant, a few. We had been; destroyed like Sodom and Gomorrah. By these quotations the apostle showed conclusively that the doctrine of the Jewish Scriptures was that only a remnant of the nation should be saved. In former judgments God had proceeded upon this principle, and he would do so now. A people may have the greatest outward privileges, and yet very few of them be saved. Of course no one can safely depend upon any outward distinctions or external privileges; unless they lead him to Christ as the all-sufficient and only Saviour, they will, by being abused, neglected, or perverted, aggravate his condemnation.>
<9:30 What shall we say; what is the conclusion? Followed not after righteousness; did not know God and did not seek his favor. Have attained; acceptance with God, by believing in the Messiah who has been offered to them.>
<9:31 The law of righteousness; or, as we may render, a law of righteousness; that is, a law which can give justification and eternal life, which, in their case, was the law of Moses.>
<9:32 Wherefore? why have they not obtained justification? Because they sought it by their own works, and as a matter of human merit; not by believing in Christ, and receiving it for his sake. They stumbled at that stumbling-stone; they were offended at Christ, and opposed salvation through him.>
<9:33 As it is written; Isa 8:14; 28:16. These passages pointed out the manner in which they would treat Christ; that the effect of believing on him would be salvation, and of rejecting him would be destruction; so that it was certain, not merely from the preaching of Paul, but from the testimony of God by the Old Testament prophets, that other foundation for human hope could no man lay than that which was laid, Jesus Christ; that by believing on him, Gentiles as well as Jews could be saved, and that by continuing to reject him, Jews as well as Gentiles would be lost. 1Co 3:11. The eternal condition of men who have the gospel will be according to their treatment of the Lord Jesus Christ. However ignorant, careless, or wicked they may have been, if they believe on him they will be justified, sanctified, and saved; if they reject him they will be lost.>
<10:1 As it is written; Isa 8:14; 28:16. These passages pointed out the manner in which they would treat Christ; that the effect of believing on him would be salvation, and of rejecting him would be destruction; so that it was certain, not merely from the preaching of Paul, but from the testimony of God by the Old Testament prophets, that other foundation for human hope could no man lay than that which was laid, Jesus Christ; that by believing on him, Gentiles as well as Jews could be saved, and that by continuing to reject him, Jews as well as Gentiles would be lost. 1Co 3:11. The eternal condition of men who have the gospel will be according to their treatment of the Lord Jesus Christ. However ignorant, careless, or wicked they may have been, if they believe on him they will be justified, sanctified, and saved; if they reject him they will be lost.>
<10:2 I bear them record; I freely and openly testify. A zeal of God; zeal for God; great zeal in religion. Not according to knowledge; not enlightened, wise, or according to truth. Men may have great zeal in religion, and yet be blinded and hardened in sin. Right zeal will be in accordance with truth and duty; holy in character, kind in spirit, and useful in tendency.>
<10:3 God's righteousness; that which he has provided in Jesus Christ. See note to chap Ro 1:17. Their own righteousness; by their outward obedience to the law. Have not submitted themselves; not given their hearts to God, or accepted his salvation through Christ.>
<10:4 The end of the law for righteousness; the true end of the law is to give eternal life; but to fallen sinful men it becomes the occasion of death. Chap Ro 7:10. Christ, by delivering those who believe on him from both the condemnation of the law and the reigning power of sin, brings them into a state of eternal life, and thus accomplishes the end of the law. That righteousness which men vainly seek by their own works, they may freely attain by believing in Christ. They may also in this way be led from the heart to yield an obedience to the law, which they otherwise never would have rendered; and to perform works which will receive a gracious and abundant reward.>
<10:5 Moses describeth; Le 18:5. The righteousness--of the law; that which can be obtained by obeying the law. Doeth those things; all the things which the law requires. Shall live by them; and thus be saved by his works.>
<10:6 The righteousness--of faith; that which men attain by believing in Christ. Speaketh on this wise; is described in this way. De 30:11-14. Say not in thy heart; do not think that this way of becoming righteous requires of you impossibilities, or things which, if disposed, you cannot do. It does not require you to go up to heaven, or down into the deep. All that it required is, to receive Christ as your Saviour and guide.>
<10:8 The word; the message of salvation through Christ. Nigh thee; close at hand, so that it requires no toilsome labor to find it or to do it. In thy mouth, and in thy heart; ready at hand to be received by thy heart, and confessed by thy lips. Verses Ro 10:9,10,11.>
<10:9 Confess with thy mouth--believe in thy heart; he mentions these two things, because both are necessary to salvation--the inward faith in Christ, and the outward confession of him. That God hath raised him from the dead; for belief in this includes every thing else. By raising Christ from the dead, God set his seal to him as the promised Messiah.>
<10:11 The scripture saith; Isa 28:16. Shall not be ashamed; his confidence in the Saviour shall not be disappointed. That faith which is represented in the Old Testament and the New as essential to salvation, is one which influences and controls the heart and life.>
<10:13 Whosoever shall call; Joe 2:32. The apostle quotes from a prophecy relating to the times of the gospel. See the context, Joe 2:28-31.>
<10:14 How then shall they call; the quotation from Joel gives the apostle occasion to magnify the office of the gospel preacher, and show the propriety, the wisdom, and goodness of preaching the gospel to the heathen, as had been done by Paul and others, and as had been foretold by the prophets.>
<10:15 As it is written; Isa 52:7. Another prophecy that has its highest fulfilment under the gospel dispensation.>
<10:16 They have not all obeyed; though the gospel had been preached extensively among Jews and Gentiles, yet only a few comparatively had embraced it, especially among the Jews. This also had been foretold in the Old Testament. Isa 53:1. Yet some, as foretold, when they heard the gospel, believed and were saved.>
<10:17 Faith cometh by hearing; the hearing of the word of God. His communication in the gospel is the means appointed and blessed to lead men to exercise faith; hence it should be preached to all, that they may hear, believe, and be saved. The hearing of Christ as revealed in the gospel, is the means which God has appointed, and which he blesses to the production of faith in him; he should therefore, as soon as practicable, be preached to all people: and those who aid in doing this, are exerting an important instrumentality for the salvation of men.>
<10:18 But I say, Have they not heard? this is said in reference to the general unbelief of men, verse Ro 10:16; as much as to say, True, few have obeyed; but is this from want of hearing? no; for their sound went into all the earth; in other words, what the Psalmist says of the instruction given by the heavens, Ps 19:1-4, is true of the preaching of the gospel. It has been extensively proclaimed among many nations. Of course, if the people do not believe, it must be their own fault.>
<10:19 Did not Israel know? have in their own scriptures the means of knowing; namely, that the Gentiles as well as the Jews were to hear the gospel, and that multitudes of them would embrace it, while the unbelieving and disobedient Jews were to be rejected? This had been foretold. First Moses saith; as much as to say, To begin with Moses, the first of all the sacred writers. The passage quoted is in De 32:21. It teaches that for their disobedience God will provoke the covenant people to anger by exalting the heathen nations above them. This has been fulfilled, first, in a temporal way, by their repeated subjection--as at this very day--to the dominion of gentile nations; secondly, in a spiritual way, by God's casting off the unbelieving Jews, and calling into the church the believing Gentiles. It is to this latter fulfilment that the apostle here refers. No people; as idolaters, not worth to be called a people in contrast with the people of God. Foolish nation; stupidly worshipping idols.>
<10:20 Esaias is very bold; speaks openly and plainly. Isa 65:1,2. Them that sought me not; those who had not before sought him, the heathen. He revealed himself to them in the gospel, and they believed on him.>
<10:21 Stretched forth my hands; in kind invitations of mercy. A disobedient and gainsaying people; who continued to oppose his messengers and reject their message. As they continued to reject him, he would reject them, and gather to himself a people from the Gentiles. This had been foretold in the Old Testament, and they might have known it. No outward connection with any visible church, and no external privileges merely, can secure for men the favor of God. He will treat them as they treat his Son. If they receive and obey him as their Saviour, he will be made of God unto them wisdom, righteousness, sanctification, and redemption. If they do not, he will cast them off. Joh 14:15,21; 1Co 1:30; 16:22.>
<11:1 Cast away his people; cast them off as a people, so as to break his covenant with them. The answer is, No; he has cast off the unbelieving part of them, and saved the believing remnant. The apostle then proceeds to show, by quotations from the Jewish scriptures, that God has always proceeded in this way. I also am an Israelite; and, as such, an example of the "remnant according to the election of grace," verse Ro 11:5.>
<11:2 His people which he foreknew; that is, the remnant of his people which he foreknew. See note to chap Ro 8:29. Wot; know. Maketh intercession; 1Ki 19:10.>
<11:4 Baal; the name of an idol which many, in the days of Elijah, worshipped. 1Ki 18:22.>
<11:5 A remnant; a small number of the Jewish nation who belong to his redeemed people, and to whom his promises to Abraham of spiritual blessings, were made. Ga 3:29. Election of grace; God's gracious choice of them to be his people. When multitudes enter the wide gate, and walk in the broad way which leadeth to destruction, the reason why some enter the strait gate, and walk in the narrow way which leadeth unto life, is they are by God graciously chosen to salvation through sanctification of the Spirit and belief of the truth, and are kept by the power of God, through faith unto salvation. 2Th 2:13,14; 1Pe 1:5.>
<11:6 No more of works; if his choice of them were of grace, it was not on the ground of any merit in them; because, if it were, it would be of debt, not of grace.>
<11:7 What then? what is the conclusion to which we come? Israel; the great body of the Jewish nation. That which he seeketh for; righteousness and acceptance with God. The election; those whom God graciously chose to be his people, and whom he gave to Jesus Christ. Joh 6:37; 10:26-30. Blinded; by their sins, in refusing to come to the light, and given up to hardness of heart, as a punishment for their transgressions. Joh 3:20. If none were chosen of God to eternal life, none would be saved, because none would take the only way of salvation. His choosing them was not because they were naturally better than others, or on account of any thing spiritually good in them; but it was a favor graciously bestowed, and which, through his grace, is connected with believing in Christ, repenting of sin, and persevering in obedience to eternal life. Of course we ought to be as grateful to God for his election of men, as for their salvation.>
<11:8 As it is written; De 29:4; Isa 6:9,10; 29:10; Mt 13:14,15; Mr 4:11,12.>
<11:9 David saith; Ps 69:22, 23. As David was, by divine appointment, an eminent type of Christ, so the destruction of David's enemies typified that of the enemies of Christ, the great antitype. And in the case of both David's and Christ's enemies the great principle was illustrated, that the wicked and rebellious among Abraham's children shall perish, even as other sinners. Let their table; representing all their earthly good.>
<11:10 Bow down their back alway; the same as "make their loins continually to shake," namely, with terror and anguish. Ps 69:23.>
<11:11 Have they stumbled that they should fall? irrecoverably fall? Will the great body of the Jews always continue in unbelief, and all for ever perish? Certainly not. Salvation is come unto the Gentiles; the rejection of the Messiah by the Jews was made the occasion of his being preached to the Gentiles, and many among them being led to believe on him. After a time the Jews also shall believe on him and be saved. For to provoke them to jealousy; to provoke the Jews to jealousy, while they witness the great privileges to which the Gentiles are admitted through faith in Christ. The apostle alludes to the passage which he had quoted from De 32:21: "I will provoke you to jealousy by them that are no people.">
<11:12 Them; the Jews. Be the riches of the world; be the occasion of great good to the world. How much more; will their restoration to the favor of God be the occasion of greater good. The ruin of some men is overruled by God as the occasion of great good to others.>
<11:13 I magnify mine office; his office as apostle to the Gentiles. This he showed to be highly honorable, as connected with the plan of God for the salvation of men.>
<11:14 Provoke to emulation; provoke to jealousy, as in verse Ro 11:11; and chap Ro 10:19, where the original Greek uses the same word. The apostle means, provoke to jealousy in such a way that they shall be stirred up to seek the blessings which they see taken from them on account of the unbelief, and given to the Gentiles through their faith.>
<11:15 Of them; the Jews. Of the world; the Gentiles. Life from the dead; as a glorious resurrection.>
<11:16 The first-fruit; the cake made from the first dough of the new harvest, which the Israelites were required to offer to the Lord, before they ate of it. See Nu 15:20. Be holy; consecrated to God. The lump; the whole mass of dough from which the offering was taken. The root; of the tree. The first-fruit of the dough and the root of the tree here represent the patriarchs of the Israelitish nation, who received the irrevocable promises made to Abraham and his seed. Their reception by God, as a peculiar people consecrated to his service, is a pledge that he will not finally cast off the nation sprung from them. Their present rejection is temporary, and not final.>
<11:17 Some of the branches; the Jews, the natural descendants of Abraham, called on this account, verse Ro 11:21, natural branches. A wild olive-tree; one that is uncultivated and bears no valuable fruit; representing the Gentiles.>
<11:18 Boast not against the branches; the natural branches, the Jews. Do not exult over them, as if you were naturally better than they, and were in no danger. The root; the original church of God consisting of Abraham's seed, into which the Gentiles are grafted by faith.>
<11:19 The branches were broken off; the Jews were cast out, that the Gentiles might be admitted.>
<11:20 Because of unbelief; their unbelief was the cause of their rejection. Thou standest by faith; continued faith in Christ and obedience to him are essential to your continuance as the people of God. High-minded; elated, proud, haughty. Fear; walk humbly in the fear and love of God, lest you too be broken off and perish. Men who have right views of God and his ways, of themselves, and their relations to him and their fellow-men, will not be proud, haughty, or censorious, but will be humble, meek, grateful, benevolent.>
<11:22 On them which fell; the unbelieving Jews. Severity; just, righteous punishment. Toward thee; the believing Gentiles. Goodness; gratuitous favor. If thou continue in his goodness; by continuing to believe and obey him.>
<11:25 Of this mystery; unattainable by mere human reason, and which has not hitherto been clearly revealed: that the rejection of the Messiah by the Jews was to be only temporary; and that when multitudes of the Gentiles should be converted to him, then the Jews also would acknowledge him, and be again received as the people of God. Until the fulness of the Gentiles be come in; the gentile nations in full number, according to the wise appointment of God.>
<11:26 All Israel; Israel as a nation, and not, as now, a remnant of believing individuals. As it is written; Isa 59:20,21; Ps 14:7; Jer 31:31-34; Heb 10:15-18. However great the present blindness of the Jews, however strong their opposition to Jesus of Nazareth, the time is coming when they will know that he is their long-promised Messiah, will embrace him as their hope of glory, and will become eminent benefactors of the world. With reference to this they have been kept as a distinct people; and all exulting over them by the Gentiles, is highly offensive to God.>
<11:28 As concerning the gospel; in respect to the spread of the gospel. Enemies for your sakes; enemies to the gospel and rejecters of it in such a way that, by the wise appointment of God, it shall be more generally preached and obeyed among the Gentiles. The rejection of the gospel by the Jews was the occasion why the preachers of it turned to the Gentiles, and this was a part of God's purpose. Compare Ac 13:46; Ac 18:6; 22:18-21; Ac 28:28. Touching the election; on account of God's choosing Abraham and his spiritual seed, and on account of the promises which he made, they were still remembered in mercy, and in due time would again be restored to the privileges and blessings of his people.>
<11:29 Gifts and calling of God; his choosing them as his people, and his promises to give them spiritual blessings. Are without repentance; will not be revoked. God will not change his determination, or fail to bestow the blessings which he has promised.>
<11:30 Through their unbelief; their unbelief was made the occasion of your having the gospel preached to you, and thus obtaining mercy.>
<11:31 Through your mercy; through the mercy bestowed upon you by God, in bringing you into his church. For the converted Gentiles shall, in their turn, be instrumental in bringing back the Jewish people to Christ.>
<11:32 Hath concluded; left shut up, as in a prison, without any hope of relief, from their own works. Them all; the whole, both Jews and Gentiles. That he might have mercy upon all; and thus Jews and Gentiles alike be made to feel and acknowledge that their salvation is of grace, not of works.>
<11:33 The reasons of the proceedings of God with men are often by them unknown, and can never be fully understood; yet we may be certain that he, in all cases, has reasons which are perfectly wise and infinitely good. No objection therefore ought to be made, by any one, to any thing that God does; but the spontaneous expression of all should be, Bless the Lord in all places of his dominion; bless the Lord, O my soul. Ps 103:22.>
<11:34 Who; can understand the mind of God? or, who ever taught him any thing? No one. He is alone the sum and source of all. Isa 40:13; Jer 23:18. He needeth not and receiveth not information from any of his creatures. Ac 17:24-28.>
<11:35 Who hath first given; who ever gave to God any thing which God did not first give to him? No one; for no one ever had any thing, except what he received from God.>
<11:36 Of him; God, as their Creator. Through him; as their Preserver and Benefactor. To him; as their great end. Are all things; all things were created, are perserved and controlled, and will be disposed of, to the promotion of his glory. To whom be glory for ever; it all belongs to Him, to Him let it all be given. Amen; so be it. Let every thing that hath breath praise the Lord. Ps 150:6.>
<12:1 The mercies of God; those which he bestows in and through Jesus Christ, as the apostle has exhibited in the former part of this epistle. For he now proceeds to draw from the deep doctrines there unfolded inferences of a practical nature. Present your bodies; as the priest presented to God the bodies of the victims slain. A living sacrifice; in contrast with the slain sacrifices of the Mosaic law. To present the body to God as a living sacrifice, is to consecrate it, with the living soul that inhibits it, to God's service. 1Co 6:15-20. Reasonable service; a service of the spirit, in contrast with a merely outward and bodily service. Compare 1Pe 2:5.>
<12:2 Be not conformed to this world; to its sinful spirit, maxims, customs, and habits. Transformed; changed, not in outward conduct merely, but in the spirit and temper of your minds. May prove; know and discern aright by your own experience. The apostle has in view that discernment of God's will which comes from actual obedience. Compare Joh 7:17. The doctrines of the gospel as inculcated by the apostles, especially justification by grace through faith in Jesus Christ, are not only consistent with, but conducive to the more pure, elevated, and universal morality; and the intelligent, cordial belief of the one will, through the grace of God, secure the other.>
<12:3 Through the grace given unto me; by virtue of my apostolic office, for which the grace of God has furnished me. To think soberly; by forming a just estimate of himself and his gifts, as compared with his brethren and their gifts. To every man the measure of faith; thus qualifying him for some services, but not for others.>
<12:4 All members have not the same office; the eye, for instance, cannot perform the office of the ear, nor the hand that of the foot. The perfection of the whole depends upon the perfection of each organ; so with the spiritual body, the church of Christ.>
<12:5 We; Christians. One body in Christ; he is the head, and we are the members of his one body. The perfection of each member and of the whole body of Christ, depends, not upon all being alike, or doing the same things, but upon all being in their proper places, and doing each his appropriate work.>
<12:6 Gifts differing according to the grace that is given; God graciously bestows upon different members of the church different talents and gifts, and all are to use them according to his will. Prophecy; this was one of the spiritual gifts. See note 1Co 12:28. According to the proportion of faith; the same as "the measure of faith," verse Ro 12:3. Let him keep himself within the limit of the spiritual qualifications bestowed on him by God.>
<12:7 Or ministry; that is, or having ministry for his gift. Ministry is here ministering to the wants of the brethren, as distinct from prophesying and teaching. Let us wait on our ministering; occupy ourselves with it in a humble and contented spirit.>
<12:8 Giveth; for the relief of others. With simplicity; singleness of aim, purity of motive, without selfish ends. Ruleth; directeth the concerns of the church. Showeth mercy; by attending on and assisting the sick, sorrowful, and distressed. With cheerfulness; with a kind, patient disposition, which will greatly increase the pleasure and benefit of his assistance.>
<12:10 In honor preferring one another; rather, going before, or setting an example to one another in courtesy, kindness, and respect.>
<12:11 In business; or in diligence, as the same word is rendered in verse Ro 12:8. The meaning is, that in the exercise of a diligent and earnest spirit we should not be remiss. Serving the Lord; by activity of body and mind, wisely and perseveringly discharging the various duties of life. 1Co 10:31.>
<12:12 In hope; in hope of future glory. Instant in prayer; habitual, fervent, persevering in the duty.>
<12:13 Distributing to the necessity of saints; supplying their wants. Given to hospitality; accustomed to provide for needy travellers and strangers, especially such as are laboring or suffering for Christ.>
<12:14 Good men will desire to do good, not to friends only, or such as do good to them, but also to enemies, and such as do evil. The characters of men are more clearly seen by their treatment of enemies, than of friends. Lu 6:32-36.>
<12:15 Rejoice--and weep; manifest a deep interest, a tender sympathy in the joys and sorrows of others.>
<12:16 The same mind; be united, live in peace. Mind not; do not aspire to, or seek after wealth, honor, or powerful earthly connections. Condescend; sympathize and associate with the poor, humble, afflicted, especially such as suffer for righteousness' sake. In your own conceits; do not have such an opinion of your own wisdom as to exalt yourself or despise others, or prevent your feeling your dependence, and obligation for all which you possess, to the grace of God. Pr 3:5-7.>
<12:17 Recompense to no man; do no evil to any one, because he does evil to you. Things honest in the sight of all men; things that are right, lovely, and of good report, as the original word implies. Conduct in such a manner as is suited to meet the enlightened and conscientious approbation of men. Pr 3:3,4.>
<12:18 As much as lieth in you; as far as you can consistently with duty, cultivate a peaceful temper, and seek to live in peace.>
<12:19 Give place unto wrath; do not take revenge upon those who injure you, but exercise a forgiving spirit. Leave the taking of vengeance to Jehovah, to whom it belongs. De 32:35. Individuals against whom crimes are committed, are not to avenge themselves by punishing the criminals. Civil government, which God has established for this purpose, is to punish criminals so far as is needful for the terror of evildoers and the security of those who do well. This is one way in which God manifests his wrath against transgressors in this world, and gives an earnest of the fulness of wrath which, unless they repent and believe on his Son, he will manifest against them in the world to come.>
<12:20 Feed him--give him drink; treat him kindly, do him good, and when he is needy supply his wants. Pr 25:21,22; Mt 5:44. Heap coals of fire on his head; which will be adapted to melt him into penitence and love.>
<12:21 Be not overcome of evil; let not evil conquer you, but do you with kindness conquer it. Kindness towards enemies is a most likely means of making them friends; and if it does not have this effect, but they continue obstinately and wickedly to be enemies to their benefactors, they will ripen for aggravated ruin.>
<13:1 The higher powers; the civil government. Are ordained of God; civil government is an ordinance of God, and magistrates are to be obeyed as his ministers, clothed with authority from him.>
<13:2 Resisteth the power; the civil government, in the exercise of its rightful authority. Damnation; condemnation, punishment. As civil government is an institution of God, it should be respected, and its just requirements conscientiously and cheerfully obeyed.>
<13:3 Rulers; in the discharge of their appropriate duties, are not a terror to good works; to persons who do right. They were not made rulers by God for this purpose, but to be a terror to the evil; to evil-doers, by being authorized to punish them. Be afraid of the power? provided you do evil, because, if the government does its duty, it will punish you. Thou shalt have praise of the same; do right, and the government, if it does its duty, will protect and encourage you.>
<13:4 He is the minister of God; the magistrate is His servant. To thee for good; made a ruler, not for his own good, but the good of the people whose interests he is bound to promote. Not the sword in vain; the sword is an instrument of punishment, and as such, an emblem in the hand of the magistrate, of rightful authority, in case men maliciously put to death their fellow-men, to punish them even with death. Ge 9:6; Nu 35:16-21,30,31. To execute wrath; not the wrath of the magistrate or of the government merely, but the wrath of God against evil-doers. As the object for which God established and upholds government is the highest good of the governed, it should be so constructed and administered as will best accomplish this end.>
<13:5 Also for conscience' sake; men should obey the laws, not merely from the fear of punishment, but from a sense of duty to God and men.>
<13:6 For this cause; because government is God's institution, and magistrates are his ministers to promote the good of the people. Pay ye tribute; taxes are justly due to the government for the payment of its officers, and for other needful expenses; and they ought to be freely, conscientiously, and punctually paid. This very thing; the discharge of the appropriate duties of their office. Men have no more right to defraud the government of its just dues, or to withhold the taxes of the duties which are needful to carry on its operations, than to defraud their fellow-men. And those who in any way do this, sin not only against men, but against God.>
<13:7 Their dues; what rightfully belongs to them. Tribute; taxes on real and personal estate. Custom; taxes on merchandise, and on foreigners. Fear--honor; pay to rulers and officers of government such respect as will conduce to the best discharge of their duties. If rulers transcend their just authority, neglect the objects for which they were appointed, and seek their own, not the good of the people--if they terrify the good, encourage the bad, and require men to commit sin--men are bound, in these things, to disobey them, and in all things to obey God. In no case are men to commit sin to accomplish any object whatever.>
<13:8 Owe no man any thing; discharge, at the proper time, all just obligations. But to love; love to men will lead you to fulfill towards them all your duties.>
<13:9 Love thy neighbor as thyself; desire and in all suitable ways seek to promote his good. Le 19:18; Lu 10:29-37. Do to him as you ought to wish, under similar circumstances, that he should do to you. Mt 7:12; Lu 6:31.>
<13:10 Supreme love to God, and that genuine love to men which springs from and accompanies it, will lead rulers and ruled to seek each other's good and that of all their fellowmen. In the government and out of it, in their official duties, in their private example, and in all their influence, good men will strive to do to others as they ought to wish others to do to them.>
<13:11 And that; and do that which I have been urging. Knowing the time; knowing how far it has advanced. Sleep; the insensibility and inactvity of sin. Now is our salvation nearer than when we believed; our final salvation with Christ, towards which believers are every day drawing nearer.>
<13:12 The night; our state of darkness and trials in this world. The day; the state of light and bliss in heaven. The works of darkness; sinful deeds of every kind. The armor of light; the armor of righteousness, which is worn by those who walk in the light.>
<13:13 Walk honestly; live in a manner becoming disciples of Christ hastening to eternity, and preparing for heaven. See note to chap Ro 12:17. Rioting and drunkenness; intemperance. Chambering and wantonness; licentiousness.>
<13:14 Put ye on the Lord Jesus Christ; clothe yourselves with his character and spirit. In order to do the most in their power to remove all existing evils, and promote the greatest good, Christians should possess, and in all things manifest, the spirit of Christ, labor to make known his character and will to all people, and set before them the motives which he has revealed, to lead them to believe on and obey him. All should look upward to Him who has the residue of the Spirit, that his heavenly influence may descend in copious effusions, and the evils of sin become as the frosts of winter on the approach of spring, and vanish as darkness before the light of day.>
<14:1 In this chapter, and part of the following, the apostle urges the duty of mutual forbearance and charity, in respect to non-essential points of difference. Among the Roman Christians these had respect to certain outward distinctions of food, days, and the like. These would be best overcome, not by scornful and bitter judgments of each other, but by the spirit of mutual love and conciliation. Him that is weak in the faith; namely, the faith of the gospel. The apostle has in mind the conscientious believer, who has not attained to such enlarged views of the liberty of the gospel as to raise him above bondage to unessential outward observances. Receive ye; to your fellowship, treat him as a Christian. Not to doubtful disputations; or, not to discernings of thoughts; in other words, not for the purpose of setting yourselves up to try and pass judgment upon his religious scruples. Persons may have erroneous views with regard to many unessential things, and yet be real Christians; and those who give evidence of being received of Christ as his disciples, should be received by us, and treated as Christian brethren.>
<14:2 Eat all things; any wholesome food. Who is weak; ignorant of what is proper on this subject. Eateth herbs; lives on vegetables and abstains from flesh, lest he should be defiled by the use of it.>
<14:3 Him that eateth; all kinds of wholesome food indiscriminately, having attained, in this respect, to a true idea of the liberty of the gospel. Despise; the sin to which men of liberal views are especially tempted. Him which eateth not; eateth not flesh, because he has scruples of conscience in respect to the use of it. Judge; in a condemnatory way. This is the sin to which conscientious men of narrow views are particularly prone. God hath received him; as a Christian, and admits him to fellowship with himself. We should never despise any on account of their errors, or their supposed inferiority to ourselves; nor condemn them for following their own consciences, not ours; but we should endeavor to enlighten them as to the will of God, and set them an example of obeying it.>
<14:4 His own master; Jesus Christ. He standeth or falleth; he will be approved or condemned, not according to the correctness of his views about the ceremonial law, or outward forms and ceremonies, but according to his character as a friend or enemy of Christ. He; the true Christian, though feeble, and in some respects erring. Shall be holden up; sustained as a Christian and accepted; for, God is able, and he has promised to do it.>
<14:5 Esteemeth one day above another; because the ceremonial law, which he erroneously thinks is still binding, makes a distinction between different days of the week. One observes the Jewish feasts and fasts, the other does not. The apostle here has no reference to the difference of days spoken of in the moral law. He speaks in this chapter about that difference which is associated with meats and drinks, divers washings, and various other things contained in the ceremonial law. Be fully persuaded; let a man examine and ascertain by the best light he can what is right, and do as he conscientiously believes that God requires. He should not, in such matters, be forced to follow another's conscience, but should be permitted, in the exercise of his inalienable right, to follow his own.>
<14:6 He giveth God thanks; the Christian who regards days and meats according to Jewish ceremonies, and the Christian who does not, both act from religious motives, and for the purpose of honoring God. This they show by thanking him for his mercies. They should therefore be received and treated by each other as friends of God. No man should do what he does not believe to be right; and the great object of every man in what he does, and in what he forbears to do, should be to honor God and benefit his fellow-men.>
<14:7 None of us liveth to himself; the great object of every Christian, in life and death, is not himself, but Jesus Christ. His language is, Not my will, but thine be done.>
<14:8 We are the Lord's; we seek his glory, are governed by his will, and belong to his redeemed people.>
<14:9 To this end; that he might be Lord of his redeemed people dead and living, on earth and in heaven, he died, rose, and ascended to glory, where he now lives; head over all things to his church, and will in due time come to judge the world in righteousness. It follows that our aim should be his glory, and the edification of his people; not the promotion of our own private ends. Christ is the rightful owner and governor of the whole family, especially of his redeemed people. All should therefore give him the homage of their hearts and the obedience of their lives, treat his friends as their friends, and love them heartily for his sake. Ga 6:10.>
<14:10 Of Christ; our rightful and proper judge.>
<14:11 For it is written; Isa 45:21-25. What is said by Isaiah of Jehovah, the apostle here applies to Christ, and thus shows that he is Jehovah, God the judge of all.>
<14:12 As Christ is to be the final judge of men, and we are to give account each of himself to him, and be accepted or condemned according to his decision, we should live as under his inspection, and make it our great object so to act that he will say to us, Well done, good and faithful servants; enter ye into the joy of your Lord. Mt 25:21.>
<14:13 Not therefore judge; not assume the place of Christ in judging his servants, but leave that to him to whom it belongs, and who will judge according to truth. Judge this rather; decide this rather in your minds. The apostle intentionally uses the word judge in a double sense; as much as to say, Instead of deciding on your brethren's conduct, decide this rather, to lay no stumbling-block before them.>
<14:14 Persuaded by the Lord Jesus; convinced by knowledge received from him. Nothing unclean of itself; the distinction between clean and unclean meats, and different days of ceremonial observance, is now done away, and it is as lawful to eat one kind of healthy food as another. To him it is unclean; if a man really believe it wrong for him to eat meat, for him it is wrong, because it is wrong to violate his conscience.>
<14:15 Be grieved with thy meat; if your eating meat grieves and injures a brother. Walkest thou not charitably; provided you continue to eat it. Love to him requires you to abstain from it. Destroy not him; by doing that which tends to ruin him or make him miserable. Christ endured the agonies of the cross to make him blessed; you, as a friend of Christ, redeemed with his blood, ought, if need be, to deny yourself for the same end.>
<14:16 Your good; your knowledge of your Christian liberty and freedom from the ceremonial law, which is a real good. Be evil spoken of; be an occasion of reproach and blame, by your using your liberty in such a manner as to injure others, or dishonor Christ. A course of conduct may be right in some respects and in some circumstances, and wrong in others, on account of the different effects which it will produce. In order, therefore, to justify an act, it is not enough that it is not in its nature sinful, but it must also be suited, in the circumstances, to do good.>
<14:17 The kingdom of God; his reign in the soul, and true obedience to him, do not consist in the observance or non-observance of distinctions between meats and drinks, and other like outward things; whence it follows that on the side of both the weak and the strong there should be forbearance and kindness. But righteousness, and peace, and joy in the Holy Ghost; it consists rather in being just, benevolent, and merciful; at peace with God and one another, rejoicing in his government, and in hope, through grace, of dwelling with him for ever in heaven.>
<14:18 Serveth Christ; though it is written, "Worship the Lord thy God, and him only shalt thou serve," Mt 4:10, yet he that serveth Christ is acceptable to God. The reason is, Christ is God. Chap Ro 9:5; Joh 1:1; Heb 1:6-8.>
<14:20 For meat destroy not; do not, for the sake of your own indulgence, injure the religious character of your brother, or do any thing which shall tend to destroy him. All things indeed are pure; all kinds of wholesome food are in themselves innocent, but if your partaking of them causes your brother to sin, or injures him, it is wrong for you to do it.>
<14:21 It is good; duty requires us to abstain from indulgences which lead others to sin, injure their character, hinder their usefulness, prevent their enjoyment, or endanger their souls. It is often a duty to avoid the doing of things which, though not in themselves wrong, will become the occasion of evil to our fellow-men.>
<14:22 Hast thou faith? do you believe that the ceremonial law is abolished, and that it is right for you to eat all kinds of food? be grateful to God for this light, but do not use it in such a manner as to injure others. Happy is he; who does not allow himself in things which his conscience condemns, or the propriety of which he doubts. Self-denial as to personal gratifications, for the sake of others, is an evidence of great excellence and a means of rich enjoyment. Those who make proper efforts to ascertain what is right, and who do only what they believe to be so, will be truly blessed in the approbation of conscience and of God; while those who do what they do not believe to be right, will be condemned both by themselves and their Maker.>
<14:23 He that doubteth; the lawfulness of any thing, and yet does it when there is no doubt about the lawfulness of abstaining from it, is damned; condemned as guilty of sin. Whatsoever is not of faith; whatever a man cannot do with a clear conscience, believing it to be right. The apostle is speaking of those things which are in themselves indifferent, and about which the true friends of Christ may honestly differ.>
<15:1 We then that are strong; enlightened on the subject in question; free from harassing doubts as to our duty. Bear the infirmities of the weak; bear with them, and endeavor to assist them.>
<15:2 Please his neighbor; make the good of others, not his own gratification, his object. Our object in trying to please men should be, not to gain applause, but to do good; and we should not strive to please them any further than will be for the glory of God, and their highest benefit.>
<15:3 Christ pleased not himself; by staying in heaven and enjoying the glory he had with the Father; but he condescended, submitted to many privations, and made great sacrifices for the good of others. As it is written; Ps 69:9. Reproaches--fell on me; and he cheerfully bore them, for the sins of men.>
<15:4 Written aforetime; in the Scriptures. For our learning; to instruct us in our duty. Patience and comfort of the Scriptures; received through the Holy Ghost from the Scriptures. The apostle uses the word patience here in the sense of the steadfast endurance of trials. See note to Ro 5:3. Might have hope; hope of future glory, which shall sustain us in trials, quicken us in duty, and thus purify and fit us for heaven. The Scriptures were all written under the guidance and according to the direction of the Holy Ghost, to afford instruction and increase the excellence, usefulness, and enjoyment of men in all countries and ages. They should therefore be put into the hands of all as soon as possible.>
<15:5 The God of patience; who, by his word and Spirit, gives patience and consolation in trials. Like-minded; alike in views and feelings, in obedience to and imitation of Christ. Hence differences of Christians on lesser points need not mar their unity in feeling.>
<15:6 One minded and one mouth; unitedly. Glorify God; by manifesting those dispositions which are the fruit of his Spirit, and which he requires.>
<15:7 Wherefore; for the reasons above mentioned. Receive ye one another; to Christian fellowship, for such reasons and with such a spirit. As Christ also received us; to fellowship with him, that God by this Christian union may be glorified. The union of Christians glorifies God. They should receive and treat as Christians all who give evidence that they are such, and do it in obedience to the will, and in imitation of the example of Christ.>
<15:8 Christ was a minister of the circumcision; he was born, lived, and died a Jew; he came as the Messiah to the Jews, exercised his ministry among them, and died to redeem them, in fulfilment of the promises which God made to their fathers.>
<15:9 That the Gentiles might glorify God for his mercy; his mercy also to them in sending them the gospel and inclining them to receive it. As it is written; Ps 18:49; originally spoken by David in view of his triumphs over all his enemies. These typified the higher triumphs of Christ, in the benefits of which the Gentiles are to share.>
<15:10 Again he saith; De 32:43. When Moses calls upon the nations to rejoice with God's people, it is manifest that they are to be admitted to a share of their privileges.>
<15:11 And again; Ps 17:1. The call upon the Gentiles to praise God implies their reception to the blessings of God's covenant in Christ.>
<15:12 Esaias saith; Isa 11:1,10. The "root of Jesse" is Christ.>
<15:13 The God of hope; the author of the hope in Christ which the prophets foretold. Habitual trust in God for all needed good is the great means of increasing joy, peace, hope, and all the graces of the Spirit in the hearts of believers; and also of leading them to abound in every good word and work.>
<15:14 Full of goodness; Paul was confident that those to whom he wrote felt kindly towards one another, and would be disposed to follow, so far as they should understand it, the will of God. Filled with all knowledge; so well acquainted with the doctrines and duties of religion, especially with regard to the subject in question, that they would be able also to admonish; or enlighten and benefit others.>
<15:15 Nevertheless; notwithstanding his good opinion of them. Because of the grace--given to me; as God had enlightened him, and made him a minister, not to Jews only, but especially to Gentiles, and as the church of Rome was composed of both, he thought it the dictate of love to write to both, and thus plainly remind them of their duty, and of such motives as were suited to induce them to do it.>
<15:16 The offering up of the Gentiles; my offering of the Gentiles to God. He figuratively compares himself to a priest, and the offering which he presents to God is the souls of the Gentiles converted through his instrumentality.>
<15:17 Whereof I may glory; ground for rejoicing and giving praise to God, that he had been made a minister and his efforts crowned with success.>
<15:18 I will not dare; as some false apostles did, who intruded themselves upon the labor of other men, and took to themselves the honor of it.>
<15:19 By the power of the Spirit of God; in working miracles and in renewing and sanctifying the hearts of men. Illyricum; a province in Europe, north-west of Macedonia, and bordering on Italy and Germany. "From Jerusalem, and round about unto Illyricum," comprehended a large portion of the then known world.>
<15:20 Not where Christ was named; his object was to preach the gospel to the destitute who had never before heard it. The ministers who go and preach the gospel to those that have never heard it, and who are successful, through the power of the Holy Ghost, in converting them to God, gathering churches, and establishing Christian institutions, are, in a high and peculiar sense, imitators of apostles, and may hope, through grace, to be distinguished partakers of their gracious and glorious reward.>
<15:21 As it is written; Isa 52:15. The course which Paul took was a fulfilment of prophecy. They that have not heard; those who had not before heard the gospel would, through such labors as those of Paul, hear and obey it.>
<15:22 For which cause; his extensive journeyings to preach the gospel. To you; the Christians at Rome.>
<15:23 Having no more place; in which to preach the gospel to those who have not heard it.>
<15:24 Spain; a country west of Italy, in the south of Europe.>
<15:25 To minister unto; to carry a contribution for the relief of their wants.>
<15:26 Macedonia--Achaia; countries of Greece.>
<15:27 It hath pleased them; to make a voluntary contribution. Their debtors they are; the Gentiles were indebted to the Jewish Christians for the gospel. Spiritual things; the blessings of salvation. Carnal things; such as would supply bodily wants.>
<15:28 Performed this; this service of carrying the contribution to Jerusalem. Sealed to them this fruit; made its benefits sure to them by delivering to them the contribution of their brethren.>
<15:30 For the Lord Jesus Christ's sake; from regard to him and the promotion of his cause. The love of the Spirit; that which he produces in the hearts of Christians towards God and towards one another. Strive together with me; in earnest, persevering prayer. Fervent, united, and persevering prayer has great influence with God, and leads him to bestow many great and precious blessings which he otherwise would not grant. The reason is, in answer to such prayer, it is in his view best to grant them; when, without such prayer, it would not be.>
<15:31 Them that do not believe; unbelieving Jews, who were everywhere opposed to him. My service; in taking to the Jewish Christians the contribution of the Gentiles.>
<15:32 Be refreshed; cheered, invigorated, and strengthened for his future labors.>
<15:33 The God of peace; the author and lover of peace, especially that peace of conscience and peace with God which passeth all understanding; peace in life, peace in death, and peace for ever. Php 4:7; Ps 37:37. If the God of peace, love, and joy be with his people, they will not want any real good; but will always, having all sufficiency in all things, be able to abound in receiving and communicating blessings, to the glory of Him of whom and through whom and to whom are all things. Chap Ro 11:36.>
<16:1 The present chapter is a beautiful illustration of the lively interest which the apostle took, not in churches alone, but also in their individual members; and as naturally growing out of this, of the extent and accuracy of his knowledge concerning them. For, in writing to a church which he had never visited, he not only salutes many by name, but accurately describes the Christian service rendered by them. Herein he is an example to all Christ's ministers. Our sister; a member of the Christian church. A servant of the church; employed in instructing the young, and in visiting the poor, sick, and afflicted. Cenchrea; the eastern seaport of Corinth, whence the apostle sent the epistle, and, as is generally supposed, by the hand of this woman. True religion unites believers, not only to Christ but to one another, in a most tender and endearing union--one which is a source of rich enjoyment, which will outlive all other unions, and be growing more delightful for ever.>
<16:2 In the Lord, as becometh saints; as a Christian, and in a Christian manner. A succorer; a helper and benefactor.>
<16:3 Helpers in Christ; assistants in spreading the gospel. Ac 18:2,3,18,26; 1Co 16:19; 2Ti 4:19.>
<16:4 Laid down their own necks; exposed their lives to great danger to save mine. Those who, from love to Christ, assist faithful ministers in their work, confer great benefits not only on them, but on the church and the world. Christians who enjoy their labors will gratefully acknowledge such benefits, and they will be acknowledged and rewarded by Christ at the great day.>
<16:5 The church--in their house; the Christians who worship there. Epenetus; he may have been a member of the family of Stephanas. 1Co 16:15.>
<16:7 My kinsmen; natural relatives. Fellow-prisoners; who had been imprisoned with him on account of their religion. 2Co 11:23. Of note; persons of distinction. In Christ; Christians. True religion does not destroy or lessen natural affection, but elevates and purifies it, and makes it the means of greatly increased usefulness and enjoyment.>
<16:9 Helper in Christ; Christian helper in promoting religion.>
<16:10 In Christ; as a Christian.>
<16:13 His mother; literally. And mine; figuratively, by affectionate care and assistance.>
<16:16 With a holy kiss; the common sign and pledge of Christian love in those days. Christian affection is always the same in its nature, but the modes of expressing it differ at different periods and in different nations. Those modes should be observed which are commonly esteemed suitable, and which are adapted to be useful.>
<16:17 Mark; carefully notice. Divisions and offences; dissensions and occasions of strife. Avoid them; give them no countenance or encouragement.>
<16:18 Serve not our Lord Jesus Christ; they do not seek his honor, but their own selfish ends. The simple; the unsuspecting and unwary.>
<16:19 Your obedience is come abroad; the report of your obedient disposition and conduct. Wise--and simple; ready and skillful to do good, but unpractised in and opposed to doing evil. In doing good, Christians should have that wisdom and skill which result from practice, experience, and habit; but they should be wholly unskilled and inexperienced in doing evil.>
<16:20 God of peace; the divine author, promoter, and lover of peace. Bruise Satan under your feet; give you the victory over him and his adherents; a victory begun in this world, but consummated in the glory of heaven.>
<16:21 My work-fellow; companion in labor.>
<16:22 Who wrote this epistle; Paul dictated and Tertius wrote it from his lips.>
<16:23 My host; the person at whose house Paul staid. The chamberlain; treasurer of Corinth, the city from which Paul wrote this epistle.>
<16:24 The grace of--Christ; his spiritual favors.>
<16:25 My gospel; the gospel of Christ which Paul preached. The mystery; the truths of the gospel, made known obscurely in the Old Testament to the Jews, were now, by the command of God, clearly revealed to Gentiles as well as Jews.>
<16:26 Made known to all nations for the obedience of faith; in order to lead them to exercise faith in Christ and be saved.>
<16:27 To God only wise; the author of all true wisdom, especially that wonderful display of it made in the gospel. As God is the author of all good, and all our mercies come through Christ, we should be disposed, for all the blessings we receive, especially for the gospel and the hope of heaven, to render to him, through Jesus Christ, glory and honor, thanksgiving and praise for ever. Amen.>
\kniha{I Corinthians}
\zkratka{1Cor}
<1:1 Sosthenes; Ac 18:17.>
<1:2 Corinth; the capital of Achaia, the south part of Greece. Sanctified in Christ; Christians. Call upon the name of Jesus Christ; pray to him. This was the practice of Christians, and distinguished them from other people. The great peculiarity of Christians, that which distinguishes them from all others, is union to the Lord Jesus Christ by faith. This union leads them to love him, pray to him, and delight to honor him by obeying his commands. It leads them also to love one another, and seek for each other, of the Father and the Son, all needed good.>
<1:3 Grace--peace; this is a prayer for spiritual blessings, and is addressed equally to God the Father and the Son.>
<1:5 In all utterance, and in all knowledge; in a comprehensive knowledge of the gospel, and the ability to unfold its doctrines and discourse concerning them with readiness and propriety. It is the constant manner of Paul, even where he has much to censure in a church, to commend it for what he finds good in it.>
<1:6 Even as the testimony of Christ was confirmed in you; as much as to say, Your utterance and knowledge of the gospel are in accordance with its truth, as it was preached by me among you. For the Corinthians, though deserving rebuke for many things, had not departed from the essential doctrines of the gospel. By the testimony of Christ is meant the apostle's testimony concerning him; in other words, the gospel which he preached, as Christ's witness for the truth. And this was not only preached but confirmed among the Corinthians; that is, it was established, and took root in their hearts.>
<1:7 Ye come behind; they were inferior to other churches in no spiritual gifts. When called to point out the faults of Christian brethren for the purpose of reforming them, it is wise freely to acknowledge their excellences, and thus by conciliating their minds prepare the way for our efforts to do them greater good.>
<1:8 Confirm you; establish and keep you in the faith and practice of the gospel. Blameless in the day of our Lord; accepted of him, and presented spotless and faultless in the great day. Jude 24.>
<1:9 God is faithful; to his promises, and will keep you by his power, through faith unto salvation. 1Pe 1:5.>
<1:10 By the name of our Lord Jesus Christ; from regard to him and his cause. All speak the same thing; be united, and refrain from contentions.>
<1:11 House of Chloe; members of her family.>
<1:12 I am of Paul; they were attached to, and ranged under different men whom they claimed as their leaders, though without their approbation; as if one were better than another, and his followers more holy. This caused divisions among them, which Paul lamented, and endeavored by this epistle to heal. Apollos; an eloquent preacher, who had visited Achaia and Corinth after the apostle. Ac 18:24-28. Cephas; the great apostle of the circumcision, whose name the adherents of the Mosaic law would be likely to use. I of Christ; the men who said this probably affected a peculiar intimacy with Christ, which raised them above the necessity of following any human teacher: for men may be proud not only of having a particular human leader but Christ.>
<1:14 Crispus and Gaius; Ac 18:8; Ro 16:23.>
<1:16 The household; the family. Chap 1Co 16:15.>
<1:17 Not to baptize; as his principal or most important business. Wisdom of words; the subtle philosophical speculations and polished rhetoric which so strongly characterized Grecian oratory. The cross of Christ; the doctrine of salvation through a crucified Redeemer. Of none effect; ineffectual to the salvation of men. The great business of ministers is to preach the gospel; and they should be careful not to muffle it with the drapery of human ornament, lest they prevent its saving effect.>
<1:18 Foolishness; they cannot discern its heavenly wisdom, and reject it as an absurd scheme, unworthy of their regard. The power of God; through which he delivers men from condemnation and ruin.>
<1:19 It is written; Isa 29:14; 33:18; Jer 8:9. Destroy the wisdom of the wise; show that worldly wisdom and efforts could never effect the salvation of men.>
<1:20 The scribe; the learned man. The disputer; the subtle, abstruse reasoner. Made foolish; showed it to be folly.>
<1:21 In the wisdom of God; after he in wisdom had suffered men to make the fullest experiments and show their utter insufficiency. The foolishness of preaching; that which those who reject Christ regard as foolishness.>
<1:22 Require a sign; a sign from heaven, some great and signal display of miraculous power. Mt 12:38. Wisdom; learned, philosophical, and literary discussions.>
<1:23 Christ crucified; the doctrine of salvation through a crucified Saviour, as the only foundation of human hope. Ac 4:12. A stumbling-block; because especially the idea of a suffering and crucified Messiah was contrary to all their preconceived notions respecting him. Compare Ro 9:32; 1Pe 2:8. Foolishness; a foolish doctrine, unworthy of their regard. The chief subjects of a minister's preaching should be the character and work of Christ, and the means of obtaining an interest in his savlation.>
<1:24 Them which are called; those who are led by the Holy Spirit to see their need of Christ as a Saviour and to believe on him.>
<1:25 The foolishness--the weakness of God; his way of salvation, which to rejecters appears so foolish and incompetent, is shown by facts to be wise and efficacious.>
<1:26 Ye see your calling, brethren; in the character and condition of those who preach the gospel, and of those who embrace it. Not many wise--mighty--noble; not many who are so considered by worldly men, are called to preach Christ, or led to believe on him.>
<1:27 27, 28. The foolish things of the world--weak things; those men and instrumentalities in this world which unbelievers regard as foolish and weak. So also base things and things which are despised. Things which are not; men and instrumentalities which are in the eyes of unbelievers as good as nothing, of no account whatever. Things that are; men and systems of philosophy of high repute.>
<1:29 Should glory; in any man or system of doctrine of man's invention. The dealings of God in selecting ministers of the gospel and subjects of his renewing grace, are calculated to humble the pride of men and lead them to feel, that for every thing wise, great, or good, and for all their success in doing good, they are indebted to his grace.>
<1:30 Of him; of the free grace of God, and by the exercise of his power. Ps 110:3; Jas 1:18; 1Jo 4:19. In Christ; united to him by faith, and for his sake entitled to receive all needed good. In him therefore they may rejoice, and in him alone, with joy unspeakable and full of glory. 1Pe 1:8,9.>
<2:1 Not with excellency of speech; that rhetorical refinement, or those subtle philosophical discussions which were admired by the Greeks. The testimony of God; concerning Jesus Christ and the way of salvation through him.>
<2:2 Not to know; to demean myself among you as one that knew nothing else; in other words, to make the doctrine of salvation through Jesus Christ, and him crucified, my only theme. Christ crucified as the atonement for sin, is the great central truth of the gospel. Other truths, in order to be rightly apprehended and have their due effect, must be seen in connection with, and in the light of this. Hence the great prominence which Paul gave to it in his preaching, and which all ministers should give to it in theirs; hence also the prominent place which this truth should hold in the contemplations of all who would grow in grace, or gain an interest in the blessings of salvation. Ga 6:14; Php 3:7-9.>
<2:3 In weakness--fear--trembling; he knew that he had many enemies. He felt deeply his insufficiency, and was fearful that he should fail of success. God, however, who knew his difficulties, had compassion on him, and encouraged him to go forward. Ac 18:6,9,10; 2Co 10:10.>
<2:4 Enticing words of man's wisdom; such as were used by heathen orators to gain applause. In demonstration of the Spirit and of power; it consisted in that demonstration of the truth which had for its foundation the accompanying Spirit and power of God.>
<2:5 Not stand in the wisdom of men; not rest on human, but on divine testimony; and be produced not by human, but by divine power. The more deeply ministers of the gospel feel their own insufficiency, and their dependence on God for success, the more likely it is that their preaching will be attended by the power of the Holy Ghost, and thus rendered effectual to the salvation of men. 2Co 12:9,10.>
<2:6 Wisdom; that which is truly wise in the estimation of God and those who are like him. Them that are perfect; who have maturity of knowledge and spiritual discernment, and are thus prepared to receive the deeper revelations of the gospel. Not the wisdom of this world; that which worldly men call wisdom. That come to naught; who perish themselves, with all their vain schemes which they oppose to the wisdom and power of the gospel.>
<2:7 We speak the wisdom of God in a mystery; in speaking the wisdom of God we proclaim a mystery. The word mystery is used here, as often elsewhere in the New Testament, to denote something beyond the power of human wisdom to discover. Even the hidden wisdom; that which had long been to a great extent unknown, but was now revealed in the gospel. Ordained before the world; purposed from eternity to reveal. Unto our glory; that it might raise us who receive it to glory. The apostle refers both to the spiritual glory which the gospel bestows upon men here, and the eternal heavenly glory in which it ends, the former being an earnest and pledge of the latter.>
<2:8 All persons, however great their advantages, who are not taught by the Holy Spirit, are exceedingly ignorant of divine things. Truths are plainly revealed of which they have no just conception, because they love darkness rather than light, their deeds being evil. In their ignorance they may commit crimes which will bring interminable evils upon themselves and others.>
<2:9 As it is written; Isa 64:4. The things which God hath prepared; it is in these things, which include all the blessings the gospel bestows on men here, and the "exceeding and eternal weight of glory" hereafter, that the glory consists spoken of in verse 1Co 2:7.>
<2:10 Unto us; the apostles and their fellow-disciples, who were taught of the Holy Spirit. Searcheth; fully understands, and therefore can reveal to us.>
<2:11 The things of a man; his unrevealed thoughts. Even so; as the unrevealed thoughts of a man are not known except to himself, so the unrevealed things of God are not known except to the Spirit of God; and he alone can reveal them. To the apostles he did reveal them, and through them they were revealed to others.>
<2:12 Not the spirit of the world; which could instruct us only in the things of this world, and make us, like itself, earthly in all our views and feelings. Might known; by his revelation. The things that are freely given to us of God; those, namely, mentioned in verse 1Co 2:9. As the Holy Spirit is fully acquainted with the mind and will of God, and is able to communicate all needed light to men, he must be divine.>
<2:13 In the words--which the Holy Ghost teacheth; the Spirit taught them not only what was to be communicated, but how to communicate it--not in preaching only, but in writing. As the Holy Ghost taught the writers of the Bible what truths to communicate and in what words to communicate them, it may safely be relied on as an exact expression of the will of God, and a perfect rule of faith and practice.>
<2:14 The natural man; the same as he who is after the flesh, and minds the things of the flesh, Ro 8:5; the man who is unenlightened by the Holy Spirit, who does not love the truth, and is the willing slave of sin. Receiveth not the things of the Spirit of God; does not rightly apprehend or appreciate them. They are foolishness; they appear foolish. Neither can he know them; he needs to be renewed and enlightened by the Holy Spirit. As without spiritual discernment no man will rightly apprehend or suitably treat the things which God has revealed, and as the author of this discernment is the Holy Spirit, all should seek his teaching; and not only attend to the words in which he communicates divine truth, but ask him to show them his meaning, cause it to make the right impression, and be the means of spiritual life to their souls. Ps 119:18; Joh 6:63.>
<2:15 He that is spiritual; he that is born of the Spirit, and therefore minds the things of the Spirit. Ro 8:5. Judgeth all things; discerneth aright all spiritual things, loves their excellence, and judges correctly concerning them. He himself is judged of no man; they who are not enlightened by the Holy Ghost, do not judge correctly concerning him. He acts from principles with which they are unacquainted.>
<2:16 For who hath known the mind of the Lord; no one but he who has been taught by the Spirit of the Lord. The natural man therefore, who has not been thus taught, cannot judge us who are spiritual, and have the mind of Christ; that is, know it, having been taught of God.>
<3:1 Spiritual; advanced in spiritual knowledge, and prepared to understand and profit by the higher and more difficult truths of the gospel. Carnal; having little religious knowledge or spiritual discernment; being still much under the influence of evil. Christians when first converted are indeed born of God, and are in some measure like him, but they are infants, not men, in the divine life. They need such instructions as are suited to those who are young, feeble, and but just beginning spiritually to live.>
<3:2 Milk; the plain, simple truths of the gospel; such as are adapted to those who are young and inexperienced in religion. Meat; truths suited to those who have made greater progress in divine things. Not able; not able rightly to apprehend and usefully to apply the more difficult parts of divine truth.>
<3:3 Carnal, and walk as men; selfish and worldly in their feelings and conduct.>
<3:4 I am of Paul--I am of Apollos; their division into parties and their violent contentions showed that they were still narrow in their views, and carnal in their feelings. Young Christians are exposed to be self-confident--to be influenced by feeling rather than judgment--to glory in men, and follow human leaders; not duly considering that they may be very zealous and earnest in efforts to increase the number and strength of their sect or party, and yet be far from that unity of spirit with Christ and his people which he requires.>
<3:5 Ministers; servants of Christ engaged in one common work, and not designed to be heads of different parties. Their object then was to convert men, not to themselves, but to Christ.>
<3:6 I have planted; Paul first preached the gospel to the Corinthians and gathered the church. Apollos watered; he came after Paul and further instructed the people. God gave the increase; all the success of both was from God.>
<3:7 He that planteth--he that watereth; preachers of the gospel are not the cause, but, under God, the instruments of their success.>
<3:8 Are one; they are engaged in one work, and for the promotion of one end, the glory of God, in the salvation of men. It is not proper, then, that they should be set up as the heads of different parties. Shall receive; from Christ, not from man. His own reward according to his own labor; the common Master of all will apportion to each his just reward, so that invidious comparisons between the different servants of Christ on the part of their fellow-Christians are entirely out of place. Ministers of Christ who are engaged in his work, are not laboring to attach men to themselves or to any human leader, but to Jesus Christ. They are all equally his servants, doing his work. And though their labor may be as needful to the salvation of men as is that of husbandmen in order to a harvest, yet their success is from God, and to him belongs the glory.>
<3:9 Laborers together with God; he as the cause, we as the instruments. Husbandry--building; the church is here compared to a cultivated field, in which husbandmen labor and God causes things to grow; and also to a building, on which he gives artisans strength to labor, and crowns their labor with his blessing.>
<3:10 I have laid the foundation; Paul first preached to the Corinthians Christ crucified, an atoning sacrifice for sin, as the only and all-sufficient foundation for human hope; and gathered such as appeared to believe on him into the church. Another buildeth; others afterwards preached to them, and admitted to their number such as professed to be converted. Take heed; let them be careful as to what doctrines they preach, and what practices they encourage; and see that both are according to the revealed will of God.>
<3:11 Jesus Christ; he is the only sure foundation of human hope; and his true church is composed of such, and such only, as trust in him. Isa 28:16; Mt 21:42; Ac 4:11; Eph 2:20; 2Ti 2:19; 1Pe 2:6. The only foundation of the true church is Jesus Christ; and none belong to it except those who believe on him. Others may have an outward connection as members, but they have no saving union with the Head. They are dead members, who will be cut off--dry branches, which bear no fruit, and will be taken away. Joh 15:2.>
<3:12 Gold, silver, precious stones; if he preach the pure truths of God, from love to him and in humble dependence on his grace, and thus build up the church. Wood, hay, stubble; if he preach error, or the speculations of men.>
<3:13 Made manifest; shown to be what it really is. The day; the day of judgment will make it known. Revealed by fire; as fire shows the difference between gold and wood, or silver and stubble, so the day of judgment will show the difference between the works of different men. A day is coming when every man's character and work will be tried. Those who have attempted to build a church on Peter or Paul, or any mere creature, or who are trusting for salvation to any outward connection with the church, without being justified by faith in Christ and governed by love to him, will be disappointed and condemned.>
<3:14 If any man's work abide; if his preaching and practice are approved, he shall receive a reward; a reward of grace.>
<3:15 If any man's work shall be burned; if it be condemned as wrong, though he himself believed and is pardoned, he shall suffer loss; he shall lose his labor, and much of the good which might have resulted from a different and better course of conduct. He himself shall be saved; yet so as by fire; as he who escapes naked from his house on fire, is saved from being consumed, but suffers loss.>
<3:16 Ye are the temple of God; elsewhere the apostle calls the bodies of individual believers the temples of the Holy Ghost, chap 1Co 6:19. Compare Isa 57:15; 66:1,2. But here, as in Eph 2:20-22; 1Pe 2:5; he has reference to the church of Christ, which is "God's building." Each true member, quickened by God's Spirit, is a living stone, and all united form a living temple, in which He dwells in a much higher and fuller sense than He dwelt in his temple of old.>
<3:17 Defile--destroy; these two words are in the original the same. If by false doctrine or unholy practice any one should defile, and thus exert his influence to destroy the church or any of its members, he would incur great guilt, and expose himself to aggravated ruin. Holy; set apart and devoted to the service of God.>
<3:18 Deceive himself; by a vain idea of his superior wisdom. Addressed especially to those who sought preeminence as leaders. Seemeth to be wise; seemeth in his own eyes, thinks himself wise. Become a fool; let him consent to be esteemed a fool by the men of the world--let him renounce dependence on that worldly wisdom for which he now values himself, feel his need of divine guidance, and seek the teaching of the Holy Ghost; receiving as true what he declares, and doing as right what he commands.>
<3:19 The wisdom of this world; that of which worldly men are proud, and in which they glory. For it is written; Job 5:12,13.>
<3:20 Again; Ps 94:11.>
<3:21 Therefore; as the result of what has been said. Glory in men; by setting up one teacher above others as his leader. All things are yours; not one teacher alone, but all the teachers of the church with all their varied gifts. And not only they, but all things else, in the sense that God makes all things work together for your good. Ro 8:28.>
<3:23 Ye are Christ's; all united in one body under Christ, who has redeemed you by his blood, and to whom alone ye belong, not to any human leader. Christ is God's; sent by the Father to redeem men, and always acting in his name and for his glory. Thus the unity in God's holy family is complete. Compare Joh 17:8.21-23, which is the best commentary on these wonderful words.>
<4:1 Ministers of Christ; not of man; whose main business therefore is not to please man, but God--compare verse 1Co 4:3--and who are not to be set up as the heads of parties. Stewards of the mysteries of God; stewards were appointed by the head of a family to provide for them and superintend their concerns. So the apostles were appointed by God to provide needful instruction for his spiritual family--to preach to them the truths of the gospel, called mysteries because they had before been comparatively unknown.>
<4:2 Fidelity to God and to the souls of men, in rightly dispensing the truths of the gospel, and in enforcing them by a uniformly holy and consistent example, is required of all ministers of Christ.>
<4:3 Judged of you; in regard to my fidelity as Christ's steward. I judge not mine own self; he was not to be approved or condemned in the day of judgment according to his own decision, any more than that of his fellow-men.>
<4:4 I know nothing before the time; namely, when the Lord shall come to judgment, as immediately afterwards stated. In the mean time they were not to decide upon and condemn the character of one another. Hidden things of darkness; those which are not seen by men. Counsels of the hearts; desires, intentions, and motives. Have praise of God; for all that he has done well. The apostle states only one side of the judgment--the approval of those who have been faithful to Christ. The condemnation of the unfaithful is implied in this.>
<4:5 Our own judgment or that of men is not a sure or safe test of our fidelity. We may not see any violations of duty, and yet He who is omniscient may see many. Hence we have need to pray, each for himself, Search me, O God, and know my heart; try me, and know my thoughts; and see if there be any wicked way in me; and lead me in the way everlasting.>
<4:6 These things; what I have said about glorying in men. In a figure transferred to myself and to Apollos; to illustrate the facts and evils of their divisions into parties, he had named only Apollos and himself as set up for leaders among them. This he did to avoid giving offence by naming others who were ambitious of such a distinction. For your sakes; that they might see their folly and renounce it. In us; by my thus putting Apollos and myself as examples. Above that which is written; that they should not think of men as any other than as they are described in the Bible, nor glory in them as leaders of separate divisions, or heads of different denominations of the Lord's people.>
<4:7 Who maketh thee to differ? as to talents, condition, character, or influence. Receive; from God. Why dost thou glory; in thyself or other men? For whatever excellence any one has, he is indebted to the grace of God; and he has no just cause to glory in himself, or to be gloried in by others.>
<4:8 Full--rich; in their own estimation. Reigned as kings without us; they imagined themselves possessed of great spiritual riches, knowledge, and power; able without the help of the apostles to succeed and prosper in their own way. Ye did reign; that ye were indeed as spiritually rich, great, powerful, and prosperous as you imagine. That we also might reign with you; then the apostle and his associates might rejoice with them in their fulness of spiritual blessings, instead of being distressed at their divisions.>
<4:9 Last; assigned to the lowest place as it regards shame and suffering, as if we were the vilest of men; "the filth of the world," and "the offscouring of all things," verse 1Co 4:13. Appointed to death; doomed to the endurance of trials and martyrdom. A spectacle; of conflicts and sufferings.>
<4:10 We are fools; accounted fools by those who boast of their wisdom. For Christ's sake; because we devote ourselves faithfully to the work of preaching Christ crucified, and seek only his honor. Ye are wise in Christ; in your own esteem. It is not safe to judge of character by outward condition or by the opinion which persons form of themselves. They may imagine that they are spiritually rich, increased in goods, and have need of nothing, when in fact they are poor, and in want of all things. Re 3:17.>
<4:14 To shame you; the apostle's severity had not for its object to overbear the Corinthians and put them to shame, but to correct their errors, as children dear to him.>
<4:15 I have begotten you; his preaching was the means of their regeneration; and as their spiritual father, it was proper for him to reprove, rebuke, and exhort them with affectionate plainness.>
<4:16 Be ye followers of me; imitate my example as your spiritual father.>
<4:17 My ways; of teaching and living, in which he proposes himself as their example.>
<4:18 Some are puffed up; elated with their imaginary importance and power, as if Paul would not dare to come to them.>
<4:19 Will know; make trial of their power to withstand his teaching and influence.>
<4:20 The kingdom of God; the reign of God in the hearts of men and in the church. Is not in word; or continued by pompous declaration and vain boastings. But in power; it consists in the power of God as manifested both in miraculous gifts, and in the awakening, enlightening, and renewing influences of the Holy Spirit operating through his preached word.>
<4:21 What will ye? would they, by disregarding his instruction, make it needful to visit them with discipline; or would they, by complying with those instructions, open the way for him to commend them and share in their joy? Christian discipline should be maintained in all Christian churches. If any of their members are guilty of immorality, and cannot by the proper use of other means be reclaimed, it is the will of God that they be excluded from the church. Mt 18:15-18; 1Co 5:4-5; 1Ti 1:20.>
<5:1 Not so much as named; except with abhorrence, as a rare crime. Members of the church who, in opposition to their profession and to all the light which they enjoy, live in sin, are sometimes suffered to commit crimes which are viewed with abhorrence even by heathen.>
<5:2 Puffed up; by their supposed excellence.>
<5:3 But present in spirit; present with you in spirit, judging and acting as though I were present with you in body.>
<5:4 In the name; by the authority. Any my spirit; acting with you as if I were present. The power of our Lord; to sanction and give effect to their proceedings.>
<5:5 Deliver such a one unto Satan; exclude him from the visible kingdom of Christ, not to destroy him, but to bring him to repentance and thus save him. For the destruction of the flesh; many think that these words imply the infliction, along with the exclusion from the church, of some bodily evil through the miraculous power conferred by Christ on the apostle, which should cooperate with the exclusion to lead him to repentance. There are but two kingdoms on earth, the kingdom of God and the kingdom of Satan. All who do not belong to the one, belong to the other.>
<5:6 Your glorying; in their supposed attainments. A little leaven; one such wicked person suffered to remain would corrupt others and injure the whole.>
<5:7 Purge out--the old leaven; the apostle alludes to the Jewish custom of carefully putting away from their houses all leaven upon the approach of the feast of the passover. Leaven is in the Scriptures a common emblem for moral corruption. The old leaven which he exhorted them to put away was the remainder of their former wickedness, as seen in this incestuous person. A new lump; fresh and unleavened; that is, pure from corruption. As ye are unleavened; profess and are under peculiar obligations to be holy. Christ our passover; the ground or reason why we are passed over, as the first-born of the Israelites were, Ex 12:23, and not destroyed, is the death of Christ. As the Israelites were to put away all leaven before partaking of the paschal lamb, Ex 12:15, which was the type of Christ, so the Corinthians were to put away all sin, that they might spiritually feed on Christ, the great antitype.>
<5:8 Keep the feast; the spiritual passover provided for us in the gospel.>
<5:9 In an epistle; generally supposed to refer to an epistle not handed down to us. As the apostle delivered many inspired discourses which it did not please the Holy Ghost to have recorded, so he may have written letters which the wisdom of God did not judge needful to make the sacred volume complete.>
<5:10 Not altogether; he did not command them wholly to abstain from intercourse with worldly men, for that would require them to retire from the world. It is not the will of God that good men should retire from the world to avoid its evils; nor is that the way to become more holy, useful, or happy. Their duty is to communicate with the wicked, for the purpose of doing them good; and to labor in the world till God shall call them out of it.>
<5:12 For what have I to do; these words give the reason for the limitation expressed in verse 1Co 5:10; as much as to say, In regard to the fornicators of this world, I propose no strict rule of absolute separation from them; they may be left to God's judgment. That are without; who do not belong to the church. Within; in the church.>
<5:13 If members of the church continue in immorality, their good, the good of the church, and the honor of Christ require that they should be excluded from it.>
<6:1 The unjust; heathen magistrates, the same as "unbelievers," verse 1Co 6:6.>
<6:2 Judge the world; be highly exalted with Christ at the day of judgment, and cooperate in his decisions.>
<6:4 Judgments; cases of difference to be settled. Of things pertaining to this life; not requiring, therefore, for their settlement the possession of high spiritual gifts. Set them to judge who are least esteemed in the church; no men who were least esteemed for integrity and intelligence in common things; but men who, not being endowed with those shining spiritual gifts so highly coveted by some--"all utterance, and all knowledge," chap 1Co 1:5--were placed by their leaders in the lowest rank. See chap 1Co 12-14. Such, according to our version, seems to be the meaning of this difficult passage. But two other renderings are proposed, which refer these words to the heathen magistrates: first, interrogatively, "Do ye set," etc. implying a stern rebuke; secondly, indicatively, "ye set," etc. The words would then be a simple statement of their wrong conduct, which the apostle proceeds, in the following verse, to censure.>
<6:7 Utterly a fault; they did wrong in going to the heathen to decide their differences. They ought to have decided them by arbitration, or in some other way, among themselves. It were better even to suffer wrong, than thus to do wrong. All differences of Christians should be settled among themselves, according to principles of equity, without appealing to civil tribunals, especially those of wicked men.>
<6:9 The unrighteous; those who are dishonest and defraud others, whether under the cover of law, or in any other way.>
<6:10 The dishonest, the unjust, the impure, and those who seek wealth, honor, or pleasure as the chief good, whatever their professions, or to whatever church they may externally belong, if they continue such, will, with drunkards, idolaters, thieves, robbers, murderers, and all the openly vicious, be for ever excluded from heaven.>
<6:12 All things are lawful; which are not forbidden of God. Not expedient; because not adapted to do good. Not be brought under the power; he would not be the slave of any appetite or passion. A wise man will govern his appetites and passions, not be governed by them; nor will he indulge them, except so far as shall tend to fit both his body and mind for the best discharge of all the duties of life.>
<6:13 Shall destroy both it and them; they are both temporary, and God shall do away both at death. It follows that meats are among the indifferent things, in respect to which the believer should be careful that he does not abuse his Christian liberty. For the Lord; the use to be made of the body is not an indifferent thing. He made and preserves it to be employed not in sinning, but in serving him. The Lord for the body; he died that not only the soul, but the body also, should be saved from the effects of sin, and raised from the grace spiritual and immortal, to serve him for ever.>
<6:15 Your bodies are the members of Christ; he has redeemed the bodies as well as the souls of believers. As such they are spiritually united to him now, and shall be throughout eternity.>
<6:17 Is one spirit; one spirit with Christ: Christ dwells in him and he in Christ. This mutual union makes him spiritual as Christ is spiritual.>
<6:18 Flee; do not stop to reason about it or think of it. Turn from it with detestation, and occupy your mind with things right and good. Is without the body; it is true of sins in general that they are without the body; in other words, do not consist in a direct prostitution and dishonor of it. Sinneth against his own body; by prostituting it in the way named in verses 1Co 6:15,16. It is the shame and guilt of the sin itself that the apostle has chiefly in mind. The dreadful physical consequences of lewdness are the brand of infamy which God puts upon it, as the greatest and most direct dishonor and abuse of the body.>
<6:19 The temple; the dwelling-place of the Holy Spirit.>
<6:20 Bought with a price; the blood of Christ. Good men feel that they are in the highest sense the property of God; and that the first principles of honesty require them in all things to glorify him.>
<7:1 The things whereof ye wrote; these were certain things about which the Christians at Corinth had, in a letter to Paul, requested him to give his views. Good for a man; best under the peculiar circumstances to which the letter of the Corinthians related. Not to touch a woman; not to be married.>
<7:2 Nevertheless; notwithstanding it may sometimes be best for persons not to be married, it is in all ordinary cases best that they should be. His own wife--her own husband; no man is allowed by God to have at once more than one wife, and no woman to have more than one husband. The union for life of one man and woman in marriage is an appointment of God, designed for the continuance and benefit of the human race. All who are in proper circumstances, and are so disposed, ought to be permitted to form such a union; and all who do form it, should faithfully discharge its duties.>
<7:3 Due benevolence; these words express the mutual duty of husband and wife towards each other, as explained in verses 1Co 7:4,5. Whatever increases temptations to evils which marriage was designed to prevent, or renders it ineffectual for the purposes for which it was instituted, should be carefully avoided.>
<7:4 Not power; not to live apart, even for a time, without mutual consent.>
<7:5 Defraud ye not one the other; deprive not one another by separation of any safeguard against temptation. Do nothing which shall tend to impurity, or give Satan advantage over you.>
<7:6 By permission; this was a subject about which persons were permitted to judge for themselves; and on which they should exercise an enlightened and conscientious Christian discretion.>
<7:7 Even as I; I wish they had the same self-control, and could live as contentedly in any condition to which Providence calls them. He evidently refers to his condition as unmarried. Proper gift of God; control over one's bodily appetites and passions is one of these gifts; but all do not have it in the same high degree.>
<7:8 It is good for them; it was well, under their then peculiar circumstances, to remain as Paul was, unmarried, provided they thought so, and could do it without inconvenience.>
<7:9 To burn; be disturbed with ungratified passion, or tempted by it to the commission of sin. God bestows on different persons different gifts, and places them in different conditions. That course which is wise for one may not be so for another. Their wishes as well as their situations and habits may be different and it may be best for them to pursue different courses.>
<7:10 Not I, but the Lord; not Paul only, but Jesus Christ. Mt 5:32; Mt 19:3-10.>
<7:12 Not the Lord; he had not given specific directions about the case which follows, and Paul, under the guidance of the Holy Spirit, proceeded to do it. Believeth not; believeth not the gospel. It is evident that some of the Corinthians had scruples about the lawfulness of living in the marriage relation with an unbeliever, and that this was one of the points on which they had asked Paul's advice. The obligations, rights, and privileges of marriage continue through life, notwithstanding any changes in religious character which may take place in either of the parties; and married persons, wherever it be practicable consistently with duty, should live together, for the purpose of promoting each other's highest temporal and eternal good.>
<7:16 For what knowest thou; another argument why the believing party should continue to live with the unbelieving. The Christian party may be instrumental in saving the other.>
<7:17 As God hath distributed to every man; assigned him his place and lot in life. The apostle here begins an exhortation to contentment in present circumstances. So let him walk; let each one continue in the condition and business to which God in his providence calls him, and do all the good he can.>
<7:18 Being circumcised; having been circumcised as a Jew. Let him not become uncircumcised; not try to undo or disavow his circumcision.>
<7:19 Is nothing; as to acceptance with God. The commandments of God; the keeping of them was all that he required, and this might be done whether a man was circumcised or not.>
<7:20 In the same calling; let a man after his conversion continue in the same business in which he was before, provided it is a moral and useful one, and the providence of God does not call him to change it.>
<7:21 Called; converted to Christ. Care not for it; be not so anxious to change your condition as to unfit you to discharge its duties. If thou mayest be made free; in the original, If thou canst be free; that is, if thou art able in doing right to be free, use it rather; be free, because freedom is a better state than servitude. In it, persons can more generally own and search the Scriptures, worship God according to the dictates of an enlightened conscience, and discharge the duties which God requires of husbands and wives, parents and children, as rational, accountable, redeemed, immortal beings. Men should continue in the situation in which God has placed them, and in the business, if it be moral and right, to which they are accustomed; unless without committing sin they can change them for the better. If they can, they are bound to do it; and in a manner accordant with the revealed will of God.>
<7:22 Called in the Lord; converted. Is the Lord's freeman; through divine power and grace, he is delivered from the condemnation and bondage of sin; and under the teaching of the Holy Ghost pursues his own free choice, cheerfully, as an affectionate child, doing the will of his Father in heaven. Is Christ's servant; not an involuntary, but a willing servant; who chooses to be his, delights in his laws, and is to receive a great reward.>
<7:23 Ye; Christians of all countries and conditions, high and low, rich and poor, bond and free. Are bought with a price; redeemed from endless bondage to sin, Satan, and death by the precious blood of Christ. Be not ye the servants of men; act from supreme regard not to them, but to Christ. Honor him, manifest his spirit in every condition, and faithfully discharge its appropriate duties.>
<7:24 Abide with God; in union with God, and acting in such a manner as to meet his approbation and enjoy his favor.>
<7:25 I have no commandment of the Lord; God had given no specific command concerning the marriage of virgins in times of great and peculiar trials. Therefore Paul, in answer to their inquiry, gave his view of the matter.>
<7:26 I suppose; in my judgment. For the present distress; on account of the peculiarly distressing circumstances in which they were then placed. So to be; to remain, during the continuance of those trials, unmarried. But some understand the words "so to be" as meaning, to be so as he is now--to remain in his present state, whether married or unmarried. And this agrees with the following verse.>
<7:28 She hath not sinned; in marrying even in those troublous times, if she chose this, and thought it best for her. Trouble in the flesh; special trials in those times of peculiar difficulty and danger. I spare you; am sparing of you; have a fatherly feeling towards you, and in giving you this advice seek your comfort.>
<7:29 Be as though they had none; live above the world and its temporary relations, remembering how soon this earthly scene, with all its joys and sorrows, will be over, and eternity come in its place. So also are the following clauses to be understood. No worldly circumstances should so disturb or occupy our minds as to unfit us in any measure for duty; nor should we desire any more worldly enjoyment than God shall graciously give us in doing his will.>
<7:31 Not abusing it; not using it to excess, depending upon it, or seeking your chief good in it. The fashion; circumstances and condition of earthly things. Passeth away; like a shadow or dream. Ps 39:6; 1Jo 2:17.>
<7:32 Without carefulness; not distracted by worldly cares. Careth for the things that belong to the Lord; is able to give his undivided attention to them.>
<7:33 Careth for the things that are of the world; he is more exposed than the unmarried, in peculiarly troublous times, to be so engrossed with cares as to be hindered from wholly following the Lord.>
<7:35 Not that I may cast a snare; his object was not to bind all to act alike, but to induce each to take the course which would be most proper, and in which he or she could best serve God.>
<7:36 Behaveth himself uncomely; acts unsuitably towards his daughter or one under his care, in withholding her from marriage. If she is of a proper age, and is disposed to be married, he would do well to consent to it.>
<7:37 Nevertheless; on the other hand. Having no necessity; if circumstances do not call for her marriage, he does well to let her remain unmarried.>
<7:38 He that giveth her in marriage; when circumstances require it, doeth well; that which is right. He that giveth her not; when circumstances do not require or favor it, doeth better; what will be more comfortable for her, that is, "for the present distress," as is to be understood throughout this chapter.>
<7:39 Only in the Lord; only within the limits of the Christian body. Others interpret these words to mean, only in the spirit of obedience to the Lord. But the former is the preferable view. So important is the institution of marriage, so honorable in all, and so numerous its blessings to those who faithfully discharge its duties, that those who, in the fear and love of God, marry, though in troublous times, do well; even in cases where, if their wishes had been different, it would have been better, at least for them, had they for a time remained unmarried.>
<7:40 So abide; continue unmarried during these troublous times, if she can so remain consistently. I have the Spirit of God; to guide me in the views expressed on this subject.>
<8:1 Things offered unto idols; the flesh of animals sacrificed to idols, on which the offerers and their friends feasted in the idol's temple, verse 1Co 8:10, and which was sometimes sold in the market, chap 1Co 10:25. We all have knowledge; namely, that an idol is nothing. The apostle refers to the claim set up by some of the Corinthians, that, because they knew the vanity of idolatry, they could eat things offered to idols anywhere, even in an idol's temple, without rendering to the idol any worship. He intimates that this knowledge is possessed by other Christians, as well as by them. Knowledge puffeth up; that is, knowledge without love. Charity; that is, love, as the word in the original is generally rendered.>
<8:2 Think that he knoweth; is puffed up with a conceit of his superior knowledge. A man proud of his knowledge is ignorant of himself, and dangerous to others.>
<8:3 Is known of him; as his friend and the object of his love and care. Compare Mt 7:23.>
<8:4 That an idol is nothing; that the deity which it represents has no existence, and that the idol is therefore a powerless and vain thing.>
<8:6 Of whom; as the source. In him; rather, as the margin, unto or for him; created to promote his glory. By whom; by whose agency. Enlightened Christians hold with unshaken confidence to the unity of God, while they acknowledge Jesus Christ as their Creator, Redeemer, and rightful Proprietor, and render to him and to the Father the homage of their hearts.>
<8:7 That knowledge; that there is but one God, and that idols are nothing. With conscience of the idol; thinking that idol deities are realities. As a thing offered unto an idol; with superstitious reverence for the supposed idol diety, as if he were a real existence. Weak; unenlightened, ignorant. Defiled; by thus joining in idol-worship.>
<8:8 Meat; food of any kind, as the original word implies. To those who have knowledge it is one of the indifferent things. For this reason they ought to be considerate and kind towards their weak brethren in respect to the food now in question, as the apostle proceeds to show.>
<8:9 A stumbling-block; an occasion of leading others into sin. No man is at liberty to do a thing merely because it is not expressly forbidden, is not wrong in itself, or will not injure him. He is bound to consider how it will affect others, and so to act as to promote their good.>
<8:10 Eat those things; as real sacrifices to idols, and thus be guilty of idolatry, and of acting in opposition to his conscience.>
<8:11 Through thy knowledge; the improper use of it.>
<8:13 If meat; if my indulgence in a particular kind of food. Make my brother to offend; be the occasion of leading him into sin. That love which leads a person to deny himself, for the sake of honoring God and doing good to men, is essential to true religion. Lu 14:24-33.>
<9:1 In the present chapter Paul asserts his personal rights and privileges as an apostle, verses 1Co 9:1-14; and then proceeds to show how he has willingly given them up for Christ's sake, therein proposing his own example for the imitation of those among the Corinthians who were disposed to insist on the use of their Christian liberty without regard to the consciences of the weak. Am I not free? as much at liberty as the other apostles in regard to the rights and privileges of my office. Seen Jesus Christ; seen him personally, and received from him my commission as an apostle. He mentions this to show that he is not behind the other apostles in regard to his apostolic office. My work; converted by my ministry.>
<9:2 Seat of mine apostleship; their conversion was evidence that Paul was an apostle of Jesus Christ.>
<9:3 Examine me; concerning his credentials as an apostle, and his privileges as such.>
<9:4 Power to eat and to drink; a right to be maintained at their expense, instead of supporting himself by manual labor. Ac 18:3; 2Co 11:7-9.>
<9:5 Power to lead about a sister, a wife; he had as good a right to be married, and have his family supported, as Peter and other apostles had. Ministers of the gospel, whether settled in Christian or missionaries to heathen lands, have a right to be married, and with their families to be supported; though it may sometimes be wise not to exercise this right.>
<9:6 Power to forbear working; to abstain from working for their own support.>
<9:8 As a man; merely on the common principles of justice. The law; the law of God. De 25:4.>
<9:9 Not muzzle the mouth of the ox; the ox had a right to support from those for whom he labored, and they could not withhold it without sinning against God, who required it of them. Take care for oxen; is it for the sake of oxen that this precept is given? that is, oxen alone, or chiefly. The apostle proceeds to show that the law in question is designed to contain a general principle for the benefit of those who labor in spiritual things.>
<9:10 For our sakes--this is written; to show that it is the will of God that ministers of the gospel should receive support from the people for whom they labor, and that all who labor are entitled to a just reward for their services. Specific directions contained in the Old Testament, even with regard to beasts and inanimate things, are often illustrations of principles, and are designed to instruct men in all ages as to the character and will of God, and the nature, variety, and extent of human duties.>
<9:11 Sown unto you spiritual things; labored for the good of your souls. Reap your carnal things; receive in return what is needful for support.>
<9:12 This power; the right to a maintenance. Suffer all things; all the inconveniences and hardships of not being supported by the people.>
<9:13 They which minister--and they which wait; the priests and the Levites, who, under the Old Testament, conducted and waited upon the religious services at the temple. Are partakers with the altar; they were supported from the offerings and contributions which the people brought to the altar.>
<9:14 Ordained; appointed; required. Live of the gospel; be supported. The support of ministers of Christ who devote their lives to the preaching of the gospel is not a charitable donation, but a debt justly due, and cannot be withheld without injustice to them, and dishonor to Christ.>
<9:15 Used none of these things; he had not required them to support him, nor did he write this to induce them to do it. He judged that such was then the peculiar state of things, that he could do more good by supporting himself. Make my glorying void; his glorying that he preached the gospel free of charge, by inducing him to take a different course.>
<9:16 Nothing to glory of; nothing from the fact of his preaching. Necessity is laid upon me; after the commission he had received from Christ, he could not, consistently with duty, refrain from preaching.>
<9:17 I have a reward; for fulfilling the charge committed to me with a willing mind. Against my will; even if reluctantly, I still have a trust which I must fulfill. When ministers of the gospel relinquish their just rights, submit to inconveniences, perform labors, and make sacrifices for the sake of doing greater good, they imitate the example of Christ, show the excellence of his religion, and may, through grace, expect from him a distinguished reward.>
<9:18 What is my reward then? in the course he was willing and joyfully pursuing for their good. It was the satisfaction arising from his disinterested labors, the approval of conscience and of God. That I abuse not my power; or right, by requiring them to support him, when this would hinder his usefulness.>
<9:19 Free from all; free from obligation to men to preach the gospel without charge. Gain the more; lead more souls to Christ.>
<9:20 Became as a Jew; complied with their customs so far as he innocently could.>
<9:21 Them that are without law; the Gentiles, who had not the written law of God. As without law; he omitted those compliances with the ceremonial law which he practised when among Jews. Under the law to Christ; bound in all things to obey him.>
<9:22 Made all things to all men; complied, in all things lawful, with their wishes.>
<9:23 Partaker thereof; of the blessings which the gospel confers.>
<9:24 Run in a race; the foot-race, with which the Corinthians were familiar.>
<9:25 Is temperate in all things; the Grecian racers subjected themselves to a very severe training, that they might thus bring their body to the most perfect condition for the race.>
<9:26 So run, not as uncertainly; not as one who runs at random, without knowing his goal or how to reach it. The apostle means, that he so lived as to be sure of obtaining the approbation of God, and receiving a crown of glory.>
<9:27 Keep under my body; literally, beat it in the face, after the manner of a boxer. This represents the severe discipline to which he subjected his appetites and passions according to God's word. Ministers of Christ who have long preached the gospel, are not on that account sure of heaven. Nor can they safely depend upon any former experience. They must habitually govern their appetites, passions, and conduct by the revealed will of God, or they will be in danger of losing their souls. If this is the case with ministers, it must be with all others; and that hope of salvation which does not lead men to obey the commands of God, will perish at the giving up of the ghost.>
<10:1 All our fathers; those who came out of Egypt.>
<10:2 Were all baptized unto Moses; shown by those signs to be under his guidance, as the acknowledged visible people of God.>
<10:3 Spiritual meat; manna, typical of spiritual blessings by Christ. Joh 6:31-35,48-51.>
<10:4 Spiritual drink; the water that flowed miraculously from the rock, and was a type of Christ. Ex 17:6; Nu 20:11. Drank of that spiritual Rock that followed them; it has been supposed that the water from the rock mentioned in Ex 17:6, followed the Israelites during their wanderings in the wilderness, till they approached Kadesh the second time. But perhaps the words "that followed them" refer to Christ the antitype, rather than to the material water that typified him. That Rock was Christ; a figure, type, or representation of Christ; as when he said, Lu 22:19, This is my body, meaning a representation of his body.>
<10:5 Many of them; who belonged to God's visible people, and enjoyed all their outward privileges. Thus the apostle warns the Corinthians not to think themselves safe from danger because they belong outwardly to Christ's church, and enjoy its ordinances and gifts. They were overthrown; Nu 14:29-35; 26:64,65. Persons may profess to be friends of God, observe his ordinances, and be favored with all external privileges, and yet fail of heaven. Unless they love God, and seek to honor him by obeying his commands, they will perish.>
<10:6 Our examples; designed to warn us against doing evil, lest we also be destroyed.>
<10:7 7-10. It is written; Ex 32:6; Nu 25:1-9; Ex 17:2,7; 15:24; 16:2-9; Nu 14:2,27-30; 16:46-49; 21:5,6.>
<10:11 They are written; in the Scriptures, as a warning to those who should live under the gospel. The ends of the world; the ends of the ages, an expression nearly equivalent to the Old Testament phrase, "the last days," by which was represented the then distant future of the Christian dispensation. Compare Heb 9:26. The judgments of God against transgressors recorded in the Old Testament, were designed to deter us from imitating their example, that we may escape their ruin.>
<10:12 Him that thinketh he standeth; securely in the favor of God. Lest he fall; into sin, and perish.>
<10:13 That ye may be able to bear; the temptation or trial, without being overcome by it. Ps 34:19.>
<10:14 Flee from idolatry; do not join in or encourage it.>
<10:15 Wise men; capable of judging correctly.>
<10:16 The cup--the bread; of which they partook in the ordinance of the Lord's supper, and in which they professed to commune with Christ. Partaking of the Lord's supper is a solemn public profession of friendship to Christ, and devotion to his service. All who unite in it should be especially careful to avoid not only the reality, but the appearance of evil, and to adorn their profession by habitual holiness of life and conversation.>
<10:17 Are one bread; as being all partakers of that one bread which represents Christ, and thus made one spiritual body in Christ.>
<10:18 Partakers of the altar; connected in a special sense with the altar, and thus with Jehovah, to whom the altar is devoted. So, should they feast in heathen temples, they would be considered as worshippers of heathen gods.>
<10:20 They sacrifice to devils; though idols are nothing, idolatry is a system under the dominion of evil spirits; and they are the real objects worshipped by idolaters.>
<10:21 Ye cannot; consistently with truth and duty. Should they join with idolaters, they would rebel against Jehovah, and provoke him to come out in judgment against them.>
<10:22 Persons cannot continue to unite with the wicked in the service of Satan, and yet be the friends of God; and those who seek their chief enjoyment in sensual gratifications, are provoking the Lord to destroy them.>
<10:23 All things; which are good to eat, may at proper times be eaten and even meat which had been offered to idols was not changed, and would not injure Paul: but it would not on that account be right for him to partake of it in idolatrous feasts, because his doing so might injure others.>
<10:24 His own; his own pleasure or profit merely. Another's; benefit, as well as his own.>
<10:25 The shambles; public markets. Asking no question; as for example, whether it may not be the flesh of an animal sacrificed to an idol. For in such a case, though it should be so, you are not thereby made in the view of men a patron of idolatry.>
<10:26 The earth is the Lord's; we may therefore use any part of it in such a manner as will honor him and do good, and we should not desire to use it in any other way.>
<10:28 Eat not; lest your example injure him who gave the information. The same action may under some circumstances be right, and under other circumstances be wrong. It is not always enough therefore to look at the action as it is in itself, disconnected from its circumstances, or at its effects on ourselves merely; but we must look also at the impression it will make and the effects it will have on others.>
<10:29 Judged of another man's conscience; why should another man make the scruples of his conscience a measure of my liberty? This, with what follows in the next verse, is said in support of the assertion just made, "Conscience, I say, not thine own.">
<10:30 By grace; rather, as the margin, with thanksgiving, namely, to God. Be a partaker; of the food set before me. Give thanks; to God, thus showing that I worship him, and no idol. In this and the preceding verse the apostle vindicates for every Christian his liberty of conscience, while in the context he urges all to use their liberty so as not to give offence to the weak.>
<10:31 Do all to the glory of God; let it be your great object to honor him, and do the greatest good in your power.>
<10:32 Give none offence; no just occasion of offence.>
<10:33 Please all men in all things; so far as is consistent with fidelity to God and to them.>
<11:1 Please all men in all things; so far as is consistent with fidelity to God and to them.>
<11:2 Keep the ordinances; the directions which I have given you. The apostle being compelled to censure certain practices in the Corinthian church, is careful to manifest towards them his candor and good will by commending their general regard to his precepts--an example which all who have occasion to censure their brethren will do well to imitate. A disposition to commend in others whatever is commendable is essential in those who are called to administer reproof; and the manifestation of such a disposition tends to prepare men to receive reproof with kindness, and to be rightly affected by it.>
<11:3 Head; rightful governor or ruler. The head of Christ is God; in the work of redemption, Christ, as Mediator, was subject to the Father, and acted in obedience to him. So Christians should be subject to Christ, and the woman to the man. It is the will of God that there should be a difference of condition, and this requires a difference in their appearance.>
<11:4 Prophesying; see note to chap 1Co 12:28. Having his head covered; the apostle regards a covering on the head as a sign of subjection. He would have the men prophecy and pray with their heads uncovered, that they may not disown the dignity which God has conferred upon them as, under Christ, the head of the human family; their uncovered heads will be a sign that they have no earthly lord. Dishonoreth his head; according to some, his own head; according to others, Christ. Both interpretations come to the same thing, since it is through the dishonor which the man puts on his own head that he dishonors Christ, by seeming thus to subject himself to an earthly head.>
<11:5 Dishonoreth her head; her husband, by appearing as if she were not in subjection to the man. That is even all one as if she were shaven; it has the same significance, and therefore puts the same dishonor upon her head. The apostle means that one thing may as properly be done as the other. But all acknowledge the latter to be a reproach to her. The former was therefore a reproach also.>
<11:6 Let her also be shorn; have her hair cut off. Let her be covered; veiled, as a token of subjection to man.>
<11:7 The image and glory of God; his representative, and reflecting his glory as ruler of this lower world. Ge 1:26-31; 2:16; The glory of the man; her excellence is an expression of his dignity and worth, since she was formed of him and for him. Ge 2:18,22,23. It is the will of God that there should be due subordination of one class of persons to another, and that this should be manifested in their dress and deportment. Our character, usefulness, and enjoyment very much depend upon suitable recognizing the relations which God has established, and acting in accordance with them.>
<11:10 Power; that is, a veil, as the token of her husband's rightful authority over her, and of her subjection to him. Because of the angels; probably the holy angels, who, as "ministering spirits, sent forth to minister for them who shall be heirs of salvation," were present in the Christian assemblies, and witnessed the propriety or impropriety of their conduct, as reflecting honor or dishonor on Christ and his cause.>
<11:11 In the Lord; according to his arrangement. Though one is subject to the other, both, on believing in Christ, are accepted of him. They are equally needful to, and should be equally respected and beloved by each other.>
<11:14 It is a shame unto him; because it makes him appear like a woman. God has made the two sexes different, and placed them in different stations; and a proper regard to him and one another requires that this difference should be seen in their apparel and deportment.>
<11:15 A covering; in the sense already explained, a token of her subjection to man. As God has made a distinction between men and women, nature and common-sense teach that in their appearance it should be duly observed.>
<11:16 Contentious; should any at Corinth contend that it was proper for women, in their worship, to appear like men, or men like women, Paul informed them that it was contrary to the teaching of the apostles and to the practice of the churches, and should be avoided.>
<11:17 In this; what he was about to mention.>
<11:18 Come together in the church; meet as a church, to worship God and celebrate the Lord's supper.>
<11:19 There must be also heresies; the word here means parties, divisions, sectaries. Such was human nature that these would exist; and one object of God in suffering them was, that it might be seen who were his friends. Divisions among professed Christians spring from their wickedness, and are productive of great evils; yet God in suffering them is wise and good. He often overrules them to show who are his true disciples.>
<11:20 This is not to eat the Lord's supper; such a mode of procedure cannot be a true eating of the Lord's supper; you cannot thus eat it in a proper or acceptable manner.>
<11:21 Every one taketh before the other; in connection with the Lord's supper they had a collation, consisting of what each one brought from his own home. This ought to have been shared by the poor equally with the rich. But instead of this, the shameful abuse prevailed here censured by the apostle.>
<11:23 Received of the Lord; it was communicated to Paul by Christ himself.>
<11:24 This is my body; a representation of it. Chap 1Co 10:4.>
<11:25 Testament in my blood; covenant, ratified by my blood. As Christ instituted the ordinance of the supper, showed the proper mode of its administration, and commanded his disciples to observe it in remembrance of him till his second coming, all should be careful to obey his command.>
<11:26 Eat this bread; Christ does not call it flesh, and it was not flesh which they ate, but it was bread, representing the flesh or body of Jesus Christ, which was broken or crucified for the sins of men; and they were to do this from time to time, as a public expression of their faith in him, and devotion to his service.>
<11:27 Unworthily; in a careless, irreverent, and wicked manner. Be guilty of the body and blood; of casting and contempt on Christ himself, resembling that which was cast upon him by his crucifiers.>
<11:28 Examine himself; as to his love to the Saviour, and his desire to honor him; as to his hatred of sin, and his longing for deliverance from it; his trust in the Redeemer, and disposition to imitate his example.>
<11:29 Damnation; judgment; he exposes himself to divine judgments. Not discerning the Lord's body; not discerning in the bread and wine the symbols of Christ's body and blood, but partaking of them irreverently, as if it were a common feast.>
<11:30 For this cause; on account of their irreverent and wicked manner of celebrating the Lord's supper. Many sleep; are dead. God had sent sickness among them, and many had died.>
<11:31 If we would judge ourselves; properly examine and decide concerning ourselves. We should not be judged; not punished of the Lord.>
<11:32 Not be condemned; the object of God in chastising his children in this world is, to lead them to repentance and reformation, that they may not in the future world be condemned. In chastising his people for their sins, God is kind. He does not afflict them because he delights in it, but for their profit, that they may be partakers of his holiness, and thus escape endless condemnation. They should therefore in trials be submissive, search out and forsake their sins, be grateful for mercies, and commit themselves and all their interests to his gracious disposal.>
<11:33 To eat; in the celebration of the Lord's supper. Tarry; let no one partake till others are ready, and do all things decently and in order, to the honor of Christ and their own spiritual good.>
<11:34 The rest; other things which might need correction, Paul would regulate when he should visit them.>
<12:1 The rest; other things which might need correction, Paul would regulate when he should visit them.>
<12:2 Ye were led; by Satan and those under his influence. These words seem to contain the reason why the Corinthians should receive instruction respecting spiritual gifts. They have just come out of the darkness and ignorance of idolatry.>
<12:3 I give you to understand; he begins by stating a general rule whereby to test the genuineness of all alleged spiritual gifts. They all unite in putting supreme honor upon Christ. Compare 1Jo 4:1-3. Calleth Jesus accursed; as an imposter. Can say that Jesus is the Lord; that is, say it in sincerity, with a true apprehension of the meaning of such a confession; in other words, acknowledge and receive him as the Messiah. Compare Mt 11:25-27; Mt 16:16,17.>
<12:4 Diversities of gifts, but the same Spirit; throughout the whole of this chapter the apostle labors to show, first, the unity of these gifts, as all having the same Spirit for their author, and all conspiring for the same common end, the glory of God in the edification of the church; secondly, their variety, as having different outward forms, and designed to accomplish different specific objects.>
<12:5 Differences of administrations; or offices, which God has established, verses 1Co 12:28-30.>
<12:6 Diversities of operations; such as produce in different cases different effects.>
<12:7 The manifestation of the Spirit; in the gifts which he bestowed. To profit withal; for the benefit of men. All Christian gifts and graces come from the Holy Spirit. He bestows different measures of grace and means of influence upon different individuals, but always for the wisest reasons.>
<12:8 8-10. Diversities of gifts and offices, spoken of in verses 1Co 12:4,5. The word of wisdom--the word of knowledge; the exact distinction between those two gifts has been a matter of doubt. Probably "wisdom" refers rather to the practical, and "knowledge" to the doctrinal in Christianity. According to this view "the word of wisdom" would be peculiar skill in explaining to men the way of life, exhibiting the motives to induce them to walk in it, and guiding their conduct in difficult situations: "the word of knowledge," on the other hand, would be that which comes from a deep insight into the doctrines of the gospel, including an understanding of the prophecies, types, and spiritual meaning of the Scriptures, and their true application and fulfilment. Faith; in a special sense. We are probably to understand an extraordinary measure of confidence in God, such as raises its possessor above the fear of man, and inspires him with the firm hope of success in the midst of dangers, difficulties, and hinderances. Gifts of healing; power to cure diseases. The working of miracles; in a general sense, miracles not confined to the healing of diseases. Prophecy; the inspired utterance of God's will in respect to the way of salvation, including also, as an occasional part of it, the foretelling of future events. See note to verse 1Co 12:28. Discerning of spirits; whether men who professed to exercise spiritual gifts, were guided by the Holy Ghost or by a false spirit. Compare 1Jo 4:1. Divers kinds of tongues; power to speak various languages. Interpretation of tongues; power to translate, or tell the meaning of one language in the words of another.>
<12:11 Dividing to every man; bestowing different gifts and in different measures upon different persons, as the Holy Spirit sees best. In bestowing miraculous powers upon the apostles and first teachers of Christianity, in calling them to their office, assigning them their fields of labor, fitting them for their work, and giving them success, the Holy Ghost has shown himself to be God; and with the Father and the Son, entitled to divine honors.>
<12:12 So also is Christ; the head of his spiritual body the church. All its members, like the different members of the human body, are united to one head, and should be, by mutual sympathy and affection, united to one another.>
<12:13 By one Spirit; the Holy Ghost. To drink into one Spirit; or, to drink of one Spirit. By his operation on our hearts, we become united in spirit to Christ the head and to one another as members of his spiritual body the church.>
<12:18 As God assigns to Christians their talents and opportunities, their condition and measure of influence, as will best promote his glory and the good of his kingdom, there is not more reason for strife among them as to which shall be the greatest, than among the members of the human body. The perfection and highest usefulness of each consists not in his possessing the talents, exerting the influence, or doing the duties of another, but in rightly discharging his own.>
<12:22 22-26. Are necessary; the strongest and most prominent members of the body are not in all cases the most essential to human life, but often those which are more feeble and concealed. Those parts which need it, we cover and adorn; and we never think of neglecting, much less of despising any part because it is feeble, or needs special care. God has so ordered, that if one member, even the most feeble or uncomely, suffer, all suffer with it; and if one rejoice, or is in health and vigor, all experience the benefit.>
<12:25 No schism; no division or contest with one another. They have one common interest, and the welfare of each is for the good of all. So it should be with the different members of the church of Christ.>
<12:27 Ye are the body of Christ; ye, the whole body of believers, constitute the one spiritual body of Christ. Members in particular; each individual is a member of Christ's body, having his particular office assigned to him by the one common Head. As all have a common interest, and are parts of one great whole, all should have a common sympathy; and by fidelity to Christ in the discharge of appropriate duties, labor for the general good.>
<12:28 28-30. These were different offices which God established in the churches at first; referred to in verse 1Co 12:5. Apostles; men who had seen Christ after his resurrection, and were commissioned by him to testify to this fact, to reveal his will, work miracles in attestation of his truth, preach the gospel, gather churches, and do what was needful for the establishment of Christianity. Prophets; those who had the gift of prophecy. The apostle here assigns them a rank next to the apostles, and elsewhere he puts prophecy first among spiritual gifts. Chap 1Co 14:1, etc. Like the prophets of the Old Testament, they spoke under the immediate inspiration of the Holy Spirit, Ac 2:17: like them, they unfolded to men the counsels of God, especially as contained in the way of salvation through Christ, Ac 13:1,2; 15:32; and like them they also at times foretold future events, Ac 11:28; 21:11. Teachers; of the gospel. The term probably includes those who had "the word of wisdom" and "the word of knowledge," verse 1Co 12:8. Helps; persons appointed to assist in visiting the sick, instructing the ignorant, and relieving the needy. Governments; persons who directed the external order of the church.>
<12:31 Covet; earnestly desire. They had coveted what was most esteemed by men; but Paul would have them desire what was most esteemed by God, and without which, whatever else they might have, they would be destitute and wretched. What this was he proceeded to show. A more excellent way; a way preeminently excellent; namely, the way of love, which he proceeds in the next chapter to point out.>
<13:1 Charity; love to God and to men. Sounding brass; empty and worthless. As love to God and to men is the best gift which God bestows, all should most earnestly desire and cherish it in themselves and in others.>
<13:2 Gift of prophecy; see note to chap 1Co 12:28. Understand all mysteries, and all knowledge; have the power of unfolding all the deep counsels of God as contained in his word, and of declaring by inspiration what has been hitherto concealed from the world. All faith; the faith of miracles.>
<13:3 Men often eagerly pursue and glory in what will be to them of no permanent benefit. Nothing which they do or possess, without love to God and to men, will save them from perdition.>
<13:4 Suffereth long; with patience, under injuries. Is kind; not to friends only, but to foes. Envieth not; is not uneasy at the prosperity of others. Vaunteth not; does not boast of its own excellence.>
<13:5 Seeketh not her own; as the chief end; is not selfish, but benevolent. Thinketh no evil; is not disposed to impute to others evil designs.>
<13:6 Rejoiceth not in iniquity; as is done by the selfish, whenever iniquity in themselves or others can be made subservient to their own personal interests, or the destruction of their enemies. Rejoiceth in the truth; in the reception and propagation of it, whoever may be the instruments, and whatever the effects on us.>
<13:7 Beareth all things; inflicted by others, so far as is consistent with duty, without being disposed to publish their misconduct or to punish it. Believeth--hopeth all things; is disposed to put the best construction upon men's conduct, and hope the best concerning them. Endureth all things; which it may be called to suffer in the path of duty. Love is so active in its nature and marked in its effects, that none need or ought to be in doubt whether they possess it.>
<13:8 Never faileth; it will continue to eternity; while the gifts of foretelling future events, or of miraculously speaking with tongues, or by inspiration understanding and communicating divine truth, will soon pass away as no longer necessary.>
<13:9 We know in part; that is, in our present state. We prophesy in part; we are not capable of either receiving or communicating truth by prophecy, except in a partial and imperfect way.>
<13:10 That which is perfect; the perfect knowledge of heaven. That which is in part; our present imperfect knowledge, with our present imperfect means of gaining it through prophecies, tongues, etc.>
<13:11 A child--a man; as the conceptions and speech of a lisping infant differ from those of an educated and full-grown man, so do our highest attainments in this life differ from what they will be in the life to come. Childish things; the imperfect conceptions and reasonings of a child. Supply, So in heaven we shall put away our imperfect conceptions of God's truth, and our imperfect helps for gaining it.>
<13:12 Now; in our present earthly state. We see through a glass, darkly; our knowledge of God and divine truth is indirect and obscure, like that of a man who looks not directly on the object itself, but only on a dim image of it, such as was reflected from the imperfect mirrors of the ancients. But then; in heavenly state. Face to face; that is, immediately, and clearly, as one looks on the face of another. Compare Nu 12:8; and Ex 33:11; De 34:10. Shall I know; God and divine truth. Even as also I am known; more literally, even as also I have been known; that is, known by God in the present state; where our knowledge of him, though real and saving, is yet so faint and imperfect, that it may be better said that we are known of God, than that we know him. Ga 4:9. The apostle plainly has in mind not the extent of our knowledge in the heavenly state, but rather its manner, as direct and clear.>
<13:13 Abideth; according to some, will outlive all miraculous gifts; according to others, will abide for ever: faith in the sense of confidence in God and Christ; and hope, as the joyous looking forward to an eternity of ever increasing blessedness. The greatest of these is charity; not only in itself, but because it is the root and ground of the other two. Things which are only temporary should never awaken our deepest interest or be our chief concern. Nothing should do this that will not last for ever.>
<14:1 It is plain from the present chapter that the Corinthians measured the worth of the several spiritual gifts not so much from their power to edify the church, as from their adaptedness to strike the beholders with wonder. For this reason they were ready to put the gift of speaking in tongues above that of prophecy. This erroneous judgment the apostle now proceeds to correct.>
<14:2 Speaketh not unto men; conveys to them no instruction or edification. But unto God; who understands his spirit, and with whom he thus has communion. Speaketh mysteries; the mysteries of the gospel; its deep truths which have heretofore been hidden, or but dimly revealed.>
<14:4 Endowments are valuable in proportion as they are useful, and it should be the object of all, in the exercise of their talents, to do good.>
<14:5 Greater is he that prophesieth; because his office is more useful to the church, verse 1Co 14:3. Except he interpret; from this and verse 1Co 14:13 it is manifest that the two gifts of speaking with tongues and of interpreting them chap 1Co 12:10, might or might not be possessed by the same person. When a man had the gift of speaking with tongues without the power to interpret them, some think that he was unable to understand his own utterances. But the meaning seems rather to be, that though he spoke intelligibly to himself, he had not the gift of interpreting intelligibly to others. See further on verse 1Co 14:14.>
<14:6 By revelation; so as to make to you a revelation. By knowledge; so as to communicate to you knowledge. By doctrine; teachings in the ordinary way, as distinguished from the utterances of prophecy. Unless he declared to them truths which they could understand and apply to practice, he would do them no good.>
<14:7 A distinction; the meaning of which shall be understood by the hearers.>
<14:8 An uncertain sound; the meaning of which is not understood.>
<14:9 Easy to be understood; better, as the margin, significant, that is, to the hearers. Speak into the air; your words shall be thrown away. All the services in public worship should be in language understood by the worshippers.>
<14:10 None of them is without signification; they all have meaning, and were designed to be understood; each language should therefore be used with those only who understand it.>
<14:11 A barbarian unto me; a foreigner, whose language is not understood.>
<14:12 Are zealous; earnestly desire. Excel to the edifying of the church; abound in these gifts in such a way as to edify the church.>
<14:13 Pray that he may interpret; either, that God will add to him the gift of interpretation; or, as the context seems rather to require, pray in such a way that he may interpret; namely, by adding interpretation to his prayer.>
<14:14 My spirit prayeth; goeth forth to God in holy and fervent desires, and is thus edified, verse 1Co 14:4. My understanding is unfruitful; according to one view, it bears no fruit to myself, since it is not enlightened by what I utter; according to another and preferable view, it bears no fruit to others, since it communicates nothing to them in an intelligible way.>
<14:15 With the understanding; with the right use of it, in words which the hearers understand. The object of a wise and good minister is not to display himself, but to impart instruction, and thus be useful to those who hear him. Of course, he will not conduct any part of the public service in a language which his hearers do not understand.>
<14:16 Bless with the spirit; praise God in language which others do not understand.>
<14:20 Not children; not carried away with sound and show, but men who look not to show, but to the edification of their brethren. To be carried away or greatly influenced by sound, external display, or any thing pertaining to manner merely, without enlightening the mind or purifying the heart, is childish, unworthy the character of men, especially of professed Christians.>
<14:21 It is written; Isa 28:11,12. Other tongues; foreign languages. In this passage Jehovah threatens the men who treated with scorn the messages of their own prophets, who spoke to them in their own language, that he will speak to them by the lips of foreigners; namely, by giving them up to their dominion. The point on which the apostle insists is, that here foreign tongues are made a sign to unbelievers.>
<14:23 All speak with tongues; in foreign languages only. Ye are mad; appear deranged to him who does not understand you.>
<14:24 All prophesy; in language intelligible to those who hear. He is convinced; convicted of sin, and condemns his former course of life.>
<14:25 Falling down on his face; as penitents were wont to do in confessing their sins and imploring mercy. In you; among you, in your assembly. The truths of the gospel plainly and kindly declared, are often so attended by the influences of the Holy Ghost, that persons who come to a place of public worship out of curiosity, or to ridicule the preacher and scoff at religion, are convicted of sin, led to condemn themselves, and join with those whom they came to oppose in sincerely worshipping God.>
<14:26 Come together; in public assemblies for instruction and worship. Every one; one had a psalm to sing; another a doctrine or truth to inculcate; another a discourse to deliver; another an interpretation to give; and all were eager to speak, not considering that the object of each should be to do good, and the exercises should be so conducted as to be suited to this end.>
<14:27 By two, or--three; let not more than two or three speak at the same meeting, and these not together, but in succession; and let some one as they proceed give the meaning, that all the hearers may understand it.>
<14:28 Let him; who can speak only in an unknown tongue. Speak to himself; in silent meditation and prayer, but not utter in public what his hearers will not understand. Unless a minister speak in a language which his hearers understand, or some one interpret to them his meaning, it is the will of God that he should not speak at all. Suppose he speaks in Latin, and says, "This is not an unknown tongue, but is perhaps the best known in the world;" if his hearers do not understand it, he sacrifices their good and violates the revealed will of God.>
<14:29 The other; in the original, the others: those who hear, let them judge whether what they hear is according to the word of God.>
<14:30 Be revealed; by the Holy Ghost to one who is not speaking, as a thing which he ought to declare. Hold his peace; so that all shall have an opportunity to speak in succession.>
<14:31 Ye may all prophesy; each may speak in his turn, and thus all be benefited.>
<14:32 Are subject; they were able to control themselves in this matter. Though they were inspired, there was no need of more than one speaking at a time. The Holy Spirit by his influences does not lessen a man's control over himself, but increases it, and leads him to do, not things which are unsuitable, but those which in themselves are right, and in their tendency useful.>
<14:33 God is not the author of confusion; the Holy Spirit, by his inspiration, does not lead to it, nor does he approve it. All under his influence can and ought to avoid it. Of peace; order, harmony, and love, as is manifest in all well-regulated churches.>
<14:34 In the churches; in the public religious assemblies.>
<14:35 If they will learn any thing; beyond what they can by hearing. It is a shame; because it is stepping out of their proper place, assuming what does not belong to them, and acting in this respect as if they were men. It is doing what God forbids.>
<14:36 What! came the word--unto you only? were you the first to receive or spread the gospel, that you, in the above-mentioned things, act so differently from other churches, as if you were wiser than they? By no means. Other churches have been much longer established, have had greater experience, and are more worthy of imitation.>
<14:37 Prophet, or spiritual; under the special guidance of the Holy Ghost, and qualified to judge in such matters. Let him acknowledge; if he does not, it will show that he does not think right. Spiritually minded persons, who are influenced by the Holy Ghost, and rightly understand truth and duty, will acknowledge that the directions given by Paul about ministers' speaking in a language known to their hearers, about women's keeping silence in the church, and about the avoidance of all tumult and confusion in public worship, are commandments of God which all should obey.>
<14:38 Be ignorant; of the fact that the directions which Paul gave were the commandments of God. Let him be ignorant; it was not best for them to debate the matter further; but leave him to meet the consequences of his conduct.>
<14:39 Covet to prophesy; as the best gift. Forbid not to speak with tongues; as a gift good in its proper place, and when exercised in a proper way.>
<15:1 I declare unto you the gospel; I make once more a statement of the gospel. He says this with especial reference to one of its great foundation doctrines, the resurrection of Christ, and in him, of all his believing people. Wherein ye stand; upon which your church is founded, and upon which all your hopes rest.>
<15:3 Received; by inspiration, and directly from the Lord Jesus Christ. Christ died for our sins; on account of them; as an atoning sacrifice, the just for the unjust. 1Pe 3:18. The Scriptures; as foretold of him in the Old Testament. Ps 22:1-31; Isa 53:1-9; Da 9:24-26; Zec 12:10; 13:7.>
<15:4 The Scriptures; Ps 16:10,11; Isa 53:10-12; Ho 6:2.>
<15:5 Cephas; Peter. Twelve; the apostles.>
<15:6 Brethren; disciples of Christ. Remain; are now alive. Are fallen asleep; are dead.>
<15:7 James; supposed to be the James who wrote the epistle, and who was called James the less.>
<15:8 Out of due time; as by an untimely birth. A proverbial expression to denote unworthiness.>
<15:9 The least of the apostles; least worthy, or most unworthy and guilty. Eph 3:8. Not meet; not worthy, or fit, Ac 9:1; 26:9-11.>
<15:10 Not in vain; it was not inoperative; it led me to labor more than they all; than any of the apostles. Not I; not on account of any thing spiritually good naturally in him. Mt 10:20.>
<15:11 They; the other apostles. We preach; the same great truths--Christ crucified for the sins of men, and raised for their justification. Ro 4:25. That Jesus Christ died for the sins of men, and rose again for the justification and salvation of all who believe on him, are fundamental doctrines of the gospel, and are taught as such by all who are prepared and called by Christ to preach it.>
<15:12 Is no resurrection of the dead? no such thing as a resurrection of the dead? The men against whose error the apostle wrote, denied, doubtless on philosophical grounds, the possibility of a resurrection from the dead. He proceeds to show, first, that such a denial involves the denial of Christ's resurrection, and consequently of the gospel itself with all the hopes that are built on it, 1Co 15:13-19,29-32; secondly, that the certain fact of Christ's resurrection is an earnest and pledge of our resurrection also, verse 1Co 15:20-28. He then proceeds further to meet objections to the doctrine, and make various explanations and revelations concerning it.>
<15:14 Vain; useless, because not true.>
<15:17 Your faith is vain; it cannot save you. Ye are yet in your sins; unpardoned, because no one can be pardoned except through the atonement and righteousness of Christ, which, if he has not risen from the dead, never has been accepted; and there is no way of salvation, except by one's own works, which to sinful man is impossible.>
<15:18 Fallen asleep in Christ; died believing in Christ's resurrection, and expecting to be saved through him. Are perished; since no salvation has been provided for them, they have died under the curse of God's law, which is the death of the soul.>
<15:19 Most miserable; because we have exposed ourselves to all manner of sufferings and hardships to no purpose, and all our hopes are destined to end in disappointment.>
<15:20 Now is Christ risen; the apostle comes now to the triumphant assertion of the fact, the proof of which he has already stated. First-fruits; a pledge that all united by faith to Christ would rise again. Them that slept; the pious dead. Elsewhere he affirms the resurrection of both the just and unjust, Ac 24:15; but here he dwells more particularly on the resurrection of the dead in Christ. That Christ arose from the dead, God has shown to be certain, and with equal certainty that all his people who die will rise also.>
<15:21 By man; Adam. By man; Jesus Christ.>
<15:22 In Adam all die; all having become sinners through him, as is stated in Ro 5:12,17-19. In Christ shall all be made alive; he shall raise to life the whole human family, Joh 5:28,29; but here the apostle has especially in view the resurrection of the righteous.>
<15:23 Christ the first-fruits; he is the first who rose from the dead to die no more; and his resurrection was a sure pledge that his people at is coming will likewise rise, to live and reign with him for ever. Mt 25:34,46; Joh 14:19.>
<15:24 The end; of the present state of things--the day of judgment. Delivered up the kingdom; that which he received as Mediator, having completed the work for which he received it. Put down all rule--authority--power; conquered all enemies.>
<15:25 Must reign; as Mediator till then, in order to fulfill the predictions of scripture concerning him. Ps 2:6-12; 46:3-7; 110:1.>
<15:26 Death; Christ will abolish or destroy this, when, at the general resurrection, he delivers his people from its power.>
<15:27 He saith; Ps 8:6. He; God.>
<15:28 The Son--be subject; the chief object of his Mediatorship will then have been accomplished.>
<15:29 This verse is connected in argument with verse 1Co 15:19. Else; if there is no resurrection, what is the advantage of being baptized and exposed to innumerable dangers, and even to death itself, in hope of one? Baptized for the dead; according to some, the meaning is this: Why, when many for their attachment to Christ are put to death, do multitudes in the face of death openly profess by baptism to be his disciples, and thus take their place?>
<15:30 And why; do we who have professed this continue, without wavering, to brave cheerfully all its dangers?>
<15:31 I protest; solemnly affirm. I die daily; am daily exposed to death, on account of my attachment to Christ as a crucified and risen Saviour.>
<15:32 Fought with beasts; exposed myself to instant and violent death. Let us eat and drink; if there is no resurrection to eternal life, let us avoid all the pain and enjoy all the pleasures we can now, according to the maximum of those who live only for the present world.>
<15:33 Be not deceived; by the false opinions and reasoning of wicked men. Evil communications; familiar canversation with the wicked in corrupting. This was a sentiment expressed by Menander, a Greek poet, whom Paul quoted.>
<15:34 Awake to righteousness; the original is very strong: Awake out of your intoxication, namely, with sin and error. Let the certainty of retribution arouse you to duty, and restrain you from sin; for some among you have not that knowledge of God which leads them to believe and obey him. To your shame; for they had means and opportunities to know and do better.>
<15:35 Some man; who, because he cannot understand the manner in which men will be raised, or with what bodies, concludes there will be no resurrection.>
<15:36 Fool; measuring your faith by your ignorance, and because you cannot understand the manner, rejecting the fact; when there is nothing more unaccountable in the case of the resurrection-body, than there is in the quickening of a seed sown, through its death, into a new body. Quickened; made to live and grow into a new plant.>
<15:37 Bare grain; a naked kernel of grain.>
<15:38 Giveth it; the seed sown. To every seed his own body; so that each grain preserves its identity, wheat producing wheat, and barley, barley. In this illustration three things are to be noticed: first, the seed sown is not quickened into a new plant except it die, that is, be itself dissolved and perish, as it always does in germination; secondly, the new plant with its seed is not the grain itself that was sown; yet, thirdly, it is the same in kind, and thus preserves its identity, each seed reproducing its own body. So the heavenly body that shall spring from the death of this earthly body, though not that body of flesh and blood that was sown in the grave, shall yet be the same body in such a sense that at the resurrection every one shall receive again his own body.>
<15:39 39-41. All flesh is not the same flesh--celestial bodies, and bodies terrestrial--one glory of the sun, and another glory of the moon; the apostle introduces another argument, drawn from the variety that exists by the power of God among natural bodies, to show how easily he may cause the bodies raised to differ greatly from those that died. The contrast between terrestrial and celestial bodies seem to hint at that between our present and our future bodies, as given in 1Co 15:42-54. Some have thought that in referring to the difference in glory among the heavenly bodies, he has in view different degrees of glory and blessedness among the righteous in heaven. This, however, he does not afterwards insist on, but only the great contrast between the corruptible body and the incorruptible.>
<15:42 Many things take place, the manner of which men cannot understand; and for them to disbelieve what God has declared, because they cannot understand the manner in which it will be accomplished, is great folly.>
<15:43 In power; strong, and not subject to disease or death.>
<15:44 A natural body--a spiritual body; in the original Greek the word rendered "natural" is the adjective corresponding to the word rendered "soul" in verse 1Co 15:45. In order better to understand the force of the quotation in that verse, we might render the present thus: It is sown a soul-body; it is raised a spirit-body: the word soul being used as it is in Ge 2:7, to denote man in his present earthly state as inhabiting an animal body, and subject to animal passions and wants; while the spiritual body will have no animal nature, and be subject to no animal wants.>
<15:45 It is written; Ge 2:7. The quotation extends only to the first clause of the verse. The first man Adam; whose nature we all inherit. Was made a living soul; see note to verse 1Co 15:44. The last Adam; Christ; to the nature of whose heavenly body our spiritual bodies will be made like. A quickening spirit; a spirit having life in himself, and bestowing spiritual life and a spiritual body upon all who are his.>
<15:47 The Lord from heaven; Isa 9:6,7; Mal 3:1; Joh 17:5; 2Co 8:9; Php 2:6. He is therefore not earthly, but heavenly in his nature.>
<15:48 Such are they also that are earthy; descendants of Adam, and naturally like him in body and soul. That are heavenly; friends of Christ, like him in body and spirit. Php 3:21.>
<15:49 The image of the heavenly; in both soul and body. Ro 8:29; 1Jo 3:2; Php 3:21.>
<15:50 Flesh and blood cannot inherit; our bodies must undergo a change, such as is effected in the resurrection, in order to fit them to live in heaven.>
<15:51 Not all sleep; Christians who shall be living at the end of the world will not die, but will experience a change similar to that which those who have died will experience in the resurrection, that they may be spiritual, incorruptible, and immortal.>
<15:54 Then; when the dead have been raised, and the living so changed as to fit them to live and reign with Christ. The saying; shall be fulfilled that is written in Isa 25:8.>
<15:55 Thy sting; that by which thou didst terrify men. Ho 13:14. Thy victory; by which thou didst hold men as vanquished.>
<15:56 The sting of death; that which makes death terrible, is sin. Ro 4:15; 6:23.>
<15:57 The victory; over sin, death, and every foe. Ro 7:25; 8:1,37.>
<15:58 Steadfast; in the faith and practice of the gospel, in habitual lively confidence of the resurrection, the day of judgment, and the retributions of eternity. Unmovable; not discouraged by opposition or difficulties; not led even to doubt about the complete fulfilment of all which God has declared. In the work of the Lord; in labors to honor him and do good. Your labor is not in vain in the Lord; what you do to honor Christ shall receive a glorious and an eternal reward. The certainty of the resurrection, of the day of judgment, and the retributions of eternity, should lead all to make it their great object to learn and do the will of God; hearkening daily to his voice, believing heartily his declarations, and obeying cheerfully and perseveringly his commands.>
<16:1 Collection; for the relief of poor saints at Jerusalem. 1Co 16:3. Faithful ministers of the gospel will, if practicable, induce their hearers liberally to contribute for the benefit of the needy; and regular, systematic beneficence will, in the end, be much more abundant than that which is merely occasional, and much more useful, both to givers and receivers.>
<16:2 Upon the first day of the week; the day set apart and observed by the apostles and Christians as the Lord's day, the Christian Sabbath. Lay by him in store; at home. That there be no gatherings; that their gifts might be ready when the apostle should come. As the first day of the week is the Lord's day, and from his resurrection has been observed by his people as sacred to his worship, it is a proper time for them to consider his goodness, and contribute, or lay by in store, as he has prospered them, for the supply of the wants of their fellow-men.>
<16:3 By--letters; this may mean letters from the brethren at Corinth, or letters by Paul, commending the messengers to his friends at Jerusalem.>
<16:5 When I shall pass through Macedonia; rather, when I have passed through Macedonia. This was an alteration of his original plan, which he had in some way made known to them, and for this some in Corinth charged him with changeableness of purpose. 2Co 1:15-17.>
<16:7 By the way; on his way to Macedonia, according to his first plan. See above. Such a visit would not only have been brief, but would have brought him to Corinth before the present epistle could have had time to produce its intended effects. See 2Co 1:23; 2:1-3. Tarry a while; on his return. In forming plans for future action, we should ever remember our dependence on God, seek to understand his will, and commit ourselves in well-doing to his merciful guidance and disposal.>
<16:8 Pentecost; this feast was celebrated in June, fifty days after the Passover, which was in April. Ac 2:1.>
<16:9 A great door and effectual; a great opportunity for successfully preaching the gospel. Many adversaries; opposers of Paul and his preaching. This made it necessary that he should remain for the present at Ephesus, where he wrote to the Corinthians this epistle.>
<16:10 Without fear; occasioned by opposition or neglect on your part. He worketh; is a wise and faithful minister.>
<16:11 Conduct him forth in peace; when he has finished his work among you, and is prepared to leave Corinth. With the brethren; whom he expected to come from Corinth to Ephesus. Compare Ac 19:21,22.>
<16:12 Christian brethren, and even pious and faithful ministers of the gospel, may differ in judgment about the best way of doing good; and while they exercise the right of private judgment as to their own duty, they should cheerfully concede the same privilege to others.>
<16:13 Watch; against temptation. Stand fast; in the faith and practice of the gospel. Like men; act in a manly and not a puerile manner. Be strong; in the grace which is in Christ Jesus.>
<16:14 Charity; love to God and men.>
<16:15 House; family. First-fruits; the first persons who were converted in Achaia by the ministry of Paul.>
<16:16 Submit yourselves; treat them with respect, and be suitably influenced by them.>
<16:17 Coming of Stephanas; he had come to Ephesus, but his family remained at Corinth. Verse 1Co 16:15. That which was lacking on your part; in ministering to my comfort. That which, by reason of my separation from you, you could not bestow, they have given by their personal presence.>
<16:18 Acknowledge ye them; as friends of Christ, and worthy of imitation.>
<16:19 The church--in their house; the Christians who worshipped there. The meeting and conference of Christians from different and distant places may be the means not only of their own comfort, but of their increased usefulness to one another, and to their fellow-men.>
<16:21 The salutation of--Paul; he employed the hand of another in writing the previous part of the epistle, but this and what follows he wrote with his own hand.>
<16:22 Anathema; accursed, that is, of God. Maran-atha; the Lord cometh, namely, to judgment. This addition to the anathema contains a solemn intimation of the time when it will be fulfilled. Was written from Philippi; the superscriptions to the epistles are not a part of the inspired Scriptures, but were added at a later period, and contain errors. From verse 1Co 16:8 of this chapter it appears that Paul wrote this epistle from Ephesus.>
<16:24 When called to administer reproof, or to proclaim the fearful doom of the incorrigibly wicked, while we should endeavor to do it with fidelity, we should also do it with affection, and in all suitable ways show that it springs not from enmity or indifference, but from love; and that it is our earnest desire that even our greatest opposers may so conduct, that the grace of our Lord Jesus Christ shall be with them now and for ever. Amen.>
\kniha{II Corinthians}
\zkratka{2Cor}
<1:4 The afflictions and consolations of faithful ministers are designed to prepare them for giving instruction and comfort to the afflicted.>
<1:5 The sufferings of Christ; sufferings like those of Christ, or endured in Christ's cause. All Christ's disciples, each in his own measure, must first suffer with Christ, that they may afterwards be partakers of his glory. Ro 8:17; 2Ti 2:12.>
<1:6 It is for your consolation and salvation; our affliction is endured in behalf of you, as of all the churches, and redounds to your comfort and salvation. Which is effectual; which salvation of yours--including also the consolation accompanying it--is effectual; that is, active and efficacious. It has a vigorous life and growth in your souls, and exerts its power in them more and more. In the enduring--we also suffer; no while you simply look on and see us suffer in your behalf, but while you share with us the same sufferings.>
<1:7 The consolation; which Christ gives to those who suffer for his sake. Great sufferings are usually accompanied by great consolations, and increase both our present and our eternal good.>
<1:8 Pressed out of measure; exceedingly distressed.>
<1:9 Sentence of death; were cut off from all human means of help, and doomed apparently to immediate death.>
<1:10 From so great a death; of the particular kind of death to which the apostle was exposed we have no certain knowledge. We only know from the present verse that it was very terrible.>
<1:11 The gift; the deliverance just referred to. By the means of many persons; by means of their intercessory prayers for the apostle and his helpers, which God heard and answered. By many; the many who have prayed for the gift. As Christians may by prayer greatly assist absent friends and bring an increased revenue of glory to God, they should pray with all prayer and supplication in the Spirit, and watch thereunto with all perseverance. Eph 6:18.>
<1:12 In simplicity; with a single view to the glory of God and the good of men. Fleshly wisdom; worldly, selfish, underhanded policy. Had our conversation; conducted ourselves in all things; according to the old meaning of the word conversation, that is, deportment, manner of life. To you-ward; in whose case there has been especial occasion for circumspection.>
<1:13 Than what ye read; the opponents of Paul in Corinth had probably accused him of insincerity in his former epistle. He assures them that he has no hidden end, but that what they read is precisely what he means. Or acknowledge; that is, recognize and know to be true from your own personal acquaintance with me: as much as to say, What I write agrees with what you already know of me. Shall acknowledge even to the end; by finding me to be a person who does not change.>
<1:14 In part; he makes this limitation because there were some in Corinth who did not acknowledge his worth and authority as an apostle of Jesus Christ.>
<1:15 In this confidence; that we are your rejoicing, and ye ours, as verse 2Co 1:14. To come unto you before; before visiting Macedonia, whence he wrote to them this letter; namely, by calling on the Corinthians on his was thither, verse 2Co 1:16. A second benefit; the words probably mean, the benefit of a second visit from me when I should return to you from Macedonia, as stated in the following verse.>
<1:17 Use lightness; was he fickleminded, as some charged him, because he did not fulfill his purpose of visiting them on his way to Macedonia? According to the flesh; as insincere, selfish, and worldly men do, changing their plans and promises to suit their own convenience.>
<1:18 Our word; or, as the margin, our preaching; for the apostle adduces the steadfastness and consistency of his preaching as the great proof of his general steadfastness. Was not yea and nay; was not now yes, and now no. Like Christ, its author, it was not a changeable, but a steadfast doctrine. We may, for good reasons, change our intentions and plans; but all men, especially ministers of the gospel, should be on their guard against every thing like fickleness of purpose or worldly policy.>
<1:19 Silvanus; the same as Silas, Ac 15:22. In him was yea; all the promises made in him were only yea--steadfast and sure, as is asserted in the following verse. It is implied that Paul, the preacher of such a Saviour, was steadfast also.>
<1:20 By us; through our preaching.>
<1:21 Hath anointed us; us Christians, by his Holy Spirit. 1Jo 2:20,27.>
<1:22 Sealed us; marked as his own. The agent of this sealing is the Holy Spirit, as immediately stated. The earnest of the Spirit in our hearts; the Holy Spirit dwelling in our hearts, and giving us a foretaste of the joys of heaven, which is the pledge of our full introduction to them. The possession and exercise of the graces of the Spirit are sure evidences of regeneration, and pledges of eternal life.>
<1:23 I call God; to witness the truth of what he said. To spare you; to save them from that painful discipline which he might have found necessary, had he visited them in their disorderly state.>
<1:24 Not for that we have dominion over your faith; as much as to say, Do not understand the words I have just written as meaning that we set ourselves up to be lords over your faith, and delight to exercise severity towards you. He intimates that he would not exercise apostolical and inspired authority in punishing their offences, if he could consistently avoid it; but would seek to promote their joy in leading them, by kind persuasion, to correct their errors and return to their duty. By faith ye stand; as much as to say, not by our exercising dominion over you, but by your own free faith in Christ.>
<2:1 In heaviness; in grief and sorrow, as he must have done, had he visited Corinth before the disorders in the church to which he referred in his first epistle had been removed.>
<2:2 If I make you sorry; a delicate way of saying that he has the strongest motives not to grieve any one of them, if he can possibly avoid it; since it is from the very person grieved that he looks for his comfort. Compare chap 2Co 1:14.>
<2:3 I wrote this same; the admonitions contained in his first epistle.>
<2:4 Faithful ministers of the gospel are often made sorrowful by those who ought to give them joy; and the efforts which most grieve or offend some of their people, may spring from sincere love to them, and a most earnest desire for their good.>
<2:5 He; the person referred to in 1Co 5:1. But in part; he had not grieved Paul only, but the sound part of the church also. Not overcharge you all; not speak as if all were guilty, or equally so.>
<2:6 Such a man; the offender referred to, who had, by the discipline of the church, been brought to repentance. This punishment; his excommunication, according to Paul's direction, 1Co 5:4,5; which was to be effected, not by Peter or Paul, but by the authority of the church.>
<2:7 Contrariwise; instead of continuing his exclusion from the church, they ought now to restore him.>
<2:8 Confirm your love toward him; by receiving him again into the church. The objects of church discipline are the repentance of offenders and the honor of religion. Whenever these are accomplished, offenders should be forgiven, and such as have been excommunicated should be received again to Christian communion.>
<2:9 To this end--did I write; his object in his first epistle was to induce them, by disciplining this man, to give evidence of their disposition to do right.>
<2:10 For if I forgave--forgave I it; or, For if I have forgiven any thing, to whomsoever I have forgiven it, it is for your sakes. As the punishment had in view their profit, not the gratification of his own private feelings, so also the forgiveness. In the person of Christ; acting as an inspired apostle for Christ and under his direction.>
<2:11 Lest Satan should get an advantage of us; by leading them to be needlessly severe, to the injury of the offender and of religion. His devices; in tempting men to sinful extremes. Satan is an artful and malignant spirit, and has many devices for injuring the cause of Christ and ruining the souls of men. Persons who disbelieve his existence, who do not oppose his influence, or are ignorant of his devices, are not qualified to discharge the duties of ministers of Christ.>
<2:12 Troas; a city on the way from Ephesus to Macedonia, where Paul expected to meet Titus and learn from him the effect of his first epistle. But in this he was disappointed. He therefore left Troas and went into Macedonia, where he met Titus, and learned the happy issue of things at Corinth. This caused him to break forth in thanksgiving to God for the success which had attended his labors. Verse 2Co 2:14.>
<2:14 The savor of his knowledge; the sweet savor of the knowledge of Christ.>
<2:15 A sweet savor of Christ; what he has said of the knowledge of Christ he now applies to those who preach Christ. To God they are a sweet savor of the knowledge of Christ; for both the message itself, and they who from love towards Christ and their fellow-men publish it, are most precious in God's sight, and that whether men receive Christ or reject him.>
<2:16 The savor of death unto death; a deadly savor, having death for its result; because, by rejecting the gospel, they turn that which was intended for their life into an occasion of death. The savor of life unto life; a life-giving savor, having life for its result. These things; the duties, responsibilities, and labors of such a solemn office. The faithful labors of preachers of the gospel are highly pleasing to the Lord, however they may be regarded by their people, and whatever may be their effects. But as those effects are momentous and eternal, and depend much on the character and conduct of ministers, their responsibilities are great, and they should earnestly seek wisdom from above to direct them in the discharge of their duties.>
<2:17 Corrupt the word of God; adulterate it by a mixture of human additions, thus destroying its efficacy.>
<3:1 Again; probably with reference to a charge of self-commendation against his former epistle. To commend ourselves; in what we have said of the dignity of our office, our purity in the discharge of it, and the triumphs in it which God awards to us, chap 2Co 2:14-17. The reader should notice in this epistle the abundant use of the plural number, where the apostle means chiefly himself, but prefers to speak in the name of his fellow-laborers also.>
<3:2 Ye are our epistle; his letter of commendation, which all could read. He says here, "written in our hearts," because he wishes thus to express the place which the Corinthian converts have in his affections. In the next verse he represents the epistle as written, by his ministration, on the hearts of the Corinthians, because it is there that the gospel has exerted its power. Such changes of figure are very common with Paul.>
<3:3 Ministered by us; written by our ministration, as his instruments. Not in tables of stone; as a mere outward law is. The allusion is to the ten commandments written on tables of stone. Fleshly tables of the heart; compare Jer 31:33; Eze 11:19; 36:26. When ministers of the gospel are instrumental in converting men from sin to holiness, it is proof that the Spirit of God accompanies their labors; and though they are the means, he is the author of their success, and to him belongs the glory.>
<3:4 Such trust have we; in regard to the success of our ministry, as just stated. Through Christ; not through our own power. To God-ward; in regard to God. In these words the apostle represents himself as always acting with reference to God's glory, and putting all his trust in him.>
<3:5 To think any thing; aright, or which would insure success.>
<3:6 Hath made us able ministers; hath given us sufficiency to be ministers, as the original means. The new testament; the new covenant of the gospel, revealing the way of salvation through Jesus Christ. Not of the letter; not of the outward form merely, but of the design, end, and spiritual meaning, the right apprehension and cordial reception of which is, through the grace of God, life-giving, while dependence upon the letter or outward form merely is ruinous to the soul. To rely for salvation on the possession of the Scriptures, on the stated reading of them, or on any outward forms and privileges, is destructive; while the right understanding of the Scriptures, and spiritual obedience to their true meaning, are saving to the soul.>
<3:7 The ministration of death; of the Mosaic law, which "worketh wrath," and brings death instead of life to sinners. Ro 4:15; 7:10. Written and engraven in stones; the ten commandments thus written here represent the whole Mosaic economy. Was glorious; in the circumstances of its institution, and in the objects it was designed to accomplish. Of this glory the splendor of Moses' countenance was the divinely appointed symbol. The apostle therefore puts the latter for the former. Could not steadfastly behold the face of Moses; Ex 34:29-35. The glory of Moses' countenance represented that of the dispensation of which he was the mediator; the veiling of his face, the obscurity which God threw over it, in consideration of the inability of his covenant people to behold directly the true spiritual end which this temporary dispensation had in view.>
<3:8 The ministration of the spirit; of the gospel, which is a spiritual dispensation, administered by the Holy Spirit, and giving life to the soul, instead of death. The chief reason why the Christian dispensation excels in glory is, that under it the Spirit is given with a fulness and power unknown before.>
<3:9 Righteousness; here the righteousness which God gives through faith in Christ, bringing to the soul justification instead of condemnation.>
<3:10 Had no glory; ceased to appear as glorious. In this respect; in comparing its glory with the greater glory of the gospel.>
<3:11 That which is done away; the Mosaic dispensation. That which remaineth; the gospel dispensation. Is glorious; as accomplishing a more glorious work, and to continue with increasing power to the end of time.>
<3:12 Seeing then that we have such hope; of the glorious results to be accomplished by the Spirit through the gospel. Great plainness of speech; not veiling what we teach under obscure types and symbols, as did the law of Moses, but declaring boldly, clearly, and freely the doctrines and duties of religion. Living faith in the gospel inspires those who preach it with glorious hopes, and leads them to preach so plainly and with such earnest, affectionate boldness, that all who are disposed may understand them, and be made wise unto salvation.>
<3:13 Not as Moses, which put a veil over his face; our message is not in any measure concealed, as was the face of Moses, in token of the darkness of that dispensation. Could not steadfastly look to the end of that which is abolished; not clearly understand the meaning and design of the ceremonies and types of the Mosaic dispensation, which was appointed to pass away.>
<3:14 But their minds were blinded; not only did God place a veil on the dispensation, but there was a veil on their minds also--that of unbelief and hardness of heart. In this natural and easy way does the veil on Moses' face lead the apostle to speak of the veil on the minds of the covenant people. The same veil; the same blindness as to the meaning of the Old Testament scriptures. Which veil is done away in Christ; the obscurity of the Old Testament prophecies, types, and figures, is removed by their fulfilment in Christ. But as the hearts of the Jews are still opposed to him, and their minds blinded, they do not see this fulfilment.>
<3:16 It; the heart of the Jewish people. Turn to the Lord; embrace Jesus Christ as the Messiah. The veil shall be taken away; they shall understand the meaning of the Old Testament scriptures, and see their application to Christ. The reason why the Jews misunderstand the Old Testament and reject the New, is their hardness of heart and blindness of mind. These, with regard to many, the Holy Ghost at some future day will remove. Then they will see that Jesus is their long promised Messiah, and will embrace him as their hope of glory.>
<3:17 The Lord; the Lord Jesus. Is that Spirit; more literally, is the spirit; the spirit in contrast with the letter, verse 2Co 3:6. The Old Testament types, figures, and prophecies, taken without him, are the letter which killeth. But in him they are the spirit which giveth life. The Spirit of the Lord; the Spirit of the Lord Jesus, which need not be here distinguished from the Holy Spirit, since it is through him that Christ works in our hearts. Liberty; from bondage to the letter. Such liberty includes free access to God and communion with him.>
<3:18 With open face; literally, with unveiled face, the veil having been, to us, taken away in Christ. The glory of the Lord; the Lord Jesus. From glory to glory; from one degree of glory to another. By the Spirit of the Lord; or, as the margin, by the Lord the Spirit; that is, by the Lord Jesus, who is the Spirit, 2Co 3:17. Both renderings come to the same thing; since it is by the Holy Spirit that the Lord Jesus transforms us into his own image.>
<4:1 As we have received mercy; the apostle has special reference to the mercy of God in calling him from being a blasphemer of Christ to be his apostle, 1Ti 1:12,13. The glorious prospects which the gospel opens to faithful ministers, and all true Christians, animate them to press onward in the path of duty with increasing zeal and fidelity to the end.>
<4:2 The hidden things of dishonesty; literally, as the margin, the hidden things of shame; shameful deeds which men practise secretly, because they are ashamed to have them known. The words immediately following show that he refers to the base arts of the false teachers who sought to supplant him in the favor of the Corinthians. Not walking in craftiness; as preachers of the gospel, not resorting to low and base arts to gain popularity. Handling the word of God deceitfully; corrupting it by a mixture of human inventions to make it more platable to worldly men. Commending ourselves; preaching as in the presence of God, and in such a manner as every enlightened conscience must approve.>
<4:3 If our gospel be hid; literally, veiled, in allusion to the veil in the hearts of the unbelieving Jews; that is, so hid that men do not see its glory. All such are still in their lost condition, unenlightened by the Holy Ghost.>
<4:4 The god of this world; Satan, under whose influence are all unbelievers. Lest the light--should shine; so shine that, by believing in Christ, they should see his glory and be made like him. Satan makes great efforts to hinder men from hearing and believing the gospel, lest its light should so shine into their minds as to be the means of their conversion and salvation.>
<4:5 Christ Jesus the Lord; and ourselves your servants; more literally, Christ Jesus as Lord, and ourselves as your servants.>
<4:6 Commanded the light; Ge 1:3. In the face of Jesus Christ; as he is revealed in the gospel.>
<4:7 This treasure; this knowledge of Christ, and of the gospel which they were to publish. Earthen vessels; feeble, frail, dying men. The excellency of the power; that it may be seen that the power which gives success is of God. The character and condition of ministers of the gospel have always been such as to show that their success was of God; and the fact that Christianity has lived and triumphed, notwithstanding their weakness and unworthiness, is a standing demonstration of its divine origin.>
<4:8 Not distressed; so as to be overcome or disabled for their duties. Perplexed; as to what course to take. Not in despair; not left utterly at a loss what to do, as those forsaken of God.>
<4:9 Persecuted; by men. Not forsaken; of God. Not destroyed; able to rise again and renew the conflict.>
<4:10 Always bearing about; wherever we go. The dying of the Lord Jesus; the violent putting to death of the Lord Jesus; in other words, always exposed, like him, to a violent death at the hand of the wicked, with all the sufferings connected with such an exposure. See next verse. The life also of Jesus; a life conformed to that of Jesus. We are to understand the life of Jesus in the widest sense, so far as he was a man. It is a life devoted to God, sustained by God, and which will be finally made triumphant in a glorious resurrection over all evil. Might be made manifest; might be clearly exhibited to men.>
<4:11 For we which live; who yet live in mortal bodies. In our mortal flesh, not merely in our spirits, but in our frail dying bodies also; for these are Christ's, will be preserved by Christ till our earthly work is done, and be glorified with Christ in the final resurrection.>
<4:12 Death--in us, but life in you; our labors, which constantly expose us to death, promote your eternal life. He does not deny that life works in himself also; but he wishes to exhibit his sufferings as contributing to their life as Christians: not the life of their souls only, but also that of their bodies, as verse 2Co 4:14 shows.>
<4:13 As it is written; Ps 116:10. Believe, and--speak; the truths of the gospel, without being disheartened by any trials to which it exposes us. The moving spring of ministerial fidelity is such confidence in God as causes his declarations to appear true, and gives to unseen realities a commanding influence over their minds.>
<4:14 Present us with you; spotless and faultless before the throne of his glory, with exceeding and eternal joy.>
<4:15 All things; the whole of God's dealings with you. This he says with especial reference to the sufferings and triumphs of God's ministers, all of whom, not certain favorite leaders, are for their sakes. For your sakes; for their salvation and that of others, and thus for the glory of God.>
<4:16 For which cause; because of the glorious hopes of the gospel and the glorious results of our labors. We faint not; under these labors and trials. Our outward man; our dying body. The inward man; the spiritual life and vigor of our souls.>
<4:18 The endless glories which await faithful ministers and Christians, and for which their present trials are preparing them, are such that, in comparison, the latter are swallowed up and lost in the eternal greatness of the former.>
<5:1 Our earthly house of this tabernacle; our body, considered as a tent in which the soul sojourns. Compare 2Pe 1:13,14. A building of God, a house not made with hands; namely, the resurrection-body. The apostle here passes over the intermediate disembodied state without noticing it. But in verses 2Co 5:6,8, he distinctly mentions it.>
<5:2 In this; in this our earthly body. Clothed upon; with our glorified heavenly body.>
<5:3 Naked; destitute of a glorified body.>
<5:4 Unclothed; it is not the unclothing of our soul by death that we desire, but the clothing of it with the glorified body. If it might be the will of God, we should be glad to have mortality swallowed up of life without death, as will be the case with those who are alive at Christ's coming.>
<5:5 Wrought us; prepared us for, and led us to expect these heavenly glories. Earnest of the Spirit; the joys which he imparts as foretastes of heaven. Every thing good in believers comes from God, and is the fruit of his Spirit. In their greatest trials he is with them, and often gives them joys which are foretastes of heaven.>
<5:6 Absent from the Lord; from the place of his special abode in heaven.>
<5:7 Walk by faith; are controlled, not by what we see, but by what we believe.>
<5:8 Absent from the body; as are the disembodied spirits of the just who rest with Jesus.>
<5:10 An abiding conviction that each individual will stand at the judgment-seat of Christ, and receive according to the deeds done in the body, is adapted to make men circumspect, and lead them most earnestly to desire and diligently to labor that they may be accepted of him.>
<5:11 The terror of the Lord; what terrible punishments he will inflict on the wicked. We persuade men; to flee from the wrath to come. Manifest unto God; he sees our sincerity. Are made manifest in your consciences; commend ourselves to your consciences as sincere.>
<5:12 Commend not ourselves; Paul means that he did not say this to gain their applause. Occasion to glory on our behalf; just ground of commending us as true and faithful servants of Christ. In appearance, and not in heart; in the outward show of virtues which had no place in their hearts. These were the vain-glorious boasters who opposed and slandered Paul.>
<5:13 Be beside ourselves; go, as some think, beyond all reasonable bounds in our efforts. Whether we be sober; go, as some maintain, to the extreme of caution and prudence. For your cause; for the sake of doing you good.>
<5:14 The love of Christ; his love to sinners constraineth us to love him, and thus labor to induce our fellow-men to love him. All dead; in trespasses and sins--dead to all desire to honor God or live to his glory, and dead to all possibility of salvation by their own works, or in any way except through faith in Christ.>
<5:15 They which live; in consequence of Christ's dying for them. Live--unto him which died for them; seek to honor him, and to induce all others to do the same. As Jesus Christ by dying for all has proved that all are spiritually dead, and as his object in this was, that those who are made spiritually alive should live not unto themselves, but unto him, a disposition to do this is essential to true religion.>
<5:16 Know we no man; we do not regard men's outward condition or connections. We have known Christ; as a Jew belonging to our nation, and expected great temporal favors from him. But now we regard him as a spiritual Saviour, and labor to induce as many as possible to believe in him.>
<5:17 In Christ; united to him by faith. A new creature; created in Christ Jesus unto good works. Eph 2:10; 4:24; Col 3:10. Old things are passed away; former views and feelings with regard to spiritual things are changed. Become new; he seeks new ends; has a new rule of action and pursues a new course of conduct; has new joys and new sorrows, new hopes and new fears, new relations and new prospects.>
<5:18 All things are of God; he is the author of this change and all its blessings. Ministry of reconciliation; the treasure spoken of in chap 2Co 4:7, to be used for the benefit of lost men.>
<5:19 Not imputing their trespasses; not punishing, but forgiving them. The word of reconciliation; the gospel, making known the way, and inviting men to be reconciled to God.>
<5:20 Ambassadors for Christ; persons appointed to act in his stead.>
<5:21 To be sin; suffer to make atonement for it. Be made the righteousness of God; for Christ's sake accepted, and treated as righteous, through faith in him. In giving his Son to die for his enemies, and in coming by the gospel through his ministers, and beseeching men to be reconciled to him, God has shown that he is exceedingly desirous of their salvation; and that if any are lost, it will be because they refuse to be reconciled to him.>
<6:1 The grace of God; his gracious offer of pardon and salvation through Christ.>
<6:2 He saith; Isa 49:8. I have heard thee; the Messiah, to whom the words quoted by Paul are addressed. Succored thee; in thy work of man's salvation. Now is the accepted time; the time foretold by the prophet when God, in a preeminent way, would hear and succor his Son in the work of man's salvation. This made it to all men the day of salvation--the day when God's grace was given to them in larger measures than ever before.>
<6:3 Giving no offence; that is, we the apostles giving no offence. The ministry; the apostolic ministry committed to us by Christ.>
<6:4 In much patience; endurance of trials, as the word in the original means. This and the next verse contain an enumeration of the circumstances in which they approved themselves as faithful ministers of God. The office of ministers of the gospel is one of high dignity and honor. They are workers together with God in the accomplishment of his great plan of mercy, and should in all things so conduct as is best suited to promote this end.>
<6:6 By pureness; of heart and life. In this and the following verse he mentions the spiritual graces and instrumentalities belonging to his ministry. He then passes again to the conditions and character, temporal and spiritual, under which it was exercised, 2Co 6:8-10. By knowledge; of the gospel, which they received from God and communicated to men. By the Holy Ghost; whose constant presence qualified them for their work.>
<6:7 The word of truth; which they proclaimed. The power of God; by which he sanctioned and gave efficacy to the truths they uttered. By the armor of righteousness; the armor furnished by "the righteousness which is of faith." For a full description of it see Eph 6:13-18, which is the best commentary on the present words.>
<6:8 As deceivers; in the view of our enemies.>
<6:9 Unknown; especially to the rich, great, and powerful of this world. Well known; by the efficacy of our labors. Dying; exposed continually to be put to death. We live; being upheld in life by the power of God.>
<6:10 Sorrowful; on account of our conflict with sin and suffering. Always rejoicing; in God, and the triumphs of his grace. Poor; as to this world. Rich; for eternity. Nothing; of the wealth of earth. All things; that will be truly beneficial on earth and in heaven. Persons united by faith to Christ may have no exclusive right to any thing, and yet be joint-heirs with Christ to all things. They may be destitute of the riches of earth, and yet entitled to all that earth and heaven can afford; dependent for their daily bread, and yet dispensing inexhaustible, ever-satisfying, eternal treasures.>
<6:11 Our mouth is open; we speak freely from the fulness of our hearts. Our heart is enlarged; with love towards you, so that there is room enough in it to take you all in.>
<6:12 Not straitened in us; there is no want of room in our hearts to receive you. Straitened in your own bowels; it is your hearts that are too narrow to receive us. The want of confidence and love is on your side, not ours.>
<6:13 For a recompense in the same; that you may recompense our largeness of heart towards you, by exercising the same towards us. Be ye also enlarged; so as to take us into your hearts with full love and confidence.>
<6:14 Be ye not unequally yoked together with unbelievers; intimately connected--Christians with heathen; believers in Christ with unbelievers. All such connections as tend to increase wickedness or encourage sin should be carefully avoided, and such a course of life be pursued as most tends to promote holiness in ourselves and our fellow-men.>
<6:15 Belial; Satan.>
<6:16 Ye; the company of believers, the church. Are the temple of the living God; in which he especially dwells. Eph 2:21, 22. God hath said; Ex 29:45; Le 26:12; Jer 31:33; Eze 11:20; 36:28; 37:27.>
<6:17 Come out from among them; do not unite with them, nor encourage or connive at any of their idolatrous or wicked practices. Le 11:44; 1Pe 1:15,16.>
<6:18 A Father; friend, saviour, guardian, protector, guide, benefactor, and portion. Sons and daughters; like God in temper, and heirs to the eternal glories of his kingdom. Almighty; infinitely powerful, able, and willing to do all he has promised. As Jehovah the Almighty God is through grace the Father of his people, and they are heirs to his great and eternal possessions, they have no need to seek alliances with the rich and great of this world. They are children of a King, and are themselves to be kings and priests unto God, and to reign with him for ever.>
<7:1 These promises; the promises of God referred to in the last chapter. Filthiness of the flesh; excessive sensual indulgences. And spirit; as pride, anger, malice, revenge, envy, covetousness. The promises of God to believers, instead of leading them to be careless in sin, excite them to the most earnest desires and strenuous efforts to be delivered from it. No hope is genuine, or will stand in the day of trial, but that which tends to purify the soul even as Christ is pure.>
<7:2 Receive us; into your hearts, as apostles and ministers of Christ; give us your affectionate confidence and ready obedience. There is here an allusion to the exhortation, "Be ye also enlarged," 2Co 6:13.>
<7:3 I speak not this to condemn you; what he has just said, 2Co 7:2, might seem to the Corinthians to be uttered in a spirit of censure. He wishes them to understand that he cannot speak to them except from the impulse of love. Said before; chap 2Co 6:11,12. To die, and live with you; such is his affection for them, that he would gladly be joined with them in life and death.>
<7:4 My glorying of you; on account of their ready compliance with his directions.>
<7:5 Fightings; great opposition to the gospel and to him for preaching it. Fears; lest his first epistle should not have produced the desired effect. The best ministers may be called to endure great afflictions both from without and within. But God is mindful of their trials; and when their sorrows are the greatest, he is preparing them for the greatest joys.>
<7:6 The coming of Titus; from Corinth to Macedonia with the news of their compliance with Paul's directions. 1Co 5:4,5.>
<7:7 Earnest desire; the original word, rendered "vehement desire" in verse 2Co 7:11, seems to denote earnest affection towards the apostle; of course with reference to the matter for which he had reproved them. This would be accompanied with diligence in complying with his wish as to the incestuous person. Mourning; for the sins into which they had fallen. Fervent mind toward me; zeal in complying with my wishes.>
<7:8 Though I did repent; this expresses his distress in having been called to write as he did in his first epistle. Chap 2Co 2:4.>
<7:9 After a godly manner; according to the will of God.>
<7:10 Godly sorrow; such as God requires; which grieves for sin because it dishonors God. Repentance to salvation; that sorrow for sin which leads a man to forsake it, and look to Christ for salvation. Not to be repented of; a change that will never be regretted or renounced. Sorrow of the world; that which is supremely selfish, and grieves principally because of the evil which sin occasions to the transgressor. Worketh death; tends to undermine health, shorten life, and hurry men to the second death.>
<7:11 Carefulness; diligence to remove the offence. Clearing of yourselves; from fault in this matter. Indignation; against the offender, and yourselves for having suffered him. Fear; towards God, and me his apostle. Vehement desire; affectionate longing towards me, who had been constrained to rebuke you for your sin. Zeal; to discipline the offender. Revenge; the infliction of just discipline and punishment upon him in your treatment of the matter. Clear; by having done your duty, according to the apostle's directions, 1Co 5:4,5, etc. The difference between worldly and godly sorrow is, one has supreme respect to the creature, the other to the Creator: one tends to inaction, murmuring, despair, and death; the other to earnest, persevering efforts for deliverance from sin, a dread of repeating it, a readiness to justify God in his threatenings against it, and a hearty reliance on Christ for pardon, sanctification, and eternal life.>
<7:12 His cause that had done the wrong; not for his only, or chiefly. His--that suffered wrong; the father. 1Co 5:1. Care for you; regard for your good. It was no private feeling towards the incestuous person or any one wronged by him that had induced the apostle to rebuke the Corinthian church. He wished rather to manifest his zeal for their purity, and thus bring them to repentance.>
<7:13 His spirit was refreshed; by your good behavior on the receipt of my first epistle.>
<7:14 Our boasting; his representation of their general readiness to do their duty.>
<7:15 With fear and trembling; lest they should not properly treat Titus and his message.>
<7:16 I have confidence in you; as Christians, that as you learn the will of God you will do it, and thus secure his favor. That repentance of sin which leads to the forsaking of it, and to a prompt, persevering discharge of duty, gives great joy to faithful ministers. And well it may, for it is evidence of true religion and of preparation for eternal life.>
<8:1 Do you to wit; cause you to know. The grace of God; here the grace of God as manifested in the liberality of the Macedonian churches.>
<8:2 The abundance of their joy; their spiritual joy in Christ. This abounded unto the riches of their liberality by prompting them to give a richly liberal gift. Their deep poverty; this abounded unto the riches of their liberality by making their gift a richer expression of faith and love. Though persecuted and poor, they had contributed largely for the benefit of others. Compare what the Saviour says of the poor widow's gift in Mr 12:43,44; Lu 21:3,4.>
<8:3 Of themselves; without being entreated.>
<8:4 Praying us with much entreaty--ministering to the saints; according to another and a more literal rendering, Asking of us with much entreaty the gift and the fellowship of the ministering to the saints; that is, asking of us the privilege of making the gift, and thus sharing in ministering to the saints. It is the poor saints in Jerusalem that are referred to. Ro 15:26; 1Co 16:3.>
<8:5 Not as we hoped; they went beyond our hopes. Unto us; to be directed by us, according to the will of God. A disposition to give one's self to the Lord, and to use what he bestows according to his will, is the fruit of divine grace; it is also a source of great joy, and leads to liberal contributions for the good of others.>
<8:6 Insomuch; on account of their great liberality. As he had begun; to make a collection for the poor saints at Jerusalem.>
<8:7 In this grace; that of liberally contributing of their substance, to supply the wants of the needy.>
<8:8 Not by commandment; he did not command as to the amount of their contribution; but from the example of others, and to show their love to God and men, he endeavors to persuade them to be liberal. The forwardness of others; the Macedonian Christians. 2Co 8:1.>
<8:9 He was rich; in all the glories of the Godhead in heaven. He became poor; by leaving the glory he had with the Father before the creation, being born of a virgin in a stable, and cradled in a manger; living in poverty, and dying in agony on the cross, the just for the unjust. Ye--might be rich; in the perfect and eternal holiness and bliss of heaven. Information and kind persuasions are more efficacious than authority in leading men to do good. Example has great influence, and the most powerful means of all is the example of Jesus Christ. Would we lead men to do the greatest good, we must direct their minds to him, and by his love strive to induce them to imitate his example.>
<8:10 Herein; as to their contribution. Begun before; begun before the Macedonian churches. As they had made a movement the preceding year before the Macedonian churches began, it was highly desirable that they should have their contribution completed. Compare what he says on this point in chap 2Co 9:2-4.>
<8:12 Not according to that he hath not; a man is not required to do beyond his ability, or give what he has not.>
<8:14 A supply for your want; should you be destitute, and they have means to relieve you. That there may be equality; to such an extent that all shall have a supply.>
<8:15 It is written; Ex 16:18. See note on this passage. The point urged by the apostle is, that now, as in the distribution of the ancient manna, every one should have his just supply. Those who have, whether they regard their own good or that of others, should cheerfully impart to those who have not. Blessings are not given to men that they should hoard them, or consume them upon their lusts, but that they should use them for the glory of God and the good of men.>
<8:16 The same earnest care; to complete this collection at Corinth for Christians in Judea.>
<8:17 The exhortation; to visit Corinth, in verse 2Co 8:6.>
<8:19 This grace--administered by us; the gift or contribution which the apostle had obtained, and was to convey to Jerusalem. Churches have a right to choose not only their ministers, but also the persons who shall receive and distribute their contributions: and those who are intrusted with charitable funds should not only be faithful in their application of them, but should show that they are so; and thus avoid the appearance, and as far as practicable, the suspicion of evil, that their influence for good may not be impaired by augmented.>
<8:20 That no man should blame us; charge us with any improper use of the money. In this abundance; in respect to this abundant contribution.>
<8:21 Not only; being really honest in the sight of God, but also appearing to be so in view of men.>
<8:22 I have; or, he hath.>
<8:23 Of Titus; about Titus, who he is, or why he is thus employed. Or our brethren; if inquiries were made about them, the answer might be given which Paul suggested. The glory of Christ; persons in whom Christ manifests his glory. This he does by his glorious work in their own souls, which makes itself visible in their whole life and spirit, Mt 5:16; and by his glorious work, through their instrumentality, upon the souls of others.>
<8:24 The proof of your love; by furnishing them with a liberal contribution. Our boasting; our commendation of your liberality.>
<9:1 The ministering to the saints; making the collection for the Christians in Judea.>
<9:2 Achaia; that part of Greece of which Corinth was the capital. Was ready; to make a collection.>
<9:3 Our boasting of you; of their readiness liberally to contribute.>
<9:4 Unprepared--be ashamed; if it should be found that no collection had been made. Wise and good ministers exceedingly desire that Christians should be prompt and liberal in their benefactions, and will be disposed to make honorable mention of such as are so, that others may be led to imitate their example.>
<9:5 Go before; before he went himself, accompanied by others of Macedonia. Bounty--not as of covetousness; as a freewill-offering, not as if extorted by importunity.>
<9:6 Shall reap--sparingly--bountifully; men will be rewarded in proportion to what, from love to Christ, they do for his cause.>
<9:7 Cheerful contributions for Christians who are in want are peculiarly pleasing to God, and the greater the amount in proportion to their means, which any rightly bestow, the greater will be their reward.>
<9:8 All grace; every good gift.>
<9:9 As it is written; Ps 112:9. His righteousness; as manifested in his works of love and mercy. Remaineth for ever; in the original Hebrew, standeth for ever, that is, endureth firm, being acknowledged and upheld by God. It follows that he himself stands firm for ever in God's favor. Compare the following clause of the psalm, "His horn shall be exalted with honor.">
<9:10 He; God. Multiply your seed sown; increase your means of doing good. The fruits of your righteousness; the blessed results to yourselves and others.>
<9:11 To all bountifulness; that they might do greater good, and thus lead many to bless God. Through us; as the dispensers of your liberality. Thanksgiving to God; from those who receive it; and so verse 2Co 9:12.>
<9:12 The administration of this service; the distribution of their bounty. Is abundant--unto God; will cause many thanks to God.>
<9:13 By the experiment of this ministration; through the experience they have of your liberality in ministering to their wants. For your professed subjection unto the gospel of Christ; literally, for the subjection of your profession towards the gospel of Christ. Their Christian profession was not empty, and in name only; it was accompanied by true obedience.>
<9:14 By their prayer; which they will offer for your good. Which long after you; or, while they long after you, with Christian affection. For; on account of. The exceeding grace of God in you; as manifested in your deeds of love and mercy.>
<9:15 Thanks be unto God for his unspeakable gift; Jesus Christ, and the grace through him which produces in men fruits of righteousness. To liberal contributions Christians are urged not only by a wise regard to their own good, but by gratitude to God for the freeness and greatness of his love in the gift of a Saviour, through whom they receive all the good which they enjoy in this world, and all which they hope for in the world to come.>
<10:1 Meekness and gentleness of Christ; which Paul wished them to imitate. Base--bold; weak and contemptible, as my enemies say, in my bodily presence; but assuming great boldness in my absence. See verse 2Co 10:10. Meekness and gentleness were distinguishing characteristics of Jesus Christ, which should be habitually imitated by his disciples. All who learn of him will find rest to the soul, and may be instrumental of imparting this blessing to others.>
<10:2 Be bold; called to exercise his apostolical authority and enforce painful discipline. As if we walked according to the flesh; were governed by a worldly policy, after the manner of selfish men.>
<10:3 In the flesh; in the body, and subject to human frailty. Not war after the flesh; are not governed by worldly or selfish considerations.>
<10:4 Not carnal; not such as worldly and selfish men use or rely on for success, as external force, wealth, talent, cunning, and fraud. Through God; by his power. Pulling down of strong-holds; overcoming strong opposition to truth and duty. Those who pretend to be ministers of the meek and lowly Jesus, and yet enforce their authority by guns, swords, and prisons, are deceivers, and show this by using such means as were never used by Christ or his apostles, and such as are suited to make not Christians, but hypocrites and infidels.>
<10:5 Casting down imaginations--every high thing; all the proud and lofty thoughts of men, which lead them to exalt themselves against the gospel.>
<10:6 To revenge; punish by virtue of our apostolic authority. Your obedience is fulfilled; when you, who are true to Christ and his cause, have had opportunity to approve yourselves by your obedience.>
<10:7 Look on things after the outward appearance; regard men simply according to their outward condition and relations.>
<10:8 Our authority; as inspired apostles. I should not be ashamed; for the result will show that I have power to do according to my words.>
<10:9 That I may not seem; supply at the beginning of this verse, And this I say, in respect to my not being ashamed. Terrify you by letters; by empty threats in my letters, which I have no power to fulfil.>
<10:10 Say they; his opposers.>
<10:12 Men who think highly of themselves, and boast of their talents, excellence, and usefulness--who compare themselves not with the law of God, but with their own defective ideas of the characters of their fellow-men, are living exhibitions of pride, weakness, and folly.>
<10:13 Without our measure; beyond the measure of our actual labors, as was done by the opponents of Paul, who intruded themselves upon the field of other men's labors, and took to themselves the credit of what other men had done. The measure of the rule; the limits of labor which God had assigned them. A measure to reach; a measure appointed by God to reach.>
<10:14 We stretch not ourselves; boast not ourselves beyond the sphere of our actual labors.>
<10:15 Enlarged; in respect to our field of labor. By you; by your cooperation. According to our rule; according to the field assigned us by God, which has lain without the field of other men's labors. An earnest desire to make known Christ to those who have never heard of him, and a readiness to labor and suffer to induce men to believe on him, are truly apostolic, and make his ministers in the highest and noblest sense successors of apostles.>
<10:16 To preach; that is, so as to preach, as the result of this enlargement. Beyond you; to the heathen farther west, who had never heard the gospel. In another man's line; in another man's field of labor. Of things made ready to our hand; of labors that we find already performed.>
<10:17 In the Lord; acknowledging him as the Author of all good.>
<10:18 Not he that commendeth himself; man is not his own judge, but the Lord; and by His decision every one must stand or fall. As men are to stand or fall, not by their own judgment or that of their fellow-men, but the judgment of God, they should be most careful to secure his approbation; and as their qualifications for usefulness and their success come from him, they should give him all the glory.>
<11:1 My folly; in relating what he had done and suffered in the cause of Christ; which, in ordinary circumstances, might have been regarded as foolish. And indeed bear with me; better, as the margin, "and indeed ye do bear with me." As much as to say, I acknowledge your indulgence heretofore, and ask for more of it on the present occasion. A judicious and modest Christian will not speak of himself and his labors unless the public good evidently requires it; and then he will do it, not to exalt himself, but to magnify the grace of God.>
<11:2 With godly jealousy; I am exceedingly anxious for your good. Espoused you to one husband; he had been the means of uniting them to Christ.>
<11:3 His subtlety; Ge 3:1-5. Your minds should be corrupted; by false teachers. From the simplicity; so as to depart from the simplicity. The simplicity that is in Christ is their simple-hearted devotion to his gospel in its purity.>
<11:4 Ye might well bear with him; in his vain-glorious assumption of superiority over me and of dominion over your faith. But this is not the case. These boastful teachers have nothing new to offer.>
<11:6 But though I be rude in speech; as my enemies object to me, chap 2Co 10:10. Thoroughly made manifest; he had given them abundant evidence of his character as an apostle.>
<11:7 Abasing myself; in laboring for his support, not receiving it from them. Ac 18:3.>
<11:8 Taking wages of them; receiving supplies from others, while laboring for you.>
<11:9 From being burdensome; by receiving support from you. Although it is the duty of a people to support their minister, and he is as justly entitled to his living as any workman is to his wages, yet there are cases where a wise and good minister will preach without compensation, and live, if need be, by manual labor or on charity, for the purpose of doing greater good to mankind.>
<11:10 Of this boasting; that I preach without receiving support from those to whom I preach.>
<11:11 Because I love you not? and therefore am unwilling to seem to be under obligation to you?>
<11:12 That I may cut off occasion from them which desire occasion; that his enemies should not be able to say that he was selfish, and preached for hire. Wherein they glory; namely, that they preach the gospel free of charges. They may be found even as we; have no ground or plausible appearance for pretending to be more benevolent or worthy of regard then we.>
<11:13 Transforming themselves; attempting to appear like apostles of Christ.>
<11:14 An angel of light; tries to seem like one.>
<11:15 Satan has ministers who pretend to preach Christ's gospel; they make professions of piety and benevolence, enter into other men's labors, and strive to draw away Christians from ministers who have been instrumental in their conversion, and who preach to them the truth as it is in Jesus.>
<11:16 Think me a fool; in seeming thus to boast of what I am and what I have done; for present circumstances render this needful. If otherwise; if it does appear foolish, let him bear with me in mentioning a few things which the case seems to require.>
<11:17 Not after the Lord; not in accordance with his usual inspired instructions. As it were foolishly; as may appear foolish, and would be, were it not for the peculiarities which now call for it.>
<11:18 Many glory after the flesh; in their birth, rank, and worldly distinctions.>
<11:19 Ye suffer fools gladly; your persuasion of your own wisdom makes it easy for you to bear with the conduct of fools. He alludes to their false teachers, who without any good reason boasted of their preeminence.>
<11:20 For ye suffer; that is, ye endure patiently. He now adduces the proof that they suffer fools gladly. Bring you into bondage; by usurping dominion over you. Devour you; devour your property. Take of you; or, take you; that is, take you by fraud, circumvent you. Exalt himself; over you. Smite you of the face; treat you with insolence and abuse. The inference is, that if they can suffer all this patiently, they ought to bear with the apostle in his boasting.>
<11:21 I speak as concerning reproach; or, I speak by way of dishonor, as if admitting the truth of the reproaches cast upon me by my enemies. This, however, he does not admit, as he proceeds to show. Any is bold; to state things of which he may boast. I am bold also; for in all the grounds of preeminence on which they pride themselves, I go beyond them.>
<11:25 A night and a day I have been in the deep; floating, it is supposed, on something after one of his shipwrecks.>
<11:28 Besides those things that are without; or, besides other things, some of which he proceeds to name.>
<11:29 Is weak; needing assistance. And I; do not sympathize with him. Offended; tempted, or led into sin. I burn not; with grief and indignation.>
<11:30 Mine infirmities; my sufferings for Christ's sake, and my need of his help. The above enumeration shows that in the Acts of the Apostles we have but a brief account of Paul's labors and sufferings for Christ's sake.>
<11:31 The labors, sacrifices, and trials of faithful ministers are all known to God; and it is a great consolation when they are able in sincerity to appeal to him for the truth of their declarations, the benevolence of their plans, and the fidelity of their efforts. Though they may here be reproached, vilified, persecuted, and slain, yet He will remember them in the day when he makes up his jewels, and will bring forth their righteousness as the light and their judgment as the noonday.>
<12:1 It is not expedient for me doubtless to glory; for him to state further with regard to his labors and sufferings. Revelations; which the Lord made to him of the glories of heaven.>
<12:2 A man in Christ; a Christian, meaning himself. The third heaven; the place where God peculiarly manifests his presence.>
<12:4 Paradise; the place of celestial blessedness. From the Scriptures, under the teaching of the Holy Spirit, we may learn as much about heaven as it is best we should know while on earth. We should therefore be contented with, and grateful for our present means of information, and so use them as to become wise to salvation, and thus be prepared to grow in the knowledge, holiness, and bliss of heaven for ever. De 29:29; 1Co 2:9.>
<12:5 Of such a one--of myself; he purposely speaks of Paul caught up to the third heavens as one person; and himself--Paul dwelling in the flesh, and subject to all its infirmities--as another. Of the former he will glory in respect to the high favors conferred upon him; but of the letter--Paul as known among men--he will glory only in respect to his infirmities.>
<12:6 To glory; in stating still further the honor God had bestowed upon him. I forbear; to mention any thing more about visions and revelations. Above that which he seeth me to be, or that he heareth of me; he chooses to be judged and estimated not according to the glorious revelations vouchsafed to him, which were invisible to men, but according to what in his life and labors was open to the view of all.>
<12:7 A thorn in the flesh; this seems to have been some bodily infirmity of a painful and humbling character. The messenger of Satan; this is best understood of the thorn in the flesh, which is called the messenger of Satan, because he made use of it to buffet the apostle. The buffeting we may well suppose came in the way of temptation to impatience, despondency, and the like unholy feelings.>
<12:8 The Lord; The Lord Jesus. That it might depart; that the trial might be removed.>
<12:9 My grace is sufficient; to enable you with patience to bear it, support and comfort you under it, and make you more happy and useful than you would be without it. Glory in my infirmities; because they fit me better for the service of Christ, and make it more manifest that it is his power which sustains me, and gives success to my labors.>
<12:10 Take pleasure in infirmities; on account of the good which they occasion. Weak; in myself. Strong; in Christ. Pride is so natural and strong even in Christians, and the bestowment on them of special mercies is so apt to increase it, that God sees it needful to visit them with special trials; and if, in answer to their prayers and the use of proper means, he does not remove those trials, they have abundant reason to acquiesce and even to rejoice in their continuance, as the best means of promoting the glory of God and the good of his kingdom.>
<12:11 Ye have compelled me; your conduct has made it needful. I be nothing; in and of myself; all my sufficiency is of God.>
<12:12 Signs of an apostle; such works as proved me to be one.>
<12:13 Were inferior to other churches; in the quality of the ministry enjoyed by you. Was not burdensome; did not receive my support from you. Forgive me this wrong; said in irony.>
<12:14 The third time I am ready to come to you; there is but one recorded visit of the apostle to Corinth before the date of this epistle; but he had purposed to visit them twice before, and now he purposed it the third time. Not yours, but you; not your money, but your salvation.>
<12:16 Be it so; his enemies said, if he did not openly receive support from them, he did covertly, for he sent men among them to take up contributions professedly for the poor, and then used the money himself. With guile; they said he obtained money by false pretences. This slander he refutes, verses 2Co 12:17,18, by appealing to what they knew.>
<12:19 That we excuse ourselves unto you; as if you were set to be our judges, and we needed to clear our character before you. We speak before God in Christ; in all sincerity, having no concealed purpose to accomplish. For your edifying; what the apostle had said by way of self-vindication had reference simply to their spiritual good, that they might be led to trust in him as a true apostle of Christ, and obey his directions by repenting of their sins and putting them away, of which there was much need, as he shows in the next verse.>
<12:20 Such as ye would not; lest he should be obliged to rebuke them for their sins, and administer severe discipline in order to bring them to repentance and reformation, and to save the church from the corrupting influence of their example.>
<12:21 Among the numerous trials which affectionate and successful ministers of Christ are called to encounter, the disappointment of their hopes with regard to many who for a time promised well, is by no means the least. Often they are called to deep anguish under the apprehension that some of their professed converts may, after all, be impenitent and sink into the horrors of the second death.>
<13:1 The third time; see chap 2Co 12:14. Two or three witnesses; the probable meaning of the apostle is, that he will administer prompt discipline according to the well-known Jewish rule. De 17:6; 19:15.>
<13:2 As if I were present, the second time; see note to 2Co 12:14. I will not spare; if I find you unreclaimed, I will exercise my apostolical authority and miraculous power in discipline. When professors of religion fall into sin, dishonor their profession, and injure the cause of Christ, his ministers will earnestly desire and faithfully endeavor by remonstrance, persuasion, and kind entreaty to reclaim them. If this is ineffectual, the discipline which Christ has appointed must be applied, and such offenders be excluded from the communion of the church. Mt 18:15-18; 1Co 5:4,5.>
<13:3 Christ speaking in me; that I am commissioned of him and act according to his will. Mighty in you; as shown by the effects which, through my agency, he has produced.>
<13:4 Through weakness; as a man in apparent weakness; abstaining from exercising his power for deliverance. Are weak in him; have fellowship with him in the weakness which he manifested when among men, and have abstained from exercising apostolical and miraculous power upon our opposers. Shall live with him; when we come among you again. By the power of God; manifested in the judgments which through us he will inflict on obstinate opposers.>
<13:5 Whether ye be in the faith; whether you have heartily believed on Jesus Christ. Is in you; by his Spirit, authority, and likeness. Except ye be reprobates; except your faith is dead, your hopes vain, and your religion worthless.>
<13:6 Not reprobates; not deceivers, nor deceived, but what we profess to be, Christians and inspired apostles, armed by our Master with divine power.>
<13:7 Ye do no evil; but do what is right, and especially in the matter about which I have written. Not that we should appear approved; by showing our apostolical authority in inflicting judgments. That which is honest; that you should reform, and not need punishment. Though we be as reprobates; though we should not show our apostolical authority, and should thus give our enemies occasion still to say, that we either could not or dared not inflict the punishment we spoke of as proving the truth of our apostleship.>
<13:8 We can do nothing; against truth and duty, however it may affect ourselves.>
<13:9 We are glad; are willing, and even rejoice to appear weak, or to continue to be called so, if it is occasioned by your well-doing. Your perfection; complete reformation and restoration to the faith and practice of the gospel. When Christians do right, and the cause of Christ prospers, his ministers rejoice, whether they have been instrumental in it or not, and however it may affect them; for they love Christ and his cause more than themselves or any earthly good.>
<13:10 Not to destruction; not for the purpose of destroying you, but of delivering you from sin, and thus promoting your salvation.>
<13:11 Farewell; an expression of earnest desire for their good. Be perfect; in the belief and practice of the truth. Be of good comfort; in the consolation which it will then afford you. Of one mind; united in feeling and conduct. Live in peace; without divisions, strife, or contentions. The God of love and peace; the author of these graces, who requires and loves them in his people. Union and peace among Christians in believing and obeying the truth, are peculiarly pleasing to God, and prepare the way for him to dwell with them, and impart to them the riches of his grace.>
<13:13 All the saints; who were with Paul.>
<13:14 The grace of the Lord Jesus Christ; the favor which he bestows upon his affectionate and obedient people. The love of God; manifested in the gift of his Son, and shed abroad in the hearts of his people. The communion of the Holy Ghost; his gracious presence, divine communications, graces, and consolations be and abide with you all. Amen; so let it be; and so, if you obey him through grace it will be for ever and ever. The grace of the Lord Jesus Christ, the love of God, and the communion of the Holy Ghost, comprehend all the blessings which the most benevolent heart can desire. They will therefore, by all the truly wise, be most earnestly sought, for themselves and their fellow-men; and to all who believe on Christ and walk in his ways, they will for his sake be given, to the glory of the Father, the Son, and the Holy Ghost, the one only living and true God, for ever. Amen.>
\kniha{Galatians}
\zkratka{Gal}
<1:1 Not of men; not deriving my office from men. Neither by man; not appointed by man. Being about to contend against a fundamental error, he asserts in the strongest terms his full apostolical authority, and goes on to show that he has received not his office alone, but the gospel which he preaches directly from Christ.>
<1:2 Christians in any house, town, or city who met together on the Lord's day to worship him and observe his ordinances, were regarded by the apostles as in a sense a church of Christ.>
<1:3 As Paul was chosen to be an apostle, and commissioned to preach the gospel, by Jesus Christ, and prayed to him as he did to the Father for the highest spiritual blessings, it is evident that he viewed him as divine.>
<1:4 Who gave himself for our sins; he asserts at the outset the fundamental doctrine of redemption through Christ, in opposition to the Judaizing teachers, who taught the Galatians to seek salvation through the works of the law. Deliver us from this present evil world; from both its corruption and its misery. This he does by making us citizens of a better world, and thus enabling us to live above the present.>
<1:6 Him that called you; God, who by Paul called them to embrace the gospel.>
<1:9 Any plan of salvation except that of free grace, through faith in Christ, is opposed to the gospel, and they who preach it are in danger of an awful condemnation.>
<1:10 For do I now persuade; seek to gain the favor of. As much as to say, Wonder not that I speak with such severity; for I seek not man's friendship, but God's.>
<1:11 Not after man; not of human, but divine origin.>
<1:12 I neither received it of man--was I taught it; lest the false teachers in Galatia should disparage Paul's apostleship, as being only of a secondary character, he takes pains to show that he has received the doctrines which he preaches immediately from Christ.>
<1:13 Conversation; manner of life.>
<1:14 Profited; made progress. Equals; in age, standing, and privileges.>
<1:15 Who separated me; set me apart from my birth for the work to which he afterwards called me.>
<1:16 Reveal his Son in me; make known to me Jesus Christ, and lead me to believe on him. I conferred not with flesh and blood; took no counsel with men, and sought not instruction from them.>
<1:17 Arabia; a country south of Damascus, a city of Syria. God assigns to all his people their appropriate work in life, and so orders events in his providence and grace as to fit them to perform it.>
<1:21 Syria; a country north of Palestine. Cilicia; a province of Asia Minor, north-west of Syria.>
<1:24 Glorified God in me; praised God for the change which he had wrought in me. When persecutors of Christ become his friends, and labor to promote the cause which before they sought to destroy, they strikingly manifest the grace of God, and furnish occasion for thanksgiving and praise.>
<2:1 Fourteen years after; after his conversion, or after his journey to Jerusalem. Chap Ga 1:18.>
<2:2 By revelation; by direction of God. Run in vain; he stated what he had preached and done among the Gentiles to certain leading individuals, and not to the whole church, lest he should fail of the object he had in view. Ac 15:2. Ministers of the gospel, while they should preach Christ and him crucified as the only foundation of hope, should use all proper means to prevent misapprehensions, remove prejudices, and counteract influences which tend to hinder the success of their labors.>
<2:3 Neither Titus--was compelled; or required to be circumcised. This showed, in direct opposition to the false teachers among the Galatians, that they did not consider circumcision needful.>
<2:4 And that because of false brethren; as much as to say, This exemption of Titus from circumcision was because of false brethren. It was a protest against their false teachings. Unawares brought in; artfully introduced. Came in; to their meetings. To spy out our liberty; their liberty to dispense with Jewish rites. Into bondage; to the ceremonial law, which Paul contended was not binding under the gospel.>
<2:5 No, not for an hour; they did not yield at all to the false brethren.>
<2:6 Of those; the leading men referred to, verse Ga 2:2. Maketh no matter; their reputation did not affect his standing as an apostle, or the correctness of his preaching and conduct. God accepteth no man's person; he does not approve of men because of their talents, reputation, rank, or condition. Added nothing; to his authority as an apostle, or his doctrines as a minister of Christ. Men are prone, even in religion, to be governed by human opinions rather than by the word of God--to regard some man as master and head of the church, rather than Jesus Christ. But the great question should be, not what does this or that man think, but what do the Scriptures teach.>
<2:7 Gospel of the uncircumcision; that he was commissioned by Christ to preach the gospel to the Gentiles, as Peter was to the Jews.>
<2:9 Pillars; men of eminence among the apostles, and chief instruments in supporting the cause of Christ. The grace; the favor bestowed on Paul in preparing him for the work to which he was called. The right hands of fellowship; by this they acknowledged them as ministers of Christ.>
<2:10 The poor; the needy Christians in Judea; obtain contributions from the Gentiles for them.>
<2:11 Withstood him; rebuked, and reproved him. Was to be blamed; for his timidity, and time-serving spirit.>
<2:12 From James; from Jerusalem, where James resided. Did eat with the Gentiles; in disregard of the ceremonial law. Of the circumcision; the Jews from Judea.>
<2:13 Dissembled; disguised their sentiments. They knew that the Jewish ceremonial was done away by the gospel, and had practically acknowledged it by eating with the Gentiles. But now they were afraid to avow their true convictions.>
<2:14 Livest after the manner of Gentiles; without observing the Jewish ceremonies. This was what Peter had been in the habit of doing. Compellest thou; setting an example, which, if they follow it, will lead them astray. Live as do the Jews; observe the ceremonial law. God foreknowing that some would be disposed to claim for Peter and his pretended successors peculiar prerogatives and honors, suffered him repeatedly to fall into great sins, and had them recorded in the Scriptures, that all might have infallible evidence that Peter was not a whit above the rest of the apostles.>
<2:15 Jews by nature; born Jews. Sinners of the Gentiles; sunk in the idolatry and vices of the Gentiles.>
<2:16 By the works of the law shall no flesh be justified; Paul and Peter, though Jews, believed this. Why then should Peter act as if it were necessary for the Gentiles to observe the ceremonial law? This was inconsistent, and adapted to make an erroneous impression on others.>
<2:17 If, while we seek--are found sinners; if, in seeking justification and salvation from Christ, not from the works of the law, we ourselves also; we who are Jews by nature as well as the Gentiles, are found sinners; found, after all our seeking, to be still in a state of guilt and condemnation: is therefore Christ the minister of sin? has he introduced a gospel which leaves those who trust in it still sinners under the condemnation of the law, so that they must turn again from Christ to the law for justification? Paul states the conclusion which must inevitably follow, if men are obliged to go back to the Jewish ceremonial for salvation, and then indignantly denies it in the words, God forbid; let it not be. Christ is not the minister of sin; but in turning away from him, I make myself a sinner, as he proceeds to show.>
<2:18 The things which I destroyed; the system of Jewish ceremonies, which, upon believing in Christ, I had destroyed, that is, given up as worthless. I make myself a transgressor; in going back from faith in Christ to the law.>
<2:19 For I through the law am dead to the law; instead of thus going back to the law for justification, I have learned through the law itself to renounce the law as the means of my salvation. A true knowledge of God's holy and spiritual law has taught me, that to a sinner, like me, it works death. Ga 3:24; Ro 3:20; 4:15; 7:10. Might live unto God; in and through Christ.>
<2:20 Crucified with Christ; through his death Paul had become dead to all expectation of salvation in any way except through faith in Christ; yet he was more active than ever, and from better motives. I live; a heavenly and divine life. Not I; not by my own power or goodness. Christ liveth in me; by his Spirit; and he is the cause of every thing right and good in me. The author and sustainer of divine life in the soul is Christ; and the means of rendering it vigorous is faith in him--forming between the soul and him a union, by virtue of which it receives of his fulness, grows in conformity to his image, and shows forth his glory.>
<2:21 Do not frustrate the grace of God; set it aside as of no efficacy, as do the false teachers, by going back to the law for justification. If righteousness come by the law; if there is any other way of being justified and saved, except through Christ, his death was needless. Any system of salvation which dispenses with the atoning sacrifice of Christ, is a renunciation of the gospel, and a virtual proclamation that his death was in vain.>
<3:1 Bewitched; fascinated, deluded. Set forth, crucified among you; or, set forth among you as crucified. Men who hope to be saved in any other way than through faith in Christ, are grossly deceived. Ac 4:10-12.>
<3:2 This only would I learn of you; as much as to say, The answer to this question will abundantly convict you of your folly. The Spirit; the Holy Spirit, in his sanctifying and miraculous influences. By the works of the law; through the efficacy of your observance of the Jewish law, to which you are now turning. By the hearing of faith; by obeying the message of the gospel, which offers you salvation through faith in Christ. The answer is plain: It was not through the works of the law, but through the hearing of faith, that they had received the Holy Spirit; why then turn away from the latter to the former?>
<3:3 In the Spirit; the Holy Spirit, as the minister of a spiritual dispensation. By the flesh; by the observance of the outward ceremonial law.>
<3:4 Suffered so many things; on account of their professed attachment to Christ. If it be yet in vain; as it would be, if they should forsake the gospel for the Jewish ceremonial law.>
<3:5 He therefore that ministereth to you the Spirit; God, who bestows upon you the Holy Spirit. By the works of the law, or by the hearing of faith? supply the answer, He does it by the hearing of faith, not by the works of the law. The effects of the gospel are conclusive evidence that is from God.>
<3:7 Children of Abraham; like him in spirit, and justified in the same way, not by works, but by faith.>
<3:8 Foreseeing; the Holy Ghost foreseeing, and in the Scriptures foretelling, that God would justify Gentiles as he did Abraham. In thee; as the spiritual father of all that believe. Ge 12:3; 18:18; Ge 22:18. Thus the glad tidings were announced to Abraham that God would bless and save the Gentiles through such faith as he exercised. The promises of God to Abraham and his seed were of spiritual blessings, justification by faith, and eternal life through Jesus Christ; not to believing Jews only, but to all who should believe, of all nations, in all ages.>
<3:9 Blessed with faithful Abraham; accepted of God in the same way.>
<3:10 Of the works of the law; are seeking justification by it. Under the curse; because they have not perfectly obeyed the law.>
<3:11 The just shall live by faith; does not promise justification by faith, but by works. But; that is, but its language is. Shall live in them; by doing them. The law knows nothing of grace, but demands absolute obedience as its only condition of justification. Justification by faith and not by works is a doctrine taught in the Old Testament as well as in the New.>
<3:12 The law is not of faith; does not promise justification by faith, but by works. But; that is, but its language is. Shall live in them; by doing them. The law knows nothing of grace, but demands absolute obedience as its only condition of justification.>
<3:13 The curse of the law; the punishment which it threatens against transgressors. Made a curse; treated as accursed, in suffering for our sake the accursed death of the cross. De 21:23; 2Co 5:21, "made him to be sin for us.">
<3:14 The blessing of Abraham; that which God promised to him and to all believers, justification through faith.>
<3:15 After the manner of men; as they view and treat a covenant that has been ratified.>
<3:16 His seed; Christ, as the head of his church; and through him, all believers, who constitute his body. He saith not, And to seeds, as of many; he does not make the promise to Abraham's seeds, as if he were speaking of the many individual children of Abraham; in other words, were making the promise to each one of the many who are his children by outward descent. But as of one, And to thy seed; he makes the promise, as speaking of one, to one seed of Abraham. Which is Christ; that is, this one seed that receives the promise is Christ, and in him all believers, who constitute his body.>
<3:17 The covenant; with Abraham and his seed. In Christ; as the promised seed. The law; given to Moses. The covenant of God with Abraham was made and confirmed in Christ; and the laws which were afterwards given, were not designed to alter it or change its conditions, but to lead men to comply with them, and thus obtain its blessings.>
<3:18 The inheritance; of the spiritual blessings promised to Abraham and his seed. By promise; that the blessing should come, through Christ, to all who believe on him. As this promise was made and ratified long before the law was given, its blessings could not come from obedience to law.>
<3:19 Wherefore then serveth the law? why was it added? Because of transgressions; the Jews were so prone to forsake God, worship idols, and commit all sorts of abominations, that the law was added to restrain them--to preserve among them the knowledge and worship of Jehovah, show them the desert of sin, their need of Christ and the nature of his salvation, and point them to him as the Lamb of God, that taketh away the sin of the world. Ordained by angels; given through the ministry of angels. In the hand of a mediator; namely, Moses. compare Ex 20:19; De 5:5,27.>
<3:20 Of one; one party. Is one; one party, the other party being man.>
<3:21 Is the law then against the promises? was it designed to open another way of life, or in any degree to conflict with the promises? Certainly not, but to aid in their accomplishment.>
<3:22 Hath concluded; declared all to be shut up under sin and condemnation, so that there is no way of escape except by faith in Christ.>
<3:23 Before faith came; before Christ, the object of faith, came; or before the way of life through him was clearly revealed.>
<3:24 The law was our schoolmaster; showing us our lost and guilty condition, and thus constraining us to come to Christ for salvation. In the ceremonial law, and in the whole Mosaic economy, God had a gracious design; and by it he produced, on all who rightly observed it, gracious effects: not by leading them to expect salvation by their ceremonial observances, but in leading them, through faith in Christ to become Abraham's seed, and thus receive the blessing.>
<3:25 No longer under a schoolmaster; having believed on Christ, he had adopted them as his children and given them in the gospel all needed instruction, so that they had no further need of the ritual observances of the law.>
<3:27 Baptized into Christ; become united to him by faith, and according to his will openly professed to be his disciples. Have put on Christ; taken him as their leader, and professed to come under the controlling influence of his Spirit.>
<3:28 One in Christ Jesus; possessed of one character; accepted in one way; belonging to one family; under one head, Christ; and equally entitled to all the blessings of salvation through him.>
<3:29 Christ's; united to Christ by believing on him. Abraham's seed; for the one seed of Abraham to whom the promise was made is Christ, verse Ga 3:16. All, then, that are united to Christ by faith are, in and through him, Abraham's seed, and heirs of the promises made to Abraham. The rich spiritual blessings which God in his covenant with Abraham promised him and his seed, do not come by natural descent; they are not affected by age, rank, sex, or outward condition; but are the fruits of grace, given for Christ's sake to all who by believing on him become children of Abraham in the sense of the covenant, and thus possess the character and sustain the relation of those to whom the promises were made.>
<4:1 In carrying out his argument against Judaism, the apostle compares the covenant people, before the coming of Christ, to an heir under age kept in a state of servitude. Differeth nothing; as to the control of his person and property. Be lord of all; owner of the whole estate.>
<4:2 The time appointed; for his taking possession of his inheritance.>
<4:3 We; the covenant people of God before the advent of Christ, into whom, after his coming, the Gentiles also are incorporated by faith. When we were children; under the Old Testament dispensation, when the people of God were treated as in their minority, and subjected to many restraints from which under the gospel they are free. Elements of the world; the Mosaic rites and ceremonies. See, for a fuller explanation of these words, the note to Col 2:8.>
<4:5 Receive the adoption of sons; pass from the condition and spirit of servants to the privileges and filial spirit of sons, in a state not of minority and servitude, but of manhood and freedom.>
<4:6 Sent forth the Spirit; God by his Spirit has given you a filial temper, and taught you to use the language not of servants, but of sons. Abba; a Chaldee word for Father. Compare Ro 8:15,16, and notes. The only sure evidence of being born of God, adopted into his family, and made heirs of the blessings of his kingdom, is the possession of a filial spirit towards our Father in heaven--a spirit of confidence, affection, submission, and obedience; connected with faith in Christ and a hearty reliance on him for salvation.>
<4:8 Ye; the gentile part of the church. No gods; idols.>
<4:9 Have known God; have been led through the gospel to the knowledge of God. Or rather are known of God; as much as to say, I might better say that ye have been known of God--known as the objects of his love and favor; for this higher knowledge of you on God's part, is the ground of your lower knowledge of him. Weak and beggarly elements; Jewish rites and forms, which can impart no real good.>
<4:10 Days, and months, and times, and years; such as were required in the ceremonial law. This has no reference to the weekly Sabbath, which was established at the creation, and set apart by God, to be observed by all men in all ages, and was required in the moral law; but to the feasts, new moons, and sabbaths required in the ceremonial law, which was never binding except on Jews and those who embraced their religion, and when Paul wrote had for years been done away.>
<4:11 I am afraid of you; he was fearful that they were depending for salvation on Jewish ceremonies, not on Christ; in which case his labor to bring them to Christ would be lost. There has always been a proneness in some professors of religion to depend for salvation upon the observance of rites, forms, and ceremonies, rather than on Christ. In such cases there is reason to fear that all efforts to save them and all their professions have hitherto been in vain.>
<4:12 Be as I am; for I am as ye are; according to some, Be united to me in love, as I am to you. Make to me the return of love which I bestow on you. Compare 2Co 6:13, and note there. Others understand him to mean, Be as I am in renouncing dependence on Judaism; for I, though by birth a Jew, have become, in this respect, like you Gentiles. Compare 1Co 9:21. Ye have not injured me; I have no injuries to charge upon you which have changed my love towards you: what I say is from pure regard to your welfare. Others suppose the apostle to mean, Hitherto ye have showed me only love and kindness, as he goes on to show.>
<4:13 Infirmity of the flesh; 1Co 2:3; 2Co 10:10; 12:7.>
<4:14 Received me--as Christ Jesus; with great cordiality, affection, and confidence.>
<4:15 Plucked out your own eyes; have made any sacrifice to comply with my wishes.>
<4:17 They; the false teachers professed a great regard for the Galatians, that they might detach them from Paul, and attach them to themselves. This would, as the apostle saw, be at the peril of their salvation, for then the awful words of our Lord would be fulfilled to them: If the blind lead the blind, both shall fall into the ditch.>
<4:18 As zeal in a good cause, united with judgment, is excellent and adapted to give a person influence, false teachers often make great professions, and express high regard for the welfare of the people. All should therefore be on their guard against wolves in sheep's clothing, and take heed not only how but what they hear, prove all things by the word of God, and hold fast that only which is thus found to be good.>
<4:19 Until Christ be formed in you; till the new man that lives by faith in Christ be fully formed in you, so that you shall no longer be in you, so that you shall no longer be in danger of being drawn away from Christ to Judaism.>
<4:20 To change my voice; from this expression of doubt and concern to one of satisfaction and joy. For I stand in doubt of you; am perplexed respecting you. He intimates his fervent desire to be delivered from this perplexity, by seeing them established in the faith of Christ, so that he should no longer be obliged to employ towards them the tone of severity.>
<4:21 Hear the law; attend to and receive the instruction which may be drawn from this portion of it to which I invite your attention.>
<4:22 It is written; Ge 16:15; 21:2,3.>
<4:23 Born after the flesh; without any special divine interposition. By promise; the special and peculiar favor of God, graciously and unexpectedly bestowed.>
<4:24 Which things; those which relate to these two sons, Ishmael and Isaac. An allegory; aptly represent the bondage of those who are under the ceremonial law and seek justification from it, and the freedom of those who embrace the gospel and expect justification only through faith in Christ. For these; these two women, Sarah and Hagar. Are the two covenants; fit representations of the two; namely, that with Abraham, which was confirmed of God in Christ, and that with Moses, which was made at mount Sinai. Gendereth to bondage; bears children to bondage; is herself a bond women, and bears children in the same condition with herself. There is here a blending together of Hagar and the covenant which she represents. The children of the Mosaic covenant represented by Hagar are those who live under it. Agar; in Hebrew, Hagar. Facts recorded by direction of the Holy Ghost in the Old Testament, are often striking illustrations of truths revealed in the New, and were designed by God to convey momentous instruction to man-kind. Hence the reason why so great a portion of the Old Testament is history; and the more it is understood, the more, by all good men, will it be valued.>
<4:25 Agar is mount Sinai; her case and that of her son Ishmael well represent the covenant at Sinai and those who are in bondage to its burdensome rites. Answereth to Jerusalem which now is; as is the case with the present inhabitants of Jerusalem who reject the Messiah, and are therefore in bondage to the Mosaic law.>
<4:26 Jerusalem which is above; the true spiritual Jerusalem, which has its centre in heaven, where Christ its head is. The Christian church, which is made up of believers in Christ, both Jews and Gentiles, may well be represented by Sarah the free princess, and Isaac her free son and heir of the covenant blessings promised through grace to his father. Of us all; all who are in Christ through faith.>
<4:27 For it is written; Isa 54:1; a prophecy which plainly relates to the Christian dispensation. Thou barren--desolate; the gentile church, or rather the church under the Christian dispensation, which knows no distinction between Jews and Gentiles. She is represented as remaining unmarried and barren till the coming of Christ. She which hath a husband; the old Jewish church, whose husband was God.>
<4:28 We; believers in Christ.>
<4:29 So it is now; as Ishmael opposed Isaac, so the unbelieving Jews, called, in verse Ga 4:25, Jerusalem which now is, and who were still in bondage to the law, persecuted Christians.>
<4:30 The scripture; Ge 21:10-12. As the bond woman and her son were cast out, so all subjection to Mosaic rites should be cast out or excluded from the Christian church; and so all who continue to seek justification by the law, will be cast off by God for rejecting the way of salvation which he has provided through his Son.>
<4:31 Not children of the bondwoman; not under the Mosaic dispensation represented by her, but under the gospel dispensation represented by the free-woman. Of course we are free from subjection to Mosaic rites and ceremonies, and cannot without great guilt and danger seek salvation from the observance of them. A state of freedom is much to be preferred to a state of bondage. One in the view of God is a fit representation of the darkness and burdensome restrictions of the Mosaic dispensation, a yoke which, the apostle says, neither the first Christians nor their fathers were able to bear. The other is a fit emblem of the light, liberty, and glory of the gospel. Under the blessings of the one, men have no right to take upon themselves or impose upon their fellow-men the disabilities and burdens of the other.>
<5:1 Stand fast; be firm, steadfast, and persevering. Yoke of bondage; to Jewish ceremonies.>
<5:2 If ye be circumcised; that is, circumcised as a profession of your dependence for salvation on the law of Moses. Ac 15:1. It was not against the simple rite of circumcision that the apostle contended, for Timothy was circumcised under his direction as a prudential measure, to avoid the prejudices of the Jews, Ac 16:3; but against circumcision as necessary to salvation, which was the error of the false teachers among the Galatians. Christ shall profit you nothing; for ye have left him for the law.>
<5:3 He is a debtor to do the whole law; for by circumcision he professes his dependence on his works for salvation, and must therefore perfectly obey the whole law. Salvation, if obtained, will then be of debt, not of grace. Chap Ga 3:12; Ro 4:4.>
<5:4 Justified by the law; are depending upon the law for justification. Fallen from grace; have renounced God's gracious mode of justification through faith in Christ.>
<5:5 We; true Christians. Righteousness; the righteousness which God gives through faith. Ro 1:17. True Christians to the end of life depend on Christ for salvation, and expect it only through faith in him. Those who depend on their works, must through their whole lives neglect no duty and commit no sin, but in all things obey perfectly the whole law of God, or they will be lost.>
<5:6 In Jesus Christ; in obtaining salvation through him. Faith which worketh by love; that confidence in him which has love for its foundation, and which leads to obedience.>
<5:8 This persuasion; that it was needful to be circumcised and observe Jewish rites in order to be saved. Of him that calleth you; of God.>
<5:9 A little leaven; error introduced by a few false teachers. Leaveneth the whole lump; corrupts the whole body of the church. As error begun in a church tends to increase and to corrupt the whole, it should be renounced and abandoned as soon as discovered; and all should watch and be on their guard against the beginning of evil.>
<5:10 None otherwise minded; that they would, on reflection, agree with him in this matter. He that troubleth you; by propagating error. Bear his judgment; receive punishment.>
<5:11 If I yet preach; that circumcision is needful to salvation, as the false teachers maintained. Then; if he had so preached he would have agreed with the Jews, and escaped their persecutions.>
<5:13 Liberty; freedom for Jewish ceremonies, and from the condemning power of the law. For an occasion to the flesh; as a pretext for the indulgence of fleshly lusts. The apostle is careful to distinguish between true Christian liberty from the bondage of Judaism and Antinomian licentiousness. Serve; do good to one another. Freedom from the ceremonial law, and through faith in Christ, from the condemning power of the moral law, and from the necessity of perfectly obeying it in order to salvation, do not lessen but increase a man's obligation to keep it; and such freedom will secure a hearty obedience.>
<5:14 All the law; the requirements of the law with regard to our fellow-men.>
<5:15 If ye bite and devour; contend with and injure one another.>
<5:16 Walk in the Spirit; live under his influence and follow his directions. Not fulfill the lust of the flesh; not follow sinful inclinations or comply with temptations to sin.>
<5:17 Lusteth against; strongly desires what the Holy Spirit forbids. The Spirit against the flesh; the Holy Spirit and all that is right in Christians oppose the indulgence of sinful desires. Hence a warfare in the soul, and thus they do not the good they otherwise would, and which they desire to do. Compare Ro 7:15-25.>
<5:18 Led of the Spirit; follow his guidance. Not under the law; as a covenant of works, but are delivered from its condemning power. No one is delivered from the condemning power of the law, or overcomes the corruptions of his heart, except under the influence of the Holy Spirit.>
<5:19 The works of the flesh; those to which corrupt human nature prompts, and when not restrained, produces.>
<5:22 The fruit of the Spirit; that which he produces in those who follow his guidance.>
<5:24 Have crucified the flesh; have, through grace, overcome the reigning power of sin, and are now habitually weakening and destroying its influence.>
<5:25 If we live in the Spirit; if our inner life be in the Spirit; that is, received from the Spirit, sustained by him, and conformed to him in character. Let us also walk in the Spirit; let our outward life also be in the Spirit; in other words, let it be conformed to him in character, so that our inward principles and outward conduct shall be in harmony with each other.>
<5:26 Vainglory; empty applause, which puffs up with pride. Provoking one another; by claims of superiority, or haughty, imperious behavior. Envying one another; for any real or supposed excellence or distinction. For every thing excellent and praise worthy, men are indebted to the grace of God. They have therefore no good reason for self-complacency or exaltation, but much for humility and gratitude.>
<6:1 Ye which are spiritual; advanced in Christian knowledge and experience. The most spiritual Christians, and those most advanced in knowledge and piety, are still exposed to aggravated sins. This should make them kind and compassionate towards all sinners, and active in efforts to reclaim them. It should make them also watchful, humble, and prayerful; remembering that but for the grace of God they might have been among the chief of sinners.>
<6:2 One another's burdens; of weakness, temptation, and sorrow. The law of Christ; to love one another as he has loved them. Joh 15:12.>
<6:3 Think himself to be something; have a high conceit of his own knowledge and attainments as a Christian.>
<6:4 Prove his own work; put it to the test by comparing it with God's word, the Bible. Then; if it is shown by that to be right. In himself alone; in the evidence which he has of his own conformity in heart and life to God's truth. And not in another; not in his fancied superiority over his neighbor. Each man should compare his views, motives, and conduct with the Bible. If they agree with that they are right, and he may rejoice in them as evidences that he is born of God and is an heir of heaven. But if they do not, they are wrong, and must be changed, or whatever he or others may think, he will be an outcast from God and all good for ever.>
<6:5 His own burden; the load imposed on him by his own sins. The word in the original is different from that used in verse Ga 6:2.>
<6:6 All good things; things needful for his support.>
<6:7 Is not mocked; will not allow men to trifle with him or his requirements.>
<6:8 Soweth to his flesh; by the indulgence of the lusts of his flesh. Compare chap Ga 5:19-21. Of the flesh; as the result of sowing to it. Reap corruption; corruption in the widest sense, the ruin of body and soul. Soweth to the Spirit; by devoting himself to the works of the Spirit, chap Ga 5:22-24. Life is the seed-time for eternity, and the fruit of what each one here sows he will there for ever reap.>
<6:9 In due season; the proper time, that which God has appointed to give the reward.>
<6:12 A fair show in the flesh; in outward observances, and thus to be in good repute with men of fleshly minds. Lest they should suffer; persecution from the Jews, if they neglected circumcision and preached the doctrines of the cross. False teachers refrain from proclaiming the truth as it is in Jesus, not only because they dislike it, but to avoid the opposition to which it would expose them, and to become popular with the wicked.>
<6:13 Glory in your flesh; in having induced you to be circumcised, and thus to join their party.>
<6:14 By whom; or, by which, referring to the cross. The world is crucified; has lost its power to control me, and I my desire to follow it.>
<6:15 In Christ Jesus; chap Ga 5:6. A new creature; Joh 3:3; 2Co 5:17.>
<6:16 This rule; the truth which he had declared. The Israel of God; all his true worshippers.>
<6:17 Let no man trouble me; with such opposition as he had received from false teachers. The marks of the Lord Jesus; scars of the wounds he had received in the cause of Christ, on account of his attachment to him and his zeal in serving him. Those who have been created in Christ Jesus unto good works, and are living not unto themselves but unto him, have the substance of true religion, and will not be disposed to contend about the shadow. They will earnestly desire and fervently pray that grace, mercy, and peace may be multiplied to all who love the Lord Jesus Christ, and walk according to the rules of his word.>
\kniha{Ephesians}
\zkratka{Eph}
<1:3 In heavenly places; the word "places" is supplied by the translators. Some propose to render, in heavenly things, things pertaining to our preparation for heaven. But everywhere else in this epistle the word means heavenly places, Eph 1:20; 2:6; 3:10; 6:12, rendered in our version "high places;" and this meaning may be retained here, as denoting the place where these spiritual blessings are prepared for us, where we shall finally enjoy them in full measure, and whence we now receive, through the Holy Ghost, the earnest of them. Compare verse Eph 1:14. In Christ; as much as to say, All these spiritual blessings come to us by virtue of our union with Christ. And so verse Eph 1:4, "He hath chosen us in him.">
<1:4 That we should be holy; he has not chosen us on the ground that we, of ourselves, make ourselves holy, but purposed that we should be made holy by the power of his Spirit. In love; referring, according to the punctuation of our version, to those whom God has chosen to be holy and without blame, as being in a state of love, which is the sum of all the Christian graces. Others join these words with the following verse: "In love having predestinated us," etc. Holiness of heart and life is sure evidence of having been predestinated to salvation, through sanctification of the Spirit and belief of the truth.>
<1:5 To himself; to be connected immediately with "the adoption of children," and meaning children which he has adopted to himself--taken by adoption into his own family.>
<1:6 Accepted in the Beloved; namely, in Christ.>
<1:7 Through his blood; making atonement for our sins.>
<1:8 In all wisdom and prudence; in the bestowal upon us of all wisdom and understanding in spiritual things, as he goes on to show in the next verse. Others refer these words to God's wisdom and prudence as exercised in bestowing upon us his grace.>
<1:9 The mystery; that which is explained in verse Eph 1:10. Men have no correct views of salvation through faith in Christ, except as God reveals it to them; and no disposition to believe on him except as God gives it.>
<1:10 The dispensation of the fulness of times; the Christian dispensation appointed by him, to be introduced when the full time should come. Gather together in one; unite into one holy kingdom. All things; in the widest sense, by subjecting every thing in heaven and earth to the dominion of Christ.>
<1:11 We; Jewish believers. An inheritance; heirship with Christ to the blessedness of heaven.>
<1:12 We--who first trusted; the gospel was first preached to the Jews, and from them were its first fruits gathered.>
<1:13 Ye, also; ye Gentiles also. Sealed; as belonging to Christ by receiving the gift of the Holy Spirit.>
<1:14 Earnest; pledge or first-fruit of heavenly felicity. Redemption of the purchased possession; complete salvation of his ransomed people. The reception of the Holy Spirit, and the blessedness which he bestows on those who follow his guidance, are sure pledges and earnests that, in due time, he will give them in perfection the blessedness of heaven.>
<1:18 The hope of his calling; the hope which he has called you to enjoy. Of his inheritance; the inheritance which he gives. In the saints; or, among the saints. These words are added to define the persons upon whom this glorious inheritance is bestowed.>
<1:19 To us-ward who believe; manifested towards us who believe, not merely in this life, but also in that to come. The exercise of this power extends over the whole work of the believer's redemption, from his calling and the quickening of his soul in regeneration to his final glorification in heaven.>
<1:20 Which he wrought in Christ; as our head. God manifests in the redemption of Christ's members the same divine power which he exercised in Christ their head. The greatness of power and grace of God manifested when he leads men to believe on Christ, and raises them from spiritual death to spiritual life, should fill them with adoring gratitude, and bind them for ever in cheerful and hearty obedience to his will.>
<1:21 Principality, and power--every name that is named; these terms describe every order of intelligent beings in heaven and on earth; every creature that bears a name.>
<1:22 Head over all things; all things in the creation. To the church; for its good.>
<1:23 Which is his body; compare Joh 15:1-7. The fulness; Christ's body the church is called his fulness, as being throughout filled with his gifts and graces. That filleth all in all; or who filleth all things with all things. For Christ is the creator of all things, and he fills them with whatever powers and privileges they possess.>
<2:1 Quickened; made alive. Dead in trespasses and sins; it is a living death which the apostle describes. They were dead to God and holiness, and alive to this world and fleshly lust. They lived in trespasses and sins, and this is spiritual death.>
<2:2 In time past; in their unconverted state. The prince of the power of the air; Satan, the ruler of the power of the air, that is, of the empire of evil spirits, whose abode is the air. Satan does much to lead men to disobey God, and when they violate divine laws they take part with Satan against Jehovah.>
<2:3 Among whom; namely, among which children of disobedience. We all; Jews and Gentiles. Had our conversation; lived. Desires of the flesh; bodily appetites and passions. Of the mind; such as pride, envy, covetousness, and ambition. By nature; naturally children of wrath, because children of disobedience. All men naturally are more pleased in gratifying their bodily appetites, and the selfish inclinations of their own hearts, than in learning and doing the will of God; thus showing that they are opposed to holiness, in love with sin, and heirs of divine wrath.>
<2:5 Together with Christ; as God raised Christ from the dead in behalf of his people and as their surety, so they, by virtue of their union with him, had been raised from spiritual death, which is the pledge of their future union with Christ in the resurrection of the body also to a glorious immortality.>
<2:6 Raised us up together--made us sit together; that is, together with Christ, as in the preceding verse. In heavenly places; see note to chap Eph 1:3. In Christ Jesus; all this takes place in and through our union with Christ.>
<2:8 And that; your being saved by grace through faith. The gift of God; all that is good in man, and all the good which he enjoys, are the gracious gift of God.>
<2:10 His workmanship; of our spiritual life, God is the author. Before ordained; it was ever the purpose and will of God, that those to whom he gives spiritual life should be holy and abound in good works. The deliverance of men from a state of sin and death, by making them alive to holiness, is of God. It springs from his love, is the fruit of his Spirit, and is given not merely to save men from perdition, but to manifest in all ages and worlds the riches of his grace, in kindness to believers, through Jesus Christ.>
<2:11 Remember; the apostle affectionately reminds the gentile converts of the unspeakable gift they have received in being introduced, through Christ, into his church. Gentiles in the flesh; in contrast with "the circumcision in the flesh;" meaning, men who bore in their flesh, as uncircumcised, the marks of their being Gentiles. Uncircumcision; uncircumcised Gentiles. Circumcision in the flesh; Jews, who had the outward sign of circumcision, but not the thing signified by it.>
<2:12 Aliens from the commonwealth of Israel; not belonging, even outwardly, to the people who were in covenant with God, had his knowledge, and maintained his worship. Covenants of promise; those made with Abraham and his seed. No hope; no hope in God, to whom ye were strangers. Without God; without the knowledge of God and an interest in his salvation.>
<2:13 In Christ Jesus; by your union with him through faith. Are made nigh; brought near to God's spiritual commonwealth and admitted into it. By the blood of Christ; making atonement for your sins.>
<2:14 Our peace; the author and ground of our peace--peace in the widest sense: first, between man and God, verses Eph 2:16-18; and then, as a consequence of this, between Jews and Gentiles, verses Eph 2:14,15. Both one; Jews and Gentiles, one body. The middle wall; the ceremonial law, which, till the death of Christ, separated Jews and Gentiles.>
<2:15 Abolished in his flesh; by his death he abolished the ceremonial law, that cause of enmity and separation between Jews and Gentiles. Contained in ordinances; thus he characterizes the Mosaic economy as a system of outward ordinances. Of twain; of the two parties, Jews and Gentiles. One new man; one new body, of which he should be the head.>
<2:16 Both; both Jews and Gentiles. In one body; in one spiritual body, namely, the Christian church. By the cross; by his bloody death on the cross as an expiation for sin. Having slain the enmity; by annulling the Jewish ceremonial law, which was the ground of the enmity between Jews and Gentiles. Thereby; literally, in it; that is, by dying upon it.>
<2:17 To you; you Gentiles, which were afar off; from God. See verse Eph 2:12. To them that were nigh; to the Jews, who, in their outward relation, were nigh to God. To both he preached peace with God, and thus with one another.>
<2:18 Peace with God, peace with conscience, and peace with one another, are the fruit of faith in Christ. By his Spirit he produces in those who believe on him a filial temper, gives them access to God as their Father, and leads them, as his children, from love to him to love one another.>
<2:20 Are built; into a spiritual temple. The foundation of the apostles and prophets; the foundation laid by them; in other words, the doctrine preached by them, the corner-stone of which is Jesus Christ. The fact, that in describing the foundation of the church, Paul, under the guidance of the Holy Ghost, says nothing of Peter, but teaches that it is built on Christ, as preached by apostles and prophets, is conclusive evidence that the belief of its being built on Peter, or any mere creature, is an error.>
<2:21 In whom; not in Peter or Paul, but in Christ; in whom all true Christians believe, and on whom they rely for salvation. Groweth; as a living temple made of living stones, 1Pe 2:5.>
<2:22 Ye also; ye Gentiles, as well as the Jews. For a habitation of God through the Spirit; God dwells in the hearts of his people who are united to him through faith and love, and thus each believer is his temple. Isa 57:15; Joh 14:23; 17:21,23,26. In like manner he dwells in his church, which is made up of believers united to him and to each other, and thus the church is, as here, his temple. Compare 1Co 3:16; 1Pe 2:5. The church of God is not composed merely of ministers of the gospel, but of all who are united by faith to Jesus Christ, and in whom he dwells by his Spirit.>
<3:1 For this cause; in view of all that has been said concerning your introduction through Christ, to the household of faith. The prisoner of Jesus Christ; one who is subjected to imprisonment for the cause of Jesus Christ. For you Gentiles; he was especially called to preach the gospel to the Gentiles, and admit them to the church without circumcision, on an equal footing with the Jews. For this he had been persecuted, and was not imprisoned. This first verse is the beginning of a sentence which is virtually resumed and continued at verse Eph 3:14, the intermediate verses being an expansion of the idea contained in the words, "for you Gentiles.">
<3:3 The mystery; namely, that explained in verse Eph 3:6, that the Gentiles, through faith in Christ, were to be partakers of his salvation on equal terms with the Jews, and without the observance of Jewish ceremonies. As I wrote afore; as I wrote a little above, chap Eph 2:12-21.>
<3:6 The gospel was designed to make all who embrace it children of God, and members of one family; to give them free access to him as their Father, and lead them to love one another as brethren. So far as it does not produce these effects on those who profess it, they have reason to fear that they have never experienced its power.>
<3:7 His power; his power in me, qualifying me for the office to which he has called me.>
<3:9 What is the fellowship of the mystery; the mystery is that, through faith in Christ, Gentiles and Jews were to be united to God and one another in holy fellowship and communion for ever. "The fellowship of the mystery" would be the fellowship of Gentiles with Jews, which the revelation of this mystery discloses. But another and better authenticated reading is, "what is the dispensation of the mystery;" that is, a dispensation which has the revelation of this mystery as its foundation principle. Hid in God; hid, as it were, among the secret counsels of God. Who created all things; and has therefore the absolute right to order all things according to his own counsel.>
<3:10 Principalities and powers; the different orders of heavenly beings. By the church; by means of God's dealings with the church. This is one of those passages which represent the angelic orders as studying with deep interest the dealings of God with men in the work of redemption. Compare 1Pe 1:12.>
<3:11 The blessings of grace, which, for Christ's sake, God bestows on those who believe, are the fruits of his eternal purpose, and are given not merely to save them, but to show to the universe the perfections of his character as they could not otherwise be made known.>
<3:13 My tribulations; on account of preaching the gospel to the Gentiles, for which he was then a prisoner at Rome. Your glory; the means of promoting your glory; that is, promotive of your heavenly glory, with all the earnests of it which ye now receive through the Holy Spirit.>
<3:14 For this cause; see note to verse Eph 3:1.>
<3:15 Of whom; of God as its author and head. The whole family; or, as the original implies, every family; namely, every one of the different orders of holy beings in heaven and earth. Is named; bears his name as the common Father of each; so that all orders of holy beings in heaven and earth are thus united into one glorious fellowship. The apostle introduces this as the climax of that great idea which he labors throughout the epistle to unfold--the union of all holy beings in God through Christ.>
<3:16 In the inner man; by a great increase of love, joy, peace, long-suffering, gentleness, goodness, faith, meekness, temperance, and all the fruits of the Spirit.>
<3:17 Dwell in your hearts; as the object of supreme affection. Rooted and grounded; fixed as trees in a deep, fruitful soil, and firm as a building on a rock.>
<3:18 Comprehend; understand more and more of the inexhaustible, eternal love of Christ, the fulness of which infinitely transcends all finite comprehension.>
<3:19 With all the fulness of God; more literally, unto all the fulness of God. So filled with his light, truth, love, holiness, and bliss, as to become in your measure like him, and shine in the glory of his image for ever. Faith in Christ is the means not only of justification, but of sanctification; rendering men stedfast and persevering in duty, enlarging their apprehensions of his love, and causing them to become more and more like him, till they are complete in the perfect image of God.>
<3:21 The glories to which God will finally exalt his people, can be comprehended by none but himself. His saints will be for ever enlarging their comprehensions; and yet, at every future period, their anticipations of what is to come will fill them with profounder adoration and a warmer zeal, and be drawing forth louder and sweeter praises to God and the Lamb for ever.>
<4:1 Therefore; on account of the glorious truths revealed in the gospel. Walk worthy of the vocation wherewith ye are called; in a manner corresponding with its high and holy nature.>
<4:3 The unity of the Spirit in the bond of peace; be united in affection and live in peace, according to the leading of the Holy Spirit. The privileges and blessings graciously bestowed upon believers, lay them under peculiar obligations to be meek and lowly in heart, patient under trials, forgiving of injuries, and active in promoting the union and harmony of all friends of God.>
<4:4 One body; the church, the body of Christ, of which all true believers are members. One Spirit; one Holy Spirit dwelling in the hearts of all, and animating all. One hope; hope of heaven, through faith in the divine Redeemer.>
<4:5 One Lord; Jesus Christ. One faith; in respect to both its object, its origin, and its inward character. It is faith in the one gospel of Christ, it is wrought in our souls by the one Spirit of God; and it is one in its nature and effects, being a faith which works by love, purifies the heart, and overcomes the world. One baptism; for all are baptized into one Saviour.>
<4:6 In you all; by his Spirit to enlighten, comfort, strengthen, sanctify, and save you. Joh 14:23; 17:23.>
<4:7 But unto every one of us is given grace; here, as in Ro 12:3-8, and 1Co 12.1-31,the apostle exhibits, in connection with the essential unity of believers, the diversity of their particular gifts; for grace here is grace qualifying us for particular offices. According to the measure of the gift of Christ; according as Christ has measured out to each his gift.>
<4:8 He saith; Ps 68:18. The apostle does not quote literally, but gives the spirit and scope of the passage, which is, that the gifts received by the ascended Saviour he bestows upon men. Ascended up; into heaven. Led captivity captive; triumphed over all his foes, and led multitudes captive as trophies of his victory.>
<4:9 The lower parts of the earth; understood by some simply of his humiliation in descending from heaven to earth. But the words more naturally mean his descent into Hades, or the world of spirits, which is mentioned as the extreme of his humiliation.>
<4:10 Far above all heavens; to the highest state of heavenly dignity, authority, and glory. Mt 28:18. Fill all things; as God, with his omnipresent power and grace.>
<4:11 Apostles--prophets, etc; see notes to the parallel passage in 1Co 12:28.>
<4:13 Till we all come; come fully. In the unity of the faith, and of the knowledge of the Son of God; better, as the margin, "into the unity," etc.; meaning that unity which full establishment in the faith and knowledge of the Son of God gives. For the greater the measure of our faith and knowledge, the greater our unity in that faith and knowledge, and thus our unity with God and each other. Unto a perfect man; a full-grown, mature man, in contrast with babes in Christ. Verse Eph 4:14. The fulness of Christ; the fulness that belongs to Christ; that is, Christ considered in his body the church. The kind and the measure of the different gifts which God bestows upon different Christians are according to his wise eternal purpose, and designed to promote the holiness and happiness of his kingdom. All these gifts therefore should be so employed as is best adapted to accomplish this end.>
<4:14 They; men who practise sleight and cunning craftiness; meaning the false teachers, who sought to unsettle believers in faith.>
<4:15 Speaking the truth; the original word means rather, walking in the truth, being truthful in word and deed. May grow up into him; so as to become mature men in him. Verse Eph 4:13. In all things; in all parts of our Christian character.>
<4:16 From whom; as the head and source of life. These words are to be connected immediately with the close of the verse, "maketh increase," etc. Compare the parallel passage, Col 2:19. Every joint supplieth; to the nourishment and growth of the body. According to the effectual working; the vital energy which is in the measure of every part; according as God has measured out to each part its office. The church is beautifully compared in this verse to the human body under the direction of the head, and rendered perfect by every member performing its appropriate office, so that there is a common interest, a common sympathy, and what promotes the good of one promotes that of all.>
<4:17 In the vanity of their mind; devoted to vain and sinful pursuits.>
<4:18 The life of God; which God gives, and which is in communion with God. Blindness; hardness and perversity. Their ignorance then is sinful, because it has a sinful cause.>
<4:19 Past feeling; having become insensible to moral and religious impressions.>
<4:20 Learned Christ; the knowledge of Christ includes the knowledge of his doctrine; for we know him as our teacher, as well as our Lord and Saviour.>
<4:22 Concerning the former conversation; as respects your former life. The old man; so thorough and radical is the change, that it is best described as a putting off of our former selves. Deceitful lusts; literally, lusts of deceit, lusts which have their ground in error and self-delusion. Compare note to verse Eph 4:24.>
<4:24 After God; after God's image. Is created in righteousness and true holiness; literally, in righteousness and holiness of truth, in that righteousness and holiness which have their ground in the knowledge and obedience of the truth. Even Christians have need of being often exhorted to be renewed in the spirit of their minds, to put off the old man, and put on the new; for they are at best only partially sanctified, and must make great advances in knowledge, piety, righteousness, and true holiness, before they will be fitted for heaven.>
<4:25 Members one of another; belong to one body, have one interest, and should no more deceive one another than ourselves.>
<4:26 Sin not; by being in anger without or beyond just cause; or by indulging it too long, in a wrong spirit, or for a wrong end.>
<4:27 Neither give place; hearken not to the devil, who will tempt you to hate such as injure you, and to seek revenge.>
<4:28 The thing which is good; in a lawful and useful business.>
<4:29 Minister grace; tend to promote the salvation of those who hear you speak.>
<4:30 Grieve not the Holy Spirit; by refusing or neglecting to follow his directions. Sealed; marked as the property, and distinguished as the children of God, by the effects which the Holy Spirit produces in you. Redemption; final and complete salvation.>
<4:32 Sacred regard to truth; freedom from violent, revengeful, or protracted anger; strict and conscientious integrity; diligence in lawful and useful business; compassion towards the needs, and a disposition to aid them--are all essential to completeness of Christian character: and no one has any more true religion than he has in these respects likeness to Christ.>
<5:1 Followers of God; imitators of him, especially in his love to men. We have reason for everlasting gratitude to God that he has given us a perfect example; and it should be our great object perfectly to imitate it.>
<5:2 Sweet-smelling savor; peculiarly pleasing and acceptable to God.>
<5:3 Covetousness; the excessive desire of worldly gain, leading to its unlawful pursuit.>
<5:4 Filthiness; obscenity in words or actions. Not convenient; not fit, proper, useful. A grateful temper, and the habit of expressing it in thanksgiving to God, form a safeguard against temptation and against improprieties of thought, word, and deed.>
<5:5 Covetous man; one who regards supremely earthly good. See note to verse Eph 5:3.>
<5:6 Let no man deceive you; by inducing you to think such persons as are mentioned, verse Eph 5:5, can go to heaven.>
<5:7 Those who maintain that continuance in sin will not exclude men from heaven are deceivers. As such they should be treated, and their doctrines be rejected.>
<5:8 Darkness; living in ignorance and sin. Light in the Lord; enlightened and renewed by virtue of your union with the Lord Jesus. As children of light; those who belong to the kingdom of light, of which God in Christ is the head.>
<5:10 Proving; in an experimental way, by the actual subjection of yourselves to the will of God. See note to Ro 12:2.>
<5:12 It is a shame even to speak of those things; they are too vile to be mentioned or even thought of but with abhorrence.>
<5:13 Are made manifest by the light; by the light that reproof sheds upon them, thus revealing their heinous character. Whatsoever doth make manifest is light; rather, as the same word has just been rendered, whatsoever is made manifest, namely, by reproof, is light--it ceases to lie hid in the dark, and is seen in its true character.>
<5:14 He saith; the Lord saith, namely, by the general tenor of his word. Sleepest; are stupid and senseless in the darkness and pollution of sin. Arise from the dead; wake from death in sin to a sense of thy guilty, lost condition, and look to Him who died for thee, and he will make thee light, and thus sanctify and save thee. The deadness of men in trespasses and sins is not such as to free them from obligation to awake and rise to newness of life. Of course ministers of the gospel are bound to call upon them thus to awake without delay; and under the influence with which God accompanies this call, all should hear and obey.>
<5:15 Circumspectly; with caution and habitual regard to what is God's will. Verse Eph 5:10. Not as fools; regardless of danger. But as wise; perceiving the evil, and avoiding it--the good, and pursuing it.>
<5:16 Redeeming the time; time considered as furnishing opportunity for serving Christ; in other words, making the most of every opportunity. The days are evil; days of wickedness, such as will oppose many hinderances to your Christian activity.>
<5:18 Be not drunk; for drunkenness will prevent you from understanding and doing the divine will, and bring upon you the wrath of God. Excess; abandoned wickedness of all sorts. The Spirit; the Holy Spirit. Invite his influences, receive his consolations, and yield yourselves to his guidance. Wine, in all countries, is intoxicating; and Christians who use it as a beverage, are exposed to the sin of drunkenness.>
<5:21 Submitting yourselves; yielding cheerful obedience to proper authority, from regard to God, who established it. This general precept he then proceeds to expand by the mention of particular cases. The only security from the most debasing crimes is, in being habitually under the influences of the Holy Spirit, actively engaged in the service of God, and in the conscientious discharge of duty.>
<5:22 As unto the Lord; as those who, in obeying their husbands, obey the Lord Jesus, because he requires such obedience.>
<5:23 And he; Christ. Is the Saviour of the body; the church, which is his body.>
<5:24 In every thing; see note to chap Eph 6:1.>
<5:25 As Christ--loved the church; with a pure, ardent, self-sacrificing love.>
<5:27 One of the great institutions of God for keeping alive in this world and securing to men the benefits of the knowledge and worship of himself, is that of the family; and the feelings and conduct of the wife towards her husband, and the husband towards his wife, may make it a nursery for and foretaste of heaven, or a preparation for and an emblem of hell.>
<5:28 Loveth himself; with allusion to the scriptural declaration, "they shall be one flesh." Ge 2:24. Their union is so intimate and indissoluble that their happiness is inseparable; and what promotes the interest of one, promotes that of the other.>
<5:29 As the Lord the church; which is "his flesh and his bones," verse Eph 5:30. As a man cares for himself, and as Christ cares for his church, so a husband should care for his wife.>
<5:30 Members of his body; so that he loves and cherishes us as his own flesh.>
<5:31 For this cause; on account of the oneness which God has formed between a man and his wife, which represents the union between Christ and believers, and is somewhat like the union between the soul and body. The union for life of one man and one woman in marriage, was designed by God to illustrate the union of Christ and his people; and the spirit which he manifests towards his church is that which they should manifest towards each other.>
<5:32 A great mystery; the mystery of the union of Christ and believers, represented by the marriage union, and which makes it proper for the Holy Ghost to speak of believers as members of the body of Christ, of his flesh and of his bones, and of their being so joined to the Lord as to be one spirit. 1Co 6:17. But I speak concerning Christ and the church; in other words, My object is to direct your thoughts to the higher relation of Christ to his church, which is shadowed forth by the lower relation of husband and wife.>
<5:33 Nevertheless; as much as to say, But to drop this mystical application of the marriage relation. Reverence; honor him, respect his authority, and thus promote the peace, usefulness, and happiness of the family. It is the duty and the privilege of husbands and wives, from supreme love to God, to love themselves and each other--to perform any labors, submit to any self-denials, and make any sacrifices which may be needful for his glory and their highest individual and mutual welfare.>
<6:1 Obey your parents; it is to be understood here, as in chap Eph 5:24, that the obedience enjoined extends to all things not contrary to Christ's commands; for the addition, in the Lord, that is, obey as those who are in the Lord, and make his will the law of their being, excludes obedience to those commands which are contrary to Christ's word.>
<6:2 With promise; with a promise annexed, namely, that of long life and great blessings. Ex 20:12.>
<6:3 The gospel inculcates perfect fidelity in the discharge of all the relative duties of life; and children who are kind, respectful, and obedient to parents, take the way to become blessings to themselves, their parents, the church of God, and the world.>
<6:4 Provoke not your children; give them no just occasion to be angry or to feel as if they were injured. As the highest good of children in this life and the life to come requires them, in all things right, to obey their parents, it is the duty of parents to take the course which is best suited to secure this, and lead their children also to obey their Father in heaven. In order to this, they must obey him themselves, daily seek his guidance and blessings, instruct their children to do his will, and present to them the motives which he has revealed. They must also accustom their children, from their earliest years, promptly to submit their wills to the will of their parents, so that it shall, by habit, become easy and pleasant.>
<6:5 Masters according to the flesh; persons to whom you justly owe service, or who by human laws have power to force you to serve them. Obey their commands whenever you can do it without disobeying the commands of your Master in heaven. With fear and trembling; reverentially, and with that fear of God which is the beginning of wisdom, and which shall make you anxious to please him. As unto Christ; for the purpose of honoring him and promoting his cause.>
<6:6 Not with eye-service; not outwardly merely, while men are looking on, or for the purpose of pleasing them--not a constrained external service, but from the heart, out of regard to God.>
<6:7 With good will; kind and cheerful readiness.>
<6:8 The same shall he receive; the servant shall be rewarded by God for obeying him, as certainly and abundantly as if he were not a servant. Servants are bound to be servants of Christ, and from love to him to obey, in things not wicked, their earthly masters; and to do it for the purpose of pleasing him: showing the excellence of his religion, and promoting its influence in the world. For doing this, they will receive from him a gracious and glorious reward.>
<6:9 Do the same things; be governed by the same supreme regard to God which is inculcated on servants, and manifest the same kind, benevolent, and cheerful readiness to please God in your conduct towards them, which they are required to manifest in their conduct towards you. Forbearing threatening; avoiding it, and seeking to win them to the love and service of God. They are your brethren, children of the same heavenly Father, redeemed by the same almighty Saviour, and you must stand with them before the same impartial Judge. Neither is there respect of persons with him; you will not be favored because you are masters., nor they less favored because they are servants. Your more elevated position increases your responsibility, and if you do not possess and manifest the spirit of Christ, will increase your condemnation. Masters are bound to be servants of Christ, doing his will from the heart, and manifesting towards their servants his spirit; doing to them in all things as, under a change of circumstances, they ought to wish their servants to do to them; knowing that Christ requires this, and that they are both to stand before him in judgment, and to receive for eternity, not according to their outward condition, but according to their character and conduct.>
<6:10 Be strong in the Lord; as those who are united by faith to the Lord Jesus, and depend on him for strength and all needed aid to perform every duty, bear every trial, and conquer every foe. In the power of his might; in the power possessed by you, which his might furnishes. Although all our strength in the Christian life comes from God, it is still our duty to have strength, because it is our duty to look to God for it in faith and humility, and when we so look, we always receive it.>
<6:11 Armor of God; that which he has provided in and through Jesus Christ, and furnishes by his word, Spirit, and providence. Wiles; devices, stratagems to deceive and destroy.>
<6:12 We wrestle not against flesh and blood; weak men like ourselves. Our great contest is not with men, but with various orders of evil spirits, styled principalities, powers, and rulers of darkness. The rulers of the darkness of this world; those evil spirits who rule this world in and through the spiritual darkness that prevails in it. Spiritual wickedness; or, spiritual powers of wickedness, armies of evil spirits. In high places; in the regions of the air. See note to chap Eph 2:2.>
<6:13 The whole armor of God; literally the panoply of God; namely, the complete armor which he has provided for you in the gospel of his Son. The evil day; when tempted to sin, assailed by enemies, and beset with trials. Having done all; having gone through with the whole conflict.>
<6:14 Having your loins girt about with truth; having for the girdle of your loins truth in word and deed. The allusion is to the military girdle, which was worn about the loins for strength, and not for mere ornament. The breastplate of righteousness; the "righteousness and holiness of truth," chap Eph 4:24, which is wrought in the soul by God's Spirit.>
<6:15 Your feet shod; the reference is to the military shoes of warriors. The preparation of the gospel of peace; the inward preparation of mind which the gospel of peace gives. For by shedding abroad in the soul "the peace of God which passeth all understanding," the gospel furnishes it with courage, zeal, and alacrity for every duty.>
<6:16 Fiery darts; in allusion to the darts fitted with burning substances that were used by the ancients. The wicked; the wicked one, the devil. He means the fiery temptations inward and outward by which Satan seeks to destroy us, and which can be quenched only by faith.>
<6:17 The helmet of salvation; in 1Th 5:8, he says, "for a helmet, the hope of salvation." The sword of the Spirit; the sword which the Spirit furnishes. The apostle would have us stand firm in the faith and practice of the gospel, and ever ready to propagate and promote it; relying with implicit confidence on Christ, and expecting the fulfilment of his declarations; acquainted with the Scriptures, and using them for doctrine, reproof, correction, and instruction in righteousness; habitually and fervently praying, in secret, in the family, and in public, not only for yourselves but for all Christians, and especially for ministers of Christ; that without fear of man they may preach the whole gospel in its just application to all their hearers.>
<6:18 None will stand firm under the banner of Christ, and fight perseveringly and successfully with Satan and his allies, but those who rely on Christ for strength, and clothe themselves with the armor which he has provided. But taking the Bible for their guide, and habitually seeking the presence of the Holy Spirit, under a Leader who never was and never will be overcome, they may go triumphantly from conquering to conquer.>
<6:21 Tychicus; by whom Paul sent this epistle from Rome to Ephesus.>
<6:22 Comfort your hearts; by hearing of the goodness of God to Paul, the success of the gospel, and the readiness of God to aid and bless all who put their trust in him.>
\kniha{Philippians}
\zkratka{Phil}
<1:4 Faithful ministers of Christ habitually pray for the spiritual prosperity of his people; and when his people so live as to honor him, it gives his ministers exceeding joy.>
<1:5 For your fellowship in the gospel; more literally, for your fellowship unto, or towards the gospel; in other words, your common interest and fellowship in the work of promoting it. From the first day; of your faith.>
<1:6 He; God. Perform it; carry it on to perfection, finish it, as in the margin. The day of Jesus Christ; of his coming to judgment.>
<1:7 Because I have you in my heart; the marginal rendering, "because ye have me in your heart," best suits the context, their affection for the apostle being to him one of the proofs that God had begun a good work in them. My grace; the grace of being allowed to suffer for Christ and labor in the defence and confirmation of his gospel, as just stated. Compare verse Php 1:29, where suffering for Christ is regarded as a gift of God; and Eph 3:8, where preaching the gospel is spoken of as a grace and a gift.>
<1:8 Long after you; to see you and promote your benefit. The bowels of Jesus Christ; tender regard for you such as Christ himself feels.>
<1:9 Judgment; discernment in spiritual things. The apostle will have the love of believers enlightened and discerning.>
<1:10 Approve things that are excellent; or, try things that differ, for the purpose of approving the good and rejecting the evil.>
<1:11 Every thing good in men is the fruit of the Holy Spirit; and where he has begun his work in their hearts, teaching them to practise piety towards God and righteousness towards men, we may confidently expect that he will carry it forward, through faith and obedience, till they are perfect in glory.>
<1:12 The things which happened unto me; my imprisonment, and all the circumstances connected with it.>
<1:13 My bonds in Christ; see note to Eph 3:1. Are manifest; are made known, not simply as bonds, but as bonds in Christ. In all the palace; in the original it is, in all the praetorium; by which is to be understood the praetorian camp, that is, the camp of the emperor's body-guards.>
<1:14 Waxing confident; by seeing how God supported Paul, and gave efficacy to his preaching, even in his bonds.>
<1:15 Envy and strife; envy of the influence of Paul, and for the purpose of raising up a party hostile to him. They were manifestly the same class of preachers who disparaged his influence at Corinth, and sought to supplant him in the affections of the church.>
<1:18 In pretence; such as the false teachers employed, covering up their base designs of self-exaltation under a show of zeal for Christ.>
<1:19 To my salvation; the apostle's salvation in the widest sense. Compare Ro 8:28.>
<1:20 With all boldness; while I exercise all boldness in the gospel, being assured that all things will work together for good to me and Christ's church through me. In my body; in that which befalls my body. The efforts of the wicked to injure the righteous and hinder the success of the gospel, God overrules for the good of his people; so that in nothing need they be discouraged, but with meekness and calmness may go forward, rejoicing that whether they live or die, Christ will be honored and his cause triumph.>
<1:21 Is Christ; his great object was the glory of Christ, and the favor of Christ was his chief joy. To die is gain; it would be more for his happiness than to continue on earth.>
<1:22 This is the fruit of my labor; this is the way in which my labor can bear fruit for the good of men.>
<1:23 In a strait betwixt two; strongly drawn two different ways. having a desire to depart; more literally, having my desire towards departing, as that way of the two which is far better; better as respects my personal enjoyment of Christ.>
<1:25 This confidence; that his longer continuance on earth would be more for their benefit.>
<1:27 Your conversation; your conduct and intercourse of life.>
<1:28 An evident token; the sustaining presence of God which he grants you, shows that he will save you and destroy those who continue to oppose you.>
<1:29 It is given; given as a privilege. Compare 1Pe 4:13,14. In the behalf of Christ; for the purpose of honoring him. By enduring trials with a proper spirit, believers honor Christ as really as by active labors; and they have no more just reason to complain or be discontented when he visits them with adversity, than when he crowns them with prosperity; for in both he consults his glory, their highest good, and the good of his cause.>
<1:30 Conflict; with the enemies of the gospel.>
<2:1 Conflict; with the enemies of the gospel.>
<2:4 While the gospel inculcates universal humility and benevolence, it produces these virtues in all who savingly embrace it, and thus shows itself to be divine.>
<2:7 Made himself of no reputation; emptied himself; for a time relinquished the glory which he had with the Father before the creation.>
<2:9 Highly exalted him; as Mediator, head over all things to his church. Mt 28:18.>
<2:10 That at the name of Jesus every knee should bow; that all in heaven and on earth should worship him. Mt 4:10; Joh 5:23.>
<2:11 Humility and benevolence are peculiarly pleasing to God. The most wondrous exhibition of them was made by Jesus Christ: and those who imitate him will, with him, receive a glorious reward; while the homage which they and all holy creatures will render him, will show that he is God.>
<2:12 Work out; by obeying God. With fear and trembling; lest you should live in sin and fall of eternal life.>
<2:13 Worketh in you; influencing you by his Spirit to do what is pleasing to him. In order to be saved, men must work out their own salvation by faith, love, and obedience, as God has appointed; and the fact that whenever they are inclined to do it, he works in them, and thus influences them to work out their salvation, gives them the greatest encouragement, without delay, to engage in this work.>
<2:15 Without rebuke; without doing any thing to deserve rebuke.>
<2:16 Holding forth; exhibiting in principle and practice the gospel of Christ. Not run--neither labored in vain; in the toil and labor bestowed on you.>
<2:17 If I be offered; literally, poured out as a drink-offering; that is, if my blood be thus poured out. Upon the sacrifice and service of your faith; upon the service of presenting your faith as a sacrifice acceptable to God. The apostle compares himself to a priest ministering spiritually at God's altar, and presenting, as his offering, the faith of the Philippians. Compare note to Ro 15:16.>
<2:18 Rejoice with me; grieve not at my sufferings, or even death for your sakes; but join me in giving thanks to God for it. The holy example of Christians is conclusive evidence of the usefulness of ministers of Christ. For the promotion of it they are willing to labor, to suffer, and if need be to die; rejoicing that they can, even in death, promote so important an object.>
<2:19 I trust in the Lord Jesus; Paul trusted in him as the God of providence as well as of grace.>
<2:20 No man; no man among those now in attendance upon me. Like-minded; in his ardent attachment to them, and desire for their good.>
<2:21 All; apparently spoken of those then in attendance on Paul. Seek their own; selfishness is natural to all, and Paul's companions, though Christians, were only partially delivered from it. All men naturally love themselves with all the heart, and soul, and strength, and mind; but they do not love God. The gospel when embraced dethrones this idol, and leads men supremely to love God, and benevolently to seek the good of their fellow-men.>
<2:23 How it will go; at his trial before the Roman emperor; whether he should be acquitted or condemned.>
<2:25 Epaphroditus; he was from Philippi, and had come to Rome to bring assistance to Paul. Chapter Php 4:18.>
<2:29 Hold such in reputation; honor those who make such sacrifices in the cause of Christ.>
<2:30 To supply your lack of service; that in your absence he might, as your messenger, assist me. Those who at the call of duty make sacrifices and suffer trials in the cause of Christ, and to benefit his friends, are peculiarly dear to him; and for their work's sake, they should be highly esteemed and honored by his people.>
<3:1 The same things; which he had before inculcated. Some suppose that he has especial reference to the exhortation to rejoice, which he had already given, chap Php 2:18, and repeats again emphatically, chap Php 4:4. No persons have so much occasion for, or possess so much real joy, as true Christians.>
<3:2 Dogs; contentious and impure false teachers. The concision; that is, the cutting, namely, of the flesh in circumcision. The apostle applies the term to those who contended for the outward mark of circumcision as essential to salvation. We should beware of the doctrines of those who rely on external observances, or on their own works for salvation; especially when, by propagating their doctrines, they seek to promote divisions among Christians.>
<3:3 We; believers in Christ. Are the circumcision; they had the true circumcision, which alone was essential, that of the heart; they were cut off from the love of sin, and justified through faith in Christ, of which the outward mark was a sign, or, as revealed, Ro 4:11, a seal of the righteousness of faith. No confidence in the flesh; in any natural goodness, external privileges, distinctions, or works of their own, as a ground of salvation. True Christians possess that, the necessity of which was indicated by circumcision; without the sign they have the thing signified, as had Abraham before he was circumcised. Ro 4:11.>
<3:5 Touching the law; in his view of it and of his mode of keeping it.>
<3:6 Touching the righteousness which is the law; as to that external observance of it on which the Pharisees relied for salvation.>
<3:7 What things were gain; those by which he once had hoped to gain eternal life. Loss; he renounced all dependence on them, sensible that dependence on them, should it continue, would cause the loss of his soul.>
<3:8 All things; as a ground of dependence for salvation.>
<3:9 Not having mine own righteousness; as a ground of acceptance. The righteousness which is of God by faith; see note to Ro 1:17.>
<3:10 Know him; Christ in his true character, by trusting in him and experiencing the transforming effects of such knowledge in my own soul. The power of his resurrection; according to some, the full power which the fact of his resurrection should have on the mind. But we may better understand here, as in Ro 6:4; Eph 1:19, 20; 2:1,5,6, the power of God manifested in the resurrection of Christ. The meaning then will be, that I may know, by blessed experience in my own person, the divine power which raised Christ from the dead: first, as manifested in quickening me more and more from my former death in sin to a life of holiness in Christ; secondly, as exerted in raising my body, as that of Christ was raised, to a glorious immortality. The fellowship of his sufferings; the endurance of trials when called to it as Christ was, in communion with him and in the exercise of his spirit. Conformable unto his death; his life of suffering which led to and ended in death; the same as "always bearing about in the body the dying of the Lord Jesus," and "always delivered unto death for Jesus' sake." 2Co 4:10,11.>
<3:11 Resurrection; that which awaits the just--in the likeness of Christ, to shine as the brightness of the firmament, and as the stars for ever. For that we may be glorified with him, we must first suffer with him. Ro 8:17; 2Ti 2:12. Souls enlightened and renewed by the Holy Spirit renounce dependence for salvation upon privileges and external distinctions, as well as upon works, and make it their great object to be justified, sanctified, and saved through faith in Him who died, the just for the unjust, to bring them to God.>
<3:12 Not as though I had already attained; the prize of my high calling, mentioned below, verse Php 3:14. Apprehend; obtain that heavenly glory for which Christ had arrested and called him when a persecutor, and for which, through faith and patience and much tribulation, He was preparing him.>
<3:13 To have apprehended; the fulness of that to which he was appointed, and which he most earnestly desired. Reaching forth; as a racer, who never looks back, but always bends forward in his race.>
<3:14 I press toward the mark; the goal at the end of the course. For the prize of the high calling of God in Christ Jesus; the prize which God's heavenly calling has in view. This prize is perfect likeness to and full glory with Christ, for which Christians were led to renounce their sins, believe on him, and devote their life to his service. Joh 17:24; 1Jo 3:2.>
<3:15 Be perfect; have attained to mature Christian knowledge. Be thus minded; have my mind as just described. Be otherwise minded; if any had not attained to so much light as to their duty, let them improve what they had, and earnestly seek the teaching of God, and he would impart more. Those who would be perfect must not imagine that they are so, or count themselves to have attained complete likeness to Christ; but they must press onward and daily seek it, in obedience to him, till the end of life.>
<3:16 Whereto we have already attained; so far as we understand duty let us heartily do it, taking the Bible for our guide, regarding God as our Father, and all his children as brethren, affectionately uniting with them in what is right, and kindly endeavoring to enlighten and convince them where they are wrong.>
<3:19 Who mind; seek as their chief good.>
<3:20 Our conversation; more exactly, our citizenship: we are citizens of heaven; our King is there; our hearts and thoughts are there; we obey its laws, and look to it as our everlasting home. As Christians are citizens of heaven, and only pilgrims and sojourners here, they should not be greatly influenced by things of earth, or chiefly occupied with its concerns. Their treasure is, and their hearts should be in heaven; whence they look for Christ perfectly to change them into his own glorious image, and raise them for ever to reign with him in the kingdom of their Father.>
<3:21 Shall change; at the resurrection. The working; the almighty power or energy. To subdue all things unto himself; consequently death also, the last enemy that should be destroyed. See 1Co 15:26, where the destruction of death is mentioned in immediate connection with the subjection of all things to Christ.>
<4:1 So stand fast; as those whose citizenship is in heaven, and who have the glorious hopes just named.>
<4:2 Euodias, and--Syntyche; two Christian women at Philippi. In the Lord; in love to him and efforts to promote his cause. Differences among Christians, especially in religious matters, are a great hinderance to the gospel, and should as soon as practicable be healed.>
<4:3 True yoke-fellow; a person whose name and office are to us unknown. The words may possibly be addressed to Epaphroditus, the bearer of the epistle, whom, as present, he does not think it necessary to name. Labored with me; not in preaching--as public teachers--but in ways appropriate to women, exerting their influence to promote the success of the gospel. Pious women may do much to promote the cause of Christ, and in a way which shall furnish increasing evidence that God has enrolled their names among the heirs of heaven.>
<4:5 Moderation; mildness, especially towards opposers. The Lord is at hand; to deliver you, and punish your persecutors. The shortness and uncertainty of time should moderate our desire for earthly things, and lead us so to use them, that in the account we must soon give, the Judge may say, "Well done, good and faithful servants; enter ye into the joy of your Lord.">
<4:6 Careful; anxious, solicitous.>
<4:7 The peace of God; that which he gives, and like that which he enjoys. Joh 14:27; Isa 26:3. Shall keep; in a state of joyous composure and security. The original word means, keep as a military watch. Habitual affectionate communion with God, asking him for all good which is needed, praising him for all that is received, and trusting him for future supplies, prevents anxious cares, inspires peace, calmness, and composure, and furnishes a delight surpassing all finite comprehension.>
<4:8 Honest; honorable and worthy of being respected. Any virtue--praise; any thing truly virtuous or praiseworthy. Think on these things; attend to and practise them. Professors of religion should be careful never to falsify their word, or be mean or dishonorable, unjust, impure, or unamiable; but conscientiously and habitually to practise whatever deserves to be respected and is praiseworthy.>
<4:10 Your care of me; in sending him assistance. Ye were--careful; ready to assist, but had not opportunity.>
<4:11 Not--in respect of want; his necessities were not the reason of his mentioning this subject. Compare verse Php 4:17.>
<4:12 To be abased; to be destitute. To abound; to have an abundance. Instructed; literally, initiated, as into something of which the world at large is ignorant. God has taught me to bear prosperity with meekness, and adversity with contentment.>
<4:13 Do all things; to which he was in duty called.>
<4:15 In the beginning of the gospel; when he first preached it to them. Communicated with me; by sending him relief.>
<4:17 Not because I desire a gift; he did not make this suggestion merely or principally for his own sake. Fruit; the fruit of your Christian liberality, as something set to your account in heaven to be graciously rewarded.>
<4:19 None need neglect any known duty, or be discontented and unhappy in any condition of life. Let them trust in the Lord and do good to all as they have opportunity, especially to his people for his sake, and he will support them, and supply all their wants--not according to the narrowness, weakness, and unworthiness of their conceptions, but according to the riches of his grace in Christ Jesus.>
<4:22 They that are of Cesar's household; persons attached to the emperor's household, who had been converted by the labors of Paul or his associates.>
\kniha{Colossians}
\zkratka{Col}
<1:2 To obtain the greatest and richest of all blessings for himself and his fellow-Christians, Paul, under the guidance of the Holy Ghost, was in the habit of seeking them from God our Father and the Lord Jesus Christ, in a manner which showed that he knew them both to be divine.>
<1:5 For the hope; that is, connecting these words with "we give thanks," on account of the hope. But we may better connect them with verse Col 1:4, and render, through the hope; meaning faith and love exercised under the influence of the hope laid up for them in heaven--the hope here standing for the inheritance which is its object.>
<1:6 In all the world; wherever the gospel was embraced it produced the same effects as in Colosse. The gospel, wherever preached, influences all who embrace it in truth, whatever may have been their past character and condition, to lives of holy obedience; and inspires them with a hope which tends to purify them, even as Christ is pure. 1Jo 3:3.>
<1:7 As ye also learned of Epaphras; learned "the grace of God in truth" of Epaphras, who seems to have been their first teacher, and was now with Paul in Rome.>
<1:8 Love; that which the Holy Spirit produces.>
<1:9 Spiritual understanding; understanding of spiritual things as revealed in the gospel.>
<1:10 Unto all pleasing; in a manner constantly or every way pleasing to God.>
<1:11 Patience; endurance under trials. That patient and joyful submission under trials which becomes the friends of Christ, it is difficult for them to exercise, and nothing will effectually secure it but the power of God.>
<1:12 Made us meet; prepared us. Saints in light; in the enjoyment of the bliss of heaven.>
<1:13 The power of darkness; dominion of sin and Satan.>
<1:15 The image of the invisible God; the invisible God made manifest. Christ is the image of God, as possessing perfect equality with the Father in substance and divine perfections. Compare his own words: "He that hath seen me, hath seen the Father." Joh 14:9. The first-born of every creature; or, the first-born of all creation. Since Christ is the creator of all things, verse Col 1:16, he is not himself one of the creation. But he is the first-born of all creation, as being before all things, verse Col 1:17, and above them as their supreme head, verse Col 1:20.>
<1:16 In heaven--in earth, visible and invisible; the apostle labors to assert in the most absolute way that the whole universe of created things is the work of Christ. Compare Joh 1:3, and Heb 3:4. "He that built all things is God." Thrones--powers; words that denote the different orders of created intelligences. Compare Eph 1:21; 3:10. For him; as their end. This is the highest possible assertion of Christ's proper deity.>
<1:17 Consist; are upheld in their present state. Compare Heb 1:3. "Upholding all things by the word of his power.">
<1:18 Head of the--church; source of its life, light, and all its blessings. First-born from the dead; the first who rose never again to die, and who will raise all the dead--some to everlasting life, and some to shame and everlasting contempt. Joh 5:28,29. Have the preeminence; as head of the created universe, and the object of supreme affection to all who put their trust in him. As Christ owns the bodies and souls of men by the right of creation and redemption, and as he made, redeemed, and preserves them for himself, not to devote themselves to his service is injustice and dishonesty.>
<1:19 All fulness; all the fulness of the Godhead; as a Saviour, then, he has all that is needful to save to the uttermost those who come unto God by him. Heb 7:25.>
<1:20 Made peace; opened the way for peace. Things in earth--things in heaven; that the opposition between heaven and earth, which sin has occasioned, may be removed, and all things in heaven and earth may be united under Christ as their head in one harmonious body.>
<1:21 Alienated; from God. By wicked works; literally, in wicked works, these being, as it were, the element in which they lived.>
<1:22 In the body of his flesh through death; by his propitiatory death on the cross in human nature. To present; to present you before God, as the final result of his work of redemption. Eph 1:4. In Eph 5:27, the nature of the figure--Christ and the church his bride--required the apostle to say, "that he might present it to himself." The two forms of speaking come to the same thing.>
<1:23 Grounded and settled; firm and steadfast in the belief and practice of the truth. To every creature which is under heaven; the apostle in these words expresses the design of the gospel and its final destiny. Those who would be saved must not merely believe in Christ, profess him before men, and begin to serve him; they must continue in the belief of the truth, in the practice of piety towards God, righteousness and benevolence towards men, and in the conscientious discharge of their various personal and relative duties to the end of life. Mt 10:22; Heb 10:38,39.>
<1:24 Fill up that which is behind; that which remains to be yet endured. The afflictions of Christ; afflictions to be undergone by Christ in his body the church; that is, in the persons of his disciples. Of these every believer has his share to fill up, and ought to rejoice in it, because God with these means works out his salvation and that of his brethren.>
<1:25 Whereof; of which church. For you; for the benefit of you Gentiles. To fulfil the word of God; to fulfil my stewardship of God's word by dispensing it faithfully. Compare Ro 15:19, where the original is, "I have fulfilled the gospel of Christ.">
<1:26 Even the mystery; added to show wherein the fulfilling of the word of God lies, namely, in unfolding the mystery, etc. The mystery is that so fully unfolded in Eph 2:13-22, namely, God's purpose to unite Jews and Gentiles through Christ on equal terms in one holy and spiritual body.>
<1:27 This mystery among the Gentiles; because it is in the reception of the Gentiles to God's spiritual fold that the glory of this mystery is especially displayed.>
<1:29 Striving; or struggling. See note to chap Col 2:1. Worketh in me; God working in Paul was the cause of his working, and the reason why his work was efficacious in reconciling men to God through Christ.>
<2:1 Conflict or struggle, in allusion to what he has just said of his striving, chap Col 1:29. In the original Greek the two words agree, like "struggling" and "struggle." It is an inward conflict, the object of which he explains in the following verse. As many as have not seen my face in the flesh; the obvious meaning of these words is, that the Colossians were among those who had never enjoyed Paul's personal ministry.>
<2:2 Be comforted, being knit together in love; that is, comforted in the way of being knit together in love. Unto all riches of the full assurance of understanding; these words give the end to which such a union of love tends. The apostle means that understanding of spiritual things which carries with it the full assurance of their reality and excellence, and which is, moreover, possessed in rich measure. To the acknowledgment; or, unto the knowledge. This is added, as a parallel clause, to explain the object to which the understanding just spoken of has reference. It is, the mystery of God--of Christ; in other words, the mystery of the plan of redemption in God and Christ, with special reference to that feature of it which unites Jews and Gentiles in one body under Christ as their head.>
<2:3 Hid; treasured up, to be communicated according to the wants of those who believe. The religion of Christ makes all who possess it truly benevolent, and leads them earnestly to desire the holiness and happiness of others.>
<2:4 With enticing words; false persuasion; words of mere human wisdom, such as are more fully explained afterwards in verses Col 2:16-23.>
<2:6 Christ Jesus the Lord; emphatically, Christ Jesus as your supreme Lord and Saviour, and no yoke of carnal ordinances. Walk ye in him; continue in the belief and practice of those truths which you received when you gave your hearts to Christ.>
<2:7 Rooted and built; firmly established, like a tree deep rooted, or a house on a rock. A heart abounding in thanksgiving to God for his mercies, especially for Jesus Christ and life through him, is a great safeguard against error, a source of the purest enjoyment, and a means of the greatest good.>
<2:8 Lest any man spoil you; literally, make booty of you; rob you of spiritual blessings by leading you to depend on something besides Christ for salvation. Rudiments of the world; the Mosaic ceremonies, so called as containing, in comparison with the gospel of Christ, only the first elements of religion, even when rightly used; while they were so perverted by the false teachers, that they fed the spirit of worldly confidence, and made those who trusted in them carnal, instead of spiritual.>
<2:9 Godhead bodily; God incarnate, or dwelling in human nature. Joh 1:14; Rom 9:5; 1Ti 3:16; Heb 1:6-8.>
<2:11 Ye are circumcised; have experienced that spiritual renovation, that cutting off or renouncing of sin, the necessity of which circumcision taught. Without hands; literal circumcision was made with hands, but the spiritual circumcision which they had experienced was wrought by the Holy Spirit, through the atonement, righteousness, and intercession of Christ. The true circumcision, that which God requires and which is essential to salvation, is not any thing which is outward merely, or wrought by men. It is the work of the Holy Spirit, and the fruit of faith in Christ.>
<2:12 Buried with him in baptism; by openly renouncing sin, and publicly professing to hate and forsake it; thus showing their deadness to its reigning power. Risen with him; from their death in sin, by believing on him, and thus experiencing in their own persons the same divine power which raised Christ from the dead.>
<2:13 Dead in your sins; while in their unconverted state. The uncircumcision of your flesh; that inward uncleanness of soul which their outward state of uncircumcision well represented. Quickened; made spiritually alive.>
<2:14 The handwriting of ordinances; the Jewish ceremonial law, which lay in the letter, and not in the spirit. Contrary to us; burdensome, opposed to our liberty and peace as Christians, and constituting a "middle wall of partition" between Jews and Gentiles. Nailing it to his cross; as a sign of its abrogation; in other words, annulling it by his expiatory death on the cross.>
<2:15 Principalities and powers; the powers of darkness, of which Satan is the leader. Compare Eph 6:12. These our Lord overcame by his death and resurrection. Joh 12:31; 14:30; 16:11. Made a show of them openly; let them in triumph, as a conqueror his captives. Compare Eph 4:8. In it; in his cross, as the means of his victory over them.>
<2:16 Judge you; pronounce you good or bad, according to your treatment of the ceremonial law. A holy-day--sabbath-days; in the original, a festival--sabbaths. The days referred to are those required to be observed in the ceremonial law--days associated by God with meats, drinks, and new moons. The passage does not refer to the Sabbath of the moral law, associated with the commands forbidding theft, murder, and adultery. This weekly Sabbath was never against men or contrary to them, but was always for them, and promotive of their highest good. The observance of it caused them to ride upon the high places of the earth, and to possess the heritage of God's people. Isa 58:13,14; Jer 17:21-27.>
<2:17 A shadow; of the Redeemer who was to come; pointing to him as the only and all-sufficient Saviour. The body is of Christ; he is the substance to which, as shadows, all the Jewish rites referred. Circumcision and all the Mosaic rites and ceremonies were designed to show men their need of inward purification, and the necessity of believing on Christ in order to obtain it. In him we have all that we need. There is no occasion, then, that we look for salvation to Jewish, or other kindred ceremonies, to saints, to the Virgin Mary, or to any one except Christ.>
<2:18 Of your reward; that which Christ bestows on those who cleave to him, and seek salvation through him. In a voluntary humility and worshipping of angels; the apostle apparently speaks of that false humility which they had who pretended that God was too great to be approached except through created beings, such as angels. Thus their false teachers sought to draw them away from Christ, as if they needed ceremonies and mediators not prescribed in the gospel; whereas neither saints nor angels nor the Virgin Mary can help us as mediators; and Christ, if we trust in him, will do for us all that we need. Vainly puffed up; whatever appearances of humility or piety any may have who teach that you need other mediators besides Christ, or other observances besides those which he has appointed, they are ignorant or selfish, worldly and wicked, deceivers or deceived. If honest in what they say, they are blind leaders of the blind. Mt 15:14. Persons who occupy themselves in matters beyond the limits of the human mind, are wanting in humility as well as in wisdom and goodness. They are generally vain and light-minded, superficial and proud.>
<2:19 Head; Christ. The body; the church, which is composed of all who truly believe in him. The increase of God; the increase which God bestows, consisting in faith, love, joy, peace, and other graces of the Spirit. Ga 5:22,23.>
<2:20 If ye be dead with Christ; compare, for this idea of dying with Christ, Ro 6:3-11, and the notes on that passage. By dying with Christ, the Colossians had renounced sin and worldly confidence in the rudiments of the world; that is, the Mosaic ceremonies; see note to verse Col 2:8. Why then, as though living in the world, and not dead to it with Christ, would they be subject to ordinances; have the vain worldly ordinances which they professed to have renounced imposed upon them? Why seek justification by Jewish ordinances, which forbid certain meats or drinks, and make vain distinctions of days. Verse Col 2:16.>
<2:21 Touch not--handle not; samples of these worldly ordinances pertaining to the Jewish distinction of meats.>
<2:22 Which all are to perish with the using; a parenthetical remark thrown in by the apostle to show that these meats can bring no real defilement to the soul; for they all perish with the using, and pass away without touching the true inner man. Compare the exactly similar argument of our Saviour, Mr 7:14-23; which is the best commentary on the present passage. After the commandments--of men; to be connected immediately with the words, "why--are ye subject to ordinances?" verse Col 2:20.>
<2:23 A show of wisdom; an empty show without the reality. He then names three things in which this vain show is made. Will-worship; of man's invention, not required of God. Humility; a vain show of it. See note to verse Col 2:18. Neglecting of the body; unsparing treatment of it by austerities of man's invention. Not in any honor; meaning, according to some, while they refuse to bestow any honor on the body, but vilify it by their false severity towards it. But we may better understand the words as referring to all the preceding part of the verse, and describing the things named which have a show of wisdom as having in them no true honor towards God, but being, on the contrary, utterly worthless. To the satisfying of the flesh; referring to all the preceding things, as having for their end not true holiness, but only the satisfying of the fleshly mind. Every thing which draws men away from Christ as the only foundation of human hope, or leads them to seek salvation in any way except through faith in him, tends to rob them of blessings which, by continued active faith and obedience, they would obtain.>
<3:1 Risen; after the example of Christ and in union with him, from your death in sin to a new divine life. See chap Col 2:12; Ro 6:3-11; Eph 2:1-6.>
<3:3 Dead; dead to your former life of sin. Your life; your new life to which you have risen in Christ through faith. This life comprehends both the present spiritual life of the soul and the glorious resurrection life of which it is the earnest; for both together constitute that one everlasting life, the beginning of which all who believe in Christ has as a present possession. Joh 3:36; 5:24; Joh 6:40, 54; 1Jo 5:13. Is hid with Christ in God; hid along with Christ, whose members ye are, in the bosom of God in heaven. It is then, first, safe from all the assaults of wicked men and evil spirits, Joh 10:27-29; secondly, invisible to the eye of sense, so that not only does the world know us not, as it knew not Christ in his humiliation, but we do not ourselves know what we shall be. 1Joh 3:1,2.>
<3:4 Shall appear; shall be manifested in glory at his second coming. Shall ye also appear; be manifested, so that all shall see the glory which God has bestowed on you. Christ is the light, life, and joy of his people. Because he lives, they shall live; and when he comes, it will be to be glorified in his saints, and admired in all them that believe. 2Th 1:10.>
<3:5 Mortify therefore; since ye are dead with Christ, act consistently in putting to death your members which are upon the earth; your bodily members as the instruments of earthly lust; in other words, the sinful passions that exert their power in your bodily members; so that from being "servants to uncleanness and to iniquity," they may become "servants to righteousness." Rom 6:19.>
<3:9 The old man; those inclinations and habits which belong to man before conversion.>
<3:10 Put on; adopted new principles and entered on a new course of life, in consequence of having been renewed in the spirit of their minds by the Holy Ghost.>
<3:11 Greek nor Jew--Barbarian, Scythian, bond nor free; bondmen and freemen, and men of all descriptions who are born of the Spirit, have equal rights and are entitled to equal privileges in the church of Christ. They are all living members of his living body, and objects of his gracious favor. The standing of persons in the Christian church, and their rights and privileges as members, do not depend on their outward circumstances or condition in life, but on their union to Him on whom they are dependent, and to whom they are accountable.>
<3:14 Charity; love. The bond of perfectness; as binding together all the other graces into one whole, and thus making the Christian character complete. Love to God and to men, dependence on Christ, and a desire to obey his will, are the source and security of right actions, and are, in all conditions and relations, essential to perfection of human character and conduct.>
<3:15 The peace of God; that which he gives, and which resembles his own.>
<3:16 16-25. On these verses see the notes on the very similar passage, Eph 5:19 - 6:9.>
<3:21 Provoke not your children; by unkindly and improperly finding fault with them, being difficult to please, or failing to commend and encourage them when they do well. Lest they be discouraged; despair of being able to please you, and so become broken in spirit and reckless in regard to your wishes. A most important admonition to all parents who would retain their influence over their children.>
<3:22 In all things; unless they command you to do wrong. Not--as men-pleasers; not merely or principally for the purpose of pleasing men, with constrained or outward service only; but willingly, heartily, from regard to God, and for the purpose of pleasing him.>
<3:24 The reward of the inheritance; the gracious reward of the heavenly inheritance which he will give to his children.>
<3:25 No respect of persons; servants and masters will stand together before God, be judged by the same law, and be rewarded or punished according to their character and conduct. It is the will of God that there should be government, law, and order--that some should command, and others obey; but no degree of power or authority gives any a right to require of others what is wrong, or if they do require it, makes it the duty of others to obey. Each one is bound supremely to regard God, and whatever may be the consequences, to make it the great object in all things to please him.>
<4:1 That which is just; which rightfully, according to the law of God, belongs to them. Equal; that which fairness and honesty require. Ye also have a Master; to whom you justly owe service, and who requires you to render to your servants all which equitably and honestly belongs to them; and to manifest towards them the spirit which you ought to wish Christ to manifest towards you. Servants have rights as really as masters. Certain things, through the grace of God, belong to them; and masters are as sacredly bound to give them what equity and honesty require, as they would be if their servants were masters, or as they are to any of their fellow-men.>
<4:2 Continue; be earnest and steadfast. Watch in the same; be vigilant in the discharge of this duty; see that ye be not dull or slothful in it, and that ye allow yourselves in nothing that can hinder it.>
<4:3 A door of utterance; open the way and give opportunity to preach the gospel. Paul often asked for the prayers of Christians on earth, but never of the Virgin Mary or any of the saints in heaven. He knew better. All who have the Bible, who read and understand it, know better; and if they obey the Bible, they all do better than to ask or desire any intercession in heaven, except that of Christ; for his intercession is all which they need. Heb 7:25.>
<4:5 Walk in wisdom; conduct with discretion and propriety. Them that are without; without the church, men of the world. Redeeming the time; see note to Eph 5:16. Christians, in all their intercourse with men, especially with irreligious men, should be open, frank, honest, and sincere; kind, amiable, benevolent, serious, and cheerful; showing by example the supreme excellence and loveliness of true religion.>
<4:6 With grace; such as grace dictates; speak what is seasonable, pertinent, instructive, useful. Seasoned with salt; not insipid and profitless, but, like well-seasoned food, wholesome and promotive of the edification of all who hear. Know how--to answer; in order to give just views and make right impressions. The apostle probably has special reference to the questions of "those without," the answers to which would require much circumspection and heavenly wisdom. Much, very much depends upon the proper use of the tongue. It may be "a world of iniquity" or a fountain of life. Every person therefore, especially every Christian, should pray and strive for wisdom and grace rightly to use his tongue; knowing that by his words he will be justified or condemned; that if any man offend not in word, the same is a perfect man; and that words fitly spoken are like apples of gold in a net-work of silver. Pr 18:21; 25:11; Mt 12:37; Jas 3:2-18.>
<4:7 My state; as a prisoner at Rome.>
<4:8 Whom I have sent unto you; Tychicus was evidently the bearer of the present epistle, as well as of that to the Ephesians. Eph 6:21.>
<4:9 Onesimus; Phm 16,17,21.>
<4:10 Son; to the sister of Barnabas. Mark, or Marcus, was the nephew of Barnabas, and this might be one reason why Barnabas wished him to go with them, when Paul thought it not best. Ac 15:37-39. Ye received commandments; probably in connection with the contention between Paul and Barnabas just referred to.>
<4:11 Jesus; the same name as Joshua in Hebrew. Of the circumcision; Jews.>
<4:12 One of you; he belonged at Colosse, but was then with Paul at Rome. Chap Col 1:7. There is nothing which Christians so much desire for others, especially for their friends, and nothing for which they so earnestly pray, as that they may understand and do the will of God; for he that doeth the will of God, abideth for ever. 1Jo 2:17; 3:24; 1Co 7:19; Mt 13:50.>
<4:13 Laodicea--Hierapolis; both cities of Phrygia, in the vicinity of Colosse, the former on the west, the latter on the north-west.>
<4:14 Luke; the author of the gospel which bears his name. Demas; Phm 24; 2Ti 4:10-12.>
<4:16 The epistle from Laodicea; these words are most naturally understood of an epistle which Paul had sent to the church of the Laodiceans, which was to be obtained from Laodicea that it might be read at Colosse. See note to 1Co 5:9.>
<4:17 Archippus; Phm 2.>
\kniha{I Thessalonians}
\zkratka{1Thess}
<1:1 In God the Father and in the Lord Jesus Christ; a form of expression abundantly employed by the apostle Paul, and full of deep meaning. It contains the idea that the life of churches, as of individual believers, has its ground in union and communion with the Father and the Son through faith.>
<1:3 Work of faith; work which has faith for its source, and is therefore itself an exercise of faith. So the following expression, labor of love, is to be understood. Patience; the steadfast endurance of trials; an endurance sustained by hope in our Lord Jesus Christ; or, more literally, hope of our Lord Jesus Christ; that is, the hope of his second coming in glory to receive his people to himself, which is made so prominent in the two epistles to the Thessalonians: verse 1Th 1:10; 1Th 2:19; 3:13; 4:13-18; 5:23; 2Th 1:7-10; 1Th 2. 1-20. The piety of believers and their activity in doing good awaken fervent gratitude in the ministers of Christ, and lead them to render hearty and devout thanksgiving to God.>
<1:4 Knowing--your election; being chosen of God, of which their reception and treatment of the gospel were evidences.>
<1:5 In power, and in the Holy Ghost, and in much assurance; the last clause of this verse shows that the immediate reference of these words is to the apostle and his associates. Their preaching was in power, and in the Holy Ghost, and in full assurance of what they uttered. But this cannot be separated from the effect on their hearers. To them also their preaching was attended with the power of the Holy Ghost; it was embraced with full conviction of its truths, and led them to break off their sins and turn to the Lord. See verse 1Th 1:6.>
<1:6 Joy of the Holy Ghost; that which he imparts. A cordial reception of the gospel, and devotion of heart and life to the service of Christ, are sure evidences of being elected, and the pledge of being kept by the power of God through faith unto salvation.>
<1:7 Macedonia and Achaia; mentioned together as adjoining Roman provinces, comprising the whole of Greece. Achaia comprised the southern part of Greece, of which Corinth was the capital.>
<1:8 The word of the Lord; as manifested in your lives. Verses 1Th 1:9,10.>
<1:9 They themselves; the persons named in the preceding verse among whom the word of the Lord sounded out. What manner of entering in we had; what power attended our preaching.>
<1:10 Christ is the cause of the deliverance of believers from the wrath to come. This they feel and acknowledge, and to him they give the glory.>
<2:1 Christ is the cause of the deliverance of believers from the wrath to come. This they feel and acknowledge, and to him they give the glory.>
<2:2 At Philippi; Ac 16:19-24. Much contention; conflict with opposers, and the dangers and difficulties thence arising.>
<2:3 Not of deceit, nor of uncleanness; it did not spring from deceit, nor impure motives. Nor in guile; we had no crafty designs of our own to accomplish under the cloak of preaching the gospel.>
<2:4 Allowed of God; approved of God, as the original means.>
<2:5 Neither--used we--a cloak of covetousness; they had not used religion to conceal any covetous or selfish purpose.>
<2:6 Been burdensome; or, as the margin, used authority, exacted honor of you. Others understand the word of requiring support from the Thessalonians, verse 1Th 2:9. The great object of ministers in preaching the gospel, should be not the praises of men, but the approbation of God.>
<2:8 Our own souls; our lives: he was so desirous of their salvation, that he was willing not only to labor, but if need be to die to promote it. In their manner of preaching and in their intercourse with men, ministers should be kind, gentle, courteous, upright, and sincere--not merely or principally for the purpose of pleasing men, but of doing them good.>
<2:9 Travail; hard and wearisome toil. Laboring; the reference is to manual labor to obtain the means of support.>
<2:13 Effectually worketh; leading them to repent of their sins and believe in Christ. When the truths of the Bible are received as coming from God, they will be mightily efficacious, under the influence of his Spirit, to enlighten, sanctify, and save. Heb 4:12,13.>
<2:15 Contrary to all men; opposed to them, scorning all Gentiles, and hating even Jews who believe on Christ.>
<2:16 To speak; proclaim the gospel. To fill up their sins; unwilling to believe themselves, or to have the Gentiles believe, they filled up the cup of their iniquities and of God's vengeance. The wrath is come; the wrath of God. It was already at their door, ready to fall upon them and consume them to the uttermost. This epistle was written but a few years before the awful overthrow of Jerusalem and the Jewish nations by the Romans. Men are naturally so wicked, that, if left to themselves, they will not only reject the Saviour, but oppose the preaching of him to others. This is exceedingly offensive to God, and exposes all who are guilty of it to his wrath.>
<2:17 Being taken from you; bereaved of you, as the original word means. The apostle felt, in his separation from the Thessalonians, like a father bereaved of his children.>
<2:18 Hindered us; the agency which Satan employed was probably that of wicked men. Efforts for the salvation of souls are hated and opposed, not only by wicked men, but by Satan; and often he succeeds in hindering good men from doing what they might otherwise accomplish.>
<2:20 Sinners who are converted and saved in answer to the prayers and through the instrumentality of Christians, will be jewels in their crown of everlasting joy.>
<3:1 No longer forbear; so desirous was he of hearing from them, that he could not consistently wait any longer.>
<3:3 Be moved; be led to renounce his religion on account of the trials to which it exposed him. We are appointed; it is a part of God's gracious plan that his people in this world should suffer trials. Christians should see and acknowledge the hand of God in their trials as well as their mercies, and never be led by them to renounce their confidence in him, or their devotion to his service.>
<3:5 The tempter; Satan. Have tempted you; to deny Christ.>
<3:8 We live; our life is bound up in your spiritual welfare. To hear of this fills us with joy. Compare chap 1Th 2:19, 20. When Christians are steadfast and persevering in the faith and practice of the gospel, those ministers of Christ who have been instrumental in their conversion give the glory to God, and rejoice with exceeding great joy.>
<3:13 At the coming of our Lord Jesus Christ; when the result of God's work of sanctification in the hearts of believers shall be made manifest in its perfection. Increasing love to Christians on account of their attachment and likeness to Christ, and earnest desires to promote the highest good of men, are powerful means of perseverance in holiness and preparation for heaven.>
<4:1 Jesus Christ and his apostles were exceedingly desirous not merely that men should be converted and have a good hope of heaven, but that they should be eminently holy; should not merely be planted as trees of righteousness in the garden of the Lord, but bear much fruit. Joh 15:8.>
<4:4 Possess his vessel; treat his body as the work and property of God and the habitation of an immortal spirit.>
<4:8 Despiseth; rejecteth these instructions. Men treat God as they treat the truths of the Bible. Those who disbelieve and reject them, disbelieve and reject him; and those who love and obey them, love and obey him.>
<4:12 Walk honestly; have a deportment that is honorable and reputable. Without; without the church; those who do not profess to love Christ. Have lack of nothing; nothing which is needful for support, comfort, and usefulness. Diligence in lawful useful business is the duty of all men, unless disabled; it is the means by which God ordinarily supplies their wants and enables them to pay their debts, support their families, and be useful to their fellow-men.>
<4:13 Asleep; asleep in Jesus, who have died in union with Christ by faith, verse 1Th 4:14. Others; the unenlightened heathen, who have no hope of a resurrection and life of blessedness with Christ in heaven.>
<4:14 Will God bring with him; raise from the dead in glorious, immortal bodies, so that they as well as those that remain alive at Christ's coming, shall appear with him in glory. 1Co 15:51-54.>
<4:15 Which are alive; when the Lord shall come to judgment. Shall not prevent; not go before, or rise to meet the Lord before those do who are dead.>
<4:16 Rise first; before the living shall be changed. But after the dead in Christ are raised, the living shall be changed, and both ascend together to be for ever with the Lord.>
<4:18 The prospect of meeting our pious friends at the day of judgment, and with them, perfect in body and soul, ascending with Christ and all his redeemed, to be for ever like him, soothes the anguish of parting with them, and fills the soul with joys unspeakable and full of glory.>
<5:1 The times and the seasons; that pertain to the Lord's coming, of which he had just been speaking. Ye have no need that I write unto you; for the reason stated in the next verse.>
<5:2 Perfectly; more literally, accurately, having been carefully instructed on that point by myself. The day of the Lord; of his second coming in glory just spoken of. As a thief in the night; suddenly and at an unexpected time. It is God's will that men should live in constant preparation for it. All over-curious computations for the purpose of fixing its exact date are vain and profitless. Mt 24:36. What the apostle here says of Christ's second coming is eminently true also of his particular coming to each one at death, which is, in truth, to him the end of the world. Heb 9:27.>
<5:3 They; the wicked, who are living in careless security. Compare Mt 24:37-39.>
<5:4 Ye; Christians. Not in darkness; the darkness of ignorance and sin. They had been enlightened by the reception of the truth. Should overtake you; surprise you in an unprepared state, as a thief does.>
<5:6 Sleep; live in stupidity and carnal security, unmindful of and uninfluenced by the great truths of the gospel. As we know that God will call us to judgment, but cannot know when, we should be always ready, and so live that whenever called we may give our account with joy, and not with grief.>
<5:7 They that sleep, sleep in the night--drunken in the night; as the natural night is the time when men indulge in natural sleep and drunkenness, so they who live in the spiritual night of ignorance and sin may be expected to give themselves up to spiritual sleep and dissoluteness. But not so we, who are of the day, as the apostle proceeds to show.>
<5:8 Breastplate--helmet; see Eph 6:13-18, and notes.>
<5:9 To obtain salvation; this was evident from their having believed in Christ. Those who believe in Christ and obey his commands show that they are elected to eternal life and are heirs of heaven.>
<5:10 Wake or sleep; be found among the living or the dead.>
<5:12 Know them which labor among you; as your ministers, with affectionate love and obedience to their instructions.>
<5:13 Their work's sake; as preachers of the gospel and promoters of your spiritual good. Ministers who take the oversight of churches are not to be esteemed merely or principally on account of their office, but must do works which are worthy of esteem, or they give no evidence of being ministers of Christ, and have no claim, as such, to the respect and confidence of his people.>
<5:14 Unruly; those who live in violation of the rules of God's word. Feeble-minded; those who are easily disheartened and discouraged. The weak; the weak in faith. Compare Ro 14:1; 15:1.>
<5:16 Rejoice; in God and his salvation. A Christian is never placed in any situation in which he has not abundant reason for exceeding great joy.>
<5:18 A dependent, grateful, and benevolent spirit, manifested in habitual thanksgiving and in supplication for ourselves and our fellow-men, is peculiarly pleasing to God.>
<5:19 Quench not the Spirit; by refusing to follow his gracious leadings, doing what you know to be contrary to his will, or neglecting to perform the duty to which he prompts you. Though the Holy Spirit is almighty, he may be resisted. His influences may be quenched, their efficacy counteracted, and by the commission of sin and the neglect of duty men may deprive themselves of his saving power.>
<5:20 Prophesyings; see note to 1Co 12:28.>
<5:21 Prove all things; by comparing them with the Bible; and if they do not agree with that, reject them; if they do, receive and believe them. Hearers of the gospel are bound to inquire, examine, and judge whether what they hear is or is not according to the Bible; and any man or body of men that denies them this right, or hinders them from exercising it, is violating the revealed will of God.>
<5:23 The coming of our Lord; his second coming in glory. See note to chap 1Th 3:13.>
<5:24 Who also will do it; God, who had begun their sanctification, would increase it till it should be perfected. God will certainly and wholly sanctify and save all who truly believe in Christ and continue to serve him.>
<5:25 The fact that Paul felt his need of the prayers of living Christians, and often asked for them, but never asked for the prayers of the dead or of the Virgin Mary, shows conclusively that it is not right to pray to them. No inspired man ever did it, and none who rightly understand and obey the Scriptures ever will do it.>
<5:27 Unto all; for it was not designed for individuals merely, but for the whole church.>
\kniha{II Thessalonians}
\zkratka{2Thess}
<1:3 The holy, consistent, and useful lives of Christians are a striking manifestation of the power of divine grace, in which the friends of Christ greatly rejoice, and for which they render hearty and devout thanksgiving to God.>
<1:5 A manifest token; your "patience and faith," namely, "in all your persecutions and tribulations," endured for Christ's sake, make it manifest that a righteous judgment is coming when God will graciously reward you and punish your foes. That ye may be counted worthy; these words express the end and issue to which the righteous judgment of God looks, in the case of those who have steadfastly suffered for Christ's sake.>
<1:7 When the Lord Jesus shall be revealed; at the day of judgment.>
<1:8 The prosperity, injustice, and cruelty of the wicked, and the adversity, meekness, patience, and submission of the righteous, show that men are not in this world treated according to their character; and that there is a coming judgment; when the righteous will be rewarded and the wicked punished according to their works.>
<1:9 Everlasting destruction from the presence of the Lord; their destruction consists in everlasting banishment from God's presence and the glory of his power, and the everlasting endurance of God's wrath with the devil and his angels. Mt 25:41,46. The endless destruction of the perseveringly wicked is just.>
<1:10 Admired in all them that believe; that is, in the persons of all them that believe; for they will be transformed into Christ's image in soul and body, and will reflect his glory. Because our testimony among you was believed; a parenthetical sentence added to show that the Thessalonians also are included in "all them that believe." In that day; in the day of Christ's second coming. These words are to be connected immediately with those before the parenthesis.>
<1:11 This calling; their calling to eternal life. Fulfil; in your souls. All the good pleasure of his goodness; as manifested in carrying forward to completion the work of sanctification in your souls. The work of faith; that is, fulfil the work of faith in your souls, by making perfect your faith with its fruits.>
<1:12 Be glorified in you, and ye in him; by their bearing his image, promoting his glory, and being admitted to his rest.>
<2:1 By the coming of our Lord; rather, in respect to the coming of our Lord, of which he had just spoken.>
<2:2 By spirit; by any pretended revelation from the Spirit of God. As from us; professing to come from us. The apostles did not teach that the day of judgment or the end of the world was near; but that the day of death, when their hearers would be called into eternity, was near, and that they should be always ready; for in such an hour as they thought not, the Son of man would in that sense come. Lu 12:35-40.>
<2:3 A falling away; a great apostasy from the faith and practice of the gospel. That man of sin; the words "man of sin" are to be understood not of any single person, but of a wicked system presided over and directed by a succession of wicked men. The words of the apostle clearly describe that great system of spiritual tyranny and wickedness of which the papacy has ever been the central power. Be revealed; show himself, and be made manifest in his true character. The son of perdition; the very words applied by our Saviour to the apostate Judas. They describe the man of sin as notoriously wicked and doomed to final destruction. See the histories of popes John II. and John VIII.; of Marcellinus; of Honorius, of whom the council of Constantinople say, "We have caused him to be accursed;" of Eugenius, whom the council of Basle call "a simonist, a perjurer, a willful heretic;" of John XIII.; of Sextus IV.; of Alexander VI., who, as a papal historian says, was "one of the greatest and most horrible monsters in nature;" and of many others. See Guicciardini, Ciaconius, and other papal historians.>
<2:4 Who opposeth and exalteth himself above all that is called God; opposeth the gospel of Christ as revealed in the Bible, and persecutes those who embrace it. See the history of Wickliffe, Huss, and Jerome of Prague, of the Waldenses, of the Inquisition, of Mary queen of England, and of St. Bartholomew's day in France. Who invades the prerogatives of God, pretending to be the head of the church, to forgive sins, and to do what God himself cannot do--grant indulgences to commit sins. See the history of pope Leo X., of the archbishop of Mayence, of Tetzel, and of papal indulgences. Who practically annuls the laws of God, and substitutes for them the commandments of men; as when the council of Trent decreed, "Whosoever shall say that it is not more blessed to remain in virginity or celibacy than to be joined in marriage, let him be accursed;" and when the pope says, "Be careful to preserve the people not only from the reading of papers, but from the reading of the Bible"--"shun with horror the reading of such deadly poison;" thus exalting himself above the word of God. See the Catechism of Dr. James Butler, Dens' Moral Theology, and other papal works. As God; assuming the right to control the conscience, receiving the titles, and claiming the honors which belong only to God--called by his deluded followers, "Our Lord God the Pope," "Another God upon earth," "King of kings and Lord of lords." See Newton on the Prophecies.>
<2:5 These things; that there would be a great apostasy before the coming of Christ to judgment. Of course that event was not near. Verse 2Th 2:2. The rise and progress of the papacy and all its abominations, in exact fulfilment of the declarations of Paul, uttered and recorded hundreds of years before, show that he was divinely inspired to make known these things; and that the epistles as well as the gospels, the New Testament as well as the Old, are the sure and infallible word of God.>
<2:6 Withholdeth; holdeth back or hindereth the development of the man of sin, and his claiming the high powers and prerogatives which he afterwards assumed.>
<2:7 The mystery of iniquity; that ambitious, proud, covetous, and domineering spirit, which the popes afterwards exhibited in assuming to be lords temporal and spiritual. He who now letteth; the Roman government, which, while it lasted, prevented the rise of the papal civil government. Until he be taken out of the way; the Roman government would continue, as long as it should last, to prevent the establishment at Rome of the papal government.>
<2:8 Then; after the downfall of the Roman empire. That Wicked; the wicked one, the man of sin and son of perdition spoken of in verse 2Th 2:3. Be revealed; manifest himself; claim to be universal bishop and lord of the kings of the earth. Shall consume with the spirit of his mouth; compare Isa 11:4, "He shall smite the earth with the rod of his mouth, and with the breath of his lips shall he slay the wicked;" Isa 49:2, "He hath made my mouth like a sharp sword;" Re 1:16, "Out of his mouth went a sharp two-edged sword." See also Re 19:21. The reference apparently is as well to the judgments that proceed from the mouth of Christ, as to the doctrines of his gospel and the power of his Spirit. The full accomplishment of this promise can alone give us its full interpretation. The same divine Spirit who by the mouth of Paul foretold the rise and progress of popery, foretold also its destruction; and the accomplishment of one part of the prophecy is conclusive evidence that, in due time, will be witnessed the perfect accomplishment of the other. Re 19:20.>
<2:9 After the working of Satan; by his aid, and like him deceitful, crafty, and wicked. Lying wonders; pretending to work miracles when they do not, and the pretence is a lie designed to delude the ignorant. 9-12. Satan has had much to do in the rise and progress of popery, and now has much to do in sustaining it, by wars and bloodshed, persecution and cruelty, deceit and falsehood, and by those pretended miracles and lying wonders, by which multitudes, who receive not the truth in the love of it, are deceived to their destruction.>
<2:10 Deceivableness of unrighteousness; all deceitful arts and practices to promote their selfish and unrighteous schemes. In them that perish; added to show over whom the wicked and lying arts of the man of sin have power. Because they received not; still further added to show why the men just spoken of perish. It is the truth; did not give the truth a loving reception, because they hated it and chose error in its stead.>
<2:11 For this cause; because they hate and reject the truth. Shall send them strong delusion; permit it in his providence to come upon them as a righteous judgment for their hatred and rejection of the truth. Believe a lie; those lying wonders and false immoral doctrines propagated by the man of sin, to the deceiving and ruining of multitudes who take pleasure in unrighteousness.>
<2:13 Chosen you to salvation through sanctification--and belief of the truth; God not only chooses his people to salvation, but he chooses the way also--"sanctification of the Spirit and belief of the truth"--a way in which they "work out" their "own salvation with fear and trembling," while God "worketh in" them "both to will and to do of his good pleasure.">
<2:14 Whereunto; to salvation in the way just mentioned.>
<2:15 The traditions; instructions which the apostle had given them in preaching and by writing. Traditions, in the sense in which the word is used by the apostles, are the doctrines and duties which they preached, and which are recorded in the Bible. These are the traditions and the only traditions which they exhorted their hearers to hold. Hence the reason why all people should have the Bible and study it, that they may understand and follow the traditions which apostles and other inspired men taught. Hence, too, the reason why popes are afraid to have the Bible freely circulated. It points out their character, and describes the wickedness of their doings. It denounces the system of which they are the head, as the "man of sin," "the son of perdition," "the wicked one," "the mystery of iniquity;" "whose coming is after the working of Satan, with signs and lying wonders, in all deceivableness of unrighteousness." If the people are permitted to read it and judge for themselves of its meaning, and are disposed to follow it, they will see that popes and their associates are antichrist, and will treat them accordingly. No wonder they issue bulls against the Bible being circulated in Italy, and that the masses of their people have not been taught to read it. No wonder their priests, even in the United States, often take the Bible away from their people, and sometimes burn it. It denounces their system as false and wicked, and describes those who are deluded by it as believing a lie. Let the Bible circulate, and let all read, believe, and obey it as the word of God, and errors of every description will vanish, will be consumed with the breath of his mouth and destroyed with the brightness of his coming.>
<3:1 Have free course; not be obstructed and hindered in its progress. Be glorified; by manifesting its divine power in the salvation of all who believe.>
<3:2 Have not faith; faith in the gospel. They reject it, and manifest unreasonable and wicked opposition towards those who preach it or receive it. Prayer is as really instrumental in the success of the gospel as preaching; and is one of the most powerful means of being delivered from, or of overcoming the opposition of the wicked.>
<3:3 The Lord is faithful; to all his promises, and may be safely trusted.>
<3:6 Disorderly; not according to the rule of God's word, as the apostle had taught them. Tradition; in the sense of precepts. See note to chap 2Th 2:15. Not to associate with members of the church who by their sins disgrace their profession, is one of the divinely appointed means for bringing them to repentance, and thus preventing their ruin. Verse 2Th 3:14.>
<3:9 Power; authority and right, according to the gospel, to receive support while preaching it. An ensample; of diligence in business and readiness to labor and suffer for the good of others. True benevolence will lead persons sometimes to omit enforcing their just rights, in order to do greater good to their fellow-men. 1Co 1:9-18.>
<3:10 Neither should he eat; he should not be supported from the earnings of others. Idleness is a great sin, and the supporting of idle persons by private or public charity, or in any way which encourages them in idleness, should be conscientiously avoided.>
<3:11 Busybodies; neglecting their own business and meddling with that which does not belong to them.>
<3:12 All who can, should be habitually diligent in useful and appropriate business. If necessary, it should be done for their own support; and if not necessary for this, it should be done for the purpose of assisting others.>
<3:17 Which is the token; mark by which the epistle may be known to be from me. The preceding part of the epistle had been written, as usual, by an amanuensis. In every epistle; these words need not be taken with any limitation; where he does not expressly mention the fact, it is still probable that the closing benediction was from his own hand.>
<3:18 Those who are blessed with the grace of our Lord Jesus Christ will be furnished, in obeying him, with all needed good for time and eternity.>
\kniha{I Timothy}
\zkratka{1Tim}
<1:2 My own son; spiritually, having been converted by his instrumentality. The connection between faithful ministers of the gospel and those who are led by them to Christ is most intimate and endearing. It may well be represented by that between parents and their children, and is a source of rich and lasting enjoyment.>
<1:3 Other doctrine; false doctrine contrary to what Paul had taught.>
<1:4 Fables; called "profane and old wives' fables," chap 1Ti 4:7; and "Jewish fables." Ti 1:14. The apostle has reference to absurd legends and stories such as abound in the writings of the later Jews. Endless genealogies; the exact nature of these is unknown. According to some, the reference is to the Jewish records of the descent from Abraham, by which their pride was nourished, and their confidence was withdrawn from Christ to fleshly relations. Others, with more probability, suppose that the apostle has in view fables respecting the generation of angelic orders of beings. Though the system of Gnosticism, which is filled with such "endless genealogies," was of later origin, they suppose that its germs may have existed in the apostle's day, and have mixed themselves with Jewish fables. Minister questions, rather than godly edifying; lead to nothing but empty questions of speculation and dispute.>
<1:5 The end of the commandment; its scope or aim; its substance, which all its particular precepts have in view. Charity; love towards God and man. The great things required of us are love to God and to men, confidence in him, and conscientious devotion to his will.>
<1:6 Swerved; turned aside. Vain jangling; empty and contentious talk.>
<1:8 Lawfully; according to its proper design. The law of God is good as a rule of duty for all men; to restrain, by fear of its penalties, those who transgress it, and to point out the punishment which they deserve, and without repentance will suffer.>
<1:9 The law is not made for a righteous man; the apostle is combating the error of those who trusted in the law as the instrument of their justification and salvation. This end it could never accomplish. To fallen men it "worketh wrath," Ro 4:15, and its end is death, Ro 7:10. "Wherefore then serveth the law? It was added because of transgressions"--to restrain the transgressions of lawless men, Ga 3:19; consequently not for the righteous, who are a law unto themselves, but for the lawless and disobedient.>
<1:10 Sound doctrine; the doctrine of the gospel, called, "The doctrine which is according to godliness," chap 1Ti 6:3, as having for its scope true godliness, and opposing itself to every form of wickedness.>
<1:12 Counted me faithful; counted him a proper person to be put into the ministry, and enabled him to discharge its duties.>
<1:13 Injurious; one who maliciously and tyrannically oppressed and put to death the people of God. Ignorantly; in ignorance of the true character of Christ and his disciples.>
<1:14 The grace of our Lord; in leading him to repent and believe on Christ. With faith and love; he mentions these as the never failing attendants and fruits of God's grace. Which is in Christ Jesus; he adds these words to show that faith and love have their ground only in the union of the soul with Christ.>
<1:16 A pattern; to show that the chief of sinners who believe in Christ may be pardoned, sanctified, and saved. Upon all who have ever repented and believed, God has bestowed free pardon and the blessings of heavenly grace, that even the chief of sinners may be encouraged to repent of their sins and embrace the Saviour as he is offered in the gospel.>
<1:18 This charge; the directions in this epistle. Prophecies; which some of the New Testament prophets had uttered concerning Timothy, before he was put into the ministry. By them; by these prophecies, under their auspices, as it were; in other words, having them in view, and being animated by them. War a good warfare; be faithful to the Captain of salvation, contending against sin and striving to save sinners.>
<1:19 Holding faith; holding fast the faith of the gospel, which has for its natural companion a good conscience. Which; which good conscience. Having put away; literally, having thrust away. They have willfully cast away a good conscience, and, as a natural consequence, concerning faith have made shipwreck; for he who allows his conscience to be defiled by sinful practices is prepared to reject the faith of the gospel, which opposes itself to every form of ungodliness.>
<1:20 Delivered unto Satan; cast out of the visible kingdom of Christ, perhaps also with the additional idea of an infliction of some bodily malady. See note to 1Co 5:5. That they may learn; literally, may be disciplined; may be taught, by the evils they suffer, not to continue in their erroneous and wicked courses.>
<2:1 First of all; in importance.>
<2:2 For all that are in authority; men in public office and stations of influence. That we may lead a quiet and peaceable life; this is the result of God's grace given to them in answer to the prayers of his people, and enabling them to administer their office with fidelity and uprightness. Prayer for rulers is one of the most powerful means of obtaining a good government, and securing for all liberty to search the Scriptures and judge of their meaning, to worship God according to the dictates of conscience, and to discharge their various duties towards God and men.>
<2:3 For this; praying for all men, especially for rulers.>
<2:4 Who will have all men to be saved; by becoming acquainted with the gospel, and by believing and obeying it. He therefore wills that it should be preached to every creature, and that his people should pray that all may embrace it.>
<2:5 5,6. One God, and one mediator--a ransom for all; these words contain the ground of the preceding exhortation to pray for all men, and declaration that God desires the salvation of all men. All have one God for their Father, and one Mediator who gave himself a ransom for all. To be testified in due time; literally, as the margin, the testimony in, or for, its own times. The apostle means that this high doctrine which he has just stated is one of the mysteries not hitherto clearly revealed, but reserved by God to be made known, through the testimony of his Spirit, in its appointed time, which is the present.>
<2:6 As there is but one Mediator between God and men, and as he has given himself a ransom for all, wills that all should hear of him, believe on him, and be saved, and has made it our duty to use all means in our power to accomplish this, it is evident that he has made provision for and desires their salvation; and that if any to whom Christ is made known are not saved, it is their own fault.>
<2:8 Lifting up holy hands; in prayer. Without wrath; unholy anger towards men, which always hinders prayer. Mr 11:25. Doubting; the wavering of faith, which also hinders prayer. Jas 1:6,7. But many prefer to render, disputing, as the same word is translated in Php 2:14.>
<2:9 In like manner; with the same holy temper. Shamefacedness; modest appearance. Sobriety; decorum. Broidered; or, plaited. Costly array; expensive ornaments or dress, which is the mark of pride and luxury, and corrupting in its influence on them and on others. Compare 1Pe 3:3.>
<2:10 The highest beauty of women, and the richest ornaments with which they can adorn themselves, are true piety and sincere active beneficence. The gospel inculcates universal propriety, and a character formed after its model is one of consummate excellence, usefulness, and enjoyment.>
<2:12 Nor to usurp authority; as she would should she undertake publicly to teach. It is the revealed will of God that public religious teachers should be men, not women. He has allotted to them different spheres of action, and the perfection of each consists not in aspiring or submitting to occupy the place of the other, but in performing their own appropriate duties.>
<2:13 Adam was first formed; an indication that he is the head of the woman, and that the office of teaching and governing belongs to him. 1Co 11:8,9. The apostle has reference to the public assemblies of believers. Compare 1Co 14:34, "Let your women keep silence in the churches.">
<2:14 Was not deceived; by the serpent in the first transgression. The serpent first assailed the woman, as being most open to his arts, and having deceived her, he made use of her to persuade her husband. Compare the words of the woman, "The serpent beguiled me, and I did eat;" and the words of God to the man, "Because thou hast hearkened to the voice of thy wife, and hast eaten," Ge 3:13,17. The headship was given to the man, not to the woman.>
<2:15 She shall be saved in childbearing; the apostle says this with reference to the original curse pronounced upon the woman, "In sorrow shalt thou bring forth children." Ge 3:16. Through faith in Christ, who is emphatically "the seed of the woman," God will not only sustain her in the perils of childbearing, but make them conducive to her spiritual and eternal salvation. If they continue; words added to show who alone have an interest in the promise just given.>
<3:1 She shall be saved in childbearing; the apostle says this with reference to the original curse pronounced upon the woman, "In sorrow shalt thou bring forth children." Ge 3:16. Through faith in Christ, who is emphatically "the seed of the woman," God will not only sustain her in the perils of childbearing, but make them conducive to her spiritual and eternal salvation. If they continue; words added to show who alone have an interest in the promise just given.>
<3:2 Blameless; of irreproachable character. Vigilant; watchful and circumspect in his deportment and office. Sober; sober-minded, properly regulating his appetites and passions. Of good behavior; orderly and decorous in all his deportment.>
<3:3 Greedy of filthy lucre; that is, of gain obtained by base arts and employments.>
<3:4 With all gravity; with reverent and decorous deportment in all things. The words refer to the deportment of his children. It is proper that a bishop as a minister of the gospel should be married: and if married, he should with discretion and fidelity discharge the various duties of the head of a family; especially should he set an example of good family government, and train up his children in the nurture and admonition of the Lord.>
<3:6 A novice; one recently converted, who has but little knowledge of Christian doctrines and duties, and has not yet become fully established in the faith. The condemnation of the devil; that which befell him for his pride.>
<3:7 Have a good report; be of unblemished reputation in view not only of Christians, but of others. The snare of the devil; which he sets by tempting men so to act as to injure themselves and the cause of religion. As the work of a bishop is sacred and momentous, it should be undertaken only by those who, by a course of good conduct, have formed the character and secured the reputation, in the church and in the world, of being good men; free from the imputation of vice, meanness, sensual indulgence, or love of money; men who have knowledge, and are able and willing to teach; who are patterns of what is upright and honorable, lovely, and of good report.>
<3:8 Grave; dignified and decorous in their deportment. Double-tongued; deceitful, saying one thing and doing or meaning another.>
<3:9 The mystery of the faith; the doctrines of the gospel, so called because they are addressed to our faith, and are a revelation of truths undiscoverable by the light of human reason. In a pure conscience; a conscience not defiled by indulgence in sinful practices. The deacons must be sound in daily life, as well as in faith.>
<3:10 Proved; tried in regard to their previous Christian life, and their qualifications for the office.>
<3:11 Must their wives be grave; in selecting deacons, regard must be had to the character of their wives, for they will greatly help or hinder their husbands in their work. But many prefer to render, "must the women be grave;" that is, those selected to be deaconesses.>
<3:13 A good degree; a good standing in the church of Christ, enlarged influence and means of usefulness. Great boldness in the faith; in professing and maintaining the faith.>
<3:14 The office work of both bishops and deacons is such, that their comfort and success in it depend much on the character and conduct of their wives. These should be pious, prudent, and discrect, especially in the use of the tongue; and say nothing which is suited to do evil to themselves or others. They should also be one in judgment and effort with their husbands in governing their children, and examples of wisdom and energy, patience and kindness, in all their concerns.>
<3:15 In the house of God; in conducting the affairs of the church. Pillar and ground of the truth; the church is the means of sustaining, extending, and perpetuating the saving knowledge of divine truth among men. The cordial reception of the great truths of the gospel, especially those which relate to the character, work, and glory of Christ, is the means of true godliness; and the church is God's institution to maintain those truths, perpetuate a knowledge of them, and extend them through the world.>
<3:16 The mystery of godliness; that great mystery of the manifestation of God in human nature, of which the apostle proceeds to speak. It is a mystery, as having been hitherto hidden in the secret counsels of God; and the mystery of godliness, as having godliness for its end in all that believe. God was manifest in the flesh; compare Joh 1:14, "And the Word was made flesh"--the same Word which was in the beginning with God, and was God, verse 1Ti 3:1. Justified in the Spirit; shown to be just in his claims as the Messiah, by the Holy Ghost, given to him without measure, and working in and by him with divine power. Seen of angels; who ministered to him and worshipped him, even in his deepest humiliation. Heb 1:6. Preached unto the Gentiles; to all nations as the almighty and only Saviour. Believed on; by multitudes of Jews and Gentiles. Received up into glory; where he ever lives to make intercession for all who come unto God by him. Heb 7:25.>
<4:1 In his farewell address to the elders of Ephesus, Paul forewarns them that after his departure grievous wolves shall enter in among them, not sparing the flock, and that also of their own selves shall men arise speaking perverse things, to draw away disciples after them. Ac 20:29,30. Whatever view may be held respecting the date of the present epistle, it is plain that the apostle here refers to the same corrupt leaders and teachers. But he connects their appearance with that great apostasy foretold in his second epistle to the Thessalonians, chap 2Th 2:3-12. Of this these "grevious wolves" were the forerunners. In and through them that "mystery of iniquity" was already working, the full development of which came when he who then hindered was taken out of the way. 2Th 2:7. From the faith; from the doctrine of faith in Christ as the only foundation of hope. Doctrines of devils; such as Satan tempts men to embrace, as the worshipping of images, praying to the Virgin Mary or departed saints, and relying on external connections and observances for salvation. The errors of popery are a fulfilment of Scripture, having been expressly and clearly foretold by the Holy Spirit. They are therefore conclusive evidence that the Bible is given by inspiration of God.>
<4:2 Speaking lies in hypocrisy; or, in the hypocrisy of those who speak lies, inculcating on the people as true what they know to be false; such as the power of the priests to forgive sins, the pretended working of miracles by the relics of saints, the liquifying of the blood of St. Januarius, and the weeping of the statue of the Virgin Mary. Seared with a hot iron; branded with the marks of their wicked deeds. Their sins are, as it were, burnt in upon their consciences. They are hardened transgressors, who carry about in their own souls the consciousness of their hypocrisy and wickedness, and are indifferent to it. To pretend to be what one is not, and by hypocrisy and lying obtain money, power, and influence awfully blinds the mind, sears the conscience, and hardens the heart.>
<4:3 Forbidding to marry; as popery forbids the clergy, and induces monks and nuns to take vows of celibacy, declaring, as did the council of Trent, "Whosoever shall say that the married state is to be preferred to a state of virginity or celibacy, let him be accursed." Commanding to abstain from meats; as popery does during Lent, on fast-days, and days of abstinence. See Butler's Catechism and Dens' Theology. To be received; for food, and eaten by believers who know the will of God, during Lent as well as at other times.>
<4:4 Every creature; which God hath made for food is good for food, and not to be abstained from, but to be eaten with gratitude to God the giver.>
<4:5 Sanctified; made holy to him who partakes of it, so that the use of it cannot defile him. By the word of God; ordaining it for man's use. Prayer; which procures from God's blessing upon it.>
<4:6 These things; the truths of which he had spoken. To point out the errors of popery and the predictions of the Bible concerning it, and to warn the people against its seductive, demoralizing, and ruinous influence, is the duty of all good ministers of the gospel.>
<4:7 Profane and old wives' fables; see note to chap 1Ti 1:4. These fables are in their spirit and influence profane, and in their character absurd and anile.>
<4:8 Bodily exercise; the discipline of the body by fastings and other austerities, considered as a religious exercise. Godliness is profitable for both worlds; and the man who makes it his great object to do his whole duty, takes the course which is best suited to promote his own highest good.>
<4:10 Saviour of all men; as preserving all men, and having opened for them a way of salvation, and commanded that it be made known to them, and that they should be entreated to embrace it. 2Co 5:20. Especially of those that believe; for to them alone does the perfect and everlasting salvation which he has provided for and offered to all men become actual.>
<4:11 Command and teach; teach all men these truths, and command them, from God, to believe and obey them. As the declarations of God are all true, and his commands good, those who have confidence in him will labor hard, and if need be suffer much to induce others to believe and obey him; knowing that this is the will of God, and that all who comply with it will be saved.>
<4:12 Thy youth; compare the admonition in the second epistle, "Flee also youthful lusts," chap 2Ti 2:22.>
<4:13 Reading; of the holy Scriptures. The immediate connection of this word with "exhortation" and "doctrine" seems to show that the public reading of the Scriptures in religious assemblages, after the manner of the Jewish synagogues, is meant. This was of course to be accompanied with the exposition of its meaning.>
<4:14 The gift that is in thee; the spiritual gift imparted to Timothy by the Holy Ghost. By prophecy; in accordance with preceding prophecies, which pointed him out as a man to be inducted into the sacred office. Compare chap 1Ti 1:18. With the laying on of the hands; the special gift of the Spirit was given to Timothy, as to others, in connection with the laying on of hands. Ac 8:17; 19:6. Presbytery; an assembly of elders or ministers of the gospel.>
<4:16 Unto thyself; to his own character and conduct. Doctrine; the truths which he taught. In them; in the belief, teaching, and practice of those truths. Them that hear; thy hearers, upon the condition of their obeying the truths taught them. The ability of ministers to do good may, by their own efforts under the blessing of God, be much increased; and it is their duty so to increase it that their progress shall be manifest, and so to devote themselves to their work that they may expect, through grace, to save both themselves and their hearers.>
<5:1 Elder; an aged Christian man.>
<5:2 Ministers of the gospel should pay special attention to the aged, and treat them with special respect and kindness. They should also, in their needful and proper interaction with females of their congregations, possess and manifest a delicate sense of propriety, and the utmost purity of feeling, conversation, and conduct.>
<5:3 Honor widows; the honor here referred to, as the context shows, was that of a reception to the list of those who were to have public maintenance from the congregation, and were employed in useful Christian labors. Widows indeed; worthy of the name of widows.>
<5:4 Children or nephews; the word translated nephews means descendants, specially grandchildren. If a destitute widow had children or grandchildren who could support her, they were bound to do so, and not let her be a charge on the church. A disposition in children to be kind and attentive to their parents and grandparents, and if need be to support them and keep them from being a public charge, is required by the gospel, and is peculiarly pleasing to God.>
<5:5 Desolate; destitute, and having no relatives to support her. The words "widow indeed, and desolate" describe both her worldly condition and her character as a Christian. If she was not only destitute but truly pious, was more than sixty years old, had been faithful to her husband and her children, hospitable when she had the means, attentive to the wants of poor Christians, and accustomed to relieve the distressed, she might be received into the number who were to be employed and supported by the church. Verse 1Ti 5:9.>
<5:6 In pleasure; in wantonness and luxurious self-indulgence. Is dead while she liveth; dead to Christ and his service, and dead in sin, while she lives only for this world's pleasures.>
<5:7 These things; what he has just said about widows and their relatives. Give in charge; command or enjoin.>
<5:8 His own; his own relatives who are dependent on him, as a destitute mother or grandmother, and especially his wife, children, and such as belong to his own family. Denied the faith; practically, by disobeying its known requirements. Worse than an infidel; in this respect, violating what unbelievers and even heathen inculcate as a duty, and often practise. Professors of religion who are able and yet unwilling to provide comfortably for their own families, for their parents, grandparents, and other relatives who are necessarily dependent on them, act in opposition not only to the revealed will of God, but to the dictates of natural religion, and bring disgrace on the Christian cause.>
<5:10 Aged and indigent females, who have been distinguished for devotion to Christ and usefulness to men, and who have no relatives to support them, should be supported by the church of which they are members; and as far as may be rendered comfortable and useful.>
<5:11 The younger widows; who make application to be employed and supported by the church. Wanton against Christ; being unwilling, through their wantonness and love of pleasure, to submit to the rules which he had enjoined.>
<5:12 Having damnation; being condemned for their inconstancy, in deserting the trust committed to them, and with reference to which they had received support.>
<5:13 They; these younger widows who are supported by the church under a promise of devoting themselves to her service.>
<5:14 I will--that the younger women marry; the younger widows of whom he has been speaking. This would be better for them and better for society. The adversary; the enemy of religion. The admission of young women into institutions where it is expected that they will never be married, thus exposing them to the manifold evils of such a condition, is directly opposed to the revealed will of God, and productive of great mischiefs to themselves and the community.>
<5:15 Some; of those spoken of in verses 1Ti 5:11-13. Turned aside after Satan; by complying with his temptations and falling into the evils mentioned above.>
<5:16 Have widows; widowed mothers or grandmothers, or any whom he or she ought to support. Widows indeed; who are destitute, are of the required age and character, and have not relatives to support them. Verse 1Ti 5:3.>
<5:17 Elders; having the superintendence of the church, some of whom labored as preachers and teachers of the gospel. Double honor; special respect, manifested, as the next verse shows, in provision for their wants. As the service of the church would occupy much of their time, especially when they devoted themselves to the work of preaching and teaching, a proportionate provision was to be made for their maintenance.>
<5:18 The scripture saith; De 25:4; Mt 10:10; Lu 10:7. It is the will of God that officers of the church, especially ministers who devote their life to the promotion of her interests, should receive not only respect and gratitude, but a just and reasonable compensation for their services: enough at least to provide comfortable support for themselves and their families.>
<5:19 Two or three witnesses; De 19:15.>
<5:20 Them that sin; and whose offences are proved.>
<5:21 The elect angels; the holy angels whom God in accordance with his eternal purpose, has preserved in a state of sinlessness. They are "all ministering spirits, sent forth to minister for them who shall be heirs of salvation." Heb 1:14. As such they are present in the assemblies of his church, and witnesses of the transactions there taking place. These things; the directions just given. Without preferring one before another; literally, without prejudgment, which is manifested in deciding a case beforehand under the influence of prejudice against a man, or prepossession in his favor. Doing nothing by partiality; the immediate reference of these words is to the hearing of accusations and the administering of rebukes, verse 1Ti 5:19,20. That they apply also to the ordaining of men for the service of the church is evident and is implied in what follows.>
<5:22 Lay hands; in ordination. Suddenly; hastily, without due investigation respecting the qualifications of the candidate. Partaker of other men's sins; as he would be, if through his negligence or sinful partiality improper men were raised to office in the church. In raising men to the sacred office, great care should be taken not to introduce improper persons. All suitable means should be used to ascertain their qualifications, and none be admitted who may not reasonably be expected to be faithful and useful.>
<5:23 Drink no longer water; water merely. A little wine; as a medicine, on account of his bodily infirmities.>
<5:24 Going before to judgment; they precede the man, as it were, to the place of judgment, and witness against him beforehand to his condemnation. They follow after; some wicked men's characters are not known at first; it is necessary to take time, make inquiries, and become more acquainted with them.>
<5:25 Likewise; so is it with good men. There is a great difference in the readiness with which men show their character. They that are otherwise; the good works that are otherwise; namely, not manifest beforehand. Cannot be hid; they will be revealed in time. Of course it is a duty to be cautious, to avoid haste, and use all proper means to obtain knowledge, in order to judge and act right.>
<6:1 Under the yoke; the yoke of servitude or bondage. Count their own masters worthy; manifest towards them a respectful, kind, forgiving, benevolent, Christian spirit. That the name of God and his doctrine be not blasphemed; that the wicked may not be led to speak against the Christian religion.>
<6:2 Believing masters; Christians. Not despise them; not withhold from them the manifestation of a respectful, obedient, Christian spirit, because they are brethren in Christ. Rather do them service; promote their interests the more cheerfully. Ga 6:10. Faithful; or, as the margin, believing, as the same word is rendered in the beginning of the verse. Beloved; of God. Partakers of the benefit; sharers with you in the benefit of the grace of the gospel. But we may better render, sharers [with you] in well-doing, or helpers [with you] in well-doing; that is, well-doing towards each other and all men. From such masters, therefore, servants have a right to expect the same kind, benevolent, forgiving, Christian spirit which is required of themselves. Eph 6:9. That they will forbear to threaten them with evil, and as they learn what their rights are, will respect them and render to them what is just and equal, Col 4:1; knowing that this is required of them by their Master in heaven. Mt 7:12; Lu 6:31. Ministers are bound to teach, that Christians in bondage, when wrongfully treated, whether in accordance with or in opposition to human laws, should possess and manifest the humble, patient, peaceful, forgiving, and obedient spirit of Christ, whether those who hold them in bondage are Christians or heathen; that they may thus show the excellence of Christ's religion, and if possible, lead all to embrace it. Ro 12:21.>
<6:3 Teach otherwise; differently from what Paul had taught as to the duty of Christian servants. The words of our Lord Jesus Christ; about the manifestation of a Christian spirit in all relations and conditions. Mt 5:39; 6:12-15; 18:21-35. According to godliness; that which accords with the word of God, and tends to promote his cause.>
<6:5 Supposing that gain is godliness; rather, that godliness is gain: in other words, that the profession of godliness is a business of worldly gain. Compare the case of Simon the sorcerer, Ac 8:18-24; what is said of the false teachers at Corinth, 2Co 11:20; and of these very "men of corrupt minds." 2Ti 3:5,6. The idea that it is always right to pursue the course in which we can make the most money, or possess the greatest influence, even though human laws do not forbid but require it, is a great error. The law of God is above human laws. By it human laws and their authors, those who obey, and those who disobey, are all to be tried, and approved or condemned.>
<6:6 Godliness--is great gain; the apostle, by a beautiful turn of thought, shows in what sense the proposition is true that godliness is gain. Not the outward form of godliness, but its inward substance with contentment is great gain--gain not of a worldly, but of a spiritual nature. It has the "promise of the life that now is, and of that which is to come." Chap 1Ti 4:8. Supreme regard to God, grateful obedience to his commands, cheerful submission to his dealings, and contentment with the allotments of his providence, will, through grace, gain all needed good in life, in death, and for ever.>
<6:7 For we brought nothing; a reason why we should have the contentment just spoken of. Riches, if we have them, are but a fleeting possession.>
<6:9 Will be; are resolved and determined to be rich. Another argument against the love of money, drawn from its hurtful nature. Fall into temptation; temptation to be dishonest, or so absorbed with earthly cares as to neglect their souls. A snare; one that Satan has set to catch their souls, by leading them to indulge the lusts of the eye, the lusts of the flesh, and the pride of life, till they sink into perdition.>
<6:10 The root of all evil; it tempts to the commission of all sorts of evil. Some; who professed to be Christians. Erred from the faith; wandered away from the faith of the gospel; with the accessory idea of their falling into sinful practices.>
<6:11 Flee these things; the love of money, with all its accompanying temptations and sins; never love money, nor be anxious to be rich. One of the most hateful and destructive sins is the supreme love of money, or of that ease, power, and influence which money will procure. A Christian should avoid this sin as he would the snare of the devil or the door of hell.>
<6:12 Fight the good fight of faith; the Christian life is here, as often elsewhere, compared to a warfare against sin and Satan. Eph 6:11-17; 2Ti 2:3. Lay hold on eternal life; as on a prize to be obtained by hard struggling. Whereunto; to the gaining of which prize. Called; by the grace of God through the gospel. Hast professed a good profession; or, confessed a good confession; in the widest sense, including not only his confession of Christ at his baptism and ordination, but also especially in times of persecution. See the following verse.>
<6:13 Witnessed a good confession; he bore his testimony to the truth concerning his person and mission in the face of death. The same steadfast confession he required of Timothy, and requires of all his followers.>
<6:14 Commandment; the whole charge contained in this epistle. Without spot, unrebukable; the commandment is kept without spot, unrebukable, when it is not marred by an imperfect obedience deserving of censure.>
<6:15 In his times, in due time; the proper time. He shall show; God shall make manifest, to whom alone belongs the ordering of the times and seasons. Ac 1:7.>
<6:16 Immortality; in and of himself; life underived, independent, and eternal.>
<6:19 A good foundation; for receiving the everlasting reward of grace in heaven. For this is given only to those who have shown their faith in Christ by being rich in good works towards man. Mt 25:34-45. Riches are the gift of God, and call for unceasing gratitude to him. They may be, and when rightly used, will be, the means of great and lasting good. Let rich men, as faithful stewards, from love to God, use their riches in promoting his glory and the benefit of their fellow-men, especially in making known his salvation to all people; and when called to leave the riches of earth, they will have the riches of heaven.>
<6:20 That which is committed to thy trust; the same as the "commandment," verse 1Ti 6:14. Oppositions; contentions and contradictions springing from science falsely so called; that is, spurious knowledge that exists only in name; that empty knowledge which puffeth up. 1Co 8:1. The apostle apparently alludes to those who explained away the vital truths of the gospel under pretence of imparting a deeper knowledge of them.>
\kniha{II Timothy}
\zkratka{2Tim}
<1:1 According to the promise of life; called to be an apostle according to the promise of life; for the purpose, namely, of furthering the knowledge of it. The blessings which come upon believers are according to the gracious purpose and promise of God in Jesus Christ.>
<1:2 Beloved son; spiritually, Paul having been the means of his conversion. When true religion leads one person to be the means of converting another, it forms between them a most affectionate and lasting bond of union.>
<1:3 From my forefathers; as I have received from them, both by example and precept, the duty of living in all good conscience before God. Ac 23:1. True religion is in all ages the same. Pious ancestors had the same spirit, and were accepted of God in the same way, with their pious posterity.>
<1:4 Thy tears; when they parted. Filled with joy; in meeting him again.>
<1:6 Stir up; as one does a smouldering fire, that it may burn brighter. This was to be done by the vigorous exercise of the gift of God; that was in him; the spiritual gifts, namely, that had been imparted to him by the Holy Ghost in connection with the laying on of hands. Compare 1Ti 4:14, and 1Co 12, where among spiritual gifts are named "the word of wisdom," "and the word of knowledge." Verse 2Ti 1:8. The faith and love of parents and grandparents will not save their children or grandchildren; they also must exercise faith and love themselves, and be active in discharging their appropriate duties.>
<1:7 Fear; timidity and cowardice, manifested in shrinking from arduous and dangerous services. Power; energy and courage in meeting and overcoming difficulties.>
<1:8 The testimony of our Lord; the work of testifying for Christ. Partaker; with me and all the faithful. The afflictions of the gospel; those which they were called to suffer in preaching it. According to the power of God; as manifested in the manner described in the following verse, and which manifestation is to us a pledge that he will always be with us, and support us in our trials. Those who delight in the company of Christians when they are in prosperity, are in danger of being ashamed of them and of forsaking them when in adversity.>
<1:9 Given us--before the world began; given us in God's eternal purpose.>
<1:10 Abolished death; put an end to its dominion over believers, and will finally deliver them for ever from its power. Brought--to light; revealed with greater clearness an immortal, glorious life for all believers. The eternal purpose of God to save his people is manifested by the gift of his Son to make atonement, and of the Holy Spirit to renew their hearts and lead them to devote life to his service.>
<1:12 For the which cause; on account of preaching this gospel. To keep that; his soul and all its concerns. That day; the day of judgment. The reproaches which haters of God and his cause sometimes cast on his people, and the sufferings thus caused, only increase their confidence in him, and their assurance that, as they are here called to suffer for him, they shall hereafter reign with him.>
<1:13 The form of sound words; the doctrines and duties of the gospel, as preached by Paul.>
<1:14 That good thing; that good charge or trust, namely, the office of preaching the gospel and presiding over the interests of the church. By the Holy Ghost; by his aid.>
<1:15 All they--in Asia; Asia in the narrower sense, the proconsular province of Asia, of which Ephesus was the capital. The word "all" is to be understood popularly of a general defection, for he immediately mentions one man of Asia who had remained faithful.>
<1:16 Onesiphorus; a Christian of Asia, chap 2Ti 4:19. Refreshed me; supplied my wants. My chain; by which he was bound to the soldier who kept him. Compare Ac 28:16.>
<1:18 True religion gives a man hearty, steadfast friends, who, in trials when others turn away, will stand by him, sympathize with him, and if possible render him any aid which he needs.>
<2:1 True religion gives a man hearty, steadfast friends, who, in trials when others turn away, will stand by him, sympathize with him, and if possible render him any aid which he needs.>
<2:2 The things that thou hast heard--among many witnesses; the solemn charge of doctrine and practice committed by the apostle to Timothy in the presence of many witnesses. Ability and disposition to teach the truths of gospel from love to God and to men, are essential qualifications in ministers of Christ; and none who do not possess them, should be introduced into the sacred office.>
<2:3 Endure hardness; such hardships as he would meet with in preaching the gospel.>
<2:4 Warreth; enlists as a soldier. The affairs of this life; the various kinds of business which other men pursue. Please him; his commander, by devoting himself undividedly to his service. So Timothy must devote himself wholly to Christ in the work of the gospel.>
<2:5 Lawfully; according to the laws which, in the Grecian games to which the apostle alluded, governed those who sought the prize.>
<2:6 The husbandman; must first labor, according to the laws which God has established, before he can be partaker of the fruits. So with ministers. Verses 2Ti 2:4,5.>
<2:7 Understanding; of the instruction which what he had said was suited to convey. Consideration and divine teaching are both needful in order to a right understanding of truth and duty. All should therefore exercise the one and seek the other. In so doing, they will find that there is no inconsistency between human agency and human dependence--that both are true and operate in delightful harmony together.>
<2:8 Was raised from the dead; so that you serve a risen Saviour, able to bestow upon you a glorious reward.>
<2:9 Wherein I suffer trouble; in the preaching of which gospel. Is not bound; opposers cannot prevent its progress.>
<2:10 The elect's sake; those whom God has chosen to salvation.>
<2:11 If we be dead with him; with Christ. See notes to Ro 6:3-11.>
<2:12 Suffer; suffer with him.>
<2:13 Faithful; to all his promises and threatenings. Deny himself; be untrue to his own character and declarations. The words can and cannot, like many other words in the Bible, are used in different senses; and if we would understand them correctly, we must consider the subject about which they are spoken, and the connection in which they stand.>
<2:14 Subverting of the hearers; turning them away from the truth.>
<2:15 Dividing the word of truth; communicating to each the portion suited to his wants.>
<2:16 They will increase; or they, that is, the men who are given to these babblings, will proceed to more ungodliness. Compare chap. 2Ti 3:13.>
<2:17 Their word; their pernicious teachings. Will eat as doth a canker; will eat into the spiritual body as a mortifying sore spreads through the natural body. The words describe both the malignant nature and the contagious character of false teachings.>
<2:18 That the resurrection is past already; perhaps they explained the doctrine of the resurrection figuratively of the regeneration of men's souls by the grace of the gospel.>
<2:19 The foundation; Jesus Christ, on whom are built the church of God and the hopes of his people. Isa 28:16; 1Co 3:10-15; Eph 2:19-22. This seal; this double inscription written upon it. Knoweth--his; and will keep them from the seductions of the wicked. This gives one mark of the true believer. Depart from iniquity; this gives the other mark. All who are on "the foundation of God" exhibit both these marks.>
<2:20 A great house; which here represents the visible church of God. Vessels of gold and of silver--of wood and of earth; a figurative way of saying that in Christ's visible church there will be found the precious and the vile. Compare the parables of the tares in the field, Mt 13:24-30,36-43, and of the net cast into the sea, verses Mt 13:47-50.>
<2:21 Purge himself from these; these vessels of dishonor, by avoiding them and their defilement.>
<2:22 Men must not only believe on Christ, but live pure and holy lives, in order to be saved.>
<2:25 Instruction is a great means of leading men to repentance; but while it is their duty without delay to repent, they are so wicked that without the grace of God they never will do it. Repentance when exercised is therefore the gift of God.>
<2:26 The snare of the devil; set to ruin them, by inducing them to embrace error. Temptations to embrace error are snares of the devil in order to ruin men. From such snares all who have been caught are bound, by believing and obeying the truth, to recover themselves.>
<3:1 In the last days; see note to 1Ti 4:1.>
<3:3 Truce-breakers; faithless men, who break treaties and refuse to fulfil their engagements. Incontinent; not restraining their fleshly lusts.>
<3:4 Heady; rash, reckless. High-minded; puffed up with a high opinion of themselves.>
<3:5 Having a form of godliness; having only its external form. These words mark the men whose character the apostle has drawn in such dark colors, as only outwardly members of the church of Christ. Denying the power; showing by their lives that they have not the spirit of true religion, and have not experienced its renewing and sanctifying power. Turn away; do not associate with or acknowledge them as Christians, and do nothing to countenance their errors. The great apostasy from the faith and practice of the gospel which has been and still is witnessed in the world, and which is manifested by its leaders exalting themselves as ecclesiastical and civil rulers, loving and amassing large sums of money, inducing men and women to stifle natural affection, to break away from and not to enter into family connections, falsely accusing men of heresy and putting them to death for reading and obeying the Bible, living in luxury and sinful pleasures, and at the same time abounding in forms and ceremonies of religion and bitterly opposing its spirit--was clearly foretold in the Bible. This is evidence that the Bible was given by the inspiration of God; and thus the wickedness of the wicked is an illustration of his truth. Ro 3:7.>
<3:6 This sort; of false teachers. Creep into houses; go slyly into families. Lead captive; get the confidence, and thus control the conscience and the purse of weak and wicked women. Inducing weak and silly women to repose such confidence in their spiritual guide as to reveal to him their secret feelings and conduct, and answer his polluted and polluting questions, is one of the most effectual ways of making them his captives, and leading them unhesitatingly into the most abominable transgressions.>
<3:7 Ever learning; from their false teachers what they would have them believe and do. Never able; under such teachers, to know the truths of the gospel. Compare Mt 15:14.>
<3:8 Jannes and Jambres; traditional names of two of the Egyptian magicians. Withstood Moses; Ex 7:11. These; these corrupt teachers. Resist the truth; by pretending to be Christ's, and yet opposing his truth.>
<3:9 No further; in their propagation of error and wickedness, than God shall suffer them. As theirs also was; the folly of Jannes and Jambres, who pretended to work miracles when they did not. In opposing what is good and in promoting what is evil, men will proceed no further than God shall suffer them; and when he sees best, he will make their folly and wickedness manifest to all.>
<3:11 Antioch--Iconium--Lystra; Ac 13:14,45,50; 14:2,5,19.>
<3:13 Men who knowingly seduce others into sin become more and more wicked, and in deceiving others they often deceive themselves, to their own destruction.>
<3:14 Continue thou; to believe, preach, and practise the truths learned from the Scriptures and from the preaching of Paul.>
<3:15 Holy Scriptures; of the Old Testament. All who have the Bible may, and if they rightly treat it will, become wise to salvation; and if they do not, it will be their own fault.>
<3:16 Inspiration of God; God directed the men who wrote it what to write, and how to write it, that as a rule of faith and practice for men it might be perfect. For doctrine; the communication of instruction. For reproof; to show men their sins. For correction; to reclaim them. For instruction; in what is right, and the presentation of the highest and best motives to induce men to do it. As all Scripture is given by inspiration of God, and like its author is perfect, and as it tends to make perfect all who believe and obey it, it should with the least possible delay be put into the hands of all people.>
<3:17 The man of God; especially the religious teacher. Perfect, thoroughly furnished; prepared and furnished on all sides. Unto all good works; pertaining to him as a teacher. These include his life as well as his labors. As the Lord said to dead matter, "Let there be light," and there was light, Ge 1:3; Ps 33:6, Heb 1:2; so he speaks in Scripture to dark, dead souls, and they become light in the Lord. He who breathed into man the breath of life, and made him a living soul, breathed into Scripture a life-giving power. Hence it is called quick or life-giving and powerful, Heb 4:12, converting the soul. "The words that I speak unto you, they are spirit, and they are life." Joh 6:63. Hence too the reason why the man of sin, whose domain is like the valley of dry bones in Ezekiel's vision, Eze 37:1-10, is opposed to giving Scripture free course among his people. If he should, it would cause a shaking among those bones, clothe them with sinews and flesh, and the Spirit whose breath it is would breathe into them spiritual life, and they would stand up for God an exceeding great army. Thus would God consume popery with the spirit of his mouth, and destroy it with the brightness of his coming. 2Th 2:8.>
<4:1 Quick; living. Appearing; in glory to judge the world. Kingdom; which shall then be fully consummated.>
<4:2 In season; at regular times of public worship. Out of season; at occasional seasons, when it can be done to advantage.>
<4:3 Sound doctrine; such as is revealed in the word of God, and adapted to promote the spiritual good of men. Teachers; who will not condemn their favorite sins. Itching ears; wishing for new teachers and new doctrines, and multiplying those who will amuse and flatter them.>
<4:4 Turn away--from the truth; because it condemns them. Fables; mystical speculations and frivolous stories.>
<4:5 Watch; be sober, earnest, and vigilant, as the original word implies. Evangelist; a minister of Christ, who went from place to place preaching the gospel, gathering churches, and assisting in establishing the institutions of Christianity. Make full proof of thy ministry; or, as in the margin, fulfil thy ministry; discharge its duties faithfully and thoroughly. While ministers of Christ have opportunity they should be most diligently and conscientiously employed in preaching the gospel, as they do not know how soon their opportunities will cease.>
<4:6 Offered; poured out as a drink-offering. My blood is about to be poured out for my attachment to Christ.>
<4:7 A good fight; more literally, the good fight; that, namely, prescribed to me by my Lord. The word in the original is one used of the Grecian games. Compare 1Co 9:26; 1Ti 6:12. Finished my course; the Christian race appointed me. This is another allusion to the Grecian games. Compare Php 3:12-14.>
<4:8 To the faithful minister, the close is the most joyful period of life. Evils, natural and moral, are behind, and all before is blessedness and glory.>
<4:10 Having loved; this life and its enjoyments so much that he left Rome to avoid the danger of losing them.>
<4:11 Luke; the writer of the gospel. Mark; Ac 12:25; 13:5; 15:37; Col 4:10.>
<4:13 The parchments; skins prepared, on which the more costly of the ancient books were written. Whether these were the writings of the apostle, or the writings of others, or merely materials for writing, is not known.>
<4:15 Ministers of the gospel are bound to be wise as well as harmless; to foresee evils and avoid them. Although they are not to be afraid of men, they are to beware of them; and so to act as not needlessly to put themselves in their power.>
<4:16 My first answer; his first hearing, probably during his present imprisonment, before the Roman emperor or a court commissioned by him. The word first implies a subsequent hearing which he had already had or expected soon to have.>
<4:17 Out of the mouth of the lion; according to some, literally, by being saved from death by wild beasts. According to others, the words are figuratively spoken of the Roman emperor Nero.>
<4:18 From every evil work; not from persecution even to death, for this the apostle expected; but from receiving any spiritual injury through the evil works of his enemies. God would keep his faith steadfast amid all his trials, and grant him a perfect and everlasting victory over them. Should godly ministers or private Christians in the hour of death be absent from earthly friends, they will not be alone. That Friend who loves them, and can do for them infinitely more than all others, will be with them; and when flesh and heart fail, he will be the strength of their heart, and their portion for ever. Ps 73:26.>
<4:19 The household of Onesiphorus; see note to chap 2Ti 1:16.>
\kniha{Titus}
\zkratka{Titus}
<1:1 According to; in order that God's elect might believe and acknowledge the truth which is according to godliness.>
<1:2 Promised; in his eternal purpose.>
<1:3 His word; which is the revelation of this eternal life. Through preaching, which is committed unto me; that is, through the ministry of preaching wherewith I am entrusted. These words are added to show how the revelation of eternal life contained in his word is to be promulgated. God sometimes converts men who have been distinguished for their opposition to him, and makes them eminently successful preachers of the gospel.>
<1:5 Set in order; complete arrangements which Paul had begun for the establishment of churches and the promotion of religion throughout the island. Over every congregation there should be a settled pastor, to preach the gospel, administer the ordinances of the New Testament, baptism and the Lord's supper, and set before the people an example of habitual and consistent piety.>
<1:6 Faithful children; trained up in the nurture and admonition of the Lord, and not guilty of disobedience to their parents or openly immoral.>
<1:7 7-9. A bishop must be blameless; 1Ti 3:2-7.>
<1:9 A blameless character, soundness in the faith, and ability to maintain and defend it, are essential requisites in ministers of the gospel; and without these none should be introduced into the sacred office.>
<1:10 The circumcision; Jews.>
<1:11 Whose mouths must be stopped; not by inquisitions or physical force, but by sound argument and conclusive reasoning--by the power of truth. Subvert; turn aside from the faith and practice of the gospel. For filthy lucre's sake; for base gain.>
<1:12 A prophet of their own; Epimenides, one of their poets. Evil beasts; ferocious and malignant. Slow bellies; slothful gluttons.>
<1:15 Unto the pure all things are pure: but unto them that are defiled--is nothing pure; there is an allusion here to the stress laid by the false teachers on Jewish distinctions of clean and unclean meats, washing of hands, and other ceremonial purifications, while they took no pains to keep their hearts and lives clean from the defilement of sin. The apostle exposes their error by laying down a general principle applicable to all the relations of life. The pure are those whose hearts and lives are controlled by the holy principles of love, faith, and obedience towards God. To them all God's creatures are pure, and they need fear no defilement from them while they use them with thanksgiving in accordance with his word. Ro 14:14-20; 1Ti 4:4,5. All their daily labors, moreover, secular as well as religious, are pure, being all performed in the spirit of love towards God and man. The impure are those whose hearts and lives are under the control of selfish and base passions. To them nothing is pure; for their mind and conscience being defiled, every outward work that proceeds from them is unclean also. Those who love and practise what is good are constantly growing better, and those who love and practise evil are constantly growing worse.>
<1:16 In works they deny him; they show that they have no such knowledge. Their words and works disagree. Unto every good work reprobate; rejected as men given over to iniquity, from whom no good work is to be expected. Not the professions but the practices of men are the index of their true character.>
<2:1 In works they deny him; they show that they have no such knowledge. Their words and works disagree. Unto every good work reprobate; rejected as men given over to iniquity, from whom no good work is to be expected. Not the professions but the practices of men are the index of their true character.>
<2:3 False accusers; given to slander.>
<2:5 Not blasphemed; that the gospel be not reproached on account of the inconsistency of its professors. The gospel inculcates upon all professing Christians the duties appropriate to their age, sex, and condition; and requires the performance of them, for the purpose of honoring God and preventing the reproach which they will otherwise bring upon his cause.>
<2:8 One of the most convincing proofs of the truth and excellence of the Christian religion is a consistently pious and beneficent life; and every friend of God should strive so to conduct, that no one can justly say any evil of him.>
<2:9 In all things; where they can do it and at the same time please God. Not answering again; not contradicting or finding fault with their masters.>
<2:10 Purloining; taking what is not theirs. Servants who conscientiously discharge their appropriate duties from regard to God, are a great ornament to the Christian religion, and do much to recommend it. They should therefore search the Scriptures, hear the gospel, and enjoy the means of grace, that their minds may be enlightened, their hearts purified, and all their powers employed most profitably to themselves and their fellow-men.>
<2:11 That bringeth salvation hath appeared to all men; or, as the margin, the grace of God that bringeth salvation to all men, hath appeared. By the grace of God, in the gospel salvation is provided for and offered to all men, and it gives actual salvation to all who receive it in faith. Ro 1:16.>
<2:13 The glorious appearing; at his second coming in glory to judge the world.>
<2:14 Our Saviour Jesus Christ, who gave himself a ransom for us, and will be our final Judge, is the great God. As such all should regard him, and ever be governed by his revealed will.>
<2:15 These things; the duties he had mentioned, verses Ti 2:1-14. Let no man despise thee; conduct in such a manner as to command the respect of all. Supreme love to Christ, manifested in acts of good-will to men, cannot be despised, but must be respected even by the wicked. Though they may hate and oppose it, their conscience condemns them, and thus forewarns them of their final doom.>
<3:1 These things; the duties he had mentioned, verses Ti 2:1-14. Let no man despise thee; conduct in such a manner as to command the respect of all. Supreme love to Christ, manifested in acts of good-will to men, cannot be despised, but must be respected even by the wicked. Though they may hate and oppose it, their conscience condemns them, and thus forewarns them of their final doom.>
<3:2 Speak evil of no man; falsely or unnecessarily. True religion makes good subjects, quiet citizens, peaceful neighbors, and renders men meek, patient, and forgiving in all the relations of life.>
<3:3 We ourselves also; as well as the unbelieving world without us now lying in wickedness. From the regenerating and renewing grace of God, which had elevated believers from a life of inward uncleanness and outward vice to one of holiness and righteousness, he draws an argument for consistency in a godly life. When not restrained by divine grace, men naturally run into all kinds of vicious, hateful excesses; and nothing but the Holy Spirit will prevent their growing worse and worse for ever.>
<3:4 Appeared; in the gift of his Son, the preaching of the gospel, and the operations of the Spirit.>
<3:8 To the gracious operations of the Holy Spirit Christians are indebted for the difference between them and the most heinous sinners. This they should feel, and should show their gratitude in doing good as they have opportunity, by precept, example, and every proper method, to all their fellow-men.>
<3:10 A heretic; in New Testament usage, sectarist, attaching himself to a factious party that rejected sound doctrine and a godly life. The first and second admonition; Mt 18:15-17; Ro 16:17,18. Kind, watchful, and efficient discipline should ever be maintained in Christian churches. Efforts, not by pains and penalties, but by sound argument and kind persuasion, should first be made to reclaim offenders. If all is unavailing, Christians should separate themselves from them, and thus bear open testimony against their transgressions; but they should never feel unkindly towards them, or seek to injure them. Persecutions, prisons, inquisitions, fires, and tortures are measures instigated by the devil and pursued by his adherents, not by the friends of Jesus Christ.>
<3:11 Subverted; turned from the right way. Sinneth; by embracing the side of error and sin. Condemned of himself; by his own conduct and conscience.>
<3:13 Nothing be wanting; needful for their journey.>
<3:14 Ours; those of our side, those who profess godliness. Maintain good works; live godly lives, and habitually and diligently pursue some useful business. For necessary uses; that they may have the means of honorably maintaining themselves and helping others, thus being not unfruitful in their lives. Joh 15:16; Eph 4:28;Php 4:17; Col 1:10.>
\kniha{Hebrews}
\zkratka{Heb}
<1:1 The train of thought in this opening chapter of the epistle is the following: God, who in past ages has given various partial revelations, has now made a full revelation of himself through his Son, who is the brightness of his glory, the maker and upholder of all things, and exalted above all the angels, as in name, so also in nature and office. At sundry times; or, in sundry parts. This marks the incompleteness of the past revelations. In divers manners; as by dreams, visions, voices from heaven, etc. All these are contrasted with the perfect manner of the present revelation, through God manifest in the flesh.>
<1:2 Heir of all things; Christ is the only begotten Son of God, in the high and incommunicable sense of possessing equality with the Father in nature. By virtue of this his sonship, God has made him heir of all that he possesses, that is, of the universe, and constituted him the sovereign Lord and Ruler of all things. Mt 28:18; Joh 16:15; Joh 17:10; Ac 2:36; 10:36; Re 17:14; 19:16. The worlds; the created universe, verse Heb 1:10; Joh 1:3; 1Co 8:6; Eph 3:9; Col 1:16,17. As the Scriptures are communications from God, we should receive them as such, diligently study, heartily believe, and faithfully obey them.>
<1:3 The brightness of his glory; in him the glory of the Father shines forth, so that in and through him the Father's glory is seen. Joh 1:14; 14:9; 2Co 4:6. The express image of his person; he in whom the very being of God is represented to us, as far as we are able to apprehend it. The Greek word rendered person, means rather substance, reality of being, as opposed to mere appearance. Compare Mt 11:27; Joh 1:18; Col 1:15. Upholding all things; sustaining the universe in being. Col 1:17. By the rod of his power; the word of his creative power. The same almighty word of his which called things into being, now sustains them. Compare Ge 1:3;, etc.; Ps 33:9; Ps 148:5. By himself; by the sacrifice of himself. Purged our sins; made expiation for them, thus opening the way for our forgiveness and purification. Ro 8:3; 2Co 5:21; Ga 3:13; Eph 1:7; 1Pe 2:24; 1Jo 1:7; Re 1:5. Sat down on the right hand; Ps 110:1; Mr 16:16; Ac 7:55. As Jesus Christ made the atonement, it is perfect and sufficient for all men, should be preached to all, and accepted by all; and is a sure foundation of eternal life to all who believe on him.>
<1:4 Better; in dignity and office. By inheritance obtained; obtained as his just right. See note to verse Heb 1:2. A more excellent name; that of the Son of God. Angels and men are called sons of God; but Christ alone is "the Son of God" in a high and peculiar sense, because, as Son, he has the same nature with the Father.>
<1:5 Thou are my Son; see the following note on the quotation from 2Sa 7:14. This day have I begotten thee; some understand these words of Christ's eternal sonship, supposing that with God, to whom time is nothing, "this day" may include eternity. But they are more commonly taken in a declarative sense of the manifestation which the Father made of Christ's sonship by his resurrection and glorification. So the apostle Paul seems to use them, Ac 13:33. Compare Ro 1:4; Col 1:18. I will be to him a Father, and he shall be to me a Son; 2Sa 7:14, compared with Ps 89:26,27. This promise was made not to Solomon as an individual, but to David's whole royal line, at the head of which, after David, Solomon stood, and which led to and terminated in Christ. Lu 1:32,33. God took David's house into the relation of sonship to himself, in the sense of making his seed heirs to his throne by an inalienable title. Ps 89:28,29,33-37. The lower sonship of David and Solomon, moreover, foreshadowed the higher sonship of Christ, in whom alone the promise here, and in Ps 2:7, is perfectly fulfilled.>
<1:6 Bringeth in the first-begotten into the world; by his incarnation and the events that followed it, thus establishing in and through him "the kingdom of heaven" among men. It is of this kingdom that the ninety-seventh Psalm, from which the apostle immediately proceeds to quote, speaks. It describes, by anticipation, the coming of God as king to destroy the wicked and save his people, verses Heb 1:3-6. His reign is one in which "the multitude of isles," the whole gentile world, is called upon to rejoice, verse Heb 1:1. The ancient Jews rightly understood the psalm of the Messiah, in whom alone it is fulfilled, and whose kingdom it describes in its whole extent to the end of time. Let all the angels of God worship him; quoted according to the Greek version from Ps 97:7, where the word "gods" in the original Hebrew means the heavenly hosts. As Christ in his deepest humiliation received the worship of angels as well as of men, and as he is now receiving it in glory, it is certain that he is God; and that in paying him divine honors they and we are not breaking, but obeying the command, Worship the Lord thy God, and him only shalt thou serve. Mt 4:10; Re 5:8-14.>
<1:7 And of the angels he saith; Ps 104:4. God indicates the nature and office of angels by calling them spirits and a flame of fire. The quotation, as usual in this epistle, is made from the Greek version of the Seventy.>
<1:8 Thy throne, O God, is for ever and ever; taken from Ps 45:6,7, where the Messiah appears in the character of the husband of the church.>
<1:9 Above thy fellows; in power and office; for God has made him "King of kings, and Lord of lords," Re 17:14; 19:16; exalting him not only above all human kings, but above all heavenly principalities and powers. Eph 1:21; Php 2:9,10; Col 1:18.>
<1:10 Thou, Lord, in the beginning; taken from Ps 102:25-27, another psalm which prays for the coming of God in glory to build up Zion, verse Ps 102:16, and which, like Ps 97.1-12, has its true fulfilment in Christ, who is God manifested in the flesh.>
<1:12 As Christ made, sustains, and governs all things, and will remain unchangeable for ever, it is safe to trust in him, and to commit all our interests to his care and disposal.>
<1:13 Sit on my right hand; Ps 110:1. The Saviour interprets these words of himself. Mt 22:41-45.>
<1:14 Ministering spirits; Ge 19:1-23; Ps 34:7; Ps 103:21; Da 6:22; Da 7:10. Christians are highly honored and greatly blessed; their attendants are more exalted than those of any earthly kings, and they are themselves to be kings and priests unto God, and to reign with him for ever and ever. Ro 5:17; 2Ti 2:12; Re 5:9,10; Re 22:5.>
<2:1 Therefore; on account of the dignity and glory of him who speaks. We have heard; from Christ and those commissioned to speak in his name. Let them slip; forget or neglect them, and lose the benefit. The character of the Saviour should secure for his instructions the most earnest attention. This all ought to give, from regard not only to him, but to their own benefit.>
<2:2 Spoken by angels; that is, by the ministration of angels. Ac 7:38,53; Ga 3:19. Was steadfast; firm, settled, established, and could not be violated with impunity.>
<2:3 The neglect of Christ's salvation is ruinous to the soul.>
<2:5 The world to come; or, the coming age. This was a common expression with the Jewish Rabbis to indicate the expected reign of the Messiah, which is the Christian dispensation in its whole course to the end of time and the final judgment.>
<2:6 One in a certain place; Ps 8:4-9.>
<2:8 For in that he put all in subjection under him, he left nothing--not put under him; as much as to say, The psalmist explicitly declares that all things are put under man. We see not yet all things put under him; that is, under man, apart from Christ.>
<2:9 But we see Jesus--crowned with glory and honor; as much as to say, The words of the psalmist have their complete fulfilment only in "the man Christ Jesus," in whose person human nature is exalter to universal dominion and crowned with glory and honor, after he has been made a little lower than the angels; in the nature which he took upon him for the purpose of suffering death to atone for sin.>
<2:10 It became him; it was suitable that God, in saving sinners, should effect it through the suffering and death of his Son, who thus became the Author of complete, eternal salvation to all who trust in him. Perfect through sufferings; perfect in his character as Mediator and Redeemer. It was through the suffering of death for the salvation of men that he was to be exalter to the mediatorial throne and have all things put under his feet. Compare Php 2:5-11, which may serve as a divine commentary on the present verse.>
<2:11 He that sanctifieth; Christ. They who are sanctified; Christians. All of one; either simply of one nature, or of one Father, as partakers of the same nature, or of one Father, as partakers of the same nature received from God. The latter view is favored by the words "many sons," immediately preceding.>
<2:12 Saying; in Ps 22:22, a psalm of which Christ is the subject.>
<2:13 I will put my trust in him; 2Sa 22:3, where David in his conflict and victory is regarded as the type of Christ. Some suppose the quotation to be from Isa 8:17, where, in the Greek version of the Seventy, the same words occur. The argument is, that trust in God is an attribute of men. Christ, by exercising it, makes himself one with men. Behold, I and the children which God hath given me; taken from Isa 8:18. Some understand the words of Isaiah as spoken directly and exclusively of the Messiah. But they may be more naturally understood of the prophet himself, who was, by God's appointment, a type or symbol of Christ in his prophetical office, as David was in his kingly office. As such, the prophet and his children were "for signs and for wonders in Israel from the Lord of hosts, who dwelleth in mount Zion," as he immediately adds. In Christ then, the great antitype, the words have their perfect fulfilment. By the expression, "I and the children which God hath given me," he declares that he has a common nature with them, which is the point to be proved.>
<2:14 The children; in allusion to the words just quoted: "Behold, I and the children which God hath given me." Through death--destroy; for it was through death that Jesus conquered and spoiled the prince of death. Joh 12:31. Him that had the power of death--the devil; by the agency of the devil sin was introduced into the world, and death through sin. Ro 5:12. Over all that are out of Christ he reigns, in and through death, as a cruel tyrant and tormentor. But Christ, by redeeming men from sin and death, takes them out of the power of Satan.>
<2:15 Through fear of death--subject to bondage; the sting of death is sin and its penalty. It is this that makes it so terrible to men. From this sting Christ delivers all who trust in him, making the death of the body to them the gateway to eternal life. Those who believe in Christ need not fear death, for it will put an end to all their sorrows, and introduce them to endless joys.>
<2:16 Took not on him the nature of angels; or, as the margin, taketh not hold of angels, for the purpose of saving them; and so in the following clause. The way in which he takes hold of the seed of Abraham is by the assumption of their nature, that he may in and through it redeem them. The seed of Abraham includes all who are Abraham's children in a spiritual sense, by the possession of his faith. Ro 4:11; Ga 3:7,16.>
<2:17 It behooved him; it was proper for him. His brethren; of the human race. To make reconciliation for the sins; more exactly, to make propitiation for the sins, which was the office of the Jewish high-priest. But he did it typically, by the blood of bulls and goats; Christ does it efficaciously, by his own blood. Chap Heb 9:12.>
<2:18 He is able; having endured sufferings and temptations, he is fitted to sympathize with and deliver others who endure them. Jesus Christ being both God and man, perfectly understands and rightly regards the claims of God and the character and interests of men, and is thus prepared to bring glory to God in the highest, and manifest most effectively good will to men.>
<3:1 Wherefore; on account of the character and work of Christ as exhibited in the previous chapter. Heavenly calling; by which God called and inclined them to prepare for heaven. Habitual contemplation of the character work, teaching, example, death, resurrection, intercession, government, and glory of Christ, is a powerful means of increasing the holiness of his people, and securing their perseverance in his service.>
<3:2 Him that appointed him; God the Father, who appointed his Son to be the author and introducer of the Christian dispensation. Faithful in all his house; in all God's house, the Jewish economy, with the household of God's covenant people contained in it. The reference is to Nu 12:7, where God says, "My servant Moses--is faithful in all my house.">
<3:3 This man; Christ, as the builder of God's house under the Christian economy. Hath builded the house; or, prepared the house, for the words include not only the building of the house itself, but also the ordering of the household belonging to it. Hath more honor than the house; than the structure itself, or any of the household pertaining to it; consequently, more honor than Moses who was not the builder of the house in which he served, but himself constituted a part of it, that is, of its household. The greatest and best of men are as much inferior to Christ as the thing made is inferior to him who made it.>
<3:4 Every house is builded by some man; or, by some one; added to unfold still further the contrast between the house and its builder. But he that built all things is God; that is, but God is he that built all things. These words are added to refer the house, of which Christ is the builder and owner, to God as its ultimate author: as much as to say, Christ is indeed the builder and Lord of the Christian dispensation with its household of faith; but he has built it as the Son under the appointment of the Father, from whom all things originally proceed. Compare, for the same idea, chap Heb 1:2, "By whom also he made the worlds." Christ made all things, Joh 1:3; Col 1:16,17; Heb 1:10,11,12; therefore Christ is God, Joh 1:1; Ro 9:15; 1Ti 3:16; Heb 1:8; 1Jo 5:20.>
<3:5 Faithful in all his house; in all God's house. See note to verse Heb 3:2. As a servant; and therefore a part of the house itself. See note to verse Heb 3:3. For a testimony of those things which were to be spoken after; or, more exactly, for a testimony of the things that should be spoken, the word "after" not belonging to the original. The meaning is, that he, as God's faithful servant, might testify to the people the things that should be spoken through him to them.>
<3:6 As a son; he was faithful over the household or spiritual family of which he was the rightful owner. Whose house are we; to which family we--Christians--belong. The confidence; in the sense of boldness or assurance, such as a well-grounded faith in Christ gives. The rejoicing of the hope; or, the glorying of the hope; that glorying in Christ and his salvation which the hope of our future inheritance in heaven gives.>
<3:7 Wherefore; since we are the household of Christ, who is so much greater than Moses. The Holy Ghost saith; Ps 95:7-10. To-day if ye will hear his voice; the command of God is always to-day; for he always demands present obedience.>
<3:11 So I sware in my wrath; Nu 14:23. My rest; the rest of Canaan, so called in De 12:9, 10, and which is a type of the rest of heaven. Perseverance in faith and obedience is essential to a well-grounded hope of salvation; and should any cease to believe and obey Christ, they would harden their hearts, grieve the Holy Ghost, and be in danger of destruction.>
<3:13 Exhort one another; to be steadfast in the belief and practice of the gospel.>
<3:14 Partakers of Christ; united to him by faith, and entitled to his favor, and the enjoyment of the rest provided by him for his people.>
<3:15 Harden not your hearts; by refusing to hearken to Christ. In the provocation; when the Israelites provoked God. Nu 14:2-11. Great watchfulness is needful to the people of God, and the diligent use of appropriate means, in order to secure their perseverance in holiness and to prevent their final apostasy and ruin.>
<3:16 Did provoke; displease God by disobedience.>
<3:17 Whose carcasses fell; Nu 26:64,65.>
<3:18 Sware he; Nu 14:12-37.>
<3:19 Could not enter in; to the rest of Canaan, typifying the rest of heaven. The great and destructive sin which cuts off the hope of heaven and makes perdition certain, is unbelief.>
<4:1 The course of argument in this chapter, to verse Heb 4:11, is as follows: There is a rest promised to us, which we should be careful not to lose by our unbelief and disobedience, after the example of the ancient Israelites in the wilderness. This cannot be the rest upon which God entered after he had finished the works of creation, nor the rest which Joshua gave to Israel in Canaan, since long after both of these the Holy Ghost still speaks of a rest which he warns us, as he did the covenant people in David's day, not to lose. His rest; God's rest provided for his people. Should seem to come short of it; regarded by some as simply a softened way of saying, Should come short of it. Others render, Should appear--that is, at the last day--to have come short of it.>
<4:2 The gospel; good news of a future rest. Unto them; the Israelites in the wilderness. The word preached; the offer to them of a future rest. Did not profit them; because they did not, by believing God, comply with its instructions.>
<4:3 Do enter into rest; literally, into the rest; that, namely, which the Holy Ghost, through David, warns us not to lose. There is a rest promised to believers now as really as there was to believers in the days of Moses; and true Christians have a foretaste of it. It is a spiritual, holy rest, like the rest of God on the Sabbath after he had finished the work of creation; and of which the right keeping of the Sabbath is to believers an emblem. As he said; Ps 95:11. If they shall enter; a Hebrew form of expression, the same as in chap Heb 3:11, meaning, they shall not enter. See also verse Heb 4:5. Although the works were finished; as the Holy Ghost, by the mouth of David, said this nearly three thousand years after God on the Sabbath rested from his work of creation, it is plain that this was not the rest referred to.>
<4:4 In a certain place; Ge 2:1-3, showing that there is a rest upon which God entered when he had finished the works of creation.>
<4:5 And in this place again; showing that God has still another rest into which he invites us to enter.>
<4:6 6, 7. Seeing therefore--harden not your hearts; some connect verse Heb 4:6 immediately with verse Heb 4:11, making the intervening verses a parenthesis. But the passage is plainer if taken without any parenthesis, thus: "Seeing therefore"--as has been shown by the preceding argument--"it remaineth"--long after God has entered upon his rest of the Sabbath--"that some must enter therein;" in other words, that it is a rest yet reserved for some, namely, for all those who accept it as it is offered; "and [seeing] they to whom it was first preached"--namely, the Israelites in the wilderness--"entered not in because of unbelief; again, he limiteth a certain day"--that is, he therefore again sets a certain day--"saying by the mouth of David, To-day; thus showing that to-day an offer is made to men of God's rest--"after so long a time"--so long a time after the rest of Canaan had been entered upon--"as it is said"--rather, as it has been said before, in the quotation already made from Ps 95:7--"To-day if ye will hear his voice, harden not your hearts" Of course, when David spoke of a time when men by believing might obtain rest, it was not the rest of Canaan, for that they had, verse Heb 4:8; nor was it the rest of the Sabbath, for that they had, verses Heb 4:3,4; but it was the rest of which these were emblems, the glorious, eternal rest of heaven.>
<4:8 Jesus; Joshua; Jesus being the same in Greek as Joshua in Hebrew, meaning Saviour. Afterward; in the days of David. Another day; or time when the rest spoken of could, by believing, be obtained.>
<4:9 Therefore; as the certain conclusion from the above-mentioned facts, the rest spoken of by God is one which is spiritual and future; the keeping of an eternal Sabbath, a holy, blessed rest in heaven. The rest promised to the faithful and obedient in the Old Testament, was not merely a temporary rest on the Sabbath, or in Canaan, but a spiritual, eternal rest in heaven; of which the rest of the Sabbath and the rest of Canaan were emblems.>
<4:10 His rest; in heaven. Hath ceased; from his work on earth. As God; ceased from his work of creation on the first Sabbath. God's method of salvation was not designed, and is not adapted to encourage idleness, but great and persevering diligence in the discharge of duty.>
<4:11 Therefore; as there is such a glorious, heavenly rest, and many through unbelief have lost it, let us give all diligence by faith and obedience to secure it, lest through unbelief we also lose it.>
<4:12 The word of God; all his declarations, whether of law or grace, whether of promise or threatening. God, who is its author, imparts to it his own divine energy. It lays open every heart, and detects all hypocrisy and unbelief. Quick; living, and powerful in its effects. Joh 6:63; 2Co 10:4; Two-edged sword; Eph 6:17; Re 1:16; 19.15. Discerner of the thoughts; lays open the secrets of the heart, and shows a man to himself. Ro 7:7. Our faith, therefore, must be hearty, active, and persevering, or we shall fail of obtaining the promised rest.>
<4:13 In his sight; the sight of God the author of this word. In order to be saved, men must be Christians in reality as well as in appearance. God sees men as they are, and will treat them according to their works.>
<4:14 Profession; of faith in Christ.>
<4:16 The throne of grace; God on his gracious throne dispensing mercy to sinners. In God is help for men; and it is their duty to come unto him in the name of Christ, that for his sake they may receive it.>
<5:1 Having several times spoken of Christ as our High-priest, chap Heb 2:17; 3:1; 4:14,15; he now proceeds to unfold at large the idea of his priesthood, chap Heb 5:1-10:18, introducing, however, a digression, chap Heb 5:11-6:19, by way of warning and exhortation. He begins by considering the qualifications and office of the earthly high-priest. He must be taken from among his brethren, that he may be able to sympathize with them, being himself a sharer of their infirmities; and he must be called of God. His office, moreover, is to offer gifts and sacrifices for sin. With this earthly priesthood the higher priesthood of Christ is then compared. For men; for the benefit of men in their spiritual concerns.>
<5:2 A kind, compassionate, and forgiving spirit is of great importance to all in the sacred office; and the consideration of their own unworthiness and of the grace of God towards them, is well suited to increase in them this heavenly temper.>
<5:3 By reason hereof; of infirmity, which, in the case of the earthly high-priest, is connected with sin. See Le 9:7.>
<5:4 This honor; of being a priest under the law of Moses, and offering sacrifices.>
<5:5 He that said; God the Father, who appointed his Son Jesus Christ to be our High-priest. Thou art my son; Ps 2:7. The sonship of Christ is here considered as including his priesthood. Chap Heb 1:5.>
<5:6 Another place; Ps 110:4. The Old Testament is so constructed that the fulness of its meaning is seen only in the light of the New; and both must be taken together in order to have the fullest understanding of the revealed will of God.>
<5:7 Offered up prayers; Mt 26:39-56. In that he feared; because he was reverently obedient and submissive to God, God heard his prayer and answered it, in bestowing upon him all that he needed to prepare him for what was before him. Lu 22:39-46.>
<5:8 A Son; the divine Son of God. Yet learned he obedience; he learned by experience what it is to obey God in the midst of manifold sufferings. Thus he was qualified to succor those who are in like circumstances of suffering. Chap Heb 2:18; 4:15.>
<5:9 Being made perfect; having triumphantly gone through with the course of suffering appointed for him, and thus become perfect as our Saviour. Though Christ has opened a way of salvation and commands us to make it known to all, yet every man for himself must enter and continue to walk in it, or he cannot be saved.>
<5:10 Called of God; verse Heb 5:6.>
<5:11 Hard to be uttered; difficult to be so explained that you will understand them. Seeing ye are dull of hearing; slow of apprehension, through your sluggishness in respect to divine truth.>
<5:12 The time; the length of time since they were converted. Milk; the simplest truths.>
<5:13 Unskillful in the word of righteousness; inexperienced, having comparatively little knowledge of the character and work of Christ, and the way of salvation through him as revealed in the Scriptures.>
<5:14 Strong meat; the more difficult parts of divine truth. Of full age; of greater experience and knowledge of divine things. To discern; to distinguish between truth and error, good and evil. Even Christians at first, and often for a long time, are ignorant of many things clearly revealed in the word of God, known by those who have made greater advances in the divine life, and which, where God gives opportunity, ought to be known by all.>
<6:1 Principles; elements or first rudiments of religion. Unto perfection; maturity in the knowledge and obedience of the gospel. The reference is especially to those deep doctrines concerning the priesthood of Christ which he is preparing to unfold. The foundation of repentance; the foundation consisting in the doctrine of repentance, and what follows. Upon this foundation we must always build, but we ought not to be always laying it. Dead works; outward forms without spiritual life. Divine grace in the hearts of God's people is progressive. It leads them to increase in knowledge and piety, till they at last become perfect in Christ Jesus.>
<6:2 Eternal judgment; judgment eternal in its consequences, having for its result the endless retributions of eternity.>
<6:3 This will we do; we will go on unto perfection, as he proceeds to do in the seventh and following chapters.>
<6:4 Enlightened; in the knowledge of the gospel. The heavenly gift; that which God bestows upon men in the gospel. Of the Holy Ghost; of the gifts which he bestows. 1Co 12:4-11.>
<6:5 Have tasted the good word of God; have had experience of its excellency and power. And the powers of the world to come; the world to come is probably here, as in chap Heb 2:5, the gospel dispensation, and its powers are those specified in chap Heb 2:4.>
<6:6 Fall away; renounce Christianity, turn against Christ, and openly apostatize from his religion. If they do this they will perish, because they renounce the only way of salvation, and treat Christ as an imposter, deserving of crucifixion.>
<6:7 Receiveth blessing from God; he rewards its fruitfulness with his blessing, making it still further fruitful. Compare the Saviour's words, "He that hath, to him shall be given." Mr 4:25.>
<6:8 Thorns and briars; only. Rejected; as worthless. Nigh unto cursing; being given up to perpetual barrenness; bearing that which is fit only to be burned. So those who renounce Christ, go back to the world, and continue in sin, the Holy Ghost will leave to perpetual barrenness and death.>
<6:9 Persuaded better things; he was persuaded, from what he had known of them, that in view of the destruction which awaited them should they apostatize, they would, through the grace of God and the use of proper means, persevere in holiness to the end, and so obtain eternal life.>
<6:10 God is not unrighteous; he would not fail to reward the acts of love which for his sake they had done to his people. Mt 10:41,42.>
<6:11 The same diligence; in the discharge of duty and the manifestation of love to Christ and his people to the end of life. To the full assurance of hope; these words express the object which our Christian diligence has in view and which it secures. As those who apostatize, and continue to renounce Christ, will perish with an awfully aggravated destruction, Christians should carefully guard against all approaches towards this sin.>
<6:12 Of them; Heb 11:32-40. As a knowledge of the destruction which awaits men who renounce the Saviour and continue in sin, is one of the means of preventing Christians from so doing, they should be thankful to him for communicating this knowledge, and for rendering it, with other means, efficacious in leading them to avoid this destruction, and perseveringly to imitate those who through faith and patience and much tribulation are now inheriting the promises.>
<6:13 Made promise to Abraham; Ge 22:16-18.>
<6:15 Obtained the promise; Ge 12:1-3; 15:5-21; 17:1-16; 18:10; 21:1,2.>
<6:16 For confirmation; to confirm treaties and agreements; the oath gives confidence and puts an end to contention.>
<6:17 Wherein; in respect to which matter, namely, the ending of all doubt and dispute by an oath. Heirs of promise; true believers, to whom God has promised eternal life. Joh 10:27-30.>
<6:18 Two immutable things; his word and oath. Impossible; for want not of natural power, but of disposition; on account of his unchangeable faithfulness, truth, and holiness. Fled for refuge; to Jesus Christ, by believing on him. The hope; of heaven set before us in the gospel.>
<6:19 Which entereth; it is immaterial whether we understand the hope of the anchor as entering, since the one is a symbol of the other. Into that within the veil; into the heavenly holy of holies within the veil, that is, into heaven itself. For here, as elsewhere in this epistle, the earthly tabernacle--in whose inner sanctuary God had his visible dwelling-place between the cherubim that overshadowed the ark, Ex 25:22; Nu 7:89; Ps 80:1; Ps 99:1--is considered as a type of the true heavenly tabernacle where God resides. Compare chap Heb 8:2; Heb 9:11,12; and especially chap Heb 9:24. Hope has great influence in the salvation of Christians, and the gospel is suited to inspire it. But in order to this, the gospel must be believed. And that hope which arises from true faith tends powerfully to make men holy, and lead them, notwithstanding all trials, to persevere in holiness to the end. Pr 10:28; 11:7; 1Jo 3:3.>
<6:20 The forerunner is for us entered; he has entered into heaven itself as our high-priest, to present his own blood before the throne as the expiation for our sins; and he has entered as our forerunner also, who will in due time bring us into his Father's presence, and present us faultless before his throne. Made a high-priest--after the order of Melchisedec; thus the writer returns to the theme which he had proposed, chap Heb 5:11.>
<7:1 The argument in the present chapter rests on the certain truth, that God appointed Melchisedec to be a type of Christ in his priestly office, and ordered every thing concerning his history in such a way as to make the type as perfect as it could be in the case of a mere earthly priest. By his wise providence it came to pass, first, that both his name and that of the place where he reigned should be typical of Christ's character and office, verse Heb 7:2; secondly, that the inspired record should give his priesthood without any such limitations in respect to descent as belonged to the Levitical priesthood, and also without any notice of either the beginning or end of his life and priesthood, verses Heb 7:3,6,8; thirdly, that he should bless Abraham, the father of all the faithful, and receive tithes from him, in both which things was made manifest Melchisedec's official superiority over him, and consequently over all his children, none of whom could pretend to be in dignity above him, verses Heb 7:4,6,7. Melchisedec--met Abraham; Ge 14:18,19.>
<7:2 King of righteousness; this is the meaning of the Hebrew word Melchisedec. Salem; that is, peace; Melchisedec was therefore in his own name and that of his city a fit type of the righteous Prince of peace, Isa 9:6; 11:4,5; 32:1.>
<7:3 Without father--end of life; the inspired record takes no notice of any of these things; and this was designed by the Holy Ghost, that his priesthood might thus typify the priesthood of Christ in a double way; first, as to our Lord's human nature, as being a priest of another order than the Levitical priests, who must always be able to show their descent from Aaron, verses Heb 7:13,14, compared with Nu 3:10; Ezr 2:62; secondly, as to his divine nature, as being in the highest sense without any of these limitations. The reader should carefully notice that the apostle describes Melchisedec, the type, in terms which, in the full meaning, hold good only of Christ the great antitype. Christ as a priest making a real and perfect atonement for sin, stands alone in divine majesty, grandeur, and glory. All other priests were only types, emblems, and shadows of him, which when he appeared vanished away.>
<7:4 How great this man was; see verses Heb 7:6,7, and notes.>
<7:5 Have a commandment to take tithes; Nu 18:21-32. Though they come; though their brethren of whom they take tithes, come out of the loins of Abraham. Thus the Levitical priests are raised above their brethren in official dignity.>
<7:6 But he; Melchisedec. Received tithes of Abraham, and blessed him that had the promises; being thus exalted, not as the Levitical priests were, above the rest of their brethren, but above Abraham himself, and thus, as the epistle goes on to show, above the Levitical priesthood also.>
<7:7 The less--the better; in official dignity.>
<7:8 Men that die--he liveth; he passes to another point in which Melchisedec's priesthood was typically superior to that of the Levitical priests: it had no limitation; all the testimony we have of him is as a living priest and king, no mention being made of his death or the end of his priesthood. The writer designedly applies to Melchisedec terms which have their full application to Christ alone. See note to verse Heb 7:3. However much one man may be elevated above another, or however sacred the employment to which he may be called, he is a sinner; he must die, and with his fellow-sinners stand at the bar of Christ, give account of the things done in the body, and be treated for eternity according to his works.>
<7:9 Levi also; and in him the Levitical priests of whom he was the father. Paid tithes in Abraham; paid tithes to Melchisedec, and thus acknowledged his superiority.>
<7:11 11-19. A new argument is now introduced. Since the Levitical priesthood and the law were given together, as parts of one whole, so that the annulling of the one is the annulling of the other, why should God have promised another priesthood, and with it another economy, except because the former priesthood with its economy was unable to give perfection? Perfection; see note to verse Heb 7:19. Under it the people received the law; it was the basis of the Mosaic law in such a way that when the law should be changed, that must be changed also, verse Heb 7:12.>
<7:13 He; Christ, spoken of in Ps 110:4. To another tribe; not that of Levi, from whom the priests under the law were to descend.>
<7:15 It is yet far more evident; that there is a change of the priesthood, and with it, of the economy.>
<7:16 Who is made; constituted a priest. Not after the law of a carnal commandment; not with a temporary and inefficacious priesthood, corresponding with the carnal ordinances of the law under which he ministers. Compare, for the meaning of these words, chap Heb 9:9,10; Heb 10:4. After the power of an endless life; with an efficacious priesthood, such as belongs to one who has endless life and is a priest for ever, verses Heb 7:17,25. As Christ has made a full and perfect atonement, and ever lives to make intercession, all should forsake their sins, trust in him, and come to him for grace to help in all times of need.>
<7:18 A disannulling; setting aside and bringing to a close the ceremonial law and its priesthood. Weakness and unprofitableness; as to the work of making a true expiation for sin, and thus opening a true way for salvation. See the following note, verse Heb 7:19.>
<7:19 The law made nothing perfect; the ceremonial law was not designed for that. It answered the local and temporary purpose for which it was intended, but its sacrifices could not, like the sacrifice of Christ, purge the conscience from dead works to serve the living God, cleanse from sin, justify and sanctify the soul, give it access to God, and inspire that hope which purifies it as Christ is pure. But the bringing in of a better hope; the gospel through the atonement, righteousness, and intercession of Christ, does all this. Of course the gospel must be immeasurably superior in its benefits to the ceremonial law. Verses Heb 7:18,19 may be more plainly and simply rendered thus: "For there is verily a disannulling of the commandment going before, for the weakness and unprofitableness thereof--for the law made nothing perfect--and there is in the bringing in of a better hope," etc.>
<7:20 20-22. Still another argument to show the superiority of Christ's priesthood to that of Aaron and his sons: He was made priest with an oath, they without an oath. The added solemnity of the oath shows the superior dignity of the priesthood.>
<7:22 A surety; one who becomes responsible for the fulfilment of a covenant. A better testament; or, a better covenant. The same Greek word is rendered now covenant, as in chap Heb 8:6, etc., and now testament, as in chap Heb 9:15, etc. The later is the appropriate rendering where there is a reference to the death of him who mediates the covenant, as in the latter of the above passages. The covenant which was ratified by the blood of Jesus secured for ever the highest and best of blessings to all who trust in him and devote their life to his service.>
<7:23 The last argument for the superiority of Christ's priesthood over that of the Levitical priests, after which there is a summing up of the perfections of our great High-priest, Christ Jesus, verses Heb 7:25-28.>
<7:25 By him; as their high-priest, not venturing before God in their own name.>
<7:26 Became us; was needed by us. Made higher than the heavens; where he ministers before God. See chap Heb 8:1,2,4; 9:24.>
<7:27 This he did once; made a full and complete atonement, so that no further sacrifice for sin would ever be needed.>
<7:28 Consecrated; or perfected as a High-priest. Compare chap Heb 2:10; 6:9. Christ is in all respects such a Deliverer as sinners need. None perish for want of an all-sufficient and willing Saviour, nor because a way of salvation is not opened, nor because God does not desire their salvation; but if any who know the gospel perish, it is because they willfully and perseveringly refuse to accept its gracious offers.>
<8:1 Consecrated; or perfected as a High-priest. Compare chap Heb 2:10; 6:9. Christ is in all respects such a Deliverer as sinners need. None perish for want of an all-sufficient and willing Saviour, nor because a way of salvation is not opened, nor because God does not desire their salvation; but if any who know the gospel perish, it is because they willfully and perseveringly refuse to accept its gracious offers.>
<8:2 Of the sanctuary, and of the true tabernacle; of the true heavenly sanctuary and tabernacle, of which the earthly is only a type, verse Heb 8:5. As Jesus Christ is a High-priest and Mediator in all respects such as we need, it is wrong to trust in or acknowledge any other.>
<8:3 That this man; Christ, if he would perform the office of a priest.>
<8:4 Not be a priest; he could not on earth officiate as a priest according to the Jewish law, because he did not belong to the tribe from which alone priests could be taken. He therefore, after having offered himself a sacrifice, ascended for the further discharge of his priestly office to heaven, of which the holy of holies was a type. Chap Heb 9:12.>
<8:5 Who serve unto the example and shadow of heavenly things; that is, who minister to the earthly tabernacle, which is but a type and shadow of the true tabernacle above where Christ ministers. As Moses was admonished; Ex 25:40. According to the pattern showed to thee in the mount; apparently a representation made to Moses in vision of a glorious structure after which he was to model the earthly sanctuary with its furniture, the tabernacle seen in vision being itself a type of the true spiritual realities of heaven. Compare Ezekiel's vision of a city and temple, chapters Eze 40.1-48.35, and John's vision of the new Jerusalem, in Revelation, chapters 21, 22.>
<8:6 He; Christ. A more excellent ministry; than the Jewish priests. Better covenant; than that formed with Israel at Sinai. Better promises; securing greater blessings. As the gospel dispensation is the last that God will ever grant to men, those who live under it and yet are not by it led to repentance and salvation, will perish with an everlasting destruction.>
<8:7 First covenant; that at Sinai. The second; that of the gospel.>
<8:8 With them; according to some, with the provisions of the covenant at Sinai, as not adapted to give perfection, chap Heb 7:11,18,19. But we may more naturally refer the words to those who lived under the covenant. It was in connection with severe rebukes that this promise was given of a new covenant which should accomplish what the old had failed to do. He saith; Jer 31:31-34. New covenant; the gospel dispensation, which is spiritual in its nature.>
<8:10 Put my laws into their mind, and--their hearts; deeply and permanently impress them on their minds, and incline their hearts to obey them. And I will be their God, and they shall be my people. In the gospel covenant, God not only makes known to his people his will, but secures their obedience to it. To his grace they are indebted for their disposition to choose him as their portion, and for all the blessings which come from his being their God and their being his people.>
<8:11 All shall know me; he would by his word and Spirit impart to them such knowledge of himself as should incline them to walk in his ways.>
<8:12 Merciful to their unrighteousness; pardon their sins, and not so remember as to punish them.>
<8:13 Made the first old; declared it to be old, and as such approaching its end, as the writer immediately proceeds to show. Momentous truth is often conveyed in the Bible by a single word, a change of which would greatly alter the sense, and give a different meaning to what is revealed. Hence the Holy Ghost directed the writers of the Bible not only what to write, but how to write it in order to convey exactly his meaning; and they spoke and wrote not in words which man's wisdom taught them, but which the Holy Ghost taught them. 1Co 2:13.>
<9:1 A comparison is now introduced between the priestly services of the first covenant, and the perfect priesthood of Christ, the Mediator of the new covenant.>
<9:2 A tabernacle; the reference is to the movable tabernacle built by God's direction in the wilderness of Sinai, which had two divisions separated from each other by a curtain--the holy place and the most holy. See Ex 26. 1-37.>
<9:3 The second veil; the first, or outer veil, answered for a door to the tabernacle. Ex 26:36,37. The second, or inner veil, separated the holy from the most holy place. Ex 26:31-33.>
<9:4 The golden censer; in which the high-priest burned incense within the veil on the great day of atonement. Le 16:12. It seems to have been kept in the holy of holies; but however this may have been, it belonged to its furniture, and is properly reckoned to it. Le 16:12. The ark; Ex 25:10-16. The golden pot; Ex 16:33,34. Aaron's rod; Nu 17:5,8,10. Tables of the covenant; the two tables of stone containing the ten commandments. Ex 25:21; Ex 40:20. When Solomon removed the ark into the temple which he had built, there was nothing in it but these two tables. 1Ki 8:9; 2Ch 5:10. But it would seem that it originally contained the pot of manna and Aaron's rod.>
<9:5 Cherubim; Ex 25:18,22.>
<9:6 Always; daily, habitually. First tabernacle; the first apartment, called the holy place.>
<9:7 The second; second apartment, called the most holy. Once; that is, on one day. He entered the most holy place on that day several times. Le 16:12,15. Blood; that of the victims offered in sacrifice. Le 16:2-19,34.>
<9:8 Into the holiest of all; that is, into God's presence. It was in the most holy place between the cherubim that he had his earthly dwelling-place under the Mosaic economy. See not to chap Heb 6:19. It was not yet revealed how men could approach God with acceptance. As a sign of this, his earthly abode was concealed by a veil, and could be approached only once a year by the high-priest, and that not without blood. But when Christ died the veil was rent, Mt 27:51, and thenceforward all his disciples became "a royal priesthood," having admission, through his blood, into the true holy of holies. The first tabernacle; that is, as in verses Heb 9:2, 6, the outer tabernacle, which represents the whole Mosaic dispensation. Was standing; that is, standing as a valid ordinance of God's appointment, and thus barring the way to the holy of holies. This continued till the rending of the veil at Christ's death. The Jewish ritual was full of meaning. God designed by it to teach men their pollution by sin, their need of spiritual cleansing, and the way in which this would be obtained, through the shedding of the blood of Christ and the renewing influences of his Spirit. Many were led by it to depend on Christ and obtain salvation through him. Heb 11:13-16.>
<9:9 A figure; a shadow of good things to come under the Christian dispensation. Perfect, as pertaining to the conscience; it could remove uncleanness and guilt only in a typical way. It had no power to quiet the conscience by removing its sense of guilt.>
<9:10 Stood--in; consisted of. Reformation; the new and better order of things under the Christian dispensation.>
<9:11 Of good things to come; of the substance of those good things of which the rites of the Mosaic economy were only the shadow. Compare chap Heb 10:1. By a greater and more perfect tabernacle; to be connected immediately with the words, "he entered in once," verse Heb 9:12. The meaning is, that just as the Jewish high-priest entered by the way of the earthly tabernacle into the earthly holy of holies, so Christ, our great High-priest, has entered through the tabernacle of the heavens not made with hands, into the true holy of holies above, there to present before God not the blood of bulls and goats, but his own blood, as an expiation for the sins of his people.>
<9:13 The ashes of a heifer; Nu 19.1-10. To the purifying of the flesh; to the removal of outward and ceremonial defilement; the flesh here representing that which is outward in man, as distinguished from that which is inward and spiritual.>
<9:14 The eternal Spirit; the Holy Spirit, given him without measure, and under whose influence he offered himself a sacrifice for the sin of men. Purge your conscience from dead works to serve the living God; cleanse your consciences from the guilt and pollution of sin; make you spiritually alive, and enable you to offer the spiritual living sacrifice of holy obedience to God. Though the Jewish ritual has ceased as a mode of worship, yet its usefulness will continue to the end of time. It shows the evil nature of sin, the way of salvation from it through faith in Jesus Christ, the object of his death as an atoning sacrifice for sin, and the safety and blessedness of all who trust in him. Ga 3:24.>
<9:15 For this cause; in view of what has just been said of the superior efficacy of his priesthood. The new testament; the words "covenant" and "testament" are, in the original, the same. The new covenant, of which Christ is the Mediator, is also a testament when considered as ratified and made valid by his expiatory death on the cross. For the redemption of the transgressions; for their forgiveness through the payment of a ransom. The power of Christ's expiatory sacrifice extends backward to the beginning of the world, as it does forward to its end. They which are called; the called of God of all ages, before and after Christ's advent.>
<9:18 Whereupon; for which reason. The first testament; or covenant; God's arrangement with his people at Sinai. Dedicated without blood; it was ratified by the blood of the sacrifice, which typified Christ, who ratified the second covenant with his own blood.>
<9:19 When Moses had spoken; Ex 24:4-11.>
<9:20 Testament; covenant. See note to verse Heb 9:15.>
<9:21 He sprinkled--all the vessels; Ex 29:12,20,36.>
<9:22 Purged; purified. Le 4:20,26,35; 17:11. As there can be no remission of sin except through the shedding of the blood of Christ and the atonement he has made, those who continue to reject him must remain under the guilt of unpardoned sin for ever.>
<9:23 Patterns of things in the heavens; the tabernacle and its furniture, typical of the true heavenly tabernacle. See note to verse Heb 9:11. Better sacrifices; the blood of Christ, which cleanses the conscience of all who believe from sin, and thus prepares them to enter with Christ their forerunner into heaven, the true holy of holies.>
<9:26 In the end of the world; in the end of the ages; in those "last days" by which the Hebrew prophets represented the then distant future of the Christian dispensation. See note to 1Co 10:11. To put away sin; to expiate it, and thus open the way for deliverance from its punishment, pollution, and power.>
<9:27 As it is appointed unto men; he points out in these words the agreement between the one death of men who are to be redeemed, and the one death of their Redeemer, the man Christ Jesus.>
<9:28 To bear the sins of many; to die on account of them, in the room and stead of sinners; the just for the unjust. 2Co 5:21; 1Pe 3:18. Them that look for him; his people, who expect his coming to judgment. Mt 25:31-46. Without sin; not as before to suffer for sin, but to give his people free, full, and everlasting salvation. As Christ has borne the sins of his people, and is coming for their deliverance from all evil and their introduction to the eternal enjoyment of all good, they ought to be ever rejoicing; giving thanks to God through Jesus Christ, and adoring him who, though he was rich, for their sakes became poor, that they through his poverty might be for ever rich.>
<10:1 The law; the ceremonial law, or Jewish economy. A shadow; an emblem of the blessings of the gospel, but not the blessings themselves, or even an exact likeness of them. The comers thereunto; to the service prescribed by the law. Perfect, that is, as elsewhere expressed, "perfect as pertaining to the conscience," chap Heb 9:9. See also below, verses Heb 10:2, 22. It could not cleanse the consciences of the worshippers from a sense of guilt.>
<10:3 A remembrance again made of sins every year; showing that the sins of those who offered sacrifices have not yet received a true expiation.>
<10:4 Should take away sins; by making an expiation for them. The sincere offerers of these victims under the law did indeed receive forgiveness; but it was by virtue of the atonement of Christ, which the Jewish sacrifices prefigured. The sacrifices under the Old Testament were not an atonement for sin, but typical of the atonement which was to be made, and pointed the believing offerer to the sacrifice of Christ.>
<10:5 He; Christ. Sacrifice and offering; such as were presented under the law God no longer desired, Ps 40:6-8; a psalm which had its fulfilment in David only in a lower and typical way, but was prefectly fulfilled in Christ the great antitype. But a body has thou prepared me; the quotation is made from the Greek version of the Seventy. The Hebrew is, "Mine ears hast thou opened," that is, to hear and do thy will. How the difference has arisen is not known. But in both the essential idea is, that the Messiah makes a perfect devotion of himself to the Father to do his will.>
<10:7 The volume of the book; the Scriptures, which foretold the coming of Christ.>
<10:8 Above, when he said; that is, after he had first said, "Sacrifice and offering," etc.--"then said he," etc., verse Heb 10:9.>
<10:9 The first; the sacrifice of the law. The second; Christ, doing the will of God in his obedience, sacrifice, and death.>
<10:10 By the which will; of God as done by Christ, especially in his suffering and death, believers are justified and sanctified. Christ crucified as an atonement for sin is the great subject of the Old Testament scriptures. Their principles and precepts, their rites and ceremonies, their sacrifices and offerings, their predictions, declarations, and promises have reference to him; and one who does not see them in this light will never apprehend the fulness, or duly appreciate the perfection of their meaning.>
<10:11 Oftentimes; morning and evening daily.>
<10:12 This man; Christ. On the right hand of God; in an exalted state of glory, which is evidence that his atonement once for all is accepted, and is efficacious in securing the salvation of all who believe.>
<10:15 The Holy Ghost--is a witness; to the above-mentioned truths, by what he has said in Jer 31:33-34. The testimony of the Holy Ghost in the Old Testament is in accordance with his testimony in the New. It is equally a part of God's revelation to men, and without understanding it, men cannot be skillful in the word of righteousness, or well fitted to communicate a knowledge of it.>
<10:17 And their sins; supply before these words, Then said he.>
<10:18 No more offering; no need of any further atonement.>
<10:19 Into the holiest; into the true holy of holies; that is, into God's presence in heaven. By the blood of Jesus; which has already been presented there in our behalf.>
<10:20 Living way; of which life is the attribute--which conducts those who are spiritually alive to life eternal; in contrast with the way of dead works, in which those who are dead in trespasses and sins walk onward to eternal death. Consecrated for us; initiated and dedicated as a new way in our behalf. Through the veil, that is to say, his flesh; as the earthly holy of holies was entered through the veil, so we have access to the heavenly holy of holies, that is, to God's presence in heaven itself, through the flesh of Christ offered as our propitiatory sacrifice for sin.>
<10:21 The house of God; his spiritual household, the church.>
<10:22 Draw near; to God on his throne of grace. Hearts sprinkled--bodies washed; the reference is to the consecration of the Levitical priests by the sprinkling of blood and the washing of water, Le 8:6,23,24,30, which shadowed forth the true consecration of believers to their spiritual priesthood by the sprinkling of the blood of Christ, and the washing of regeneration and renewing of the Holy Ghost. Tit 3:5; 1Pe 1:2; 2:5. In approaching God, Christians should discard and reject all mediators except Jesus Christ. They need no other; and to trust in another is to reject him.>
<10:23 Hold fast; by continuing steadfast in the belief of the truths of the gospel and in the practice of its duties.>
<10:24 Provoke; excite each other to abound more and more in love and good works.>
<10:25 The assembling; for public and social worship. Exhorting one another; to continue in steadfast adherence to truth and duty. The day approaching; when Christ will save his friends and destroy his foes. Assembling for social worship is essential to the promotion of the divine glory, to the greatest progress in holiness, and to the highest usefulness among men.>
<10:26 Sin willfully; by renouncing Christ after having embraced him, and rejecting his gospel after having known and acknowledged it to be his. Chap Heb 6:4-8. No other atonement will ever be made, and if we reject this after having known its efficacy, and willfully turn away, refusing to trust in it for salvation, we shall perish.>
<10:28 Died without mercy; De 13:6-10.>
<10:29 He be thought worthy; who has been set apart to the service of Christ, and yet treats him as a vile malefactor, and despitefully spurns the blessed influences of his Spirit.>
<10:30 That hath said; De 32:35,36.>
<10:31 To fall; especially after such aggravated transgressions. Of the living God; as a just, almighty, and eternally avenging God. The knowledge of that certain and awful destruction which awaits believers if they renounce Christ, is a powerful means of preventing it, and one which God blesses in keeping them by his mighty power through faith unto salvation.>
<10:32 Call to remembrance; remember the grace of Christ, which sustained you in your former trials.>
<10:34 Ye have; for yourselves, in heaven. Substance; possession.>
<10:35 Confidence; in the ability and willingness of Christ to support, deliver, and save. Great-reward; in peace of mind here and endless glory hereafter.>
<10:36 Patience; in suffering as well as in doing the will of God. The promise; of eternal life.>
<10:37 He that shall come will come; an application to the coming of Christ of the promise made in Hab 2:3, where the coming is also one that has in view the destruction of the oppressors of God's people, and their salvation.>
<10:38 The just shall live by faith--no pleasure in him; quoted for substance from the Greek version of Hab 2:4. Draw back; give up confidence in Christ, deny him, and renounce his cause to escape suffering, or for any other reason. No pleasure; God will abhor him.>
<10:39 We; true Christians. To the saving of the soul; Job 17:9; Joh 4:13,14; 10:28,29; 1Pe 1:2-9. The assurance God has given that he will keep his people in the floods of tribulation, however high they may rise, and in the fires of affliction, however fiercely they may burn, is suited to inspire strong and living confidence in him, and firm, energetic, persevering devotion to his service.>
<11:1 At the close of the preceding chapter, mention was made of "them that believe to the saving of the soul." Now follows a description of faith and an illustration of its power from the example of the ancient worthies. Substance; the Greek word has two distinct meanings: first, as rendered by our version, substance; the meaning will then be, that faith is that which gives to things hoped for subsistence in the views and feelings of the soul, and leads it to regard and treat them as real; secondly, confidence, as in 2Co 11:17. According to this, faith is the firm persuasion of things hoped for. The evidence of things not seen; their demonstration, that which sets before the mind unseen realities as if they were seen. Faith is a glorious reality and mightily efficacious. It works powerfully, and produces effects which nothing else can. It is in the highest and best sense rational, and is as essential with regard to things unseen, as the eye is to things seen.>
<11:2 A good report; commendation from God and good men.>
<11:3 We understand; through God's testimony. Worlds; heaven and earth. Things which are seen; the whole visible universe. Things which do appear; things visible to sense. The matter itself of which heaven and earth are made was called into being by God's power, and afterwards reduced to order and beauty. Faith quickens, purifies, elevates, and ennobles the human soul. It raises it to higher spheres, gives it keener vision and a purer atmosphere, enables it to look backward and forward, above, beneath, and around, and avail itself to an untold extent of the length and the breadth, the height and the depth of the vision and knowledge, the wisdom, grace, and joy of God.>
<11:4 More excellent sacrifice; because offered in a more excellent spirit. There is probably a reference also to the kind of offering. It was not merely a thank-offering, like that of Cain, but a propitiatory sacrifice. Testifying of his gifts; expressing in some visible form his approbation of them. Ge 4:4-7. Yet speaketh; by his example and its effects. Two persons may engage in the same external worship and yet their service be totally different in the sight of God. Whatever is done, in order to be accepted of him, must be done with faith, in spirit and in truth.>
<11:5 He pleased God; by walking with him. Ge 5:24. He had confidence in him, lived in communion with him, opened his heart to him, and consulted him as his bosom-friend.>
<11:7 Moved with fear; because he believed God's word that the flood would certainly come. Ge 6:14-22. By the which; by which faith of his, with its accompanying fruits. He condemned the world; his example of faith condemned their unbelief. The righteousness which is by faith; the righteousness which God gives through faith. See note to Ro 1:17. Fear is a powerful means of the salvation of men. God designs to awaken it, and it is right that it should have influence. There is great reason for it, and he who attempts to show that there is not, acts against God and against the best interests of mankind.>
<11:8 Not knowing whither he went; God's words were, "Unto a land that I will show thee," Ge 12:1; and such was his confidence in God that he was willing to go anywhere, as God should direct. It is not necessary for us to know all that God will do with us, in order to trust in and obey him; or to be able to see the reasons of his declarations, in order to believe them; or to understand the manner in which his promises can be accomplished, in order to expect their fulfilment.>
<11:9 As in a strange country; he bought no land except what he wanted for a burying-ground, but lived as a stranger in tents, expecting his permanent abode and possessions in heaven. Ge 13:3,18; 18:1,9.>
<11:10 A city which hath foundations; which hath everlasting foundations; that heavenly city which God himself has built for those who love him. Chap Heb 12:22; 13:14. Old Testament saints had knowledge of a future state, and expected their reward in another world.>
<11:11 11, 12. Sarah; Ge 21:1,2; 22:17.>
<11:12 Great events for this world as well as the future, depend on the exercise of faith in God; and things which affect vast multitudes for time and eternity, are accomplished through its influence, which would otherwise be impossible.>
<11:13 The promises; the things which God had promised. Embraced them; looked forward to the fulfilment of the promises with earnest desire and confident expectation. Confessed that they were strangers and pilgrims; see Ge 23:4, where the literal pilgrimage of Abraham shadows forth this life as a pilgrimage; and so it is spoken of by Jacob, Ge 47:9, and still more fully afterwards by David, 1Ch 29:15.>
<11:14 A country; which they had not found and could not find in this world.>
<11:16 God is not ashamed; because they place such confidence in him and desire such pure and elevated joys, he has prepared for them a permanent abode and unending bliss in heaven. God is ashamed of those who have no confidence in him and prepares for them no habitation in heaven. He will not acknowledge them as his people, Mr 8:38, or bring them to his blest abode.>
<11:19 In a figure; when raised alive from the altar where he expected him to die. There are no difficulties in believing God's declarations and obeying his commands, over which faith cannot triumph.>
<11:20 Things to come; which God had promised, and which Isaac confidently expected. Ge 27:27-40.>
<11:21 Blessed both the sons; Ge 48:5-20. Upon the top of his staff; Ge 47:31; the quotation is from the Greek version of the Seventy. The same Hebrew letters, according as they are differently pronounced, may signify bed or staff. Taken either way, the sense of the passage is substantially the same. Faith lives and worships God in death.>
<11:22 The departing of the children of Israel; out of Egypt, because God had promised it. Ge 50:24,25.>
<11:23 Proper; beautiful. Not afraid; to disobey the king's command, because they trusted in God to protect them. There are cases in which to obey civil rulers is wrong. In such cases, faith will keep even a woman from obeying the most despotic king.>
<11:25 Affliction with the people of God; because he expected in so doing to receive the blessings which God had promised them.>
<11:26 The reproach of Christ; here, and in chap Heb 13:13, the reproach which Christ in all ages bears in the person of his covenant people, as he once bore it in his own person; for what is done to his people is done to him. Compare Mt 10:40; 18:5,6; Lu 9:48; Lu 10:16, and especially Mt 25:34-45. Of this reproach each disciple must for Christ's sake bear his share, before he can share with Christ in his glory. 2Ti 2:12. The reward; to be given him by God. No earthly sacrifices are too great for faith to make in order to obey God, and no loss is encountered in such a cause which faith does not esteem unspeakable gain.>
<11:27 Forsook Egypt; Ex 12:31-51. As seeing him who is invisible; as seeing by faith the unseen God--a beautiful illustration of what is said, verse Heb 11:1, of faith. Faith has eyes to see invisible things, and a heart to feel their power. It has a head to plan, a tongue to speak, and a hand to work for God.>
<11:28 The passover, and the sprinkling of blood; Moses observed them as God directed, expecting according to His promise, that in so doing he and the Israelites would be safe. Ex 12:21-30. Faith regards the blood of Christ as the foundation of human hope, and looks to it as the only safeguard from the destroyer.>
<11:29 They passed through the Red sea; trusting in God to preserve them. Ex 14:22-29.>
<11:30 Compassed about; with confidence that God would cause the walls of the city to fall as he had said. Jos 6:15-20.>
<11:31 Rahab; believed that what God had spoken concerning Israel would be accomplished, and she acted accordingly. Jos 2:1-21; 6:23.>
<11:32 Gideon; Jud 6.1-8.35. Barak; Jud 4.1-5.31. Samson; Jud 13.1-16.31. Jephthae; Jud 11.1-12.15. David; 1Sa 16:1-13. Samuel; 1Sa 1:20. The prophets; Mt 5:12.>
<11:33 33, 34. For examples, see the references. 33-40. No victories ever won compare with those of faith. Its triumphs no earthly tongue can speak or pen describe. They are written in the book of life, and will be told with immortal tongues, by multitudes which no man can number, in strains of glory rising higher and higher, and growing sweeter and sweeter to endless ages.>
<11:35 Women received their dead; 1Ki 17:17-23; 2Ki 4:32-37. Others were tortured; from this point onward examples are included of those who lived after the record of the Old Testament was closed, some of whose sufferings for the truth's sake are recorded in the books of Maccabees, and in Josephus' account of the same times. A better resurrection; to a life of everlasting glory.>
<11:39 A good report; they are in Scripture commended as good men, and their faith by which they persevered in duty held up as worthy of imitation to all succeeding ages. The promise; the great thing promised, namely, the Messiah and the blessings of the gospel.>
<11:40 Some better thing; the fulfilment of God's promises in the coming of Christ and the blessings which he conferred. Not be made perfect; without the fulfilment of these promises, which we witness, and in the faith of which they lived, and died, and went to glory.>
<12:1 Compassed about--cloud of witnesses; the reference is to the Grecian games, in which the racers were surrounded by a vast multitude of spectators. Here the witnesses are those who have themselves run the heavenly race and obtained the reward of faith. Every weight; every thing which can hinder our progress in the way to heaven, just as the earthly racers lay aside every incumbrance, especially the sin to which we are most exposed.>
<12:2 Looking unto Jesus; not merely as an example, but also as the author and finisher of faith and of all which was needed for perseverance in duty, even to eternal life. The joy; of redeeming multitudes which no man can number from eternal sinning and suffering, and raising them to eternal holiness and bliss. Despising the shame; the shame of being crucified. It is right to regard our own happiness, to be influenced by the hope of future reward, and for the sake of obtaining it to perform labors, make sacrifices, suffer trials, and endure, when called to it, even death itself, that we may follow Christ and be partakers of his joy.>
<12:3 Consider him; meditate much on the character and work of Christ, especially his patience under sufferings, that you may be strengthened and encouraged in following his example.>
<12:4 Ye have not yet; been called as Christ was to suffer death for resisting sin.>
<12:5 The exhortation; Pr 3:11,12; Re 3:19.>
<12:7 Chastening; trials designed to correct your faults and make you better.>
<12:8 All; the children of God. Are ye bastards; treated as such; your faults are not corrected, but you are left to go unreformed to ruin.>
<12:9 Fathers of our flesh; earthly parents. Father of spirits; God.>
<12:10 After their own pleasure; as they chose or thought best. Be partakers; become holy like him. God never sends trials because he has any pleasure in afflicting his people, but to make them more useful and happy than they would be without them. Hence a cheerful and hearty submission is required not only by the glory of God, but by our own highest good.>
<12:11 The peaceable fruit of righteousness; the chastisement yields, like a good tree, the good fruit of righteousness, which always has for its companion "the peace of God which passeth all understanding." Php 4:7.>
<12:12 Lift up; encourage and animate the desponding. Isa 35:3,4.>
<12:13 Make straight paths for your feet; walk in the plain way of duty, and that not merely for your own sake, but for the sake of feeble and halting among your brethren; that they, by your good example, may be kept in the right way, and healed of their spiritual infirmities.>
<12:14 Follow peace with all; so far as duty will permit. See the Lord; dwell with or enjoy him.>
<12:15 Any man; that is, as the connection shows, any man who belongs to your Christian community. Fail of the grace of God; of his grace which bestows eternal life, by being found at last unholy and unprepared for heaven. Any root of bitterness; any doctrine or practice adapted to lead men to apostatize from Christ and perish. The words quoted, from De 29:18, were originally applied to such a root of bitterness, consisting in apostasy from Jehovah to idolatry.>
<12:16 Fornicator--profane person; examples of the "root of bitterness" just referred to. A profane person is here one who, like Esau, despises sacred things and gives up spiritual blessings for sensual enjoyments. One morsel of meat; Ge 25:29-34. Birthright; right by birth to high temporal and spiritual blessings.>
<12:17 No place of repentance; whether we refer the word repentance to Esau, as some do, or with others, to Isaac, the sense remains substantially the same. In the former case the meaning will be that Esau could not make his own repentance avail to change his father's mind; in the latter, that he could not induce Isaac to repent by taking the blessing of the birthright from Jacob, and giving it to him. He had sold it for a mess of pottage, and it was gone for ever. Ge 27:34-40. So would be the blessings of following Christ, if they should renounce him to escape suffering or to enjoy worldly good. Great care, watchfulness, and prayer, are needful even in Christians, lest they should fail of heaven; and great effort is needful in sinners, however high their privileges and however enlightened or closely connected with Christians they may be, in order to enter and pursue the way that leads to life.>
<12:18 18-29. Now follows an exhortation which contains, first, and encouragement drawn from the gracious character of the Christian dispensation, as contrasted with the severity of the Mosaic law; secondly, a warning against apostasy under such a glorious dispensation, in view of its greater guilt and severer punishment. The mount; Sinai and the terrors which surrounded it at the giving of the law. Ex 19:9-25; 20:1-22. Here, as in Ga 4:24,25, Sinai represents the whole Mosaic economy. That might be touched; the reference is not merely to its material nature, but to the peril of touching it. Verse Heb 12:20.>
<12:20 If so much as a beast touch the mountain; much more a man. Ex 19:12,13,21-24. This prohibition shadowed forth the distance from himself at which the holy God, under the Mosaic economy, kept sinful men. Compare chap Heb 9:8.>
<12:22 Unto mount Zion--the city of the living God, the heavenly Jerusalem; to the true spiritual Zion and Jerusalem, of which the earthly Zion with its city was an emblem; that is, to the privileges, hopes, and blessings of the Christian dispensation and the holy family of God under it. Compare the words of the apostle, Ga 4:26: "Jerusalem which is above is free, which is the mother of us all." An innumerable company of angels; who make a part of God's universal family, of which Christ is the head. Eph 1:10; Col 1:20.>
<12:23 The general assembly and church; here distinguished from "the spirits of just men made perfect;" probably meaning therefore the church on earth, so far as it consists of true believers. The first-born; the word in the original is plural. It describes either all God's true children, as each admitted, in and through Christ, to the privileges of first-born sons, that is, to a preeminent place in God's favor; or, as some think, the more eminent among them, as patriarchs, prophets, and apostles. Which are written in heaven; enrolled there in the Lamb's book of life. The spirits of just men; who await in God's presence the resurrection of the just. Made perfect; they have gone through the conflict with sin and suffering, obtained the victory, and been made perfect in holiness and blessedness; not in the sense of having reached the consummation of their bliss--which is reserved for the final resurrection--but in the sense of being for ever freed from sin and suffering.>
<12:24 The blood of sprinkling; which cleanses our consciences from the guilt and defilement of sin, and thus speaks peace to them. Chap Heb 9:14; 10:22; 1Pe 1:2. That of Abel; which called to God for vengeance. Ge 4:10.>
<12:25 Him that speaketh; in the revelations, the promises, and the threatenings of the gospel. Refused him that spake on earth; apostatized from the Jewish religion revealed by Moses. De 13:6-10. Him that speaketh from heaven; God, by Jesus Christ. Chap Heb 1:2. The responsibilities of men are in proportion to their blessings; and if they abuse or neglect them, they will proportionably enhance their condemnation.>
<12:26 Then; when he gave the law at mount Sinai. Ex 19:18. Not the earth only, but also heaven; Hag 2:6,7, where the words are, "I will shake the heavens, and the earth, and the sea, and the dry land; and I will shake all nations, and the Desire of all nations shall come." It is a shaking and removal of every thing that is in its nature transitory and perishable, not merely the old Mosaic dispensation, but also every human power opposed to the kingdom of Christ. Compare, for the figure, Isa 13:13; Joe 3:16; Mt 24:29, and the notes on those passages.>
<12:27 Things that are shaken; that is, as the margin, things that can be shaken. See the note to the preceding verse. Things that are made; nearly equivalent to things "made with hands," and therefore transitory. Chap Heb 9:11. Which cannot be shaken; the kingdom of Christ and the eternal spiritual blessings connected with it. This shaking is the thrice repeated overturning of Eze 21:27. It covers the whole history of Christ's kingdom from its beginning to its perfect establishment.>
<12:28 Let us have grace; though the grace by which alone we can render acceptable service to God is his gift, yet we are responsible for possessing it, since it is freely offered to all, and all will have it who do not repel and reject it by a disobedient spirit. Serve God; perseveringly, to whatever troubles it may expose us. Godly fear; having respect to all God's commandments. Ps 119:6; Jer 32:40.>
<12:29 A consuming fire; De 4:24. He is such to all rejecters of our Lord Jesus Christ, especially those who have apostatized from him. Hence all who have set out in the way to heaven should persevere, whatever trials may assail them, till faith is swallowed up in vision, and hope in endless joy. All good reasons are on the side of perseverance in obeying God, trusting in Christ for what is needed to do this and to be accepted in it. The contrary course is most unreasonable and wicked, will be condemned by God and all good beings, and will fill those who pursue it with consuming terrors for ever.>
<13:1 Love to Christians on account of their likeness to Christ is a fruit of the Spirit, and an evidence of being born of God. It is also a means of promoting our love to Christ and the enjoyment of his presence. God is love; and he that dwelleth in love dwelleth in God, and God in him. 1Jo 4:16. This is a reason why Satan hates Christian love, and so often employs such as speak lies and sow discord among brethren to prevent it; and also a reason why such persons are mentioned in the Bible as children of the devil, and as one of the seven abominations which the Lord abhors. Joh 8:44; Pr 6:19.>
<13:2 Entertained angels; Ge 18:2-19; 19:1-3.>
<13:3 Remember; sympathize with, pray for, and be ready, as you have opportunity, to assist them that are in bonds; whatever be the kind of bondage, especially those who are bound or imprisoned on account of their religion. In the body; and of course liable to similar trials. Love to Christians for Christ's sake will lead all who possess it deeply to sympathize with such as are in bonds, to pray for them, and in all suitable ways endeavor to benefit them.>
<13:4 Honorable; right, proper, and for ministers of the gospel as well as others. As marriage is God's institution for the happiness of man and the prevention of fornication and other abominable vices, he who forbids or discourages it increases the temptations to these crimes, and exposes himself to the just indignation of God.>
<13:5 Conversation; manner of life. Covetousness; inordinate regard for money or such things as money will procure. I will never leave thee, nor forsake thee; the words occur in De 31:6, as a promise to all Israel, and again in 1Ch 28:20, as a promise to Solomon. The apostle has simply put them in the first person. Christians, in principle, precept, and practice, should show their abhorrence of covetousness, should be contented with the allotments of Providence, be grateful for mercies, and never fear the want of any needful good, for the Lord has engaged to supply them. Ps 34:4-10.>
<13:7 Them which have the rule; your leaders and guides; those who have proclaimed to you the will of God. The end; of their earthly course; their peaceful, happy, joyful death. Ac 7:59,60; Php 1:23. These words should not be connected with the following verse, as is plain from the construction of the original.>
<13:8 Jesus Christ--for ever; rather, Jesus Christ is the same, etc. Of course he can sustain, comfort, and bless you, as he did them.>
<13:9 Be not carried about; from one opinion to another; or, according to another reading, carried away, namely, from the right path. Adhere steadfastly to your steadfast Saviour, and the truth concerning him as ye have received it. Not with meats; the Jewish distinctions of meats, and the whole ceremonial law connected with them. The words intimate that it is the grace of Christ alone, not these carnal ordinances, that has power to establish the heart. To treat external rites and ceremonies as the chief thing does much evil and exposes men to the loss of their souls, because it tends to prevent their reliance on the Lord Jesus Christ.>
<13:10 We; Christians. Have an altar; a spiritual altar. We are spiritual priests, and partake by faith of Christ's body sacrificed for us; from which they are excluded who still depend on Jewish sacrifices.>
<13:11 Whose blood is brought into the sanctuary; the reference is to the bullock and goat that were offered as sin-offerings in the great day of atonement, and whose blood was carried by the high-priest into the holy of holies, Le 16:27. In being thus burned without the camp they typified the sacrifice of Jesus, who suffered without the gate of Jerusalem, which city corresponded to the camp in the wilderness.>
<13:12 With his own blood; carrying it as their great High-priest into the true holy of holies, that is, into God's presence in heaven. Chap Heb 9:12,24. Without the gate; Joh 19:17,18.>
<13:13 Let us go forth--without the camp; some, taking the Israelitish camp as a symbol of the Mosaic economy, suppose the meaning to be, Let us forsake Judaism, cleave to Christ, and suffer with him. But the more natural meaning of this verse is, Let us follow Jesus in his shame and suffering. Bearing his reproach; bearing it with him, as members of his body. See note to chap Heb 11:26. Christians should not fear any reproach or shame which they are called of God to suffer for his sake, but like Christ should cheerfully endure any cross, despising the shame, that they may with him sit down on the right hand of the majesty on high; remembering that their shame will be short, and their honors eternal.>
<13:14 Here have we no continuing city; our stay on earth will be short. This is a reason why we should cheerfully bear reproach with Jesus, that we may share with him the glory of the heavenly city that is to come.>
<13:15 By him; as our great High-priest, not by Jewish priests on Jewish altars. The sacrifice of praise; the spiritual thank-offering of praise, which the Mosaic thank-offerings shadowed forth. The fruit of our lips; quoted after the Greek version from Ho 14:2, where the Hebrew has, "the calves of our lips;" that is, sacrifices of praise. Eph 5:19,20; 1Th 5:16,18.>
<13:16 To communicate; impart blessings as you have opportunity to the needy. It is not enough for men to be pious, devotional, and grateful: they must also be beneficent, disposed to communicate of their blessings to others. This they are prone to forget, and they need often to be reminded that it is with such things God is well pleased. They would thus give evidence that their natural selfishness, which if continued will ruin them, is in a way of being subdued.>
<13:17 Them that have the rule; who guide you, by making known to you the will of God. Submit yourselves; to be governed by his will, which they declare. Watch for your souls; their object is to promote your salvation, by obeying God to whom they are accountable. With joy; in having been instrumental of saving you. Unprofitable for you; if you refuse to follow their guidance when they point out the path of duty, you not only grieve them, but injure yourselves and incur the wrath of God.>
<13:18 Honestly; uprightly, in a manner suitable to a minister of the gospel and an inspired apostle. Enlightened ministers of Christ often express a desire for the prayers of Christians on earth, but never ask or desire others to ask for them the prayers of Peter, Paul, Mary, or any of the saints in heaven.>
<13:19 Restored to you the sooner; be sooner able to visit you. Prayer is efficacious not only with regard to spiritual, but temporal things. It often enables persons to do what they otherwise could not do, and to confer and receive blessings of which they and others without prayer would fail. But in order to receive the full benefits of the prayers of others, men must pray themselves; and not only for themselves, but also for their fellow-men; and must be disposed in this way to confer the blessings on others which they wish to receive from them.>
<13:20 Through the blood of the everlasting covenant; these words are best connected with the preceding part of the verse. It was by virtue of the expiatory blood of Christ, by which he ratified the everlasting covenant of grace, that God raised him from the dead and exalted him to universal dominion. The question here is not one of mere power, but of fitness. It was meet that, in view of his propitiation for the sins of the world through the bloody death of the cross, God should exalt him, as he did by his resurrection and ascension to heaven. Compare Php 2:9-11.>
<13:21 Every thing thought, felt, or done by men which is holy and acceptable to God, is the fruit of his working in them both to will and to do. It comes to them in consequence of the death of Christ as a propitiation for their sins and the sins of the world; and to him all who feel and act rightly will give glory for ever.>
<13:23 If he come shortly; to me; for though set at liberty he was yet absent from the writer.>
<13:24 All--and all the saints; the ministers and brethren of the churches. They; the Christians in Italy, whence this epistle was written, the great object of which was to enable the Jewish Christians rightly to understand the Old Testament, especially its rites and ceremonies, and to persuade them to persevere in their attachment and obedience to Jesus Christ.>
\kniha{James}
\zkratka{Jas}
<1:2 Temptations; trials suited to develop their character, and if rightly borne, to make them better. God does not afflict or expose his children to temptation because he takes pleasure in their distresses or exposures, but for their benefit, that they may be made wiser and better; and although no trials or exposures in themselves are joyous, but grievous, yet as they are the means when rightly improved of increasing holiness and usefulness, they should be received not only with submission, but with gratitude.>
<1:3 The trying of your faith; that is, when the trial is rightly endured. Patience; in its usual scriptural sense of steadfast endurance.>
<1:4 Have her perfect work; produce its full and appropriate effects, through your enduring to the end all the trials which God appoints to you. Mt 24:13. Perfect and entire; complete in all parts of the Christian character.>
<1:5 Lack wisdom; to feel and act rightly under all circumstances, especially in trials. To all; who ask according to God's directions. All who have the Bible may be made wise to salvation, and be guided aright in all their concerns. If they are not, it is because they do not aright seek wisdom from the Lord, or knowing his will, do not obey it.>
<1:6 In faith; in confidence that God will do as he has declared, and give to those who thus ask him the wisdom which they need. Nothing wavering; not doubting the truth of his declarations. Like a wave; not fixed or settled in purposes, plans, or efforts.>
<1:8 Double-minded; one who halts between faith and unbelief.>
<1:9 Of low degree; afflicted and depressed in his circumstances. Exalted; spiritually, by being made a partaker of the heavenly inheritance.>
<1:10 The rich; in worldly possessions. Made low; spiritually, by being brought into a lowly and humble state of mind. The apostle exhibits, in this and the preceding verse, the two sides of Christian character which are appropriate to the two conditions of rich and poor.>
<1:11 Fade away; earthly glory is transient; and a man may well rejoice in what leads him to feel this, and secure the glory which is abiding. Thus will the poor be kept from envying the rich, and the rich from glorying in their wealth and despising the poor.>
<1:12 Endureth temptation; bears his trials with a right spirit.>
<1:13 When he is tempted; to commit sin. Neither tempteth he any man; to commit sin: that is not God's design in sending trials, or in any thing he does: what he does is designed to promote holiness and happiness. If men commit sin, or grow worse under any of his dealings, they pervert and abuse them; the fault is theirs, not his. There is that in men which may account for the evil they commit, without ascribing it to God; and as he never tempts any one to commit sin, no one, when so tempted, or if he does comply with it, should attempt to cast any of the blame on God.>
<1:14 Of his own lust; his desire to obtain something which he cannot without doing wrong.>
<1:15 Lust; the inward desire of the soul after forbidden objects, here considered as the parent of sinful deeds. Sin; in the life. Is finished; in its consequences. Death; eternal death, which is, to all who continue in sin, its proper result.>
<1:16 Do not err; in the matter now under consideration, by thinking of God as if he could tempt to sin.>
<1:17 Every good gift--is from above; God is the author of every thing in men which is good, and they are the authors of every thing in them which is evil.>
<1:18 Begat he us; by the regeneration of our souls, and thus made us his spiritual children. With the word of truth; which is the instrument of his Spirit. A kind of first-fruits; the gospel was first preached to the Jews, and the primitive believers were, like the first sheaf offered at the sanctuary, the earnest of the ingathering of all nations to Christ. See Le 23:9-14. As every thing good in men comes from God, and every thing evil from themselves, they should renounce self-dependence, and give God the glory of whatever good they enjoy.>
<1:19 Wherefore; in consistency with your new character as God's children. Swift to hear; the instructions which God gives him. Slow to speak; either by way of usurping the office of a teacher, chap Jas 3:1, or of censure, chap Jas 3:9,10. Both these faults proceed from pride, and are allied to sinful anger, which the apostle next forbids.>
<1:21 All filthiness; of flesh and spirit, 2Co 7:1. Superfluity of naughtiness; malice in the heart flowing out in the life. The engrafted word; the word of divine truth engrafted--or, as some render, implanted--into your hearts, as a heavenly stock which is to bear fruits of righteousness. It is not engrafted in such a way that our freedom and responsibility in admitting or rejecting it are set aside: hence we are exhorted to receive it with meekness. Divine truth received in love and obeyed is efficacious in the salvation of men.>
<1:24 He beholdeth; the gospel shows a man himself. Forgetteth; if a man does not obey the gospel, its impressions on him will be short.>
<1:25 Looketh into; looketh closely into. The apostle means a practical looking, that is, connected with obedience. The perfect law of liberty; the gospel, which gives true freedom to the soul, and is a perfect rule of action. That it delivers the soul from the bondage of the Mosaic law is also true, but that is a view not discussed in the present epistle.>
<1:26 Bridleth not his tongue; does not regulate it by the revealed will of God. Deceiveth his own heart; by thinking that he is pious, when he continues to cherish and indulge a slanderous spirit.>
<1:27 Pure religion; he described religion by its fruits, and that on two sides: first, that of love and mercy; secondly, that of purity from worldly defilements. Fatherless and widows; these represent all objects of Christian compassion and kindness. From the world; from all the enticements to sin which the world offers. That religion which does not govern the tongue and make men beneficent and holy, is not the religion of Christ, and will not secure salvation.>
<2:1 Have not; hold not. The faith of our Lord Jesus Christ; his gospel considered as addressed to our faith. Compare Ac 6:7. With respect of persons; let not the outward condition of persons regulate your judgment of their character, or your treatment of them. The Lord Jesus Christ is the fountain of honor as well as blessedness; and those are the most honorable and happy who most resemble him, and are most faithful in his service.>
<2:4 Judges of evil thoughts; under the influence of evil thoughts. It is wrong to judge of persons by outward appearances merely; and the manifestation of such a spirit does evil both to those who are guilty of it, and to others.>
<2:5 Chosen the poor; to be his disciples more often than the rich.>
<2:6 Do not rich men oppress you; unbelieving rich men: as much as to say, Why show such partiality to the rich? Are not they your chief persecutors?>
<2:7 That worthy name; the name of Christ.>
<2:8 The royal law; the law of love, called royal by way of preeminence. Compare Mt 22:37-40. Real and ardent love is the best guide to a proper treatment of our fellow-men.>
<2:9 Convinced of the law; shown by it to be transgressors.>
<2:10 He is guilty; he as really violates the law, if he allows himself in one transgression, as if he committed many; and if he continues in it, he will as certainly be condemned.>
<2:12 The law of liberty; see note to chap Jas 1:25. Obedience, if sincere, will be universal; and real love to God, or living faith in him, will lead men to have respect to all his commandments. Ps 119:6.>
<2:13 He--that hath showed no mercy; to others in distress, will have no mercy shown to him. Mercy rejoiceth; exults in the consciousness of its security against judgment; the judgment, namely, of the last day; for then the merciful man will not be condemned. Mt 5:7.>
<2:14 Can faith save him? that kind of faith which is inactive, dead, and never does good? No.>
<2:16 Be ye warmed; to give words only, when deeds are due and in our power, is to offend God and wound our needy brother.>
<2:17 Is dead; it is not the kind of faith which justifies the soul: that is, a living, operative faith, which works by love, purifies the heart, overcomes the world, and produces good works.>
<2:19 Thou doest well; in believing that there is one God, thou believest the truth; but it will do you no good, any more than it does devils who believe the same, unless it lead you to obey him.>
<2:20 Faith without works; is not the kind which Abraham had, and by which he was justified. His faith led him to obey God, even His most difficult and trying commands. Ge 22:9,12. Justifying faith produces good works; and if a man does not love to obey God and do good, he may conclude that he is not in a justified state.>
<2:22 Made perfect; shown to be complete, of the right kind, by producing its appropriate fruit.>
<2:23 The scripture was fulfilled; by the works which Abraham did. His works showed that he believed God in reality, as the Scriptures declared, with a living, and not merely a dead faith.>
<2:24 Not by faith only; not by that faith which is alone, and never produces good works; but by good works a man is shown to have living faith, and to be in a justified state.>
<2:25 By works; they proved that she had faith. Between the teaching of James in this chapter and that of Paul there is perfect agreement. When the question is, What is the ground of justification before God? Paul answers, Faith, and not the works of the law. But when the inquiry is as here, What kind of faith is acceptable to God? both answer, Not a dead faith, but "faith which worketh by love;" in other words, operates to produce good fruits through love, which is "the fulfilling of the law." Ga 5:6; Ro 13:10.>
<2:26 There is no contradiction between the meaning of the apostles James and Paul, with regard to justification. The case of Abraham exemplifies the doctrine of both. Paul treats of faith as justifying us before God: James treats of the fruit, or effects of faith.>
<3:1 Be not many masters; let not many aspire to be teachers or guides in religion; a sin which always abounds where men have the idea that an empty faith without the fruits of obedience is all that is necessary to salvation. We; who exercise the office of teachers. The greater condemnation; or, as the margin, the greater judgment. We shall be subjected to a severer trial; and if found wanting, to a greater punishment. Right views of the responsibility of religious teachers and guides, of the difficulties of their work, the strict account which they must render to God, and the awful ruin which will come on those who are unfaithful, tend effectually to prevent improper aspirations for power in the church.>
<3:2 Offend all; we all offend in many things: of course, in the office of teaching also, where there is especial danger of our offending in word. This ought to deter us from forwardness in arrogating to ourselves this work. Offend not in word; as much as to say, Sins of word are so difficult to avoid that he who can keep himself from fault in this respect is a perfect man; one who has his appetites, passions, and all propensities of body and spirit, under wise control.>
<3:4 Listeth; chooseth.>
<3:5 As everlasting consequences depend upon the use of the tongue, all, and especially ministers of the gospel, should earnestly pray that they may always so speak as shall tend most to honor God and benefit their fellow-men.>
<3:6 Is a fire; setting society in a blaze, like fire in dry matter. Defileth the whole body; when used in spreading moral pollution. The course of nature; produces universal destruction, like a general conflagration. It is set on fire of hell; instigated in its destructive courses by infernal spirits. The devil employs some men more than they are aware of; and things which they would start at, should they hear him utter them, they at his suggestion utter for him, and thus do his work and ripen to be companions with him and his angels.>
<3:9 Bless we God; thus professing our love for him. Curse we men; thus showing that our profession of love towards God is empty and insincere. Compare 1Jo 4:20. Made after the similitude of God; bear his image as rational and immortal beings, and ought therefore to be the objects of our love.>
<3:11 11, 12. The stream must answer to the fountain, the fruit to the tree. The heart that sends forth the bitter water of curses towards men, cannot have the good water of love towards God. Its professions of love and praise must be empty and worthless.>
<3:13 Who is a wise man; would any man show his wisdom? Let him do it by eagerly assuming the office of a teacher, and occupying his tongue with words of contention and bitterness, but by showing out of a good life his good works with meekness of wisdom; with that meekness and gentleness which always belong to true wisdom. Knowledge and wisdom, discretion and meekness, a good understanding of the Scriptures, and a life conformed to their precepts, are essential requisites in ministers of the gospel and guides of the church.>
<3:14 Glory not, and lie not against the truth; by falsely boasting of these as the fruits of true wisdom, which is to slander God's truth.>
<3:15 This wisdom; that which is envious, quarrelsome, and leads to contention.>
<3:17 Pure; in its nature, and in its effects on the person himself and on others. All who are guided by heavenly wisdom will manifest a heavenly spirit; and none have reason to expect acceptance with Christ any further than they have evidence of likeness to him in their temper and conduct.>
<3:18 The fruit of righteousness; that spirit which righteousness produces, leads peaceable men in a peaceable way to disseminate such views as tend to promote peace with God and peace with one another; thus bringing glory to God in the highest, and proclaiming peace on earth, goodwill to men.>
<4:1 A large part of this and the following chapter is addressed to that portion of the Jewish Christian community who had departed from "the doctrine that is according to godliness," and become conformed, in their spirit and conduct, to the exceedingly corrupt and turbulent mass of unbelieving Jews among whom they lived. Wars and fightings; contentions and quarrels, from those of individuals and neighborhoods, to those of provinces and states. The age in which the apostle wrote was one of the most turbulent and seditious on record; and the unworthy members of the Christian community here addressed did not escape its defilements. That war in your members; have their seat in your members, and impel you to fight and war for their gratification. Compare Ro 7:23.>
<4:2 Have not; real satisfying enjoyment, because you do not seek it in the right things or in the right way. Wars are the fruit of human wickedness. Let all men be at peace with God, and obey the command, "Whatsoever ye would that men should do to you, do ye even so to them," and wars will cease.>
<4:3 Receive not; because you do not ask of God with a right spirit, or for the right end. You seek to gratify yourselves; whereas you should seek to glorify God and do good to men.>
<4:4 Ye adulterers and adulteresses; the sin of adultery stands here as the embodiment of all the fleshly lusts in which these unworthy professors of Christianity indulged. The same sin is charged by the apostles Peter and Jude upon the false teachers and their followers, 2Pe 2:10,14,18; Jude 4,7,8. The world; the men of the world, with all the worldly objects to which they devote themselves, "the lust of the flesh, and the lust of the eyes, and the pride of life." Compare 1Jo 2:15,16.>
<4:5 In vain; without any urgent necessity of warning us. The spirit that dwelleth in us; the human spirit that belongs to us as a fallen sinful race. Lusteth to envy; to the exciting of envious desires. It is the constant doctrine of the Old Testament, that men naturally desire what others have, and that this spirit in the form of envy, jealousy, and covetousness, produces immense evil. Ec 4:4; Job 5:2; Pr 14:30; Pr 27:4; Ge 26:14; 30:1; 37:11; Ps 73:3; Ps 106:16. The conduct of men in all ages has shown this doctrine to be true. You therefore ought to take warning, and earnestly strive against its power. Some prefer to read this passage as two questions, thus: "Do ye think that the scripture speaketh in vain? Doth the Spirit that hath dwelt in us"--the Holy Spirit whom God has given to dwell in our hearts--"lust to envy?"--excite lusts that lead to envy?>
<4:6 He; God; or, according to the second of the above interpretations, the Holy Spirit. Giveth more grace; to those who humbly seek him, to overcome this evil propensity. Resisteth the proud; repels those who come in pride and self-sufficiency, trusting in themselves and despising others.>
<4:7 Submit--to God; be resigned to his will, be content with the allotments of his providence, and lay aside all envious desires. Resist the devil; by refusing to do wrong, for the accomplishment of any object. The devil is a living and busy agent, exciting and increasing human wickedness, tempting men to envy, violence, and fraud. But he may be, and he ought to be effectually resisted, by refusing to comply with his temptations.>
<4:8 Draw nigh to God; in prayer for all needed wisdom and strength to resist temptation and to persevere, whatever be the consequences, in doing right. He will draw nigh to you; to sustain, comfort, and provide for you. Cleanse your hands--purify your hearts; be outwardly and inwardly pure. Double-minded; those who are drawn different ways by conscience and passion, and are halting between two opinions.>
<4:9 Be afflicted; in view of your sins, and the judgments of God that are hanging over you. See note to chap Jas 5:1.>
<4:10 Humble yourselves; instead of fighting and warring for worldly emoluments, honor, and pleasure, commit yourselves quietly to God, and he shall lift you up to true honor in his own time and way.>
<4:11 Judgeth the law; he practically condemns it. It is the royal law of love which the apostle has specially in mind, which forbids slander, and every thing else that can injure our neighbor.>
<4:12 One lawgiver; Jesus Christ, and only one. All therefore are bound to yield cheerful, prompt, unwavering obedience to him. Who are thou; frail, sinful, dying man, that thou usurpest the place of Christ, and pronouncest sentence of condemnation on his servants, when thou must thyself soon stand before him in judgment and be treated for eternity according to thy works? Christ is the only rightful Lord and Lawgiver of his church. His disciples are all brethren; and when any one undertakes to lord it over others, he rebels against Christ, and exposes himself to be destroyed.>
<4:13 13, 14. A year--the morrow; all our plans should be made in view of the uncertainty of human life.>
<4:15 Ye ought; to feel your dependence on God for the continuance of life and for every blessing, and to act accordingly. In forming plans for the future, we should never forget our dependence on God, or neglect to seek his guidance and blessing.>
<4:16 In your boastings; of what great things you will do, as if you were able of yourselves to accomplish your plans. Is evil; because it is treating God and yourselves contrary to truth.>
<4:17 It is sin; because it is neglect of known duty. Knowledge of duty increases obligation to perform it; and the neglect of known duty is sin.>
<5:1 Ye rich men; for the class of persons addressed. That shall come upon you; on account of your sins, especially the wicked manner in which you have obtained and used riches. There is a reference here and in verses Jas 5:3,5, to the awful calamity that was about to come upon the Jewish people, and in which these rich men would be involved in common with the rest of their country-men. But this is not the full meaning of the words. That judgment shadowed forth the more dreadful retributions that shall overtake all sinners at the last day.>
<5:2 Are corrupted; riches in those days consisted much in large quantities of grain and clothing, which were liable to become worthless by decay.>
<5:3 Cankered; from having been hoarded up, instead of being used. Eat your flesh; the same rust that bears witness against them shall consume them as fire; that is, shall bring upon them the consuming judgments of God. Treasure; of ill-gotten wealth. For the last days; more literally, in the last days; when the vengeance of God is ready to fall upon you. See note to verse Heb 5:1. Riches bring with them great responsibilities; and to hoard them, instead of using them in doing good, is a great sin, and one which draws down upon their possessor the judgments of heaven.>
<5:4 Crieth; to God for vengeance. The Lord of sabbath; the God of hosts; the Almighty. Persons who work for others have a righteous claim to full compensation; and the withholding of it from them is fraud, which will be visited with divine indignation.>
<5:5 Nourished your hearts; made them fat by luxurious living. Fat is said to have no feeling, and the luxurious have few sympathies. As in a day of slaughter; he compares them to beasts that feed to excess on the very day of slaughter. See note to verse Heb 5:1.>
<5:6 He doth not resist you; after the example of his Lord, he commits his cause to God, knowing that He will execute judgment in his behalf. Compare 1Pe 2:23. With these words the apostle passes from the false professors of Christianity, whose sins he has been rebuking, to the truly humble and godly in Christ, whom he exhorts to patience under the trials that they are enduring.>
<5:8 Stablish your hearts; in the faith and practice of the gospel. The coming of the Lord; for the deliverance of his friends and the destruction of his enemies. The giving of directions by God to his people, as to the manifestation of a Christian spirit under wrongs, was not designed to justify or excuse those who wronged them, but to show the excellence of the Christian religion, and to increase the graces and promote the happiness of believers.>
<5:9 Grudge not one against another; the apostle here rebukes that murmuring and complaining spirit between brethren which has its root in worldly lust and envy. The Judge; who will punish all sin, and render to all their just reward.>
<5:11 The patience of Job; his endurance of trials. The end of the Lord; the happy end to which he brought Job's sufferings. Job 42:10,12.>
<5:12 Swear not; in ordinary conversation. Let your yea, be yea; and your nay, nay; let your yea and nay be steadfast and reliable, so that they shall need no oath to support them. Profaneness is a great sin, every form of which should be carefully avoided by all, and especially by professors of religion.>
<5:14 The elders of the church; who in the days of the apostles were often endued with miraculous powers. Anointing him with oil; as was customary among the Jews. Mr 6:13. In the name of the Lord; Ac 3:6,16. The appeal of popish priests to the directions here given for the healing of the sick, to justify them in anointing with oil those who are expected to die, or administering what they call extreme unction, is a gross perversion; and it is only by perverting the scriptures, that they can gain even a seeming support for their practice.>
<5:15 They shall be forgiven him; on his being penitent, and exercising faith in Christ.>
<5:16 One to another; where you have injured one another. Be healed; that the calamities which your sins have occasioned may be removed. Effectual; sincere, earnest, believing. Availeth much; has great influence in procuring blessings from God. Interpreting this verse as if it said, Confess your sins to the priest, is another gross perversion of scripture, which, when the Bible shall be read by all, will be seen. It is not strange, therefore, that the pope warns his people against reading it and judging of its meaning; because when they do, they will see that it condemns him.>
<5:17 Elias; Elijah. 1Ki 17:1; 1Ki 18:42-45. Subject to like passions as we are; as much as to say, Do not think of Elias as some superhuman being, whose prevalent intercession with God you are forbidden to imitate. He was a fellow-man with you, and a sharer with you of all the infirmities of human nature.>
<5:18 Prayer, humble, believing, earnest, and persevering, has great influence with God, and is the means of procuring unspeakable blessings for men. All should therefore pray for whatever they need, in the name of Christ the one only Mediator; confessing their sins, not to the priest, but to God, who alone has power to forgive them, and thanking him for his mercies.>
<5:19 Err from the truth; depart from the faith and practice of the gospel. Convert him; turn him from his error to the belief and practice of truth and duty. Brethren in Christ who turn aside from the path of truth and duty, must be brought into it again, or they will perish; and he who, from love to Christ and to them, is instrumental in doing this, and thus saving their souls from death and hiding a multitude of sins, will be hailed by them as an everlasting benefactor, and with them bless God for ever and ever.>
<5:20 Hide a multitude of sins; cover them, by leading the person who has committed them to obtain, through repentance and faith in Christ, forgiveness of them. Compare Ps 32:1; "Blessed is he whose transgression is forgiven, whose sin is covered.">
\kniha{I Peter}
\zkratka{1Pet}
<1:1 Strangers; Jewish Christians, scattered throughout Pontus, Galatia, etc., provinces of Asia Minor. The people of God are scattered throughout various countries that they may show the nature of true religion, and thus lead others to embrace it and receive its benefits.>
<1:2 According to the foreknowledge--through sanctification--unto obedience and sprinkling; the apostle states three particulars concerning their election: first, it is in accordance with God's foreknowledge; in other words, it is the carrying out in action of his eternal foreknowledge, which includes his purpose; secondly, it is through the sanctification of the Holy Spirit, as the agent; thirdly, it is unto obedience and the sprinkling of Christ's blood, as the end. All who are elected, then, are elected to be brought into a state of obedience and purification from sin, through the power of God's Spirit, and the efficacy of Christ's blood, called the blood of sprinkling, because it cleanses the conscience from the guilt and defilement of sin. Heb 9:18-23.>
<1:3 A lively hope; more literally, a living hope. The Christian's hope is a living principle, and sustains his spiritual life. By the resurrection of Jesus Christ from the dead; this lies at the foundation of the regenerating power of the Spirit, with the living hope that accompanies it; for the Spirit works through the truth, verse 1Pe 1:22; Jas 1:18.>
<1:5 In the last time; at the end of the world. Those who believe on Christ must persevere in holiness in order to be saved; and the manner in which God keeps them affords the greatest encouragement to do this.>
<1:6 If need be; if it seem good to God that it be so for your spiritual profit. Compare Heb 12:6-11.>
<1:7 The trial of your faith; that is, the result of its being tried; or, which amounts to the same thing, your faith itself after it has been so tried; for it is not the trial that is precious, but its product. Than of gold; more literally, than gold. Though it be tried with fire; judged worth such a trial, notwithstanding its perishable nature, while your faith is imperishable.>
<1:9 Receiving--the salvation of your souls; eternal life begins in the soul with the beginning of faith in Christ, which has this for its end; though the full possession of it is reserved for the life to come. God never sends trials on his people, or allows snares to beset them, except when their good requires it. Their faith often needs to be tried, to show whether it is genuine; and when these tests prove that it is, and that its end will be salvation, this greatly increases their joy.>
<1:10 The prophets have inquired; they sought to know more fully the meaning of the prophecies which they uttered concerning Christ, and the blessings he would bestow on his people.>
<1:11 What; what time in respect to its distance from them. What manner of time; in respect to the character of the events belonging to it.>
<1:12 That not unto themselves; not unto themselves chiefly. Their revelations related to our times, and were given mainly for our benefit. Did minister; minister by way of prophetic annuciation, the things which are now announced to you as facts. Which things; foretold by the prophets, and revealed in the gospel. Angels desire to look; the church on earth and its salvation are objects of sturdy and admiration in heaven.>
<1:13 Wherefore; as the things revealed had engaged the attention of prophets, apostles, and angels. Gird up the loins of your mind; be awake, attentive, and steadfast in the faith and practice of the gospel. The grace; their salvation at the day of judgment. The fact that Christ and his salvation are the great subjects of Scripture, and have been regarded with the most intense interest by good men in all ages, and even by the angels in heaven, should lead all on earth to give them their earnest, devout, active, and persevering devotion.>
<1:14 Ignorance; their unconverted state.>
<1:15 He; God.>
<1:17 If ye call on the Father; or, if ye call upon him as Father. In fear; that fear of God which would lead them to obey him, and that fear of sin which would lead them to avoid it.>
<1:18 Vain conversation; vain life; their vain reliance on the ceremonies of the law and the traditions of the father. It may be the duty of persons to change their religion although it has come down to them from their ancestors, and is supported by ancient traditions; and in many cases, unless they do change it, they will lose their souls.>
<1:20 Foreordained; as a Saviour.>
<1:21 By him; it is through Christ alone that we come to God as our heavenly Father, and exercise true faith in him. Joh 14:6. Might be in God; rest on what God, not man, has done.>
<1:22 Unto unfeigned love; having this for its proper result.>
<1:23 Not of corruptible seed; not as at first, of mortal parents, born to die. The word of God, which liveth and abideth for ever; and therefore communicates and nourishes life which will be eternal. Truth is the means of regeneration and sanctification; and men by believing and obeying it are instrumental of their own eternal life.>
<1:24 For all flesh is as grass; a contrast between the weakness and transitoriness of man and the power and eternal duration of God's word, taken from Isa 40:6-8.>
<1:25 As the gospel is the appointed means of saving the soul, all should be taught to read it. It should also be preached to all, and they should be allowed and disposed to hear it, especially on the Sabbath; and for this purpose to rest from worldly business, travelling, and amusement, and meet together unitedly to seek the blessings of grace.>
<2:1 Men must cease to do evil if they would rightly understand and appreciate the truths of the gospel, or be savingly benefited in receiving them.>
<2:2 The sincere milk of the word; the pure spiritual truths of the gospel. Grow thereby; many of the best copies read, "grow thereby unto salvation;" that is, grow up, through the spiritual nourishment of the truth, into a state of salvation.>
<2:3 Tasted; learned by your own blessed experience. Ps 34:8. The Lord; the Lord Jesus.>
<2:4 A living stone; Christ, the foundation of the church and of the hopes of his people. Ps 118:22; Isa 28:16; Isa 53:5.>
<2:5 Lively; living. A spiritual house; the church of God, which is a spiritual temple consisting of living stones, built upon Jesus Christ, the living corner-stone. 1Co 3:16; Eph 2:20-22. A holy priesthood; why they are called a priesthood he immediately explains. It is because they offer to God, through Jesus Christ, not the outward sacrifices of the Levitical priesthood, but the spiritual sacrifices; of a broken heart and a contrite spirit, mingled with the incense of thanksgiving and praise. Ps 51:17; Ho 14:2; Heb 13:15; Re 1:6; Re 5:10.>
<2:6 In the scripture; Isa 28:16. The Scriptures show that Jesus Christ is the only foundation of hope, and that those who build on any other foundation will in the end be disappointed.>
<2:7 The stone; Christ. The builders; Jewish rulers. Mt 21:42; Ac 4:11.>
<2:8 Stumble at the word; are offended at the gospel and reject it. Appointed; by God, who will bring upon them the punishment they deserve.>
<2:9 But ye are a chosen--priesthood--show forth the praises; these expressions are a combination of Ex 19:5,6, and Isa 43:20,21, according to the Greek version. The idea is, that what God said of the literal Israel under the old economy, holds good of "the Israel of God" under the new, embracing all of every nation who believe in Christ. The praises; rather, as in the margin, the virtues; meaning the glorious attributes of God. The priests of the New Testament dispensation, spoken of in the Bible, are Christians; and the sacrifices which they offer are the sacrifices of love and devotion. These are acceptable to God through Jesus Christ, who by one offering of himself obtained eternal life for all who put their trust in him.>
<2:10 Not a people--obtained mercy; quoted from Ho 1:9,10; 2:1, upon the same principle as above.>
<2:11 As strangers and pilgrims; there seems to be a double allusion here: first, to their literal dispersion in foreign lands, chap 1Pe 1:1; secondly, to their being pilgrims and strangers upon earth, which their literal sojourn among foreigners well shadowed forth. 1Ch 29:15; Heb 11:9,10, compared with verses 1Pe 2:13-16. From all gratifications which injure the soul, or tend to hinder its salvation, the gospel requires total abstinence.>
<2:12 Conversation; deportment, manner of life. The day of visitation; the time when the gospel is accompanied by the Holy Spirit. Honesty, uprightness, and a kind and courteous demeanor should be conscientiously observed by the followers of Christ, that they may manifest the excellence of religion, and as far as possible lead all men to embrace it.>
<2:13 Every ordinance of man; all human laws which are not in opposition to the law of God. For the Lord's sake; for the purpose of honoring him.>
<2:14 The praise of them that do well; their protection, security, and comfort.>
<2:15 Put to silence; a good life best confounds slanderers.>
<2:16 As free; free from the service of Satan, and from slavish bondage to human ordinances. Compare Ga 5:13. For a cloak of maliciousness; not abusing your liberty by making it a cover for doing wrong to man.>
<2:17 Honor all men; by showing them proper respect. The brotherhood; Christians, who are all equally children of God. Fear God; in such a manner as shall lead you to obey him. The king; the one who is at the head of civil government. True religion teaches us to conduct with propriety in all conditions and relations of life, and to exercise those feelings towards others which we ought to wish others to exercise towards us.>
<2:18 All fear; all proper respect. The froward; wicked, peevish, morose.>
<2:21 Hereunto were ye called; to exercise a kind and forgiving spirit when injured, and thus honor Christ, who, when injured, manifested such a spirit. Isa 53:7-9; Ac 8:32.>
<2:23 To him; God, who although he commands his people to manifest a Christian spirit towards all, will nevertheless condemn and punish those who oppress or injure them. Mt 25:40-46. The commands of God to exercise right feelings when suffering under wrongs, were not designed to excuse the authors of those wrongs; and to quote these commands for such a purpose is a gross perversion of Scripture.>
<2:24 Bare our sins; expiated them by suffering in his own person the curse of them, and thus delivering us from it. Joh 1:29; Ga 3:13. The tree; the cross. Dead to sins; freed from their guilt and power. By whose stripes; in consequence of whose sufferings. Isa 53:5. Ye were healed; delivered from sin in its condemnation and pollution.>
<2:25 The Shepherd and Bishop; Jesus Christ, the overseer and watchman of our souls. Isa 40:11.>
<3:1 Be in subjection to your own husbands; treat them as the rightful head of the family. The word; the Scriptures and the preaching of the gospel. Be won; led to embrace the gospel.>
<3:2 Chaste conversation; pure deportment. Fear; a reverential demeanor, such as becomes the wife. Compare verses 1Pe 3:5,6. The salvation of relatives should be earnestly sought, and a uniformly Christian deportment is one of the most powerful means of promoting it.>
<3:4 Let it be the hidden man of the heart; instead of outward adornments visible to man, let it consist in the inward spiritual state of the heart, invisible to sense, which alone God regards. 1Sa 16:7. In; consisting or lying in. That which is not corruptible--quiet spirit; or, the incorruptible ornament of a meek and quiet spirit, like that which Jesus manifested, and which those possess who imitate him. Mt 11:29.>
<3:6 Calling him lord; thus acknowledging her subjection to him as her rightful head. Ge 18:12; 1Co 11:3. Amazement; such apprehension of danger as would prevent them from doing their duty. The most excellent, lovely, and enduring ornaments of women are not those which are external, but those which are internal-- purity of heart, meekness, contentment, and delight in doing good.>
<3:7 According to knowledge; knowledge of the nature and duties of the marriage relation. Giving honor; due respect, kind attention, and affectionate assistance; such as love guided by wisdom dictates. Heirs together; mutual partakers of divine grace, equally entitles to the blessings of the gospel. Daily family prayer is one of the most powerful means of grace; and husbands and wives should so live that uniting in it will be delightful, and a means of fitting them for the joys of earth and the bliss of heaven.>
<3:10 10-12. Quoted from Ps 34:12-16.>
<3:11 Eschew; avoid. Ensue; follow, practise.>
<3:12 Over the righteous; for their protection and benefit. Against them; he disapproves and will punish them.>
<3:13 Will harm you; the general effect of a righteous life is to deter men from harming us. Even should wicked men persecute us for righteousness' sake, God will overrule this for our good, as Peter immediately shows.>
<3:14 Happy; Mt 5:10. Of their terror; of any evil which they threaten. This, and the first clause of the next verse, are taken from Isa 8:12,13.>
<3:15 Sanctify the Lord God; treat him as God, trust in him to protect you and do for you what you need. To give an answer; state the reasons why you believe the gospel, and hope to be saved by it.>
<3:16 A good conscience; one that is enlightened, whose dictates you follow, and whose approbation you enjoy. They; the wicked. Good conversation; consistent life. So great is human wickedness that men will often be called to suffer for doing right; but instead of being discouraged, they should, with greater steadfastness and fidelity, go forward in duty, trusting in God to enable them so to live as never to be called to suffer for any other cause.>
<3:18 In the flesh; in his human nature. Quickened; made alive again; raised from the dead. By the Spirit; his own divine Spirit. Joh 10:17,18.>
<3:19 By which; divine Spirit. He went and preached; by Noah. Unto the spirits; which, when Peter wrote, were confined in torment as in a prison. Mt 5:25,26.>
\kniha{II Peter}
\zkratka{2Pet}
<1:15 These things; the truths and duties he had inculcated. One of the best ways of doing the greatest good for time and eternity, is to lead all people, as far as possible, rightly to understand and permanently to remember the truths God has revealed; and thus keep before them the motives he presents to lead men to believe and obey him.>
<1:16 Coming of our Lord; his second coming in that divine majesty of which the apostle and his two companions had a glimpse on the mount of transfiguration.>
<1:18 In the holy mount; Mt 17:1-5.>
<1:19 More sure; better fitted to carry universal conviction, because it is more comprehensive, resting not on a single revelation, but on a whole system of revelations. Word of prophecy; the prophecies of Scripture concerning the Messiah. The day dawn; the day of mature knowledge. The day-star; which is the forerunner of the perfect day.>
<1:20 First; as first in importance. Is of any private interpretation; that is, as the original seems to mean, comes of the prophet's own interpretation. He does not invent his prophecies. They are not his own private unfolding of God's counsels, but that which the Holy Ghost makes through him, as the apostle immediately proceeds to show, verse 2Pe 1:21.>
<1:21 As the Holy Ghost is the author of scripture prophecies, they cannot be made to mean whatever men may choose, or any thing except what God intended, and what in his providence has been or will be exactly accomplished.>
<2:1 False prophets; in allusion to the "more sure word of prophecy" spoken of in chap 2Pe 1:19; as much as to say, I have indeed commended to you the study of the prophets; but beware of the false prophets, who will come, as in ancient times, under the guise of true prophets. The people; under the Old Testament dispensation. Damnable; destructive. The Lord that bought them; by dying as a propitiation for their sins. 1Jo 2:2. False teachers have always abounded, who, by erroneous doctrines and unholy practices, have brought ruin upon themselves and others. All should therefore take heed not only how they hear, but what they hear; should prove all things by the Bible, and hold fast that which is good.>
<2:2 The way of truth; which the gospel reveals. Shall be evil spoken of; shall be brought into reproach and discredit by the ungodly lives of these false teachers and those who follow them.>
<2:3 With feigned words; covering over their base ends with a fair show of godliness. Make merchandise; should treat them not as immortal beings for whom Christ died, but in the way in which they thought they could gain the most money out of them. Slumbereth not; is certainly and speedily coming. When men are so pleased with error as liberally to pay for it, many will engage in its propagation.>
<2:4 For if God spared not the angels; verses 2Pe 2:4-8 are all connected with verse 2Pe 2:9, thus: "For if God spared not the angels--and spared not the old world--and turning the cities of Sodom and Gomorrah to ashes, condemned them--and delivered just Lot--[these examples show that] the Lord knoweth," etc>
<2:5 Saved Noah; Ge 7.1-24.>
<2:6 6-9. Sodom and Gomorrah; Ge 19:16-25.>
<2:8 When a professing Christian for worldly purposes becomes intimately connected with the wicked, he may expect them to be occasions of vexation and sorrow, if they do not prove the means of his ruin.>
<2:9 Facts as well as the declarations of the Bible testify to the justice of God, and to the certainty that, though he may bear long with the wicked, yet if they continue in sin they will not go unpunished.>
<2:10 But chiefly them; that is, but especially those of the unjust men just spoken of. Dignities; such magistrates and persons in official or elevated stations as God requires should be treated with respect, and should be obeyed in all their lawful commands.>
<2:11 Against them; against the dignities that oppose them in the execution of God's commands. See Jude 9. Angels, and those who are in temper like them, will not rail even against the wicked; and those who do, show that they are wicked themselves.>
<2:13 Riot in the daytime; openly and shamelessly, while common transgressors are content to riot in the night. Ro 13:13; 1Th 5:7.>
<2:14 Cannot cease; not for want of natural power, but of disposition. Licentiousness and the love of money in professors of religion are decisive marks of hypocrisy, and show that those who live in these sins are heirs of destruction.>
<2:15 Bosor; answering to the Hebrew Beor. Nu 22:5. Who loved the wages of unrighteousness; he desired permission to curse Israel that he might receive from Balak the promised reward. Nu 22. So these false teachers have in view their own private gain, verse 2Pe 2:3.>
<2:17 These are wells without water; an apt description of these boastful false teachers who came under the guise of godly men, but who had no true goodness themselves, and could impart no profit to their followers. Clouds; empty and windy clouds, that promise rain only to disappoint the husbandmen. The mist of darkness; the gloom of thick darkness.>
<2:18 Great swelling words of vanity; making, after the fashion of such men, large professions of their own light and knowledge, and large promises of good to others. Through the lusts of the flesh--wantonness; by turning the true doctrine of Christian liberty into licentiousness, and teaching men that the gospel gives license to indulge in fleshly lusts. Ga 5:13; 1Pe 2:16; Jude 4. Were clean escaped; or, according to another reading, "were scarcely escaped;" and therefore could be easily drawn back again into the company of the wicked.>
<2:19 Liberty; false liberty, which gave license to fleshly lusts. See note to the preceding verse.>
<2:20 The latter end is worse with them than the beginning; professors of religion who go back again into sin, become worse in character and condition than they were before.>
<2:22 Men may break off outward sins and profess religion without becoming holy. But they will be extremely apt to go back again; and when they do, they prove that they never had true religion, or were made "partakers of the divine nature." They never had a change of heart, or were "born of God.">
<3:1 Men may break off outward sins and profess religion without becoming holy. But they will be extremely apt to go back again; and when they do, they prove that they never had true religion, or were made "partakers of the divine nature." They never had a change of heart, or were "born of God.">
<3:2 Ministers of the gospel should labor not only to communicate a knowledge of its truths, but to lead all so to remember them as to act habitually under their influence.>
<3:3 The last days; see note to 1Ti 4:1.>
<3:4 The promise of his coming; fulfilment of the promise that Christ would come to judgment. All things continue as they were; this assertion of the scoffers was false, as the apostle proceeds to show. Infidels and scoffers at religion are evidences of the truth of the Bible. It foretold that they would come and act out their wickedness, and by doing it they fulfil the Scriptures. Thus the wickedness of men illustrates the truth of God.>
<3:5 Standing out of the water and in the water; rather, consisting out of water and by water. the reference is to the chaotic watery mass out of which the earth was formed, Ge 1:2. At the command of God it rose out of this, and took its form of dry land; so that it consisted out of water, and by means of water.>
<3:6 The world--perished; and as it had once been destroyed, it would be destroyed again; not as before with water, but with fire.>
<3:7 The same word; that command or power of God by which the world was created. Verse 2Pe 3:5; Ge 1:1-10. The same power of God which created the world keeps it in being, and will keep it till the time appointed for its dissolution.>
<3:8 A thousand years as one day; in comparison with eternity, and as to the certainty of what God has declared. What he has determined to accomplish a thousand years hence, is just as sure as if he had determined to accomplish it to-morrow. Compare the words of Moses: "A thousand years in thy sight are but as yesterday when it is past." Ps 90:4.>
<3:9 His promise; of a future judgment, when he will save his people and destroy their enemies. Count slackness; impute slackness to him, because he waits so long before executing his threatened judgments. Long-suffering; by waiting so long before he brings destruction on the wicked, he shows his desire that they should repent and be saved. By continuing men in life, offering them the gospel, and beseeching them to embrace it, God shows that he is unwilling they should perish, and would delight in their repentance and salvation.>
<3:10 The day of the Lord; when he will come to judgment. As a thief; suddenly, unexpectedly.>
<3:11 As this world with all it contains is to be burnt up, none should place their hearts upon it, or seek it as their chief good; but all should place their affections on things above, and lay up their treasure in heaven.>
<3:12 Hasting unto; preparing for and earnestly desiring the salvation which will be given to God's people.>
<3:13 His promise; Isa 65:17; 66:22; Re 21:1.>
<3:14 The "new heavens and the new earth" promised by God, is that state of perfect holiness and bliss into which, after the Judgment, God will receive his people; and for which the highest holiness and bliss on earth are but a preparation and a foretaste.>
<3:15 The long-suffering of our Lord is salvation; his delay to come to judgment is designed not to show that he will never come, but to give men opportunity to secure their salvation. Hath written unto you; you believers. We need not understand any particular church, since this epistle is general. The writings of Paul contain abundant notices of the second coming of Christ, and exhortations to wait for it in patience. See especially 1Th 4:13-18; 2Th 1:5-10; Heb 10:35-39.>
<3:16 These things; Christ's coming to judgment, and the necessity of diligent preparation in order to meet him in peace. Unlearned and unstable; ignorant persons who have no settled principles, and do not love the truth, which reproves their sinful lives. Wrest; pervert, misunderstand and misapply. As ignorance of the Scriptures greatly increases the danger of their perversion, and enables false teachers the more to delude and destroy the people, the Bible should be universally circulated, and all persons encouraged daily to read it--with earnest prayer for the teaching of the Holy Spirit, that they may rightly understand it, and by believing and obeying it be made wise to salvation.>
<3:18 Grow in grace; increase your knowledge of Christ, and your likeness to him. The grand safeguard against the seductions of error, and the most powerful means of increase in holiness, is increasing knowledge of Jesus Christ; that experimental knowledge which is obtained under the teaching of the Holy Ghost by daily searching the Scriptures, and which prepares us to unite with saints on earth and in heaven, saying with the heart, "To Him be glory both now and for ever. Amen.">
\kniha{I John}
\zkratka{1John}
<1:1 That which was from the beginning; that which was in the beginning, and therefore existed from the beginning. He means the Son of God in his eternal nature. Joh 1:1. Which we have heard; when made flesh and dwelling among us. Joh 1:14. Our hands have handled; Lu 24:39; Joh 20:27. The Word of life; the Word is here used, as in Joh 1:1,14, for Christ's divine and eternal person; and he is called "the Word of life," because he has life in himself, and is the author of life natural and spiritual. Joh 1:4.>
<1:2 The life was manifested; by becoming flesh. Joh 1:14. Was with the Father; dwelt with him from eternity. Joh 1:1,18; Joh 16:28; 17:5,24. The evidence that He was in the beginning with God became a man, not in appearance only, but in reality--that he took upon him human nature, and died, the just for the unjust, to bring men to God, is abundant and perfectly conclusive. All therefore who act rationally will believe these truths, and trust in Christ for salvation.>
<1:3 Have fellowship with us; in our union and communion, through faith, with the Father and the Son.>
<1:4 That your joy may be full; by your being thus brought into full fellowship with God and Christ. The religion of Christ is benevolent, leading all who enjoy its benefits to desire that others should enjoy them, and labor to extend them to all people.>
<1:5 God is light; his nature is light. He is perfect knowledge and purity. No darkness; the opposite of light: no ignorance or impurity. God is in all respects perfect; and all that he does is perfectly holy, wise, just, and good.>
<1:6 If we say; the apostle deduces from what he has just said of God's nature a most weighty inference. Since He is light, we must walk in the light, or we cannot have fellowship with Him. Walk in darkness; live in error and sin. Do not the truth; do not obey it, or act in accordance with it. Those who think they love God and yet live in sin are deceived.>
<1:7 Walk in the light; know and obey the truth. Fellowship one with another; joyful communion with each other and with God. Cleanseth us from all sin; expiates the guilt of all our sin, and cleanses our souls from all its pollution. This cleansing, so far as it is a work of sanctification, is not a momentary act, but a process which God carries forward till it ends in our perfect and everlasting holiness. It is given to those who walk in the light as God is in the light, seeking daily to know and do all God's will. The atonement of Christ is the ground, faith in him the means, and the Holy Ghost the author of sanctification; and all who truly believe, confess and forsake their sins, will, at the close of their probation, become completely and unchangeably perfect.>
<1:8 Say that we have no sin; that we are without sin, and need no forgiveness. Men who think they are now sinless are deceived; and those who say they have not sinned, commit aggravated sin by treating God as a liar.>
<1:9 Confess our sins; to God, and forsake them. Pr 28:13. Faithful; to his promises of forgiveness to the penitent. Pr 28:13. Just; to himself and all the great interests of his kingdom. Cleanse us; from the guilt and the defilement of sin, so as at last to present us spotless before the throne of his glory with exceeding joy. Jude 24.>
<1:10 We make him a liar; treat him as a liar, for he says all have sinned; and the facts that all die, and that all who are saved must be saved through the death of Christ, prove this. Ro 3:23; 5:12; 2Co 5:14.>
<2:1 My little children; believers; an endearing appellation from an aged apostle. These things; what he has just said of God's readiness to forgive the sins of those who confess them. 1Jo 1:7,9. That ye sin not; the offer of forgiveness is made to us that we may be encouraged to forsake sin and return to God. Ps 130:4. An advocate; in the original the word is the same that is rendered "Comforter," Joh 14:16-26; 15:26; 16:7, where it is applied to the Holy Spirit. Under the general idea of Helper, or Counsellor, it includes both these special meanings of Advocate and Comforter. The gospel of Jesus Christ is the true antidote both to presumption and despair.>
<2:2 Propitiation for our sins--also for sins of the whole world; by making propitiation for the sins of the whole world, he has opened a way in which all who believe in him shall be saved. One great object of all true ministers of the gospel is to keep Christians from the commission of sin; and the most efficacious way of doing this is to preach Christ to them as the propitiation for sin.>
<2:3 That we know him; to know God, in the scriptural sense of the words, is to have experimental acquaintance and communion with him as our Father and Friend. Such knowledge and fellowship are always connected with sincere obedience. Where this is wanting the profession of knowing God is vain and false. Chap 1Jo 1:6; 3:6-24; Mt 7:23; Joh 14:15-21,23.>
<2:5 Love--perfected; by bringing forth its proper fruits, and thus showing that it is genuine and saving. Hereby; by keeping his commandments.>
<2:6 He that saith he abideth in him; he who professes to be in union with Christ, must show the reality of his profession by walking as Christ walked. Union with Christ is the good tree, and this is known by its fruits. Obedience to God is sure evidence of a saving knowledge of him.>
<2:7 No new commandment; no commandment now revealed to you by me for the first time. From the beginning; from the time when the gospel was first preached to you. The apostle has special reference to the commandment of love, which is "the fulfilling of the law." This they had from the beginning, chap 1Jo 3:11; Joh 13:34,35; 15:12,17, and in this sense it was an old commandment. Compare 2Jo 5. Love to men was inculcated in the Old Testament. Christ not only taught it more clearly, but perfectly exemplified it, and thus presented to men new motives, and laid them under new obligations habitually to exercise it.>
<2:8 Again, a new commandment; as much as to say, I have called it an old commandment; but there is a sense in which it is new. Which thing is true; it is true that it is a new commandment. In him and in you; in the case of Christ who has given it, and in the case of you who have received it. Because the darkness is past; more literally, is passing away; he means the darkness that existed before the light of Christ's gospel was revealed. The true light; which Christ has brought into the world. Joh 8:12; 9:5; 12:35,36. This light is all summed up in the great commandment of love, which Christ has given and his disciples have received as a new commandment, because it is exemplified by himself in a new way and enforced upon them by new motives. Compare, besides the texts quoted above from John, 1Jo 3:16; 1Jo 4:9-11; Eph 4:32; 5:2,25; Php 2:5-8.>
<2:9 Is in darkness; having neither understood nor received this new commandment.>
<2:10 None occasion of stumbling in him; his soul is illuminated with the light of love: he sees the right way, and walks safely in it without danger of stumbling.>
<2:11 Is in darkness; being blinded by hatred, he walks on in darkness, and stumbles into perdition.>
<2:12 12-14. For the right understanding of these verses it is important to observe, first, that in them the apostle represents himself as appealing to Christians on the ground of the experience and knowledge which they already possess; compare verse 1Jo 2:21; secondly, that the repetitions of the words, "I write," "I have written," are for the sake of emphasis, both forms referring to the present epistle; thirdly, that the term "little children" is referred by some to all Christians, who are then distributed into "fathers" and "young men;" while others suppose that Christians of three different ages are addressed. Your sins are forgiven; very appropriate to children upon either of the above interpretations, as forgiveness of sin lies at the foundation of the Christian life. His name's sake; on account of what Christ has done.>
<2:13 Him that is from the beginning; Christ, who is from the beginning. A mature knowledge of Christ in his divine character is appropriate to fathers. Young men; to whom strength is especially becoming, verse 1Jo 2:14. Little children--known the Father; known God the Father as your father, and thus come into the relation of sons to him. This also is appropriate to children, according to either of the above-named interpretations.>
<2:14 Are strong--abideth in you; the apostle mentions not simply their strength, but the means also by which it is maintained--by God's word abiding in them. The wicked one; the devil. The gospel is suited to persons at every period and in all relations of life. None who can understand are too young to embrace it, and experience its saving power; none are too vigorous and active, or too full of business, to be governed by its spirit and perform its duties; and none too old to inculcate its principles and exemplify its precepts.>
<2:15 Love not the world; to love the world, and the things that are in the world, is to make them our treasure, and put our trust in them, instead of in God. Compare Mt 6:19-24.>
<2:16 Is not of the Father; does not come from him, and is not on his side, but stands in opposition to him. He created the world, and gave it to men to be used in his service, not to be abused as the minister of fleshly lust. Is of the world; comes from the world as the nourisher of earthly lust, and is opposed to God, and his service.>
<2:17 Passeth away; and should not therefore be made the object of our love. That doeth the will of God; in opposition to loving the world and its lusts. Abideth for ever; is blessed union with God, who is an imperishable portion. That love of worldly enjoyment which leads men supremely to seek it, is inconsistent with the love of God; and however much of it any may obtain, it will be unsatisfying and temporary; while that love of God which leads them to find their chief enjoyment in learning and doing his will, will be satisfying and eternal.>
<2:18 The last time; the same last time as that spoken of in 1Ti 4:1; 2Ti 3:1; 2Pe 3:3. It agrees with the time foretold by our Lord when iniquity should abound, and false Christs and false prophets should arise. Mt 24:10-12; Mr 13:22; Lu 21:8. It had a fulfilment in the last days of the apostolic age, but a higher fulfilment is reserved for the last days connected with Christ's second coming. Anti-christ; the opposer of Christ and his cause. Whereby we know; because it had been predicted that in the last times such persons will arise. See the reference above given.>
<2:19 They went out; apostatized. From us; from the church or company of Christians. Not of us; not real Christians. That they were not all of us; or, that all are not of us; that some who belong to our body are not really of us, but Christians only in name. When professors of religion apostatize, embrace error, and live in sin, they show to all that they are not the children of God. Job 17:9; Joh 4:14.>
<2:20 Ye; real Christians. An unction; anointing, or the enlightening and sanctifying influences of the Holy Spirit. All things; all things essential to your preservation from fatal error, and your perseverance in the faith and practice of the gospel.>
<2:21 No lie; error or false doctrine.>
<2:22 A liar; an asserter of false doctrines. Denieth the Father and the Son; that is, denies the Father in denying the Son, as he goes on to state in the next verse.>
<2:23 Hath not the Father; not right views of him, no supreme regard to him, and no interest in his favor. Such is the union between the Father and the Son, that men who reject and oppose the Son, reject and oppose the Father; while all who love and obey the Son, love and obey the Father also.>
<2:24 If that--remain; if you continue to believe and obey the truths you first embraced. In the Son, and in the Father; in holy union and fellowship with them, the foundation of which is the true knowledge of them joined with love.>
<2:27 The anointing; see notes to verse 1Jo 2:20. This anointing teaches us not without, but through the revealed word of God; and whoever lays claim to it must be tried by this word. Shall abide in him; being kept from the seductions of the wicked. The reason why real Christians persevere in holiness to the end is, that the Holy Ghost continues to teach them the good and the right way, and to incline them to walk in it. When they deviate from it, he leads them to think on their ways, and turn their feet unto God's testimonies. Thus he works in them to will and to do, while they work out their salvation with fear and trembling, and so keeps them by his mighty power, through faith and obedience, unto eternal life.>
<2:28 When he shall appear; when Christ shall appear in glory to judge the world.>
<2:29 Every one that doeth righteousness is born of him; as much as to say, He that doeth righteousness, and no other; since they who are born of God must be like God in character.>
<3:1 Upon us; who have received Christ through faith. Joh 1:12; Knew him not; did not understand his true character. In making guilty, polluted rebels and heirs of endless perdition holy--not merely servants but children, heirs of God, and partakers of endless life and glory--the grace of God surpasses all finite conception, and will be a theme of adoring praises from multitudes that no man can number, for ever and ever.>
<3:2 Not yet appear; the fulness of their future excellence and bliss could not here be known. Appear; in glory.>
<3:3 This hope; the hope of being like Christ and seeing him as he is. Purifeth himself, even as he is pure; he strives now to be pure as Christ is pure. The apostle here gives the distinguishing mark of a true hope, as contrasted with every false hope. Every man who has the hope of the gospel, by the habitual contemplation of Christ, and earnest, prayerful, persevering efforts to imitate his example, becomes, through the grace of God, more and more like him.>
<3:4 Transgresseth also the law; the essence of all sin is the transgression of God's law; in other words, sin is contrariety to the revealed will of God, which must for ever be the rule of our action.>
<3:5 Was manifested to take away our sins--in him is no sin; two reasons why God's children cannot allow themselves in sin. It is contrary to both the work of Christ, and his character. Christ takes away our sin by expiating it, and cleansing our hearts from its pollution.>
<3:6 6-10. In these verses the apostle is combating the error of those who sought to separate fellowship with God from a life of holiness; or who at least did not consider the inseparable connection of the two, and boasted that they had fellowship with God, while their lives were devoted to sinful lusts. Chapter 1Joh 1:6; 2:4,9. To sin then, or commit sin, must mean, in this connection, to lead a life of sin, to sin allowedly and habitually. This no true believer does. To be righteous as Christ is righteous is the aim of his life. His daily effort is to keep the whole law of God; and wherein he fails through the remaining corruption of his fallen nature, he confesses to God his guilt, asks His forgiveness, and addresses himself anew to the work of keeping His commandments, not in the letter only, but "in spirit and in truth." Abideth in him; is united to him by faith, and lives in fellowship with him. Men who live in the love and practice of known sin, secret or open, of omission or commission, of heart or of life, have no interest in Christ, and have never experienced his salvation.>
<3:7 Doeth righteousness; in his life. Is righteous; in his character. The tree--a righteous character--is known by its fruit--doing righteousness.>
<3:8 Is of the devil; is a child of the devil and like him in character. Might destroy the works of the devil; it follows that Christ's disciples cannot do these works.>
<3:9 His seed; God's seed; that is, the new moral nature which he has received from God, and which is maintained in his heart by the indwelling of the Holy Ghost. He cannot sin; not for want of power, but disposition; he does not desire or consent to live in sin. The reason is, he loves those things which please God, and hates those which displease him.>
<3:10 Are manifest; by the different courses which they pursue. One class work righteousness and love Christians, the other do not. Those who live in sin take an active part against Christ and his cause, and in favor of the cause of the devil; and if they continue this course, they will be treated as the servants of Satan, and be made for ever partakers of his plagues.>
<3:12 That wicked one; the devil, the father of all murderers.>
<3:13 Marvel not--if the world hate you; because you are not of the world, and the world knows you not. Verse 1Jo 3:1; Joh 15:17-21.>
<3:14 From death unto life; spiritually. Because we love the brethren; for true love towards them is inseparable from love towards God, and love is the essence of the new divine life. To dwell in love, is to have in our souls the beginning of eternal life. Chap 1Jo 4:7. In death; spiritual death; in an unholy state and under condemnation.>
<3:15 Is a murderer; in heart; he cherishes the feelings from which the outward act of murder proceeds. Love to real Christians on account of their religion, is evidence of love to Christ and acceptance with him; while hatred of them is Satan-like, and tends to envy, slander, persecution, and murder.>
<3:16 Hereby perceive we the love of God; the words "of God" are not in the original. The literal rendering is, "Herein"--by the example that follows--"we know love," we see and understand its true nature. He; Christ. To lay down our lives; to have that love which makes us ready to die for our brethren, and actually to do so when God calls us. Such a love is constantly active in doing good, as the apostle proceeds to show. We may be called on to sacrifice life, but never to give up our salvation, for the good of others.>
<3:17 The possession of property involves high responsibilities, increases obligation, and multiplies duties. By the manner in which men use it they show their character.>
<3:19 Hereby; by loving the brethren in reality, and being disposed, as we have opportunity, to do them good. Are of the truth; belong to the side of the truth, believe and love it. Assure our hearts; quiet their fears by the assurance of his gracious acceptance.>
<3:20 Our heart condemn us; as wanting in love, and for this reason withholding aid from the destitute when we ought to bestow it. God is greater; more perfectly acquainted with our sins, and will more certainly condemn us. The approbation of an enlightened, healthy conscience is needful to a well-grounded hope of the approbation of God; and the condemnation of an enlightened conscience is evidence of the condemnation of God.>
<3:22 Whatsoever we ask; in this state of filial confidence, which comes from the consciousness of keeping God's commandments; that is, as the context shows, of being led by love to obey God. We receive of him; we always receive of God an answer to our sincere and believing petitions, though not always in the particular form in which we present them, because God sees that another form is better for us.>
<3:24 Hereby; namely, by what follows. By the Spirit which he hath given us; the Holy Spirit bears witness with our spirits that we are the children of God, and that, as such, we dwell in God and God in us. The possession of the spirit of Christ, and its manifestation in the fruits of the Spirit, prove that one is born of God and an heir of heaven.>
<4:1 Every spirit; speaking to you through one who claims to be a prophet. Try the spirits; the "discerning of spirits" was one of the special and temporary spiritual gifts, 1Co 12:10; but here the apostle proposes such tests as all might employ, verses 1Jo 4:2,3. The doctrines and practice of all religious teachers should be tried by the word of God. If they agree with this they should be received, and if not should be rejected. Hence the right and the duty of all men to be acquainted with the word of God, that they may rightly judge and act in this matter.>
<4:2 Confesseth that Jesus Christ is come in the flesh; or confesseth Jesus Christ as having come in the flesh. Many think that the apostle refers to a very ancient form of error which denied our Lord's humanity by maintaining that his body was a delusive show, existing only in vision; whence it would follow that his expiation for sin on the cross with his own blood was not real, but a vain show also. In all such passages as the present, the confession is to be understood as sincere, and as accompanied by a corresponding obedient reception of Christ in his proper character as he is revealed in the gospel.>
<4:3 Is that spirit of antichrist; it is one of the forms in which the spirit of antichrist is manifested. Religious teachers who do not confess that Christ took upon him human nature, and became the propitiation for the sins of men, are not of God. 1Jo 2:2.>
<4:4 Overcome them; the false prophets, through whom the spirit of antichrist works, seeking to seduce you from the truth. He that is in you; God, who dwells in you by the Holy Spirit, enlightening, sanctifying, and strengthening you, and thus preserving you from the wiles of these false teachers.>
<4:5 They; the false teachers. Are of the world; belong in their spirit to the world, and are governed by its principles. Speak they of the world--the world heareth them; their doctrine proceeds from a worldly spirit and is worldly in its character. For this reason it is agreeable to worldly men.>
<4:6 We are of God; the apostles and those who taught like them had the Spirit of God and proclaimed the truth of God. This they proved by their works, God working with them by miracles and gifts of the Holy Ghost. Mr 16:20; Joh 21:24. He that knoweth God; the true Christian. Hereby; by their believing and obeying the truths taught by the apostles, or disbelieving and rejecting them. False teachers proclaim doctrines which are more agreeable to worldly men than the doctrines of the Bible, and flatter them with hopes of heaven though they live in sin. For this reason those who love their sins follow them, while those who hate their sins embrace the doctrines and follow the precepts of the Bible.>
<4:7 Love is of God; he is its author, and those who exercise it are his children, spiritually born of him.>
<4:8 Knoweth not God; has no true acquaintance and fellowship with him. God is love; this is the sum of his moral nature. To have communion with God we must be like him in love.>
<4:11 The most wondrous exhibition of the love of God was the gift of his Son, to be the propitiation for the sins of the world; and the right apprehension and cordial reception of this truth is the most powerful means of leading men to love God, and to manifest it in love to men.>
<4:12 Is perfected; by producing in us its proper fruits, and is thus shown to be genuine, complete.>
<4:13 Because he hath given us of his Spirit; see note to chap 1Jo 3:24.>
<4:14 We have seen; Joh 1:14.>
<4:15 Confess that Jesus is the Son of God; truly, sincerely, heartily; receiving him as the Son of God.>
<4:16 Loving God and good men unites the soul to him in a most intimate, endearing, elevated, ennobling, and blissful union; the joy of which, even in its foretaste on earth, is often unspeakable and full of glory. 1Pe 1:8.>
<4:17 Herein; according to some, this word refers backward to the preceding verse. The meaning will then be, that by our dwelling in love, and thus in God and he in us, our love is made perfect; and the words following, "that we may have boldness," etc., will express the end towards which that love is directed. According to others, the reference is forward, precisely as in Joh 15:8, "Herein is my Father glorified, that ye bear much fruit." The meaning will then be, that the perfection of our love consists in its giving us boldness in the day of judgment; and consequently now, in anticipation of that day. Because; the ground of this boldness. As he is; as Christ is, in respect to love. He does not say, as Christ was, because Christ's love is not changed by his removal to heaven. So are we in this world; we manifest in the world the same love which Christ manifested on earth, and now has in heaven.>
<4:18 Fear hath torment; literally, fear hath punishment. It is this towards which fear looks, and the dread of it fills the soul with misery.>
<4:19 Because he first loved us; his love to us opened the way for and was the procuring cause of our love to him. The gift of the Saviour and the way of life which he has opened, the gift of the Holy Spirit, the preaching of the gospel and all the means of grace, the regeneration of men, their sanctification and hope of glory, their perseverance in holiness, and their eternal life, are all the fruit and manifestation of the infinite and eternal love of God, and will call forth from all the redeemed the most exalted praises to God and the Lamb for ever. Re 5:8-14.>
<5:1 Believeth that Jesus is the Christ; heartily, so as to trust in him for salvation. Him that begat--him that is begotten; the spiritual child bears the image of God his Father. Hence the love of the Father implies the love of all his children.>
<5:2 By this we know; obedience is the test of love towards God; and the love of God includes in itself the love of the brethren, chap 1Jo 3:17; 4:20,21.>
<5:3 Not grievous; not burdensome and oppressive. Compare Mt 11:30.>
<5:4 For whatsoever is born of God overcometh the world; a proof of the assertion just made, that God's commands are not grievous. Nothing is opposed to the fulfilment of them but the love of the world; and this is overcome by all who are born of God. This is the victory--our faith; for through faith we see Jesus the Son of God as our Saviour, and with him the unseen and eternal realities of heaven, 2Co 4:18; 1Pe 1:8; and thus we overcome the fear of man and the love of things seen and temporal.>
<5:6 By water; in his baptism, when he was by the testimony of the Father solemnly proclaimed as the Messiah. And blood; in his bloody death on the cross, when he made expiation for the sins of the world, which was the great work of his earthly mission. Not by water only, but by water and blood; thus testifying that his work of redemption includes atonement for sin as well as spiritual cleansing--that without the shedding of his blood there could be no remission of sins, any more than there could be communion with God and the enjoyment of his love without the inward sanctification of the Holy Ghost. That beareth witness; not only to the Messiahship of Jesus, but also to the nature of his work as the Messiah. The apostle has in view the testimony of the Holy Ghost not only in his miraculous gifts, but also, and especially, in his inward witness in the hearts of believers. Compare verse 1Jo 5:10; Jo 16:14.>
<5:8 Agree in one; they unite in one and the same testimony concerning the character and office of Jesus as the Messiah. The necessity and efficacy of the atonement of Christ, of faith in him, and of the purifying influences of the Holy Spirit, in order to salvation, are taught by the word, the Spirit, the ordinances, and the works of God; and without believing them, we shall never gain the victory over the world, the flesh, and the devil, or come off conquerors through him that loved us and gave himself for us.>
<5:9 Greater; more certainly true and worthy of belief. This; the testimony above referred to.>
<5:10 Hath the witness in himself; evidence of the truth of God's testimony by the effects which the Holy Ghost produces on him in his believing it. Made him a liar; acts towards him as if he were one. The record that God gave; in the ways above mentioned. Disbelief of the testimony of God is a great and aggravated crime.>
<5:11 Given to us eternal life; made known to us the way of life, and given it to all who take that way. This life is in his Son; he is the author of it, and it is obtained by faith in him.>
<5:12 That hath the Son; as his Saviour, by believing on him.>
<5:13 And that ye may believe; believe with more steadfastness.>
<5:15 We have the petitions; our prayers are accepted, and will be answered in that way and time which will be for the glory of God and our own highest good. See note to chap 1Jo 3:22. Prayers offered according to the will of God are always accepted of him; and in the bestowment of those blessings which are most for his glory, for the good of the offerer and of the universe, they are in the highest and best sense answered.>
<5:16 Give him life; by leading him to repent of his sins and believe in Christ. A sin unto death; one which will not be repented of nor forgiven. Mt 12:31,32.>
<5:17 A sin not unto death; one that may be repented of and forgiven.>
<5:18 Sinneth not; willfully, deliberately, perseveringly; but if he sin, he repents, has an Advocate with the Father, and will be forgiven. See note to chap 1Jo 3:6-10. That wicked one; the devil. Toucheth him not; assails him not in such a way as to overcome and destroy him. Lu 22:31,32.>
<5:19 Lieth in wickedness; under the influences of the wicked one.>
<5:20 May know him that is true; the Father whom the Son has revealed to us. And we are in him--Jesus Christ; the literal rendering of these words is, And we are in him that is true, in his Son Jesus Christ. The meaning seems to be, that we are in the true God by being in his Son Jesus Christ; or, which comes to the same thing, that being in the Son is being in the Father. Compare Joh 17:21, "that they may be one in us." As Jesus Christ is the true God, the author of eternal life, and has promised to give it to all who believe on him, all have the best reasons and strongest motives to trust in him, and to continue steadfast in their adherence to truth and duty, till faith shall be swallowed up in vision and hope in endless joy.>
<5:21 From idols; from idolatry literal and spiritual; from worshipping or regarding supremely any created thing.>
\kniha{II John}
\zkratka{2John}
<1:1 The elder; John, the writer of this epistle. Compare 1Pe 5:1. Elect lady; one chosen of God and distinguished as a Christian.>
<1:2 For the truth's sake; this lady's family were beloved on account of their love of the gospel and their practice of its duties.>
<1:4 That I found of thy children; when I found some of thy children. The apostle had learned with joy that this was true of some of them. Friends of the truth are friends of each other, and earnestly desire each other's highest good; and when they hear that the children of their friends have embraced the gospel, and are walking in the love and practice of it, they rejoice with great joy.>
<1:5 Not as though I wrote a new commandment; compare 1Jo 2:7.>
<1:6 This is love--his commandments; the proper expression and evidence of love to God and men, is the keeping of his commandments. Compare 1Jo 5:2, 3. This is the commandment; what the apostle has just declared; that love, namely, which consists in obedience to God's commandments.>
<1:7 That Jesus Christ is come in the flesh; see note to 1Jo 4:2. Those who pretend to be religious teachers, and yet do not believe that Jesus Christ has come into the world and redeemed us unto God by his blood, are deceivers. Whatever love they may profess towards God or men, they are opposers of Christ and the great interest of his kingdom.>
<1:8 That we lost not; through the seductions of these deceivers leading you to depart from the truth, any portion of the blessings we have obtained, or which the gospel offers.>
<1:9 Doctrine of Christ; that taught by him and his apostles, especially his having come in the flesh and made an atonement for the sin of the world. Hath not God; not a right knowledge of him, nor an interest in his favor.>
<1:10 This doctrine; of Christ, as taught by him and his apostles. Receive him not into your house; do nothing to aid or encourage him in his propagation of error.>
<1:11 Biddeth him God speed; if one aid or encourage another in sinning, he becomes partaker of his guilt. Those teachers who reject the great truths taught by Christ and his apostles, of his divinity, incarnation, and atonement; of justification by faith in him, regeneration by the Spirit of God, and the necessity of perseverance in holiness in order to salvation, are not of God, and not to be received: no such attention should be paid to them as will aid or encourage them in the propagation of their errors.>
\kniha{III John}
\zkratka{3John}
<1:1 Gaius; a Christian whose piety and beneficence had greatly endeared him to the apostle.>
<1:2 As thy soul prospereth; that he might be as much favored in his health and outward condition as he was in his piety and beneficence. It is desirable that good men should not only be eminent in piety and good works, but also have health and be in unembarrassed outward circumstances. They should therefore conscientiously and diligently use all suitable means to secure these important blessings.>
<1:4 My children; Christians, especially those who had been converted through his instrumentality.>
<1:5 The brethren; Christians who were in want. Strangers; those who were driven from home by persecution, or who had left it in the service of Christ.>
<1:6 After a godly sort; with that kind of assistance which becomes disciples of Christ towards his ministers who go to preach the gospel and supply the destitute. It is the duty of ministers from love to Christ not only to preach the gospel at home, but to go to the heathen and preach it, where Christ has never been known; and when they do this, it cannot be expected that the heathen, at first, should support them. It is not wise to ask it; and it is in such cases a duty, and should be esteemed a privilege, for Christians at home to support them.>
<1:7 His name's sake; from love to Christ. Went forth; to preach the gospel to the heathen. Taking nothing; of their hearers for their support, but were supported by Christian friends and their own efforts.>
<1:8 Be fellow-helpers; assist them in spreading the gospel.>
<1:9 I wrote unto the church; requesting them to assist the brethren in their benevolent efforts. Diotrephes; who opposed the apostle, and influenced the church not to comply with his request.>
<1:10 The brethren; whom the apostle had recommended to their hospitality and aid. Those who love power and seek to have preeminence in the church, are very apt to be haters of good men and of what they do for Christ--to be opposed to the right of private judgment, and to persecute those who exercise it. But all such deeds are evil, let who will perform them, and they will be remembered and treated as evil in the day when God shall render to every one according to his works.>
<1:11 That which is evil; as exemplified by Diotrephes. He that doeth good; to the friends of Christ, from love to him. Is of God; belongs to God, as one of his children who is like him, and has communion with him. Hath not seen God; has wrong views of him, and is opposed to him.>
<1:12 Hath good report; is well spoken of, justly, as a good man, whose works attest his piety and benevolence.>
<1:14 Kind salutations of friends are profitable, both to those who give and those who receive them; and real kindness habitually and kindly expressed, is the essence of true politeness, the ornament of dignified refinement, and the source of pure, elevated, and purifying joy.>
\kniha{Jude}
\zkratka{Jude}
<1:3 Needful; on account of their danger from false teachers. The faith; the truths taught by Christ and his apostles. Ministers in addressing their people should select such subjects as are pertinent to their circumstances; especially should they warn their hearers against prevailing errors which tend to draw them from the faith and practice of the gospel.>
<1:4 Unawares; by stealth. Of old ordained; whose coming, character, and punishment have been foretold by ancient prophets, and by Christ and his apostles. Turning the grace of our God into lasciviousness; so perverting the doctrine of divine grace as to make it an excuse for living in the indulgence of fleshly lusts, and teaching others to do the same. Denying the only Lord God, and our Lord Jesus Christ; in doctrine by rejecting the truths revealed by God through Christ, and in practice by trampling under foot Christ's commands.>
<1:6 Their first estate; or, as in the margin, their principality; which seems to denote the rank and office assigned to them by their Creator among the heavenly hosts. Left their own habitation; became discontented with their condition, and refused to do the will of God, in the place assigned to them 2Pe 2:4. We know nothing further concerning their fall than the brief hints of Scripture, and all speculation on the subject is vain.>
<1:7 Suffering the vengeance of eternal fire; they were cast into endless perdition with the devil and his angels. Mt 25:41. Of this the flames which consumed their cities and made them desolate for ever were a solemn symbol. Ge 19:24,25. The destruction of the Israelites, of the inhabitants of Sodom, and of the angels that sinned, is recorded for the warning of sinners in all ages, and to show that however great the blessings men may enjoy, if they reject the gospel, or continue in sin, they will inevitably and awfully perish.>
<1:8 Dreamers; the false teachers referred to. Despise dominion; spurn obedience to law, human and divine. Of dignities; persons called by God to stations of authority or honor. 1Pe 2:10.>
<1:9 The archangel; the word archangel means a chief angel, or ruler of angels. the inspired writers of the New Testament occasionally refer, as is done here and in verse Jude 14, to events not recorded in the Old Testament, but handed down in tradition. Compare 2Ti 3:8. We know nothing more of the event here referred to than what Jude has given us. Durst not; not because he feared the devil, but because he feared God, and feared to commit sin by using reproachful language. Rebuke thee; restrain thy rage, control, and punish thee. Holy beings will not use reproachful epithets even towards the devil, much less towards men, especially magistrates, and those whom God requires us to honor. Those who delight in such language show themselves to be servants of the evil one.>
<1:10 These; false, wicked teachers. Know not; do not rightly understand. Know naturally; by instinct, such as the indulgence of animal appetites and passions.>
<1:11 The way of Cain; relying on their own wisdom and goodness, and not on the wisdom and grace of God; envying, hating, and destroying those who were better and more highly favored than themselves. Ge 4:4-8. The error of Balaam; loving and coveting money. Nu 22:7-21; 2Pe 2:15. Core; Korah. Nu 16:1-33.>
<1:12 12, 13. Spots; rather, sunken rocks, exposing to destruction the voyager that comes upon them. Feasts of charity; among the brethren, where purity and temperance ought to have prevailed. Clouds--trees--raging waves--wandering stars; those false teachers were in many respects like these things; disappointing all just expectations, corrupting and exposing to destruction all who came under their influence, and themselves doomed to destruction.>
<1:16 Having men's persons in admiration; paying court to the corrupt, the rich, and the great, to further their own selfish designs. Pride, covetousness, and sensuality have ever been besetting sins with false teachers of religion: and they have flattered the wicked, the rich, and the great, to obtain means for their own selfish gratification. Holy men have always opposed them, pointed out their errors, and foretold their certain destruction if they continue in sin.>
<1:17 17-21. A thorough acquaintance with the declarations of Scripture is a great safeguard against the seductions of error, and one of the chief means of preservation from sin. It is one by which the Holy Ghost, who dwells in believers, operates in enlightening their minds and purifying their hearts; guiding them in duty, shielding them from danger, keeping them in the fear and love of God, in the patient waiting for Jesus Christ, and in habitual preparation for his coming and kingdom.>
<1:19 Separate themselves; by withdrawing themselves and leading off their followers from the faithful, who adhere to the doctrines and duties of the gospel. Sensual; wickedly indulging their appetites and passions. Having not the Spirit; not under his guidance, nor partakers of his salvation.>
<1:20 Building up yourselves; by increasing in the knowledge and love of God, of his truth and will, and in devotion of body and soul to his service. In the Holy Ghost; according to his directions, under his influence, and by his aid.>
<1:21 Keep yourselves in the love of God; by hearkening diligently to his voice in the Scriptures, believing heartily his declarations, and cheerfully, steadfastly, and perseveringly obeying his commands. Looking for the mercy; desiring and expecting salvation only through rich grace in Christ.>
<1:22 Of some; who have been bewildered as to truth and duty, seduced into error and sin. Making a difference; according to their character, condition, and wants; treating them gently and kindly, and thus alluring them back to truth and duty.>
<1:23 Others save with fear; present alarming considerations to arouse them, as you would were they asleep in a house on fire. Hating even the garment; abhorring and avoiding every thing connected with these transgressions, or tending to defile you. In our efforts to reclaim and save men, great wisdom is needful, to adapt the means used to their various cases. Some must be allured by kindness, and efforts for them should be gentle, as those of a nurse with her children. Others must be aroused by terrors, and urged by the thunders of coming wrath. 1Th 2:7; 2Co 5:11.>
<1:24 Falling; from truth and duty into error and sin. In all efforts for our own good and that of others we should depend for success wholly upon the grace of God, who is able to save us from sin, to keep us from falling into it, and to present us faultless before the presence of his glory, with exceeding and eternal joy. To him belongs the glory of all the good which is done or enjoyed, and to him should be given all the praise for ever. Amen.>
\kniha{Revelation of John}
\zkratka{Rev}
<1:1 The Revelation of Jesus Christ; that made by Jesus Christ. Which God gave unto him; here, as uniformly in the New Testament, Christ is represented as acting according to the commission which he has received from God the Father. Compare Joh 3:34; 5:20; 7:16; 10:32; Joh 12:49. Must shortly come to pass; these words may be understood as meaning that the series of events here foretold must soon begin to be accomplished. But this limitation is not necessary, since the constant representation of Scripture is, that with the Lord a thousand years are but as one day, and that the coming of Christ and the end of all things is always at hand, chapter Re 22:20; 1Pe 4:7; 2Pe 3:8,12; and especially Lu 18:8. He sent; whether we understand Jesus Christ, as in chap Re 22:16, or God, as in chap Re 22:6, is unimportant, since in the matter of this revelation the Father and the Son are one. By his angel; making use of his ministry, chap Re 22:6,8,16. John; the apostle John. The Lord reveals as many things as it is needful for his people in this life to know; and many things which are now dark and mysterious will hereafter be made plain. Joh 13:7.>
<1:2 The word of God; the word revealed by God. The testimony of Jesus Christ; the testimony borne to the truth by Jesus Christ, "the faithful and true Witness," chap Re 3:14.>
<1:3 Keep those things; remember the truths herein taught, and do the duties required. The time is at hand; see note to verse Re 1:1.>
<1:4 To the seven churches which are in Asia; we are to understand here the Roman province of Proconsular Asia, embracing the provinces of Mysia, Lydia, Caria, and as it would seem, the western part of Phrygia also, in which Laodicea was situated. From the naming of these seven it does not follow that there were not other churches in Asia. The number seven, which is the symbol of completeness, prevails throughout this book, and is designedly chosen here. Which is, and which was, and which is to come; that is, the self-existent and eternal God, who has life in himself. The words seem to be an exposition of the meaning of the Hebrew word Jehovah. See note to Ex 6:3. The seven spirits which are before his throne; the same as the "seven lamps of fire burning before the throne," chap Re 4:5. As this and the following verse contain a benediction from the Father and the Son, we must suppose that it is them, as elsewhere, and not any created spirits. In accordance with the emblematical character of this book, he is described under the number seven, to denote his manifold and perfect divine operations. Compare the seven "eyes of the Lord which run to and fro through the whole earth," Zec 4:10; and the seven eyes of the Lamb, "which are the seven spirits of God sent forth into all the earth," chap Re 5:6; both which represent one and the same Holy Spirit proceeding from the Father and the Son.>
<1:5 The First-begotten of the dead; the first who rose to die no more, and the leader and head of all who shall be by his divine power raised from the dead to eternal life. Him; Jesus Christ.>
<1:6 Kings and priests; to reign with him in glory, chap Re 22:5, and to offer to God through him spiritual sacrifices, 1Pe 2:5. The source of grace, mercy, and peace, is the self-existent, eternal, unchangeable Jehovah; and for the manifestation of himself as the Father, the Son, and the Holy Ghost, in redeeming and sanctifying men, he is worthy of, and will receive the highest glory for ever.>
<1:7 He cometh; for the deliverance of his friends and the ruin of his enemies. They also which pierced him--all kindreds of the earth shall wail; there is here an allusion to Zec 12:10-14, but with a different application of the words. In Zechariah it is a penitential mourning; but here, as in Mt 24:30, where the same words are used, it is a mourning of terror in view of Christ's coming to take vengeance on the wicked. Compare 2Th 1:8.>
<1:8 Alpha and Omega; these are the first and last letters of the Greek alphabet, and by thus applying them to himself, Christ shows that he is the cause and end of all things. Compare Isa 44:6. Is--was--is to come; a description of Christ as Jehovah, self-existent, unchangeable, and eternal. See note to verse Re 1:4.>
<1:9 Brother--companion; a fellow-Christian, who, with others, was suffering persecution on account of his religion. Patmos; a desolate island in the Aegean sea. For the word of God; on account of my fidelity in preaching it. He had been banished to Patmos by the persecutors of Christianity.>
<1:10 In the Spirit; under his miraculous and prophetical influence. The Lord's day; the first day of the week, commemorating the Lord's resurrection, and observed as a day of divine worship, the Christian Sabbath. A great voice; that of Jesus Christ. Verse Re 1:13. The fact that the first day of the week was regarded by the apostles and first Christians as, in a special sense, the Lord's day, and that it was known and kept as such, devoted to divine worship and acts of beneficence throughout the churches, indicates the will of God that it should be observed in all coming ages as the Christian Sabbath. 1Co 16:2.>
<1:11 Ephesus; the capital city of Proconsular Asia, lying near the Mediterranean sea. Smyrna; a seaport of the Mediterranean about forty miles north of Ephesus. Pergamos; on the river Caicus, about twenty miles from the sea, and sixty miles north of Smyrna. Thyatira; a city in the province of Lydia north-east of Smyrna. Sardis; a city east of Smyrna, and about thirty miles south-east of Thyatira. Philadelphia; about seventy miles east of Smyrna. Loadicea; a city in the west of Phrygia, about a hundred miles east of Ephesus.>
<1:12 Seven golden candlesticks; these represented the seven churches in the places above mentioned. Verse Re 1:20.>
<1:13 One like unto the Son of man; compare Da 7:13, where "one like the Son of man came with the clouds of heaven, and came to the Ancient of days." In both cases it is Christ, who, when on earth, called himself "the Son of man." In the description of his person that follows, the writer combines what is said of "the Ancient of days," that is, God, Da 7:9, and of the "man clothed in linen," Da 10:5,6. Thus he ascribes to Christ the characters of deity.>
<1:15 Fine brass; the word used in the original is generally thought to denote a mixed metal composed of gold and silver, and distinguished for its brilliancy. As if they burned in a furnace; shining with intense brightness.>
<1:16 Seven stars; representing the angels of the seven churches, verse Re 1:20. A sharp two-edged sword; with which he smites the nations, chap Re 19:15; compare Isa 11:4; 49:2, which are also prophecies of the Messiah. The symbol denotes the efficacy of his doctrine, and of the judgments uttered by him against the wicked.>
<1:17 As dead; being overcome by the divine majesty and glory of the Redeemer. The first and the last; a direct ascription to himself of the attributes of deity. See Isa 43:10; Isa 44:6. A full view of the Saviour's glory would be more than any man in his life could bear; and in the future life, while it will be unfolding to the admiring eye of his people with greater and greater clearness for ever, all that they will see will only enlarge their conceptions of the infinitude of what remains unseen.>
<1:18 The keys of hell and of death; supreme power over hell and death. Hell, in the original Hades, is here the place of the dead.>
<1:19 The things which thou hast seen; in the vision just described. Which are; the present state of the seven churches, chaps Re 2.1-3.22. Which shall be; the revelations of future events which he is about to receive.>
<1:20 The seven stars are the angels of the seven churches; probably their spiritual leaders. Are the seven churches; represent them. The fact that Jesus Christ said, "The seven candlesticks are the seven churches," does not require us to believe that a candlestick is literally a church; nor do his words, "This is my body," Mt 26:26, require us to believe that bread is literally flesh. What he meant in each case, is that one is an emblem of the other; and it is his meaning, not the mere sound of the words, by which we should be governed.>
<2:1 The seven epistles to the seven churches of Asia have a remarkable agreement in their structure. They all begin with the same form of address, with which is connected one or more of the attributes of the Son of God, as given in the first chapter. Then follow the words, "I know thy works," with reproofs, commendations, warnings, and encouragements adapted to the case of each church. They all close with the solemn call: "He that hath an ear, let him hear what the Spirit saith unto the churches;" and with a promise "to him that overcometh," which varies with each church. In the case of the first three churches, the call to hear precedes the promise; in that of the other four, the reverse is true. We are to understand each address as sustained by all the attributes of the Son of God named in the other addresses, and each promise as including all the good contained in the other promises. From the seven different conditions of the seven churches addressed, arises such a manifoldness and completeness of instruction as adapts these epistles to the spiritual wants of all Christ's churches in all ages. They are expressed with wonderful vividness and power, and should be earnestly studied by all--teachers and taught--who hope to find, at the last day, their names not blotted out of the Lamb's book of life, but confessed by him before his Father, and before his angels, chap Re 3:5. Holdeth the seven stars in his right hand; these words express Christ's supreme power and authority over all the rulers and teachers of his churches. From him they receive it; and to him they must render their account at the last day. Walketh in the midst--candlesticks; words which represent Christ's constant presence with his churches. For their qualifications for usefulness, and for their fidelity and success, ministers and Christians are dependent upon Jesus Christ. He sees their thoughts and feelings as well as their outward conduct, and he requires that they be not only sincerely, but earnestly devoted to his service.>
<2:2 Evil; corrupt in doctrine and practice. Which say they are apostles; false teachers who claimed for themselves the authority of apostles, such as are described by Paul in his second epistle to the Corinthians, chap 2Co 11:13, etc, and foretold in his address to the elders of Ephesus, Ac 20:29,30.>
<2:4 Left thy first love; for the abatement of which no steadfastness in outward services can be a compensation, since it is the heart that Christ desires.>
<2:5 Do the first works; devote thyself as earnestly and heartily to my service as at the beginning. Remove thy candlestick; extinguish the light of thy church--an awful warning which Christ fulfilled long ago to the church in Ephesus, that has been for centuries extinct, and which he has fulfilled to many unfaithful churches since.>
<2:6 Nicolaitanes; a corrupt sect, who seem to have turned Christian liberty into licentiousness.>
<2:7 Overcometh; in the conflict with sin. Compare Mt 24:13; Eph 6:13. The tree of life; compare chap Re 22:2. To eat of the tree of life in the earthly paradise was to our first parents the token. By sin they lost the right to eat of it, and fell under the sentence of death. But Christ restores what was lost in Adam in a higher and nobler form.>
<2:9 But thou art rich; spiritually rich, notwithstanding thy deep poverty in temporal things. Here, as in all the epistles, the angel of the church represents the church itself, and what is said to him is said to the church also. Say they are Jews; a corrupt sect of Judaizers seems to be referred to. They boasted of their Jewish origin, and magnified the institutions of Judaism, but lacked the spirit of true Jews, Ro 2:28,29, and had, instead of it, the spirit of Satan. To true believers, Christ is a faithful, ever-present, all-sufficient friend; making them rich in the deepest poverty, honorable in the greatest abasement, and blessed in the heaviest trials. 2Co 4:17,18.>
<2:10 The devil shall cast some of you into prison; you shall be cast by his instigation. Ten days; a symbolical designation for a short time. The persecution of Christians on account of their religion is instigated by Satan; and those who engage in it are his servants, doing his work, and ripening for the place prepared for him and his angels. Mt 25:41.>
<2:11 The second death; the punishment of the wicked in the future world.>
<2:12 The sharp sword with two edges; see note to chap Re 1:16.>
<2:13 Satan's seat; the place in which and from which he exerts great influence.>
<2:14 Hold the doctrine of Balaam; imitate Balaam in their spirit and teachings. When Balaam could not obtain permission to curse Israel, he counselled Balak to seduce the Israelites to fornication and idolatry through the agency of the women of Moab. Nu 25:1-9; 31:16; 2Pe 2:15,16; Jude 11,12.>
<2:17 The hidden manna; the true spiritual manna laid up in heaven for Christ's faithful servants; alluding to the literal manna that was laid up before the Lord in holy of holies. Ex 16:32-34. A white stone; there is a reference here to the practice in common use among the ancients of making inscriptions on small stones for various purposes. White is the color of victory. A new name written; expressive of the new character and new privileges bestowed upon the bearer. No man knoweth, saving he that receiveth it; an intimation that the love of God shed abroad in the hearts of his children here, and the heavenly inheritance of which it is the foretaste and earnest, can be known only by possession; perhaps, also, that each child of God has his own individual experience, which he alone can understand. The blessedness of true religion is great beyond description, and known only to those who enjoy it.>
<2:19 Last--more than the first; instead of declining, they had increased in good works.>
<2:20 Jezebel; a wicked woman like the wife of king Ahab; pretending to be a religious teacher, yet seducing the people into error and sin. The doctrine that she taught was the same as "the doctrine of Balaam," chap Re 2:14.>
<2:22 I will cast her into a bed; a bed of sickness, instead of the bed of fornication which she encourages. Thus Christ will punish her and her followers with great judgments.>
<2:23 Her children; her followers. The discrimination of character which Christ will make as to each individual, will show his complete knowledge of the heart, and that nothing has ever been thought, said, or done, with which he was not perfectly acquainted.>
<2:24 The depths of Satan, as they speak; this Jezebel and her followers were in the habit of speaking of the depths of knowledge which they possessed; but the apostle calls them the depths of Satan.>
<2:26 Over the nations; they shall not prevail against him, but he shall prevail over them.>
<2:27 He shall rule them; reign with Christ over all his foes. As I received of my Father; Ps 2:8,9.>
<2:28 Give him the morning-star; to shine in glory with Christ the true Morning-star. Chap Re 22:16.>
<3:1 A name; the merely outward profession and form of religion. Persons may be regular in the outward form of religion, and yet destitute of its spirit. Without an effectual change, such cannot escape the judgments of God.>
<3:2 Things which remain; their remaining attachment to truth and duty.>
<3:3 Received and heard; the blessings bestowed, and the truths inculcated upon them. As a thief; suddenly and unexpectedly.>
<3:4 Not defiled their garments; not embraced error or indulged in sin. In white; a state of purity and blessedness.>
<3:5 I will not blot out his name--I will confess his name before my Father, and before his angels; Mt 10:32,33; 25:34-40. Character is personal; and amid great and abounding iniquity individuals may faithfully serve God and be ripening for glory.>
<3:7 The key of David; that is, the key of the house of David. Compare Isa 22:22, from which passage the language is borrowed, but with a far higher application. To have the key of David's house, is to exercise supreme dominion there, which is expressed by the acts of opening and shutting at will; in other words, it is to be supreme on David's throne. Compare Lu 1:32,33: "The Lord God shall give unto him the throne of his father David: and he shall reign over the house of Jacob for ever;" where the "house of Jacob" is the true spiritual Israel, embracing all who have through faith become "Abraham's seed, and heirs according to the promise," Ga 3:29.>
<3:8 An open door; in allusion to the declaration of the preceding verse: "he that openeth, and no man shutteth." The words seem to mean full liberty in professing and preaching the gospel. Compare 1Co 16:9; 2Co 2:12; Col 4:3.>
<3:9 Synagogue of Satan--say they are Jews; see note to chap Re 2:9. Worship before thy feet; humble themselves before thee. God can at any time humble the most bitter persecutors of his people, and make them their cordial friends, or utterly destroy them.>
<3:10 The hour of temptation; a season of fiery trial, apparently in the shape of severe persecution.>
<3:11 Thy crown; the crown of life which Christ will bestow upon all who continue faithful to him.>
<3:12 A pillar in the temple of my God; give him a permanent place in God's spiritual temple. Compare Eph 2:20-22; 1Pe 2:5. The name of my God--the name of the city of my God--my new name; thus marking him as belonging for ever to God, to the city of God, and to Christ, who has redeemed him by his own blood. The new name of Christ is that which belongs to him in his glorified state as the conqueror of death and all the powers of darkness. It therefore marks its possessor as admitted to share Christ's glory with him. Compare verse Re 3:21; Joh 17:24.>
<3:14 The Amen; he who will cause all his words to be accomplished. The beginning of the creation; its Author and Lord. See notes to Col 1:15-17.>
<3:15 Neither cold nor hot; lukewarm, indifferent in religion. Thou wert cold or hot; madest no pretension to my service, or else wert zealous in it.>
<3:16 Spew thee out; reject with abhorrence. God abhors indifference in religion no less really than he does open infidelity or open immorality.>
<3:17 I am rich; have knowledge and religion enough. Wretched; on account of their ignorance of their wants and their indifference to religion.>
<3:18 Gold--white raiment--eye-salve; representing the rich spiritual blessings which Christ will give to those who look to him. Isa 45:22. The more cold and formal men are in religion, the more self-confident they are--the less they feel their need of Christ and his salvation; and without a great change, they will never obtain the blessings of his favor.>
<3:19 I rebuke and chasten; to deliver them from sin, and prepare them for heaven.>
<3:20 I stand at the door; representing his readiness and desire to bestow all needed good upon all who serve him. Sup with him, and he with me; which would be to their rich mutual joy. Christ is ready to save men; but in order to be saved by him, they must receive him in faith and love as their Redeemer, and devote life to his service. If they are lost, it will be on account of their opposition to him, and their refusal to accept of his salvation.>
<3:21 To sit with me; in a state of high and eternal exaltation and blessedness.>
<4:1 And the first voice--talking with me; or, And the first voice which I heard as of a trumpet talking with me; the voice, namely, mentioned in chap Re 1:10.>
<4:2 I was in the Spirit; rapt in prophetic vision. With the command, "Come up hither," he was immediately carried in vision through the open door into heaven.>
<4:3 A jasper; a precious stone of various colors, as purple, blue, green. In chap Re 21:11, the light of the new Jerusalem is compared with it for brilliancy. A sardine stone; a precious stone of a bright red color. Compare Eze 1:27, where he who sits on the throne has "the appearance of fire." An emerald; a gem of a soft green color. Created objects can but very imperfectly represent the divine majesty and glory of the Lord Jesus Christ. The whole creation can afford, But some faint shadows of my Lord; Nature, to make his beauties known, Must mingle colors not her own.>
<4:4 Four and twenty elders; the representatives of God's people under the Old and New Testament dispensations, twelve for each, answering to the twelve tribes of Israel, and the twelve apostles of Christ.>
<4:5 Out of the throne proceeded lightnings, and thunderings, and voices; representing the awful majesty, holiness, and power of God. Seven lamps of fire burning before the throne; see note to chap Re 1:4.>
<4:6 A sea of glass like unto crystal; chap Re 15:2; an expanse of crystal-line clearness and splendor. It answers to the "paved work of a sapphire stone, and as it were the body of heaven in its clearness," Ex 24:10; and to the firmament "as the color of the terrible crystal" on which the throne of God rested, Eze 1:22,26. Four beasts; rather, four living creatures. The word in the original is different from that applied to the persecuting beasts in chap Re 11:7; 13:1,11, etc. The agreement between these four living creatures and the cherubim of Ezekiel's vision, chaps Rev 1.10, is so remarkable, that we must suppose that in both cases the same thing is represented. In both places they appear as the immediate attendants upon God's throne, of which in Ezekiel they are the bearers; in both places they have the same four faces, only that in Ezekiel each has all the four, while here they are distributed one to each; in both places, moreover, their bodies are full of eyes. In their six wings, and in their ceaseless cry, "Holy, holy, holy," they agree with the seraphim of Isaiah. They seem to represent all the created powers and agencies by which God administers his providential government over the world; which are all pervaded by his omniscient Spirit, and stand ever ready to do his bidding; which all show forth his praises, and execute with unerring certainty his high purposes. Full of eyes; representing their ever wakeful vigilance and discernment of God's will. In Ezekiel they and the wheels by them are all pervaded by the one Spirit of God: "Whithersoever the Spirit was to go they went; thither was their spirit to go," chap Re 1:20. None of God's creatures are omniscient, but his omniscience directs all their movements.>
<4:7 Like a lion; representing power, majesty, and dominion. A calf; a young bullock or ox, an emblem of laborious and patient endurance. Face as a man; indicative of reason, intelligence, and kindness. A flying eagle; denoting swiftness, keen-sightedness, and elevation. Another view of these four faces is, that they represent the heads of the four divisions into which the Hebrews distributed the living creation--man, cattle, beasts, birds--uniting in themselves the powers and attributes of all; in other words, that all which is great and excellent in creation ministers to God's will.>
<4:8 Six wings; expressive of swiftness in executing the purposes of God.>
<4:10 The glory and blessedness of saints in heaven, the clearness with which they discern the will of God, and the alacrity, delight, and perfection with which they obey it, no human language can fully describe, and no man on earth adequately conceive.>
<4:11 That good pleasure of God which gave birth to creation, and constantly upholds it, awakens the liveliest gratitude in the hearts of his people, and will draw forth to him the most ardent ascriptions of glory and honor, thanksgiving and praise, for ever and ever.>
<5:1 A book; a scroll, written on both sides to denote the fulness of its contents, rolled up and sealed with seven seals. Compare Eze 2:9,10. The successive opening of the seals represents the gradual unfolding of the counsels of God in the history of this world. This represented the purposes of God with regard to events which were future, and which no one could know except God, and those to whom he should reveal them.>
<5:2 Loose the seals; so as to read the book, and make known the events described in it.>
<5:3 No one can understand the designs of infinite love, except so far as Christ shall reveal them. All should therefore look to him as their prophet, as well as their priest and king; that by his word they may be made wise to salvation, and be furnished thoroughly for every good work.>
<5:4 Look thereon; so as to learn what it contained.>
<5:5 The Lion of the tribe of Juda; Christ, who sprung from the tribe of Judah, and unites in himself the attributes of the lion and the lamb. The Root of David; see note to chap Re 22:16. Hath prevailed; literally, hath conquered. He has conquered death and hell, been exalted to the right hand of God, and received all power in heaven and earth, in which is included the right to unseal the book of God's decrees.>
<5:6 Seven horns; the symbol of perfect power. Seven eyes; the symbol of perfect knowledge. They are immediately explained to mean the seven spirits of God; that is, the Holy Spirit in his fulness of wisdom, sent forth by the Father and the Son. Joh 14:16,26; Joh 15:26; 16:7; Ac 2:33. Exceedingly diverse as well as infinite glories unite in the Son of God: the lion and the lamb; power and meekness; riches and poverty; authority and subjection; majesty and love; dignity and condescension; justice and mercy; holiness and grace.>
<5:8 The four beasts and four and twenty elders; here, as in verse Re 5:14, and in chapter Re 4:9,10, the four living ones begin the adoration of God, and then the four and twenty elders follow. Golden vials; rather, golden bowls or goblets. The harps, the bowls, and the new song seem to belong to the elders, and not to the four living creatures. Compare verse Re 5:14, where the worship of the four living creatures is distinguished from that of the four and twenty elders. Which are the prayers; showing the acceptableness to God of humble, believing, affectionate, and fervent prayer.>
<5:9 A new song; the song of redeeming love, through the atonement and righteousness of Christ.>
<5:10 Kings and priests; see note to chap Re 1:6. On the earth; ultimately over all the earth, and for ever in heaven.>
<5:11 Many angels; they are distinct from the four living creatures, though in a certain sense, included in them as a part of God's universal creation, just as the multitudes of the redeemed in chap Re 7:9; 14:1; 15:2; 19:6, are included in the four and twenty elders, who represent the entire church of God in all ages.>
<5:12 Power--blessing; let the reader notice the seven ascriptions--power, riches, wisdom, strength, honor, glory, blessing--which represent the fulness of adoration given to the Lamb, because in him dwells the fulness of the godhead.>
<5:13 Every creature; the song of adoration that began with the living creatures and elders, and was then taken up by the angels, now spreads itself through the whole created universe; and as the echo of it comes back to the throne of God, the living creatures say, Amen, and the elders fall down and worship. In paying divine honors to Christ, Christians on earth imitate saints and angels in heaven, and prepare to mingle in their society, join in their employments, and partake of their joys for ever.>
<6:1 Here begin the proper prophecies of the book extending onward from the writer's day to the end of the world. For the general plan of the series and the principles on which its several parts are to be interpreted. In the present chapter every thing depends on the interpretation of the sixth seal. There are those who suppose that the seven seals and the seven trumpets run, either wholly or in part, parallel with each other in time, each carrying the history of the church and the world down to the era of millennial glory. Such of course apply the sixth seal to the mighty revolutions, commotions, and overturnings that immediately precede the millennial reign of Christ. But it seems impossible to reconcile this view with the plain words of the apostle in chap Re 8:1,2, which represent the seven trumpets as included under the seventh seal, and therefore following the sixth. Taking then this latter as the true view, we may inquire to what great event in past history the sixth seal applies. They who suppose that the Apocalypse was written before the destruction of Jerusalem very naturally refer the sixth seal to that awful catastrophe, and they find an interpretation of the five preceding seals in our Saviour's words which describe the signs preceding that event, Mt 24:6-14, where the triumphant progress of the gospel amidst wars, famines, earthquakes, pestilences, and bitter persecutions, is set forth, and the great catastrophe itself is described, verse Mt 24.29, in imagery remarkably agreeing with that of the opening of the sixth seal. If, according to the more usual supposition, this book was written after the destruction of Jerusalem by the Romans, A.D. 70, there are but two events to which the sixth seal can, with any degree of probability, be referred--the overthrow of paganism by Christianity in the first half of the fourth century, or the dissolution of the old Roman empire by the invasion of the northern barbarians. The imagery employed seems more appropriate to the latter event than to the former. If we apply the sixth seal to that mighty revolution by which the fact of the civilized world was permanently changed, it will be best to understand it as representing the dissolution of the old Roman empire, not in its successive stages, but in its entireness; in other words, the breaking up of that great central power which had, for so many centuries, kept the world in subordination, thus preparing the way for the series of desolating invasions from the north which had their origin in the decay of the Roman state, and which completed the work of its destruction. One of the four beasts; according to the interpretation of the four living creatures that has been given, that they represent the sum of the created powers and agencies by which God administers his providential government over the world, the call to "come and see" proceeding from them will signify that the events predicted are of a providential character.>
<6:2 A white horse; here, as in Zec 1:8; Zec 6:1-8, the horses denote dispensations, the character of which is indicated by their color and the other emblems employed. A white horse is the symbol of victory. The rider plainly represents Christ. It is therefore a symbol of victory and under his guidance, and redounding to the enlargement of his church.>
<6:4 Red; an emblem of war and bloodshed. Men left without restraint to the indulgence of their lusts and passions, become the tormentors and destroyers of one another.>
<6:5 Black; a symbol of devastation, mourning, and woe. Balances; indicating that food would be but scantily supplied.>
<6:6 A measure; about enough to sustain a man for a day. A penny; the price of a day's labor. Hurt not the oil and the wine; these would be needed to keep men from starving, so great would be the scarcity of food. Men are dependent on God for the blessings of this life, as well as the life to come. Without his aid, the earth will not yield her increase, and men cannot obtain the necessary means of subsistence.>
<6:8 A pale horse; the original denotes the ghastly paleness of a corpse. By this awful symbol destruction in multiplied forms is indicated. Hell; that is, Hades, the abode of the dead. Hades follows death to swallow in its abyss those whom death has slain. The fourth part of the earth; see note to chapter Re 8:7. With sword--hunger--death, and with the beasts of the earth; four destroying agents to slay the fourth part of men. Compare Eze 14:21, from which the imagery is taken; also Jer 15:3, where also four destroyers are named. Not only famine, but pestilence and all destructive judgments are under divine control; and whenever God pleases, he can desolate cities, sweep off nations, and consign their inhabitants to utter ruin.>
<6:9 The souls of them that were slain; the souls of the martyrs in Christ's cause represent a period of severe persecution. These are seen under the altar, which may mean either the altar of burnt-offering in the court before the temple, or the altar of incense in the outer sanctuary. If, as seems probably, the altar of burnt-offering is meant, the idea will be that they have been sacrificed on God's altar as victims in his cause, and their blood poured out beneath it. Those who understand the altar of incense, which was the symbol of intercessory prayer, explain their position from their words as recorded in verse Re 6:10.>
<6:11 White robes; expressive of victory and blessedness. Should rest yet for a little season; an intimation that the full time for avenging their blood has not yet come, but that more must first be added to their numbers. Persecutors, by putting Christians to death, do not annihilate them or their influence.>
<6:12 When he had opened the sixth seal; according to either of the interpretations of this seal above given, the course of events indicated in the preceding five seals had a remarkable fulfilment in history. For an account of the events preceding the overthrow of paganism by Christianity, and of the old Roman empire by the northern invaders, the history of the decline and fall of the Roman empire should be studied, with the fuller commentaries on the Apocalypse, in which the interpretation of these prophetic symbols is discussed at large. 12-14. Earthquake--the sun became black--moon became as blood--stars of heaven fell--the heaven departed--every mountain and island were moved; here as often elsewhere, symbols of great commotions, dissolutions of civil governments, fall of illustrious men, and multitudes overwhelmed in ruin. Compare Isa 13:10; 24:19,20,23; 34:4; Jer 4:23-25; Eze 32:7,8; Joel 2:12; 3:15,16; Am 8:9; Mt 24:29; with the notes on those passages.>
<6:15 Hid themselves; under the judgments of God, fled, and attempted by concealment to elude the search of their destroyers.>
<6:16 Full on us, and hide us; representing their great consternation when Christ should appear, in answer to the prayers of the martyrs, to deliver his people and take vengeance on their foes. Compare Ho 10:8. When Christ comes to take vengeance on his foes, they can neither elude nor withstand him. No dens nor caverns, rocks nor mountains, can hide them; nor can any created power screen them from the indignation of him who sitteth upon the throne, and from the wrath of the Lamb.>
<7:1 After these things; after the events of the sixth seal. The four corners; east, west, north, and south. Holding the four winds; restraining the fury of human passions, and producing a period of calmness and quiet.>
<7:2 Ascending from the east; perhaps as the seat of the rising sun. Having the seal; to mark the servants of God, that they might be distinguished from others.>
<7:3 Hurt not; let not the troubles come till we have placed God's seal or mark upon his people. Desolating judgments are often delayed for a season, to give opportunity for the spread of the gospel, and for the gathering in of such as embrace it to the church of Christ.>
<7:4 A hundred and forty and four thousand; a definite is here put for an indefinite, but very large number, who had embraced the gospel and were made partakers of divine grace, and thus were sealed by the Holy Spirit to the day of redemption. Of all the tribes of the children of Israel; Israel is here "the Israel of God," including all, whether Jews or Gentiles, who are Abraham's children in a spiritual sense. In the enumeration of the twelve tribes that follows, Dan is omitted, and Joseph is reckoned once in Manasses and again for Ephraim. None are secure from coming wrath, except those who are born of God, who manifest the fruits of his Spirit, and are thus distinguished as belonging to him.>
<7:9 Stood before the throne; saved through the preaching of the gospel, not from among the Jews only, but from all nations; showing the spiritual progress of the gospel during the events that have been symbolically set forth. Palms; in token of their victory over sin, sorrow, and death.>
<7:10 Saints in heaven take a deep interest in the concerns, and greatly rejoice in the triumphs of saints on earth, and with them unite in ascribing the glory of their salvation to God and to the Lamb.>
<7:11 Worshipped God; in view of the wondrous manifestation of himself in the salvation of his people.>
<7:12 Blessing--and might; notice again the number seven, as in chap Re 5:12.>
<7:13 What; who. These; the redeemed sinners whom John saw in heaven.>
<7:14 Great tribulation; distressing trials which they endured on account of their religion. Made them white in the blood of the Lamb; cleansed from sin and made righteous, through faith in the atoning blood and perfect righteousness of Jesus Christ.>
<7:15 Therefore; on account of their union by faith to Jesus Christ, and its effects in purifying their hearts and preparing them for heaven. The ground of human salvation is the atonement of Christ, and faith in him is the means of obtaining it. This is acknowledged by saints on earth and in heaven.>
<7:16 Hunger no more; endure no more evils of any kind.>
<7:17 Wipe away all tears; remove all sorrows, and fill them with perfect joy for ever.>
<8:1 Silence in heaven--half an hour; indicating quiet for a short time, eager expectation of what was to follow, and silent aspirations to God.>
<8:2 Seven angels; messengers prepared to do the will of God. Seven trumpets; instruments of alarm, and indications of approaching wars and desolations.>
<8:3 Another angel; supposed by many to be the Messenger of the covenant, the High-priest of our profession, Jesus Christ, offering the petitions of this people, and making intercession for them. Heb 7:25. These petitions manifestly have reference to the impending judgments that are about to fall on the earth, and so long as they are continued the judgments are delayed, verse Re 8:1. The object of the petitions we may understand to be, as in chapter Re 6:10, the avenging of the blood of the saints. Censer; a pan, or small vessel, in which incense was burnt, and from which arose a smoke of fragrant odor.>
<8:4 Ascended up before God; in token of the acceptance of their prayers. The prayers of saints being presented by the great interceding Angel, and perfumed with his merits, ascend with acceptance before God, and will be answered in rich and lasting blessings on his friends, and in the ruin of his foes.>
<8:5 The censer; with which he had offered incense. Filled it with fire; a symbol of the divine wrath about to be inflicted on the wicked. Cast it into the earth; as the place where the divine judgments were to be executed. Voices--earthquake; all symbols and precursors of the coming judgments, and the commotions and overturnings connected with them.>
<8:6 The seven trumpets; in accordance with the view that has been given above, the four trumpets of the present chapter are commonly understood as emblematic of the successive invasions by which the destruction of the western empire was completed; while the two woe-trumpets that follow in the next chapter relate to the overthrow of the eastern empire by the Saracens and Turks.>
<8:7 Hail--fire--blood; symbols of slaughter and ruin. The third part; a definite part to denote a large part. Compare Eze 5:2,12.>
<8:9 The third part--died; and the third part of the ships were destroyed; showing that great numbers would perish, business be suspended, and vast amounts of property be destroyed.>
<8:11 Wormwood; indicating the bitter and fatal distresses which the presence of this star would produce upon the wicked, especially the persecutors of God's people. Continuance in sin inevitably leads to misery; and the greatness of the numbers, wealth, and power of persevering transgressors will do nothing towards diminishing the certainty, the greatness, or the perpetuity of their torment.>
<8:12 Third part of the sun--moon--stars; for the darkening of the heavenly bodies as the symbol of the overthrow of nations, see above note to chap Re 6:12-14.>
<8:13 By reason of the other voices; because the calamities which they would indicate would be exceedingly great and destructive. Interpreters generally apply the preceding four trumpets to the four principal invasions of the barbarians--of the Goths under Alaric, of the Vandals under Genseric, of the Huns under Attila, and of the Heruli under Odoacer, extending from about A.D. 410 to A.D. 476. The details must be sought in the history of these times, and in the more extended commentaries on the Apocalypse. However great or long continued the calamities of the wicked in this world, they are only warnings and foretastes of greater and more lasting calamities which, if they continue in sin, they will suffer in the world to come.>
<9:1 By reason of the other voices; because the calamities which they would indicate would be exceedingly great and destructive. Interpreters generally apply the preceding four trumpets to the four principal invasions of the barbarians--of the Goths under Alaric, of the Vandals under Genseric, of the Huns under Attila, and of the Heruli under Odoacer, extending from about A.D. 410 to A.D. 476. The details must be sought in the history of these times, and in the more extended commentaries on the Apocalypse. However great or long continued the calamities of the wicked in this world, they are only warnings and foretastes of greater and more lasting calamities which, if they continue in sin, they will suffer in the world to come.>
<9:2 A smoke out of the pit; the smoke arising out of the bottomless pit and darkening the sun and the air may be taken as an apt emblem of the Mohammedan delusion. Out of this smoke come the swarms of locusts which well represent the hosts of the Saracens; for these fierce invaders had their origin in this satanic delusion, and were thoroughly animated by its spirit. The star fallen from heaven that opens the bottomless pit will then be Mohammed, the introducer of this pestilent superstition, with all who aided and abetted him in it. The description of the star as fallen from heaven, is thought by many to symbolize the fact that Mohammedism had its occasion in the deep corruption of Christianity that preceded it.>
<9:3 As the scorpions of the earth have power; see notes to verse Re 9:5,10. Infernal spirits are ever ready, when permitted, to increase the ignorance, wickedness, cruelty, and wretchedness of men. But they are under divine control, and can proceed no further than God, for wise and good reasons, sees fit to suffer them.>
<9:4 Not hurt the grass--any tree; contrary to the nature of natural locusts, showing that these locusts represent cruel enemies sent by God to scourge those men which have not the seal of God; such as are not true Christians, and have rejected divine truth.>
<9:5 Five months; the period of the duration of natural locusts. It here denotes a time appointed and limited by God; according to some, 150 years--a day being taken for a year--which was about the period during which the Saracens extended their conquests, though their empire lasted much longer. As the torment of a scorpion; compare verse Re 9:10. The Saracens were cruel and bigoted conquerors, compelling all to receive their pestilent superstition under the penalty of death or tribute. This seems to be especially the torment of their stings. Wherever they went they left behind them the poison of their false religion.>
<9:6 Seek death; as a relief from the calamities brought upon them by these cruel invaders. When wicked men here suffer a part only of the evils which their sins deserve, life itself often becomes a burden, and they seek for death to relieve them. But there is effectual and permanent relief only in forsaking their sins and turning heartily to the Lord, who will then abundantly pardon.>
<9:7 7-9. Like unto horses--hair--teeth--breastplates--wings; compare the description of locusts in Joe 2:4,5. The Arabs wore their hair long, with turbans of gay colors, which seems to be enigmatically set forth by their having "crowns like gold," and "hair as the hair of women." That a part of the characters given should be intended to identify them from their personal appearance, while another part represents their qualities as warriors, is not unnatural in such a symbolic description as the present.>
<9:11 A king over them; representing the succession of their caliphs. Abaddon--Apollyon; that is, Destroyer, as both names signify. In corrupting and ruining men, the wicked on earth and in hell unite under one great leader; showing that they belong to the same company, are engaged in the same work, and are preparing for the same torment.>
<9:12 One woe is past; one of the three woes foretold in chapter Re 8:13. Two woes more hereafter; an intimation of their separation from each other by noticeable intervals of time, while the woes of the first four trumpets came in immediate succession and were partly blended with each other.>
<9:13 The golden altar; the altar of incense, which stood in the outer sanctuary immediately before the ark of the covenant where God dwelt between the cherubim, and from which it was separated by the inner veil.>
<9:14 Loose the four angels; representing desolating powers which in the course of providence had been restrained, but were to be suffered for a time to scourge, desolate, and destroy a great portion of the earth. The number four may be here, as in chap Re 7:1, a symbol of universality. In the great river Euphrates; a symbol of the region whence these four angels should come. The Turks or Othmans, to whom this woe-trumpet seems to refer, came from the vicinity of the Euphrates.>
<9:15 An hour--a day--a month--a year; that is, 391 days, and the twelfth part of a day--interpreted by many of so many prophetic years during which they should extend their conquests, which ended with the fall of Constantinople, A.D. 1453. God's messengers of vengeance are often for a season restrained; but when restraint is removed, they commence their work of desolation.>
<9:16 Two hundred thousand; a definite, for a very large indefinite number.>
<9:17 Fire--jacinth, and brimstone; of red, purple, and yellow color. Supposed by many to be an enigmatical description of the Othman cavalry, with whom these were favorite colors.>
<9:18 Fire--smoke--brimstone; symbols of their awfully destructive powers.>
<9:19 In their tails; like the locusts that preceded them they do injury with their tails, taking up and propagating by force the same pestilent superstition.>
<9:20 The rest of the men; men in the countries which were overrun by those destroyers who were not killed. Repented not; this and the preceding judgment had no influence to bring them to repent of their worship of demons and idols.>
<9:21 Murders--sorceries--fornication--thefts; by continuing to commit these various crimes, they were ripening for still further manifestations of divine wrath. No judgments of heaven which men endure will, without the grace of God, lead them to repentance, make them holy, or fit them for heaven.>
<10:1 The tenth and eleventh chapters of the Apocalypse may be regarded as an episode, referring to the history and sufferings of Christ's church during the time of the preceding woe-trumpets, and until the sounding of the seventh trumpet. Angel; this angel seems to be the Son of God, or an emblematical representation of his glory. Compare chap Re 1:13-16; 14:14. Clothed with a cloud; chap Re 1:7; 14:14; Mt 24:30; Ac 1:9; 1Th 4:17.>
<10:2 A little book open; containing the revelations of this and the following chapter to verse 15, the seventh trumpet. Upon the sea, and--on the earth; in token of supreme dominion over both.>
<10:3 Seven thunders uttered their voices; each thunder containing, like each of the preceding trumpets, a revelation of some coming event.>
<10:4 Write them not; we cannot therefore know their contents, unless, as some suppose, they are coincident with the seven last plagues. Christ graciously communicates to his people, or gives them the means of learning, all that it is here best they should know; and the knowledge which would only injure them he wisely withholds.>
<10:6 That there should be time no longer; or, that there should be delay no longer; that is, as immediately explained, no longer after the sounding of the seventh angel. No one of the preceding trumpets has brought a fulfilment of the mystery of God, but the seventh trumpet shall finish it.>
<10:7 The mystery of God; his glorious plan for overthrowing the kingdom of Satan, and establishing the kingdom of Christ, which is the great theme of the Apocalypse. Though many things which God has promised by his prophets are for a time delayed, yet in due season they will all be perfectly accomplished. Till then his people should labor, and if need be suffer, with patience and in hope.>
<10:9 Eat it up; a symbol for attentively reading, thoroughly understanding, and diligently considering what it foretold.>
<10:10 Sweet as honey--bitter; the reception of the revelation was pleasant, but its contents filled him with distress, for they related to the afflictions of God's people. Compare, for this whole symbol of eating the book, Ezekiel's eating the roll, Eze 3:1-3. Joys and sorrows will be intermingled in coming events. They should be met as they occur, with submission and gratitude; and if rightly improved, they will both conspire to work out an exceeding and eternal weight of glory.>
<10:11 Before many peoples; concerning them, and what should in future befall them. John in his writings was to reach many and remote lands that he himself never visited.>
<11:1 The present chapter gives the contents of the little book expressed in two striking emblems, the measuring of God's temple, and the prophesying of the two witnesses. A reed like unto a rod--measure the temple--and them that worship; compare Eze 40:3, etc. The holy city, Jerusalem, with its temple and court, represents the body of those who profess Christianity: measuring denotes God's act of acknowledgment and approval; leaving unmeasured, his act of rejection. The temple and altar therefore, with their attendant worshippers, represent "the Israel of God," whom he owns as his true people; while the outer court of the temple and the city thronged with Gentiles, represent the multitude of both church officers and people who are Christian only in name. The whole symbol represents a period during which there would be some spiritual worshippers among the professed followers of Christ, while multitudes would be given up to spiritual darkness, idolatry, and death.>
<11:2 Forty and two months; the period of the duration of the beast that rises out of the sea, chapter Re 13:5, where see the notes. The time of the trampling under foot of the holy city, and that of the prophesying of the two witnesses, both agree with the continuance of the two beasts of chapter thirteen. The difference in character between sincere worshippers of God and those who oppose him or worship him only in name, he perfectly knows, and he will make a corresponding difference in their condition for ever.>
<11:3 My two witnesses; representing the few who continued faithful to God during this long period of general apostasy. Two witnesses are probably named, because two were required by the Mosaic law to constitute valid testimony. De 17:6; 19:15. A thousand two hundred and threescore days; 1260 days, the same as "forty and two months," verse Re 11:2, reckoning thirty days to a month. Clothed in sackcloth; expressive of their afflicted and persecuted condition.>
<11:4 The two olive-trees--the two candlesticks; compare Zec 4:2-6,11-14, from which the imagery is taken, but with free changes. Oil is a symbol of divine grace: a lamp replenished with oil and shining brightly, represents the light of a holy life and holy doctrine. The two witnesses are God's two olive-trees and two candlesticks, because they are the repositories of his grace, and the lights which he has appointed to shine in this dark world.>
<11:5 5, 6. The images of these two verses represent the jealous care with which God watches over his faithful servants, and the punishment with which he visits their persecutors. Fire proceedeth out of their mouth; an allusion to the act of Elijah in calling down fire from heaven. 2Ki 1:10,12. To shut heaven, that it rain not; as Elijah did by his intercession with God. 1Ki 17:1; Jas 5:17. To turn them to blood--smite the earth with all plagues; as Moses did at God's command. Ex 7.1-12.51. Though God bears long, and for a time bestows many favors upon the wicked, not willing that they should perish, but that they should come to repentance; yet, in the end, if they turn not, he will whet his sword, his hand will take hold on judgment, and there will be none to deliver. De 32:41.>
<11:7 The beast that ascendeth out of the bottomless pit; Satan and his emissaries, or persecutors instigated by the evil one. See further on chap Re 13.1-18. The word rendered beast here, and in chap Re 13.1-18, is a different word in the original from that in chap Re 4:6. There it means living creatures; here, a wild, savage beast. Shall overcome them, and kill them; the various conjectures concerning the slaying of the witnesses are uncertain. The time has not yet come for the clear understanding of this prophecy.>
<11:8 The great city; the seat of the persecuting power; supposed to be Rome, or places distinguished for wickedness under her control.>
<11:9 Three days and a half; a limited and short period. Not suffer their dead bodies to be put in graves; showing the dishonor and contempt with which the faithful servants of God would be treated, not only while they lived, but after they were dead.>
<11:10 Make merry; in prospect of being in future freed from the influence of those whom they hated, and had slain.>
<11:11 The Spirit of life from God entered into them; they were spiritually resuscitated. New and faithful servants of God were raised up, religion greatly revived, and the number of those who embraced it so multiplied, that the blood of the martyrs was seen to be the seed of the church.>
<11:12 Ascended up to heaven in a cloud; indicating the honor God bestowed upon them, and the special favor with which he treated them. The children of God, in bearing testimony for him, will live till their work is accomplished; and though they should come to a violent and ignominious death even in great numbers, yet God will raise up others to fill their places, cause his kingdom to triumph, clothe his friends with honor, and cover their opposers with confusion and disgrace.>
<11:13 The same hour; with the resurrection of these witnesses. A great earthquake; see note to chap Re 6:12-14. Slain of men seven thousand; literally, seven thousand names of men, representing a great destruction of the wicked high in power and place, who had hated and killed the saints. Chap Re 13:10.>
<11:15 Great voices in heaven; rejoicing over the rapid and triumphant spread of the gospel.>
<11:17 In the spread of the gospel and the multiplication of those who embrace it, in the honor of the saints and their triumph over all who oppose them, the inhabitants of heaven greatly rejoice, and render fervent thanksgiving to God.>
<11:18 The time of the dead, that they should be judged; probably meaning the time when the pious dead, who have been slain for Christ's sake, shall be avenged. Reward--thy servants--and--destroy them which destroy the earth; save his friends and destroy his enemies, especially those who had been engaged in destroying his people.>
<11:19 The temple of God was opened in heaven; this verse belongs in all probability to the following series of prophecies, which it appropriately introduces. Compare chap Re 4:1, "A door was opened in heaven." But here the holy of holies is laid open, where God dwelt between the cherubim of the ark; apparently indicating that the apostle is about to receive a more interior and spiritual view of the condition and conflicts of the church. See the remarks prefixed to the next chapter. The ark of his testament; the same as the ark of the covenant. Ex 25:10-22. It was the symbol of God's immediate presence, and of the certain fulfilment of his promises. Lightnings--thunderings--earthquake, and great hail; emblems of God's presence, and of the judgments about to be executed on the persecutors of his people.>
<12:1 With the seventh trumpet the mystery of God was to be finished. Chapter Re 10:7. This has already sounded, and "the kingdoms of this world are become the kingdoms of our Lord, and of his Christ." Chap Re 11:15. We cannot therefore, with any degree of probability, suppose that the long series of persecutions and trials predicted in this and the following chapters belongs to the seventh trumpet. Both the numbers contained in these prophecies, and their general character, identify them with those previously recorded. Accordingly there is a general agreement among expositors that the vision here goes back to the primitive days of Christianity, and gives a new series of revelations containing a more interior and spiritual view of the history of the church, that of the preceding series having been more outward and providential. A woman; undoubtedly a symbol of God's church. Clothed with the sun; with the glory of Christ's presence, and the graces of his Spirit. The moon; according to some, a symbol of all sublunary things; others, with more reason, regard it as a symbol of the less glory of the Mosaic economy. A crown of twelve stars; the twelve apostles of the New Testament, answering to the twelve tribes of the Old. Twelve is the symbol of God's people. Compare chapter Re 21:12,14, where the twelve angels of the twelve gates represent the twelve tribes of Israel; and the twelve foundations the twelve apostles. We can in this world but faintly conceive the glories with which saints in heaven are crowned, and to which, after their days of trial, all true believers will be for ever exalted.>
<12:2 Travailing in birth; a symbol of the fruitfulness of the church in times of great trial. Compare Isa 54:1; 66:8.>
<12:3 A great red dragon; the pagan Roman empire, considered as the agent of the devil, and animated with his spirit. Red or purple was the distinguishing color of the Roman emperors, as it has since been of the popes and cardinals. Seven heads; explained in chap Re 17:9, to mean the seven hills of Rome and her seven kings; that is, as commonly interpreted, the seven forms of government which prevailed in Rome. See note to chap Re 17:9. Ten horns; the ten kingdoms into which the Roman empire was ultimately divided. See Da 7:24, where the Roman empire is represented by the same general symbol. Seven crowns upon his heads; not, as afterwards, upon his horns. The Roman empire is always represented in prophecy in its whole duration. But the seven crowns upon its heads indicate that the seat of empire is yet in Rome; not, as afterwards, in the ten kingdoms which rise out of the old empire. Compare chap Re 13:1, where the crowns are on the horns, and the explanation of the angel, chap Re 17:12.>
<12:4 Drew the third part of the stars of heaven; probably representing the subjection of the kings and rulers of a large part of the world to the Roman power. The enmity between the seed of the woman and the seed of the serpent, Ge 3:15, has always existed, and has been manifested in various ways, especially in the persecution by the wicked of the children of God.>
<12:5 A man-child; this man-child is Christ, the seed of the mystic woman, considered as the head and representative of all his disciples. It includes, therefore, not only him, but all who are through faith united to him. Caught up unto God, and to his throne; representing the exaltation of Christ, and through him the protection of his people and their victory over their enemies.>
<12:6 Fled into the wilderness; spoken here by way of anticipation. See note to verse Re 12:14. God is mindful of his people in all their trials, kindly provides for them all the blessings which he sees best, and will one day give them dominion over all the earth. Da 7:27.>
<12:7 War in heaven; representing the conflict for supremacy between the truth of Christianity and the old system of pagan delusion. Michael seems here to represent all the agencies employed by Christ, as the dragon does the devil acting in and through his agents, especially the persecuting emperors and their servants.>
<12:9 Was cast out; truth and its friends prevailed, and idolatry was overthrown. Was cast out into the earth; excluded from his former position of power and office. The dragon in heaven is thought by many to mean the devil enthroned in the chief place of power; the dragon on earth, to mean the devil cast out of that place, but still active against the church Ver Re 12:13-17. Whoever may be the instruments of persecuting the people of God, Satan is their leader; they are his servants, and are doing his work. Ro 6:16.>
<12:10 The accuser of our brethren; Satan, who, so long as he retained his place of power, persecuted and destroyed Christ's servants by false accusations. In the name here given to Satan, there seems to be an allusion to the manifestation made of his character in the case of Job, chap Re 1:9-11; 2:4,5; and which was again made in the calumnies which he raised against Christians in the primitive ages.>
<12:11 By the blood of the Lamb--by the word of their testimony; not by carnal weapons, but by faith in the efficacy of Christ's atonement, and their faithful testimony to his truth. Loved not their lives unto the death; would not renounce the truth to save their lives. The children of God will be triumphant, and come off conquerors, and more than conquerors, over all their foes; not by worldly stratagem or force, but by the power of truth and love exemplified in the cross, and set home by the Holy Ghost.>
<12:12 Great wrath; at his overthrow in heaven. A short time; his time to persecute God's people and hinder the truth. The twelve hundred and sixty days that remain to him are short absolutely, and short in comparison with the ages during which he has been the god of this world.>
<12:14 Two wings of a great eagle; representing the assistance granted by God to his church, to escape the rage of her persecutors, or to endure and survive it. The wilderness; the sojourn of the church in the wilderness agrees with the prophesying of the two witnesses in sackcloth. It is another representation of the same thing. A time, and times, and half a time; the same as three years and a half, forty-two months, and twelve hundred and sixty days. See Da 7:25.>
<12:15 The serpent cast out of his mouth water as a flood after the woman; thought by many to represent the inundation of northern barbarians, by which Satan hoped to overwhelm Christianity.>
<12:16 The earth helped the woman; these pagan hosts, instead of destroying God's church, in many ways befriended her. They settled down in the regions conquered by them and embraced Christianity.>
<12:17 The remnant of her seed; representing those who remained faithful to the truth of the gospel. The manner in which he made war upon them is immediately set forth in the following chapter. Sin unrestrained is outrageous, cruel, and persevering. When the agents of Satan are foiled in one way, they try another; and unless changed by the power of the Holy Ghost, however often they may be disappointed, they will continue their opposition for ever.>
<13:1 A beast; the symbol of a great evil and persecuting power. Out of the sea; out of the troubles, commotions, and revolutions of that period. Compare Da 7:2. Seven heads--ten horns--ten crowns; it is a continuation of the same great persecuting power that has been described in the preceding chapter, but at a later age, and in another form. The horns now wear the crowns, not the heads, as before, chap Re 12:1; indicating that the power has been transferred to them. This beast is then identical with the fourth beast of Daniel's vision, and represents the ten kingdoms that arose out of the ruins of the old Roman empire. See Da 7:24. The name of blasphemy; showing his opposition to God and his Christ, and his arrogant assumption of the prerogatives that belong to them.>
<13:2 Leopard--bear--lion; symbolic of his savage and cruel character, which unites in itself the properties of the three first beasts of Daniel's vision. Da 7:4-6. The dragon gave him his power--seat--authority; Satan, who had in past ages made use of pagan Rome as an instrument of persecuting God's church, now transfers to him the same power and authority to be used against Christians.>
<13:3 One of his heads as it were wounded to death; smitten with a mortal wound. Of the various interpretations proposed for this difficult passage, that seems most probable which refers it to the extinction of the old Roman empire under the imperial form in the latter part of the fifth century, and its revival again under Charlemagne, who was, at the close of the eighth century, crowned by the Roman pontiff as emperor of Rome under the title of Caesar and Augustus. See further in notes to verses Re 13:14,15; chapter Re 17:10,11. Wondered after the beast; followed him with wonder and homage, as explained in the following verses. Persecutors, when overthrown in one form, often rise in another, and continue, under the instigation and by the aid of Satan, that accuser of the brethren and murderer from the beginning, to prosecute their work of death.>
<13:4 Worshipped the dragon; by worshipping or paying divine honors to the beast, his agent in persecuting the saints.>
<13:5 Speaking great things and blasphemies; usurping the prerogatives of God, and subjecting men's consciences to his control. This beast exerts his power in connection with the second beast, verses Re 13:12-15, and the two together usurp God's place, and require men to pay to them divine honors. Compare 2Th 2:4. Forty and two months; the same as twelve hundred and sixty days, chap Re 11:3, and "a time, and times, and the dividing of time." Chap Re 12:14; Da 7:25. These days are commonly understood as symbolical of so many years.>
<13:6 His tabernacle; his people among whom he dwells. Them that dwell in heaven; by speaking of them in opposition to their true character, as if they aided in his cruel designs. The mouths which God has made are often opened in blasphemy against him, and the faculties which he has given and preserves, often employed in opposing his cause.>
<13:7 All kindreds--tongues, and nations; showing the extent of his influence, reaching over a great portion of the world.>
<13:8 All; the wicked, who inhabit the countries subject to his power, and who have no true religion. There is no security against embracing the most dangerous errors, and joining in the most abominable practices, except in that distinguishing grace of God which leads men to trust in the Redeemer, and in well-doing to commit the keeping of their souls to him.>
<13:9 Let him hear; consider what is said, and receive the instruction which it is suited to impart; not go with the multitude after the beast, but continue, under all his persecutions, steadfast in the faith and practice of the gospel.>
<13:10 Shall go into captivity--killed with the sword; the time would come when the persecutors would be destroyed. God would render vengeance to his enemies. Till then, his people, while active and persevering in duty, should wait with patience. When persecutors of God's people have gone as far as he sees fit to suffer them, and have accomplished what he intended, he will turn his hand against them, and punish them according to their deserts. Jer 25:9-14; 27:6,7; 50:1-19.>
<13:11 Another beast; representing an ecclesiastical power, pretended ministers of religion uniting with the civil power, described under the first beast, in persecuting the saints. Out of the earth; in a quiet, silent way; an exact description of the rise of the spiritual power of the papacy, which grew up stealthily and by degrees. This beast may be considered as including all the other kindred forms of ecclesiastical domination, which arose side by side with the papacy, and constitute with it one vast system of spiritual tyranny. Like a lamb; professing to be very mild, meek, and humble. Spake as a dragon; showing himself to be the opposite of what he professed to be.>
<13:12 Exerciseth--the power--and causeth the earth--to worship the first beast; unites his ecclesiastical power with the civil power of the first beast in laboring to accomplish the same cruel, selfish, and wicked designs.>
<13:13 Doeth great wonders--maketh fire come down; the probable meaning of these words is, that having, by his signs and lying wonders, deceived the multitude and gained control over them, he uses the power thus acquired to destroy, as if by fire from heaven, those who will not submit themselves to his usurped authority.>
<13:14 An image to the beast; a living representative of his power. According to some, this image represents the succession of Roman pontiffs; but they are rather the representatives of the second beast. Others, therefore, understand the new succession of Roman emperors, referred to in the note to verse Re 13:3, who were animated by the spirit of the papacy, and exerted their power in its interest.>
<13:15 Speak; proclaim and send out his decrees, requiring all to bow to him; and if they will not, causing them to be put to death. Ecclesiastical and civil rulers have often been united in persecuting Christians, and endeavoring to force them to disobey God.>
<13:16 Causeth all; compels them, under the penalty mentioned in the following verse. To receive a mark; a sign to distinguish them as his followers, and as acknowledging his authority.>
<13:17 That no man might buy or sell; literally fulfilled in the history of the papal power, whose policy has been to place those who would not bow to it under an interdict, deprive them of the means of living, and thus starve them into compliance. Satan often acts as if he were the god of this world, and his subjects as if they had a right to govern it. If men will not submit to them, they treat them as worthy of death; thus usurping the prerogatives of Jehovah, and acting as if they were above him.>
<13:18 Here is wisdom; wisdom is required, in order to determine to whom the above representations apply, and who is meant by them. Let him that hath understanding; of this matter. Count the number; of the name of the beast. Six hundred threescore and six; the letters which compose the Greek word Lateinos, signifying the Latin man, when used as numerals, make the number six hundred threescore and six. This is the case also with some other names. But in order to be sure that any one is the true name, it must, not only in this, but in other respects, answer the description given of the beast.>
<14:1 A hundred forty and four thousand; here, as in chap Re 7:4; the representatives of the multitude of the redeemed during the times of trouble and persecution that have been foretold. His Father's name written in their foreheads; in contrast with the worshippers of the beast, who have his mark in their right hand, or in their foreheads. Chap Re 13:16.>
<14:3 A new song; the song of redemption through the blood of the Lamb. Whatever trials believers may suffer in this world, through the aid of their great High-priest and Intercessor they will all arrive safely in heaven, and their arrival will awaken songs of thanksgiving throughout all the hosts above.>
<14:4 Not defiled with women; with spiritual fornication; compare chap Re 17.1-18, where the apostate church is represented as a harlot, and her followers as those who commit fornication with her. Virgins; espoused to Christ as chaste virgins. 2Co 11:2. First-fruits; those who had first been converted, and were the earnest of multitudes who were to follow.>
<14:5 Without fault; having been sanctified and presented spotless before God. Jude 24.>
<14:6 6-11. Now follow three visions, representing the rapid and wide spread of the gospel through the whole earth, the fall of the mystic Babylon, and the punishment of her adherents. With the dawn of the Reformation began the fulfilment of these promises, and it is progressing in our day.>
<14:7 Worship him; worship God. This is said in contrast with the worship of the beast. The persecution of saints even unto death is often followed by a rapid and extensive spread of the gospel, and the greatest rage of opposers by their speedy and utter ruin.>
<14:8 Babylon; compare Isa 21:9. Babylon was the chief seat of persecution against the church of God under the Old Testament; and this name is given to the chief seat of such persecutions under the New Testament. Is fallen; an announcement of the overthrow of this great persecuting power. Drink of the wine of the wrath of her fornication; Babylon is compared to a harlot holding in her hand a wine-cup of wrath, and making all nations drunk with it. The figure is taken from Jer 25:15-28, where God, through the literal Babylon, administers to the nations the wine-cup of his fury. The meaning is, that the mystic Babylon, by seducing the nations to commit spiritual fornication with her, brings upon them the wrath of God. For this sin her doom is here foretold.>
<14:9 Worship the beast and his image; see note to chap Re 13:14. We have here the ecclesiastical uniting with the civil power in killing the saints who refused to submit to him.>
<14:10 The wine of the wrath of God; the effect of God's wrath is here, as often elsewhere, compared to a wine-cup, which produces in those who drink of it reeling and madness.>
<14:11 Union with opposers of God and his cause, for the sake of avoiding present evil, or obtaining fancied good, is aggravated sin, and will be followed with awful punishment.>
<14:12 The patience of the saints; of those who should stand out against all the blandishments, wiles, and threats of the beast and his image, continuing steadfast in the faith of Christ, though it should cost them the sacrifice of life. The words contain a solemn intimation of the severe trials to which God's faithful servants would be subjected during the reign of the beast.>
<14:13 From henceforth; they entered immediately into rest, and were blessed. Of course there was no purgatory for them to pass through; but when absent from the body, they were present with the Lord. 2Co 5:8. There is, to saints, no state either of insensibility or of suffering after death, but they enter at once into rest. The day they leave the body they are happy with Christ. Lu 23:43.>
<14:14 14-20. Now follows a double vision representing the execution of God's vengeance upon the persecutors of his people. A white cloud; the symbol of Christ's presence in power and great glory to take vengeance on the wicked. Compare note to chap Re 10:1. A golden crown; to signify that he is "King of kings, and Lord of lords." Chap Re 17:14; 19:16. A sharp sickle; here, as in Joe 3:13, the ripeness of the harvest and vintage indicates that the measure of man's wickedness is full, and the reaping is a symbol of the execution of God's wrath.>
<14:15 Another angel came out of the temple; the dwelling-place of Jehovah, signifying that he was the bearer of a message from Him.>
<14:17 Another angel came out of the temple; having a commission immediately from God.>
<14:18 The altar; apparently the altar of burnt-offering. Fire; the symbol of God's destroying wrath. Both saints and sinners are continued in this world till they are fully ripe, the one for endless bliss, and the other for endless woe.>
<14:19 Wine-press of the wrath of God; the treading of grapes is expressive of his vengeance upon his enemies.>
<14:20 Without the city; apparently the holy city Jerusalem, which represents the people of God; signifying that they will be exempted from this awful judgment. Blood--even unto the horse-bridles; showing the greatness of the slaughter, and of the multitudes who perished. A thousand and six hundred furlongs; two hundred Roman miles. Some have supposed an allusion to the length of Palestine; others, to the extent of the pope's dominions in Italy. But all such conjectures are uncertain.>
<15:1 Seven last plagues; those which would accomplish the wrath of God against the beast, and result in his final and utter overthrow. Whether these seven plagues are a more detailed account of the harvest and vintage described in the preceding chapter, or follow after them as additional judgments, can be known only by their fulfilment.>
<15:2 Sea of glass mingled with fire; a smooth, transparent pavement, clear as crystal, variegated with fiery colors. Chap Re 4:6. Another glorious vision of the redeemed in heaven is granted to the apostle, to show that during all this period of abounding wickedness, God was still gathering home his faithful servants to the rest and blessedness of his presence.>
<15:3 The song of Moses--and--of the Lamb; praising and adoring God for his deliverance of his people from Egyptian bondage by Moses, and from the bondage of sin by Christ, and for his victories over all their foes.>
<15:4 The salvation of the righteous and the destruction of the wicked are both from God. One is a display of his grace, the other of his justice; and in both he is glorious, blessed, and worthy of everlasting confidence, affection, and praise.>
<15:5 The tabernacle of the testimony; the holy of holies, the peculiar dwelling place of God.>
<15:6 Came out of the temple; expressive of their being commissioned and sent of God.>
<15:7 One of the four beasts gave; we seem to have here an intimation that these seven last plagues proceed from the all-comprehending providence of God. Compare Eze 10:2,6,7.>
<15:8 Smoke; the symbol of God's presence, as a holy and jealous God prepared to execute vengeance on the wicked. 1Ki 8:10; Isa 6:4. No man was able to enter; because of the smoke: alluding to the cloud which covered the tabernacle, and filled the temple when they were dedicated. Ex 40:34,35; 1Ki 8:10,11. Though God often waits long upon the wicked and his judgments seem to linger, in due time they will come; and continuance in sin against all warnings and entreaties will bring inevitable and overwhelming destruction.>
<16:1 Out of the temple; coming from the temple, where God dwelt. Men are apt to look no further than to second causes; but the holy Scriptures refer all the judgments which fall upon the world for its wickedness to God as their author. They come from him, and execute his holy purposes. Pour out the vials of the wrath of God; the seven last plagues belong to the seventh trumpet, under which, or at least, near to which, we seem to be living. To attempt the application of them to particular events in history, seems to be premature. Upon the earth; upon the inhabitants of the earth, especially the persecutors of God's people. All the seven vials belong alike to the inhabitants of the earth, whatever be the particular symbols on which they are poured out.>
<16:2 Upon the earth; signifying that some distressing judgment falls on the worshippers of the beast. Men are so entirely in the hands of God, he can in so many ways and with such perfect ease destroy them, that it is the height of folly as well as wickedness to oppose his cause, or to refuse his grace.>
<16:3 Upon the sea; probably here, as often elsewhere, a symbol of revolutions accompanied with a terrible amount of bloodshed, and preparing the way for the overthrow of the beast.>
<16:4 The rivers and fountains--became blood; seeming to indicate a succession of bloody wars.>
<16:5 The angel of the waters; that had charge of the waters. Compare chap Re 7:1, where four angels have charge of four winds.>
<16:6 They are worthy; deserve their awful doom. Men often suffer calamities greater than those they have wickedly inflicted, and may read their sins in their punishment.>
<16:7 Out of the altar; under which are the souls of the martyrs whose blood has been shed by these persecutors. Chap Re 6:9.>
<16:8 Upon the sun; not extinguishing him, but kindling in him an unnatural and scorching heat. The symbol, according to some, denotes the turning of civil power into a means of oppressing men.>
<16:9 They repented not; they grew no better under their torment, being given up to hardness of heart and blindness of men. The elements, which ordinarily are sources of rich blessings, become, when commissioned by God, sources of exquisite anguish to his foes.>
<16:10 The seat of the beast; the centre of his power, authority, and influence. They; his followers.>
<16:12 The great river Euphrates--the water thereof was dried up; that thus the hinderance to the way of the kings of the east might be removed. Many think that the decay of the Ottoman power, as a preparation for some great movement yet in the future, is here predicted. Compare the notes on the sixth trumpet, chap Re 9:13-21.>
<16:13 Like frogs; loathsome, creeping, unclean things. The dragon; the devil. See note to chap Re 12:3. The beast; the first beast that rose up out of the sea. chap Re 13:1. The false prophet; the second beast that rose out of the earth, chap Re 13:11, was in league with the first beast, chap Re 13:12-15, and wrought miracles before him. That this beast is here to be understood is certain from chap Re 19:20. Here then is represented a league between the civil and ecclesiastical persecuting powers under the direction of Satan, and the three frogs seem to denote their emissaries and agents, exciting the nations to a general warfare against Christ and his people. This brings on the final decisive conflict so often foretold in the holy writ, and again set forth in chap Re 19:11-21. There are times when certain forms of spiritual delusion seem contagious. Nations become morally insane. Satan and his servants, the secular and ecclesiastical persecuting powers, send out their vile agents to revive their drooping interests, increase their influence, and concentrate their powers. But the final issue will always be a new victory of the truth.>
<16:14 That great day of God; when he will inflict full vengeance on his foes. Satan and wicked men are leagued together, and cooperate in opposing God; and often the greater their efforts the nearer they are to destruction.>
<16:15 As a thief; suddenly, unexpectedly. A solemn intimation that the day here spoken of will come suddenly and unexpectedly, and find multitudes unprepared for its approach. Watcheth, and keepeth his garments; is awake and active in duty. Lest he walk naked; as a man would whose garments, through his carelessness, had been stolen.>
<16:16 He gathered them; or, as the idiom of the original Greek admits, they gathered them; the three unclean spirits, namely, "which go forth"--the verb in the original is here also singular--"unto the kings of the earth and of the whole world, to gather them," verse Re 16:14. Armageddon; that is, the mount of Megiddo. The Hebrew word Megiddo seems to mean a place of troops. Megiddo was celebrated as the place of two memorable overthrows: that of the kings who oppressed Israel, Jud 5:19; and that of Josiah and his army, 2Ch 35:22-24; Zec 12:11. Hence Armageddon, like "the valley of decision," Joe 3:14, is a symbolic name for a place of great slaughter. Compare chap Re 19:17-21.>
<16:17 Into the air; the place of storms and tempests, which immediately follow, verses Re 16:18,21, and are, as well as earthquakes, symbolic of mighty commotions and overturnings among the nations. It is done; the destruction of the enemies of God is complete. So certain was it, that he spoke of it as already accomplished.>
<16:19 The great city; Babylon. The cities of the nations; these with Babylon represent the centres of the power and influence of Satan, the beast, false prophet, and all their anti-christian associates. Great Babylon; see chap Re 17:5.>
<16:20 Every island fled away--the mountains were not found; representing the overthrow by mighty revolutions of all the powers opposed to Christ and his people.>
<16:21 A great hail; a symbol of God's awful judgments on the wicked. The weight of a talent; the Attic talent was equal to about fifty-seven pounds; the Hebrew talent to more than a hundred pounds. The weight of the stones represents the awful severity of the judgments. No sufferings, however great or long continued, will of themselves bring sinners to repentance, or lead them to submit to God and obey him.>
<17:1 After the accomplishment of the mystery of God by the pouring out of the seven vials, the prophet has a further and more particular vision of the same great persecuting power whose doom has been foretold. He is carried by one of the seven angels into the wilderness, and there sees a harlot riding upon a scarlet-colored beast, which is manifestly the same as the beast that rose out of the sea. Chap Re 13:1. A harlot is the appropriate symbol of an apostate church, and her riding upon the beast represents the fact that the beast supports her, and she uses it for the accomplishment of her base purposes. She is therefore the same for substance as the two-horned beast that "exerciseth all the power of the first beast before him." Chap Re 13:11,12. The great whore; the great persecuting power, whose destruction had been foretold; called by this name on account of her awfully corrupting influence. Sitteth upon many waters; ruleth over many nations. Verse Re 17:15.>
<17:2 Have committed fornication; by their idolatrous devotion to her service. Made drunk with the wine of her fornication; an allusion to the wine-cup which harlots give to their deluded votaries. The meaning is that the inhabitants of the earth have been deluded, corrupted, and made wretched by her errors, vices, and control. The wicked character and seductive influence of those who have been distinguished for their persecutions of God's people, he had represented under a great variety of images; all suited to show their detestable character, and lead all the right-minded to abhor them.>
<17:3 Into the wilderness; probably to be understood symbolically of the fact that her presence makes a spiritual wilderness. A woman; representing this idolatrous persecuting power, who, with all deceivableness of unrighteousness, by pretended miracles, shows, splendid decorations, indulgences, jubilees, and blandishments of various sorts, had been deceiving and enslaving the nations, promising all good to those who should follow, and all evil to those who should oppose her. 2Th 2:9-12; 1Ti 4:1-3. Scarlet-colored; scarlet is the well-known color of popes and cardinals. Seven heads and ten horns; see below on verses Re 17:9-12.>
<17:4 Purple--scarlet--gold--and pearls; indicating her vast wealth and luxury, and the gorgeous and splendid decorations by which she dazzled and captivated the deluded multitude. Great external parade, pomp, and show are given in the Bible as characteristics of that corrupt secular and ecclesiastical power denominated antichrist, the mystery of iniquity, the beast, the great whore, the mother of harlots and abominations of the earth.>
<17:5 Mystery; apparently indicating the symbolic character of her name.>
<17:6 Drunken with the blood of the saints, and--martyrs of Jesus; expressive of the vast multitude whom she, by her inquisitions, wars, and in various other ways had caused to be put to death, because they would not yield to her seductions. I wondered; at the revelation which was made, and at the shameless and awful wickedness revealed of this mystery of iniquity. Another characteristic is, while professing to have the temper of a lamb, and to be allied to heaven, it shows by its acts the heart of a dragon, instigated from hell. It has put to death, as heretics and schismatics, so many of the humble, believing followers of Jesus Christ, that it is described as drunken with their blood.>
<17:8 Was, and is not--shall ascend--go into perdition; the beast is the Roman power considered through the whole of its duration. It was, as the old Roman empire, and in this form it was destroyed and ceased to be; then it ascended out of the bottomless pit as papal Rome, and in this form it shall finally go into perdition. Shall wonder; wonder after the beast, chap Re 13:3; that is, shall follow him with idolatrous admiration. When they behold the beast; during the continuance of the beast, or this persecuting power, in its last form and before its final destruction. There is no certain security, except to the true children of God, against the seductive arts and fascinating influence of that power which is represented as the great whore, on a scarlet-colored beast, arrayed in purple, and decked with gold, precious stones, and pearls, and with her wine making the nations drunk.>
<17:9 Hath wisdom; to understand the meaning of this description, and to whom it properly applies. Seven mountains; on which Rome, the seat of her empire and that of the beast which supported her, was built.>
<17:10 And there are seven kings; the seven heads signify also seven kings. This is understood by many as representing the seven forms of civil government which prevailed in Rome--kings, consuls, dictators, decemvirs, military tribunes, emperors, and the exarchate of Ravenna; for here, as in verse Re 17:12, a king denotes not an individual, but a succession of rulers. Five are fallen; the first five of the above list; which are those enumerated by the historian Livy, as having existed in his day. One is; the imperial form. When he cometh; when the last king cometh, that is, the last form of government.>
<17:11 The beast that was, and is not; that is, the beast in his last form ascending out of the bottomless pit, verse Re 17:8. Is the eighth; in the order of succession, since he comes after the other seven. Is of the seven; he belongs to them, as being a continuation of the same power which they have exercised before him.>
<17:12 Ten kings; ten governments, which should arise out of the ruins of the old Roman empire. One hour; through one period of time. With the beast; their rise is coincident with that of the beast. At first they give their power to the beast; but afterwards they turn against it.>
<17:14 Make war with the Lamb; act in open and deadly hostility to Christ and his cause.>
<17:16 The ten horns which thou sawest; that nations which for a time supported the persecuting power. Shall hate the whore--make her desolate--eat her flesh, and burn her with fire; shall turn against her, and help to destroy the persecuting power which she represented. Though she may for a time and to a great extent succeed in deceiving the nations, yet it will be only so long as God shall suffer it, and till his word concerning it is fulfilled. Then many will be undeceived, see their folly and her abominations, and turn in wrath against her, and she shall come to her end and have none to help her.>
<17:17 The words of God; those which he has uttered concerning the wickedness of this beast, or persecuting power.>
<17:18 That great city; Rome; those who there exerted influence and exercised dominion over Italy, and over a great portion of the earth.>
<18:1 The present chapter contains a vision of the fall of the mystic Babylon, expressed in magnificent imagery, taken mainly from the prophecies of the overthrow of Babylon and Tyre contained in the Old Testament. Another angel; coming to herald the fall of Babylon. The glory in which he appears represents the bright and glorious displays of Christ's power connected with her overthrow.>
<18:2 The habitation of devils--every foul spirit--every unclean and hateful bird; the meaning is that Babylon is abandoned, as a place utterly desolated and uninhabited, to be the abode of these unclean beings. Compare what is said of ancient Babylon, Isa 13:21,22; Jer 50:39; and of Edom, Isa 34:11-15. From these passages the imagery is plainly taken. Bodies of men, as well as individuals, are responsible to God for their conduct; and when they have filled up the measure of their sins, and he comes out in judgment, no numbers, wealth, or power can withstand or ward off his wrath.>
<18:3 Have drunk of the wine of the wrath of her fornication; see note to chap Re 17:2. Babylon is represented as a rich, powerful, and luxurious harlot, enriching the nations by her commerce with them, while she corrupts them by her fornications. Compare Isa 47.1-15; Na 3:4.>
<18:4 Come out of her, my people; separate yourselves from her, and have no fellowship with her errors and crimes. Compare Jer 51:6,45. To avoid communion with those who oppose the truth and persecute the people of God, is the only way to escape the ruin which awaits them.>
<18:5 Have reached unto heaven; compare Jer 51:9.>
<18:6 Reward her--as she rewarded you--fill to her double; Ps 137:8; Jer 50:15,29. Compare with these verses Isa 47:8,9.>
<18:8 Burned with fire; expressive of the certainty, dreadfulness, and completeness of her ruin.>
<18:9 9-19. This description of the articles in which Babylon trafficked with the nations, and the lamentation of the kings and merchants of the earth over her fall, agrees in many respects with the prophecy of the overthrow of Tyre in Ezekiel, chaps Eze 26.1-27.35, which should be read in connection with it. We are not to insist on the particulars enumerated. The whole is a picture of her great wealth and the abundance of her resources.>
<18:10 10-19. Alas, alas; showing the vast interests which multitudes of the votaries of wealth, fashion, and pleasure had in her continuance, and their disappointment, sorrow, vexation, and despair at her downfall.>
<18:15 Men who grow rich by wickedness exceedingly desire its continuance; and when their wicked patrons are cut off, they wail at the ceasing of their gains.>
<18:20 Rejoice; all the good on earth and in heaven will exult when her power to injure ceases. Events which bring consternation to the wicked fill the righteous with joy; not because they delight in the misery of the wicked, but because God is holy, just, and good in all his judgments.>
<18:21 Took up a stone--cast it into the sea; an allusion to Jer 51.63,64.>
<18:22 The sound of a millstone; compare Jer 25:10. The orientals grind their meal daily in hand-mills. The cessation of the sound of the millstone is therefore a sign of utter desolation.>
<18:23  All nations deceived--the blood of prophets--saints, and of all that were slain; deceitfulness and cruelty were among her grand characteristics; and for them and her other numerous sins she is visited with these awful judgments.>
<18:24 The persecution of Christians by rulers or people, civil or ecclesiastical, even if in compliance with human laws, is never overlooked or forgotten by Jehovah. It is an aggravated sin, which in due time he will surely punish.>
<19:1 The apostle hears the multitude of the heavenly hosts rejoicing over the fall of Babylon, and sees the bride, the Lamb's wife, arrayed in white linen, ready for the consummation of her marriage to her Lord. After this he has another vision, of the final conflict between Christ and the powers of darkness, which ends in their utter overthrow and the ushering in of the age of millennial peace and glory. Alleluia; in Hebrew, hallelujah; meaning, praise ye the Lord.>
<19:3 For ever and ever; expressive of the perpetuity of her torment.>
<19:4 Amen; Alleluia; be it so, praise ye the Lord; showing their hearty acquiescence in the infliction of his judgments.>
<19:6 Hearty acquiescence in all the dealings of God is the duty of all creatures, and the delight of all the holy.>
<19:7 The marriage of the Lamb; the church has been from the first espoused to Christ as a chaste virgin, but now he takes her into full union with himself, and thus gives her rest from all her conflicts and sufferings. His wife; representing the multitude of his faithful followers. She appears as a bride arrayed in pure and white linen, in contrast with the filthy harlot of Rome described in chap Re 17.1-18.>
<19:8 Is the righteousness of saints; representing their righteousness.>
<19:9 They which are called unto the marriage-supper; as the bride represents the faithful people of God taken as a whole, so they which are called to the marriage-supper represent the faithful servants of Christ considered individually.>
<19:10 Do it not; no creature is to be worshipped. And of thy brethren; that is, and the fellow-servant of thy brethren. Worship God; and him only shalt thou serve. Mt 4:10. The testimony of Jesus is the spirit of prophecy; the grand scope and end of the spirit of prophecy is to bear witness concerning Jesus. With this spirit John was endowed as well as the angel. They were in this respect alike; one therefore was not to worship the other, for they were fellow-servants--each by prophecy making known the testimony of Christ. As no creature in heaven or on earth is to be worshipped, as men are commanded to worship God only, and as both men and angels do worship Christ, and that in obedience to divine command, it is certain that he is God. Joh 1:1; Heb 1:6;Re 5:13.>
<19:11 Heaven opened; in token of still further communications which were to be made of the purposes of Christ. The final conflict here described seems to be identical with "the battle of that great day of God Almighty," chap Re 16:14. There the dragon, the beast, and the false prophet gather their hosts into the valley of Armageddon. Here the beast and the false prophet are taken and cast into the lake of fire; and then, chap Re 20:1-3, the dragon, who had instigated this war against Christ, is bound and cast into the bottomless pit for a thousand years. There also, as here, the conflict is immediately followed by the complete triumph of Christ and his reign over men. Compare chap Re 16:17; with chap Re 20:1-3. A white horse; here, as in chap Re 6:2, the symbol of victory over his foes. Faithful and True; he is Jesus Christ, "the faithful and true Witness." Chap Re 3:14. He doth judge and make war; compare Isa 11:3,4, where the same attributes and works are ascribed to the Messiah.>
<19:12 That no man knew, but he himself; the apostle saw in vision the name, but no one save its divine bearer could know it. The meaning, according to some, is, that it was a secret inscription which Christ only could read. Others suppose it to be the name given in verse Re 19:13--"The Word of God"--which might be read outwardly, while no one but Christ could comprehend its meaning, since the name contains the deep mystery of his nature and office.>
<19:13 He was clothed with a vesture dipped in blood; compare Isa 63:1-3, and notes. His name is--The Word of God; applied here, as in Joh 1:1, to the second person of the god-head.>
<19:14 The armies which were in heaven; representing the multitude of Christ's redeemed followers.>
<19:15 Out of his mouth goeth a sharp sword; see note to chap Re 1:16. He treadeth the wine-press; for the symbol of treading the wine-press, see note to chap Re 14:19,20; Isa 66:3.>
<19:16 King of kings, and Lord of lords; indicating his universal and supreme dominion. The supremacy of the Lord Jesus Christ, and his determination to crush his enemies and save his friends, are most clearly revealed in the Bible; and in his manifestations of himself, all will see that he is a just God and an almighty Saviour.>
<19:17 The supper of the great God; which the great God prepares for you; representing the immense destruction of his enemies, on the flesh of whom the fowls of heaven were invited to feed. The imagery is taken from Eze 39:17-20, where God invites all the feathered fowl and wild beasts to come to the table which he has provided for them.>
<19:20 The beast; the first beast, described in chap Re 13:1. The false prophet; the same as the second beast, chap Re 13:11.>
<19:21 The remnant; of the armies that followed them, verse Re 19:19.>
<20:1 The remnant; of the armies that followed them, verse Re 19:19.>
<20:2 Satan; though his agents, the beast, the false prophet, and those who cooperated with them, had been destroyed, Satan still lived, and if permitted, would tempt men to persecute the church.>
<20:3 That he should deceive the nations no more; not be permitted to have influence over men, to seduce them into error, tempt them to sin, or afflict them by persecution. He must be loosed; again suffered to tempt men, excite their evil passions, and influence them to array themselves against Christ and his cause. Though Satan is a powerful, malignant, and artful spirit, who has for ages deceived the nations and led vast multitudes to ruin, yet he is under divine control. He can go no further than God shall suffer: when God sees best, He can bind, imprison, and so restrain him, that men shall no longer be under his influence, or annoyed by his wiles.>
<20:4 Thrones, and they sat upon them; representing the exalted and favored state of the friends of God. The souls of them; that were put to death for their attachment to Christ. They lived; best understood figuratively as meaning that they lived in the persons of their successors, as Elijah came and lived in the person of John. Mal 4:5; Mt 11:14; 17:10-13. The men who lived during the thousand years were men of like spirit with those martyrs who suffered for the cause of Jesus, as John was of like spirit with Elijah.>
<20:5 The rest of the dead lived not again; those who were put to death for their opposition to Christ being destroyed by him, chap Re 19:20,21, there would be none to persecute the followers of Christ till the close of the thousand years; then such men would again be found, as described in verses Re 20:7-9.>
<22:14 The city; the heavenly city, the new Jerusalem, the eternal abode of God and his people.>
<22:15 Dogs--and whosoever loveth and maketh a lie; a description of various classes of sinners, including all who do not love, believe, and obey the truth. Dogs represent here the rapacious and unclean. Compare Php 3:2. Much of the future misery of the wicked will spring from the character and conduct of their associates. In this world the wicked are mutual tempters; in the world to come they will be mutual tormentors.>
<22:16 The root and the offspring of David; these words are commonly interpreted to mean that Christ is the root of David--the ground of his being--in respect to his divine nature; and his offspring in respect to his human nature. Compare Ps 110:1; Mt 22:42-46; Ac 2:34-36. But a comparison with Isa 11:1, to which there is a plain reference, leads rather to the idea that Christ is called the root and offspring of David as growing out of his root; that is, as being his true progeny according to the promises of the Old Testament. The--morning star; ushering in upon his people the splendors of eternal day.>
<22:17 The bride; the church of Christ. Come; come unto Christ, and receive freely the blessings of eternal life. Mt 11:28-30; Isa 45:22. As the Holy Spirit, the church, and Jesus Christ invite sinners to come to him, all should accept and echo the invitation, and publish it, as far as possible, in every language of every people, that whosoever will may come to Christ, and receive of him the free, gracious gift of eternal life.>
<22:18 Add unto these things; unto the words of this prophecy; and by parity of reason, shall add to any part of divine revelation what God has not revealed.>
<22:19 Take away from the words of--this prophecy; take away a part of what God has revealed, and reject it as not inspired by him. See note to the preceding verse. To attempt to require of men what God does not require, or to absolve them from what he does require, is a great sin, and exposes those who practise it to his endless curse.>
<22:20 He; Jesus Christ. I come quickly; to call each one to give an account of his stewardship, and to enter, according to his conduct and character, on the retributions of eternity. Whatever we do for our own salvation, or that of others, we must do soon; for in the grave to which we are hastening, there is no work. Ec 9:10.>
<22:21 The grace of our Lord Jesus Christ; a desire and prayer that Christ's divine favor might be bestowed upon those for whom the apostle wrote, according to all their wants for time and eternity. However poor, polluted, and wretched any may be, through the grace of Christ and by trusting in him all may be rich, holy, and happy for ever. May his grace be the portion of the writer, and all the readers, for the Redeemer's sake; and to the Father, the Son, and the Holy Ghost shall be the glory for ever. Amen.>

