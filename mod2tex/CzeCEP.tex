\source{CzeCEP}

\book{Genesis}{Gen}
#1:1 Na počátku stvořil Bůh nebe a zemi.
#1:2 Země byla pustá a prázdná a nad propastnou tůní byla tma. Ale nad vodami vznášel se duch Boží.
#1:3 I řekl Bůh: „Buď světlo!“ A bylo světlo.
#1:4 Viděl, že světlo je dobré, a oddělil světlo od tmy.
#1:5 Světlo nazval Bůh dnem a tmu nazval nocí. Byl večer a bylo jitro, den první.
#1:6 I řekl Bůh: „Buď klenba uprostřed vod a odděluj vody od vod!“
#1:7 Učinil klenbu a oddělil vody pod klenbou od vod nad klenbou. A stalo se tak.
#1:8 Klenbu nazval Bůh nebem. Byl večer a bylo jitro, den druhý.
#1:9 I řekl Bůh: „Nahromaďte se vody pod nebem na jedno místo a ukaž se souš!“ A stalo se tak.
#1:10 Souš nazval Bůh zemí a nahromaděné vody nazval moři. Viděl, že to je dobré.
#1:11 Bůh také řekl: „Zazelenej se země zelení: bylinami, které se rozmnožují semeny, a ovocným stromovím rozmanitého druhu, které na zemi ponese plody se semeny!“ A stalo se tak.
#1:12 Země vydala zeleň: rozmanité druhy bylin, které se rozmnožují semeny, a rozmanité druhy stromoví, které nese plody se semeny. Bůh viděl, že to je dobré.
#1:13 Byl večer a bylo jitro, den třetí.
#1:14 I řekl Bůh: „Buďte světla na nebeské klenbě, aby oddělovala den od noci! Budou na znamení časů, dnů a let.
#1:15 Ta světla ať jsou na nebeské klenbě, aby svítila nad zemí.“ A stalo se tak.
#1:16 Učinil tedy Bůh dvě veliká světla: větší světlo, aby vládlo ve dne, a menší světlo, aby vládlo v noci; učinil i hvězdy.
#1:17 Bůh je umístil na nebeskou klenbu, aby svítila nad zemí,
#1:18 aby vládla ve dne a v noci a oddělovala světlo od tmy. Viděl, že to je dobré.
#1:19 Byl večer a bylo jitro, den čtvrtý.
#1:20 I řekl Bůh: „Hemžete se vody živočišnou havětí a létavci létejte nad zemí pod nebeskou klenbou!“
#1:21 I stvořil Bůh veliké netvory a rozmanité druhy všelijakých hbitých živočichů, jimiž se zahemžily vody, stvořil i rozmanité druhy všelijakých okřídlených létavců. Viděl, že to je dobré.
#1:22 A Bůh jim požehnal: „Ploďte a množte se a naplňte vody v mořích. Létavci nechť se rozmnoží na zemi.“
#1:23 Byl večer a bylo jitro, den pátý.
#1:24 I řekl Bůh: „Vydej země rozmanité druhy živočichů, dobytek, plazy a rozmanité druhy zemské zvěře!“ A stalo se tak.
#1:25 Bůh učinil rozmanité druhy zemské zvěře i rozmanité druhy dobytka a rozmanité druhy všelijakých zeměplazů. Viděl, že to je dobré.
#1:26 I řekl Bůh: „Učiňme člověka, aby byl naším obrazem podle naší podoby. Ať lidé panují nad mořskými rybami a nad nebeským ptactvem, nad zvířaty a nad celou zemí i nad každým plazem plazícím se po zemi.“
#1:27 Bůh stvořil člověka, aby byl jeho obrazem, stvořil ho, aby byl obrazem Božím, jako muže a ženu je stvořil.
#1:28 A Bůh jim požehnal a řekl jim: „Ploďte a množte se a naplňte zemi. Podmaňte ji a panujte nad mořskými rybami, nad nebeským ptactvem, nade vším živým, co se na zemi hýbe.“
#1:29 Bůh také řekl: „Hle, dal jsem vám na celé zemi každou bylinu nesoucí semena i každý strom, na němž rostou plody se semeny. To budete mít za pokrm.
#1:30 Veškeré zemské zvěři i všemu nebeskému ptactvu a všemu, co se plazí po zemi, v čem je živá duše, dal jsem za pokrm veškerou zelenou bylinu.“ A stalo se tak.
#1:31 Bůh viděl, že všechno, co učinil, je velmi dobré. Byl večer a bylo jitro, den šestý. 
#2:1 Tak byla dokončena nebesa i země se všemi svými zástupy.
#2:2 Sedmého dne dokončil Bůh své dílo, které konal; sedmého dne přestal konat veškeré své dílo.
#2:3 A Bůh požehnal a posvětil sedmý den, neboť v něm přestal konat veškeré své stvořitelské dílo.
#2:4 Toto je rodopis nebe a země, jak byly stvořeny. V den, kdy Hospodin Bůh učinil zemi a nebe,
#2:5 nebylo na zemi ještě žádné polní křovisko ani nevzcházela žádná polní bylina, neboť Hospodin Bůh nezavlažoval zemi deštěm, a nebylo člověka, který by zemi obdělával.
#2:6 Jen záplava vystupovala ze země a napájela celý zemský povrch.
#2:7 I vytvořil Hospodin Bůh člověka, prach ze země, a vdechl mu v chřípí dech života. Tak se stal člověk živým tvorem.
#2:8 A Hospodin Bůh vysadil zahradu v Edenu na východě a postavil tam člověka, kterého vytvořil.
#2:9 Hospodin Bůh dal vyrůst ze země všemu stromoví žádoucímu na pohled, s plody dobrými k jídlu, uprostřed zahrady pak stromu života a stromu poznání dobrého a zlého.
#2:10 Z Edenu vychází řeka, aby napájela zahradu. Odtud dál se rozděluje ve čtyři hlavní toky.
#2:11 Jméno prvního je Píšon; ten obtéká celou zemi Chavílu, v níž je zlato,
#2:12 a zlato té země je skvělé; je tam také vonná pryskyřice a kámen karneol.
#2:13 Jméno druhé řeky je Gíchón; ta obtéká celou zemi Kúš.
#2:14 Jméno třetí řeky je Chidekel; ta teče východně od Asýrie. Čtvrtá řeka je Eufrat.
#2:15 Hospodin Bůh postavil člověka do zahrady v Edenu, aby ji obdělával a střežil.
#2:16 A Hospodin Bůh člověku přikázal: „Z každého stromu zahrady smíš jíst.
#2:17 Ze stromu poznání dobrého a zlého však nejez. V den, kdy bys z něho pojedl, propadneš smrti.“
#2:18 I řekl Hospodin Bůh: „Není dobré, aby člověk byl sám. Učiním mu pomoc jemu rovnou.“
#2:19 Když vytvořil Hospodin Bůh ze země všechnu polní zvěř a všechno nebeské ptactvo, přivedl je k člověku, aby viděl, jak je nazve. Každý živý tvor se měl jmenovat podle toho, jak jej nazve.
#2:20 Člověk tedy pojmenoval všechna zvířata a nebeské ptactvo i všechnu polní zvěř. Ale pro člověka se nenašla pomoc jemu rovná.
#2:21 I uvedl Hospodin Bůh na člověka mrákotu, až usnul. Vzal jedno z jeho žeber a uzavřel to místo masem.
#2:22 A Hospodin Bůh utvořil z žebra, které vzal z člověka, ženu a přivedl ji k němu.
#2:23 Člověk zvolal: „Toto je kost z mých kostí a tělo z mého těla! Ať muženou se nazývá, vždyť z muže vzata jest.“
#2:24 Proto opustí muž svého otce i matku a přilne ke své ženě a stanou se jedním tělem.
#2:25 Oba dva byli nazí, člověk i jeho žena, ale nestyděli se. 
#3:1 Nejzchytralejší ze vší polní zvěře, kterou Hospodin Bůh učinil, byl had. Řekl ženě: „Jakže, Bůh vám zakázal jíst ze všech stromů v zahradě?“
#3:2 Žena hadovi odvětila: „Plody ze stromů v zahradě jíst smíme.
#3:3 Jen o plodech ze stromu, který je uprostřed zahrady, Bůh řekl: ‚Nejezte z něho, ani se ho nedotkněte, abyste nezemřeli.‘“
#3:4 Had ženu ujišťoval: „Nikoli, nepropadnete smrti.
#3:5 Bůh však ví, že v den, kdy z něho pojíte, otevřou se vám oči a budete jako Bůh znát dobré i zlé.“
#3:6 Žena viděla, že je to strom s plody dobrými k jídlu, lákavý pro oči, strom slibující vševědoucnost. Vzala tedy z jeho plodů a jedla, dala také svému muži, který byl s ní, a on též jedl.
#3:7 Oběma se otevřely oči: poznali, že jsou nazí. Spletli tedy fíkové listy a přepásali se jimi.
#3:8 Tu uslyšeli hlas Hospodina Boha procházejícího se po zahradě za denního vánku. I ukryli se člověk a jeho žena před Hospodinem Bohem uprostřed stromoví v zahradě.
#3:9 Hospodin Bůh zavolal na člověka: „Kde jsi?“
#3:10 On odpověděl: „Uslyšel jsem v zahradě tvůj hlas a bál jsem se. A protože jsem nahý, ukryl jsem se.“
#3:11 Bůh mu řekl: „Kdo ti pověděl, že jsi nahý? Nejedl jsi z toho stromu, z něhož jsem ti zakázal jíst?“
#3:12 Člověk odpověděl: „Žena, kterou jsi mi dal, aby při mně stála, ta mi dala z toho stromu a já jsem jedl.“
#3:13 Proto řekl Hospodin Bůh ženě: „Cos to učinila?“ Žena odpověděla: „Had mě podvedl a já jsem jedla.“
#3:14 I řekl Hospodin Bůh hadovi: „Protožes to učinil, buď proklet, vyvržen ode všech zvířat a ode vší polní zvěře. Polezeš po břiše, po všechny dny svého života žrát budeš prach.
#3:15 Mezi tebe a ženu položím nepřátelství, i mezi símě tvé a símě její. Ono ti rozdrtí hlavu a ty jemu rozdrtíš patu.“
#3:16 Ženě řekl: „Velice rozmnožím tvé trápení i bolesti těhotenství, syny budeš rodit v utrpení, budeš dychtit po svém muži, ale on nad tebou bude vládnout.“
#3:17 Adamovi řekl: „Uposlechl jsi hlasu své ženy a jedl jsi ze stromu, z něhož jsem ti zakázal jíst. Kvůli tobě nechť je země prokleta; po celý svůj život z ní budeš jíst v trápení.
#3:18 Vydá ti jenom trní a hloží a budeš jíst polní byliny.
#3:19 V potu své tváře budeš jíst chléb, dokud se nenavrátíš do země, z níž jsi byl vzat. Prach jsi a v prach se navrátíš.“
#3:20 Člověk svou ženu pojmenoval Eva (to je Živa), protože se stala matkou všech živých.
#3:21 Hospodin Bůh udělal Adamovi a jeho ženě kožené suknice a přioděl je.
#3:22 I řekl Hospodin Bůh: „Teď je člověk jako jeden z nás, zná dobré i zlé. Nepřipustím, aby vztáhl ruku po stromu života, jedl a byl živ navěky.“
#3:23 Proto jej Hospodin Bůh vyhnal ze zahrady v Edenu, aby obdělával zemi, z níž byl vzat.
#3:24 Tak člověka zapudil. Východně od zahrady v Edenu usadil cheruby s míhajícím se plamenným mečem, aby střežili cestu ke stromu života. 
#4:1 I poznal člověk svou ženu Evu a ta otěhotněla a porodila Kaina. Tu řekla: „Získala jsem muže, a tím Hospodina.“
#4:2 Dále porodila jeho bratra Ábela. Ábel se stal pastýřem ovcí, ale Kain se stal zemědělcem.
#4:3 Po jisté době přinesl Kain Hospodinu obětní dar z plodin země.
#4:4 Také Ábel přinesl oběť ze svých prvorozených ovcí a z jejich tuku. I shlédl Hospodin na Ábela a na jeho obětní dar,
#4:5 na Kaina však a na jeho obětní dar neshlédl. Proto Kain vzplanul velikým hněvem a zesinal v tváři.
#4:6 I řekl Hospodin Kainovi: „Proč jsi tak vzplanul? A proč máš tak sinalou tvář?
#4:7 Což nepřijmu i tebe, budeš-li konat dobro? Nebudeš-li konat dobro, hřích se uvelebí ve dveřích a bude po tobě dychtit; ty však máš nad ním vládnout.“
#4:8 I promluvil Kain ke svému bratru Ábelovi... Když byli na poli, povstal Kain proti svému bratru Ábelovi a zabil jej.
#4:9 Hospodin řekl Kainovi: „Kde je tvůj bratr Ábel?“ Odvětil: „Nevím. Cožpak jsem strážcem svého bratra?“
#4:10 Hospodin pravil: „Cos to učinil! Slyš, prolitá krev tvého bratra křičí ke mně ze země.
#4:11 Budeš nyní proklet a vyvržen ze země, která rozevřela svá ústa, aby z tvé ruky přijala krev tvého bratra.
#4:12 Budeš-li obdělávat půdu, už ti nedá svou sílu. Budeš na zemi psancem a štvancem.“
#4:13 Kain Hospodinu odvětil: „Můj zločin je větší, než je možno odčinit.
#4:14 Hle, vypudil jsi mě dnes ze země. Budu se muset skrývat před tvou tváří. Stal jsem se na zemi psancem a štvancem. Každý, kdo mě najde, bude mě moci zabít.“
#4:15 Ale Hospodin řekl: „Nikoli, kdo by Kaina zabil, bude postižen sedmeronásobnou pomstou.“ A Hospodin poznamenal Kaina znamením, aby jej nikdo, kdo ho najde, nezabil.
#4:16 Kain odešel od tváře Hospodinovy a usadil se v zemi Nódu, východně od Edenu.
#4:17 I poznal Kain svou ženu, ta otěhotněla a porodila Enocha. Tu se dal do stavby města a nazval to město Enoch, podle jména svého syna.
#4:18 Enochovi se narodil Írad, Írad zplodil Mechújáela, Mechíjáel zplodil Metúšáela, Metúšáel zplodil Lámecha.
#4:19 Lámech pojal dvě ženy; jedna se jmenovala Áda a druhá se jmenovala Sila.
#4:20 Áda porodila Jábala, který se stal praotcem těch, kdo přebývají ve stanu a u stáda.
#4:21 Jeho bratr se jmenoval Júbal; ten se stal praotcem všech hrajících na citaru a flétnu.
#4:22 Také Sila porodila, a to Túbal-kaina, mistra všech řemeslníků obrábějících měď a železo. Sestrou Túbal-kainovou byla Naama.
#4:23 Tu řekl Lámech svým ženám: „Ádo a Silo, poslyšte můj hlas, ženy Lámechovy, naslouchejte mé řeči: Zabil jsem muže za své zranění, pacholíka za svou jizvu.
#4:24 Bude-li sedmeronásobně pomstěn Kain, tedy Lámech sedmdesátkrát a sedmkrát.“
#4:25 I poznal opět Adam svou ženu a ta porodila syna a dala mu jméno Šét (to je Do klína vložený). Řekla: „Bůh mi vložil do klína jiného potomka místo Ábela, kterého zabil Kain.“
#4:26 Šétovi se narodil syn; dal mu jméno Enóš. Tehdy se začalo vzývat jméno Hospodinovo. 
#5:1 Toto je výčet rodopisu Adamova: V den, kdy Bůh stvořil člověka, učinil jej k podobě Boží.
#5:2 Jako muže a ženu je stvořil, požehnal jim a v den, kdy je stvořil, dal jim jméno Adam (to je Člověk).
#5:3 Ve věku sto třiceti let zplodil Adam syna ke své podobě, podle svého obrazu, a dal mu jméno Šét.
#5:4 Po zplození Šéta žil Adam ještě osm set let a zplodil syny a dcery.
#5:5 Všech dnů Adamova života bylo devět set třicet let, a umřel.
#5:6 Ve věku sto pěti let zplodil Šét Enóše.
#5:7 Po zplození Enóše žil Šét osm set sedm let a zplodil syny a dcery.
#5:8 Všech dnů Šétových bylo devět set dvanáct let, a umřel.
#5:9 Ve věku devadesáti let zplodil Enóš Kénana.
#5:10 Po zplození Kénana žil Enóš osm set patnáct let a zplodil syny a dcery.
#5:11 Všech dnů Enóšových bylo devět set pět let, a umřel.
#5:12 Ve věku sedmdesáti let zplodil Kénan Mahalalela.
#5:13 Po zplození Mahalalela žil Kénan osm set čtyřicet let a zplodil syny a dcery.
#5:14 Všech dnů Kénanových bylo devět set deset let, a umřel.
#5:15 Ve věku šedesáti pěti let zplodil Mahalalel Jereda.
#5:16 Po zplození Jereda žil Mahalalel osm set třicet let a zplodil syny a dcery.
#5:17 Všech dnů Mahalalelových bylo osm set devadesát pět let, a umřel.
#5:18 Ve věku sto šedesáti dvou let zplodil Jered Henocha.
#5:19 Po zplození Henocha žil Jered osm set let a zplodil syny a dcery.
#5:20 Všech dnů Jeredových bylo devět set šedesát dvě léta, a umřel.
#5:21 Ve věku šedesáti pěti let zplodil Henoch Metúšelacha.
#5:22 A chodil Henoch s Bohem po zplození Metúšelacha tři sta let a zplodil syny a dcery.
#5:23 Všech dnů Henochových bylo tři sta šedesát pět let.
#5:24 I chodil Henoch s Bohem. A nebylo ho, neboť ho Bůh vzal.
#5:25 Ve věku sto osmdesáti sedmi let zplodil Metúšelach Lámecha.
#5:26 Po zplození Lámecha žil Metúšelach sedm set osmdesát dvě léta a zplodil syny a dcery.
#5:27 Všech dnů Metúšelachových bylo devět set šedesát devět let, a umřel.
#5:28 Ve věku sto osmdesáti dvou let zplodil Lámech syna.
#5:29 Dal mu jméno Noe (to je Odpočinutí). Řekl: „Ten nám dá potěšení a odpočinutí od naší práce a od námahy našich rukou, kterou nám přináší země prokletá Hospodinem.“
#5:30 Po zplození Noeho žil Lámech pět set devadesát pět let a zplodil syny a dcery.
#5:31 Všech dnů Lámechových bylo sedm set sedmdesát sedm let, a umřel.
#5:32 Když bylo Noemu pět set let, zplodil Noe Šéma, Cháma a Jefeta. 
#6:1 Když se lidé počali na zemi množit a rodily se jim dcery, viděli synové božští,
#6:2 jak půvabné jsou dcery lidské, a brali si za ženy všechny, jichž se jim zachtělo.
#6:3 Hospodin však řekl: „Můj duch se nebude člověkem věčně zaneprazdňovat. Vždyť je jen tělo. Ať je jeho dnů sto dvacet let.“
#6:4 Za oněch dnů, kdy synové božští vcházeli k dcerám lidským a ty jim rodily, vznikaly na zemi zrůdy, ba ještě i potom. To jsou ti bohatýři dávnověku, mužové pověstní.
#6:5 I viděl Hospodin, jak se na zemi rozmnožila zlovůle člověka a že každý výtvor jeho mysli i srdce je v každé chvíli jen zlý.
#6:6 Litoval, že na zemi učinil člověka, a trápil se ve svém srdci.
#6:7 Řekl: „Člověka, kterého jsem stvořil, smetu z povrchu země, člověka i zvířata, plazy i nebeské ptactvo, neboť lituji, že jsem je učinil.“
#6:8 Ale Noe našel u Hospodina milost.
#6:9 Toto je rodopis Noeho: Noe byl muž spravedlivý, bezúhonný ve svém pokolení; Noe chodil s Bohem.
#6:10 A Noe zplodil tři syny: Šéma, Cháma a Jefeta.
#6:11 Země však byla před Bohem zkažená a plná násilí.
#6:12 Bůh pohleděl na zemi; byla zcela zkažená, protože všechno tvorstvo pokazilo na zemi svou cestu.
#6:13 I řekl Bůh Noemu: „Rozhodl jsem se skoncovat se vším tvorstvem, neboť země je plná lidského násilí. Zahladím je i se zemí.
#6:14 Udělej si archu z goferového dřeva. V arše uděláš komůrky a vysmolíš ji uvnitř i zvenčí smolou.
#6:15 A uděláš ji takto: Délka archy bude tři sta loket, šířka padesát loket a výška třicet loket.
#6:16 Archa bude mít světlík; na loket odshora jej ukončíš a do boku archy vsadíš dveře. Uděláš v ní spodní, druhé i třetí patro.
#6:17 Hle, já uvedu potopu, vody na zemi, a zahladím tak zpod nebe všechno tvorstvo, v němž je duch života. Všechno, co je na zemi, zhyne.
#6:18 S tebou však učiním smlouvu. Vejdeš do archy a s tebou tvoji synové, tvá žena i ženy tvých synů.
#6:19 A ze všeho, co je živé, ze všeho tvorstva, uvedeš vždy po páru do archy, aby s tebou zůstali naživu; samec a samice to budou.
#6:20 Z rozmanitých druhů ptactva a z rozmanitých druhů zvířat a ze všech zeměplazů rozmanitých druhů, z každého po páru vejdou k tobě, aby se zachovali při životě.
#6:21 Ty pak si naber k obživě různou potravu, nashromáždi si ji, a bude tobě i jim za pokrm.“
#6:22 Noe udělal všechno přesně tak, jak mu Bůh přikázal. 
#7:1 I řekl Hospodin Noemu: „Vejdi ty a celý tvůj dům do archy, neboť vidím, že ty jsi v tomto pokolení jediný můj spravedlivý.
#7:2 Ze všech zvířat čistých vezmeš s sebou po sedmi párech, samce se samicí, ale ze zvířat, která nejsou čistá, jen po páru, samce se samicí.
#7:3 Také z nebeského ptactva po sedmi párech, samce a samici, aby zůstalo naživu potomstvo na celé zemi,
#7:4 neboť již za sedm dní sešlu na zemi déšť, který potrvá čtyřicet dní a čtyřicet nocí. Smetu z povrchu země vše, co povstalo, co jsem učinil.“
#7:5 Noe udělal všechno, jak mu Hospodin přikázal.
#7:6 Šest set let bylo Noemu, když nastala potopa, vody na zemi.
#7:7 Před vodami potopy vešel Noe a s ním jeho synové i jeho žena a ženy jeho synů do archy.
#7:8 Z čistých zvířat i ze zvířat, která nejsou čistá, z ptactva i ze všeho, co se plazí po zemi,
#7:9 vždy po páru vešli samec a samice k Noemu do archy, jak mu Bůh přikázal.
#7:10 Po sedmi dnech pak pronikly na zemi vody potopy.
#7:11 V šestistém roce života Noeho, sedmnáctý den druhého měsíce, se provalily všechny prameny obrovské propastné tůně a nebeské propusti se otevřely.
#7:12 Nad zemí se strhl lijavec a trval čtyřicet dní a čtyřicet nocí.
#7:13 Právě toho dne vešli Noe i Šém, Chám a Jefet, synové Noeho, i žena Noeho a tři ženy jeho synů s nimi do archy,
#7:14 oni i všechna zvěř rozmanitých druhů, všechen dobytek rozmanitých druhů, všichni plazící se zeměplazi rozmanitých druhů i všechno ptactvo rozmanitých druhů, každý pták, každý okřídlenec.
#7:15 Vešli k Noemu do archy vždy pár po páru ze všeho tvorstva, v němž je duch života.
#7:16 Vcházeli, samec a samice ze všeho tvorstva, jak mu Bůh přikázal. A Hospodin za ním zavřel.
#7:17 Potopa na zemi trvala čtyřicet dní, vod přibývalo, až nadnesly archu, takže se zdvihla od země.
#7:18 Vody zmohutněly a stále jich na zemi přibývalo. Archa plula po hladině vod.
#7:19 Vody na zemi převelice zmohutněly, až přikryly všechny vysoké hory, které jsou pod nebesy.
#7:20 Do výšky patnácti loket vystoupily vody, když byly přikryty hory.
#7:21 A zahynulo všechno tvorstvo, které se na zemi pohybuje, ptactvo, dobytek i zvěř a také všechna na zemi se hemžící havěť, i každý člověk.
#7:22 Všechno, co mělo v chřípích dech ducha života, cokoli bylo na suché zemi, pomřelo.
#7:23 Tak smetl Bůh vše, co povstalo, co bylo na povrchu země: od lidí až po zvířata, po plazy a nebeské ptactvo, všechno bylo smeteno ze země. Zachován byl pouze Noe a to, co s ním bylo v arše.
#7:24 Mohutně stály vody na zemi po sto padesát dnů. 
#8:1 Bůh však pamatoval na Noeho i na všechnu zvěř a všechen dobytek, který s ním byl v arše. Způsobil, že nad zemí zavanul vítr, a vody se uklidnily.
#8:2 Byly ucpány prameny propastné tůně i nebeské propusti a byl zadržen lijavec z nebe.
#8:3 Když přešlo sto padesát dnů, začaly vody ze země ustupovat a opadávat,
#8:4 takže sedmnáctého dne sedmého měsíce archa spočinula na pohoří Araratu.
#8:5 A vody ustupovaly a opadávaly až do desátého měsíce; prvního dne desátého měsíce se objevily vrcholky hor.
#8:6 Když pak přešlo čtyřicet dnů, otevřel Noe v arše okno, které udělal,
#8:7 a vypustil krkavce; ten vylétával a vracel se, dokud se vody na zemi nevysušily.
#8:8 Pak vypustil holubici, kterou měl u sebe, aby viděl, zda vody z povrchu země ustoupily.
#8:9 Holubice však nenalezla místečka, kde by její noha mohla spočinout, a vrátila se k němu do archy, neboť vody dosud pokrývaly povrch celé země. Vztáhl tedy ruku, vzal ji a vnesl ji k sobě do archy.
#8:10 Čekal ještě dalších sedm dní a znovu vypustil holubici z archy.
#8:11 A holubice k němu v době večerní přilétla, a hle, měla v zobáčku čerstvý olivový lístek. Tak Noe poznal, že vody ze země ustoupily.
#8:12 Čekal ještě dalších sedm dní a opět vypustil holubici; už se však k němu zpátky nevrátila.
#8:13 Léta šestistého prvého, první den prvního měsíce, začaly vody na zemi vysychat. Tu Noe odsunul příklop archy a spatřil, že povrch země osychá.
#8:14 Dvacátého sedmého dne druhého měsíce byla již země suchá.
#8:15 I promluvil Bůh k Noemu:
#8:16 „Vyjdi z archy, ty a s tebou tvá žena i tvoji synové a ženy tvých synů.
#8:17 Vyveď s sebou všechno tvorstvo, jež je s tebou, všechnu zvěř i ptactvo a dobytek a všechnu havěť plazící se po zemi. Ať se na zemi hemží, ať se na zemi plodí a množí.“
#8:18 Noe tedy vyšel a s ním jeho synové a jeho žena a ženy jeho synů.
#8:19 Všechna zvěř, všechna havěť a všechno ptactvo, vše, co se plazí po zemi, vyšlo podle svých čeledí z archy.
#8:20 Noe pak vybudoval Hospodinu oltář a vzal ze všech čistých dobytčat i ze všeho čistého ptactva a zapálil na tom oltáři oběti zápalné.
#8:21 I ucítil Hospodin libou vůni a řekl si v srdci: „Už nikdy nebudu zlořečit zemi kvůli člověku, přestože každý výtvor lidského srdce je od mládí zlý, už nikdy nezhubím všechno živé, jako jsem učinil.
#8:22 Setba i žeň a chlad i žár, léto i zima a den i noc nikdy nepřestanou po všechny dny země.“ 
#9:1 Bůh Noemu a jeho synům požehnal a řekl jim: „Ploďte a množte se a naplňte zemi.
#9:2 Bázeň před vámi a děs z vás padnou na všechnu zemskou zvěř i na všechno nebeské ptactvo; se vším, co se hýbe na zemi, i se všemi mořskými rybami jsou vám vydáni do rukou.
#9:3 Každý pohybující se živočich vám bude za pokrm; jako zelenou bylinu vám dávám i toto všechno.
#9:4 Jen maso oživené krví nesmíte jíst.
#9:5 A krev, která vás oživuje, budu vyhledávat. Budu za ni volat k odpovědnosti každé zvíře i člověka; za život člověka budu volat k odpovědnosti každého jeho bratra.
#9:6 Kdo prolije krev člověka, toho krev bude člověkem prolita, neboť člověka Bůh učinil, aby byl obrazem Božím.
#9:7 Vy pak se ploďte a množte, hemžete se na zemi a množte se na ní.“
#9:8 Bůh řekl Noemu a jeho synům:
#9:9 „Hle, já ustavuji svou smlouvu s vámi a s vaším potomstvem
#9:10 i s každým živým tvorem, který je s vámi, s ptactvem, s dobytkem i s veškerou zemskou zvěří, která je s vámi, se všemi, kdo vyšli z archy, včetně zemské zvěře.
#9:11 Ustavuji s vámi svou smlouvu. Už nebude vyhlazeno všechno tvorstvo vodami potopy a nedojde již k potopě, která by zahladila zemi.“
#9:12 Dále Bůh řekl: „Toto je znamení smlouvy, jež kladu mezi sebe a vás i každého živého tvora, který je s vámi, pro pokolení všech věků:
#9:13 Položil jsem na oblak svou duhu, aby byla znamením smlouvy mezi mnou a zemí.
#9:14 Kdykoli zahalím zemi oblakem a na oblaku se ukáže duha,
#9:15 rozpomenu se na svou smlouvu mezi mnou a vámi i veškerým živým tvorstvem, a vody již nikdy nezpůsobí potopu ke zkáze všeho tvorstva.
#9:16 Ukáže-li se na oblaku duha, pohlédnu na ni a rozpomenu se na věčnou smlouvu mezi Bohem a veškerým živým tvorstvem, které je na zemi.“
#9:17 Řekl pak Bůh Noemu: „Toto je znamení smlouvy, kterou jsem ustavil mezi sebou a veškerým tvorstvem, které je na zemi.“
#9:18 Synové Noeho, kteří vyšli z archy, byli Šém, Chám a Jefet; Chám je otec Kenaanův.
#9:19 Tito tři jsou synové Noeho; podle nich se rozdělila celá země.
#9:20 I začal Noe obdělávat půdu a vysadil vinici.
#9:21 Napil se pak vína, opil se a odkryl uprostřed svého stanu.
#9:22 Chám, otec Kenaanův, spatřil svého otce obnaženého a pověděl to venku oběma svým bratřím.
#9:23 Ale Šém a Jefet vzali plášť, vložili si jej na ramena a jdouce pozpátku přikryli nahotu svého otce. Tvář měli odvrácenou, takže nahotu svého otce nespatřili.
#9:24 Když Noe procitl z opojení a zvěděl, co mu provedl jeho nejmladší syn,
#9:25 řekl: „Proklet buď Kenaan, ať je nejbídnějším otrokem svých bratří!“
#9:26 Dále řekl: „Požehnán buď Hospodin, Bůh Šémův. Ať je Kenaan jejich otrokem!
#9:27 Kéž Bůh Jefetovi dopřeje bydlet ve stanech Šémových. Ať je Kenaan jejich otrokem!“
#9:28 Po potopě žil Noe tři sta padesát let.
#9:29 Všech dnů Noeho bylo devět set padesát let, a umřel. 
#10:1 Toto je rodopis synů Noeho: Šém, Chám a Jefet. Po potopě se jim narodili synové.
#10:2 Synové Jefetovi: Gomer a Mágog a Mádaj, Jávan a Túbal, Mešek a Tíras.
#10:3 Synové Gomerovi: Aškenaz a Rífat a Togarma.
#10:4 Synové Jávanovi: Elíša a Taršíš, Kitejci a Dódanci.
#10:5 Z nich vzešly ostrovní pronárody v různých zemích, pronárody různého jazyka a různých čeledí.
#10:6 Synové Chámovi: Kúš a Misrajim, Pút a Kenaan.
#10:7 Synové Kúšovi: Seba a Chavíla, Sabta a Raema a Sabteka. Synové Raemovi: Šeba a Dedán.
#10:8 Kúš pak zplodil Nimroda; ten se stal na zemi prvním bohatýrem.
#10:9 Byl to bohatýrský lovec před Hospodinem; proto se říká: Jako Nimrod, bohatýrský lovec před Hospodinem.
#10:10 Počátkem jeho království byl Babylón, Erek, Akad a Kalné v zemi Šineáru.
#10:11 Z této země vyšel do Asýrie a vystavěl Ninive - i Rechobót-ír a Kelach
#10:12 a Resen mezi Ninivem a Kelachem - to je to veliké město.
#10:13 Misrajim zplodil Lúďany a Anámce, Lehábany a Naftúchany
#10:14 i Patrúsany a Kaslúchany - z nich vyšli Pelištejci - a Kaftórce.
#10:15 Kenaan zplodil Sidóna, svého prvorozeného, a Chéta,
#10:16 Jebúsejce, Emorejce a Girgašejce
#10:17 i Chivejce, Arkejce a Síňana,
#10:18 Arváďana a Semárce a Chamáťana. Potom se čeledi kenaanské rozptýlily.
#10:19 Pomezí kenaanské se táhlo od Sidónu směrem přes Gerar až ke Gáze, směrem přes Sodomu, Gomoru, Admu a Sebójím až k Leše.
#10:20 To jsou synové Chámovi v různých zemích, pronárody různých čeledí a jazyků.
#10:21 Také Šémovi, praotci všech Heberovců, staršímu bratrovi Jefeta, se narodili synové.
#10:22 Synové Šémovi: Élam a Ašúr, Arpakšád a Lúd a Aram.
#10:23 Synové Aramovi: Ús a Chúl, Geter a Maš.
#10:24 Arpakšád zplodil Šelacha a Šelach zplodil Hebera.
#10:25 Heberovi se narodili dva synové: jméno jednoho bylo Peleg (to je Rozčlenění), neboť za jeho dnů byla země rozčleněna; a jméno jeho bratra bylo Joktán.
#10:26 Joktán pak zplodil Almódada a Šelefa, Chasarmáveta a Jeracha,
#10:27 Hadórama a Úzala a Diklu,
#10:28 Óbala a Abímaela a Šebu,
#10:29 Ofíra a Chavílu a Jóbaba; ti všichni jsou synové Joktánovi.
#10:30 Jejich sídliště bylo od Méši směrem k Sefáru, hoře na východě.
#10:31 To jsou synové Šémovi různých čeledí a různých jazyků, různé pronárody v různých zemích.
#10:32 To jsou čeledi synů Noeho podle jejich rodopisu v různých pronárodech; z nich pak po potopě vzešly všechny pronárody na zemi. 
#11:1 Celá země byla jednotná v řeči i v činech.
#11:2 Když táhli na východ, nalezli v zemi Šineáru pláň a usadili se tam.
#11:3 Tu si řekli vespolek: „Nuže, nadělejme cihel a důkladně je vypalme.“ Cihly měli místo kamene a asfalt místo hlíny.
#11:4 Nato řekli: „Nuže, vybudujme si město a věž, jejíž vrchol bude v nebi. Tak si učiníme jméno a nebudeme rozptýleni po celé zemi.“
#11:5 I sestoupil Hospodin, aby zhlédl město i věž, které synové lidští budovali.
#11:6 Hospodin totiž řekl: „Hle, jsou jeden lid a všichni mají jednu řeč. A toto je teprve začátek jejich díla. Pak nebudou chtít ustoupit od ničeho, co si usmyslí provést.
#11:7 Nuže, sestoupíme a zmateme jim tam řeč, aby si navzájem nerozuměli.“
#11:8 I rozehnal je Hospodin po celé zemi, takže upustili od budování města.
#11:9 Proto se jeho jméno nazývá Bábel (to je Zmatek), že tam Hospodin zmátl řeč veškeré země a lid rozehnal po celé zemi.
#11:10 Toto je rodopis Šémův: Když bylo Šémovi sto let, zplodil Arpakšáda, druhého roku po potopě.
#11:11 Po zplození Arpakšáda žil Šém pět set let a zplodil syny a dcery.
#11:12 Arpakšád pak žil třicet pět let a zplodil Šelacha.
#11:13 Po zplození Šelacha žil Arpakšád čtyři sta tři léta a zplodil syny a dcery.
#11:14 Šelach pak žil třicet let a zplodil Hebera.
#11:15 Po zplození Hebera žil Šelach čtyři sta tři léta a zplodil syny a dcery.
#11:16 Ve věku třiceti čtyř let zplodil Heber Pelega.
#11:17 Po zplození Pelega žil Heber čtyři sta třicet let a zplodil syny a dcery.
#11:18 Ve věku třiceti let zplodil Peleg Reúa.
#11:19 Po zplození Reúa žil Peleg dvě stě devět let a zplodil syny a dcery.
#11:20 Ve věku třiceti dvou let zplodil Reú Serúga.
#11:21 Po zplození Serúga žil Reú dvě stě sedm let a zplodil syny a dcery.
#11:22 Ve věku třiceti let zplodil Serúg Náchora.
#11:23 Po zplození Náchora žil Serúg dvě stě let a zplodil syny a dcery.
#11:24 Ve věku dvaceti devíti let zplodil Náchor Teracha.
#11:25 Po zplození Teracha žil Náchor sto devatenáct let a zplodil syny a dcery.
#11:26 Ve věku sedmdesáti let zplodil Terach Abrama, Náchora a Hárana.
#11:27 Toto je rodopis Terachův: Terach zplodil Abrama, Náchora a Hárana. Háran zplodil Lota.
#11:28 Háran umřel před svým otcem Terachem v rodné zemi, v Kaldejském Uru.
#11:29 Abram a Náchor si vzali ženy: Žena Abramova se jmenovala Sáraj a žena Náchorova Milka, dcera Hárana, otce Milky a Jisky.
#11:30 Sáraj však byla neplodná, neměla dítě.
#11:31 I vzal Terach svého syna Abrama a vnuka Lota, syna Háranova, a snachu Sáraj, ženu svého syna Abrama, a vyšli spolu z Kaldejského Uru. Cestou do země kenaanské přišli do Cháranu a usadili se tam.
#11:32 Dnů Terachových bylo dvě stě pět let, když v Cháranu umřel. 
#12:1 I řekl Hospodin Abramovi: „Odejdi ze své země, ze svého rodiště a z domu svého otce do země, kterou ti ukážu.
#12:2 Učiním tě velkým národem, požehnám tě, velké učiním tvé jméno. Staň se požehnáním!
#12:3 Požehnám těm, kdo žehnají tobě, prokleji ty, kdo ti zlořečí. V tobě dojdou požehnání veškeré čeledi země.“
#12:4 A Abram se vydal na cestu, jak mu Hospodin přikázal. Šel s ním také Lot. Abramovi bylo sedmdesát pět let, když odešel z Cháranu.
#12:5 Vzal svou ženu Sáraj a Lota, syna svého bratra, se vším jměním, jehož nabyli, i duše, které získali v Cháranu. Vyšli a ubírali se do země kenaanské a přišli tam.
#12:6 Abram prošel zemí až k místu Šekemu, až k božišti Móre; tehdy v té zemi byli Kenaanci.
#12:7 I ukázal se Abramovi Hospodin a řekl: „Tuto zemi dám tvému potomstvu.“ Proto tam Abram vybudoval oltář Hospodinu, který se mu ukázal.
#12:8 Odtud táhl dál na horu, která je východně od Bét-elu, a postavil svůj stan mezi Bét-elem na západě a Ajem na východě. Také tam vybudoval Hospodinu oltář a vzýval Hospodinovo jméno.
#12:9 Pak se vydal na další cestu směrem k Negebu.
#12:10 I nastal v zemi hlad. Tu Abram sestoupil do Egypta, aby tam pobyl jako host, neboť na zemi těžce doléhal hlad.
#12:11 Když už se chystal vejít do Egypta, řekl své ženě Sáraji: „Vím dobře, že jsi žena krásného vzhledu.
#12:12 Až tě spatří Egypťané, řeknou si: ‚To je jeho žena.‘ Mne zabijí a tebe si ponechají živou.
#12:13 Říkej tedy, žes mou sestrou, aby se mi kvůli tobě dobře dařilo a abych tvou zásluhou zůstal naživu.“
#12:14 Když pak Abram vešel do Egypta, spatřili Egypťané tu ženu, jak velice je krásná.
#12:15 Spatřila ji také faraónova knížata a vychválila ji faraónovi. Byla proto vzata do domu faraóna
#12:16 a ten kvůli ní prokázal Abramovi mnoho dobrého, takže měl brav a skot a osly i otroky a otrokyně i oslice a velbloudy.
#12:17 Ale faraóna a jeho dům ranil Hospodin velikými ranami kvůli Abramově ženě Sáraji.
#12:18 Farao tedy Abrama předvolal a řekl: „Jak ses to ke mně zachoval? Proč jsi mi nepověděl, že to je tvá žena?
#12:19 Proč jsi říkal: ‚To je má sestra‘? Vždyť já jsem si ji vzal za ženu. Tady ji máš, vezmi si ji a jdi!“
#12:20 A farao o něm vydal svým lidem příkaz. Vyhostili jej i jeho ženu se vším, co měl. 
#13:1 I vystoupil Abram z Egypta se svou ženou a se vším, co měl, do Negebu; byl s ním i Lot.
#13:2 Abram byl velice zámožný, měl stáda, stříbro i zlato.
#13:3 Postupoval po stanovištích od Negebu až k Bét-elu, na místo mezi Bét-elem a Ajem, kde byl zprvu jeho stan,
#13:4 k místu, kde předtím postavil oltář; tam vzýval Abram Hospodinovo jméno.
#13:5 Také Lot, který putoval s Abramem, měl brav a skot i stany.
#13:6 Země jim však nevynášela tolik, aby mohli sídlit pospolu, a jejich jmění bylo tak značné, že nemohli sídlit pohromadě.
#13:7 Proto došlo k rozepři mezi pastýři stáda Abramova a pastýři stáda Lotova. Tehdy v zemi sídlili Kenaanci a Perizejci.
#13:8 Tu řekl Abram Lotovi: „Ať nejsou rozepře mezi mnou a tebou a mezi pastýři mými a tvými, vždyť jsme muži bratři.
#13:9 Zdalipak není před tebou celá země? Odděl se prosím ode mne. Dáš-li se nalevo, já se dám napravo. Dáš-li se ty napravo, já se dám nalevo.“
#13:10 Lot se rozhlédl a spatřil celý okrsek Jordánu směrem k Sóaru, že je celý zavlažován, že je jako zahrada Hospodinova, jako země egyptská. To bylo předtím, než Hospodin zničil Sodomu a Gomoru.
#13:11 Proto si Lot vybral celý okrsek Jordánu a odtáhl na východ. Tak se od sebe oddělili.
#13:12 Abram se usadil v zemi kenaanské a Lot se usadil v městech toho okrsku a stanoval až u Sodomy.
#13:13 Sodomští muži však byli před Hospodinem velice zlí a hříšní.
#13:14 Poté, co se Lot od něho oddělil, řekl Hospodin Abramovi: „Rozhlédni se z místa, na němž jsi, pohlédni na sever i na jih, na východ i na západ,
#13:15 neboť celou tu zemi, kterou vidíš, dám tobě a tvému potomstvu až navěky.
#13:16 A učiním, že tvého potomstva bude jako prachu země. Bude-li kdo moci sečíst prach země, pak bude i tvé potomstvo sečteno.
#13:17 Teď projdi křížem krážem tuto zemi, neboť ti ji dávám.“
#13:18 Hnul se tedy Abram se stany, přišel a usadil se při božišti Mamre, které je u Chebrónu. I tam vybudoval Hospodinu oltář. 
#14:1 V oněch dnech šineárský král Amráfel, elasarský král Arjók, élamský král Kedorlaómer a král pronárodů Tideál
#14:2 vedli válku proti Bérovi, králi sodomskému, Biršovi, králi gomorskému, Šineábovi, králi ademskému, Šemeberovi, králi sebójskému, a králi z Bely, což je Sóar.
#14:3 Tito všichni tvořili spolek při dolině Sidímu, kde je nyní Solné moře.
#14:4 Dvanáct let otročili Kedorlaómerovi, třináctého roku se vzbouřili.
#14:5 Čtrnáctého roku přitáhl Kedorlaómer a králové, kteří byli s ním, a pobili Refájce v Aštarót-karnajimu, Zuzejce v Hámu, Emejce na planině kirjatajimské
#14:6 a Chorejce v jejich horách seírských až k El-páranu, který leží proti stepi.
#14:7 Pak přitáhli obchvatem k Énmišpátu, což je Kádeš, a pobili vše na poli Amálekovců i Emorejce, kteří sídlili v Chasesón-támaru.
#14:8 Tu vytáhl král sodomský a král gomorský a král ademský a král sebójský a král belský, totiž sóarský, a seřadili se v dolině Sidímu k boji proti nim, to jest:
#14:9 proti élamskému králi Kedorlaómerovi, králi pronárodů Tideálovi, šineárskému králi Amráfelovi a elasarskému králi Arjókovi; čtyři králové stáli proti pěti.
#14:10 Dolina Sidím je plná asfaltových studní. Král sodomský a gomorský se do nich při útěku propadli. Ti, kteří zůstali, utekli do hor.
#14:11 Útočníci pak pobrali všechno jmění Sodomy a Gomory i všechny potraviny a odtáhli.
#14:12 Vzali s sebou též Abramova synovce Lota s jeho jměním, sídlil totiž v Sodomě, a odtáhli.
#14:13 Tu přišel uprchlík a pověděl o tom Hebreji Abramovi, který bydlil při božišti Emorejce Mamreho, bratra Eškólova a bratra Anérova; ti byli s Abramem spjati smlouvou.
#14:14 Když Abram uslyšel, že jeho bratr byl zajat, vytrhl se svými třemi sty osmnácti zasvěcenci, zrozenými v jeho domě, a sledoval útočníky až k Danu.
#14:15 V noci se pak proti nim se svými služebníky rozestavil a pobíjel je a pronásledoval až po Chóbu, jež je na sever od Damašku.
#14:16 Všechno jmění přinesl zpět a nazpět přivedl též svého bratra Lota s jeho jměním, i ženy a lid.
#14:17 Když se vracel po vítězství nad Kedorlaómerem a nad králi, kteří stáli na jeho straně, vyšel mu vstříc král sodomský k dolině Šáve, což je Dolina královská.
#14:18 A šálemský král Malkísedek přinesl chléb a víno; byl totiž knězem Boha Nejvyššího.
#14:19 Požehnal mu: „Požehnán buď Abram Bohu Nejvyššímu, jemuž patří nebesa i země.
#14:20 Požehnán buď sám Bůh Nejvyšší, jenž ti vydal do rukou tvé protivníky.“ Tehdy mu dal Abram desátek ze všeho.
#14:21 Pak řekl Abramovi král Sodomy: „Dej mi lidi, a jmění si nech.“
#14:22 Abram však sodomskému králi odvětil: „Pozdvihl jsem ruku k přísaze Hospodinu, Bohu Nejvyššímu, jemuž patří nebesa i země,
#14:23 že z ničeho, co je tvé, nevezmu nitku ani řemínek k opánkům, abys neřekl: ‚Já jsem učinil Abrama bohatým.‘
#14:24 Sám nechci nic, jen to, co snědla družina, a podíl pro muže, kteří šli se mnou; Anér, Eškól a Mamre, ti ať vezmou svůj podíl.“ 
#15:1 Po těchto událostech se stalo k Abramovi ve vidění slovo Hospodinovo: „Nic se neboj, Abrame, já jsem tvůj štít, tvá přehojná odměna.“
#15:2 Abram však řekl: „Panovníku Hospodine, co mi chceš dát? Jsem stále bezdětný. Nárok na můj dům bude mít damašský Elíezer.“
#15:3 Abram dále řekl: „Ach, nedopřáls mi potomka. To má být mým dědicem správce mého domu?“
#15:4 Hospodin však prohlásil: „Ten tvým dědicem nebude. Tvým dědicem bude ten, který vzejde z tvého lůna.“
#15:5 Vyvedl ho ven a pravil: „Pohleď na nebe a sečti hvězdy, dokážeš-li je spočítat.“ A dodal: „Tak tomu bude s tvým potomstvem.“
#15:6 Abram Hospodinovi uvěřil a on mu to připočetl jako spravedlnost.
#15:7 A řekl mu: „Já jsem Hospodin, já jsem tě vyvedl z Kaldejského Uru, abych ti dal do vlastnictví tuto zemi.“
#15:8 Abram odvětil: „Panovníku Hospodine, podle čeho poznám, že ji obdržím?“
#15:9 I řekl mu: „Vezmi pro mne tříletou krávu a tříletou kozu a tříletého berana, hrdličku a holoubě.“
#15:10 Vzal tedy pro něho to všechno, rozpůlil a dal vždy jednu půlku proti druhé; ptáky však nepůlil.
#15:11 Tu se na ta mrtvá těla slétli dravci a Abram je odháněl.
#15:12 Když se slunce chýlilo k západu, padly na Abrama mrákoty. A hle, padl na něho přístrach a veliká temnota.
#15:13 Tu Hospodin Abramovi řekl: „Věz naprosto jistě, že tvoji potomci budou žít jako hosté v zemi, která nebude jejich; budou tam otročit a budou tam pokořováni po čtyři sta let.
#15:14 Avšak proti pronárodu, jemuž budou otročit, povedu při. Potom odejdou s velkým jměním.
#15:15 Ty vejdeš ke svým otcům v pokoji, budeš pohřben v utěšeném stáří.
#15:16 Sem se vrátí teprve čtvrté pokolení, neboť dosud není dovršena míra Emorejcovy nepravosti.“
#15:17 Když pak slunce zapadlo a nastala tma tmoucí, hle, objevila se dýmající pec a mezi těmi rozpůlenými kusy prošla ohnivá pochodeň.
#15:18 V ten den uzavřel Hospodin s Abramem smlouvu: „Tvému potomstvu dávám tuto zemi od řeky Egyptské až k řece veliké, řece Eufratu,
#15:19 zemi Kénijců, Kenazejců a Kadmónců,
#15:20 Chetejců, Perizejců a Refájců,
#15:21 Emorejců, Kenaanců, Girgašejců a Jebúsejců.“ 
#16:1 Sáraj, žena Abramova, mu nerodila. Měla egyptskou otrokyni, která se jmenovala Hagar.
#16:2 Jednou řekla Sáraj Abramovi: „Hle, Hospodin mi nedopřál, abych rodila, vejdi tedy k mé otrokyni, snad budu mít syna z ní.“ Abram Sárajiny rady uposlechl.
#16:3 Vzala tedy Abramova žena Sáraj svou otrokyni, Hagaru egyptskou, deset let po tom, co se Abram usadil v kenaanské zemi, a dala ji svému muži Abramovi za ženu.
#16:4 I vešel k Hagaře a ona otěhotněla. Když viděla, že je těhotná, přestala si své paní vážit.
#16:5 Tu řekla Sáraj Abramovi: „Mé příkoří musíš odčinit. Sama jsem ti dala svoji otrokyni do náruče, ale ona, jakmile uviděla, že je těhotná, přestala si mě vážit. Ať mezi mnou a tebou rozsoudí Hospodin.“
#16:6 Abram Sáraji odvětil: „Hle, otrokyně je v tvých rukou, nalož s ní, jak uznáš za dobré.“ Od té doby ji Sáraj pokořovala tak, že Hagar od ní uprchla.
#16:7 Nalezl ji Hospodinův posel ve stepi nad pramenem vody, nad pramenem při cestě do Šúru,
#16:8 a otázal se jí: „Hagaro, otrokyně Sáraje, odkud jsi přišla a kam jdeš?“ Odvětila: „Prchám od své paní Sáraje.“
#16:9 Hospodinův posel jí řekl: „Navrať se ke své paní a pokoř se pod její ruku.“
#16:10 Dále jí řekl: „Velice rozmnožím tvé potomstvo, takže je nebude možno ani spočítat.“
#16:11 A dodal: „Hle, jsi těhotná, porodíš syna a dáš mu jméno Izmael (to je Slyší Bůh), neboť Hospodin tě ve tvém pokoření slyšel.
#16:12 Bude to člověk nezkrotný, jeho ruka bude proti všem a ruce všech budou proti němu; bude stát proti všem svým bratřím.“
#16:13 I nazvala Hagar Hospodina, který k ní promluvil, „Bůh vševidoucí“, neboť řekla: „Zda právě zde jsem nesměla pohlédnout za tím, který mě vidí?“
#16:14 Proto se ta studně nazývá Studnicí Živého, který mě vidí; je mezi Kádešem a Beredem.
#16:15 Hagar porodila Abramovi syna. Abram nazval svého syna, kterého Hagar porodila, Izmael.
#16:16 Abramovi bylo osmdesát šest let, když mu Hagar porodila Izmaela. 
#17:1 Když bylo Abramovi devětadevadesát let, ukázal se mu Hospodin a řekl: „Já jsem Bůh všemohoucí, choď stále přede mnou, buď bezúhonný!
#17:2 Mezi sebe a tebe kladu svou smlouvu; převelice tě rozmnožím.“
#17:3 Tu padl Abram na tvář a Bůh k němu mluvil:
#17:4 „Já jsem! A toto je má smlouva s tebou: Staneš se praotcem hlučícího davu pronárodů.
#17:5 Nebudeš se už nazývat Abram; tvé jméno bude Abraham. Určil jsem tě za otce hlučícího davu pronárodů.
#17:6 Převelice tě rozplodím a učiním z tebe pronárody, i králové z tebe vzejdou.
#17:7 Smlouvu mezi sebou a tebou i tvým potomstvem ve všech pokoleních činím totiž smlouvou věčnou, že budu Bohem tobě i tvému potomstvu.
#17:8 A tobě i tvému potomstvu dávám do věčného vlastnictví zemi, v níž jsi hostem, tu celou zemi kenaanskou. A budu jim Bohem.“
#17:9 Bůh dále Abrahamovi řekl: „Ty i tvoje potomstvo budete mou smlouvu zachovávat ve všech pokoleních.
#17:10 Znamením mé smlouvy mezi mnou a vámi i tvým potomstvem, kterou budete zachovávat, bude toto: Každý mezi vámi, kdo je mužského pohlaví, bude obřezán.
#17:11 Dáte obřezat své neobřezané tělo a to bude znamením smlouvy mezi mnou a vámi.
#17:12 Po všechna pokolení každý, kdo je mezi vámi mužského pohlaví, bude osmého dne po narození obřezán, doma zrozený i koupený za stříbro od kteréhokoli cizince, který není z tvého potomstva.
#17:13 Musí být obřezán každý zrozený v tvém domě i koupený za stříbro. Tak bude má smlouva pro znamení na vašem těle smlouvou věčnou.
#17:14 Neobřezanec, který by nedal své neobřezané tělo obřezat, bude ze svého lidu vyobcován; porušil mou smlouvu.“
#17:15 Bůh také Abrahamovi řekl: „Svou ženu nebudeš už nazývat Sáraj, její jméno bude Sára (to je Kněžna).
#17:16 Požehnám ji a dám ti také z ní syna; požehnám ji a stane se matkou pronárodů a vzejdou z ní králové národů.“
#17:17 Tu padl Abraham na tvář, usmál se a v duchu si řekl: „Což se může narodit syn stoletému? Cožpak bude Sára rodit v devadesáti?“
#17:18 Proto Abraham Bohu řekl: „Kéž by Izmael žil v tvé blízkosti!“
#17:19 Bůh však pravil: „A přece ti tvá žena Sára porodí syna a nazveš ho Izák (to je Smíšek). Svou smlouvu s ním ustavím pro jeho potomstvo jako smlouvu věčnou.
#17:20 A pokud jde o Izmaela, vyslyšel jsem tě: Hle, požehnám mu a rozplodím a rozmnožím ho převelice; zplodí dvanáct knížat a učiním z něho veliký národ.
#17:21 Ale svoji smlouvu ustavím s Izákem, kterého ti porodí Sára příštího roku v tomto čase.“
#17:22 Bůh skončil rozmluvu s Abrahamem a vystoupil od něho.
#17:23 Abraham tedy vzal svého syna Izmaela a všechny zrozené ve svém domě i všechny koupené za stříbro, všechnu svou čeleď mužského pohlaví, a obřezal jejich neobřezané tělo hned toho dne, kdy k němu Bůh promluvil.
#17:24 Abrahamovi bylo devětadevadesát let, když jeho neobřezané tělo bylo obřezáno.
#17:25 Jeho synu Izmaelovi bylo třináct let, když bylo jeho neobřezané tělo obřezáno.
#17:26 Abraham i jeho syn Izmael byli obřezáni v týž den.
#17:27 Také všechna jeho čeleď, ať doma zrození či za stříbro od cizince koupení, byli obřezáni spolu s ním. 
#18:1 I ukázal se Hospodin Abrahamovi při božišti Mamre, když seděl za denního horka ve dveřích stanu.
#18:2 Rozhlédl se a spatřil: Hle, naproti němu stojí tři muži. Jakmile je spatřil, vyběhl jim ze dveří stanu vstříc, sklonil se k zemi
#18:3 a řekl: „Panovníku, jestliže jsem u tebe nalezl milost, nepomíjej svého služebníka.
#18:4 Dám přinést trochu vody, umyjte si nohy a zasedněte pod strom.
#18:5 Rád bych vám podal sousto chleba, abyste se posilnili; potom půjdete dál. Přece nepominete svého služebníka.“ Odvětili: „Učiň, jak říkáš.“
#18:6 Abraham rychle odběhl do stanu k Sáře a řekl: „Rychle vezmi tři míry bílé mouky, zadělej a připrav podpopelné chleby.“
#18:7 Sám se rozběhl k dobytku, vzal mladé a pěkné dobytče a dal mládenci, aby je rychle připravil.
#18:8 Potom vzal máslo a mléko i dobytče, jež připravil, a předložil jim to. Zatímco jedli, stál u nich pod stromem.
#18:9 Pak se ho otázali: „Kde je tvá žena Sára?“ Odpověděl: „Tady ve stanu.“
#18:10 I řekl jeden z nich: „Po obvyklé době se k tobě určitě vrátím, a hle, tvá žena bude mít syna.“ Sára naslouchala za ním ve dveřích stanu.
#18:11 Abraham i Sára byli staří, sešlí věkem, a Sáře již ustal běh ženský.
#18:12 Zasmála se v duchu a řekla si: „Když už jsem tak sešlá, má se mi dostat takové rozkoše? I můj pán je stařec.“
#18:13 Tu Hospodin Abrahamovi řekl: „Pročpak se Sára směje a říká: ‚Což mohu opravdu rodit, když už jsem tak stará?‘
#18:14 Je to snad pro Hospodina nějaký div? V jistém čase, po obvyklé době, se k tobě vrátím a Sára bude mít syna.“
#18:15 Sára však zapírala: „Nesmála jsem se“, protože se bála. On však řekl: „Ale ano, smála ses.“
#18:16 Muži se odtud zvedli a zamířili k Sodomě. Abraham šel s nimi, aby je doprovodil.
#18:17 Tu Hospodin řekl: „Mám Abrahamovi zamlčet, co hodlám učinit?
#18:18 Abraham se jistě stane velikým a zdatným národem a budou v něm požehnány všechny pronárody země.
#18:19 Důvěrně jsem se s ním sblížil, aby přikazoval svým synům a všem, kteří přijdou po něm: ‚Dbejte na Hospodinovu cestu a jednejte podle spravedlnosti a práva, ať Hospodin Abrahamovi splní, co mu přislíbil‘.“
#18:20 Hospodin dále pravil: „Křik ze Sodomy a Gomory je tak silný a jejich hřích je tak těžký,
#18:21 že už musím sestoupit a podívat se. Jestliže si počínají tak, jak je patrno z křiku, který ke mně přichází, je po nich veta; zjistím si, jak tomu je.“
#18:22 Zatímco se muži odtud ubírali k Sodomě, Abraham zůstal stát před Hospodinem.
#18:23 I přistoupil Abraham a řekl: „Vyhladíš snad se svévolníkem i spravedlivého?
#18:24 Možná, že je v tom městě padesát spravedlivých; vyhladíš snad i je a nepromineš tomu místu, přestože je v něm padesát spravedlivých?
#18:25 Přece bys neudělal něco takového a neusmrtil spolu se svévolníkem spravedlivého; pak by na tom byl spravedlivý stejně jako svévolník. To bys přece neudělal. Což Soudce vší země nejedná podle práva?“
#18:26 Hospodin odvětil: „Najdu-li v Sodomě, v tom městě, padesát spravedlivých, prominu kvůli nim celému místu.“
#18:27 Abraham pokračoval: „Dovoluji si k Panovníkovi mluvit, ač jsem prach a popel:
#18:28 Možná, že bude do těch padesáti spravedlivých pět chybět. Zahladíš pro těch pět celé město?“ Odvětil: „Nezahladím, najdu-li jich tam čtyřicet pět.“
#18:29 On však k němu mluvil ještě dále: „Možná, že se jich tam najde čtyřicet.“ Pravil: „Neudělám to kvůli těm čtyřiceti.“
#18:30 I řekl: „Ať se Panovník nerozhněvá, když budu mluvit dále: Možná, že se jich tam najde třicet.“ Pravil: „Neučiním to, najdu-li jich tam třicet.“
#18:31 Řekl pak: „Hle, dovoluji si promluvit k Panovníkovi znovu: Možná, že se jich tam najde dvacet.“ Pravil: „Nezahladím je kvůli těm dvaceti.“
#18:32 Nato řekl: „Ať se Panovník nerozhněvá, promluvím-li ještě jednou: Možná, že se jich tam najde deset.“ Pravil: „Nezahladím je ani kvůli těm deseti.“
#18:33 Hospodin po skončení rozmluvy s Abrahamem odešel a Abraham se vrátil ke svému místu. 
#19:1 I přišli ti dva poslové večer do Sodomy. Lot seděl v sodomské bráně. Když je spatřil, povstal jim vstříc, sklonil se tváří k zemi
#19:2 a řekl: „Snažně prosím, moji páni, uchylte se do domu svého služebníka. Přenocujte, umyjte si nohy a za časného jitra půjdete svou cestou.“ Odvětili: „Nikoli, přenocujeme na tomto prostranství.“
#19:3 On však na ně velice naléhal; uchýlili se tedy k němu a vešli do jeho domu. Připravil jim hostinu, dal napéci nekvašené chleby a pojedli.
#19:4 Dříve než ulehli, mužové toho města, muži sodomští, lid ze všech koutů, mladí i staří, obklíčili dům.
#19:5 Volali na Lota a řekli mu: „Kde máš ty muže, kteří k tobě této noci přišli? Vyveď nám je, abychom je poznali!“
#19:6 Lot k nim vyšel ke vchodu, dveře však za sebou zavřel.
#19:7 Řekl: „Bratří moji, nedělejte prosím nic zlého.
#19:8 Hleďte, mám dvě dcery, které muže nepoznaly. Jsem ochoten vám je vyvést a dělejte si s nimi, co se vám zlíbí. Jenom nic nedělejte těmto mužům! Vešli přece do stínu mého přístřeší.“
#19:9 Oni však vzkřikli: „Kliď se!“ A hrozili: „Sám je tu jen jako host a bude dělat soudce! Že s tebou naložíme hůř než s nimi!“ Obořili se na toho muže, na Lota, a chystali se vyrazit dveře.
#19:10 Vtom ti muži vlastníma rukama vtáhli Lota k sobě do domu a zavřeli dveře.
#19:11 Ale muže, kteří byli u vchodu do domu, malé i velké, ranili slepotou, takže nebyli schopni nalézt vchod.
#19:12 Tu řekli ti muži Lotovi: „Máš-li zde ještě někoho, zetě, syny, dcery, všechny, kteří v tomto městě patří k tobě, vyveď je z tohoto místa.
#19:13 My přinášíme tomuto místu zkázu, protože křik z něho je před Hospodinem tak velký, že nás Hospodin poslal, abychom je zničili.“
#19:14 Lot tedy vyšel a promluvil ke svým zeťům, kteří si měli vzít jeho dcery. Řekl jim: „Vyjděte hned z tohoto místa, poněvadž Hospodin chystá tomuto městu zkázu.“ Ale zeťům to připadalo, jako by žertoval.
#19:15 Když vzešla jitřenka, nutili poslové Lota: „Ihned vezmi svou ženu a obě dcery, které tu máš, abys pro nepravost města nezahynul.“
#19:16 Ale on váhal. Ti muži ho tedy uchopili za ruku, i jeho ženu a obě dcery - to shovívavost Hospodinova byla s ním -, vyvedli ho a dovolili mu odpočinout až za městem.
#19:17 Když je Hospodin vyváděl ven, řekl: „Uteč, jde ti o život. Neohlížej se zpět a v celém tomto okrsku se nezastavuj. Uteč na horu, abys nezahynul.“
#19:18 Lot jim však odvětil: „Ne tak prosím, Panovníku.
#19:19 Hle, tvůj služebník našel u tebe milost. Prokazuješ mi velké milosrdenství, že mě chceš zachovat při životě. Já však nemohu na tu horu utéci, aby mě nepostihlo něco zlého a abych nezemřel.
#19:20 Hle, tamto město je blízko, tam bych se mohl utéci, je jen maličké. Smím tam utéci? Což není opravdu maličké? Tak zůstanu naživu.“
#19:21 I řekl mu: „Vyhovím ti i v této věci; město, o kterém mluvíš, nepodvrátím.
#19:22 Uteč tam rychle, neboť nemohu nic učinit, dokud tam nevejdeš.“ Proto bylo to město nazváno Sóar (to je Maličké).
#19:23 Slunce vycházelo nad zemí, když Lot vešel do Sóaru.
#19:24 Hospodin začal chrlit na Sodomu a Gomoru síru a oheň; od Hospodina z nebe to bylo.
#19:25 Tak podvrátil ta města i celý okrsek a zničil všechny obyvatele měst, i co rostlo na rolích.
#19:26 Lotova žena šla vzadu, ohlédla se a proměnila se v solný sloup.
#19:27 Za časného jitra se Abraham vrátil k místu, kde stál před Hospodinem.
#19:28 Vyhlížel směrem k Sodomě a Gomoře a spatřil, jak po celé krajině toho okrsku vystupuje ze země dým jako dým z hutě.
#19:29 Ale Bůh, když vyhlazoval města toho okrsku, pamatoval na Abrahama: poslal Lota pryč ze středu zkázy, když vyvracel města, v nichž se Lot usadil.
#19:30 Lot pak vystoupil ze Sóaru a usadil se se svými dvěma dcerami na hoře, protože se bál usadit se v Sóaru. Usadil se s oběma dcerami v jeskyni.
#19:31 Tu řekla prvorozená té mladší: „Náš otec je stařec a není muže v zemi, aby k nám podle obyčeje celé země vešel.
#19:32 Pojď, dáme otci napít vína a budeme s ním ležet. Tak dáme život potomstvu ze svého otce.“
#19:33 Daly mu tedy té noci pít víno. Pak vešla prvorozená a ležela s otcem. On však nic nevěděl, ani když ulehla, ani když vstala.
#19:34 Příštího dne řekla prvorozená té mladší: „Hle, na dnešek jsem ležela s otcem já. Dáme mu pít víno i této noci a vejdeš ty a budeš s ním ležet. Tak dáme život potomstvu ze svého otce.“
#19:35 Daly mu tedy pít víno i této noci a mladší přišla a ležela s ním. On však nic nevěděl, ani když ulehla, ani když vstala.
#19:36 Tak obě Lotovy dcery otěhotněly se svým otcem.
#19:37 I porodila prvorozená syna a nazvala ho Moáb (to je Z otce zplozený); ten je praotcem Moábců až podnes.
#19:38 A mladší porodila také syna a nazvala ho Ben-amí (to je Syn mého příbuzného); ten je praotcem Amónovců až podnes. 
#20:1 Abraham táhl odtud do krajiny negebské a usadil se mezi Kádešem a Šúrem; pobýval jako host v Geraru.
#20:2 O své ženě Sáře Abraham řekl: „To je má sestra.“ Abímelek, král gerarský, poslal pro Sáru a vzal si ji.
#20:3 Té noci však přišel k abímelekovi ve snu Bůh a řekl mu: „Kvůli ženě, kterou sis vzal, zemřeš; vždyť je provdaná.“
#20:4 Proto se k ní abímelek nepřiblížil a řekl: „Panovníku, což vybiješ pronárod, i když je spravedlivý?
#20:5 Což mi on sám neřekl: ‚To je má sestra‘ ? Ano, i ona řekla: ‚To je můj bratr‘. Učinil jsem to v bezúhonnosti srdce a s čistýma rukama.“
#20:6 A Bůh mu ve snu odvětil: „I já vím, že jsi to učinil v bezúhonnosti srdce a sám jsem ti zabránil, aby ses proti mně neprohřešil; proto jsem ti nedovolil dotknout se jí.
#20:7 Teď však ženu toho muže navrať, neboť je to prorok. Bude se za tebe modlit a zůstaneš naživu. Nevrátíš-li ji, věz, že určitě zemřeš se všemi, kteří jsou tvoji.“
#20:8 Za časného jitra svolal abímelek všechny své služebníky a všechno jim vypověděl; a ty muže pojala velká bázeň.
#20:9 Abímelek tedy zavolal Abrahama a řekl mu: „Jak ses to k nám zachoval? Čím jsem se proti tobě prohřešil, že jsi na mne a na mé království uvedl takový hřích? Dopustil ses u mne něčeho, co se přece nedělá.“
#20:10 Dále se abímelek Abrahama otázal: „Co tě k tomu mělo, že jsi udělal takovou věc?“
#20:11 Abraham odvětil: „Řekl jsem si, že na tomto místě určitě není bázeň Boží a že mě kvůli mé ženě zabijí.
#20:12 Ona také vskutku je má sestra; je to dcera mého otce, ale ne dcera mé matky; stala se mou ženou.
#20:13 Když mě bohové po odchodu z otcova domu nechali bloudit, řekl jsem jí: ‚Prokazuj mi toto milosrdenství: na každém místě, kam přijdeme, říkej o mně, že jsem tvůj bratr‘.“
#20:14 Abímelek tedy vzal brav a skot, otroky a služebnice a dal je Abrahamovi a navrátil mu jeho ženu Sáru.
#20:15 Abímelek také řekl: „Hle, má země je před tebou; usaď se, kde uznáš za dobré.“
#20:16 Sáře pak řekl: „Hle, tvému bratru jsem dal tisíc šekelů stříbra; to bude pro tebe na zadostiučinění v očích všech, kteří jsou s tebou. Tím vším budeš obhájena.“
#20:17 I modlil se Abraham k Bohu a Bůh uzdravil abímeleka i jeho ženu a jeho otrokyně, takže rodily.
#20:18 Hospodin totiž kvůli Abrahamově ženě Sáře uzavřel v abímelekově domě každé lůno. 
#21:1 Hospodin navštívil Sáru, jak řekl, a splnil jí, co slíbil.
#21:2 Sára otěhotněla a Abrahamovi, ačkoli byl stár, porodila syna v čase, který mu Bůh předpověděl.
#21:3 Abraham dal svému narozenému synu, kterého mu Sára porodila, jméno Izák.
#21:4 Když mu bylo osm dní, Abraham svého syna Izáka obřezal, jak mu Bůh přikázal.
#21:5 Abrahamovi bylo sto let, když se mu syn Izák narodil.
#21:6 Tu Sára řekla: „Bůh mi dopřál, že se mohu smát. Se mnou ať se směje každý, kdo o tom uslyší.“
#21:7 A dodala: „Kdo by byl Abrahamovi řekl, že Sára bude kojit syny? A přece jsem mu porodila syna, ačkoli je stár.“
#21:8 Dítě rostlo a bylo odstaveno. V den, kdy Izáka odstavili, vystrojil Abraham veliké hody.
#21:9 Sára však viděla, že syn, jehož Abrahamovi porodila Hagar egyptská, je poštívač.
#21:10 Řekla Abrahamovi: „Zapuď tu otrokyni i jejího syna! Přece nebude syn té otrokyně dědicem spolu s mým synem Izákem.“
#21:11 Ale Abraham se tím velmi trápil; vždyť šlo o jeho syna.
#21:12 Bůh však Abrahamovi řekl: „Netrap se pro chlapce a pro tu otrokyni; poslechni Sáru ve všem, co ti říká, neboť tvé potomstvo bude povoláno z Izáka.
#21:13 Učiním však národ i ze syna otrokyně, neboť také on je tvým potomkem.“
#21:14 Za časného jitra vzal Abraham chléb a měch vody a dal Hagaře. Vložil jí dítě na ramena a propustil ji. Šla a bloudila po Beeršebské stepi.
#21:15 Když voda v měchu došla, odložila dítě pod jedním křoviskem.
#21:16 Odešla a usedla opodál, co by lukem dostřelil, neboť si řekla: „Nemohu se dívat, jak dítě umírá.“ Usedla tam, zaúpěla a rozplakala se.
#21:17 Bůh uslyšel hlas chlapce a Boží posel z nebe zavolal na Hagaru. Pravil jí: „Co je ti, Hagaro? Neboj se! Bůh slyšel hlas chlapce na tom místě, kde je.
#21:18 Vstaň, vezmi chlapce a pečuj o něj, já z něho učiním veliký národ.“
#21:19 Tu jí Bůh otevřel oči a ona spatřila studni s vodou. Šla, naplnila měch vodou a dala chlapci napít.
#21:20 A Bůh byl s chlapcem. Když vyrostl, usadil se ve stepi a stal se lučištníkem.
#21:21 Usadil se v Páranské stepi a jeho matka mu dala ženu z egyptské země.
#21:22 V té době řekl abímelek a píkol, velitel jeho vojska, Abrahamovi: „Bůh je s tebou ve všem, co činíš.
#21:23 Proto mi na místě přísahej při Bohu, že neoklameš mne ani mého nástupce a následníka. Milosrdenství, jaké jsem já prokázal tobě, prokazuj i ty mně a zemi, v níž jsi pobýval jako host.“
#21:24 Abraham tedy řekl: „Jsem hotov přísahat.“
#21:25 Ale domlouval abímelekovi kvůli studni s vodou, kterou abímelekovi služebníci uchvátili.
#21:26 Abímelek na to odvětil: „Nevím, kdo to udělal. Tys mi to neoznámil a já jsem o tom dodnes neslyšel.“
#21:27 I vzal Abraham brav a skot a dal abímelekovi. Pak spolu uzavřeli smlouvu.
#21:28 Sedm ovcí ze stáda postavil Abraham zvlášť.
#21:29 Tu se abímelek Abrahama otázal: „Co s těmi sedmi ovcemi, které jsi postavil zvlášť?“
#21:30 A on řekl: „Sedm ovcí si ode mne vezmi na svědectví, že jsem tuto studni vykopal já.“
#21:31 Proto se to místo nazývá Beer-šeba (to je Studně přísahy), že tam oba přísahali.
#21:32 Po uzavření smlouvy v Beer-šebě se abímelek a píkol, velitel jeho vojska, hned navrátili do pelištejské země.
#21:33 A Abraham zasadil v Beer-šebě tamaryšek a vzýval tam jméno Hospodina, Boha věčného.
#21:34 Abraham pobyl v pelištejské zemi jako host mnoho dní. 
#22:1 Po těch událostech chtěl Bůh Abrahama vyzkoušet. Řekl mu: „Abrahame!“ Ten odvětil: „Tu jsem.“
#22:2 A Bůh řekl: „Vezmi svého jediného syna Izáka, kterého miluješ, odejdi do země Mórija a tam ho obětuj jako oběť zápalnou na jedné hoře, o níž ti povím!“
#22:3 Za časného jitra osedlal tedy Abraham osla, vzal s sebou dva své služebníky a svého syna Izáka, naštípal dříví k zápalné oběti a vydal se k místu, o němž mu Bůh pověděl.
#22:4 Když se Abraham třetího dne rozhlédl a spatřil v dálce to místo,
#22:5 řekl služebníkům: „Počkejte tu s oslem, já s chlapcem půjdeme dále, vzdáme poctu Bohu a pak se k vám vrátíme.“
#22:6 Abraham vzal dříví k oběti zápalné a vložil je na svého syna Izáka; sám vzal oheň a obětní nůž. A šli oba pospolu.
#22:7 Tu Izák svého otce Abrahama oslovil: „Otče!“ Ten odvětil: „Copak, můj synu?“ Izák se otázal: „Hle, oheň a dříví je zde. Kde však je beránek k zápalné oběti?“
#22:8 Nato Abraham řekl: „Můj synu, Bůh sám si vyhlédne beránka k oběti zápalné.“ A šli oba spolu dál.
#22:9 Když přišli na místo, o němž mu Bůh pověděl, vybudoval tam Abraham oltář, narovnal dříví, svázal svého syna Izáka do kozelce a položil ho na oltář, nahoru na dříví.
#22:10 I vztáhl Abraham ruku po obětním noži, aby svého syna zabil jako obětního beránka.
#22:11 Vtom na něho z nebe volá Hospodinův posel: „Abrahame, Abrahame!“ Ten odvětil: „Tu jsem.“
#22:12 A posel řekl: „Nevztahuj na chlapce ruku, nic mu nedělej! Právě teď jsem poznal, že jsi bohabojný, neboť jsi mi neodepřel svého jediného syna.“
#22:13 Abraham se rozhlédl a vidí, že vzadu je beran, který uvízl svými rohy v houští. Šel tedy, vzal berana a obětoval jej v zápalnou oběť místo svého syna.
#22:14 Tomu místu dal Abraham jméno „Hospodin vidí“. Dosud se tu říká: „Na hoře Hospodinově se uvidí.“
#22:15 Hospodinův posel zavolal pak z nebe na Abrahama podruhé:
#22:16 „Přisáhl jsem při sobě, je výrok Hospodinův, protože jsi to učinil a neodepřel jsi mi svého jediného syna,
#22:17 jistotně ti požehnám a tvé potomstvo jistotně rozmnožím jako nebeské hvězdy a jako písek na mořském břehu. Tvé potomstvo obdrží bránu svých nepřátel
#22:18 a ve tvém potomstvu dojdou požehnání všechny pronárody země, protože jsi uposlechl mého hlasu.“
#22:19 Abraham se pak vrátil k služebníkům. Vydali se spolu na cestu do Beer-šeby, neboť tam Abraham sídlil.
#22:20 Po těchto událostech bylo Abrahamovi oznámeno: „Hle, také Milka porodila tvému bratru Náchorovi syny:
#22:21 prvorozeného Úsa a jeho bratra Búza, též Kemúela, otce Aramova,
#22:22 Keseda a Chazóa, Pildáše, Jidláfa a Betúela.
#22:23 Betúel pak zplodil Rebeku.“ Těch osm porodila Milka Abrahamovu bratru Náchorovi.
#22:24 Jeho ženina jménem Reúma také porodila, a to Tabecha a Gachama, Tachaše a Maaku. 
#23:1 Sára byla živa sto dvacet sedm let; to jsou léta Sářina života.
#23:2 Zemřela v Kirjat-arbě, což je Chebrón v kenaanské zemi. Abraham přišel, aby nad Sárou naříkal a oplakával ji.
#23:3 Pak od zemřelé vstal a promluvil k Chetejcům:
#23:4 „Jsem tu u vás host a přistěhovalec. Dejte mi u vás do vlastnictví hrob, kam bych zemřelou pohřbil.“
#23:5 Chetejci Abrahamovi odpověděli:
#23:6 „Slyš nás, pane. Jsi mezi námi jako kníže Boží; pochovej svou zemřelou v nejlepším z našich hrobů. Nikdo z nás ti neodepře svůj hrob, abys mohl zemřelou pohřbít.“
#23:7 Abraham se nato Chetejcům, lidu té země, uklonil
#23:8 a promluvil k nim dále: „Souhlasíte-li, abych zde svou zemřelou pohřbil, slyšte vy mne: Přimluvte se za mne u Efróna, syna Sócharova,
#23:9 ať mi přenechá svou makpelskou jeskyni, která je na konci jeho pole; ať mi ji před vámi přenechá za plnou cenu stříbra, abych měl vlastní hrob.“
#23:10 Efrón zasedal uprostřed Chetejců. Efrón chetejský tedy Abrahamovi v přítomnosti Chetejců odpověděl přede všemi, kteří měli přístup do rady v bráně toho města:
#23:11 „Nikoli, pane; slyš ty mne. To pole je tvé, i ta jeskyně na něm je tvá; dávám ti ji před očima synů svého lidu. Tam pochovej svou zemřelou.“
#23:12 I uklonil se Abraham před lidem té země
#23:13 a v přítomnosti lidu té země pravil Efrónovi: „Raději slyš ty mne. Dám za to pole stříbro. Vezmi je ode mne a já tam svou zemřelou pochovám.“
#23:14 Efrón Abrahamovi odpověděl:
#23:15 „Pane, slyš ty mne. Pozemek má cenu čtyř set šekelů stříbra. Co to pro nás znamená? Jen pochovej svou zemřelou.“
#23:16 Abraham souhlasil a odvážil Efrónovi stříbro, jak on sám v přítomnosti Chetejců navrhl, čtyři sta šekelů stříbra, běžných mezi obchodníky.
#23:17 Tak připadlo Efrónovo pole v Makpele naproti Mamre, pole i s jeskyní na něm a všechno stromoví na poli i na všem jeho pomezí kolem,
#23:18 do Abrahamova majetku před očima Chetejců, všech, kdo měli přístup do rady v bráně jeho města.
#23:19 Abraham potom svou ženu Sáru pohřbil v té jeskyni na poli v Makpele naproti Mamre, jež je u Chebrónu v kenaanské zemi.
#23:20 Tak připadlo pole Chetejců i s jeskyní na něm Abrahamovi, aby měl vlastní hrob. 
#24:1 Abraham byl stařec pokročilého věku. Hospodin mu ve všem požehnal.
#24:2 I řekl Abraham služebníku, správci svého domu, který vládl vším, co mu patřilo: „Polož ruku na můj klín.
#24:3 Zavazuji tě přísahou při Hospodinu, Bohu nebes a Bohu země, abys nebral pro mého syna ženu z dcer Kenaanců, mezi nimiž sídlím.
#24:4 Půjdeš do mé země a do mého rodiště a vezmeš odtamtud ženu pro mého syna Izáka.“
#24:5 Služebník mu na to odvětil: „Co když mě ta žena nebude chtít následovat sem, do této země? Mám tvého syna zavést zpátky do země, z níž jsi vyšel?“
#24:6 Abraham řekl: „Chraň se tam mého syna zavést!
#24:7 Hospodin, Bůh nebes, který mě vzal z domu mého otce a z mé rodné země, promluvil ke mně a přísahal mi, že tuto zemi dá mému potomstvu; on sám vyšle před tebou svého posla, a ty budeš moci vzít odtamtud ženu pro mého syna.
#24:8 Kdyby tě snad ta žena nechtěla následovat, budeš své přísahy zproštěn. Jen tam mého syna nezaváděj!“
#24:9 I položil služebník ruku na klín svého pána Abrahama a odpřisáhl mu to.
#24:10 Služebník tedy vzal deset velbloudů svého pána a jel. Vybaven všelijakými drahocennostmi svého pána vydal se na cestu do aramského Dvojříčí, do města Náchorova.
#24:11 Před městem zastavil velbloudy u studně s vodou. Bylo to navečer, v době, kdy ženy vycházívají čerpat vodu.
#24:12 Tu řekl: „Hospodine, Bože mého pána Abrahama, dopřej mi to prosím a prokaž milosrdenství mému pánu Abrahamovi.
#24:13 Hle, stojím nad pramenem a dcery mužů města vycházejí čerpat vodu.
#24:14 Dívce, která přijde, řeknu: Nakloň svůj džbán, abych se napil. Odvětí-li: ‚Jen pij, a také tvé velbloudy napojím‘, předurčil jsi ji pro svého služebníka Izáka. Podle toho poznám, že jsi mému pánu prokázal milosrdenství.“
#24:15 Ještě než domluvil, přišla Rebeka, která se narodila Betúelovi, synu Milky, ženy Abrahamova bratra Náchora. Měla na rameni džbán.
#24:16 Byla to dívka velmi půvabného vzhledu, panna, muž ji dosud nepoznal. Sestoupila k pramenu, naplnila svůj džbán a vystoupila.
#24:17 Služebník jí přiběhl naproti a řekl: „Dej mi prosím doušek vody ze džbánu!“
#24:18 Odvětila: „Jen se napij, můj pane.“ A rychle spustila džbán na ruku a dala mu pít.
#24:19 Když mu dala napít, řekla: „Načerpám i tvým velbloudům, aby se napili.“
#24:20 Rychle vylila vodu ze svého džbánu do napajedla a znovu odběhla ke studni čerpat, aby napojila i všechny jeho velbloudy.
#24:21 Muž ji přitom mlčky pozoroval, aby poznal, zda Hospodin dává jeho cestě zdar, či nikoli.
#24:22 Když se velbloudi napili, vzal muž zlatý nosní kroužek o váze půl šekelu a dva náramky pro ni o váze deseti šekelů zlata.
#24:23 Řekl: „Pověz mi prosím, čí jsi dcera? Bylo by pro nás v domě tvého otce místo k přenocování?“
#24:24 Odvětila mu: „Jsem dcera Betúela, syna Milky, kterého porodila Náchorovi.“
#24:25 A dodala: „Slámy i obroku máme dost, i místo k přenocování.“
#24:26 Tu padl muž na kolena, klaněl se Hospodinu
#24:27 a řekl: „Požehnán buď Hospodin, Bůh mého pána Abrahama, že od mého pána neodňal své milosrdenství a svou věrnost. Dovedl mě až do domu bratří mého pána.“
#24:28 Dívka odběhla a oznámila rodině své matky, co a jak se stalo.
#24:29 Rebeka měla bratra jménem Lábana. Lában běžel ven k tomu muži u pramene.
#24:30 Viděl totiž nosní kroužek a na sestřiných rukou náramky a slyšel slova své sestry Rebeky, o čem s ní ten muž mluvil. Přišel k tomu muži, který dosud stál u velbloudů nad pramenem,
#24:31 a řekl: „Pojď, požehnaný Hospodinův! Proč stojíš venku? Už jsem připravil dům i místo pro velbloudy.“
#24:32 I vešel ten muž do domu a odstrojil velbloudy. Lában dal velbloudům slámu a obrok; jemu a jeho mužům dal vodu k umytí nohou.
#24:33 Pak dal před něho prostřít, aby pojedl. On však řekl: „Nebudu jíst, dokud nevyřídím svou záležitost.“ Lában odvětil: „Mluv.“
#24:34 Pravil tedy: „Jsem služebník Abrahamův.
#24:35 Hospodin mému pánu velmi požehnal, takže je zámožný. Dal mu brav a skot, stříbro a zlato, otroky a otrokyně, velbloudy a osly.
#24:36 A Sára, žena mého pána, ač velmi stará, porodila mému pánu syna. Jemu dá všechno, co má.
#24:37 Mne pak můj pán zavázal přísahou: ‚Nevezmeš pro mého syna ženu z dcer Kenaanců, v jejichž zemi sídlím.
#24:38 Půjdeš do domu mého otce a k mé čeledi a vezmeš odtamtud ženu pro mého syna.‘
#24:39 Nato jsem svému pánu řekl: ‚Možná, že se mnou ta žena nepůjde.‘
#24:40 Odvětil mi: ‚Hospodin, před nímž chodím, pošle s tebou svého posla a dá tvé cestě zdar. Ty pak vezmeš pro mého syna ženu z mé čeledi a z domu mého otce.
#24:41 Svého závazku se zprostíš tehdy, když dojdeš k mé čeledi: Nedají-li ti ji, budeš svého závazku zproštěn.‘
#24:42 Dnes jsem přišel k prameni a řekl jsem: ‚Hospodine, Bože mého pána Abrahama, kéž bys dal zdar cestě, po níž jdu!‘
#24:43 Hle, stojím nad pramenem. Dívce, která přijde čerpat vodu, řeknu: ‚Dej mi prosím napít ze svého džbánu trochu vody.‘
#24:44 Jestliže mi ona řekne: ‚Jen se napij, a také tvým velbloudům načerpám‘, bude to žena, kterou synu mého pána předurčil Hospodin.
#24:45 Ještě jsem sám u sebe nedomluvil, když tu přišla Rebeka se džbánem na rameni. Sestoupila k pramenu a čerpala vodu. Řekl jsem jí: ‚Dej mi prosím napít!‘
#24:46 Rychle spustila džbán s ramene a řekla: ‚Jen se napij, a také tvé velbloudy napojím.‘ Napil jsem se a ona napojila i velbloudy.
#24:47 Pak jsem se jí zeptal: ‚Čí jsi dcera?‘ Odvětila: ‚Jsem dcera Betúela, syna Náchora, kterého mu porodila Milka.‘ Navlékl jsem jí tento kroužek do nosu a náramky na ruce.
#24:48 Padl jsem na kolena a klaněl se Hospodinu. Dobrořečil jsem Hospodinu, Bohu mého pána Abrahama, že mě vedl pravou cestou, abych mohl dceru pánova bratra přivést pro jeho syna.
#24:49 A teď mi povězte, chcete-li mému pánu prokázat milosrdenství a věrnost. Ne-li, povězte mi, a já se obrátím napravo nebo nalevo.“
#24:50 Lában i Betúel odpověděli: „Toto vyšlo od Hospodina a my nemůžeme tobě říci ani zlé ani dobré.
#24:51 Hle, tady je Rebeka. Vezmi si ji a jdi, ať se stane ženou syna tvého pána, jak mluvil Hospodin.“
#24:52 Jakmile Abrahamův služebník uslyšel jejich slova, poklonil se Hospodinu až k zemi.
#24:53 Potom vyňal stříbrné a zlaté předměty a látky a dal je Rebece, též jejímu bratrovi a její matce dal vzácné dary.
#24:54 Pak jedli a pili, on i mužové, kteří byli s ním, a přenocovali. Když ráno vstali, řekl: „Propusťte mě k mému pánu.“
#24:55 Ale její bratr a matka odvětili: „Ať s námi dívka ještě nějaký čas zůstane, aspoň deset dní; potom půjde.“
#24:56 On však jim řekl: „Nezdržujte mě, když Hospodin dopřál mé cestě zdaru. Propusťte mě, ať mohu jít k svému pánu.“
#24:57 Odvětili: „Zavoláme dívku a zeptáme se přímo jí.“
#24:58 Zavolali Rebeku a otázali se jí: „Půjdeš s tímto mužem?“ Řekla: „Půjdu.“
#24:59 Propustili tedy svou sestru Rebeku a její chůvu i Abrahamova služebníka a jeho muže.
#24:60 Rebece požehnali slovy: „Sestro naše, buď matkou nesčíslných tisíců, tvé potomstvo ať obsadí bránu těch, kteří je nenávidí!“
#24:61 Rebeka a její dívky vsedly hned potom na velbloudy a následovaly toho muže. Tak dostal služebník Rebeku a odešel.
#24:62 Izák právě přicházel od Studnice Živého, který mě vidí, neboť sídlil v Negebu.
#24:63 Vyšel totiž k večeru na pole a byl zamyšlen. Tu se rozhlédl a spatřil přicházet velbloudy.
#24:64 I Rebeka se rozhlédla a spatřila Izáka. Sesedla z velblouda
#24:65 a otázala se služebníka: „Kdo je ten muž, který nám jde po poli vstříc?“ Služebník odvětil: „To je můj pán.“ Vzala tedy roušku a zahalila se.
#24:66 Služebník podal Izákovi zprávu o všem, co vykonal.
#24:67 Izák pak uvedl Rebeku do stanu své matky Sáry. Vzal si ji a stala se jeho ženou. A zamiloval si ji. Tak našel útěchu po smrti své matky. 
#25:1 Abraham si vzal opět ženu; jmenovala se Ketúra.
#25:2 Porodila mu Zimrána a Jokšána, Medána a Midjána, Jišbáka a Šúacha.
#25:3 Jokšán zplodil Šebu a Dedána. Synové Dedánovi byli Ašúrejci a Letúšejci a Leumejci.
#25:4 Midjánovi synové byli Éfa a Éfer a Chanók, Abída a Eldáa. Ti všichni jsou synové Ketúřini.
#25:5 Abraham však všechno, co měl, odkázal Izákovi.
#25:6 Synům svých ženin, které měl, dal dary a ještě za svého života je poslal od svého syna Izáka pryč na východ, do země východní.
#25:7 Toto jsou léta Abrahamova života, jichž se dožil: sto sedmdesát pět let.
#25:8 I zesnul Abraham a zemřel v utěšeném stáří, stár a sytý dnů, a byl připojen k svému lidu.
#25:9 Jeho synové Izák a Izmael ho pochovali do makpelské jeskyně na poli Efróna, syna Chetejce Sóchara, naproti Mamre.
#25:10 To pole koupil Abraham od Chetejců. Tam byl pochován Abraham i jeho žena Sára.
#25:11 Po Abrahamově smrti požehnal Bůh jeho synu Izákovi. A Izák sídlil u Studnice Živého, který mě vidí.
#25:12 Toto je rodopis Abrahamova syna Izmaela, kterého Abrahamovi porodila Sářina otrokyně Hagar egyptská.
#25:13 Toto jsou jména Izmaelových synů, podle nichž jsou pojmenovány jejich rody: Izmaelův prvorozený Nebajót, Kédar, Adbeel a Mibsám,
#25:14 Mišma, Dúma a Masa,
#25:15 Chadad, Téma, Jetúr, Náfiš a Kedma.
#25:16 To jsou Izmaelovi synové a to jsou jejich jména podle jejich dvorců a hradišť; dvanáct předáků jejich národů.
#25:17 A toto jsou léta Izmaelova života: sto třicet sedm let. I zesnul a zemřel a byl připojen k svému lidu.
#25:18 Jeho synové bydlili od Chavíly až k Šúru proti Egyptu a směrem k Asýrii. Izmael se položil proti všem svým bratřím.
#25:19 Toto je rodopis Abrahamova syna Izáka: Abraham zplodil Izáka.
#25:20 Izákovi bylo čtyřicet let, když si vzal za ženu Rebeku, dceru Aramejce Betúela z Rovin aramských, sestru Aramejce Lábana.
#25:21 Izák prosil Hospodina za svou ženu, protože byla neplodná. Hospodin jeho prosby přijal, a jeho žena Rebeka otěhotněla.
#25:22 Děti se však začaly v jejím těle strkat. Tu řekla: „Je-li tomu tak, co mě čeká?“ A šla se dotázat Hospodina.
#25:23 Hospodin jí řekl: „Ve tvém životě jsou dva pronárody. Oba národy se rozejdou, jen co z tebe vyjdou. Jeden národ bude zdatnější než druhý, bezpočetný bude sloužit počtem skrovnějšímu.“
#25:24 Potom se naplnily dny, kdy měla rodit. A hle, v jejím životě byla dvojčata.
#25:25 První vyšel celý červenohnědý a chlupatý jako kožíšek; toho pojmenovali Ezau.
#25:26 Potom vyšel jeho bratr a rukou držel Ezaua za patu; ten dostal jméno Jákob. Při jejich narození bylo Izákovi šedesát let.
#25:27 Když chlapci dospěli, stal se Ezau mužem znalým lovu, mužem pole, kdežto Jákob byl muž bezúhonný a sídlil ve stanech.
#25:28 Izák miloval Ezaua, protože z lovu měl co do úst, kdežto Rebeka milovala Jákoba.
#25:29 Jákob jednou připravil krmi. Tu přišel Ezau z pole znavený
#25:30 a řekl Jákobovi: „Dej mi zhltnout trochu toho červeného, toho krvavého, jsem znaven k smrti.“ Proto se jmenuje Edóm (to je Červený).
#25:31 Jákob však řekl: „Prodej mi dnes své prvorozenství!“
#25:32 Ezau na to odvětil: „Stejně mám blízko k smrti, k čemu je mi prvorozenství!“
#25:33 Jákob řekl: „Odpřisáhni mi to dnes.“ A on mu to odpřisáhl, a tak své prvorozenství prodal Jákobovi.
#25:34 Jákob dal pak Ezauovi chléb a čočkovou krmi. Ten pojedl, napil se, vstal a odešel. Tak Ezau pohrdl prvorozenstvím. 
#26:1 V zemi nastal opět hlad, jiný než onen první, který byl za dnů Abrahamových. Izák tedy odešel do Geraru k abímelekovi, pelištejskému králi.
#26:2 Ukázal se mu totiž Hospodin a pravil: „Nesestupuj do Egypta. Přebývej v zemi, o níž ti řeknu.
#26:3 Pobývej v této zemi jako host. Já budu s tebou a požehnám ti, tobě a tvému potomstvu dám všechny tyto země. Tak splním přísahu, jíž jsem se zapřisáhl tvému otci Abrahamovi:
#26:4 ‚Tvé potomstvo rozmnožím jako nebeské hvězdy; tvému potomstvu dám všechny tyto země. V tvém potomstvu dojdou požehnání všechny pronárody země.‘
#26:5 To proto, že Abraham uposlechl mého hlasu a dbal na to, co jsem mu svěřil: na má přikázání, nařízení a zákony.“
#26:6 Izák se tedy usadil v Geraru.
#26:7 Když se muži toho místa vyptávali na jeho ženu, řekl: „Je to má sestra.“ Bál se totiž říci, že je to jeho žena, aby ho muži toho místa kvůli Rebece nezabili, neboť byla půvabného vzhledu.
#26:8 Jednou, když tam byl už delší dobu, vyhlédl z okna abímelek, pelištejský král, a spatřil Izáka, jak se laská se svou ženou Rebekou.
#26:9 Předvolal tedy abímelek Izáka a řekl: „To je určitě tvá žena. Pročpak jsi mi řekl, že je to tvá sestra?“ Izák odvětil: „Řekl jsem to, abych kvůli ní nepřišel o život.“
#26:10 Abímelek pravil: „Cos nám to učinil? Málem by byl někdo z lidu spal s tvou ženou a uvedl bys na nás vinu!“
#26:11 Proto abímelek přikázal všemu lidu: „Kdo by se dotkl tohoto muže nebo jeho ženy, bude bez milosti usmrcen.“
#26:12 Izák začal v té zemi sít a sklidil toho roku stonásobně; tak mu Hospodin požehnal.
#26:13 Tak se ten muž vzmohl a vzmáhal se stále víc, až se stal velice zámožným.
#26:14 Měl stáda bravu a stáda skotu i četnou čeládku. Pelištejci mu proto záviděli.
#26:15 Pelištejci zasypali všechny studně, které vykopali Abrahamovi služebníci za Izákova otce Abrahama, a naplnili je prachem.
#26:16 Abímelek řekl Izákovi: „Odejdi od nás, neboť jsi mnohem mocnější než my.“
#26:17 Izák tedy odtud odešel, utábořil se v Gerarském úvalu a usadil se tam.
#26:18 Znovu kopal studně, které vykopali za dnů jeho otce Abrahama a které po Abrahamově smrti Pelištejci zasypali. Pojmenoval je stejně jako jeho otec.
#26:19 Izákovi služebníci kopali v tom úvalu a přišli na studni pramenité vody.
#26:20 Ale gerarští pastýři se s pastýři Izákovými přeli: „Ta voda patří nám!“ Proto tu studni pojmenoval Esek (to je Váda), že se s ním vadili.
#26:21 Vykopali tedy jinou studni. O tu se také přeli, proto ji pojmenoval Sitná (to je Sočení).
#26:22 Pak postoupil dál a vykopal další studni; o tu se již nepřeli. Pojmenoval ji Rechobót (to je Prostorná) a řekl: „Teď už nám Hospodin poskytl prostor, abychom se mohli na zemi rozplodit.“
#26:23 Odtud vystoupil do Beer-šeby.
#26:24 Tu noc se mu ukázal Hospodin a pravil: „Já jsem Bůh tvého otce Abrahama. Neboj se, jsem s tebou. Požehnám ti a rozmnožím tvé potomstvo kvůli Abrahamovi, svému služebníku.“
#26:25 Izák tam vybudoval oltář a vzýval Hospodinovo jméno. Postavil tam také svůj stan a jeho služebníci tam vyhloubili studni.
#26:26 Tu k němu přišel z Geraru abímelek se svým přítelem Achuzatem a píkol, velitel jeho vojska.
#26:27 Izák se jich otázal: „Proč jste ke mně přišli? Vždyť jste mě nenáviděli a vypověděli jste mě.“
#26:28 Odvětili: „Shledali jsme, že je s tebou Hospodin. Proto jsme si řekli: Ať nás stihne kletba, nás i tebe, porušíme-li smlouvu, kterou s tebou chceme uzavřít.
#26:29 Neučiníš nám nic zlého, jako jsme se ani my nedotkli tebe. Prokazovali jsme ti jen dobro a propustili jsme tě v pokoji. Jsi přece Hospodinův požehnaný!“
#26:30 Izák jim vystrojil hody, i jedli a pili.
#26:31 Za časného jitra se zavázali vzájemnou přísahou. Potom je Izák propustil a oni od něho odešli v pokoji.
#26:32 Právě toho dne přišli Izákovi služebníci a pověděli mu o studni, kterou vykopali. Řekli mu: „Našli jsme vodu!“
#26:33 Nazval ji Šibea (to je Přísežná). Proto se to město jmenuje Beer-šeba (to je Studně přísahy) až podnes.
#26:34 Když bylo Ezauovi čtyřicet let, vzal si za ženu Jehúditu, dceru Chetejce Beera, a Basematu, dceru Chetejce Elóna.
#26:35 Ty působily Izákovi a Rebece jen hořké trápení. 
#27:1 Když Izák zestárl, jeho oči vyhasly, takže neviděl. I zavolal svého staršího syna Ezaua a řekl mu: „Můj synu!“ On mu odvětil: „Tu jsem.“
#27:2 Izák řekl: „Hle, jsem už starý a neznám den své smrti.
#27:3 Vezmi si nyní zbraně, toulec a luk, vyjdi na pole a něco pro mě ulov.
#27:4 Připrav mi oblíbenou pochoutku a přines mi ji, ať se najím, abych ti mohl požehnat, dříve než umřu.“
#27:5 Když Izák mluvil se svým synem Ezauem, Rebeka naslouchala. Sotva Ezau odešel na pole, aby něco ulovil a přinesl úlovek,
#27:6 poradila Rebeka svému synu Jákobovi: „Hle, slyšela jsem rozmluvu tvého otce s tvým bratrem Ezauem; řekl mu:
#27:7 ‚Přines mi úlovek a připrav mi pochoutku, ať se najím, abych ti před smrtí požehnal před Hospodinem.‘
#27:8 A proto, můj synu, poslechni mě ve všem, co ti přikážu.
#27:9 Dojdi ke stádu a dones mi z něho dvě pěkná kůzlata. Připravím z nich tvému otci oblíbenou pochoutku
#27:10 a ty mu ji doneseš; on se nají a před smrtí ti požehná.“
#27:11 Jákob však své matce Rebece odvětil: „Můj bratr Ezau je přece chlupatý, a já jsem holý.
#27:12 Co když si otec na mě sáhne? Bude mě mít za podvodníka a místo požehnání na sebe uvedu zlořečení.“
#27:13 Matka mu řekla: „Takové zlořečení ať padne na mne, můj synu. Jen mě poslechni, jdi a dones mi kůzlata.“
#27:14 Šel tedy pro ně a přinesl je matce; ona pak připravila otcovu oblíbenou pochoutku.
#27:15 Potom Rebeka vzala šaty svého staršího syna Ezaua, ty nejlepší, které měla doma u sebe, a oblékla svého mladšího syna Jákoba.
#27:16 Jeho ruce i hladký krk ovinula kůzlečími kožkami.
#27:17 Nakonec dala svému synu Jákobovi do rukou připravenou pochoutku a chléb.
#27:18 I vešel k svému otci a řekl: „Můj otče!“ On odvětil: „Tu jsem. Který jsi ty, můj synu?“
#27:19 Jákob řekl otci: „Já jsem Ezau, tvůj prvorozený. Učinil jsem, co jsi mi uložil. Nuže, posaď se prosím a pojez z mého úlovku, abys mi mohl požehnat.“
#27:20 Izák však synovi řekl: „Jak to, žes to tak rychle našel, můj synu?“ Odvětil: „To mi dopřál Hospodin, tvůj Bůh.“
#27:21 Izák řekl Jákobovi: „Přistup, synu, sáhnu si na tebe, jsi-li můj syn Ezau nebo ne.“
#27:22 Jákob tedy přistoupil k svému otci Izákovi, on na něho sáhl a řekl: „Hlas je to Jákobův, ale ruce jsou Ezauovy.“
#27:23 Nepoznal ho, protože jeho ruce byly chlupaté jako ruce jeho bratra Ezaua. A požehnal mu.
#27:24 Řekl: „Ty jsi tedy můj syn Ezau.“ On odvětil: „Jsem.“
#27:25 Pak řekl: „Předlož mi, ať pojím z úlovku svého syna, abych ti mohl požehnat.“ I předložil mu a on jedl. Přinesl mu i víno a on pil.
#27:26 Nato jeho otec Izák řekl: „Přistup prosím a polib mě, můj synu!“
#27:27 Přistoupil tedy a políbil ho. Když Izák ucítil vůni jeho šatu, požehnal mu slovy: „Hle, vůně mého syna jako vůně pole, jemuž žehná Hospodin.
#27:28 Dej ti Bůh z rosy nebes a ze žírnosti země, i hojnost obilí a moštu.
#27:29 Ať ti slouží lidská pokolení, ať se ti klanějí národy. Budeš panovat nad svými bratry a synové tvé matky se ti budou klanět. Kdo prokleje tebe, bude proklet, kdo žehnat bude tobě, sám bude požehnán.“
#27:30 Když Izák udělil Jákobovi požehnání a sotvaže Jákob od svého otce odešel, přišel jeho bratr Ezau z lovu.
#27:31 Také on připravil pochoutku, přinesl ji otci a řekl: „Nechť můj otec povstane a nají se z úlovku svého syna, aby mi mohl požehnat.“
#27:32 Jeho otec Izák se ho otázal: „Kdo jsi?“ Odvětil: „Jsem tvůj syn Ezau, tvůj prvorozený.“
#27:33 Tu se Izák roztřásl a zalomcovalo jím zděšení: „Kdo to vlastně ulovil úlovek a přinesl mi jej? Ode všeho jsem pojedl, dříve než jsi přišel! A požehnal jsem mu! Požehnaný také zůstane.“
#27:34 Jak Ezau uslyšel otcova slova, dal se do hrozného a hořkého křiku a naléhal na otce: „Požehnej mně, také mně, otče!“
#27:35 On odvětil: „Přišel lstivě tvůj bratr a vzal ti požehnání.“
#27:36 Ezau řekl: „Právem dostal jméno Jákob (to je Úskočný). Už dvakrát se mnou jednal úskočně. Připravil mě o mé prvorozenství a nyní i o požehnání.“ A otázal se: „Pro mne už požehnání nemáš?“
#27:37 Izák Ezauovi odpověděl: „Hle, ustanovil jsem, aby nad tebou panoval, a všechny jeho bratry jsem mu dal za služebníky. Zabezpečil jsem jej i obilím a moštem. Co bych mohl udělat pro tebe, můj synu?“
#27:38 Ezau otci odpověděl: „Což máš jen to jediné požehnání, otče? Požehnej mně, také mně, otče!“ A Ezau zaúpěl a rozplakal se.
#27:39 Jeho otec Izák tedy odpověděl: „Tvé sídliště bude bez žírnosti země, bez rosy shůry, rosy nebeské.
#27:40 Bude tě živit meč, avšak svému bratru budeš sloužit. Jen když se budeš toulat bez domova, setřeseš jeho jho ze své šíje.“
#27:41 I zanevřel Ezau na Jákoba pro požehnání, jímž mu jeho otec požehnal. A řekl sám k sobě: „Mému otci se blíží dny truchlení; zabiji svého bratra Jákoba.“
#27:42 Když byla Rebece oznámena slova jejího staršího syna Ezaua, dala si zavolat svého mladšího syna Jákoba a řekla mu: „Tvůj bratr Ezau pomýšlí na pomstu. Chce tě zabít.
#27:43 Proto mě nyní poslechni, můj synu: Ihned uprchni k mému bratru Lábanovi do Cháranu.
#27:44 Zůstaneš u něho nějaký čas, dokud rozhořčení tvého bratra nepomine.
#27:45 Až se bratrův hněv od tebe odvrátí a on zapomene, cos mu udělal, pošlu pro tebe a vezmu tě odtamtud. Proč mám být zbavena vás obou v jednom dni?“
#27:46 Izákovi pak Rebeka řekla: „Zprotivilo se mi žít s dcerami chetejskými. Vezme-li si Jákob ženu z chetejských dcer, jako jsou tyhle dcery této země, k čemu mi život?“ 
#28:1 I povolal Izák Jákoba a požehnal mu. Přikázal mu: „Neber si ženu z dcer kenaanských.
#28:2 Odejdi do Rovin aramských do domu Betúela, otce své matky, a odtud si vezmi ženu z dcer jejího bratra Lábana.
#28:3 Kéž ti Bůh všemohoucí požehná, rozplodí tě a rozmnoží. Bude z tebe společenství lidských pokolení.
#28:4 Kéž ti dá požehnání Abrahamovo, tobě i tvému potomstvu, abys obdržel zemi, v níž jsi hostem a kterou Bůh dal Abrahamovi.“
#28:5 Pak Izák Jákoba propustil a ten šel do Rovin aramských k Lábanovi, synu Aramejce Betúela, bratru Jákobovy a Ezauovy matky Rebeky.
#28:6 Ezau viděl, že Izák Jákobovi požehnal a poslal ho do Rovin aramských, aby si odtamtud vzal ženu; že mu při požehnání přikázal, aby si nebral ženu z dcer kenaanských;
#28:7 že Jákob otce i matku uposlechl a odešel do Rovin aramských.
#28:8 Viděl též, že kenaanské dcery jsou jeho otci Izákovi odporné.
#28:9 Šel tedy Ezau k Izmaelovi a přibral si ke svým ženám ještě Machalatu, dceru Abrahamova syna Izmaela, sestru Nebajótovu.
#28:10 Jákob vyšel z Beer-šeby a šel do Cháranu.
#28:11 Dorazil na jedno místo a přenocoval tam, neboť slunce již zapadlo. Vzal jeden z kamenů, které na tom místě byly, postavil jej v hlavách a na tom místě ulehl.
#28:12 Měl sen: Hle, na zemi stojí žebřík, jehož vrchol dosahuje k nebesům, a po něm vystupují a sestupují poslové Boží.
#28:13 Nad ním stojí Hospodin a praví: „Já jsem Hospodin, Bůh tvého otce Abrahama a Bůh Izákův. Zemi, na níž ležíš, dám tobě a tvému potomstvu.
#28:14 Tvého potomstva bude jako prachu země. Rozmůžeš se na západ i na východ, na sever i na jih. V tobě a v tvém potomstvu dojdou požehnání všechny čeledi země.
#28:15 Hle, já jsem s tebou. Budu tě střežit všude, kam půjdeš, a zase tě přivedu do této země. Nikdy tě neopustím, ale učiním, co jsem ti slíbil.“
#28:16 Tu procitl Jákob ze spánku a zvolal: „Jistě je na tomto místě Hospodin, a já jsem to nevěděl!“
#28:17 Bál se a řekl: „Jakou bázeň vzbuzuje toto místo! Není to nic jiného než dům Boží, je to brána nebeská.“
#28:18 Za časného jitra vzal Jákob kámen, který měl v hlavách, a postavil jej jako posvátný sloup; svrchu jej polil olejem.
#28:19 Tomu místu dal jméno Bét-el (to je Dům Boží). Původně se to město jmenovalo Lúz.
#28:20 Jákob se tu zavázal slibem: „Bude-li Bůh se mnou, bude-li mě střežit na cestě, na niž jsem se vydal, dá-li mi chléb k jídlu a šat k odívání
#28:21 a navrátím-li se v pokoji do domu svého otce, bude mi Hospodin Bohem.
#28:22 Tento kámen, který jsem postavil jako posvátný sloup, stane se domem Božím. A ze všeho, co mi dáš, odvedu ti poctivě desátky.“ 
#29:1 Jákob vykročil lehkým krokem a přišel do země synů Východu.
#29:2 Pojednou spatřil v poli studni, u níž odpočívala tři stáda ovcí. Z té studně napájeli stáda. Na jejím otvoru byl veliký kámen.
#29:3 Když přihnali všechna stáda, odvalovali kámen z otvoru studně a napájeli ovce; potom zase kámen přivalili zpět na otvor studně.
#29:4 I řekl jim Jákob: „Bratří, odkud jste?“ Odvětili: „Jsme z Cháranu.“
#29:5 Otázal se jich: „Znáte Lábana, syna Náchorova?“ Řekli: „Známe.“
#29:6 Zeptal se jich: „Daří se mu dobře?“ Odpověděli: „Dobře. Však tady přichází jeho dcera Ráchel s ovcemi.“
#29:7 Tu řekl: „Ještě je jasný den, není čas shánět dobytek. Napojte ovce a jděte pást.“
#29:8 Odpověděli: „Nemůžeme, dokud nebudou sehnána všechna stáda; pak odvalíme kámen z otvoru studně a napojíme ovce.“
#29:9 Když ještě s nimi rozmlouval, přišla Ráchel s ovcemi svého otce; byla totiž pastýřka.
#29:10 Jakmile Jákob uviděl Ráchel, dceru Lábana, bratra své matky, a jeho ovce, přistoupil, odvalil kámen z otvoru studně a napojil ovce Lábana, bratra své matky.
#29:11 Jákob pak Ráchel políbil a hlasitě zaplakal.
#29:12 Oznámil jí, že je synovec jejího otce a syn Rebeky. Běžela to povědět svému otci.
#29:13 Jakmile Lában uslyšel zprávu o Jákobovi, synu své sestry, běžel mu vstříc, objal ho, políbil a uvedl do svého domu. A on vypravoval Lábanovi všechno, co se přihodilo.
#29:14 Lában mu řekl: „Ty jsi má krev a mé tělo!“ Pobyl tedy u něho celý měsíc.
#29:15 Potom řekl Lában Jákobovi: „Což mi budeš sloužit zadarmo jen proto, že jsi můj příbuzný? Pověz mi, jaká má být tvá mzda.“
#29:16 Lában měl dvě dcery. Starší se jmenovala Lea, mladší Ráchel.
#29:17 Lea měla mírné oči, Ráchel byla krásné postavy, krásného vzezření.
#29:18 Jákob si Ráchel zamiloval; proto řekl: „Budu ti sloužit sedm let za tvou mladší dceru Ráchel.“
#29:19 Lában souhlasil: „Lépe, když ji dám tobě než někomu jinému; zůstaň u mne.“
#29:20 Jákob tedy sloužil za Ráchel sedm let; bylo to pro něho jako několik dní, protože ji miloval.
#29:21 Potom řekl Jákob Lábanovi: „Dej mi mou ženu, má lhůta už uplynula. Toužím po ní.“
#29:22 Lában shromáždil všechny muže toho místa a uspořádal hody.
#29:23 Večer vzal svou dceru Leu a uvedl ji k Jákobovi, a on k ní vešel.
#29:24 Za služebnici své dceři Leji dal Lában otrokyni Zilpu.
#29:25 Ráno Jákob viděl, že to je Lea. Vyčítal Lábanovi: „Cos mi to provedl? Což jsem u tebe nesloužil za Ráchel? Proč jsi mě oklamal?“
#29:26 Lában odvětil: „U nás není zvykem, aby se mladší vdávala dříve než prvorozená.
#29:27 Zůstaň u ní po celý svatební týden a dáme ti i tu mladší za službu, kterou si u mne odsloužíš v dalších sedmi letech.“
#29:28 Jákob tak učinil a zůstal u ní po celý týden. Pak mu Lában dal za ženu svou dceru Ráchel.
#29:29 Za služebnici dal své dceři Ráchel otrokyni Bilhu.
#29:30 I vešel Jákob také k Ráchel a miloval ji více než Leu, a sloužil u něho ještě dalších sedm let.
#29:31 Když Hospodin viděl, že Lea není milována, otevřel její lůno. Ráchel však zůstala neplodná.
#29:32 Lea otěhotněla, porodila syna a pojmenovala ho Rúben (to je Hleďte-syn); řekla totiž: „Hospodin viděl mé pokoření; nyní mě už bude můj muž milovat.“
#29:33 Otěhotněla znovu, porodila syna a řekla: „Hospodin uslyšel, že nejsem milována, a dal mi také tohoto.“ Pojmenovala ho tedy Šimeón (to je Vyslyš-Bůh)
#29:34 A znovu otěhotněla, porodila syna a řekla: „Tentokrát se už můj muž přidruží ke mně, poněvadž jsem mu porodila tři syny.“ Proto se jmenuje Lévi (to je Přidružitel).
#29:35 A znovu otěhotněla, porodila syna a řekla: „Zase mohu vzdávat chválu Hospodinu.“ Proto ho pojmenovala Juda (to je Ten, který vzdává chválu). A přestala rodit. 
#30:1 Když Ráchel viděla, že Jákobovi nerodí, žárlila na svou sestru a naléhala na Jákoba: „Dej mi syny! Nedáš-li, umřu.“
#30:2 Jákob vzplanul proti Ráchel hněvem a okřikl ji: „Což mohu za to, že Bůh odpírá plod tvému životu?“
#30:3 Odvětila: „Tu je má otrokyně Bilha; vejdi k ní! Porodí na má kolena, a tak i já budu mít z ní syny.“
#30:4 Dala mu tedy svou služku Bilhu za ženu a Jákob k ní vešel.
#30:5 Bilha otěhotněla a porodila Jákobovi syna.
#30:6 Tu řekla Ráchel: „Bůh mě obhájil, také můj hlas uslyšel a dal mi syna.“ Proto ho pojmenovala Dan (to je Obhájce).
#30:7 Ráchelina služka Bilha otěhotněla ještě jednou a porodila Jákobovi druhého syna.
#30:8 Ráchel opět řekla: „V úporném boji o Boží přízeň jsem zápasila se svou sestrou a obstála jsem.“ A pojmenovala ho Neftalí (to je Vybojovaný).
#30:9 Když Lea viděla, že přestala rodit, vzala svou služku Zilpu a dala ji Jákobovi za ženu.
#30:10 Také Lejina služka Zilpa porodila Jákobovi syna.
#30:11 Tu Lea řekla: „Jaké štěstí!“ A dala mu jméno Gád (to je Štěstí).
#30:12 Pak Lejina služka Zilpa porodila Jákobovi druhého syna.
#30:13 Lea opět řekla: „Jaké blaho pro mne; všechny dcery mě budou blahoslavit.“ A dala mu jméno Ašer (to je Blahoslav).
#30:14 Ve dnech, kdy se žala pšenice, vyšel Rúben a nalezl na poli jablíčka lásky a přinesl je své matce Leji. Ráchel však na Leu naléhala: „Dej mi prosím několik těch jablíček lásky od svého syna!“
#30:15 Ale ona ji odbyla: „Copak je to málo, žes mi vzala muže? Chceš mi vzít i jablíčka lásky od mého syna?“ Ráchel řekla: „Tak ať za ta jablíčka lásky od tvého syna spí tuto noc s tebou.“
#30:16 Když Jákob přicházel navečer z pole, vyšla mu Lea vstříc a řekla: „Musíš vejít ke mně, najala jsem tě za mzdu, za jablíčka lásky od svého syna.“ I spal té noci s ní.
#30:17 A Bůh Leu vyslyšel; otěhotněla a porodila Jákobovi pátého syna.
#30:18 Lea řekla: „Bůh mi dal mzdu za to, že jsem svému muži dala svou služku.“ A pojmenovala ho Isachar (to je Za-mzdu-najatý).
#30:19 Lea otěhotněla ještě jednou a porodila Jákobovi šestého syna.
#30:20 Lea opět řekla: „Jak pěkným darem obdaroval Bůh právě mne! Tentokrát už bude můj muž zůstávat se mnou. Porodila jsem mu šest synů!“ A dala mu jméno Zabulón (to je Zůstávající).
#30:21 Potom porodila dceru a dala jí jméno Dína.
#30:22 I rozpomenul se Bůh na Ráchel, vyslyšel ji a otevřel její lůno.
#30:23 Otěhotněla, porodila syna a řekla: „Bůh odňal mé pohanění.“
#30:24 Dala mu jméno Josef (to je Přidej-Bůh) a dodala: „Kéž mi Hospodin přidá ještě dalšího syna.“
#30:25 Když Ráchel porodila Josefa, řekl Jákob Lábanovi: „Propusť mě, abych mohl odejít do svého domova a do své země.
#30:26 Vydej mi mé ženy a děti, za něž jsem ti sloužil, a já půjdu. Sám přece víš, jakou službu jsem ti prokázal.“
#30:27 Lában mu odvětil: „Kéž bych získal tvoji přízeň! Zjistil jsem, že mi Hospodin kvůli tobě žehná.“
#30:28 Dále řekl: „Urči si přesně mzdu a dám ti ji.“
#30:29 Jákob mu odpověděl: „Sám přece víš, jak jsem ti sloužil a jak prospívalo tvé stádo, když jsem byl při něm.
#30:30 Vždyť to byla hrstka, kterou jsi měl, než jsem přišel, a rozmnožila se nesmírně. Hospodin ti žehnal na každém mém kroku. Ale kdy budu konečně pracovat také pro svou rodinu?“
#30:31 Lában se otázal: „Co ti mám dát?“ Jákob odvětil: „Nemusíš mi dávat vůbec nic. Budu dále pást a střežit tvé ovce a kozy, když pro mne uděláš tuto věc:
#30:32 Projdu dnes všechny tvé ovce a kozy a vyřadím z nich každé skvrnité a strakaté mládě, všechna načernalá jehňata a strakatá a skvrnitá kůzlata; to bude má mzda.
#30:33 Od zítřka se bude má poctivost osvědčovat takto: Když přijdeš přehlížet mou mzdu, všechno, co nebude mezi mými kůzlaty skvrnité a strakaté a mezi jehňaty načernalé, pokládej za kradené.“
#30:34 Lában řekl: „Nuže, ať se stane podle tvého slova.“
#30:35 A téhož dne sám vyřadil pruhované a strakaté kozly a všechny skvrnité a strakaté kozy, všechno, na čem bylo něco bílého, a všechna načernalá jehňata, a svěřil je svým synům.
#30:36 Stanovil také vzdálenost tří dnů cesty mezi sebou a Jákobem. A Jákob pásl ostatní Lábanovy ovce a kozy.
#30:37 Jákob si nabral čerstvé pruty topolové, mandloňové a platanové a sloupal z nich na některých místech pruhy kůry až do běla.
#30:38 Pruty, z nichž sloupal kůru, nakladl do napájecích žlabů, k nimž ovce a kozy přicházely pít, přímo před ně. I běhaly se, když přicházely pít.
#30:39 Ovce a kozy se běhaly při pohledu na pruty a vrhaly pruhovaná, skvrnitá a strakatá mláďata.
#30:40 Jákob jehňata odděloval; avšak ovce a kozy z bravu Lábanova obracel k pruhovaným a všem načernalým. Tak si pořídil vlastní stáda, ale ta nestavěl proti ovcím a kozám Lábanovým.
#30:41 Pokaždé, když se z ovcí a koz běhaly ty nejstatnější kusy, kladl před ně Jákob do žlabů pruty, aby se běhaly před pruty.
#30:42 Když však byly ovce a kozy neduživé, pruty nekladl. Neduživé tedy patřily Lábanovi a statné Jákobovi.
#30:43 Tak se ten muž převelice vzmohl a měl mnoho ovcí a koz, i služky a služebníky, i velbloudy a osly. 
#31:1 Jákob se doslechl, jaké řeči vedou Lábanovi synové: „Jákob pobral všechno, co patřilo našemu otci, a z toho, co měl otec, získal takové postavení.“
#31:2 Viděl i na Lábanovi, že se už k němu nemá jako dřív.
#31:3 Tu řekl Hospodin Jákobovi: „Navrať se do země svých otců a do svého rodiště; já budu s tebou.“
#31:4 Jákob si proto dal zavolat Ráchel a Leu na pole k svému stádu
#31:5 a řekl jim: „Vidím na vašem otci, že se ke mně nemá jako dřív. Ale Bůh mého otce je se mnou.
#31:6 Vy samy víte, že jsem vašemu otci sloužil ze všech sil.
#31:7 Ale váš otec mě obelstil, desetkrát změnil mou mzdu. Bůh mu však nedovolil, aby mi provedl něco zlého.
#31:8 Když říkal: ‚Tvou mzdou bude vše skvrnité‘, všechny ovce a kozy vrhaly mláďata skvrnitá. A když říkal: ‚Tvou mzdou bude vše pruhované‘, všechny ovce a kozy vrhaly mláďata pruhovaná.
#31:9 Tak odňal Bůh stáda vašemu otci a dal je mně.
#31:10 V době, kdy se ovce a kozy běhaly, rozhlédl jsem se a ve snu jsem náhle spatřil, že samci, skákající na ovce a kozy, byli pruhovaní, skvrnití a stříkaní.
#31:11 Ve snu mi tehdy Boží posel řekl: ‚Jákobe!‘ Odvětil jsem: ‚Tu jsem.‘
#31:12 I řekl mi: ‚Rozhlédni se a viz: Všichni samci, skákající na ovce a kozy, jsou pruhovaní, skvrnití a stříkaní. Viděl jsem totiž všechno, co ti Lában dělá.
#31:13 Já jsem Bůh z Bét-elu, kde jsi olejem pomazal posvátný sloup a kde ses mi zavázal slibem. Vyjdi teď hned z této země a navrať se do své rodné země.‘“
#31:14 Ráchel i Lea mu odpověděly: „Máme vůbec ještě v domě svého otce dědičný podíl?
#31:15 Cožpak pro něho nejsme jako cizí? Vždyť nás prodal a stříbro shrábl.
#31:16 Celé bohatství, které Bůh odňal našemu otci, patří nám a našim synům. Jen udělej všechno, co ti Bůh řekl.“
#31:17 Jákob tedy vstal, posadil své syny i ženy na velbloudy,
#31:18 sehnal také všechna svá stáda, vzal všechno nabyté jmění, vlastní stáda získaná v Rovinách aramských, a ubíral se ke svému otci Izákovi do země kenaanské.
#31:19 Když Lában odešel ke stříži svého bravu, ukradla Ráchel otcovy domácí bůžky.
#31:20 A Jákob Aramejce Lábana přelstil, takže nevyšlo najevo, že chce uprchnout.
#31:21 Uprchl se vším, co měl. Přebrodil řeku Eufrat a dal se směrem k pohoří Gileádu.
#31:22 Třetího dne bylo Lábanovi oznámeno, že Jákob uprchl.
#31:23 Tu vzal s sebou své bratry a pronásledoval ho po sedm dní, až ho dohonil na pohoří Gileádu.
#31:24 K Aramejci Lábanovi však přišel v noci ve snu Bůh a řekl mu: „Měj se na pozoru! Mluv s Jákobem jen v dobrém, ne ve zlém!“
#31:25 Lában dostihl Jákoba, který si na té hoře postavil stan. Sám se svými bratry si rovněž postavil stan na pohoří Gileádu.
#31:26 A Lában Jákobovi vyčítal: „Cos to udělal? Přelstils mě. Mé dcery jsi odvedl jako válečné zajatce.
#31:27 Proč jsi ode mne uprchl tajně jako zloděj a nic jsi mi neoznámil? Byl bych tě radostně vyprovodil s písničkami, s bubínkem a citarou.
#31:28 Nedopřáls mi ani políbit mé vnuky a dcery; jednal jsi věru jako pomatenec.
#31:29 Bylo v mé moci naložit s vámi zle. Ale Bůh vašeho otce mi na dnešek řekl: ‚Měj se na pozoru, mluv s Jákobem jen v dobrém, ne ve zlém.‘
#31:30 Když už jsi tedy odešel, protože se ti tolik stýskalo po otcovském domě, proč jsi ukradl mé bohy?“
#31:31 Na to Jákob Lábanovi odpověděl: „Bál jsem se a říkal jsem si, že bys mě mohl o své dcery připravit.
#31:32 Ale u koho najdeš své bohy, ten nezůstane naživu! Před našimi bratry si prohlédni všechno, co mám, a vezmi si své.“ Jákob totiž nevěděl, že je Ráchel ukradla.
#31:33 Lában vešel do stanu Jákobova a do stanu Lejina i do stanu obou služek, ale nic nenašel. Vyšel tedy ze stanu Lejina a vešel do stanu Ráchelina.
#31:34 Ráchel však bůžky vzala, vložila je do torby na velbloudím sedle a sedla si na ně. Lában zpřeházel celý stan, ale nic nenašel.
#31:35 Ráchel otci řekla: „Nechť se můj pán nehněvá, že před ním nemohu povstat; stalo se mi, co se stává ženám.“ Lában slídil, ale bůžky nenašel.
#31:36 Tu se Jákob rozhněval a začal se s Lábanem přít. Vyčítal mu: „Jaký je můj přestupek? Jaký je můj hřích, že ses za mnou tak hnal?
#31:37 Když jsi zpřeházel všechny mé věci, co jsi našel z věcí svého domu? Polož to před mé i své bratry, ať mezi námi dvěma rozhodnou!
#31:38 Byl jsem u tebe celých dvacet let. Tvé ovce a kozy nezmetaly. Berany z tvého bravu jsem nejídal.
#31:39 Co bylo rozsápáno zvěří, jsem ti nevykazoval, nahrazoval jsem to ze svého; vymáhals to na mně, byl jsem okrádán ve dne i v noci.
#31:40 Ve dne mě sužovalo vedro a v noci chlad. Spánek prchal od mých očí.
#31:41 Celých dvacet let jsem ti v tvém domě sloužil: čtrnáct let za tvé dvě dcery a šest let za tvůj brav. Mou mzdu jsi změnil desetkrát.
#31:42 Kdyby se mnou nebyl Bůh mého otce, Bůh Abrahamův a Strach Izákův, propustil bys mě teď s prázdnou. Bůh viděl mou trýzeň a námahu mých rukou a na dnešek sám rozsoudil.“
#31:43 Lában na to Jákobovi odpověděl: „Toto jsou mé dcery a toto jsou moji synové. I brav je můj, ano, všechno, co vidíš, patří mně. Co však za těchto okolností mohu dnes udělat pro své dcery, pro ně a pro syny, které porodily?
#31:44 Nuže, uzavřeme teď spolu smlouvu, a Bůh ať je svědkem mezi mnou a tebou!“
#31:45 Jákob vzal tedy kámen a vztyčil jej jako posvátný sloup.
#31:46 Svým bratřím řekl: „Nasbírejte kameny.“ Vzali kameny, udělali val a na tom valu pojedli.
#31:47 Lában jej nazval Jegar-sahaduta (to je Násep svědectví) a Jákob jej nazval Gal-ed (to je Val-svědek).
#31:48 A Lában řekl: „Tento val je ode dneška svědkem mezi mnou a tebou.“ - Proto se jmenuje Gal-ed
#31:49 nebo Mispa (to je Hlídka), neboť Jákob řekl: „Hospodin ať je na hlídce mezi mnou a tebou, že už spolu nebudeme nic mít.“ -
#31:50 Na to Lában: „Pokoříš-li mé dcery a vezmeš-li si jiné ženy mimo ně, hleď, ne někdo z lidí, ale Bůh je svědkem mezi mnou a tebou!“
#31:51 Lában dále Jákobovi řekl: „Hle, tu je val a tu je posvátný sloup, který jsem vztyčil mezi sebou a tebou.
#31:52 Svědkem je tento val, svědkem je i posvátný sloup, že já nepřekročím tento val proti tobě a ty nepřekročíš tento val a tento sloup proti mně se zlým úmyslem.
#31:53 Ať mezi námi soudí Bůh Abrahamův a bůh Náchorův, bůh jejich otce!“ A Jákob se zapřisáhl při Strachu svého otce Izáka.
#31:54 Pak Jákob připravil na té hoře obětní hod. Povolal své bratry, aby pojedli chléb. Jedli tedy chléb a přenocovali na té hoře. 
#32:1 Za časného jitra políbil Lában své vnuky a dcery, požehnal jim a vracel se opět ke svému domovu.
#32:2 Jákob šel svou cestou. Tu se s ním srazili Boží poslové.
#32:3 Jakmile je Jákob spatřil, zvolal: „To je tábor Boží“, a to místo pojmenoval Machanajim (to je Tábořiště).
#32:4 Pak Jákob vyslal napřed posly ke svému bratru Ezauovi do země Seíru, na pole Edómské,
#32:5 a přikázal jim: „Vyřiďte mému pánu Ezauovi toto: Tvůj otrok Jákob vzkazuje: Až dosud jsem prodléval jako host u Lábana.
#32:6 Mám voly a osly, ovce, služebníky a služky. Posílám o tom zprávu tobě, svému pánu, abych získal tvoji přízeň.“
#32:7 Poslové se vrátili k Jákobovi a řekli: „Přišli jsme k tvému bratru Ezauovi, ale on ti už jde vstříc a je s ním čtyři sta mužů.“
#32:8 Tu padla na Jákoba veliká bázeň a tíseň. Rozdělil proto lid, který byl s ním, i brav a skot a velbloudy do dvou táborů,
#32:9 neboť si řekl: „Přijde-li Ezau k prvnímu táboru a pobije jej, může ještě zbylý tábor vyváznout.“
#32:10 Dále Jákob řekl: „Bože mého otce Abrahama a Bože mého otce Izáka, Hospodine, tys mi pravil: ‚Navrať se do své země a do svého rodiště a já se postarám, aby se ti dobře vedlo.‘
#32:11 Nejsem hoden veškerého tvého milosrdenství a vší tvé věrnosti, které jsi svému služebníku prokázal. Tento Jordán jsem překročil s holí, a teď mám dva tábory.
#32:12 Vytrhni mě prosím z ruky mého bratra, z ruky Ezauovy, neboť se ho bojím, aby nepřišel a nezabil mě, matku nad dětmi.
#32:13 Ty jsi přece řekl: ‚Určitě se postarám o tvé dobro a tvé potomstvo rozmnožím jako mořský písek, jejž nelze pro množství sečíst.‘“
#32:14 A přenocoval tam té noci. Pak vzal z toho, co vyzískal, dar na usmířenou pro svého bratra Ezaua:
#32:15 dvě stě koz a dvacet kozlů, dvě stě bahnic a dvacet beranů,
#32:16 třicet velbloudic se sajícími mláďaty, čtyřicet krav a deset býků, dvacet oslic a deset oslů.
#32:17 To vše předal svým služebníkům, každé stádečko zvlášť, a řekl jim: „Jděte napřed a ponechte mezi jednotlivými stádečky odstup.“
#32:18 A prvnímu přikázal: „Až se s tebou setká můj bratr Ezau a zeptá se tě čí jsi a kam jdeš a čí je to, co ženeš před sebou,
#32:19 odvětíš: ‚Je to dar od tvého otroka Jákoba, poslaný jeho pánu Ezauovi; on sám je za námi.‘“
#32:20 Tak přikázal i druhému a třetímu a všem, kteří šli za stádečky. Řekl: „V tomto smyslu mluvte s Ezauem, až na něj narazíte.
#32:21 A dodejte: ‚Také tvůj otrok Jákob je za námi, neboť řekl: Darem, který jde přede mnou, chci ho usmířit a teprve potom spatřit jeho tvář. Snad mě přijme milostivě.‘“
#32:22 Šli tedy napřed s darem, zatímco on přenocoval oné noci v táboře.
#32:23 A té noci vstal, vzal obě své ženy i obě své služky a jedenáct svých synů a přebrodil se přes Jabok.
#32:24 Vzal je a převedl je přes potok se vším, co měl.
#32:25 Pak zůstal Jákob sám a tu s ním kdosi zápolil, dokud nevzešla jitřenka.
#32:26 Když viděl, že Jákoba nepřemůže, poranil mu při zápolení kyčelní kloub, takže se mu vykloubil.
#32:27 Neznámý řekl: „Pusť mě, vzešla jitřenka.“ Jákob však odvětil: „Nepustím tě, dokud mi nepožehnáš.“
#32:28 Otázal se: „Jak se jmenuješ?“ Odpověděl: „Jákob.“
#32:29 Tu řekl: „Nebudou tě už jmenovat Jákob (to je Úskočný), nýbrž Izrael (to je Zápasí Bůh), neboť jsi jako kníže zápasil s Bohem i s lidmi a obstáls.“
#32:30 A Jákob ho žádal: „Pověz mi přece své jméno!“ Ale on odvětil: „Proč se ptáš na mé jméno?“ A požehnal mu tam.
#32:31 I pojmenoval Jákob to místo Peníel (to je Tvář Boží), neboť řekl: „Viděl jsem Boha tváří v tvář a byl mi zachován život.“
#32:32 Slunce mu vzešlo, když minul Penúel, ale v kyčli byl chromý.
#32:33 Synové Izraelovi nejedí až podnes šlachu při kyčelním kloubu, protože Bůh poranil Jákobovi šlachu kyčelního kloubu. 
#33:1 Potom se Jákob rozhlédl a vidí, že přichází Ezau a s ním čtyři sta mužů. I rozdělil zvlášť děti Lejiny a Rácheliny a obou otrokyň.
#33:2 Dopředu postavil otrokyně a jejich děti, za ně Leu s jejími dětmi a Ráchel s Josefem dozadu.
#33:3 Sám se ubíral před nimi a sedmkrát se poklonil až k zemi, než k svému bratrovi přistoupil.
#33:4 Ezau se k němu rozběhl a objal ho, padl mu kolem krku a políbil ho; oba zaplakali.
#33:5 Pak se Ezau rozhlédl a spatřil ženy a děti. Tázal se: „Koho to máš s sebou?“ Jákob odvětil: „To jsou děti, jimiž Bůh milostivě obdaroval tvého otroka.“
#33:6 Mezitím přistoupily otrokyně se svými dětmi a poklonily se.
#33:7 Pak přistoupila i Lea a její děti a poklonily se. Naposled přistoupil Josef a Ráchel a poklonili se.
#33:8 Ezau se otázal: „K čemu je celý ten tábor, s kterým jsem se setkal?“ Odvětil: „Abych získal přízeň svého pána.“
#33:9 Ezau řekl: „Mám dost, bratře. Ponech si, co máš.“
#33:10 Ale Jákob naléhal: „Jestliže jsem získal tvoji přízeň, přijmi prosím ode mne ten dar, vždyť smím vidět tvou tvář, a to je jako bych viděl tvář Boží. Tak přívětivě jsi mě přijal!
#33:11 Přijmi prosím z mého požehnání, co jsem ti přinesl, neboť Bůh se nade mnou smiloval a mám všeho dost.“ Tak ho nutil, až Ezau přijal.
#33:12 A navrhl: „Vydejme se na cestu, půjdu s tebou.“
#33:13 Ale on mu odvětil: „Můj pán ví, že děti jsou útlé a že mám s sebou březí ovce a krávy. Budou-li hnány po celý den, všechny ovce uhynou.
#33:14 Nechť se prosím můj pán ubírá před svým otrokem a já potáhnu pomalu, jak může jít stádo, které je přede mnou, a jak mohou děti; pak přijdu k svému pánu do Seíru.“
#33:15 Ezau na to řekl: „Dovol, abych ti tu ponechal několik svých lidí.“ On však odvětil: „K čemu to? Jen když jsem získal přízeň svého pána.“
#33:16 A tak se Ezau toho dne vrátil svou cestou do Seíru,
#33:17 kdežto Jákob vytáhl do Sukótu a vystavěl dům a pro svůj dobytek udělal přístřešky. Proto pojmenoval to místo Sukót (to je Přístřešky).
#33:18 Potom přišel Jákob cestou z Rovin aramských pokojně k městu Šekemu, které je v zemi kenaanské. Utábořil se před městem
#33:19 a od synů Chamóra, otce Šekemova, koupil za sto kesít díl pole, na němž si postavil stan.
#33:20 I zřídil tam oltář a nazval jej Él je Bůh Izraelův. 
#34:1 Dína, kterou Jákobovi porodila Lea, si vyšla, aby se podívala na dcery té země.
#34:2 Uviděl ji Šekem, syn Chivejce Chamóra, knížete země, vzal ji a ležel s ní, a tak ji ponížil.
#34:3 Přilnul však k Díně, dceři Jákobově, celou duší, zamiloval si tu dívku a vemlouval se do jejího srdce.
#34:4 Svému otci Chamórovi pak Šekem řekl: „Vezmi mně to děvčátko za ženu.“
#34:5 Když Jákob uslyšel, že Šekem jeho dceru Dínu poskvrnil, byli jeho synové na poli s dobytkem. Jákob mlčel, dokud nepřišli.
#34:6 Šekemův otec Chamór pak vyšel k Jákobovi, aby s ním promluvil.
#34:7 Jákobovi synové přišli z pole, jakmile o tom uslyšeli. Bolestně to ty muže ranilo a velmi se rozlítili. Vždyť spáchal v Izraeli hanebnost, ležel s dcerou Jákobovou, a to je nepřípustné.
#34:8 Chamór s nimi mluvil takto: „Můj syn Šekem lpí na vaší dceři celou duší. Dejte mu ji prosím za ženu.
#34:9 Spřízněte se s námi, dávejte své dcery nám a berte si dcery naše.
#34:10 Můžete sídlit s námi, je před vámi celá země. Usaďte se, volně v ní obchodujte a mějte ji jako vlastní.“
#34:11 Také Šekem jejímu otci a jejím bratrům řekl: „Kéž bych získal vaši přízeň. Dám vám, oč mě požádáte.
#34:12 Určete mi jakkoli veliké věno i dar. Dám vám, oč mě požádáte, jen mi tu dívku dejte za ženu!“
#34:13 Avšak Jákobovi synové odpověděli Šekemovi a jeho otci Chamórovi lstivě. Jednali tak proto, že jejich sestru Dínu poskvrnil.
#34:14 Řekli jim: „To my nemůžeme udělat, abychom dali svou sestru muži neobřezanému, byla by to pro nás potupa.
#34:15 Svolíme jen tehdy, budete-li jako my: dá-li se u vás každý mužského pohlaví obřezat.
#34:16 Pak vám budeme dávat své dcery a budeme si brát dcery vaše, usídlíme se u vás a staneme se jedním lidem.
#34:17 Jestliže nás neuposlechnete a nedáte se obřezat, vezmeme svou dceru a odejdeme.“
#34:18 Chamórovi i jeho synu Šekemovi se jejich slova líbila.
#34:19 Mládenec nemeškal a vykonal to, neboť si Jákobovu dceru oblíbil. V otcově domě byl ze všech nejváženější.
#34:20 Chamór a jeho syn Šekem přišli k bráně svého města a promluvili k mužům svého města:
#34:21 „Tito muži se k nám chovají pokojně. Ať se tedy usídlí v zemi a volně v ní obchodují. Ať je pro ně tato země na všechny strany otevřená. My si budeme brát za ženy jejich dcery a budeme jim dávat dcery své.
#34:22 Avšak aby s námi bydlili jako jeden lid, svolí ti muži jen tehdy, dá-li se u nás každý mužského pohlaví obřezat, jako jsou obřezáni oni.
#34:23 Nebudou tak jejich stáda a majetek i všechen jejich dobytek patřit nám? Buďme jen svolni a oni se u nás usídlí.“
#34:24 Poslechli tedy Chamóra a jeho syna Šekema všichni, kdo měli právo vycházet k bráně jeho města; dal se obřezat každý mužského pohlaví, kdo vycházel k bráně jeho města.
#34:25 Ale třetího dne, když byli v bolestech, vtrhli bezpečně do města s mečem v ruce dva Jákobovi synové, Šimeón a Lévi, bratři Díny, a všechny mužského pohlaví povraždili.
#34:26 Zavraždili mečem také Chamóra a jeho syna Šekema, vzali z Šekemova domu Dínu a odešli.
#34:27 Na pobité potom přišli Jákobovi synové a za poskvrnění své sestry město vyloupili.
#34:28 Pobrali jejich brav a skot i jejich osly a co bylo v městě i na poli.
#34:29 Zajali všechny jejich děti a jejich ženy a uloupili všechno jejich jmění, vše, co bylo v domě.
#34:30 I řekl Jákob Šimeónovi a Lévimu: „Přivedete mě do zkázy, způsobili jste, že vzbuzuji nelibost u obyvatel země, u Kenaanců a Perizejců. Mám málo lidí. Seberou-li se proti mně, pobijí mě a budu vyhuben já i můj dům.“
#34:31 Odvětili: „Což směl s naší sestrou zacházet jako s děvkou?“ 
#35:1 I řekl Bůh Jákobovi: „Vstaň a vystup do Bét-elu, usaď se tam a udělej tam oltář Bohu, který se ti ukázal, když jsi prchal před svým bratrem Ezauem.“
#35:2 Jákob tedy řekl svému domu i všem, kteří byli s ním: „Zbavte se cizích bohů, které máte mezi sebou! Očisťte se, převlékněte si šat,
#35:3 budeme putovat do Bét-elu. Chci tam udělat oltář Bohu, který mi odpověděl v den mého soužení a byl se mnou na cestě, kterou jsem šel.“
#35:4 Odevzdali tedy Jákobovi všechny cizí bůžky, které u sebe měli, i všechny náušnice a Jákob je zakopal pod posvátným stromem u Šekemu.
#35:5 Potom táhli dál. Na okolní města padl děs Boží; proto Jákobovy syny nepronásledovali.
#35:6 Tak přišel Jákob i všechen lid, který byl s ním, do Lúzu, to je do Bét-elu v zemi kenaanské.
#35:7 Tam vybudoval oltář a vzýval na tom místě Boha Bét-elu, neboť se mu tam zjevil sám Bůh, když prchal před svým bratrem.
#35:8 Zde zemřela Rebečina chůva Debora a byla pochována dole u Bét-elu pod posvátným dubem, který pojmenoval Posvátný dub pláče.
#35:9 I ukázal se Bůh znovu Jákobovi, když přišel z Rovin aramských, a požehnal mu
#35:10 slovy: „Tvé jméno bylo Jákob; už nebudeš zván Jákob, tvé jméno bude Izrael.“ A dal mu jméno Izrael.
#35:11 Dále mu Bůh řekl: „Já jsem Bůh všemohoucí. Ploď a množ se; vzejde z tebe národ a společenství pronárodů, i králové vzejdou z tvých beder.
#35:12 Zemi, kterou jsem dal Abrahamovi a Izákovi, tu dám tobě; tvému potomstvu dám tuto zemi.“
#35:13 Potom Bůh vystoupil od něho z toho místa, kde s ním mluvil.
#35:14 A Jákob na tom místě, kde s ním mluvil, postavil posvátný sloup, sloup kamenný, a vykonal na něm úlitbu: polil jej svrchu olejem.
#35:15 Místu, kde s ním mluvil Bůh, dal Jákob jméno Bét-el.
#35:16 Potom z Bét-elu odtáhli. A když už byli nedaleko Efraty, Ráchel porodila; měla však těžký porod.
#35:17 Když těžce rodila, pravila jí porodní bába: „Neboj se, máš zase syna!“
#35:18 Ve chvíli, kdy umírala a život z ní unikal, pojmenovala ho Ben-óni (to je Syn mého zmaru), ale otec ho nazval Ben-jamín (to je Syn zdaru).
#35:19 Ráchel umřela a byla pohřbena u cesty do Efraty, což je Betlém.
#35:20 Jákob nad jejím hrobem postavil pamětní sloup, a to je památník Ráchelina hrobu až dodnes.
#35:21 Potom táhl Izrael dál a postavil svůj stan opodál Migdal-ederu.
#35:22 Když Izrael přebýval v oné zemi, šel Rúben a ležel s ženinou svého otce Bilhou. Izrael se o tom doslechl... Synů Jákobových bylo dvanáct.
#35:23 Synové Lejini: Rúben, Jákobův prvorozený, Šimeón a Lévi, Juda, Isachar a Zabulón.
#35:24 Synové Ráchelini: Josef a Benjamín.
#35:25 Synové Rácheliny otrokyně Bilhy: Dan a Neftalí.
#35:26 A synové Lejiny otrokyně Zilpy: Gád a Ašer. To jsou synové Jákobovi, kteří se mu narodili v Rovinách aramských.
#35:27 Jákob přišel k svému otci Izákovi do Mamre, do Kirjat-arby, což je Chebrón, kde Abraham a Izák pobývali jako hosté.
#35:28 Izák se dožil sto osmdesáti let.
#35:29 I zesnul Izák a zemřel a byl připojen k svému lidu, stár a sytý dnů. Jeho synové Ezau a Jákob ho pochovali. 
#36:1 Toto je rodopis Ezauův, to je Edómův.
#36:2 Ezau si vzal ženy z dcer kenaanských: Ádu, dceru Chetejce Elóna, a Oholíbamu, dceru Anovu, vnučku Chivejce Sibeóna,
#36:3 dále Basematu, dceru Izmaelovu, sestru Nebajótovu.
#36:4 Áda porodila Ezauovi Elífaza a Basemat porodila Reúela.
#36:5 Oholíbama porodila Jeúše, Jaelama a Kóracha. To jsou synové Ezauovi, kteří se mu narodili v kenaanské zemi.
#36:6 Ezau pak vzal své ženy, syny a dcery i všechny lidi svého domu, stádo a všechen dobytek i všechen majetek, jehož v kenaanské zemi nabyl, a odešel od svého bratra Jákoba pryč do seírské země.
#36:7 Jmění, jehož nabyli, bylo totiž tak značné, že nemohli sídlit pospolu, a země, v níž pobývali jako hosté, jim nemohla pro jejich stáda stačit.
#36:8 Proto se Ezau usadil v Seírském pohoří. Ezau, to je Edóm.
#36:9 To je tedy rodopis Ezaua, praotce Edómu, v Seírském pohoří.
#36:10 Toto jsou jména synů Ezauových: Elífaz, syn Ezauovy ženy Ády, Reúel, syn Ezauovy ženy Basematy.
#36:11 Synové Elífazovi jsou: Téman, Ómar, Sefó, Gátam a Kenaz.
#36:12 Ženina Ezauova syna Elífaza byla Timna a ta porodila Elífazovi Amáleka. To jsou vnuci Ezauovy ženy Ády.
#36:13 A toto jsou synové Reúelovi: Nachat a Zerach, Šama a Miza. To jsou vnuci Ezauovy ženy Basematy.
#36:14 A toto jsou synové Ezauovy ženy Oholíbamy, dcery Anovy, vnučky Sibeónovy: porodila Ezauovi Jeúše, Jaelama a Kóracha.
#36:15 Toto jsou pohlaváři synů Ezauových. Synové Ezauova prvorozence Elífaza: pohlavár Téman, pohlavár Ómar, pohlavár Sefó, pohlavár Kenaz,
#36:16 pohlavár Kórach, pohlavár Gátam, pohlavár Amálek. To jsou elífazovští pohlaváři v edómské zemi, vnuci Ádini.
#36:17 A toto jsou synové Ezauova syna Reúela: pohlavár Nachat, pohlavár Zerach, pohlavár Šama, pohlavár Miza. To jsou reúelovští pohlaváři v edómské zemi, vnuci Ezauovy ženy Basematy.
#36:18 A toto jsou synové Ezauovy ženy Oholíbamy: pohlavár Jeúš, pohlavár Jaelam, pohlavár Kórach. To jsou pohlaváři Ezauovy ženy Oholíbamy, dcery Anovy.
#36:19 To jsou synové Ezauovi a to jsou jejich pohlaváři. To je Edóm.
#36:20 Toto jsou synové Chorejce Seíra, praobyvatelé této země: Lótan a Šóbal, Sibeón a Ana,
#36:21 Dišón, Eser a Díšan. To jsou chorejští pohlaváři, Seírovi synové v edómské zemi.
#36:22 Lótanovi synové jsou Chorí a Hémam; Lótanova sestra je Timna.
#36:23 A toto jsou synové Šóbalovi: Alván, Manachat a Ébal, Šefó a Ónam.
#36:24 Toto jsou synové Sibeónovi: Aja a Ana. To byl ten Ana, který našel ve stepi horké prameny, když pásl osly svého otce Sibeóna.
#36:25 A toto jsou Anovy děti: Dišón a Oholíbama, dcera Anova.
#36:26 A toto jsou synové Díšanovi: Chemdán a Ešbán, Jitrán a Keran.
#36:27 Toto jsou synové Eserovi: Bilhán, Zaavan a Akan.
#36:28 Toto jsou synové Díšanovi: Ús a Aran.
#36:29 Toto jsou chorejští pohlaváři: pohlavár Lótan, pohlavár Šóbal, pohlavár Sibeón, pohlavár Ana,
#36:30 pohlavár Dišón, pohlavár Eser, pohlavár Díšan. To jsou chorejští pohlaváři podle seznamu pohlavárů v seírské zemi.
#36:31 A toto jsou králové, kteří kralovali v edómské zemi, dříve než kraloval král synům izraelským.
#36:32 V Edómu kraloval Bela, syn Beórův, a jeho město se jmenovalo Dinhaba.
#36:33 Když Bela zemřel, stal se po něm králem Zerachův syn Jóbab z Bosry.
#36:34 Když zemřel Jóbab, stal se po něm králem Chušam z témanské země.
#36:35 Když zemřel Chušam, stal se po něm králem Bedadův syn Hadad, který porazil Midjána na Moábském poli. Jeho město se jmenovalo Avít.
#36:36 Když zemřel Hadad, stal se po něm králem Samla z Masreky.
#36:37 Když zemřel Samla, stal se po něm králem Šaul z Rechobótu nad Řekou.
#36:38 Když zemřel Šaul, stal se po něm králem Akbórův syn Baal-chanan.
#36:39 Když zemřel Akbórův syn Baal-chanan, stal se po něm králem Hadar. Jeho město se jmenovalo Paú a jméno jeho ženy bylo Mehetabel; byla to dcera Matredy, vnučka Mé-zahabova.
#36:40 Toto jsou tedy jména ezauovských pohlavárů podle jejich čeledí a míst, jmenovitě: pohlavár Timna, pohlavár Alva, pohlavár Jetet,
#36:41 pohlavár Oholíbama, pohlavár Ela, pohlavár Pínon,
#36:42 pohlavár Kenaz, pohlavár Téman, pohlavár Mibsár,
#36:43 pohlavár Magdíel, pohlavár Iram. To jsou edómští pohlaváři podle sídlišť v zemi jejich vlastnictví. To je Ezau, praotec Edómu. 
#37:1 I usadil se Jákob v zemi, v níž jeho otec pobýval jako host, v zemi kenaanské.
#37:2 Toto je rodopis Jákobův. Sedmnáctiletý Josef pásal se svými bratry ovce. Byl to mládenec, který býval se syny žen svého otce, Bilhy a Zilpy. Josef přinášel svému otci o svých bratrech zlé zprávy.
#37:3 Izrael Josefa miloval ze všech svých synů nejvíce; vždyť to byl syn jeho stáří. Proto mu udělal pestře tkanou suknici.
#37:4 Když bratři viděli, že ho otec miluje nade všechny bratry, začali ho nenávidět a nepromluvili na něho pokojného slova.
#37:5 Jednou měl Josef sen a pověděl jej svým bratrům; nenáviděli ho pak ještě více.
#37:6 Řekl jim totiž: „Slyšte prosím, jaký jsem měl sen:
#37:7 Vážeme na poli snopy. Tu povstane můj snop a zůstane stát. A hle, vaše snopy obcházely kolem něho a klaněly se mému snopu.“
#37:8 Bratři mu odpověděli: „To budeš nad námi kralovat jako král či mezi námi vládnout jako vladař?“ A nenáviděli ho pro jeho sny a pro jeho slova ještě víc.
#37:9 Měl pak ještě jiný sen a vypravoval jej svým bratrům: „Měl jsem opět sen: Klanělo se mi slunce, měsíc a jedenáct hvězd.“
#37:10 To vyprávěl otci a bratrům. Otec ho okřikl: „Jaký žes to měl sen? Že i já, tvá matka a tvoji bratři přijdeme, abychom se před tebou skláněli k zemi?“
#37:11 Bratři na něho žárlili, ale otec na to nepřestával myslet.
#37:12 Bratři pak odešli, aby pásli ovce svého otce v Šekemu.
#37:13 Tu Izrael řekl Josefovi: „Zdalipak nepasou tvoji bratři v Šekemu? Pojď, rád bych tě za nimi poslal.“ On mu odvětil: „Tu jsem.“
#37:14 Izrael mu řekl: „Jdi a podívej se, je-li s tvými bratry a s ovcemi vše v pořádku, a podej mi zprávu.“ Poslal ho tedy z chebrónské doliny a on přišel do Šekemu.
#37:15 Tu ho nalezl nějaký muž, jak bloudí po poli, a zeptal se ho: „Co hledáš?“
#37:16 Odvětil: „Hledám své bratry. Pověz mi prosím, kde pasou.“
#37:17 Muž mu řekl: „Odtáhli odtud. Slyšel jsem, jak říkají: ‚Pojďme do Dótanu.‘“ Josef tedy šel za svými bratry a nalezl je v Dótanu.
#37:18 Jakmile ho v dálce spatřili, ještě než se k nim přiblížil, smluvili se proti němu, že ho usmrtí.
#37:19 Řekli si mezi sebou: „Hle, mistr snů sem přichází!
#37:20 Pojďte, zabijme ho! Pak ho vhodíme do některé cisterny a řekneme: Sežrala ho divá zvěř. A uvidíme, co bude z jeho snů!“
#37:21 Když to uslyšel Rúben, rozhodl se vysvobodit ho z jejich rukou. Zvolal: „Přece ho nebudeme ubíjet!“
#37:22 Dále jim Rúben řekl: „Neprolévejte krev. Vhoďte ho do cisterny, která je ve stepi, ale ruku na něj nevztahujte!“ Chtěl ho z rukou bratrů vysvobodit a přivést k otci.
#37:23 Jakmile Josef přišel k bratrům, strhli z něho suknici, tu suknici pestře tkanou, kterou měl na sobě.
#37:24 Vzali ho a hodili do cisterny. Cisterna byla prázdná, bez vody.
#37:25 Pak se posadili, aby jedli chléb. Tu se rozhlédli a spatřili, jak od Gileádu přichází karavana Izmaelců; jejich velbloudi nesli ladanum, mastix a masti. Táhli s tím dolů do Egypta.
#37:26 Juda řekl bratrům: „Čeho tím dosáhneme, když svého bratra zabijeme a jeho krev zatajíme?
#37:27 Pojďte, prodejme ho Izmaelcům, ale sami na něho nesahejme; vždyť je to náš rodný bratr.“ Bratři ho uposlechli.
#37:28 Když midjánští obchodníci jeli kolem, vytáhli Josefa z cisterny a prodali ho Izmaelcům za dvacet šekelů stříbra. Ti přivedli Josefa do Egypta.
#37:29 Když se Rúben vrátil k cisterně, vidí, že tam Josef není. Roztrhl své roucho,
#37:30 vrátil se k bratrům a naříkal: „Ten hoch tam není. Co si jen, co si jen počnu?“
#37:31 Tu vzali bratři Josefovu suknici, porazili kozla a suknici namočili v krvi.
#37:32 Tu pestře tkanou suknici pak dali donést otci se vzkazem: „Tohle jsme nalezli. Pozorně si to prosím prohlédni: Je to suknice tvého syna, nebo není?“
#37:33 Když si ji prohlédl, zvolal: „Suknice mého syna! Sežrala ho divá zvěř! Rozsápán, rozsápán je Josef!“
#37:34 I roztrhl Jákob svůj šat, přes bedra přehodil žíněné roucho a truchlil pro syna mnoho dní.
#37:35 Přišli všichni jeho synové a všechny jeho dcery, aby ho potěšili, ale on se potěšit nedal. Naříkal: „Ve smutku sestoupím za synem do podsvětí.“ Tak oplakával otec Josefa.
#37:36 A Medanci ho prodali do Egypta faraónovu dvořanu Potífarovi, veliteli tělesné stráže. 
#38:1 Stalo se pak v té době, že Juda odešel od svých bratrů a připojil se k adulamskému muži jménem Chíra.
#38:2 Tam spatřil dceru kenaanského muže, který se jmenoval Šúa. Vzal si ji za ženu a vešel k ní.
#38:3 Otěhotněla, porodila syna a Juda mu dal jméno Er.
#38:4 Znovu otěhotněla, porodila syna a dala mu jméno Ónan.
#38:5 Nato porodila ještě dalšího syna a dala mu jméno Šela. Bylo to v Kezíbu, kde jej porodila.
#38:6 Juda dal Erovi, svému prvorozenému, manželku jménem Támar.
#38:7 Judův prvorozený Er však byl v očích Hospodinových zlý, a proto jej Hospodin usmrtil.
#38:8 Juda tedy řekl Ónanovi: „Vejdi k bratrově ženě, vezmi si ji podle švagrovského práva a postarej se tak svému bratru o potomstvo.“
#38:9 Ale Ónan věděl, že to potomstvo nebude patřit jemu; proto kdykoli vcházel k ženě svého bratra, vypouštěl semeno na zem, aby svému bratru nezplodil potomka.
#38:10 Jeho počínání bylo v očích Hospodinových zlé, proto usmrtil i jeho.
#38:11 Tehdy řekl Juda své snaše Támaře: „Usaď se v domě svého otce jako vdova, dokud nedospěje můj syn Šela.“ Říkal si však: „Jen aby také on nezemřel jako jeho bratři!“ Támar tedy odešla a usadila se v otcovském domě.
#38:12 Uplynulo mnoho dní. Judova manželka, dcera Šúova, zemřela. Když se Juda utěšil, putoval se svým adulamským přítelem Chírou za střihači svých ovcí do Timny.
#38:13 Oznámili Támaře: „Tvůj tchán putuje do Timny ke stříži svých ovcí.“
#38:14 Tu odložila vdovské šaty, zastřela se rouškou, zahalila se a posadila se při vstupu do Énajimu, který je u cesty do Timny. Viděla totiž, že nebyla dána Šelovi za ženu, ačkoli už dospěl.
#38:15 Když ji Juda uviděl, považoval ji za nevěstku, protože si zastřela tvář.
#38:16 Obrátil se k ní, sedící při cestě, a řekl: „Dovol prosím, abych k tobě vešel.“ Nevěděl totiž, že je to jeho snacha. Otázala se: „Co mi dáš za to, že ke mně vejdeš?“
#38:17 Odvětil: „Pošlu ti kůzle ze stáda.“ Řekla: „Ale dáš mi zástavu, než je pošleš.“
#38:18 Otázal se: „Co ti mám dát jako zástavu?“ Odvětila: „Své pečetidlo se šňůrkou a hůl, kterou máš v ruce.“ Dal jí to a vešel k ní. A ona s ním otěhotněla.
#38:19 Hned nato odešla, odložila roušku a oděla se do vdovských šatů.
#38:20 Juda pak poslal kůzle po svém adulamském příteli, aby vyzvedl od té ženy zástavu; ten ji však nenalezl.
#38:21 Vyptával se mužů toho místa: „Kde je ta kněžka, která byla v Énajimu u cesty?“ Odpověděli: „Žádná kněžka zde nebyla.“
#38:22 Vrátil se k Judovi a řekl: „Nenašel jsem ji. Také mužové toho místa pravili: ‚Žádná kněžka zde nebyla.‘“
#38:23 Juda řekl: „Ať si to nechá; jen když nebudeme v opovržení. Vždyť jsem to kůzle poslal, ale tys ji nenašel.“
#38:24 Asi po třech měsících bylo Judovi oznámeno: „Tvá snacha Támar se dopustila smilstva a dokonce je již z toho smilstva těhotná.“ Juda řekl: „Vyveďte ji, ať je upálena.“
#38:25 Když už ji vedli, poslala svému tchánovi vzkaz: „Jsem těhotná s mužem, jemuž patří tyhle věci.“ A dodala: „Pohleď jen pozorně, čí je toto pečetidlo, šňůrka a hůl!“
#38:26 Juda si je pozorně prohlédl a řekl: „Je spravedlivější než já; nedal jsem ji svému synu Šelovi.“ A už k ní nikdy nevešel.
#38:27 Nastala chvíle jejího porodu, a hle, v jejím životě byla dvojčata.
#38:28 Když rodila, jedno vystrčilo ruku. Porodní bába na ni rychle přivázala karmínovou nitku a řekla: „Toto vyjde první.“
#38:29 Ono však stáhlo ruku zpět, a hle, vyšel jeho bratr. Řekla: „Jakou trhlinou ses prodral!“ A pojmenovali ho Peres (to je Trhlina).
#38:30 Potom vyšel jeho bratr, který měl na ruce karmínovou nitku. Toho pojmenovali Zerach (to je Rozbřesk). 
#39:1 Josef byl odveden dolů do Egypta. Od Izmaelců, kteří ho tam dovedli, si ho koupil Egypťan Potífar, faraónův dvořan, velitel tělesné stráže.
#39:2 S Josefem však byl Hospodin, takže ho provázel zdar; byl v domě svého egyptského pána.
#39:3 Jeho pán viděl, že je s ním Hospodin a že všemu, co on činí, dopřává Hospodin zdaru.
#39:4 Josef proto získal jeho přízeň a posluhoval mu. Potífar ho ustanovil správcem svého domu a svěřil mu všechno, co měl.
#39:5 A od té chvíle, co ho Egypťan ustanovil ve svém domě nade vším, co měl, žehnal Hospodin jeho domu kvůli Josefovi. Hospodinovo požehnání bylo na všem, co měl, v domě i na poli.
#39:6 Ponechal tedy všechno, co měl, v rukou Josefových. Nestaral se přitom o nic, leda o chléb, který jedl. Josef byl krásné postavy, krásného vzhledu.
#39:7 Po těchto událostech se stalo, že se žena jeho pána do Josefa zahleděla a naléhala: „Spi se mnou!“
#39:8 Ale on odmítl a ženě svého pána řekl: „Pokud mě tu můj pán má, nestará se o nic, co je v domě; svěřil mi všechno, co má.
#39:9 V tomto domě není nikdo větší než já. Nevyňal z mé správy nic, jen tebe, protože jsi jeho manželka. Jak bych se tedy mohl dopustit takové špatnosti a prohřešit se proti Bohu!“
#39:10 A třebaže se Josefovi nabízela den co den, nevyhověl jí, aby k ní ulehl a byl s ní.
#39:11 Jednoho dne přišel do domu, aby vykonával svou práci. Nikdo z domácích v domě nebyl.
#39:12 Tu ho chytila za oděv se slovy: „Spi se mnou!“ Ale on jí nechal svůj oděv v ruce, utekl a vyběhl ven.
#39:13 Když viděla, že jí nechal svůj oděv v ruce a utekl ven,
#39:14 křikem přivolala služebnictvo a vykládala jim: „Hleďte, přivedli nám Hebreje, a on se u nás bude miliskovat! Přišel za mnou a chtěl se mnou spát. Proto jsem se dala do takového křiku.
#39:15 Jak slyšel, že se dávám do křiku a volám, nechal svůj oděv u mne, utekl a vyběhl ven.“
#39:16 Uložila oděv u sebe, dokud nepřišel jeho pán domů.
#39:17 Jemu vykládala totéž: „Přišel za mnou ten hebrejský otrok, jehož jsi k nám přivedl, a chtěl se se mnou miliskovat.
#39:18 Když jsem se dala do křiku a volala, nechal svůj oděv u mne a utekl ven.“
#39:19 Jakmile Josefův pán uslyšel slova své ženy, která ho ujišťovala: „Jak říkám, tohle mi provedl tvůj otrok“, vzplanul hněvem,
#39:20 vzal Josefa a vsadil ho do pevnosti, tam, kde byli vězněni královi vězňové. Tak se Josef ocitl v pevnosti.
#39:21 Ale Hospodin byl s ním, rozprostřel nad ním své milosrdenství a zjednal mu přízeň u velitele pevnosti;
#39:22 ten Josefovi svěřil všechny vězně v pevnosti. Řídil vše, co se tam mělo dělat.
#39:23 Velitel pevnosti nedohlížel na nic, co mu svěřil, poněvadž s Josefem byl Hospodin; všemu, co činil, dopřával Hospodin zdaru. 
#40:1 I stalo se po těch událostech, že číšník egyptského krále a pekař se prohřešili proti svému pánu, egyptskému králi.
#40:2 Farao se na oba své dvořany rozlítil, na nejvyššího číšníka a na nejvyššího pekaře,
#40:3 a dal je do vazby v domě velitele tělesné stráže při pevnosti, kde byl uvězněn Josef.
#40:4 Velitel tělesné stráže Josefa ustanovil, aby jim posluhoval. Nějaký čas byli ve vazbě.
#40:5 Tu oba dva měli sen, číšník i pekař egyptského krále, uvěznění v pevnosti; téže noci měl každý svůj sen volající po výkladu.
#40:6 Když k nim Josef ráno přišel, viděl, jak jsou sklíčeni.
#40:7 Zeptal se faraónových dvořanů, kteří s ním byli v domě jeho pána ve vazbě: „Proč jste dnes tak zamlklí?“
#40:8 Odvětili mu: „Měli jsme sen, avšak není tu nikdo, kdo by jej vyložil.“ Josef jim nato řekl: „Což vykládat sny není věc Boží? Jen mi je vypravujte.“
#40:9 Nejvyšší číšník vypravoval tedy Josefovi svůj sen: „Ve snu jsem pojednou před sebou viděl vinnou révu
#40:10 a na té révě tři výhonky. Sotva réva vypučela, hned rozkvetla a její hrozny dozrály.
#40:11 V ruce jsem měl faraónovu číši. Bral jsem hrozny, vytlačoval je do faraónovy číše a podával jsem mu ji do ruky.“
#40:12 Josef mu pravil: „Toto je výklad snu: Tři výhonky jsou tři dny;
#40:13 již po třech dnech tvou hlavu farao povýší a tvou hodnost ti vrátí. Budeš faraónovi podávat do ruky jeho číši podle dřívějšího práva, kdy jsi býval jeho číšníkem.
#40:14 Vzpomeneš-li si na mě, až se ti dobře povede, prokaž mně milosrdenství: upozorni na mě faraóna a vyvedeš mě z tohoto domu.
#40:15 Vždyť jsem byl ukraden z hebrejské země a zde jsem se nedopustil naprosto ničeho, zač by mě měli vsadit do jámy.“
#40:16 Když nejvyšší pekař viděl, že Josef dobře vykládá, řekl mu: „Také já jsem měl sen: Hle, měl jsem na hlavě tři košíky pečiva.
#40:17 V nejhořejším košíku byly všelijaké pokrmy, upečené pro faraóna. A ptáci je jedli z košíku na mé hlavě.“
#40:18 Josef odpověděl: „Toto je výklad snu: Tři košíky jsou tři dny.
#40:19 Již po třech dnech tvou hlavu farao povýší nad tebe - oběsí tě na dřevě. A ptáci budou z tebe rvát maso.“
#40:20 Stalo se pak třetího dne, v den faraónových narozenin, že farao vystrojil hody pro všechny své služebníky. Uprostřed svých služebníků povýšil hlavu nejvyššího číšníka a hlavu nejvyššího pekaře:
#40:21 nejvyššího číšníka znovu dosadil do jeho číšnického úřadu, aby podával faraónovi do ruky číši,
#40:22 a nejvyššího pekaře oběsil, jak jim vyložil Josef.
#40:23 Nejvyšší číšník si však na Josefa nevzpomněl; zapomněl na něho. 
#41:1 Po dvou letech se stalo, že farao měl sen: Stojí u Nilu.
#41:2 Pojednou z Nilu vystupuje sedm krav nápadně krásných a vykrmených a popásají se na říční trávě.
#41:3 A hle, za nimi vystupuje z Nilu jiných sedm krav, nápadně šeredných a vyhublých, a postaví se vedle těch sedmi krav na břehu.
#41:4 A ty nápadně šeredné a vyhublé sežraly sedm krav nápadně krásných a vykrmených. Vtom farao procitl.
#41:5 Když zase usnul, měl druhý sen. Sedm klasů bohatých a pěkných vyrůstá z jednoho stébla.
#41:6 A hle, za nimi vyráží sedm klasů hluchých a sežehlých východním větrem.
#41:7 A ty hluché klasy pohltily sedm klasů bohatých a plných. Farao procitl; takový to byl sen.
#41:8 Když nastalo jitro, byl tak rozrušen, že si dal zavolat všechny egyptské věštce a mudrce. Vyprávěl jim své sny, ale žádný mu je nedovedl vyložit.
#41:9 Až promluvil k faraónovi nejvyšší číšník: „Musím dnes připomenout svůj prohřešek:
#41:10 Farao se kdysi na své služebníky rozlítil a dal mě spolu s nejvyšším pekařem do vazby v domě velitele tělesné stráže.
#41:11 Jedné noci jsme oba měli sen; každý z nás měl sen, který volal po výkladu.
#41:12 Byl tam s námi hebrejský mládenec, otrok velitele tělesné stráže. Vypravovali jsme mu své sny a on nám je vyložil; každému vyložil, co jeho sen znamená.
#41:13 A vskutku, jak nám vyložil, tak se i stalo; mně farao vrátil hodnost, ale pekaře dal oběsit.“
#41:14 Farao si tedy dal zavolat Josefa. Okamžitě ho propustili z jámy. Oholil se, převlékl si plášť a přišel k faraónovi.
#41:15 Farao Josefovi řekl: „Měl jsem sen a nikdo mi jej nedovede vyložit. Doslechl jsem se, že tobě stačí sen slyšet a už jej vyložíš.“
#41:16 Josef faraónovi odpověděl: „Ne já, ale Bůh dá faraónovi uspokojivou odpověď.“
#41:17 Farao tedy k Josefovi mluvil: „Zdálo se mi, že stojím na břehu Nilu.
#41:18 Pojednou z Nilu vystupuje sedm krav vykrmených a krásného vzhledu a popásají se na říční trávě.
#41:19 A hle, za nimi vystupuje jiných sedm krav, nevzhledných, velice bídného vzrůstu a vychrtlých. Něco tak šeredného jsem neviděl v celé egyptské zemi.
#41:20 A ty vychrtlé a šeredné krávy sežraly prvních sedm krav vykrmených.
#41:21 Ačkoli se dostaly do jejich útrob, nebylo znát, že tam jsou. Zůstaly nápadně šeredné jako předtím. Vtom jsem procitl.
#41:22 Pak jsem ve snu viděl: Z jednoho stébla vyrůstá sedm klasů plných a pěkných.
#41:23 A hle, za nimi vyráží sedm klasů jalových, hluchých a sežehlých východním větrem.
#41:24 A ty hluché klasy pohltily sedm klasů pěkných. Řekl jsem to věštcům, ale nikdo mi nedovedl podat výklad.“
#41:25 Josef faraónovi odvětil: „Faraónův sen je jeden a týž. Bůh faraónovi oznámil, co učiní.
#41:26 Sedm pěkných krav, to je sedm let. Také sedm pěkných klasů je sedm let. Je to jeden sen.
#41:27 Sedm vychrtlých a šeredných krav, vystupujících za nimi, je sedm let, stejně jako sedm prázdných a východním větrem sežehlých klasů; to bude sedm let hladu.
#41:28 Když jsem faraónovi řekl: Bůh faraónovi ukázal, co učiní, mínil jsem toto:
#41:29 Přichází sedm let veliké hojnosti v celé egyptské zemi.
#41:30 Po nich však nastane sedm let hladu a všechna hojnost v egyptské zemi bude zapomenuta. Hlad zemi úplně zničí.
#41:31 V zemi nebude po hojnosti ani potuchy pro hlad, který potom nastane, neboť bude velmi krutý.
#41:32 Dvakrát byl sen faraónovi opakován proto, že slovo od Boha je nezvratné a Bůh to brzy vykoná.
#41:33 Ať se tedy farao nyní poohlédne po zkušeném a moudrém muži a dosadí ho za správce egyptské země.
#41:34 Nechť farao ustanoví v zemi dohlížitele a po sedm let hojnosti nechť vybírá pětinu výnosu egyptské země.
#41:35 Ať po dobu příštích sedmi úrodných let shromažďují všechnu potravu a ve městech ať uskladňují pod faraónovu moc obilí a hlídají je.
#41:36 Tato potrava zabezpečí zemi na sedm let hladu, která přijdou na egyptskou zemi. A země nezajde hladem.“
#41:37 Tato řeč se faraónovi i všem jeho služebníkům zalíbila.
#41:38 Farao svým služebníkům tedy řekl: „Zda najdeme podobného muže, v němž je duch Boží?“
#41:39 Josefovi pak řekl: „Když ti to vše dal Bůh poznat, nikdo nebude tak zkušený a moudrý jako ty.
#41:40 Budeš správcem mého domu a všechen můj lid bude poslouchat tvé rozkazy. Budu tě převyšovat jen trůnem.“
#41:41 Farao mu dále řekl: „Hleď, ustanovuji tě správcem celé egyptské země.“
#41:42 A farao sňal z ruky svůj prsten, dal jej na ruku Josefovu, oblékl ho do šatů z jemné látky a na šíji mu zavěsil zlatý řetěz.
#41:43 Dal ho vozit ve voze pro svého zástupce a volat před ním: „Na kolena!“ Tak ho učinil správcem celé egyptské země.
#41:44 Farao Josefovi ještě řekl: „Já jsem farao. Bez tebe nikdo nehne rukou ani nohou v celé egyptské zemi.“
#41:45 A farao Josefa pojmenoval Safenat Paneach (to je po egyptsku Zachránce světa) a dal mu za manželku Asenatu, dceru Potífery, kněze z Ónu. Tak vzešel Josef nad egyptskou zemí jako slunce.
#41:46 Josefovi bylo třicet let, když stanul před faraónem, králem egyptským. Josef pak vyšel od faraóna a procházel celou egyptskou zemí.
#41:47 Země vydávala po sedm let přebohatou hojnost.
#41:48 Shromažďoval tedy všechnu potravu po sedm let hojnosti, která v egyptské zemi nastala, a zásoby ukládal ve městech; v každém městě uložil potravu z okolních polí.
#41:49 Nashromáždil takové množství obilí, jako je písku v moři, takže přestali počítat, neboť se už počítat nedalo.
#41:50 Ještě než přišel rok hladu, narodili se Josefovi dva synové, které mu porodila Asenat, dcera Potífery, kněze z Ónu.
#41:51 Prvorozenému dal Josef jméno Manases (to je Bůh dal zapomenutí), neboť řekl: „Bůh mi dal zapomenout na všechno mé trápení a na celý dům mého otce.“
#41:52 Druhému dal jméno Efrajim (to je Bůh dal plodnost), neboť řekl: „Bůh mě učinil plodným v zemi mého utrpení.“
#41:53 Sedm let hojnosti v egyptské zemi skončilo
#41:54 a nastalo sedm let hladu, jak řekl Josef. Ve všech zemích byl hlad, ale v celé egyptské zemi měli chléb.
#41:55 Když všechen lid egyptské země začal hladovět a křičel k faraónovi o chléb, pravil farao celému Egyptu: „Jděte k Josefovi a učiňte, cokoli vám řekne.“
#41:56 Hlad byl po celé zemi. Tu Josef otevřel všechny sklady a prodával Egyptu obilí, neboť hlad tvrdě doléhal na egyptskou zemi.
#41:57 A všechny země přicházely do Egypta, aby nakupovaly u Josefa obilí, protože hlad tvrdě dolehl na celý svět. 
#42:1 Když Jákob viděl, že v Egyptě prodávají obilí, vytkl svým synům: „Co se díváte jeden na druhého?“
#42:2 A řekl: „Slyšel jsem, že v Egyptě prodávají obilí. Sestupte tam a nakupte je pro nás, ať zůstaneme naživu a nezemřeme.“
#42:3 Deset Josefových bratrů tedy sestoupilo nakoupit v Egyptě obilí.
#42:4 Josefova bratra Benjamína s nimi Jákob neposlal, protože si řekl: „Aby snad nepřišel o život!“
#42:5 Izraelovi synové přišli spolu s jinými nakoupit obilí, protože v kenaanské zemi byl hlad.
#42:6 Josef byl říšským správcem a prodával obilí všemu lidu země. Když přišli jeho bratři, skláněli se před ním tváří k zemi.
#42:7 Josef spatřil své bratry a poznal je, ale sám se jim nedal poznat a mluvil s nimi tvrdě. Otázal se jich: „Odkud jste přišli?“ Odvětili: „Z kenaanské země, abychom nakoupili potravu.“
#42:8 Ačkoli Josef své bratry poznal, oni ho nepoznali.
#42:9 Tu si vzpomněl na sny, které se mu o nich zdály. Křikl na ně: „Jste vyzvědači! Přišli jste obhlédnout nechráněná místa země.“
#42:10 Ohradili se: „Nikoli, pane; tvoji otroci přišli nakoupit potravu.
#42:11 Všichni jsme synové jednoho muže, jsme poctiví lidé. Tvoji otroci nikdy nebyli vyzvědači.“
#42:12 Ale on trval na svém: „Ne, přišli jste obhlédnout nechráněná místa země.“
#42:13 Odvětili: „Tvých otroků bylo dvanáct. Jsme bratři, synové jednoho muže z kenaanské země. Nejmladší je teď u otce a jeden - ten už není.“
#42:14 Ale Josef stál na svém: „Je to tak, jak jsem řekl. Jste vyzvědači.
#42:15 Takto budete prověřeni: Jakože živ je farao, nevyjdete odtud, dokud sem nepřijde váš nejmladší bratr.
#42:16 Vyšlete jednoho z vás, aby ho přivedl; vy zůstanete v poutech. Tak budou vaše výpovědi ověřeny, mluvíte-li pravdu. Když ne, jakože živ je farao, jste vyzvědači.“
#42:17 A vsadil je společně na tři dny do vazby.
#42:18 Třetího dne jim Josef řekl: „Toto udělejte a zůstanete naživu. Bojím se Boha.
#42:19 Jestliže jste poctiví, zůstane jeden z vás spoután ve vězení; ostatní půjdete a donesete obilí, aby vaše rodiny nehladověly.
#42:20 Svého nejmladšího bratra přiveďte ke mně; tak se prokáže pravdivost vašich výpovědí a nezemřete.“ I učinili tak.
#42:21 A řekli si navzájem: „Jistě jsme se provinili proti svému bratru; viděli jsme jeho tíseň, když nás prosil o smilování, ale nevyslyšeli jsme ho. Proto jsme přišli do tísně teď my.“
#42:22 Rúben jim odpověděl: „Cožpak jsem vám neříkal, abyste se na tom hochovi neprohřešovali? Neposlechli jste, a teď jsme voláni za jeho krev k odpovědnosti.“
#42:23 Nevěděli, že jim Josef rozumí, neboť s nimi mluvil skrze tlumočníka.
#42:24 Josef se od nich odvrátil a zaplakal. Pak se k nim obrátil a mluvil s nimi. Potom z nich vybral Šimeóna a před jejich očima ho spoutal.
#42:25 Nato dal příkaz, aby naplnili jejich měchy obilím, vrátili každému do jeho pytle stříbro a dali jim potravu na cestu. Učinili tak.
#42:26 Bratři naložili nakoupené obilí na osly a odjeli.
#42:27 Když pak na místě, kde nocovali, rozvázal jeden z nich pytel, aby dal svému oslu obrok, uviděl své stříbro navrchu v žoku.
#42:28 Zvolal na bratry: „Mé stříbro je tady! Zde v žoku!“ Zůstali jako bez sebe, roztřásli se a říkali jeden druhému: „Co nám to jen Bůh učinil?“
#42:29 Když přišli k svému otci Jákobovi do kenaanské země, pověděli mu všechno, co je potkalo:
#42:30 „Ten muž, pán země, mluvil s námi tvrdě a měl nás za vyzvědače.
#42:31 Říkali jsme mu, že jsme poctiví lidé, že nejsme vyzvědači,
#42:32 že nás bylo dvanáct bratrů, synů našeho otce; jeden, ten že už není, a nejmladší že je teď u otce v kenaanské zemi.
#42:33 Ale ten muž, pán země, odvětil: ‚Že jste poctiví, poznám podle toho: Jednoho ze svých bratrů necháte u mne. Vezměte obilí, aby vaše rodiny nehladověly, a jděte.
#42:34 Svého nejmladšího bratra pak přiveďte ke mně. Tak poznám, že nejste vyzvědači, že jste poctiví. Potom vám vašeho bratra vydám a můžete v zemi volně obchodovat.‘“
#42:35 Když vyprazdňovali pytle, našel každý ve svém pytli váček se stříbrem. Jakmile oni i otec spatřili váčky se stříbrem, padla na ně bázeň.
#42:36 Otec Jákob jim řekl: „Připravujete mě o děti. Nemám Josefa ani Šimeóna, a Benjamína mi chcete vzít. To všechno na mne dolehlo!“
#42:37 Nato Rúben svému otci pravil: „Můžeš usmrtit dva z mých synů, jestliže ti Benjamína nepřivedu. Svěř mi ho, já ti ho přivedu zpátky!“
#42:38 Ale on řekl: „Můj syn s vámi do Egypta nesestoupí. Jeho bratr je mrtev, zůstal sám. Kdyby na cestě, kterou se budete ubírat, přišel o život, uvalili byste na mé šediny žal a přivedli mě do podsvětí. 
#43:1 Hlad těžce doléhal na zemi dále.
#43:2 Když spotřebovali obilí, které dovezli z Egypta, řekl jim otec: „Nakupte nám znovu trochu potravy.“
#43:3 Juda odvětil: „Ten muž nás důrazně varoval a prohlásil: ‚Mou tvář nespatříte, nebude-li s vámi váš bratr.‘
#43:4 Jsi-li ochoten pustit s námi našeho bratra, sestoupíme do Egypta a nakoupíme ti potravu.
#43:5 Nejsi-li však ochoten ho pustit, nesestoupíme. Ten muž nám přece řekl: ‚Mou tvář nespatříte, nebude-li s vámi váš bratr.‘“
#43:6 Izrael se otázal: „Proč jste jednali vůči mně tak bezohledně a pověděli tomu muži, že máte ještě bratra?“
#43:7 Odvětili: „Ten muž se nás neodbytně vyptával na nás a na náš rod. Ptal se: ‚Je ještě naživu váš otec? Máte ještě bratra?‘ Pověděli jsme mu jen to, nač se přímo zeptal. Což jsme vůbec mohli tušit, že řekne: ‚Přiveďte svého bratra dolů‘?“
#43:8 Potom se na otce Izraele obrátil Juda: „Pusť toho chlapce se mnou, ať můžeme jít. Tak zůstaneme naživu a nezemřeme - ani my ani ty ani naši maličcí.
#43:9 Já sám se za něho zaručuji, můžeš mě volat k odpovědnosti. Jestli ti ho nepřivedu a nepostavím před tebe, prohřešil jsem se proti tobě na celý život.
#43:10 Vždyť kdybychom nebyli otáleli, mohli jsme už být dvakrát zpátky.“
#43:11 Otec Izrael jim řekl: „Když to tak musí být, učiňte toto: Vezměte si do nádob něco opěvovaných vzácností země a doneste je dolů tomu muži jako dar: trochu mastixu a trochu medu, ladanum a masti, pistácie a mandle.
#43:12 Vezměte s sebou dvojnásobnou částku stříbra a stříbro, které vám bylo vráceno do žoků, osobně vraťte; snad to byl omyl.
#43:13 Vezměte i svého bratra a hned se navraťte k tomu muži.
#43:14 Sám Bůh všemohoucí ať vás obdaří před tváří toho muže slitováním, aby propustil vašeho druhého bratra i Benjamína. Teď zůstanu úplně bez dětí!“
#43:15 Ti muži tedy vzali onen dar, vzali s sebou dvojnásobnou částku stříbra a Benjamína, vydali se na cestu, sestoupili do Egypta a postavili se před Josefem.
#43:16 Když Josef spatřil, že Benjamín je s nimi, řekl správci svého domu: „Uveď ty muže do domu, poraz hned dobytče a připrav je, neboť ti muži budou jíst v poledne se mnou.“
#43:17 Muž udělal, co mu Josef řekl, a uvedl je do jeho domu.
#43:18 Ale ti muži se báli, že byli uvedeni do domu Josefova, a říkali si: „To jsme předvedeni kvůli tomu stříbru, které nám bylo posledně vráceno do žoků. Teď se na nás vyřítí, přepadnou nás a zajmou nás i s našimi osly jako otroky.“
#43:19 Proto přistoupili k muži, který byl správcem Josefova domu, a ve dveřích domu se s ním domlouvali.
#43:20 Říkali: „Dovol, pane; my jsme sem posledně sestoupili, abychom nakoupili potravu.
#43:21 Když jsme vstoupili do noclehárny, rozvázali jsme žoky. A hle, každý měl své stříbro ve svém žoku navrchu, v plné váze. Vracíme je tedy osobně zpět.
#43:22 Přinesli jsme s sebou ještě jiné stříbro, abychom nakoupili potravu. Nevíme, kdo nám vložil naše stříbro do žoků.“
#43:23 Ale on řekl: „Upokojte se, nic se nebojte. Bůh váš a Bůh vašeho otce vám dal do žoků ten skrytý poklad. Vaše stříbro jsem přece přijal.“ A vyvedl k nim Šimeóna.
#43:24 Když správce uvedl ty muže do Josefova domu, dal jim vodu, aby si umyli nohy, a jejich oslům dal obrok.
#43:25 Oni zatím připravili dar pro Josefa, až v poledne přijde; slyšeli totiž, že tam mají stolovat.
#43:26 Jakmile Josef vkročil do domu, přinesli mu tam svůj dar a klaněli se mu až k zemi.
#43:27 Zeptal se jich, jak se jim vede, a otázal se: „Zdalipak se vede dobře vašemu starému otci, o němž jste mluvili? Je ještě naživu?“
#43:28 Odpověděli: „Tvému otroku, našemu otci, se vede dobře, je dosud živ.“ Padli na kolena a klaněli se mu.
#43:29 Tu se rozhlédl a spatřil svého bratra Benjamína, syna své matky, a tázal se: „Toto je váš nejmladší bratr, o kterém jste se mnou mluvili?“ A dodal: „Bůh ti buď milostiv, můj synu!“
#43:30 Nato se Josef rychle vzdálil. Byl hluboce pohnut a dojat nad bratrem až k pláči; vešel proto do pokojíku a rozplakal se tam.
#43:31 Potom si umyl obličej, vyšel a s přemáháním řekl: „Podávejte jídlo.“
#43:32 Podávali zvlášť jemu, zvlášť jim a zvlášť Egypťanům, kteří s ním jídali; Egypťané totiž nesmějí stolovat s Hebreji, poněvadž to je pro ně ohavnost.
#43:33 A seděli před ním od prvorozeného, jak náleželo prvorozenému, až po nejmladšího, každý podle svého věku. Jeden jako druhý trnuli úžasem.
#43:34 Potom je Josef uctil ze svého stolu; nejvíce ze všech, pětkrát víc než ostatní, však uctil Benjamína. Hodovali s ním a hojně se s ním napili. 
#44:1 Josef pak přikázal správci svého domu: „Naplň žoky těch mužů potravou, kolik jen budou moci unést, a stříbro každého z nich vlož navrch do jeho žoku.
#44:2 Můj kalich, ten stříbrný, vlož navrch do žoku nejmladšího spolu se stříbrem za nakoupené obilí.“ I učinil, jak mu Josef uložil.
#44:3 Ráno za svítání byli ti muži propuštěni i se svými osly.
#44:4 Když vyšli z města a nebyli ještě daleko, řekl Josef správci svého domu: „Pronásleduj ty muže. Až je dostihneš, řekni jim: ‚Proč jste se za dobro odvděčili zlem?
#44:5 Což jste nevzali to, z čeho můj pán pije a čeho používá k věštění? Zle jste se zachovali, že jste to udělali!‘“
#44:6 Když je správce dostihl, řekl jim ta slova.
#44:7 Odvětili mu: „Proč mluví můj pán taková slova? Tvoji otroci jsou daleci toho, aby udělali něco takového!
#44:8 Vždyť stříbro, které jsme našli navrchu ve svých žocích, jsme ti přinesli ze země kenaanské zpět. Jak bychom mohli ukrást stříbro nebo zlato z domu tvého pána?
#44:9 U koho z tvých otroků se to najde, ten ať zemře a my se staneme otroky svého pána!“
#44:10 Správce souhlasil: „Dobře, ať je po vašem. Ten, u koho se to najde, se stane mým otrokem; vy budete bez viny.“
#44:11 Každý rychle složil svůj žok na zem a rozvázal jej.
#44:12 Nastala prohlídka; začala nejstarším a skončila u nejmladšího. Kalich se našel v žoku Benjamínově.
#44:13 Tu roztrhli svůj šat, náklad naložili na osly a vrátili se do města.
#44:14 Tak přišel Juda se svými bratry do Josefova domu. Josef tam ještě byl; i padli před ním k zemi.
#44:15 Josef se na ně rozkřikl: „Co jste to spáchali za čin! Což nevíte, že muž jako já všechno uhodne?“
#44:16 Juda odvětil: „Co můžeme svému pánu říci? Jak to omluvíme? Čím se ospravedlníme? Sám Bůh stíhá tvé otroky za jejich provinění. Teď jsme otroky svého pána, my i ten, u něhož se kalich našel.“
#44:17 Josef řekl: „Takového jednání jsem dalek; mým otrokem se stane jen ten, u něhož se našel kalich, a vy odejděte v pokoji k svému otci.“
#44:18 Juda k němu přistoupil a řekl: „Dovol, můj pane, aby tvůj otrok směl promluvit ke svému pánu. Ať proti tvému otroku nevzplane tvůj hněv. Ty jsi přece jako sám farao.
#44:19 Můj pán se svých otroků vyptával: ‚Máte ještě otce nebo bratra?‘
#44:20 My jsme svému pánu odvětili: ‚Máme ještě starého otce a malého hocha, kterého zplodil ve svém stáří. Jeho bratr je mrtev, zůstal tedy po své matce sám; jeho otec jej miluje.‘
#44:21 Nato jsi svým otrokům řekl: ‚Přiveďte ho sem ke mně, chci ho vidět na vlastní oči.‘
#44:22 My jsme svému pánu odvětili: ‚Ten chlapec nemůže svého otce opustit. Opustí-li jej, otec zemře.‘
#44:23 Ale ty jsi svým otrokům řekl: ‚Nepřijde-li sem s vámi váš nejmladší bratr, nikdy nespatříte mou tvář.‘
#44:24 Když jsme pak přišli k tvému otroku, mému otci, oznámili jsme mu slova mého pána.
#44:25 Později náš otec řekl: ‚Nakupte nám znovu trochu potravy!‘
#44:26 My jsme odvětili: ‚Nemůžeme tam sestoupit. Sestoupíme tam jen tehdy, bude-li s námi náš nejmladší bratr. Tvář toho muže nespatříme, nebude-li náš nejmladší bratr s námi.‘
#44:27 Tvůj otrok, náš otec, nám řekl: ‚Vy víte, že mi má žena porodila dva syny.
#44:28 Jeden mi odešel. Naříkal jsem: Rozsápán je, rozsápán. Už jsem ho nespatřil.
#44:29 Vezmete-li mi i tohoto a přijde-li o život, uvalíte na mé šediny neštěstí a přivedete mě do podsvětí.‘
#44:30 Co teď, až přijdu k tvému otroku, svému otci, a chlapec, na němž lpí celou svou duší, s námi nebude?
#44:31 Jakmile spatří, že chlapec s námi není, umře. Tvoji otroci uvalí žal na šediny tvého otroka, našeho otce, a přivedou ho do podsvětí.
#44:32 Tvůj otrok se za toho chlapce, aby ho otec pustil, zaručil takto: ‚Nepřivedu-li ho k tobě, prohřeším se proti svému otci na celý život.‘
#44:33 Proto dovol, aby tvůj otrok zůstal u svého pána v otroctví namísto tohoto chlapce, a chlapec ať smí se svými bratry odejít.
#44:34 Jak bych mohl k svému otci přijít, kdyby chlapec nebyl se mnou? Což bych se mohl dívat na utrpení, které by mého otce postihlo?“ 
#45:1 Josef se už nemohl ovládnout přede všemi, kdo stáli kolem něho, a křikl: „Jděte všichni pryč!“ Tak u něho nezůstal nikdo, když se dal poznat svým bratrům.
#45:2 Hlasitě se rozplakal. Slyšeli to Egypťané, slyšel to dům faraónův.
#45:3 Tu řekl bratrům: „Já jsem Josef. Můj otec vskutku ještě žije?“ Bratři mu však nemohli odpovědět; tak se ho zhrozili.
#45:4 Josef je proto vyzval: „Přistupte ke mně.“ Když přistoupili, řekl jim: „Já jsem váš bratr Josef, kterého jste prodali do Egypta.
#45:5 Avšak netrapte se teď a nevyčítejte si, že jste mě sem prodali, neboť mě před vámi vyslal Bůh pro zachování života.
#45:6 V zemi trvá po dva roky hlad a ještě pět let nebude orba ani žeň.
#45:7 Bůh mě poslal před vámi, aby zajistil vaše potomstvo na zemi a aby vás zachoval při životě pro veliké vysvobození.
#45:8 A tak jste mě sem neposlali vy, ale Bůh. On mě učinil otcem faraónovým, pánem celého jeho domu a vladařem v celé egyptské zemi.
#45:9 Putujte rychle k otci a řekněte mu: ‚Toto praví tvůj syn Josef: Bůh mě učinil pánem celého Egypta. Nerozpakuj se a sestup ke mně.
#45:10 Budeš bydlit v zemi Gošenu, a tak mi budeš nablízku i se svými syny a vnuky, s bravem a skotem i se vším, co je tvé.
#45:11 Postarám se tam o tebe, neboť bude ještě pět let hladu, abys neměl nouzi ani ty ani tvůj dům ani nic z toho, co je tvé.‘
#45:12 Vidíte na vlastní oči, i můj bratr Benjamín to vidí, že jsem to já sám, kdo s vámi mluví.
#45:13 Povězte otci, jakou vážnost mám v Egyptě, a vše, co jste viděli. Pospěšte si a přiveďte ho sem.“
#45:14 Padl svému bratru Benjamínovi kolem krku a rozplakal se a Benjamín plakal na jeho šíji.
#45:15 Políbil také všechny bratry, sklonil se k nim a plakal. Teprve potom se bratři rozhovořili.
#45:16 Do faraónova domu se donesla zpráva: „Přišli Josefovi bratři.“ Bylo to milé faraónovi i jeho služebníkům.
#45:17 Farao tedy Josefovi řekl: „Pověz bratrům: Učiňte toto: Naložte na soumary náklad a dojděte do kenaanské země,
#45:18 vezměte otce a své rodiny a přijďte ke mně. Dám vám to nejlepší, co egyptská země má, a budete jíst tuk země.
#45:19 Přikazuji ti, abys jim řekl: Učiňte toto: Vezměte z egyptské země pro své děti a ženy povozy, dovezte svého otce a přijďte.
#45:20 Nelitujte ze svých věcí ničeho, protože budete mít to nejlepší z celé egyptské země.“
#45:21 Izraelovi synové tak učinili. Josef jim podle faraónova příkazu vydal povozy a dal jim zásobu potravy na cestu.
#45:22 Všem jim dal sváteční pláště, Benjamínovi však dal tři sta šekelů stříbra a pět svátečních plášťů.
#45:23 A svému otci poslal tyto věci: deset oslů, kteří nesli nejlepší egyptské věci, deset oslic, které nesly obilí a chléb, a stravu otci na cestu.
#45:24 Pak své bratry propustil. Když odcházeli, řekl jim: „Jen mezi sebou nevyvolávejte po cestě hádky!“
#45:25 I vystoupili z Egypta a přišli do země kenaanské k svému otci Jákobovi.
#45:26 Oznámili mu: „Josef ještě žije, a dokonce je vladařem nad celou egyptskou zemí!“ On však zůstal netečný, protože jim nevěřil.
#45:27 Vypravovali mu tedy všechno, co k nim Josef mluvil. Teprve když spatřil povozy, které Josef poslal, aby ho odvezly, okřál duch jejich otce Jákoba.
#45:28 „Stačí“, zvolal Izrael, „můj syn Josef žije! Půjdu, abych ho ještě před smrtí uviděl.“ 
#46:1 Izrael se vydal na cestu se vším, co měl. Když přišel do Beer-šeby, obětoval Bohu svého otce Izáka oběti.
#46:2 I řekl Bůh Izraelovi v nočních viděních: „Jákobe! Jákobe!“ A on odvětil: „Tu jsem.“
#46:3 Bůh pravil: „Já jsem Bůh, Bůh tvého otce. Neboj se sestoupit do Egypta; učiním tě tam velikým národem.
#46:4 Já sestoupím do Egypta s tebou a já tě také určitě vyvedu. Josef ti vlastní rukou zatlačí oči.“
#46:5 Jákob se tedy zvedl z Beer-šeby a Izraelovi synové vysadili svého otce Jákoba i své dítky a ženy na povozy, které pro ně farao poslal.
#46:6 Vzali svá stáda i jmění, jehož nabyli v zemi kenaanské, a přišli do Egypta, Jákob a s ním celé jeho potomstvo.
#46:7 Přivedl do Egypta všechno své potomstvo, své syny a vnuky, své dcery a vnučky.
#46:8 Toto jsou jména synů Izraelových, kteří vstoupili do Egypta: Jákob a jeho synové. Jákobův prvorozený syn Rúben;
#46:9 jeho synové Chanók a Palú, Chesrón a Karmí.
#46:10 Synové Šimeónovi: Jemúel, Jámin a Óhad, Jákin a Sóchar a Šaul, syn ženy kenaanské.
#46:11 Synové Léviho: Geršón, Kehat a Merarí.
#46:12 Synové Judovi: Er, Ónan a Šela, Peres a Zerach; Er a Ónan však zemřeli v zemi Kenaanu. Synové Peresovi byli Chesrón a Chámul.
#46:13 Synové Isacharovi: Tóla a Púva, Jób a Šimrón.
#46:14 Synové Zabulónovi: Sered, Elón a Jachleel.
#46:15 Ti všichni jsou synové Lejini, které Jákobovi porodila v Rovinách aramských stejně jako dceru Dínu. Celkem třiatřicet synů a dcer.
#46:16 Synové Gádovi: Sifjón a Chagí, Šúni a Esbón, Éri, Aródi a Aréli.
#46:17 Synové Ašerovi: Jimna a Jišva, Jišví a Bería a jejich sestra Serach; synové Beríovi byli Cheber a Malkíel.
#46:18 To jsou synové Zilpy, kterou dal Lában své dceři Leji, a ona je porodila Jákobovi, šestnáct duší.
#46:19 Synové Jákobovy ženy Ráchel: Josef a Benjamín.
#46:20 Josefovi se narodili v zemi egyptské Manases a Efrajim, které mu porodila Asenat, dcera ónského kněze Potífery.
#46:21 Synové Benjamínovi: Bela, Beker a Ašbel, Géra a Naamán, Échi a Róš, Mupím a Chupím a Ard.
#46:22 To jsou synové Ráchelini, kteří se narodili Jákobovi, celkem čtrnáct duší.
#46:23 Synové Danovi: Chuším.
#46:24 Synové Neftalího: Jachseel a Gúní, Jeser a Šilem.
#46:25 To jsou synové Bilhy, kterou dal Lában své dceři Ráchel, a ona je porodila Jákobovi, celkem sedm duší.
#46:26 Celkový počet těch, kdo vstoupili do Egypta s Jákobem, z jehož beder vzešli, byl šedesát šest duší, kromě žen Jákobových synů.
#46:27 K tomu synové Josefovi, kteří se mu narodili v Egyptě, dva. Všech duší domu Jákobova, které vešly do Egypta, bylo sedmdesát.
#46:28 Jákob poslal před sebou k Josefovi Judu, aby dal předem pokyny do Gošenu. Pak vstoupili do země Gošenu.
#46:29 Josef dal zapřáhnout do vozu a vyjel svému otci Izraelovi vstříc do Gošenu. Když se setkali, padl mu kolem krku a na jeho šíji se rozplakal.
#46:30 Izrael Josefovi řekl: „Teď už mohu zemřít, když jsem spatřil tvou tvář a vím, že jsi ještě živ.“
#46:31 Josef svým bratrům a rodině svého otce pravil: „Pojedu a ohlásím to faraónovi. Řeknu mu: ‚Přišli ke mně z kenaanské země moji bratři a rodina mého otce.
#46:32 Ti muži jsou pastýři ovcí, chovateli stád. Přišli se svým bravem a skotem a se vším, co mají.‘
#46:33 Až vás farao zavolá a otáže se, jaké je vaše zaměstnání,
#46:34 řeknete: ‚Tvoji otroci jsou od svého mládí až doposud chovateli stád, stejně jako naši otcové.‘ To proto, abyste se mohli usídlit v zemi Gošenu, neboť pro Egypťany jsou všichni pastýři ovcí ohavností.“ 
#47:1 Josef šel a oznámil faraónovi: „Můj otec a moji bratři se svým bravem a skotem i se vším, co mají, přišli ze země kenaanské. Jsou tu v zemi Gošenu.“
#47:2 Potom vzal ze svých bratrů pět mužů a postavil je před faraóna.
#47:3 Farao se jeho bratrů otázal: „Jaké je vaše zaměstnání?“ Oni mu odvětili: „Tvoji otroci jsou pastýři ovcí stejně jako naši otcové.“
#47:4 A řekli faraónovi: „Přišli jsme, abychom v této zemi pobývali jako hosté, poněvadž pro ovce tvých otroků nebyla žádná pastva; na kenaanskou zemi těžce dolehl hlad. Dej svolení, ať tvoji otroci mohou sídlit v zemi Gošenu.“
#47:5 Farao řekl Josefovi: „Tvůj otec a tvoji bratři přišli přece k tobě.
#47:6 Egyptská země je před tebou. Usídli tedy otce a bratry v nejlepší části země. Ať sídlí v zemi Gošenu. Máš-li za to, že jsou mezi nimi schopní muži, ustanov je správci nad mými stády.“
#47:7 Pak uvedl Josef svého otce Jákoba a postavil jej před faraóna, a Jákob faraónovi požehnal.
#47:8 Farao se Jákoba otázal: „Kolik je let tvého života?“
#47:9 Jákob mu odvětil: „Dnů mého putování je sto třicet let. Léta mého života byla nečetná a zlá, nedosáhla let života mých otců za dnů jejich putování.“
#47:10 Když Jákob faraónovi požehnal, vyšel od něho.
#47:11 Josef pak svého otce a bratry usadil a dal jim trvalé vlastnictví v zemi egyptské, v nejlepší její části, v zemi Ramesesu, podle faraónova rozkazu.
#47:12 A opatřoval svého otce i bratry chlebem, celý dům svého otce až po ty nejmenší.
#47:13 V celé zemi nebyl chléb; hlad doléhal velmi těžce. Země egyptská i země kenaanská byly hladem vyčerpány.
#47:14 Za obilí, které nakupovali, vybral Josef všechno stříbro, co se ho jen v zemi egyptské a kenaanské našlo, a odvedl je do faraónova domu.
#47:15 Tak došlo v zemi egyptské i kenaanské stříbro a všichni Egypťané přicházeli k Josefovi a žádali: „Dej nám chleba; copak ti máme umírat před očima jen proto, že nemáme stříbro?“
#47:16 Josef rozhodl: „Když už nemáte stříbro, dejte svá stáda a já vám za ně dám chléb.“
#47:17 Přiváděli tedy svá stáda k Josefovi a Josef jim dával chléb za koně, za stáda bravu a skotu a za osly. Pečoval o ně tím, že jim v onom roce poskytoval za všechna jejich stáda chléb.
#47:18 Tak uplynul onen rok. V příštím roce však přišli znovu a řekli mu: „Nebudeme před pánem tajit, že stříbro došlo a stáda dobytka už též patří pánovi. Jak pán vidí, zůstala nám už jen těla a půda.
#47:19 Proč ti máme umírat před očima, my i naše půda? Kup nás i s naší půdou za chléb a budeme i se svou půdou faraónovými otroky. Vydej osivo, abychom zůstali naživu a nezemřeli a naše půda aby nezpustla.“
#47:20 Josef tedy skoupil všechnu egyptskou půdu pro faraóna. Všichni Egypťané prodávali svá pole, neboť na ně tvrdě doléhal hlad. Země se tak stala vlastnictvím faraónovým.
#47:21 A lid od jednoho konce egyptského pomezí až ke druhému uvedl pod správu měst.
#47:22 Pouze půdu kněží nekupoval, neboť kněží měli od faraóna své důchody a žili z důchodů, které jim farao dával; proto svou půdu nemuseli prodat.
#47:23 Josef potom řekl lidu: „Dnes jsem koupil pro faraóna vás i vaši půdu. Zde máte osivo a půdu osejte.
#47:24 Pětinu z úrody budete odevzdávat faraónovi a čtyři díly vám zůstanou k osetí pole a pro obživu vaši a vašich domů a pro obživu vašich dětí.“
#47:25 Odpověděli: „Tys nás zachoval při životě! Jen když získáme přízeň svého pána! Budeme faraónovými otroky.“
#47:26 Josef tedy vydal o egyptské půdě nařízení dodnes platné: pětina úrody patří faraónovi. Faraónovým vlastnictvím se nestala jedině půda kněží.
#47:27 Izrael se usadil v egyptské zemi, na území Gošenu. Zabydleli se v ní, rozplodili se a velmi se rozmnožili.
#47:28 Jákob žil v egyptské zemi ještě sedmnáct let; všech let Jákobova života bylo sto čtyřicet sedm.
#47:29 Když se přiblížil den Izraelovy smrti, zavolal svého syna Josefa a řekl mu: „Jestliže jsem získal tvoji přízeň, vlož prosím ruku na můj klín a prokaž mi milosrdenství a věrnost: Nepohřbívej mě prosím v Egyptě!
#47:30 Až ulehnu ke svým otcům, vynes mě z Egypta a pochovej mě v jejich hrobě.“ Odpověděl: „Zachovám se podle tvých slov.“
#47:31 Izrael řekl: „Přísahej mi.“ Tedy mu přísahal. I poklonil se Izrael k hlavám lůžka. 
#48:1 Po těchto událostech pověděli Josefovi: „Tvůj otec je nemocen.“ Vzal tedy Josef s sebou oba své syny, Manasesa a Efrajima.
#48:2 I oznámili Jákobovi: „Přichází k tobě tvůj syn Josef.“ Tu se Izrael vzchopil a posadil se na lůžku.
#48:3 Jákob Josefovi řekl: „Bůh všemohoucí se mi ukázal v Lúzu v kenaanské zemi a požehnal mi
#48:4 slovy: ‚Hle, rozplodím tě a rozmnožím a učiním tě společenstvím lidských pokolení. Tuto zemi dávám do věčného vlastnictví tvému potomstvu.‘
#48:5 Oba synové, kteří se ti v egyptské zemi narodili před mým příchodem k tobě do Egypta, budou nyní moji. Efrajim a Manases jsou moji jako Rúben a Šimeón.
#48:6 Děti pak, které zplodíš po nich, budou tvé; ve svém dědictví se budou nazývat jmény svých bratrů.
#48:7 Když jsem přicházel z Rovin aramských, zemřela mi před očima Ráchel; bylo to v kenaanské zemi na cestě nedaleko Efraty a pochoval jsem ji tam u cesty do Efraty, to je do Betléma.“
#48:8 Izrael pohleděl na Josefovy syny a zeptal se: „Kdo to je?“
#48:9 Josef otci odvětil: „To jsou moji synové, které mi zde Bůh dal.“ Otec řekl: „Nuže, přiveď je ke mně, abych jim požehnal.“
#48:10 Izrael měl totiž oči obtížené stářím a špatně viděl. Josef je k němu přivedl, on je políbil a objal.
#48:11 Izrael Josefovi řekl: „Nedoufal jsem, že ještě někdy uvidím tvou tvář, a hle, Bůh mi dopřál vidět i tvé potomky.“
#48:12 Poté je Josef odvedl od jeho kolenou a sklonil se tváří až k zemi.
#48:13 Vzal oba, pravou rukou Efrajima a postavil ho k Izraelově levici, levou rukou Manasesa a postavil ho k Izraelově pravici.
#48:14 I vztáhl Izrael pravici a položil ji na hlavu Efrajima, který byl mladší, a levici na hlavu Manasesovu. Zkřížil ruce úmyslně, ačkoli prvorozený byl Manases.
#48:15 Požehnal Josefovi takto: „Bůh, před nímž ustavičně chodívali moji otcové Abraham a Izák, Bůh, Pastýř, který mě vodí od počátku až dodnes,
#48:16 Anděl, Vykupitel, jenž před vším zlým mě chránil, ať požehná těm chlapcům. Ať se v nich hlásá mé jméno a jméno mých otců Abrahama a Izáka; ať se nesmírně rozmnoží uprostřed země.“
#48:17 Josef viděl, že otec položil pravou ruku na hlavu Efrajimovu, a nelíbilo se mu to. Uchopil tedy otcovu ruku, aby ji přenesl z hlavy Efrajimovy na hlavu Manasesovu.
#48:18 Řekl mu: „Ne tak, otče, vždyť prvorozený je tento. Polož svou pravici na jeho hlavu.“
#48:19 Otec však odmítl: „Vím, synu, vím. Také on se stane lidem a také on vzroste. Jeho mladší bratr jej však přeroste a jeho potomstvo naplní pronárody.“
#48:20 A toho dne jim požehnal: „Tvým jménem bude Izrael žehnat: Bůh tě učiň jako Efrajima a Manasesa.“ Tak povýšil Efrajima nad Manasesa.
#48:21 Josefovi Izrael řekl: „Hle, já umírám, ale Bůh bude s vámi a přivede vás zpátky do země vašich otců.
#48:22 Tobě dávám o jeden díl víc než tvým bratrům; získal jsem jej z rukou Emorejců svým mečem a svým lukem.“ 
#49:1 I povolal Jákob své syny a řekl: „Sejděte se, oznámím vám, co v budoucnu vás potká.
#49:2 Shromážděte se a slyšte, synové Jákobovi, slyšte Izraele, svého otce.
#49:3 Rúbene, tys můj prvorozený, síla má a prvotina mého mužství; povzneseností a mocí překypuješ.
#49:4 Přetekls jak vody; nebudeš však první, protože jsi vstoupil na otcovo lože. Tehdy znesvětils je. Na mé lůžko vstoupil!
#49:5 Šimeón a Lévi, bratři, jejich zbraně - nástroj násilí.
#49:6 V jejich kruh ať nevstupuje moje duše, s jejich spolkem zajedno ať není moje sláva, neboť v hněvu povraždili muže, ve svém rozvášnění ochromili býky.
#49:7 Buď proklet jejich hněv, že byl tak prudký, jejich prchlivost, že byla tak krutá. Rozdělím je v Jákobovi, rozptýlím je v Izraeli.
#49:8 Tobě, Judo, tobě vzdají čest tví bratři. Na šíji nepřátel dopadne tvá ruka; synové tvého otce se ti budou klanět.
#49:9 Mládě lví je Juda. S úlovkem, můj synu, vystoupil jsi vzhůru. Stočil se a odpočíval jako lev, jak lvice. Kdo ho donutí, aby povstal?
#49:10 Juda nikdy nebude zbaven žezla ani palcátu, jenž u nohou mu leží, dokud nepřijde ten, který z něho vzejde; toho budou poslouchat lidská pokolení.
#49:11 Své oslátko si přiváže k vinné révě, mládě své oslice k révoví. Oděv svůj vypere ve víně, háv knížecí v krvi hroznů.
#49:12 Oči bude mít tmavší než víno, zuby bělejší než mléko.
#49:13 Zabulón se rozloží až k břehům moře, tam, kde lodě kotví, dosáhne až k Sidónu svým bokem.
#49:14 Isachar, to kostnatý je osel. Mezi dvěma ohradami odpočívá.
#49:15 Uviděl, jak dobré odpočinutí mít bude a že rozkošná je země; sehnul hřbet a břemena nosil, podrobil se otrockým pracím.
#49:16 Dan, ten povede pře svého lidu jako jeden z kmenů Izraele.
#49:17 Hadem na cestě buď Dan, buď na stezce růžkatou zmijí, jež do paty uštkne koně, že se jeho jezdec skácí nazpět. -
#49:18 Ve tvou spásu naději jsem složil, Hospodine!
#49:19 Na Gáda se vrhne horda, on však hordě té do týla vpadne.
#49:20 Ašerův chléb bude tučnost sama, lahůdky i králům bude skýtat.
#49:21 Neftalí, laň vypuštěná, promlouvá úchvatnými slovy.
#49:22 Josef, toť mladý plodonosný štěp, plodonosný štěp nad pramenem, přes zeď pnou se jeho ratolesti.
#49:23 Hořkostí ho naplnili, ohrožovali ho, střelci úklady mu nastrojili.
#49:24 Jeho luk si zachová svou pružnost, jeho paže svoji svěžest. Z rukou Přesilného Jákobova vzejde pastýř, kámen Izraele,
#49:25 z rukou Boha tvého otce. Kéž pomáhá tobě, kéž ti Všemohoucí žehná shora hojným požehnáním nebes, hojným požehnáním tůně propastné, jež odpočívá dole, hojným požehnáním prsů, požehnáním lůna.
#49:26 Požehnání tvého otce překonají požehnání horstev věčných, pahorků dávnověkých dary vytoužené. Ať přijdou na hlavu Josefovu, na temeno zasvěcence mezi bratry.
#49:27 Benjamín svůj úlovek rve jako vlk, co odvlekl, požírá hned ráno, večer dělí kořist.“
#49:28 To jsou všechny izraelské kmeny, celkem dvanáct, a toto k nim mluvil jejich otec, když jim žehnal. Každému požehnal zvláštním požehnáním.
#49:29 Také jim přikázal: „Až budu připojen ke svému lidu, pochovejte mě k mým otcům do jeskyně, která je na poli Chetejce Efróna,
#49:30 do jeskyně na poli v Makpele, naproti Mamre v kenaanské zemi; to pole koupil Abraham od Chetejce Efróna, aby měl vlastní hrob.
#49:31 Tam pochovali Abrahama a jeho ženu Sáru, tam pochovali Izáka a jeho ženu Rebeku, a tam jsem pochoval Leu.
#49:32 To pole bylo i s jeskyní získáno od Chetejců.“
#49:33 Když Jákob dokončil příkazy svým synům, uložil se opět na lože a zesnul. Tak byl připojen k svému lidu. 
#50:1 Tu padl Josef na tvář svého otce, plakal nad ním a líbal ho.
#50:2 Potom přikázal svým služebníkům lékařům, aby otce nabalzamovali. Lékaři balzamovali Izraele
#50:3 plných čtyřicet dní; tak dlouho totiž trvá balzamování. Egypťané ho oplakávali sedmdesát dní.
#50:4 Když přešly dny smutku, promluvil Josef k nejbližším faraónovým: „Jestliže jsem získal vaši přízeň, předložte faraónovi mou prosbu:
#50:5 Můj otec mě zapřisáhl slovy: ‚Hle, já umírám. Pochovej mě v mé hrobce, kterou jsem si vytesal v kenaanské zemi.‘ Dovol, abych tam vystoupil a pochoval svého otce; pak se navrátím.“
#50:6 Farao řekl: „Jen vystup a pohřbi svého otce, jak tě zapřisáhl.“
#50:7 I šel Josef pochovat svého otce a šli s ním všichni faraónovi služebníci, starší jeho domu i všichni starší egyptské země,
#50:8 celý Josefův dům a jeho bratři i dům jeho otce. V zemi Gošenu zanechali pouze své dítky, svůj brav a skot.
#50:9 Vytáhlo s ním též vozatajstvo a jezdectvo. Byl to velmi slavný průvod.
#50:10 Když přišli do Goren-atádu, který je u Jordánu, dali se do velikého a velmi ponurého nářku. Josef tu uspořádal za svého otce sedmidenní smuteční slavnost.
#50:11 Kenaanci, obyvatelé té země, viděli smuteční slavnost v Goren-atádu a pravili: „Egypt má těžký smutek!“ Proto pojmenovali to místo, které je u Jordánu, Ábel-misrajim (to je Smutek Egypta).
#50:12 Pak synové učinili, jak jim otec přikázal.
#50:13 Donesli ho do kenaanské země a pochovali ho v jeskyni na poli v Makpele; to pole naproti Mamre koupil Abraham od Chetejce Efróna, aby měl vlastní hrob.
#50:14 Po otcově pohřbu se Josef vrátil do Egypta se svými bratry a se všemi, kteří s ním vyšli pochovat jeho otce.
#50:15 Když si Josefovi bratři uvědomili, že jejich otec je mrtev, řekli si: „Jen aby na nás Josef nezanevřel a neoplatil nám všechno zlo, kterého jsme se na něm dopustili.“
#50:16 Proto mu vzkázali: „Tvůj otec před smrtí přikázal:
#50:17 Josefovi řekněte toto: ‚Ach, odpusť prosím svým bratrům přestoupení a hřích, neboť se na tobě dopustili zlého činu. Odpusť prosím služebníkům Boha tvého otce to přestoupení.‘“ Josef se nad jejich vzkazem rozplakal.
#50:18 Pak přišli bratři sami, padli před ním a řekli: „Tu jsme, měj nás za otroky!“
#50:19 Josef jim však odvětil: „Nebojte se. Což jsem Bůh?
#50:20 Vy jste proti mně zamýšleli zlo, Bůh však zamýšlel dobro; tím, co se stalo, jak dnes vidíme, zachoval naživu četný lid.
#50:21 Nebojte se už tedy; postarám se o vás i o vaše děti.“ Tak je těšil a promlouval jim k srdci.
#50:22 Josef sídlil v Egyptě i s domem svého otce a byl živ sto deset let.
#50:23 Syny Efrajimovy viděl do třetího pokolení. Též synové Makíra, syna Manasesova, se zrodili na Josefova kolena.
#50:24 Potom Josef svým bratrům řekl: „Já umírám, ale Bůh vás jistě navštíví a vyvede vás odtud do země, kterou přísežně slíbil Abrahamovi, Izákovi a Jákobovi.“
#50:25 A zapřisáhl syny Izraelovy slovy: „Bůh vás jistě navštíví a pak odtud vynesete mé kosti.“
#50:26 I umřel Josef, když mu bylo sto deset let; nabalzamovali ho a položili do rakve v Egyptě.  

\book{Exodus}{Exod}
#1:1 Toto jsou jména synů Izraelových, kteří přišli do Egypta s Jákobem; každý přišel se svou rodinou:
#1:2 Rúben, Šimeón, Lévi a Juda,
#1:3 Isachar, Zabulón a Benjamín,
#1:4 Dan a Neftalí, Gád a Ašer.
#1:5 Všech, kdo vzešli z Jákobových beder, bylo sedmdesát. Josef už byl v Egyptě.
#1:6 Potom zemřel Josef a všichni jeho bratři i celé to pokolení.
#1:7 Ale Izraelci se rozplodili, až se to jimi hemžilo, převelice se rozmnožili a byli velice zdatní; byla jich plná země.
#1:8 V Egyptě však nastoupil nový král, který o Josefovi nevěděl.
#1:9 Ten řekl svému lidu: „Hle, izraelský lid je početnější a zdatnější než my.
#1:10 Musíme s ním nakládat moudře, aby se nerozmnožil. Kdyby došlo k válce, jistě by se připojil k těm, kdo nás nenávidí, bojoval by proti nám a odtáhl by ze země.“
#1:11 Ustanovili tedy nad ním dráby, aby jej ujařmovali robotou. Musel stavět faraónovi města pro sklady, Pitom a Raamses.
#1:12 Avšak jakkoli jej ujařmovali, množil se a rozmáhal dále, takže měli z Izraelců hrůzu.
#1:13 Proto začali Egypťané Izraelce surově zotročovat.
#1:14 Ztrpčovali jim život tvrdou otročinou při výrobě cihel a všelijakou prací na poli. Všechnu otročinu, kterou na ně uvalili, jim ještě ztěžovali surovostí.
#1:15 Egyptský král poručil hebrejským porodním bábám, z nichž jedna se jmenovala Šifra a druhá Púa:
#1:16 „Když budete pomáhat Hebrejkám při porodu a při slehnutí zjistíte, že to je syn, usmrťte jej; bude-li to dcera, ať si je naživu.“
#1:17 Avšak porodní báby se bály Boha a rozkazem egyptského krále se neřídily. Nechávaly hochy naživu.
#1:18 Egyptský král si porodní báby předvolal a řekl jim: „Co to děláte, že necháváte hochy naživu?“
#1:19 Porodní báby faraónovi odvětily: „Hebrejky nejsou jako ženy egyptské; jsou plné života. Porodí dříve, než k nim porodní bába přijde.“
#1:20 Bůh pak těm porodním bábám prokazoval dobrodiní a lid se množil a byl velmi zdatný.
#1:21 Protože se porodní báby bály Boha, požehnal jejich domům.
#1:22 Ale farao všemu svému lidu rozkázal: „Každého syna, který se jim narodí, hoďte do Nilu; každou dceru nechte naživu.“ 
#2:1 Muž z Léviova domu šel a vzal si lévijskou dceru.
#2:2 Žena otěhotněla a porodila syna. Když viděla, jak je půvabný, ukrývala ho po tři měsíce.
#2:3 Ale déle už ho ukrývat nemohla. Proto pro něho připravila ze třtiny ošatku, vymazala ji asfaltem a smolou, položila do ní dítě a vložila do rákosí při břehu Nilu.
#2:4 Jeho sestra se postavila opodál, aby zvěděla, co se s ním stane.
#2:5 Tu sestoupila faraónova dcera, aby se omývala v Nilu, a její dívky se procházely podél Nilu. Vtom uviděla v rákosí ošatku a poslala svou otrokyni, aby ji přinesla.
#2:6 Otevřela ji a spatřila dítě, plačícího chlapce. Bylo jí ho líto a řekla: „Je z hebrejských dětí.“
#2:7 Jeho sestra se faraónovy dcery otázala: „Mám jít a zavolat kojnou z hebrejských žen, aby ti dítě odkojila?“
#2:8 Faraónova dcera jí řekla: „Jdi!“ Děvče tedy šlo a zavolalo matku dítěte.
#2:9 Faraónova dcera jí poručila: „Odnes to dítě, odkoj mi je a já ti zaplatím.“ Žena vzala dítě a odkojila je.
#2:10 Když dítě odrostlo, přivedla je k faraónově dceři a ona je přijala za syna. Pojmenovala ho Mojžíš (to je Vytahující). Řekla: „Vždyť jsem ho vytáhla z vody.“
#2:11 V oněch dnech, když Mojžíš dospěl, vyšel ke svým bratřím a viděl jejich robotu. Spatřil nějakého Egypťana, jak ubíjí Hebreje, jednoho z jeho bratří.
#2:12 Rozhlédl se na všechny strany, a když viděl, že tam nikdo není, ubil Egypťana a zahrabal do písku.
#2:13 Když vyšel druhého dne, spatřil dva Hebreje, jak se rvali. Řekl tomu, který nebyl v právu: „Proč chceš ubít svého druha?“
#2:14 Ohradil se: „Kdo tě ustanovil nad námi za velitele a soudce? Máš v úmyslu mě zavraždit, jako jsi zavraždil toho Egypťana?“ Mojžíš se ulekl a řekl si: „Jistě se o věci už ví!“
#2:15 Farao o tom vskutku uslyšel a chtěl dát Mojžíše zavraždit. Ale Mojžíš před faraónem uprchl a usadil se v midjánské zemi; posadil se u studny.
#2:16 Midjánský kněz měl sedm dcer. Ty přišly, vážily vodu a plnily žlaby, aby napojily stádo svého otce.
#2:17 Tu přišli pastýři a odháněli je. Ale Mojžíš vstal, ochránil je a napojil jejich stádo.
#2:18 Když přišly ke svému otci Reúelovi, zeptal se: „Jak to, že jste dnes přišly tak brzo?“
#2:19 Odpověděly: „Nějaký Egypťan nás vysvobodil z rukou pastýřů. Také nám ochotně navážil vodu a napojil stádo.“
#2:20 Reúel se zeptal svých dcer: „Kde je? Proč jste tam toho muže nechaly? Zavolejte ho, ať pojí chléb!“
#2:21 Mojžíš se rozhodl, že u toho muže zůstane, a on mu dal svou dceru Siporu za manželku.
#2:22 Ta porodila syna a Mojžíš mu dal jméno Geršom (to je Hostem-tam). Řekl: „Byl jsem hostem v cizí zemi.“
#2:23 Po mnoha letech egyptský král zemřel, ale Izraelci vzdychali a úpěli v otročině dál. Jejich volání o pomoc vystupovalo z té otročiny k Bohu.
#2:24 Bůh vyslyšel jejich sténání, Bůh se rozpomněl na svou smlouvu s Abrahamem, Izákem a Jákobem,
#2:25 Bůh na syny Izraele pohleděl, Bůh se k nim přiznal. 
#3:1 Mojžíš pásl ovce svého tchána Jitra, midjánského kněze. Jednou vedl ovce až za step a přišel k Boží hoře, k Chorébu.
#3:2 Tu se mu ukázal Hospodinův posel v plápolajícím ohni uprostřed trnitého keře. Mojžíš viděl, jak keř v ohni hoří, ale není jím stráven.
#3:3 Řekl si: „Zajdu se podívat na ten veliký úkaz, proč keř neshoří.“
#3:4 Hospodin viděl, že odbočuje, aby se podíval. I zavolal na něho Bůh zprostředku keře: „Mojžíši, Mojžíši!“ Odpověděl: „Tu jsem.“
#3:5 Řekl: „Nepřibližuj se sem! Zuj si opánky, neboť místo, na kterém stojíš, je půda svatá.“
#3:6 A pokračoval: „Já jsem Bůh tvého otce, Bůh Abrahamův, Bůh Izákův a Bůh Jákobův.“ Mojžíš si zakryl tvář, neboť se bál na Boha pohledět.
#3:7 Hospodin dále řekl: „Dobře jsem viděl ujařmení svého lidu, který je v Egyptě. Slyšel jsem jeho úpění pro bezohlednost jeho poháněčů. Znám jeho bolesti.
#3:8 Sestoupil jsem, abych jej vysvobodil z moci Egypta a vyvedl jej z oné země do země dobré a prostorné, do země oplývající mlékem a medem, na místo Kenaanců, Chetejců, Emorejců, Perizejců, Chivejců a Jebúsejců.
#3:9 Věru, úpění Izraelců dolehlo nyní ke mně. Viděl jsem také útlak, jak je Egypťané utlačují.
#3:10 Nuže pojď, pošlu tě k faraónovi a vyvedeš můj lid, Izraelce, z Egypta.“
#3:11 Ale Mojžíš Bohu namítal: „Kdo jsem já, abych šel k faraónovi a vyvedl Izraelce z Egypta?“
#3:12 Odpověděl: „Já budu s tebou! A toto ti bude znamením, že jsem tě poslal: Až vyvedeš lid z Egypta, budete sloužit Bohu na této hoře.“
#3:13 Avšak Mojžíš Bohu namítl: „Hle, já přijdu k Izraelcům a řeknu jim: Posílá mě k vám Bůh vašich otců. Až se mě však zeptají, jaké je jeho jméno, co jim odpovím?“
#3:14 Bůh řekl Mojžíšovi: „JSEM, KTERÝ JSEM.“ A pokračoval: „Řekni Izraelcům toto: JSEM posílá mě k vám.“
#3:15 Bůh dále Mojžíšovi poručil: „Řekni Izraelcům toto: ‚Posílá mě k vám Hospodin, Bůh vašich otců, Bůh Abrahamův, Bůh Izákův a Bůh Jákobův.‘ To je navěky mé jméno, jím si mě budou připomínat od pokolení do pokolení.
#3:16 Jdi, shromažď izraelské starší a pověz jim: ‚Ukázal se mi Hospodin, Bůh vašich otců, Bůh Abrahamův, Izákův a Jákobův, a řekl: Rozhodl jsem se vás navštívit, vím, jak s vámi v Egyptě nakládají,
#3:17 a prohlásil jsem: Vyvedu vás z egyptského ujařmení do země Kenaanců, Chetejců, Emorejců, Perizejců, Chivejců a Jebúsejců, do země oplývající mlékem a medem.‘
#3:18 Až tě vyslechnou, půjdeš ty a izraelští starší k egyptskému králi a řeknete mu: ‚Potkal se s námi Hospodin, Bůh Hebrejů. Dovol nám nyní odejít do pouště na vzdálenost tří dnů cesty a přinést oběť Hospodinu, našemu Bohu.‘
#3:19 Vím, že vám egyptský král nedovolí jít, leda z donucení.
#3:20 Proto vztáhnu ruku a budu bít Egypt všemožnými svými divy, které učiním uprostřed něho. Potom vás propustí.
#3:21 Zjednám tomuto lidu u Egypťanů přízeň. Až budete odcházet, nepůjdete s prázdnou.
#3:22 Každá žena si vyžádá od sousedky a spolubydlící stříbrné a zlaté ozdoby a pláště. Vložíte je na své syny a dcery. Tak vypleníte Egypt.“ 
#4:1 Mojžíš však znovu namítal: „Nikoli, neuvěří mi a neuposlechnou mě, ale řeknou: Hospodin se ti neukázal.“
#4:2 Hospodin mu řekl: „Co to máš v ruce?“ Odpověděl: „Hůl.“
#4:3 Hospodin řekl: „Hoď ji na zem.“ Hodil ji na zem a stal se z ní had. Mojžíš se dal před ním na útěk.
#4:4 Ale Hospodin Mojžíšovi poručil: „Vztáhni ruku a chyť ho za ocas.“ Vztáhl tedy ruku, uchopil ho a v dlani se mu z něho stala hůl.
#4:5 „Aby uvěřili, že se ti ukázal Hospodin, Bůh jejich otců, Bůh Abrahamův, Bůh Izákův a Bůh Jákobův.“
#4:6 Dále mu Hospodin řekl: „Vlož si ruku za ňadra.“ Vložil tedy ruku za ňadra. Když ruku vytáhl, byla malomocná, bílá jako sníh.
#4:7 Tu poručil: „Dej ruku zpět za ňadra.“ Dal ruku zpět za ňadra. Když ji ze záňadří vytáhl, byla opět jako ostatní tělo.
#4:8 „A tak jestliže ti neuvěří a nedají na první znamení, uvěří druhému znamení.
#4:9 Jestliže však neuvěří ani těmto dvěma znamením a neuposlechnou tě, nabereš vodu z Nilu a vyleješ ji na suchou zemi. Z vody, kterou nabereš z Nilu, se stane na suché zemi krev.“
#4:10 Ale Mojžíš Hospodinu namítal: „Prosím, Panovníku, nejsem člověk výmluvný; nebyl jsem dříve, nejsem ani nyní, když ke svému služebníku mluvíš. Mám neobratná ústa a neobratný jazyk.“
#4:11 Hospodin mu však řekl: „Kdo dal člověku ústa? Kdo působí, že je člověk němý nebo hluchý, vidící nebo slepý? Zdali ne já, Hospodin?
#4:12 Nyní jdi, já sám budu s tvými ústy a budu tě učit, co máš mluvit!“
#4:13 Ale Mojžíš odmítl: „Prosím, Panovníku, pošli si, koho chceš.“
#4:14 Tu Hospodin vzplanul proti Mojžíšovi hněvem a řekl: „Což nemáš bratra Árona, toho lévijce? Znám ho, ten umí mluvit. Jde ti už naproti a bude se srdečně radovat, až tě uvidí.
#4:15 Budeš k němu mluvit a vkládat mu slova do úst. Já budu s tvými ústy i s jeho ústy a budu vás poučovat, co máte činit.
#4:16 On bude mluvit k lidu za tebe, on bude tobě ústy a ty budeš jemu Bohem.
#4:17 A tuto hůl vezmi do ruky; budeš jí konat znamení.“
#4:18 Mojžíš odešel a vrátil se ke svému tchánu Jitrovi. Řekl mu: „Rád bych šel a vrátil se ke svým bratřím, kteří jsou v Egyptě, a podíval se, zda ještě žijí.“ Jitro Mojžíšovi odvětil: „Jdi v pokoji.“
#4:19 Hospodin pak řekl Mojžíšovi ještě v Midjánu: „Jen se vrať do Egypta, neboť zemřeli všichni, kteří ti ukládali o život.“
#4:20 Mojžíš tedy vzal svou ženu a syny, posadil je na osla a vracel se do egyptské země. A do ruky si vzal Boží hůl.
#4:21 Hospodin dále Mojžíšovi poručil: „Až se vrátíš do Egypta, hleď, abys před faraónem udělal všechny zázraky, jimiž jsem tě pověřil. Já však zatvrdím jeho srdce a on lid nepropustí.
#4:22 Potom faraónovi řekneš: Toto praví Hospodin: Izrael je můj prvorozený syn.
#4:23 Vzkázal jsem ti: Propusť mého syna, aby mi sloužil. Ale ty jsi jej propustit odmítl. Za to zabiji tvého prvorozeného syna.“
#4:24 Když se na cestě chystali nocovat, střetl se s ním Hospodin a chtěl ho usmrtit.
#4:25 Tu vzala Sipora kamenný nůž, obřezala předkožku svého syna, dotkla se jeho nohou a řekla: „Jsi můj ženich, je to zpečetěno krví.“
#4:26 A Hospodin ho nechal být. Tehdy se při obřízkách říkalo: „Jsi ženich, je to zpečetěno krví.“
#4:27 Hospodin řekl Áronovi: „Jdi na poušť naproti Mojžíšovi.“ Áron šel, setkal se s ním u Boží hory a políbil ho.
#4:28 Mojžíš oznámil Áronovi všechna Hospodinova slova, s nimiž ho poslal, a všechna znamení, kterými ho pověřil.
#4:29 Pak šel Mojžíš s Áronem a shromáždili všechny izraelské starší.
#4:30 Áron vyřídil všechna slova, která mluvil Hospodin k Mojžíšovi, a Mojžíš učinil před očima lidu ona znamení.
#4:31 A lid uvěřil. Když slyšeli, že Hospodin navštívil Izraelce a že pohleděl na jejich ujařmení, padli na kolena a klaněli se. 
#5:1 Mojžíš s Áronem pak předstoupili před faraóna a řekli: „Toto praví Hospodin, Bůh Izraele: Propusť můj lid, ať mi v poušti slaví slavnost.“
#5:2 Farao však odpověděl: „Kdo je Hospodin, že bych ho měl uposlechnout a propustit Izraele? Hospodina neznám a Izraele nepropustím!“
#5:3 Řekli: „Potkal se s námi Bůh Hebrejů. Dovol nám nyní odejít do pouště na vzdálenost tří dnů cesty a přinést oběť Hospodinu, našemu Bohu, aby nás nenapadl morem nebo mečem.“
#5:4 Egyptský král je okřikl: „Proč, Mojžíši a Árone, odvádíte lid od jeho prací? Jděte za svými robotami!“
#5:5 A farao pokračoval: „Hle, lidu země je teď mnoho, a vy chcete, aby nechali svých robot?“
#5:6 Onoho dne přikázal farao poháněčům lidu a dozorcům:
#5:7 „Propříště nebudete vydávat lidu slámu k výrobě cihel jako dříve. Ať si jdou slámu nasbírat sami!
#5:8 A uložíte jim dodat stejné množství cihel, jaké vyráběli dříve. Nic jim neslevujte, jsou líní. Proto křičí: Pojďme obětovat svému Bohu.
#5:9 Ať na ty muže těžce dolehne otročina, aby měli co dělat a nedali na lživé řeči.“
#5:10 Poháněči lidu a dozorci vyšli a ohlásili lidu: „Toto praví farao: Nedám vám žádnou slámu.
#5:11 Sami si jděte nabrat slámu, kde ji najdete. Ale z vaší pracovní povinnosti se nic nesleví.“
#5:12 Lid se rozběhl po celé egyptské zemi, aby na strništích sbíral slámu.
#5:13 Poháněči je honili: „Plňte svůj denní úkol, jako když sláma byla.“
#5:14 Dozorci z řad Izraelců, které nad nimi ustanovili faraónovi poháněči, byli biti. Vytýkalo se jim: „Proč jste v těchto dnech nevyrobili tolik cihel jako dříve?“
#5:15 Dozorci z řad Izraelců tedy přišli a úpěli před faraónem: „Proč se svými otroky takhle jednáš?
#5:16 Tvým otrokům se nedodává sláma, ale pokud jde o cihly, poroučejí nám: ‚Dělejte!‘ Hle, tvoji otroci jsou biti a tvůj lid bude pykat za hřích“.
#5:17 Farao odpověděl: „Jste lenoši líní, proto říkáte: ‚Pojďme obětovat Hospodinu.‘
#5:18 Hned jděte dělat! Sláma vám dodávána nebude, ale dodávku cihel odvedete.“
#5:19 Dozorci z řad Izraelců viděli, že je s nimi zle, když bylo řečeno: „Nesmíte snížit svůj denní úkol výroby cihel.“
#5:20 Když vycházeli od faraóna, narazili na Mojžíše a Árona, kteří se s nimi chtěli setkat.
#5:21 Vyčítali jim: „Ať se nad vámi ukáže Hospodin a rozsoudí. Vy jste pokáleli naši pověst u faraóna a jeho služebníků. Dali jste jim do ruky meč, aby nás povraždili.“
#5:22 Mojžíš se obrátil k Hospodinu a řekl: „Panovníku, proč jsi dopustil na tento lid zlo? Proč jsi mě vlastně poslal?
#5:23 Od chvíle, kdy jsem předstoupil před faraóna, abych mluvil tvým jménem, nakládá s tímto lidem ještě hůře. A ty svůj lid stále nevysvobozuješ.“ 
#6:1 Hospodin Mojžíšovi odvětil: „Nyní uvidíš, co faraónovi udělám. Donutím ho, aby je propustil; donutím ho, aby je vypudil ze své země.“
#6:2 Bůh promluvil k Mojžíšovi a ujistil ho: „Já jsem Hospodin.
#6:3 Ukázal jsem se Abrahamovi, Izákovi a Jákobovi jako Bůh všemohoucí. Ale své jméno Hospodin jsem jim nedal poznat.
#6:4 Ustavil jsem s nimi také svou smlouvu, že jim dám kenaanskou zemi, zemi jejich putování, kde pobývali jako hosté.
#6:5 Rovněž jsem uslyšel sténání Izraelců, které si Egypťané podrobili v otroctví, a rozpomenul jsem se na svou smlouvu.
#6:6 Proto řekni Izraelcům: Já jsem Hospodin. Vyvedu vás z egyptské roboty, vysvobodím vás z vašeho otroctví a vykoupím vás vztaženou paží a velkými soudy.
#6:7 Vezmu si vás za lid a budu vám Bohem. Poznáte, že já jsem Hospodin, váš Bůh, který vás vyvede z egyptské roboty.
#6:8 Dovedu vás do země, kterou jsem přísežně slíbil dát Abrahamovi, Izákovi a Jákobovi. Vám ji dám do vlastnictví. Já jsem Hospodin.“
#6:9 Mojžíš to tak Izraelcům vyhlásil, ale ti nebyli pro malomyslnost a tvrdou otročinu s to Mojžíšovi naslouchat.
#6:10 Hospodin dále mluvil k Mojžíšovi:
#6:11 „Předstup před faraóna, krále egyptského, a vyřiď mu, ať propustí Izraelce ze své země.“
#6:12 Mojžíš Hospodinu namítl: „Když mi nenaslouchají Izraelci, jak by mě poslechl farao! Nejsem způsobilý mluvit.“
#6:13 Ale Hospodin Mojžíšovi a Áronovi domluvil a dal jim příkazy pro Izraelce i pro faraóna, krále egyptského, aby připravili odchod Izraelců z egyptské země.
#6:14 Toto jsou představitelé otcovských rodů: Rúbenovci, potomci Izraelova prvorozeného: Chanók a Palú, Chesrón a Karmí. To jsou čeledi Rúbenovy.
#6:15 Šimeónovci: Jemúel, Jámin, Ohad, Jákin, Sochar a Šaul, syn Kenaanky. To jsou čeledi Šimeónovy.
#6:16 Toto jsou jména Léviovců podle jejich rodopisu: Geršón, Kehat a Merarí. Lévi byl živ sto třicet sedm let.
#6:17 Geršónovci: Libní a Šimeí podle svých čeledí.
#6:18 Kehatovci: Amrám, Jishár, Chebrón a Uzíel. Kehat byl živ sto třicet tři léta.
#6:19 Meraríovci: Machlí a Muší. To jsou lévijské čeledi podle jejich rodopisu.
#6:20 Amrám si vzal za ženu Jókebedu, svou tetu. Ta mu porodila Árona a Mojžíše. Amrám byl živ sto třicet sedm let.
#6:21 Synové Jishárovi: Kórach, Nefeg a Zikrí.
#6:22 Synové Uzíelovi: Míšael, Elsáfan a Sitrí.
#6:23 Áron si vzal za ženu Elíšebu, dceru Amínadabovu, sestru Nachšónovu. Ta mu porodila Nádaba, Abíhúa, Eleazara a Ítamara.
#6:24 Synové Kórachovi: Asír, Elkána a Abíasaf. To jsou kórachovské čeledi.
#6:25 Eleazar, syn Áronův, si vzal za ženu jednu z dcer Pútíelových. Ta mu porodila Pinchasa. To jsou představitelé lévijských rodů podle svých čeledí.
#6:26 Z tohoto pokolení pocházejí ten Áron a Mojžíš, k nimž mluvil Hospodin: „Vyveďte z egyptské země Izraelce seřazené po oddílech.“
#6:27 Oni to byli, kdo mluvili k faraónovi, králi egyptskému, že mají vyvést Izraelce z Egypta. To tedy byli Mojžíš a Áron.
#6:28 To bylo tehdy, když Hospodin mluvil k Mojžíšovi v egyptské zemi.
#6:29 Hospodin promluvil k Mojžíšovi: „Já jsem Hospodin! Řekni faraónovi, králi egyptskému, všechno, co k tobě mluvím.“
#6:30 Mojžíš však Hospodinu namítl: „Nejsem způsobilý mluvit. Jak by mě farao poslechl?“ 
#7:1 Hospodin řekl Mojžíšovi: „Pohleď, ustanovil jsem tě, abys byl pro faraóna Bohem, a Áron, tvůj bratr, bude tvým prorokem.
#7:2 Ty mu povíš všechno, co ti přikážu, a Áron, tvůj bratr, bude mluvit s faraónem, aby propustil Izraelce ze své země.
#7:3 Já však zatvrdím faraónovo srdce a učiním v egyptské zemi mnoho svých znamení a zázraků.
#7:4 Farao vás neposlechne, ale já vložím na Egypt svou ruku. Vyvedu zástupy svého lidu, syny Izraele, z egyptské země, ale ji postihnu velkými soudy.
#7:5 Egypťané poznají, že já jsem Hospodin, až vztáhnu svou ruku na Egypt a vyvedu Izraelce z jejich středu.“
#7:6 Mojžíš a Áron učinili přesně tak, jak jim Hospodin přikázal.
#7:7 Mojžíšovi bylo osmdesát let a Áronovi osmdesát tři léta, když mluvili s faraónem.
#7:8 Hospodin dále řekl Mojžíšovi a Áronovi:
#7:9 „Až k vám farao promluví: ‚Prokažte se nějakým zázrakem‘, řekneš Áronovi: ‚Vezmi svou hůl a hoď ji před faraóna‘, a stane se drakem.“
#7:10 Mojžíš s Áronem tedy předstoupili před faraóna a učinili, jak Hospodin přikázal. Áron hodil svou hůl před faraóna i před jeho služebníky a ona se stala drakem.
#7:11 Farao však také povolal mudrce a čaroděje, a egyptští věštci učinili svými kejklemi totéž.
#7:12 Hodili každý svou hůl na zem a ony se staly draky. Ale Áronova hůl jejich hole pohltila.
#7:13 Srdce faraónovo se však zatvrdilo a neposlechl je, jak Hospodin předpověděl.
#7:14 Hospodin řekl Mojžíšovi: „Srdce faraónovo je neoblomné. Nechce lid propustit.
#7:15 Jdi k faraónovi ráno. Až půjde k vodě, postav se naproti němu na břehu Nilu a vezmi si do ruky hůl, která se proměnila v hada.
#7:16 Řekneš mu: Hospodin, Bůh Hebrejů, mě k tobě posílá se vzkazem: Propusť můj lid, aby mi na poušti sloužil. Ale ty jsi dosud neposlechl.
#7:17 Toto praví Hospodin: Podle toho poznáš, že já jsem Hospodin: Holí, kterou mám v ruce, teď udeřím do vody v Nilu, a ta se promění v krev.
#7:18 Ryby, které jsou v Nilu, leknou a Nil bude páchnout. Marně budou Egypťané usilovat, aby se mohli napít vody z Nilu.“
#7:19 Hospodin dále řekl Mojžíšovi: „Vyzvi Árona: ‚Vezmi svou hůl a vztáhni ruku nad egyptské vody, nad průplavy, nad říční ramena, nad jezera, vůbec nad všechny nahromaděné vody.‘ Stanou se krví. V celé egyptské zemi bude krev, i ve džberech a džbánech.“
#7:20 Mojžíš a Áron učinili, jak Hospodin přikázal. Áron pozdvihl hůl a před očima faraóna a jeho služebníků udeřil do vody v Nilu a všechna voda Nilu se proměnila v krev.
#7:21 Ryby v Nilu lekly, Nil začal páchnout a Egypťané nemohli vodu z Nilu pít. A krev byla v celé egyptské zemi.
#7:22 Ale totéž učinili egyptští věštci svými kejklemi. Faraónovo srdce se zatvrdilo a neposlechl je, jak Hospodin předpověděl.
#7:23 Farao se obrátil a vešel do svého domu, a ani toto si nevzal k srdci.
#7:24 Všichni Egypťané kopali kolem Nilu, aby přišli na pitnou vodu, protože vodu z Nilu pít nemohli.
#7:25 To trvalo plných sedm dní poté, co Hospodin zasáhl Nil.
#7:26 Potom Hospodin řekl Mojžíšovi: „Předstup před faraóna a řekni mu: Toto praví Hospodin: Propusť můj lid, aby mi sloužil.
#7:27 Budeš-li se zdráhat jej propustit, napadnu celé tvé území žábami.
#7:28 Nil se bude žábami hemžit, vylezou a vniknou do tvého domu, do tvé ložnice a na tvé lože i do domu tvých služebníků a mezi tvůj lid, do tvých pecí a díží.
#7:29 I po tobě, po tvém lidu a po všech tvých služebnících polezou žáby.“ 
#8:1 Hospodin dále řekl Mojžíšovi: „Vyzvi Árona: ‚Vztáhni ruku se svou holí nad průplavy, nad říční ramena i nad jezera a vyveď na egyptskou zemi žáby.‘“
#8:2 Áron vztáhl ruku nad egyptské vody a žáby vylézaly, až pokryly egyptskou zemi.
#8:3 Ale totéž učinili věštci svými kejklemi a i oni vyvedli na egyptskou zemi žáby.
#8:4 Tu povolal farao Mojžíše a Árona a řekl: „Proste Hospodina, aby mě i můj lid zbavil žab. Pak propustím lid, aby obětoval Hospodinu.“
#8:5 Mojžíš faraónovi odvětil: „Rač mi sdělit, kdy mám prosit za tebe, za tvé služebníky a za tvůj lid, aby Hospodin vyhladil žáby u tebe i v tvých domech. Zůstanou jen v Nilu.“
#8:6 Farao odpověděl: „Zítra.“ Mojžíš řekl: „Ať je podle tvého slova, abys poznal, že nikdo není jako Hospodin, náš Bůh.
#8:7 Žáby se stáhnou od tebe i z tvých domů, od tvých služebníků a od tvého lidu. Zůstanou jen v Nilu.“
#8:8 Nato odešel Mojžíš s Áronem od faraóna a Mojžíš úpěnlivě volal k Hospodinu kvůli žábám, kterými faraóna postihl.
#8:9 Hospodin učinil podle Mojžíšovy prosby a žáby v domech, ve dvorcích i na polích pošly.
#8:10 Shrabali je na hromady a kupy a zápach z nich naplnil zemi.
#8:11 Když však farao viděl, že nastala úleva, zůstal v srdci neoblomný a neposlechl je, jak Hospodin předpověděl.
#8:12 Hospodin řekl Mojžíšovi: „Vyzvi Árona: ‚Vztáhni svou hůl a udeř do prachu na zemi!‘ Stanou se z něho po celé egyptské zemi komáři.“
#8:13 I učinili tak. Áron vztáhl ruku s holí a udeřil do prachu na zemi a na lidech i na dobytku se objevili komáři. Po celé egyptské zemi se ze všeho prachu země stali komáři.
#8:14 Když totéž chtěli učinit věštci svými kejklemi, totiž vyvést komáry, nemohli. A komáři byli na lidech i na dobytku.
#8:15 Věštci tedy řekli faraónovi: „Je to prst Boží.“ Srdce faraónovo se však zatvrdilo a neposlechl je, jak Hospodin předpověděl.
#8:16 Hospodin řekl Mojžíšovi: „Za časného jitra se postav před faraóna, až vyjde k vodě. Řekneš mu: Toto praví Hospodin: Propusť můj lid, aby mi sloužil!
#8:17 Jestliže můj lid nepropustíš, pošlu na tebe, na tvé služebníky, na tvůj lid i na tvé domy mouchy. Domy Egypťanů budou plné much, i ta půda, na které žijí.
#8:18 Ale zemi Gošen, kde se zdržuje můj lid, v onen den podivuhodně odliším. Tam mouchy nebudou, abys poznal, že já jsem Hospodin i uprostřed této země.
#8:19 Učiním rozdíl mezi lidem svým a lidem tvým. Toto znamení se stane zítra.“
#8:20 A Hospodin tak učinil. Dotěrné mouchy vnikly do domu faraónova, do domu jeho služebníků a na celou egyptskou zemi. Země byla těmi mouchami zamořena.
#8:21 Tu povolal farao Mojžíše a Árona a řekl: „Nuže, přineste oběť svému Bohu zde v zemi.“
#8:22 Mojžíš odpověděl: „Nebylo by správné, abychom to učinili. To, co máme obětovat Hospodinu, svému Bohu, je Egypťanům ohavností. Copak by nás neukamenovali, kdybychom před nimi obětovali, co je jim ohavností?
#8:23 Odejdeme do pouště na vzdálenost tří dnů cesty a tam budeme obětovat Hospodinu, svému Bohu, jak nám nařídil.“
#8:24 Farao řekl: „Propustím vás tedy, abyste obětovali Hospodinu, svému Bohu, na poušti. Jenom neodcházejte příliš daleko. Proste za mne.“
#8:25 Mojžíš odvětil: „Až od tebe odejdu, budu prosit Hospodina a zítra odletí mouchy od faraóna, od jeho služebníků i od jeho lidu. Jen ať nás opět farao neobelstí, že by nechtěl propustit lid, aby obětoval Hospodinu.“
#8:26 Pak Mojžíš od faraóna odešel a prosil Hospodina.
#8:27 A Hospodin učinil, jak Mojžíš řekl. Mouchy odletěly od faraóna, od jeho služebníků i od jeho lidu. Ani jediná nezůstala.
#8:28 Ale farao zůstal v srdci neoblomný i tentokrát a lid nepropustil. 
#9:1 Hospodin řekl Mojžíšovi: „Předstup před faraóna a promluv k němu: Toto praví Hospodin, Bůh Hebrejů: Propusť můj lid, aby mi sloužil!
#9:2 Budeš-li se zdráhat jej propustit a zatvrdíš-li se proti nim ještě víc,
#9:3 tu na tvá stáda, která jsou na poli, na koně, na osly, na velbloudy, na skot i na brav, dolehne Hospodinova ruka velmi těžkým morem.
#9:4 Hospodin však bude podivuhodně rozlišovat mezi stády izraelskými a stády egyptskými, takže nezajde nic z toho, co patří Izraelcům.
#9:5 Hospodin také určil lhůtu: Zítra toto učiní Hospodin v celé zemi.“
#9:6 A nazítří to Hospodin učinil. Všechna egyptská stáda pošla, ale z izraelských stád nepošel jediný kus.
#9:7 Farao si to dal zjistit, a vskutku z izraelských stád nepošel jediný kus; přesto zůstalo srdce faraónovo neoblomné a lid nepropustil.
#9:8 Hospodin řekl Mojžíšovi a Áronovi: „Naberte si plné hrsti sazí z pece a Mojžíš ať je rozhazuje faraónovi před očima směrem k nebi.
#9:9 Bude z nich po celé egyptské zemi poprašek, který způsobí na lidech i na dobytku po celé egyptské zemi vředy hnisavých neštovic.“
#9:10 Nabrali tedy saze z pece, postavili se před faraóna a Mojžíš je rozhazoval směrem k nebi. Na lidech i na dobytku se objevily vředy hnisavých neštovic.
#9:11 Ani věštci se nemohli postavit před Mojžíše pro vředy, neboť vředy byly na věštcích i na všech Egypťanech.
#9:12 Hospodin však zatvrdil faraónovo srdce, takže je neposlechl, jak Hospodin Mojžíšovi předpověděl.
#9:13 Hospodin řekl Mojžíšovi: „Za časného jitra se postav před faraóna. Řekneš mu: Toto praví Hospodin, Bůh Hebrejů: Propusť můj lid, aby mi sloužil!
#9:14 Tentokrát zasáhnu do srdce všemi svými údery tebe i tvé služebníky a tvůj lid, abys poznal, že na celé zemi není nikdo jako já.
#9:15 Vždyť už tehdy, když jsem vztáhl ruku, abych bil tebe i tvůj lid morem, mohl jsi být vyhlazen ze země.
#9:16 Avšak proto jsem tě zachoval, abych na tobě ukázal svou moc a aby se po celé zemi vypravovalo o mém jménu.
#9:17 Stále jednáš proti mému lidu zpupně a nechceš jej propustit.
#9:18 Proto spustím zítra v tuto dobu tak hrozné krupobití, jaké v Egyptě nebylo ode dne jeho vzniku až do nynějška.
#9:19 Nuže, dej odvést do bezpečí svá stáda a všechno, co máš na poli. Všechny lidi i dobytek, vše, co bude zastiženo na poli a nebude shromážděno do domu, potluče krupobití, takže zemřou.“
#9:20 Kdo z faraónových služebníků se Hospodinova slova ulekl, zahnal své otroky a svá stáda do domů.
#9:21 Kdo si slovo Hospodinovo nevzal k srdci, nechal své otroky a svá stáda na poli.
#9:22 Hospodin řekl Mojžíšovi: „Vztáhni svou ruku k nebi. Na celou egyptskou zemi dolehne krupobití, na lidi, na dobytek i na všecky polní byliny v egyptské zemi.“
#9:23 Když Mojžíš vztáhl svou hůl k nebi, dopustil Hospodin hromobití a krupobití. Na zemi padal oheň. Tak Hospodin spustil krupobití na egyptskou zemi.
#9:24 Nastalo krupobití a uprostřed krupobití šlehal oheň; něco tak hrozného nebylo v celé zemi egyptské od dob, kdy se dostala do moci tohoto pronároda.
#9:25 Krupobití potlouklo v celé egyptské zemi všechno, co bylo na poli, od lidí po dobytek; krupobití potlouklo také všechny polní byliny a polámalo všechno polní stromoví.
#9:26 Jenom v zemi Gošenu, kde sídlili Izraelci, krupobití nebylo.
#9:27 Tu si farao dal předvolat Mojžíše a Árona a řekl jim: „Opět jsem zhřešil. Hospodin je spravedlivý, a já i můj lid jsme svévolníci.
#9:28 Proste Hospodina. Božího hromobití a krupobití je už dost. Propustím vás, nemusíte tu už dál zůstat.“
#9:29 Mojžíš mu odvětil: „Jen co vyjdu z města, rozprostřu své dlaně k Hospodinu. Hromobití přestane a krupobití skončí, abys poznal, že země je Hospodinova.
#9:30 Vím ovšem, že ty ani tvoji služebníci se stále ještě nebudete Hospodina Boha bát.“
#9:31 Potlučen byl len a ječmen, protože ječmen byl už v klasech a len nasazoval tobolky.
#9:32 Pšenice a špalda však potlučeny nebyly, protože jsou pozdní.
#9:33 Mojžíš vyšel od faraóna z města a rozprostřel dlaně k Hospodinu. Hromobití a krupobití přestalo a déšť už nezaplavoval zemi.
#9:34 Když farao viděl, že přestal déšť a krupobití i hromobití, hřešil dále. Zůstal v srdci neoblomný, on i jeho služebníci.
#9:35 Srdce faraónovo se zatvrdilo a Izraelce nepropustil, jak Hospodin skrze Mojžíše předpověděl. 
#10:1 Hospodin řekl Mojžíšovi: „Předstup před faraóna. Já jsem totiž učinil jeho srdce i srdce jeho služebníků neoblomné, abych mohl uprostřed nich provést tato svá znamení
#10:2 a ty abys mohl vypravovat svým synům i vnukům o tom, co jsem v Egyptě dokázal, i o znameních, která jsem mezi nimi udělal, ať víte, že já jsem Hospodin.“
#10:3 Mojžíš a Áron tedy předstoupili před faraóna a řekli mu: „Toto praví Hospodin, Bůh Hebrejů: Jak dlouho se budeš zdráhat pokořit se přede mnou? Propusť můj lid, aby mi sloužil.
#10:4 Budeš-li se zdráhat propustit můj lid, pak na tvé území uvedu zítra kobylky.
#10:5 Přikryjí povrch země, takže nebude možno zemi ani vidět, a sežerou zbytek toho, co vyvázlo, co vám zůstalo po krupobití. Ožerou také všechny stromy, které vám na polích znovu raší.
#10:6 Naplní tvé domy, domy všech tvých služebníků i domy všech Egypťanů. Něco takového neviděli tvoji otcové ani dědové od doby, kdy začali obdělávat půdu, až dodnes.“ Nato se Mojžíš obrátil a odešel od faraóna.
#10:7 Faraónovi služebníci řekli: „Jak dlouho nám bude tento člověk léčkou? Propusť ty muže, ať slouží Hospodinu, svému Bohu. Což jsi dosud nepoznal, že hrozí Egyptu zánik?“
#10:8 Mojžíš a Áron byli přivedeni zpět k faraónovi. Ten jim řekl: „Nuže, služte Hospodinu, svému Bohu. Kdo všechno má jít?“
#10:9 Mojžíš odvětil: „Půjdeme se svou mládeží i se starci, půjdeme se svými syny i dcerami, se svým bravem i skotem, neboť máme slavnost Hospodinovu.“
#10:10 Farao jim však řekl: „To tak! Myslíte si, že Hospodin bude s vámi, když vás propustím s dětmi? To jste si zamanuli špatnou věc.
#10:11 Kdepak! Vy muži si jděte a služte Hospodinu, když o to tak stojíte.“ A vyhnali je od faraóna.
#10:12 Hospodin řekl Mojžíšovi: „Vztáhni nad egyptskou zemi ruku, aby přilétly na egyptskou zemi kobylky a sežraly všechny byliny země, všechno, co zůstalo po krupobití.“
#10:13 Mojžíš tedy vztáhl nad egyptskou zemi hůl a Hospodin přihnal na zemi východní vítr. Ten vál po celý den a celou noc. Když nastalo jitro, přinesl východní vítr kobylky.
#10:14 Kobylky přilétly na celou egyptskou zemi a spustily se na celé území Egypta v takovém množství, že tolik kobylek nebylo nikdy předtím ani potom.
#10:15 Přikryly povrch celé země, až se na zemi zatmělo, a sežraly všechny byliny na zemi i všechno ovoce na stromech, co zbylo po krupobití. Na stromech a na polních bylinách po celé egyptské zemi nezbylo nic zeleného.
#10:16 Farao rychle povolal Mojžíše a Árona. Řekl jim: „Zhřešil jsem proti Hospodinu, vašemu Bohu, i proti vám.
#10:17 Sejmi prosím můj hřích ještě tentokrát a proste Hospodina, svého Boha, aby jen odvrátil ode mne tuto smrt.“
#10:18 Mojžíš od faraóna odešel a prosil Hospodina.
#10:19 Tu Hospodin obrátil vítr a velmi silný mořský vítr odnesl kobylky a prudce je vrhl do Rákosového moře, takže na celém egyptském území nezůstala jediná kobylka.
#10:20 Avšak Hospodin zatvrdil faraónovo srdce, takže Izraelce nepropustil.
#10:21 Hospodin řekl Mojžíšovi: „Vztáhni svou ruku k nebi a egyptskou zemi zahalí temnota, taková temnota, že se dá nahmatat.“
#10:22 Mojžíš vztáhl ruku k nebi. Tu nastala po celé egyptské zemi tma tmoucí a trvala po tři dny.
#10:23 Lidé neviděli jeden druhého; po tři dny se nikdo neodvážil hnout ze svého místa. Ale všichni Izraelci měli ve svých obydlích světlo.
#10:24 Farao povolal Mojžíše a řekl: „Odejděte! Služte Hospodinu! Zanechte tu jenom svůj brav a skot. Také vaše děti mohou jít s vámi.“
#10:25 Mojžíš odpověděl: „Ty sám nám dáš potřebné k obětním hodům a k zápalným obětem, abychom je připravili Hospodinu, svému Bohu.
#10:26 Půjdou s námi i naše stáda, ani pazneht tu nezůstane. Budeme z nich brát k službě Hospodinu, svému Bohu. My ještě nevíme, čím budeme Hospodinu sloužit, dokud tam nepřijdeme.“
#10:27 Avšak Hospodin zatvrdil faraónovo srdce a on je nedovolil propustit.
#10:28 Farao řekl: „Odejdi ode mne. Dej si pozor, ať mi už nepřijdeš na oči. Neboť v den, kdy mi přijdeš na oči, zemřeš!“
#10:29 Mojžíš odpověděl: „Jak jsi řekl. Už ti na oči nepřijdu.“ 
#11:1 Hospodin řekl Mojžíšovi: „Ještě jednu ránu uvedu na faraóna a na Egypt. Potom vás odtud propustí, nadobro vyhostí, přímo vás odtud vyžene.
#11:2 Vybídni lid, ať si vyžádá každý muž od svého souseda a každá žena od své sousedky stříbrné a zlaté šperky.“
#11:3 A Hospodin zjednal lidu v očích Egypťanů přízeň. Také sám Mojžíš platil v egyptské zemi za velice významného v očích faraónových služebníků i v očích lidu.
#11:4 Mojžíš řekl faraónovi: „Toto praví Hospodin: O půlnoci projdu Egyptem.
#11:5 Všichni prvorození v egyptské zemi zemřou, od prvorozeného syna faraónova, který sedí na jeho trůnu, po prvorozeného syna otrokyně, která mele na mlýnku, i všechno prvorozené z dobytka.
#11:6 Po celé egyptské zemi se bude rozléhat veliký křik, jakého nebylo a už nebude.
#11:7 Ale na žádného Izraelce ani pes nezavrčí, ani na člověka ani na dobytče, abyste poznali, že Hospodin podivuhodně rozlišuje mezi Egyptem a Izraelem.
#11:8 Všichni tito tvoji služebníci sestoupí ke mně, budou se mi klanět a říkat: Odejdi ty i všechen lid, který jde za tebou! Teprve potom odejdu.“ Nato Mojžíš, planoucí hněvem, od faraóna odešel.
#11:9 Hospodin řekl Mojžíšovi: „Farao vás neposlechne, a tak mých zázraků v egyptské zemi ještě přibude.“
#11:10 Mojžíš a Áron všechny ty zázraky před faraónem učinili, ale Hospodin zatvrdil faraónovo srdce, takže Izraelce ze své země nepropustil. 
#12:1 Hospodin řekl Mojžíšovi a Áronovi v egyptské zemi:
#12:2 „Tento měsíc bude pro vás začátkem měsíců. Bude pro vás prvním měsícem v roce.
#12:3 Vyhlaste celé izraelské pospolitosti: Desátého dne tohoto měsíce si každý vezmete beránka podle svých rodů, beránka na rodinu.
#12:4 Kdyby byla rodina malá a na beránka by nestačila, přibere si každý souseda, který bydlí nejblíže jeho rodiny, aby doplnil počet osob. Podle toho, kolik kdo sní, stanovíte počet na beránka.
#12:5 Budete mít beránka bez vady, ročního samce. Vezmete jej z ovcí nebo z koz.
#12:6 Budete jej opatrovat až do čtrnáctého dne tohoto měsíce. Navečer bude celé shromáždění izraelské pospolitosti beránky zabíjet.
#12:7 Pak vezmou trochu krve a potřou jí obě veřeje i nadpraží u domů, v nichž jej budou jíst.
#12:8 Tu noc budou jíst maso upečené na ohni a k němu budou jíst nekvašené chleby s hořkými bylinami.
#12:9 Nebudete z něho jíst nic syrového ani vařeného ve vodě, nýbrž jen upečené na ohni s hlavou i s nohama a vnitřnostmi.
#12:10 Nic z něho nenecháte do rána. Co z něho zůstane do rána, spálíte ohněm.
#12:11 Budete jej jíst takto: Budete mít přepásaná bedra, opánky na nohou a hůl v ruce. Sníte jej ve chvatu. To bude Hospodinův hod beránka.
#12:12 Tu noc projdu egyptskou zemí a všecko prvorozené v egyptské zemi pobiji, od lidí až po dobytek. Všechna egyptská božstva postihnu svými soudy. Já jsem Hospodin.
#12:13 Na domech, v nichž budete, budete mít na znamení krev. Když tu krev uvidím, pominu vás a nedolehne na vás zhoubný úder, až budu bít egyptskou zemi.
#12:14 Ten den vám bude dnem pamětním, budete jej slavit jako slavnost Hospodinovu. Budete jej slavit po všechna svá pokolení. To je provždy platné nařízení.
#12:15 Po sedm dní budete jíst nekvašené chleby. Hned prvního dne odstraníte ze svých domů kvas. Každý, kdo by od prvního do sedmého dne jedl něco kvašeného, bude z Izraele vyobcován.
#12:16 Prvního dne budete mít bohoslužebné shromáždění. I sedmého dne budete mít bohoslužebné shromáždění. V těch dnech se nebude konat žádné dílo. Smíte si připravit jen to, co každý potřebuje k jídlu.
#12:17 Budete dbát na ustanovení o nekvašených chlebech, neboť právě toho dne jsem vyvedl vaše oddíly z egyptské země. Na tento den budete bedlivě dbát. To je provždy platné nařízení pro všechna vaše pokolení.
#12:18 Od večera čtrnáctého dne prvního měsíce budete jíst nekvašené chleby až do večera jedenadvacátého dne téhož měsíce.
#12:19 Po sedm dní se nenajde ve vašich domech kvas. Každý, kdo by jedl něco kvašeného, bude vyobcován z pospolitosti Izraele, i host a domorodec.
#12:20 Nebudete jíst nic kvašeného. Ve všech svých obydlích budete jíst nekvašené chleby.“
#12:21 Mojžíš svolal všechny izraelské starší a řekl jim: „Jděte si vzít kus z bravu podle vašich čeledí a zabijte velikonočního beránka.
#12:22 Potom vezměte svazek yzopu, namočte jej v misce s krví a krví z misky potřete nadpraží a obě veřeje. Ať nikdo z vás až do rána nevychází ze dveří svého domu.
#12:23 Až Hospodin bude procházet zemí, aby udeřil na Egypt, uvidí krev na nadpraží a na obou veřejích. Hospodin ty dveře pomine a nedopustí, aby do vašeho domu vešel zhoubce a udeřil na vás.
#12:24 Dbejte na toto ustanovení. To je provždy platné nařízení pro tebe i pro tvé syny.
#12:25 Až přijdete do země, kterou vám Hospodin dá, jak přislíbil, dbejte na tuto službu.
#12:26 Až se vás pak vaši synové budou ptát, co pro vás tato služba znamená,
#12:27 odpovíte: ‚Je to velikonoční obětní hod Hospodinův. On v Egyptě pominul domy synů Izraele. Když udeřil na Egypt, naše domy vysvobodil.‘“ Lid padl na kolena a klaněl se.
#12:28 Izraelci pak odešli a učinili přesně tak, jak Hospodin Mojžíšovi a Áronovi přikázal.
#12:29 Když nastala půlnoc, pobil Hospodin v egyptské zemi všechno prvorozené, od prvorozeného syna faraónova, který seděl na jeho trůnu, až po prvorozeného syna zajatce v žalářní kobce, i všechno prvorozené z dobytka.
#12:30 Tu farao v noci vstal, i všichni jeho služebníci a všichni Egypťané, a v celém Egyptě nastal veliký křik, protože nebylo domu, kde by nebyl mrtvý.
#12:31 Ještě v noci povolal Mojžíše a Árona a řekl: „Seberte se a odejděte z mého lidu, vy i Izraelci. Jděte, služte Hospodinu, jak jste žádali.
#12:32 Vezměte také svůj brav i skot, jak jste žádali, a jděte. Vyproste požehnání i pro mne.“
#12:33 Egypťané naléhali na lid a spěchali s jeho propuštěním ze země, protože si říkali: „Všichni pomřeme!“
#12:34 Lid tedy vzal těsto ještě nevykynuté, zabalil díže do plášťů a nesl na ramenou.
#12:35 Izraelci jednali podle Mojžíšova rozkazu; vyžádali si též od Egypťanů stříbrné a zlaté šperky a pláště.
#12:36 A Hospodin zjednal lidu přízeň v očích Egypťanů a oni jim vyhověli. Tak vyplenili Egypt.
#12:37 Izraelci vytáhli z Ramesesu do Sukótu, kolem šesti set tisíc pěších mužů kromě dětí.
#12:38 Vyšlo s nimi také mnoho přimíšeného lidu a obrovská stáda bravu a skotu.
#12:39 Z těsta, které vynesli z Egypta, napekli nekvašené podpopelné chleby, protože ještě nevykynulo. Byli totiž z Egypta vyhnáni a nemohli otálet. Ani potravu na cestu si nestačili připravit.
#12:40 Doba pobytu, kterou Izraelci v Egyptě strávili, byla čtyři sta třicet let.
#12:41 Když uplynulo čtyři sta třicet let, přesně na den vyšly všechny Hospodinovy zástupy z egyptské země.
#12:42 Byla to noc jejich bdění pro Hospodina, když je vyváděl z egyptské země. Tato noc je všem synům Izraele po všechna pokolení nocí bdění pro Hospodina.
#12:43 Hospodin řekl Mojžíšovi a Áronovi: „Toto je nařízení o hodu beránka: Nebude z něho jíst žádný cizinec.
#12:44 Ale bude jej jíst každý služebník koupený za stříbro, bude-li obřezán.
#12:45 Přistěhovalec ani nádeník jej jíst nebude.
#12:46 Musí být sněden v témž domě. Z jeho masa nevyneseš nic z domu; žádnou jeho kost nezlámete.
#12:47 Tak to bude dělat celá izraelská pospolitost.
#12:48 Jestliže by u tebe pobýval host a chtěl by připravit Hospodinu hod beránka, nechť je u něho obřezán každý mužského pohlaví a potom bude smět přistoupit a tak učinit a bude jako domorodec v zemi. Ale žádný neobřezanec jej jíst nebude.
#12:49 Stejný řád bude platit pro domorodce i pro hosta, který bude pobývat mezi vámi.“
#12:50 Všichni Izraelci učinili přesně tak, jak Hospodin Mojžíšovi a Áronovi přikázal.
#12:51 Právě v ten den vyvedl Hospodin Izraelce seřazené po oddílech z egyptské země. 
#13:1 Hospodin promluvil k Mojžíšovi:
#13:2 „Posvěť mi všechno prvorozené, co mezi Izraelci otvírá lůno, ať z lidí či z dobytka. Je to moje!“
#13:3 Mojžíš řekl lidu: „Pamatujte na tento den, kdy jste vyšli z Egypta, z domu otroctví. Hospodin vás odtud vyvedl pevnou rukou. Proto se nesmí jíst nic kvašeného.
#13:4 Vycházíte dnes, v měsíci ábíbu.
#13:5 Až tě Hospodin uvede do země Kenaanců, Chetejců, Emorejců, Chivejců a Jebúsejců, o níž se zavázal tvým otcům přísahou, že ji dá tobě, zemi oplývající mlékem a medem, budeš v ní tohoto měsíce konat tuto službu:
#13:6 Sedm dní budeš jíst nekvašené chleby. Sedmého dne bude slavnost Hospodinova.
#13:7 Nekvašené chleby se budou jíst po sedm dní. Nespatří se u tebe nic kvašeného, na celém tvém území se u tebe nespatří žádný kvas.
#13:8 V onen den svému synovi oznámíš: ‚To je pro to, co mi prokázal Hospodin, když jsem vycházel z Egypta‘.
#13:9 A budeš to mít jako znamení na své ruce a jako připomínku mezi svýma očima, aby v tvých ústech zůstal Hospodinův zákon, neboť pevnou rukou tě vyvedl Hospodin z Egypta.
#13:10 Budeš dbát na toto nařízení ve stanovený čas rok co rok.
#13:11 Až tě Hospodin uvede do země Kenaanců, jak přísežně zaslíbil tobě i tvým otcům, a až ti ji dá,
#13:12 všechno, co otvírá lůno, odevzdáš Hospodinu. Všichni samečci, které tvůj dobytek vrhne nejprve, budou patřit Hospodinu.
#13:13 Každého osla, který se narodil jako první, vyplatíš jehnětem. Kdybys jej nemohl vyplatit, zlomíš mu vaz. Také každého prvorozeného ze svých synů vyplatíš.
#13:14 Až se tě tvůj syn v budoucnu zeptá, co to znamená, odpovíš mu: ‚Hospodin nás vyvedl pevnou rukou z Egypta, z domu otroctví.
#13:15 Když se farao zatvrdil a nechtěl nás propustit, pobil Hospodin v egyptské zemi všechno prvorozené, od prvorozeného z lidí až po prvorozené z dobytka. Proto obětuji Hospodinu všechny samce, kteří otvírají lůno, a každého prvorozeného ze svých synů vyplácím.‘
#13:16 To bude jako znamení na tvé ruce a jako pásek na čele mezi tvýma očima. Neboť pevnou rukou nás vyvedl Hospodin z Egypta.“
#13:17 Když farao lid propustil, nevedl je Bůh cestou směřující do země Pelištejců, i když byla kratší. Bůh totiž řekl: „Aby lid nelitoval, když uvidí, že mu hrozí válka, a nevrátil se do Egypta.“
#13:18 Proto Bůh vedl lid oklikou, cestou přes poušť k Rákosovému moři. Izraelci vytáhli z egyptské země rozděleni do bojových útvarů.
#13:19 Mojžíš vzal s sebou Josefovy kosti. Ten totiž zavázal Izraelce přísahou: „Až vás Bůh navštíví, vynesete odtud s sebou mé kosti.“
#13:20 I vytáhli ze Sukótu a utábořili se v Étamu na pokraji pouště.
#13:21 Hospodin šel před nimi ve dne v sloupu oblakovém, a tak je cestou vedl, v noci ve sloupu ohnivém, a tak jim svítil, že mohli jít ve dne i v noci.
#13:22 Sloup oblakový se nevzdálil od lidu ve dne, ani sloup ohnivý v noci. 
#14:1 Hospodin promluvil k Mojžíšovi:
#14:2 „Rozkaž Izraelcům, aby se obrátili a utábořili před Pí-chírotem mezi Migdólem a mořem; utáboříte se před Baal-sefónem, přímo proti němu při moři.
#14:3 Farao si o Izraelcích řekne: Bloudí v zemi, zavřela se za nimi poušť.
#14:4 Tu zatvrdím faraónovo srdce a on vás bude pronásledovat. Já se však na faraónovi a na všem jeho vojsku oslavím, takže Egypťané poznají, že já jsem Hospodin.“ I učinili tak.
#14:5 Egyptskému králi bylo oznámeno, že lid uprchl. Srdce faraóna a jeho služebníků se obrátilo proti lidu. Řekli: „Co jsme to udělali, že jsme Izraele propustili z otroctví?“
#14:6 Farao dal zapřáhnout do svého válečného vozu a vzal s sebou svůj lid.
#14:7 Vzal též šest set vybraných vozů, totiž všechny vozy egyptské. Na všech byla tříčlenná osádka.
#14:8 Hospodin zatvrdil srdce faraóna, krále egyptského, a ten Izraelce pronásledoval. Ale Izraelci navzdory všemu vyšli.
#14:9 Egypťané je pronásledovali a dostihli je, když tábořili při moři, dostihli je všichni faraónovi koně, vozy, jeho jízda a vojsko, při Pí-chírotu před Baal-sefónem.
#14:10 Když se farao přiblížil, Izraelci se rozhlédli a viděli, že Egypťané táhnou za nimi. Tu se Izraelci velmi polekali a úpěli k Hospodinu.
#14:11 A osopili se na Mojžíše: „Což nebylo v Egyptě dost hrobů, že jsi nás odvedl, abychom zemřeli na poušti? Cos nám to udělal, že jsi nás vyvedl z Egypta?
#14:12 Došlo na to, o čem jsme s tebou mluvili v Egyptě: Nech nás být, ať sloužíme Egyptu. Vždyť pro nás bylo lépe sloužit Egyptu než zemřít na poušti.“
#14:13 Mojžíš řekl lidu: „Nebojte se! Vydržte a uvidíte, jak vás dnes Hospodin zachrání. Jak vidíte Egypťany dnes, tak je už nikdy neuvidíte.
#14:14 Hospodin bude bojovat za vás a vy budete mlčky přihlížet.“
#14:15 Hospodin řekl Mojžíšovi: „Proč ke mně úpíš? Pobídni Izraelce, ať táhnou dál.
#14:16 Ty pak pozdvihni svou hůl, vztáhni ruku nad moře a rozpoltíš je, a tak Izraelci půjdou prostředkem moře po suchu.
#14:17 Já zatvrdím srdce Egypťanů, takže půjdou za nimi. Oslavím se na faraónovi a na všem jeho vojsku, na jeho vozech i jízdě.
#14:18 Egypťané poznají, že já jsem Hospodin, až budu oslaven tím, co učiním s faraónem, s jeho vozy a jízdou.“
#14:19 Tu se zvedl Boží posel, který šel před izraelským táborem, a šel teď za nimi. Oblakový sloup se před nimi totiž zvedl, postavil se za ně
#14:20 a vstoupil mezi tábor egyptský a izraelský. Jedněm byl oblakem a temnotou, druhým osvěcoval noc; po celou noc se jedni k druhým nepřiblížili.
#14:21 Mojžíš vztáhl ruku nad moře a Hospodin hnal moře silným východním větrem, který vál po celou noc, až proměnil moře v souš. Vody byly rozpolceny.
#14:22 Izraelci šli prostředkem moře po suchu. Vody jim byly hradbou zprava i zleva.
#14:23 Egypťané je pronásledovali a vešli za nimi doprostřed moře, všichni faraónovi koně, vozy i jízda.
#14:24 Za jitřního bdění vyhlédl Hospodin ze sloupu ohnivého a oblakového na egyptský tábor a vyvolal v egyptském táboře zmatek.
#14:25 Způsobil, že se uvolnila kola jejich vozů, takže je stěží mohli ovládat. Tu si Egypťané řekli: „Utecme před Izraelem, neboť za ně bojuje proti Egyptu Hospodin.“
#14:26 Hospodin řekl Mojžíšovi: „Vztáhni ruku nad moře! Vody se obrátí na Egypťany, na jejich vozy a jízdu.“
#14:27 Mojžíš vztáhl ruku nad moře, a když nastávalo jitro, moře opět nabylo své moci. Egypťané utíkali proti němu a Hospodin je vehnal doprostřed moře.
#14:28 Vody se vrátily, přikryly vozy i jízdu celého faraónova vojska, které vešlo za Izraelci do moře. Nezůstal z nich ani jediný.
#14:29 Ale Izraelci přešli prostředkem moře po suchu a vody jim byly hradbou zprava i zleva.
#14:30 Onoho dne zachránil Hospodin Izraele z moci Egypta. Izrael viděl na břehu moře mrtvé Egypťany.
#14:31 Tak uviděl Izrael velikou moc, kterou osvědčil Hospodin na Egyptu. Lid se bál Hospodina a uvěřili Hospodinu i jeho služebníku Mojžíšovi. 
#15:1 Tehdy zpíval Mojžíš a synové Izraele Hospodinu tuto píseň. Vyznávali: „Hospodinu chci zpívat, neboť se slavně vyvýšil, smetl do moře koně i s jezdcem.
#15:2 Hospodin je má záštita a píseň, stal se mou spásou. On je můj Bůh, a já ho velebím, Bůh mého otce, a já ho vyvyšuji.
#15:3 Hospodin je bojovný rek; Hospodin je jeho jméno.
#15:4 Vozy faraónovy i jeho vojsko svrhl v moře, v moři Rákosovém utonul výkvět jeho vozatajstva.
#15:5 Tůně propastné je zavalily, klesli do hlubin jak kámen.
#15:6 Tvá pravice, Hospodine, velkolepá v síle, tvá pravice, Hospodine, zdrtí nepřítele.
#15:7 Nesmírně vyvýšen rozmetáš útočníky, vysíláš své rozhorlení, jako oheň strniště je pozře.
#15:8 Dechem tvého chřípí počaly se kupit vody, příboje zůstaly stát jako hráze, sesedly se tůně propastné v klín moře.
#15:9 Nepřítel si řekl: ‚Pustím se za nimi, doženu je, rozdělím kořist, ukojím jimi svou duši, meč vytasím, podrobí si je má ruka.‘
#15:10 Zadul jsi svým dechem a moře je zavalilo, potopili se jak olovo v nesmírných vodách.
#15:11 Kdo je mezi bohy jako ty, Hospodine? Kdo je jako ty, tak velkolepý ve svatosti, hrozný v chvályhodných skutcích, konající divy?
#15:12 Vztáhl jsi pravici a země je pohltila.
#15:13 Svým milosrdenstvím jsi vedl tento lid, který jsi vykoupil, provázel jsi jej svou mocí ke své svaté nivě.
#15:14 Uslyšely o tom národy a zmocnil se jich neklid, bolest sevřela obyvatele Pelišteje.
#15:15 Tehdy se zhrozili edómští pohlaváři, moábské vůdce zachvátilo chvění, všichni obyvatelé Kenaanu propadli zmatku.
#15:16 Padla na ně hrůza a strach; pro velikost tvé paže zmlknou jako kámen, dokud, Hospodine, neprojde tvůj lid, dokud neprojde ten lid, který sis získal.
#15:17 Přivedeš a zasadíš je na hoře svého dědictví, kde jsi, Hospodine, připravil své sídlo k přebývání, kde tvé ruce, Panovníku, svatyni si přichystaly.
#15:18 Hospodin kraluje navěky a navždy.“
#15:19 Když totiž faraónovi koně s jeho vozy a jízdou vešli do moře, Hospodin na ně obrátil mořské vody. Ale Izraelci šli po suchu prostředkem moře.
#15:20 Tu vzala prorokyně Mirjam, sestra Áronova, do ruky bubínek a všechny ženy vyšly za ní s bubínky v tanečním reji.
#15:21 A Mirjam střídavě s muži prozpěvovala: „Zpívejte Hospodinu, neboť se slavně vyvýšil, smetl do moře koně i s jezdcem.“
#15:22 Mojžíš vedl Izraele od Rákosového moře dál. Vyšli na poušť Šúr a táhli pouští po tři dny, aniž narazili na vodu.
#15:23 Došli až do Mary, ale nemohli vodu z Mary pít, protože byla hořká. Pojmenovali ji proto Mara (to je Hořká).
#15:24 Tu lid proti Mojžíšovi reptal: „Co budeme pít?“
#15:25 Mojžíš úpěl k Hospodinu a Hospodin mu ukázal dřevo. Když je hodil do vody, voda zesládla. Tam dal Hospodin lidu nařízení a právní ustanovení a podrobil jej tam zkoušce.
#15:26 Řekl: „Jestliže opravdu budeš poslouchat Hospodina, svého Boha, dělat, co je v jeho očích správné, naslouchat jeho přikázáním a dbát na všechna jeho nařízení, nepostihnu tě žádnou nemocí, kterou jsem postihl Egypt. Neboť já jsem Hospodin, já tě uzdravuji.“
#15:27 Pak přišli do Élimu. Tam bylo dvanáct vodních pramenů a sedmdesát palem. Tam při vodách se utábořili. 
#16:1 Pak vytáhli z Élimu. Celá pospolitost Izraelců přišla na poušť Sín, která je mezi Élimem a Sínajem, patnáctý den druhého měsíce poté, co vyšli z egyptské země.
#16:2 Celá pospolitost Izraelců na poušti opět reptala proti Mojžíšovi a Áronovi.
#16:3 Izraelci jim vyčítali: „Kéž bychom byli zemřeli Hospodinovou rukou v egyptské zemi, když jsme sedávali nad hrnci masa, když jsme jídávali chléb do sytosti. Vždyť jste nás vyvedli na tuto poušť, jen abyste celé toto shromáždění umořili hladem.“
#16:4 Hospodin řekl Mojžíšovi: „Já vám sešlu chléb jako déšť z nebe. Ať lid vychází a sbírá, co denně spotřebují. Tak je podrobím zkoušce, budou-li se řídit mým zákonem, či nikoli.
#16:5 Když budou připravovat, co přinesou, ať je toho šestého dne dvakrát tolik, než co nasbírají každodenně.“
#16:6 Mojžíš a Áron řekli všem Izraelcům: „Večer poznáte, že vás z egyptské země vyvedl Hospodin.
#16:7 A ráno spatříte Hospodinovu slávu, ačkoli slyšel vaše reptání proti sobě. Co jsme my, že reptáte proti nám?“
#16:8 Pak Mojžíš dodal: „Poznáte to podle toho, že vám Hospodin dá večer k jídlu maso a ráno k nasycení chléb, ačkoli slyšel reptání, jak jste proti němu reptali. Co jsme my? Nereptáte proti nám, ale proti Hospodinu.“
#16:9 Áronovi Mojžíš řekl: „Vyzvi celou pospolitost Izraelců: ‚Přistupte před Hospodina, neboť slyšel vaše reptání.‘“
#16:10 Když mluvil Áron k celé pospolitosti Izraelců, obrátili se k poušti, a vtom se ukázala v oblaku Hospodinova sláva.
#16:11 Tu Hospodin promluvil k Mojžíšovi:
#16:12 „Slyšel jsem reptání Izraelců. Vyhlas jim: ‚Navečer se najíte masa a ráno se nasytíte chlebem, abyste poznali, že já jsem Hospodin, váš Bůh.‘“
#16:13 Když pak nastal večer, přiletěly křepelky a snesly se na tábor. A ráno padala kolem tábora rosa.
#16:14 Když rosa přestala padat, hle, na povrchu pouště leželo po zemi cosi jemně šupinatého, jemného jako jíní.
#16:15 Když to Izraelci viděli, říkali jeden druhému: „Man hú?“ (to je: „Co je to?„) Nevěděli totiž, co to je. Mojžíš jim řekl: „To je chléb, který vám dal Hospodin za pokrm.
#16:16 Hospodin přikázal toto: Nasbírejte si ho každý, kolik potřebujete k jídlu. Každý vezmete podle počtu osob ve svém stanu ómer na hlavu.“
#16:17 Izraelci tak učinili a nasbírali někdo více, někdo méně.
#16:18 Pak odměřovali po ómeru. Ten, kdo nasbíral mnoho, neměl nadbytek, a kdo nasbíral málo, neměl nedostatek. Nasbírali tolik, kolik každý k jídlu potřeboval.
#16:19 Mojžíš jim řekl: „Nikdo ať si nenechává nic do rána!“
#16:20 Ale oni Mojžíše neposlechli a někteří si něco do rána nechali. To však zčervivělo a páchlo. Mojžíš se na ně rozlítil.
#16:21 Sbírali to tak ráno co ráno, kolik každý k jídlu potřeboval. Když však začalo hřát slunce, rozpustilo se to.
#16:22 Šestého dne nasbírali toho chleba dvakrát tolik, totiž dva ómery na osobu. Tu přišli všichni předáci pospolitosti a oznámili to Mojžíšovi.
#16:23 Ten jim řekl: „Toto praví Hospodin: Zítra je slavnost odpočinutí, Hospodinův svatý den odpočinku. Co je třeba napéci, napečte, a co je třeba uvařit, uvařte. A vše, co přebývá, uložte a opatrujte do rána.“
#16:24 Uložili to tedy do rána, jak Mojžíš přikázal. A nezapáchalo to, ani se do toho nedali červi.
#16:25 Mojžíš pak řekl: „Snězte to dnes, protože dnes je Hospodinův den odpočinku. Dnes nenajdete na poli nic.
#16:26 Šest dní budete sbírat, ale sedmý den je den odpočinku. Ten den nebude nic padat.“
#16:27 Když přesto někteří z lidu sedmého dne vyšli, aby sbírali, nic nenašli.
#16:28 Hospodin řekl Mojžíšovi: „Jak dlouho se budete zpěčovat a nebudete dbát mých příkazů a řádů?
#16:29 Hleďte, vždyť Hospodin vám dal den odpočinku. Proto vám dává šestého dne chléb na dva dny. Zůstaňte každý, kde jste, ať nikdo sedmého dne nevychází ze svého místa.“
#16:30 Lid tedy sedmého dne odpočíval.
#16:31 Dům izraelský pojmenoval ten pokrm mana. Byl jako koriandrové semeno, bílý, a chutnal jako medový koláč.
#16:32 Mojžíš řekl: „Hospodin přikázal toto: Naplň tím ómer, aby to bylo opatrováno po všechna vaše pokolení, aby viděla chléb, kterým jsem vás na poušti živil, když jsem vás vyvedl z egyptské země.“
#16:33 Áronovi Mojžíš řekl: „Vezmi jeden džbán, nasyp do něho plný ómer many a ulož to před Hospodinem, aby to bylo opatrováno po všechna vaše pokolení.“
#16:34 Áron to tedy uložil před schránou svědectví, aby to bylo opatrováno, jak Hospodin Mojžíšovi přikázal.
#16:35 Izraelci jedli manu po čtyřicet let, dokud nepřišli do země, v níž se měli usadit; jedli manu, dokud nepřišli na pokraj kenaanské země.
#16:36 Ómer je desetina éfy. 
#17:1 Celá pospolitost synů Izraele táhla z pouště Sínu od stanoviště ke stanovišti podle Hospodinova rozkazu. Utábořili se v Refídimu, ale lid neměl vodu k pití.
#17:2 Tu se lid dostal do sváru s Mojžíšem a naléhali: „Dejte nám vodu, chceme pít!“ Mojžíš se jich zeptal: „Proč se se mnou přete? Proč pokoušíte Hospodina?“
#17:3 Lid tam žíznil po vodě a reptal proti Mojžíšovi. Vyčítali: „Proto jsi nás vyvedl z Egypta, abys nás, naše syny a stáda umořil žízní?“
#17:4 Mojžíš úpěl k Hospodinu: „Jak se mám vůči tomuto lidu zachovat? Taktak že mě neukamenují.“
#17:5 Hospodin Mojžíšovi řekl: „Vyjdi před lid. Vezmi s sebou některé z izraelských starších. Také hůl, kterou jsi udeřil do Nilu, si vezmi do ruky a jdi.
#17:6 Já tam budu stát před tebou na skále na Chorébu. Udeříš do skály a vyjde z ní voda, aby lid mohl pít.“ Mojžíš to udělal před očima izraelských starších.
#17:7 To místo pojmenoval Massa a Meriba (to je Pokušení a Svár) podle sváru Izraelců a proto, že pokoušeli Hospodina pochybováním: „Je mezi námi Hospodin nebo není?“
#17:8 Tu přitáhl Amálek, aby v Refídimu bojoval s Izraelem.
#17:9 Mojžíš rozkázal Jozuovi: „Vyber nám muže a vyjdi do boje proti Amálekovi. Já se zítra postavím na vrchol pahorku s Hospodinovou holí v ruce.“
#17:10 Jozue učinil, jak mu Mojžíš rozkázal, a dal se s Amálekem do boje. Mojžíš, Áron a Chúr vystoupili na vrchol pahorku.
#17:11 Dokud Mojžíš držel ruku nahoře, vítězil Izrael, když ruku spustil, vítězil Amálek.
#17:12 Když Mojžíšovi umdlévaly ruce, vzali kámen a podložili jej pod Mojžíše, aby se na něj posadil. Áron a Chúr, každý z jedné strany, mu podpírali ruce, takže vytrval s rukama nahoře až do západu slunce.
#17:13 I porazil Jozue Amáleka a jeho lid ostřím meče.
#17:14 Hospodin řekl Mojžíšovi: „Zapiš na památku do knihy a předej Jozuovi, že zcela vymažu zpod nebes památku na Amáleka.“
#17:15 I vybudoval Mojžíš oltář a pojmenoval jej: ‚Hospodin je má korouhev.‘
#17:16 Řekl totiž: „Je vztažena ruka nad Hospodinovým trůnem. Hospodin vyhlašuje boj proti Amálekovi do posledního pokolení.“ 
#18:1 Jitro, midjánský kněz, Mojžíšův tchán, uslyšel o všem, co Bůh učinil Mojžíšovi a svému izraelskému lidu, že Hospodin vyvedl Izraele z Egypta.
#18:2 Tu vzal Jitro, Mojžíšův tchán, Siporu, Mojžíšovu manželku, kterou on poslal zpět,
#18:3 a dva její syny; první se jmenoval Geršom (to je Hostem-tam), neboť Mojžíš řekl: „Byl jsem hostem v cizí zemi“,
#18:4 a druhý se jmenoval Elíezer (to je Bůh-je-pomoc), neboť řekl: „Bůh mého otce je má pomoc, vysvobodil mě od faraónova meče“.
#18:5 Jitro, Mojžíšův tchán, přišel k němu s jeho syny a manželkou na poušť, kde on tábořil, k hoře Boží.
#18:6 Vzkázal Mojžíšovi: „Já, tvůj tchán Jitro, jsem přišel k tobě, i tvoje manželka a s ní oba její synové.“
#18:7 Mojžíš tedy vyšel svému tchánovi vstříc, poklonil se a políbil ho. Popřáli si navzájem pokoje a vešli do stanu.
#18:8 Mojžíš vypravoval svému tchánovi o všem, co Hospodin kvůli Izraeli učinil faraónovi a Egypťanům, a o všech útrapách, které je potkaly na cestě, a jak je Hospodin vysvobodil.
#18:9 Jitro měl radost ze všeho dobrého, co Hospodin Izraeli prokázal, a že jej vysvobodil z moci Egypta.
#18:10 Řekl: „Požehnán buď Hospodin, že vás vysvobodil z moci Egypta a z ruky faraónovy, že vysvobodil tento lid z područí Egypta.
#18:11 Nyní jsem poznal, že Hospodin je větší než všichni bohové; odplatil jim podle toho, jak se vypínali nad Izraele.“
#18:12 Jitro, tchán Mojžíšův, pak připravil Bohu zápalnou oběť a obětní hod. Áron a všichni izraelští starší přistoupili, aby s Mojžíšovým tchánem pojedli před Bohem chléb.
#18:13 Nazítří se Mojžíš posadil, aby soudil lid. Lid musel stát před Mojžíšem od rána do večera.
#18:14 Mojžíšův tchán se díval na celé jeho jednání s lidem a řekl: „Jakým způsobem to s lidem jednáš? Proč sám sedíš a všechen lid kolem tebe stojí od rána do večera?“
#18:15 Mojžíš tchánovi odpověděl: „Lid ke mně přichází dotazovat se Boha.
#18:16 Když něco mají, přijde ta záležitost přede mne a já rozsoudím mezi oběma stranami; učím je znát Boží nařízení a řády.“
#18:17 Mojžíšův tchán mu odpověděl: „Není to dobrý způsob, jak to děláš.
#18:18 Úplně se vyčerpáš, stejně jako tento lid, který je s tebou. Je to pro tebe příliš obtížné. Sám to nezvládneš.
#18:19 Poslechni mě, poradím ti a Bůh bude s tebou: Ty zastupuj lid před Bohem a přednášej jejich záležitosti Bohu.
#18:20 Budeš jim vysvětlovat nařízení a řády a učit je znát cestu, po které mají chodit, i skutky, které mají činit.
#18:21 Vyhlédni si pak ze všeho lidu schopné muže, kteří se bojí Boha, muže důvěryhodné, kteří nenávidí úplatek. Dosaď je nad nimi za správce nad tisíci, sty, padesáti a deseti.
#18:22 Oni budou soudit lid, kdykoli bude třeba. Každou důležitou záležitost přednesou tobě, každou menší záležitost rozsoudí sami. Ulehči si své břímě, ať je nesou s tebou.
#18:23 Jestliže se podle toho zařídíš, budeš moci obstát, až ti Bůh vydá další příkazy. Také všechen tento lid dojde na své místo v pokoji.“
#18:24 Mojžíš svého tchána uposlechl a učinil všechno, co řekl.
#18:25 Vybral schopné muže ze všeho Izraele a ustanovil je za představitele lidu, za správce nad tisíci, sty, padesáti a deseti.
#18:26 Ti soudili lid, kdykoli bylo třeba; obtížné záležitosti přednášeli Mojžíšovi a všechny menší záležitosti soudili sami.
#18:27 Potom Mojžíš svého tchána propustil a ten se ubíral do své země. 
#19:1 Třetího měsíce potom, co Izraelci vyšli z egyptské země, téhož dne, přišli na Sínajskou poušť.
#19:2 Vytáhli z Refídimu, přišli na Sínajskou poušť a utábořili se v poušti; Izrael se tam utábořil naproti hoře.
#19:3 Mojžíš vystoupil k Bohu. Hospodin k němu zavolal z hory: „Toto povíš domu Jákobovu a oznámíš synům Izraele:
#19:4 Vy sami jste viděli, co jsem učinil Egyptu. Nesl jsem vás na orlích křídlech a přivedl vás k sobě.
#19:5 Nyní tedy, budete-li mě skutečně poslouchat a dodržovat mou smlouvu, budete mi zvláštním vlastnictvím jako žádný jiný lid, třebaže má je celá země.
#19:6 Budete mi královstvím kněží, pronárodem svatým. To jsou slova, která promluvíš k synům Izraele.“
#19:7 Mojžíš přišel, zavolal starší lidu a předložil jim všechno, co mu Hospodin přikázal.
#19:8 Všechen lid odpověděl jednomyslně: „Budeme dělat všechno, co nám Hospodin uložil.“ Mojžíš tlumočil odpověď lidu Hospodinu.
#19:9 Hospodin řekl Mojžíšovi: „Hle, přijdu k tobě v hustém oblaku, aby lid slyšel, až s tebou budu mluvit, a aby ti provždy věřili.“ Mojžíš totiž Hospodinu oznámil slova lidu.
#19:10 Hospodin dále Mojžíšovi řekl: „Jdi k lidu a dnes i zítra je posvěcuj; ať si vyperou pláště
#19:11 a ať jsou připraveni na třetí den, neboť třetího dne sestoupí Hospodin před zraky všeho lidu na horu Sínaj.
#19:12 Vymezíš kolem lidu hranici a řekneš: Střezte se vystoupit na horu nebo i dotknout se jejího okraje. Kdokoli se hory dotkne, musí zemřít;
#19:13 nedotkne se ho žádná ruka, bude ukamenován nebo zastřelen. Ať je to dobytče nebo člověk, nezůstane naživu. Teprve až se dlouze zatroubí na roh, smějí na horu vystoupit.“
#19:14 Mojžíš sestoupil z hory k lidu, posvětil lid a oni si vyprali pláště.
#19:15 Řekl také lidu: „Buďte připraveni na třetí den; nepřistupujte k ženě.“
#19:16 Když nadešel třetí den a nastalo jitro, hřmělo a blýskalo se, na hoře byl těžký oblak a zazněl velmi pronikavý zvuk polnice. Všechen lid, který byl v táboře, se třásl.
#19:17 Mojžíš vyvedl lid z tábora vstříc Bohu a postavili se při úpatí hory.
#19:18 Celá hora Sínaj byla zahalena kouřem, protože Hospodin na ni sestoupil v ohni. Kouř z ní stoupal jako z hutě a celá hora se silně chvěla.
#19:19 Zvuk polnice víc a více sílil. Mojžíš mluvil a Bůh mu hlasitě odpovídal.
#19:20 Hospodin totiž sestoupil na horu Sínaj, na vrchol hory. Zavolal Mojžíše na vrchol hory a Mojžíš tam vystoupil.
#19:21 Hospodin Mojžíšovi řekl: „Sestup a varuj lid, aby se nikdo nepokoušel proniknout k Hospodinu ve snaze ho uvidět. Mnoho by jich padlo.
#19:22 Také kněží, kteří přistupují k Hospodinu, se musí posvětit, aby se na ně Hospodin neobořil.“
#19:23 Mojžíš řekl Hospodinu: „Lid nemůže vystoupit na horu Sínaj, neboť ty sám jsi nás varoval slovy: Vymez podél hory hranici a horu posvěť.“
#19:24 Hospodin mu řekl: „Teď sestup, potom vystoupíš spolu s Áronem; ale kněží ani lid nesmějí proniknout a vystoupit k Hospodinu, aby se na ně neobořil.“
#19:25 Mojžíš tedy sestoupil k lidu a řekl jim to. 
#20:1 Bůh vyhlásil všechna tato přikázání:
#20:2 „Já jsem Hospodin, tvůj Bůh; já jsem tě vyvedl z egyptské země, z domu otroctví.
#20:3 Nebudeš mít jiného boha mimo mne.
#20:4 Nezobrazíš si Boha zpodobením ničeho, co je nahoře na nebi, dole na zemi nebo ve vodách pod zemí.
#20:5 Nebudeš se ničemu takovému klanět ani tomu sloužit. Já jsem Hospodin, tvůj Bůh, Bůh žárlivě milující. Stíhám vinu otců na synech do třetího i čtvrtého pokolení těch, kteří mě nenávidí,
#20:6 ale prokazuji milosrdenství tisícům pokolení těch, kteří mě milují a má přikázání zachovávají.
#20:7 Nezneužiješ jména Hospodina, svého Boha. Hospodin nenechá bez trestu toho, kdo by jeho jména zneužíval.
#20:8 Pamatuj na den odpočinku, že ti má být svatý.
#20:9 Šest dní budeš pracovat a dělat všechnu svou práci.
#20:10 Ale sedmý den je den odpočinutí Hospodina, tvého Boha. Nebudeš dělat žádnou práci ani ty ani tvůj syn a tvá dcera ani tvůj otrok a tvá otrokyně ani tvé dobytče ani tvůj host, který žije v tvých branách.
#20:11 V šesti dnech učinil Hospodin nebe i zemi, moře a všechno, co je v nich, a sedmého dne odpočinul. Proto požehnal Hospodin den odpočinku a oddělil jej jako svatý.
#20:12 Cti svého otce i matku, abys byl dlouho živ na zemi, kterou ti dává Hospodin, tvůj Bůh.
#20:13 Nezabiješ.
#20:14 Nesesmilníš.
#20:15 Nepokradeš.
#20:16 Nevydáš proti svému bližnímu křivé svědectví.
#20:17 Nebudeš dychtit po domě svého bližního. Nebudeš dychtit po ženě svého bližního ani po jeho otroku ani po jeho otrokyni ani po jeho býku ani po jeho oslu, vůbec po ničem, co patří tvému bližnímu.“
#20:18 Všechen lid pozoroval hřmění a blýskání, zvuk polnice a kouřící se horu. Lid to pozoroval, chvěl se a zůstal stát opodál.
#20:19 Řekli Mojžíšovi: „Mluv s námi ty a budeme poslouchat. Bůh ať s námi nemluví, abychom nezemřeli.“
#20:20 Mojžíš lidu odpověděl: „Nebojte se! Bůh přišel proto, aby vás vyzkoušel, aby bylo zřejmé, že se ho budete bát a přestanete hřešit.“
#20:21 Lid zůstal stát opodál a Mojžíš přistoupil k mračnu, v němž byl Bůh.
#20:22 Hospodin řekl Mojžíšovi: „Toto řekneš synům Izraele: Viděli jste, že jsem s vámi mluvil z nebe.
#20:23 Neuděláte si mé zpodobení, neuděláte si bohy stříbrné ani zlaté.
#20:24 Uděláš mi oltář z hlíny a budeš na něm obětovat ze svého bravu a skotu své oběti zápalné i pokojné. Na každém místě, kde určím, aby se připomínalo mé jméno, přijdu k tobě a požehnám ti.
#20:25 Jestliže mi budeš dělat oltář z kamenů, neotesávej je; kdybys je opracoval dlátem, znesvětil bys je.
#20:26 Nebudeš vystupovat k mému oltáři po stupních, abys u něho neodkrýval svou nahotu.“ 
#21:1 Toto jsou právní ustanovení, která jim předložíš:
#21:2 Když koupíš hebrejského otroka, bude sloužit šest let; sedmého roku odejde jako propuštěnec bez výkupného.
#21:3 Jestliže přišel sám, odejde sám, měl-li ženu, odejde jeho žena s ním.
#21:4 Jestliže mu dal jeho pán ženu, která mu porodila syny nebo dcery, zůstane žena a její děti u svého pána, a on odejde sám.
#21:5 Prohlásí-li otrok výslovně: „Zamiloval jsem si svého pána, svou ženu a syny, nechci odejít jako propuštěnec“,
#21:6 přivede ho jeho pán před Boha, totiž přivede ho ke dveřím nebo k veřejím, probodne mu ucho šídlem a on zůstane provždy jeho otrokem.
#21:7 Když někdo prodá svou dceru za otrokyni, nebude s ní nakládáno jako s jinými otroky.
#21:8 Jestliže se znelíbí svému pánu, který si ji vzal za družku, dovolí ji vyplatit, ale nemá právo prodat ji cizímu lidu a naložit s ní věrolomně.
#21:9 Jestliže ji dal za družku svému synovi, bude s ní jednat podle práva dcer.
#21:10 Jestliže on si vezme ještě jinou, nesmí ji zkrátit na stravě, ošacení a manželském právu.
#21:11 Jestliže jí nezajistí tyto tři věci, smí ona odejít bez zaplacení výkupného.
#21:12 Kdo někoho uhodí a ten zemře, musí zemřít.
#21:13 Neměl-li to v úmyslu, ale Bůh dopustil, aby to jeho ruka způsobila, určím ti místo, kam se uteče.
#21:14 Když se však někdo opováží lstivě zavraždit svého bližního, vezmeš ho i od mého oltáře, aby zemřel.
#21:15 Kdo uhodí svého otce nebo matku, musí zemřít.
#21:16 Kdo někoho ukradne, ať už jej prodá nebo jej u něho naleznou, musí zemřít.
#21:17 Kdo zlořečí svému otci nebo matce, musí zemřít.
#21:18 Když se muži dostanou do sporu a jeden druhého uhodí kamenem nebo pěstí, ale on nezemře, nýbrž je upoután na lůžko
#21:19 a zase vstane a může vycházet o holi, bude pachatel bez viny; poskytne pouze náhradu za jeho vyřazení z práce a zajistí mu léčení.
#21:20 Jestliže někdo uhodí svého otroka nebo otrokyni holí, takže mu zemřou pod rukou, musí být usmrcený pomstěn.
#21:21 Jestliže však vydrží den či dva, nebude pomstěn, neboť byl jeho majetkem.
#21:22 Když se muži budou rvát a udeří těhotnou ženu, takže potratí, ale nepřijde o život, musí pachatel zaplatit pokutu, jakou mu uloží muž té ženy; odevzdá ji prostřednictvím rozhodčích.
#21:23 Jestliže o život přijde, dáš život za život.
#21:24 Oko za oko, zub za zub, ruku za ruku, nohu za nohu,
#21:25 spáleninu za spáleninu, modřinu za modřinu, jizvu za jizvu.
#21:26 Když někdo udeří do oka svého otroka nebo otrokyni a vyrazí mu je, v náhradu za oko ho propustí na svobodu.
#21:27 Jestliže vyrazí zub svému otroku nebo otrokyni, v náhradu za zub ho propustí na svobodu.
#21:28 Když býk potrká muže nebo ženu, takže zemřou, musí být býk ukamenován a jeho maso se nesmí jíst; majitel býka však bude bez viny.
#21:29 Jestliže však jde o býka trkavého již od dřívějška a jeho majitel byl varován, ale nehlídal ho, a býk usmrtí muže nebo ženu, bude býk ukamenován a také jeho majitel zemře.
#21:30 Jestliže mu bude uloženo výkupné, dá jako výplatu za svůj život všechno, co mu bude uloženo.
#21:31 Jestliže býk potrká syna nebo dceru, bude s ním naloženo podle téhož právního ustanovení.
#21:32 Jestliže býk potrká otroka nebo otrokyni, dá majitel býka třicet šekelů stříbra jejich pánu a býk bude ukamenován.
#21:33 Když někdo odkryje nebo vyhloubí studnu a nepřikryje ji, takže do ní spadne býk nebo osel,
#21:34 majitel studny poskytne jeho majiteli náhradu ve stříbře a mrtvé zvíře bude patřit jemu.
#21:35 Když něčí býk utrká k smrti sousedova býka, živého býka prodají a rozdělí se na polovinu o stříbro i o mrtvé zvíře.
#21:36 Bylo-li však známo, že jde o býka trkavého od dřívějška a jeho majitel ho nehlídal, poskytne plnou náhradu, býka za býka, a mrtvé zvíře bude jeho.
#21:37 Když někdo ukradne býka nebo beránka a porazí jej nebo prodá, dá náhradou za býka pět kusů hovězího dobytka a za beránka čtyři ovce. 
#22:1 Jestliže je zloděj přistižen při vloupání a je zbit, takže zemře, nebude lpět krev na tom, kdo ho ubil.
#22:2 Jestliže se to však stane po východu slunce, krev na něm bude lpět. Zloděj musí poskytnout plnou náhradu. Jestliže nic nemá, bude prodán za hodnotu ukradeného.
#22:3 Jestliže se u něho vskutku nalezne to, co ukradl, živé, ať je to býk či osel nebo ovce, poskytne dvojnásobnou náhradu.
#22:4 Když někdo nechá spást cizí pole nebo vinici tím, že pustí svůj dobytek, aby se pásl na jiném poli, poskytne náhradu z nejlepšího, co je na jeho poli a na jeho vinici.
#22:5 Když vypukne oheň a zachvátí trní a sežehne požaté nebo ještě stojící obilí nebo pole, poskytne ten, kdo požár zavinil, plnou náhradu.
#22:6 Když někdo svěří svému bližnímu k opatrování stříbro nebo předměty a ty budou z domu toho muže ukradeny, poskytne zloděj, bude-li přistižen, dvojnásobnou náhradu.
#22:7 Jestliže zloděj nebude přistižen, bude majitel toho domu předveden před Boha, zda sám nevztáhl ruku po výtěžku práce svého bližního.
#22:8 Ve všech majetkových přestupcích, ať jde o býka, osla, ovci, plášť či cokoli ztraceného, o čem někdo řekne: „To je ono“, přijde záležitost obou před Boha; koho Bůh označí jako svévolníka, ten poskytne svému bližnímu dvojnásobnou náhradu.
#22:9 Když někdo někomu svěří do opatrování osla, býka, ovci nebo jakékoli dobytče a ono pojde nebo utrpí úraz nebo bude odehnáno, aniž to kdo viděl,
#22:10 rozhodne mezi oběma přísaha při Hospodinu, že nevztáhl ruku po výtěžku práce svého bližního; majitel zvířete to přijme a druhý nemusí poskytnout náhradu.
#22:11 Jestliže mu však bylo skutečně ukradeno, poskytne majiteli náhradu.
#22:12 Jestliže bylo vskutku rozsápáno, přinese je na svědectví; za rozsápané náhradu poskytovat nebude.
#22:13 Když si někdo vyžádá od svého bližního dobytče a ono utrpí úraz nebo uhyne, poskytne plnou náhradu, nebyl-li majitel přítomen.
#22:14 Jestliže majitel byl přítomen, nemusí poskytnout náhradu; jde-li o námezdného dělníka, jde škoda na vrub jeho mzdy.
#22:15 Když někdo svede pannu, která nebyla zasnoubena, a vyspí se s ní, vezme si ji za ženu a dá za ni plné věno.
#22:16 Jestliže by se její otec rozhodně zdráhal mu ji dát, zaplatí svůdce obnos ve výši věna panen.
#22:17 Čarodějnici nenecháš naživu.
#22:18 Kdokoli by obcoval s dobytčetem, musí zemřít.
#22:19 Kdo by obětoval bohům a ne samotnému Hospodinu, propadne klatbě.
#22:20 Hostu nebudeš škodit ani ho utlačovat, neboť i vy jste byli hosty v egyptské zemi.
#22:21 Žádnou vdovu a sirotka nebudete utiskovat.
#22:22 Jestliže je přece budeš utiskovat a oni budou ke mně úpět, jistě jejich úpění vyslyším.
#22:23 Vzplanu hněvem a pobiji vás mečem, takže z vašich žen budou vdovy a z vašich synů sirotci.
#22:24 Jestliže půjčíš stříbro někomu z mého lidu, zchudlému, který je s tebou, nebudeš se k němu chovat jako lichvář, neuložíš mu úrok.
#22:25 Jestliže se rozhodneš vzít do zástavy plášť svého bližního, do západu slunce mu jej vrátíš,
#22:26 neboť jeho plášť, kterým si chrání tělo, je jeho jedinou přikrývkou. V čem by spal? Stane se, že bude ke mně úpět a já ho vyslyším, poněvadž jsem milostivý.
#22:27 Nebudeš zlořečit Bohu ani nebudeš proklínat předáka ve svém lidu.
#22:28 Neopozdíš se s dávkami z hojnosti svých úrod a vylisované šťávy svých hroznů a oliv. Dáš mi prvorozeného ze svých synů.
#22:29 Se svým skotem a bravem naložíš tak, že zůstane sedm dní u matky, osmého dne jej dáš mně.
#22:30 Buďte mými muži svatými. Maso zvířete rozsápaného na poli nebudete jíst, hodíte je psovi. 
#23:1 Nebudeš šířit falešnou pověst. Nespřáhneš se se svévolníkem, aby ses stal zlovolným svědkem.
#23:2 Nepřidáš se k většině, páchá-li zlo. Nebudeš vypovídat ve sporu s ohledem na většinu a převracet právo.
#23:3 Ani nemajetnému nebudeš v jeho sporu nadržovat.
#23:4 Když narazíš na býka svého nepřítele nebo na jeho zatoulaného osla, musíš mu jej vrátit.
#23:5 Když uvidíš, že osel toho, kdo tě nenávidí, klesá pod svým břemenem, zanecháš ho snad, aniž ho vyprostíš? Spolu s ním ho vyprostíš.
#23:6 Nebudeš převracet právo ubožáka v jeho sporu.
#23:7 Buď dalek každého podvodu; nepřipustíš, aby byl zabit nevinný a spravedlivý, neboť svévolníka neospravedlním.
#23:8 Nebudeš brát úplatek, neboť úplatek oslepuje i ty, kdo mají oči otevřené, a vede k překrucování záležitostí spravedlivých.
#23:9 Nebudeš utlačovat hosta; víte přece, jak bývá hostu v duši, neboť jste byli hosty v egyptské zemi.
#23:10 Po šest let budeš osévat svou zemi a sklízet z ní úrodu.
#23:11 Sedmého roku ji necháš ležet ladem. Nebudeš ji obdělávat, aby jedli ubožáci z tvého lidu, a co zbude, spase polní zvěř. Tak naložíš i se svou vinicí a se svým olivovím.
#23:12 Po šest dnů budeš konat svou práci, ale sedmého dne přestaneš, aby odpočinul tvůj býk i osel a aby si mohl oddechnout syn tvé otrokyně i host.
#23:13 Na všechno, co jsem vám řekl, budete bedlivě dbát. Jméno jiného boha nebudete připomínat; ať je není slyšet z tvých úst.
#23:14 Třikrát v roce budeš slavit mé slavnosti:
#23:15 Budeš zachovávat slavnost nekvašených chlebů. Po sedm dní budeš jíst nekvašené chleby, jak jsem ti přikázal, a to v určený čas měsíce ábíbu (to je měsíce klasů), neboť tehdy jsi vyšel z Egypta. Nikdo se neukáže před mou tváří s prázdnou.
#23:16 Budeš zachovávat též slavnost žně, prvních snopků z výtěžku toho, co jsi zasel na poli, a slavnost sklizně plodin na konci roku, kdy sklízíš z pole výsledek své práce.
#23:17 Třikrát v roce se ukáže každý, kdo je mužského pohlaví, před Pánem Hospodinem.
#23:18 Nebudeš obětovat dobytče tak, aby krev mého obětního hodu vytekla na něco kvašeného. Nebudeš přechovávat tuk z mé slavnosti přes noc až do rána.
#23:19 Prvotiny raných plodů své role přineseš do domu Hospodina, svého Boha. Nebudeš vařit kůzle v mléku jeho matky.
#23:20 Hle, posílám před tebou posla, aby tě opatroval na cestě a aby tě uvedl na místo, které jsem připravil.
#23:21 Měj se před ním na pozoru a poslouchej ho, nevzdoruj mu, neboť přestupky vám nepromine, poněvadž v něm je mé jméno.
#23:22 Když jej však budeš opravdu poslouchat a činit všechno, co mluvím, stanu se nepřítelem tvých nepřátel a protivníkem tvých protivníků.
#23:23 Můj posel půjde před tebou a uvede tě k Emorejcům, Chetejcům a Perizejcům, ke Kenaancům, Chivejcům a Jebúsejcům, a já je zničím.
#23:24 Nebudeš se klanět jejich bohům a nebudeš jim sloužit. Nebudeš se dopouštět toho, co páchají. Úplně je rozmetáš a na kusy roztříštíš jejich posvátné sloupy.
#23:25 Budete sloužit Hospodinu, svému Bohu, a on požehná tvému chlebu a tvé vodě. Vzdálím od tebe nemoc;
#23:26 ve tvé zemi nebude ženy, která by potratila nebo která by byla neplodná; obdařím tě plností let.
#23:27 Pošlu před tebou svou hrůzu a uvedu ve zmatek všechen lid, k němuž přicházíš; obrátím před tebou všechny tvé nepřátele na útěk.
#23:28 Pošlu před tebou děsy, aby před tebou vypudili Chivejce, Kenaance a Chetejce.
#23:29 Nevypudím je však před tebou za jeden rok, aby země nezpustla a aby se k tvé škodě nerozmnožila polní zvěř.
#23:30 Vypudím je před tebou postupně, dokud se nerozplodíš a nepřevezmeš zemi do dědictví.
#23:31 A určím tvé pomezí od Rákosového moře až k moři Pelištejců a od pouště až k řece Eufratu. Dám totiž do vašich rukou obyvatele země a vypudíš je před sebou.
#23:32 Neuzavřeš smlouvu s nimi nebo s jejich bohy.
#23:33 Nebudou v tvé zemi sídlit, aby tě nesvedli ke hříchu proti mně. Kdybys sloužil jejich bohům, stalo by se ti to léčkou. 
#24:1 Potom Mojžíšovi řekl: „Vystup k Hospodinu, ty i Áron, Nádab a Abíhú a sedmdesát z izraelských starších. Budete se zdálky klanět.
#24:2 K Hospodinu přistoupí jen Mojžíš. Ostatní se přibližovat nebudou. Lid nesmí vystoupit vzhůru spolu s ním.“
#24:3 Když Mojžíš přišel nazpět, vypravoval lidu všechna slova Hospodinova a předložil mu všechna právní ustanovení. Všechen lid odpověděl jako jedněmi ústy. Řekli: „Budeme dělat všechno, o čem Hospodin mluvil.“
#24:4 Nato Mojžíš zapsal všechna Hospodinova slova. Za časného jitra postavil pod horou oltář a dvanáct posvátných sloupů podle dvanácti izraelských kmenů.
#24:5 Pak pověřil izraelské mládence, aby přinesli oběti zápalné a obětovali Hospodinu býčky k hodům oběti pokojné.
#24:6 Mojžíš vzal polovinu krve a vlil ji do mís a druhou polovinou pokropil oltář.
#24:7 Potom vzal Knihu smlouvy a předčítal lidu. Prohlásili: „Poslušně budeme dělat všechno, o čem Hospodin mluvil.“
#24:8 Mojžíš vzal krev, pokropil lid a řekl: „Hle, krev smlouvy, kterou s vámi uzavírá Hospodin na základě všech těchto slov.“
#24:9 Pak Mojžíš a Áron, Nádab a Abíhú a sedmdesát z izraelských starších vystoupili vzhůru.
#24:10 Uviděli Boha Izraele. Pod jeho nohama bylo cosi jako průzračný safír, jako čisté nebe.
#24:11 Ale nevztáhl ruku na nejpřednější z Izraelců, ačkoli uzřeli Boha; i jedli a pili.
#24:12 Hospodin řekl Mojžíšovi: „Vystup ke mně na horu a pobuď tam. Dám ti kamenné desky - zákon a přikázání, které jsem napsal, abys jim vyučoval.“
#24:13 I povstal Mojžíš a Jozue, který mu přisluhoval, a Mojžíš vystoupil na Boží horu.
#24:14 Starším řekl: „Zůstaňte zde, dokud se k vám nevrátíme. Budou tu s vámi Áron a Chúr. Kdo něco má, ať se obrací na ně.“
#24:15 Mojžíš tedy vystoupil na horu a horu přikryl oblak.
#24:16 A Hospodinova sláva přebývala na hoře Sínaji a oblak ji přikrýval po šest dní. Sedmého dne zavolal Hospodin na Mojžíše zprostřed oblaku.
#24:17 Hospodinova sláva se jevila pohledu Izraelců jako stravující oheň na vrcholku hory.
#24:18 Mojžíš vstoupil doprostřed oblaku. Vystoupil na horu a byl na hoře čtyřicet dní a čtyřicet nocí. 
#25:1 Hospodin promluvil k Mojžíšovi:
#25:2 „Mluv k synům Izraele, ať pro mne vyberou oběť pozdvihování. Vyberete oběť pozdvihování pro mne od každého, kdo ji ze srdce dobrovolně odevzdá.
#25:3 Toto bude oběť pozdvihování, kterou od nich vyberete: zlato, stříbro a měď;
#25:4 látka purpurově fialová, nachová a karmínová, jemné plátno a kozí srst;
#25:5 načerveno zbarvené beraní kůže, tachaší kůže a akáciové dřevo;
#25:6 olej na svícení, balzámy na olej k pomazání a na kadidlo z vonných látek;
#25:7 karneolové drahokamy a kameny pro zasazení do nárameníku a náprsníku.
#25:8 Ať mi udělají svatyni a já budu bydlit uprostřed nich.
#25:9 Uděláte všechno přesně podle toho, co ti ukazuji jako vzor svatého příbytku i vzor všech bohoslužebných předmětů.
#25:10 Udělají z akáciového dřeva schránu dva a půl lokte dlouhou, jeden a půl lokte širokou a jeden a půl lokte vysokou.
#25:11 Obložíš ji čistým zlatem, uvnitř i zvnějšku ji obložíš a opatříš ji dokola zlatou obrubou.
#25:12 Uliješ pro ni čtyři zlaté kruhy a připevníš je k čtyřem jejím hranám: dva kruhy na jednom boku a dva kruhy na druhém.
#25:13 Zhotovíš tyče z akáciového dřeva a potáhneš je zlatem.
#25:14 Tyče prostrčíš skrz kruhy po stranách schrány, aby bylo možno na nich schránu nést.
#25:15 Tyče zůstanou v kruzích, nebudou vytahovány.
#25:16 Do schrány uložíš svědectví, které ti dám.
#25:17 Zhotovíš příkrov z čistého zlata dlouhý dva a půl lokte a široký jeden a půl lokte.
#25:18 Potom zhotovíš dva cheruby ze zlata; dáš je vytepat na oba konce příkrovu.
#25:19 Jednoho cheruba uděláš na jednom konci a druhého cheruba na druhém konci. Uděláte cheruby na příkrov, na oba jeho konce.
#25:20 Cherubové budou mít křídla rozpjatá vzhůru; svými křídly budou zastírat příkrov. Tvářemi budou obráceni k sobě; budou hledět na příkrov.
#25:21 Příkrov dáš nahoru na schránu a do schrány uložíš svědectví, které ti dám.
#25:22 Tam se budu s tebou setkávat a z místa nad příkrovem mezi oběma cheruby, kteří budou na schráně svědectví, budu s tebou mluvit o všem, co ti pro Izraelce přikážu.
#25:23 Zhotovíš stůl z akáciového dřeva dlouhý dva lokte, široký jeden loket a vysoký jeden a půl lokte.
#25:24 Obložíš jej čistým zlatem a opatříš jej dokola zlatou obrubou.
#25:25 Uděláš mu také dokola na dlaň širokou lištu a k liště uděláš dokola zlatou obrubu.
#25:26 Opatříš jej čtyřmi zlatými kruhy a připevníš je ke čtyřem rohům při jeho čtyřech nohách.
#25:27 Kruhy budou těsně pod lištou, aby držely tyče na nošení stolu.
#25:28 Tyče zhotovíš z akáciového dřeva a potáhneš je zlatem; na nich se stůl bude nosit.
#25:29 Zhotovíš k němu též mísy, pánvičky, konvice a obětní misky používané k úlitbě; zhotovíš je z čistého zlata.
#25:30 Pravidelně budeš klást přede mne na stůl předkladný chléb.
#25:31 Zhotovíš svícen z čistého zlata. Svícen bude mít vytepaný dřík a prut; jeho kalichy, číšky a květy budou s ním zhotoveny z jednoho kusu.
#25:32 Z jeho stran bude vycházet šest prutů, tři pruty svícnu po jedné straně a tři pruty svícnu po druhé straně.
#25:33 Na jednom prutu budou tři kalichy podobné mandloňovému květu: číška a květ. A tři kalichy podobné mandloňovému květu na druhém prutu: číška a květ. Tak to bude na všech šesti prutech vycházejících ze svícnu.
#25:34 Na svícnu budou čtyři kalichy podobné mandloňovému květu s číškami a květy.
#25:35 Jedna číška bude pod jednou dvojicí prutů, druhá číška bude pod druhou dvojicí prutů a třetí číška bude pod třetí dvojicí prutů; tak to bude u všech šesti prutů vycházejících ze svícnu.
#25:36 Příslušné číšky a pruty budou s ním zhotoveny z jednoho kusu; všechno bude vytepané z čistého zlata.
#25:37 Ke svícnu zhotovíš také sedm kahánků; kahánky ať jsou nasazeny tak, aby osvětlovaly prostor před ním.
#25:38 Zhotovíš k němu z čistého zlata i nůžky na knoty a pánve na oharky.
#25:39 Bude zhotoven se všemi těmito předměty z jednoho talentu čistého zlata.
#25:40 Hleď, abys všechno udělal podle vzoru, který ti byl ukázán na hoře. 
#26:1 Zhotovíš příbytek z deseti pruhů jemně tkaného plátna a z látky purpurově fialové, nachové a karmínové; zhotovíš je s umně vetkanými cheruby.
#26:2 Jeden pruh bude dlouhý dvacet osm loket a široký čtyři lokte. Takový bude jeden pruh. Všechny pruhy budou mít stejné rozměry.
#26:3 Pět pruhů bude navzájem po délce spojeno, a právě tak bude navzájem spojeno druhých pět.
#26:4 Z fialového purpuru zhotovíš poutka na lemu toho pruhu, který bude na konci spojeného kusu. Stejně to uděláš na lemu krajního pruhu druhého spojeného kusu.
#26:5 Uděláš také padesát poutek na jednom koncovém pruhu a padesát poutek uděláš na koncovém pruhu druhého spojeného kusu. Jednotlivá poutka budou naproti sobě.
#26:6 Zhotovíš i padesát zlatých spon a sepneš jimi pruhy dohromady, takže příbytek bude spojen v jeden celek.
#26:7 Zhotovíš též houně z kozí srsti pro stan nad příbytkem; jedenáct takových houní uděláš.
#26:8 Jedna houně bude dlouhá třicet loket a široká čtyři lokte. Taková bude jedna houně. Všech jedenáct houní bude mít stejné rozměry.
#26:9 Zvlášť spojíš pět houní a zvlášť šest houní. Šestou houni na přední straně stanu přeložíš.
#26:10 Uděláš také padesát poutek na lemu jedné houně, která bude na kraji spojeného kusu, a padesát poutek uděláš na lemu koncové houně druhého spojeného kusu.
#26:11 Zhotovíš i padesát bronzových spon, provlékneš je poutky a stan sepneš, takže bude spojen v jeden celek.
#26:12 Z toho, co bude ze stanových houní přesahovat, bude přesahující polovina houně viset přes zadní stranu příbytku.
#26:13 Loket z té i oné strany, o nějž budou stanové houně větší, bude tvořit na bočních stranách příbytku převis, aby byl cele přikryt.
#26:14 Zhotovíš pro stan též přikrývku z beraních kůží zbarvených načerveno a navrch přikrývku z tachaších kůží.
#26:15 Pro příbytek zhotovíš také desky z akáciového dřeva, aby se daly postavit.
#26:16 Každá deska bude dlouhá deset loket a široká půldruhého lokte.
#26:17 Každá bude mít dva čepy a jedna bude připojena k druhé; tak to provedeš na všech deskách pro příbytek.
#26:18 Zhotovíš pro příbytek tyto desky: dvacet desek pro pravou jižní stranu.
#26:19 Těch dvacet desek opatříš dole čtyřiceti stříbrnými patkami, po dvou patkách pod každou desku, k oběma jejím čepům.
#26:20 I pro druhý bok příbytku, k severní straně, dvacet desek
#26:21 a čtyřicet stříbrných patek, po dvou patkách pod každou desku.
#26:22 Pro zadní stranu příbytku, k západu, zhotovíš šest desek.
#26:23 Navíc zhotovíš dvě desky pro oba úhly příbytku při zadní straně.
#26:24 Zdola budou přiloženy k sobě, navrchu budou těsně spojeny jedním kruhem. Tak tomu bude s oběma deskami v obou úhlech.
#26:25 Bude tam tedy osm desek se stříbrnými patkami, celkem šestnáct patek, po dvou patkách pod každou deskou.
#26:26 Zhotovíš také svlaky z akáciového dřeva, pět pro desky na jednom boku příbytku,
#26:27 pět na druhém boku příbytku a dalších pět svlaků pro desky na zadní straně příbytku, k západu.
#26:28 Prostřední svlak bude probíhat v poloviční výši desek od jednoho konce k druhému.
#26:29 Desky potáhneš zlatem a zhotovíš k nim zlaté kruhy pro vsunutí svlaků; také svlaky potáhneš zlatem.
#26:30 Příbytek postavíš podle ustanovení, jak ti bylo ukázáno na hoře.
#26:31 Zhotovíš také oponu z látky purpurově fialové, nachové a karmínové a z jemně tkaného plátna. Zhotovíš ji s umně vetkanými cheruby.
#26:32 Zavěsíš ji na čtyřech sloupech z akáciového dřeva, potažených zlatem; háčky na nich budou ze zlata. Sloupy budou na čtyřech stříbrných patkách.
#26:33 Zavěsíš oponu na spony a dovnitř tam za oponu vneseš schránu svědectví. Opona vám bude oddělovat svatyni od velesvatyně.
#26:34 Ve velesvatyni vložíš na schránu svědectví příkrov.
#26:35 Vně před oponu postavíš stůl a naproti stolu svícen při jižním boku příbytku; stůl dáš k severnímu boku.
#26:36 Ke vchodu do stanu zhotovíš pestře vyšitý závěs z látky fialově purpurové, nachové a karmínové a z jemně tkaného plátna.
#26:37 K závěsu zhotovíš pět sloupů z akáciového dřeva a potáhneš je zlatem; háčky na nich budou ze zlata; odliješ pro ně také pět bronzových patek. 
#27:1 Zhotovíš oltář z akáciového dřeva pět loket dlouhý a pět loket široký. Oltář bude čtyřhranný, vysoký tři lokte.
#27:2 Jeho čtyři úhly opatříš rohy; ty budou s ním zhotoveny z jednoho kusu. Potáhneš jej bronzem.
#27:3 Zhotovíš k němu i hrnce na vybírání popela, lopaty a kropenky, vidlice a pánve na oheň. Všechno příslušné náčiní zhotovíš z bronzu.
#27:4 Zhotovíš k němu také mřížový rošt z bronzu. Na čtyřech koncích mříže připevníš čtyři bronzové kruhy.
#27:5 A zasadíš ji zdola pod obložení oltáře; mříž bude dosahovat až do poloviny oltáře.
#27:6 Zhotovíš pro oltář i tyče; budou z akáciového dřeva a potáhneš je bronzem.
#27:7 Tyče budou provléknuty skrze kruhy. Když bude oltář přenášen, budou tyče po obou jeho bocích.
#27:8 Uděláš jej dutý z desek, jak ti bylo ukázáno na hoře; přesně tak jej udělají.
#27:9 K příbytku uděláš také nádvoří. Na jižní, pravé straně nádvoří budou zástěny z jemně tkaného plátna, pro jednu stranu dlouhé sto loket.
#27:10 Bude tam dvacet sloupů na dvaceti bronzových patkách; háčky ke sloupům a příčné tyče budou stříbrné.
#27:11 Stejně tak budou zástěny i na severní straně v délce sto loket; i tam bude dvacet sloupů na dvaceti bronzových patkách; háčky ke sloupům a příčné tyče budou stříbrné.
#27:12 Našíř nádvoří na západní straně budou zástěny v délce padesáti loket. A bude tam deset sloupů na deseti patkách.
#27:13 Šířka nádvoří na přední, východní straně bude padesát loket.
#27:14 Na jednom křídle bude patnáct loket zástěn; budou tam tři sloupy na třech patkách.
#27:15 I na druhém křídle bude patnáct loket zástěn; budou tam také tři sloupy na třech patkách.
#27:16 V bráně nádvoří bude pestře vyšitý závěs široký dvacet loket z látky fialově purpurové, nachové a karmínové a z jemně tkaného plátna. Budou tam čtyři sloupy na čtyřech patkách.
#27:17 Všechny sloupy na nádvoří budou dokola spojeny stříbrnými příčnými tyčemi; jejich háčky budou stříbrné a patky bronzové.
#27:18 Délka nádvoří bude sto loket, šířka padesát a výška zástěn pět loket; budou z jemně tkaného plátna; patky sloupů budou bronzové.
#27:19 Všechno náčiní příbytku pro veškerou bohoslužbu v něm i všechny jeho kolíky a všechny kolíky pro nádvoří budou z bronzu.
#27:20 Ty pak přikážeš Izraelcům, aby ti přinášeli čistý vytlačený olivový olej ke svícení, aby bylo možno udržovat ustavičně svítící kahan.
#27:21 Ve stanu setkávání před oponou, která bude zakrývat svědectví, bude o něj pečovat Áron se svými syny před Hospodinem od večera do rána. To je mezi Izraelci provždy platné nařízení pro všechna pokolení. 
#28:1 Přikaž, aby předstoupil tvůj bratr Áron a s ním jeho synové z řad Izraelců, aby mi sloužili jako kněží: Áron a Áronovi synové Nádab, Abíhú, Eleazar a Ítamar.
#28:2 Uděláš pro svého bratra Árona svaté roucho k slávě a ozdobě.
#28:3 Promluvíš se všemi dovednými řemeslníky, které jsem naplnil duchem moudrosti; ti zhotoví Áronovi roucho, aby byl posvěcen a mohl mi sloužit jako kněz.
#28:4 Toto budou roucha, která udělají: náprsník a nárameník, říza a tkaná suknice, turban a šerpa; taková svatá roucha udělají tvému bratru Áronovi a jeho synům, aby mi sloužili jako kněží.
#28:5 Použijí zlata, látky fialově purpurové, nachové a karmínové a jemného plátna.
#28:6 Zhotoví umně utkaný nárameník ze zlata, z látky fialově purpurové, nachové a karmínové a z jemně tkaného plátna.
#28:7 Obě vrchní části budou na obou koncích spojeny; nárameník musí být spojen.
#28:8 Tkaný pás, jímž bude nárameník upevňován, bude zhotoven týmž způsobem ze zlata, z látky fialově purpurové, nachové a karmínové a z jemně tkaného plátna.
#28:9 Vezmeš dva kameny karneoly a vyryješ do nich jména synů Izraele:
#28:10 šest jmen do jednoho kamene a zbývajících šest jmen do druhého kamene podle pořadí narození.
#28:11 Jako kamenorytec vyrývá pečeť, tak vyryješ do obou kamenů jména synů Izraele. Vsadíš je do zlatých obrouček
#28:12 a umístíš oba kameny na vrchních částech nárameníku, aby byly připomínkou synů Izraele. Áron bude jejich jména nosit na obou ramenou před Hospodinem, aby je připomínal.
#28:13 Zhotovíš i zlaté obroučky
#28:14 a dva řetízky z čistého zlata. Uděláš je jako stočené šňůry a upevníš tyto stočené řetízky na obroučky.
#28:15 Zhotovíš také umně utkaný náprsník Božích rozhodnutí. Uděláš jej týmž způsobem jako nárameník; ze zlata, z látky fialově purpurové, nachové a karmínové a z jemně tkaného plátna.
#28:16 Bude čtvercový a dvojitý, jednu píď dlouhý a jednu píď široký.
#28:17 Vysadíš jej drahými kameny ve čtyřech řadách: v první řadě rubín, topas a smaragd;
#28:18 v druhé řadě malachit, safír a jaspis;
#28:19 v třetí řadě opál, achát a ametyst;
#28:20 ve čtvrté řadě chrysolit, karneol a onyx. Všechny budou zasazeny do zlaté obruby.
#28:21 Kameny budou označeny jmény synů Izraele; bude jich dvanáct podle jejich jmen; na každém bude vyryto jeho jméno jako na pečeti podle dvanáctera kmenů.
#28:22 Uděláš k náprsníku řetízky z čistého zlata jako stočenou šňůru.
#28:23 K náprsníku zhotovíš též dva zlaté kroužky a připevníš oba kroužky k oběma horním okrajům náprsníku.
#28:24 Obě zlaté šňůry připevníš k oběma kroužkům na horních okrajích náprsníku.
#28:25 Oba druhé konce obou šňůr připevníš ke dvěma obroučkám a umístíš vpředu na vrchních částech nárameníku.
#28:26 Potom zhotovíš jiné dva zlaté kroužky a připevníš je k oběma okrajům náprsníku, na lem jeho vnitřní strany při nárameníku.
#28:27 Zhotovíš i další dva zlaté kroužky a připevníš je k oběma vrchním částem nárameníku dole na jeho přední straně těsně u švu, nad tkaným pásem nárameníku.
#28:28 Náprsník opatřený kroužky přivážou ke kroužkům nárameníku purpurově fialovou šňůrkou, aby byl nad tkaným pásem nárameníku; náprsník nebude od nárameníku odstávat.
#28:29 Áron tak bude nosit jména synů Izraele na náprsníku Božích rozhodnutí na svém srdci, kdykoli bude vstupovat do svatyně, aby je ustavičně připomínal před Hospodinem.
#28:30 A ty vložíš do náprsníku Božích rozhodnutí posvátné losy urím a tumím; budou na srdci Áronově, kdykoli bude předstupovat před Hospodina. Áron bude ustavičně nosit Boží rozhodnutí pro syny Izraele na svém srdci před Hospodinem.
#28:31 K nárameníku zhotovíš řízu, celou z fialového purpuru.
#28:32 Uprostřed bude otvor pro hlavu, dokola opatřený tkaným lemem jako otvor v krunýři, aby se neroztrhl.
#28:33 Dole na obrubě připevníš granátová jablka z látky fialově purpurové, nachové a karmínové dokola na obrubě a mezi nimi zlaté zvonečky.
#28:34 Zlatý zvoneček a granátové jablko, zlatý zvoneček a granátové jablko budou střídavě dokola na obrubě řízy.
#28:35 To bude mít Áron na sobě, když bude konat službu. Kdykoli bude vstupovat do svatyně před Hospodina a kdykoli bude vycházet, bude se ozývat jejich zvuk, aby nezemřel.
#28:36 Zhotovíš také květ z čistého zlata a vyryješ na něj, jako se vyrývá pečeť: „Svatý Hospodinu“.
#28:37 Připevníš jej purpurově fialovou šňůrkou na turban; bude vpředu na turbanu.
#28:38 Bude na Áronově čele. Áron bude odpovědný za nepravost při přinášení svatých věcí, které Izraelci oddělí jako svaté, za všechny jejich svaté dary. Květ bude na jeho čele trvale, aby nalezli zalíbení před Hospodinem.
#28:39 Utkáš také suknici z jemného plátna a zhotovíš turban z jemného plátna a pestře vyšitou šerpu.
#28:40 I Áronovým synům uděláš suknice a zhotovíš pro ně šerpy. Zhotovíš jim také mitry k slávě a ozdobě.
#28:41 Oblékneš do toho svého bratra Árona a s ním jeho syny. Dáš jim pomazání a uvedeš je v úřad. Posvětíš je, aby mi sloužili jako kněží.
#28:42 Rovněž jim uděláš lněné spodky, aby zakryly nahotu jejich těla. Budou sahat od beder po stehna.
#28:43 Áron a jeho synové je budou nosit, kdykoli budou vstupovat do stanu setkávání nebo kdykoli budou přistupovat k oltáři, aby konali službu ve svatyni; tak neponesou následky nepravosti a nezemřou. To je provždy platné nařízení pro něho i pro jeho potomky. 
#29:1 Toto s nimi uděláš, až je budeš světit, aby mi sloužili jako kněží: Vezmeš jednoho mladého býčka a dva berany bez vady,
#29:2 nekvašené chleby, nekvašené bochánky zadělané olejem a nekvašené oplatky pomazané olejem; uděláš je z bílé pšeničné mouky.
#29:3 Dáš to do jednoho koše a přineseš to v tom koši jako dar; přivedeš i býčka a oba berany.
#29:4 Potom předvedeš Árona a jeho syny ke vchodu do stanu setkávání a omyješ je vodou.
#29:5 Vezmeš kněžská roucha, oblékneš Áronovi suknici a řízu pod nárameník, i nárameník a náprsník. Pak ho opásáš umně utkaným nárameníkovým pásem.
#29:6 Na hlavu mu vložíš turban a na turban připevníš svatou čelenku.
#29:7 Nato vezmeš olej pomazání, vyleješ jej na jeho hlavu a pomažeš ho.
#29:8 Pak předvedeš jeho syny, oblékneš jim suknice,
#29:9 přepášeš je šerpou, Árona i jeho syny, a vstavíš jim na hlavu mitry. I stane se jim kněžství provždy platným nařízením. Tak uvedeš v úřad Árona a jeho syny.
#29:10 Přivedeš býčka jako dar před stan setkávání a Áron se svými syny vloží ruce na jeho hlavu.
#29:11 Býčka porazíš před Hospodinem u vchodu do stanu setkávání.
#29:12 Vezmeš trochu krve z býčka a potřeš prstem rohy oltáře; všechnu ostatní krev vyleješ ke spodku oltáře.
#29:13 Pak vezmeš všechen tuk pokrývající vnitřnosti, jaterní lalok, obě ledviny i s tukem, který je na nich, a obrátíš je na oltáři v obětní dým.
#29:14 Maso býčka, jeho kůži a jeho výměty spálíš na ohni venku za táborem. To je oběť za hřích.
#29:15 Potom vezmeš jednoho berana. Áron se svými syny vloží ruce na jeho hlavu.
#29:16 Berana porazíš. Vezmeš trochu jeho krve a pokropíš oltář dokola.
#29:17 Pak rozsekáš berana na díly. Omyješ jeho vnitřnosti a hnáty a přiložíš to k ostatním dílům a k hlavě.
#29:18 Celého berana obrátíš na oltáři v obětní dým. To je zápalná oběť Hospodinu, libá vůně, ohnivá oběť pro Hospodina.
#29:19 Potom vezmeš druhého berana. Áron se svými syny vloží ruce na jeho hlavu.
#29:20 Berana porazíš. Vezmeš trochu jeho krve a potřeš jí lalůček Áronova ucha a lalůček pravého ucha jeho synů i palec jejich pravé ruky a palec jejich pravé nohy. Zbylou krví pokropíš oltář dokola.
#29:21 Pak vezmeš trochu krve z oltáře a trochu oleje pomazání a postříkáš Árona i jeho roucha a s ním jeho syny i roucha jeho synů. Tím bude posvěcen Áron i jeho roucha a s ním jeho synové a roucha jeho synů.
#29:22 Potom vezmeš z berana tuk a tučný ocas, tuk přikrývající vnitřnosti, jaterní lalok, obě ledviny i s tukem, který je na nich, a pravou kýtu. To je beran vysvěcení.
#29:23 Vezmeš také z koše s nekvašenými věcmi, který je před Hospodinem, jeden pecen chleba, jeden bochánek připravený na oleji a jeden oplatek.
#29:24 Vše vložíš do dlaní Áronových a do dlaní jeho synů a podáváním to nabídneš Hospodinu jako oběť podávání.
#29:25 Potom to vezmeš z jejich rukou a obrátíš na oltáři v obětní dým nad zápalnou obětí jako libou vůni před Hospodinem. To je ohnivá oběť pro Hospodina.
#29:26 Vezmeš pak hrudí z Áronova berana vysvěcení a podáváním je nabídneš Hospodinu jako oběť podávání. To je tvůj podíl.
#29:27 Posvětíš hrudí z oběti podávání a kýtu z oběti pozdvihování, tedy ty části, které byly nabídnuty Hospodinu podáváním a pozdvihováním z berana vysvěcení za Árona a jeho syny.
#29:28 To je mezi Izraelci provždy platné nařízení pro Árona a jeho syny; je to oběť pozdvihování. Oběť pozdvihování bude mezi Izraelci přinášena při jejich hodech oběti pokojné, jejich oběť pozdvihování pro Hospodina.
#29:29 Svatá roucha Áronova zůstanou po něm jeho synům, aby v nich byli pomazáváni a uváděni v úřad.
#29:30 Po sedm dní je bude oblékat ten z jeho synů, který po něm bude knězem, který bude přicházet ke stanu setkávání, aby konal službu ve svatyni.
#29:31 Pak vezmeš berana vysvěcení a uvaříš jeho maso na svatém místě.
#29:32 Áron a jeho synové budou jíst maso toho berana s chlebem, který bude v koši u vchodu do stanu setkávání.
#29:33 Budou jíst ti, kteří byli zproštěni vin, aby byli uvedeni v úřad a posvěceni; nepovolaný jíst nesmí, poněvadž to je svaté.
#29:34 Zůstane-li něco z masa berana vysvěcení a z chleba až do rána, spálíš zbytek na ohni; nesmí se jíst, poněvadž to je svaté.
#29:35 S Áronem a jeho syny uděláš všechno podle toho, co jsem ti přikázal. Po sedm dní je budeš uvádět v úřad.
#29:36 Denně budeš připravovat býčka jako oběť za hřích kromě smírčích obětí. Očistíš oltář od hříchu tím, že na něm vykonáš smírčí obřady. Pak jej pomažeš a posvětíš.
#29:37 Po sedm dní budeš vykonávat na oltáři smírčí obřady; tak jej posvětíš a oltář bude velesvatý. Cokoli se dotkne oltáře, bude svaté.
#29:38 Toto pak budeš přinášet na oltáři: každodenně dva jednoroční beránky.
#29:39 Jednoho beránka přineseš ráno a druhého beránka přineseš navečer.
#29:40 A k prvnímu beránkovi též desetinu éfy bílé mouky zadělané čtvrtinou hínu vytlačeného oleje a jako úlitbu čtvrtinu hínu vína.
#29:41 Druhého beránka přineseš navečer; podobně jako jitřní obětní dar a příslušnou úlitbu jej přineseš v libou vůni, ohnivou oběť pro Hospodina.
#29:42 Tuto každodenní zápalnou oběť budete přinášet u vchodu do stanu setkávání před Hospodinem po všechna vaše pokolení. Tam se s vámi budu setkávat, abych tam k tobě mluvil.
#29:43 Budu se tam setkávat se syny Izraele a místo bude posvěceno mou slávou.
#29:44 Posvětím stan setkávání a oltář; také Árona a jeho syny posvětím, aby mi sloužili jako kněží.
#29:45 Budu přebývat uprostřed Izraelců a budu jejich Bohem.
#29:46 Poznají, že já jsem Hospodin, jejich Bůh, že já jsem je vyvedl z egyptské země, abych přebýval uprostřed nich. Já jsem Hospodin, jejich Bůh. 
#30:1 Zhotovíš také oltář k pálení kadidla. Z akáciového dřeva jej zhotovíš.
#30:2 Bude čtyřhranný: loket dlouhý, loket široký a dva lokte vysoký. Jeho rohy budou z jednoho kusu s ním.
#30:3 Potáhneš jej čistým zlatem, jeho vršek i jeho stěny dokola a jeho rohy, a opatříš jej dokola zlatou obrubou.
#30:4 Zhotovíš pro něj rovněž dva zlaté kruhy, a to pod obrubou na obou jeho bocích; k oběma bočnicím je zhotovíš, aby držely tyče, na nichž bude nošen.
#30:5 Tyče zhotovíš z akáciového dřeva a potáhneš je zlatem.
#30:6 Oltář postavíš před oponu, která je před schránou svědectví, před příkrov, přikrývající schránu svědectví, kde se s tebou budu setkávat.
#30:7 Na něm bude Áron pálit kadidlo z vonných látek. Bude je pálit každého rána, když bude ošetřovat kahany.
#30:8 Bude je pálit i navečer, když bude kahany rozsvěcovat. Každodenně bude před Hospodinem pálit kadidlo po všechna vaše pokolení.
#30:9 Nebudete na něm obětovat jiné kadidlo ani oběť zápalnou nebo přídavnou ani na něm nebudete přinášet úlitbu.
#30:10 Jednou za rok vykoná Áron na jeho rozích smírčí obřady. Z krve smírčí oběti za hřích bude na něm vykonávat smírčí obřady jednou za rok po všechna vaše pokolení. Oltář bude velesvatý Hospodinu.“
#30:11 Hospodin promluvil k Mojžíšovi:
#30:12 „Když budeš pořizovat soupis Izraelců povolaných do služby, dá každý při sčítání výkupné Hospodinu za svůj život, aby je při sčítání nestihla nenadálá pohroma.
#30:13 Toto dá každý, kdo přejde mezi povolané do služby: půl šekelu podle váhy určené svatyní; šekel je dvacet zrn. Tato půlka šekelu je oběť pozdvihování pro Hospodina.
#30:14 Každý, kdo přejde mezi povolané do služby, od dvacetiletých výše, odvede Hospodinu oběť pozdvihování.
#30:15 Bohatý nebude dávat více a nemajetný nedá méně než půl šekelu, když se bude odvádět Hospodinu oběť pozdvihování na vykonání smírčích obřadů za vaše životy.
#30:16 Vezmeš od Izraelců obnos na smírčí oběti a věnuješ jej na službu při stanu setkávání. To bude pro Izraelce jako připomínka před Hospodinem, když se za vás budou vykonávat smírčí obřady.“
#30:17 Hospodin dále mluvil k Mojžíšovi:
#30:18 „Zhotovíš také bronzovou nádrž k omývání s bronzovým podstavcem a umístíš ji mezi stanem setkávání a oltářem a naleješ tam vodu.
#30:19 Áron a jeho synové si jí budou omývat ruce a nohy.
#30:20 Když budou přicházet ke stanu setkávání nebo když budou přistupovat k oltáři, aby konali službu a obraceli ohnivou oběť Hospodinu v obětní dým, budou se omývat vodou, aby nezemřeli.
#30:21 Budou si omývat ruce i nohy, aby nezemřeli. To je provždy platné nařízení pro něho i pro jeho potomstvo po všechna jejich pokolení.“
#30:22 Hospodin dále mluvil k Mojžíšovi:
#30:23 „Ty si pak vezmi nejvzácnější balzámy: pět set šekelů tekuté myrhy, poloviční množství, totiž dvě stě padesát šekelů balzámové skořice, dvě stě padesát šekelů puškvorce,
#30:24 pět set šekelů kasie podle váhy šekelu svatyně a jeden hín olivového oleje.
#30:25 Z toho připravíš olej svatého pomazání, vonnou mast odborně smísenou. To je olej svatého pomazání.
#30:26 Pomažeš jím stan setkávání a schránu svědectví,
#30:27 stůl se vším náčiním, svícen s náčiním a kadidlový oltář,
#30:28 rovněž oltář pro zápalnou oběť se vším náčiním a nádrž s podstavcem.
#30:29 Posvětíš je a budou velesvaté. Cokoli se jich dotkne, bude svaté.
#30:30 Pomažeš Árona a jeho syny a posvětíš je, aby mi sloužili jako kněží.
#30:31 Izraelcům pak vyhlásíš: To je olej svatého pomazání pro mne po všechna vaše pokolení.
#30:32 Nesmí se vylít na tělo nepovolaného člověka. Nepřipravíte podobný olej stejného složení. Je svatý a zůstane pro vás svatý.
#30:33 Každý, kdo namíchá podobnou mast nebo z ní dá nepovolanému, bude vyobcován ze svého lidu.“
#30:34 Hospodin řekl Mojžíšovi: „Vezmi si vonné látky, totiž čerstvou pryskyřici, vonné lastury, klovatinu galbanum a čisté kadidlo, od všeho stejný díl.
#30:35 Z toho připravíš kadidlovou směs, odborně smísenou, posolenou, čistou a svatou.
#30:36 Část jemně rozetři a přines před svědectví ve stanu setkávání, kde se s tebou budu setkávat. To vám bude velesvaté.
#30:37 Kadidlo, které uděláš - ve stejném složení si podobné neuděláte - bude ti svaté, je jen pro Hospodina.
#30:38 Kdokoli by zhotovil podobné, aby vdechoval jeho vůni, bude vyobcován ze svého lidu.“ 
#31:1 Hospodin promluvil k Mojžíšovi:
#31:2 „Hleď, povolal jsem jménem Besaleela, syna Urího, vnuka Chúrova, z pokolení Judova.
#31:3 Naplnil jsem ho Božím duchem, totiž moudrostí, důvtipem a znalostí každého díla,
#31:4 aby uměl dovedně pracovat se zlatem, stříbrem a mědí,
#31:5 opracovávat kameny pro vsazování a obrábět dřevo k zhotovení jakéhokoli díla.
#31:6 Dal jsem mu také k ruce Oholíaba, syna Achísamakova z pokolení Danova. A do srdce každého dovedného řemeslníka jsem vložil moudrost, aby zhotovili všecko, co jsem ti přikázal:
#31:7 stan setkávání a schránu svědectví, též příkrov na ni a všecko náčiní stanu,
#31:8 stůl s náčiním, svícen z čistého zlata s veškerým náčiním a kadidlový oltář;
#31:9 i oltář pro zápalné oběti s veškerým náčiním a nádrž s podstavcem;
#31:10 také jemně tkaná roucha, totiž svatá roucha pro kněze Árona a roucha jeho synů ke kněžské službě;
#31:11 i olej k pomazání a kadidlo z vonných látek pro svatyni. Ať učiní všechno, jak jsem ti přikázal.“
#31:12 Hospodin řekl Mojžíšovi:
#31:13 „Promluv k Izraelcům: Dbejte na mé dny odpočinku; to je znamení mezi mnou a vámi pro všechna vaše pokolení, abyste věděli, že já Hospodin vás posvěcuji.
#31:14 Budete dbát na den odpočinku; má být pro vás svatý. Kdo jej znesvětí, musí zemřít. Každý, kdo by v něm dělal nějakou práci, bude vyobcován ze společenství svého lidu.
#31:15 Šest dní se bude pracovat, ale sedmého dne bude slavnost odpočinutí, Hospodinův svatý den odpočinku. Každý, kdo by dělal nějakou práci v den odpočinku, musí zemřít.
#31:16 Ať tedy Izraelci dbají na den odpočinku a dodržují jej po všechna svá pokolení jako věčnou smlouvu.
#31:17 To je provždy platné znamení mezi mnou a syny Izraele. V šesti dnech totiž učinil Hospodin nebe a zemi, ale sedmého dne odpočinul a oddechl si.“
#31:18 Když přestal k Mojžíšovi na hoře Sínaji mluvit, dal mu dvě desky svědectví; byly to kamenné desky psané Božím prstem. 
#32:1 Když lid viděl, že Mojžíš dlouho nesestupuje z hory, shromáždil se k Áronovi a naléhali na něho: „Vstaň a udělej nám boha, který by šel před námi. Vždyť nevíme, co se stalo s Mojžíšem, s tím člověkem, který nás vyvedl z egyptské země.“
#32:2 Áron jim řekl: „Strhněte zlaté náušnice z uší svých žen, synů a dcer a přineste je ke mně!“
#32:3 I strhal si všechen lid z uší zlaté náušnice a přinesli je k Áronovi.
#32:4 On je od nich vzal, připravil formu a odlil z toho sochu býčka. A oni řekli: „To je tvůj bůh, Izraeli, který tě vyvedl z egyptské země.“
#32:5 Když to Áron viděl, vybudoval před ním oltář. Potom Áron provolal: „Zítra bude Hospodinova slavnost.“
#32:6 Nazítří za časného jitra obětovali oběti zápalné a přinesli oběti pokojné. Pak se lid usadil k jídlu a pití. Nakonec se dali do nevázaných her.
#32:7 I promluvil Hospodin k Mojžíšovi: „Sestup dolů. Tvůj lid, který jsi vyvedl z egyptské země, se vrhá do zkázy.
#32:8 Brzy uhnuli z cesty, kterou jsem jim přikázal. Odlili si sochu býčka a klanějí se mu, obětují mu a říkají: ‚To je tvůj bůh, Izraeli, který tě vyvedl z egyptské země.‘“
#32:9 Hospodin dále Mojžíšovi řekl: „Viděl jsem tento lid, je to lid tvrdé šíje.
#32:10 Teď mě nech, ať proti nim vzplane můj hněv a skoncuji s nimi; z tebe však udělám veliký národ.“
#32:11 Mojžíš však prosil Hospodina, svého Boha, o shovívavost: „Hospodine, proč plane tvůj hněv proti tvému lidu, který jsi vyvedl velikou silou a pevnou rukou z egyptské země?
#32:12 Proč mají Egypťané říkat: ‚Vyvedl je se zlým úmyslem, aby je v horách povraždil a nadobro je smetl z povrchu země.‘ Upusť od svého planoucího hněvu. Dej se pohnout k lítosti nad zlem, jež proti svému lidu zamýšlíš.
#32:13 Rozpomeň se na Abrahama, na Izáka a na Izraele, své služebníky, kterým jsi sám při sobě přísahal a vyhlásil: Rozmnožím vaše potomstvo jako nebeské hvězdy a celou tuto zemi, jak jsem řekl, dám vašemu potomstvu, aby ji navěky mělo v dědictví.“
#32:14 A Hospodin se dal pohnout k lítosti nad zlem, o němž mluvil, že je dopustí na svůj lid.
#32:15 Mojžíš se obrátil a sestupoval z hory s dvěma deskami svědectví v ruce. Desky byly psány po obou stranách, byly popsané po líci i po rubu.
#32:16 Ty desky byly dílo Boží, i písmo vyryté na deskách bylo Boží.
#32:17 Tu Jozue uslyšel, jak lid hlučí, a řekl Mojžíšovi: „V táboře je válečný ryk.“
#32:18 Ale on odvětil: „To nezní zpěvy vítězů, to nezní zpěvy poražených, já slyším halas rozpustilých písní.“
#32:19 Když se Mojžíš přiblížil k táboru a uviděl býčka a křepčení, vzplanul hněvem, odhodil desky z rukou a pod horou je roztříštil.
#32:20 Pak vzal býčka, kterého udělali, spálil jej ohněm, rozemlel na prach, nasypal do vody a dal Izraelcům pít.
#32:21 Áronovi Mojžíš řekl: „Co ti tento lid udělal, že jsi naň uvedl tak veliký hřích?“
#32:22 Áron odvětil: „Nechť můj pán tolik neplane hněvem! Ty víš, že tento lid je nakloněn ke zlému.
#32:23 Řekli mi: ‚Udělej nám boha, který by šel před námi. Vždyť nevíme, co se stalo s Mojžíšem, s tím člověkem, který nás vyvedl z egyptské země.‘
#32:24 Řekl jsem jim: Kdo má zlaté náušnice, ať si je strhne a donese mi je. Hodil jsem to do ohně, a tak vznikl tento býček.“
#32:25 Mojžíš viděl, jak si lid bezuzdně počíná; to Áron jej nechal počínat si bezuzdně ke škodolibosti jejich protivníků.
#32:26 Tu se Mojžíš postavil do brány tábora a zvolal: „Kdo je Hospodinův, ke mně!“ Seběhli se k němu všichni Léviovci.
#32:27 Řekl jim: „Toto praví Hospodin, Bůh Izraele: Všichni si připněte k boku meč, projděte táborem tam i zpět od brány k bráně a zabijte každý svého bratra, každý svého přítele, každý svého nejbližšího.“
#32:28 Léviovci vykonali, co jim Mojžíš rozkázal; toho dne padlo z lidu na tři tisíce mužů.
#32:29 Potom Mojžíš řekl: „Ujměte se dnes svého úřadu pro Hospodina každý, kdo povstal proti svému synovi či bratru, aby vám dnes dal požehnání.“
#32:30 Nazítří pak Mojžíš řekl lidu: „Dopustili jste se velikého hříchu; avšak nyní vystoupím k Hospodinu, snad jej za váš hřích usmířím.“
#32:31 Mojžíš se tedy vrátil k Hospodinu a řekl: „Ach, tento lid se dopustil velikého hříchu, udělali si zlatého boha.
#32:32 Můžeš jim ten hřích ještě odpustit? Ne-li, vymaž mě ze své knihy, kterou píšeš!“
#32:33 Ale Hospodin Mojžíšovi odpověděl: „Vymažu ze své knihy toho, kdo proti mně zhřešil.
#32:34 A ty nyní jdi a veď lid, jak jsem ti řekl. Hle, můj posel půjde před tebou. A až přijde den mého trestu, potrestám jejich hřích na nich.“
#32:35 Hospodin udeřil na lid, protože si dali udělat býčka, kterého zhotovil Áron. 
#33:1 Hospodin promluvil k Mojžíšovi: „Vyjdi odtud, ty i lid, který jsi vyvedl ze země egyptské, do země, kterou jsem přísežně slíbil Abrahamovi, Izákovi a Jákobovi: Dám ji tvému potomstvu.
#33:2 Pošlu před tebou svého posla a vypudím Kenaance, Emorejce, Chetejce, Perizejce, Chivejce i Jebúsejce.
#33:3 Půjdete do země oplývající mlékem a medem. Já však nepůjdu uprostřed vás, abych vás cestou nevyhubil, neboť jste lid tvrdošíjný.“
#33:4 Když lid uslyšel tuto zlou zprávu, začali truchlit a nikdo na sebe nevzal žádnou ozdobu.
#33:5 Potom řekl Hospodin Mojžíšovi: „Řekni Izraelcům: Jste tvrdošíjný lid. Kdybych šel jediný okamžik uprostřed vás, musel bych vás vyhladit. Nyní však ze sebe složte své ozdoby, ať vím, jak mám s vámi naložit.“
#33:6 Od hory Chorébu se tedy Izraelci zbavili všech ozdob.
#33:7 Mojžíš vzal stan a postavil si jej venku za táborem opodál tábora a nazval jej stanem setkávání. Když někdo hledal Hospodina, vycházel ke stanu setkávání, který byl venku za táborem.
#33:8 A když Mojžíš vycházel ke stanu, všechen lid povstával; zůstali stát, každý u vchodu do svého stanu, a hleděli za Mojžíšem, dokud nevešel do stanu.
#33:9 Kdykoli Mojžíš vcházel do stanu, sestupoval oblakový sloup a stál u vchodu do stanu. A Hospodin rozmlouval s Mojžíšem.
#33:10 Všechen lid viděl oblakový sloup, stojící u vchodu do stanu; tu všechen lid povstával a klaněli se, každý u vchodu do svého stanu.
#33:11 A Hospodin mluvil s Mojžíšem tváří v tvář, jako když někdo mluví se svým přítelem. Potom se Mojžíš vracel do tábora. Ale mládenec Jozue, syn Núnův, který mu přisluhoval, se ze stanu nevzdaloval.
#33:12 Mojžíš řekl Hospodinu: „Hleď, ty mi říkáš: Vyveď tento lid. Ale nesdělil jsi mi, koho chceš se mnou poslat, ačkoli jsi řekl: ‚Já tě znám jménem, našel jsi u mne milost.‘
#33:13 Jestliže jsem tedy nyní u tebe našel milost, dej mi poznat svou cestu, abych poznal tebe a našel u tebe milost; pohleď, vždyť tento pronárod je tvůj lid.“
#33:14 Odvětil: „Já sám půjdu s vámi a dám vám odpočinutí.“
#33:15 Mojžíš mu řekl: „Kdyby s námi neměla být tvá přítomnost, pak nás odtud nevyváděj!
#33:16 Podle čeho jiného by se poznalo, že jsem u tebe našel milost já i tvůj lid, ne-li podle toho, že s námi půjdeš; tím budeme odlišeni, já i tvůj lid, od každého lidu na tváři země.“
#33:17 Hospodin Mojžíšovi odvětil: „Učiním i tuto věc, o které mluvíš, protože jsi u mne našel milost a já tě znám jménem.“
#33:18 I řekl: „Dovol mi spatřit tvou slávu!“
#33:19 Hospodin odpověděl: „Všechna má dobrota přejde před tebou a vyslovím před tebou jméno Hospodin. Smiluji se však, nad kým se smiluji, a slituji se, nad kým se slituji.“
#33:20 Dále pravil: „Nemůžeš spatřit mou tvář, neboť člověk mě nesmí spatřit, má-li zůstat naživu.“
#33:21 Hospodin pravil: „Hle, u mne je místo; postav se na skálu.
#33:22 Až tudy půjde moje sláva, postavím tě do skalní rozsedliny a zakryji tě svou dlaní, dokud nepřejdu.
#33:23 Až dlaň odtáhnu, spatříš mě zezadu, ale mou tvář nespatří nikdo.“ 
#34:1 Hospodin řekl Mojžíšovi: „Vytesej si dvě kamenné desky jako ty první, já na ty desky napíšu slova, která byla na prvních deskách, jež jsi roztříštil.
#34:2 Připrav se na ráno; vystoupíš zrána na horu Sínaj a postavíš se tam na vrcholku hory ke mně.
#34:3 Nikdo s tebou nevystoupí, a též ať se na celé hoře nikdo neukáže, ani brav nebo skot ať se nepase poblíž té hory.“
#34:4 Mojžíš tedy vytesal dvě kamenné desky, jako ty první. Za časného jitra vystoupil na horu Sínaj, jak mu Hospodin přikázal, a do rukou vzal obě kamenné desky.
#34:5 Tu sestoupil Hospodin v oblaku. Mojžíš tam zůstal stát s ním a vzýval Hospodinovo jméno.
#34:6 Když Hospodin kolem něho přecházel, zavolal: „Hospodin, Hospodin! Bůh plný slitování a milostivý, shovívavý, nejvýš milosrdný a věrný,
#34:7 který osvědčuje milosrdenství tisícům pokolení, který odpouští vinu, přestoupení a hřích; avšak viníka nenechává bez trestu, stíhá vinu otců na synech i na vnucích do třetího a čtvrtého pokolení.“
#34:8 Mojžíš rychle padl na kolena tváří k zemi, klaněl se
#34:9 a řekl: „Jestliže jsem, Panovníku, nalezl u tebe milost, putuj prosím, Panovníku, uprostřed nás. Je to lid tvrdošíjný; promiň nám však vinu a hřích a přijmi nás jako dědictví.“
#34:10 Hospodin odpověděl: „Hle, uzavírám s vámi smlouvu. Před veškerým tvým lidem učiním podivuhodné věci, jaké nebyly stvořeny na celé zemi ani mezi všemi pronárody. Všechen lid, uprostřed něhož jsi, uvidí Hospodinovo dílo; neboť to, co já s tebou učiním, bude vzbuzovat bázeň.
#34:11 Bedlivě dbej na to, co ti dnes přikazuji. Hle, vypudím před tebou Emorejce, Kenaance, Chetejce, Perizejce, Chivejce a Jebúsejce.
#34:12 Dej si pozor, abys neuzavíral smlouvu s obyvateli té země, do které vejdeš, aby se nestali uprostřed tebe léčkou.
#34:13 Proto jejich oltáře poboříte, jejich posvátné sloupy roztříštíte a jejich posvátné kůly pokácíte.
#34:14 Nebudeš se klanět jinému bohu, protože Hospodin, jehož jméno je Žárlivý, je Bůh žárlivě milující.
#34:15 Neuzavřeš smlouvu s obyvateli té země. Budou se svými bohy smilnit a svým bohům obětovat a tebe pozvou, abys jedl z jejich obětního hodu;
#34:16 a ty budeš brát z jejich dcer manželky pro své syny a jejich dcery budou se svými bohy smilnit a svádět tvé syny, aby smilnili s jejich bohy.
#34:17 Nebudeš si odlévat sochy bohů.
#34:18 Budeš dbát na slavnost nekvašených chlebů. Sedm dní budeš jíst nekvašené chleby, jak jsem ti přikázal, ve stanovený čas v měsíci ábíbu; neboť v měsíci ábíbu jsi vyšel z Egypta.
#34:19 Všechno, co otvírá lůno, bude patřit mně, i každý samec z prvého vrhu tvého stáda, skotu i bravu.
#34:20 Osla, který se narodil jako první, vyplatíš ovcí; jestliže jej nevyplatíš, zlomíš mu vaz. Vyplatíš každého svého prvorozeného syna. Nikdo se neukáže před mou tváří s prázdnou.
#34:21 Šest dní budeš pracovat, ale sedmého dne odpočineš; i při orbě a při žni odpočineš.
#34:22 Budeš slavit slavnost týdnů, prvních snopků pšeničné žně, a slavnost sklizně na přelomu roku.
#34:23 Třikrát v roce se každý z vás, kdo je mužského pohlaví, ukáže před Pánem Hospodinem, Bohem Izraele.
#34:24 Vyženu před tebou pronárody a rozšířím tvé pomezí. Nikdo nebude žádostiv tvé země, když se půjdeš třikrát v roce ukázat před Hospodinem, svým Bohem.
#34:25 Nebudeš porážet dobytče tak, aby krev mého obětního hodu vytekla na něco kvašeného. Nebudeš přechovávat přes noc do rána nic z oběti při slavnosti hodu beránka.
#34:26 Prvotiny raných plodů své role přineseš do domu Hospodina, svého Boha. Nebudeš vařit kůzle v mléku jeho matky.“
#34:27 Hospodin řekl Mojžíšovi: „Napiš si tato slova, neboť podle těchto slov uzavírám s tebou a s Izraelem smlouvu.“
#34:28 A byl tam s Hospodinem čtyřicet dní a čtyřicet nocí; chleba nepojedl a vody se nenapil, nýbrž psal na desky slova smlouvy, desatero přikázání.
#34:29 Když pak Mojžíš sestupoval z hory Sínaje, měl při sestupu z hory desky svědectví v rukou. Mojžíš nevěděl, že mu od rozhovoru s Hospodinem září kůže na tváři.
#34:30 Když Áron a všichni Izraelci uviděli, jak Mojžíšovi září kůže na tváři, báli se k němu přistoupit.
#34:31 Ale Mojžíš je zavolal, i vrátili se k němu Áron a všichni předáci pospolitosti a Mojžíš k nim promluvil.
#34:32 Potom přistoupili všichni Izraelci a on jim přikázal všechno, o čem s ním Hospodin mluvil na hoře Sínaji.
#34:33 Když k nim Mojžíš přestal mluvit, dal si na tvář závoj.
#34:34 Kdykoli Mojžíš vstupoval před Hospodina, aby s ním mluvil, odkládal závoj, dokud nevyšel. Pak vycházel, aby k Izraelcům mluvil, co mu bylo přikázáno.
#34:35 Izraelci spatřili Mojžíšovu tvář a viděli, jak mu kůže na tváři září. Proto si Mojžíš dával na tvář závoj, pokud nešel mluvit s Hospodinem. 
#35:1 Mojžíš shromáždil celou pospolitost Izraelců a řekl jim: „Hospodin přikázal, abyste dodržovali toto:
#35:2 Šest dní se bude pracovat, ale sedmého dne budete mít slavnost odpočinutí, Hospodinův svatý den odpočinku; kdo by v ten den dělal nějakou práci, zemře.
#35:3 V den odpočinku nerozděláte oheň v žádném svém obydlí.“
#35:4 Mojžíš řekl celé pospolitosti Izraelců: „Toto přikázal Hospodin:
#35:5 Vyberte mezi sebou pro Hospodina oběť pozdvihování. Každý ať ze srdce dobrovolně přinese jako Hospodinovu oběť pozdvihování zlato, stříbro a měď,
#35:6 látku purpurově fialovou, nachovou nebo karmínovou, jemné plátno, kozí srst,
#35:7 načerveno zbarvené beraní kůže, tachaší kůže, akáciové dřevo,
#35:8 olej na svícení, balzámy na olej k pomazání a na kadidlo z vonných látek,
#35:9 karneolové drahokamy a kameny pro zasazení do nárameníku a náprsníku.
#35:10 A každý, kdo je mezi vámi dovedný, ať přijde a dělá vše, co Hospodin přikázal:
#35:11 příbytek, jeho stan a přikrývku, spony a desky, svlaky, sloupy a patky,
#35:12 schránu a tyče k ní, příkrov a vnitřní oponu,
#35:13 stůl a tyče k němu i veškeré náčiní, také předkladné chleby,
#35:14 svícen k svícení a náčiní k němu i kahánky a olej k svícení,
#35:15 kadidlový oltář a tyče k němu, olej k pomazání a kadidlo z vonných látek i vstupní závěs ke vchodu do příbytku,
#35:16 oltář pro zápalnou oběť s bronzovým roštem, tyče k němu a veškeré náčiní, nádrž a podstavec k ní,
#35:17 zástěny pro nádvoří, jeho sloupy a patky i závěs pro bránu do nádvoří,
#35:18 kolíky pro příbytek, kolíky pro nádvoří a příslušná lana,
#35:19 tkaná roucha pro přisluhování ve svatyni, svatá roucha pro kněze Árona a roucha jeho synům pro kněžskou službu.“
#35:20 Celá pospolitost Izraelců vyšla od Mojžíše
#35:21 a každý, koho srdce pudilo a kdo byl ochotné mysli, přicházel a přinášel Hospodinovu oběť pozdvihování pro dílo na stanu setkávání a pro veškerou službu v něm i pro svatá roucha.
#35:22 Přicházeli muži i ženy a všichni ze srdce dobrovolně přinášeli spínadla, kroužky, prsteny a přívěsky, všelijaké zlaté předměty, totiž každý, kdo nabídl Hospodinu zlato jako oběť podávání.
#35:23 Každý, kdo měl látku purpurově fialovou, nachovou a karmínovou, jemné plátno a kozí srst, načerveno zbarvené beraní kůže a tachaší kůže, přinášel je.
#35:24 Každý, kdo chtěl přinést oběť pozdvihování ve stříbře a v mědi, přinesl to Hospodinu jako oběť pozdvihování. Každý, kdo měl akáciové dřevo k jakémukoli bohoslužebnému dílu, přinášel je.
#35:25 I všechny dovedné ženy vlastnoručně předly a přinášely, co napředly, látku purpurově fialovou, nachovou a karmínovou a jemné plátno.
#35:26 Všechny ženy, které srdce pudilo a které to dovedly, předly kozí srst.
#35:27 Předáci přinášeli karneoly a kameny pro zasazení do nárameníku a náprsníku
#35:28 i balzám a olej na svícení i na olej k pomazání a na kadidlo z vonných látek.
#35:29 I přinášeli Izraelci dobrovolný dar Hospodinu, každý muž a žena, které pudilo srdce, dobrovolně přinášeli všechno potřebné k dílu, které skrze Mojžíše přikázal Hospodin vykonat.
#35:30 Mojžíš řekl Izraelcům: „Hleďte, Hospodin povolal jménem Besaleela, syna Urího, vnuka Chúrova, z pokolení Judova.
#35:31 Naplnil ho Božím duchem, aby měl moudrost, důvtip a znalost pro každé dílo
#35:32 a uměl dovedně pracovat se zlatem, stříbrem a mědí,
#35:33 opracovávat kameny pro vsazování a obrábět dřevo, aby dovedl udělat každé navržené dílo.
#35:34 Dal mu i schopnost vyučovat, jemu i Oholíabovi, synu Achísamakovu z pokolení Danova.
#35:35 Obdařil je dovedností zhotovovat jakékoli dílo řemeslnické, umělecké, výšivkářské, na látce purpurově fialové, nachové a karmínové a na jemném plátně, i dílo tkalcovské; aby dovedli udělat jakékoli dílo a pracovat s vynalézavostí.“ 
#36:1 Besaleel s Oholíabem a všichni, kdo byli dovední a jimž dal Hospodin moudrost a důvtip, aby se vyznali ve veškerém díle pro službu ve svatyni, dali se tedy do práce podle toho, jak Hospodin přikázal.
#36:2 Mojžíš povolal Besaleela s Oholíabem a všechny dovedné, jež Hospodin obdařil moudrostí, všechny, jež srdce pudilo, aby přistoupili k práci na tom díle.
#36:3 Převzali od Mojžíše každou oběť pozdvihování, kterou Izraelci přinesli, aby bylo zhotoveno dílo pro službu ve svatyni. Lid mu přinášel i dále každé ráno dobrovolné dary.
#36:4 Všichni moudří, konající veškeré dílo pro svatyni, přicházeli jeden za druhým od práce, kterou konali,
#36:5 a říkali Mojžíšovi: „Lid přináší víc, než je třeba k provedení díla, které Hospodin přikázal vykonat.“
#36:6 Mojžíš tedy přikázal a jeho vzkaz roznesli po táboře: „Muži a ženy, nechystejte už nic pro oběť pozdvihování ve prospěch svatyně.“ Tak zabránili lidu, aby přinášel další dary.
#36:7 Měli dostatek potřeb pro veškeré dílo, které se mělo udělat, a ještě zbývalo.
#36:8 Všichni, kdo byli dovední mezi těmi, kdo na díle pracovali, zhotovili příbytek z desíti pruhů jemně tkaného plátna a látky purpurově fialové, nachové a karmínové; zhotovili je s umně vetkanými cheruby.
#36:9 Jeden pruh byl dvacet osm loket dlouhý a čtyři lokte široký. Všechny pruhy měly stejné rozměry.
#36:10 Pět pruhů spojil po délce jeden s druhým a druhých pět pruhů spojil právě tak jeden s druhým.
#36:11 Z fialového purpuru zhotovil poutka na lemu toho pruhu, který byl na konci spojeného kusu. Stejně to udělal na lemu krajního pruhu druhého spojeného kusu.
#36:12 Udělal také padesát poutek na jednom koncovém pruhu a padesát poutek na koncovém pruhu druhého spojeného kusu, vždy jedno proti druhému.
#36:13 Zhotovil i padesát zlatých spon a sepnul jimi pruhy jeden s druhým, takže příbytek byl spojen v jeden celek.
#36:14 Zhotovil též houně z kozí srsti pro stan nad příbytkem; těch houní udělal jedenáct.
#36:15 Jedna houně byla třicet loket dlouhá a čtyři lokte široká. Všech jedenáct houní mělo stejné rozměry.
#36:16 Zvlášť spojil pět houní a zvlášť šest houní.
#36:17 Udělal také padesát poutek na lemu houně, která byla na kraji spojeného kusu, a padesát poutek udělal na lemu koncové houně druhého spojeného kusu.
#36:18 Zhotovil také padesát bronzových spon a spojil stan v jeden celek.
#36:19 Zhotovil pro stan též přikrývku z beraních kůží zbarvených načerveno a navrch přikrývku z tachaších kůží.
#36:20 Pro příbytek zhotovil také desky z akáciového dřeva, aby se daly postavit.
#36:21 Každá deska byla deset loket dlouhá a půldruhého lokte široká.
#36:22 Každá měla dva čepy a jedna byla připojena k druhé; tak to provedl na všech deskách pro příbytek.
#36:23 Zhotovil pro příbytek tyto desky: dvacet desek pro jižní, pravou stranu.
#36:24 Těch dvacet desek opatřil dole čtyřiceti stříbrnými patkami, po dvou patkách pod každou desku, k oběma jejím čepům.
#36:25 I pro druhý bok příbytku, k severní straně, zhotovil dvacet desek
#36:26 a k nim čtyřicet stříbrných patek, po dvou patkách pod každou desku.
#36:27 Pro zadní stranu příbytku, k západu, zhotovil šest desek.
#36:28 Navíc zhotovil dvě desky pro oba úhly příbytku při zadní straně.
#36:29 Zdola byly přiloženy k sobě, navrchu byly těsně spojeny jedním kruhem. Tak učinil s oběma deskami v obou úhlech.
#36:30 Bylo tam tedy osm desek se stříbrnými patkami, celkem šestnáct patek, po dvou patkách pod každou deskou.
#36:31 Zhotovil také svlaky z akáciového dřeva, pět pro desky na jednom boku příbytku,
#36:32 pět na druhém boku příbytku a dalších pět svlaků pro desky na zadní straně příbytku, k západu.
#36:33 Udělal i prostřední svlak, aby probíhal v poloviční výši desek od jednoho konce k druhému.
#36:34 Desky potáhl zlatem a zhotovil k nim zlaté kruhy pro vsunutí svlaků; rovněž svlaky potáhl zlatem.
#36:35 Zhotovil také oponu z látky purpurově fialové, nachové a karmínové a z jemně tkaného plátna. Zhotovil ji s umně vetkanými cheruby.
#36:36 Udělal pro ni čtyři sloupy z akáciového dřeva a potáhl je zlatem; háčky na nich byly ze zlata; odlil pro ně také čtyři stříbrné patky.
#36:37 Ke vchodu do stanu zhotovil pestře vyšitý závěs z látky purpurově fialové, nachové a karmínové a z jemně tkaného plátna.
#36:38 K tomu pět sloupů s háčky; jejich hlavice a příčné tyče potáhl zlatem. Jejich pět patek bylo z bronzu. 
#37:1 Besaleel zhotovil schránu z akáciového dřeva dva a půl lokte dlouhou, jeden a půl lokte širokou a jeden a půl lokte vysokou.
#37:2 Obložil ji uvnitř i zvnějšku čistým zlatem a opatřil ji dokola zlatou obrubou.
#37:3 Odlil pro ni čtyři zlaté kruhy a připevnil je k čtyřem jejím hranám; dva kruhy na jednom boku a dva kruhy na druhém.
#37:4 Zhotovil tyče z akáciového dřeva a potáhl je zlatem.
#37:5 Tyče prostrčil kruhy po stranách schrány, aby bylo možno schránu nést.
#37:6 Zhotovil příkrov z čistého zlata dva a půl lokte dlouhý a jeden a půl lokte široký.
#37:7 Potom zhotovil dva cheruby ze zlata; dal je vytepat na oba konce příkrovu.
#37:8 Jeden cherub byl na jednom konci a druhý cherub na druhém konci. Udělal cheruby na příkrov, na oba jeho konce.
#37:9 Cherubové měli křídla rozpjatá vzhůru; svými křídly zastírali příkrov. Tvářemi byli obráceni k sobě; hleděli na příkrov.
#37:10 Zhotovil stůl z akáciového dřeva dva lokte dlouhý, jeden loket široký a jeden a půl lokte vysoký.
#37:11 Obložil jej čistým zlatem a opatřil jej dokola zlatou obrubou.
#37:12 Udělal mu také dokola na dlaň širokou lištu a k liště udělal dokola zlatou obrubu.
#37:13 Odlil pro něj čtyři zlaté kruhy a připevnil je ke čtyřem rohům při jeho čtyřech nohách.
#37:14 Kruhy byly těsně u lišty, aby držely tyče na nošení stolu.
#37:15 Tyče zhotovil z akáciového dřeva a potáhl je zlatem; sloužily k nošení stolu.
#37:16 Zhotovil také nádoby, které patří na stůl, mísy, pánvičky, obětní misky a konvice, používané k úlitbě, vše z čistého zlata.
#37:17 Zhotovil svícen z čistého zlata. Svícen měl vytepaný dřík a prut; jeho kalichy, číšky a květy byly s ním zhotoveny z jednoho kusu.
#37:18 Z jeho stran vycházelo šest prutů, tři pruty svícnu po jedné straně a tři pruty svícnu po druhé straně.
#37:19 Na jednom prutu byly tři kalichy podobné mandloňovému květu: číška a květ. A tři kalichy podobné mandloňovému květu na druhém prutu: číška a květ. Tak to bylo na všech šesti prutech vycházejících ze svícnu.
#37:20 Na svícnu byly čtyři kalichy podobné mandloňovému květu s číškami a květy.
#37:21 Jedna číška byla pod jednou dvojicí prutů, druhá číška byla pod druhou dvojicí prutů a třetí číška byla pod třetí dvojicí prutů; tak to bylo u všech šesti prutů vycházejících ze svícnu.
#37:22 Příslušné číšky a pruty byly s ním zhotoveny z jednoho kusu; všechno bylo v celku vytepáno z čistého zlata.
#37:23 Ke svícnu zhotovil také sedm kahánků, nůžky na knoty a pánve z čistého zlata na oharky.
#37:24 Svícen se všemi předměty zhotovil z jednoho talentu čistého zlata.
#37:25 Zhotovil kadidlový oltář z akáciového dřeva, čtyřhranný, loket dlouhý, loket široký a dva lokte vysoký; jeho rohy byly z jednoho kusu s ním.
#37:26 Potáhl jej čistým zlatem, jeho vršek i jeho stěny dokola a jeho rohy, a opatřil jej dokola zlatou obrubou.
#37:27 Zhotovil pro něj též dva zlaté kruhy, a to pod obrubou při jeho bocích, k oběma bočnicím, aby držely tyče, na nichž byl nošen.
#37:28 Tyče zhotovil z akáciového dřeva a potáhl je zlatem.
#37:29 Připravil také olej svatého pomazání a kadidlo z vonných látek, čisté, odborně smísené. 
#38:1 Zhotovil oltář pro zápalnou oběť z akáciového dřeva, čtyřhranný, pět loket dlouhý, pět loket široký a tři lokte vysoký.
#38:2 Jeho čtyři úhly opatřil rohy; ty s ním byly zhotoveny z jednoho kusu. Potáhl jej bronzem.
#38:3 Zhotovil také všechno oltářní náčiní: hrnce, lopaty, kropenky, vidlice a pánve na oheň. Všechno náčiní zhotovil z bronzu.
#38:4 Zhotovil pro oltář také mřížový rošt z bronzu pod obložení oltáře, sahající do jeho poloviny.
#38:5 Na čtyřech koncích bronzového roštu odlil čtyři kruhy na prostrčení tyčí.
#38:6 Zhotovil i tyče z akáciového dřeva a potáhl je bronzem.
#38:7 Tyče provlékl kruhy po bocích oltáře, aby jej bylo možno na nich přenášet. Udělal jej dutý z desek.
#38:8 Zhotovil bronzovou nádrž s bronzovým podstavcem ze zrcadel žen, které konaly službu u vchodu do stanu setkávání.
#38:9 Udělal také nádvoří. Na jižní, pravé straně nádvoří byly zástěny z jemně tkaného plátna, v délce sto loket.
#38:10 Bylo tam dvacet sloupů na dvaceti bronzových patkách; háčky ke sloupům a příčné tyče k nim byly stříbrné.
#38:11 Na severní straně v délce sto loket bylo dvacet sloupů na dvaceti bronzových patkách; háčky ke sloupům a příčné tyče k nim byly stříbrné.
#38:12 Na západní straně byly zástěny v délce padesáti loket. Bylo tam deset sloupů na deseti patkách; háčky ke sloupům a příčné tyče k nim byly stříbrné.
#38:13 Na přední, východní straně byla šířka padesát loket.
#38:14 Na jednom křídle bylo patnáct loket zástěn; byly tam tři sloupy na třech patkách.
#38:15 I na druhém křídle, z této strany brány do nádvoří jako z oné, bylo patnáct loket zástěn; byly tam také tři sloupy na třech patkách.
#38:16 Všechny zástěny kolem nádvoří byly z jemně tkaného plátna.
#38:17 Patky ke sloupům byly bronzové, háčky ke sloupům a příčné tyče k nim byly stříbrné, i obložení jejich hlavic bylo stříbrné; všechny sloupy na nádvoří byly spojeny stříbrnými příčnými tyčemi.
#38:18 Závěs v bráně na nádvoří byl pestře vyšit z látky fialově purpurové, nachové a karmínové a z jemně tkaného plátna v délce dvaceti loket, vysoký či široký pět loket, právě tak jako zástěny na nádvoří.
#38:19 Byly tu čtyři sloupy na čtyřech bronzových patkách, jejich háčky byly stříbrné, i obložení jejich hlavic a příčné tyče byly stříbrné.
#38:20 Všechny kolíky pro příbytek a pro nádvoří dokola byly z bronzu.
#38:21 Toto jsou ti, kdo byli povoláni k službě u příbytku, u příbytku svědectví, pověření na Mojžíšův rozkaz lévijskou službou, za dozoru Ítamara, syna kněze Árona:
#38:22 Besaleel, syn Uríův, vnuk Chúrův z pokolení Judova, zhotovil všechno, co Hospodin Mojžíšovi přikázal.
#38:23 S ním Oholíab, syn Achísamakův z pokolení Danova, řemeslník, umělec, zhotovující výšivky z látky purpurově fialové, nachové a karmínové a z jemného plátna.
#38:24 Všechno zlato, zpracované při tom díle, při celém díle na svatyni, bylo zlato obětované podáváním, dvacet devět talentů a sedm set třicet šekelů podle váhy určené svatyní.
#38:25 Stříbra bylo od těch, kdo byli z pospolitosti povoláni k službě, sto talentů a tisíc sedm set sedmdesát pět šekelů podle váhy určené svatyní.
#38:26 Půl šekelu na hlavu, polovina šekelu podle váhy určené svatyní, za každého, kdo přešel mezi ty, kteří byli povoláni do služby od dvacetiletých výše, za šest set tři tisíce a pět set padesát mužů.
#38:27 Sto talentů stříbra bylo použito k odlití patek pro svatyni a patek k oponě, sto patek z jednoho sta talentů, jeden talent na jednu patku.
#38:28 Z tisíce sedmi set sedmdesáti pěti šekelů udělal háčky ke sloupům, obložil jejich hlavice a spojil je příčkami.
#38:29 Mědi obětované podáváním bylo sedmdesát talentů a dva tisíce čtyři sta šekelů.
#38:30 Z toho udělal patky ke vchodu do stanu setkávání, bronzový oltář a k němu bronzový rošt a všechno náčiní k oltáři,
#38:31 patky kolem nádvoří, patky k bráně do nádvoří i všechny kolíky pro příbytek a všechny kolíky pro nádvoří dokola. 
#39:1 Z látky fialově purpurové, nachové a karmínové zhotovili tkaná roucha pro přisluhování ve svatyni; zhotovili též svatá roucha pro Árona, jak Hospodin Mojžíšovi přikázal.
#39:2 Zhotovil nárameník ze zlata, z látky fialově purpurové, nachové a karmínové a z jemně tkaného plátna.
#39:3 Vytepali zlaté plátky a z nich nastříhal nitky, aby je umně vetkali do látky fialově purpurové, nachové a karmínové a do jemného plátna.
#39:4 Vrchní části mu udělali spojené; na obou koncích byl spojen.
#39:5 Tkaný pás, jímž byl nárameník upevňován, byl zhotoven týmž způsobem ze zlata, z látky fialově purpurové, nachové a karmínové a z jemně tkaného plátna, jak Hospodin Mojžíšovi přikázal.
#39:6 Vsadili též kameny karneoly do zlatých obrouček a vyryli do nich jména synů Izraele, jako se ryjí pečeti.
#39:7 Kameny umístil na vrchních částech nárameníku, aby byly připomínkou na syny Izraele, jak Hospodin Mojžíšovi přikázal.
#39:8 Týmž způsobem jako nárameník zhotovil umně utkaný náprsník ze zlata, z látky fialově purpurové, nachové a karmínové a z jemně tkaného plátna.
#39:9 Byl čtvercový a dvojitý; náprsník udělali jednu píď dlouhý a jednu píď široký, dvojitý.
#39:10 Vysadili jej čtyřmi řadami kamenů: v první řadě rubín, topas a smaragd;
#39:11 v druhé řadě malachit, safír a jaspis;
#39:12 v třetí řadě opál, achát a ametyst;
#39:13 ve čtvrté řadě chrysolit, karneol a onyx. Všechny byly vsazeny do zlatých obrouček.
#39:14 Kameny byly označeny jmény synů Izraele; bylo jich dvanáct podle jejich jmen; na každém bylo vyryto jeho jméno jako na pečeti podle dvanácti kmenů.
#39:15 K náprsníku udělali řetízky z čistého zlata jako stočenou šňůru.
#39:16 Zhotovili též dvě zlaté obroučky a dva zlaté kroužky; oba kroužky pak připevnili k oběma horním okrajům náprsníku.
#39:17 Obě zlaté šňůry připevnili k oběma kroužkům na horních okrajích náprsníku.
#39:18 Oba druhé konce obou šňůr připevnili k oběma obroučkám a umístili je vpředu na vrchních částech nárameníku.
#39:19 Potom zhotovili jiné dva zlaté kroužky a připevnili je k oběma okrajům náprsníku, na lem jeho vnitřní strany při nárameníku.
#39:20 Zhotovili i další dva zlaté kroužky a připevnili je k oběma vrchním částem nárameníku dole na jeho přední straně těsně u švu, nad tkaným pásem nárameníku.
#39:21 Náprsník opatřený kroužky přivázali ke kroužkům nárameníku purpurově fialovou šňůrkou, aby byl nad tkaným pásem nárameníku; aby náprsník od nárameníku neodstával, jak Hospodin Mojžíšovi přikázal.
#39:22 K nárameníku zhotovil řízu, tkanou práci, celou z fialového purpuru.
#39:23 Uprostřed řízy byl otvor jako otvor v krunýři a otvor byl dokola olemován, aby se neroztrhl.
#39:24 K dolní obrubě řízy připevnili granátová jablka z látky fialově purpurové, nachové a karmínové a z jemně tkaného plátna.
#39:25 Zhotovili i zvonečky z čistého zlata a zvonečky upevnili mezi granátová jablka na dolní obrubu řízy dokola, mezi granátová jablka.
#39:26 Zvoneček a granátové jablko, zvoneček a granátové jablko byly střídavě dokola na obrubě řízy, když Áron konal službu, jak Hospodin Mojžíšovi přikázal.
#39:27 Zhotovili suknice z jemného plátna, tkanou práci, pro Árona a jeho syny;
#39:28 dále turban z jemného plátna a ozdobné mitry z jemného plátna i lněné spodky z jemně tkaného plátna;
#39:29 také pestře vyšitou šerpu z jemně tkaného plátna a z látky fialově purpurové, nachové a karmínové, jak Hospodin Mojžíšovi přikázal.
#39:30 Zhotovili květ z čistého zlata, svatou čelenku, a vepsali na něj nápis, jako se vyrývá pečeť: „Svatý Hospodinu.“
#39:31 Připevnili jej purpurově fialovou šňůrkou nahoře na turbanu, jak Hospodin Mojžíšovi přikázal.
#39:32 Tak byla všechna práce na příbytku stanu setkávání ukončena. Izraelci provedli všechno přesně tak, jak Hospodin Mojžíšovi přikázal.
#39:33 Přinesli k Mojžíšovi příbytek: stan s veškerým náčiním, sponami, deskami, svlaky, sloupy a patkami;
#39:34 přikrývku z beraních kůží, zbarvených načerveno, přikrývku z kůží tachaších a vnitřní oponu;
#39:35 schránu svědectví s tyčemi a příkrovem;
#39:36 stůl s veškerým náčiním a předkladné chleby;
#39:37 svícen z čistého zlata s kahánky připravenými k nasazení a s veškerým náčiním, i olej k svícení;
#39:38 zlatý oltář a olej k pomazání i kadidlo z vonných látek a závěs ke vchodu do stanu;
#39:39 bronzový oltář a jeho bronzový rošt, s tyčemi a s veškerým náčiním, nádrž s podstavcem;
#39:40 zástěny pro nádvoří, sloupy s patkami, závěs pro bránu do nádvoří s lany a kolíky i všecko náčiní pro službu v příbytku, pro stan setkávání;
#39:41 tkaná roucha pro přisluhování ve svatyni, svatá roucha pro kněze Árona a roucha jeho synům pro kněžskou službu.
#39:42 Všechnu práci vykonali Izraelci přesně tak, jak Hospodin Mojžíšovi přikázal.
#39:43 Mojžíš celé dílo prohlédl; ano, vykonali je přesně tak, jak Hospodin přikázal. A Mojžíš jim požehnal. 
#40:1 Hospodin promluvil k Mojžíšovi:
#40:2 „Prvního dne prvního měsíce postavíš příbytek stanu setkávání.
#40:3 Tam umístíš schránu svědectví a zastřeš schránu oponou.
#40:4 Přineseš také stůl a všechno na něm uspořádáš, přineseš i svícen a nasadíš na něj kahánky.
#40:5 Zlatý kadidlový oltář dáš před schránu svědectví a pověsíš závěs ke vchodu do příbytku.
#40:6 Oltář pro zápalnou oběť postavíš před vchod do příbytku stanu setkávání.
#40:7 Mezi stan setkávání a oltář umístíš nádrž a naleješ do ní vodu.
#40:8 Dokola postavíš nádvoří a do brány k nádvoří pověsíš závěs.
#40:9 Potom vezmeš olej pomazání a pomažeš příbytek a všechno, co je v něm, a posvětíš jej i s veškerým náčiním, a bude svatý.
#40:10 Pomažeš také oltář pro zápalnou oběť i s veškerým náčiním a posvětíš jej; a oltář bude velesvatý.
#40:11 Pomažeš také nádrž s podstavcem a posvětíš ji.
#40:12 Pak přivedeš Árona a jeho syny ke vchodu do stanu setkávání a omyješ je vodou.
#40:13 Nato oblékneš Áronovi svatá roucha, pomažeš ho a posvětíš a bude mi sloužit jako kněz.
#40:14 Přivedeš i jeho syny a oblékneš do suknic.
#40:15 Pomažeš je, jako jsi pomazal jejich otce, a budou mi sloužit jako kněží. Toto pomazání je uvede v trvalé kněžství po všechna pokolení.“
#40:16 Mojžíš učinil všechno přesně tak, jak mu Hospodin přikázal.
#40:17 Příbytek byl postaven první den prvního měsíce druhého roku.
#40:18 Mojžíš postavil příbytek: rozmístil jeho patky, zasadil desky, přiložil svlaky a postavil sloupy.
#40:19 Nad příbytkem rozprostřel stan a nahoru na stan položil přikrývku, jak Hospodin Mojžíšovi přikázal.
#40:20 Potom vzal svědectví a dal je do schrány; podél schrány zasunul tyče a nahoru na schránu položil příkrov.
#40:21 Schránu vnesl do příbytku, zavěsil vnitřní oponu a zastřel schránu svědectví, jak Hospodin Mojžíšovi přikázal.
#40:22 Stůl postavil do stanu setkávání ke straně příbytku na sever, vně před oponu.
#40:23 Uspořádal na něm chléb před Hospodinem, jak Hospodin Mojžíšovi přikázal.
#40:24 Svícen postavil do stanu setkávání naproti stolu ke straně příbytku na jih
#40:25 a nasadil kahánky před Hospodinem, jak Hospodin Mojžíšovi přikázal.
#40:26 Zlatý oltář umístil ve stanu setkávání před oponu.
#40:27 Pálil na něm kadidlo z vonných látek, jak Hospodin Mojžíšovi přikázal.
#40:28 Na vchod do příbytku pověsil závěs.
#40:29 Oltář pro zápalnou oběť postavil u vchodu do příbytku stanu setkávání. Na něm obětoval zápalnou a přídavnou oběť, jak Hospodin Mojžíšovi přikázal.
#40:30 Mezi stan setkávání a oltář umístil nádrž a nalil do ní vodu k omývání.
#40:31 Mojžíš, Áron a jeho synové si z ní omyli ruce a nohy.
#40:32 Když přistupovali ke stanu setkávání a když se přibližovali k oltáři, omývali se, jak Hospodin Mojžíšovi přikázal.
#40:33 Kolem příbytku a oltáře postavil nádvoří a do brány nádvoří pověsil závěs. Tak Mojžíš dokončil celé to dílo.
#40:34 Tu oblak zahalil stan setkávání a příbytek naplnila Hospodinova sláva.
#40:35 Mojžíš nemohl přistoupit ke stanu setkávání, neboť nad ním přebýval oblak a příbytek naplňovala Hospodinova sláva.
#40:36 Kdykoli se oblak z příbytku zvedl, vytáhli Izraelci ze všech svých stanovišť.
#40:37 Jestliže se oblak nezvedal, nevytáhli, dokud se nezvedl.
#40:38 Hospodinův oblak býval nad příbytkem ve dne a v noci v něm planul oheň před očima celého domu izraelského na všech jejich stanovištích.  

\book{Leviticus}{Lev}
#1:1 I zavolal Hospodin Mojžíše a promluvil k němu ze stanu setkávání:
#1:2 „Mluv k synům Izraele a řekni jim: Když někdo z vás přinese dar Hospodinu, přinesete svůj dar z dobytka, ze skotu nebo z bravu.
#1:3 Jestliže jeho darem bude zápalná oběť ze skotu, přivede samce bez vady. Přivede jej ke vchodu do stanu setkávání, aby došel zalíbení před Hospodinem.
#1:4 Vloží ruku na hlavu zápalné oběti; ta mu získá zalíbení a zprostí ho viny.
#1:5 Dobytče pak porazí před Hospodinem a Áronovci, kněží, přinesou krev v oběť; krví pokropí dokola oltář, který je u vchodu do stanu setkávání.
#1:6 Stáhne kůži ze zápalné oběti a rozseká oběť na díly.
#1:7 Synové kněze Árona vloží na oltář oheň a na oheň narovnají dříví.
#1:8 Díly, hlavu a tuk narovnají pak Áronovci, kněží, na dříví, které je na ohni na oltáři.
#1:9 Vnitřnosti však a hnáty omyje kněz vodou a všechno obrátí na oltáři v obětní dým. To bude zápalná oběť; jako oběť ohnivá bude libou vůní pro Hospodina.
#1:10 Jestliže jeho dar pro zápalnou oběť bude z bravu, z ovcí nebo z koz, přivede samce bez vady.
#1:11 Porazí ho před Hospodinem při severní straně oltáře a Áronovci, kněží, pokropí jeho krví oltář dokola.
#1:12 Rozseká oběť na díly i s hlavou a s tukem a kněz to narovná na dříví, které je na ohni na oltáři.
#1:13 Vnitřnosti však a hnáty omyje kněz vodou; všechno přinese v oběť a na oltáři obrátí v obětní dým. To bude zápalná oběť; jako oběť ohnivá bude libou vůní pro Hospodina.
#1:14 Jestliže jeho darem Hospodinu bude zápalná oběť z ptactva, přinese svůj dar z hrdliček nebo z holoubat.
#1:15 Kněz přinese ptáče k oltáři, nehtem mu natrhne hlavu a na oltáři obrátí v obětní dým; jeho krev nechá vykapat na stěnu oltáře.
#1:16 Odstraní vole s vývržkem a pohodí to k východní straně oltáře, kde je popel z tuku.
#1:17 Pak mu kněz natrhne křídla, ale neodtrhne je; a na oltáři je obrátí v obětní dým na ohni ze dříví. To bude zápalná oběť; jako oběť ohnivá bude libou vůní pro Hospodina. 
#2:1 Když někdo přinese Hospodinu darem přídavnou oběť, bude jeho darem bílá mouka. Poleje ji olejem, vloží na to kadidlo
#2:2 a donese ji Áronovcům, kněžím. Kněz z ní vezme plnou hrst bílé mouky s olejem i všechno kadidlo a jako připomínku ji na oltáři obrátí v obětní dým. Jako oběť ohnivá bude libou vůní pro Hospodina.
#2:3 Zbytek přídavné oběti připadne Áronovi a jeho synům jako velesvatý podíl z ohnivých obětí Hospodinových.
#2:4 Když přineseš darem přídavnou oběť pečenou v peci, budou to nekvašené bochánky z bílé mouky zadělané olejem a nekvašené oplatky pomazané olejem.
#2:5 Jestliže bude tvým darem přídavná oběť připravená na pánvi, bude z bílé mouky zadělané olejem, nekvašená.
#2:6 Rozdrobíš ji na sousta a poleješ ji olejem. To bude přídavná oběť.
#2:7 Jestliže tvým darem bude přídavná oběť připravená v kotlíku, ať je připravena z bílé mouky s olejem.
#2:8 Takto připravenou přídavnou oběť přineseš Hospodinu: předáš ji knězi a on ji donese k oltáři.
#2:9 Kněz oddělí hrst přídavné oběti a jako připomínku ji na oltáři obrátí v obětní dým. Jako oběť ohnivá bude libou vůní pro Hospodina.
#2:10 Zbytek přídavné oběti připadne Áronovi a jeho synům jako velesvatý podíl z ohnivých obětí Hospodinových.
#2:11 Žádná přídavná oběť, kterou přinesete Hospodinu, nebude připravována kvašením; žádný kvas a žádný med neobrátíte v obětní dým jako ohnivou oběť Hospodinu.
#2:12 Můžete je přinést Hospodinu jako dar z prvotin, ale nebudou na oltáři obětovány v libou vůni.
#2:13 Každý dar své přídavné oběti solí osolíš. Nenecháš svou přídavnou oběť bez soli smlouvy svého Boha. S každým svým darem přineseš sůl.
#2:14 Jestliže přineseš Hospodinu přídavnou oběť raných plodů, přineseš klasy pražené na ohni a drcené zrní z nového obilí jako přídavnou oběť svých raných plodů.
#2:15 Přidáš k ní olej a položíš na to kadidlo. To bude přídavná oběť.
#2:16 Kněz obrátí v obětní dým jako připomínku hrst rozdrceného zrní s olejem i všechno kadidlo. To bude ohnivá oběť Hospodinu. 
#3:1 Jestliže někdo připraví jako svůj dar hod oběti pokojné ze skotu, ať už býka nebo krávu, přivede před Hospodina zvíře bez vady.
#3:2 Vloží ruku na hlavu svého daru a porazí oběť před vchodem do stanu setkávání. Áronovci, kněží, pokropí krví oltář dokola.
#3:3 Z hodu oběti pokojné přinese Hospodinu jako ohnivou oběť tuk pokrývající vnitřnosti i všechen tuk, který je na vnitřnostech,
#3:4 dále obě ledviny i s tukem, který je na nich i na slabinách, a jaterní lalok; odejme jej nad ledvinami.
#3:5 Áronovci to jako zápalnou oběť obrátí na oltáři v obětní dým na ohni ze dříví. Jako oběť ohnivá bude libou vůní pro Hospodina.
#3:6 Jestliže jeho dar Hospodinu k hodu oběti pokojné bude z bravu, přivede samce nebo samici bez vady.
#3:7 Přivede-li jako svůj dar jehně, přivede je před Hospodina,
#3:8 vloží ruku na hlavu svého daru a porazí je před stanem setkávání. Áronovci pokropí jeho krví oltář dokola.
#3:9 Z hodu oběti pokojné přinese Hospodinu jako ohnivou oběť jeho tuk, celý tučný ocas, který odejme těsně u kostrče, a tuk pokrývající vnitřnosti, totiž všechen tuk, který je na vnitřnostech,
#3:10 dále obě ledviny i s tukem, který je na nich i na slabinách, a jaterní lalok; odejme jej nad ledvinami.
#3:11 Kněz to na oltáři obrátí v obětní dým. To bude pokrm ohnivé oběti Hospodinu.
#3:12 Jestliže bude jeho darem koza, přivede ji před Hospodina,
#3:13 vloží ruku na její hlavu a porazí ji před stanem setkávání. Áronovci pokropí její krví oltář dokola.
#3:14 Přinese z ní Hospodinu svůj dar jako ohnivou oběť: tuk pokrývající vnitřnosti i všechen tuk, který je na vnitřnostech,
#3:15 dále obě ledviny s tukem, který je na nich i na slabinách, a jaterní lalok; odejme jej nad ledvinami.
#3:16 Kněz to na oltáři obrátí v obětní dým. To bude pokrm ohnivé oběti, libá vůně. Všechen tuk patří Hospodinu.
#3:17 To bude provždy platné nařízení pro všechna vaše pokolení ve všech vašich sídlištích: Nebudete jíst žádný tuk ani žádnou krev.“ 
#4:1 Hospodin promluvil k Mojžíšovi:
#4:2 „Mluv k Izraelcům: Když se někdo neúmyslně prohřeší proti kterémukoli příkazu Hospodinovu něčím, co se dělat nesmí, a dopustí se něčeho proti některému z nich, pak platí:
#4:3 Jestliže se prohřeší pomazaný kněz a uvalí tím vinu na lid, ať přivede Hospodinu za svůj hřích, jehož se dopustil, mladého býčka bez vady k oběti za hřích.
#4:4 Dovede býčka ke vchodu do stanu setkávání před Hospodina, vloží ruku na hlavu býčka a porazí jej před Hospodinem.
#4:5 Pak vezme pomazaný kněz trochu krve z býčka a přinese ji ke stanu setkávání.
#4:6 Kněz namočí prst v krvi a sedmkrát z ní stříkne před Hospodinem na oponu svatyně.
#4:7 Trochou krve potře kněz též rohy oltáře pro pálení kadidla z vonných látek před Hospodinem, oltáře, který je ve stanu setkávání; všechnu ostatní krev toho býčka pak vyleje ke spodku oltáře pro zápalné oběti, jenž je u vchodu do stanu setkávání.
#4:8 Všechen tuk z býčka obětovaného za hřích z něho odstraní, tuk pokrývající vnitřnosti i všechen tuk, který je na vnitřnostech,
#4:9 dále obě ledviny s tukem, který je na nich i na slabinách, a jaterní lalok; odejme jej nad ledvinami,
#4:10 jak to bývá odstraňováno z býka pro hod oběti pokojné. Kněz to na oltáři pro zápalné oběti obrátí v obětní dým.
#4:11 Kůži z býčka a všechno maso z něho včetně hlavy a hnátů i vnitřnosti a výměty,
#4:12 celý zbytek býčka vynese ven za tábor na čisté místo, kam se sype popel z tuku, a na dříví jej spálí ohněm; kam se sype popel z tuku, tam bude spálen.
#4:13 Jestliže se celá izraelská pospolitost neúmyslně něčeho dopustí a zůstane shromáždění skryto, že se dopustili proti kterémukoli příkazu Hospodinovu něčeho, co se dělat nesmí, přece se provinili.
#4:14 Když vyjde najevo hřích, kterého se dopustili proti příkazu, přivede shromáždění mladého býčka k oběti za hřích a dovedou jej před stan setkávání.
#4:15 Starší pospolitosti vloží před Hospodinem ruce na hlavu býčka a porazí jej před Hospodinem.
#4:16 Pak přinese pomazaný kněz trochu krve z býčka ke stanu setkávání.
#4:17 Kněz namočí v krvi prst a sedmkrát z ní stříkne před Hospodinem na oponu.
#4:18 Trochou krve potře též rohy oltáře, který je před Hospodinem ve stanu setkávání; všechnu ostatní krev pak vyleje ke spodku oltáře pro zápalné oběti, jenž je u vchodu do stanu setkávání.
#4:19 Všechen tuk z něho odstraní a na oltáři jej obrátí v obětní dým.
#4:20 Naloží s býčkem jako s býčkem obětovaným za hřích, stejně s ním naloží. Kněz za ně vykoná smírčí obřady, a bude jim odpuštěno.
#4:21 Býčka vynese ven za tábor a spálí jej stejně, jako spálil předešlého býčka. To bude oběť za hřích shromáždění.
#4:22 Prohřeší-li se předák a dopustí se neúmyslně proti kterémukoli příkazu Hospodina, svého Boha, něčeho, co se dělat nesmí, provinil se.
#4:23 Je-li mu oznámeno, že se dopustil hříchu, přivede jako dar kozla, samce bez vady.
#4:24 Vloží ruku na hlavu kozla a porazí ho na místě, kde se poráží dobytek pro zápalnou oběť před Hospodinem. To bude oběť za hřích.
#4:25 Pak vezme kněz trochu krve z oběti za hřích na prst a potře rohy oltáře pro zápalné oběti; ostatní krev vyleje ke spodku oltáře pro zápalné oběti.
#4:26 Všechen tuk na oltáři obrátí v obětní dým jako tuk z hodu oběti pokojné. Pro jeho hřích vykoná za něho kněz smírčí obřady, a bude mu odpuštěno.
#4:27 Jestliže se neúmyslně prohřeší někdo z lidu země a dopustí se proti některému příkazu Hospodinovu něčeho, co se dělat nesmí, provinil se.
#4:28 Je-li mu oznámeno, že se dopustil hříchu, přivede jako svůj dar kozu, samici bez vady, za hřích, kterého se dopustil.
#4:29 Vloží ruku na hlavu zvířete obětovaného za hřích a porazí oběť za hřích na místě pro zápalné oběti.
#4:30 Pak vezme kněz trochu krve na prst a potře rohy oltáře pro zápalné oběti; všechnu ostatní krev vyleje ke spodku oltáře.
#4:31 Všechen tuk odejme, jako bývá odňat tuk z hodu oběti pokojné, a kněz jej na oltáři obrátí v obětní dým, v libou vůni pro Hospodina. Kněz za něho vykoná smírčí obřady, a bude mu odpuštěno.
#4:32 Jestliže přivede jako dar k oběti za hřích jehně, přivede samičku bez vady.
#4:33 Vloží ruku na hlavu zvířete obětovaného za hřích a porazí je v oběť za hřích na místě, kde se poráží dobytek pro zápalné oběti.
#4:34 Pak vezme kněz trochu krve z oběti za hřích na prst a potře rohy oltáře pro zápalné oběti; všechnu ostatní krev vyleje ke spodku oltáře.
#4:35 Všechen tuk odejme, jako bývá odňat tuk z beránka pro hod oběti pokojné, a kněz jej na oltáři obrátí v obětní dým jako ohnivou oběť pro Hospodina. Pro hřích, jehož se dopustil, vykoná za něho kněz smírčí obřady, a bude mu odpuštěno. 
#5:1 Prohřeší-li se někdo tím, že slyšel vyslovit kletbu a byl toho svědkem, ať už to viděl nebo se o tom dozvěděl, jestliže to neoznámí, ponese svou vinu.
#5:2 Anebo když se někdo dotkne čehokoli nečistého, buď zdechliny nečistého divokého zvířete nebo zdechliny nečistého domácího zvířete nebo zdechliny nečisté drobné havěti, i když mu to nebylo známo, je nečistý a provinil se.
#5:3 Anebo když se někdo dotkne lidské nečistoty, jakékoli nečistoty, jíž se může znečistit, i když mu to nebylo známo, ale potom se to dozví, provinil se.
#5:4 Anebo když někdo nerozvážně pronese přísahu, že udělá něco zlého nebo dobrého, vším, k čemu se člověk přísahou nerozvážně zavázal, i když mu to nebylo známo, ale potom se to dozví, každou jednotlivostí se provinil.
#5:5 Když se tedy čímkoli z toho provinil, ať vyzná, čím se prohřešil,
#5:6 a přivede Hospodinu jako pokutu za prohřešek, jehož se dopustil, samici z bravu, ovci nebo kozu, v oběť za hřích. Pro jeho hřích vykoná za něho kněz smírčí obřady.
#5:7 Jestliže není v jeho možnostech obětovat beránka nebo kůzle, přinese Hospodinu jako pokutu za svůj prohřešek dvě hrdličky nebo dvě holoubata, jedno k oběti za hřích, druhé k oběti zápalné.
#5:8 Donese je knězi. Ten bude nejprve obětovat to, které je určeno k oběti za hřích. Zespodu mu nehtem natrhne hlavu, ale neodtrhne ji.
#5:9 Trochu krve oběti za hřích stříkne na stěnu oltáře a zbytek krve nechá vykapat ke spodku oltáře. To bude oběť za hřích.
#5:10 To druhé připraví podle řádu zápalné oběti. Pro hřích, jehož se dopustil, vykoná za něho kněz smírčí obřady, a bude mu odpuštěno.
#5:11 Jestliže není s to dát ani dvě hrdličky nebo dvě holoubata, přinese ten, kdo zhřešil, jako dar desetinu éfy bílé mouky v oběť za hřích; nepoleje ji olejem a nedá na ni kadidlo, protože to je oběť za hřích.
#5:12 Donese ji knězi a kněz z ní vezme plnou hrst na připomínku a na oltáři ji obrátí v obětní dým jako ohnivé oběti Hospodinovy. To bude oběť za hřích.
#5:13 Pro hřích, jehož se čímkoli z toho dopustil, vykoná za něho kněz smírčí obřady, a bude mu odpuštěno; zbytek oběti bude patřit knězi jako při oběti přídavné.“
#5:14 Hospodin promluvil k Mojžíšovi:
#5:15 „Jestliže se někdo těžce zpronevěří tím, že se neúmyslně prohřeší proti svatým věcem Hospodinovým, jako pokutu přivede Hospodinu z bravu berana bez vady, jehož cenu stanovíš v šekelech stříbra podle váhy určené svatyní, jako oběť za vinu.
#5:16 Nahradí také, čím se proti svatým věcem prohřešil, a přidá nad to pětinu. Dá to knězi a kněz za něho obětí berana za vinu vykoná smírčí obřady, a bude mu odpuštěno.
#5:17 Jestliže se někdo prohřeší a dopustí se něčeho proti kterémukoli Hospodinovu příkazu, co se dělat nesmí, aniž to věděl, provinil se a ponese následky své nepravosti.
#5:18 Přivede ke knězi z bravu berana bez vady, jehož cenu určíš, jako oběť za vinu. Za neúmyslný přestupek, jehož se dopustil, aniž věděl, vykoná za něho kněz smírčí obřady, a bude mu odpuštěno.
#5:19 To bude oběť za vinu. Jistěže se provinil proti Hospodinu.“
#5:20 Hospodin dále mluvil k Mojžíšovi:
#5:21 „Prohřeší-li se někdo a těžce se zpronevěří Hospodinu tím, že zapře svému bližnímu věc, kterou měl v úschově, ať svěřenou či zabavenou, nebo svého bližního vydírá
#5:22 nebo najde ztracenou věc a křivopřísežně to zapře, ať se člověk dopustí kterékoli z těch věcí, zhřeší.
#5:23 Jestliže se tedy prohřešil a provinil, navrátí, co násilně zabavil nebo co uchvátil vydíráním nebo co mu bylo svěřeno do úschovy, anebo ztracenou věc, kterou našel,
#5:24 anebo cokoli, o čem křivě přísahal. Plně to nahradí a nad to přidá pětinu. Dá to tomu, čí to je, v den své oběti za vinu.
#5:25 Jako pokutu přivede Hospodinu z bravu berana bez vady, jehož cenu určíš; přivede jej ke knězi jako oběť za vinu.
#5:26 Kněz za něho vykoná před Hospodinem smírčí obřady, a bude mu odpuštěno všechno, čeho se dopustil a čím se provinil.“ 
#6:1 Hospodin promluvil k Mojžíšovi:
#6:2 „Přikaž Áronovi a jeho synům: Toto je řád zápalné oběti: Zápalná oběť bude na ohništi na oltáři po celou noc až do rána a oheň na oltáři bude stále udržován.
#6:3 Kněz si obleče lněný šat a na tělo si obleče lněné spodky. Vybere popel z tuku zápalné oběti strávené ohněm na oltáři a vysype jej vedle oltáře.
#6:4 Pak si svleče roucho, obleče si jiné a vynese popel z tuku ven za tábor na čisté místo.
#6:5 Oheň na oltáři bude stále udržován, nesmí vyhasnout. Kněz jím bude každé ráno zapalovat dříví, narovná na něm zápalnou oběť a obrátí na něm tuk z pokojných obětí v obětní dým.
#6:6 Oheň na oltáři bude stále udržován, nesmí vyhasnout.
#6:7 Toto je řád přídavné oběti: Áronovci ji přinesou před Hospodina k přední straně oltáře.
#6:8 Kněz z té přídavné oběti vyzdvihne v hrsti část bílé mouky s olejem a všechno kadidlo, které je na přídavné oběti, a na oltáři to obrátí v obětní dým; bude to libá vůně na připomínku pro Hospodina.
#6:9 Co z přídavné oběti zbude, snědí Áron a jeho synové. Bude se to jíst nekvašené na svatém místě; na nádvoří stanu setkávání to budou jíst.
#6:10 Nebude se péci nic kvašeného. Dal jsem jim to jako jejich podíl ze svých ohnivých obětí. Bude to velesvaté jako oběť za hřích a za vinu.
#6:11 Smí to jíst každý mužský příslušník mezi Áronovci. To bude pro všechna vaše pokolení provždy platné nařízení o ohnivých obětech Hospodinových. Cokoli se jich dotkne, bude svaté.“
#6:12 Hospodin dále mluvil k Mojžíšovi:
#6:13 „Toto je dar Árona a jeho synů, který v den jeho pomazání přinesou Hospodinu: desetinu éfy bílé mouky jako pravidelnou přídavnou oběť, polovinu ráno a polovinu večer.
#6:14 Přineseš ji připravenou s olejem na pánvi, smaženou. Přídavnou oběť přineseš rozdrobenou na sousta. Bude to libá vůně pro Hospodina.
#6:15 Připraví ji pomazaný kněz z Áronových synů, který nastoupí na jeho místo. To bude provždy platné Hospodinovo nařízení. Jako celopal bude obrácena v obětní dým.
#6:16 Každá přídavná oběť kněžská bude celopal. Nesmí se jíst.“
#6:17 Hospodin dále mluvil k Mojžíšovi:
#6:18 „Mluv k Áronovi a jeho synům: Toto je řád oběti za hřích: Na místě, na kterém se poráží dobytče pro zápalnou oběť, bude se porážet před Hospodinem i pro oběť za hřích. Je to velesvaté.
#6:19 Kněz, který ji obětuje za hřích, ji smí jíst. Bude se jíst na svatém místě, na nádvoří stanu setkávání.
#6:20 Cokoli se dotkne jejího masa, bude svaté. Stříkne-li trochu krve z ní na roucho, vypereš postříknuté místo na svatém místě.
#6:21 Hliněné nádobí, v němž se bude vařit, bude rozbito. Bude-li se vařit v měděné nádobě, vydrhne se a opláchne vodou.
#6:22 Každý mužský příslušník kněžských rodin ji smí jíst. Je to velesvaté.
#6:23 Ale žádná oběť za hřích, z jejíž krve se část vnáší do stanu setkávání k vykonání smírčích obřadů ve svatyni, nebude se jíst; bude spálena ohněm. 
#7:1 Toto je řád oběti za vinu; je to velesvaté.
#7:2 Na místě, na němž se poráží dobytče pro zápalnou oběť, bude se porážet i pro oběť za vinu. Jeho krví pokropí kněz oltář dokola.
#7:3 Pak přinese všechen tuk z něho, tučný ocas a tuk pokrývající vnitřnosti,
#7:4 dále obě ledviny s tukem, který je na nich i na slabinách, a jaterní lalok; odejme jej nad ledvinami.
#7:5 Kněz to na oltáři obrátí v obětní dým jako ohnivou oběť Hospodinu. To bude oběť za vinu.
#7:6 Každý mužský příslušník kněžských rodin ji smí jíst. Bude se jíst na svatém místě. Je to velesvaté.
#7:7 Oběť za vinu má stejný řád jako oběť za hřích; patří knězi, který jí vykonává smírčí obřady.
#7:8 Knězi, který přináší něčí zápalnou oběť, patří kůže z oběti, kterou přináší; je jeho.
#7:9 Rovněž každá přídavná oběť, která je pečena v peci, i každá, která je připravena v kotlíku nebo na pánvi, patří knězi, který ji přináší; je jeho.
#7:10 Každá přídavná oběť, zadělaná olejem i suchá, patří všem Áronovcům, jednomu jako druhému.
#7:11 Toto je řád hodu oběti pokojné, která se přináší Hospodinu:
#7:12 Jestliže ji někdo přinese jako projev vděčnosti, ať přinese k děkovné oběti nekvašené bochánky zadělané olejem a nekvašené oplatky pomazané olejem. Bochánky budou z bílé zasmažené mouky, zadělané olejem.
#7:13 Vedle bochánků přinese kvašený chléb jako svůj dar ke svému hodu díků oběti pokojné.
#7:14 Z každého daru přinese po jednom kusu jako oběť pozdvihování pro Hospodina. Patří knězi, který kropí krví z pokojné oběti; je jeho.
#7:15 Maso jeho hodu díků oběti pokojné se musí jíst v den, kdy bylo přineseno; nezůstane z něho nic do rána.
#7:16 Jestliže se dar obětního hodu týká slibu nebo dobrovolného závazku, bude se jíst v den, kdy byl obětní hod přinesen, a příštího dne se smí jíst zbytek.
#7:17 Zbytek masa z obětního hodu ať je však třetího dne spálen ohněm.
#7:18 Jestliže by se přesto jedlo maso z hodu oběti pokojné třetího dne, nenalezne Hospodin zalíbení v tom, kdo jej přinesl; dar mu nebude připočten k dobru. Je závadný a každý, kdo by z něho jedl, ponese následky své nepravosti.
#7:19 Ani maso, které přišlo do styku s něčím nečistým, nesmí se jíst; bude spáleno ohněm. Jiné maso může jíst každý, kdo je čistý.
#7:20 Bude-li však někdo jíst maso Hospodinova hodu oběti pokojné a přitom bude nečistý, bude vyobcován ze svého lidu.
#7:21 Když by se někdo dotkl čehokoli nečistého, lidské nečistoty nebo nečistého zvířete nebo čehokoli nečistého, hodného opovržení, a jedl by z masa Hospodinova hodu oběti pokojné, bude vyobcován ze svého lidu.“
#7:22 Hospodin promluvil k Mojžíšovi:
#7:23 „Mluv k Izraelcům: Nebudete jíst žádný tuk z býka ani z ovce a kozy.
#7:24 Tuku zdechlého nebo rozsápaného zvířete se může užívat k různému účelu, ale jíst jej rozhodně nesmíte.
#7:25 Každý, kdo by jedl tuk z dobytčete, z něhož přinesl ohnivou oběť Hospodinu, bude vyobcován ze svého lidu.
#7:26 Také nebudete jíst žádnou krev, ať se usadíte kdekoli, ani krev ptactva ani dobytčat.
#7:27 Kdokoli by jedl jakoukoli krev, bude vyobcován ze svého lidu.“
#7:28 Hospodin dále mluvil k Mojžíšovi:
#7:29 „Mluv k Izraelcům: Kdo připravuje svůj hod oběti pokojné Hospodinu, přinese Hospodinu dar ze svého hodu oběti pokojné.
#7:30 Vlastní rukou přinese ohnivé oběti Hospodinu: přinese tuk, který je na hrudí, i hrudí, aby je podáváním nabídl Hospodinu jako oběť podávání.
#7:31 Kněz obrátí tuk na oltáři v obětní dým. Hrudí pak připadne Áronovi a jeho synům.
#7:32 Pravou kýtu dáte knězi jako oběť pozdvihování ze svých hodů oběti pokojné.
#7:33 Tomu z Áronovců, kdo přinese z pokojných obětí krev a tuk, bude pravá kýta patřit jako podíl.
#7:34 Hrudí z oběti podávání a kýtu z oběti pozdvihování jsem totiž Izraelcům odebral z jejich hodů oběti pokojné a dávám je knězi Áronovi a jeho synům. Budou je dostávat od Izraelců podle provždy platného nařízení.
#7:35 To je příděl Áronův i příděl jeho synů z ohnivých obětí Hospodinových; v den, kdy jim dovolil přistoupit, aby sloužili Hospodinu jako kněží,
#7:36 v den, kdy je pomazáním oddělil od Izraelců, přikázal Hospodin, aby jim byl dáván podle nařízení provždy platného pro všechna jejich pokolení.“
#7:37 To je řád pro zápalnou oběť, pro přídavnou oběť, pro oběť za hřích a oběť za vinu, i pro oběť při vysvěcování kněží a pro hod oběti pokojné,
#7:38 jak to přikázal Hospodin Mojžíšovi na hoře Sínaji v den, kdy přikázal synům Izraele, aby přinášeli své dary Hospodinu na Sínajské poušti. 
#8:1 Hospodin promluvil k Mojžíšovi:
#8:2 „Vezmi Árona i s jeho syny, též roucha a olej pomazání i býčka k oběti za hřích, dva berany a košík nekvašených chlebů,
#8:3 a shromáždi celou pospolitost ke vchodu do stanu setkávání.“
#8:4 Mojžíš učinil, jak mu Hospodin přikázal. Pospolitost se shromáždila ke vchodu do stanu setkávání
#8:5 a Mojžíš pospolitosti řekl: „Toto nám přikázal Hospodin udělat.“
#8:6 Pak vyzval Mojžíš Árona a jeho syny, aby přistoupili, a omyl je vodou.
#8:7 Přehodil mu suknici a přepásal ho šerpou, oblékl mu řízu, vložil na něj nárameník a přepásal ho tkaným pásem nárameníku; opásal ho jím.
#8:8 Navlékl mu také náprsník a do náprsníku dal posvátné losy urím a tumím.
#8:9 Na hlavu mu nasadil turban a dopředu na turban připevnil zlatý květ, svatou čelenku, jak přikázal Mojžíšovi Hospodin.
#8:10 Potom vzal Mojžíš olej pomazání, pomazal příbytek s veškerým příslušenstvím a posvětil je.
#8:11 Sedmkrát jím stříkl na oltář a pomazal oltář i všechno jeho náčiní i nádrž s podstavcem, aby je posvětil.
#8:12 Nalil též trochu oleje pomazání na Áronovu hlavu a pomazal ho, aby ho posvětil.
#8:13 Pak vyzval Mojžíš Áronovy syny, aby přistoupili, oblékl jim suknice, přepásal je šerpou a vstavil jim na hlavu mitry, jak přikázal Mojžíšovi Hospodin.
#8:14 Potom dal přivést býčka k oběti za hřích a Áron i jeho synové vložili ruce na hlavu býčka určeného k oběti za hřích.
#8:15 Mojžíš ho porazil, vzal krev, prstem potřel rohy oltáře dokola a očistil oltář od hříchu. Krev vylil ke spodku oltáře a posvětil jej k vykonávání smírčích obřadů.
#8:16 Vzal také všechen tuk, který je na vnitřnostech, jaterní lalok i obě ledviny s jejich tukem a na oltáři to obrátil v obětní dým.
#8:17 Ale býčka, jeho kůži, maso a výměty spálil na ohni mimo tábor, jak přikázal Mojžíšovi Hospodin.
#8:18 Potom dal přivést berana k zápalné oběti a Áron i jeho synové vložili ruce na hlavu toho berana.
#8:19 Mojžíš ho porazil a krví pokropil oltář dokola.
#8:20 Berana rozsekal na díly a obrátil hlavu, díly a tuk v obětní dým.
#8:21 Vnitřnosti a hnáty omyl vodou a pak celého berana obrátil na oltáři v obětní dým. To je zápalná oběť v libou vůni, ohnivá oběť pro Hospodina, jak přikázal Mojžíšovi Hospodin.
#8:22 Pak dal přivést druhého berana, berana vysvěcení, a Áron se svými syny vložili ruce na hlavu toho berana.
#8:23 Mojžíš ho porazil, vzal trochu jeho krve a potřel Áronovi lalůček pravého ucha, palec na pravé ruce a palec u pravé nohy.
#8:24 Pak vyzval Áronovy syny, aby přistoupili, a krví jim potřel lalůček pravého ucha, palec na pravé ruce a palec u pravé nohy. Potom pokropil krví oltář dokola.
#8:25 Vzal také tuk, tučný ocas i všechen tuk, který je na vnitřnostech, jaterní lalok i obě ledviny s tukem a pravou kýtu;
#8:26 a z košíku s nekvašenými chleby, který stojí před Hospodinem, vzal jeden nekvašený bochánek, jeden bochánek chleba zadělaného olejem a jeden oplatek a položil to na kusy tuku a na pravou kýtu.
#8:27 To vše dal do rukou Áronovi i jeho synům a podáváním to nabídl jako oběť podávání Hospodinu.
#8:28 Pak to Mojžíš vzal z jejich rukou a na oltáři obrátil v obětní dým nad zápalnou obětí. To je oběť při uvádění kněží v úřad, libá vůně, ohnivá oběť pro Hospodina.
#8:29 Mojžíš vzal i hrudí a podáváním je nabídl jako oběť podávání Hospodinu; to je Mojžíšův podíl z berana vysvěcení, jak přikázal Mojžíšovi Hospodin.
#8:30 Potom vzal Mojžíš trochu oleje pomazání a krve z oltáře a stříkl na Árona a na jeho roucha a s ním i na jeho syny a na jejich roucha. Tak posvětil Árona a jeho roucha a s ním i jeho syny a jejich roucha.
#8:31 I řekl Mojžíš Áronovi a jeho synům: „Vařte maso u vchodu do stanu setkávání a tam je jezte s chlebem z košíku pro oběť vysvěcení, jak jsem přikázal: Áron i jeho synové je budou jíst.
#8:32 Co zbude z masa a chleba, spálíte ohněm.
#8:33 Po sedm dní nebudete odcházet od vchodu do stanu setkávání až do dne, kdy skončí vaše uvádění v úřad; po sedm dní budete totiž uváděni v úřad.
#8:34 Hospodin přikázal, abyste byli i nadále stejným způsobem jako dnes zprošťováni vin.
#8:35 Zůstanete u vchodu do stanu setkávání dnem i nocí po sedm dní a budete držet stráž Hospodinovu, abyste nezemřeli. Tak mi bylo přikázáno.“
#8:36 I vykonal Áron a jeho synové všechno, co Hospodin skrze Mojžíše přikázal. 
#9:1 Osmého dne zavolal Mojžíš Árona, jeho syny a izraelské starší.
#9:2 Áronovi nařídil: „Vezmi si mladého býčka k oběti za hřích a berana k zápalné oběti, oba bez vady, a přiveď je před Hospodina.
#9:3 Pak promluvíš k Izraelcům: Vezměte kozla k oběti za hřích a býčka a beránka, oba roční, bez vady, k oběti zápalné,
#9:4 též býka a berana k oběti pokojné, aby byli obětováni před Hospodinem s obětí přídavnou zadělanou olejem, neboť se vám dnes ukáže Hospodin.“
#9:5 Vzali tedy, co přikázal Mojžíš, před stan setkávání a celá pospolitost přistoupila; zůstali stát před Hospodinem.
#9:6 Tu řekl Mojžíš: „Vykonejte, co přikázal Hospodin, a ukáže se vám Hospodinova sláva.“
#9:7 Áronovi pak Mojžíš řekl: „Přistup k oltáři, obětuj svou oběť za hřích a svou oběť zápalnou a vykonej za sebe i za lid smírčí obřady. Obětuj také dar lidu a vykonej za něj smírčí obřady, jak přikázal Hospodin.“
#9:8 Áron tedy přistoupil k oltáři. Porazil svého býčka v oběť za hřích
#9:9 a Áronovi synové mu přinesli krev. Omočil v krvi prst a potřel rohy oltáře, krev pak vylil ke spodku oltáře.
#9:10 Tuk, ledviny a jaterní lalok z oběti za hřích obrátil na oltáři v obětní dým, jak přikázal Hospodin Mojžíšovi.
#9:11 Maso a kůži spálil ohněm venku za táborem.
#9:12 Potom porazil dobytče určené k zápalné oběti. Áronovi synové mu podali krev a on jí pokropil oltář dokola.
#9:13 Podali mu zápalnou oběť rozsekanou na díly a hlavu a on to na oltáři obrátil v obětní dým.
#9:14 Vnitřnosti však a hnáty omyli a Áron je na oltáři obrátil v obětní dým nad zápalnou obětí.
#9:15 Pak dal přinést dar lidu; vzal kozla lidu k oběti za hřích, porazili ho a obětovali za hřích jako předešlou oběť.
#9:16 Dal přinést i oběť zápalnou a obětoval ji podle řádu.
#9:17 Dal přinést také oběť přídavnou, vzal z ní plnou hrst a na oltáři ji obrátil v obětní dým, kromě jitřní zápalné oběti.
#9:18 Potom porazil býka a berana jako hod oběti pokojné za lid; Áronovi synové mu podali krev a on jí pokropil oltář dokola.
#9:19 Podali mu i tuk z býka a z berana, tučný ocas, bránici, ledviny a jaterní lalok.
#9:20 Položili tuk na hrudí a on jej na oltáři obrátil v obětní dým.
#9:21 Hrudí a pravou kýtu nabídl Áron podáváním Hospodinu jako oběť podávání, jak přikázal Mojžíš.
#9:22 Potom pozvedl Áron ruce k lidu a dal jim požehnání. A sestoupil z místa, kde obětoval oběť za hřích a oběť zápalnou i pokojnou.
#9:23 Mojžíš s Áronem nato vešli do stanu setkávání. Když vyšli, dali lidu požehnání. Vtom se ukázala všemu lidu Hospodinova sláva.
#9:24 Od Hospodina vyšel oheň a pozřel na oltáři zápalnou oběť i obětovaný tuk. Všechen lid to spatřil, zajásal a padli na tvář. 
#10:1 Áronovi synové Nádab a Abíhú vzali každý svou kadidelnici, dali do ní oheň a na něj položili kadidlo. Přinesli před Hospodina cizí oheň, jaký jim nepřikázal.
#10:2 I vyšel oheň od Hospodina a pozřel je, takže zemřeli před Hospodinem.
#10:3 Mojžíš řekl Áronovi: „Toto mluvil Hospodin: Na těch, kteří jsou mi blízko, ukážu svou svatost, před veškerým lidem osvědčím svou slávu.“ Áron mlčel.
#10:4 Mojžíš tedy zavolal Míšaela a Elsáfana, syny Áronova strýce Uzíela, a poručil jim: „Přistupte a vyneste své bratry pryč od svatyně ven za tábor!“
#10:5 Přistoupili tedy a vynesli je v jejich suknicích ven za tábor podle slova Mojžíšova.
#10:6 Mojžíš pak řekl Áronovi a jeho synům Eleazarovi a Ítamarovi: „Nebudete mít své hlavy kvůli nim neupravené a neroztrhnete svá roucha, abyste nezemřeli. Na celou pospolitost by dolehl hněv. Vaši bratří, celý dům izraelský, ať pláčou nad požárem, který zanítil Hospodin,
#10:7 ale vy nebudete odcházet od vchodu do stanu setkávání, abyste nezemřeli, neboť na vás je olej Hospodinova pomazání.“ Udělali tedy podle Mojžíšova slova.
#10:8 Hospodin promluvil k Áronovi:
#10:9 „Ty ani tvoji synové s tebou nesmíte pít víno nebo opojný nápoj, když budete vcházet do stanu setkávání, abyste nezemřeli. To je provždy platné nařízení pro všechna vaše pokolení.
#10:10 Musíte rozlišovat svaté od nesvatého a nečisté od čistého
#10:11 a učit Izraelce všem nařízením, která vám Hospodin uložil skrze Mojžíše.“
#10:12 Mojžíš dále mluvil k Áronovi a k jeho synům, kteří zůstali naživu, Eleazarovi a Ítamarovi: „Vezměte přídavnou oběť, co zbude z ohnivých obětí Hospodinových, a jezte to nekvašené u oltáře; je to velesvaté.
#10:13 Budete to jíst na svatém místě, neboť to je pravoplatný podíl tvůj a tvých synů z ohnivých obětí Hospodinových. Tak mi bylo přikázáno.
#10:14 Hrudí z oběti podávání a kýtu z oběti pozdvihování budete jíst na čistém místě, ty i tvoji synové a tvoje dcery s tebou. Jsou ti dány jako pravoplatný podíl tvůj i tvých synů z hodů oběti pokojné Izraelců.
#10:15 Kýtu z oběti pozdvihování a hrudí z oběti podávání ať přinášejí při ohnivých obětech tuku, aby to podáváním nabídli Hospodinu jako oběť podávání. To bude provždy pravoplatný podíl tvůj a tvých synů s tebou, jak přikázal Hospodin.“
#10:16 Mojžíš pak začal pátrat po kozlu k oběti za hřích a zjistil, že byl spálen. Rozlítil se na Áronovy syny, Eleazara a Ítamara, kteří zůstali naživu:
#10:17 „Proč jste nejedli oběť za hřích na svatém místě? Vždyť je to velesvaté! Vám ji dal, abyste nesli nepravost pospolitosti a konali za ni před Hospodinem smírčí obřady.
#10:18 Tady však nebyla ani její krev vnesena dovnitř svatyně. Měli jste jíst oběť ve svatyni, jak jsem přikázal.“
#10:19 Áron Mojžíšovi odpověděl: „Hle, oni dnes přinesli před Hospodina svou oběť za hřích a svou oběť zápalnou. Mne přece potkaly takové zlé věci. Kdybych dnes jedl oběť za hřích, líbilo by se to Hospodinu?“
#10:20 Když to Mojžíš vyslechl, zalíbilo se mu to. 
#11:1 Hospodin promluvil k Mojžíšovi a Áronovi a řekl jim:
#11:2 „Mluvte k Izraelcům: Ze všech zvířat na zemi smíte jíst tyto živočichy:
#11:3 Všechno, co má rozdělená kopyta tak, že jsou kopyta rozpolcená úplně, přežvýkavce mezi zvířaty, ty smíte jíst.
#11:4 Ale z přežvýkavců a z těch, kdo mají rozdělená kopyta, nesmíte jíst velblouda, který sice přežvykuje, ale nemá rozdělená kopyta; bude pro vás nečistý;
#11:5 damana, který také přežvykuje, ale nemá rozdělená kopyta; bude pro vás nečistý;
#11:6 zajíce, který také přežvykuje, ale nemá rozdělená kopyta; bude pro vás nečistý;
#11:7 vepře, který sice má rozdělená kopyta tak, že jsou úplně rozpolcená, ale nepřežvykuje; bude pro vás nečistý.
#11:8 Jejich maso nesmíte jíst, jejich zdechliny se nedotknete; budou pro vás nečistí.
#11:9 Ze všeho, co je ve vodě, smíte jíst toto: Všechno ve vodách, v mořích a potocích, co má ploutve a šupiny, smíte jíst.
#11:10 Z veškeré vodní havěti v mořích a potocích, ze všech živočichů, kteří jsou ve vodách, bude pro vás hodné opovržení všechno, co nemá ploutve ani šupiny.
#11:11 Budou pro vás hodni opovržení. Nesmíte jíst jejich maso a jejich zdechliny budete mít v opovržení.
#11:12 Všechno ve vodě, co nemá ploutve ani šupiny, budete mít v opovržení.
#11:13 Z létajících živočichů budete mít v opovržení tyto, nesmějí se jíst, jsou hodni opovržení: orla, orlosupa a mořského orla,
#11:14 luňáka a různé druhy jestřábů,
#11:15 všechny druhy havranů,
#11:16 pštrosa, sovu, racka a různé druhy sokolů,
#11:17 kulicha, kormorána a výra,
#11:18 sovu pálenou, pelikána a mrchožrouta,
#11:19 čápa a různé druhy volavek, dudka a netopýra.
#11:20 Všechna létající havěť chodící po čtyřech bude pro vás hodna opovržení.
#11:21 Jen to smíte jíst z veškeré létající havěti chodící po čtyřech, co má skákavé nohy, jimiž skáče po zemi.
#11:22 Smíte z nich jíst tyto: různé druhy kobylek, jako arbe, soleám, chargól a chágáb.
#11:23 Ale ostatní létající čtyřnohá havěť bude pro vás hodna opovržení.
#11:24 Tou byste se poskvrnili. Každý, kdo se dotkne jejich zdechliny, bude nečistý až do večera
#11:25 a každý, kdo by něco z jejich zdechliny nesl, vypere si šaty a bude nečistý až do večera.
#11:26 Každé zvíře, které má rozdělená kopyta, ale ne rozpolcená úplně, a které nepřežvykuje, bude pro vás nečisté. Každý, kdo se ho dotkne, bude nečistý.
#11:27 Také všechno ze čtyřnohých živočichů, co chodí po tlapách, bude pro vás nečisté. Každý, kdo se dotkne jejich zdechliny, bude nečistý až do večera.
#11:28 Kdo by nesl jejich zdechlinu, vypere si šaty a bude nečistý až do večera. To všechno je pro vás nečisté.
#11:29 Dále pro vás bude nečisté z havěti hemžící se po zemi: krysa, myš, různé druhy ještěrek,
#11:30 gekoni, scinkové a chameleón.
#11:31 Ti jsou pro vás nečistí z veškeré havěti. Každý, kdo se jich dotkne, když pojdou, bude nečistý až do večera.
#11:32 Také všechno, nač padne něco z nich, když pojdou, bude nečisté, ať to je dřevěný předmět nebo oděv, kůže či pytlovina, cokoli, čeho se užívá k práci; bude to vloženo do vody a bude to nečisté až do večera; pak to bude čisté.
#11:33 Když něco z nich spadne do hliněné nádoby, bude nečisté všechno, co je uvnitř, a nádobu rozbijete.
#11:34 Každý pokrm určený k jídlu, který přišel do styku s onou vodou, bude nečistý. Též každý nápoj, který se pije z takové nádoby, bude nečistý.
#11:35 Zkrátka všechno, nač padne něco z těch zdechlin, bude nečisté. Pec i krb budou nečisté, musí se zbořit; budou pro vás nečisté.
#11:36 Jen vodní pramen a jímka zadržující vodu budou čisté. Cokoli přijde do styku s těmi zdechlinami, bude nečisté.
#11:37 Když však něco z těch zdechlin padne na semeno, určené k setí, to zůstane čisté.
#11:38 Ale když bude na semeno nalita voda a padne na ně něco z těch zdechlin, bude pro vás nečisté.
#11:39 Když pojde některé ze zvířat, která máte dovoleno jíst, ten, kdo se dotkne jeho zdechliny, bude nečistý až do večera.
#11:40 Kdo z té zdechliny něco sní, vypere si šaty a bude nečistý až do večera. Též kdo by tu zdechlinu nesl, vypere si šaty a bude nečistý až do večera.
#11:41 Všechna havěť hemžící se po zemi je hodna opovržení, nesmí se jíst.
#11:42 Nic z veškeré havěti hemžící se po zemi, nic, co leze po břiše nebo chodí po čtyřech, ani co má více noh, nesmíte jíst, je to hodno opovržení.
#11:43 Neuvádějte v opovržení sami sebe pro nějakou hemžící se havěť; nesmíte se jimi poskvrnit, abyste se jimi neznečistili.
#11:44 Já jsem Hospodin, váš Bůh. Posvěťte se a buďte svatí, neboť já jsem svatý. Neposkvrňujte sami sebe žádnou havětí plazící se po zemi.
#11:45 Já jsem Hospodin, který jsem vás vyvedl z egyptské země, abych byl vaším Bohem. Proto buďte svatí, neboť já jsem svatý.“
#11:46 To je řád o zvířatech, o ptácích a o všech živých tvorech pohybujících se ve vodách i o všech, kteří se hemží po zemi.
#11:47 Je nutno rozlišovat mezi nečistým a čistým, mezi živočichy, kteří se smějí jíst, a těmi, kteří se jíst nesmějí. 
#12:1 Hospodin promluvil k Mojžíšovi:
#12:2 „Mluv k Izraelcům: Když žena otěhotní a porodí chlapce, bude nečistá po sedm dní; bude nečistá jako v době svého obvyklého krvácení.
#12:3 Osmého dne bude obřezána jeho předkožka.
#12:4 Matka pak setrvá v očišťování od krve ještě po třicet tři dny. Nedotkne se ničeho svatého a nevejde do svatyně, dokud neskončí dny jejího očišťování.
#12:5 Jestliže porodí děvče, bude nečistá dva týdny jako při svém krvácení a ještě po šedesát šest dní setrvá v očišťování od krve.
#12:6 Když skončí dny jejího očišťování po synu nebo dceři, přivede ročního beránka k zápalné oběti a holoubě nebo hrdličku k oběti za hřích knězi ke vchodu do stanu setkávání.
#12:7 On je přinese jako oběť před Hospodina a vykoná za ni smírčí obřady; tak bude očištěna od svého krvotoku. To je řád pro ženu při narození chlapce nebo děvčete.
#12:8 Jestliže si nemůže opatřit jehně, ať vezme dvě hrdličky nebo dvě holoubata, jedno k zápalné oběti a jedno k oběti za hřích. Kněz za ni vykoná smírčí obřady a bude čistá.“ 
#13:1 Hospodin promluvil k Mojžíšovi a Áronovi:
#13:2 „Když se někomu objeví na kůži otok nebo vyrážka či bělavá skvrna a na jeho kůži bude příznak malomocenství, bude přiveden ke knězi Áronovi nebo k některému z jeho synů, kněží.
#13:3 Kněz mu prohlédne postižené místo na kůži. Když chloupky na postiženém místě zbělely a ono je napohled hlubší než okolní kůže, je postižen malomocenstvím. Kněz ho prohlédne a prohlásí jej za nečistého.
#13:4 Jestliže je to jen bílá skvrna na kůži a napohled není hlubší než okolní kůže a chloupky na ní nezbělely, přikáže kněz postiženého na sedm dní uzavřít.
#13:5 Sedmého dne jej kněz prohlédne. Vidí-li, že postižené místo zůstalo beze změny a nerozšířilo se na kůži, uzavře ho na dalších sedm dní.
#13:6 Sedmého dne ho kněz opět prohlédne. Je-li postižené místo nevýrazné a nerozšířilo se po kůži, prohlásí jej kněz za čistého. Je to jen vyrážka. Vypere své šaty a bude čistý.
#13:7 Jestliže se však potom, co se ukázal knězi, aby byl očištěn, vyrážka po kůži značně rozšířila, ukáže se knězi znovu.
#13:8 Kněz ho prohlédne. Rozšířila-li se vyrážka po kůži, kněz jej prohlásí za nečistého. Je to malomocenství.
#13:9 Když se na někom objeví rána malomocenství, bude přiveden ke knězi.
#13:10 Kněz ho prohlédne. Je-li na kůži bílý otok a chloupky na něm zbělely a v otoku se objevilo živé maso,
#13:11 je to pokročilé malomocenství na kůži. Kněz jej prohlásí za nečistého. Nedá ho uzavřít, protože je nečistý.
#13:12 Jestliže se malomocenství na kůži rozmohlo tak silně, že pokrylo celou kůži a je postižen všude, kam se kněz podívá, od hlavy až k patě,
#13:13 kněz ho prohlédne, a jestliže malomocenství už pokrylo celé tělo, prohlásí postiženého za čistého. Zbělel celý, je čistý.
#13:14 Jakmile se však na něm ukáže živé maso, bude nečistý.
#13:15 Když kněz uvidí živé maso, prohlásí jej za nečistého; živé maso je nečisté, je to malomocenství.
#13:16 Když se živé maso ztratí a místo zbělí, přijde ke knězi.
#13:17 Kněz ho prohlédne. Jestliže postižené místo zbělelo, prohlásí kněz postiženého za čistého; je čistý.
#13:18 Kdo měl na kůži vřed, který se zhojil,
#13:19 a na místě vředu se objeví bělavý otok nebo bělavě načervenalá skvrna, musí se ukázat knězi.
#13:20 Kněz ho prohlédne. Je-li místo napohled nižší než okolní kůže a chloupky na něm zbělely, prohlásí jej kněz za nečistého. Je postižen malomocenstvím; rozmohlo se z vředu.
#13:21 Uvidí-li však kněz, že na něm nejsou bílé chloupky a není nižší než okolní kůže, ale že je nevýrazné, přikáže ho uzavřít na sedm dní.
#13:22 Jestliže se po kůži velice rozšíří, prohlásí jej kněz za nečistého; je postižen malomocenstvím.
#13:23 Zůstane-li skvrna stejná, nebude-li se rozšiřovat, je to jizva po vředu. Kněz ho prohlásí za čistého.
#13:24 Když má někdo na kůži spáleninu a na spálenině se objeví bělavě načervenalá či bílá skvrna,
#13:25 kněz ji prohlédne. Jestliže chloupky na skvrně zbělely a je napohled hlubší než okolní kůže, je to malomocenství, rozmohlo se na spálenině. Kněz jej prohlásí za nečistého, je postižen malomocenstvím.
#13:26 Uvidí-li kněz, že chloupky na skvrně nejsou bílé a místo není nižší než okolní kůže, ale že je nevýrazné, přikáže ho uzavřít na sedm dní.
#13:27 Sedmého dne ho kněz prohlédne. Jestliže se skvrna dál na kůži rozšířila, prohlásí jej kněz za nečistého. Je postižen malomocenstvím.
#13:28 Zůstane-li skvrna stejná, nebude-li se po kůži rozšiřovat a bude-li nevýrazná, byl to otok po spálenině. Kněz ho prohlásí za čistého; je to jizva po spálenině.
#13:29 Když se objeví ať muži či ženě nějaké postižené místo na hlavě nebo na bradě,
#13:30 kněz je prohlédne. Bude-li napohled hlubší než okolní kůže a budou-li na něm jemné nažloutlé chloupky, prohlásí jej kněz za nečistého. Je to prašivina, druh malomocenství na hlavě nebo na bradě.
#13:31 Když kněz prohlédne místo postižené prašivinou a ono není napohled hlubší než okolní kůže a nejsou na něm černé chloupky, přikáže uzavřít postiženého prašivinou na sedm dní.
#13:32 Sedmého dne kněz prohlédne postižené místo. Jestliže se prašivina nerozšířila a nejsou na ní nažloutlé chloupky a napohled není hlubší než okolní kůže,
#13:33 postižený se oholí, avšak místo zasažené prašivinou neoholí. Kněz přikáže uzavřít postiženého prašivinou na dalších sedm dní.
#13:34 Sedmého dne kněz opět prašivinu prohlédne. Jestliže se prašivina po kůži nerozšířila a napohled není hlubší než okolní kůže, prohlásí jej kněz za čistého. Vypere si šaty a bude čistý.
#13:35 Jestliže se však prašivina po jeho očištění velice rozšíří,
#13:36 kněz ho prohlédne. Jestliže se rozšířila prašivina po kůži, nebude kněz pátrat po nažloutlých chloupcích. Postižený je nečistý.
#13:37 Jestliže však prašivina zůstala beze změny a vyrostly na ní černé chloupky, je prašivina zhojena; je čistý. Kněz ho prohlásí za čistého.
#13:38 Když se objeví ať muži či ženě na kůži skvrny, bělavé skvrny,
#13:39 kněz je prohlédne. Jsou-li skvrny na kůži nevýrazně bělavé, je to lišej, který na kůži vyrazil. Je čistý.
#13:40 Když někomu olysá hlava a má pleš, je čistý.
#13:41 Jestliže mu hlava olysá na spáncích a má lysinu, je čistý.
#13:42 Kdyby se však na pleši či na lysině objevila bíločervená rána, je to malomocenství vyrážející na pleši nebo na lysině.
#13:43 Kněz ho prohlédne. Jestliže je na jeho pleši či na lysině otok a postižené místo je bělavě načervenalé, podobající se kožnímu malomocenství,
#13:44 je ten muž malomocný. Je nečistý. Kněz ho musí prohlásit za nečistého. Je postižen na hlavě.
#13:45 Malomocný, který je postižen, bude mít šaty roztržené, vlasy na hlavě neupravené, vousy zahalené a bude volat: ‚Nečistý, nečistý!‘
#13:46 Po všechny dny, co bude postižen, zůstane nečistý. Je nečistý. Bude bydlet v odloučení, jeho obydlí bude mimo tábor.
#13:47 Když postihne malomocenství oděv, ať už oděv vlněný nebo lněný,
#13:48 látku nebo tkaninu lněnou či vlněnou nebo kůži či nějaký předmět z kůže,
#13:49 a bude postižené místo nažloutlé nebo načervenalé, na oděvu či na kůži, na látce či na tkanině nebo na jakémkoli koženém předmětu, je to rána malomocenství. Ukáže se knězi.
#13:50 Kněz postižené místo prohlédne a přikáže postiženou věc uzavřít na sedm dní.
#13:51 Sedmého dne postižené místo opět prohlédne. Když se postižení rozšířilo na oděvu nebo na látce či na tkanině nebo na kůži a na všem, co se z kůže zhotovuje, je to postiženo zhoubným malomocenstvím. Je to nečisté.
#13:52 Spálí oděv či tu látku nebo tu tkaninu, ať už je to vlněné nebo lněné, i každý předmět z kůže, na němž se ta rána objevila. Je to zhoubné malomocenství. Spálí se to ohněm.
#13:53 Jestliže však kněz uvidí, že se to postižené místo na oděvu či na látce nebo na tkanině či na jakémkoli koženém předmětu nerozšiřuje,
#13:54 nařídí postiženou věc vyprat a uzavřít ji na dalších sedm dní.
#13:55 Po vyprání kněz opět postiženou věc prohlédne. Jestliže se postižené místo nezměnilo, i když se nerozšířilo, je to nečisté. Spálíš to ohněm, ať je ta věc zasažena na líci nebo na rubu.
#13:56 Jestliže však kněz uvidí, že postižené místo je po vyprání nevýrazné, odtrhne je od oděvu nebo od kůže, od látky či od tkaniny.
#13:57 Bude-li však rána ještě patrná na oděvu či na látce nebo na tkanině či na jakémkoli koženém předmětu a bude vyrážet dále, spálíš postiženou věc ohněm.
#13:58 Oděv, látka, tkanina nebo jakýkoli předmět z kůže, který jsi vypral a rána z toho zmizela, vypere se podruhé; pak to bude čisté.
#13:59 Toto je řád rány malomocenství na oděvu vlněném nebo lněném, na látce, tkanině či jakémkoli koženém předmětu při rozhodování o čistém nebo nečistém.“ 
#14:1 Hospodin promluvil k Mojžíšovi:
#14:2 „Toto je řád týkající se malomocného v den jeho očišťování: Bude přiveden ke knězi.
#14:3 Kněz vyjde ven z tábora a prohlédne ho. Je-li rána malomocenství na malomocném zhojena,
#14:4 rozkáže kněz, aby vzal pro očišťování dva živé čisté ptáky, cedrové dřevo, karmínové barvivo a yzop.
#14:5 Kněz rozkáže zabít jednoho ptáka nad hliněnou nádobou s pramenitou vodou.
#14:6 Pak vezme živého ptáka, cedrové dřevo, karmínové barvivo a yzop a omočí to i s živým ptákem v krvi ptáka zabitého nad pramenitou vodou.
#14:7 Stříkne sedmkrát na očišťovaného od malomocenství a očistí ho. Živého ptáka vypustí do pole.
#14:8 Očištěný si vypere šaty, oholí si celé tělo, omyje se vodou a bude čistý. Potom vejde do tábora, ale usadí se vně svého stanu na sedm dní.
#14:9 Sedmého dne si oholí všecky vlasy na hlavě, bradu, obočí, všechny chloupky si oholí, vypere si šaty, omyje se celý vodou a bude čistý.
#14:10 Osmého dne vezme dva beránky bez vady, jednu roční ovečku bez vady, tři desetiny éfy bílé mouky zadělané olejem jako přídavnou oběť a jeden džbánek oleje.
#14:11 Očišťující kněz postaví očišťovaného muže i ty věci před Hospodina, ke vchodu do stanu setkávání.
#14:12 Pak vezme kněz jednoho beránka a přinese jej se džbánkem oleje v oběť za vinu a podáváním to nabídne jako oběť podávání před Hospodinem.
#14:13 Beránka pak porazí na místě, kde se poráží oběť za hřích a oběť zápalná, na místě svatém, neboť knězi patří jak oběť za hřích, tak oběť za vinu. Je to velesvaté.
#14:14 Kněz vezme trochu krve z oběti za vinu a potře očišťovanému pravý lalůček ucha a palec na pravé ruce a u pravé nohy.
#14:15 Ze džbánku vezme kněz trochu oleje a odleje do levé dlaně.
#14:16 Pak omočí kněz prst pravé ruky v oleji na své levé dlani a sedmkrát stříkne prstem olej před Hospodinem.
#14:17 Trochou zbylého oleje na své dlani potře kněz očišťovanému lalůček pravého ucha a palec na pravé ruce a u pravé nohy tam, kde byl potřen krví z oběti za vinu.
#14:18 Zbytkem oleje z dlaně potře kněz hlavu očišťovaného. Tak za něj vykoná smírčí obřady před Hospodinem.
#14:19 Kněz připraví oběť za hřích, aby zprostil očišťovaného nečistoty; potom se porazí dobytče k oběti zápalné.
#14:20 Kněz bude obětovat zápalnou a přídavnou oběť na oltáři. Tak za něj kněz vykoná smírčí obřady. Bude čistý.
#14:21 Je-li dotyčný nemajetný a nemůže dát tolik, vezme k smírčím obřadům jednoho beránka v oběť za vinu jako oběť podávání a jednu desetinu éfy bílé mouky zadělané olejem jako přídavnou oběť a džbánek oleje,
#14:22 též dvě hrdličky nebo dvě holoubata podle svých možností. Jedno bude k oběti za hřích a druhé k zápalné oběti.
#14:23 Osmého dne je přinese knězi ke vchodu do stanu setkávání před Hospodina na své očištění.
#14:24 Kněz vezme beránka v oběť za vinu i džbánek oleje a podáváním to nabídne jako oběť podávání před Hospodinem.
#14:25 Pak porazí beránka v oběť za vinu a vezme trochu krve oběti za vinu a potře očišťovanému lalůček pravého ucha a palec na pravé ruce a u pravé nohy.
#14:26 Z oleje si kněz naleje trochu do levé dlaně.
#14:27 Sedmkrát stříkne prstem pravé ruky před Hospodinem olej, který má na své levé dlani.
#14:28 Trochou oleje na své dlani potře očišťovanému lalůček pravého ucha a palec na pravé ruce a u pravé nohy, na místě, kde byl potřen krví z oběti za vinu.
#14:29 Zbytkem oleje z dlaně potře kněz hlavu očišťovaného, a tím vykoná za něho smírčí obřady před Hospodinem.
#14:30 Potom připraví jednu z hrdliček nebo jedno z holoubat, jež dotyčný podle svých možností přinesl.
#14:31 Podle jeho možností bude jedno obětí za hřích a druhé zápalnou obětí, spolu s obětí přídavnou. Tak vykoná kněz smírčí obřady za očišťovaného před Hospodinem.
#14:32 To je řád týkající se toho, kdo byl stižen malomocenstvím a nemůže tolik dát na své očištění.“
#14:33 Hospodin dále mluvil k Mojžíšovi a Áronovi:
#14:34 „Až vejdete do kenaanské země, kterou vám dávám do vlastnictví, a já raním malomocenstvím některý dům v zemi, kterou budete mít ve vlastnictví,
#14:35 půjde majitel domu a oznámí knězi: ‚Zdá se mi, jako by byl můj dům postižen.‘
#14:36 Dříve než přijde kněz postižené místo prohlédnout, nařídí dům vyklidit, aby nebylo nečisté všecko, co je v domě. Potom přijde kněz, aby dům prohlédl.
#14:37 Prohlédne postižené místo. Jsou-li na postižené zdi domu nažloutlé nebo načervenalé dolíčky, napohled nižší než okolní zeď,
#14:38 vyjde kněz z domu ke vchodu do domu a přikáže uzavřít ten dům na sedm dní.
#14:39 Sedmého dne se kněz vrátí a opět to prohlédne. Jestliže se postižení na zdech domu rozšířilo,
#14:40 rozkáže postižené kameny vyjmout a vynést je ven za město na místo nečisté.
#14:41 Dům oškrabou uvnitř dokola a hlínu, kterou oškrabali, vysypou ven za město na místo nečisté.
#14:42 Pak vezmou jiné kameny a nahradí ony kameny a vezmou jinou hlínu a vymažou dům.
#14:43 Jestliže se postižení znovu v domě rozmůže i potom, co vyňali kameny a dům oškrabali a vymazali,
#14:44 přijde kněz opět. Uvidí-li, že se postižení na domě šíří, je to zhoubné malomocenství na domě. Je nečistý.
#14:45 Strhnou ten dům a kameny, dříví i všechnu hlínu z domu vynesou ven za město na místo nečisté.
#14:46 Kdo by vešel do toho domu ve dnech, kdy byl uzavřen, bude nečistý až do večera.
#14:47 Kdo by spal v tom domě, vypere si šaty. Kdo by v tom domě jedl, vypere si šaty.
#14:48 Jestliže kněz přijde a uvidí, že se rána na domě poté, co dům vymazali, nerozšiřuje, prohlásí dům za čistý, protože rána byla zhojena.
#14:49 Vezme dva ptáky, cedrové dřevo, karmínové barvivo a yzop, aby dům očistil od hříchu.
#14:50 Jednoho ptáka zabije nad hliněnou nádobou s pramenitou vodou.
#14:51 Pak vezme cedrové dřevo, yzop, karmínové barvivo a živého ptáka, omočí je v krvi zabitého ptáka a v pramenité vodě a sedmkrát stříkne na dům.
#14:52 Tak očistí ten dům od hříchu krví ptáka, pramenitou vodou, živým ptákem, cedrovým dřevem, yzopem a karmínovým barvivem.
#14:53 Živého ptáka vypustí ven z města do pole. Tak vykoná smírčí obřady za dům. Bude čistý.
#14:54 To je řád pro všechny rány malomocenství a prašiviny,
#14:55 pro ránu malomocenství na oděvu a na domě,
#14:56 pro otok, vyrážku a bělavou skvrnu,
#14:57 k poučení o tom, kdy je co nečisté a kdy čisté. To je řád týkající se malomocenství.“ 
#15:1 Hospodin promluvil k Mojžíšovi a Áronovi:
#15:2 „Mluvte k Izraelcům a řekněte jim: Když některý muž trpí výtokem, je pro svůj výtok nečistý.
#15:3 Toto je řád týkající se jeho nečistoty při výtoku: Ať jeho tělo výtok vypouští nebo je v jeho těle výtok zadržen, je to jeho nečistota.
#15:4 Každé lůžko, na němž by ležel trpící výtokem, bude nečisté, každý předmět, na němž by seděl, bude nečistý.
#15:5 Člověk, který by se dotkl jeho lůžka, vypere si šaty, omyje se vodou a bude nečistý až do večera.
#15:6 Kdo by si sedl na předmět, na němž seděl trpící výtokem, vypere si šaty, omyje se vodou a bude nečistý až do večera.
#15:7 Kdo by se dotkl těla toho, kdo trpí výtokem, vypere si šaty, omyje se vodou a bude nečistý až do večera.
#15:8 Kdyby trpící výtokem plivl na čistého, ten si vypere šaty, omyje se vodou a bude nečistý až do večera.
#15:9 Každé sedadlo, na němž by seděl trpící výtokem, bude nečisté.
#15:10 Každý, kdo by se dotkl čehokoli, co bylo pod nemocným, bude nečistý až do večera. Kdo by ty věci nesl, vypere si šaty, omyje se vodou a bude nečistý až do večera.
#15:11 Každý, koho by se dotkl trpící výtokem, aniž si opláchl ruce ve vodě, vypere si šaty, omyje se vodou a bude nečistý až do večera.
#15:12 Hliněná nádoba, jíž by se dotkl trpící výtokem, bude rozbita. Každá nádoba dřevěná bude opláchnuta vodou.
#15:13 Když trpící výtokem bude od svého výtoku čist, odpočítá si pro své očišťování ještě sedm dní. Vypere si šaty, omyje se celý pramenitou vodou a bude čistý.
#15:14 Osmého dne si vezme dvě hrdličky nebo dvě holoubata, přijde před Hospodina ke vchodu do stanu setkávání a dá je knězi.
#15:15 Kněz je připraví: jedno v oběť za hřích, druhé v oběť zápalnou. Tak za něho kněz vykoná pro jeho výtok smírčí obřady před Hospodinem.
#15:16 Muž, který měl výron semene, omyje si celé tělo vodou a bude nečistý až do večera.
#15:17 Každý oděv a všechno kožené, na nichž ulpí výron semene, budou vyprány ve vodě a budou nečisté až do večera.
#15:18 Kdyby muž mající výron semene obcoval se ženou, omyjí se oba vodou a budou nečistí až do večera.
#15:19 Když má žena výtok, totiž svůj pravidelný krvavý výtok, bude v období svého krvácení nečistá sedm dní. Každý, kdo by se jí dotkl, bude nečistý až do večera.
#15:20 Všechno, na čem by ležela v období svého krvácení, bude nečisté, a všechno, na čem by seděla, bude nečisté.
#15:21 Každý, kdo by se dotkl jejího lůžka, vypere si šaty, omyje se vodou a bude nečistý až do večera.
#15:22 Každý, kdo by se dotkl jakéhokoli předmětu, na kterém seděla, vypere si šaty, omyje se vodou a bude nečistý až do večera.
#15:23 Jestliže se dotkne něčeho, co bylo na lůžku či na předmětu, na němž seděla, bude nečistý až do večera.
#15:24 Jestliže s ní bude muž obcovat a ulpí na něm její krvácení, bude nečistý sedm dní a každé lůžko, na němž by ležel, bude nečisté.
#15:25 Když má žena dlouhotrvající krvotok mimo období svého krvácení nebo když výtok trvá déle než obvyklé krvácení, bude trvat její nečistota po všechny dny výtoku. Bude nečistá jako v období svého krvácení.
#15:26 Každé lůžko, na němž by ležela kterýkoli den svého výtoku, bude jako lůžko v období jejího krvácení. Každý předmět, na němž by seděla, bude nečistý jako při nečistotě jejího krvácení.
#15:27 Každý, kdo by se jich dotkl, bude nečistý; vypere si šaty, omyje se vodou a bude nečistý až do večera.
#15:28 Jestliže bude očištěna od svého výtoku, odpočítá si sedm dní a potom bude čistá.
#15:29 Osmého dne si vezme dvě hrdličky nebo dvě holoubata a přinese je knězi ke vchodu do stanu setkávání.
#15:30 Kněz připraví jedno v oběť za hřích a druhé v oběť zápalnou. Tak za ni kněz vykoná pro nečistotu jejího výtoku smírčí obřady před Hospodinem.
#15:31 Tak budete zdržovat Izraelce, aby se neznečistili a pro své znečištění nezemřeli, kdyby znečistili můj příbytek, který je uprostřed nich.“
#15:32 To je řád týkající se toho, kdo trpí výtokem nebo má výron semene, jímž je znečišťován,
#15:33 i té, která je nečistá svým krvácením; týká se toho, kdo trpí výtokem, ať to je muž nebo žena, i toho, kdo by obcoval s nečistou. 
#16:1 Hospodin promluvil k Mojžíšovi po smrti dvou synů Áronových, kteří zemřeli, když svévolně předstoupili před Hospodina.
#16:2 Hospodin řekl Mojžíšovi: „Promluv ke svému bratru Áronovi, ať nevstupuje v libovolné době do svatyně dovnitř za oponu k příkrovu, který je na schráně, aby nezemřel, až se objevím v oblaku nad příkrovem.
#16:3 Jen tak smí Áron přistupovat ke svatyni: s mladým býčkem k oběti za hřích a s beranem k oběti zápalné.
#16:4 Oblékne si svatou lněnou suknici, pod ní bude mít na těle lněné spodky, přepáše se lněnou šerpou a ovine si lněný turban. Je to svaté roucho. Celý se omyje vodou, než je oblékne.
#16:5 Od pospolitosti Izraelců vezme dva kozly k oběti za hřích a jednoho berana k oběti zápalné.
#16:6 Potom přivede Áron býčka jako svou oběť za hřích a vykoná za sebe a za svůj dům smírčí obřady.
#16:7 Vezme i oba kozly a postaví je před Hospodina, u vchodu do stanu setkávání.
#16:8 O obou kozlech bude Áron losovat: jeden los pro Hospodina, druhý pro Azázela.
#16:9 Pak přivede Áron kozla, na kterého padl los pro Hospodina, a připraví jej v oběť za hřích.
#16:10 Kozel, na kterého padl los pro Azázela, bude postaven živý před Hospodina, aby na něm vykonal smírčí obřady a vyhnal jej k Azázelovi na poušť.
#16:11 Potom přivede Áron býčka jako svou oběť za hřích, aby za sebe a za svůj dům vykonal smírčí obřady, a porazí ho jako svou oběť za hřích.
#16:12 Vezme z oltáře, od Hospodina, kadidelnici plnou žhavého uhlí a dvě hrsti jemného kadidla z vonných věcí a vnese to dovnitř za oponu.
#16:13 Kadidlo vloží na oheň před Hospodina. Oblak z kadidla zahalí příkrov, který je na schráně svědectví, aby nezemřel.
#16:14 Potom vezme trochu krve z býčka a stříkne ji prstem na příkrov z přední strany a před příkrov stříkne prstem trochu krve sedmkrát.
#16:15 Pak porazí kozla, oběť za hřích lidu, a vnese jeho krev dovnitř za oponu. S jeho krví naloží stejně jako s krví býčka; stříkne ji na příkrov a před příkrov.
#16:16 Tak vykoná smírčí obřady za svatyni pro nečistotu Izraelců, pro jejich přestoupení a všechny jejich hříchy. Stejně bude postupovat při stanu setkávání, který stojí u nich, uprostřed jejich nečistot.
#16:17 Nikdo z lidí nesmí být ve stanu setkávání, když vejde do svatyně k vykonávání smírčích obřadů, dokud nevyjde. Tam vykoná smírčí obřady za sebe i za svůj dům a za celé izraelské shromáždění.
#16:18 Potom vyjde k oltáři, který je před Hospodinem, a vykoná smírčí obřady za něj. Vezme trochu krve z býčka a krve z kozla a potře rohy oltáře dokola.
#16:19 Sedmkrát na něj stříkne prstem trochu krve. Tak jej očistí od nečistot Izraelců a posvětí ho.
#16:20 Když dokončí smírčí obřady za svatyni, za stan setkávání a za oltář, přivede živého kozla.
#16:21 Áron vloží obě ruce na hlavu živého kozla. Vyzná nad ním všechny nepravosti Izraelců a všechna jejich přestoupení se všemi jejich hříchy a vloží je na hlavu kozla; pak ho dá připraveným mužem vyhnat do pouště.
#16:22 Kozel na sobě ponese všechny jejich nepravosti do odlehlé země. Toho kozla vyžene na poušť.
#16:23 Potom vstoupí Áron do stanu setkávání a svlékne lněné roucho, které si oblékl při vstupu do svatyně, a zanechá je tam.
#16:24 Celý se omyje vodou na svatém místě a obleče si své roucho. Pak vyjde, vykoná zápalnou oběť za sebe i zápalnou oběť za lid a smírčí obřady za sebe i za lid.
#16:25 Tuk z oběti za hřích obrátí na oltáři v obětní dým.
#16:26 Ten, kdo vyhnal kozla pro Azázela, vypere si šaty a celý se omyje vodou. Teprve pak smí vstoupit do tábora.
#16:27 Býčka k oběti za hřích i kozla k oběti za hřích, jejichž krev byla vnesena do svatyně k smírčím obřadům, dá vynést ven za tábor. Jejich kůže, maso a výměty spálí ohněm.
#16:28 Ten, kdo je spálil, vypere si šaty a celý se omyje vodou. Teprve pak smí vstoupit do tábora.
#16:29 To bude pro vás provždy platné nařízení. Desátého dne sedmého měsíce se budete pokořovat a nebudete vykonávat žádnou práci, ani domorodec ani ten, kdo mezi vámi přebývá jako host.
#16:30 V tento den za vás vykoná smírčí obřady a očistí vás ode všech vašich hříchů. Budete před Hospodinem čisti.
#16:31 Bude to pro vás den odpočinku, slavnost odpočinutí. Budete se pokořovat. To je provždy platné nařízení.
#16:32 Smírčí obřady bude vykonávat kněz, který je pomazán a uveden v úřad, aby sloužil jako kněz místo svého otce, a který smí oblékat svaté lněné roucho.
#16:33 Ten vykoná smírčí obřady za velesvatyni, vykoná smírčí obřady za stan setkávání a za oltář; také za kněze i za shromáždění všeho lidu vykoná smírčí obřady.
#16:34 To pro vás bude provždy platné nařízení: jednou v roce vykoná za Izraelce smírčí obřady, aby byli zbaveni všech svých hříchů.“ I učinil Mojžíš, jak mu Hospodin přikázal. 
#17:1 Hospodin promluvil k Mojžíšovi:
#17:2 „Mluv k Áronovi a jeho synům i ke všem Izraelcům a řekni jim: Toto přikázal Hospodin:
#17:3 Kdokoli z domu izraelského porazí býka, jehně nebo kozu v táboře nebo je porazí venku za táborem
#17:4 a ke vchodu do stanu setkávání je nepřivede, aby přinesl dar Hospodinu před Hospodinův příbytek, ten za tu krev zaplatí. Prolil krev, proto bude vyobcován ze společenství svého lidu.
#17:5 To proto, aby Izraelci, když přivádějí své oběti, které si zvykli obětovat na poli, přiváděli je Hospodinu ke vchodu do stanu setkávání, ke knězi. Budou je obětovat Hospodinu jako hody oběti pokojné.
#17:6 Kněz pokropí krví Hospodinův oltář u vchodu do stanu setkávání a obrátí tuk v obětní dým, v libou vůni pro Hospodina.
#17:7 Už nebudou obětovat své oběti běsům, s nimiž smilnili. Toto jim bude provždy platné nařízení pro všechna jejich pokolení.
#17:8 Dále jim oznam: Kdokoli z domu izraelského i z těch, kdo mezi vámi přebývají jako hosté, bude obětovat zápalnou oběť nebo připraví obětní hod
#17:9 a ke vchodu do stanu setkávání to nepřivede, aby to připravil Hospodinu, bude vyobcován ze svého lidu.
#17:10 Kdokoli z domu izraelského i z těch, kdo mezi vámi přebývají jako hosté, bude jíst jakoukoli krev, proti tomu se obrátím; každého, kdo by jedl krev, vyobcuji ze společenství jeho lidu.
#17:11 V krvi je život těla. Já jsem vám ji určil na oltář k vykonávání smírčích obřadů za vaše životy. Je to krev; pro život, který je v ní, se získává smíření.
#17:12 Proto jsem rozkázal Izraelcům: Nikdo z vás nebude jíst krev, ani ten, kdo mezi vámi přebývá jako host, nebude jíst krev.
#17:13 Kdokoli z Izraelců i z těch, kdo mezi vámi přebývají jako hosté, uloví zvíře nebo ptáka, které se smí jíst, nechá vytéci jeho krev a přikryje ji prachem,
#17:14 neboť život každého tvora je v jeho krvi, ta ho oživuje. Proto jsem Izraelcům řekl: Nebudete jíst krev žádného tvora, neboť život každého tvora je v jeho krvi. Každý, kdo by ji jedl, bude vyobcován.
#17:15 Každý, kdo by jedl zdechlinu nebo rozsápané zvíře, ať domorodec či host, vypere si šaty, omyje se vodou a bude nečistý až do večera. Potom bude opět čistý.
#17:16 Jestliže by je nevypral a celý se neomyl, ponese následky své nepravosti.“ 
#18:1 Hospodin promluvil k Mojžíšovi:
#18:2 „Mluv k Izraelcům a řekni jim: Já jsem Hospodin, váš Bůh.
#18:3 Nesmíte jednat po způsobu egyptské země, v níž jste sídlili, ani po způsobu kenaanské země, do které vás vedu. Nebudete se řídit jejich zvyklostmi.
#18:4 Budete jednat podle mých řádů a dbát na má nařízení; jimi se budete řídit, neboť já jsem Hospodin, váš Bůh.
#18:5 Budete dbát na moje nařízení a na moje řády. Člověk, který podle nich bude jednat, bude z nich žít. Já jsem Hospodin.
#18:6 Nikdo se nepřiblíží k některé ze svých blízkých příbuzných, aby odkryl její nahotu. Já jsem Hospodin.
#18:7 Neodkryješ nahotu svého otce tím, že bys odkryl nahotu své matky. Je to tvá matka, neodkryješ její nahotu.
#18:8 Neodkryješ nahotu ani jiné ženy svého otce. Je to také nahota tvého otce.
#18:9 Neodkryješ nahotu žádné své sestry, dcery svého otce nebo své matky, ať zplozené v manželství nebo mimo. Neodkryješ jejich nahotu.
#18:10 Neodkryješ nahotu dcery svého syna nebo své dcery. Neodkryješ jejich nahotu, je to tvá nahota.
#18:11 Neodkryješ nahotu dcery jiné ženy svého otce. Tvůj otec ji zplodil, je to tvá sestra, neodkryješ její nahotu.
#18:12 Neodkryješ nahotu sestry svého otce. Je to blízká příbuzná tvého otce.
#18:13 Neodkryješ nahotu sestry své matky. Je to blízká příbuzná tvé matky.
#18:14 Neodkryješ nahotu bratra svého otce tím, že by ses přiblížil k jeho ženě. Je to tvá teta.
#18:15 Neodkryješ nahotu své snachy. Je to žena tvého syna, neodkryješ její nahotu.
#18:16 Neodkryješ nahotu ženy svého bratra. Je to nahota tvého bratra.
#18:17 Neodkryješ nahotu některé ženy a její dcery. Nevezmeš si ani dceru jejího syna nebo její dcery proto, abys tím odkryl její nahotu. Jsou to blízké příbuzné. Byl by to mrzký čin.
#18:18 Nevezmeš si za ženu ani sestru své ženy, abys tu první soužil tím, že bys odkrýval její nahotu za jejího života.
#18:19 Nepřiblížíš se k ženě, abys odkryl její nahotu v období jejího krvácení, kdy je nečistá.
#18:20 Nebudeš obcovat se ženou svého bližního a neposkvrníš se s ní.
#18:21 Nedopustíš, aby někdo z tvých potomků byl přiveden v oběť Molekovi. Neznesvětíš jméno svého Boha. Já jsem Hospodin.
#18:22 Nebudeš obcovat s mužem jako s ženou. Je to ohavnost.
#18:23 Nebudeš obcovat s žádným dobytčetem a neznečistíš se s ním. Ani žena se nepostaví před dobytče, aby se s ní pářilo. Je to zvrhlost.
#18:24 Ničím z toho všeho se neznečišťuj, neboť tím vším se znečišťují pronárody, které před vámi vyháním.
#18:25 Jimi byla znečištěna země. Proto jsem ji ztrestal pro její nepravost a ona vyvrhla své obyvatele.
#18:26 Vy tedy dbejte na má nařízení a na mé řády, ať se nedopustíte žádné z těchto ohavností, ani domorodec, ani ten, kdo přebývá mezi vámi jako host.
#18:27 Všech těchto ohavností se totiž dopouštěli mužové této země, kteří byli před vámi, takže země byla znečištěna.
#18:28 Nebo i vás země vyvrhne, kdybyste ji znečistili, jako vyvrhla pronárod, který byl před vámi.
#18:29 Každý, kdo se dopustí kterékoli z těchto ohavností, bude vyobcován ze společenství svého lidu.
#18:30 Proto dbejte na to, co jsem vám uložil. Nesmíte jednat podle ohavných zvyklostí, které byly páchány před vámi. Neznečišťujte se jimi. Já jsem Hospodin, váš Bůh.“ 
#19:1 Hospodin promluvil k Mojžíšovi:
#19:2 „Mluv k celé pospolitosti Izraelců a řekni jim: Buďte svatí, neboť já Hospodin, váš Bůh, jsem svatý.
#19:3 Každý měj v úctě svou matku a svého otce. Dbejte na mé dny odpočinku. Já jsem Hospodin, váš Bůh.
#19:4 Neobracejte se k bůžkům a neodlévejte si sochy bohů. Já jsem Hospodin, váš Bůh.
#19:5 Až budete obětovat Hospodinu hod oběti pokojné, obětujte jej tak, aby ve vás našel zalíbení.
#19:6 V den obětního hodu a v následující den se maso sní. Co by zůstalo do třetího dne, bude spáleno ohněm.
#19:7 Jestliže by se přece jedlo třetího dne, je závadné; Hospodin nenajde v obětníkovi zalíbení.
#19:8 Kdo by z něho jedl, ponese následky své nepravosti, neboť znesvětil, co je svaté Hospodinu. Bude vyobcován ze svého lidu.
#19:9 Až budete ve své zemi sklízet obilí, nepožneš své pole až do samého kraje a nebudeš paběrkovat, co zbylo po žni.
#19:10 Ani svou vinici úplně nevysbíráš, nebudeš na své vinici paběrkovat spadaná zrnka; ponecháš je pro zchudlého a pro hosta. Já jsem Hospodin, váš Bůh.
#19:11 Nebudete krást ani obelhávat a podvádět svého bližního.
#19:12 Nebudete křivě přísahat v mém jménu, sice znesvětíš jméno svého Boha. Já jsem Hospodin.
#19:13 Nebudeš utiskovat a odírat svého druha. Výdělek dělníka, kterého si najmeš, nezůstane u tebe do rána.
#19:14 Nebudeš zlořečit hluchému a slepému nepoložíš do cesty překážku, ale budeš se bát svého Boha. Já jsem Hospodin.
#19:15 Nedopustíte se bezpráví při soudu. Nebudeš nadržovat nemajetnému ani brát ohled na mocného; budeš soudit svého bližního spravedlivě.
#19:16 Nebudeš se chovat ve svém lidu jako utrhač, nebudeš ukládat svému bližnímu o život. Já jsem Hospodin.
#19:17 Nebudeš ve svém srdci chovat nenávist ke svému bratru, ale budeš trestat svého bližního podle práva, a neponeseš následky jeho hříchu.
#19:18 Nebudeš se mstít synům svého lidu a nezanevřeš na ně, ale budeš milovat svého bližního jako sebe samého. Já jsem Hospodin.
#19:19 Dbejte na má nařízení. Když budeš připouštět dobytek, nesmíš křížit dvojí druh. Své pole nebudeš osívat dvojím druhem semene. Nebudeš nosit šaty utkané z dvojího druhu vláken.
#19:20 Kdyby někdo obcoval s ženou, která jako otrokyně byla určena pro jiného, ale ještě nebyla ani vykoupena penězi ani propuštěna na svobodu, bude případ vyšetřen, ale nebudou potrestáni smrtí, protože nebyla ještě propuštěna.
#19:21 Muž přinese Hospodinu svou oběť za vinu ke vchodu do stanu setkávání, berana v oběť za vinu.
#19:22 Obětí berana za vinu vykoná za něho kněz smírčí obřad před Hospodinem pro hřích, jehož se dopustil; hřích, kterým se prohřešil, mu bude odpuštěn.
#19:23 Až přijdete do země a budete sázet různé ovocné stromy, bude vám jejich ovoce nedotknutelné jako neobřízka; po tři roky bude pro vás neobřezané, nesmí se jíst.
#19:24 Čtvrtého léta bude všechno jejich ovoce posvěceno k oslavě Hospodina.
#19:25 Teprve v pátém roce budete jíst jejich ovoce, abyste nadále sklízeli bohatou úrodu. Já jsem Hospodin, váš Bůh.
#19:26 Nebudete jíst maso s krví, nebudete se obírat hadačstvím ani věštěním.
#19:27 Nebudete si přistřihovat na hlavě vlasy dokola, nezohavíš si okraj plnovousu.
#19:28 Nebudete svá těla zjizvovat pro mrtvého ani si dělat nějaké tetování. Já jsem Hospodin.
#19:29 Nezneuctíš svou dceru tím, že bys ji učinil nevěstkou, aby země nepropadla smilstvu a nebyla naplněna mrzkostí.
#19:30 Dbejte na mé dny odpočinku a mějte v úctě mou svatyni. Já jsem Hospodin.
#19:31 Neobracejte se k duchům zemřelých a nevyhledávejte vědmy a neposkvrňujte se jimi. Já jsem Hospodin, váš Bůh.
#19:32 Před šedinami povstaň a starci vzdej poctu. Boj se svého Boha. Já jsem Hospodin.
#19:33 Bude-li přebývat s tebou ve vaší zemi někdo jako host, nebudete mu škodit.
#19:34 Ten, kdo bude s vámi přebývat jako host, bude vám jako domorodec mezi vámi. Budeš ho milovat jako sebe samého, protože i vy jste byli hosty v zemi egyptské. Já jsem Hospodin, váš Bůh.
#19:35 Nedopustíte se bezpráví při soudu, při měření, vážení a odměřování.
#19:36 Budete mít poctivé váhy, poctivá závaží, poctivé míry velké i malé. Já jsem Hospodin, váš Bůh, já jsem vás vyvedl z egyptské země.
#19:37 Proto dbejte na všechna má nařízení a všechny mé řády dodržujte. Já jsem Hospodin.“ 
#20:1 Hospodin promluvil k Mojžíšovi:
#20:2 „Řekni Izraelcům: Kdokoli z Izraelců i z těch, kdo budou v Izraeli přebývat jako hosté, by daroval Molekovi někoho ze svých potomků, musí zemřít; lid země ho ukamenuje.
#20:3 Já sám se obrátím proti tomu muži a vyobcuji jej ze společenství jeho lidu, neboť daroval někoho ze svých potomků Molekovi, a tak znečistil mou svatyni a znesvětil mé svaté jméno.
#20:4 Jestliže lid země přivře oči, aby neviděl, že ten muž daroval někoho ze svých potomků Molekovi, a nepotrestá ho smrtí,
#20:5 postavím se sám proti tomu muži i proti jeho čeledi a vyobcuji jej ze společenství jejich lidu se všemi, kdo ho v tom smilstvu následovali a také smilnili s Molekem.
#20:6 Kdyby se někdo uchyloval k duchům zemřelých a k vědmám, aby s nimi smilnil, proti takovému se obrátím a vyobcuji jej ze společenství jeho lidu.
#20:7 Posvěťte se a buďte svatí, protože já jsem Hospodin, váš Bůh.
#20:8 Dbejte na má nařízení a dodržujte je; já jsem Hospodin, já vás posvěcuji.
#20:9 Kdo by zlořečil svému otci nebo své matce, musí zemřít. Svému otci a matce zlořečil, jeho krev padni na něho.
#20:10 Kdo se dopustí cizoložství s ženou někoho jiného, kdo cizoloží s ženou svého bližního, musí zemřít, cizoložník i cizoložnice.
#20:11 Kdo by spal s ženou svého otce, odkryl nahotu svého otce. Oba musejí zemřít, jejich krev padni na ně.
#20:12 Kdyby někdo spal se svou snachou, oba musejí zemřít; dopustili se zvrhlosti, jejich krev padni na ně.
#20:13 Kdyby muž spal s mužem jako s ženou, oba se dopustili ohavnosti; musejí zemřít, jejich krev padni na ně.
#20:14 Kdyby někdo pojal ženu a zároveň i její matku, spáchal mrzký čin. Budou upáleni, muž i ty ženy, aby nedocházelo mezi vámi k mrzkým činům.
#20:15 Kdyby muž obcoval s dobytčetem, musí zemřít. Dobytče zabijete.
#20:16 Kdyby se žena přiblížila k nějakému dobytčeti, aby se s ní pářilo, zabijete ženu i dobytče. Musejí zemřít, jejich krev padni na ně.
#20:17 Kdyby si někdo vzal svou sestru, dceru svého otce nebo své matky, aby spatřil její nahotu a ona spatřila jeho nahotu, je to potupa; budou vyobcováni před zraky synů svého lidu. Odkryl nahotu své sestry, ponese následky své nepravosti.
#20:18 Kdyby někdo spal se ženou v období její nečistoty, odkryl její nahotu a obnažil zdroj jejího krvácení a ona by nechala odkrýt zdroj svého krvácení, budou oba vyobcováni ze společenství svého lidu.
#20:19 Neodkryješ nahotu sestry své matky nebo svého otce; znamenalo by to obnažit svého příbuzného. Ponesou svou vinu.
#20:20 Kdo by spal se svou tetou, odkryl nahotu svého strýce. Ponesou následky svého hříchu, zemřou bezdětní.
#20:21 Kdyby někdo pojal ženu svého bratra, znamená to znečištění. Odkryl nahotu svého bratra, zůstanou bezdětní.
#20:22 Budete dbát na všechna má nařízení a na všechny moje řády a jednat podle nich, aby vás nevyvrhla země, do které vás vedu, abyste v ní sídlili.
#20:23 Nebudete se řídit zvyklostmi pronároda, který před vámi vyháním. Zprotivil jsem si je, protože se toho všeho dopouštěli.
#20:24 Ale vám jsem řekl, že dostanete do vlastnictví jejich půdu. Vám ji dám do vlastnictví, zemi oplývající mlékem a medem. Já jsem Hospodin, váš Bůh, já jsem vás oddělil od každého jiného lidu.
#20:25 Proto rozlišujte mezi čistými a nečistými zvířaty a mezi nečistým a čistým ptactvem, abyste neuvedli sami sebe v opovržení kvůli zvířatům, ptactvu a všelijakým zeměplazům, které jsem oddělil, abyste je měli za nečisté.
#20:26 Budete svatí pro mne, neboť já Hospodin jsem svatý. Oddělil jsem vás od každého jiného lidu, abyste byli moji.
#20:27 Muž či žena, v nichž by byl duch zemřelých nebo duch věštecký, musejí zemřít. Ukamenují je. Jejich krev padni na ně.“ 
#21:1 Hospodin řekl Mojžíšovi: „Mluv ke kněžím, Áronovcům, a řekni jim: Žádný kněz se neposkvrní při někom mrtvém ze svého lidu
#21:2 s výjimkou svého nejbližšího příbuzného: matky, otce, syna, dcery, bratra
#21:3 a svobodné sestry; byla mu blízká, protože se neprovdala; při ní se smí poskvrnit.
#21:4 Neposkvrní se ve svém lidu ani jako manžel, aby nebyl znesvěcen.
#21:5 Kněží si nebudou vyholovat na hlavě lysinu ani zastřihovat okraj svého vousu ani své tělo zjizvovat.
#21:6 Mají být svatí pro svého Boha; neznesvětí jeho jméno, neboť přinášejí ohnivé oběti Hospodinovy, chléb svého Boha. Proto budou svatí.
#21:7 Nevezmou si za ženu nevěstku nebo zneuctěnou ženu, ani se neožení se ženou zapuzenou od muže, neboť kněz je svatý pro svého Boha.
#21:8 Měj ho tedy za svatého, neboť přináší na oltář chléb tvého Boha. Bude pro tebe svatý, protože svatý jsem já Hospodin; já vás posvěcuji.
#21:9 Když se dcera některého kněze znesvětí smilstvem, znesvětila svého otce; bude upálena.
#21:10 Kněz, přední mezi svými bratry, na jehož hlavu byl vylit olej pomazání a který byl uveden v úřad, aby oblékal kněžské roucho, nebude mít vlasy na hlavě neupravené a neroztrhne své roucho.
#21:11 Nepřistoupí k nikomu mrtvému, neposkvrní se ani při svém otci ani při své matce.
#21:12 Nevyjde ze svatyně a neznesvětí svatyni svého Boha, neboť byl zasvěcen olejem pomazání svého Boha. Já jsem Hospodin.
#21:13 Za ženu si vezme jen pannu.
#21:14 S vdovou nebo zapuzenou nebo zneuctěnou či s nevěstkou, se žádnou takovou se neožení. Vezme si za ženu pannu ze svého lidu,
#21:15 aby neznesvětil své potomstvo ve svém lidu. Já jsem Hospodin, já ho posvěcuji.“
#21:16 Hospodin dále mluvil k Mojžíšovi:
#21:17 „Mluv k Áronovi: Když se v pokoleních tvého potomstva vyskytne muž, který by měl nějakou vadu, nepřiblíží se, aby přinášel chléb svého Boha.
#21:18 Nepřiblíží se žádný muž s vadou: nikdo slepý nebo kulhavý nebo se znetvořenou tváří nebo s některým údem příliš dlouhým,
#21:19 nebo kdo by měl zlomenou nohu nebo zlomenou ruku,
#21:20 nebo hrbatý nebo zakrnělý nebo se skvrnou na oku nebo postižený svrabem nebo lišejem nebo s rozdrcenými varlaty.
#21:21 Nikdo z potomstva kněze Árona, kdo by měl nějakou vadu, se nepřiblíží, aby přinášel ohnivé oběti Hospodinovy. Má vadu, nepřiblíží se, aby přinášel chléb svého Boha.
#21:22 Chléb svého Boha z velesvatých i svatých darů smí jíst.
#21:23 Ale nesmí přistupovat k oponě a nepřiblíží se k oltáři, neboť má vadu, aby neznesvětil prostory mé svatyně. Já jsem Hospodin, já je posvěcuji.“
#21:24 Mojžíš to vyhlásil Áronovi, jeho synům a všem Izraelcům. 
#22:1 Hospodin promluvil k Mojžíšovi:
#22:2 „Mluv k Áronovi a jeho synům, ať se zdržují svatých darů Izraelců, které mi oddělují jako svaté, a ať neznesvěcují mé svaté jméno. Já jsem Hospodin.
#22:3 Řekni jim: Kdyby se někdo ve vašich pokoleních ze všeho vašeho potomstva přiblížil ke svatým darům, které Izraelci oddělují Hospodinu jako svaté, a byl nečistý, bude vyobcován od mé tváře. Já jsem Hospodin.
#22:4 Kdokoli z potomstva Áronova by byl malomocný nebo by trpěl výtokem, nesmí jíst ze svatých darů, dokud nebude čistý. Kdyby se dotkl čehokoli, co je znečištěno mrtvým, nebo měl výron semene
#22:5 nebo by se někdo dotkl jakékoli havěti, jež by ho znečistila, nebo člověka, který by ho znečistil jakoukoli svou nečistotou,
#22:6 každý, kdo by se něčeho z toho dotkl, bude nečistý až do večera; nesmí jíst ze svatých darů, dokud se celý neomyje vodou.
#22:7 Po západu slunce bude čistý a potom může jíst ze svatých darů, protože to je jeho pokrm.
#22:8 Zdechlinu nebo něco rozsápaného nebude jíst, aby se tím neznečistil. Já jsem Hospodin.
#22:9 Všichni budou dbát na to, co jsem jim uložil, aby nenesli následky hříchu a nezemřeli pro něj, kdyby to znesvětili. Já jsem Hospodin, já je posvěcuji.
#22:10 Žádný nepovolaný nebude jíst nic svatého; ani ten, kdo by právě dlel u kněze, ani člověk jím za mzdu najatý nebude jíst nic svatého.
#22:11 Když si však kněz koupí člověka za stříbro, ten z toho jíst může; i narození v jeho domě budou jíst z jeho pokrmu.
#22:12 Když se kněžská dcera vdá za nekněze, nesmí jíst ze svatých darů obětovaných pozdvihováním.
#22:13 Ale když kněžská dcera ovdoví nebo bude zapuzena a nemá potomka a vrátí se do domu svého otce jako za svého mládí, bude jíst z pokrmu otcova; ovšem žádný nepovolaný z něho jíst nebude.
#22:14 Kdyby někdo jedl něco svatého neúmyslně, přidá k tomu pětinu a dá jako svatý dar knězi.
#22:15 Kněží nedovolí znesvětit svaté dary Izraelců, které pozdvihují pro Hospodina;
#22:16 uvalili by tíhu provinění na ty, kdo by jedli jejich svaté dary. Já jsem Hospodin, já je posvěcuji.“
#22:17 Hospodin dále mluvil k Mojžíšovi:
#22:18 „Mluv k Áronovi a jeho synům i ke všem Izraelcům. Řekneš jim: Kdo by z izraelského domu nebo z těch, kdo přebývají v Izraeli jako hosté, přinesl svůj dar při různých svých slibech nebo dobrovolných závazcích a přinesl jej jako oběť zápalnou Hospodinu,
#22:19 aby ve vás našel zalíbení, dá samce bez vady ze skotu, z ovcí nebo z koz.
#22:20 Nepřinesete nic, co by mělo vadu, neboť byste tím nedošli zalíbení.
#22:21 Když někdo připraví hod oběti pokojné pro Hospodina ze skotu nebo z bravu, aby splnil slib, nebo jako dobrovolný dar, pak jen oběť bez vady dojde zalíbení; nesmí na ní být žádná vada.
#22:22 Nic slepého nebo polámaného či zmrzačeného nebo vředovitého, svrabovitého či lišejovitého nepřinesete Hospodinu a nevložíte z toho na oltář ohnivou oběť pro Hospodina.
#22:23 Býka nebo jehně s příliš dlouhými nebo zakrslými údy můžeš dát jako dobrovolný dar, ale při slibu nenajdou zalíbení.
#22:24 Zvíře s varlaty rozmáčknutými, roztlučenými, odříznutými nebo pořezanými nepřinesete Hospodinu. To ve své zemi dělat nebudete.
#22:25 Ani od cizince nepřinesete pokrm svému Bohu z takových zvířat, neboť jsou porušená, mají vadu; ta vám nezískají zalíbení.“
#22:26 Hospodin dále mluvil k Mojžíšovi:
#22:27 „Když se narodí býček nebo beránek nebo kozlík, ať je sedm dní pod svou matkou; počínaje osmým dnem bude se zalíbením přijat jako dar ohnivé oběti pro Hospodina.
#22:28 Kus skotu nebo bravu neporazíte v týž den s jeho mládětem.
#22:29 Když budete obětovat Hospodinu obětní hod díků, obětujte tak, aby ve vás našel zalíbení.
#22:30 Týž den bude sněden, nic z něho nenechávejte do rána. Já jsem Hospodin.
#22:31 Dbejte na mé příkazy a jednejte podle nich. Já jsem Hospodin.
#22:32 Neznesvětíte mé svaté jméno. Ať jsem posvěcen mezi Izraelci. Já jsem Hospodin, já vás posvěcuji.
#22:33 Vyvedl jsem vás z egyptské země, abych vám byl Bohem, já Hospodin.“ 
#23:1 Hospodin promluvil k Mojžíšovi:
#23:2 „Mluv k Izraelcům a řekni jim: Slavnosti Hospodinovy, které budete svolávat, jsou bohoslužebná shromáždění. Jsou to mé slavnosti.
#23:3 Šest dní se bude pracovat, ale sedmého dne bude den odpočinku, slavnost odpočinutí, bohoslužebné shromáždění; nebudete vykonávat žádnou práci. Je to Hospodinův den odpočinku ve všech vašich sídlištích.
#23:4 Toto jsou slavnosti Hospodinovy, bohoslužebná shromáždění, jež budete svolávat v jejich určený čas:
#23:5 Čtrnáctého dne prvního měsíce navečer bude Hospodinův hod beránka.
#23:6 Patnáctého dne téhož měsíce začne Hospodinova slavnost nekvašených chlebů; po sedm dní budete jíst nekvašené chleby.
#23:7 Prvého dne budete mít bohoslužebné shromáždění; nebudete konat žádnou všední práci.
#23:8 Po sedm dní budete přinášet ohnivou oběť Hospodinu. Sedmého dne bude bohoslužebné shromáždění; nebudete konat žádnou všední práci.“
#23:9 Hospodin dále mluvil k Mojžíšovi:
#23:10 „Mluv k Izraelcům a řekni jim: Až přijdete do země, kterou vám dávám, a budete sklízet obilí, přinesete snopek jako prvotiny své žně knězi.
#23:11 Podáváním nabídne snopek Hospodinu, aby ve vás našel zalíbení; druhého dne po dni odpočinku jej nabídne kněz podáváním.
#23:12 V den, kdy za vás bude nabízet váš snopek podáváním, připravíte Hospodinu k zápalné oběti ročního beránka bez vady
#23:13 a k němu přídavnou oběť, dvě desetiny éfy bílé mouky zadělané olejem, jako ohnivou oběť Hospodinu v libou vůni. K tomu i úlitbu, čtvrtinu hínu vína.
#23:14 Až do toho dne, kdy přinesete dar svému Bohu, nebudete jíst chléb ani zrní pražené ani čerstvé. To je provždy platné nařízení pro všechna vaše pokolení ve všech vašich sídlištích.
#23:15 Potom si odpočítáte ode dne po dni odpočinku, ode dne, kdy jste přinesli snopek k oběti podávání, plných sedm týdnů.
#23:16 Ke dni po sedmém dni odpočinku napočítáte padesát dní a pak přinesete Hospodinu novou přídavnou oběť.
#23:17 Přinesete ze svých sídlišť chléb k oběti podávání; budou to dva chleby ze dvou desetin éfy bílé mouky, budou upečeny kvašené, jakožto prvotiny Hospodinu.
#23:18 S chlebem přivedete sedm ročních beránků bez vady, jednoho mladého býčka a dva berany. To bude zápalná oběť Hospodinu s příslušnými přídavnými oběťmi a úlitbami, ohnivá oběť, libá vůně pro Hospodina.
#23:19 Také připravíte jednoho kozla k oběti za hřích a dva roční beránky k hodu oběti pokojné.
#23:20 Kněz to nabídne Hospodinu podáváním i s chlebem z prvotin jako oběť podávání, včetně obou beránků. Bude to svaté Hospodinu, určené pro kněze.
#23:21 V týž den svoláte lid. Budete mít bohoslužebné shromáždění; nebudete konat žádnou všední práci. To je provždy platné nařízení pro všechna vaše pokolení ve všech vašich sídlištích.
#23:22 Až budete sklízet obilí ve své zemi, nepožneš své pole až do okraje ani nebudeš po žni paběrkovat; zanecháš paběrky pro zchudlého a pro hosta. Já jsem Hospodin, váš Bůh.“
#23:23 Hospodin dále mluvil k Mojžíšovi:
#23:24 „Mluv k Izraelcům: Prvního dne sedmého měsíce budete mít slavnost odpočinutí s pamětným vytrubováním a bohoslužebným shromážděním.
#23:25 Nebudete vykonávat žádnou všední práci, ale přinesete Hospodinu ohnivou oběť.“
#23:26 Hospodin dále mluvil k Mojžíšovi:
#23:27 „Desátého dne téhož sedmého měsíce bude den smíření. Budete mít bohoslužebné shromáždění; budete se pokořovat a přinesete ohnivou oběť Hospodinu.
#23:28 Toho dne nebudete konat žádnou práci. Je to den smíření, kdy se za vás budou konat smírčí obřady před Hospodinem, vaším Bohem.
#23:29 Kdo se toho dne nebude pokořovat, bude vyobcován ze svého lidu.
#23:30 Kdyby někdo konal toho dne nějakou práci, vyhubím ho ze společenství jeho lidu.
#23:31 Nebudete vykonávat žádnou práci! To je provždy platné nařízení pro všechna vaše pokolení ve všech vašich sídlištích.
#23:32 Bude to pro vás den odpočinku, slavnost odpočinutí; budete se pokořovat od večera devátého dne toho měsíce, od jednoho večera do druhého budete zachovávat svůj den odpočinku.“
#23:33 Hospodin dále mluvil k Mojžíšovi:
#23:34 „Mluv k Izraelcům: Od patnáctého dne téhož sedmého měsíce budou po sedm dní Hospodinovy slavnosti stánků.
#23:35 Prvního dne bude bohoslužebné shromáždění; nebudete konat žádnou všední práci.
#23:36 Po sedm dní budete přinášet Hospodinu ohnivou oběť, osmého dne budete mít bohoslužebné shromáždění a přinesete ohnivou oběť Hospodinu. To je slavnostní shromáždění, nebudete konat žádnou všední práci.
#23:37 To jsou Hospodinovy slavnosti, které budete svolávat jako bohoslužebná shromáždění, abyste přinášeli Hospodinu ohnivou oběť, oběť zápalnou a přídavnou, obětní hod a úlitby, všechno v příslušném dni,
#23:38 mimo oběti o Hospodinových dnech odpočinku a kromě vašich darů při různých slibech a dobrovolných závazcích, jež budete dávat Hospodinu.
#23:39 Nuže, patnáctého dne sedmého měsíce, když sklidíte zemskou úrodu, budete po sedm dnů slavit slavnost Hospodinovu. Prvního dne bude slavnost odpočinutí i osmého dne bude slavnost odpočinutí.
#23:40 Prvého dne si vezmete plody ušlechtilých stromů, palmové ratolesti, větve myrtových keřů a potočních topolů a sedm dní se budete radovat před Hospodinem, svým Bohem.
#23:41 Tuto slavnost budete Hospodinu slavit po sedm dní v roce. To je provždy platné nařízení pro všechna vaše pokolení. Budete ji slavit v sedmém měsíci.
#23:42 Sedm dní budete bydlet ve stáncích; všichni domorodci v Izraeli budou bydlet ve stáncích,
#23:43 aby všechna vaše pokolení věděla, že jsem přechovával Izraelce ve stáncích, když jsem je vyvedl z egyptské země. Já jsem Hospodin, váš Bůh.“
#23:44 Mojžíš vyhlásil Izraelcům Hospodinovy slavnosti. 
#24:1 Hospodin promluvil k Mojžíšovi:
#24:2 „Přikaž Izraelcům, ať ti přinesou pročištěný olivový olej, vymačkaný, k svícení, aby mohl být každodenně zapalován kahan.
#24:3 Vně, před oponou svědectví, ve stanu setkávání jej bude Áron každodenně obsluhovat od večera do rána před Hospodinem. To je provždy platné nařízení pro všechna vaše pokolení.
#24:4 Na svícnu z čistého zlata bude každodenně před Hospodinem obsluhovat kahánky.
#24:5 Vezmeš bílou mouku a upečeš z ní dvanáct bochníků: každý bochník bude ze dvou desetin éfy.
#24:6 Položíš je před Hospodinem na stůl z čistého zlata do dvou sloupců, šest do sloupce.
#24:7 Na každý sloupec dáš čisté kadidlo; bude při chlebu na připomínku jako ohnivá oběť pro Hospodina.
#24:8 Ať je Áron před Hospodinem uspořádá vždycky v den odpočinku, a to ustavičně. To bude mezi Izraelci platit jako smlouva věčná.
#24:9 Budou patřit Áronovi a jeho synům, kteří je budou jíst na svatém místě; je to pro něho velesvaté, podíl z ohnivých obětí Hospodinových. To je provždy platné nařízení.“
#24:10 Jednou si vyšel syn jedné izraelské ženy a jakéhosi Egypťana mezi Izraelce; v táboře se dostal syn té Izraelky do hádky s nějakým Izraelcem.
#24:11 Syn izraelské ženy lál Jménu a zlořečil. Přivedli ho tedy k Mojžíšovi; jméno jeho matky bylo Šelomít, dcera Dibrího, z pokolení Danova.
#24:12 I dali ho do vazby, až by se jim dostalo jasného pokynu z Hospodinových úst.
#24:13 Hospodin promluvil k Mojžíšovi:
#24:14 „Vyveď toho zlolajce ven z tábora; tam všichni, kdo ho slyšeli, vloží ruce na jeho hlavu a celá pospolitost ho ukamenuje.
#24:15 Izraelcům pak řekneš: Kdokoli by zlořečil svému Bohu, ponese následky svého hříchu.
#24:16 Kdo bude lát jménu Hospodinovu, musí zemřít. Celá pospolitost ho ukamenuje. Jak host, tak domorodec zemře, jestliže bude lát Jménu.
#24:17 Když někdo ubije nějakého člověka, musí zemřít.
#24:18 Kdo ubije nějaké dobytče, nahradí je, kus za kus.
#24:19 Když někdo zmrzačí svého bližního, bude mu způsobeno, co sám způsobil:
#24:20 zlomenina za zlomeninu, oko za oko, zub za zub; jak zmrzačil člověka, tak ať se stane jemu.
#24:21 Kdo ubije dobytče, nahradí je; kdo ubije člověka, zemře.
#24:22 Budete mít jediné právo jak pro hosta, tak pro domorodce. Já jsem Hospodin, váš Bůh.“
#24:23 Tak promluvil Mojžíš k Izraelcům. Zlolajce vyvedli ven z tábora a ukamenovali. Izraelci učinili, jak Hospodin Mojžíšovi přikázal. 
#25:1 Hospodin promluvil k Mojžíšovi na hoře Sínaji:
#25:2 „Mluv k Izraelcům a řekni jim: Až přijdete do země, kterou vám dávám, bude země slavit odpočinutí, odpočinutí Hospodinovo.
#25:3 Šest let budeš osívat své pole, šest let budeš prořezávat svou vinici a shromažďovat z ní úrodu,
#25:4 ale sedmého roku bude mít země rok odpočinutí, slavnost odpočinutí, odpočinutí Hospodinovo. Nebudeš osívat své pole ani prořezávat svou vinici.
#25:5 Co po tvé žni samo vyroste, nebudeš sklízet, a hrozny z vinice, kterou jsi neobdělal, nebudeš sbírat. Země bude mít rok odpočinutí.
#25:6 Co země v odpočinutí sama zplodí, bude vaší potravou, pro tebe, tvého otroka i otrokyni, tvého nádeníka i přistěhovalce, kteří u tebe pobývají jako hosté.
#25:7 I tvůj dobytek a zvěř, která je v tvé zemi, bude mít všechnu její úrodu za potravu.
#25:8 Odpočítáš si pak sedm roků odpočinutí, sedmkrát sedm let, a vyjde ti období sedmi roků odpočinutí: čtyřicet devět let.
#25:9 Desátého dne sedmého měsíce dáš ryčně troubit na polnici; v den smíření budete troubit na polnici v celé vaší zemi.
#25:10 Padesátý rok posvětíte a vyhlásíte v zemi svobodu všem jejím obyvatelům. Bude to pro vás léto milostivé, kdy se každý vrátíte ke svému vlastnictví a všichni se vrátí ke své čeledi.
#25:11 Padesátý rok vám bude létem milostivým. Nebudete v něm sít ani sklízet, co samo vyroste, ani sbírat hrozny z neobdělaných vinic.
#25:12 Je to léto milostivé. Budete je mít za svaté. Smíte jíst z pole, co urodí.
#25:13 V tomto milostivém létě se každý vrátí ke svému vlastnictví.
#25:14 Když budete prodávat něco svému bližnímu nebo něco od něho budeš kupovat, nepoškozujte jeden druhého.
#25:15 Podle počtu let po létě milostivém budeš kupovat od svého bližního, podle počtu let, v nichž budeš brát užitek, bude prodávat tobě.
#25:16 Čím více let zbývá, tím vyšší bude kupní cena, a čím méně let zbývá, tím nižší bude kupní cena; bude ti prodávat podle počtu let, v nichž budeš brát užitek.
#25:17 Nikdo nepoškodíte svého bližního, ale budeš se bát svého Boha. Já jsem Hospodin, váš Bůh.
#25:18 Budete dodržovat má nařízení, dbát na moje řády a podle nich jednat. Tak budete bezpečně přebývat na zemi.
#25:19 Země vám vydá své plody a budete jíst dosyta a přebývat na ní bezpečně.
#25:20 Snad řeknete: Co budeme jíst sedmého roku, když nebudeme sít ani sklízet žádnou úrodu?
#25:21 Na můj rozkaz sestoupí na vás šestého roku mé požehnání, takže vydá úrodu na tři roky.
#25:22 Osmého roku budete sít, ale jíst budete z předloňské úrody až do devátého roku; dokud se nesveze jeho úroda, budete jíst ze staré.
#25:23 Země nesmí být prodávána bez práva na zpětnou koupi, neboť země je má. Vy jste u mne jen hosté a přistěhovalci.
#25:24 Proto po celé zemi, jež bude vaším vlastnictvím, zajistíte možnost zemi vyplatit.
#25:25 Když tvůj bratr zchudne a odprodá něco ze svého vlastnictví, přijde k němu jeho příbuzný jako zastánce a vyplatí, co jeho bratr prodal.
#25:26 Kdyby někdo neměl zastánce, ale byl by potom sám schopen opatřit si potřebné výplatné,
#25:27 sečte léta od doby prodeje, proplatí zbytek tomu, komu prodal, a vrátí se ke svému vlastnictví.
#25:28 Jestliže není schopen zaplatit, zůstane, co prodal, v držbě kupce až do léta milostivého. V létě milostivém bude vyvázáno z držby a on se vrátí ke svému vlastnictví.
#25:29 Když někdo prodá obytný dům v hrazeném městě, má právo jej vyplatit do roka od prodeje; rok trvá právo k vyplacení.
#25:30 Jestliže nebude vyplacen do jednoho roku, zůstane dům v hrazeném městě tomu, kdo jej koupil, i jeho pokolením, bez práva na zpětnou koupi; v létě milostivém nebude vyvázán z držby.
#25:31 Domy ve vsích neobehnaných hradbami budou posuzovány jako pole v zemi; bude na nich právo k vyplacení a v létě milostivém budou vyvázány z držby.
#25:32 Pokud jde o lévijská města, budou mít lévijci trvale právo vyplácet městské domy, jež jsou jejich vlastnictvím.
#25:33 Někdo z lévijců buď použije práva k vyplacení, nebo bude prodaný dům v městě, které je v jejich vlastnictví, vyvázán z držby v létě milostivém, neboť domy v lévijských městech jsou jejich vlastnictvím uprostřed Izraelců.
#25:34 Předměstské pole u jejich měst nesmí být prodáno, neboť to je jejich trvalé vlastnictví.
#25:35 Když tvůj bratr zchudne a nebude moci vedle tebe obstát, ujmeš se ho jako hosta a přistěhovalce a bude žít s tebou.
#25:36 Nebudeš od něho brát lichvářský úrok, ale budeš se bát svého Boha. Tvůj bratr bude žít s tebou.
#25:37 Své stříbro mu nepůjčuj lichvářsky, na poskytované potravě nechtěj vydělávat.
#25:38 Já jsem Hospodin, váš Bůh, já jsem vás vyvedl z egyptské země, abych vám dal kenaanskou zemi a byl vám Bohem.
#25:39 Když tvůj bratr vedle tebe zchudne a prodá ti sebe sama, nezotročíš ho otrockou službou.
#25:40 Bude u tebe jako nádeník, jako přistěhovalec; bude u tebe sloužit až do milostivého léta.
#25:41 Pak od tebe odejde a s ním i jeho děti a vrátí se ke své čeledi; vrátí se k vlastnictví svých otců.
#25:42 Jsou to přece moji služebníci, které jsem vyvedl z egyptské země, nesmějí být prodáváni jako otroci.
#25:43 Nebudeš nad ním surově panovat, ale budeš se bát svého Boha.
#25:44 Otrok a otrokyně, které budeš mít, ať jsou z pronárodů okolo vás; z nich si budete kupovat otroka a otrokyni.
#25:45 Také si je můžete koupit z dětí přistěhovalců, kteří u vás pobývají jako hosté, a z čeledi těch, kteří jsou u vás, z dětí, které zplodí ve vaší zemi; ti budou vaším vlastnictvím.
#25:46 Po sobě je odkážete svým synům, aby je zdědili jako své vlastnictví; trvale jim budou sloužit. Pokud jde o vaše bratry Izraelce, nikdo nebude nad svým bratrem surově panovat.
#25:47 Když u tebe nějaký host nebo přistěhovalec dosáhne blahobytu a tvůj bratr u něho zchudne, takže sebe sama prodá hostu, který se k tobě přistěhoval, nebo některému příslušníku hostovy čeledi,
#25:48 ač se prodal, bude mít právo se vykoupit. Někdo z jeho bratrů dá za něj výkupné
#25:49 nebo ho vykoupí jeho strýc nebo bratranec anebo ho vykoupí někdo z blízkých příbuzných jeho čeledi anebo si opatří potřebný obnos a vykoupí se sám.
#25:50 Provede vyúčtování se svým kupcem od roku, kdy se mu prodal, až do milostivého léta. Stříbro, za které se prodal, přepočte se podle počtu let, jako by po ty dny byl u něho nádeníkem.
#25:51 Jestliže zůstává ještě mnoho let, vrátí podle nich ze stříbra kupní ceny odpovídající část výkupného.
#25:52 Jestliže zbývá do milostivého léta málo roků, vyúčtuje s ním a odevzdá své výkupné podle příslušných let.
#25:53 Z roku na rok bude u něho jako nádeník a on nebude nad ním před tebou surově panovat.
#25:54 Jestliže nebude tak či onak vykoupen, bude vyvázán z držby v milostivém létě i se svými dětmi.
#25:55 Izraelci mi totiž patří jako služebníci, jsou to služebníci moji, které jsem vyvedl z egyptské země. Já jsem Hospodin, váš Bůh. 
#26:1 Ve své zemi si nebudete dělat bůžky, ani si nevztyčíte tesanou sochu nebo posvátný sloup, ani nepostavíte přitesaný kámen, jimž byste se klaněli. Já jsem Hospodin, váš Bůh.
#26:2 Dbejte na mé dny odpočinku a mějte v úctě mou svatyni. Já jsem Hospodin.
#26:3 Jestliže se budete řídit mými nařízeními, dbát na má přikázání a plnit je,
#26:4 dám vám ve vhodném čase vydatné deště, země vydá svůj výnos a stromoví na poli ponese ovoce.
#26:5 Výmlat bude trvat až do vinobraní a vinobraní zas až do setby. Najíte se svého pokrmu dosyta a budete sídlit ve své zemi bezpečně.
#26:6 Zemi daruji pokoj a nikdo vás nebude děsit, když budete spát. Učiním také přítrž řádění zlé zvěře v zemi a vaší zemí nebude procházet meč.
#26:7 Budete pronásledovat své nepřátele, takže padnou před vámi mečem.
#26:8 Pět z vás jich bude pronásledovat sto a sto z vás jich bude pronásledovat deset tisíc. Tak padnou vaši nepřátelé před vámi mečem.
#26:9 Obrátím se k vám a způsobím, že se rozplodíte a rozmnožíte. Svou smlouvu s vámi upevním.
#26:10 Budete jíst starou úrodu z loňska, dokud loňskou neodstraníte kvůli nové.
#26:11 Zřídím si mezi vámi příbytek a nezprotivíte se mi.
#26:12 Budu procházet mezi vámi a budu vám Bohem a vy budete mým lidem.
#26:13 Já jsem Hospodin, váš Bůh, já jsem vás vyvedl z egyptské země, abyste tam nebyli otroky. Rozlámal jsem břevna vašeho jha, abyste mohli chodit zpříma.
#26:14 Jestliže mě nebudete poslouchat a nebudete plnit všechny tyto příkazy,
#26:15 jestliže zavrhnete má nařízení a zprotivíte si mé řády, takže nebudete plnit všechny mé příkazy, ale budete porušovat mou smlouvu,
#26:16 pak já vám učiním toto: Navštívím vás hrůzou, úbytěmi a zimnicí, pohasnou vám oči a budete plni zoufalství. Nadarmo budete sít obilí, snědí je vaši nepřátelé.
#26:17 Postavím se proti vám a budete poraženi svými nepřáteli, a budou nad vámi panovat ti, kdo vás nenávidí, budete utíkat, i když vás nikdo nebude pronásledovat.
#26:18 Jestliže mě ani potom nebudete poslouchat, budu vás trestat za vaše hříchy ještě sedmkrát víc.
#26:19 Roztříštím vaši pyšnou moc a učiním, že nebe vám bude jako železo a země jako měď.
#26:20 Nadarmo bude spotřebována vaše síla. Země vám nedá svůj výnos a stromoví v zemi neponese ovoce.
#26:21 Jestliže mi i pak budete odporovat a nebudete mě chtít poslouchat, přidám vám sedmkrát víc ran pro vaše hříchy.
#26:22 Pustím na vás polní zvěř. Ta vás připraví o děti, vyhubí váš dobytek a ztenčí vaše řady; vaše cesty zpustnou.
#26:23 Jestliže se tím nedáte ode mne napomenout a budete mi dále odporovat,
#26:24 i já se vám budu stavět na odpor a budu vás bít sedmkrát víc za vaše hříchy.
#26:25 Uvedu na vás meč pomsty, aby pomstil porušování smlouvy. Shromáždíte-li se do svých měst, pošlu mezi vás mor a budete vydáni do rukou nepřítele.
#26:26 Až vám zlomím hůl chleba, bude deset žen pro vás péci chléb v jediné peci a budou vám odvažovat chléb na příděl. Budete jíst, ale nenasytíte se.
#26:27 Jestliže mě ani přesto neuposlechnete a budete mi dále odporovat,
#26:28 i já se vám budu stavět na odpor v rozhořčení a budu vás trestat sedmkrát víc za vaše hříchy.
#26:29 Budete jíst maso svých synů a budete jíst maso svých dcer.
#26:30 Zahladím vaše posvátná návrší a roztříštím vaše kadidlové oltáříky, vaše mrtvoly pohodím na mršiny vašich hnusných model a zprotivíte se mi.
#26:31 Obrátím vaše města v trosky, vaše svatyně zpustoším a nepřivoním k libé vůni vašich obětí.
#26:32 Sám zpustoším zemi tak, že nad ní strnou i vaši nepřátelé, kteří se v ní usadí.
#26:33 Vás pak rozptýlím mezi pronárody, vpadnu vám s taseným mečem do zad, takže se vaše země stane pustinou a vaše města se promění v trosky.
#26:34 Tehdy si země vynahradí své roky odpočinutí po celou dobu, co bude zpustošena, až vy budete v zemi svých nepřátel. Tehdy bude země odpočívat a vynahradí si své roky odpočinutí.
#26:35 Po celou dobu zpustošení bude odpočívat, protože nemohla odpočívat ve vašich letech odpočinutí, když jste v ní sídlili.
#26:36 Naplním ustrašeností srdce těch, kdo z vás zůstanou v zemích svých nepřátel, že je zažene i šelest odvátého listí. Budou utíkat jako před mečem a padat, i když je nikdo nebude pronásledovat.
#26:37 Budou klopýtat jeden přes druhého jako před mečem, ačkoli je nikdo nebude pronásledovat. Neobstojíte před svými nepřáteli.
#26:38 Zahynete mezi pronárody a pohltí vás země vašich nepřátel.
#26:39 Kteří z vás zůstanou, zaniknou pro svou nepravost v zemích svých nepřátel, a také pro nepravosti svých otců zaniknou.
#26:40 Pak budou vyznávat nepravost svou i nepravost svých otců, zpronevěru, jíž se na mně dopustili, a že mi odporovali.
#26:41 Proto i já jsem se jim postavil na odpor a zavedl jsem je do země jejich nepřátel, dokud nebude jejich neobřezané srdce zkrušeno a dokud neodpykají své provinění.
#26:42 I připomenu si svou smlouvu s Jákobem a též svou smlouvu s Izákem, také si připomenu svou smlouvu s Abrahamem, i tu zemi si připomenu.
#26:43 Země totiž bude od nich opuštěna a vynahradí si své roky odpočinutí, bude zpustošena, bez nich, dokud oni neodpykají své provinění, protože znovu a znovu zavrhovali mé řády a má nařízení si protivili.
#26:44 Avšak i když budou v zemi svých nepřátel, nezavrhnu je a nezprotivím si je, jako bych s nimi měl skoncovat a zrušit svou smlouvu s nimi. Já jsem Hospodin, jejich Bůh.
#26:45 Připomenu jim smlouvu s jejich předky, které jsem vyvedl z egyptské země před zraky pronárodů, abych jim byl Bohem, já Hospodin.“
#26:46 Toto jsou nařízení, řády a zákony, jež Hospodin vydal na hoře Sínaji prostřednictvím Mojžíše, aby byly mezi ním a syny Izraele. 
#27:1 Hospodin promluvil k Mojžíšovi:
#27:2 „Mluv k Izraelcům a řekni jim: Když někdo složí zvláštní slib, osoby zasvěcené Hospodinu budou vyplaceny podle tvého odhadu.
#27:3 Tvůj odhadní obnos u muže od dvaceti do šedesáti let bude činit padesát šekelů stříbra podle váhy určené svatyní.
#27:4 Půjde-li o ženu, bude obnos činit třicet šekelů.
#27:5 Půjde-li o osobu od pěti do dvaceti let, bude obnos činit při osobě mužského pohlaví dvacet šekelů a při ženském deset šekelů.
#27:6 Půjde-li o dítě od jednoho měsíce do pěti let, bude obnos činit při dítěti mužského pohlaví pět šekelů stříbra a při ženském tři šekely stříbra.
#27:7 Půjde-li o osobu šedesátiletou a starší, bude obnos činit při osobě mužského pohlaví patnáct šekelů a při ženském deset šekelů.
#27:8 Jestliže někdo nebude s to dát příslušný obnos, bude postaven před kněze a kněz mu odhadne obnos; kněz odhadne obnos podle platební schopnosti slibujícího.
#27:9 Půjde-li o dobytek, z něhož se přináší dar Hospodinu, každý kus daný Hospodinu bude svatý.
#27:10 Nevymění jej a nezamění dobrý za špatný ani špatný za dobrý. Jestliže pak přece zamění dobytče za jiné, bude svaté to i ono.
#27:11 Půjde-li o jakékoli nečisté dobytče, z něhož se dar Hospodinu nepřináší, postaví dobytče před kněze.
#27:12 Kněz je odhadne, je-li dobré nebo špatné, a bude podle knězova odhadu.
#27:13 Jestliže však je chce někdo vyplatit, přidá k tvému odhadnímu obnosu navíc jednu pětinu.
#27:14 Když někdo zasvětí svůj dům jako svatý Hospodinu, odhadne jej kněz, je-li dobrý nebo špatný; zůstane při odhadu, jak jej provedl kněz.
#27:15 Jestliže však ten, kdo svůj dům zasvětil, jej chce vyplatit, přidá k tvému odhadnímu obnosu pětinu stříbra navíc a bude jeho.
#27:16 Oddělí-li někdo kus pole ze svého vlastnictví jako svatý pro Hospodina, odhadneš jej podle osevu; plocha osetá z jednoho chómeru ječmene bude mít cenu padesáti šekelů stříbra.
#27:17 Oddělí-li své pole jako svaté již od milostivého léta, zůstane při tvém odhadním obnosu.
#27:18 Oddělí-li své pole jako svaté po milostivém létě, vypočítá mu kněz cenu podle počtu let zbývajících do dalšího milostivého léta, a ubere se z tvého odhadního obnosu.
#27:19 Jestliže však ten, kdo oddělil pole jako svaté, je chce vyplatit, přidá k tvému odhadnímu obnosu pětinu stříbra navíc a bude patřit jemu.
#27:20 Jestliže pole nevyplatí, ale prodá je někomu jinému, nemůže být již vyplaceno.
#27:21 Takové pole při vyvázání z držby v létě milostivém bude svaté Hospodinu jako pole propadlé klatbě; bude náležet knězi jako jeho vlastnictví.
#27:22 Jestliže někdo oddělí jako svaté pro Hospodina pole, jež získal koupí, a nebylo částí jeho vlastnictví,
#27:23 vypočítá mu kněz výši obnosu až do milostivého léta a on vyplatí obnos téhož dne jako svatý Hospodinu.
#27:24 V milostivém létě bude pole vráceno tomu, od koho bylo koupeno, jehož vlastnictvím pozemek je.
#27:25 Veškeré tvé odhadní obnosy budou v šekelech podle váhy určené svatyní. Šekel je dvacet zrn.
#27:26 Jenom ať nikdo neodděluje jako svaté prvorozené dobytče; náleží beztak Hospodinu jakožto prvorozené; jak býk, tak jehně patří Hospodinu.
#27:27 Jestliže jde o dobytče nečisté, vyplatí je podle tvého odhadu a přidá pětinu obnosu navíc; nebude-li vyplaceno, bude prodáno za obnos tebou stanovený.
#27:28 Jenom nic klatého, co někdo jako takové oddal Hospodinu z čehokoli, co má, z lidí, z dobytka i z polností, jež má ve vlastnictví, nesmí být prodáváno ani vypláceno. Všechno klaté je velesvaté a náleží Hospodinu.
#27:29 Nikdo z lidí, kdo je klatbou oddán Bohu jako klatý, nemůže být vyplacen; musí zemřít.
#27:30 Všechny desátky země z obilí země a z ovoce stromů budou Hospodinovy; jsou svaté Hospodinu.
#27:31 Jestliže si však někdo přeje vyplatit něco ze svého desátku, přidá pětinu obnosu navíc.
#27:32 Každý desátek ze skotu a bravu, každý desátý kus, který při počítání prochází pod holí, bude svatý Hospodinu.
#27:33 Nebude se prohlížet, je-li dobrý či špatný, a nezamění se za jiný; jestliže jej pak přece zamění, bude ten i onen svatý. Nesmí být vyplacen.“
#27:34 Toto jsou příkazy, které vydal Hospodin Mojžíšovi pro syny Izraele na hoře Sínaji.  

\book{Numbers}{Num}
#1:1 Hospodin promluvil k Mojžíšovi na Sínajské poušti ve stanu setkávání prvního dne druhého měsíce ve druhém roce po jejich vyjití z egyptské země.
#1:2 „Pořiďte soupis celé pospolitosti Izraelců podle čeledí otcovských rodů. Ve jmenném seznamu bude každý jednotlivec mužského pohlaví
#1:3 od dvacetiletých výše, každý, kdo je v Izraeli schopen vycházet do boje. Ty a Áron je spočítáte po oddílech.
#1:4 Za každé pokolení bude s vámi jeden muž, vždy představitel otcovského rodu.
#1:5 Toto jsou jména mužů, kteří budou stát při vás: za Rúbenův kmen Elísúr, syn Šedeúrův,
#1:6 za Šimeónův Šelumíel, syn Suríšadajův,
#1:7 za Judův Nachšón, syn Amínadabův,
#1:8 za Isacharův Netaneel, syn Súarův,
#1:9 za Zabulónův Elíab, syn Chelónův,
#1:10 za Josefovce, za Efrajimův kmen Elíšama, syn Amíhudův, za Manasesův Gamlíel, syn Pedásurův,
#1:11 za Benjamínův Abídan, syn Gideóního,
#1:12 za Danův Achíezer, syn Amíšadajův,
#1:13 za Ašerův Pagíel, syn Okranův,
#1:14 za Gádův Eljásaf, syn Deúelův,
#1:15 za Neftalíův Achíra, syn Énanův.“
#1:16 To byli zástupci pospolitosti, předáci otcovských pokolení; ti byli veliteli izraelských šiků.
#1:17 Mojžíš a Áron přibrali tyto muže uvedené jménem.
#1:18 Prvního dne druhého měsíce svolali celou pospolitost, aby všichni hlásili svůj původ podle čeledí otcovských rodů; do jmenného seznamu byl zapsán každý jednotlivec od dvacetiletých výše,
#1:19 jak přikázal Hospodin Mojžíšovi. Spočítal je na Sínajské poušti.
#1:20 Synové Rúbena, Izraelova prvorozeného, byli uvedeni ve svých rodopisech podle čeledí otcovských rodů; do jmenného seznamu byl zapsán každý jednotlivec mužského pohlaví od dvacetiletých výše, každý, kdo byl schopen vycházet do boje.
#1:21 Z Rúbenova pokolení bylo povolaných do služby čtyřicet šest tisíc pět set.
#1:22 Šimeónovci ve svých rodopisech podle čeledí otcovských rodů; z nich povolaní do služby byli zapsáni do jmenného seznamu, každý jednotlivec mužského pohlaví od dvacetiletých výše, každý, kdo byl schopen vycházet do boje.
#1:23 Z Šimeónova pokolení bylo povolaných do služby padesát devět tisíc tři sta.
#1:24 Gádovci ve svých rodopisech podle čeledí otcovských rodů; do jmenného seznamu byl zapsán od dvacetiletých výše každý, kdo byl schopen vycházet do boje.
#1:25 Z Gádova pokolení bylo povolaných do služby čtyřicet pět tisíc šest set padesát.
#1:26 Judovci ve svých rodopisech podle čeledí otcovských rodů; do jmenného seznamu byl zapsán od dvacetiletých výše každý, kdo byl schopen vycházet do boje.
#1:27 Z Judova pokolení bylo povolaných do služby sedmdesát čtyři tisíce šest set.
#1:28 Isacharovci ve svých rodopisech podle čeledí otcovských rodů; do jmenného seznamu byl zapsán od dvacetiletých výše každý, kdo byl schopen vycházet do boje.
#1:29 Z Isacharova pokolení bylo povolaných do služby padesát čtyři tisíce čtyři sta.
#1:30 Zabulónovci ve svých rodopisech podle čeledí otcovských rodů; do jmenného seznamu byl zapsán od dvacetiletých výše každý, kdo byl schopen vycházet do boje.
#1:31 Ze Zabulónova pokolení bylo povolaných do služby padesát sedm tisíc čtyři sta.
#1:32 Josefovci: Efrajimovci ve svých rodopisech podle čeledí otcovských rodů; do jmenného seznamu byl zapsán od dvacetiletých výše každý, kdo byl schopen vycházet do boje.
#1:33 Z Efrajimova pokolení bylo povolaných do služby čtyřicet tisíc pět set.
#1:34 Manasesovci ve svých rodopisech podle čeledí otcovských rodů; do jmenného seznamu byl zapsán od dvacetiletých výše každý, kdo byl schopen vycházet do boje.
#1:35 Z Manasesova pokolení bylo povolaných do služby třicet dva tisíce dvě stě.
#1:36 Benjamínovci ve svých rodopisech podle čeledí otcovských rodů; do jmenného seznamu byl zapsán od dvacetiletých výše každý, kdo byl schopen vycházet do boje.
#1:37 Z Benjamínova pokolení bylo povolaných do služby třicet pět tisíc čtyři sta.
#1:38 Danovci ve svých rodopisech podle čeledí otcovských rodů; do jmenného seznamu byl zapsán od dvacetiletých výše každý, kdo byl schopen vycházet do boje.
#1:39 Z Danova pokolení bylo povolaných do služby šedesát dva tisíce sedm set.
#1:40 Ašerovci ve svých rodopisech podle čeledí otcovských rodů; do jmenného seznamu byl zapsán od dvacetiletých výše každý, kdo byl schopen vycházet do boje.
#1:41 Z Ašerova pokolení bylo povolaných do služby čtyřicet jeden tisíc pět set.
#1:42 Neftalíovci ve svých rodopisech podle čeledí otcovských rodů; do jmenného seznamu byl zapsán od dvacetiletých výše každý, kdo byl schopen vycházet do boje.
#1:43 Z Neftalíova pokolení bylo povolaných do služby padesát tři tisíce čtyři sta.
#1:44 To jsou povolaní do služby, které spočítali Mojžíš a Áron s izraelskými předáky; bylo jich dvanáct, po jednom z každého otcovského rodu.
#1:45 Všech Izraelců povolaných do služby podle otcovských rodů od dvacetiletých výše, všech, kdo v Izraeli byli schopni vycházet do boje,
#1:46 všech povolaných do služby bylo šest set tři tisíce pět set padesát.
#1:47 Avšak lévijci podle svého otcovského pokolení mezi ně započteni nebyli.
#1:48 Hospodin totiž promluvil k Mojžíšovi:
#1:49 „Pokolení Léviho mezi Izraelce nezapočítáš ani nepořídíš jejich soupis.
#1:50 Lévijce přidělíš k příbytku svědectví, ke všemu jeho nářadí i ke všemu, co k němu patří. Budou nosit příbytek a všechno jeho nářadí a budou u něho přisluhovat; budou tábořit kolem příbytku.
#1:51 Než příbytek potáhne dál, lévijci jej složí; při táboření jej lévijci postaví. Přiblíží-li se někdo nepovolaný, zemře.
#1:52 Izraelci budou tábořit každý ve svém táboře, každý při svém praporu po oddílech.
#1:53 Lévijci však budou tábořit kolem příbytku svědectví, aby na pospolitost Izraelců nedolehl hrozný hněv. Lévijci budou držet stráž u příbytku svědectví.“
#1:54 Izraelci učinili vše přesně tak, jak přikázal Hospodin Mojžíšovi. 
#2:1 Hospodin promluvil k Mojžíšovi a Áronovi:
#2:2 „Izraelci budou tábořit každý u svého praporu, pod znaky svého otcovského rodu; budou tábořit opodál kolem stanu setkávání.
#2:3 Vpředu, na východní straně, se utáboří po oddílech prapor tábora Judova. Předákem Judovců bude Nachšón, syn Amínadabův;
#2:4 v jeho voji bude sedmdesát čtyři tisíce šest set povolaných do služby.
#2:5 U něho bude tábořit pokolení Isachar. Předákem Isacharovců bude Netaneel, syn Súarův;
#2:6 v jeho voji bude padesát čtyři tisíce čtyři sta povolaných do služby.
#2:7 Dále pokolení Zabulón. Předákem Zabulónovců bude Elíab, syn Chelónův;
#2:8 v jeho voji bude padesát sedm tisíc čtyři sta povolaných do služby.
#2:9 Všech povolaných do služby z tábora Judova bude po oddílech sto osmdesát šest tisíc čtyři sta; ti potáhnou první.
#2:10 Prapor tábora Rúbenova po oddílech bude na jižní straně. Předákem Rúbenovců bude Elísur, syn Šedeúrův;
#2:11 v jeho voji bude čtyřicet šest tisíc pět set povolaných do služby.
#2:12 U něho bude tábořit pokolení Šimeón. Předákem Šimeónovců bude Šelumíel, syn Suríšadajův;
#2:13 v jeho voji bude padesát devět tisíc tři sta povolaných do služby.
#2:14 Dále pokolení Gád. Předákem Gádovců bude Eljasáf, syn Reúelův;
#2:15 v jeho voji bude čtyřicet pět tisíc šest set padesát povolaných do služby.
#2:16 Všech povolaných do služby z tábora Rúbenova bude po oddílech sto padesát jeden tisíc čtyři sta padesát; ti potáhnou druzí.
#2:17 Pak potáhne stan setkávání a tábor lévijců uprostřed ostatních táborů. Potáhnou tak, jak budou tábořit, každý na určeném místě, prapor za praporem.
#2:18 Prapor tábora Efrajimova po oddílech bude na západní straně. Předákem Efrajimovců bude Elíšama, syn Amíhudův;
#2:19 v jeho voji bude čtyřicet tisíc pět set povolaných do služby.
#2:20 U něho bude pokolení Manases. Předákem Manasesovců bude Gamlíel, syn Pedásurův;
#2:21 v jeho voji bude třicet dva tisíce dvě stě povolaných do služby.
#2:22 Dále pokolení Benjamín. Předákem Benjamínovců bude Abídan, syn Gideóního;
#2:23 v jeho voji bude třicet pět tisíc čtyři sta povolaných do služby.
#2:24 Všech povolaných do služby z tábora Efrajimova bude po oddílech sto osm tisíc jedno sto; ti potáhnou třetí.
#2:25 Prapor tábora Danova po oddílech bude na straně severní. Předákem Danovců bude Achíezer, syn Amíšadajův;
#2:26 v jeho voji bude šedesát dva tisíce sedm set povolaných do služby.
#2:27 U něho bude tábořit pokolení Ašer. Předákem Ašerovců bude Pagíel, syn Okranův;
#2:28 v jeho voji bude čtyřicet jeden tisíc pět set povolaných do služby.
#2:29 Dále pokolení Neftalí. Předákem Neftalíovců bude Achíra, syn Énanův;
#2:30 v jeho voji bude padesát tři tisíce čtyři sta povolaných do služby.
#2:31 Všech povolaných do služby z tábora Danova bude sto padesát sedm tisíc šest set; ti potáhnou poslední, prapor za praporem.“
#2:32 To jsou Izraelci povolaní do služby podle otcovských rodů. Všech povolaných do služby v táborech po oddílech bylo šest set tři tisíce pět set padesát.
#2:33 Avšak lévijci nebyli mezi Izraelce započteni, jak přikázal Hospodin Mojžíšovi. Tak tábořili prapor za praporem, tak i táhli dál, každý se svou čeledí, u svého otcovského rodu. 
#2:34 
#3:1 Toto je rodopis Áronův - a Mojžíšův - v den, kdy Hospodin promluvil k Mojžíšovi na hoře Sínaji.
#3:2 Toto jsou jména Áronových synů: Prvorozený Nádab, pak Abíhú, Eleazar a Ítamar.
#3:3 To jsou jména Áronových synů, pomazaných kněží, které Mojžíš uvedl v úřad, aby sloužili jako kněží.
#3:4 Nádab a Abíhú však zemřeli před Hospodinem, když na Sínajské poušti přinesli před Hospodina cizí oheň. Syny neměli. Proto vedle svého otce Árona sloužili jako kněží Eleazar a Ítamar.
#3:5 Hospodin promluvil k Mojžíšovi:
#3:6 „Přiveď pokolení Léviho a postav je před kněze Árona, aby mu přisluhovali.
#3:7 Budou držet stráž před stanem setkávání za něho i za celou pospolitost, aby se mohla konat služba při příbytku.
#3:8 Budou mít na starosti všechno nářadí stanu setkávání. Budou držet stráž za Izraelce, aby se mohla konat při příbytku služba.
#3:9 Lévijce dáš Áronovi a jeho synům; budou mu odděleni z Izraelců jako dar.
#3:10 Árona pak a jeho syny pověříš, aby střežili své kněžství. Jestliže se přiblíží někdo nepovolaný, zemře.“
#3:11 Hospodin dále mluvil k Mojžíšovi:
#3:12 „Hle, já jsem si vzal lévijce z Izraelců místo všech prvorozených mezi Izraelci, všech, kteří otvírají lůno. Lévijci jsou moji.
#3:13 Mně patří všechno prvorozené. V den, kdy jsem pobil všechno prvorozené v egyptské zemi, oddělil jsem pro sebe jako svaté všechno prvorozené v Izraeli, člověka i dobytče. Jsou moji. Já jsem Hospodin.“
#3:14 Hospodin dále mluvil k Mojžíšovi na Sínajské poušti:
#3:15 „Spočítej Léviovce podle čeledí otcovských rodů. Spočítáš všechny mužského pohlaví, od jednoměsíčních výše.“
#3:16 Mojžíš je na rozkaz Hospodinův spočítal, jak měl přikázáno.
#3:17 A toto jsou Léviovci uvedení svými jmény: Geršón, Kehat a Merarí.
#3:18 Toto jsou jména Geršónovců podle čeledí: Libní a Šimeí.
#3:19 A Kehatovci podle čeledí: Amrám a Jishár, Chebrón a Uzíel.
#3:20 A Meraríovci podle čeledí: Machlí a Muší. To jsou lévijské čeledi podle otcovských rodů.
#3:21 Ke Geršónovi patří čeleď libníjská a čeleď šimeíská. To jsou geršónské čeledi.
#3:22 Povolaných do služby bylo mezi nimi sedm tisíc pět set, což odpovídalo počtu všech mužského pohlaví od jednoměsíčních výše.
#3:23 Geršónské čeledi měly tábořit za příbytkem na západní straně.
#3:24 Předákem geršónského rodu byl Eljásaf, syn Láelův.
#3:25 Ve stanu setkávání měli Geršónovci na starosti příbytek, totiž stan, jeho přikrývku a závěs na vchodu do stanu setkávání,
#3:26 zástěny nádvoří i závěs na vchodu do nádvoří, které je kolem příbytku a oltáře, stanová lana a veškerou práci s tím spojenou.
#3:27 Ke Kehatovi patří čeleď amrámská a čeleď jishárská, čeleď chebrónská a čeleď uzíelská. To jsou kehatské čeledi.
#3:28 Bylo jich osm tisíc šest set, což odpovídalo počtu všech mužského pohlaví od jednoměsíčních výše; ti drželi stráž při svatyni.
#3:29 Čeledi Kehatovců měly tábořit podél příbytku na jižní straně.
#3:30 Předákem rodu kehatských čeledí byl Elísáfan, syn Uzíelův.
#3:31 Na starosti měli schránu, stůl, svícen, oltáře, nářadí svatyně potřebné k obsluze, závěs a veškerou práci s tím spojenou.
#3:32 Předním z předáků léviovských byl Eleazar, syn kněze Árona, pověřený dohledem nad těmi, kteří drželi stráž při svatyni.
#3:33 K Merarímu patří čeleď machlíjská a čeleď mušíjská. To jsou meraríjské čeledi.
#3:34 Povolaných do služby bylo mezi nimi šest tisíc dvě stě, což odpovídalo počtu všech mužského pohlaví od jednoměsíčních výše.
#3:35 Předákem rodu meraríjských čeledí byl Súríel, syn Abíchajilův; měly tábořit podél příbytku na severní straně.
#3:36 Meraríovci byli pověřeni starostí o desky příbytku a jeho svlaky, o sloupy, podstavce, všechno nářadí a veškerou práci s tím spojenou,
#3:37 též o sloupy kolem nádvoří, jejich podstavce, stanové kolíky a lana.
#3:38 Vpředu před příbytkem, před stanem setkávání, na východní straně budou tábořit Mojžíš, Áron a jeho synové; budou držet stráž při svatyni, stráž za Izraelce. Jestliže se přiblíží někdo nepovolaný, zemře.
#3:39 Všech lévijců povolaných do služby, které spočítal Mojžíš a Áron podle čeledí na rozkaz Hospodinův, všech mužského pohlaví od jednoměsíčních výše bylo dvacet dva tisíce.
#3:40 Hospodin řekl Mojžíšovi: „Spočítej všechny prvorozené Izraelce mužského pohlaví od jednoměsíčních výše a pořiď jejich jmenný seznam.
#3:41 Pak místo všech prvorozených mezi Izraelci vezmeš pro mne lévijce. Já jsem Hospodin! Též místo všech prvorozených izraelských dobytčat vezmeš dobytek lévijců.“
#3:42 Mojžíš spočítal všechny prvorozené mezi Izraelci, jak mu Hospodin přikázal.
#3:43 A bylo všech prvorozených mužského pohlaví od jednoměsíčních výše, kteří byli zapsáni do jmenného seznamu, povolaných do služby dvacet dva tisíce dvě stě sedmdesát tři.
#3:44 Hospodin dále mluvil k Mojžíšovi:
#3:45 „Místo všech prvorozených mezi Izraelci vezmi lévijce a místo jejich dobytka dobytek lévijců. Lévijci jsou moji. Já jsem Hospodin!
#3:46 Jako výplatné za dvě stě třiasedmdesát prvorozených Izraelců, kteří přesahují počet lévijců,
#3:47 vybereš po pěti šekelech stříbra na hlavu; vybereš je podle váhy určené svatyní, šekel je dvacet zrn.
#3:48 Stříbro předáš Áronovi a jeho synům jako výplatné za ty, kdo přesahují jejich počet.“
#3:49 Mojžíš vybral výplatné za ty, kdo přesahovali počet vyplacených lévijců.
#3:50 Vybral stříbro za prvorozené Izraelce, tisíc tři sta šedesát pět šekelů podle váhy určené svatyní.
#3:51 Výplatné pak předal Mojžíš na rozkaz Hospodinův Áronovi a jeho synům, jak Hospodin Mojžíšovi přikázal. 
#4:1 Hospodin promluvil k Mojžíšovi a Áronovi:
#4:2 „Pořiď soupis Kehatovců z řad Léviovců podle čeledí otcovských rodů
#4:3 od třicetiletých výše až po padesátileté, všech, kdo jsou schopni nastoupit do služby, totiž konat dílo při stanu setkávání.
#4:4 Toto bude služba Kehatovců při stanu setkávání: péče o nejsvětější předměty.
#4:5 Než tábor potáhne dál, přijde Áron se svými syny a sejmou oponu zavěšenou uvnitř. Tou přikryjí schránu svědectví;
#4:6 na to pak dají přehoz z tachaší kůže, rozprostřou navrch pokrývku celou purpurově fialovou a provléknou tyče.
#4:7 Na stůl pro předkladné chleby rozprostřou pokrývku purpurově fialovou a na ni dají mísy, pánvičky, obětní misky a konvice pro úlitbu; bude na něm i každodenní chléb.
#4:8 Na to vše rozprostřou karmínovou pokrývku, přikryjí přikrývkou z tachaší kůže a provléknou tyče.
#4:9 Pak vezmou purpurově fialovou pokrývku a přikryjí svícen k svícení i jeho kahánky, nůžky na knoty, pánve na oheň a všechny nádoby na olej, potřebné k jeho obsluze.
#4:10 Vloží jej se vším náčiním do přikrývky z tachaší kůže a dají na nosidla.
#4:11 Na zlatý oltář rozprostřou pokrývku purpurově fialovou, přikryjí jej přikrývkou z tachaší kůže a provléknou tyče.
#4:12 Pak vezmou všechno bohoslužebné náčiní potřebné k obsluze ve svatyni, dají na pokrývku purpurově fialovou, přikryjí je přikrývkou z tachaší kůže a dají na nosidla.
#4:13 Vyberou popel z oltáře a na oltář rozprostřou nachovou pokrývku;
#4:14 na ni dají všechno náčiní potřebné k jeho obsluze: pánve na oheň, vidlice, lopaty, kropenky a všechno ostatní oltářní náčiní, rozprostřou na ně přehoz z tachaší kůže a provléknou tyče.
#4:15 Tak Áron a jeho synové úplně přikryjí svaté předměty a všechno nářadí svatyně, než tábor potáhne dál; teprve potom přijdou Kehatovci, aby to nesli. Nedotknou se svatých předmětů, aby nezemřeli. To bude povinnost Kehatovců při stanu setkávání.
#4:16 Eleazar, syn kněze Árona, bude pověřen dohledem na olej k svícení, na kadidlo z vonných látek, každodenní přídavnou oběť a olej k pomazávání, bude pověřen dohledem na celý příbytek se vším, co je v něm, ve svatyni i v jejích nádobách.“
#4:17 Hospodin dále mluvil k Mojžíšovi a Áronovi:
#4:18 „Nesmíte dopustit, aby kmen kehatských čeledí z řad lévijců byl vyhlazen.
#4:19 Aby zůstali naživu a nezemřeli, udělejte pro ně toto: Než přistoupí k nejsvětějším předmětům, přijde Áron se svými syny a každému určí jeho pracovní povinnost.
#4:20 Nesmějí vejít, aby se dívali, jak jsou baleny svaté předměty, aby nezemřeli.“
#4:21 Hospodin dále mluvil k Mojžíšovi:
#4:22 „Pořiď také soupis Geršónovců podle čeledí otcovských rodů.
#4:23 Spočítáš je od třicetiletých výše až po padesátileté, všechny, kdo jsou schopni nastoupit do služby, totiž konat službu při stanu setkávání.
#4:24 Toto bude služba a pracovní povinnost geršónských čeledí:
#4:25 Budou nosit stanové plachty příbytku, totiž stan setkávání, jeho přikrývku i přikrývku z tachaší kůže, která je na něm navrchu, závěs ze vchodu do stanu setkávání,
#4:26 zástěny nádvoří i závěs ze vchodu brány do nádvoří, které je kolem příbytku a oltáře, stanová lana i všechno nářadí potřebné k práci. Budou konat všechno, co jim bude uloženo.
#4:27 Veškerá služba Geršónovců se bude konat na rozkaz Árona a jeho synů; týká se to veškeré jejich pracovní povinnosti. Dohlédnete na ně, aby svou povinnost plnili svědomitě.
#4:28 To bude služba geršónovských čeledí při stanu setkávání; budou ji bedlivě konat za dozoru Ítamara, syna kněze Árona.
#4:29 Meraríovce rovněž spočítáš podle čeledí otcovských rodů.
#4:30 Spočítáš je od třicetiletých výše až po padesátileté, všechny, kdo jsou schopni nastoupit do služby, totiž konat službu při stanu setkávání.
#4:31 Jejich povinností, na kterou budou dbát při veškeré své službě při stanu setkávání, je péče o desky příbytku a jeho svlaky, o sloupy a podstavce,
#4:32 též o sloupy kolem nádvoří, jejich podstavce, stanové kolíky a lana a o všechno nářadí pro veškerou práci s tím spojenou. Jmenovitě vypočítáte všechny předměty, o něž budou povinni dbát.
#4:33 To bude služba čeledí Meraríovců a veškerá jejich práce při stanu setkávání za dozoru Ítamara, syna kněze Árona.“
#4:34 Mojžíš s Áronem a předáky pospolitosti spočítal Kehatovce podle čeledí otcovských rodů,
#4:35 od třicetiletých výše až po padesátileté, všechny, kdo byli schopni nastoupit do služby, totiž do služby při stanu setkávání.
#4:36 Povolaných do služby podle čeledí bylo dva tisíce sedm set padesát.
#4:37 To jsou povolaní do služby z kehatských čeledí, všichni, kdo slouží při stanu setkávání, jak je spočítal Mojžíš s Áronem na rozkaz Hospodinův daný skrze Mojžíše.
#4:38 Geršónovců povolaných do služby podle čeledí otcovských rodů
#4:39 od třicetiletých výše až po padesátileté, všech, kdo byli schopni nastoupit do služby, totiž do služby při stanu setkávání,
#4:40 povolaných do služby podle čeledí a otcovských rodů bylo dva tisíce šest set třicet.
#4:41 To jsou povolaní do služby z čeledí Geršónovců, všichni, kdo slouží při stanu setkávání, jak je spočítal Mojžíš s Áronem na rozkaz Hospodinův.
#4:42 Povolaných do služby z čeledí Meraríovců podle čeledí otcovských rodů
#4:43 od třicetiletých výše až po padesátileté, všech, kdo byli schopni nastoupit do služby, totiž do služby při stanu setkávání,
#4:44 povolaných do služby podle čeledí bylo tři tisíce dvě stě.
#4:45 To jsou povolaní do služby z čeledí Meraríovců, jak je spočítal Mojžíš s Áronem na rozkaz Hospodinův daný skrze Mojžíše.
#4:46 Všech lévijců povolaných do služby, jak je spočítal Mojžíš s Áronem a izraelskými předáky podle čeledí otcovských rodů,
#4:47 od třicetiletých výše až po padesátileté, všech, kdo přicházeli vykonávat uloženou práci a jakoukoli pracovní povinnost při stanu setkávání,
#4:48 povolaných do služby bylo osm tisíc pět set osmdesát.
#4:49 Sečetl je na rozkaz Hospodinův daný skrze Mojžíše, každého podle jeho pracovní povinnosti; to jsou povolaní do služby, jak přikázal Hospodin Mojžíšovi. 
#5:1 Hospodin promluvil k Mojžíšovi:
#5:2 „Přikaž Izraelcům, ať vyhostí z tábora každého malomocného, každého, kdo trpí výtokem, i každého, kdo se znečistil při mrtvém.
#5:3 Vyhostíte osoby jak mužského, tak i ženského pohlaví, vyhostíte je ven za tábor, aby neznečišťovaly tábory těch, uprostřed nichž já přebývám.“
#5:4 Izraelci tak učinili a vyhostili je ven za tábor. Udělali to tak, jak mluvil Hospodin k Mojžíšovi.
#5:5 Hospodin dále mluvil k Mojžíšovi:
#5:6 „Mluv k Izraelcům: Když se muž nebo žena dopustí nějakého hříchu, jak se ho lidé dopouštějí zpronevěrou Hospodinu, a budou shledáni vinnými,
#5:7 vyznají svůj hřích, jehož se dopustili, a viník dá odškodnění a navíc přidá pětinu. To dá tomu, proti komu se provinil.
#5:8 Nemá-li ten člověk žádného zastánce, jemuž by se odškodnění mělo předat, připadne stanovené odškodnění Hospodinu a předá se knězi, kromě berana smírčí oběti, potřebného k vykonání smírčího obřadu za něho.
#5:9 Každá oběť pozdvihování se všemi svatými dary, které Izraelci přinesou knězi, bude jeho.
#5:10 Svaté dary od kohokoli budou jeho; co kdo dá knězi, bude jeho.“
#5:11 Hospodin dále mluvil k Mojžíšovi:
#5:12 „Mluv k Izraelcům a řekni jim: Kdyby se žena některého muže dostala na scestí a zpronevěřila se mu
#5:13 a někdo se s ní tělesně stýkal, třebaže to zůstane skryto jejímu muži a ona se neprozradí, poskvrnila se, i když proti sobě nemá svědka a nebyla přistižena.
#5:14 Když se muže zmocní žárlivost a bude žárlit na svou ženu, která se poskvrnila, anebo se ho zmocní žárlivost a bude žárlit na svou ženu, která se neposkvrnila,
#5:15 přivede svou ženu ke knězi a přinese za ni jako oběť desetinu éfy ječné mouky, ale nepoleje ji olejem ani na ni nedá kadidlo, neboť to je obětní dar žárlivosti, připomínkový obětní dar, který má připomenout vinu.
#5:16 Kněz ženu přivede a postaví ji před Hospodina.
#5:17 Pak vezme kněz do hliněné nádoby svatou vodu, nabere trochu prachu z podlahy v příbytku a dá jej do vody.
#5:18 Potom kněz postaví ženu s rozpuštěnými vlasy před Hospodina a dá jí do rukou připomínkový obětní dar, to je obětní dar žárlivosti. Kněz bude mít v ruce hořkou vodu prokletí
#5:19 a bude zapřísahat ženu slovy: ‚Jestliže se s tebou nikdo nestýkal a ty ses nedostala na scestí a neposkvrnila ses před svým mužem, buď nedotčena touto hořkou vodou prokletí.
#5:20 Ale jestliže ses dostala na scestí a byla svému muži nevěrná a poskvrnila se tím, že se s tebou stýkal někdo jiný kromě tvého muže...‘,
#5:21 tu kněz zapřisáhne ženu přísežnou kletbou a řekne jí: ‚Ať s tebou Hospodin naloží uprostřed tvého lidu podle přísežné kletby! Hospodin ať způsobí, aby tvůj klín potratil a břicho ti nadulo.
#5:22 Ať tato voda prokletí vnikne do tvých útrob, aby ti břicho nadulo a tvůj klín potratil.‘ Žena odpoví: ‚Amen, amen.‘
#5:23 Kněz pak napíše tyto kletby na listinu a spláchne je do hořké vody
#5:24 a dá ženě vypít tu hořkou vodu prokletí, takže ji voda prokletí naplní hořkostí.
#5:25 Potom vezme kněz z ženiny ruky obětní dar žárlivosti, bude obětní dar nabízet podáváním Hospodinu a přinese jej k oltáři.
#5:26 Vezme z obětního daru hrst na připomínku a obrátí ji na oltáři v obětní dým. Potom dá ženě napít té vody.
#5:27 Když jí dal napít té vody, tedy v případě, že se poskvrnila zpronevěrou svému muži, naplní ji voda prokletí hořkostí, břicho jí naduje, její klín potratí a žena bude ve svém lidu kletbou.
#5:28 Jestliže se však žena neposkvrnila, nýbrž je čistá, bude podezření zproštěna a bude mít potomky.
#5:29 Toto je řád v případě žárlivosti, když se žena dostane na scestí, je nevěrná svému muži a poskvrní se.
#5:30 Muž, jehož se zmocní žárlivost a jenž bude na svou ženu žárlit, postaví ji před Hospodina a kněz s ní vykoná vše podle tohoto řádu.
#5:31 Muž bude prost viny, žena svou vinu ponese.“ 
#6:1 Hospodin promluvil k Mojžíšovi:
#6:2 „Mluv k Izraelcům a řekni jim: Když se muž nebo žena rozhodne složit mimořádný slib nazírský a zasvětí se Hospodinu,
#6:3 ať se vystříhá vína a opojného nápoje. Nebude pít nic kvašeného, víno ani opojný nápoj, nebude pít šťávu z hroznů ani jíst čerstvé nebo sušené hrozny.
#6:4 Po celou dobu svého nazírství neokusí nic z toho, co pochází z vinné révy, od nezralých hroznů až po výhonky.
#6:5 Po celou dobu platnosti slibu svého nazírství si nedá břitvou oholit hlavu. Dokud neskončí doba, po kterou se zasvětil Hospodinu, bude svatý a nechá si na hlavě volně růst vlasy.
#6:6 Po celou dobu svého zasvěcení Hospodinu nesmí přijít do styku se zemřelým.
#6:7 Nesmí se poskvrnit ani tehdy, když mu zemře otec nebo matka, bratr nebo sestra, protože na jeho hlavě je nazírské znamení jeho Boha.
#6:8 Je svatý Hospodinův po celou dobu svého nazírství.
#6:9 Kdyby však zcela nenadále poblíž něho někdo zemřel a on poskvrnil svou hlavu s nazírským znamením, oholí si ji v den, kdy se bude očišťovat, oholí ji sedmého dne.
#6:10 V osmý den přinese knězi dvě hrdličky nebo dvě holoubátka ke vchodu do stanu setkávání.
#6:11 Kněz připraví jedno v oběť za hřích a druhé v oběť zápalnou, aby za něho vykonal smírčí obřady proto, že se prohřešil při mrtvém; toho dne mu zase posvětí hlavu.
#6:12 On se pak znovu zasvětí Hospodinu po dobu svého nazírství a přinese ročního beránka jako oběť za vinu; avšak předešlá doba bude neplatná, protože své nazírství poskvrnil.
#6:13 Toto je řád pro nazíra: V den, kdy skončí doba jeho nazírství, přivede ho kněz ke vchodu do stanu setkávání.
#6:14 Jako svůj dar Hospodinu přinese jednoho ročního beránka bez vady k oběti zápalné, jednu roční ovci bez vady k oběti za hřích a jednoho berana bez vady k oběti pokojné;
#6:15 dále koš nekvašených chlebů, totiž bochánky z bílé mouky zadělané olejem a nekvašené oplatky pomazané olejem, a příslušné přídavné oběti a úlitby.
#6:16 Kněz to přinese před Hospodina a bude obětovat jeho oběť za hřích a jeho oběť zápalnou.
#6:17 Berana připraví Hospodinu jako hod oběti pokojné spolu s košem nekvašených chlebů; kněz připraví i jeho přídavnou oběť a jeho úlitbu.
#6:18 Nazír si pak u vchodu do stanu setkávání oholí na hlavě nazírské znamení, vezme vlasy z hlavy, své nazírské znamení, a dá je na oheň, který bude pod pokojnou obětí.
#6:19 Nato kněz vezme vařené plece z berana, z koše jeden nekvašený bochánek a jednu nekvašenou oplatku a vloží to do rukou nazírovi, poté co si oholil své nazírské znamení.
#6:20 Kněz to nabídne Hospodinu podáváním jako oběť podávání. Je to svatý příděl pro kněze spolu s hrudím oběti podávání a kýtou oběti pozdvihování. Potom už může nazír pít víno.“
#6:21 Takový je řád pro nazíra, který Hospodinu slíbil dar za svoje nazírství kromě toho, co dá sám od sebe navíc. Učiní podle svého slibu, který složil, podle řádu svého nazírství.
#6:22 Hospodin dále promluvil k Mojžíšovi:
#6:23 „Mluv k Áronovi a jeho synům: Budete žehnat synům Izraele těmito slovy:
#6:24 ‚Ať Hospodin ti žehná a chrání tě,
#6:25 ať Hospodin rozjasní nad tebou svou tvář a je ti milostiv,
#6:26 ať Hospodin obrátí k tobě svou tvář a obdaří tě pokojem.‘
#6:27 Tak vloží mé jméno na Izraelce a já jim požehnám.“ 
#7:1 V den, kdy Mojžíš dokončil stavbu příbytku, pomazal a posvětil jej i všechno jeho nářadí, též oltář s veškerým nářadím. Když to vše pomazal a posvětil,
#7:2 přinesli izraelští předáci, představitelé otcovských rodů, své dary; byli to předáci pokolení, ti, kteří stáli nad povolanými do služby.
#7:3 Přivedli jako svůj dar před Hospodina šest krytých povozů a dvanáct kusů skotu, po jednom povozu od dvou předáků a od každého předáka jednoho býka; ty přinesli jako dar před příbytek.
#7:4 Hospodin řekl Mojžíšovi:
#7:5 „Přijmi je od nich. Budou tu pro službu při stanu setkávání. Dáš je lévijcům, každé čeledi podle její služby.“
#7:6 Mojžíš přijal povozy i skot a dal je lévijcům.
#7:7 Dva povozy a čtyři kusy skotu dal Geršónovcům pro jejich službu.
#7:8 Čtyři povozy a osm kusů skotu dal Meraríovcům pro jejich službu za dozoru Ítamara, syna kněze Árona.
#7:9 Kehatovcům nedal nic, protože jejich služba se týkala svatých předmětů, a ty se nosily na ramenou.
#7:10 Předáci přinesli také dar k zasvěcení oltáře. V den, kdy byl oltář pomazán, přinesli před něj svůj dar.
#7:11 Hospodin řekl Mojžíšovi: „Každého dne přinese vždy jeden z předáků svůj dar k zasvěcení oltáře.“
#7:12 Prvního dne tedy přinesl svůj dar Nachšón, syn Amínadabův, z pokolení Judova.
#7:13 Jeho dar: jedna stříbrná mísa o váze sto třiceti šekelů, jedna stříbrná kropenka o sedmdesáti šekelech podle váhy určené svatyní - obojí plné bílé mouky zadělané olejem k oběti přídavné -,
#7:14 jedna zlatá pánvička o deseti šekelech, plná kadidla,
#7:15 jeden mladý býček, jeden beran, jeden roční beránek k oběti zápalné,
#7:16 jeden kozel k oběti za hřích,
#7:17 k hodu oběti pokojné dva kusy skotu, pět beranů, pět kozlů a pět ročních beránků. To byl dar Nachšóna, syna Amínadabova.
#7:18 Druhého dne přinesl svůj dar Netaneel, syn Súarův, předák Isacharův.
#7:19 Přinesl darem jednu stříbrnou mísu o váze sto třiceti šekelů, jednu stříbrnou kropenku o sedmdesáti šekelech podle váhy určené svatyní - obojí plné bílé mouky zadělané olejem k oběti přídavné -,
#7:20 jednu zlatou pánvičku o deseti šekelech, plnou kadidla,
#7:21 jednoho mladého býčka, jednoho berana, jednoho ročního beránka k oběti zápalné,
#7:22 jednoho kozla k oběti za hřích,
#7:23 k hodu oběti pokojné dva kusy skotu, pět beranů, pět kozlů a pět ročních beránků. To byl dar Netaneela, syna Súarova.
#7:24 Třetího dne předák Zabulónovců, Elíab, syn Chelónův.
#7:25 Jeho dar: jedna stříbrná mísa o váze sto třiceti šekelů, jedna stříbrná kropenka o sedmdesáti šekelech podle váhy určené svatyní - obojí plné bílé mouky zadělané olejem k oběti přídavné -,
#7:26 jedna zlatá pánvička o deseti šekelech, plná kadidla,
#7:27 jeden mladý býček, jeden beran, jeden roční beránek k oběti zápalné,
#7:28 jeden kozel k oběti za hřích,
#7:29 k hodu oběti pokojné dva kusy skotu, pět beranů, pět kozlů a pět ročních beránků. To byl dar Elíaba, syna Chelónova.
#7:30 Čtvrtého dne předák Rúbenovců Elísúr, syn Šedeúrův.
#7:31 Jeho dar: jedna stříbrná mísa o váze sto třiceti šekelů, jedna stříbrná kropenka o sedmdesáti šekelech podle váhy určené svatyní - obojí plné bílé mouky zadělané olejem k oběti přídavné -,
#7:32 jedna zlatá pánvička o deseti šekelech, plná kadidla,
#7:33 jeden mladý býček, jeden beran, jeden roční beránek k oběti zápalné,
#7:34 jeden kozel k oběti za hřích,
#7:35 k hodu oběti pokojné dva kusy skotu, pět beranů, pět kozlů a pět ročních beránků. To byl dar Elísúra, syna Šedeúrova.
#7:36 Pátého dne předák Šimeónovců Šelumíel, syn Suríšadajův.
#7:37 Jeho dar: jedna stříbrná mísa o váze sto třiceti šekelů, jedna stříbrná kropenka o sedmdesáti šekelech podle váhy určené svatyní - obojí plné bílé mouky zadělané olejem k oběti přídavné -,
#7:38 jedna zlatá pánvička o deseti šekelech, plná kadidla,
#7:39 jeden mladý býček, jeden beran, jeden roční beránek k oběti zápalné,
#7:40 jeden kozel k oběti za hřích,
#7:41 k hodu oběti pokojné dva kusy skotu, pět beranů, pět kozlů a pět ročních beránků. To byl dar Šelumíela, syna Suríšadajova.
#7:42 Šestého dne předák Gádovců Eljásaf, syn Deúelův.
#7:43 Jeho dar: jedna stříbrná mísa o váze sto třiceti šekelů, jedna stříbrná kropenka o sedmdesáti šekelech podle váhy určené svatyní - obojí plné bílé mouky zadělané olejem k oběti přídavné -,
#7:44 jedna zlatá pánvička o deseti šekelech, plná kadidla,
#7:45 jeden mladý býček, jeden beran, jeden roční beránek k oběti zápalné,
#7:46 jeden kozel k oběti za hřích,
#7:47 k hodu oběti pokojné dva kusy skotu, pět beranů, pět kozlů a pět ročních beránků. To byl dar Eljásafa, syna Deúelova.
#7:48 Sedmého dne předák Efrajimovců Elíšama, syn Amíhudův.
#7:49 Jeho dar: jedna stříbrná mísa o váze sto třiceti šekelů, jedna stříbrná kropenka o sedmdesáti šekelech podle váhy určené svatyní - obojí plné bílé mouky zadělané olejem k oběti přídavné -,
#7:50 jedna zlatá pánvička o deseti šekelech, plná kadidla,
#7:51 jeden mladý býček, jeden beran, jeden roční beránek k oběti zápalné,
#7:52 jeden kozel k oběti za hřích,
#7:53 k hodu oběti pokojné dva kusy skotu, pět beranů, pět kozlů a pět ročních beránků. To byl dar Elíšamy, syna Amíhudova.
#7:54 Osmého dne předák Manasesovců Gamlíel, syn Pedásurův.
#7:55 Jeho dar: jedna stříbrná mísa o váze sto třiceti šekelů, jedna stříbrná kropenka o sedmdesáti šekelech podle váhy určené svatyní - obojí plné bílé mouky zadělané olejem k oběti přídavné -,
#7:56 jedna zlatá pánvička o deseti šekelech, plná kadidla,
#7:57 jeden mladý býček, jeden beran, jeden roční beránek k oběti zápalné,
#7:58 jeden kozel k oběti za hřích,
#7:59 k hodu oběti pokojné dva kusy skotu, pět beranů, pět kozlů a pět ročních beránků. To byl dar Gamlíela, syna Pedásurova.
#7:60 Devátého dne předák Benjamínovců Abídan, syn Gideóního.
#7:61 Jeho dar: jedna stříbrná mísa o váze sto třiceti šekelů, jedna stříbrná kropenka o sedmdesáti šekelech podle váhy určené svatyní - obojí plné bílé mouky zadělané olejem k oběti přídavné -,
#7:62 jedna zlatá pánvička o deseti šekelech, plná kadidla,
#7:63 jeden mladý býček, jeden beran, jeden roční beránek k oběti zápalné,
#7:64 jeden kozel k oběti za hřích,
#7:65 k hodu oběti pokojné dva kusy skotu, pět beranů, pět kozlů a pět ročních beránků. To byl dar Abídana, syna Gideóního.
#7:66 Desátého dne předák Danovců Achíezer, syn Amíšadajův.
#7:67 Jeho dar: jedna stříbrná mísa o váze sto třiceti šekelů, jedna stříbrná kropenka o sedmdesáti šekelech podle váhy určené svatyní - obojí plné bílé mouky zadělané olejem k oběti přídavné -,
#7:68 jedna zlatá pánvička o deseti šekelech, plná kadidla,
#7:69 jeden mladý býček, jeden beran, jeden roční beránek k oběti zápalné,
#7:70 jeden kozel k oběti za hřích,
#7:71 k hodu oběti pokojné dva kusy skotu, pět beranů, pět kozlů a pět ročních beránků. To byl dar Achíezera, syna Amíšadajova.
#7:72 Jedenáctého dne předák Ašerovců Pagíel, syn Okranův.
#7:73 Jeho dar: jedna stříbrná mísa o váze sto třiceti šekelů, jedna stříbrná kropenka o sedmdesáti šekelech podle váhy určené svatyní - obojí plné bílé mouky zadělané olejem k oběti přídavné -,
#7:74 jedna zlatá pánvička o deseti šekelech, plná kadidla,
#7:75 jeden mladý býček, jeden beran, jeden roční beránek k oběti zápalné,
#7:76 jeden kozel k oběti za hřích,
#7:77 k hodu oběti pokojné dva kusy skotu, pět beranů, pět kozlů a pět ročních beránků. To byl dar Pagíela, syna Okranova.
#7:78 Dvanáctého dne předák Neftalíovců Achíra, syn Énanův.
#7:79 Jeho dar: jedna stříbrná mísa o váze sto třiceti šekelů, jedna stříbrná kropenka o sedmdesáti šekelech podle váhy určené svatyní - obojí plné bílé mouky zadělané olejem k oběti přídavné -,
#7:80 jedna zlatá pánvička o deseti šekelech, plná kadidla,
#7:81 jeden mladý býček, jeden beran, jeden roční beránek k oběti zápalné,
#7:82 jeden kozel k oběti za hřích,
#7:83 k hodu oběti pokojné dva kusy skotu, pět beranů, pět kozlů a pět ročních beránků. To byl dar Achíry, syna Énanova.
#7:84 To byl dar k zasvěcení oltáře od izraelských předáků v den, kdy byl pomazán: dvanáct stříbrných mis, dvanáct stříbrných kropenek, dvanáct zlatých pánviček.
#7:85 Sto třicet šekelů stříbra měla jedna mísa a jedna kropenka sedmdesát; stříbro všech nádob vážilo dva tisíce čtyři sta šekelů podle váhy určené svatyní.
#7:86 Zlatých pánviček bylo dvanáct, plných kadidla, každá pánvička po deseti šekelech podle váhy určené svatyní; zlato všech pánviček vážilo sto dvacet šekelů.
#7:87 Všech kusů skotu k oběti zápalné bylo dvanáct býčků, k tomu dvanáct beranů, dvanáct ročních beránků s příslušnými přídavnými obětmi a dvanáct kozlů k oběti za hřích;
#7:88 všech kusů skotu k hodu oběti pokojné bylo čtyřiadvacet býčků, k tomu šedesát beranů, šedesát kozlů a šedesát ročních beránků. To byl dar k zasvěcení oltáře poté, co byl pomazán.
#7:89 Když Mojžíš vcházel do stanu setkávání, aby Bůh k němu mluvil, slyšel hlas, jak k němu mluví od příkrovu na schráně svědectví, z místa mezi dvěma cheruby; i mluvil s ním. 
#8:1 Hospodin promluvil k Mojžíšovi:
#8:2 „Mluv k Áronovi a řekni mu: Když budeš nasazovat kahánky, dbej, aby těch sedm kahánků svítilo dopředu před svícen.“
#8:3 Áron tak učinil. Nasadil kahánky na svícen směrem dopředu, jak přikázal Hospodin Mojžíšovi.
#8:4 Svícen byl vyroben z tepaného zlata; jak dřík, tak květy na něm byly z tepaného zlata. Mojžíš zhotovil svícen podle vzoru, který mu ukázal Hospodin.
#8:5 Hospodin dále mluvil k Mojžíšovi:
#8:6 „Vezmi z Izraelců lévijce a očisť je.
#8:7 Aby byli čisti, vykonej s nimi toto: Pokrop je vodou rozhřešení, pak ať si dají břitvou oholit celé tělo a vyperou si roucha. Tak se očistí.
#8:8 Potom vezmou mladého býčka a jako oběť přídavnou bílou mouku zadělanou olejem. A ty vezmeš druhého mladého býčka k oběti za hřích.
#8:9 Přikážeš, aby lévijci předstoupili před stan setkávání, a svoláš celou pospolitost Izraelců.
#8:10 Přivedeš lévijce před Hospodina a Izraelci na ně budou vkládat ruce.
#8:11 Nato Áron nabídne lévijce Hospodinu podáváním jako oběť podávání od Izraelců, a budou způsobilí vykonávat Hospodinovu službu.
#8:12 Lévijci vloží ruce na hlavy těch býků a ty obětuj Hospodinu jednoho jako oběť za hřích a druhého jako oběť zápalnou k vykonání smírčích obřadů za lévijce.
#8:13 Pak lévijce postavíš před Árona a jeho syny, nabídneš je Hospodinu jako oběť podávání
#8:14 a oddělíš lévijce od ostatních Izraelců. Lévijci jsou moji.
#8:15 Lévijci půjdou sloužit při stanu setkávání teprve potom, až je očistíš a nabídneš jako oběť podávání.
#8:16 Budou mi odděleni z Izraelců jako dar; vzal jsem si je místo všech, kteří otvírají lůno, místo všech prvorozených mezi Izraelci.
#8:17 Mně patří všechno prvorozené mezi Izraelci, člověk i dobytče. Oddělil jsem je pro sebe jako svaté v den, kdy jsem pobil v egyptské zemi všechno prvorozené.
#8:18 Vzal jsem si lévijce místo všech prvorozených mezi Izraelci.
#8:19 Dávám je Áronovi a jeho synům jako darované z Izraelců, aby za Izraelce vykonávali službu při stanu setkávání. Budou za Izraelce vykonávat smírčí obřady, aby je nepostihla pohroma, když přistoupí ke svatyni.“
#8:20 Mojžíš a Áron i celá pospolitost Izraelců vykonali s lévijci to, co o nich přikázal Hospodin Mojžíšovi; to s nimi Izraelci vykonali.
#8:21 Lévijci se očistili od hříchů, vyprali si roucha a Áron je podáváním nabídl Hospodinu jako oběť podávání, a tak za ně vykonal smírčí obřady, aby byli čisti.
#8:22 Teprve potom šli lévijci vykonávat službu při stanu setkávání před Áronem a jeho syny. Jak Hospodin Mojžíšovi o lévijcích příkázal, tak s nimi učinili.
#8:23 Hospodin dále mluvil k Mojžíšovi:
#8:24 „Toto je řád pro lévijce: Od pětadvaceti let bude schopen nastoupit do služby, totiž konat službu při stanu setkávání.
#8:25 Od padesáti let bude této služby zproštěn a nebude už sloužit.
#8:26 Může ovšem přisluhovat svým bratřím při stanu setkávání ve strážné službě, ale vlastní službu konat nebude. Tak to učiníš s lévijci vzhledem k tomu, co jim bylo svěřeno.“ 
#9:1 Hospodin promluvil k Mojžíšovi na Sínajské poušti v prvním měsíci, druhého roku po jejich vyjití z egyptské země:
#9:2 „Ať Izraelci slaví ve stanovený čas hod beránka.
#9:3 Budete jej slavit ve stanovený čas čtrnáctého dne tohoto měsíce navečer, budete jej slavit podle všech příslušných nařízení a směrnic.“
#9:4 I vyzval Mojžíš Izraelce, aby slavili hod beránka.
#9:5 A slavili hod beránka prvního měsíce, čtrnáctého dne toho měsíce navečer, na Sínajské poušti. Izraelci učinili všechno tak, jak Hospodin Mojžíšovi přikázal.
#9:6 Byli tu však muži, kteří se poskvrnili při mrtvém člověku a nemohli onoho dne hod beránka slavit. Proto předstoupili onoho dne ti muži před Mojžíše - a Árona -
#9:7 a řekli mu: „My jsme se poskvrnili při mrtvém člověku. Proč máme být vyloučeni a nesmíme přinést dar Hospodinu ve stanovený čas spolu s ostatními Izraelci?“
#9:8 Mojžíš jim odpověděl: „Zůstaňte tu, dokud neuslyším, co vám přikáže Hospodin.“
#9:9 Hospodin dále mluvil k Mojžíšovi:
#9:10 „Mluv k Izraelcům: Když se někdo z vás nebo z vašich budoucích pokolení poskvrní při mrtvém anebo bude na daleké cestě, bude také slavit hod beránka Hospodinu.
#9:11 Budou jej slavit druhého měsíce čtrnáctého dne navečer. Budou ho jíst s nekvašenými chleby a hořkými bylinami.
#9:12 Neponechají z něho nic do rána a nezlámou mu žádnou kost. Budou slavit hod beránka podle všech nařízení o něm.
#9:13 Ale kdo by byl čistý a nebyl na cestě a zanedbá slavení hodu beránka, bude vyobcován ze svého lidu, protože nepřinesl Hospodinu dar ve stanovený čas. Takový člověk ponese následky svého hříchu.
#9:14 Host, který s vámi bude pobývat, může slavit hod beránka Hospodinu, ale bude jej slavit podle nařízení o hodu beránka, jak je určeno. Stejné nařízení jako pro vás bude platit i pro hosta a domorodce v zemi.“
#9:15 V den, kdy byl postaven příbytek, přikryl jej oblak a zahalil stan svědectví; od večera do rána se jevil nad příbytkem jako ohnivá zář.
#9:16 Tak tomu bylo každodenně. Přikrýval jej oblak a v noci měl vzhled ohnivé záře.
#9:17 Kdykoli se oblak od stanu vznesl, hned táhli Izraelci dál. Kde se oblak pozdržel, tam se Izraelci utábořili.
#9:18 Na Hospodinův rozkaz táhli Izraelci dál a na Hospodinův rozkaz tábořili. Tábořili po celou dobu, pokud se zdržoval oblak nad příbytkem.
#9:19 Když oblak setrvával nad příbytkem po mnoho dní, drželi Izraelci Hospodinovu stráž a netáhli dál.
#9:20 Někdy však byl oblak nad příbytkem jen několik dní. Na Hospodinův rozkaz tábořili, na Hospodinův rozkaz táhli dál.
#9:21 Někdy tu byl oblak jen od večera do rána; když se oblak ráno vznesl, i oni táhli dál. Ať to bylo ve dne nebo v noci, jakmile se oblak vznesl, táhli dál.
#9:22 Ať setrvával oblak nad příbytkem třeba dva dny nebo měsíc nebo rok, dokud nad ním setrvával, Izraelci tábořili a netáhli dál. Když se vznesl, táhli dál.
#9:23 Na Hospodinův rozkaz tábořili, na Hospodinův rozkaz táhli dál. Drželi Hospodinovu stráž na Hospodinův rozkaz daný skrze Mojžíše. 
#10:1 Hospodin promluvil k Mojžíšovi:
#10:2 „Zhotov si dvě stříbrné trubky. Uděláš je z tepaného stříbra. Budou ti sloužit k svolávání pospolitosti a k povelu, že tábory mají táhnout dál.
#10:3 Zatroubí-li se na ně, sejde se k tobě ke vchodu do stanu setkávání celá pospolitost.
#10:4 Jestliže se zatroubí jen na jednu, sejdou se k tobě předáci, náčelníci izraelských šiků.
#10:5 Když ryčně zatroubíte po prvé, vytáhnou tábory tábořící na východ od stanu setkávání.
#10:6 Když ryčně zatroubíte po druhé, vytáhnou tábory tábořící na jihu. Ryčně se bude troubit při nástupu cesty.
#10:7 Budete-li však svolávat shromáždění, zatroubíte sice, ale ne ryčně.
#10:8 Na trubky budou troubit Áronovi synové, kněží. To vám bude provždy platným nařízením pro všechna vaše pokolení.
#10:9 Až potom ve své zemi vytáhnete do boje proti protivníku, který vás bude sužovat, a trubkami budete ryčně troubit, připomenete se Hospodinu, svému Bohu, a budete osvobozeni od svých nepřátel.
#10:10 Na trubky budete troubit v den své radosti, o slavnostech, při novoluní, při svých zápalných obětech nebo svých hodech oběti pokojné, aby vás připomínaly vašemu Bohu. Já jsem Hospodin, váš Bůh.“
#10:11 Ve druhém roce, dvacátého dne druhého měsíce, se vznesl oblak od příbytku svědectví.
#10:12 I táhli Izraelci ze Sínajské pouště dál, po jednotlivých stanovištích, až se oblak pozdržel na poušti Páranské.
#10:13 Tak poprvé táhli dál na Hospodinův rozkaz daný skrze Mojžíše.
#10:14 Jako první táhl po oddílech prapor tábora Judovců. Nad jejich vojem byl velitelem Nachšón, syn Amínadabův.
#10:15 Nad vojem pokolení Isacharovců byl Netaneel, syn Súarův.
#10:16 Nad vojem pokolení Zabulónovců byl Elíab, syn Chelónův.
#10:17 Pak byl složen příbytek a vytáhli Geršónovci a Meraríovci, kteří příbytek nosili.
#10:18 Dále táhl po oddílech prapor tábora Rúbenovců. Nad jejich vojem byl Elísúr, syn Šedeúrův.
#10:19 Nad vojem pokolení Šimeónovců byl Šelumíel, syn Suríšadajův.
#10:20 Nad vojem pokolení Gádovců byl Eljásaf, syn Deúelův.
#10:21 Pak táhli Kehatovci, kteří nosili svatyni. Než došli, druzí už postavili příbytek.
#10:22 Dále táhl po oddílech prapor tábora Efrajimovců. Nad jejich vojem byl Elíšama, syn Amíhudův.
#10:23 Nad vojem pokolení Manasesovců byl Gamlíel, syn Pedásurův.
#10:24 Nad vojem pokolení Benjamínovců byl Abídan, syn Gideóního. -
#10:25 Pak táhl po oddílech prapor tábora Danovců, uzavírající všechny tábory. Nad jejich vojem byl Achíezer, syn Amíšadajův.
#10:26 Nad vojem pokolení Ašerovců byl Pagíel, syn Okranův.
#10:27 Nad vojem pokolení Neftalíovců byl Achíra, syn Énanův.
#10:28 To byl postup Izraelců při tažení po oddílech; tak táhli dál.
#10:29 Mojžíš řekl Chóbabovi, synu Midjánce Reúela, svého tchána: „Táhneme k místu, o kterém Hospodin prohlásil: Dám vám je. Pojď s námi, prokážeme ti dobro. Vždyť Hospodin přiřkl Izraeli vše dobré.“
#10:30 On mu však odpověděl: „Nepůjdu; chci jít do své země a do svého rodiště.“
#10:31 Mojžíš řekl: „Neopouštěj nás prosím! Znáš přece na poušti místa, kde můžeme tábořit; buď naším průvodcem.
#10:32 Když půjdeš spolu s námi a dostaví se to dobro, které nám chce Hospodin prokázat, prokážeme dobro i my tobě.“
#10:33 I táhli od hory Hospodinovy tři dny cesty a schrána Hospodinovy smlouvy táhla po tři dny cesty před nimi, aby jim vyhlédla místo odpočinutí.
#10:34 A když táhli z tábora, býval ve dne nad nimi Hospodinův oblak.
#10:35 Kdykoli schrána měla táhnout dál, říkal Mojžíš: „Povstaň, Hospodine, ať se rozprchnou tvoji nepřátelé, ať před tebou utečou, kdo tě nenávidí!“
#10:36 Když se zastavila k odpočinku, říkal: „Navrať se, Hospodine, k desetitisícům izraelských šiků!“ 
#11:1 Lid si začal Hospodinu stěžovat na těžkosti. Hospodin to slyšel a vzplanul hněvem. Tu vyšlehl mezi nimi Hospodinův oheň a pohltil okraj tábora.
#11:2 Lid volal k Mojžíšovi. I modlil se Mojžíš k Hospodinu a oheň uhasl.
#11:3 Proto pojmenovali ono místo Tabéra (to je Spáleniště), že mezi nimi vyšlehl Hospodinův oheň.
#11:4 Chátru přimíšenou mezi nimi popadla žádostivost. Rovněž Izraelci začali znovu s pláčem volat: „Kdo nám dá najíst masa?
#11:5 Vzpomínáme na ryby, které jsme měli v Egyptě k jídlu zadarmo, na okurky a melouny, na pór, cibuli a česnek.
#11:6 Jsme už celí seschlí, nevidíme nic jiného než tu manu.“
#11:7 Mana byla jako koriandrové semeno a měla vzhled vonné pryskyřice.
#11:8 Lid ji chodíval sbírat, pak ji mleli mlýnkem nebo drtili v hmoždíři, vařili v kotlíku nebo z ní připravovali podpopelné chleby; měla chuť jako pečivo zadělané olejem.
#11:9 Když v noci padala na tábor rosa, padala na něj i mana.
#11:10 Mojžíš slyšel, že lid pláče, každý u vchodu do svého stanu, čeleď vedle čeledi. Hospodin vzplanul velikým hněvem. I Mojžíš na to hleděl s nevolí
#11:11 a vyčítal Hospodinu: „Proč zacházíš se svým služebníkem tak zle? Proč jsem u tebe nenalezl milost, žes na mě vložil všechen tento lid jako břímě?
#11:12 Copak jsem všechen tento lid počal já? Copak jsem ho porodil já, že mi říkáš: Nes jej v náručí, jako chůva nemluvňátko, do země, kterou jsi přisáhl dát jeho otcům?
#11:13 Kde vezmu maso, abych je dal všemu tomuto lidu? Volají ke mně s pláčem: ‚Dej nám maso, chceme jíst.‘
#11:14 Nemohu sám unést všechen tento lid, je to nad mé síly.
#11:15 Když už se mnou chceš takto jednat, raději mě zabij, jestliže jsem u tebe nalezl milost, abych se nemusel dívat na svoje trápení.“
#11:16 Hospodin Mojžíšovi odvětil: „Shromažď mi sedmdesát mužů z izraelských starších, o nichž víš, že jsou staršími a správci v lidu. Vezmi je ke stanu setkávání, ať se tam postaví s tebou.
#11:17 Sestoupím a budu tam s tebou mluvit. A odeberu z ducha, který je na tobě, a vložím jej na ně. Ponesou pak břímě lidu s tebou, neponeseš je už sám.
#11:18 A lidu řekneš: Posvěťte se pro zítřek. Budete jíst maso, protože jste s pláčem volali k Hospodinu: ‚Kdo nám dá najíst masa? Jak dobře nám bylo v Egyptě!‘ Hospodin vám dá maso a najíte se.
#11:19 Nebudete je jíst jen jeden den, ani dva dny, ani pět dní, ani deset dní, ba ani dvacet dní,
#11:20 ale po celý měsíc, až vám poleze z chřípí a zhnusí se vám. To proto, že jste zavrhli Hospodina, který je mezi vámi, když jste před ním s pláčem volali: ‚Proč jsme jen odešli z Egypta?‘“
#11:21 Mojžíš namítl: „Šest set tisíc pěších je v lidu, uprostřed něhož jsem, a ty řekneš: Dám jim maso a budou jíst po celý měsíc!
#11:22 To se pro ně budou porážet ovce a skot, aby se na všechny dostalo? Nebo se pro ně vychytají všechny ryby v moři, aby se na všechny dostalo?“
#11:23 Hospodin Mojžíšovi odvětil: „Což ruka Hospodinova je na to krátká? Hned uvidíš, uskuteční-li se mé slovo, nebo ne.“
#11:24 Mojžíš tedy vyšel ven a oznámil Hospodinova slova lidu. Shromáždil sedmdesát mužů ze starších lidu a rozestavil je kolem stanu.
#11:25 Hospodin sestoupil v oblaku, promluvil k němu a odebral z ducha, který byl na něm, a dal jej těm sedmdesáti starším. Sotva na nich duch spočinul, prorokovali, ale potom už nikdy.
#11:26 Dva muži však zůstali v táboře, jeden se jmenoval Eldad a druhý Médad. I na nich spočinul duch, ačkoliv nepřišli ke stanu; byli totiž mezi zapsanými. Ti prorokovali v táboře.
#11:27 Tu přiběhl mládenec a oznámil Mojžíšovi: „Eldad a Médad v táboře prorokují.“
#11:28 Nato se ozval Jozue, syn Núnův, který už jako jinoch přisluhoval Mojžíšovi, a zvolal: „Mojžíši, můj pane, zabraň jim v tom!“
#11:29 Ale Mojžíš mu řekl: „Ty kvůli mně žárlíš? Kéž by všechen Hospodinův lid byli proroci! Kéž by jim Hospodin dal svého ducha!“
#11:30 Potom se Mojžíš odebral s izraelskými staršími do tábora.
#11:31 Vtom se zvedl vítr seslaný Hospodinem, přihnal od moře křepelky a rozhodil je po táboře, asi do vzdálenosti jednodenní cesty na tu i na onu stranu kolem tábora, do výše kolem dvou loket nad zemí.
#11:32 Lid se hned pustil do sbírání křepelek a sbíral je po celý den i po celou noc a po celý příští den; i ten, kdo nasbíral málo, měl deset chómerů. Rozložili si je okolo tábora.
#11:33 Ještě měli v zubech maso, ještě nebylo ani rozžvýkáno, když Hospodin vzplanul proti lidu hněvem a počal lid bít převelikou ranou.
#11:34 Proto dali tomu místu jméno Kibrót-taava (to je Hroby žádostivosti), že tam pohřbili z lidu ty, kdo propadli žádostivosti.
#11:35 Z Kibrót-taavy táhl lid do Chaserótu. A v Chaserótu zůstali. 
#12:1 Mirjam s Áronem mluvila proti Mojžíšovi kvůli kúšské ženě, kterou si vzal; pojal totiž za ženu Kúšanku.
#12:2 Říkali: „Což Hospodin mluví jenom prostřednictvím Mojžíše? Což nemluví i naším prostřednictvím?“ Hospodin to slyšel.
#12:3 Mojžíš však byl nejpokornější ze všech lidí, kteří byli na zemi.
#12:4 Hned nato řekl Hospodin Mojžíšovi i Áronovi a Mirjamě: „Vy tři vyjděte ke stanu setkávání!“ Všichni tři tedy vyšli
#12:5 a Hospodin sestoupil v oblakovém sloupu a stanul u vchodu do stanu. Zavolal Árona a Mirjam a oba předstoupili.
#12:6 Řekl: „Poslyšte má slova: Bude-li mezi vámi prorok, já Hospodin se mu dám poznat ve vidění, mluvit s ním budu ve snu.
#12:7 Ne tak je tomu s mým služebníkem Mojžíšem. Má trvalé místo v celém mém domě.
#12:8 S ním mluvím od úst k ústům ve viděních, ne v hádankách; smí patřit na zjev Hospodinův. Jak to, že se tedy nebojíte mluvit proti mému služebníku Mojžíšovi?“
#12:9 Hospodin vzplanul proti nim hněvem a odešel.
#12:10 Když oblak od stanu odstoupil, hle, Mirjam byla malomocná, bílá jako sníh. Áron se obrátil k Mirjamě, a hle, byla malomocná.
#12:11 Tu řekl Áron Mojžíšovi: „Dovol, můj pane, nenech nás pykat za hřích, jehož jsme se ve své pošetilosti dopustili.
#12:12 Ať není Mirjam jako mrtvě narozené děcko, jehož tělo je z poloviny stráveno, hned jak vyšlo z matčina lůna.“
#12:13 I volal Mojžíš k Hospodinu: „Bože, prosím uzdrav ji, prosím!“
#12:14 Hospodin Mojžíšovi odvětil: „Kdyby jí otec naplil do tváře, nenesla by hanbu po sedm dní? Sedm dní bude vyloučena z tábora, potom se zase připojí.“
#12:15 Tak byla Mirjam sedm dní vyloučena z tábora a lid netáhl dál, dokud se Mirjam nepřipojila.
#12:16 Potom táhl lid z Chaserótu a utábořili se na poušti Páranské. 
#13:1 Hospodin promluvil k Mojžíšovi:
#13:2 „Pošli muže, aby prozkoumali kenaanskou zemi, kterou dávám Izraelcům. Pošlete po jednom muži z jejich otcovských pokolení, vždy jejich předáka!“
#13:3 Mojžíš je tedy na rozkaz Hospodinův poslal z Páranské pouště. Všichni ti muži byli náčelníci Izraelců.
#13:4 Toto jsou jejich jména: z Rúbenova pokolení Šamúa, syn Zakúrův,
#13:5 z Šimeónova pokolení Šafat, syn Chóríův,
#13:6 z Judova pokolení Káleb, syn Jefunův,
#13:7 z Isacharova pokolení Jigál, syn Josefův,
#13:8 z Efrajimova pokolení Hóšea, syn Núnův,
#13:9 z Benjamínova pokolení Paltí, syn Rafúův,
#13:10 ze Zabulónova pokolení Gadíel, syn Sódíův,
#13:11 z Josefova pokolení, z pokolení Manasesova, Gadí, syn Súsíův,
#13:12 z Danova pokolení Amíel, syn Gemalíův,
#13:13 z Ašerova pokolení Setúr, syn Míkaelův,
#13:14 z Neftalíova pokolení Nachbí, syn Vofsíův,
#13:15 z Gádova pokolení Geúel, syn Makíův.
#13:16 To jsou jména mužů, které poslal Mojžíš prozkoumat zemi. Hóšeu, syna Núnova, nazval Mojžíš Jozue.
#13:17 Mojžíš je poslal prozkoumat kenaanskou zemi a řekl jim: „Jděte vzhůru Negebem, vystupte pak na pohoří
#13:18 a zjistěte, jaká to je země a jaký lid v ní sídlí; zda je silný nebo slabý, zda ho je málo nebo mnoho;
#13:19 a zda je země, v níž sídlí, dobrá nebo zlá, též zda jsou města, v nichž sídlí, otevřená nebo opevněná;
#13:20 též zda je země úrodná či neúrodná, zda v ní jsou stromy nebo ne. Buďte odvážní a přineste něco z ovoce té země!“ Byl totiž čas raných hroznů.
#13:21 Šli tedy a prozkoumali zemi od pouště Sinu až k Rechóbu při cestě do Chamátu.
#13:22 Šli vzhůru Negebem a jeden z nich přišel až do Chebrónu, kde žili Achíman, Šešaj a Talmaj, zplozenci Anákovi; Chebrón byl vystavěn sedm let před egyptským Sóanem.
#13:23 Přišli až k úvalu Eškolu, kde uřízli ratolest s jedním vinným hroznem, dva ji museli nést na sochoru, a několik granátových jablek a fíků.
#13:24 To místo se nazývá Eškolský úval (to je Úval hroznů) podle hroznu, který tam Izraelci odřízli.
#13:25 Po čtyřiceti dnech průzkumu země se vrátili zpět.
#13:26 Přišli konečně k Mojžíšovi a Áronovi a k celé pospolitosti Izraelců na Páranskou poušť do Kádeše, podali jim a celé pospolitosti zprávu a ukázali jim ovoce té země.
#13:27 Ve svém vyprávění mu řekli: „Vstoupili jsme do země, do níž jsi nás poslal. Vskutku oplývá mlékem a medem. A toto je její ovoce.
#13:28 Jenomže lid, který v té zemi sídlí, je mocný a města jsou opevněná a nesmírně veliká. Dokonce jsme tam viděli potomky Anákovy.
#13:29 Na jihu země sídlí Amálek, na pohoří jsou usazeni Chetejci, Jebúsejci a Emorejci, při moři a podél Jordánu Kenaanci.“
#13:30 Káleb však uklidňoval lid bouřící se proti Mojžíšovi. Říkal: „Vzhůru! Pojďme! Obsadíme tu zemi a jistě se jí zmocníme.“
#13:31 Ale muži, kteří šli spolu s ním, tvrdili: „Nemůžeme vytáhnout proti tomu lidu, vždyť je silnější než my.“
#13:32 Pomluvami zhaněli Izraelcům zemi, kterou prozkoumali: „Země, kterou jsme při průzkumu prošli, je země, která požírá své obyvatele, a všechen lid, který jsme v ní spatřili, jsou muži obrovité postavy.
#13:33 Viděli jsme tam zrůdy - Anákovci totiž patří ke zrůdám - a zdálo se nám, že jsme nepatrní jako kobylky, vskutku jsme v jejich očích byli takoví.“ 
#14:1 Celá pospolitost se pozdvihla; dali se do křiku a lid tu noc proplakal.
#14:2 Všichni Izraelci reptali proti Mojžíšovi a Áronovi a celá pospolitost jim vyčítala: „Kéž bychom byli zemřeli v egyptské zemi nebo na této poušti! Kéž bychom zemřeli!
#14:3 Proč nás Hospodin přivedl do této země? Abychom padli mečem? Aby se naše ženy a děti staly kořistí? Nebude pro nás lépe vrátit se do Egypta?“
#14:4 I řekli si vespolek: „Ustanovme si náčelníka a vraťme se do Egypta!“
#14:5 Tu padli Mojžíš a Áron na tvář před celým shromážděním pospolitosti Izraelců.
#14:6 Jozue, syn Núnův, a Káleb, syn Jefunův, dva z těch, kdo dělali průzkum v zemi, roztrhli svá roucha
#14:7 a domlouvali celé pospolitosti Izraelců: „Země, kterou jsme při průzkumu procházeli, je země převelice dobrá.
#14:8 Jestliže nám Hospodin bude přát, uvede nás do této země a dá nám ji. Je to země oplývající mlékem a medem.
#14:9 Nechtějte se přece bouřit proti Hospodinu. Nebojte se lidu té země. Sníme je jako chleba. Jejich ochrana od nich odstoupila, kdežto s námi je Hospodin. Nebojte se jich!“
#14:10 Celá pospolitost však křičela, aby je ukamenovali. Vtom se všem Izraelcům ukázala při stanu setkávání Hospodinova sláva.
#14:11 Hospodin řekl Mojžíšovi: „Jak dlouho mě bude tento lid znevažovat? Jak dlouho mi nebude věřit přes všechna znamení, která jsem uprostřed něho konal?
#14:12 Budu ho bít morem a vydědím jej, a z tebe učiním větší a zdatnější národ, než je on.“
#14:13 Mojžíš však řekl Hospodinu: „Uslyší o tom Egypťané, neboť z jejich středu jsi vyvedl tento lid svou mocí,
#14:14 a budou o tom vykládat obyvatelům této země. Ti slyšeli, že ty, Hospodine, jsi uprostřed tohoto lidu, že ty, Hospodine, se zjevuješ tváří v tvář. Tvůj oblak stojí nad nimi, v sloupu oblakovém chodíš před nimi ve dne a v sloupu ohnivém v noci.
#14:15 Když tento lid do jednoho usmrtíš, pronárody, které slyšely o tobě zprávu, řeknou:
#14:16 ‚Protože Hospodin nebyl s to uvést tento lid do země, kterou přísežně zaslíbil, pobil je na poušti.‘
#14:17 Nuže, nechť se prosím vyvýší tvá moc, Panovníku, jak jsi prohlásil. Řekl jsi:
#14:18 ‚Hospodin je shovívavý a nesmírně milosrdný, odpouští vinu a přestupek, ale viníka nenechá bez trestu, vinu otců stíhá na synech do třetího i čtvrtého pokolení.‘
#14:19 Promiň prosím tomuto lidu vinu podle svého velikého milosrdenství, jako jsi mu ji odpouštěl od Egypta až sem.“
#14:20 Hospodin odvětil: „Na tvou přímluvu promíjím.
#14:21 Avšak, jakože jsem živ, Hospodinova sláva naplní celou zemi.
#14:22 Proto žádný z mužů, kteří viděli mou slávu a má znamení, jež jsem činil v Egyptě i na poušti, a pokoušeli mě už aspoň desetkrát a neposlouchali mě,
#14:23 nespatří zemi, kterou jsem přísežně zaslíbil jejich otcům; nespatří ji žádný, kdo mě znevažoval.
#14:24 Jen svého služebníka Káleba, protože byl jiného ducha a cele se mi oddal, uvedu do země, do níž vstoupil, a jeho potomstvo ji obsadí.
#14:25 V dolině sídlí Amálekovci a Kenaanci; zítra se tedy obraťte a táhněte pouští cestou k Rákosovému moři.“
#14:26 Hospodin promluvil k Mojžíšovi a Áronovi:
#14:27 „Jak dlouho mám snášet tuto zlou pospolitost, která proti mně stále reptá? Slyšel jsem reptání Izraelců, jak proti mně reptají.
#14:28 Vyřiď jim: Jakože jsem živ, je výrok Hospodinův, naložím s vámi tak, jak jste si o to říkali.
#14:29 Na této poušti padnou vaše mrtvá těla, vás všech povolaných do služby, kolik je vás všech od dvacetiletých výše, kdo jste proti mně reptali.
#14:30 Věru že nevejdete do země kromě Káleba, syna Jefunova, a Jozua, syna Núnova, ačkoliv jsem pozvedl ruku k přísaze, že v ní budete přebývat.
#14:31 Ale vaše děti, o nichž jste tvrdili, že se stanou kořistí, do ní uvedu, takže poznají zemi, kterou jste zavrhli.
#14:32 Avšak vy, vaše mrtvá těla padnou na této poušti
#14:33 a vaši synové budou pastevci na poušti po čtyřicet let. Ponesou následky vašeho smilstva, dokud vaše mrtvá těla nebudou do jednoho ležet na poušti.
#14:34 Podle počtu dnů, v nichž jste dělali průzkum země, ponesete své viny. Za každý den jeden rok, za čtyřicet dnů čtyřicet let. Tak pocítíte mou nevoli.
#14:35 Já Hospodin jsem promluvil. S celou touto zlou pospolitostí, která se proti mně spikla, naložím vskutku tímto způsobem: na této poušti do posledního vymřou.“
#14:36 Muži, které Mojžíš poslal prozkoumat zemi a kteří po návratu podněcovali celou pospolitost k reptání proti němu tím, že pohaněli tuto zemi pomluvami,
#14:37 ti muži, kteří pohaněli zemi zlými pomluvami, zemřeli zasaženi Hospodinem.
#14:38 Z oněch mužů, kteří šli prozkoumat zemi, zůstali naživu Jozue, syn Núnův, a Káleb, syn Jefunův.
#14:39 Když Mojžíš oznámil všem Izraelcům tato slova, začal lid velice truchlit.
#14:40 Za časného jitra vystoupili nahoru na pohoří. Řekli: „Jsme připraveni vystoupit k místu, o němž Hospodin mluvil. Víme, že jsme zhřešili.“
#14:41 Ale Mojžíš je varoval: „Proč zase přestupujete Hospodinův rozkaz? Nezdaří se vám to.
#14:42 Nevystupujte tam - vždyť ve vašem středu není Hospodin -, abyste neutrpěli porážku od svých nepřátel.
#14:43 Amálekovci a Kenaanci tam budou proti vám a vy padnete mečem, protože jste se odvrátili od Hospodina. Hospodin s vámi nebude.“
#14:44 Přesto opovážlivě vystoupili nahoru na pohoří; ale schrána Hospodinovy smlouvy ani Mojžíš se nehnuli ze středu tábora.
#14:45 Vtom sestoupili Amálekovci a Kenaanci, kteří sídlili v onom pohoří, a bili je a pobíjeli až do Chormy. 
#15:1 Hospodin promluvil k Mojžíšovi:
#15:2 „Mluv k Izraelcům a řekni jim: Až vejdete do země, kterou vám dávám, abyste v ní sídlili,
#15:3 a budete chtít připravit Hospodinu ohnivou oběť - oběť zápalnou nebo obětní hod - při plnění slibu nebo k dobrovolnému daru anebo o slavnostních shromážděních, a připravíte pro Hospodina libou vůni ze skotu nebo z bravu,
#15:4 tedy ten, kdo bude Hospodinu přinášet svůj dar, přinese jako oběť přídavnou desetinu éfy bílé mouky zadělané čtvrtinou hínu oleje.
#15:5 Na každého beránka připravíš čtvrtinu hínu vína k úlitbě, jak při oběti zápalné, tak při obětním hodu.
#15:6 Na berana připravíš jako oběť přídavnou dvě desetiny bílé mouky zadělané třetinou hínu oleje
#15:7 a vína k úlitbě třetinu hínu; to přineseš jako libou vůni pro Hospodina.
#15:8 Budeš-li připravovat dobytče v oběť zápalnou nebo k obětnímu hodu při plnění slibu nebo jako oběť pokojnou Hospodinu,
#15:9 přineseš spolu s dobytčetem jako oběť přídavnou tři desetiny bílé mouky zadělané polovinou hínu oleje
#15:10 a vína přineseš k úlitbě polovinu hínu. To bude k ohnivé oběti jako libá vůně pro Hospodina.
#15:11 Tak se to bude dělat při každém býku nebo beranu nebo při ovcích či kozách,
#15:12 podle toho, kolik jich budete obětovat. Tak to připravíte na každý kus z celkového počtu.
#15:13 Tak to bude dělat podle těchto ustanovení každý domorodec, když bude přinášet ohnivou oběť jako libou vůni pro Hospodina.
#15:14 A pokud někdo pobývá s vámi jako host nebo kdo žije mezi vámi už po několik pokolení a bude chtít připravit ohnivou oběť jako libou vůni pro Hospodina, bude dělat všechno, jako to děláte vy.
#15:15 Ve shromáždění bude platit totéž nařízení pro vás i pro toho, kdo bude pobývat mezi vámi jako host. To je před Hospodinem provždy platné nařízení pro všechna vaše pokolení, jak pro vás, tak pro hosta.
#15:16 Týmž řádem a touž směrnicí se budete řídit vy i host, který pobývá mezi vámi.“
#15:17 Hospodin dále mluvil k Mojžíšovi:
#15:18 „Mluv k Izraelcům a řekni jim: Až vejdete do země, do níž vás vedu,
#15:19 a budete jíst z pokrmů země, odevzdáte pro Hospodina oběť pozdvihování.
#15:20 Bochánek připravený z prvotin vaší obilní tluče odevzdáte jako oběť pozdvihování. Odevzdáte jej jako oběť pozdvihování z humna.
#15:21 Z prvotin své obilní tluče budete dávat Hospodinu oběť pozdvihování po všechna svá pokolení.“
#15:22 „Jestliže neúmyslně nedodržíte některý z příkazů, které udělil Hospodin Mojžíšovi,
#15:23 cokoli z toho, co vám Hospodin prostřednictvím Mojžíše přikázal ode dne, kdy začal Hospodin vydávat příkazy pro všechna vaše pokolení, až dosud,
#15:24 jestliže se to stalo neúmyslně nedopatřením pospolitosti, připraví celá pospolitost mladého býčka k zápalné oběti jako libou vůni pro Hospodina, i s příslušnou obětí přídavnou a úlitbou, jak je určeno, a jednoho kozla k oběti za hřích.
#15:25 Kněz vykoná za celou pospolitost Izraelců smírčí obřady, a bude jim odpuštěno. Stalo se to neúmyslně a oni přinesli svůj dar, ohnivou oběť pro Hospodina, i oběť za hřích před Hospodinem za to, co učinili neúmyslně.
#15:26 Celé pospolitosti Izraelců bude odpuštěno, i hostu, který přebývá mezi vámi, neboť všechen lid to učinil neúmyslně.
#15:27 Jestliže se jednotlivec prohřeší neúmyslně, přivede jednoroční kozu k oběti za hřích.
#15:28 Kněz vykoná smírčí obřady za toho, kdo se neúmyslně prohřešil, kdo se před Hospodinem dopustil něčeho neúmyslně; vykoná za něho smírčí obřady, a bude mu odpuštěno.
#15:29 Budete mít tentýž řád pro domorodce mezi Izraelci i pro hosta, který pobývá mezi vámi, když se někdo něčeho dopustí neúmyslně.
#15:30 Ale ten, kdo něco spáchá úmyslně, ať domorodec nebo host, hanobí Hospodina. Takový bude vyobcován ze společenství svého lidu.
#15:31 Pohrdl Hospodinovým slovem a porušil jeho příkaz. Takový musí být bezpodmínečně vyobcován, lpí na něm vina.“
#15:32 Když Izraelci prodlévali na poušti, přistihli muže, který v den odpočinku sbíral dříví.
#15:33 Ti, kdo jej přistihli, jak sbírá dříví, předvedli jej před Mojžíše a Árona a před celou pospolitost.
#15:34 Dali ho střežit, neboť nebylo zřejmé, co se s ním má stát.
#15:35 Hospodin řekl Mojžíšovi: „Ten muž musí zemřít. Celá pospolitost jej ukamenuje venku za táborem.“
#15:36 Celá pospolitost ho vyvedla ven za tábor a ukamenovali ho, takže zemřel, jak Hospodin Mojžíšovi přikázal.
#15:37 Hospodin řekl Mojžíšovi:
#15:38 „Mluv k Izraelcům a řekni jim, aby si po všechna pokolení dělali na okraji svých šatů třásně a nad třásně ať se dávají na okraj svého roucha purpurově fialovou stuhu.
#15:39 Budete mít třásně, abyste si při pohledu na ně připomínali všechna Hospodinova přikázání a plnili je, abyste se neřídili vlastním srdcem a vlastníma očima, jako se jimi řídí smilníci,
#15:40 abyste si připomínali a plnili všechna má přikázání a byli svatými pro svého Boha.
#15:41 Já jsem Hospodin, váš Bůh, já jsem vás vyvedl z egyptské země, abych vám byl Bohem. Já jsem Hospodin, váš Bůh.“ 
#16:1 Lévijec Kórach, syn Jishára, syna Kehatova, přibral Dátana a Abírama, syny Elíabovy, též Óna, syna Peletova, Rúbenovce,
#16:2 a s dvěma sty padesáti muži povstali proti Mojžíšovi; byli to Izraelci, předáci pospolitosti, kteří zastupovali lid při slavnostech, muži pověstní.
#16:3 Shromáždili se proti Mojžíšovi a Áronovi a vyčítali jim: „Příliš mnoho si osobujete. Celá pospolitost, všichni v ní jsou svatí a Hospodin je uprostřed nich. Proč se povznášíte nad Hospodinovo shromáždění?“
#16:4 Když to Mojžíš uslyšel, padl na tvář.
#16:5 Potom promluvil ke Kórachovi a celé jeho skupině: „Ráno oznámí Hospodin, kdo je jeho a kdo je svatý, komu tedy dovolí, aby k němu přistupoval. Koho vyvolí, tomu dovolí, aby k němu přistupoval.
#16:6 Udělejte toto: Vezměte si kadidelnice, Kórach a celá jeho skupina,
#16:7 a zítra do nich dejte oheň a vložte na ně před Hospodinem kadidlo. Muž, kterého vyvolí Hospodin, ten bude svatý. Příliš mnoho si osobujete, Léviovci!“
#16:8 Potom řekl Mojžíš Kórachovi: „Nuže poslyšte, Léviovci.
#16:9 Je vám to málo, že vás Bůh Izraele oddělil od pospolitosti Izraele a dovolil, abyste k němu přistupovali, vykonávali službu při Hospodinově příbytku, stáli před pospolitostí a přisluhovali jí?
#16:10 Dovolil přistupovat tobě i všem tvým bratřím Léviovcům s tebou. A vy se domáháte i kněžského úřadu.
#16:11 To znamená, že ty a celá tvoje skupina se srocujete proti Hospodinu. Co je Áron, že proti němu reptáte?“
#16:12 Nato dal Mojžíš předvolat Dátana a Abírama, syny Elíabovy. Odpověděli: „Nepřijdeme.
#16:13 Což je to málo, že jsi nás vyvedl ze země oplývající mlékem a medem, abys nás umořil na poušti? To se ještě opovažuješ dělat ze sebe nad námi velitele?
#16:14 Ještě jsi nás neuvedl do země oplývající mlékem a medem, ještě jsi nám nedal do dědictví pole ani vinice. To chceš vyloupnout těmto mužům oči? Nepřijdeme.“
#16:15 Mojžíš zlobně vzplanul a volal k Hospodinu: „Neobracej se k jejich obětním darům. Ani jediného osla jsem od nich nevzal a nikomu z nich jsem neudělal nic zlého.“
#16:16 Mojžíš potom Kórachovi nařídil: „Ty a celá tvoje skupina buďte zde před Hospodinem. Zítra tu budete, ty, oni i Áron.
#16:17 Každý si vezměte svou kadidelnici; dáte do ní kadidlo a přinesete každý svou kadidelnici před Hospodina, dvě stě padesát kadidelnic. Také ty a Áron, každý svou kadidelnici.“
#16:18 Každý si tedy vzal svou kadidelnici, dali do nich oheň, vložili do nich kadidlo a postavili se ke vchodu do stanu setkávání, i Mojžíš a Áron.
#16:19 Kórach shromáždil proti nim celou pospolitost ke vchodu do stanu setkávání. Vtom se celé pospolitosti ukázala Hospodinova sláva.
#16:20 Hospodin promluvil k Mojžíšovi a Áronovi:
#16:21 „Oddělte se od této pospolitosti. Chci s nimi rázně skoncovat.“
#16:22 Oba padli na tvář a volali: „Bože, Bože duchů veškerého tvorstva, jen jediný muž zhřešil a ty bys byl rozlícen na celou pospolitost?“
#16:23 Hospodin promluvil k Mojžíšovi:
#16:24 „Domluv pospolitosti: Vykliďte prostor kolem příbytku Kórachova, Dátanova a Abíramova!“
#16:25 Mojžíš se hned odebral k Dátanovi a Abíramovi a starší izraelští šli za ním.
#16:26 Promluvil k pospolitosti: „Odstupte od stanů těchto svévolných mužů a nedotýkejte se ničeho, co je jejich, abyste nebyli smeteni spolu se všemi jejich hříchy.“
#16:27 I vyklidili okolí příbytku Kórachova, Dátanova a Abíramova. Dátan a Abíram však vyšli a postavili se u vchodu do svých stanů se svými ženami, syny a dětmi.
#16:28 Mojžíš řekl: „Podle toho poznáte, že mě poslal Hospodin, abych činil všechny tyto skutky, a že nedělám nic z vlastní vůle:
#16:29 Jestliže tito lidé zemřou, jako umírá každý člověk, a postihne je obecný lidský úděl, neposlal mě Hospodin.
#16:30 Jestliže však Hospodin stvoří něco mimořádného a půda rozevře svůj chřtán a pohltí je se vším, co je jejich, takže sestoupí zaživa do podsvětí, poznáte, že tito muži znevážili Hospodina.“
#16:31 Sotva to všechno domluvil, rozpoltila se pod nimi půda,
#16:32 země otevřela svůj chřtán a pohltila je i jejich obydlí a všechny lidi, kteří byli s Kórachem, i všechen majetek.
#16:33 Sestoupili do podsvětí zaživa se vším, co bylo jejich, a země se nad nimi zavřela; zmizeli zprostředku shromáždění.
#16:34 Celý Izrael, který byl kolem nich, při jejich křiku utekl. Řekli si: „Aby i nás země nepohltila.“
#16:35 Od Hospodina pak vyšlehl oheň a pozřel těch dvě stě padesát mužů přinášejících kadidlo. 
#17:1 Hospodin promluvil k Mojžíšovi:
#17:2 „Řekni Eleazarovi, synu kněze Árona, ať vynese ze spáleniště kadidelnice, neboť jsou svaté, a oheň z nich dej rozmetat opodál.
#17:3 Z kadidelnic těch, kteří pro svůj hřích přišli o život, se udělají tepané pláty k obložení oltáře; přinesli je před Hospodina, a proto jsou svaté; budou pro Izraelce znamením.“
#17:4 Kněz Eleazar tedy vzal měděné kadidelnice, které přinesli ti, kdo byli spáleni, a vytepali z nich obložení k oltáři
#17:5 jako výstražnou připomínku pro Izraelce, aby nikdo nepovolaný, kdo by nepocházel z potomstva Áronova, nepřistupoval a nechtěl pálit před Hospodinem kadidlo, nebo dopadne jako Kórach a jeho skupina, jak mu to řekl Hospodin prostřednictvím Mojžíše.
#17:6 Druhého dne reptala celá pospolitost Izraelců proti Mojžíšovi a Áronovi: „Vy jste příčinou smrti Hospodinova lidu.“
#17:7 Když se však pospolitost shromáždila proti Mojžíšovi a Áronovi a obrátili se k stanu setkávání, hle, přikryl jej oblak a ukázala se Hospodinova sláva.
#17:8 I šel Mojžíš s Áronem před stan setkávání.
#17:9 Hospodin k Mojžíšovi promluvil:
#17:10 „Vzdalte se od této pospolitosti! Chci s nimi rázně skoncovat.“ Tu padli oba na tvář
#17:11 a Mojžíš řekl Áronovi: „Vezmi kadidelnici, dej do ní oheň z oltáře a vlož kadidlo a rychle jdi k pospolitosti. Vykonej za ně smírčí obřady, neboť již vyšlehl od Hospodina hrozný hněv, pohroma již začala.“
#17:12 Áron vzal kadidelnici, jak mu řekl Mojžíš, a běžel doprostřed shromáždění. Ale pohroma v lidu již začala. Zapálil kadidlo, aby vykonal smírčí obřady za lid.
#17:13 Postavil se mezi mrtvé a živé a pohroma se zastavila.
#17:14 Mrtvých při té pohromě bylo čtrnáct tisíc sedm set, kromě mrtvých v případě Kórachově.
#17:15 Pak se Áron vrátil k Mojžíšovi ke vchodu do stanu setkávání. Pohroma byla zastavena.
#17:16 Hospodin promluvil k Mojžíšovi:
#17:17 „Mluv k Izraelcům a vezmi od nich po holi za každý otcovský rod, od všech předáků dvanáct holí za jejich otcovské rody. Jméno každého napíšeš na jeho hůl.
#17:18 Áronovo jméno napíšeš na hůl Léviho; za každého představitele otcovského rodu bude jedna hůl.
#17:19 Uložíš je do stanu setkávání před schránu svědectví, tam, kde se s vámi setkávám.
#17:20 Hůl muže, kterého vyvolím, vypučí. Tak zkrotím reptání Izraelců, kteří proti vám reptají.“
#17:21 Mojžíš k Izraelcům promluvil a všichni jejich předáci mu odevzdali dvanáct holí, za každého předáka po holi za jeho otcovský rod; Áronova hůl byla uprostřed mezi jejich holemi.
#17:22 Mojžíš uložil hole ve stanu svědectví před Hospodinem.
#17:23 Když pak nazítří Mojžíš vešel do stanu svědectví, hle, Áronova hůl za dům Léviho vypučela, vyrazilo poupě, rozkvetl květ a dozrály mandle.
#17:24 Všechny hole vynesl Mojžíš od Hospodina ke všem Izraelcům, aby to viděli, a každý si vzal svou hůl.
#17:25 Hospodin řekl Mojžíšovi: „Dones Áronovu hůl zpátky před schránu svědectví, aby byla opatrována jako znamení pro vzpurné. Tak skoncuješ s jejich reptáním proti mně a nezemřou.“
#17:26 I učinil Mojžíš přesně tak, jak mu Hospodin přikázal.
#17:27 Ale Izraelci řekli Mojžíšovi: „Ach, zajdeme, zhyneme, my všichni zhyneme.
#17:28 Každý, kdo se jen přiblíží k Hospodinovu příbytku, zemře. Máme snad do jednoho zhynout?“ 
#18:1 Hospodin řekl Áronovi: „Ty a s tebou tvoji synové i celý tvůj rod budete odpovědni za každou nepravost ve svatyni; ty a s tebou tvoji synové budete odpovědni za každou nepravost svého kněžství.
#18:2 Také svým bratřím, pokolení Léviho, kmenu svého otce, dovolíš, aby přistupovali s tebou. Přidruží se k tobě a budou ti přisluhovat, kdežto ty a s tebou tvoji synové budete před stanem svědectví.
#18:3 Budou držet stráž u tebe i stráž při celém stanu. Ale k předmětům ve svatyni a k oltáři se nepřiblíží, aby nezemřeli ani oni ani vy.
#18:4 Přidruží se k tobě a budou držet stráž při stanu setkávání a konat při stanu veškerou službu. Nikdo nepovolaný se k vám nepřiblíží.
#18:5 Budete držet stráž při svatyni a stráž při oltáři, aby už na Izraelce nedolehl hrozný hněv.
#18:6 Já sám jsem vzal vaše bratry lévijce z Izraelců pro vás jako dar, jako darované Hospodinu, aby vykonávali službu při stanu setkávání.
#18:7 Ty však a s tebou tvoji synové budete dbát na své kněžské povinnosti ve všem, co se týká oltáře i služby uvnitř před oponou; tam budete sloužit. Vaši kněžskou službu vám dávám darem. Kdyby se přiblížil někdo nepovolaný, zemře.“
#18:8 Hospodin promluvil k Áronovi: „Hle, dávám ti na starost své oběti pozdvihování; dávám je provždy platným nařízením tobě i tvým synům jako příděl ze všech svatých darů Izraelců.
#18:9 Z nejsvětějších darů, z toho, co nepřijde na oheň, ti bude patřit toto: všechny jejich obětní dary při každé jejich oběti přídavné i při každé jejich oběti za hřích a za vinu, které mi budou dávat; nejsvětější dary budou patřit tobě a tvým synům.
#18:10 Na velesvatém místě je budeš jíst; bude je jíst každý mužského pohlaví; svaté patří tobě.
#18:11 Toto bude tvoje: oběť pozdvihování z jejich darů. Při všech obětech podávání, přinášených Izraelci, dávám ji tobě a s tebou i tvým synům a dcerám provždy platným nařízením; smí je jíst každý, kdo je v tvém domě čistý.
#18:12 Dávám ti všechen nejlepší čerstvý olej a všechen nejlepší mošt a obilí, jejich prvotiny, které budou odevzdávat Hospodinu.
#18:13 Tvoje budou rané plody ze všeho, co bude v jejich zemi, co budou přinášet Hospodinu; smí je jíst každý, kdo je v tvém domě čistý.
#18:14 Tvé bude i všechno, co je v Izraeli postiženo klatbou.
#18:15 Všechno, co z veškerého tvorstva otvírá lůno a co se přináší v oběť Hospodinu, jak z lidí, tak z dobytka, bude tvoje; avšak prvorozené z lidí bezpodmínečně vyplatíš, také vyplatíš prvorozené z nečistého dobytka.
#18:16 Budeš je vyplácet od stáří jednoho měsíce; výplatné se bude vyměřovat ve stříbře: pět šekelů podle váhy určené svatyní; jeden šekel je dvacet zrn.
#18:17 Nebudeš vyplácet jen prvorozené ze skotu nebo prvorozené z ovcí či z koz; je to svaté. Jejich krví pokropíš oltář a jejich tuk obrátíš v obětní dým. To je ohnivá oběť, libá vůně pro Hospodina.
#18:18 Jejich maso bude tvoje, stejně jako jsou tvoje hrudí z oběti podávání a pravá kýta.
#18:19 Všechny oběti pozdvihování svatých darů, které pozdvihují Izraelci k Hospodinu, dávám tobě a s tebou i tvým synům a dcerám provždy platným nařízením. Je to smlouva potvrzená solí, provždy platná před Hospodinem pro tebe i pro tvé potomstvo.“
#18:20 Hospodin řekl Áronovi: „V jejich zemi nebudeš mít mezi nimi dědictví ani podíl. Tvůj podíl i tvé dědictví uprostřed Izraelců jsem já.
#18:21 Léviovcům dávám v Izraeli za dědictví všechny desátky za jejich službu, neboť oni konají službu při stanu setkávání.
#18:22 Izraelci ať se už ke stanu setkávání nepřibližují, aby se neobtížili hříchem a nezemřeli.
#18:23 Službu při stanu setkávání budou konat pouze lévijci; sami budou odpovědni za každou svou nepravost. To je provždy platné nařízení pro všechna vaše pokolení. A nebudou mít uprostřed Izraelců dědictví.
#18:24 Zato dávám lévijcům za dědictví desátky Izraelců, které pozdvihují k Hospodinu v oběť pozdvihování. Proto jsem jim řekl, že nebudou mít uprostřed Izraelců dědictví.“
#18:25 Hospodin promluvil k Mojžíšovi:
#18:26 „Mluv k lévijcům a řekni jim: Když převezmete od Izraelců desátky, které vám od nich dávám do dědictví, budete z nich pozdvihovat v oběť Hospodinu desátý díl z desátku.
#18:27 To vám bude uznáno za vaši oběť pozdvihování, jako by šlo o obilí z humna a o šťávu z lisu.
#18:28 Tak budete pozdvihovat také vy oběť pozdvihování Hospodinu, totiž ze všech svých desátků, které od Izraelců přijmete, a z nich odevzdáte Hospodinovu oběť pozdvihování knězi Áronovi.
#18:29 Ze všech darů vám náležejících budete pozdvihovat každou oběť pozdvihování Hospodinu, svatý díl ze všeho nejlepšího.
#18:30 Dále jim řekneš: Když budete pozdvihovat to nejlepší z nich, bude to lévijcům uznáno jako výtěžek z humna a lisu.
#18:31 Smíte to jíst spolu se svými rodinami kdekoli, neboť to je vaše mzda za vaši službu při stanu setkávání.
#18:32 Neobtížíte se hříchem, když z toho budete pozdvihováním oddělovat to nejlepší, neznesvětíte svaté dary Izraelců a nezemřete.“ 
#19:1 Hospodin promluvil k Mojžíšovi a Áronovi:
#19:2 „Toto je nařízení zákona, které přikázal Hospodin: Mluv k Izraelcům, ať k tobě přivedou červenohnědou krávu bez vady a bez poskvrny, na niž dosud nebylo vloženo jho.
#19:3 Předáte ji knězi Eleazarovi; ten ji vyvede ven z tábora a za jeho přítomnosti bude poražena.
#19:4 Kněz Eleazar potom nabere na prst trochu její krve a sedmkrát stříkne krví směrem ke stanu setkávání.
#19:5 Kráva pak před jeho očima bude spálena; spálí se její kůže i maso a krev včetně výmětů.
#19:6 Potom vezme kněz cedrové dřevo s yzopem a karmínovým barvivem a hodí je na spalovanou krávu.
#19:7 Až si kněz vypere roucha a omyje se celý vodou, může vejít do tábora, ale bude nečistý až do večera.
#19:8 I ten, kdo ji spaluje, vypere si ve vodě roucha, celý se omyje vodou a bude nečistý až do večera.
#19:9 Někdo čistý sebere popel ze spálené krávy a uloží jej na čisté místo venku za táborem. Tam se bude uchovávat pro pospolitost Izraelců, pro přípravu očistné vody k očišťování od hříchu.
#19:10 I ten, kdo sebere popel z krávy, vypere si roucha a bude nečistý až do večera. To bude provždy platné nařízení pro Izraelce i pro hosta, který pobývá mezi nimi:
#19:11 Kdo by se dotkl kteréhokoli mrtvého člověka, bude nečistý po sedm dní.
#19:12 Třetího a sedmého dne se bude tou vodou očišťovat od hříchu a bude čistý. Jestliže se však nebude očišťovat od hříchu třetího a sedmého dne, nebude čistý.
#19:13 Každý, kdo by se dotkl mrtvého, člověka, který zemřel, a neočistil by se od hříchu, poskvrní Hospodinův příbytek. Takový bude z Izraele vyobcován, neboť nebyl pokropen očistnou vodou; je nečistý, jeho nečistota na něm zůstává lpět.
#19:14 Toto je řád: Když někdo zemře ve stanu, každý, kdo do toho stanu vstoupí, a každý, kdo v tom stanu je, bude nečistý po sedm dní.
#19:15 Také každá otevřená nádoba, pokud na ní není přivázána poklice, bude nečistá.
#19:16 Každý, kdo se na poli dotkne skoleného mečem nebo mrtvého anebo lidských kostí či hrobu, bude nečistý po sedm dní.
#19:17 Pro nečistého vezmou trochu prachu z krávy spálené k očišťování od hříchu a na něj se do nádoby nalije čerstvá voda.
#19:18 Někdo čistý vezme yzop, smočí v té vodě a stříkne na stan, na všechno nádobí i na všechny lidi, kteří tam jsou, i na toho, kdo se dotkl kostí nebo skoleného anebo mrtvého či hrobu.
#19:19 Třetího a sedmého dne stříkne čistý na nečistého, a tak jej očistí sedmého dne od hříchu. Ten si vypere roucho, omyje se vodou a večer bude čistý.
#19:20 Když se někdo znečistí a od hříchu se neočistí, bude vyobcován ze shromáždění, neboť poskvrnil Hospodinovu svatyni; nebyl pokropen očistnou vodou, je nečistý.
#19:21 To jim bude provždy platné nařízení. Ten, kdo bude stříkat očistnou vodou, vypere si roucha, a kdo se dotkne očistné vody, bude nečistý až do večera.
#19:22 Také vše, čeho se dotkne nečistý, bude nečisté, i člověk, který se ho dotkne, bude nečistý až do večera.“ 
#20:1 Celá pospolitost Izraelců dorazila v prvním měsíci na poušť Sin. Lid se usadil v Kádeši. Tam zemřela Mirjam a byla tam i pochována.
#20:2 Pospolitost neměla vody. Proto se srotili proti Mojžíšovi a Áronovi.
#20:3 Lid se dal s Mojžíšem do sváru. Naříkali: „Kéž bychom byli také zahynuli, když zahynuli naši bratří před Hospodinem!
#20:4 Proč jste zavedli Hospodinovo shromáždění na tuto poušť? Abychom tu pomřeli, my i náš dobytek?
#20:5 Proč jste nás vyvedli z Egypta? Abyste nás uvedli na toto zlé místo? Na místo, kde nelze sít obilí ani pěstovat fíky nebo vinnou révu či granátová jablka, ba není tady ani voda k napití.“
#20:6 I odešli Mojžíš a Áron od shromáždění ke vchodu do stanu setkávání a padli na tvář. Tu se jim ukázala Hospodinova sláva.
#20:7 Hospodin promluvil k Mojžíšovi:
#20:8 „Vezmi hůl, svolej spolu se svým bratrem Áronem pospolitost a před jejich očima promluvte ke skalisku, ať vydá vodu. Vyvedeš jim tak vodu ze skaliska a napojíš pospolitost i jejich dobytek.“
#20:9 Mojžíš tedy vzal hůl, která byla před Hospodinem, jak mu přikázal.
#20:10 I svolal Mojžíš s Áronem shromáždění před skalisko. Řekl jim: „Poslyšte, odbojníci! To vám z tohoto skaliska máme vyvést vodu?“
#20:11 Nato Mojžíš pozdvihl ruku, dvakrát udeřil svou holí do skaliska a vytryskl proud vody, takže se napila pospolitost i jejich dobytek.
#20:12 Hospodin však Mojžíšovi a Áronovi řekl: „Protože jste mi neuvěřili, když jste měli před syny Izraele dosvědčit mou svatost, neuvedete toto shromáždění do země, kterou jim dám.“
#20:13 To jsou Vody Meríba (to je Vody sváru), protože se Izraelci přeli s Hospodinem; on však mezi nimi prokázal svou svatost.
#20:14 Mojžíš vyslal z Kádeše posly k edómskému králi se žádostí: „Toto praví tvůj bratr Izrael: Ty znáš všechny útrapy, které nás potkaly.
#20:15 Naši otcové sestoupili do Egypta. V Egyptě jsme sídlili po dlouhý čas. Egypťané však s námi i s našimi otci nakládali zle.
#20:16 Úpěli jsme k Hospodinu a on nás vyslyšel. Seslal svého posla a vyvedl nás z Egypta. A tak jsme v Kádeši, v městě na pomezí tvého území.
#20:17 Nech nás prosím projít tvou zemí. Nepotáhneme přes pole ani vinice a nebudeme pít vodu ze studní. Půjdeme Královskou cestou a neodbočíme napravo ani nalevo, dokud neprojdeme tvým územím.“
#20:18 Ale Edóm mu řekl: „Mou zemí procházet nebudeš, jinak proti tobě vytáhnu s mečem.“
#20:19 Izraelci mu odpověděli: „Potáhneme po silnici. Jestliže budeme pít tvou vodu, my nebo náš dobytek, zaplatíme ji. O nic jiného nejde, než abychom mohli pěšky projít.“
#20:20 Edóm zase odmítl: „Procházet nebudeš.“ A vytáhl proti němu s početným lidem a s velkou mocí.
#20:21 Tak odmítl Edóm povolit Izraeli průchod svým územím, a Izrael se mu vyhnul.
#20:22 Z Kádeše táhli dál a celá pospolitost Izraelců dorazila k hoře Hóru.
#20:23 Na hoře Hóru na pomezí edómské země řekl Hospodin Mojžíšovi a Áronovi:
#20:24 „Áron bude připojen ke svému lidu a nevejde do země, kterou dám Izraelcům, protože jste jednali vzpurně proti mému rozkazu při Vodách sváru.
#20:25 Vyveď Árona a jeho syna Eleazara vzhůru na horu Hór.
#20:26 Svlékni Áronovi jeho roucho a oblec do něho jeho syna Eleazara. Áron bude připojen k svému lidu, zemře tam.“
#20:27 Mojžíš vykonal, co mu přikázal Hospodin. Před očima celé pospolitosti vystoupili na horu Hór.
#20:28 Mojžíš svlékl Áronovi jeho roucho a oblékl do něho jeho syna Eleazara. I zemřel tam Áron na vrcholu hory. Mojžíš s Eleazarem s hory pak sestoupili.
#20:29 Když celá pospolitost spatřila, že Áron zesnul, celý dům izraelský Árona oplakával po třicet dní. 
#21:1 Když uslyšel aradský král, Kenaanec sídlící v Negebu, že Izrael přichází cestou z Atárímu, dal se s Izraelem do boje a některé z nich zajal.
#21:2 Tu se Izrael zavázal Hospodinu slibem: „Jestliže skutečně vydáš do mých rukou tento lid, zničím jejich města jako klatá.“
#21:3 Hospodin Izraele vyslyšel a Kenaance mu vydal. Vyhubil je i s jejich městy jako klaté a nazval to místo Chorma (to je Klatbě propadlé).
#21:4 Z hory Hóru táhli dál cestou k Rákosovému moři, aby obešli edómskou zemi. Avšak lid propadl na té cestě malomyslnosti
#21:5 a mluvil proti Bohu a proti Mojžíšovi: „Proč jste nás vyvedli z Egypta? Abyste nás na poušti umořili? Vždyť tu není chléb ani voda! Tato nuzná strava se nám už protiví.“
#21:6 I poslal Hospodin na lid ohnivé hady. Ti lid štípali, takže v Izraeli mnoho lidí pomřelo.
#21:7 Lid přišel k Mojžíšovi a přiznával: „Zhřešili jsme, když jsme mluvili proti Hospodinu a proti tobě. Modli se k Hospodinu, aby nás těch hadů zbavil.“ Mojžíš se tedy za lid modlil.
#21:8 Hospodin Mojžíšovi řekl: „Udělej si hada Ohnivce a připevni ho na žerď. Když se na něj kterýkoli uštknutý podívá, zůstane naživu.“
#21:9 Mojžíš tedy udělal bronzového hada a připevnil ho na žerď. Jestliže někoho uštkl had a on pohlédl na hada bronzového, zůstal naživu.
#21:10 Izraelci táhli dál a utábořili se v Obótu.
#21:11 Pak vytáhli z Obótu a utábořili se v Ijé-abárímu, na poušti ležící na východ od Moábu.
#21:12 Odtud táhli dál a utábořili se v úvalu Zeredu.
#21:13 Odtud táhli dál a utábořili se na druhé straně Arnónu, na poušti vycházející z emorejského území; Arnón je totiž moábskou hranicí mezi Moábem a Emorejci.
#21:14 Proto se říká v Knize Hospodinových bojů: „Přemohl Vahéba v bouři i úvaly, Arnón
#21:15 a sráz úvalů, který spadá tam, co sídlí Ar, který se dotýká pomezí moábského.“
#21:16 Odtud táhli do Beéru. To je Beér (neboli Studna), kde řekl Hospodin Mojžíšovi: „Shromáždi lid a já jim dám vodu.“
#21:17 Tehdy zpíval Izrael tuto píseň: „Vytryskni, studnice! Zazpívejte o ní!
#21:18 Studnici kopali velmožové, hloubili ji urození z lidu palcátem a svými holemi.“ A z pouště táhli do Matany,
#21:19 z Matany do Nachalíelu, z Nachalíelu do Bamótu,
#21:20 z Bamótu do údolí v poli Moábském při vrcholu Pisgy, který vyčnívá nad pustinou Ješímónem.
#21:21 Izrael vyslal k emorejskému králi Síchonovi posly se vzkazem:
#21:22 „Chtěl bych projít tvou zemí. Neodbočíme do polí ani do vinic a nebudeme pít vodu ze studní. Půjdeme Královskou cestou, dokud neprojdeme tvým územím.“
#21:23 Síchon však Izraeli svým územím projít nedovolil, ale shromáždil všechen svůj lid a vytáhl proti Izraeli na poušť. Dorazil do Jahsy a bojoval s Izraelem.
#21:24 Izrael ho však pobil ostřím meče a obsadil jeho zemi od Arnónu po Jabok až k Amónovcům; hranice Amónovců byla totiž pevná.
#21:25 Izrael zabral všechna tato města. Tak se Izrael usadil ve všech emorejských městech i v Chešbónu a ve všech jeho vesnicích.
#21:26 Chešbón byl totiž městem Síchona, krále emorejského. Ten bojoval proti někdejšímu moábskému králi a odňal z jeho rukou všechnu jeho zemi až k Arnónu.
#21:27 Proto říkají průpovídkáři: „Vejděte do Chešbónu! Pěkně je vystavěno a opevněno Síchonovo město!
#21:28 Avšak oheň vyšlehne z Chešbónu, plamen ze Síchonovy tvrze, pozře Ar Moábský, baaly posvátných návrší Arnónu.
#21:29 Běda tobě, Moábe, jsi ztracen, lide Kemóšův! Učinil z jeho synů utečence, z jeho dcer zajatkyně Síchona, krále Emorejců.
#21:30 Ale my jsme jim vrhli los. Zanikl Chešbón po Díbón, zpustošili jsme jej po Nófach, tam u Médeby.“
#21:31 Tak se Izrael usadil v emorejské zemi.
#21:32 Mojžíš pak poslal zvědy, aby prošli Jaezer. Dobyli jeho vesnice a podrobili si Emorejce, kteří tam byli.
#21:33 Nato se obrátili a vystoupili cestou k Bášanu. Bášanský král Óg vytáhl se vším svým lidem k boji proti nim do Edreí.
#21:34 Hospodin však řekl Mojžíšovi: „Neboj se ho, neboť ho vydám do tvých rukou se vším jeho lidem i s jeho zemí. Naložíš s ním, jako jsi naložil se Síchonem, králem emorejským, sídlícím v Chešbónu.“
#21:35 A pobili ho i s jeho syny a všechen jeho lid, takže nikdo nevyvázl, a jeho zemi obsadili. 
#22:1 Izraelci táhli dál a utábořili se v Moábských pustinách v Zajordání naproti Jerichu.
#22:2 Balák, syn Sipórův, viděl všechno, co učinil Izrael Emorejcům.
#22:3 Moáb se velmi obával toho lidu, že je ho tak mnoho, Moáb měl z Izraelců hrůzu.
#22:4 Proto řekl Moáb midjánským starším: „Teď toto společenství spase všechno kolem nás, jako spásá vůl zeleň na poli.“ Toho času byl Balák, syn Sipórův, moábským králem.
#22:5 Ten vyslal posly k Bileámovi, synu Beórovu, do Petóru, který je nad Řekou, do země příslušníků jeho lidu, aby ho takto pozvali: „Hle, z Egypta vyšel lid a pokryl povrch země; usazuje se naproti mně.
#22:6 Pojď nyní prosím a proklej mi tento lid, neboť je zdatnější nežli já. Snad ho pak porazím a vyženu ze země. Vím, že komu ty požehnáš, je požehnán, a koho ty prokleješ, je proklet.“
#22:7 Moábští a midjánští starší šli s odměnou pro věštce v rukou. Přišli k Bileámovi a vyřídili mu Balákova slova.
#22:8 Vybídl je: „Tuto noc přenocujte zde; dám vám odpověď, až ke mně promluví Hospodin.“ Moábští velmožové zůstali tedy u Bileáma.
#22:9 Bůh přišel k Bileámovi a řekl: „Kdo jsou ti muži u tebe?“
#22:10 Bileám odpověděl Bohu: „Poslal je ke mně moábský král Balák, syn Sipórův, se vzkazem:
#22:11 ‚Hle, z Egypta vyšel lid a pokryl povrch země. Pojď nyní a zatrať mi jej, snad ho v boji přemohu a vyženu ho.‘“
#22:12 Bůh však Bileámovi poručil: „Nepůjdeš s nimi a neprokleješ ten lid, neboť je požehnaný!“
#22:13 Ráno Bileám vstal a řekl Balákovým velmožům: „Jděte do své země. Hospodin odmítl pustit mě s vámi.“
#22:14 Moábští velmožové se vydali na cestu, přišli k Balákovi a řekli: „Bileám odmítl s námi jít.“
#22:15 Balák tedy poslal znovu velmože, ve větším počtu a váženější než předchozí.
#22:16 Přišli k Bileámovi a řekli mu: „Toto praví Balák, syn Sipórův: Nijak se nerozpakuj ke mně přijít.
#22:17 Převelice tě poctím a učiním vše, co mi řekneš. Jen pojď a zatrať mi ten lid!“
#22:18 Bileám však Balákovým služebníkům odpověděl: „I kdyby mi Balák dal dům plný stříbra a zlata, nemohl bych přestoupit rozkaz Hospodina, svého Boha, a učinit cokoli, ať malého nebo velkého.
#22:19 Přesto prosím zůstaňte zde i této noci, ať zvím, co bude Hospodin dále se mnou mluvit.“
#22:20 V noci přišel Bůh k Bileámovi a řekl mu: „Když ti muži tě přišli pozvat, jdi s nimi. Ale budeš dělat jenom to, co ti poručím.“
#22:21 Bileám tedy ráno vstal, osedlal svou oslici a jel s moábskými velmoži.
#22:22 Vzplanul Bůh hněvem, že jede, a Hospodinův posel se mu postavil do cesty jako protivník. On pak jel na své oslici a byli s ním dva jeho mládenci.
#22:23 Oslice spatřila Hospodinova posla, jak stojí v cestě s taseným mečem v ruce, uhnula z cesty a šla polem. Bileám oslici bil, aby ji zase zavedl na cestu.
#22:24 Tu se postavil Hospodinův posel na pěšinu mezi vinicemi, kde byly zídky z obou stran.
#22:25 Oslice spatřila Hospodinova posla, přitiskla se ke zdi, přitiskla ke zdi i Bileámovu nohu a on ji znovu bil.
#22:26 Hospodinův posel opět přešel a postavil se v soutěsce, kde nebylo možno uhnout napravo ani nalevo.
#22:27 Oslice spatřila Hospodinova posla a klekla pod Bileámem. Bileám vzplanul hněvem a bil oslici holí.
#22:28 Tu otevřel Hospodin oslici ústa a ona řekla Bileámovi: „Co jsem ti udělala, že mě již potřetí biješ?“
#22:29 Bileám oslici odpověděl: „Protože si ze mne děláš blázny! Mít v ruce meč, byl bych tě už zabil.“
#22:30 Oslice Bileámovi odpověděla: „Což nejsem tvá oslice, na níž jezdíš odjakživa až do dneška? Udělala jsem ti někdy něco takového?“ Řekl: „Ne.“
#22:31 I sňal Hospodin clonu z Bileámových očí a on spatřil Hospodinova posla, jak stojí v cestě s taseným mečem v ruce. Poklonil se a padl na tvář.
#22:32 Hospodinův posel mu řekl: „Proč jsi svou oslici už třikrát bil? Hle, to já jsem vyšel jako tvůj protivník, protože mám tu cestu za pochybenou.
#22:33 Oslice mě spatřila a vyhnula se mi, teď už potřetí. Kdyby se mi nevyhnula, byl bych tě věru zabil a ji nechal naživu.“
#22:34 Bileám odvětil Hospodinovu poslu: „Zhřešil jsem. Nevěděl jsem, že ty ses mi postavil do cesty. Avšak jestliže se ti to nelíbí, vrátím se zpátky.“
#22:35 Hospodinův posel odpověděl Bileámovi: „Jdi s těmi muži, ale nebudeš mluvit nic než to, co já budu mluvit k tobě.“ Bileám tedy šel s Balákovými velmoži.
#22:36 Sotva Balák uslyšel, že Bileám přichází, vyšel mu vstříc do moábského města, které leží při Arnónu na samých hranicích.
#22:37 Balák řekl Bileámovi: „Copak jsem pro tebe neposílal naléhavě a nezval tě? Proč jsi ke mně nepřišel? Což tě nejsem s to náležitě uctít?“
#22:38 Bileám odpověděl Balákovi: „Teď jsem k tobě přišel. Budu však vůbec s to něco promluvit? Budu moci mluvit jen to, co mi vloží do úst Bůh.“
#22:39 Pak šel Bileám s Balákem a přišli do Kirjat-chusótu.
#22:40 Balák dal připravit k obětním hodům skot a brav a poslal pro Bileáma i pro velmože, kteří byli s ním.
#22:41 Za jitra pak vzal Balák Bileáma a vyvedl ho na Baalova posvátná návrší. Odtamtud bylo vidět zadní voj lidu. 
#23:1 Bileám řekl Balákovi: „Vybuduj zde sedm oltářů a přiveď mi sem sedm býků a sedm beranů.“
#23:2 Balák učinil, oč ho Bileám požádal. Pak Balák s Bileámem obětovali na každém oltáři býka a berana.
#23:3 Bileám řekl Balákovi: „Postav se u své zápalné oběti a já půjdu, snad se Hospodin se mnou setká. Co mi ukáže, to ti oznámím.“ Potom odešel na holé návrší.
#23:4 A Bůh se s Bileámem setkal. Ten mu řekl: „Připravil jsem sedm oltářů a obětoval jsem na každém oltáři býka a berana.“
#23:5 I vložil Hospodin Bileámovi do úst slovo a řekl: „Vrať se k Balákovi a mluv takto!“
#23:6 Vrátil se tedy k němu a on stál u své zápalné oběti se všemi moábskými velmoži.
#23:7 I pronesl svou průpověď: „Z Aramu přivedl mě Balák, z východních hor moábský král: Pojď mi proklít Jákoba, pojď zaklínat Izraele!
#23:8 Jak mám zatratit, když Bůh nezatracuje? Jak mám zaklínat, když Hospodin nezaklíná?
#23:9 Vidím ho z temene skal, z pahorků na něj hledím: Je to lid, který přebývá odděleně, nepočítá se mezi pronárody.
#23:10 Kdo sečte prach Jákobův, kdo spočítá byť jen čtvrtinu Izraele? Kéž umřu smrtí lidí přímých, kéž je můj konec jako jeho!“
#23:11 Tu se Balák na Bileáma osopil: „Cos mi to provedl? Vzal jsem tě, abys mé nepřátele zatratil, a ty jen žehnáš!“
#23:12 On však odpověděl: „Což nemusím dbát toho, abych mluvil, co mi vkládá do úst Hospodin?“
#23:13 Balák mu řekl: „Pojď prosím se mnou na jiné místo, odkud bys jej viděl. Uvidíš zase jen jeho zadní voj, neuvidíš jej celý. Zatratíš mi jej odtamtud.“
#23:14 Vzal ho na pole hlídek, na vrchol Pisgy; vybudoval sedm oltářů a obětoval na každém oltáři býka a berana.
#23:15 Bileám řekl Balákovi: „Postav se zde u své zápalné oběti, já se pokusím setkat s Hospodinem tamto.“
#23:16 A Hospodin se setkal s Bileámem. Vložil mu do úst slovo a řekl: „Vrať se k Balákovi a mluv takto!“
#23:17 Když k němu přišel, on stál u své zápalné oběti a s ním moábští velmožové. Balák se ho tázal: „Co mluvil Hospodin?“
#23:18 I pronesl svou průpověď: „Nuže, slyš, Baláku! Pozorně mi naslouchej, Sipórův synu!
#23:19 Bůh není člověk, aby lhal, ani lidský syn, aby litoval. Zdali řekne, a neučiní, promluví, a nedodrží?
#23:20 Hle, dostal jsem úkol žehnat. On dal požehnání a já to nezvrátím.
#23:21 Nehledí na kouzla proti Jákobovi, nedbá těch, kdo přejí bídu Izraeli. Hospodin, jeho Bůh, je s ním, hlaholí to v něm královským holdem.
#23:22 Bůh, který jej vyvedl z Egypta, je mu jako rohy jednorožců.
#23:23 Proti Jákobovi není zaklínadla, proti Izraeli není věštby. Od tohoto času bude hlásáno o Jákobovi, zvěstováno o Izraeli, co mu Bůh prokázal.
#23:24 Jaký to lid! Povstává jako lvice, zvedá se jako lev. Neulehne, dokud nezhltne úlovek, dokud nevypije krev skolených!“
#23:25 Balák řekl Bileámovi: „Když jej nemůžeš zatratit, aspoň mu nežehnej!“
#23:26 Ale Bileám odpověděl Balákovi: „Což jsem ti neříkal: Všechno, co bude mluvit Hospodin, vykonám?“
#23:27 Balák řekl Bileámovi: „Pojď prosím, vezmu tě na jiné místo, snad bude mít Bůh za správné, abys mi jej zatratil odtamtud.“
#23:28 I vzal Balák Bileáma na vrchol Peóru, který ční nad pustinou Ješímónem.
#23:29 Bileám řekl Balákovi: „Vybuduj zde sedm oltářů a přiveď mi sem sedm býků a sedm beranů.“
#23:30 Balák tedy učinil, co řekl Bileám, a obětoval na každém oltáři býka a berana. 
#24:1 Bileám viděl, že se Hospodinu líbí, aby Izraeli žehnal. Neuchýlil se tedy jako podvakrát předtím k zaklínadlům, nýbrž obrátil se tváří k poušti.
#24:2 Když se Bileám rozhlédl, spatřil, jak Izrael táboří kmen vedle kmene. Tu se ho zmocnil Boží duch
#24:3 a on pronesl svou průpověď: „Výrok Bileáma, syna Beórova, výrok muže jasnozřivého,
#24:4 výrok toho, jenž slyší řeč Boží, jenž mívá vidění od Všemocného; když upadá do vytržení, má odkryté oči.
#24:5 Jak skvělé jsou tvé stany, Jákobe, tvé příbytky, Izraeli!
#24:6 Rozprostírají se jako úvaly, jako zahrady nad řekou, jako vonné stromoví vysázené Hospodinem, jako cedry při vodách.
#24:7 Z jeho věder se řinou vody, jeho símě je hojně zavlažováno, jeho král bude vyvýšen nad Agaga, jeho království bude povzneseno.
#24:8 Bůh, který ho vyvedl z Egypta, je mu jako rohy jednorožců. Zhltne pronárody, své protivníky, rozhryže jim kosti, protkne je svými šípy.
#24:9 Stočil se a lehl jako lev, jak lvice. Donutí ho někdo, aby povstal? Kdo ti bude žehnat, buď požehnán, kdo tě bude proklínat, buď proklet.“
#24:10 Tu Balák vzplanul proti Bileámovi hněvem. Tleskl rukama a osopil se na Bileáma: „Povolal jsem tě, abys mé nepřátele zatratil, a ty jim jen žehnáš, dokonce již potřetí.
#24:11 Tak si běž, odkud jsi přišel! Přislíbil jsem ti převelikou poctu, avšak Hospodin tě té pocty zbavil.“
#24:12 Bileám odvětil Balákovi: „Což jsem neříkal už tvým poslům, které jsi ke mně poslal:
#24:13 I kdyby mi Balák dal svůj dům plný stříbra a zlata, nemohl bych přestoupit Hospodinův rozkaz a učinit z vlastní vůle něco dobrého nebo zlého. Budu mluvit, co řekne Hospodin!
#24:14 A teď tedy jdu ke svému lidu. Ale ještě ti sdělím, co tento lid v budoucnu učiní tvému lidu.“
#24:15 I pronesl svou průpověď: „Výrok Bileáma, syna Beórova, výrok muže jasnozřivého,
#24:16 výrok toho, jenž slyší řeč Boží, jenž má poznání od Nejvyššího, jenž mívá vidění od Všemocného; když upadá do vytržení, má odkryté oči.
#24:17 Vidím jej, ne však přítomného, hledím na něj, ne však zblízka. Vyjde hvězda z Jákoba, povstane žezlo z Izraele. Protkne spánky Moába, témě všech Šétovců.
#24:18 Bude podroben i Edóm, podroben bude Seír, Jákobovi nepřátelé. Izrael si povede zdatně.
#24:19 Panovat bude ten, jenž vzejde z Jákoba, a zahubí toho, kdo vyvázne z města.“
#24:20 Když uviděl Amáleka, pronesl svou průpověď: „Amálek je prvotina pronárodů, ale skončí v záhubě.“
#24:21 Když uviděl Kénijce, pronesl svou průpověď: „Tvé sídlo je mohutné, tvé hnízdo leží na skalisku,
#24:22 stejně však přijde vniveč, Kajine, jen co tě Ašúr odvede do zajetí!“
#24:23 Pak pronesl svou průpověď: „Běda! Kdo zůstane naživu, až to Bůh vykoná?
#24:24 Loďstvo od kitejských břehů pokoří Ašúra, pokoří Ebera. Ale v záhubě skončí i ono.“
#24:25 I vydal se Bileám na cestu a vrátil se, odkud přišel. Rovněž Balák šel svou cestou. 
#25:1 Když Izrael pobýval v Šitímu, začal lid smilnit s Moábkami.
#25:2 Zvaly totiž lid k obětním hodům svého božstva a lid hodoval a klaněl se jejich božstvu.
#25:3 Tak se Izrael spřáhl s Baal-peórem. Proto Hospodin vzplanul proti Izraeli hněvem.
#25:4 I řekl Hospodin Mojžíšovi: „Vezmi všechny představitele lidu a dej jim na slunci zpřerážet údy kvůli Hospodinu, ať se Hospodinův planoucí hněv odvrátí od Izraele.“
#25:5 Mojžíš tedy izraelským soudcům řekl: „Pobijte každý ze svých mužů ty, kdo se spřáhli s Baal-peórem.“
#25:6 I přišel jakýsi Izraelec a přivedl ke svým bratřím Midjánku před očima Mojžíše i celé pospolitosti Izraelců, právě když plakali u vchodu do stanu setkávání.
#25:7 Když to spatřil Pinchas, syn Eleazara, syna kněze Árona, vystoupil zprostředku pospolitosti, vzal do ruky oštěp,
#25:8 vešel za izraelským mužem do ženské části stanu a probodl je oba, izraelského muže i tu ženu až k jejím rodidlům. Tak byla mezi Izraelci pohroma zastavena.
#25:9 Ale mrtvých při té pohromě bylo čtyřiadvacet tisíc.
#25:10 Nato promluvil Hospodin k Mojžíšovi:
#25:11 „Pinchas, syn Eleazara, syna kněze Árona, odvrátil mé rozhořčení od Izraelců tím, že dal uprostřed nich průchod mé žárlivosti, takže jsem ve své žárlivosti s Izraelci neskoncoval.
#25:12 Proto vyhlas: Hle, daruji mu svou smlouvu pokoje.
#25:13 Ta bude pro něho i pro jeho potomstvo smlouvou trvalého kněžství za to, že horlil pro svého Boha. Bude za Izraelce vykonávat smírčí obřady.“
#25:14 Jméno zabitého izraelského muže, který byl zabit spolu s Midjánkou, bylo Zimrí, syn Sálův; byl to předák jednoho rodu Šimeónova.
#25:15 Jméno zabité midjánské ženy bylo Kozbí, dcera Súrova; ten byl kmenový náčelník jednoho rodu v Midjánsku.
#25:16 Hospodin dále mluvil k Mojžíšovi:
#25:17 „Napadni Midjánce a pobijte je.
#25:18 Oni napadli vás svými nástrahami, které vám nastražili prostřednictvím Peóra a své sestry Kozbí, dcery midjánského náčelníka, zabité v den pohromy, která vznikla kvůli Peórovi.“
#25:19 Po té pohromě 
#26:1 řekl Hospodin Mojžíšovi a Eleazarovi, synu kněze Árona:
#26:2 „Pořiďte soupis celé pospolitosti Izraelců, od dvacetiletých výše podle otcovských rodů, všech, kteří jsou v Izraeli schopni vycházet do boje.“
#26:3 I promluvil k nim Mojžíš s knězem Eleazarem na Moábských pustinách u Jordánu naproti Jerichu
#26:4 a pořídili soupis od dvacetiletých výše, jak přikázal Hospodin Mojžíšovi. Toto byli Izraelci, kteří vyšli z egyptské země:
#26:5 Rúben, Izraelův prvorozený. Rúbenovci: Chanók a od něho chanócká čeleď, od Palúa palúská čeleď,
#26:6 od Chesróna chesrónská čeleď, od Karmího karmíjská čeleď.
#26:7 To jsou rúbenské čeledi; povolaných do služby z nich bylo čtyřicet tři tisíce sedm set třicet. -
#26:8 Palúovci: Eliáb
#26:9 a Eliábovi synové: Nemúel, Dátan a Abíram; to byl ten Dátan a Abíram, zástupci pospolitosti, kteří se vzepřeli proti Mojžíšovi a Áronovi spolu se skupinou Kórachovou, když se ona vzepřela proti Hospodinu.
#26:10 Země otevřela svůj chřtán a pohltila je i Kóracha. Tehdy ta skupina zhynula a oheň pozřel dvě stě padesát mužů. Stali se výstražným znamením.
#26:11 Kórachovi synové však nezemřeli.
#26:12 Šimeónovci podle čeledí: od Nemúela nemúelská čeleď, od Jámina jáminská čeleď, od Jákina jákinská čeleď,
#26:13 od Zeracha zerašská čeleď, od Šaula šaulská čeleď.
#26:14 To jsou čeledi šimeónské; bylo jich dvacet dva tisíce dvě stě.
#26:15 Gádovci podle čeledí: od Sefóna sefónská čeleď, od Chagího chagíjská čeleď, od Šúniho šúnijská čeleď,
#26:16 od Ozního ozníjská čeleď, od Ériho érijská čeleď,
#26:17 od Aróda aródská čeleď, od Aréliho arélijská čeleď.
#26:18 To jsou čeledi Gádovců s jejich povolanými do služby; těch bylo čtyřicet tisíc pět set.
#26:19 Judovi synové byli též Er a Ónan; Er a Ónan však zemřeli v kenaanské zemi.
#26:20 Judovci podle čeledí byli: od Šély šélská čeleď, od Peresa pereská čeleď, od Zeracha zerašská čeleď.
#26:21 Peresovci byli: od Chesróna chesrónská čeleď, od Chamúla chamúlská čeleď.
#26:22 To jsou čeledi Judovy s jejich povolanými do služby; těch bylo sedmdesát šest tisíc pět set.
#26:23 Isacharovci podle čeledí: Tóla a od něho tólská čeleď, od Púvy púvská čeleď,
#26:24 od Jašúba jašúbská čeleď, od Šimróna šimrónská čeleď.
#26:25 To jsou čeledi Isacharovy s jejich povolanými do služby; těch bylo šedesát čtyři tisíce tři sta.
#26:26 Zabulónovci podle čeledí: od Sereda seredská čeleď, od Elóna elónská čeleď, od Jachleela jachleelská čeleď.
#26:27 To jsou čeledi zabulónské s jejich povolanými do služby; těch bylo šedesát tisíc pět set.
#26:28 Josefovi synové podle čeledí: Manases a Efrajim.
#26:29 Manasesovci: od Makíra makírská čeleď. Makír zplodil Gileáda; od Gileáda gileádská čeleď.
#26:30 To jsou Gileádovci: Íezer a od něho íezerská čeleď, od Cheleka chelecká čeleď,
#26:31 Asríel a od něho asríelská čeleď, Šekem a od něho šekemská čeleď,
#26:32 Šemída a od něho šemídská čeleď, Chefer a od něho cheferská čeleď.
#26:33 Selofchad, syn Cheferův, neměl syny, jenom dcery; jména Selofchadových dcer jsou: Machla, Nóa, Chogla, Milka a Tirsa.
#26:34 To jsou čeledi Manasesovy s jejich povolanými do služby; těch bylo padesát dva tisíce sedm set.
#26:35 To jsou Efrajimovci podle čeledí: od Šútelacha šútelašská čeleď, od Bekera bekerská čeleď, od Tachana tachanská čeleď.
#26:36 To jsou Šútelachovci: od Erána eránská čeleď.
#26:37 To jsou čeledi Efrajimovců s jejich povolanými do služby; těch bylo třicet dva tisíce pět set. To jsou Josefovci podle čeledí.
#26:38 Benjamínovci podle čeledí: od Bely belská čeleď, od Ašbela ašbelská čeleď, od Achírama achíramská čeleď,
#26:39 od Šefúfama šúfamská čeleď, od Chúfama chúfamská čeleď.
#26:40 Belovci byli: Ard a Naamán; od Arda ardská čeleď, od Naamána naamánská čeleď.
#26:41 To jsou Benjamínovci podle čeledí; povolaných do služby z nich bylo čtyřicet pět tisíc šest set.
#26:42 Danovci podle čeledí: od Šúchama šúchamská čeleď. To jsou čeledi Danovy podle čeledí.
#26:43 Všech šúchamských čeledí s jejich povolanými do služby bylo šedesát čtyři tisíce čtyři sta.
#26:44 Ašerovci podle čeledí: od Jimny jimenská čeleď, od Jišvího jišvíjská čeleď, od Beríji beríjská čeleď.
#26:45 Od Beríovců také: od Chebera cheberská čeleď, od Malkíela malkíelská čeleď.
#26:46 Jméno Ašerovy dcery bylo Serach.
#26:47 To jsou čeledi Ašerovců s jejich povolanými do služby; těch bylo padesát tři tisíce čtyři sta.
#26:48 Neftalíovci podle čeledí: od Jachseela jachseelská čeleď, od Gúního gúníjská čeleď,
#26:49 od Jesera jeserská čeleď, od Šiléma šilémská čeleď.
#26:50 To jsou čeledi Neftalíovy podle čeledí; povolaných do služby z nich bylo čtyřicet pět tisíc čtyři sta.
#26:51 Toto byli Izraelci povolaní do služby, celkem šest set jeden tisíc sedm set třicet.
#26:52 Hospodin promluvil k Mojžíšovi:
#26:53 „Těm bude rozdělena země v dědictví podle jmenného seznamu.
#26:54 Většímu přidělíš větší dědictví, menšímu menší dědictví, každému bude dáno dědictví podle počtu povolaných do služby.
#26:55 Země bude rozdělena losem; budou dědit podle jmen otcovských pokolení.
#26:56 Dědictví ať je každému rozděleno losem, jak velikému, tak malému.“
#26:57 To jsou povolaní do služby z pokolení Léviho podle čeledí: od Geršóna geršónská čeleď, od Kehata kehatská čeleď, od Merarího meraríjská čeleď.
#26:58 To jsou čeledi Léviho: čeleď libníjská, čeleď chebrónská, čeleď machlíjská, čeleď mušíjská a čeleď kórachovská. Kehat zplodil Amráma.
#26:59 Jméno Amrámovy ženy bylo Jókebed, dcera Léviho, která se narodila Lévimu v Egyptě. Porodila Amrámovi Árona a Mojžíše a jejich sestru Mirjam.
#26:60 Áronovi se narodil Nádab, Abíhú, Eleazar a Ítamar.
#26:61 Nádab a Abíhú však zemřeli, když přinesli cizí oheň před Hospodina.
#26:62 Povolaných do služby, všech mužského pohlaví od jednoměsíčních výše, bylo dvacet tři tisíce. Nebyli připočteni mezi Izraelce, neboť jim mezi Izraelci nebylo dáno dědictví.
#26:63 To byli ti, které povolal do služby Mojžíš s knězem Eleazarem při sčítání Izraelců na Moábských pustinách u Jordánu naproti Jerichu.
#26:64 Nebyl mezi nimi nikdo z těch, kteří byli povoláni do služby tehdy, když Mojžíš s knězem Áronem sčítal Izraelce na Sínajské poušti.
#26:65 Hospodin jim totiž řekl, že určitě zemřou na poušti. Nezbyl z nich žádný, jenom Káleb, syn Jefunův, a Jozue, syn Núnův. 
#27:1 Tehdy přistoupily dcery Selofchada, syna Chefera, syna Gileáda, syna Makíra, syna Manasesa z čeledi Manasesa, syna Josefova. Jména jeho dcer byla: Machla, Nóa, Chogla, Milka a Tirsa.
#27:2 Postavily se před Mojžíše a před kněze Eleazara i před předáky a celou pospolitost ke vchodu do stanu setkávání a řekly:
#27:3 „Náš otec zemřel na poušti, ale nebyl ve skupině těch, kdo se smluvili proti Hospodinu, ve skupině Kórachově; umřel za svůj hřích a neměl syny.
#27:4 Proč má být vymazáno jméno našeho otce z jeho čeledi proto, že nemá syna? Dej nám trvalé vlastnictví mezi bratry našeho otce.“
#27:5 Mojžíš tedy předložil jejich právní záležitost Hospodinu.
#27:6 Hospodin řekl Mojžíšovi:
#27:7 „Selofchadovy dcery mluví o tom oprávněně. Přiděl jim dědičně trvalé vlastnictví mezi bratry jejich otce a převeď na ně dědictví po jejich otci.
#27:8 Mluv k Izraelcům takto: Když někdo zemře a nemá syna, převedete dědictví po něm na jeho dceru.
#27:9 Jestliže nemá ani dceru, dáte dědictví po něm jeho bratrům.
#27:10 Jestliže nemá ani bratry, dáte dědictví po něm bratrům jeho otce.
#27:11 Jestliže nejsou ani bratři po jeho otci, dáte dědictví po něm jeho nejbližšímu příbuznému z jeho čeledi; ten je obdrží.“ To se stalo pro Izraelce právním nařízením, jak přikázal Hospodin Mojžíšovi.
#27:12 Hospodin řekl Mojžíšovi: „Vystup na toto pohoří Abárím a pohlédni na zemi, kterou jsem dal Izraelcům.
#27:13 Až ji uvidíš, budeš i ty připojen ke svému lidu, tak jako k němu byl připojen tvůj bratr Áron.
#27:14 Vzepřeli jste se mému rozkazu na poušti Sinu, když se pospolitost pustila se mnou do sváru, místo abyste dosvědčili před jejich zraky při těch vodách mou svatost.“ To jsou kádešské Vody sváru na poušti Sinu.
#27:15 Mojžíš promluvil k Hospodinu:
#27:16 „Nechť Hospodin, Bůh duchů všech tvorů, ustanoví nad pospolitostí muže,
#27:17 který by před nimi vycházel a vcházel a který by je vyváděl a přiváděl, aby Hospodinova pospolitost nebyla jako ovce bez pastýře.“
#27:18 I řekl Hospodin Mojžíšovi: „Vezmi k sobě Jozua, syna Núnova, muže, v němž je duch, a vlož na něho ruku.
#27:19 Postavíš ho před kněze Eleazara i před celou pospolitost a dáš mu před jejich zraky pověření.
#27:20 Předáš mu díl své velebnosti, aby ho celá pospolitost synů izraelských poslouchala.
#27:21 Bude se stavět před kněze Eleazara a ten se bude pro něho Hospodina vyptávat na rozhodnutí pomocí urímu. Na jeho rozkaz budou vycházet a na jeho rozkaz budou vcházet, on a s ním všichni Izraelci, tedy celá pospolitost.“
#27:22 Mojžíš učinil, co mu Hospodin přikázal. Vzal Jozua a postavil ho před kněze Eleazara i před celou pospolitost,
#27:23 vložil na něho ruce a dal mu pověření, jak o tom mluvil Hospodin skrze Mojžíše. 
#28:1 Hospodin promluvil k Mojžíšovi:
#28:2 „Přikaž Izraelcům a řekni jim: Dbejte na to, abyste mi ve stanovený čas přinášeli darem můj pokrm jako ohnivou oběť, libou vůni pro mne.
#28:3 Řekneš jim: Toto bude ohnivá oběť, kterou budete přinášet Hospodinu: denně dva roční beránky bez vady jako každodenní zápalnou oběť.
#28:4 Jednoho beránka připravíš ráno a druhého navečer.
#28:5 K tomu jako oběť přídavnou desetinu éfy bílé mouky zadělané čtvrtinou hínu odkapaného oleje.
#28:6 Taková každodenní zápalná oběť byla připravena v libou vůni na hoře Sínaji. To bude ohnivá oběť pro Hospodina.
#28:7 Příslušná úlitba k tomu bude čtvrtina hínu na jednoho beránka; úlitba opojného nápoje se pro Hospodina vykoná ve svatyni.
#28:8 Navečer připravíš druhého beránka, připravíš ho stejně jako ráno s obětí přídavnou a příslušnou úlitbou. To bude ohnivá oběť, libá vůně pro Hospodina.
#28:9 V den odpočinku budete obětovat dva roční beránky bez vady a jako oběť přídavnou dvě desetiny bílé mouky zadělané olejem i příslušnou úlitbu.
#28:10 To bude pravidelná zápalná oběť pro den odpočinku, kromě každodenní zápalné oběti a příslušné úlitby.
#28:11 Na počátku vašich měsíců budete přinášet jako zápalnou oběť Hospodinu dva mladé býčky, jednoho berana, sedm ročních beránků bez vady,
#28:12 jako oběť přídavnou tři desetiny bílé mouky zadělané olejem na každého býčka, dvě desetiny bílé mouky zadělané olejem jako oběť přídavnou na každého berana
#28:13 a po desetině bílé mouky zadělané olejem jako oběť přídavnou na každého beránka. Zápalná oběť bude připravena v libou vůni jako ohnivá oběť pro Hospodina.
#28:14 Příslušná úlitba bude polovina hínu vína na býčka, třetina hínu na berana a čtvrtina hínu na beránka. To bude zápalná oběť o novoluní, po všechny měsíce v roce.
#28:15 Mimo každodenní zápalnou oběť a příslušnou úlitbu bude Hospodinu také připraven jeden kozel v oběť za hřích.
#28:16 V prvním měsíci, čtrnáctého dne toho měsíce, bude Hospodinův hod beránka.
#28:17 Patnáctého dne toho měsíce bude slavnost; po sedm dní se budou jíst nekvašené chleby.
#28:18 Prvního dne bude bohoslužebné shromáždění. Nebudete vykonávat žádnou všední práci.
#28:19 Jako ohnivou oběť přinesete Hospodinu v oběť zápalnou dva mladé býčky, jednoho berana a sedm ročních beránků; budou bez vady.
#28:20 Připravíte též jako příslušnou oběť přídavnou bílou mouku zadělanou olejem, tři desetiny na býčka, dvě desetiny na berana,
#28:21 a na každého beránka z těch sedmi připravíš po desetině.
#28:22 Dále jednoho kozla jako oběť za hřích k vykonání smírčích obřadů za sebe.
#28:23 To vše budete připravovat kromě jitřní zápalné oběti, která patří ke každodenní oběti zápalné.
#28:24 Budete to připravovat denně po sedm dní jako pokrm ohnivé oběti v libou vůni pro Hospodina, a to kromě každodenní oběti zápalné a příslušné úlitby.
#28:25 Sedmého dne pak budete mít bohoslužebné shromáždění. Nebudete vykonávat žádnou všední práci.
#28:26 V den prvních snopků, když o svátcích týdnů budete přinášet Hospodinu novou oběť přídavnou, budete mít bohoslužebné shromáždění. Nebudete vykonávat žádnou všední práci.
#28:27 Přinesete jako oběť zápalnou v libou vůni pro Hospodina dva mladé býčky, jednoho berana, sedm ročních beránků
#28:28 a jako příslušnou oběť přídavnou bílou mouku zadělanou olejem, tři desetiny na každého býčka, dvě desetiny na každého berana
#28:29 a po desetině na každého beránka z těch sedmi.
#28:30 Dále jednoho kozla k vykonání smírčích obřadů za sebe.
#28:31 To připravíte kromě každodenní oběti zápalné a příslušné oběti přídavné; budou bez vady. Přinesete též příslušné úlitby. 
#29:1 Sedmého měsíce, prvého dne toho měsíce, budete mít bohoslužebné shromáždění. Nebudete vykonávat žádnou všední práci. Bude to pro vás den troubení.
#29:2 Připravíte jako oběť zápalnou v libou vůni pro Hospodina jednoho mladého býčka, jednoho berana, sedm ročních beránků bez vady
#29:3 a jako příslušnou oběť přídavnou bílou mouku zadělanou olejem, tři desetiny na býčka, dvě desetiny na berana
#29:4 a jednu desetinu na každého beránka z těch sedmi.
#29:5 Dále jednoho kozla jako oběť za hřích k vykonání smírčích obřadů za sebe.
#29:6 To kromě novolunní oběti zápalné a příslušné oběti přídavné i každodenní oběti zápalné a příslušné oběti přídavné i úliteb, jak jsou určeny. To bude ohnivá oběť v libou vůni pro Hospodina.
#29:7 Desátého dne toho sedmého měsíce budete mít bohoslužebné shromáždění a budete se pokořovat. Nebudete vykonávat žádnou práci.
#29:8 Přinesete jako oběť zápalnou pro Hospodina v libou vůni jednoho mladého býčka, jednoho berana a sedm ročních beránků; budou bez vady.
#29:9 Též jako příslušnou oběť přídavnou bílou mouku zadělanou olejem, tři desetiny na býčka, dvě desetiny na každého berana
#29:10 a po desetině na každého beránka z těch sedmi.
#29:11 Dále jednoho kozla jako oběť za hřích, kromě smírčí oběti za hřích a každodenní oběti zápalné a příslušné oběti přídavné i úliteb.
#29:12 Patnáctého dne sedmého měsíce budete mít bohoslužebné shromáždění. Nebudete vykonávat žádnou všední práci, ale po sedm dní budete slavit slavnost Hospodinu.
#29:13 Přinesete jako zápalnou oběť oběť ohnivou, libou vůni pro Hospodina, třináct mladých býčků, dva berany a čtrnáct ročních beránků; budou bez vady.
#29:14 Též jako příslušnou oběť přídavnou bílou mouku zadělanou olejem, tři desetiny na každého býčka z těch třinácti, dvě desetiny na každého berana z obou
#29:15 a po desetině na každého beránka z těch čtrnácti.
#29:16 Dále jednoho kozla jako oběť za hřích, kromě každodenní oběti zápalné, příslušné oběti přídavné a úlitby.
#29:17 Druhého dne dvanáct mladých býčků, dva berany a čtrnáct ročních beránků bez vady,
#29:18 též příslušnou oběť přídavnou a úlitby podle počtu býčků, beranů a beránků, jak je určeno.
#29:19 Dále jednoho kozla jako oběť za hřích, kromě každodenní oběti zápalné a příslušné oběti přídavné i úliteb.
#29:20 Třetího dne jedenáct býčků, dva berany a čtrnáct ročních beránků bez vady,
#29:21 též příslušnou oběť přídavnou a úlitby podle počtu býčků, beranů a beránků, jak je určeno.
#29:22 Dále jednoho kozla jako oběť za hřích, kromě každodenní oběti zápalné a příslušné oběti přídavné i úlitby.
#29:23 Čtvrtého dne deset býčků, dva berany a čtrnáct ročních beránků bez vady,
#29:24 též příslušnou oběť přídavnou a úlitby podle počtu býčků, beranů a beránků, jak je určeno.
#29:25 Dále jednoho kozla jako oběť za hřích, kromě každodenní oběti zápalné a příslušné oběti přídavné i úlitby.
#29:26 Pátého dne devět býčků, dva berany a čtrnáct ročních beránků bez vady,
#29:27 též příslušnou oběť přídavnou a úlitby podle počtu býčků, beranů a beránků, jak je určeno.
#29:28 Dále jednoho kozla jako oběť za hřích, kromě každodenní oběti zápalné a příslušné oběti přídavné i úlitby.
#29:29 Šestého dne osm býčků, dva berany a čtrnáct ročních beránků bez vady,
#29:30 též příslušnou oběť přídavnou a úlitby podle počtu býčků, beranů a beránků, jak je určeno.
#29:31 Dále jednoho kozla jako oběť za hřích, kromě každodenní oběti zápalné a příslušné oběti přídavné i úliteb.
#29:32 Sedmého dne sedm býčků, dva berany a čtrnáct ročních beránků bez vady,
#29:33 též příslušnou oběť přídavnou a úlitby podle počtu býčků, beranů a beránků, jak je určeno.
#29:34 Dále jednoho kozla jako oběť za hřích, kromě každodenní oběti zápalné a příslušné oběti přídavné i úlitby.
#29:35 Osmého dne budete mít slavnostní shromáždění. Nebudete vykonávat žádnou všední práci.
#29:36 Jako zápalnou oběť přinesete oběť ohnivou, libou vůni pro Hospodina: jednoho býčka, jednoho berana a sedm ročních beránků bez vady,
#29:37 též příslušnou oběť přídavnou a úlitby na býčka, na berana a na beránky podle jejich počtu, jak je určeno.
#29:38 Dále jednoho kozla jako oběť za hřích, kromě každodenní oběti zápalné a příslušné oběti přídavné i úlitby.
#29:39 To budete konat pro Hospodina v časech vám stanovených, kromě svých slibů a dobrovolných darů při vašich obětech zápalných i přídavných, při úlitbách a obětech pokojných.“ 
#30:1 Mojžíš řekl Izraelcům všechno, jak mu Hospodin přikázal.
#30:2 Mojžíš promluvil k představitelům pokolení Izraelců: „Hospodin přikázal tuto věc:
#30:3 Když učiní slib Hospodinu muž nebo složí přísahu, že na sebe vezme nějaký závazek, nezruší své slovo. Splní vše, co vlastními ústy pronesl.
#30:4 Když učiní slib Hospodinu žena a zaváže se k něčemu v otcovském domě ve svém dívčím věku
#30:5 a její otec uslyší její slib a závazek, který na sebe vzala, a bude k tomu mlčet, tedy budou všechny její sliby platit, i každý závazek, který na sebe vzala, bude platný.
#30:6 Jestliže však jí otec zabrání toho dne, kdy to uslyší, nebudou platit žádné její sliby a závazky, které na sebe vzala. Hospodin jí odpustí, protože jí v tom otec zabránil.
#30:7 Jestliže se vdá a bude ji vázat slib, nebo se o své újmě vlastními ústy k něčemu zavázala
#30:8 a její muž o tom uslyší a bude k tomu mlčet toho dne, kdy se o tom doslechne, budou její sliby platit, i závazek, který na sebe vzala, bude platný.
#30:9 Jestliže však jí zabrání její muž toho dne, kdy to uslyší, a zruší její slib, který ji váže, i to, k čemu se o své újmě vlastními ústy zavázala, Hospodin jí odpustí.
#30:10 Slib vdovy a zapuzené ženy, každý závazek, který na sebe vzala, bude pro ni platit.
#30:11 Jestliže něco slíbila nebo se k něčemu přísežně zavázala žena v domě svého muže
#30:12 a její muž to slyšel a mlčel k tomu a nezabránil jí, budou platit všechny její sliby, i každý závazek, který na sebe vzala, bude platný.
#30:13 Jestliže je však její muž toho dne, kdy je slyšel, výslovně zrušil, tedy vše, co pronesla svými rty jako svůj slib nebo závazek, který na sebe vzala, pozbývá platnosti. Její muž je zrušil a Hospodin jí odpustí.
#30:14 Každý slib i každý přísežný závazek, že se bude pokořovat, její muž buď nechá platit nebo jej zruší.
#30:15 Jestliže však její muž před ní trvale mlčí den ze dne, nechává platit všechny její sliby nebo všechny závazky, které ji zavazují; učinil je platnými tím, že před ní mlčel toho dne, kdy je slyšel.
#30:16 Jestliže je však výslovně zruší teprve později, než je slyšel, ponese následky její nepravosti on.“
#30:17 To jsou nařízení, která přikázal Hospodin Mojžíšovi pro vztah mezi mužem a jeho ženou, mezi otcem a jeho dcerou v dívčím věku, dokud je v otcovském domě. 
#31:1 Hospodin promluvil k Mojžíšovi:
#31:2 „Vykonáš za syny Izraele pomstu na Midjáncích a pak budeš připojen ke svému lidu.“
#31:3 Mojžíš tedy promluvil k lidu: „Vyzbrojte své muže k boji, aby se vrhli na Midjána a vykonali na Midjánu Hospodinovu pomstu.
#31:4 Pošlete do boje z každého izraelského pokolení po tisíci mužích.“
#31:5 I dodali z izraelských šiků vždy tisíc mužů na pokolení, celkem to bylo dvanáct tisíc vyzbrojených k boji.
#31:6 Mojžíš je poslal do boje po tisíci na pokolení a s nimi Pinchasa, syna kněze Eleazara, se svatou zbrojí a válečnými trubkami.
#31:7 Vypravili se do boje proti Midjánu, jak přikázal Hospodin Mojžíšovi, a pobili všechny muže.
#31:8 Zabili také midjánské krále Evího, Rekema, Súra, Chúra a Rebaa, pět midjánských králů, mimo ostatní skolené. I Bileáma, syna Beórova, zabili mečem.
#31:9 Midjánské ženy a jejich děti však Izraelci zajali a všechen jejich dobytek, všechna stáda a veškeré bohatství uloupili.
#31:10 Všechna jejich města, ve kterých sídlili, i všechna jejich hradiště vypálili.
#31:11 Pak vzali všechnu kořist i všechen lup, lidi i dobytek,
#31:12 a přivedli zajatce i lup a kořist před Mojžíše a kněze Eleazara i před pospolitost Izraelců do tábora na Moábských pustinách u Jordánu naproti Jerichu.
#31:13 Mojžíš a kněz Eleazar i všichni předáci pospolitosti jim vyšli vstříc ven z tábora.
#31:14 Mojžíš se však rozlítil na vůdce vojska, na velitele nad tisíci a nad sty, přicházející z vojenské výpravy.
#31:15 Vytkl jim: „Cože jste nechali naživu všechny ženy?
#31:16 Právě ony daly Izraelcům podnět k věrolomnosti vůči Hospodinu ve věci Peórově podle slov Bileámových, takže na pospolitost Hospodinovu dolehla pohroma.
#31:17 Nyní zabijte z dětí všechny mužského pohlaví i každou ženu, která poznala muže a obcovala s ním.
#31:18 Všechny děti ženského pohlaví a ženy, jež nepoznaly muže a neobcovaly s ním, nechte naživu pro sebe.
#31:19 A vy se utáboříte na sedm dní venku za táborem. Každý, kdo někoho zabil, i každý, kdo se dotkl skoleného, bude se očišťovat od hříchu třetího a sedmého dne, jak vy, tak vaši zajatci.
#31:20 Očistíte od hříchu i každé roucho a každou koženou nádobu, každý výrobek z kozích kůží i všechno dřevěné nářadí.“
#31:21 Kněz Eleazar řekl bojovníkům, kteří se účastnili bitvy: „Toto je nařízení zákona, který vydal Hospodin Mojžíšovi:
#31:22 Zlato, stříbro, měď, železo, cín a olovo,
#31:23 vše, co snese oheň, přepálíte ohněm a bude to čisté; musí se to však očistit od hříchu také očistnou vodou. Všechno, co nesnese oheň, properete vodou.
#31:24 Sedmého dne si vyperete roucha a budete čisti. Potom vstoupíte do tábora.“
#31:25 Hospodin řekl Mojžíšovi:
#31:26 „Ty s knězem Eleazarem a s představiteli rodů pospolitosti pořídíš soupis lupu, jak zajatých lidí, tak dobytka.
#31:27 Lup rozdělíš na polovinu mezi válečníky, kteří se účastnili boje, a ostatní pospolitost.
#31:28 Od bojovníků, kteří se účastnili boje, oddělíš dávku pro Hospodina po jednom z pěti set, jak z lidí, tak i skotu, oslů, ovcí a koz.
#31:29 Vezmete je z jejich poloviny a dáš je knězi Eleazarovi jako Hospodinovu oběť pozdvihování.
#31:30 Z poloviny patřící Izraelcům vezmeš po jednom ukořistěném z padesáti, jak z lidí, tak ze skotu, oslů, ovcí a koz a ze všeho dobytka, a dáš je lévijcům, držícím stráž u Hospodinova příbytku.“
#31:31 Mojžíš a kněz Eleazar vykonali, co Hospodin přikázal Mojžíšovi.
#31:32 Z lupu naloupeného válečným lidem zbývalo: ovcí a koz šest set sedmdesát pět tisíc,
#31:33 skotu sedmdesát dva tisíce kusů,
#31:34 oslů šedesát jeden tisíc;
#31:35 lidských duší, totiž žen, které nepoznaly muže a neobcovaly s ním, bylo celkem třicet dva tisíce.
#31:36 Polovina tvořící podíl těch, kdo se účastnili boje, činila tři sta třicet sedm tisíc pět set ovcí a koz.
#31:37 Dávka z ovcí a koz pro Hospodina činila šest set sedmdesát pět kusů.
#31:38 Skotu bylo třicet šest tisíc kusů a dávka pro Hospodina dvaasedmdesát kusů.
#31:39 Oslů bylo třicet tisíc pět set a dávka pro Hospodina jedenašedesát kusů.
#31:40 Lidských duší bylo šestnáct tisíc a dávka pro Hospodina dvaatřicet osob.
#31:41 Mojžíš tedy dal knězi Eleazarovi tuto dávku jako Hospodinovu oběť pozdvihování, jak Hospodin Mojžíšovi přikázal.
#31:42 Z poloviny patřící Izraelcům, kterou oddělil Mojžíš od poloviny pro účastníky tažení -
#31:43 a byla to polovina patřící pospolitosti, totiž ovcí a koz tři sta třicet sedm tisíc pět set,
#31:44 skotu třicet šest tisíc kusů,
#31:45 oslů třicet tisíc pět set,
#31:46 lidských duší šestnáct tisíc -
#31:47 tedy z této poloviny patřící Izraelcům vzal Mojžíš po jednom ukořistěném z padesáti, jak z lidí, tak z dobytka, a dal je lévijcům, držícím stráž u Hospodinova příbytku, jak Hospodin Mojžíšovi přikázal.
#31:48 Tu přistoupili k Mojžíšovi vůdcové vojenských útvarů, velitelé nad tisíci a nad sty,
#31:49 a řekli mu: „Tvoji služebníci pořídili soupis bojovníků, kterým veleli, a nechybí ani jeden z nás.
#31:50 Každý přinášíme Hospodinu jako dar, co kdo našel, zlaté předměty, spony a náramky, pečetní prsteny, náušnice a přívěsky, abychom za sebe vykonali před Hospodinem smírčí obřady.“
#31:51 I přijal Mojžíš a kněz Eleazar od nich to zlato, vesměs umně zpracované.
#31:52 Všeho zlata věnovaného na oběť pozdvihování, jež odevzdali Hospodinu, bylo šestnáct tisíc sedm set padesát šekelů, totiž od velitelů nad tisíci a nad sty.
#31:53 Z mužstva loupil každý pro sebe.
#31:54 Přijal tedy Mojžíš a kněz Eleazar to zlato od velitelů nad tisíci a nad sty a vnesli je do stanu setkávání, aby připomínalo před Hospodinem syny Izraele. 
#32:1 Rúbenovci a Gádovci měli převeliké množství stád. Když spatřili jaezerskou a gileádskou zemi, shledali, že to je vhodné místo pro stáda.
#32:2 Gádovci a Rúbenovci tedy přišli a řekli Mojžíšovi, knězi Eleazarovi a předákům pospolitosti:
#32:3 „Atarót, Díbón, Jaezer, Nimra, Chešbón, Eleále, Sebám, Nebó a Beón,
#32:4 země, kterou Hospodin před izraelskou pospolitostí porazil, je zemí vhodnou pro stáda. A tvoji služebníci mají stáda.“
#32:5 A dodali: „Jestliže jsme u tebe nalezli přízeň, nechť je tvým služebníkům přidělena do vlastnictví tato země. Nevoď nás přes Jordán.“
#32:6 Mojžíš Gádovcům a Rúbenovcům vytkl: „Což vaši bratří půjdou do boje a vy si zůstanete tady?
#32:7 Proč berete Izraelcům odvahu přejít do země, kterou jim dal Hospodin?
#32:8 Zrovna tak se zachovali vaši otcové, když jsem je poslal z Kádeš-barneje, aby zhlédli zemi.
#32:9 Došli až do úvalu Eškólu a zhlédli zemi, ale Izraelcům vzali odvahu, takže nechtěli vejít do země, kterou jim dal Hospodin.
#32:10 Onoho dne vzplanul Hospodin hněvem a zapřisáhl se:
#32:11 Muži ve věku od dvaceti let výše, kteří vyšli z Egypta, neuvidí zemi, kterou jsem přísežně zaslíbil Abrahamovi, Izákovi a Jákobovi, protože se mi cele neoddali,
#32:12 kromě Kenazejce Káleba, syna Jefunova, a Jozua, syna Núnova; ti se totiž cele oddali Hospodinu.
#32:13 Tehdy vzplanul Hospodin proti Izraeli hněvem a čtyřicet let jim dal bloudit po poušti, dokud nezaniklo celé to pokolení, které se dopouštělo toho, co je zlé v Hospodinových očích.
#32:14 A teď jste nastoupili vy na místo svých otců, zplozenci hříšných lidí, abyste ještě stupňovali Hospodinův hněv planoucí proti Izraeli.
#32:15 Když se od něho odvrátíte, nechá Izraele na poušti ještě déle. Tak uvedete na všechen tento lid zkázu.“
#32:16 Přistoupili k němu a prohlásili: „Vystavíme zde ohrady pro svá stáda a města pro své děti.
#32:17 My však potáhneme ve zbroji v čele Izraelců, dokud je neuvedeme na jejich místo. Naše děti zatím zůstanou v městech opevněných proti obyvatelům této země.
#32:18 Nenavrátíme se do svých domovů, dokud se všichni Izraelci neujmou svého dědictví.
#32:19 Nebudeme požadovat dědictví s nimi tam někde za Jordánem, když nám připadne dědictví na této straně Jordánu k východu.“
#32:20 Mojžíš jim odpověděl: „Jestliže dodržíte toto slovo, vyzbrojíte se před Hospodinem do boje
#32:21 a všichni ve zbroji přejdete před Hospodinem Jordán, potom až on vyžene své nepřátele
#32:22 a země bude před Hospodinem podmaněna, vrátíte se a budete bez viny u Hospodina i u Izraele. Pak bude tato země vaším vlastnictvím před Hospodinem.
#32:23 Nezachováte-li se však takto, prohřešíte se proti Hospodinu. Vězte, že se vám váš hřích vymstí.
#32:24 Vystavějte si města pro své děti a ohrady pro své ovce a dodržte, co jste slíbili.“
#32:25 Gádovci a Rúbenovci Mojžíšovi odpověděli: „Tvoji služebníci učiní, jak náš pán přikazuje.
#32:26 Naše děti, ženy, stáda a všechen dobytek budou zde v gileádských městech.
#32:27 Ale tvoji služebníci přejdou Jordán, všichni vojensky vyzbrojeni před Hospodinem, aby bojovali, jak praví náš pán.“
#32:28 I vydal Mojžíš o nich rozkaz knězi Eleazarovi a Jozuovi, synu Núnovu, i představitelům otcovských rodů pokolení Izraelců.
#32:29 Řekl jim: „Překročí-li Gádovci a Rúbenovci s vámi Jordán, všichni ve zbroji, aby bojovali před Hospodinem, a země bude před vámi podmaněna, dáte jim do vlastnictví zemi gileádskou.
#32:30 Nepřekročí-li jej však s vámi ve zbroji, ať mají vlastnictví mezi vámi v zemi kenaanské.“
#32:31 Gádovci a Rúbenovci odpověděli: „Učiníme, jak promluvil k tvým služebníkům Hospodin.
#32:32 Ano, přejdeme ve zbroji před Hospodinem do země kenaanské, aby nám zůstalo ve vlastnictví naše dědictví v Zajordání.“
#32:33 Mojžíš jim tedy dal, totiž Gádovcům, Rúbenovcům a polovině kmene Manasesa, syna Josefova, království emorejského krále Síchona a království bášanského krále Óga, celou tu zemi, její města s přilehlým územím, všechna města té země kolem.
#32:34 Gádovci pak znovu vystavěli Díbón, Atarót a Aróer,
#32:35 Atarót-šófan, Jaezer a Jogbehu,
#32:36 Bét-nimru a Bét-háran, opevněná města a ohrady pro ovce.
#32:37 Rúbenovci znovu vystavěli Chešbón, Eleále a Kirjátajim,
#32:38 Nebó a Baal-meón, jimž změnili jména, a Sibmu; městům, jež vystavěli, dali jména.
#32:39 Synové Makíra, syna Manasesova, táhli do Gileádu a dobyli jej. Podrobili si Emorejce, kteří v něm sídlili.
#32:40 Mojžíš dal Gileád Makírovi, synu Manasesovu, a ten se tam usadil.
#32:41 Zatím Jaír, syn Manasesův, dobyl na svém tažení jejich vesnice a nazval je vesnicemi Jaírovými.
#32:42 A Nóbach dobyl na svém tažení Kenat i jeho osady a nazval je svým jménem Nóbach. 
#33:1 Toto byla stanoviště Izraelců, když vytáhli z egyptské země po oddílech pod vedením Mojžíše a Árona.
#33:2 Na Hospodinův rozkaz sepsal Mojžíš místa, odkud vycházeli, podle jejich stanovišť. Toto byla tedy jejich stanoviště podle výchozích bodů:
#33:3 V prvním měsíci, patnáctého dne prvého měsíce, vytáhli z Ramesesu. Druhého dne po hodu beránka vyšli Izraelci navzdory všemu před očima celého Egypta,
#33:4 zatímco Egypťané pohřbívali ty, které mezi nimi Hospodin pobil, všechny prvorozence. Hospodin totiž vykonal soudy nad jejich božstvy.
#33:5 Izraelci tedy vytáhli z Ramesesu a utábořili se v Sukótu.
#33:6 Vytáhli ze Sukótu a utábořili se v Étamu, který je na okraji pouště.
#33:7 Vytáhli z Étamu, obrátili se zpět k Pí-chírótu, který leží naproti Baal-sefónu, a utábořili se před Migdólem.
#33:8 Vytáhli z Pené-chírótu a prošli prostředkem moře na poušť. Vykonali třídenní pochod pouští Étamem a utábořili se v Maře.
#33:9 Vytáhli z Mary a přišli do Élimu. V Élimu bylo dvanáct pramenů a sedmdesát palem; tam se utábořili.
#33:10 Vytáhli z Élimu a utábořili se u Rákosového moře.
#33:11 Vytáhli od Rákosového moře a utábořili se na Sínské poušti.
#33:12 Vytáhli ze Sínské pouště a utábořili se v Dofce.
#33:13 Vytáhli z Dofky a utábořili se v Alúši.
#33:14 Vytáhli z Alúše a utábořili se v Refídímu; tam nebyla voda, aby se lid mohl napít.
#33:15 Vytáhli z Refídímu a utábořili se na Sínajské poušti.
#33:16 Vytáhli ze Sínajské pouště a utábořili se v Kibrót-taavě.
#33:17 Vytáhli z Kibrót-taavy a utábořili se v Chaserótu.
#33:18 Vytáhli z Chaserótu a utábořili se v Ritmě.
#33:19 Vytáhli z Ritmy a utábořili se v Rimón--peresu.
#33:20 Vytáhli z Rimón-peresu a utábořili se v Libně.
#33:21 Vytáhli z Libny a utábořili se v Rise.
#33:22 Vytáhli z Risy a utábořili se v Kehelatě.
#33:23 Vytáhli z Kehelaty a utábořili se na hoře Šeferu.
#33:24 Vytáhli od hory Šeferu a utábořili se v Charadě.
#33:25 Vytáhli z Charady a utábořili se v Makhelótu.
#33:26 Vytáhli z Makhelótu a utábořili se v Tachatu.
#33:27 Vytáhli z Tachatu a utábořili se v Terachu.
#33:28 Vytáhli z Terachu a utábořili se v Mitce.
#33:29 Vytáhli z Mitky a utábořili se v Chašmóně.
#33:30 Vytáhli z Chašmóny a utábořili se v Moserótu.
#33:31 Vytáhli z Moserótu a utábořili se v Bené-jaakánu.
#33:32 Vytáhli z Bené-jaakánu a utábořili se v Chór-gidgádu.
#33:33 Vytáhli z Chór-gidgádu a utábořili se v Jotbatě.
#33:34 Vytáhli z Jotbaty a utábořili se v Abróně.
#33:35 Vytáhli z Abróny a utábořili se v Esjón-geberu.
#33:36 Vytáhli z Esjón-geberu a utábořili se na poušti Sinu; to je Kádeš.
#33:37 Vytáhli z Kádeše a utábořili se na hoře Hóru na okraji Edómské pouště.
#33:38 Tehdy na Hospodinův rozkaz vystoupil kněz Áron na horu Hór a zemřel tam čtyřicátého roku po vyjití Izraelců z egyptské země, pátého měsíce, prvého dne toho měsíce.
#33:39 Áronovi bylo sto třiadvacet let, když na hoře Hóru zemřel.
#33:40 Tu uslyšel Kenaanec, aradský král, který sídlil v Negebu v kenaanské zemi, že přicházejí Izraelci.
#33:41 Vytáhli od hory Hóru a utábořili se v Salmóně.
#33:42 Vytáhli ze Salmóny a utábořili se ve Funónu.
#33:43 Vytáhli z Funónu a utábořili se v Obótu.
#33:44 Vytáhli z Obótu a utábořili se v Ijé-abárímu na moábském území.
#33:45 Vytáhli z Ijímu a utábořili se v Dibón-gádu.
#33:46 Vytáhli z Dibón-gádu a utábořili se v Almón-diblátajimu.
#33:47 Vytáhli z Almón-diblátajimu a utábořili se v horách Abárímských proti Nebó.
#33:48 Vytáhli z hor Abárímských a utábořili se na Moábských pustinách u Jordánu naproti Jerichu.
#33:49 Tábořili u Jordánu od Bét-ješimótu až k Ábel-šitímu na Moábských pustinách.
#33:50 Hospodin promluvil k Mojžíšovi na Moábských pustinách u Jordánu naproti Jerichu:
#33:51 „Mluv k Izraelcům a řekni jim: Až přejdete Jordán do kenaanské země,
#33:52 vyženete před sebou obyvatele té země a zničíte všechny jejich modlářské výtvory. Zničíte všechny jejich lité modly a popleníte všechna jejich posvátná návrší.
#33:53 Podrobíte si tu zemi a usadíte se v ní. Tu zemi jsem dal vám, abyste ji obsadili.
#33:54 Rozdělíte si zemi losem jako dědictví mezi své čeledi; větší dáte větší dědictví a menší dáte menší dědictví. Kam komu padne los, to bude jeho. Rozdělíte ji v dědictví svým otcovským pokolením.
#33:55 Nevyženete-li však obyvatele té země před sebou, tedy ti, které z nich zanecháte, budou vám trnem v očích a bodcem v boku, budou vás sužovat v té zemi, ve které se usadíte.
#33:56 A jak jsem zamýšlel učinit jim, učiním vám.“ 
#34:1 Hospodin promluvil k Mojžíšovi:
#34:2 „Přikaž Izraelcům a řekni jim: Přijdete do kenaanské země, do té země, která vám připadne do dědictví, do kenaanské země vymezené těmito hranicemi:
#34:3 Jižní okraj vaší země povede od pouště Sinu podél Edómu. Na východě povede vaše jižní hranice od konce Solného moře,
#34:4 potom se vaše hranice stočí od jihu ke Svahu štírů, bude pokračovat k Sinu a na jihu bude vybíhat ke Kádeš-barneji, potáhne se na Chasar-adár a bude pokračovat k Asmónu.
#34:5 Pak se hranice stočí od Asmónu k Egyptskému potoku a bude vybíhat k moři.
#34:6 Vaší západní hranicí bude Velké moře s pobřežním územím; to bude vaše západní hranice.
#34:7 Toto pak bude vaše severní hranice: od Velkého moře si vyznačíte jako mezník horu Hór.
#34:8 Od hory Hóru si vyznačíte hranici až k cestě do Chamátu a hranice bude vybíhat k Sedadu.
#34:9 Hranice pak povede k Zifrónu a bude vybíhat k Chasar-énanu. To bude vaše severní hranice.
#34:10 Východní hranici si vyměříte od Chasar-énanu k Šefámu,
#34:11 poté sestoupí hranice od Šefámu k Rible na východ od Ajinu, pak sestupuje, až narazí na východní svah při Kineretském moři.
#34:12 Dál sestoupí hranice k Jordánu a bude vybíhat k Solnému moři. To bude vaše země, vymezená hranicemi kolem dokola.“
#34:13 Mojžíš tedy vydal Izraelcům takovýto rozkaz: „Toto je země, kterou dostanete losem do dědictví. Hospodin přikázal dát ji devíti a půl pokolením.
#34:14 Pokolení Rúbenovců a pokolení Gádovců již přijala svá dědictví pro svoje otcovské rody; i polovina pokolení Manasesova přijala dědictví.
#34:15 Dvě a půl pokolení přijala svá dědictví v Zajordání, na východ od Jericha, k východu slunce.“
#34:16 Hospodin promluvil k Mojžíšovi:
#34:17 „Toto jsou jména mužů, kteří za vás převezmou zemi do dědictví: kněz Eleazar a Jozue, syn Núnův.
#34:18 K převzetí země do dědictví přiberete po jednom předáku z každého pokolení.
#34:19 Toto jsou jména těch mužů: za Judovo pokolení Káleb, syn Jefunův,
#34:20 za pokolení Šimeónovců Šemúel, syn Amíhúdův,
#34:21 za Benjamínovo pokolení Elídad, syn Kislónův,
#34:22 za pokolení Danovců předák Bukí, syn Joglíův,
#34:23 za Josefovce z pokolení Manasesovců předák Chaníel, syn Efódův,
#34:24 z pokolení Efrajimovců předák Kemúel, syn Šiftánův,
#34:25 za pokolení Zabulónovců předák Elísafan, syn Parnakův,
#34:26 za pokolení Isacharovců předák Paltíel, syn Azanův,
#34:27 za pokolení Ašerovců předák Achíhúd, syn Šelomíův,
#34:28 za pokolení Neftalíovců předák Pedahél, syn Amíhúdův.“
#34:29 To jsou ti, kterým přikázal Hospodin, aby Izraelcům přidělili dědictví v kenaanské zemi. 
#35:1 Hospodin promluvil k Mojžíšovi na Moábských pustinách u Jordánu naproti Jerichu:
#35:2 „Vydej rozkaz Izraelcům, aby ze svého dědičného vlastnictví dali lévijcům města k obývání; též okolní pastviny náležející k městům dáte lévijcům.
#35:3 Budou tak mít pro sebe města k bydlení a jejich pastviny pro svůj dobytek, který je jejich jměním, i pro všechno své zvířectvo.
#35:4 Městské pastviny, jež dáte lévijcům, budou sahat od městské zdi do vzdálenosti jednoho tisíce loket dokola.
#35:5 Naměříte tedy vně za městem dva tisíce loket jako východní okraj, dva tisíce loket jako jižní okraj, dva tisíce loket jako západní okraj a dva tisíce loket jako severní okraj, takže město bude uprostřed. To budou mít za městské pastviny.
#35:6 Z měst, jež dáte lévijcům, bude šest měst útočištných, jež určíte k tomu, aby se tam mohl utéci ten, kdo zabil; k nim přidáte dalších dvaačtyřicet měst.
#35:7 Všech měst, jež dáte lévijcům, bude čtyřicet osm, i s pastvinami.
#35:8 Až budete z vlastnictví Izraelců dávat města, dáte více od těch, kdo mají víc, a méně od těch, kdo mají míň; každý dá lévijcům ze svých měst podle míry svého dědictví, kterého se mu dostalo.“
#35:9 Hospodin dále mluvil k Mojžíšovi:
#35:10 „Mluv k Izraelcům a řekni jim: Až přejdete Jordán do kenaanské země,
#35:11 vyberete si města, která budete mít za města útočištná. Tam se uteče ten, kdo zabil, pokud by někoho zabil neúmyslně.
#35:12 Ta města vám budou útočištěm před mstitelem, tak aby ten, kdo zabil, nemusel zemřít, dokud nebude stát na soudu před pospolitostí.
#35:13 Z měst, která dáte, budete mít šest měst jako útočiště.
#35:14 Tři města určíte v Zajordání a tři města v kenaanské zemi. Budou to útočištná města.
#35:15 Pro Izraelce i pro hosta a přistěhovalce mezi nimi bude těchto šest měst útočištěm, aby se tam utekl každý, kdo někoho zabil neúmyslně.
#35:16 Jestliže však někdo udeří někoho železným předmětem, takže ten člověk zemře, je to vrah a vrah musí zemřít.
#35:17 Jestliže uchopí kámen, kterým může přivodit smrt, a udeří někoho, takže zemře, je to vrah a vrah musí zemřít.
#35:18 Též jestliže uchopí dřevěný předmět, kterým může přivodit smrt, a udeří někoho, takže zemře, je to vrah a vrah musí zemřít.
#35:19 Vraha usmrtí krevní mstitel; jakmile ho dopadne, usmrtí ho.
#35:20 Jestliže někdo do někoho z nenávisti strčí nebo ve zlém úmyslu něco po někom hodí, takže zemře,
#35:21 nebo z nepřátelství někoho udeří rukou, takže zemře, musí ten, kdo druhého zabil, zemřít: je to vrah. Krevní mstitel vraha usmrtí, jakmile ho dopadne.
#35:22 Jestliže však někdo nešťastnou náhodou bez nepřátelství do někoho strčí nebo bez zlého úmyslu na někoho shodí nějaký předmět
#35:23 nebo z nepozornosti na něj nechá padnout kámen, který může přivodit smrt, takže ten člověk zemře, přitom však mu nebyl nepřítelem a nechystal mu nic zlého,
#35:24 pospolitost rozhodne mezi tím, kdo zabil, a krevním mstitelem podle těchto právních ustanovení.
#35:25 Tím pospolitost vyprostí toho, kdo zabil, z rukou krevního mstitele a umožní mu návrat do jeho útočištného města, kam se utekl. Zůstane v něm až do smrti velekněze, který byl pomazán svatým olejem.
#35:26 Jestliže však ten, kdo zabil, opustí území svého útočištného města, kam se utekl,
#35:27 a krevní mstitel ho najde mimo území jeho útočištného města a zabije ho, nebude na něm lpět krev.
#35:28 Ten, kdo zabil, musí zůstat ve svém útočištném městě až do smrti velekněze; teprve po smrti velekněze se může vrátit do svého vlastního území.
#35:29 Toto vám bude právním nařízením pro všechna vaše pokolení ve všech vašich sídlištích.
#35:30 Každého vraha, který někoho zabil, bude možno odsoudit k smrti jen podle výpovědi několika svědků; k jeho usmrcení nestačí výpověď jednoho svědka.
#35:31 Za život vraha nepřijmete výkupné; jako svévolník je hoden smrti a musí zemřít.
#35:32 Nepřijmete výkupné, aby se někdo, kdo utekl do svého útočištného města, potom zase usadil v zemi před smrtí kněze.
#35:33 Neposkvrňujte zemi, v níž jste. Právě krev poskvrňuje zemi a země nemůže být zproštěna viny za krev, která byla na ni prolita, jinak než krví toho, kdo krev prolil.
#35:34 Neznečistíš zemi, ve které sídlíte, uprostřed níž já přebývám, neboť já Hospodin přebývám uprostřed synů Izraele.“ 
#36:1 Tehdy přistoupili představitelé rodů z čeledi synů Gileáda, syna Makíra, syna Manasesova z josefovských čeledí, aby promluvili s Mojžíšem a předáky, představiteli rodů Izraelců.
#36:2 Řekli: „Hospodin přikázal našemu pánu, aby Izraelcům přidělil losem do dědictví zemi. Našemu pánu bylo též od Hospodina přikázáno dát dědictví našeho bratra Selofchada jeho dcerám.
#36:3 Když se však stanou ženami některého z příslušníků druhých izraelských kmenů, bude naše otcovské dědictví zkráceno o jejich dědictví a připojeno k dědictví toho pokolení, k němuž budou náležet. Tak bude los našeho dědictví zkrácen.
#36:4 Budou-li pak Izraelci slavit milostivé léto, bude jejich dědictví připojeno k dědictví toho pokolení, k němuž budou náležet, a dědictví našeho otcovského pokolení bude o jejich dědictví zkráceno.“
#36:5 Z Hospodinova rozkazu tedy přikázal Mojžíš Izraelcům: „Pokolení Josefovců mluví o tom oprávněně.
#36:6 Tuto věc přikázal Hospodin o dcerách Selofchadových: Ať se vdají, za koho se jim zlíbí, jen ať se vdávají v čeledi svého otcovského pokolení.
#36:7 Dědictví Izraelců nesmí přecházet z pokolení na pokolení; Izraelci budou spjati každý s dědictvím svého otcovského pokolení.
#36:8 Každá dcera, která obdrží dědictví od některého izraelského pokolení, vdá se za někoho z čeledi svého otcovského pokolení, aby Izraelci podrželi ve vlastnictví dědictví po svých otcích, každý to své.
#36:9 Dědictví nebude přecházet z pokolení na pokolení. Izraelská pokolení budou všechna spjata se svým dědictvím.“
#36:10 Co Hospodin Mojžíšovi přikázal, to Selofchadovy dcery splnily.
#36:11 Selofchadovy dcery Machla, Tirsa, Chogla, Milka a Nóa se staly manželkami synů svých strýců.
#36:12 Staly se manželkami mužů z čeledi synů Manasesa, syna Josefova, a jejich dědictví zůstalo při pokolení jejich otcovské čeledi.
#36:13 Toto jsou příkazy a právní řády, které přikázal Hospodin synům Izraele skrze Mojžíše na Moábských pustinách u Jordánu naproti Jerichu.  

\book{Deuteronomy}{Deut}
#1:1 Toto jsou slova, která mluvil Mojžíš k celému Izraeli v Zajordání, ve stepní pustině naproti Súfu mezi Páranem, Tófelem, Lábanem, Chaserótem a Dí-zahabem.
#1:2 Cesta od Chorébu směrem k pohoří Seíru do Kádeš-barneje trvá jedenáct dní.
#1:3 Ve čtyřicátém roce, prvního dne jedenáctého měsíce, sdělil Mojžíš synům Izraele vše, co mu pro ně Hospodin přikázal,
#1:4 když porazil emorejského krále Síchona, jenž sídlil v Chešbónu, a bášanského krále Óga, jenž sídlil v Aštarótu a v Edreí.
#1:5 V Zajordání v moábské zemi začal Mojžíš vysvětlovat tento zákon:
#1:6 Hospodin, náš Bůh, k nám promluvil na Chorébu: „Dosti dlouho jste již pobyli na této hoře.
#1:7 Obraťte se a táhněte dál, vstupte na pohoří Emorejců a ke všem jejich sousedům na pustině, v pohoří i v Přímořské nížině, v Negebu a na mořském pobřeží, jděte do kenaanské země a na Libanón až k veliké řece, řece Eufratu.
#1:8 Hleď, předal jsem vám zemi. Jděte obsadit zemi, o níž Hospodin přísahal vašim otcům Abrahamovi, Izákovi a Jákobovi, že ji dá jim a jejich potomstvu.“
#1:9 V oné době jsem vám řekl: „Nemohu vás sám unést.
#1:10 Hospodin, váš Bůh, vás rozmnožil; je vás dnes takové množství jako hvězd na nebi.
#1:11 Nechť vás Hospodin, Bůh vašich otců, rozhojní ještě tisíckrát, nechť vám žehná, jak vám přislíbil.
#1:12 Jak bych mohl sám unést vaše těžkosti, břemena a spory!
#1:13 Přiveďte ze svých kmenů moudré, rozumné a zkušené muže a já je učiním vašimi náčelníky.“
#1:14 Odpověděli jste mi: „Ta věc, kterou navrhuješ učinit, je dobrá.“
#1:15 Vzal jsem tedy představitele vašich kmenů, moudré a zkušené muže, a ustanovil jsem je za vaše náčelníky, za velitele nad tisíci, nad sty, padesáti a deseti a za správce vašich kmenů.
#1:16 V oné době jsem také přikázal vašim soudcům: „Vyslechněte své bratry a suďte každého v rozepři s jeho bratrem i s bezdomovcem spravedlivě.
#1:17 Při soudu nebuďte straničtí, vyslechněte jak malého, tak velkého a nikoho se nelekejte; soud je věcí Boží. Záležitost, která bude pro vás obtížná, předložte mně a já ji vyslechnu.“
#1:18 V oné době jsem vám přikázal všechno, co máte dělat.
#1:19 Od Chorébu jsme pak táhli dál a šli jsme celou tou velikou a hroznou pouští, kterou jste viděli, cestou k Emorejskému pohoří, jak nám přikázal Hospodin, náš Bůh, až jsme přišli do Kádeš-barneje.
#1:20 Tu jsem vám řekl: „Přišli jste až k Emorejskému pohoří, které nám dává Hospodin, náš Bůh.
#1:21 Hleď, Hospodin, tvůj Bůh, předal tu zemi tobě. Vytáhni a obsaď ji, jak ti poručil Hospodin, Bůh tvých otců. Neboj se a neděs.“
#1:22 Ale vy všichni jste ke mně přistoupili s návrhem: „Pošleme před sebou muže, aby nám obhlédli zemi a podali nám zprávu o cestě, kterou máme táhnout, i o městech, do nichž máme vstoupit.“
#1:23 Pokládal jsem to za dobré, proto jsem z vás vybral dvanáct mužů, po jednom z každého kmene.
#1:24 Ti se vypravili na výzvědy, vystoupili na pohoří a přišli až do úvalu Eškólu.
#1:25 Vzali s sebou z plodů té země, přinesli k nám dolů a podali nám zprávu: „Země, kterou nám dává Hospodin, náš Bůh, je dobrá.“
#1:26 Ale vy jste nechtěli táhnout a vzpírali jste se rozkazu Hospodina, svého Boha.
#1:27 Žehrali jste na něj ve svých stanech a říkali: „Hospodin nás vyvedl z egyptské země z nenávisti, aby nás vydal do rukou Emorejců, a tak nás vyhladil.
#1:28 Kam to táhneme? Naši bratři nás zbavili odvahy, když řekli: Viděli jsme tam lid větší a vyšší než jsme my, města veliká a opevněná až k nebi, dokonce i Anákovce.“
#1:29 Tu jsem vám řekl: „Nemějte strach a nebojte se jich.
#1:30 Hospodin, váš Bůh, který jde před vámi, bude bojovat za vás, jak to před vašima očima učinil s vámi v Egyptě
#1:31 i v poušti, kde jsi viděl, jak tě Hospodin, tvůj Bůh, nesl, jako nosí muž svého syna, po celé cestě, kterou jste prošli, až jste došli k tomuto místu.“
#1:32 Přesto teď nevěříte Hospodinu, svému Bohu,
#1:33 který chodí cestou před vámi, aby vám vyhlédl místo k táboření, a ukazuje vám cestu, po níž máte jít, za noci v ohni a ve dne v oblaku.
#1:34 Když Hospodin vyslechl vaše slova, rozlítil se a přísahal:
#1:35 „Věru, nikdo z mužů tohoto zlého pokolení nespatří tu dobrou zemi, kterou jsem přísahal dát jejich otcům.
#1:36 Jenom Káleb, syn Jefunův, ten ji spatří, jemu a jeho synům dám zemi, na níž stanula jeho noha, protože se cele oddal Hospodinu.“
#1:37 Kvůli vám se Hospodin rozhněval i na mne. Řekl: „Ani ty tam nevejdeš.
#1:38 Jozue, syn Núnův, který ti je k službě, ten tam vejde. Posilni ho, neboť on rozdělí Izraeli zemi v dědictví.
#1:39 Dám ji vašim dětem, o nichž jste říkali, že se stanou kořistí nepřátel; vaši synové, kteří dnes ještě nerozeznávají dobro od zla, ti tam vejdou. Jim ji dám a oni ji obsadí.
#1:40 Ale vy se obraťte a táhněte do pouště cestou k Rákosovému moři.“
#1:41 Tu jste mi odpověděli: „Zhřešili jsme proti Hospodinu. Chceme táhnout a bojovat tak, jak nám přikázal Hospodin, náš Bůh.“ Pak se každý z vás opásal svou válečnou zbrojí a chystali jste se lehkovážně vystoupit na pohoří.
#1:42 Hospodin mi však řekl: „Vyřiď jim: Netáhněte a nebojujte, neboť já mezi vámi nebudu, a budete od svých nepřátel poraženi.“
#1:43 Mluvil jsem k vám, ale vy jste neposlechli, vzepřeli jste se Hospodinovu rozkazu a opovážlivě jste vystoupili na pohoří.
#1:44 Emorejci, sídlící na tom pohoří, vyrazili proti vám, hnali se za vámi jako vosy a rozprášili vás po Seíru až do Chormy.
#1:45 Tu jste se vrátili a plakali před Hospodinem, ale Hospodin váš hlas nevyslyšel, nepopřál vám sluchu.
#1:46 Proto jste museli po mnoho dní zůstat v Kádeši; zůstali jste tam téměř rok. 
#2:1 Potom jsme se obrátili a táhli dál do pouště cestou k Rákosovému moři, jak Hospodin ke mně mluvil. Po mnoho dní jsme obcházeli pohoří Seír.
#2:2 Tu mi Hospodin řekl:
#2:3 „Už dosti dlouho obcházíte toto pohoří. Obraťte se na sever.
#2:4 Lidu přikaž: Procházíte územím svých bratří Ezauovců, sídlících v Seíru. Budou se vás bát, ale mějte se napozoru.
#2:5 Nedrážděte je, neboť z jejich země vám nedám ani šlépěj půdy. Pohoří Seír jsem dal do vlastnictví Ezauovi.
#2:6 Za stříbro si od nich budete kupovat pokrm, abyste měli co jíst, za stříbro si od nich budete opatřovat vodu, abyste měli co pít.
#2:7 Hospodin, tvůj Bůh, ti při veškeré práci tvých rukou žehnal; znal tvé putování touto velikou pouští. Už po čtyřicet let je Hospodin, tvůj Bůh, s tebou; v ničem jsi nestrádal.“
#2:8 Přešli jsme tedy mimo území svých bratří Ezauovců, sídlících v Seíru, stranou cesty vedoucí pustinou od Élatu a Esjón-geberu. Tam jsme se obrátili a pokračovali směrem na Moábskou step.
#2:9 Tu mi Hospodin řekl: „Nedotírej na Moába a nedráždi ho k válce, neboť z jeho země ti nedám do vlastnictví nic. Ar jsem dal do vlastnictví synům Lotovým.“
#2:10 (Předtím v něm sídlili Emejci, veliký a početný lid, vysoký jako Anákovci;
#2:11 ti byli stejně jako Anákovci pokládáni za Refájce, Moábci je však nazývali Emejci.
#2:12 V Seíru zase předtím sídlili Chorejci, ale Ezauovci si je podrobili; vyhladili je před sebou a usadili se na jejich místě, jako učinil Izrael se svou vlastní zemí, kterou jim dal Hospodin.)
#2:13 „Nuže vzhůru, přejděte potok Zered.“ I přešli jsme potok Zered.
#2:14 Doba, po kterou jsme putovali z Kádeš-barneje, až jsme překročili potok Zered, činila třicet osm let, dokud z tábora nevyhynulo do jednoho celé pokolení bojovníků, jak jim přísahal Hospodin.
#2:15 To právě Hospodinova ruka byla proti nim a uváděla je ve zmatek, dokud z tábora nevyhynuli do jednoho.
#2:16 Když pak všichni bojovníci z lidu do jednoho vymřeli,
#2:17 Hospodin ke mně promluvil:
#2:18 „Dnes procházíš moábským územím, městem Arem,
#2:19 a přibližuješ se k Amónovcům. Nedotírej na ně a nedráždi je, neboť ze země Amónovců ti nedám do vlastnictví nic. Dal jsem ji do vlastnictví synům Lotovým.“
#2:20 (Také ona byla pokládána za zemi Refájců; předtím v ní sídlili Refájci, Amónovci je však nazývali Zamzumci.
#2:21 Byl to veliký a početný lid, vysoký jako Anákovci. Hospodin je před nimi vyhladil, takže si je podrobili a usadili se na jejich místě,
#2:22 jak to učinil pro Ezauovce sídlící v Seíru. Vyhladil před nimi Chorejce, takže si je podrobili a usadili se na jejich místě, a zde jsou až dosud.
#2:23 Avejce, sídlící ve dvorcích až do Gázy, vyhladili zase Kaftórci, kteří vytáhli z Kaftóru a usadili se na jejich místě.)
#2:24 „Vzhůru, táhněte dál a přejděte potok Arnón. Hleď, vydal jsem ti do rukou chešbónského krále Síchona, Emorejce, a jeho zemi. Začni ji obsazovat a vydráždi je tak k válce.
#2:25 Dnešním dnem začínám nahánět strach a bázeň z tebe národům pod celým nebem. Až o tobě uslyší zprávu, budou se před tebou třást a chvět úzkostí.“
#2:26 Tehdy jsem poslal posly z pouště Kedemótu k chešbónskému králi Síchonovi s pokojným vzkazem:
#2:27 „Rád bych prošel tvou zemí. Půjdu jenom po cestě, neodbočím napravo ani nalevo.
#2:28 Za stříbro mi prodáš pokrm, abych měl co jíst, a za stříbro mi dáš vodu, abych měl co pít. Chci pouze pěšky projít,
#2:29 jak mi to dovolili Ezauovci sídlící v Seíru a Moábci sídlící v Aru, jen co přejdu přes Jordán do země, kterou nám dává Hospodin, náš Bůh.“
#2:30 Ale chešbónský král Síchon nesvolil, abychom jí prošli, neboť Hospodin, tvůj Bůh, zatvrdil jeho ducha a dal mu srdce zpupné, aby ti ho vydal do rukou, jak se dnešního dne stalo.
#2:31 Tu mi Hospodin řekl: „Hleď, už ti vydávám Síchona a jeho zemi. Začni jeho zemi obsazovat.“
#2:32 Síchon vytrhl proti nám do Jahsy k boji s veškerým svým lidem.
#2:33 A Hospodin, náš Bůh, nám ho vydal. Porazili jsme ho i jeho syny a všechen jeho lid.
#2:34 V oné době jsme dobyli všechna jeho města, vyhubili jsme každé jeho město jako klaté, muže, ženy i děti; nenechali jsme nikoho vyváznout.
#2:35 Jen dobytek a kořist z měst, která jsme dobyli, jsme si ponechali jako lup.
#2:36 Od Aróeru, který je na břehu potoka Arnónu, a od města, které je tam v úvalu, až do Gileádu nebyla pro nás žádná tvrz nedobytná; to vše nám vydal Hospodin, náš Bůh.
#2:37 Jen k zemi Amónovců ses nepřiblížil, k celému pásmu při potoku Jaboku, ani k městům v pohoří a vůbec nikam, kam nám Hospodin, náš Bůh, zakázal. 
#3:1 Potom jsme se obrátili a táhli vzhůru cestou k Bášanu. Tu vytrhl proti nám k boji do Edreí bášanský král Óg s veškerým svým lidem.
#3:2 Ale Hospodin mi řekl: „Neboj se ho, neboť jsem ti ho vydal do rukou, i všechen jeho lid a jeho zemi. Naložíš s ním, jako jsi naložil s emorejským králem Síchonem, který sídlil v Chešbónu.“
#3:3 Tak nám Hospodin, náš Bůh, vydal do rukou i bášanského krále Óga s veškerým jeho lidem. Pobili jsme je a nikdo z nich nevyvázl.
#3:4 V oné době jsme dobyli všechna jeho města. Nebylo žádné tvrze, kterou bychom jim nevzali, celkem šedesát měst, celý kraj Argób, Ógovo království v Bášanu.
#3:5 Všechna tato města byla opevněna vysokými hradbami s branami a závorami; mimoto jsme dobyli velmi mnoho otevřených měst.
#3:6 Vyhubili jsme je jako klatá, jako jsme učinili chešbónskému králi Síchonovi; vyhubili jsme každé město jako klaté, muže, ženy i děti.
#3:7 Všechen dobytek však a kořist z měst jsme si ponechali jako lup.
#3:8 Tak jsme v oné době vzali z rukou dvou emorejských králů, kteří byli v Zajordání, zemi od potoka Arnónu až k Chermónskému pohoří.
#3:9 (Chermón nazývají Sidónci Sirjón a Emorejci jej nazývají Senír.)
#3:10 Vzali jsme všechna města náhorní roviny, celý Gileád, celý Bášan až po Salku a Edreí, města Ógova království v Bášanu.
#3:11 Neboť právě bášanský král Óg zůstal ze zbytku Refájců. Hle, jeho lože, lože železné, je v Rabě Amónovců. Je dlouhé devět loket a široké čtyři lokte, měřeno běžným loktem.
#3:12 V oné době jsme tedy obsadili toto území od Aróeru při potoku Arnónu. Polovinu Gileádského pohoří s tamními městy jsem dal Rúbenovcům a Gádovcům.
#3:13 Zbytek Gileádu s celým Bášanem, královstvím Ógovým, jsem dal polovině kmene Manasesova, celý kraj Argób; celý Bášan se nazývá zemí Refájců.
#3:14 Manasesovec Jaír zabral celý kraj Argób až k pomezí gešúrskému a maakatskému a nazval jej - totiž Bášan - svým jménem: vesnice Jaírovy; jmenují se tak dodnes.
#3:15 Makírovi jsem dal Gileád.
#3:16 Rúbenovcům a Gádovcům jsem dal část Gileádu až k potoku Arnónu (hranici tvoří střed úvalu) a až k potoku Jaboku k hranici Amónovců,
#3:17 a pustinu s Jordánem jako hranici od Kineretu po Pusté moře, moře Solné, na východ od úpatí Pisgy.
#3:18 V oné době jsem vám přikázal: „Hospodin, váš Bůh, vám dal tuto zemi, abyste ji obsadili. Všichni, kteří jste schopni boje, potáhnete ve zbroji před svými bratry Izraelci.
#3:19 Jen vaše ženy, děti a stáda, vím, že máte četná stáda, zůstanou ve vašich městech, která jsem vám dal.
#3:20 Až Hospodin dopřeje odpočinutí vašim bratřím jako vám, až i oni obsadí zemi, kterou jim dává Hospodin, váš Bůh, za Jordánem, pak se vrátíte každý do svého vlastnictví, které jsem vám dal.“
#3:21 Jozuovi jsem v oné době přikázal: „Na vlastní oči jsi viděl všechno, co Hospodin, váš Bůh, učinil těm dvěma králům. Tak učiní Hospodin všem královstvím, jimiž budeš procházet.
#3:22 Nebojte se jich, vždyť Hospodin, váš Bůh, bojuje za vás.“
#3:23 V oné době jsem prosil Hospodina o smilování:
#3:24 „Panovníku Hospodine, ty jsi začal ukazovat svému služebníkovi svou velikost a svou pevnou ruku. Však také který bůh na nebi nebo na zemi by dovedl vykonat takové činy a takové bohatýrské skutky jako ty!
#3:25 Kéž smím přejít Jordán a spatřit tu dobrou zemi, co je za Jordánem, dobrou hornatou zemi a Libanón!“
#3:26 Ale Hospodin proti mně kvůli vám vzplanul prchlivostí a nevyslyšel mě. Řekl mi: „Máš dost. O této věci už ke mně nemluv.
#3:27 Vystup na vrchol Pisgy a rozhlédni se na západ, na sever, na jih a na východ; na vlastní oči se podívej. Tento Jordán však nepřejdeš.
#3:28 A dej příkazy Jozuovi, dodej mu síly a odvahy, neboť on ho přejde v čele tohoto lidu a rozdělí jim do dědictví zemi, kterou uvidíš.“
#3:29 Tak jsme zůstali v údolí naproti Bét-peóru. 
#4:1 Nyní tedy, Izraeli, slyš nařízení a práva, která vás učím dodržovat, abyste zůstali naživu a mohli obsadit zemi, kterou vám dává Hospodin, Bůh vašich otců.
#4:2 K tomu, co vám přikazuji, nic nepřidáte a nic z toho neuberete, ale budete dbát na příkazy Hospodina, svého Boha, které vám udílím.
#4:3 Na vlastní oči jste viděli, co Hospodin učinil kvůli Baal-peórovi, že každého muže, který chodil za Baal-peórem, Hospodin, tvůj Bůh, vyhladil z tvého středu.
#4:4 Ale vy, kteří jste se přimkli k Hospodinu, svému Bohu, jste dodnes všichni živi.
#4:5 Hleď, učil jsem vás nařízením a právům, jak mi přikázal Hospodin, můj Bůh, abyste je dodržovali v zemi, kterou přicházíte obsadit.
#4:6 Bedlivě je dodržujte. To bude vaše moudrost a rozumnost před zraky lidských pokolení. Když uslyší všechna tato nařízení, řeknou: „Jak moudrý a rozumný lid je tento veliký národ!“
#4:7 Což se najde jiný veliký národ, jemuž jsou jeho bohové tak blízko, jako je nám Hospodin, náš Bůh, kdykoli k němu voláme?
#4:8 A má jiný veliký národ nařízení a práva tak spravedlivá jako celý tento zákon, který vám dnes předkládám?
#4:9 Jenom si dej pozor a velice se střez zapomenout na věci, které jsi viděl na vlastní oči, aby nevymizely z tvého srdce po všechny dny tvého života. Seznam s nimi své syny i vnuky.
#4:10 Nezapomeň na den, kdy jsi stál před Hospodinem, svým Bohem, na Chorébu, kdy mi řekl Hospodin: „Shromáždi mi lid a dám jim slyšet svá slova, aby se mě naučili bát po všechny dny svého života na zemi a učili tomu i své syny.“
#4:11 Tenkrát jste se přiblížili a stáli pod horou. Hora planula ohněm až do samých nebes a kolem byla tma, oblak a mrákota.
#4:12 I promluvil k vám Hospodin zprostředku ohně. Slyšeli jste zvuk slov, avšak žádnou podobu jste neviděli; slyšeli jste jen hlas.
#4:13 Oznámil vám svou smlouvu, kterou vám přikázal dodržovat, desatero přikázání, a napsal je na dvě kamenné desky.
#4:14 A mně Hospodin v oné době přikázal učit vás nařízením a právům, abyste je dodržovali v zemi, do které táhnete a kterou máte obsadit.
#4:15 V den, kdy k vám Hospodin mluvil na Chorébu zprostředku ohně, jste neviděli žádnou podobu; velice se tedy střezte,
#4:16 abyste se nezvrhli a neudělali si tesanou sochu, žádné sochařské zpodobení: zobrazení mužství nebo ženství,
#4:17 zobrazení jakéhokoli zvířete, které je na zemi, zobrazení jakéhokoli okřídleného ptáka, který létá po nebi,
#4:18 zobrazení jakéhokoli plaza, který se plazí po zemi, zobrazení jakékoli ryby, která je ve vodách pod zemí;
#4:19 abys nepovznášel své zraky k nebesům, a když bys viděl slunce, měsíc a hvězdy, všechen nebeský zástup, nedal se svést a neklaněl se jim a nesloužil tomu, co dal Hospodin, tvůj Bůh, jako podíl všem národům pod celým nebem.
#4:20 Ale vás Hospodin vzal a vyvedl z tavicí pece, z Egypta, abyste byli jeho dědičným lidem, jak tomu je dnes.
#4:21 Na mne se však Hospodin kvůli vám rozhněval a přísahal, že nepřejdu Jordán a že nevstoupím do té dobré země, kterou ti Hospodin, tvůj Bůh, dává do dědictví.
#4:22 Umřu v této zemi, nepřejdu Jordán, ale vy jej přejdete a tu dobrou zemi obsadíte.
#4:23 Dávejte si pozor, abyste nezapomněli na smlouvu Hospodina, svého Boha, kterou s vámi uzavřel, a neudělali si tesanou sochu, zpodobení čehokoli, co ti Hospodin, tvůj Bůh, zakázal,
#4:24 neboť Hospodin, tvůj Bůh, je oheň sžírající, Bůh žárlivě milující.
#4:25 Budeš plodit syny a vnuky a dožijete se v té zemi stáří; zvrhnete-li se však a uděláte si tesanou sochu, zpodobení čehokoli, a dopustíte se toho, co je zlé v očích Hospodina, tvého Boha, co ho uráží,
#4:26 beru si dnes proti vám za svědky nebe i zemi, že v zemi, do níž přejdete přes Jordán, abyste ji obsadili, rychle a nadobro vyhynete. Nebudete v ní dlouho živi, ale budete vyhlazeni.
#4:27 Hospodin vás rozptýlí mezi kdejaký lid, že z vás v pronárodech, kam vás Hospodin zavede, zbude jen malý počet.
#4:28 Tam budete sloužit bohům vyrobeným lidskýma rukama, bohům ze dřeva a z kamene, kteří nevidí a neslyší, nejedí a nečichají.
#4:29 Odtamtud budete hledat Hospodina, svého Boha; nalezneš ho, budeš-li ho opravdu hledat celým svým srdcem a celou svou duší.
#4:30 Ve svém soužení, až tě v posledních dnech toto všechno stihne, navrátíš se k Hospodinu, svému Bohu, a budeš ho poslouchat.
#4:31 Vždyť Hospodin, tvůj Bůh, je Bůh milosrdný, nenechá tě klesnout a nepřipustí tvou zkázu, nezapomene na smlouvu s tvými otci, kterou jim stvrdil přísahou.
#4:32 Jen se ptej na dřívější časy, které byly před tebou, od chvíle, kdy Bůh stvořil na zemi člověka. Ptej se od jednoho konce nebes ke druhému, zda se stala tak veliká věc anebo bylo o něčem takovém slýcháno.
#4:33 Zdali kdy slyšel lid hlas Boha mluvícího zprostředku ohně, jako jsi slyšel ty, a zůstal naživu,
#4:34 anebo zdali se pokusil nějaký bůh přijít a vzít pronárod zprostředku jiného pronároda zkouškami, znameními, zázraky a bojem, pevnou rukou a vztaženou paží, velkými hroznými činy, jak to vše s vámi učinil Hospodin, váš Bůh, v Egyptě tobě před očima.
#4:35 Tobě bylo odhaleno poznání, že Hospodin je Bůh; kromě něho není žádný jiný.
#4:36 Z nebe ti dal slyšet svůj hlas, aby tě napomenul, na zemi ti dal spatřit svůj veliký oheň a jeho slova jsi slyšel zprostředku ohně.
#4:37 Protože miloval tvé otce, vyvolil po nich i jejich potomstvo a tebe vyvedl svou velikou mocí z Egypta,
#4:38 aby před tebou vyhnal pronárody větší a zdatnější, než jsi ty, a přivedl tě do jejich země a dal ti ji do dědictví, jak tomu je dnes.
#4:39 Proto poznej dnes a vezmi si k srdci, že Hospodin je Bůh nahoře na nebesích i dole na zemi; žádný jiný není.
#4:40 Dbej na jeho nařízení a příkazy, které ti dnes udílím, aby se vedlo dobře tobě i tvým synům po tobě a abys byl dlouho živ na zemi, kterou ti pro všechny dny dává Hospodin, tvůj Bůh.
#4:41 Tehdy oddělil Mojžíš tři města v Zajordání, na východě,
#4:42 aby se tam mohl utéci ten, kdo zabil, pokud zabil svého bližního neúmyslně a neměl ho dříve v nenávisti. Nechť se uteče do některého z těchto měst a zůstane naživu:
#4:43 do rúbenovského Beseru ve stepi na náhorní rovině, do gádovského Rámotu v Gileádu a do manasesovského Gólanu v Bášanu.
#4:44 Toto je zákon, který předložil Mojžíš Izraelcům.
#4:45 Toto jsou svědectví, nařízení a práva, která přednesl Mojžíš Izraelcům po jejich vyjití z Egypta
#4:46 v Zajordání, v údolí naproti Bét-peóru, v zemi emorejského krále Síchona, jenž sídlil v Chešbónu; porazili ho Mojžíš a Izraelci po svém vyjití z Egypta.
#4:47 Obsadili jeho zemi i zemi bášanského krále Óga, obou emorejských králů v Zajordání, na východě:
#4:48 od Aróeru, který je na břehu potoka Arnónu, až k hoře Síonu, což je Chermón,
#4:49 a celou pustinu za Jordánem na východě až k Pustému moři pod úpatí Pisgy. 
#5:1 Mojžíš svolal celý Izrael a řekl jim: Slyš, Izraeli, nařízení a práva, která vám dnes vyhlašuji. Učte se jim a bedlivě je dodržujte.
#5:2 Hospodin, náš Bůh, s námi uzavřel na Chorébu smlouvu.
#5:3 Tuto smlouvu neuzavřel Hospodin jen s našimi otci, ale s námi všemi, kteří jsme tu dnes naživu.
#5:4 Tváří v tvář mluvil s vámi Hospodin na hoře zprostředku ohně.
#5:5 Já jsem stál v oné době mezi Hospodinem a vámi, abych vám oznámil Hospodinovo slovo, protože jste se báli ohně a nevystoupili jste na horu. Řekl:
#5:6 „Já jsem Hospodin, tvůj Bůh; já jsem tě vyvedl z egyptské země, z domu otroctví.
#5:7 Nebudeš mít jiného boha mimo mne.
#5:8 Nezobrazíš si Boha zpodobením ničeho, co je nahoře na nebi, dole na zemi nebo ve vodách pod zemí.
#5:9 Nebudeš se ničemu takovému klanět ani tomu sloužit. Já Hospodin, tvůj Bůh, jsem Bůh žárlivě milující. Stíhám vinu otců na synech i do třetího a čtvrtého pokolení těch, kteří mě nenávidí,
#5:10 ale prokazuji milosrdenství tisícům pokolení těch, kteří mě milují a má přikázání zachovávají.
#5:11 Nezneužiješ jména Hospodina, svého Boha. Hospodin nenechá bez trestu toho, kdo by jeho jména zneužíval.
#5:12 Dbej na den odpočinku, aby ti byl svatý, jak ti přikázal Hospodin, tvůj Bůh.
#5:13 Šest dní budeš pracovat a dělat všechnu svou práci.
#5:14 Ale sedmý den je den odpočinutí Hospodina, tvého Boha. Nebudeš dělat žádnou práci ani ty ani tvůj syn a tvá dcera ani tvůj otrok a tvá otrokyně ani tvůj býk a tvůj osel, žádné tvé dobytče ani tvůj host, který žije v tvých branách, aby odpočinul tvůj otrok a tvá otrokyně tak jako ty.
#5:15 Pamatuj, že jsi byl otrokem v egyptské zemi a že tě Hospodin, tvůj Bůh, odtud vyvedl pevnou rukou a vztaženou paží. Proto ti přikázal Hospodin, tvůj Bůh, dodržovat den odpočinku.
#5:16 Cti svého otce i matku, jak ti přikázal Hospodin, tvůj Bůh, abys byl dlouho živ a dobře se ti vedlo na zemi, kterou ti dává Hospodin, tvůj Bůh.
#5:17 Nezabiješ.
#5:18 Nesesmilníš.
#5:19 Nepokradeš.
#5:20 Nevydáš proti svému bližnímu falešné svědectví.
#5:21 Nebudeš dychtit po ženě svého bližního. Nebudeš toužit po domě svého bližního ani po jeho poli ani po jeho otroku ani po jeho otrokyni ani po jeho býku ani po jeho oslu, vůbec po ničem, co patří tvému bližnímu.“
#5:22 Tato slova mluvil Hospodin k celému vašemu shromáždění na hoře zprostředku ohně, oblaku a mrákoty mocným hlasem a víc nepřipojil. Napsal je na dvě kamenné desky a dal je mně.
#5:23 Jakmile jste uslyšeli hlas zprostředku tmy, zatímco hora planula ohněm, přistoupili jste ke mně, všichni představitelé vašich kmenů a vaši starší,
#5:24 a řekli jste: „Hle, Hospodin, náš Bůh, nám ukázal svou slávu a velikost. Slyšeli jsme jeho hlas zprostředku ohně a viděli jsme dnešního dne, že Bůh mluví s člověkem a ten může zůstat naživu.
#5:25 Ale proč bychom teď měli zemřít? Vždyť nás tento veliký oheň pozře. Uslyšíme-li ještě dál hlas Hospodina, svého Boha, zemřeme.
#5:26 Kdo ze všeho tvorstva by mohl slyšet hlas živého Boha, mluvícího zprostředku ohně, jako slyšíme my, a zůstat naživu?
#5:27 Přistup k němu ty a vyslechni všechno, co Hospodin, náš Bůh, řekne. Ty nám pak vypovíš všechno, co k tobě promluví Hospodin, náš Bůh, a my to vyposlechneme a učiníme.“
#5:28 Hospodin vyslechl vaše slova, když jste mluvili ke mně, a řekl mi: „Vyslechl jsem slova tohoto lidu, jak mluvili k tobě. Dobře to všechno pověděli.
#5:29 Kéž mají stále takové srdce, aby se mě báli po všechny dny a dbali na všechny mé příkazy, aby se jim i jejich synům vždycky vedlo dobře.
#5:30 Jdi, řekni jim: ‚Vraťte se do svých stanů.‘
#5:31 Ty však zde u mne stůj a já ti sdělím všechna přikázání, nařízení a práva, kterým je budeš učit, aby je dodržovali v zemi, kterou jim dávám do vlastnictví.“
#5:32 Bedlivě dodržujte, co vám přikázal Hospodin, váš Bůh, neuchylujte se napravo ani nalevo.
#5:33 Musíte se ve všem držet cesty, kterou vám Hospodin, váš Bůh, přikázal jít, a tak zůstanete naživu, dobře se vám povede a budete dlouho živi v zemi, kterou máte obsadit. 
#6:1 Toto jsou přikázání, nařízení a práva, kterým vás Hospodin, váš Bůh, přikázal vyučovat, abyste je dodržovali v zemi, do níž táhnete a kterou máte obsadit:
#6:2 Aby ses bál Hospodina, svého Boha, a bedlivě dbal na všechna jeho nařízení a příkazy, které ti udílím, ty i tvůj syn a tvůj vnuk, po všechny dny svého života, abys byl dlouho živ.
#6:3 Poslouchej je, Izraeli, a bedlivě je dodržuj. Tak se ti povede dobře a velmi se rozmnožíte v zemi oplývající mlékem a medem, jak ti přislíbil Hospodin, Bůh tvých otců.
#6:4 Slyš, Izraeli, Hospodin je náš Bůh, Hospodin jediný.
#6:5 Budeš milovat Hospodina, svého Boha, celým svým srdcem a celou svou duší a celou svou silou.
#6:6 A tato slova, která ti dnes přikazuji, budeš mít v srdci.
#6:7 Budeš je vštěpovat svým synům a budeš o nich rozmlouvat, když budeš sedět doma nebo půjdeš cestou, když budeš uléhat nebo vstávat.
#6:8 Uvážeš si je jako znamení na ruku a budeš je mít jako pásek na čele mezi očima.
#6:9 Napíšeš je také na veřeje svého domu a na své brány.
#6:10 Až tě Hospodin, tvůj Bůh, přivede do země, o které přísahal tvým otcům Abrahamovi, Izákovi a Jákobovi, že ti ji dá, a dá ti veliká a dobrá města, která jsi nestavěl,
#6:11 domy plné všeho dobrého, které jsi nenaplnil, vykopané studny, které jsi nevykopal, vinice a olivoví, které jsi nevysadil, a budeš jíst a nasytíš se,
#6:12 pak si dávej pozor, abys nezapomněl na Hospodina, který tě vyvedl z egyptské země, z domu otroctví.
#6:13 Hospodina, svého Boha, se budeš bát, jemu budeš sloužit, při jeho jménu přísahat.
#6:14 Nesmíte chodit za jinými bohy z božstev těch národů, které jsou kolem vás,
#6:15 neboť uprostřed tebe je Bůh žárlivě milující, Hospodin, tvůj Bůh. Ať Hospodin, tvůj Bůh, nevzplane proti tobě hněvem a nevyhladí tě z povrchu země.
#6:16 Nepokoušejte Hospodina, svého Boha, jako jste ho pokoušeli v Masse.
#6:17 Musíte bedlivě dbát na příkazy Hospodina, svého Boha, na jeho svědectví a nařízení, která ti přikázal.
#6:18 Budeš dělat jen to, co je správné a dobré v Hospodinových očích, aby se ti dobře vedlo, až půjdeš obsadit tu dobrou zemi, o níž Hospodin přísahal tvým otcům,
#6:19 že z ní před tebou vypudí všechny tvé nepřátele; tak přece mluvil Hospodin.
#6:20 Až se tě tvůj syn v budoucnu zeptá: „Co to jsou ta svědectví, nařízení a práva, která vám přikázal Hospodin, náš Bůh?“,
#6:21 odvětíš svému synu: „Byli jsme faraónovými otroky v Egyptě a Hospodin nás vyvedl z Egypta pevnou rukou.
#6:22 Před našimi zraky činil Hospodin znamení a zázraky veliké a zlé proti Egyptu, proti faraónovi i proti celému jeho domu.
#6:23 Ale nás odtamtud vyvedl, aby nás uvedl sem a dal nám zemi, kterou přísežně přislíbil našim otcům.
#6:24 Hospodin nám přikázal, abychom dodržovali všechna tato nařízení, báli se Hospodina, svého Boha, aby s námi bylo dobře po všechny dny, aby nás zachoval při životě, jak tomu je dnes.
#6:25 Bude se nám počítat za spravedlnost, budeme-li bedlivě dodržovat každý tento příkaz před Hospodinem, svým Bohem, jak nám přikázal.“ 
#7:1 Až tě Hospodin, tvůj Bůh, uvede do země, kterou přicházíš obsadit, zažene před tebou početné pronárody, Chetejce, Girgašejce, Emorejce, Kenaance, Perizejce, Chivejce a Jebúsejce, sedm pronárodů početnějších a zdatnějších než ty.
#7:2 Hospodin, tvůj Bůh, ti je předá, abys je pobil. Vyhubíš je jako klaté, neuzavřeš s nimi smlouvu a nesmiluješ se nad nimi,
#7:3 nespřízníš se s nimi, svou dceru neprovdáš za syna někoho z nich ani jeho dceru nevezmeš pro svého syna.
#7:4 To by odvrátilo tvého syna ode mne, takže by sloužili jiným bohům. Hospodin by proti vám vzplanul hněvem a rychle by tě vyhladil.
#7:5 Proto s nimi naložíte takto: jejich oltáře rozboříte, jejich posvátné sloupy roztříštíte, jejich posvátné kůly pokácíte, jejich tesané sochy spálíte.
#7:6 Jsi přece svatý lid Hospodina, svého Boha; tebe si Hospodin, tvůj Bůh, vyvolil ze všech lidských pokolení, která jsou na tváři země, abys byl jeho lidem, zvláštním vlastnictvím.
#7:7 Nikoli proto, že byste byli početnější než kterýkoli jiný lid, přilnul k vám Hospodin a vyvolil vás. Vás je přece méně než kteréhokoli lidu.
#7:8 Ale protože vás Hospodin miluje a zachovává přísahu, kterou se zavázal vašim otcům, vyvedl vás Hospodin pevnou rukou a vykoupil tě z domu otroctví, z rukou faraóna, krále egyptského.
#7:9 Poznej tedy, že Hospodin, tvůj Bůh, je Bůh, Bůh věrný, zachovávající smlouvu a milosrdenství do tisícího pokolení těm, kteří ho milují a dbají na jeho přikázání.
#7:10 Avšak tomu, kdo ho nenávidí, odplácí přímo a uvrhne ho do záhuby. Nebude odkládat; tomu, kdo ho nenávidí, odplatí přímo.
#7:11 Proto bedlivě dbej na přikázání, nařízení a práva, která ti dnes přikazuji dodržovat.
#7:12 Za to, že tato práva budete poslouchat a bedlivě je dodržovat, bude Hospodin, tvůj Bůh, zachovávat tobě smlouvu a milosrdenství, jak přísahal tvým otcům.
#7:13 Bude tě milovat a bude ti žehnat a rozmnoží tě. Požehná plodu tvého života i plodu tvé role, tvému obilí, tvému moštu, tvému oleji, vrhu tvého skotu a přírůstku tvého bravu na zemi, o níž se přísahou zavázal tvým otcům, že ti ji dá.
#7:14 Budeš požehnaný nad každý jiný lid, nevyskytne se neplodný nebo neplodná u tebe ani u tvého dobytka.
#7:15 Hospodin od tebe odvrátí každou nemoc, nevloží na tebe žádnou ze zhoubných chorob egyptských, které jsi poznal, ale uvalí je na všechny, kdo tě nenávidí.
#7:16 Pohltíš každý lid, který ti vydá Hospodin, tvůj Bůh, nebudeš ho litovat, nebudeš sloužit jeho bohům; bylo by ti to léčkou.
#7:17 Snad si v srdci řekneš: „Tyto pronárody jsou početnější než já. Jak bych si je mohl podrobit?“
#7:18 Neboj se jich! Jen si vzpomeň, jak naložil Hospodin, tvůj Bůh, s faraónem a s celým Egyptem,
#7:19 vzpomeň si na veliké zkoušky, které jsi viděl na vlastní oči, na znamení a zázraky, na pevnou ruku a vztaženou paži, jimiž tě vyvedl Hospodin, tvůj Bůh. Tak naloží Hospodin, tvůj Bůh, s každým lidem, kterého se bojíš.
#7:20 I děsy na ně pošle Hospodin, tvůj Bůh, dokud nezhynou ti, kteří zbyli a skryli se před tebou.
#7:21 Nesmíš mít před nimi strach, neboť Hospodin, tvůj Bůh, Bůh veliký a vzbuzující bázeň, je uprostřed tebe.
#7:22 Hospodin, tvůj Bůh, zažene ty pronárody před tebou poznenáhlu. Nemůžeš s nimi rychle skoncovat, aby se proti tobě nerozmohla polní zvěř.
#7:23 Hospodin, tvůj Bůh, ti je však předá ochromené velikým zděšením, dokud nebudou vyhlazeni.
#7:24 Vydá ti do rukou jejich krále a zničíš jejich jméno pod nebem; nikdo se proti tobě nepostaví, dokud je nevyhladíš.
#7:25 Tesané sochy jejich bohů spálíte. Nebudeš žádostiv stříbra ani zlata z nich, nevezmeš si je, abys nepadl do léčky. Před Hospodinem, tvým Bohem, je to ohavnost.
#7:26 Nevneseš do svého domu ohavnou modlu; propadl bys klatbě jako ona. Budeš ji mít v opovržení, budeš ji mít za hnusnou ohavnost, neboť je klatá. 
#8:1 Bedlivě dodržujte každý příkaz, který ti dnes přikazuji, abyste zůstali naživu, rozmnožili se a obsadili zemi, kterou přísežně přislíbil Hospodin vašim otcům.
#8:2 Připomínej si celou tu cestu, kterou tě Hospodin, tvůj Bůh, vodil po čtyřicet let na poušti, aby tě pokořil a vyzkoušel a poznal, co je v tvém srdci, zda budeš dbát na jeho přikázání, či nikoli.
#8:3 Pokořoval tě a nechal tě hladovět, potom ti dával jíst manu, kterou jsi neznal a kterou neznali ani tvoji otcové. Tak ti dával poznat, že člověk nežije pouze chlebem, ale že člověk žije vším, co vychází z Hospodinových úst.
#8:4 Po těch čtyřicet let tvůj šat na tobě nezvetšel a noha ti neotekla.
#8:5 Uznej tedy ve svém srdci, že tě Hospodin, tvůj Bůh, vychovával, jako vychovává muž svého syna.
#8:6 Proto budeš dbát na přikázání Hospodina, svého Boha, chodit po jeho cestách a jeho se bát.
#8:7 Vždyť Hospodin, tvůj Bůh, tě uvádí do dobré země, do země s potoky plnými vody, s prameny vod propastných tůní, vyvěrajícími na pláni i v pohoří,
#8:8 do země, kde roste pšenice i ječmen, vinná réva, fíkoví a granátová jablka, do země olivového oleje a medu,
#8:9 do země, v níž budeš jíst chléb bez nedostatku, v které nebudeš postrádat ničeho, do země, jejíž kamení je železo a z jejíchž hor budeš těžit měď.
#8:10 Budeš jíst dosyta a budeš dobrořečit Hospodinu, svému Bohu, za tu dobrou zemi, kterou ti dal.
#8:11 Střez se však, abys nezapomněl na Hospodina, svého Boha, a nepřestal dbát na jeho přikázání, práva a nařízení, která ti dnes udílím.
#8:12 Až se dosyta najíš a vystavíš si pěkné domy a usídlíš se,
#8:13 až se ti rozmnoží skot a brav, až budeš mít hodně stříbra a zlata, až se ti rozmnoží všechno, co máš,
#8:14 jen ať se tvé srdce nevypíná, takže bys zapomněl na Hospodina, svého Boha, který tě vyvedl z egyptské země, z domu otroctví.
#8:15 Vodil tě velikou a hroznou pouští, kde jsou ohniví hadi a štíři, žíznivým krajem bez vody, vyvedl ti vodu z křemene skály,
#8:16 krmil tě na poušti manou, kterou tvoji otcové neznali, aby tě pokořil a vyzkoušel a aby ti nakonec prokázal dobro.
#8:17 Neříkej si v srdci: „Tohoto blahobytu jsem se domohl svou silou a zdatností svých rukou.“
#8:18 Pamatuj na Hospodina, svého Boha, neboť k nabytí blahobytu ti dává sílu on, aby utvrdil svou smlouvu, kterou přísahal tvým otcům, jak tomu je dnes.
#8:19 Jestliže však přesto na Hospodina, svého Boha, zapomeneš a budeš chodit za jinými bohy, sloužit jim a klanět se jim, dosvědčuji vám dnes, že docela vyhynete.
#8:20 Jako pronárody, které Hospodin před vámi vyhubí, tak vyhynete za to, že jste neposlouchali Hospodina, svého Boha. 
#9:1 Slyš, Izraeli, dnes přejdeš Jordán, aby sis podrobil pronárody větší a zdatnější, než jsi ty, města veliká a opevněná až k nebi,
#9:2 veliký a vysoký lid, Anákovce, o kterých víš. Slyšel jsi o nich úsloví: „Kdo se postaví proti Anákovcům?“
#9:3 Dnes poznáš, že Hospodin, tvůj Bůh, který jde před tebou, je jako sžírající oheň. On je vyhladí a on je před tebou pokoří, takže si je podrobíš a rychle je vyhubíš, jak ti Hospodin slíbil.
#9:4 Neříkej si však v srdci, až je Hospodin, tvůj Bůh, před tebou vypudí: „Pro mou spravedlnost mě Hospodin přivedl, abych obsadil tuto zemi.“ Vždyť tyto pronárody vyhání před tebou pro jejich zvůli.
#9:5 Přicházíš obsadit jejich zemi ne pro svou spravedlnost a přímost svého srdce. Hospodin, tvůj Bůh, vyhání před tebou tyto pronárody pro jejich zvůli a proto, aby splnil, co přísahal Hospodin tvým otcům, Abrahamovi, Izákovi a Jákobovi.
#9:6 Věz tedy, že ne pro tvou spravedlnost ti Hospodin, tvůj Bůh, dává tuto dobrou zemi, abys ji obsadil, neboť jsi lid tvrdé šíje.
#9:7 Pamatuj, nezapomeň, jak jsi rozlítil Hospodina, svého Boha, na poušti. Ode dne, kdy jsi vyšel z egyptské země, až do svého příchodu na toto místo jste byli vůči Hospodinu vzpurní.
#9:8 Rozlítili jste Hospodina na Chorébu. Hospodin se na vás rozhněval tak, že vás chtěl vyhladit.
#9:9 Když jsem vystoupil na horu, abych přijal kamenné desky, desky smlouvy, kterou s vámi Hospodin uzavřel, zůstal jsem na hoře čtyřicet dní a čtyřicet nocí, chleba jsem nejedl a vody nepil.
#9:10 A Hospodin mi dal obě kamenné desky psané prstem Božím, na nichž byla všechna přikázání, o kterých s vámi Hospodin mluvil na hoře zprostředku ohně v den shromáždění.
#9:11 Po uplynutí čtyřiceti dní a čtyřiceti nocí mi Hospodin dal obě kamenné desky, desky smlouvy.
#9:12 A Hospodin mi poručil: „Vstaň a rychle odtud sestup, neboť tvůj lid, který jsi vyvedl z Egypta, se vrhá do zkázy. Brzo sešli z cesty, kterou jsem jim přikázal, odlili si modlu.“
#9:13 Dále mi Hospodin řekl: „Vidím, jak je tento lid tvrdošíjný.
#9:14 Nech mě, já je vyhladím a vymažu jejich jméno pod nebem, z tebe však učiním pronárod zdatnější a početnější, než jsou oni.“
#9:15 Obrátil jsem se a sestoupil z hory, zatímco hora planula ohněm. Obě desky smlouvy jsem nesl v rukou.
#9:16 Tu jsem spatřil, jak jste zhřešili proti Hospodinu, svému Bohu. Odlili jste si sochu býčka! Brzo jste sešli z cesty, kterou vám přikázal Hospodin, váš Bůh.
#9:17 Uchopil jsem obě desky, odhodil jsem je a před vašimi zraky je roztříštil.
#9:18 Pak jsem se vrhl před Hospodinem k zemi jako poprvé a ležel jsem čtyřicet dní a čtyřicet nocí, chleba jsem nejedl a vody nepil, pro všechen váš hřích, kterým jste zhřešili, když jste se dopustili toho, co je zlé v Hospodinových očích, a tak jste ho urazili.
#9:19 Lekal jsem se hněvu a rozhořčení, jímž se Hospodin proti vám rozlítil, aby vás vyhladil. A Hospodin mě vyslyšel i tentokrát.
#9:20 Hospodin se velmi rozhněval i na Árona a chtěl ho zahladit. V oné době jsem se proto modlil i za Árona.
#9:21 A váš hříšný výtvor, který jste udělali, toho býčka, jsem vzal a spálil v ohni, roztloukl jsem jej a rozemlel jsem jej nadobro až na jemný prach; prach z něho jsem vsypal do potoka, který stéká z hory.
#9:22 Také v Tabéře a v Masse a v Kibrót-taavě jste rozlítili Hospodina.
#9:23 A když vás Hospodin poslal z Kádeš-barneje se slovy: „Táhněte vzhůru a obsaďte zemi, kterou jsem vám dal“, vzdorovali jste rozkazu Hospodina, svého Boha, nevěřili jste mu a neuposlechli jste ho.
#9:24 Byli jste vůči Hospodinu vzpurní ode dne, co vás znám.
#9:25 Tu jsem se vrhal před Hospodina, čtyřicet dní a čtyřicet nocí jsem se vrhal před něho, protože Hospodin řekl, že vás vyhladí.
#9:26 Modlil jsem se k Hospodinu: „Panovníku Hospodine, neuvaluj zkázu na svůj lid, na své dědictví, které jsi vykoupil svou velikou mocí a vyvedl z Egypta pevnou rukou.
#9:27 Rozpomeň se na své služebníky, na Abrahama, Izáka a Jákoba, a nepřihlížej k zatvrzelosti tohoto lidu, k jeho zvůli a hříchu,
#9:28 aby neříkali v zemi, z níž jsi nás vyvedl: ‚Protože je Hospodin nemohl uvést do země, kterou jim slíbil, a měl je v nenávisti, vyvedl je a na poušti je usmrtil.‘
#9:29 Oni jsou přece tvůj lid a tvé dědictví, které jsi vyvedl svou velikou silou a vztaženou paží.“ 
#10:1 V oné době mi Hospodin řekl: „Vytesej si dvě kamenné desky, jako byly ty první, a vystup ke mně na horu. Také si uděláš dřevěnou schránu.
#10:2 Na ty desky napíši táž slova, která byla na prvních deskách, jež jsi roztříštil. Pak je vložíš do schrány.“
#10:3 Udělal jsem tedy schránu z akáciového dřeva a vytesal jsem dvě kamenné desky, jako byly ty první, a vystoupil jsem na horu s oběma deskami v rukou.
#10:4 I napsal na desky totéž, co bylo napsáno poprvé, desatero přikázání, která k vám mluvil Hospodin na hoře zprostředku ohně v den shromáždění; pak mi je Hospodin dal.
#10:5 Obrátil jsem se a sestoupil z hory. Desky jsem vložil do schrány, kterou jsem udělal, aby tam byly, jak mi Hospodin přikázal.
#10:6 Izraelci potom táhli z Beerótu Jaakanovců do Mósery; tam zemřel Áron a byl tam pochován. Jako kněz po něm sloužil jeho syn Eleazar.
#10:7 Odtud vytáhli do Gudgódu a z Gudgódu do Jotby, do země potoků plných vody.
#10:8 V oné době oddělil Hospodin kmen Léviho, aby nosil schránu Hospodinovy smlouvy, aby stál před Hospodinem, přisluhoval mu a dával požehnání v jeho jménu. Tak tomu je dodnes.
#10:9 Proto se Lévimu nedostalo podílu ani dědictví s jeho bratry. Sám Hospodin je jeho dědictvím, jak k němu mluvil Hospodin, tvůj Bůh.
#10:10 Já jsem pak zůstal na hoře jako poprvé čtyřicet dní a čtyřicet nocí a Hospodin mě vyslyšel i tentokrát; Hospodin nechtěl tvou zkázu.
#10:11 A Hospodin mi řekl: „Vzhůru, vytáhni před lidem, aby šli obsadit zemi, o které jsem přísahal jejich otcům, že jim ji dám.“
#10:12 Nyní tedy, Izraeli, co od tebe požaduje Hospodin, tvůj Bůh? Jen aby ses bál Hospodina, svého Boha, chodil po všech jeho cestách, miloval ho a sloužil Hospodinu, svému Bohu, celým svým srdcem a celou svou duší,
#10:13 abys dbal na Hospodinova přikázání a nařízení, která ti dnes udílím, aby s tebou bylo dobře.
#10:14 Hle, Hospodinu, tvému Bohu, patří nebesa i nebesa nebes, země a všechno, co je na ní.
#10:15 Avšak Hospodin přilnul jenom k tvým otcům, zamiloval si je a vyvolil jejich potomstvo, vás, ze všech národů, jak tomu je dnes.
#10:16 Obřežte tedy svá neobřezaná srdce a už nebuďte tvrdošíjní.
#10:17 Vždyť Hospodin, váš Bůh, je Bůh bohů a Pán pánů, Bůh veliký, všemocný a vzbuzující bázeň, který nebere ohled na osobu a nepřijímá úplatek,
#10:18 ale zjednává právo sirotku a vdově, miluje hosta a dává mu chléb a šat.
#10:19 Milujte tedy hosta, neboť jste byli hosty v egyptské zemi.
#10:20 Hospodina, svého Boha, se budeš bát, jemu budeš sloužit, k němu se přimkneš a v jeho jménu budeš přísahat.
#10:21 On je tvá chvála. On je tvůj Bůh, který s tebou učinil tyto veliké a hrozné věci, které jsi viděl na vlastní oči.
#10:22 Tvoji otcové sestoupili do Egypta v počtu sedmdesáti duší, ale nyní tě Hospodin, tvůj Bůh, učinil tak početným jako nebeské hvězdy. 
#11:1 Budeš milovat Hospodina, svého Boha, budeš dbát na to, co ti svěřil, na jeho nařízení, práva a přikázání po všechny dny.
#11:2 Znáte dnes přece to, co vaši synové nepoznali a neviděli, totiž napomínání Hospodina, vašeho Boha, jeho velikost, jeho pevnou ruku a vztaženou paži,
#11:3 jeho znamení a činy, které vykonal uprostřed Egypta na faraónovi, králi egyptském, a celé jeho zemi,
#11:4 všechno, co učinil egyptskému vojsku, jeho koním a vozům, jak je zaplavil vodou Rákosového moře, když vás pronásledovali, ano, Hospodin je vyhubil - tak tomu je dodnes;
#11:5 a co s vámi učinil na poušti, než jste přišli až na toto místo,
#11:6 a jak naložil s Dátanem a Abíramem, syny Rúbenovce Elíaba, jak země rozevřela svůj chřtán a pohltila je uprostřed celého Izraele i s jejich rodinami a stany i se všemi jejich přívrženci.
#11:7 Na vlastní oči jste přece spatřili celé to veliké Hospodinovo dílo, které vykonal.
#11:8 Dbejte proto na každý příkaz, který vám dnes udílím, abyste byli rozhodní, až půjdete obsadit zemi, do níž táhnete a kterou máte obsadit,
#11:9 a abyste byli dlouho živi na zemi, o které Hospodin přísahal, že ji dá vašim otcům a jejich potomstvu, v zemi oplývající mlékem a medem.
#11:10 Země, kterou přicházíš obsadit, není totiž jako země egyptská, z níž jste vyšli, kterou jsi oséval semenem a zavlažoval šlapáním čerpadla jako zelinářskou zahradu.
#11:11 Země, do níž táhnete a kterou máte obsadit, je země hor a plání. Pije vodu z nebeského deště.
#11:12 Je to země, o niž Hospodin, tvůj Bůh, pečuje. Oči Hospodina, tvého Boha, jsou na ni neustále upřeny, od začátku roku až do konce.
#11:13 Jestliže budete opravdu poslouchat má přikázání, která vám dnes udílím, totiž abyste milovali Hospodina, svého Boha, a sloužili mu celým svým srdcem a celou svou duší,
#11:14 dám vaší zemi déšť v pravý čas, déšť podzimní i jarní, a budeš sklízet své obilí, svůj mošt a olej
#11:15 a na tvém poli dám růst bylině pro tvůj dobytek; budeš jíst dosyta.
#11:16 Střezte se, aby se vaše srdce nedalo zlákat, takže byste se odchýlili, sloužili jiným bohům a klaněli se jim.
#11:17 To by Hospodin vzplanul proti vám hněvem a zavřel by nebesa, takže by nebylo deště, z půdy by se nic neurodilo a vy byste v té dobré zemi, kterou vám Hospodin dává, rychle vyhynuli.
#11:18 Tato má slova si vložte do srdce a do své duše, přivažte si je jako znamení na ruku a ať jsou jako pásek na čele mezi vašima očima.
#11:19 Budete jim vyučovat své syny a rozmlouvat o nich, ať budeš sedět doma nebo půjdeš cestou, ať budeš uléhat nebo vstávat.
#11:20 Napíšeš je na veřeje svého domu a na své brány,
#11:21 abyste na zemi, o které Hospodin přísahal vašim otcům, že jim ji dá, dlouho žili vy i vaši synové, aby vaše dny byly jako dny nebes nad zemí.
#11:22 Jestliže budete bedlivě a cele dbát na toto přikázání, které vám přikazuji dodržovat, totiž abyste milovali Hospodina, svého Boha, chodili po všech jeho cestách a přimkli se k němu,
#11:23 vyžene Hospodin před vámi všechny tyto pronárody, takže si podrobíte národy větší a zdatnější, než jste vy.
#11:24 Každé místo, na něž vaše noha šlápne, bude vaše: od pouště po Libanón, od Řeky, řeky Eufratu, až k Zadnímu moři, to vše bude vaše území.
#11:25 Nikdo se proti vám nepostaví, Hospodin, váš Bůh, uvalí strach a bázeň před vámi na celou zemi, kde stanete, jak k vám mluvil.
#11:26 Hleď, dnes vám předkládám požehnání i zlořečení:
#11:27 požehnání, když budete poslouchat příkazy Hospodina, svého Boha, které já vám dnes přikazuji,
#11:28 a zlořečení, když nebudete poslouchat příkazy Hospodina, svého Boha, sejdete-li z cesty, kterou vám dnes přikazuji, a budete-li chodit za jinými bohy, k nimž se nemáte znát.
#11:29 Až tě uvede Hospodin, tvůj Bůh, do země, kterou přicházíš obsadit, budeš dávat požehnání na hoře Gerizímu a zlořečení na hoře Ébalu.
#11:30 Ty hory jsou za Jordánem, za cestou na západ, v zemi Kenaanců sídlících v pustině naproti Gilgálu při božišti Móre.
#11:31 Přejdete totiž Jordán a půjdete obsadit zemi, kterou vám dává Hospodin, váš Bůh. Obsadíte ji a usadíte se v ní.
#11:32 Proto bedlivě dodržujte všechna nařízení a práva, která vám dnes předkládám. 
#12:1 Toto jsou nařízení a práva, která budete bedlivě dodržovat v zemi, kterou ti dal do vlastnictví Hospodin, Bůh tvých otců, po všechny dny, v nichž budete živi na zemi.
#12:2 Úplně zničíte všechna místa, kde pronárody, jež si podrobíte, sloužily svým bohům na vysokých horách i na pahorcích a pod každým zeleným stromem.
#12:3 Jejich oltáře zboříte, jejich posvátné sloupy roztříštíte, jejich posvátné kůly spálíte, tesané sochy jejich bohů pokácíte a jejich jméno z toho místa vyhladíte.
#12:4 Pro Hospodina, svého Boha, nesmíte udělat nic takového,
#12:5 ale budete vyhledávat místo, které si vyvolí Hospodin, váš Bůh, ze všech vašich kmenů, aby tam spočinulo jeho jméno; tam budeš přicházet.
#12:6 Tam budete přinášet své zápalné oběti a oběti svých hodů i desátky a oběť pozdvihování svých rukou, záslibné a dobrovolné dary i prvorozené kusy ze svého skotu a bravu.
#12:7 Budete tam jíst před Hospodinem, svým Bohem, a vy i váš dům se budete radovat ze všeho, k čemu jste přiložili ruku, v čem ti požehnal Hospodin, tvůj Bůh.
#12:8 Nebudete už dělat to, co zde děláme dnes, co každý sám pokládá za správné.
#12:9 Dosud jste totiž nevešli do místa odpočinutí a neujali se dědictví, které ti dává Hospodin, tvůj Bůh.
#12:10 Ale až přejdete Jordán a usadíte se v zemi, kterou vám přiděluje do dědictví Hospodin, váš Bůh, až vám dá odpočinutí ode všech vašich okolních nepřátel, takže budete sídlit bezpečně,
#12:11 potom na místo, které si vyvolí Hospodin, váš Bůh, aby tam přebývalo jeho jméno, budete přinášet všechno, co vám přikazuji: své zápalné oběti a oběti svých hodů i desátky a oběť pozdvihování svých rukou i všechno nejlepší ze svých záslibných darů, které jste Hospodinu přislíbili.
#12:12 Budete se radovat před Hospodinem, svým Bohem, vy i vaši synové a vaše dcery, vaši otroci a vaše otrokyně i lévijec, který žije v tvých branách, protože nemá mezi vámi podíl ani dědictví.
#12:13 Střez se, abys neobětoval své zápalné oběti na jakémkoli místě, které by sis vyhlédl.
#12:14 Jenom na tom místě, které si vyvolí Hospodin v jednom z tvých kmenů, budeš obětovat své zápalné oběti, tam budeš vykonávat všechno, co ti přikazuji.
#12:15 Porážet dobytek a jíst maso můžeš ovšem ve všech svých branách, kdykoli po něm zatoužíš, podle požehnání, které ti dal Hospodin, tvůj Bůh; smí je jíst nečistý i čistý i nečistý jako gazelu nebo jelena.
#12:16 Jenom krev jíst nebudete. Vylejete ji na zem jako vodu.
#12:17 Ale ve svých branách nesmíš jídat desátek ze svého obilí, moštu a čerstvého oleje ani prvorozené kusy ze svého skotu a bravu ani jakýkoli záslibný dar, který jsi přislíbil, ani dobrovolné dary ani oběť pozdvihování svých rukou.
#12:18 Jen před Hospodinem, svým Bohem, na místě, které si Hospodin, tvůj Bůh, vyvolí, tam je budeš jíst ty i tvůj syn a tvá dcera, tvůj otrok a tvá otrokyně i lévijec, který žije v tvých branách. Před Hospodinem, svým Bohem, se budeš radovat ze všeho, k čemu jsi přiložil ruku.
#12:19 Pokud budeš žít ve své zemi, varuj se toho, abys lévijce nechal opuštěného.
#12:20 Až Hospodin, tvůj Bůh, rozšíří tvé území, jak ti slíbil, a ty řekneš: „Chtěl bych jíst maso“, protože zatoužíš jíst maso, můžeš je jíst podle své chuti.
#12:21 Když bude od tebe daleko místo, které Hospodin, tvůj Bůh, vyvolí, aby tam spočinulo jeho jméno, porazíš kus ze svého skotu či bravu, který ti dal Hospodin, jak jsem ti přikázal, a budeš jíst ve svých branách zcela podle své chuti.
#12:22 Budeš je jíst tak, jak se jí gazela nebo jelen; může je jíst stejně nečistý jako čistý.
#12:23 Jenom rozhodně odmítej jíst krev, neboť krev je život; proto nebudeš jíst s masem i život.
#12:24 Nebudeš ji jíst, nýbrž ji vyleješ na zem jako vodu.
#12:25 Nebudeš ji jíst, aby se dobře vedlo tobě i tvým synům po tobě, když budeš dělat, co je správné v Hospodinových očích.
#12:26 Jen to svaté, co máš obětovat, a své záslibné dary budeš nosit na místo, které Hospodin vyvolí.
#12:27 Připravíš své zápalné oběti, maso i krev, na oltáři Hospodina, svého Boha. Krev tvých obětí k hodům bude vylita na oltář Hospodina, tvého Boha, kdežto maso sníš.
#12:28 Bedlivě poslouchej všechna tato slova, která ti přikazuji, aby se vždycky dobře vedlo tobě i tvým synům po tobě, když budeš dělat, co je v očích Hospodina, tvého Boha, dobré a správné.
#12:29 Až Hospodin, tvůj Bůh, před tebou vyplení pronárody, které si jdeš podrobit, a až si je podrobíš a usadíš se v jejich zemi,
#12:30 dej si pozor, abys neuvízl v léčce, která po nich zbude, až budou před tebou už vyhlazeny: abys nevyhledával jejich bohy a neřekl: „Jak sloužily tyto pronárody svým bohům, tak to budu dělat i já!“
#12:31 Něco takového nesmíš udělat pro Hospodina, svého Boha, neboť všechno, co ony činily pro své bohy, Hospodin nenávidí jako ohavnost; vždyť ony pro své bohy spalují dokonce své syny a dcery. 
#13:1 Všechno, co vám přikazuji, budete bedlivě dodržovat. Nic k tomu nepřidáš a nic z toho neubereš.
#13:2 Kdyby povstal ve tvém středu prorok nebo někdo, kdo hádá ze snů, a nabídl ti znamení nebo zázrak,
#13:3 i kdyby se dostavilo to znamení nebo ten zázrak, o němž ti mluvil, když říkal: „Pojďme za jinými bohy“, které jsi neznal, „a služme jim“,
#13:4 neuposlechneš slov takového proroka nebo toho, kdo hádá ze snů. To vás zkouší Hospodin, váš Bůh, aby poznal, zda milujete Hospodina, svého Boha, celým svým srdcem a celou svou duší.
#13:5 Hospodina, svého Boha, budete následovat a jeho se budete bát, budete dbát na jeho přikázání a poslouchat ho, jemu budete sloužit a k němu se přimknete.
#13:6 Avšak takový prorok nebo ten, kdo hádá ze snů, bude usmrcen, protože přemlouval k odpadnutí od Hospodina, vašeho Boha, který vás vyvedl z egyptské země a vykoupil tě z domu otroctví. Chtěl tě svést z cesty, po které ti Hospodin, tvůj Bůh, přikázal chodit. I odstraníš zlo ze svého středu.
#13:7 Kdyby tě tvůj bratr, syn tvé matky, nebo tvůj syn nebo tvá dcera nebo tvá vlastní žena nebo tvůj nejmilejší přítel potají ponoukal: „Pojďme sloužit jiným bohům“, které jsi neznal ty ani tvoji otcové,
#13:8 některým z bohů těch národů, které jsou kolem vás, ať blízko tebe nebo daleko od tebe, od jednoho konce země až k druhému konci země,
#13:9 nepřivolíš mu a neuposlechneš ho, nebudeš ho litovat ani s ním mít soucit ani ho krýt.
#13:10 Musíš ho zabít. Nejprve se proti němu pozdvihne tvoje ruka, aby ho usmrtila, potom ruce všeho lidu.
#13:11 Budeš ho kamenovat, dokud nezemře, protože tě usiloval odvést od Hospodina, tvého Boha, který tě vyvedl z egyptské země, z domu otroctví.
#13:12 Celý Izrael ať o tom uslyší a ať se bojí. Nikdy ať se mezi vámi nestane něco tak zlého.
#13:13 Kdybys uslyšel o některém svém městě, které ti dává Hospodin, tvůj Bůh, aby ses v něm usadil,
#13:14 že z tvého středu vyšli muži ničemní a svádějí obyvatele města slovy: „Pojďme sloužit jiným bohům“, které jste neznali,
#13:15 budeš pátrat a zkoumat a dobře se vyptávat; bude-li to jistá pravda, že se taková ohavnost ve tvém středu stala,
#13:16 úplně vybiješ obyvatele toho města ostřím meče, zničíš je ostřím meče jako klaté, i všechno, co je v něm, i jeho dobytek.
#13:17 Všechnu kořist z něho shromáždíš na jeho prostranství a město i všechnu kořist z něho spálíš jako celopal pro Hospodina, svého Boha. Zůstane navěky pahorkem sutin, nikdy nebude vystavěno.
#13:18 Ať ti neulpí na ruce nic z toho, co propadlo klatbě, aby se Hospodin od svého planoucího hněvu odvrátil a udělil ti slitování. Slituje se nad tebou a rozmnoží tě, jak přísahal tvým otcům,
#13:19 když budeš poslouchat Hospodina, svého Boha, dbát na všechny jeho příkazy, které ti dnes udílím, a dodržovat, co je správné v očích Hospodina, tvého Boha. 
#14:1 Jste synové Hospodina, svého Boha. Nebudete si pro mrtvého dělat smuteční zářezy, ani mezi očima lysinu.
#14:2 Jsi přece svatý lid Hospodina, svého Boha. Tebe si Hospodin vyvolil ze všech lidských pokolení, která jsou na tváři země, abys byl jeho lidem, zvláštním vlastnictvím.
#14:3 Nesmíš jíst nic ohavného.
#14:4 Smíte jíst tato zvířata: býka, ovci a kozu,
#14:5 jelena, gazelu, daňka, kozorožce, antilopu díšona, tura stepního a divokou kozu,
#14:6 zkrátka všechna zvířata, která mají kopyta rozdělená tak, že jsou obě kopyta úplně rozpolcená, přežvýkavce mezi zvířaty; ty jíst smíte.
#14:7 Z přežvýkavců, z těch, co mají kopyta rozdělená rozpolcením, nesmíte však jíst velblouda, zajíce a damana. Jsou to přežvýkavci, ale kopyta nemají úplně rozdělená; budou pro vás nečistí.
#14:8 Ani vepře; má sice kopyta rozdělená, ale nepřežvykuje; bude pro vás nečistý. Jejich maso nesmíte jíst, jejich zdechliny se nedotknete.
#14:9 Ze všeho, co je ve vodě, smíte jíst toto: všechno, co má ploutve a šupiny. To smíte jíst.
#14:10 Co nemá ploutve ani šupiny, jíst nesmíte; bude to pro vás nečisté.
#14:11 Smíte jíst všechno čisté ptactvo.
#14:12 Jen tyto z nich jíst nesmíte: orla, orlosupa a orla mořského,
#14:13 luňáka, jestřába a různé druhy supů,
#14:14 žádný druh havranů,
#14:15 pštrosa, sovu, racka a různé druhy sokolů,
#14:16 kulicha, výra a sovu pálenou,
#14:17 pelikána, mrchožrouta a kormorána,
#14:18 čápa a různé druhy volavek, dudka a netopýra.
#14:19 Nečistá bude pro vás i všechna létající havěť; nesmí se jíst.
#14:20 Všechno čisté ptactvo jíst smíte.
#14:21 Nesmíte jíst žádnou zdechlinu. Buď ji dáš bezdomovci, který žije v tvých branách, aby ji jedl, nebo ji prodáš cizinci. Vždyť jsi svatý lid Hospodina, svého Boha. Nebudeš vařit kůzle v mléce jeho matky.
#14:22 Budeš odvádět desátky z veškeré úrody své setby, která každoročně na poli vyroste.
#14:23 Na místě, které si on vyvolí, aby tam přebývalo jeho jméno, budeš jíst před Hospodinem, svým Bohem, desátky ze svého obilí, moštu a čerstvého oleje i prvorozených kusů svého skotu a bravu, aby ses učil bát Hospodina, svého Boha, po všechny dny.
#14:24 Kdybys měl dlouhou cestu a nemohl je donést, protože místo, které vyvolí Hospodin, tvůj Bůh, aby tam spočinulo jeho jméno, bude od tebe vzdáleno, až Hospodin, tvůj Bůh, ti požehná,
#14:25 směníš je za stříbro, stříbro zavážeš, vezmeš do ruky a půjdeš na místo, které si Hospodin, tvůj Bůh, vyvolí.
#14:26 Tam koupíš za to stříbro všechno, po čem zatoužíš, skot, brav, víno, opojný nápoj, cokoli budeš chtít, a budeš jíst před Hospodinem, svým Bohem, a radovat se i se svým domem.
#14:27 Ani lévijce, který žije v tvých branách, nenecháš opuštěného, protože nemá s tebou podíl ani dědictví.
#14:28 Každého třetího roku odneseš všechny desátky ze své úrody toho roku a složíš je ve svých branách.
#14:29 I přijde lévijec, protože nemá s tebou podíl ani dědictví, a bezdomovec, sirotek a vdova, kteří žijí v tvých branách, a budou jíst dosyta, aby ti Hospodin, tvůj Bůh, žehnal při každé práci, kterou bude tvá ruka konat. 
#15:1 Každého sedmého roku budeš slavit léto promíjení dluhu.
#15:2 Toto je způsob promíjení dluhu: Každý věřitel svému bližnímu promine, co mu půjčil. Nebude na svého bližního, svého bratra, naléhat, protože je vyhlášeno Hospodinovo promíjení dluhu.
#15:3 Na cizince naléhat smíš, ale co máš u svého bratra, to mu tvá ruka promine.
#15:4 Ať není u tebe potřebného, neboť Hospodin ti bohatě požehná v zemi, kterou ti Hospodin, tvůj Bůh, dává do dědictví, abys ji obsadil.
#15:5 Jen ochotně poslouchej Hospodina, svého Boha, a bedlivě a cele dodržuj tento příkaz, který ti dnes přikazuji.
#15:6 Vždyť ti Hospodin, tvůj Bůh, požehná, jak ti řekl. Budeš poskytovat půjčky mnohým pronárodům, ale sám si nebudeš vypůjčovat. Budeš ovládat mnohé pronárody, ale tebe neovládnou.
#15:7 Bude-li u tebe potřebný někdo z tvých bratří, v některé z tvých bran v tvé zemi, kterou ti dává Hospodin, tvůj Bůh, nebude tvé srdce zpupné a nezavřeš svou ruku před svým potřebným bratrem.
#15:8 Ochotně mu otvírej svou ruku a poskytni mu dostatečnou půjčku podle toho, kolik ve svém nedostatku potřebuje.
#15:9 Dej si pozor, aby v tvém srdci nevyvstala ničemná myšlenka, že se blíží sedmý rok, rok promíjení dluhů; že tedy budeš na svého potřebného bratra nevlídný a nedáš mu nic. On by kvůli tobě volal k Hospodinu a na tobě by byl hřích.
#15:10 Dávej mu štědře a nebuď skoupý, když mu máš něco dát, neboť kvůli tomu ti Hospodin, tvůj Bůh, požehná ve všem, co děláš, ve všem, k čemu přiložíš ruku.
#15:11 Potřebný ze země nevymizí. Proto ti přikazuji: Ve své zemi ochotně otvírej ruku svému utištěnému a potřebnému bratru.
#15:12 Bude-li ti prodán tvůj bratr, Hebrej nebo Hebrejka, bude tvým otrokem po šest let, ale sedmého roku jej propustíš na svobodu.
#15:13 Když ho propustíš na svobodu, nepropustíš ho s prázdnou.
#15:14 Štědře jej obdaruješ ze svého stáda i humna a lisu. V čem ti Hospodin, tvůj Bůh, požehnal, z toho mu dáš.
#15:15 Pamatuj, že jsi byl otrokem v egyptské zemi a že tě Hospodin, tvůj Bůh, vykoupil. Proto ti dnes toto přikazuji.
#15:16 Kdyby ti řekl: „Nechci od tebe odejít“, protože miluje tebe i tvůj dům a je mu u tebe dobře,
#15:17 vezmeš šídlo, připíchneš mu ucho ke dveřím a bude provždy tvým otrokem. Stejně učiníš i své otrokyni.
#15:18 Nebude ti zatěžko propustit ho na svobodu, neboť ti vysloužil dvojnásobnou mzdu nádenickou za šest let, a Hospodin, tvůj Bůh, ti požehná ve všem, co budeš konat.
#15:19 Každého prvorozeného samce, který se narodí v tvém skotu nebo bravu, oddělíš jako svatého pro Hospodina, svého Boha. Nebudeš vykonávat žádnou práci se svým prvorozeným býčkem, nebudeš stříhat prvorozené kusy ze svého bravu.
#15:20 Před Hospodinem, svým Bohem, ho budeš jíst se svým domem rok co rok na místě, které Hospodin vyvolí.
#15:21 Avšak bude-li na něm nějaká vada, bude-li kulhavý nebo slepý nebo bude mít jakoukoli zlou vadu, nebudeš ho obětovat Hospodinu, svému Bohu.
#15:22 Budeš ho jíst ve svých branách, stejně nečistý jako čistý, jako gazelu nebo jelena.
#15:23 Jen jeho krev nesmíš jíst. Vyleješ ji na zem jako vodu. 
#16:1 Dbej na měsíc ábíb (to je měsíc klasů), abys slavil hod beránka Hospodinu, svému Bohu, protože tě Hospodin, tvůj Bůh, v měsíci ábíbu vyvedl v noci z Egypta.
#16:2 Hospodinu, svému Bohu, budeš obětovat hod beránka z bravu nebo skotu na místě, které Hospodin vyvolí, aby tam přebývalo jeho jméno.
#16:3 Nebudeš při něm jíst nic kvašeného. Po sedm dní budeš při něm jíst nekvašené chleby, chléb poroby, neboť jsi chvatně vyšel z egyptské země. Po všechny dny svého života si budeš připomínat den, kdy jsi vyšel z egyptské země.
#16:4 Na celém tvém území se u tebe po sedm dní neuvidí kvas. Z masa, které připravíš prvního dne navečer, nezůstane nic přes noc do rána.
#16:5 Nebudeš moci připravovat hod beránka v kterékoli ze svých bran, které ti dává Hospodin, tvůj Bůh.
#16:6 Jenom na místě, které Hospodin, tvůj Bůh, vyvolí, aby tam přebývalo jeho jméno, připravíš hod beránka večer při západu slunce v ten čas, kdy jsi vyšel z Egypta.
#16:7 Budeš vařit a budeš jíst na místě, které vyvolí Hospodin, tvůj Bůh. Ráno se pak vrátíš a půjdeš ke svým stanům.
#16:8 Šest dní budeš jíst nekvašené chleby, sedmého dne bude slavnostní shromáždění pro Hospodina, tvého Boha. Nebudeš vykonávat žádnou práci.
#16:9 Odpočítáš si sedm týdnů. Od chvíle, kdy přiložíš srp ke stojícímu obilí, začneš počítat sedm týdnů.
#16:10 Pak budeš slavit Hospodinu, svému Bohu, slavnost týdnů. Dáš dobrovolně podle své možnosti z toho, v čem ti Hospodin, tvůj Bůh, požehná.
#16:11 A budeš se radovat před Hospodinem, svým Bohem, na místě, které vyvolí Hospodin, tvůj Bůh, aby tam přebývalo jeho jméno, ty i tvůj syn a tvá dcera, tvůj otrok a tvá otrokyně, lévijec, který žije v tvých branách, i bezdomovec, sirotek a vdova, kteří jsou mezi vámi.
#16:12 Budeš si připomínat, že jsi byl v Egyptě otrokem. Proto budeš bedlivě dodržovat tato nařízení.
#16:13 Slavnost stánků budeš slavit po sedm dní, až shromáždíš úrodu ze svého humna a lisu.
#16:14 Při své slavnosti se budeš radovat ty i tvůj syn a tvá dcera, tvůj otrok a tvá otrokyně, lévijec i bezdomovec, sirotek a vdova, kteří žijí v tvých branách.
#16:15 Sedm dní budeš slavit slavnost Hospodina, svého Boha, na místě, které Hospodin vyvolí. Vždyť Hospodin, tvůj Bůh, ti požehná na veškeré tvé úrodě a při všem, co budeš dělat. Oddej se proto radosti.
#16:16 Každý z vás, kdo je mužského pohlaví, se ukáže třikrát v roce před tváří Hospodina, tvého Boha, na místě, které on vyvolí: při slavnosti nekvašených chlebů, při slavnosti týdnů a při slavnosti stánků. A neukáže se před Hospodinovou tváří s prázdnou.
#16:17 Každý přinese, co může dát, podle požehnání Hospodina, tvého Boha, které ti udělil.
#16:18 Ve všech svých branách, které ti Hospodin, tvůj Bůh, pro tvoje kmeny dává, si ustanovíš soudce a správce. Budou soudit lid podle spravedlivého práva.
#16:19 Nepřevrátíš právo, nebudeš nikomu stranit, nepřijmeš úplatek. Úplatek oslepuje oči moudrých a překrucuje slova spravedlivých.
#16:20 Budeš usilovat o spravedlnost, a jen o spravedlnost, abys zůstal naživu a obsadil zemi, kterou ti dává Hospodin, tvůj Bůh.
#16:21 Nevsadíš si posvátný kůl ani žádný strom při oltáři Hospodina, svého Boha, který si uděláš.
#16:22 Nevztyčíš si posvátný sloup; to Hospodin, tvůj Bůh, nenávidí. 
#17:1 Nebudeš obětovat Hospodinu, svému Bohu, dobytče ze skotu nebo bravu, na němž je vada, cokoli špatného, protože to má Hospodin, tvůj Bůh, za ohavnost.
#17:2 Vyskytne-li se u tebe v některé z tvých bran, které ti Hospodin, tvůj Bůh, dává, muž nebo žena, kteří by se dopustili toho, co je zlé v očích Hospodina, tvého Boha, přestoupili by jeho smlouvu
#17:3 a odešli sloužit jiným bohům a klanět se jim, slunci nebo měsíci anebo celému nebeskému zástupu, což jsem nepřikázal,
#17:4 a bude-li ti to oznámeno nebo o tom uslyšíš, dobře si to prošetříš. Bude-li to jistá pravda, že byla spáchána v Izraeli taková ohavnost,
#17:5 vyvedeš toho muže nebo tu ženu, kteří se dopustili té zlé věci, ke svým branám a toho muže nebo tu ženu budete kamenovat, dokud nezemřou.
#17:6 Ten, kdo má zemřít, bude usmrcen na základě výpovědi dvou nebo tří svědků. Nebude usmrcen na základě výpovědi jediného svědka.
#17:7 Svědkové na něho vztáhnou ruku jako první, aby ho usmrtili, potom všechen ostatní lid. Tak odstraníš zlo ze svého středu.
#17:8 Naskytne-li se ti nějaký mimořádný právní případ, rozsoudit, kdo je vinen vraždou, rozepří, ublížením na těle, tedy jakýkoli sporný případ ve tvých branách, vydáš se k místu, které si Hospodin, tvůj Bůh, vyvolí.
#17:9 Přijdeš k lévijským kněžím a k soudci, který tam bude v těch dnech, dotážeš se a oni ti oznámí právní nález.
#17:10 Budeš pak jednat podle výroku, který ti oznámí z onoho místa, které vyvolí Hospodin. Bedlivě dodržíš všechno, o čem tě oni poučí.
#17:11 Budeš jednat podle znění zákona, o němž tě poučí, a podle práva, které ti vysvětlí. Neuchýlíš se napravo ani nalevo od toho, co ti oni oznámí.
#17:12 Kdo bude jednat opovážlivě, že by neposlechl kněze, který tam stojí ve službě Hospodina, tvého Boha, nebo soudce, ten zemře. Tak odstraníš zlo z Izraele.
#17:13 Všechen lid ať to slyší a bojí se a ať už nejedná opovážlivě.
#17:14 Až vejdeš do země, kterou ti dává Hospodin, tvůj Bůh, obsadíš ji a usadíš se v ní. I řekneš si: „Ustanovím nad sebou krále jako všechny pronárody kolem mne.“
#17:15 Ustanovíš tedy nad sebou za krále jen toho, koho si vyvolí Hospodin, tvůj Bůh. Ustanovíš nad sebou krále ze svých bratří. Nesmíš dosadit nad sebou cizince, který není tvým bratrem.
#17:16 Avšak ať nemá mnoho koní a ať neuvádí lid zpátky do Egypta, aby měl více koní. Hospodin vám přece řekl: „Nikdy se už touto cestou nevracejte.“
#17:17 Také ať nemá mnoho žen, aby se jeho srdce neodvrátilo od Boha. Ani stříbra a zlata ať nemá příliš mnoho.
#17:18 Když dosedne na svůj královský trůn, dá si napsat do knihy opis tohoto zákona, opatrovaného lévijskými kněžími.
#17:19 Bude jej mít u sebe a bude v něm číst po všechny dny svého života, aby se učil bát Hospodina, svého Boha, a bedlivě dodržoval všechna slova tohoto zákona a tato nařízení;
#17:20 aby se jeho srdce nevypínalo nad jeho bratry a neuchyloval se od příkazu napravo ani nalevo; aby byl dlouho živ ve svém království uprostřed Izraele on i jeho synové. 
#18:1 Lévijští kněží, celý Léviův kmen, nebudou mít podíl ani dědictví s Izraelem. Budou živi z ohnivých obětí Hospodinových, tedy z jeho dědictví.
#18:2 Nebudou mít dědictví uprostřed svých bratří. Jejich dědictvím bude sám Hospodin, jak jim přislíbil.
#18:3 Kněžím podle práva náleží od lidu, od těch, kteří přinášejí k obětnímu hodu dobytče ze skotu nebo z bravu, toto: knězi dají plece, obě čelisti a žaludek.
#18:4 Budeš mu dávat prvotiny svého obilí, moštu a oleje i prvotiny své ovčí stříže.
#18:5 Vždyť si jej vyvolil Hospodin, tvůj Bůh, ze všech tvých kmenů, aby po všechny dny stál on a jeho synové ve službě ve jménu Hospodinově.
#18:6 Přijde-li lévijec z některé z tvých bran odkudkoli z Izraele, kde pobývá jako host, přijde-li pln touhy na místo, které Hospodin vyvolí,
#18:7 bude konat službu ve jménu Hospodina, svého Boha, jako ostatní jeho bratři lévijci, kteří tam stojí před Hospodinem.
#18:8 Bude jíst týž podíl, kromě toho, co by prodal po otcích.
#18:9 Až vstoupíš do země, kterou ti dává Hospodin, tvůj Bůh, neuč se jednat podle ohavností oněch pronárodů.
#18:10 Ať se u tebe nevyskytne nikdo, kdo by provedl svého syna nebo svou dceru ohněm, věštec obírající se věštbami, mrakopravec ani hadač ani čaroděj
#18:11 ani zaklínač ani ten, kdo se doptává duchů zemřelých, ani jasnovidec ani ten, kdo se dotazuje mrtvých.
#18:12 Každého, kdo činí tyto věci, má Hospodin v ohavnosti. Právě pro tyto ohavnosti Hospodin, tvůj Bůh, před tebou vyhání ony pronárody.
#18:13 Budeš se dokonale držet Hospodina, svého Boha.
#18:14 Tyto pronárody, které si podrobíš, poslouchají mrakopravce a věštce, ale tobě to Hospodin, tvůj Bůh, nedovolil.
#18:15 Hospodin, tvůj Bůh, ti povolá z tvého středu, z tvých bratří, proroka jako jsem já. Jeho budete poslouchat,
#18:16 zcela podle toho, co jsi žádal od Hospodina, svého Boha, na Chorébu v den shromáždění: „Kéž neslyším už hlas Hospodina, svého Boha, a nevidím už ten veliký oheň, abych nezemřel.“
#18:17 Hospodin mi řekl: „Dobře to pověděli.
#18:18 Povolám jim proroka z jejich bratří, jako jsi ty. Do jeho úst vložím svá slova a on jim bude mluvit vše, co mu přikáži.
#18:19 Kdo by má slova, která on bude mluvit mým jménem, neposlouchal, toho já sám budu volat k odpovědnosti.
#18:20 Avšak prorok, který by opovážlivě mluvil mým jménem něco, co jsem mu mluvit nepřikázal, nebo který by mluvil jménem jiných bohů, takový prorok zemře.“
#18:21 V srdci si asi říkáš: „Jak poznáme slovo, které Hospodin nepromluvil?“
#18:22 Nuže, promluví-li prorok jménem Hospodinovým a věc se nestane a nesplní, nepromluvil to slovo Hospodin. Opovážlivě je mluvil ten prorok sám; nelekej se toho. 
#19:1 Až vyhladí Hospodin, tvůj Bůh, pronárody, jejichž zemi ti Hospodin, tvůj Bůh, dává, a ty si je podrobíš a usadíš se v jejich městech a domech,
#19:2 oddělíš si tři města uprostřed své země, kterou ti Hospodin, tvůj Bůh, dává, abys ji obsadil.
#19:3 Upravíš si k nim cestu a rozdělíš na tři díly území své země, kterou ti přiděluje do dědictví Hospodin, tvůj Bůh, aby tam mohl utéci každý, kdo zabil.
#19:4 A tak tomu bude s tím, kdo zabil a uteče se tam, aby zůstal naživu: Kdo zabil svého bližního neúmyslně, aniž jej kdy předtím nenáviděl,
#19:5 třeba ten, kdo by šel se svým bližním do lesa rubat dříví a rozmáchl se sekerou při kácení stromu a železo by se vysmeklo z topůrka a zasáhlo jeho bližního tak, že by zemřel, ten ať se uteče do jednoho z těch měst, aby zůstal naživu.
#19:6 Jinak by ho jako vraha pronásledoval v rozhořčení svého srdce krevní mstitel, a kdyby cesta byla příliš daleká, dostihl by ho a zabil by ho. On však není hoden smrti, protože zabitého neměl nikdy předtím v nenávisti.
#19:7 Proto ti přikazuji: „Oddělíš si tři města.“
#19:8 Jestliže Hospodin, tvůj Bůh, rozšíří tvé území, jak přísahal tvým otcům, a dá ti celou zemi, o níž řekl, že ji dá tvým otcům,
#19:9 dbej cele na tento příkaz, který ti dnes udílím, totiž abys miloval Hospodina, svého Boha, a chodil po jeho cestách po všechny dny, a přidej k těmto třem městům ještě další tři.
#19:10 Ve tvé zemi, kterou ti dává Hospodin, tvůj Bůh, do dědictví, nebude prolévána nevinná krev, abys nebyl vinen prolitou krví.
#19:11 Kdyby však někdo svého bližního nenáviděl a strojil mu úklady, povstal proti němu a ubil ho k smrti a pak utekl do některého z těch měst,
#19:12 starší z jeho města pro něho pošlou, vezmou ho odtud a předají ho do rukou krevního mstitele, aby zemřel.
#19:13 Nebudeš ho litovat. Tak odstraníš nevinně prolitou krev z Izraele a bude s tebou dobře.
#19:14 Neposuneš mezník svého bližního, kterým předkové vymezili tvůj dědičný díl, jejž zdědíš v zemi, kterou ti dává Hospodin, tvůj Bůh, abys ji obsadil.
#19:15 Nepovstane jen jediný svědek proti někomu v jakémkoli zločinu, v jakémkoli prohřešku a při jakémkoli hříchu, jehož se někdo dopustil. Soudní výrok bude vynesen podle výpovědi dvou nebo tří svědků.
#19:16 Povstane-li proti někomu zlovolný svědek, aby ho nařkl z odpadnutí od Hospodina,
#19:17 postaví se oba muži, kteří mají spor, před Hospodina, před kněze a soudce, kteří tam v těch dnech budou,
#19:18 a soudcové případ dobře vyšetří. Zjistí-li, že je to křivý svědek, že nařkl svého bratra křivě,
#19:19 učiníte jemu, jak on zamýšlel učinit svému bratru. Tak odstraníš zlo ze svého středu.
#19:20 Ať to ostatní uslyší a ať se bojí. Nikdy ať se mezi vámi nestane něco tak zlého.
#19:21 Nebudeš ho litovat. Život za život, oko za oko, zub za zub, ruka za ruku, noha za nohu. 
#20:1 Když vytáhneš do boje proti svým nepřátelům a spatříš koně a vozbu, lid početnější, než jsi sám, nebudeš se jich bát, neboť Hospodin, tvůj Bůh, který tě přivedl z egyptské země, je s tebou.
#20:2 Než podstoupíte boj, přistoupí kněz a promluví k lidu.
#20:3 Řekne jim: „Slyš, Izraeli! Dnes podstupujete boj proti svým nepřátelům. Neklesejte na mysli, nebojte se, nebuďte ustrašení, nemějte z nich hrůzu.
#20:4 Vždyť Hospodin, váš Bůh, jde s vámi, aby za vás bojoval s vašimi nepřáteli a zachránil vás.“
#20:5 Pak promluví k lidu správcové: „Ten, kdo vystavěl nový dům a ještě jej nezasvětil, ať se vrátí domů, aby nezemřel v boji a nezasvětil jej někdo jiný.
#20:6 Ten, kdo vysadil vinici a ještě z ní nesklízel, ať se vrátí domů, aby nezemřel v boji a nesklízel z ní někdo jiný.
#20:7 Ten, kdo se zasnoubil se ženou a ještě si ji nevzal, ať se vrátí domů, aby nezemřel v boji a nevzal si ji někdo jiný.“
#20:8 Dále ještě správcové promluví k lidu: „Ten, kdo je bojácný a zbabělý, ať se vrátí domů, aby jeho bratři neztratili odvahu jako on.“
#20:9 Až správcové dokončí promluvu k lidu, postaví se v čelo lidu velitelé oddílů.
#20:10 Když přitáhneš k městu, abys proti němu bojoval, nabídneš mu mír.
#20:11 Jestliže ti odpoví mírem a otevře ti brány, tu všechen lid, který je v něm, podrobíš nuceným pracím a budou ti sloužit.
#20:12 Jestliže k míru s tebou nesvolí, ale povede s tebou boj, oblehneš je.
#20:13 Až ti je Hospodin, tvůj Bůh, vydá do rukou, pobiješ v něm ostřím meče všechny osoby mužského pohlaví.
#20:14 Ale ženy, děti a dobytek i vše, co bude v městě, všechnu kořist z něho si ponecháš jako lup. Budeš užívat kořisti po svých nepřátelích, kterou ti dal Hospodin, tvůj Bůh.
#20:15 Tak naložíš se všemi městy od tebe velice vzdálenými, která nepatří k městům těchto pronárodů zde.
#20:16 Ale v městech těchto národů, které ti dává Hospodin, tvůj Bůh, do dědictví, nenecháš naživu naprosto nikoho.
#20:17 Zničíš je jako klaté, Chetejce, Emorejce, Kenaance, Perizejce, Chivejce a Jebusejce, jak ti přikázal Hospodin, tvůj Bůh,
#20:18 aby vás neučili jednat podle všelijakých svých ohavností, které činili kvůli svým bohům. Prohřešili byste se proti Hospodinu, svému Bohu.
#20:19 Když budeš dlouhou dobu obléhat nějaké město a v boji se ho zmocníš, nezničíš jeho stromoví a nebudeš je vytínat sekerou. Vždyť z něho můžeš jíst. Nekácej je! Což je strom na poli člověk, abys jej také obléhal?
#20:20 Můžeš zničit a skácet jenom strom, o němž víš, že to není strom s ovocem k jídlu; použiješ ho při stavbě obléhacího náspu proti městu, které s tebou vede boj, dokud město nepadne. 
#21:1 Bude-li v zemi, kterou ti dává Hospodin, tvůj Bůh, abys ji obsadil, nalezen skolený, ležící na poli, a nebude známo, kdo jej ubil,
#21:2 vyjdou tvoji starší a soudcové a vyměří vzdálenost k městům, která budou v okolí skoleného.
#21:3 Když se zjistí, které město je skolenému nejblíže, vezmou starší toho města jalovici, jíž nebylo použito k práci a která netahala pod jhem.
#21:4 Starší toho města přivedou jalovici dolů k potoku s tekoucí vodou, kde se nepracovalo a neselo, a tam u potoka zlomí jalovici vaz.
#21:5 Pak přistoupí kněží Léviovci, protože je vyvolil Hospodin, tvůj Bůh, aby mu sloužili a jménem Hospodinovým dávali požehnání. Podle jejich výroku se stane při každém sporu a při každém ublížení na těle.
#21:6 Všichni starší toho města, které je nejblíže skolenému, si omyjí ruce nad jalovicí, jíž byl u potoka zlomen vaz,
#21:7 a dosvědčí: „Naše ruce tuto krev neprolily a naše oči nic neviděly.
#21:8 Hospodine, zprosť viny svůj izraelský lid, který jsi vykoupil, a nestíhej nevinně prolitou krev uprostřed svého izraelského lidu.“ Tak budou zproštěni viny za prolitou krev.
#21:9 Když vykonáš, co je v Hospodinových očích správné, odstraníš nevinně prolitou krev ze svého středu.
#21:10 Když vytáhneš do boje proti svým nepřátelům a Hospodin, tvůj Bůh, ti je vydá do rukou a ty zajmeš zajatce
#21:11 a spatříš mezi zajatci ženu krásné postavy, přilneš k ní a budeš si ji chtít vzít za ženu,
#21:12 přivedeš ji do svého domu. Ať si oholí hlavu a ostříhá nehty
#21:13 a odloží svůj plášť, v němž byla zajata, a zůstane v tvém domě. Po dobu jednoho měsíce bude oplakávat svého otce a svou matku. Potom k ní smíš vejít, budeš jejím manželem a ona bude tvou ženou.
#21:14 Jestliže se ti pak znelíbí, propustíš ji a bude volná. Nesmíš ji prodat za stříbro ani s ní hrubě zacházet, poté co jsi ji ponížil.
#21:15 Má-li někdo dvě ženy, z nichž by jednu miloval a druhou by nemiloval, a porodí mu syny milovaná i nemilovaná, ale prvorozený syn bude synem nemilované,
#21:16 v den, kdy bude dávat, co mu patří, svým synům do dědictví, nemůže dát právo prvorozenství synu milované na úkor prvorozeného syna nemilované.
#21:17 Vezme ohled na prvorozeného syna nemilované a dá mu dvojnásobný díl všeho, co má, protože on je prvotina jeho síly, jemu náleží právo prvorozenství.
#21:18 Má-li někdo syna nepoddajného a vzpurného, který neposlouchá otce ani matku, a když ho kárají, neposlechne je,
#21:19 ať ho jeho otec i matka uchopí a vyvedou ke starším jeho města, k bráně jeho místa,
#21:20 a řeknou starším jeho města: „Tento náš syn je nepoddajný a vzpurný, neposlouchá nás, je to modlářský žrout a pijan.“
#21:21 A tak ho všichni muži jeho města uházejí kamením a zemře. Tak odstraníš zlo ze svého středu. Ať to slyší celý Izrael a bojí se.
#21:22 Bude-li nad někým pro hřích vynesen rozsudek smrti, bude-li usmrcen a ty jej pověsíš na kůl,
#21:23 nenecháš jeho mrtvolu na kůlu přes noc, ale bezpodmínečně ho pohřbíš týž den, protože ten, kdo byl pověšen, je zlořečený Bohem. Neposkvrníš svou zemi, kterou ti dává do dědictví Hospodin, tvůj Bůh. 
#22:1 Nebudeš netečně přihlížet, jak se zaběhl býk tvého bratra nebo jeho ovce. Vrátíš je svému bratru.
#22:2 Jestliže tvůj bratr není blízko tebe a ty ho neznáš, zaženeš dobytče do svého domu a bude u tebe. Až tvůj bratr bude po něm pátrat, vrátíš mu je.
#22:3 Stejně naložíš s jeho oslem, stejně naložíš s jeho pláštěm, stejně naložíš s každou ztracenou věcí svého bratra, která se mu ztratila a tys ji nalezl. Nesmíš být netečný.
#22:4 Nebudeš netečně přihlížet, jak na cestě klesl osel tvého bratra nebo jeho býk. Pozdvihneš jej spolu s ním.
#22:5 Žena na sebe nevezme, co patří muži, a muž neobleče, co nosí žena. Hospodin, tvůj Bůh, má v ohavnosti každého, kdo to činí.
#22:6 Když přijdeš cestou na ptačí hnízdo s holátky nebo s vejci, s matkou sedící na holátkách nebo na vejcích na nějakém stromě nebo na zemi, nevezmeš matku od mláďat.
#22:7 Matku pustíš, mláďátka si smíš vzít. Tak ti bude dobře a budeš dlouho živ.
#22:8 Když vystavíš nový dům, uděláš na střeše zábradlí. Neuvalíš na svůj dům vinu za prolitou krev, kdyby z něho někdo spadl.
#22:9 Neosázíš svou vinici dvojím druhem, jinak bude nedotknutelné všechno, jak sadba, kterou jsi zasadil, tak úroda z vinice.
#22:10 Nebudeš orat s volem a oslem zapřaženými spolu.
#22:11 Nebudeš si oblékat tkaninu ze směsi vlny a lnu.
#22:12 Uděláš si střapce na všech čtyřech rozích své pokrývky, kterou se přikrýváš.
#22:13 Když si muž vezme ženu a vejde k ní, ale pak ji bude nenávidět,
#22:14 obviní ji ze špatnosti a bude o ní roznášet zlou pověst tím, že bude říkat: „Vzal jsem si tuto ženu, přiblížil jsem se k ní, ale zjistil jsem, že není panna“,
#22:15 tedy otec té dívky a její matka přinesou důkaz dívčina panenství do brány ke starším města.
#22:16 Otec té dívky řekne starším: „Tomu muži jsem dal za ženu svou dceru, ale on ji teď nenávidí.
#22:17 Hle, obviňuje ji ze špatnosti a říká: Zjistil jsem, že tvoje dcera nebyla panna. Tady je důkaz panenství mé dcery.“ A rozprostřou roušku před staršími města.
#22:18 I vezmou starší toho města onoho muže a ztrestají ho.
#22:19 Dají mu pokutu sto šekelů stříbra a předají je otci té dívky, protože roznášel zlou pověst o izraelské panně. Ta zůstane jeho ženou. Po celý svůj život ji nesmí propustit.
#22:20 Jestliže však byla ta řeč pravdivá a u té dívky nebylo shledáno, že je panna,
#22:21 tedy vyvedou dívku ke vchodu do domu jejího otce, mužové jejího města ji ukamenují a zemře, neboť tím, že smilnila v domě svého otce, dopustila se v Izraeli hanebnosti. Tak odstraníš zlo ze svého středu.
#22:22 Když bude přistižen muž, že ležel s vdanou ženou, oba zemřou: muž, který ležel se ženou, i ta žena. Tak odstraníš zlo z Izraele.
#22:23 Když dívku, pannu zasnoubenou muži, najde nějaký muž v městě a bude s ní ležet,
#22:24 vyvedete oba dva k bráně toho města, ukamenujete je a zemřou: dívka proto, že v městě nekřičela, a muž proto, že ponížil ženu svého bližního. Tak odstraníš zlo ze svého středu.
#22:25 Jestliže nalezne muž zasnoubenou dívku na poli a zmocní se jí a bude s ní ležet, zemře jenom ten muž, který s ní ležel.
#22:26 Dívce neuděláš nic. Dívka se nedopustila hříchu hodného smrti. Je to podobný případ, jako když někdo povstane proti svému bližnímu a zabije ho.
#22:27 Vždyť ji našel na poli, zasnoubená dívka křičela, ale nebyl tu nikdo, kdo by ji zachránil.
#22:28 Když najde muž dívku, pannu, která není zasnoubena, a chytí ji, bude s ní ležet a budou přistiženi,
#22:29 muž, který s ní ležel, dá otci té dívky padesát šekelů stříbra. Stane se jeho ženou, protože ji ponížil. Po celý svůj život ji nesmí propustit. 
#23:1 Nikdo si nesmí vzít ženu svého otce a odkrýt tak cíp pláště svého otce.
#23:2 Do Hospodinova shromáždění nevstoupí, kdo má rozdrcená varlata nebo uříznutý pyj.
#23:3 Do Hospodinova shromáždění nevstoupí míšenec; ani jeho desáté pokolení nevstoupí do Hospodinova shromáždění.
#23:4 Do Hospodinova shromáždění nikdy nevstoupí Amónec nebo Moábec; ani jejich desáté pokolení nevstoupí do Hospodinova shromáždění,
#23:5 za to, že vám nevyšli vstříc s chlebem a vodou na cestě, když jste táhli z Egypta, a že Moáb najal proti tobě Bileáma, syna Beórova, z Petóru v Aramském Dvojříčí, aby ti zlořečil.
#23:6 Ale Hospodin, tvůj Bůh, nechtěl Bileáma slyšet. Proto zvrátil Hospodin, tvůj Bůh, zlořečení tobě v požehnání, neboť Hospodin, tvůj Bůh, tě miloval.
#23:7 Po všechny své dny nikdy neusiluj o pokoj s nimi ani o dobrodiní od nich.
#23:8 Nebudeš si hnusit Edómce, neboť to je tvůj bratr. Nebudeš si hnusit Egypťana, neboť jsi byl hostem v jeho zemi.
#23:9 Synové, kteří se jim narodí, smějí v třetím pokolení vstoupit do Hospodinova shromáždění.
#23:10 Když vytáhneš vojensky proti svým nepřátelům, vystříhej se jakékoli špatnosti.
#23:11 Bude-li mezi vámi někdo, kdo by nebyl čistý pro noční výron semene, vyjde ven za tábor. Nevstoupí dovnitř do tábora.
#23:12 K večeru se omyje vodou a při západu slunce smí vstoupit dovnitř do tábora.
#23:13 Vně za táborem budeš mít vyhrazené místo, kam budeš chodit na stranu.
#23:14 Mezi svým nářadím budeš mít kolík. Než si venku dřepneš, vyhrabeš jím důlek a své výkaly zase přikryješ.
#23:15 Vždyť Hospodin, tvůj Bůh, chodí po tvém táboře, aby tě vysvobodil a aby ti vydal tvé nepřátele. Ať je tedy tvůj tábor svatý, aby u tebe nespatřil nic mrzkého a neodvrátil se od tebe.
#23:16 Nevydáš otroka jeho pánu, když se k tobě před svým pánem uchýlil.
#23:17 Bude bydlet s tebou uprostřed tebe na místě, které si zvolí, v některé z tvých bran, kde mu bude dobře. Neutiskuj ho.
#23:18 Žádná z izraelských dcer se nezasvětí smilstvu; ani žádný z izraelských synů se nezasvětí smilstvu.
#23:19 Nepřineseš nevěstčí mzdu ani psovskou odměnu do domu Hospodina, svého Boha, ať jde o jakýkoli slib, neboť obojí je Hospodinu, tvému Bohu, ohavností.
#23:20 Svému bratru nebudeš půjčovat na úrok, na žádný úrok ani za stříbro ani za pokrm ani za cokoli, co se půjčuje na úrok.
#23:21 Cizinci můžeš půjčovat na úrok, ale svému bratru na úrok půjčovat nesmíš, aby ti Hospodin, tvůj Bůh, požehnal ve všem, k čemu přiložíš svou ruku na zemi, kterou jdeš obsadit.
#23:22 Když se Hospodinu, svému Bohu, zavážeš slibem, neotálej ho splnit, neboť Hospodin, tvůj Bůh, to bude určitě od tebe vyžadovat a byl by na tobě hřích.
#23:23 Když se zdržíš slibování, nebude na tobě hřích.
#23:24 Budeš bedlivě dodržovat to, co vyřkly tvoje rty a co jsi dobrovolně slíbil Hospodinu, svému Bohu, k čemu ses svými ústy zavázal.
#23:25 Když vejdeš do vinice svého bližního, smíš se najíst hroznů dosyta podle libosti, ale nebudeš nic dávat do nádoby.
#23:26 Když vejdeš do obilí svého bližního, smíš si rukou natrhat klasů, ale nebudeš obilí svého bližního žnout srpem. 
#24:1 Když si muž vezme ženu a ožení se s ní, ona však u něho nenalezne přízeň, neboť na ní shledal něco odporného, napíše jí rozlukový list, dá jí ho do rukou a vykáže ji ze svého domu.
#24:2 Ona vyjde z jeho domu, odejde a vdá se za jiného muže.
#24:3 Ale ten druhý muž k ní také pojme nenávist, napíše jí rozlukový list, dá jí ho do rukou a vykáže ji ze svého domu. Anebo ten druhý muž, který si ji vzal za ženu, zemře.
#24:4 Tu její první manžel, který ji vykázal, si ji znovu za ženu vzít nemůže, když byla poskvrněna, neboť by to před Hospodinem byla ohavnost. Neuvalíš hřích na zemi, kterou ti Hospodin, tvůj Bůh, dává do dědictví.
#24:5 Když se muž právě oženil, nebude vycházet do boje a nebude mu ukládán žádný úkol. Po jeden rok bude uvolněn pro svůj dům, aby se radoval se svou ženou, kterou si vzal.
#24:6 Nikdo nesmí zabavit mlýnské kameny, spodní ani běhoun, neboť takový člověk jako by zabavil sám život.
#24:7 Když bude někdo přistižen, že ukradl někoho ze svých bratří Izraelců, hrubě s ním nakládal a prodal ho, tedy ten zloděj zemře; tak odstraníš zlo ze svého středu.
#24:8 Měj se na pozoru při ráně malomocenství, abys velmi bedlivě dodržoval všechno, o čem vás poučí lévijští kněží. Bedlivě dodržujte, co jsem jim přikázal.
#24:9 Pamatuj, co Hospodin, tvůj Bůh, učinil Mirjamě na cestě, když jste táhli z Egypta.
#24:10 Když poskytneš svému bližnímu nějakou půjčku, nevejdeš do jeho domu, abys na něm vymáhal zástavu.
#24:11 Zůstaneš venku a muž, kterému jsi půjčku poskytl, vynese ti zástavu ven.
#24:12 Jestliže je ten muž zchudlý, s jeho zástavou neulehneš.
#24:13 Vrátíš mu zástavu při západu slunce, aby ulehl ve svém plášti a žehnal ti. To bude tvá spravedlnost před Hospodinem, tvým Bohem.
#24:14 Nebudeš utiskovat nádeníka, zchudlého a potřebného ze svých bratří ani ze svých hostí, kteří žijí v tvé zemi v tvých branách.
#24:15 Dáš mu jeho mzdu ještě téhož dne, než nad ním zapadne slunce, neboť je zchudlý a svým životem na ní závisí, aby nevolal proti tobě k Hospodinu a aby na tobě nebyl hřích.
#24:16 Nebudou usmrcováni otcové za syny a synové nebudou usmrcováni za otce, každý bude usmrcen pro vlastní hřích.
#24:17 Nepřevrátíš právo bezdomovce ani sirotka a vdově nezabavíš roucho.
#24:18 Ale pamatuj, že jsi byl otrokem v Egyptě a že Hospodin, tvůj Bůh, tě odtud vykoupil. Proto ti přikazuji, abys to dodržoval.
#24:19 Když budeš sklízet ze svého pole a zapomeneš na poli snop, nevrátíš se pro něj. Bude patřit bezdomovci, sirotku a vdově, aby ti Hospodin, tvůj Bůh, požehnal při každé práci tvých rukou.
#24:20 Když oklátíš plody ze své olivy, nebudeš ještě setřásat zbylé. Ty budou patřit bezdomovci, sirotku a vdově.
#24:21 Když budeš na vinici sbírat hrozny, nebudeš po sobě paběrkovat. Bude to patřit bezdomovci, sirotku a vdově.
#24:22 Pamatuj, že jsi byl otrokem v egyptské zemi. Proto ti přikazuji, abys to dodržoval. 
#25:1 Když mezi muži dojde ke sporu a oni se dostaví k soudu, aby je rozsoudili, ať je spravedlivý ospravedlněn a svévolník odsouzen.
#25:2 Je-li svévolník hoden mrskání, dá ho soudce položit a mrskat za své přítomnosti; dá mu vysázet počet ran podle jeho svévole.
#25:3 Smí jej dát zmrskat nejvýše čtyřiceti ranami, aby tvůj bratr nebyl před tebou zlehčen, kdyby mu při mrskání přidal více ran.
#25:4 Při mlácení nedáš dobytčeti náhubek.
#25:5 Když budou bydlet bratři spolu a jeden z nich zemře bez syna, nevdá se žena zemřelého jinam, za cizího muže. Vejde k ní její švagr a vezme si ji za ženu právem švagrovství.
#25:6 Prvorozený syn, kterého porodí, ponese jméno jeho zemřelého bratra, aby jeho jméno nebylo z Izraele vymazáno.
#25:7 Avšak jestliže si ten muž nebude chtít vzít svou švagrovou, vystoupí jeho švagrová do brány ke starším a řekne: „Můj švagr se zdráhá zachovat svému bratrovi jméno v Izraeli, není svolný užít vůči mně práva švagrovství.“
#25:8 Předvolají ho tedy starší jeho města a promluví s ním. Když bude stát na svém a řekne: „Nechci si ji vzít“,
#25:9 přistoupí k němu před staršími jeho švagrová, zuje mu střevíc, plivne mu do tváře a prohlásí: „To si zaslouží muž, který nechce svému bratru zbudovat dům.“
#25:10 I bude mít v Izraeli přezdívku: „Dům vyzutého“.
#25:11 Když se muži budou spolu rvát jeden s druhým a přistoupí žena jednoho, aby vyrvala svého muže z rukou toho, který ho bije, a vztáhne ruku a uchopí ho za ohanbí,
#25:12 bez slitování jí utneš ruku.
#25:13 Nebudeš mít ve svém váčku dvojí závaží, větší a menší.
#25:14 Nebudeš mít ve svém domě dvojí míru, větší a menší.
#25:15 Budeš mít správné a poctivé závaží, správnou a poctivou míru, abys byl na zemi, kterou ti dává Hospodin, tvůj Bůh.
#25:16 Vždyť Hospodin, tvůj Bůh, má v ohavnosti každého, kdo páchá něco takového, každého, kdo se dopouští bezpráví.
#25:17 Pamatuj, co ti učinil Amálek, když jste táhli z Egypta.
#25:18 Střetl se s tebou na cestě a zničil ti zadní voj, všechny churavé za tebou, když tys byl ochablý a unavený; a nebál se Boha.
#25:19 Proto až ti Hospodin, tvůj Bůh, dá odpočinutí ode všech tvých okolních nepřátel v zemi, kterou ti Hospodin, tvůj Bůh, dává do dědictví, abys ji obsadil, vymažeš památku Amálekovu zpod nebes. Nezapomeň! 
#26:1 Až přijdeš do země, kterou ti Hospodin, tvůj Bůh, dává do dědictví, a obsadíš ji a usadíš se v ní,
#26:2 vezmeš z prvotin všech polních plodů, které vytěžíš ze své země, kterou ti dává Hospodin, tvůj Bůh, vložíš je do koše a půjdeš k místu, které vyvolí Hospodin, tvůj Bůh, aby tam přebývalo jeho jméno.
#26:3 Přijdeš ke knězi, který tam bude v těch dnech, a řekneš mu: „Vyznávám dnes Hospodinu, tvému Bohu, že jsem vstoupil do země, o které přísahal Hospodin našim otcům, že nám ji dá.“
#26:4 Kněz vezme koš z tvé ruky a postaví jej před oltář Hospodina, tvého Boha.
#26:5 Ty pak promluvíš před Hospodinem, svým Bohem, takto: „Můj otec byl Aramejec bloudící bez domova, sestoupil do Egypta a s hrstkou lidí tam pobýval jako host. Tam se stal velikým, zdatným a početným národem.
#26:6 Ale Egypťané s námi zle nakládali, trýznili nás a tvrdě nás zotročovali.
#26:7 Tu jsme úpěli k Hospodinu, Bohu našich otců, a Hospodin nás vyslyšel, shlédl na naše pokoření, plahočení a útlak.
#26:8 Hospodin nás vyvedl z Egypta pevnou rukou a vztaženou paží, za veliké hrůzy, se znameními a zázraky,
#26:9 a přivedl nás na toto místo a dal nám tuto zemi, zemi oplývající mlékem a medem.
#26:10 Nyní tedy, hle, přinesl jsem prvotiny z plodů role, kterou jsi mi dal, Hospodine.“ A postavíš koš před Hospodina, svého Boha, a pokloníš se před Hospodinem, svým Bohem.
#26:11 Budeš se radovat ze všeho dobrého, co dal Hospodin, tvůj Bůh, tobě a tvému domu, ty i lévijec i bezdomovec, který bude mezi vámi.
#26:12 Když pak třetího roku, v roce desátků, odvedeš všechny desátky ze své úrody a dáš je lévijci, bezdomovci, sirotku a vdově, aby jedli v tvých branách a nasytili se,
#26:13 řekneš před Hospodinem, svým Bohem: „Vynesl jsem z domu, co bylo svaté, a dal jsem to lévijci a bezdomovci, sirotku a vdově, zcela podle tvého příkazu, který jsi mi dal. Nepřestoupil jsem ani neopomenul žádné z tvých přikázání.
#26:14 Nejedl jsem z toho, když jsem měl zármutek, neodstranil jsem z toho nic, když jsem byl nečistý, nedal jsem z toho nic pro mrtvého. Poslechl jsem Hospodina, svého Boha, a učinil jsem vše, jak jsi mi přikázal.
#26:15 Shlédni ze svého svatého obydlí, z nebes, a požehnej svému izraelskému lidu i půdě, kterou jsi nám dal, jak jsi přísežně slíbil našim otcům, tu zemi oplývající mlékem a medem.“
#26:16 Dnešního dne ti Hospodin, tvůj Bůh, přikazuje, abys dodržoval tato nařízení a právní ustanovení. Budeš je bedlivě dodržovat celým svým srdcem a celou svou duší.
#26:17 Prohlásil jsi při Hospodinu, že ti bude Bohem a ty že budeš chodit po jeho cestách a dbát na jeho nařízení, přikázání a právní ustanovení a že ho budeš poslouchat.
#26:18 A Hospodin prohlásil dnes tobě, že budeš jeho lidem, zvláštním vlastnictvím, jak k tobě mluvil. Budeš dbát na všechna jeho přikázání
#26:19 a on tě vyvýší nade všechny pronárody, které učinil. Budeš mu chválou, věhlasem a okrasou, budeš svatým lidem Hospodina, svého Boha, jak mluvil. 
#27:1 I přikázal Mojžíš a izraelští starší lidu: Dbejte na každý příkaz, který vám dnes udílím.
#27:2 V den, kdy přejdete Jordán do země, kterou ti dává Hospodin, tvůj Bůh, postavíš si veliké kameny a obílíš je vápnem.
#27:3 Napíšeš na ně všechna slova tohoto zákona, až přejdeš Jordán a vejdeš do země, kterou ti dává Hospodin, tvůj Bůh, do země oplývající mlékem a medem, jak k tobě mluvil Hospodin, Bůh tvých otců.
#27:4 Až tedy přejdete Jordán, postavíte na hoře Ébalu ony kameny, jak vám dnes přikazuji, a obílíš je vápnem.
#27:5 Vybuduješ tam oltář Hospodinu, svému Bohu, oltář z kamenů, ale nebudeš je opracovávat železem.
#27:6 Z neopracovaných kamenů vybuduješ oltář Hospodinu, svému Bohu, a budeš na něm obětovat zápalné oběti Hospodinu, svému Bohu.
#27:7 Budeš tam také obětovat a jíst pokojné oběti a radovat se před Hospodinem, svým Bohem.
#27:8 A na kameny napíšeš co nejzřetelněji všechna slova tohoto zákona.
#27:9 I promluvil Mojžíš a lévijští kněží k celému Izraeli: Ztiš se a slyš, Izraeli, stal ses dnešního dne lidem Hospodina, svého Boha.
#27:10 Budeš poslouchat Hospodina, svého Boha, a plnit jeho příkazy a jeho nařízení, která ti dnes přikazuji.
#27:11 Onoho dne Mojžíš lidu dále přikázal:
#27:12 Až přejdete Jordán, postaví se na hoře Gerizímu, aby žehnali lidu: Šimeón, Lévi, Juda, Isachar, Josef a Benjamín.
#27:13 A tito se postaví na hoře Ébalu, aby zlořečili: Rúben, Gád, Ašer, Zabulón, Dan a Neftalí.
#27:14 Lévijci střídavě se všemi izraelskými muži budou pronášet zvýšeným hlasem:
#27:15 „Buď proklet muž, který zhotoví tesanou nebo litou sochu, ohavnost před Hospodinem, výrobek rukou řemeslníka, a uloží ji v úkrytu.“ A všechen lid odpoví a řekne: „Amen.“
#27:16 „Buď proklet, kdo zlehčuje svého otce a svou matku.“ A všechen lid řekne: „Amen.“
#27:17 „Buď proklet, kdo posouvá mezník svého bližního.“ A všechen lid řekne: „Amen.“
#27:18 „Buď proklet, kdo svádí slepce z cesty.“ A všechen lid řekne: „Amen.“
#27:19 „Buď proklet, kdo převrací právo bezdomovce, sirotka a vdovy.“ A všechen lid řekne: „Amen.“
#27:20 „Buď proklet, kdo obcuje se ženou svého otce, neboť odkryl cíp pláště svého otce.“ A všechen lid řekne: „Amen.“
#27:21 „Buď proklet, kdo obcuje s nějakým zvířetem.“ A všechen lid řekne: „Amen.“
#27:22 „Buď proklet, kdo obcuje se svou sestrou, dcerou svého otce nebo dcerou své matky.“ A všechen lid řekne: „Amen.“
#27:23 „Buď proklet, kdo obcuje se svou tchyní.“ A všechen lid řekne: „Amen.“
#27:24 „Buď proklet, kdo tajně ubije svého bližního.“ A všechen lid řekne: „Amen.“
#27:25 „Buď proklet, kdo vezme úplatek, aby ubil člověka a prolil nevinnou krev.“ A všechen lid řekne: „Amen.“
#27:26 „Buď proklet, kdo nebude plnit slova tohoto zákona a dodržovat je.“ A všechen lid řekne: „Amen.“ 
#28:1 Jestliže budeš opravdově poslouchat Hospodina, svého Boha, a bedlivě dodržovat všechny jeho příkazy, které ti dnes udílím, vyvýší tě Hospodin, tvůj Bůh, nad všechny pronárody země.
#28:2 A spočinou na tobě všechna tato požehnání, když budeš poslouchat Hospodina, svého Boha:
#28:3 Požehnaný budeš ve městě a požehnaný budeš na poli.
#28:4 Požehnaný bude plod tvého života i plodiny tvé role, plod tvého dobytka, vrh tvého skotu a přírůstek tvého bravu.
#28:5 Požehnaný bude tvůj koš a tvá díže.
#28:6 Požehnaný budeš při svém vcházení a požehnaný při svém vycházení.
#28:7 Hospodin dá, že tvoji nepřátelé, kteří proti tobě povstanou, budou před tebou poraženi. Jednou cestou proti tobě vytáhnou a sedmi cestami budou před tebou utíkat.
#28:8 Hospodin přikáže, aby s tebou bylo požehnání ve tvých sýpkách a na všem, k čemu přiložíš ruku. Hospodin, tvůj Bůh, ti bude žehnat v zemi, kterou ti dává.
#28:9 Hospodin si tě ustaví za svatý lid, jak ti přisáhl, když budeš na příkazy Hospodina, svého Boha, dbát a chodit po jeho cestách.
#28:10 Všechny národy země budou vidět, že se nazýváš Hospodinovým jménem, a budou se tě bát.
#28:11 Hospodin ti dá nadbytek dobrého: plodu tvého života a plodu tvého dobytka i plodin tvé role v zemi, o které přísahal Hospodin tvým otcům, že ti ji dá.
#28:12 Hospodin ti otevře svou štědrou pokladnici, nebesa, aby v pravý čas dal tvé zemi déšť a požehnal každé práci tvých rukou. Budeš půjčovat mnohým pronárodům, ale sám si nebudeš muset půjčovat.
#28:13 Hospodin tě učiní hlavou a ne chvostem, budeš vždycky stoupat výš a neklesneš níž, budeš-li poslouchat příkazy Hospodina, svého Boha, které ti dnes udílím, abys je bedlivě dodržoval.
#28:14 Neuchýlíš se napravo ani nalevo od žádného ze slov, která vám dnes přikazuji, tím, že bys chodil za jinými bohy a sloužil jim.
#28:15 Jestliže však nebudeš Hospodina, svého Boha, poslouchat a nebudeš bedlivě dodržovat všechny jeho příkazy a nařízení, která ti dnes udílím, dopadnou na tebe všechna tato zlořečení:
#28:16 Prokletý budeš ve městě a prokletý budeš na poli.
#28:17 Prokletý bude tvůj koš i tvá díže.
#28:18 Prokletý bude plod tvého života i plodiny tvé role, vrh tvého skotu i přírůstek tvého bravu.
#28:19 Prokletý budeš při svém vcházení a prokletý při svém vycházení.
#28:20 Hospodin na tebe pošle prokletí, zděšení a zmar ve všem, k čemu přiložíš ruku a co budeš dělat, dokud nebudeš zahlazen. Zanikneš rychle pro své zlé jednání, protože jsi mě opustil.
#28:21 Hospodin způsobí, že se tě bude držet mor, dokud s tebou neskoncuje v zemi, kterou přicházíš obsadit.
#28:22 Hospodin tě bude bít úbytěmi a zimnicí, palčivou horečkou, mečem, obilnou rzí a snětí; ty tě budou pronásledovat, dokud nezanikneš.
#28:23 Nebe nad tvou hlavou bude měděné a země pod tebou bude železná.
#28:24 Místo deště dá Hospodin tvé zemi písek a prach; budou na tebe padat z nebe, dokud nebudeš zahlazen.
#28:25 Hospodin dopustí, že budeš před svými nepřáteli poražen. Jednou cestou proti nepříteli vytáhneš a sedmi cestami budeš před ním utíkat. Budeš obrazem hrůzy ve všech královstvích země.
#28:26 Tvá mrtvola bude pokrmem všemu nebeskému ptactvu i zemskému zvířectvu a nikdo je nezaplaší.
#28:27 Hospodin tě raní egyptskými vředy a boulemi, svrabem a prašivinou, které nebudeš moci vyléčit.
#28:28 Hospodin tě raní šílenstvím, slepotou a pomatením mysli.
#28:29 O poledni budeš tápat jako tápe slepý ve tmě. Na svých cestách nebudeš mít zdar, stále budeš týrán a odírán, po všechny dny, a nikdo tě nezachrání.
#28:30 Zasnoubíš se s ženou, a jiný muž ji zneuctí, vystavíš si dům, a nebudeš v něm bydlet, vysázíš vinici, a nebudeš z ní sklízet.
#28:31 Tvůj býk bude poražen před tvýma očima, avšak jíst z něho nebudeš. Tvůj osel ti bude uchvácen před očima, a nevrátí se ti. Tvůj brav bude vydán tvému nepříteli, a nikdo tě nezachrání.
#28:32 Tvoji synové a tvé dcery budou vydány jinému lidu; tvé oči se na to budou dívat, po celé dny se budou pro ně rozplývat v slzách, ale nic nezmůžeš.
#28:33 Plodiny tvé role i veškerý tvůj výtěžek pozře lid, který jsi neznal, napořád budeš týrán a utiskován, po všechny dny.
#28:34 Zešílíš z podívané, na kterou se budeš muset dívat.
#28:35 Hospodin tě raní na kolenou i na stehnech, od chodidla až po temeno zlými vředy, které nebudeš moci vyléčit.
#28:36 Hospodin od tebe odvede i tvého krále, jehož nad sebou ustanovíš, k pronárodu, který jsi neznal ty ani tvoji otcové. Tam budeš sloužit jiným bohům, dřevu a kameni.
#28:37 Budeš předmětem úděsu, pořekadel a posměchu mezi všemi národy, kam tě Hospodin odvleče.
#28:38 Vyseješ na poli mnoho osiva, ale sklidíš málo, neboť to ožerou kobylky.
#28:39 Vysázíš a obděláš vinice, avšak víno pít ani uskladňovat nebudeš, neboť to sežerou červi.
#28:40 Všude po svém území budeš mít olivy, ale olejem se nepomažeš, neboť olivy ti opadají.
#28:41 Zplodíš syny a dcery, ale tobě nezůstanou, neboť půjdou do zajetí.
#28:42 Všechno tvé stromoví a plodiny tvé role napadne hmyz.
#28:43 Bezdomovec, který je ve tvém středu, bude stoupat stále výš nad tebe, ty pak budeš klesat čím dál tím níž.
#28:44 On ti bude půjčovat, ale ty mu nebudeš moci půjčit. On bude hlavou, a ty budeš chvostem.
#28:45 Dopadnou na tebe všechna tato zlořečení a budou tě pronásledovat, dokud nebudeš zahlazen, neboť jsi neposlouchal Hospodina, svého Boha, a nedbal jsi na příkazy a nařízení, která ti udělil.
#28:46 Budou na tobě a na tvém potomstvu navěky znamením a zázrakem.
#28:47 Zato, že jsi měl hojnost všeho, ale nesloužil Hospodinu, svému Bohu, s radostí a vděčným srdcem,
#28:48 budeš otročit svým nepřátelům, které na tebe pošle Hospodin, v hladu a žízni, v nahotě a v nedostatku všeho. On ti vloží na šíji železné jho, dokud tě nevyhladí.
#28:49 Hospodin na tebe přivede pronárod zdaleka, od konce země, jako přilétá orel, pronárod, jehož jazyku nebudeš rozumět,
#28:50 pronárod kruté tváře, který nebere ohled na starce a nad chlapcem se nesmiluje.
#28:51 Bude požírat plod tvého dobytka i plodiny tvé role, dokud nebudeš zahlazen. Nenechá ti obilí ani mošt a olej, vrh tvého skotu ani přírůstek tvého bravu, dokud tě nevyhubí.
#28:52 Sevře tě ve všech tvých branách, dokud nepadnou tvé vysoké a pevné hradby, na něž spoléháš po celé své zemi. Sevře tě ve všech tvých branách po celé tvé zemi, kterou ti dal Hospodin, tvůj Bůh.
#28:53 A v tísni obležení, kterým tě bude tísnit tvůj nepřítel, budeš jíst plod svého života, maso svých synů a dcer, které ti dal Hospodin, tvůj Bůh.
#28:54 Změkčilý a zhýčkaný mezi vámi bude nepřejícně hledět na svého bratra i na svou vlastní ženu a na syny, kteří mu ještě zbyli,
#28:55 aby nemusel některému z nich dát z masa svých dětí, které pojídá, ze strachu, že by jemu nezbylo nic v tísni obležení, kterým tě bude tísnit tvůj nepřítel ve všech tvých branách.
#28:56 Ta změkčilá a zhýčkaná mezi vámi, která se pro zhýčkanost a změkčilost ani nepokusila postavit nohou na zem, bude nepřejícně hledět na svého vlastního muže i na svého syna a na svou dceru,
#28:57 i na své plodové lůžko, které z ní vychází, i na své děti, které porodí. Tajně je sní pro nedostatek všeho v tísni obležení, kterým tě bude tísnit tvůj nepřítel ve tvých branách.
#28:58 Jestliže nebudeš bedlivě dodržovat všechna slova tohoto zákona, napsaná v této knize, a bát se toho slavného a hrozného jména Hospodina, svého Boha,
#28:59 dopustí Hospodin na tebe i na tvé potomstvo neobyčejné rány, rány veliké a vytrvalé i nemoci zlé a vytrvalé.
#28:60 Obrátí na tebe všechny egyptské choroby, jichž se lekáš, a ony na tobě ulpí.
#28:61 Hospodin na tebe uvede též každou nemoc a každou ránu, o níž není psáno v knize tohoto zákona, dokud nebudeš zahlazen.
#28:62 I zbude vás maličko, ač vás bylo mnoho jako nebeských hvězd, protože jsi neposlouchal Hospodina, svého Boha.
#28:63 A jako se Hospodin nad vámi veselil, když vám prokazoval dobro a rozmnožoval vás, tak se bude Hospodin nad vámi veselit, když vás bude hubit a zahlazovat. Budete vyrváni ze země, kterou přicházíš obsadit.
#28:64 Hospodin tě rozptýlí do všech národů od jednoho konce země do druhého. Tam budeš sloužit jiným bohům, které jsi neznal ty ani tvoji otcové, dřevu nebo kameni.
#28:65 Mezi těmi pronárody nebudeš mít klidu, ani tvá noha si neodpočine. Hospodin ti tam dá chvějící se srdce, pohaslé oči a zoufalou duši.
#28:66 Tvůj život bude viset na vlásku, v noci i ve dne se budeš chvět strachem a nebudeš jist svým životem.
#28:67 Ráno budeš říkat: „Kéž by byl večer!“ a večer budeš říkat: „Kéž by bylo ráno!“ pro strach svého srdce, kterým se budeš chvět, a pro podívanou, na kterou se budeš muset dívat.
#28:68 Loděmi tě Hospodin přivede zpátky do Egypta; vydáš se cestou, o níž jsem prohlásil, že ji už nespatříš. Tam se budete nabízet na prodej svým nepřátelům za otroky a otrokyně, ale nikdo vás nekoupí.
#28:69 Toto jsou slova smlouvy, o které Hospodin přikázal Mojžíšovi, aby ji uzavřel se syny Izraele v moábské zemi, kromě té smlouvy, kterou s nimi uzavřel na Chorébu. 
#29:1 Mojžíš svolal celý Izrael a řekl jim: Na vlastní oči jste viděli, co učinil Hospodin v egyptské zemi faraónovi a všem jeho služebníkům i celé jeho zemi,
#29:2 na vlastní oči jsi viděl veliké zkoušky, znamení a ony veliké zázraky.
#29:3 Ale Hospodin vám nedal srdce, aby chápalo, ani oči, aby viděly, ani uši, aby slyšely, až do tohoto dne.
#29:4 Když jsem vás vodil čtyřicet let pouští, pláště vám nezvetšely a opánky ti na noze nezpuchřely.
#29:5 Nejedli jste chléb a nepili jste víno ani opojný nápoj, abyste poznali, že já jsem Hospodin, váš Bůh.
#29:6 Když jste přišli k tomuto místu, vytáhl proti nám k boji chešbónský král Síchon a bášanský král Óg, ale pobili jsme je
#29:7 a zabrali jsme jejich zemi a dali ji v dědictví Rúbenovcům, Gádovcům a polovině kmene Manasesovců.
#29:8 Dbejte tedy na slova této smlouvy a dodržujte je, abyste měli úspěch ve všem, co budete dělat.
#29:9 Vy všichni stojíte dnes před Hospodinem, svým Bohem, vaši představitelé podle vašich kmenů, vaši starší, vaši správci, všichni izraelští mužové,
#29:10 vaše děti, vaše ženy i tvůj host, který je ve tvém táboře, i tvůj drvoštěp a nosič vody,
#29:11 abys vešel ve smlouvu Hospodina, svého Boha, kterou Hospodin, tvůj Bůh, s tebou dnes pod kletbou uzavírá:
#29:12 že si z tebe dnes ustavuje svůj lid a chce ti být Bohem, jak k tobě mluvil a jak přísahal tvým otcům, Abrahamovi, Izákovi a Jákobovi.
#29:13 Avšak nejen s vámi uzavírám pod kletbou tuto smlouvu,
#29:14 jak s tím, který tu dnes s námi stojí před Hospodinem, naším Bohem, tak s tím, který tu dnes s námi není.
#29:15 Vždyť víte, jak jsme přebývali v egyptské zemi a jak jsme procházeli územím pronárodů; sami jste jím prošli.
#29:16 Viděli jste jejich ohyzdné a hnusné modly, dřevo a kámen, stříbro a zlato, které mají.
#29:17 Ať není mezi vámi muž ani žena, čeleď ani kmen, jejichž srdce by se dnes odvrátilo od Hospodina, našeho Boha, takže by šli sloužit bohům těch pronárodů. Ať není mezi vámi kořen plodící jed a pelyněk.
#29:18 Mohlo by se stát, že by někdo slyšel slova této kletby a v duchu by si dobrořečil: „Budu mít pokoj, i když si budu žít v zarputilosti srdce, vždyť to dopadne stejně s opilým jako se střízlivým.“
#29:19 Takovému nebude Hospodin ochoten odpustit. Tehdy vzplane Hospodinův hněv i jeho rozhorlení proti tomu muži a dolehne na něj každá kletba zapsaná v této knize a Hospodin vymaže zpod nebes jeho jméno.
#29:20 Hospodin ho k jeho zkáze odloučí od všech izraelských kmenů, podle všech kleteb smlouvy, zapsaných v knize tohoto zákona.
#29:21 I řekne příští pokolení, vaši synové, kteří povstanou po vás, i cizinec, který přijde z daleké země, až uvidí rány té země a její nemoci, které na ni Hospodin dopustil,
#29:22 celou zemi spálenou sírou a solí, že není osévána, že v ní neklíčí a nevzchází žádná bylina, že je vyvrácená jako Sodoma a Gomora, Adma a Sebójím, které vyvrátil Hospodin ve svém hněvu a rozhořčení,
#29:23 všechny pronárody řeknou: „Proč tak Hospodin s touto zemí naložil? Proč tento veliký planoucí hněv?“
#29:24 A řeknou: „Proto, že opustili smlouvu Hospodina, Boha svých otců, kterou s nimi uzavřel, když je vyvedl z egyptské země,
#29:25 a šli sloužit jiným bohům a klaněli se jim, bohům, jež neznali a které jim neurčil.
#29:26 Proto Hospodin vzplanul proti té zemi hněvem a uvedl na ni všechno zlořečení zapsané v této knize.
#29:27 V hněvu, v rozhořčení a ve velikém rozlícení je Hospodin vyvrátil z jejich země a vyvrhl je do jiné země, jak je tomu dnes.“
#29:28 Skryté věci patří Hospodinu, našemu Bohu, zjevné však patří navěky nám a našim synům, abychom dodržovali všechna slova tohoto zákona. 
#30:1 Když na tebe toto všechno přijde, požehnání i zlořečení, jež jsem ti předložil, a ty si to vezmeš k srdci, kdekoli budeš ve všech pronárodech, do kterých tě zapudí Hospodin, tvůj Bůh,
#30:2 a navrátíš se k Hospodinu, svému Bohu, a budeš ho poslouchat, ty i tvoji synové, celým svým srdcem a celou svou duší podle všeho, co ti dnes přikazuji,
#30:3 změní Hospodin, tvůj Bůh, tvůj úděl, slituje se nad tebou a shromáždí tě zase ze všech národů, kam tě Hospodin, tvůj Bůh, rozptýlil.
#30:4 Kdybys byl zapuzen až na kraj světa, Hospodin, tvůj Bůh, tě odtud shromáždí a vezme tě odtamtud.
#30:5 Hospodin, tvůj Bůh, tě uvede do země, kterou obsadili tvoji otcové, a ty ji znovu obsadíš a on ti bude prokazovat dobrodiní a rozmnoží tě víc než tvé otce.
#30:6 Hospodin, tvůj Bůh, obřeže tvé srdce i srdce tvého potomstva a budeš milovat Hospodina, svého Boha, celým svým srdcem a celou svou duší a budeš živ.
#30:7 Všechny tyto kletby pak vloží Hospodin, tvůj Bůh, na tvé nepřátele a na ty, kdo tě nenávistně pronásledovali.
#30:8 Ty budeš opět poslouchat Hospodina a dodržovat všechny jeho příkazy, které ti dnes udílím.
#30:9 Hospodin, tvůj Bůh, ti dá nadbytek dobrého v každé práci tvých rukou, plodu tvého života a plodu tvého dobytka i plodin tvé role. Hospodin se bude opět nad tebou veselit k tvému dobru, jako se veselil nad tvými otci,
#30:10 budeš-li poslouchat Hospodina, svého Boha, a dbát na jeho přikázání a nařízení, zapsaná v knize tohoto zákona, a navrátíš-li se k Hospodinu, svému Bohu, celým svým srdcem a celou svou duší.
#30:11 Tento příkaz, který ti dnes udílím, není pro tebe ani nepochopitelný, ani vzdálený.
#30:12 Není v nebi, abys musel říkat: „Kdo nám vystoupí na nebe, vezme jej pro nás a ohlásí nám jej, abychom ho plnili?“
#30:13 Ani za mořem není, abys musel říkat: „Kdo se nám přeplaví přes moře, vezme jej pro nás a ohlásí nám jej, abychom ho plnili?“
#30:14 Vždyť to slovo je ti velmi blízko, ve tvých ústech a ve tvém srdci, abys je dodržoval.
#30:15 Hleď, předložil jsem ti dnes život a dobro i smrt a zlo;
#30:16 když ti dnes přikazuji, abys miloval Hospodina, svého Boha, chodil po jeho cestách a dbal na jeho přikázání, nařízení a právní ustanovení, pak budeš žít a rozmnožíš se; Hospodin, tvůj Bůh, ti bude žehnat v zemi, kterou přicházíš obsadit.
#30:17 Jestliže se však tvé srdce odvrátí a nebudeš poslouchat, ale dáš se svést a budeš se klanět jiným bohům a sloužit jim,
#30:18 oznamuji vám dnes, že úplně zaniknete. Nebudete dlouho živi v zemi, kam přecházíš přes Jordán, abys ji obsadil.
#30:19 Dovolávám se dnes proti vám svědectví nebes i země: Předložil jsem ti život i smrt, požehnání i zlořečení; vyvol si tedy život, abys byl živ ty i tvé potomstvo
#30:20 a miloval Hospodina, svého Boha, poslouchal ho a přimkl se k němu. Na něm závisí tvůj život a délka tvých dnů, abys mohl sídlit v zemi, o které přísahal Hospodin tvým otcům, Abrahamovi, Izákovi a Jákobovi, že jim ji dá. 
#31:1 Mojžíš šel a vyhlásil celému Izraeli tato slova.
#31:2 Řekl jim: „Je mi dnes sto dvacet let, nemohu již vycházet ani vcházet, a Hospodin mi řekl: ‚Nepřejdeš tento Jordán.‘
#31:3 Hospodin, tvůj Bůh, půjde před tebou, vyhladí před tebou ony pronárody a ty si je podrobíš. Jozue, ten půjde před tebou, jak mluvil Hospodin.
#31:4 Hospodin s nimi naloží, jako naložil se Síchonem a Ógem, emorejskými králi, které vyhladil, a s jejich zemí.
#31:5 Hospodin vám je vydá a naložíte s nimi zcela podle příkazu, který jsem vám dal.
#31:6 Buďte rozhodní a udatní, nebojte se a nemějte z nich strach, neboť sám Hospodin, tvůj Bůh, jde s tebou, nenechá tě klesnout a neopustí tě.“
#31:7 Mojžíš zavolal Jozua a řekl mu v přítomnosti celého Izraele: „Buď rozhodný a udatný, neboť ty vejdeš s tímto lidem do země, kterou Hospodin přísahal dát jejich otcům. Ty jim ji předáš do dědictví.
#31:8 Sám Hospodin půjde před tebou, bude s tebou, nenechá tě klesnout a neopustí tě; neboj se a neděs.“
#31:9 I napsal Mojžíš tento zákon a předal jej kněžím, Léviovcům, kteří nosili schránu Hospodinovy smlouvy, i všem izraelským starším.
#31:10 Mojžíš jim přikázal: „Každého sedmého roku, v roce určeném k promíjení dluhu, o slavnosti stánků,
#31:11 až přijde celý Izrael, aby se ukázal před tváří Hospodina, tvého Boha, na místě, které vyvolí, budeš předčítat tento zákon před celým Izraelem, aby jej slyšeli.
#31:12 Shromáždi lid, muže i ženy a děti i hosta, který žije v tvých branách, aby slyšeli a učili se bát Hospodina, vašeho Boha, a bedlivě dodržovali všechna slova tohoto zákona.
#31:13 Též jejich synové, kteří ho ještě neznají, ať poslouchají a učí se bát Hospodina, vašeho Boha, po všechny dny, co budete žít v zemi, do níž přejdete přes Jordán, abyste ji obsadili.“
#31:14 Hospodin řekl Mojžíšovi: „Hle, přiblížily se dny tvé smrti. Povolej Jozua a postavte se do stanu setkávání a já mu dám příkaz.“ Šel tedy Mojžíš s Jozuem a postavili se do stanu setkávání.
#31:15 I ukázal se Hospodin ve stanu v oblakovém sloupu. Oblakový sloup stál nad vchodem do stanu.
#31:16 Hospodin řekl Mojžíšovi: „Hle, ty ulehneš ke svým otcům a tento lid povstane a bude smilnit s cizími bohy země, do níž vstupuje, opustí mě a poruší mou smlouvu, kterou jsem s ním uzavřel.
#31:17 V onen den vzplane proti němu můj hněv. Opustím je a skryji před nimi svou tvář. Bude za pokrm svým nepřátelům a stihne ho mnohé zlo a soužení. V onen den řekne: ‚Zdali mě nestihlo toto zlo proto, že můj Bůh není uprostřed nás?‘
#31:18 A já onoho dne skryji nadobro svou tvář pro všechno to zlo, jehož se dopustil, když se obrátil k jiným bohům.
#31:19 Nyní si tedy napište tuto píseň. Uč jí syny Izraele, vlož jim ji do úst, aby byla tato píseň mým svědkem proti synům Izraele.
#31:20 Já ho totiž uvedu do té země, oplývající mlékem a medem, kterou jsem přísežně slíbil jeho otcům, ale on bude jíst, nasytí se a ztuční a obrátí se k jiným bohům. Jim budou sloužit, mne zneváží a mou smlouvu zruší.
#31:21 Až pak na něj dolehne mnohé zlo a soužení, bude tato píseň, jež neupadne v zapomenutí v ústech jeho potomstva, vypovídat proti němu jako svědek. Znám výtvory jeho mysli, vím, co udělá dnes, ještě dřív, než ho uvedu do země, kterou jsem mu přísežně slíbil.“
#31:22 Mojžíš tedy onoho dne napsal tu píseň a učil jí Izraelce.
#31:23 Hospodin pak přikázal Jozuovi, synu Núnovu, a řekl: „Buď rozhodný a udatný, neboť ty uvedeš syny Izraele do země, kterou jsem jim přísežně slíbil, a já budu s tebou.“
#31:24 Když Mojžíš dokončil zápis slov tohoto zákona do knihy,
#31:25 přikázal lévijcům nosícím schránu Hospodinovy smlouvy:
#31:26 „Vezměte knihu tohoto zákona a uložte ji po straně schrány smlouvy Hospodina, vašeho Boha. Tam bude proti tobě svědkem.
#31:27 Vždyť já znám tvou vzdorovitost a tvou tvrdou šíji: Hle, už dnes, kdy ještě mezi vámi žiji, vzdorujete Hospodinu; tím spíše po mé smrti!
#31:28 Svolejte ke mně všechny starší svých kmenů a své správce. Chci před nimi vyhlásit tato slova a dovolat se proti nim svědectví nebes i země.
#31:29 Vím, že po mé smrti zabřednete do zkázy a sejdete z cesty, kterou jsem vám přikázal. V posledních dnech vás pak potká zlo, protože jste páchali to, co je zlé v Hospodinových očích, a uráželi ho dílem svých rukou.“
#31:30 Nato Mojžíš vyhlásil před celým shromážděním Izraele slova této písně až do konce. 
#32:1 Naslouchejte, nebesa, budu mluvit, poslouchej, země, řeči mých úst.
#32:2 Ať kane jako déšť mé naučení, nechť se snáší má řeč jako rosa, jako prška na mladou trávu, jako vlahé krůpěje na bylinu.
#32:3 Hlásám Hospodinovo jméno, přiznejte velikost našemu Bohu!
#32:4 On je Skála. Jeho dílo je dokonalé, na všech jeho cestách je právo. Bůh je věrný a bez podlosti, je spravedlivý a přímý.
#32:5 Pokolení pokřivené a potměšilé do zkázy se vrhlo. Pro svá poskvrnění přestalo být jeho syny.
#32:6 Takto odplácíte Hospodinu, lide zbloudilý a nemoudrý? Cožpak není on tvůj Otec? Vždyť mu patříš. On tě učinil a zpevnil.
#32:7 Rozpomeň se na dávné dny, snaž se porozumět létům zašlých pokolení, vyptávej se svého otce, on ti poví, svých starců, oni ti řeknou:
#32:8 Když Nejvyšší přiděloval pronárodům dědictví, když rozsazoval lidské syny, stanovil hranice kdejakého lidu podle počtu synů Izraele.
#32:9 Hospodinovým podílem je jeho lid, vyměřeným dílem jeho dědictví je Jákob.
#32:10 Našel ho v zemi divokých pouští, v pustotě kvílících pustin, zahrnul ho svou péčí, chránil ho jako zřítelnici oka.
#32:11 Jako bdí orel nad svým hnízdem a nad svými mláďaty se vznáší, svá křídla rozprostírá, své mládě bere a na své peruti je nosí,
#32:12 tak Hospodin sám ho vedl, žádný cizí bůh s ním nebyl.
#32:13 Dovolil mu jezdit po posvátných návrších země, aby jedl, čím oplývá pole, kojil ho medem ze skaliska, olejem z křemene skály,
#32:14 smetanou krav a mlékem ovcí spolu s tukem jehňat a beranů bášanského plemene i kozlů; sytil ho bělí pšeničného zrna. Pil jsi i ohnivé víno, krev hroznů.
#32:15 Ješurún ztučněl a zbujněl, ztučněl jsi, obrostl tukem a ztloustl. Bohem, který ho učinil, opovrhl, potupil Skálu své spásy.
#32:16 Bohy cizími ho popouzeli k žárlivosti, ohavnými modlami ho uráželi.
#32:17 Obětovali běsům, a ne Bohu, božstvům, jež ani neznali, těm novým, nedávno povstalým, před nimiž se vaši otcové nechvěli hrůzou.
#32:18 Opomněl jsi Skálu, která tě zplodila, zapomněls na Boha, který tě v bolestech zrodil.
#32:19 Hospodin to viděl. Odvrhl je, uražen skutky svých synů a dcer.
#32:20 Řekl: „Skryji před nimi svou tvář, uvidím, jaký vezmou konec. Je to proradné pokolení, synové bez věrnosti.
#32:21 Popouzeli mě lžibohem k žárlivosti, svými přeludy mě uráželi; já je popudím lžilidem, pobloudilým pronárodem jim urážky splatím.
#32:22 Mým hněvem se vznítil oheň, až do nejhlubšího podsvětí šlehá, pohltí zemi i to, co se z ní těží, sežehne základy hor.
#32:23 Samé zlo na ně shrnu, vystřílím na ně své šípy.
#32:24 Budou vysíleni hladem, stráveni nákazou, přehořkou morovou ranou. Vydám je zubům šelem a jedovatým plazům v prachu.
#32:25 Venku jim připraví sirobu meč, v komnatách zachvátí strach jinocha i pannu, kojence i muže šedivého.
#32:26 Řekl bych: Rozpráším je, vyhladím jejich památku mezi lidmi.
#32:27 Ale obávám se urážek nepřítele a že to jejich protivníci nepochopí, že řeknou: ‚Vyvýšila se naše vlastní ruka; Hospodin nic takového nevykonal.‘“
#32:28 Je to pronárod, jenž ztratil soudnost, nejsou schopni porozumět.
#32:29 Kdyby byli moudří, jednali by prozíravě, pochopili by, jak skončí.
#32:30 Jak to, že jeden zažene tisíc a dva zaženou na útěk deset tisíc! Ne-li proto, že je prodala jejich Skála, že je Hospodin vydal v plen?
#32:31 Jejich skála není jako Skála naše, to mohou posoudit i naši nepřátelé.
#32:32 Jejich réva je z révy sodomské, z vinic Gomory, jejich hrozny, hrozny jedovaté, mají trpká zrnka.
#32:33 Jejich víno, jedovina dračí, krutý zmijí jed.
#32:34 „Což to není uschováno u mne, zapečetěno v mých pokladnicích?
#32:35 Má je pomsta i odplata, přijde včas a jejich noha zakolísá, den jejich běd se blíží, řítí se na ně, co je jim uchystáno.“
#32:36 Hospodin svůj lid obhájí, bude mít se svými služebníky soucit, až uzří, že ubývá sil, že je konec se zajatým i se zanechaným.
#32:37 Tu řekne: „Kdepak jsou jejich bohové, skála, k níž se přivinuli?
#32:38 Ti, kteří jídali tuk jejich obětí, popíjeli víno jejich úliteb? Ať povstanou a pomohou vám, ať jsou vám skrýší!
#32:39 Nyní hleďte, jsem jedině já, jiný bůh vedle mne není, já usmrcuji i obživuji, zdeptal jsem, a zase zhojím, není, kdo by vytrhl z mé ruky.
#32:40 Pozvedám ruku k nebi a pravím: Já jsem živ navěky!
#32:41 Nabrousím-li svůj blýskavý meč, chopí-li se moje ruka soudu, vykonám nad svými protivníky pomstu, odplatím těm, kdo mě nenávidí.
#32:42 Opojím své šípy krví - můj meč bude požírat maso -, krví skolených a zajatých, hlavou nepřítele s vlajícími vlasy.“
#32:43 Pronárody, zaplesejte s jeho lidem, vždyť on pomstí krev svých služebníků, vykoná nad svými protivníky pomstu a svou zemi, svůj lid, zprostí viny.
#32:44 Mojžíš předstoupil a přednesl všechna slova této písně lidu, on i Hóšea, syn Núnův.
#32:45 Když Mojžíš domluvil k celému Izraeli všechna tato slova,
#32:46 řekl jim: „Upněte své srdce ke všem slovům, která vám dnes dosvědčuji. Přikážete svým synům, aby bedlivě dodržovali všechna slova tohoto zákona.
#32:47 Pro vás není to slovo prázdné, ono je váš život. Pro toto slovo budete dlouho žít v zemi, do níž přejdete přes Jordán, abyste ji obsadili.“
#32:48 Ještě téhož dne promluvil Hospodin k Mojžíšovi:
#32:49 „Vystup zde na pohoří Abarím, na horu Nebó, která je v moábské zemi, naproti Jerichu, a pohleď na kenaanskou zemi, kterou dávám synům Izraele do vlastnictví.
#32:50 Vystup na tu horu, na ní zemřeš a budeš připojen ke svému lidu, jako zemřel tvůj bratr Áron na hoře Hóru a byl připojen ke svému lidu,
#32:51 protože jste se mi zpronevěřili uprostřed Izraelců u Vod sváru v Kádeši na poušti Sinu, že jste uprostřed Izraelců nedosvědčili mou svatost.
#32:52 Proto uzříš zemi jen odtud, ale tam do země, kterou dávám synům Izraele, nevstoupíš.“ 
#33:1 Toto je požehnání, kterým Mojžíš, muž Boží, žehnal před svou smrtí syny Izraele.
#33:2 Pravil: „Hospodin přišel ze Sínaje, jako slunce jim vzešel ze Seíru, zaskvěl se z hory Páranu, přišel s desetitisíci svatými, po jeho pravici jim z ohně vzešel Zákon.“
#33:3 Ano, lidská pokolení jsou mu milá. Všichni jeho svatí jsou v tvé ruce, přivinuli se k tvým nohám, budou se učit z tvých řečí
#33:4 Zákonu, který nám přikázal Mojžíš jako odkaz Jákobovu shromáždění.
#33:5 Tehdy se Hospodin stal králem v Ješurúnovi, když se shromažďovali představitelé lidu spolu s izraelskými kmeny.
#33:6 „Živ buď, Rúbene, neumírej, i když je tvých mužů nepatrný počet.“
#33:7 A toto je pro Judu. Pravil: „Slyš, Hospodine, hlas Judův, přiveď ho k jeho lidu. Vlastníma rukama vede svou při. Buď mu pomocí proti jeho protivníkům.“
#33:8 O Lévim pravil: „Tvé tumím a urím, Hospodine, patří muži tobě zbožně oddanému. Vyzkoušel jsi ho při pokušení v Masse, měl jsi s ním svár při Vodách sváru u Meríby.
#33:9 Řekl o svém otci a o své matce: ‚Nehledím na ně‘; na své bratry se neohlíží, své syny nezná. Oni dbali na tvou řeč, zachovávali tvou smlouvu.
#33:10 Budou vyučovat tvým právům Jákoba a tvému zákonu Izraele, budou klást před tebe kadidlo a na tvůj oltář celopaly.
#33:11 Žehnej, Hospodine, jeho zdatnosti a měj zalíbení v díle jeho rukou. Zdeptej bedra těch, kdo povstávají proti němu, aby už nepovstali, kdo ho nenávidí!“
#33:12 O Benjamínovi pravil: „Hospodinův miláček to je. Ať u něho přebývá v bezpečí. On ho bude chránit po celý čas, vždyť přebývá mezi jeho úbočími.“
#33:13 O Josefovi pravil: „Požehnána buď od Hospodina jeho země výtečnou rosou nebes i propastnou tůní, jež odpočívá dole,
#33:14 výtečnými úrodami vyzrálými sluncem, vším výtečným, co přináší měsíc,
#33:15 nejlepšími plody pravěkých hor, výtečnostmi pahorků dávnověkých,
#33:16 výtečnostmi země a všeho, co je na ní, a zalíbením toho, jenž přebývá v keři. To nechť přijde na hlavu Josefovu, na temeno zasvěcence mezi bratry.
#33:17 Je plný důstojnosti jako prvorozený býk, jeho rohy jsou rohy jednorožců; nabere na ně lidská pokolení i s dálavami země. Takové ať jsou desetitisíce Efrajimovy, takové ať jsou tisíce Manasesovy.“
#33:18 O Zabulónovi pravil: „Raduj se, Zabulóne, při svém vycházení, i ty, Isachare, ve svých stanech.
#33:19 Svolají na horu lidská pokolení a budou tam obětovat oběti spravedlnosti. Budou sát hojnost z moří, poklady ukryté v písku.“
#33:20 O Gádovi pravil: „Požehnán buď ten, jenž Gádovi rozšiřuje prostor. Uložil se jako lvice, rozsápe paži i lebku.
#33:21 Vyhlédl si prvotinu tam, kde je uschován velitelský podíl. Přišel s představiteli lidu, vykonal Hospodinovu spravedlnost a jeho práva vůči Izraeli.“
#33:22 O Danovi pravil: „Dan je lví mládě, vyskočí z Bášanu.“
#33:23 O Neftalím pravil: „Neftalí je nasycen přízní, je plný Hospodinova požehnání, ovládne moře i jih.“
#33:24 O Ašerovi pravil: „Nad ostatní syny požehnán buď Ašer, buď oblíben u svých bratří, v oleji ať smáčí svou nohu.
#33:25 Tvé závory ať jsou z železa a bronzu, tvá síla ať provází všechny tvé dny.“
#33:26 „Nikdo není jako Bůh, Ješurúne; na pomoc ti jede po nebesích, na oblacích ve své velebnosti.
#33:27 Domov je v odvěkém Bohu, na jeho věčných pažích. Zahnal před tebou nepřítele a řekl: ‚Vyhlaď ho!‘
#33:28 Izrael přebývá bezpečně - jedinečný je Jákobův Pramen - v zemi obilí a moštu; též jeho nebesa kanou rosou.
#33:29 Blaze tobě, Izraeli! Kdo je ti roven, lide vysvobozený Hospodinem? On je štítem tvé pomoci a mečem tvé velebnosti. Před tebou selže síla tvých nepřátel, pošlapeš jejich posvátná návrší!“ 
#34:1 Mojžíš vystoupil z Moábských pustin na horu Nebó, na vrchol Pisgy; ta je naproti Jerichu. Hospodin mu ukázal celou zemi: Gileád až po Dan,
#34:2 celý kraj Neftalího i zemi Efrajimovu a Manasesovu a celou zemi Judovu až k Zadnímu moři,
#34:3 i Negeb a okrsek Jerišské pláně, Palmové město, až k Sóaru.
#34:4 Pak mu Hospodin řekl: „Toto je země, o které jsem přísahal Abrahamovi, Izákovi a Jákobovi slovy: Dám ji tvému potomstvu. Dal jsem ti ji spatřit na vlastní oči, ale nepřejdeš tam.“
#34:5 I zemřel Mojžíš, služebník Hospodinův, tam v moábské zemi podle Hospodinovy výpovědi.
#34:6 Pochoval ho v údolí v moábské zemi naproti Bét-peóru. Nikdo až dodnes nezná jeho hrob.
#34:7 Mojžíšovi bylo sto dvacet let, když umřel. Zrak mu nepohasl a svěžest ho neopustila.
#34:8 Synové Izraele oplakávali Mojžíše v Moábských pustinách po třicet dní, dokud neuplynuly dny pláče a truchlení pro Mojžíše.
#34:9 Jozue, syn Núnův, byl naplněn duchem moudrosti. Mojžíš totiž na něho vložil své ruce. Izraelci ho poslouchali a činili, jak přikázal Hospodin Mojžíšovi.
#34:10 Nikdy však již v Izraeli nepovstal prorok jako Mojžíš, jemuž by se dal Hospodin poznat tváří v tvář,
#34:11 se všemi znameními a zázraky, jež ho poslal Hospodin konat v egyptské zemi na faraónovi a všech jeho služebnících i celé jeho zemi,
#34:12 i se všemi skutky pevné ruky, se vším, co vzbuzovalo velikou bázeň, co konal Mojžíš před zraky celého Izraele.  

\book{Joshua}{Josh}
#1:1 Po smrti Mojžíše, služebníka Hospodinova, řekl Hospodin Jozuovi, synu Núnovu, který Mojžíšovi přisluhoval:
#1:2 „Mojžíš, můj služebník, zemřel. Nyní tedy vstaň a přejdi s veškerým tímto lidem přes tento Jordán do země, kterou dávám Izraelcům.
#1:3 Dal jsem vám každé místo, na které vaše noha šlápne, jak jsem přislíbil Mojžíšovi.
#1:4 Vaše pomezí povede od stepi a tohoto Libanónu až k veliké řece, řece Eufratu, podél celé země Chetejců až k Velkému moři, kde zapadá slunce.
#1:5 Po všechny dny tvého života se proti tobě nikdo nepostaví. Jako jsem byl s Mojžíšem, budu i s tebou. Nenechám tě klesnout a neopustím tě.
#1:6 Buď rozhodný a udatný, neboť ty rozdělíš tomuto lidu zemi v dědictví, jak jsem se přísežně zavázal jejich otcům, že jim ji dám.
#1:7 Jen buď rozhodný a velmi udatný, bedlivě plň vše, co je v zákoně, který ti přikázal Mojžíš, můj služebník. Neodchyluj se od něho napravo ani nalevo; tak budeš jednat prozíravě všude, kam půjdeš.
#1:8 Kniha tohoto zákona ať se nevzdálí od tvých úst. Rozjímej nad ním ve dne v noci, abys mohl bedlivě plnit vše, co je v něm zapsáno. Potom tě bude na tvé cestě provázet zdar, potom budeš jednat prozíravě.
#1:9 Nepřikázal jsem ti snad: Buď rozhodný a udatný, neměj strach a neděs se, neboť Hospodin, tvůj Bůh, bude s tebou všude, kam půjdeš?“
#1:10 I přikázal Jozue správcům lidu:
#1:11 „Projděte táborem a přikažte lidu: Připravte si zásobu potravin, neboť po třech dnech přejdete tento Jordán a půjdete obsadit zemi, kterou vám Hospodin, váš Bůh, dává do vlastnictví.“
#1:12 Rúbenovcům, Gádovcům a polovině kmene Manasesova pak Jozue řekl:
#1:13 „Pamatujte na to, co vám přikázal Mojžíš, služebník Hospodinův. Řekl: ‚Hospodin, váš Bůh, vás přivede do odpočinutí a dá vám tuto zemi.‘
#1:14 Vaše ženy, děti a stáda nechť zůstanou v zemi, kterou vám dal Mojžíš v Zajordání. Ale vy ostatní, udatní bohatýři, vojensky seřazeni přejdete před svými bratřími řeku a budete jim pomáhat,
#1:15 dokud Hospodin nepřivede do odpočinutí vaše bratry jako vás. Až i oni obsadí zemi, kterou jim dává Hospodin, váš Bůh, vrátíte se do své země a obsadíte ji; tu vám dal Mojžíš, služebník Hospodinův, v Zajordání na východě.“
#1:16 Odpověděli Jozuovi: „Učiníme všechno, co jsi nám přikázal, a půjdeme všude, kam nás pošleš.
#1:17 Budeme tě poslouchat stejně, jako jsme poslouchali Mojžíše. Jen ať Hospodin, tvůj Bůh, je s tebou, jako byl s Mojžíšem.
#1:18 Kdo se tvému rozkazu postaví na odpor a tvých slov neuposlechne ve všem, co mu přikážeš, je hoden smrti. Jen buď rozhodný a udatný.“ 
#2:1 Jozue, syn Núnův, vyslal potají ze Šitímu dva muže jako zvědy. Řekl: „Jděte, prohlédněte tu zemi i Jericho.“ Šli tedy a vstoupili do domu jedné ženy jménem Rachab, nevěstky, a tam přespali.
#2:2 Králi Jericha bylo ohlášeno: „V noci sem přišli nějací izraelští muži, aby obhlédli zemi.“
#2:3 Jerišský král dal Rachabě rozkaz: „Vyveď ty muže, kteří k tobě přišli a vstoupili do tvého domu. Přišli proto, aby obhlédli celou zemi.“
#2:4 Ale ta žena muže odvedla, ukryla a řekla: „Ano, ti muži ke mně přišli, ale já jsem nevěděla, odkud jsou.
#2:5 Když při setmění zavírali bránu, ti muži odešli. Nevím, kam šli. Rychle je pronásledujte, ať je dostihnete.“
#2:6 Ona však je vyvedla na střechu a skryla je v pazdeří, které měla na střeše složené;
#2:7 a oni ty muže pronásledovali směrem k Jordánu až k brodům. Jakmile pronásledovatelé vyšli, hned za nimi bránu zavřeli.
#2:8 Zvědové se ještě neuložili k spánku, když k nim vstoupila na střechu.
#2:9 Řekla těm mužům: „Vím, že Hospodin dal zemi vám. Padla na nás hrůza před vámi a všichni obyvatelé země propadli před vámi zmatku.
#2:10 Slyšeli jsme, jak Hospodin před vámi vysušil vody Rákosového moře, když jste vycházeli z Egypta, a jak jste v Zajordání naložili se dvěma emorejskými králi, se Síchonem a Ógem, které jste zahubili jako klaté.
#2:11 Jakmile jsme to uslyšeli, ztratili jsme odvahu a pozbyli jsme ducha, poněvadž Hospodin, váš Bůh, je Bohem nahoře na nebi i dole na zemi.
#2:12 Zavažte se mi nyní prosím přísahou při Hospodinu, že také vy prokážete milosrdenství domu mého otce, jako jsem já prokázala milosrdenství vám. Dejte mi věrohodné znamení,
#2:13 že ponecháte naživu mého otce a matku, mé bratry a sestry i vše, co jim náleží, a že nás vysvobodíte před smrtí.“
#2:14 Muži jí odpověděli: „Jsme odhodláni za vás zemřít. Nesmíte však vyzradit toto naše ujednání. Až nám Hospodin vydá zemi, prokážeme ti milosrdenství a osvědčíme věrnost.“
#2:15 Potom je spustila po provaze z okna; její dům byl totiž v hradební zdi, bydlela na hradbách.
#2:16 A řekla jim: „Jděte na tamtu horu, aby na vás pronásledovatelé nenarazili, a skrývejte se tam po tři dny, dokud se nevrátí ti, kdo vás pronásledují; potom jděte svou cestou.“
#2:17 Muži ji upozornili: „Budeme zproštěni přísahy, jíž jsi nás zavázala,
#2:18 jestliže neuvážeš, až vstoupíme do země, tuto šňůru z karmínových vláken v okně, z něhož jsi nás spustila, a neshromáždíš k sobě do domu svého otce a matku, své bratry a celý svůj dům.
#2:19 Kdo vyjde ze dveří tvého domu ven, jeho krev padne na jeho hlavu, a my budeme bez viny. Avšak krev každého, kdo bude s tebou v domě, padne na naši hlavu, kdyby na něj někdo vztáhl ruku.
#2:20 Vyzradíš-li toto naše ujednání, budeme zproštěni přísahy, jíž jsi nás zavázala.“
#2:21 Odvětila: „Staň se podle vašich slov.“ Nato je propustila a oni odešli. Pak uvázala na okno karmínovou šňůru.
#2:22 Oni došli až na horu, kde zůstali tři dny, dokud se pronásledovatelé nevrátili; ti prohledali celou cestu, nikoho však nenašli.
#2:23 Oba muži se tedy vrátili; sestoupili z hory, přešli Jordán a přišli k Jozuovi, synu Núnovu. Vyprávěli mu o všem, co se jim přihodilo.
#2:24 Řekli Jozuovi: „Hospodin nám dal celou zemi do rukou. Všichni obyvatelé země propadli před námi zmatku.“ 
#3:1 Za časného jitra vytáhl Jozue se všemi Izraelci ze Šitímu, až přišli k Jordánu; tam přenocovali, dříve než jej přešli.
#3:2 Když uplynuly tři dny, prošli správcové táborem
#3:3 a přikázali lidu: „Jakmile spatříte schránu smlouvy Hospodina, svého Boha, a lévijské kněze, kteří ji nesou, vytáhnete ze svého místa a půjdete za ní.
#3:4 Mezi vámi a ní bude ovšem odstup zhruba dvou tisíc loket; nepřibližujte se k ní. Půjdete za ní, abyste poznali cestu, kudy se máte ubírat, neboť nikdy předtím jste touto cestou neprocházeli.“
#3:5 Jozue pak vyzval lid: „Posvěťte se, neboť Hospodin zítra mezi vámi učiní podivuhodné věci.“
#3:6 Potom Jozue řekl kněžím: „Zvedněte schránu smlouvy a ubírejte se před lidem.“ I zvedli schránu smlouvy a šli před lidem.
#3:7 Hospodin řekl Jozuovi: „Dnešního dne jsem tě začal před zraky celého Izraele vyvyšovat, aby poznali, že jsem s tebou, jako jsem byl s Mojžíšem.
#3:8 Kněžím nesoucím schránu smlouvy přikážeš toto: Až vstoupíte na pokraj jordánských vod, zůstanete v Jordánu stát.“
#3:9 Jozue vyzval Izraelce: „Přistupte sem a slyšte slova Hospodina, svého Boha.“
#3:10 A Jozue pokračoval: „Poznáte, že uprostřed vás je živý Bůh. Ten před vámi vyžene Kenaance, Chetejce, Chivejce, Perizejce, Girgašejce, Emorejce i Jebúsejce.
#3:11 Hle, před vámi přejde Jordán schrána smlouvy Pána celé země.
#3:12 Nyní si vyberte z izraelských kmenů dvanáct mužů, po jednom muži z každého kmene.
#3:13 Jakmile nohy kněží nesoucích schránu Hospodina, Pána celé země, spočinou ve vodách Jordánu, jordánské vody se rozestoupí; vody řítící se shora zůstanou stát jako jednolitá hráz.“
#3:14 Tak se stalo. Lid vytáhl ze svých stanů, aby přešel Jordán, a kněží nesoucí schránu smlouvy šli před lidem.
#3:15 Když ti, kteří nesli schránu, přišli k Jordánu a nohy kněží nesoucích schránu se smočily na pokraji vod - Jordán je vždycky v čas žně rozvodněn daleko z břehů -,
#3:16 tu vody řítící se shora zůstaly stát a postavily se jako jednolitá hráz velmi daleko odtud u města Adamu, které leží při Saretánu; a ty, které tekly dolů k Pustému moři, k moři Solnému, se vytratily. Tak přešel lid naproti Jerichu.
#3:17 Kněží nesoucí schránu Hospodinovy smlouvy stáli nepohnutě na suché zemi uprostřed Jordánu a celý Izrael přecházel po suchu, dokud celý ten pronárod do jednoho nepřešel Jordán. 
#4:1 Když celý ten pronárod do jednoho přešel Jordán, řekl Hospodin Jozuovi:
#4:2 „Vyberte si z lidu dvanáct mužů, po jednom z každého kmene,
#4:3 a přikažte jim: Vyneste zprostřed Jordánu, odtud, kde nepohnutě stály nohy kněží, dvanáct kamenů, vezměte je s sebou a složte je na místě, kde této noci přenocujete.“
#4:4 Jozue tedy povolal dvanáct mužů, které z Izraelců určil, po jednom z každého kmene,
#4:5 a řekl jim: „Jděte před schránu Hospodina, svého Boha, doprostřed Jordánu. Každý si vyzvedne na rameno jeden kámen, podle počtu kmenů Izraelců.
#4:6 To bude mezi vámi na znamení. Až se v budoucnu budou vaši synové vyptávat: ‚Čím jsou pro vás tyto kameny?‘,
#4:7 odpovíte jim: ‚Vody Jordánu se rozestoupily před schránou Hospodinovy smlouvy; když procházela Jordánem, vody Jordánu se rozestoupily.‘ Tyto kameny budou Izraelcům pamětným znamením navěky.“
#4:8 Izraelci učinili, jak Jozue přikázal. Vynesli zprostřed Jordánu dvanáct kamenů podle počtu kmenů Izraelců, jak Jozuovi uložil Hospodin, a přešli s nimi na místo, kde se chystali přenocovat; tam je složili.
#4:9 Též na místě uprostřed Jordánu, kde stály nohy kněží nesoucích schránu smlouvy, postavil Jozue dvanáct kamenů a ty tam jsou až dodnes.
#4:10 Kněží nesoucí schránu stáli uprostřed Jordánu, dokud nebylo vykonáno úplně všechno, co podle Hospodinova příkazu uložil Jozue lidu, jak Jozuovi přikázal Mojžíš. Lid pak rychle přešel.
#4:11 Když přešel všechen lid do jednoho, přešla také Hospodinova schrána i kněží, kteří byli před lidem.
#4:12 V čele Izraelců přešli vojensky seřazeni též Rúbenovci, Gádovci a polovina kmene Manasesova, jak jim uložil Mojžíš.
#4:13 Kolem čtyřiceti tisíc vojáků táhlo ve zbroji před Hospodinem do boje na Jerišské pustiny.
#4:14 Onoho dne vyvýšil Hospodin Jozua před očima celého Izraele, takže měli před ním bázeň po celý jeho život, jako měli bázeň před Mojžíšem.
#4:15 Hospodin řekl Jozuovi:
#4:16 „Přikaž kněžím nesoucím schránu svědectví, ať vystoupí z Jordánu.“
#4:17 Jozue tedy kněžím přikázal: „Vystupte z Jordánu.“
#4:18 Sotva však kněží nesoucí schránu Hospodinovy smlouvy zprostřed Jordánu vystoupili, hned jak se nohy kněží odtrhly ode dna a stanuly na suchu, vrátily se vody Jordánu na své místo a rozlévaly se jako předtím daleko z břehů.
#4:19 Desátého dne prvního měsíce vystoupil lid z Jordánu. Utábořili se v Gilgálu, při východním okraji Jericha.
#4:20 Oněch dvanáct kamenů, které vzali z Jordánu, postavil Jozue v Gilgálu.
#4:21 Řekl Izraelcům: „Až se v budoucnu budou vaši synové vyptávat svých otců: ‚Co je to za kameny?‘,
#4:22 seznamte své syny s tímto: Izrael přešel tento Jordán po suchu.
#4:23 Hospodin, váš Bůh, vysušil před vámi vody Jordánu, dokud jste nepřešli, jako to učinil Hospodin, váš Bůh, s Rákosovým mořem, které před vámi vysušil, dokud jste nepřešli,
#4:24 aby poznaly všechny národy země, jak mocná je ruka Hospodinova, a abyste se báli Hospodina, svého Boha, po všechny dny.“ 
#5:1 Když uslyšeli všichni emorejští králové, kteří sídlili na západ od Jordánu, a všichni kenaanští králové při moři, že Hospodin vysušil před Izraelci vody Jordánu, dokud nepřešli, ztratili odvahu a pozbyli před Izraelci ducha.
#5:2 V tom čase vyzval Hospodin Jozua: „Zhotov si kamenné nože a znovu zaveď, podruhé, pro Izraelce obřízku.“
#5:3 Jozue si tedy zhotovil kamenné nože a obřezal Izraelce u Pahorku předkožek.
#5:4 Důvod, proč je Jozue obřezal, byl tento: Všechen lid mužského pohlaví, který vyšel z Egypta, všichni bojovníci, pomřeli na poušti cestou po vyjití z Egypta.
#5:5 Všechen lid, který vyšel, byl obřezán; neobřezávali však nikoho z lidu, kdo se narodil na poušti cestou po vyjití z Egypta.
#5:6 Čtyřicet let chodili Izraelci pouští, dokud do jednoho nezmizel celý ten pronárod bojovníků, kteří vyšli z Egypta, protože neposlouchali Hospodina. O nich se Hospodin zapřisáhl, že jim nedovolí spatřit zemi, o které přisáhl jejich otcům, že nám ji dá, zemi oplývající mlékem a medem.
#5:7 Místo nich povolal jejich syny a ty Jozue obřezal, neboť byli neobřezaní; cestou je totiž neobřezávali.
#5:8 Když byl celý ten pronárod do jednoho obřezán, zůstali na svém místě v táboře, dokud se nezhojili.
#5:9 Hospodin Jozuovi řekl: „Dnes jsem od vás odvalil egyptskou potupu.“ Proto pojmenoval to místo Gilgál (to je Odvalení); jmenuje se tak až dodnes.
#5:10 Izraelci tábořili v Gilgálu. Čtrnáctého dne toho měsíce navečer slavili na Jerišských pustinách hod beránka.
#5:11 Druhého dne po hodu beránka začali jíst nekvašené chleby a pražené zrní z výtěžku země, právě toho dne.
#5:12 Toho druhého dne, kdy začali jíst z výtěžku země, přestala také mana; teď už Izraelci manu neměli, ale toho roku jedli z úrody kenaanské země.
#5:13 Když byl Jozue u Jericha, rozhlédl se, a hle, naproti němu stojí muž a má v ruce tasený meč. Jozue šel k němu a zeptal se ho: „Patříš k nám nebo k našim protivníkům?“
#5:14 Odvětil: „Nikoli. Jsem velitel Hospodinova zástupu, právě jsem přišel.“ I padl Jozue tváří k zemi, klaněl se a otázal se: „Jaký rozkaz má můj pán pro svého služebníka?“
#5:15 Velitel Hospodinova zástupu Jozuovi odpověděl: „Zuj si z nohou opánky, neboť místo, na němž stojíš, je svaté.“ A Jozue tak učinil. 
#6:1 Jericho se před Izraelci důkladně uzavřelo; nikdo nevycházel ani nevcházel.
#6:2 Ale Hospodin řekl Jozuovi: „Hleď, vydal jsem ti do rukou Jericho i jeho krále s udatnými bohatýry.
#6:3 Vy všichni bojovníci obejdete město vždy jednou kolem. To budeš dělat po šest dní.
#6:4 Sedm kněží ponese před schránou sedm polnic z beraních rohů. Sedmého dne obejdete město sedmkrát a kněží zatroubí na polnice.
#6:5 Až zazní táhlý tón z beraního rohu, jakmile uslyšíte zvuk polnice, vyrazí všechen lid mohutný válečný pokřik. Hradby města se zhroutí a lid vstoupí do města, každý tam, kde právě bude“.
#6:6 Jozue, syn Núnův, tedy povolal kněze a nařídil jim: „Přineste schránu smlouvy a sedm kněží ať nese před Hospodinovou schránou sedm polnic z beraních rohů.“
#6:7 Lidu řekl: „Pojďte a obcházejte město a ozbrojenci ať se ubírají před Hospodinovou schránou.“
#6:8 Stalo se, jak Jozue lidu nařídil: Sedm kněží nesoucích sedm polnic z beraních rohů se ubíralo před Hospodinem a troubilo na polnice a schrána Hospodinovy smlouvy šla za nimi.
#6:9 Ozbrojenci šli před kněžími troubícími na polnice a shromážděný zástup šel za schránou, šli a troubili na polnice.
#6:10 Jozue lidu dále přikázal: „Zdržíte se válečného pokřiku, ani nehlesnete, ani slovo nevyjde z vašich úst, až do dne, kdy vám nařídím: Strhněte pokřik. Pak strhnete pokřik.“
#6:11 Na jeho pokyn obešla Hospodinova schrána město jednou kolem. Pak šli do tábora a v táboře přenocovali.
#6:12 Za časného jitra přinesli Jozue a kněží Hospodinovu schránu.
#6:13 Sedm kněží neslo před Hospodinovou schránou sedm polnic z beraních rohů a za pochodu troubilo na polnice. Ozbrojenci šli před nimi a shromážděný zástup šel za Hospodinovou schránou; šli a troubili na polnice.
#6:14 I druhého dne obešli město jednou a vrátili se do tábora. To dělali po šest dní.
#6:15 Ale sedmého dne za časného jitra, hned jak vzešla jitřenka, obešli město týmž způsobem sedmkrát; jen onoho dne obcházeli město sedmkrát.
#6:16 Když obcházeli po sedmé, zatroubili kněží na polnice a Jozue lidu nařídil: „Strhněte válečný pokřik, neboť vám Hospodin město vydal.
#6:17 Město se vším všudy je klaté před Hospodinem. Naživu zůstane jen nevěstka Rachab se všemi, kteří jsou u ní v domě, neboť skryla posly, které jsme vyslali.
#6:18 Jenom se mějte na pozoru před tím, co je klaté, abyste nebyli vyhubeni jako klatí, kdybyste vzali něco klatého. Přivolali byste na izraelský tábor klatbu a uvrhli jej do zkázy.
#6:19 Všechno stříbro a zlato i bronzové a železné předměty budou zasvěceny Hospodinu, dá se to na Hospodinův poklad.“
#6:20 Když kněží zatroubili na polnice, lid strhl válečný pokřik. Jakmile lid zaslechl zvuk polnice, strhl mohutný pokřik. Hradby se zhroutily a lid vstoupil do města, každý tam, kde právě byl. Tak dobyli město.
#6:21 Všechno, co bylo v městě, vyhubili ostřím meče jako klaté, muže i ženy, mladíky i starce, též skot a brav i osly.
#6:22 Ale oběma mužům, kteří jako zvědové prošli zemí, Jozue řekl: „Vstupte do domu té ženy nevěstky a vyveďte tu ženu odtud spolu se vším, co jí náleží, jak jste jí přísahali.“
#6:23 Ti mládenci, zvědové, tedy vešli dovnitř a vyvedli Rachabu i jejího otce a matku i bratry se vším, co jí náleželo; vyvedli všechno její příbuzenstvo a poskytli jim místo za izraelským táborem.
#6:24 Město se vším všudy vypálili. Jen stříbro a zlato i bronzové a železné předměty dali na poklad Hospodinova domu.
#6:25 Ale nevěstku Rachabu, dům jejího otce a všechno, co jí náleželo, nechal Jozue naživu. Tak zůstala uprostřed Izraele až dodnes, poněvadž skryla posly, které poslal Jozue na výzvědy do Jericha.
#6:26 V onen čas se Jozue zapřisáhl: „Proklet buď před Hospodinem muž, který znovu vybuduje toto město Jericho. Na svém prvorozeném položí jeho základy, na svém nejmladším postaví jeho brány.“
#6:27 Hospodin byl s Jozuem a pověst o něm se roznesla po celé zemi. 
#7:1 Izraelci se při provádění klatby dopustili zpronevěry. Akán, syn Karmího, syna Zabdího, syna Zerachova z pokolení Judova, vzal něco z věcí propadlých klatbě. Proto vzplanul Hospodin proti Izraelcům hněvem.
#7:2 Jozue vyslal muže z Jericha do Aje, který byl u Bét-ávenu, na východ od Bét-elu, a řekl jim: „Jděte do té země na výzvědy!“ Muži tedy šli na výzvědy do Aje.
#7:3 Když se vrátili k Jozuovi, hlásili mu: „Není třeba, aby táhl všechen lid. Ať vytáhnou jen dva nebo tři tisíce mužů a přepadnou Aj. Nezatěžuj tím všechen lid, vždyť jich je málo.“
#7:4 I vydalo se tam z lidu na tři tisíce mužů, ale museli před ajskými muži utéci.
#7:5 Ajští mužové z nich pobili asi třicet šest mužů. Pronásledovali je před branou až k lomům a na stráni je pobíjeli. Lid ztratil odvahu, rozplynula se jako voda.
#7:6 Tu roztrhl Jozue své roucho, padl před Hospodinovou schránou tváří k zemi, ležel tak až do večera spolu s izraelskými staršími a házeli si prach na hlavu.
#7:7 Jozue lkal: „Ach, Panovníku Hospodine, proč jsi vlastně převedl tento lid přes Jordán? Chceš nás vydat do rukou Emorejců, aby nás zahubili? Kéž bychom byli raději zůstali na druhé straně Jordánu!
#7:8 Dovol prosím, Panovníku, co mám říci, když se Izrael dal na útěk před svými nepřáteli?
#7:9 Až to uslyší Kenaanci a ostatní obyvatelé země, obklíčí nás a vyhladí naše jméno ze země. Co pak učiníš pro své veliké jméno?“
#7:10 Hospodin řekl Jozuovi: „Vstaň! Co je s tebou, že jsi padl na tvář?
#7:11 Izrael zhřešil. Přestoupili mou smlouvu, kterou jsem jim vydal. Vzali z věcí propadlých klatbě, kradli, zatajili to a uložili mezi své věci.
#7:12 Proto nemohou Izraelci před svými nepřáteli obstát a obracejí se před svými nepřáteli na útěk, protože sami propadli klatbě. Nebudu už s vámi, jestliže ze svého středu nevymýtíte to, co propadlo klatbě.
#7:13 Vzhůru, posvěť lid a vyzvi jej: ‚Posvěťte se na zítřek! Toto praví Hospodin, Bůh Izraele: Uprostřed tebe je něco, co propadlo klatbě, Izraeli! Nemůžeš obstát před svými nepřáteli, dokud to, co propadlo klatbě, neodstraníte ze svého středu.
#7:14 Ráno budete přistupovat po kmenech. Kmen, který Hospodin označí, bude přistupovat po čeledích. Čeleď, kterou Hospodin označí, bude přistupovat po domech. Dům, který Hospodin označí, bude přistupovat po mužích.
#7:15 Ten, kdo bude označen, že si přisvojil věc propadlou klatbě, bude spálen se vším, co mu patří, protože přestoupil Hospodinovu smlouvu a dopustil se v Izraeli hanebnosti.‘“
#7:16 Za časného jitra rozkázal Jozue Izraeli přistupovat po kmenech. Losem byl označen kmen Judův.
#7:17 Rozkázal, aby přistupovala judská čeleď. Označena byla čeleď Zerachejců. Rozkázal, aby přistupovala zerašská čeleď po mužích. Označen byl Zabdí.
#7:18 Pak rozkázal, aby přistupoval jeho dům po mužích. Označen byl Akán, syn Karmího, syna Zabdího, syna Zerachova z pokolení Judova.
#7:19 Tu vyzval Jozue Akána: „Synu, přiznej slávu Hospodinu, Bohu Izraele, a vzdej mu chválu. Doznej se mi, co jsi učinil; nic přede mnou nezapírej!“
#7:20 Akán odpověděl Jozuovi: „Ano, zhřešil jsem proti Hospodinu, Bohu Izraele; učinil jsem toto:
#7:21 Viděl jsem mezi kořistí jeden pěkný šineárský plášť, dvě stě šekelů stříbra a jeden zlatý jazyk o váze padesáti šekelů. Vzplanul jsem žádostí a vzal jsem si to. Je to ukryto v zemi uvnitř mého stanu a stříbro je vespod.“
#7:22 Jozue poslal posly, ti běželi do stanu, a hle, bylo to ukryto v jeho stanu a stříbro bylo vespod.
#7:23 Vzali ty věci ze stanu, přinesli k Jozuovi a všem Izraelcům a pohodili je před Hospodinem.
#7:24 Za účasti všeho Izraele vzal Jozue Akána, syna Zerachova, i stříbro, plášť a zlatý jazyk, i jeho syny a dcery, býky a osly, jeho brav i stan a všechno, co mu patřilo, a ubírali se s tím vzhůru do doliny Akóru.
#7:25 Jozue řekl: „Zkázu, kterou jsi uvalil na nás, nechť uvalí Hospodin v tento den na tebe.“ Všechen Izrael jej kamenoval; spálili je a zaházeli je kamením.
#7:26 Navršili nad ním velkou hromadu kamení; je tam až dodnes. A Hospodin upustil od svého planoucího hněvu. Proto se to místo jmenuje Emek Akór (to je Dolina zkázy) až dodnes. 
#8:1 Potom řekl Hospodin Jozuovi: „Neboj se a neděs. Vezmi s sebou všechen bojeschopný lid a vytáhni proti Aji. Hleď, vydal jsem ti do rukou ajského krále i jeho lid, jeho město i jeho zemi.
#8:2 Naložíš s Ajem a jeho králem, jako jsi naložil s Jerichem a jeho králem. Avšak kořist a dobytek si necháte jako lup. Postav proti němu za městem zálohy.“
#8:3 Jozue a všechen bojeschopný lid vytáhli proti Aji. Jozue vybral třicet tisíc mužů, udatných bohatýrů, a v noci je rozmístil.
#8:4 Přikázal jim: „Hleďte, za městem se položte do zálohy proti městu. Příliš se od města nevzdalujte, buďte všichni připraveni.
#8:5 Já a všechen lid, který bude se mnou, se k městu přiblížíme. Jakmile oni vytrhnou proti nám jako předtím, dáme se před nimi na útěk.
#8:6 Ať za námi vytrhnou; odlákáme je od města, neboť si řeknou: ‚Utíkají před námi jako předtím.‘ Proto se dáme před nimi na útěk.
#8:7 Pak vyrazíte ze zálohy a město obsadíte; Hospodin, váš Bůh, vám je vydal do rukou.
#8:8 Až se města zmocníte, zapálíte je. Tak učiníte podle slova Hospodinova. Hleďte, to jsem vám přikázal.“
#8:9 Jozue je tedy rozmístil a oni se stáhli do zálohy a usadili se mezi Bét-elem a Ajem, západně od Aje. Jozue pak strávil tuto noc uprostřed lidu.
#8:10 Za časného jitra vykonal Jozue přehlídku lidu. Pak táhl s izraelskými staršími před lidem na Aj.
#8:11 S ním vytáhl všechen bojeschopný lid. Postupovali a došli až naproti městu. Utábořili se na sever od Aje. Mezi nimi a Ajem bylo údolí.
#8:12 Jozue vybral asi pět tisíc mužů a postavil je do zálohy mezi Bét-elem a Ajem, západně od města.
#8:13 Lid, celý tábor, zaujal postavení severně od města a zadní voj západně od města. Jozue odešel té noci doprostřed doliny.
#8:14 Když to zpozoroval ajský král, on i všechen jeho lid, mužové města, za časného jitra rychle vytrhli do boje proti Izraeli v určený čas na okraj pustiny. Nevěděl však, že jsou proti němu za městem zálohy.
#8:15 Když se Jozue a všechen Izrael měli s nimi střetnout, dali se na útěk směrem k poušti.
#8:16 Tu byl křikem přivolán všechen lid, který byl v městě, aby je pronásledoval. Pronásledovali Jozua a dali se od města odlákat.
#8:17 V Aji, totiž v Bét-elu, nezůstal nikdo, kdo by nevytrhl za Izraelem. Nechali město otevřené a pronásledovali Izraele.
#8:18 I řekl Hospodin Jozuovi: „Pozvedni oštěp, který držíš v ruce, směrem k Aji, neboť jsem ti jej vydal do rukou.“ Jozue pozvedl oštěp, který držel v ruce, směrem k městu.
#8:19 Jakmile pozvedl ruku, zálohy se ihned vyřítily ze svého postavení, přihnaly se, vpadly do města, dobyly je a ihned město zapálily.
#8:20 Ajští muži se ohlédli a spatřili, jak z města vystupuje k nebi kouř. Nemohli však utéci sem ani tam, neboť i lid, který utíkal do pouště, se obrátil proti pronásledovatelům.
#8:21 Když totiž Jozue a všechen Izrael viděli, že záloha dobyla město a že z města vystupuje kouř, obrátili se a pobíjeli ajské muže.
#8:22 A ti z města jim vytrhli vstříc, takže Ajští byli Izraelem sevřeni z jedné i z druhé strany. Izraelci je pobíjeli, nenechali nikoho vyváznout a uniknout.
#8:23 Ale ajského krále chytili živého a přivedli ho k Jozuovi.
#8:24 Izrael pobil všechny obyvatele Aje na poli, v poušti, kam je pronásledovali. Všichni do jednoho padli ostřím meče. Pak se celý Izrael vrátil do Aje a vybil jej ostřím meče.
#8:25 Všech mužů i žen, kteří toho dne padli, bylo dvanáct tisíc; padli všichni ajští muži.
#8:26 Ale Jozue nespustil ruku s pozvednutým oštěpem dříve, dokud nebyli všichni obyvatelé Aje vyhubeni jako klatí.
#8:27 Jen dobytek a kořist z města si Izrael nechal jako lup podle slova Hospodinova, které přikázal Jozuovi.
#8:28 Jozue vypálil Aj a učinil jej navěky pahorkem sutin, místem zpustošeným, jak je tomu až dodnes.
#8:29 Ajského krále dal pověsit na kůl, kde byl až do večera. Když slunce zapadalo, rozkázal Jozue, aby sňali jeho mrtvé tělo z kůlu. Pohodili je u vchodu do městské brány a navršili nad ním velikou hromadu kamení, která tam je až dodnes.
#8:30 Tehdy zbudoval Jozue oltář Hospodinu, Bohu Izraele, na hoře Ébalu,
#8:31 jak Izraelcům přikázal Mojžíš, služebník Hospodinův, tak jak je napsáno v Knize Mojžíšova zákona: oltář z neotesaných kamenů, neopracovaných železem. Na něm přinesli Hospodinu oběti zápalné a obětovali oběti pokojné.
#8:32 Tam napsal v přítomnosti Izraelců na kamenech opis Mojžíšova zákona.
#8:33 Všechen Izrael se svými staršími, správci a soudci stál po obou stranách schrány proti lévijským kněžím, kteří nosili schránu Hospodinovy smlouvy, i hosté a domorodci, polovina směrem k hoře Gerizímu a druhá polovina směrem k hoře Ébalu, jak kdysi přikázal Mojžíš, služebník Hospodinův, aby dávali požehnání izraelskému lidu.
#8:34 Potom předčítal všechna slova Zákona, požehnání i zlořečení, přesně jak to je zapsáno v Knize Zákona.
#8:35 Nebylo jediného slova přikázaného Mojžíšem, které by Jozue nepřečetl před celým shromážděním Izraele, včetně žen a dětí i těch, kdo žili mezi nimi jako hosté. 
#9:1 Když se o tom doslechli všichni králové, kteří byli z této strany Jordánu, v pohoří i v Přímořské nížině a na celém pobřeží Velkého moře směrem k Libanónu, totiž král chetejský, emorejský, kenaanský, perizejský, chivejský a jebúsejský,
#9:2 shromáždili se, aby jednomyslně bojovali proti Jozuovi a Izraeli.
#9:3 I obyvatelé Gibeónu se doslechli, jak Jozue naložil s Jerichem a Ajem.
#9:4 Vymyslili si tedy také lest, že půjdou a budou se vydávat za poselstvo. Vložili na osly vetché pytle a zpuchřelé, popraskané a svazované vinné měchy,
#9:5 na nohou měli vetché spravované střevíce a na sobě odřené pláště. I všechen chléb, jímž se zásobili, byl vyschlý a rozdrobený.
#9:6 Tak přišli k Jozuovi do tábora v Gilgálu a řekli jemu i mužům izraelským: „Přišli jsme z daleké země. Uzavřete s námi smlouvu!“
#9:7 Ale izraelští muži Chivejcům odpověděli: „Možná, že sídlíte mezi námi; jakpak bychom mohli s vámi uzavřít smlouvu!“
#9:8 Řekli nato Jozuovi: „Jsme tvoji otroci.“ Jozue se jich otázal: „Kdo jste a odkud přicházíte?“
#9:9 Odpověděli mu: „Z velmi daleké země přišli tvoji otroci za jménem Hospodina, tvého Boha. Slyšeli jsme o něm zprávy, co všechno učinil v Egyptě
#9:10 a jak naložil s dvěma emorejskými králi v Zajordání, s chešbónským králem Síchonem a s bášanským králem Ógem z Aštarótu.
#9:11 Naši stařešinové a všichni obyvatelé naší země nás vyzvali: ‚Vezměte si s sebou zásoby na cestu, jděte jim vstříc a řekněte jim: Jsme vaši otroci. Uzavřete s námi smlouvu.‘
#9:12 Toto je náš chléb. Byl ještě horký, když jsme se jím zásobili při odchodu z našich domů, než jsme se vydali k vám. A teď se podívejte: je vyschlý a rozdrobený.
#9:13 A tyto vinné měchy byly nové, když jsme je plnili. Podívejte se, jak popraskaly. A toto jsou naše pláště a střevíce; zvětšely, protože cesta byla velmi dlouhá.“
#9:14 Izraelští muži si tedy vzali z jejich zásob, aniž se ptali na Hospodinův výrok.
#9:15 Jozue sjednal mír a uzavřel s nimi smlouvu, že je nechá naživu. Předáci izraelské pospolitosti jim to potvrdili přísahou.
#9:16 Když uplynuly tři dny po tom, co s nimi uzavřeli smlouvu, doslechli se, že jsou z jejich blízkosti a že sídlí mezi nimi.
#9:17 Izraelci totiž táhli dál a třetího dne přišli do jejich měst. Byla to města Gibeón, Kefíra, Beerót a Kirjat-jearím.
#9:18 Ale Izraelci je nepobili, poněvadž se jim předáci izraelské pospolitosti zavázali přísahou při Hospodinu, Bohu Izraele. Proto celá pospolitost proti předákům reptala
#9:19 a všichni předáci vysvětlovali celé pospolitosti: „My jsme jim přísahali při Hospodinu, Bohu Izraele, takže se jich nyní nemůžeme dotknout.
#9:20 Musíme je nechat naživu, aby na nás nedolehlo Hospodinovo rozlícení pro přísahu, kterou jsme se jim zavázali.“
#9:21 O nich pak předáci řekli: „Ať zůstanou naživu, ale budou pro celou pospolitost sekat dříví a čerpat vodu.“ Tak o nich mluvili předáci.
#9:22 Potom si je zavolal Jozue a promluvil k nim: „Proč jste nás obelstili? Tvrdili jste: ‚Jsme od vás velmi daleko‘, a zatím sídlíte mezi námi!
#9:23 Buďte za to prokleti. Nepřestanete být otroky. Budete sekat dříví a čerpat vodu pro dům mého Boha.“
#9:24 Odpověděli Jozuovi: „Tvým otrokům bylo oznámeno, co Hospodin, tvůj Bůh, přikázal svému služebníku Mojžíšovi, že vám dá celou zemi a že před vámi vyhladí všechny její obyvatele. Velmi jsme se báli, že nás připravíte o život. Proto jsme to udělali.
#9:25 Teď jsme v tvých rukou. Nalož s námi jak uznáš za dobré a správné.“
#9:26 I naložil s nimi tak, že je vyprostil z rukou Izraelců, takže je nezabili.
#9:27 Ale onoho dne jim Jozue uložil, aby sekali dříví a čerpali vodu pro pospolitost a pro Hospodinův oltář, a to až dodnes, na místě, které vyvolí. 
#10:1 Adonísedek, král jeruzalémský, se doslechl, že Jozue dobyl Aj a zničil jej jako klatý - jako naložil s Jerichem a s jeho králem, tak naložil s Ajem a s jeho králem - a že obyvatelé Gibeónu uzavřeli s Izraelem mír a jsou nyní mezi nimi.
#10:2 Tu se začali velmi bát, neboť Gibeón byl veliké město jako jedno z měst královských; byl větší než Aj a všichni jeho muži byli bohatýři.
#10:3 Proto poslal Adonísedek, král jeruzalémský, vzkaz Hóhamovi, králi chebrónskému, Pirámovi, králi jarmútskému, Jafíovi, králi lakíšskému, a Debírovi, králi eglónskému:
#10:4 „Přitáhněte mi na pomoc, ať porazíme Gibeón, neboť uzavřel mír s Jozuem a Izraelci!“
#10:5 Pět emorejských králů se tedy vydalo společně na pochod, králové jeruzalémský, chebrónský, jarmútský, lakíšský a eglónský, každý se svými šiky. Oblehli Gibeón a zahájili proti němu boj.
#10:6 Gibeónští muži poslali k Jozuovi do tábora v Gilgálu s prosbou: „Nenechávej své otroky bez pomoci! Pospěš rychle k nám a zachraň nás. Pomoz nám, neboť se proti nám srotili všichni emorejští králové, kteří sídlí v pohoří.“
#10:7 I vytáhl Jozue z Gilgálu a s ním všechen bojeschopný lid, samí udatní bohatýři.
#10:8 I řekl Hospodin Jozuovi: „Neboj se jich, neboť jsem ti je vydal do rukou. Žádný z nich před tebou neobstojí.“
#10:9 Jozue táhl z Gilgálu po celou noc a náhle je přepadl.
#10:10 A Hospodin je před Izraelem uvedl ve zmatek. Připravil jim u Gibeónu zdrcující porážku. Pronásledoval je směrem k bétchorónskému svahu a pobíjel je až do Azeky a Makedy.
#10:11 Když před Izraelem utíkali a byli na bétchorónské stráni, vrhal na ně Hospodin z nebe balvany až do Azeky; tak umírali. Těch, kteří zemřeli po kamenném krupobití, bylo více než těch, které Izraelci pobili mečem.
#10:12 Tehdy mluvil Jozue k Hospodinu, v den, kdy Hospodin vydal Izraelcům Emorejce. Volal před očima Izraele: „Zmlkni, slunce, v Gibeónu, měsíci, v dolině Ajalónu.“
#10:13 A slunce zmlklo a měsíc stál, dokud lid nevykonal pomstu nad svými nepřáteli. To je zapsáno, jak známo, v Knize Přímého. Slunce stálo v polovině nebes a nepospíchalo k západu po celý den.
#10:14 Nikdy předtím ani potom nebylo dne, jako byl onen, aby Hospodin tak vyslyšel něčí hlas, neboť Hospodin bojoval za Izraele.
#10:15 Pak se Jozue spolu s celým Izraelem vrátili do Gilgálu do tábora.
#10:16 Těch pět králů se však dalo na útěk a skrylo se v jeskyni u Makedy.
#10:17 Jozuovi bylo oznámeno: „Našli jsme pět králů, ukrytých v jeskyni u Makedy.“
#10:18 Jozue řekl: „Přivalte balvany k otvoru jeskyně a postavte k ní několik mužů, aby krále střežili.
#10:19 Vy však neustávejte a pronásledujte své nepřátele, přepadněte jejich zadní voj a nedovolte jim vejít do jejich měst, neboť Hospodin, váš Bůh, vám je vydal do rukou.“
#10:20 Jozue s Izraelci dovršil převelikou, zdrcující porážku nepřátel. Rozdrtil je. Ti, kteří z nich přece jen vyvázli, stáhli se do opevněných měst.
#10:21 Všechen lid se vrátil pokojně k Jozuovi do tábora u Makedy. Nikdo se neodvážil proti Izraelcům hlesnout.
#10:22 Jozue pak rozkázal: „Uvolněte otvor jeskyně a přiveďte ke mně z jeskyně těch pět králů!“
#10:23 Učinili tak a přivedli k němu z jeskyně těch pět králů: krále jeruzalémského, chebrónského, jarmútského, lakíšského a eglónského.
#10:24 Když k němu tyto krále přivedli, svolal Jozue všechny izraelské muže a rozkázal velitelům bojovníků, kteří s ním táhli: „Přistupte a stoupněte svýma nohama na šíje těmto králům!“ Přistoupili a stoupli jim svýma nohama na šíje.
#10:25 Pak jim Jozue řekl: „Nebojte se a neděste, buďte rozhodní a udatní, neboť právě tak naloží Hospodin se všemi vašimi nepřáteli, proti nimž bojujete.“
#10:26 Jozue je potom dal pobít a usmrtit a pověsit na pěti kůlech. Na kůlech zůstali viset až do večera.
#10:27 Při západu slunce vydal Jozue rozkaz, aby je sňali z kůlů a vhodili do jeskyně, kde se skrývali. K otvoru jeskyně navršili balvany, které tam jsou až dodnes.
#10:28 Onoho dne dobyl Jozue Makedu a vybil ji ostřím meče i s jejím králem. Vyhubil obyvatele jako klaté i vše živé v ní; nenechal nikoho vyváznout. Naložil s králem Makedy, jako naložil s králem jerišským.
#10:29 Potom táhl Jozue spolu s celým Izraelem z Makedy do Libny a bojoval s Libnou.
#10:30 I ji a jejího krále vydal Hospodin do rukou Izraele. Vybil ji ostřím meče i vše živé v ní; nenechal nikoho vyváznout. Naložil s jejím králem, jako naložil s králem jerišským.
#10:31 Potom táhl Jozue spolu s celým Izraelem z Libny do Lakíše, oblehl jej a bojoval proti němu.
#10:32 I vydal Hospodin Lakíš do rukou Izraele. Druhého dne jej dobyl, vybil jej ostřím meče i vše živé v něm, stejně jako naložil s Libnou.
#10:33 Tehdy přitáhl Horám, král gezerský, Lakíši na pomoc. Ale Jozue pobil jej i jeho lid; nenechal nikoho vyváznout.
#10:34 Potom táhl Jozue spolu s celým Izraelem z Lakíše do Eglónu. Oblehli jej a bojovali proti němu.
#10:35 Dobyli jej onoho dne a vybili jej ostřím meče; vše živé v něm onoho dne vyhubil jako klaté, stejně jako naložil s Lakíšem.
#10:36 Potom přitáhl Jozue spolu s celým Izraelem z Eglónu do Chebrónu a bojovali proti němu.
#10:37 Dobyli jej a vybili ostřím meče, jeho krále, všechna jeho města i vše živé v něm; nenechal nikoho vyváznout, stejně jako naložil s Eglónem. Město a vše živé v něm vyhubil jako klaté.
#10:38 Potom se Jozue obrátil spolu s celým Izraelem do Debíru a bojoval proti němu.
#10:39 Dobyl jej a vybil jej ostřím meče, jeho krále i všechna jeho města, a vyhubili jako klaté vše živé v něm; nenechal nikoho vyváznout. Jako naložil s Chebrónem, tak naložil s Debírem a jeho králem, stejně jako naložil s Libnou a jejím králem.
#10:40 Tak vybil Jozue celou zemi, pohoří i Negeb, Přímořskou nížinu i srázy, a všechny jejich krále. Nikoho nenechal vyváznout, vše, co dýchalo, vyhubil jako klaté, jak přikázal Hospodin, Bůh Izraele.
#10:41 Jozue je vybil od Kádeš-barneje až ke Gáze, i celou zemi Gošen až po Gibeón.
#10:42 Tak se zmocnil Jozue jedním rázem všech těchto králů a jejich zemí, neboť Hospodin, Bůh Izraele, bojoval za Izraele.
#10:43 Pak se Jozue spolu s celým Izraelem navrátil do Gilgálu do tábora. 
#11:1 Když se o tom doslechl Jabín, král chasórský, poslal k Jóbabovi, králi madónskému, ke králi šimrónskému, ke králi akšáfskému
#11:2 a k jiným králům na severu v pohoří, v Jordánském úvalu jižně od Kinarótu, v Přímořské nížině i v Dórské pahorkatině na západě.
#11:3 Na východě i na západě byli Kenaanci, v pohoří Emorejci, Chetejci, Perizejci a Jebúsejci, pod Chermónem v zemi Mispě Chivejci.
#11:4 Ti všichni vytáhli se svými šiky, lid početný jako písek na mořském břehu, a velmi mnoho koní a vozů.
#11:5 Všichni tito králové se spojili, přitáhli a společně se utábořili u vod Merómu, aby bojovali s Izraelem.
#11:6 Tu řekl Hospodin Jozuovi: „Neboj se jich, neboť já je zítra v tuto dobu vydám Izraeli všechny skolené; jejich koně ochromíš a jejich vozy spálíš ohněm.“
#11:7 A tak Jozue a všechen bojeschopný lid na ně nečekaně přitáhli k vodám Merómu a přepadli je.
#11:8 A Hospodin je vydal Izraeli do rukou. Pobíjeli je a pronásledovali až k velkému Sidónu, k Misrefót-majimu a na východě k planině Mispě; pobíjeli je, že z nich nezůstal nikdo, kdo by vyvázl.
#11:9 Jozue pak s nimi naložil, jak mu řekl Hospodin: jejich koně ochromil a jejich vozy spálil ohněm.
#11:10 V téže době se Jozue obrátil a dobyl Chasór a jeho krále zabil mečem; Chasór byl totiž dříve hlavou všech těchto království.
#11:11 Všechno živé v něm vybili ostřím meče a vyhubili jako klaté. Nic, co dýchalo, nezůstalo naživu. A Chasór vypálil Jozue ohněm.
#11:12 Dobyl i všechna města těchto králů; všechny jejich krále Jozue zajal a pobil ostřím meče. Vyhubil je jako klaté, jak přikázal služebník Hospodinův Mojžíš.
#11:13 Avšak z měst, postavených na svých pahorcích, nevypálil Izrael žádné. Jozue vypálil jenom Chasór.
#11:14 Všechnu kořist z těch měst a dobytek si Izraelci nechali jako lup. Zato každého člověka zabili ostřím meče, až všechny vyhladili. Neponechali nic, co dýchalo.
#11:15 Jak přikázal Hospodin svému služebníku Mojžíšovi, tak Mojžíš přikázal Jozuovi a tak Jozue učinil. Nevynechal nic ze všeho toho, co Hospodin Mojžíšovi přikázal.
#11:16 Jozue zabral celou tu zemi: pohoří i celý Negeb, celou zemi Gošen, Přímořskou nížinu i Jordánskou pustinu, izraelské pohoří i přilehlou Přímořskou nížinu,
#11:17 od Lysé hory, zvedající se k Seíru, až k Baal-gádu na Libanónské planině pod Chermónským pohořím. Všechny jejich krále zajal, pobil a usmrtil.
#11:18 Se všemi těmito králi vedl Jozue po dlouhá léta boje.
#11:19 Nebylo města, jež by s Izraelci uzavřelo mír, kromě Chivejců, sídlících v Gibeónu. Všechno zabrali bojem.
#11:20 To sám Hospodin zatvrdil jejich srdce, že se dali do boje s Izraelem, aby je vyhubil jako klaté a nesmiloval se nad nimi, nýbrž aby je zahladil, jak přikázal Hospodin Mojžíšovi.
#11:21 Toho času přitáhl Jozue a vyhladil Anákovce z pohoří, z Chebrónu, z Debíru, z Anábu a z celého judského pohoří i z celého pohoří izraelského; Jozue vyhubil je i jejich města jako klaté.
#11:22 V zemi Izraelců nezanechal žádné Anákovce; nějací zůstali jen v Gáze, v Gatu a v Ašdódu.
#11:23 Přesně jak Hospodin řekl Mojžíšovi, tak zabral Jozue celou zemi a dal ji Izraeli za dědictví, podle toho, jak byli rozděleni na kmeny. A země žila v míru bez válek. 
#12:1 Toto pak jsou králové té země, které Izraelci pobili a jejichž zemi obsadili za Jordánem k východu slunce, od potoka Arnónu až k Chermónskému pohoří a celou Jordánskou pustinu k východu:
#12:2 Síchon, král emorejský, který sídlil v Chešbónu a vládl od Aróeru na břehu potoka Arnónu, totiž od středu Potočního údolí přes polovici Gileádu až k potoku Jaboku, který tvoří hranici Amónců,
#12:3 od Jordánské pustiny východně od Kinerótského moře (to je Genezaretského jezera) až k moři Pustému, to je k moři Solnému, na východ, směrem k Bét-ješimótu, a na jihu až pod sráz Pisgy.
#12:4 Dále pomezí Óga, krále bášanského, ze zbytku Refájců; ten sídlil v Aštarótu a v Edreí
#12:5 a vládl nad Chermónským pohořím, nad Salkou i nad celým Bášanem až k pomezí gešúrskému a maakatskému a nad půlkou Gileádu při pomezí Síchona, krále chešbónského.
#12:6 Mojžíš, služebník Hospodinův, a Izraelci je pobili a Mojžíš, služebník Hospodinův, dal toto území do vlastnictví Rúbenovcům, Gádovcům a polovině kmene Manasesova.
#12:7 A toto jsou králové té země, které pobil Jozue a Izraelci na západ od Jordánu, od Baal-gádu na planině Libanónské až k Lysé hoře zvedající se k Seíru. Jozue ji dal do vlastnictví izraelským kmenům podle toho, jak byly rozděleny,
#12:8 v pohoří i v Přímořské nížině, v Jordánské pustině i na srázech, v poušti i v Negebu, mezi Chetejci, Emorejci, Kenaanci, Perizejci, Chivejci a Jebúsejci:
#12:9 Král jerišský jeden, král ajský, při Bét-elu, jeden,
#12:10 král jeruzalémský jeden, král chebrónský jeden,
#12:11 král jarmútský jeden, král lakíšský jeden,
#12:12 král eglónský jeden, král gezerský jeden,
#12:13 král debírský jeden, král gederský jeden,
#12:14 král chormský jeden, král aradský jeden,
#12:15 král libenský jeden, král adulámský jeden,
#12:16 král makedský jeden, král bételský jeden,
#12:17 král tapúašský jeden, král cheferský jeden,
#12:18 král afecký jeden, král šáronský jeden,
#12:19 král mádónský jeden, král chasórský jeden,
#12:20 král šímrónsko-merónský jeden, král akšáfský jeden,
#12:21 král taanacký jeden, král megidský jeden,
#12:22 král kedešský jeden, král joknoámský, pod Karmelem, jeden,
#12:23 král dórský, z Dórské pahorkatiny jeden, král pronárodů z Gilgálu jeden,
#12:24 král tirský jeden; všech králů jedenatřicet. 
#13:1 Jozue byl stařec pokročilého věku. Hospodin mu řekl: „Ty jsi zestárl a sešel věkem, ale zbývá ještě velmi mnoho země, kterou je nutno obsadit.
#13:2 Toto je země, kterou zbývá přidělit: všechny oblasti pelištejské a celá oblast gešúrská,
#13:3 od ramene řeky při hranicích Egypta až na sever k pomezí Ekrónu, jež se počítá za kenaanské, pět pelištejských knížectví, Gázské, Ašdódské, Aškalónské, Gatské a Ekrónské, a oblast avijská
#13:4 na jihu; dále celá kenaanská země i Meára, patřící Sidóňanům, po Afek, až k pomezí Emorejců,
#13:5 země gebalská a celý Libanón na východ od Baal-gádu pod Chermónským pohořím až k cestě do Chamátu.
#13:6 Všechny obyvatele pohoří od Libanónu až k Misrefót-majimu, všechny Sidóňany, vyženu před Izraelci já sám. Jen přiděl zemi losem Izraeli za dědictví, jak jsem ti přikázal.
#13:7 Rozděl nyní tuto zemi do dědictví devíti kmenům a polovině kmene Manasesova.
#13:8 S druhou polovinou vzali svůj dědičný podíl Rúbenovci a Gádovci; Mojžíš jim jej dal v Zajordání na východě. Mojžíš, služebník Hospodinův, jim dal tento díl:
#13:9 od Aróeru na břehu potoka Arnónu a od města, které je uprostřed Potočního údolí, celou náhorní rovinu od Médeby k Dibónu,
#13:10 všechna města Síchona, krále emorejského, který kraloval v Chešbónu, až k pomezí Amónovců;
#13:11 Gileád a pomezí gešúrské a maakatské, celé pohoří Chermón a celý Bášan až do Salky,
#13:12 celé království Óga v Bášanu, který kraloval v Aštarótu a v Edreí. Ten zůstal ze zbytku Refájců, které Mojžíš porazil a podrobil si je.
#13:13 Ale Gešúrejce a Maakaťany si Izraelci nepodrobili. Proto Gešúr a Maakat sídlí uprostřed Izraele až dodnes.
#13:14 Jenom kmenu Léviho nedal dědičný podíl. Bude jíst z ohnivých obětí Hospodina, Boha Izraele; on bude jeho dědičným podílem, jak jej ujistil.
#13:15 Mojžíš tedy dal pokolení Rúbenovců pro jejich čeledi dědičný podíl.
#13:16 Patřilo jim území od Aróeru na břehu potoka Arnónu a města, které je uprostřed Potočního údolí, celá náhorní rovina až k Médebě,
#13:17 Chešbón a všechna jeho města, která leží na náhorní rovině, Dibón, Bamót-baal, Bét-baal-meón,
#13:18 Jahsa, Kedemót, Mefaat,
#13:19 Kirjatajim, Sibma a Seret-šachar na hoře v dolině,
#13:20 Bétpeór, srázy Pisgy a Bét-ješimót,
#13:21 všechna města na náhorní rovině, celé království Síchona, krále emorejského, který kraloval v Chešbónu. Toho Mojžíš porazil spolu s midjánskými předáky Evím, Rekemem, Súrem, Chúrem a Rebaem, vasaly Síchonovými, obyvateli té země.
#13:22 Též Bileáma, syna Beórova, věštce, popravili Izraelci mečem jako ostatní skolené.
#13:23 Hranicí Rúbenovců byl Jordán a přilehlé území. Tato města a jejich dvorce jsou dědičným podílem Rúbenovců pro jejich čeledi.
#13:24 Mojžíš dal také pokolení Gádovu, Gádovcům, pro jejich čeledi dědičný podíl.
#13:25 Patřilo jim toto území: Jaezer a všechna gileádská města, polovina země Amónovců až k Aróeru naproti Rabě
#13:26 a od Chešbónu až k Ramat-mispě a Betonímu a od Machanajimu až k pomezí Lidbiru,
#13:27 v dolině Bét-rám a Bét-nimra, Sukót a Safón, zbytek království Síchona, krále chešbónského, s Jordánem jako hranicí až k břehům moře Kineretského z východní strany Jordánu.
#13:28 Tato města a jejich dvorce jsou dědičným podílem Gádovců pro jejich čeledi.
#13:29 Mojžíš dal i polovině kmene Manasesova dědičný podíl; patřil zajordánské polovině pokolení Manasesovců pro jejich čeledi.
#13:30 Jejich území se prostíralo od Machanajimu: celý Bášan, celé království Óga, krále bášanského, se všemi vesnicemi Jaírovými v Bášanu, celkem šedesát měst,
#13:31 polovice Gileádu i Aštarót a Edreí, Ógova královská města v Bášanu. Připadly synům Makíra, syna Manasesova, totiž polovině Makírovců, pro jejich čeledi.
#13:32 To jsou dědičné podíly, které Mojžíš přidělil v Moábských pustinách naproti Jerichu za Jordánem k východu.
#13:33 Kmenu Léviho však Mojžíš dědičný podíl nedal. Hospodin, Bůh Izraele, bude jejich dědičným podílem, jak je ujistil. 
#14:1 Izraelci obdrželi v kenaanské zemi tyto dědičné podíly: přidělili jim je dědičně kněz Eleazar a Jozue, syn Núnův, a představitelé rodů izraelských pokolení.
#14:2 Dědičný podíl, a to pro devět a půl pokolení, jim byl přidělen losem, jak přikázal Hospodin skrze Mojžíše.
#14:3 Mojžíš dal totiž dědičný podíl dvěma a půl pokolením v Zajordání. Lévijcům ovšem mezi nimi dědičný podíl nedal.
#14:4 Zato Josefovci tvořili dvě pokolení, Manasesovo a Efrajimovo; a tak nedali lévijcům podíl v zemi, jen města k bydlení a pastviny pro jejich dobytek a stáda.
#14:5 Jak přikázal Hospodin Mojžíšovi, tak Izraelci učinili, když rozdělovali zemi.
#14:6 Judovci přistoupili v Gilgálu k Jozuovi a Káleb, syn Kenazejce Jefuna, mu řekl: „Ty znáš rozhodnutí, o němž mluvil Hospodin v Kádeš-barneji k Mojžíšovi, muži Božímu, ohledně mne a tebe.
#14:7 Bylo mi čtyřicet let, když mě Mojžíš, služebník Hospodinův, vyslal z Kádeš-barneje, abych jako zvěd prošel zemí, a já jsem mu podal zprávu podle nejlepšího svědomí.
#14:8 Moji bratří, kteří táhli vzhůru se mnou, zavinili, že lid ztratil odvahu. Ale já jsem se cele oddal Hospodinu, svému Bohu.
#14:9 Onoho dne se zavázal Mojžíš přísahou: ‚Ano, země, na niž stoupla tvá noha, bude navěky patřit tobě a tvým synům jako dědičný podíl, neboť ses cele oddal Hospodinu, mému Bohu.‘
#14:10 Nuže hle, Hospodin mě uchoval naživu, jak přislíbil. Čtyřicet pět let uplynulo od chvíle, co Hospodin takto promluvil k Mojžíšovi, když Izrael putoval pouští. A hle, je mi už osmdesát pět let.
#14:11 Ale ještě dnes jsem právě tak silný jako tenkrát, když mě Mojžíš vyslal. Moje síla je dnes stejná, jako byla tehdy, abych bojoval a vycházel a vcházel.
#14:12 Nuže, dej mi to pohoří, o němž mluvil onoho dne Hospodin. Slyšel jsi přece v onen den, že jsou tam Anákovci a velká opevněná města. Snad bude Hospodin se mnou a podaří se mi podrobit si je, jak mluvil Hospodin.“
#14:13 Jozue mu požehnal a přidělil Chebrón jako dědičný podíl Kálebovi, synu Jefunovu.
#14:14 Proto patří Chebrón až dodnes jako dědičný podíl Kálebovi, synu Kenazejce Jefuna, poněvadž se cele oddal Hospodinu, Bohu Izraele.
#14:15 Předtím se Chebrón jmenoval Kirjat-arba (to je Město Arbovo); Arba byl největší člověk mezi Anákovci. A země žila v míru bez válek. 
#15:1 Pokolení Judovců připadl pro jejich čeledi při pomezí Edómu a pouště Sinu v Negebu, úplně na jihu, tento los:
#15:2 Patřilo jim území Negebu od břehů Solného moře, od zálivu směřujícího k Negebu.
#15:3 Směrem od Negebu se táhne ke Svahu štírů, pokračuje k poušti Sinu a stoupá na jih od Kádeš-barneje, míří k Chesrónu, stoupá k Adáru a odbočuje ke Karce.
#15:4 Pokračuje k Asmónu, táhne se k Potoku egyptskému, takže území vybíhá k moři. To je vaše jižní hranice.
#15:5 Východní hranicí bude Solné moře až k ústí Jordánu. Na severu začíná hranice mořským zálivem při ústí Jordánu.
#15:6 Odtud stoupá hranice k Bét-chogle a pokračuje severně od Bét-araby. Pak stoupá ke kameni Bohana, syna Rúbenova.
#15:7 Hranice stoupá dále od doliny Akóru k Debíru a obrací se na sever ke Gilgálu naproti Adumímskému svahu jižně od potoka. Hranice pak pokračuje podél vody Slunečního pramene a vybíhá k prameni Rogelu.
#15:8 Hranice potom stoupá k Údolí syna Hinómova jižně od skalního hřebenu jebúsejského, což je Jeruzalém, a vystupuje na vrcholek hory, která na západě leží proti Údolí hinómskému a je na severním konci doliny Refájců.
#15:9 Od vrcholu hory zabočuje hranice k prameni vod Neftóachu a vede k městům pohoří Efrónu; poté zabočuje k Baale, to je Kirjat-jearímu.
#15:10 Zde hranice odbočuje od Baaly na západ k hoře Seíru a pokračuje od severu ke skalnímu hřebenu Lesní hory, to je Kesalónu, sestupuje od Bét-šemeše a pokračuje k Timně.
#15:11 Potom hranice vede ke skalnímu hřebenu Ekrónu na severu od města a zabočuje k Šikerónu, pokračuje k Baalské hoře a vede k Jabneelu; hranice vybíhá až k moři.
#15:12 Západní hranicí je Velké moře a jeho pobřeží. To je území Judovců pro jejich čeledi, vymezené hranicemi dokola.
#15:13 Kálebovi, synu Jefunovu, dal podíl uprostřed Judovců podle Hospodinova příkazu Jozuovi, totiž Kirjat-arbu, to je město Arby, otce Anákova, což je Chebrón.
#15:14 Odtud vyhnal Káleb tři Anákovce: Šešaje, Achímana a Talmaje, zplozence Anákovy.
#15:15 Odtud pak táhl vzhůru proti obyvatelům Debíru; Debír se předtím jmenoval Kírjat-sefer (to je Město knihy).
#15:16 Káleb tenkrát prohlásil: „Kdo porazí Kirjat-sefer a dobude jej, tomu dám za manželku svou dceru Aksu.“
#15:17 Dobyl jej Otníel, syn Kenazův, Kálebův bratr; dal mu tedy svou dceru Aksu za manželku.
#15:18 Když přišla, navedla jej, aby žádal jejího otce o pole. Sesedla z osla. Káleb se jí otázal: „Co si přeješ?“
#15:19 Odpověděla: „Obdař mě požehnáním! Když jsi mě provdal do jižního suchopáru, dej mi také vodní zřídla.“ Dal jí tedy zřídla, horní a dolní.
#15:20 To je dědičný podíl pokolení Judovců pro jejich čeledi.
#15:21 Nejvzdálenější města, která připadla pokolení Judovců, při édomském pomezí v Negebu, byla: Kabseel, Eder a Jagúr,
#15:22 Kína, Dimóna a Adeáda,
#15:23 Kedeš, Chasór a Jitnán,
#15:24 Zíf, Telem a Bealót,
#15:25 Chasór-chadata a Kerijót-chesrón, což je Chasór,
#15:26 Amám, Šema a Mólada,
#15:27 Chasar-gada, Chešmón a Bét-pelet,
#15:28 Chasar-šúal, Beer-šeba a Bizjótja,
#15:29 Baala, Ijím a Esem,
#15:30 Eltólad, Kesíl a Chorma,
#15:31 Siklag, Madmana a Sansana,
#15:32 Lebaót, Šilchím, Ajin a Rimón, celkem dvacet devět měst a jejich dvorce.
#15:33 V Přímořské nížině: Eštaól, Sorea a Ašna,
#15:34 Zanóach, Én-ganím, Tapúach a Énam,
#15:35 Jarmút, Adulám, Sóko a Azeka,
#15:36 Šaarajim, Adítajim, Gedera a Gederótajim, čtrnáct měst a jejich dvorce. -
#15:37 Senan, Chadaša a Migdal-gad,
#15:38 Dilean, Mispe a Jokteel,
#15:39 Lakíš, Boskat a Eglón,
#15:40 Kabón, Lachmas a Kitlíš,
#15:41 Gederót, Bét-dágon, Naama a Makeda, šestnáct měst a jejich dvorce. -
#15:42 Libna, Eter a Ašan,
#15:43 Jiftach, Ašna a Nesíb,
#15:44 Keíla, Akzíb a Maréša, devět měst a jejich dvorce. -
#15:45 Ekrón s vesnicemi a dvorci,
#15:46 od Ekrónu až k moři všechno, co leží stranou Ašdódu, a jejich dvorce,
#15:47 Ašdód s vesnicemi a dvorci, Gáza s vesnicemi a dvorci, až k Potoku egyptskému a Velkému moři s pobřežím.
#15:48 V pohoří: Šamír, Jatír a Sóko,
#15:49 Dana, Kirjat-sana, což je Debír,
#15:50 Anab, Eštemo a Aním,
#15:51 Gošen, Cholón a Gílo, jedenáct měst a jejich dvorce. -
#15:52 Arab, Dúma a Ešeán,
#15:53 Janím, Bét-tapúach a Afeka,
#15:54 Chumta, Kirjat-arba, což je Chebrón, a Síor, devět měst a jejich dvorce. -
#15:55 Maón, Karmel, Zíf a Júta,
#15:56 Jizreel, Jokdeám a Zanóach,
#15:57 Kajin, Gibea a Timna, deset měst a jejich dvorce. -
#15:58 Chalchúl, Bét-súr a Gedór,
#15:59 Maarat, Bét-anót a Eltekón, šest měst a jejich dvorce. -
#15:60 Kirjat-baal, což je Kirjat-jearím, a Raba, dvě města a jejich dvorce.
#15:61 Ve stepi: Bét-araba, Midín a Sekaka,
#15:62 Nibšán, Ír-melach a Én-gedí, šest měst a jejich dvorce. -
#15:63 Ale jebúsejské obyvatele Jeruzaléma nebyli Judovci s to si podrobit. Tak sídlí Jebúsejci s Judovci v Jeruzalémě až dodnes. 
#16:1 Josefovcům vyšel tento los: od Jordánu při Jerichu k Jerišským vodám na východ; hranice stoupá pouští od Jericha do hor k Bét-elu.
#16:2 Pak vede od Bét-elu k Lúzu a pokračuje k pomezí Arkejců do Atarótu.
#16:3 Sestupuje na západ k pomezí Jaflétejců až k pomezí Dolního Bét-chorónu a Gezeru a vybíhá k moři.
#16:4 Takový byl dědičný podíl Josefovců, Manasesa a Efrajima.
#16:5 Efrajimovcům připadlo pro jejich čeledi toto pomezí: Hranice jejich dědičného podílu jde na východ od Atarót-adaru až k Hornímu Bét-chorónu.
#16:6 Dál vede hranice k moři na sever od Mikmetatu. Pak odbočuje hranice na východ k Taanat-šílu a míjí je na východ k Janóachu.
#16:7 Od Janóachu sestupuje k Atarótu a k Naaře, dotýká se Jericha a vede k Jordánu.
#16:8 Od Tapúachu jde hranice na západ k potoku Káně a vybíhá k moři. To je dědičný podíl pokolení Efrajimovců pro jejich čeledi.
#16:9 K tomu města oddělená pro Efrajimovce uprostřed dědičného podílu Manasesovců, všechna města i s dvorci.
#16:10 Avšak nepodrobili si Kenaance, kteří sídlili v Gezeru. A tak sídlí Kenaanci uprostřed Efrajima až dodnes. Byli podrobeni otrockým pracím. 
#17:1 Tento los připadl pokolení Manasesovu; to byl Josefův prvorozený. Makírovi, Manasesovu prvorozenému, otci Gileádovu, za to, že byl válečník, připadly Gileád a Bášan.
#17:2 Také ostatním Manasesovcům připadly podíly pro jejich čeledi: Abíezerovcům, Chelekovcům, Asríelovcům, Šekemovcům, Cheferovcům a Šemídáovcům. To byli potomci Josefova syna Manasesa, mužské potomstvo podle svých čeledí.
#17:3 Avšak Selofchad, syn Cheferův, vnuk Gileádův, pravnuk Manasesova syna Makíra, neměl syny, nýbrž jen dcery. Jmenovaly se: Machla, Nóa, Chogla, Milka a Tirsa.
#17:4 Ty předstoupily před kněze Eleazara a před Jozua, syna Núnova, a před předáky se slovy: „Hospodin přikázal Mojžíšovi, aby nám dal dědičný podíl mezi našimi bratřími.“ Dali jim tedy podle Hospodinova příkazu dědičný podíl mezi bratry jejich otce.
#17:5 Manasesovi bylo tudíž přiděleno deset oblastí, kromě území gileádského a bášanského, která jsou v Zajordání,
#17:6 neboť Manasesovy dcery obdržely mezi jeho potomky dědičný podíl. Země gileádská připadla ostatním Manasesovcům.
#17:7 Manasesova hranice jde od Ašeru k Michmetu, který leží naproti Šekemu, a vede na jih k obyvatelům Én-tapúachu.
#17:8 Území tapúašské připadlo Manasesovi, ale samotný Tapúach, při Manasesově pomezí, patřil Efrajimovcům.
#17:9 Hranice pak sestupuje k potoku Káně. Na jih od potoka mezi městy Manasesovými jsou města patřící Efrajimovi. Hranice Manasesova se táhne severně podél potoka a vybíhá k moři.
#17:10 Jižní část patří Efrajimovi, severní Manasesovi, a jeho hranici tvoří moře. Na severu se dotýkají Ašera a na východě Isachara.
#17:11 Manasesovi připadly v území Isacharově a Ašerově: Bét-šeán s vesnicemi, Jibleám s vesnicemi, obyvatelé Dóru s vesnicemi, obyvatelé Én-dóru s vesnicemi, obyvatelé Taanaku s vesnicemi, obyvatelé Megida s vesnicemi a tři návrší.
#17:12 Ale Manasesovci nebyli s to podrobit si tato města a to umožnilo Kenaancům zůstat v té zemi.
#17:13 Teprve když se Izraelci vzmohli, podrobili Kenaance nuceným pracím. Ale nebyli s to podrobit si je úplně.
#17:14 I vytýkali Josefovci Jozuovi: „Proč jsi nám dal za dědictví jeden los a jednu oblast? Jsme lid početný, protože Hospodin nám tak mnoho požehnal.“
#17:15 Jozue jim řekl: „Jste-li lid početný, vytáhněte do lesů, tam na území Perizejců a Refájců, a vykácejte je, když je pro vás Efrajimské pohoří těsné.“
#17:16 Josefovci namítali: „Pohoří nám nepostačí. A všichni Kenaanci, kteří sídlí v dolině, mají železné vozy, jak v Bét-šeánu s jeho vesnicemi, tak v dolině Jizreelu.“
#17:17 Jozue odpověděl domu Josefovu, Efrajimovi a Manasesovi: „Jste početný lid a máte velkou sílu. Nebudete mít jeden los.
#17:18 Bude vám patřit pohoří. Protože je zalesněné, vykácíte je, a kam až vybíhá, bude vám patřit. Podrobíte si Kenaance, třebaže mají železné vozy a jsou silní.“ 
#18:1 Celá pospolitost Izraelců se shromáždila do Šíla a postavili tam stan setkávání. Zemi si už podmanili.
#18:2 Mezi Izraelci zůstalo však ještě sedm kmenů, kterým nepřidělili dědičný podíl.
#18:3 Tu řekl Jozue Izraelcům: „Jak dlouho budete otálet, než konečně obsadíte zemi, kterou vám dal Hospodin, Bůh vašich otců?
#18:4 Určete si za každý kmen tři muže. Pošlu je, aby se vydali na cestu a prošli zemi. Až pořídí její popis podle příslušných dědičných podílů, přijdou ke mně.
#18:5 Rozdělí si ji na sedm podílů. Juda zůstane na svém území na jihu a dům Josefův zůstane na svém území na severu.
#18:6 Vy pak pořídíte popis země, rozdělený na sedm podílů, a přinesete jej sem ke mně a já vám ji přidělím losem zde před Hospodinem, naším Bohem.
#18:7 Lévijci nebudou mít mezi vámi žádný podíl, jejich dědičným podílem bude Hospodinovo kněžství. A Gád, Rúben a polovice Manasesova kmene dostali své dědičné podíly na východě v Zajordání, jak jim je dal Mojžíš, služebník Hospodinův.“
#18:8 Mužové se tedy vydali na cestu a Jozue přikázal těm, kteří šli pořídit popis země: „Jděte a projděte zemi. Až pořídíte její popis, vrátíte se ke mně a já zde v Šílu před Hospodinem vylosuji vaše podíly.“
#18:9 Mužové šli, prošli zemi a pořídili její popis podle měst, rozdělený na sedm podílů. Pak přišli k Jozuovi do tábora v Šílu
#18:10 a Jozue jim v Šílu před Hospodinem přidělil losem podíly. Jozue tam rozdělil Izraelcům zemi podle jejich podílů.
#18:11 První los byl vytažen pokolení Benjamínovců pro jejich čeledi: bylo jim vylosováno území mezi Judovci a Josefovci.
#18:12 Toto je jejich hranice na severní straně: začíná u Jordánu a stoupá ke skalnímu hřebenu severně od Jericha. Směrem na západ vystupuje do pohoří a vybíhá k Bétávenské poušti.
#18:13 Odtud pokračuje hranice k Lúzu, ke skalnímu hřebenu jižně od Lúzu, to je Bét-elu, a sestupuje k Atarót-adaru proti hoře, která je na jih od Dolního Bét-chorónu.
#18:14 Pak hranice odbočuje a stáčí se na západní straně k jihu od hory, která je jižně proti Bét-chorónu, a vybíhá ke Kirjat-baalu, to je Kirjat-jearímu, městu Judovců. To je západní strana.
#18:15 Jižní strana: od okraje Kirjat-jearímu se táhne hranice na západ a vede k prameni vod Neftóachu.
#18:16 Pak sestupuje k úpatí hory, která je naproti Údolí syna Hinómova; leží na severu v dolině Refájců. Údolím hinómským sestupuje ke skalnímu hřebenu Jebúsejců na jihu. Pak sestupuje k prameni Rogelu.
#18:17 Zabočuje od severu a vede k Slunečnímu prameni a táhne se podle Kruhů, které leží naproti Adumímskému svahu. Potom sestupuje ke kameni Rúbenovce Bohana.
#18:18 Pokračuje severně ke skalnímu hřebenu nad Jordánskou pustinou a sestupuje do pustiny.
#18:19 Pak pokračuje na sever ke skalnímu hřebenu u Bét-chogly a vybíhá k severní zátoce Solného moře při ústí Jordánu na jihu. To je jižní hranice.
#18:20 Na východní straně tvoří jejich hranici Jordán. To je dědičný podíl Benjamínovců pro jejich čeledi, vymezený svými hranicemi dokola.
#18:21 Pokolení Benjamínovců a jejich čeledím patřila města: Jericho, Bét-chogla a Emek- kesis,
#18:22 Bét-araba, Semárajim a Bét-el,
#18:23 Avím, Para a Ofra,
#18:24 Kefar-amóní, Ofní a Geba, dvanáct měst s dvorci;
#18:25 Gibeón, Ráma a Beerót,
#18:26 Mispe, Kefíra a Mósa,
#18:27 Rekem, Jirpeel a Tarala,
#18:28 Sela, Elef a město Jebúsejcovo, to je Jeruzalém, Gibeat, Kirjat, čtrnáct měst i s dvorci. To je dědičný podíl Benjamínovců pro jejich čeledi. 
#19:1 Druhý los vyšel Šimeónovi, totiž pokolení Šimeónovců pro jejich čeledi. Jejich dědičný podíl byl uprostřed podílu Judovců.
#19:2 Patřila jim dědičným podílem: Beer-šeba, Šeba a Mólada,
#19:3 Chasal-šúal, Bála a Asem,
#19:4 Eltólad, Betúl a Chorma,
#19:5 Siklag, Bét-markabót a Chasarsúsa,
#19:6 Bét-lebaót a Šarúchen, třináct měst i s dvorci;
#19:7 Ajin, Rimón, Eter a Ašan, čtyři města i s dvorci.
#19:8 Také všechny dvorce, které byly v okolí těch měst až po Baalat-beer a Ramat-negeb. To je dědičný podíl pokolení Šimeónovců pro jejich čeledi.
#19:9 Dědičný podíl Šimeónovců byl v oblasti Judovců. Protože podíl Judovců byl pro ně příliš velký, dostali Šimeónovci dědičný podíl uprostřed jejich podílu.
#19:10 Třetí los byl vytažen Zabulónovcům pro jejich čeledi. Pomezí jejich dědičného podílu sahalo až k Sarídu.
#19:11 Jejich hranice stoupá na západ, k Mareále. Dotýká se Dabešetu a sleduje potok naproti Jokneámu.
#19:12 Od Sarídu jde zpět přímo na východ k pomezí Kislótu pod Táborem, táhne se k Daberatu a vystupuje k Jáfiji.
#19:13 Odtud pokračuje přímo na východ do Gat-cheferu a Et-kasínu, vede k Rimónu a zabočuje k Néji.
#19:14 Tam se hranice stáčí na sever k Chanatónu a vybíhá do údolí Jiftach-elu.
#19:15 Patří jim také Katat, Nahalal, Šimrón, Jidala a Bét-lechem, dvanáct měst i s dvorci.
#19:16 Tato města i s dvorci jsou dědičným podílem Zabulónovců pro jejich čeledi.
#19:17 Čtvrtý los vyšel Isacharovi, totiž Isacharovcům pro jejich čeledi.
#19:18 Jejich pomezí sahalo až k Jizreelu a patří k němu Kesulót a Šúnem,
#19:19 Chafarajim, Šíon a Anacharat,
#19:20 Rabít, Kišjón a Ebes,
#19:21 Remet, Én-ganím, Én-chada a Bét-pases.
#19:22 Hranice se dotýká Tábora, táhne se k Šachasímu a Bét-šemeši a vybíhá k Jordánu, šestnáct měst i s dvorci.
#19:23 Ta města i s dvorci jsou dědičným podílem pokolení Isacharovců pro jejich čeledi.
#19:24 Pátý los vyšel pokolení Ašerovců pro jejich čeledi.
#19:25 Připadlo jim pomezí: Chelkat, Chalí, Beten a Akšáf,
#19:26 Alamelek, Ameád a Mišál. Hranice se na západě dotýká Karmelu a ramene říčky Libnátu.
#19:27 Obrací se na východ k Bét-dágonu, dotýká se území zabulónského a údolí Jiftach-elu na severu, táhne se k Bét-emeku a Neíelu a vede nalevo ke Kabúlu
#19:28 a přes Ebrón, Rechób, Chamón a Kánu až k velkému Sidónu.
#19:29 Pak se hranice vrací k Rámě a až k opevněnému městu Týru. Vrací se k Chose a mezi Chebelem a Akzíbem vybíhá k moři.
#19:30 Patří jim také Uma, Afek a Rechób, dvaadvacet měst i s dvorci.
#19:31 Tato města i s dvorci jsou dědičným podílem pokolení Ašerovců pro jejich čeledi.
#19:32 Šestý los vyšel Neftalíovcům, totiž Neftalíovcům pro jejich čeledi.
#19:33 Připadalo jim pomezí od Chelefu, od božiště v Saananímu, přes Adamí-nekeb a Jabneel k Lakúmu; vybíhá k Jordánu.
#19:34 Pak se hranice vrací na západ k Aznótu pod Táborem a táhne se odtud k Chúkoku. Na jihu se dotýká území zabulónského, na západě ašerského a na východě u Jordánu judského.
#19:35 Patří jim i opevněná města: Sidím, Ser a Chamát, Rakat, Kineret,
#19:36 Adama, Ráma a Chasór,
#19:37 Kedeš, Edreí a Én-chasór,
#19:38 Jirón, Migdal-el, Chorem, Bét-anat a Bét-šemeš, devatenáct měst i s dvorci.
#19:39 Ta města i s dvorci jsou dědičným podílem pokolení Neftalíovců pro jejich čeledi.
#19:40 Sedmý los vyšel pokolení Danovců pro jejich čeledi.
#19:41 Pomezí jejich dědičného podílu tvořily: Sorea, Eštaól a Ír-šemeš,
#19:42 Šaalabín, Ajalón a Jitla,
#19:43 Elón, Timnata a Ekrón,
#19:44 Elteke, Gibetón a Baalat,
#19:45 Jehúd, Bené-berak a Gat-rimón,
#19:46 vody Jarkónu a Rakónu s pomezím naproti Jafě.
#19:47 Avšak Danovci o své území přišli. Vytáhli proto do boje s Lešemem a dobyli jej. Vybili jej ostřím meče, obsadili jej a usídlili se v něm. Přejmenovali Lešem na Dan podle svého praotce Dana.
#19:48 Tato města i s dvorci jsou dědičným podílem pokolení Danovců pro jejich čeledi.
#19:49 Tak ukončili přidělování země do dědictví podle jednotlivých území. Jozuovi, synu Núnovu, dali Izraelci dědičný podíl mezi sebou.
#19:50 Podle Hospodinova rozkazu dali mu město, které si vyžádal, totiž Timnat-serach v Efrajimském pohoří. Vybudoval město a usadil se v něm.
#19:51 Toto jsou dědičné podíly, které kněz Eleazar, Jozue, syn Núnův, a představitelé rodů přidělili losem izraelským pokolením v Šílu před Hospodinem u vchodu do stanu setkávání. A skončilo přidělování země. 
#20:1 Hospodin promluvil k Jozuovi:
#20:2 „Mluv k Izraelcům: Určete si útočištná města, jak jsem vám uložil skrze Mojžíše.
#20:3 Tam ať se uteče ten, kdo zabil, pokud někoho zabil nedopatřením, neúmyslně. Budou vám za útočiště před krevním mstitelem.
#20:4 Kdo by se utekl do některého z těchto měst, postaví se při vchodu do městské brány a přednese svou záležitost starším toho města. Pak ho přijmou k sobě do města a určí mu místo, aby s nimi bydlil.
#20:5 Když ho bude pronásledovat krevní mstitel, nesmějí mu toho, kdo zabil, vydat do rukou, poněvadž zabil svého bližního neúmyslně, aniž ho kdy předtím nenáviděl.
#20:6 Bude bydlet v tom městě, dokud se nepostaví před pospolitost k soudu, případně do smrti velekněze, který v těch dnech bude vykonávat svůj úřad. Tehdy se ten, kdo zabil, smí vrátit do svého města a ke své rodině, do města odkud utekl.“
#20:7 I oddělili jako svatý Kedeš v Galileji v pohoří Neftalíově, Šekem v pohoří Efrajimově a Kirjatarbu, to je Chebrón, v pohoří Judově.
#20:8 V Zajordání pak určili na východ od Jericha Beser ve stepi na Náhorní rovině v pokolení Rúbenově, Rámot v Gileádu v pokolení Gádově a Gólan v Bášanu v pokolení Manasesově.
#20:9 To jsou smluvní města pro všechny Izraelce i pro bezdomovce, kteří budou mezi vámi pobývat. Tam ať se uteče každý, kdo by někoho zabil nedopatřením; nezemře rukou krevního mstitele, dokud se nepostaví před shromáždění. 
#21:1 Představitelé lévijských rodů přistoupili ke knězi Eleazarovi a k Jozuovi, synu Núnovu, a k představitelům rodů izraelských pokolení.
#21:2 Mluvili k nim v Šílu, v kenaanské zemi, takto: „Hospodin přikázal skrze Mojžíše, aby nám byla dána města k bydlení i pastviny při nich pro náš dobytek.“
#21:3 Izraelci dali tedy lévijcům podle Hospodinova rozkazu ze svého dědičného podílu tato města s pastvinami:
#21:4 Los vyšel čeledím Kehatovců. Synům kněze Árona mezi lévijci bylo losem určeno třináct měst z pokolení Judova, Šimeónova a Benjamínova.
#21:5 Zbývajícím Kehatovcům bylo losem určeno deset měst z čeledí pokolení Efrajimova, z pokolení Danova a poloviny pokolení Manasesova.
#21:6 Geršónovcům bylo losem určeno třináct měst z čeledí pokolení Isacharova, Ašerova, Neftalíova a poloviny pokolení Manasesova v Bášanu.
#21:7 Meraríovcům bylo určeno dvanáct měst z pokolení Rúbenova, Gádova a Zabulónova podle jejich čeledí. -
#21:8 Izraelci určili losem lévijcům tato města i s pastvinami, jak to přikázal Hospodin skrze Mojžíše.
#21:9 Z pokolení Judovců a Šimeónovců dali jménem uvedená města.
#21:10 Připadla Áronovcům z čeledí Kehatovců mezi lévijci, protože jim padl los nejprve.
#21:11 Dali jim Kirjat-arbu, město otce Anókova, to je Chebrón, v pohoří Judském, a pastviny kolem něho.
#21:12 Ale pole toho města a jeho dvorce dali do trvalého vlastnictví Kálebovi, synu Jefunovu.
#21:13 Synům kněze Árona dali jako útočištné město pro toho, kdo zabil, Chebrón i s pastvinami; dále Libnu i s pastvinami,
#21:14 Jatír i s pastvinami a Eštemóu i s pastvinami,
#21:15 Cholón i s pastvinami a Debír i s pastvinami,
#21:16 Ajin i s pastvinami, Jutu i s pastvinami, Bét-šemeš i s pastvinami, devět měst z těchto dvou kmenů. -
#21:17 Z pokolení Benjamínova: Gibeón i s pastvinami, Gebu i s pastvinami,
#21:18 Anatót i s pastvinami a Almón i s pastvinami, čtyři města.
#21:19 Všech měst pro Áronovce, kněží, bylo třináct, i s pastvinami.
#21:20 Čeledím Kehatovců, lévijcům, kteří zbyli z Kehatovců, byla losem určena města z pokolení Efrajimova.
#21:21 Jako útočištné město pro toho, kdo zabil, jim dali Šekem i s pastvinami v pohoří Efrajimském, dále Gezer i s pastvinami,
#21:22 Kibsajim i s pastvinami a Bét-chorón i s pastvinami, čtyři města. -
#21:23 Z pokolení Danova: Elteke i s pastvinami, Gibetón i s pastvinami,
#21:24 Ajalón i s pastvinami, Gat-rimón i s pastvinami, čtyři města. -
#21:25 Z poloviny pokolení Manasesova: Taanak i s pastvinami a Gat-rimón i s pastvinami, dvě města.
#21:26 Všech měst pro čeledi zbývajících Kehatovců bylo deset, i s pastvinami.
#21:27 Geršónovcům z lévijských čeledí připadlo z poloviny pokolení Manasesova jako útočištné město pro toho, kdo zabil, Gólan v Bášanu i s pastvinami; dále Beeštera i s pastvinami, dvě města. -
#21:28 Z pokolení Isacharova: Kišjón i s pastvinami, Daberat i s pastvinami,
#21:29 Jarmút i s pastvinami, Én-ganím i s pastvinami, čtyři města. -
#21:30 Z pokolení Ašerova: Mišál i s pastvinami, Abdón i s pastvinami,
#21:31 Chelkat i s pastvinami a Rechob i s pastvinami, čtyři města. -
#21:32 Z pokolení Neftalíova: jako útočištné město pro toho, kdo zabil, Kedeš v Galileji i s pastvinami; dále Chamót-dór i s pastvinami a Kartan i s pastvinami, tři města. -
#21:33 Všech měst pro Geršónovce a jejich čeledi bylo třináct, i s pastvinami.
#21:34 Čeledím Meraríovců, zbývajícím lévijcům, připadlo z pokolení Zabulónova: Jokneám i s pastvinami, Karta i s pastvinami,
#21:35 Dimna i s pastvinami, Nahalal i s pastvinami, čtyři města. -
#21:36 Z pokolení Rúbenova: Beser i s pastvinami; dále Jahsa i s pastvinami,
#21:37 Kedemót i s pastvinami a Méfaat i s pastvinami, čtyři města. -
#21:38 Z pokolení Gádova: jako útočištné město pro toho, kdo zabil, Rámot v Gileádu i s pastvinami; dále Machanajim i s pastvinami,
#21:39 Chešbón i s pastvinami, Jaazer i s pastvinami, celkem čtyři města. -
#21:40 Všech měst pro Meraríovce a jejich čeledi, zbývající z lévijských čeledí, bylo určeno losem dvanáct.
#21:41 Všech lévijských měst uprostřed trvalého vlastnictví Izraelců bylo čtyřicet osm, i s pastvinami.
#21:42 Ke každému z těchto měst patřily okolní pastviny; to platilo o každém z těchto měst.
#21:43 Hospodin dal Izraeli celou zemi, jak se přísahou zavázal jejich otcům, že jim ji dá. Obsadili ji a usadili se v ní.
#21:44 A Hospodin jim dal odpočinutí všudy vůkol, zcela tak, jak přisáhl jejich otcům. Žádný ze všech jejich nepřátel před nimi neobstál. Hospodin jim vydal do rukou všechny nepřátele.
#21:45 Ani jedno ze všech dobrých slov, která Hospodin mluvil k domu izraelskému, nezapadlo: všechno se uskutečnilo. 
#22:1 Tehdy svolal Jozue Rúbenovce, Gádovce a polovinu pokolení Manasesova
#22:2 a řekl jim: „Zachovali jste všechno, co vám přikázal Mojžíš, služebník Hospodinův. Uposlechli jste mě ve všem, co jsem vám přikázal.
#22:3 Neopustili jste své bratry po celý dlouhý čas až dodnes, ale dbali jste na to, co vám svěřil a přikázal Hospodin, váš Bůh.
#22:4 Teď dopřál Hospodin, váš Bůh, vašim bratřím odpočinutí, jak jim přislíbil. Vraťte se tedy ke svým stanům, do země svého trvalého vlastnictví, kterou vám dal Mojžíš, služebník Hospodinův, v Zajordání.
#22:5 Jen se usilovně snažte zachovávat přikázání a zákon, které vám přikázal Mojžíš, služebník Hospodinův, totiž milovat Hospodina, svého Boha, chodit po všech jeho cestách a dbát na jeho přikázání, přimknout se k němu a sloužit mu z celého srdce a z celé duše.“
#22:6 Jozue jim požehnal a propustil je a oni odešli ke svým stanům.
#22:7 Polovině kmene Manasesova dal Mojžíš území v Bášanu a druhé polovině dal Jozue území spolu s jejich bratřími na západ od Jordánu. Když je tehdy Jozue s požehnáním propouštěl k jejich stanům,
#22:8 řekl jim: „S velikými poklady se navraťte ke svým stanům, s velkým množstvím dobytka, stříbrem, zlatem, mědí a železem a s velkým množstvím rouch; rozdělte si s bratřími kořist po nepřátelích.“
#22:9 Rúbenovci, Gádovci a polovina kmene Manasesova se vrátili od Izraelců v Šílu, jež je v kenaanské zemi, a táhli do země gileádské, do země svého trvalého vlastnictví, kterou obdrželi na rozkaz Hospodinův skrze Mojžíše.
#22:10 Když přišli ke Kruhům jordánským, které jsou ještě v zemi kenaanské, zbudovali tam Rúbenovci, Gádovci a polovina kmene Manasesova nad Jordánem oltář, veliký, nápadný oltář.
#22:11 Izraelci se doslechli, že Rúbenovci, Gádovci a polovina kmene Manasesova zbudovali oltář naproti kenaanské zemi u jordánských Kruhů, směrem k Izraelcům.
#22:12 Když to Izraelci uslyšeli, shromáždila se celá pospolitost Izraelců do Šíla, aby se proti nim vypravili do boje.
#22:13 Izraelci vyslali k Rúbenovcům, Gádovcům a polovině kmene Manasesova do země gileádské kněze Pinchasa, syna Eleazarova,
#22:14 a s ním deset předáků, po jednom předáku z každého rodu všech izraelských pokolení. Každý byl představitelem svého otcovského rodu v izraelských šicích.
#22:15 Přišli do gileádské země k Rúbenovcům, Gádovcům a k polovině kmene Manasesova a jednali s nimi:
#22:16 „Toto praví celá Hospodinova pospolitost: Jakou to zpronevěrou jste se zpronevěřili Bohu Izraele, že jste se dnes odvrátili od Hospodina! Zbudovali jste si oltář, abyste se dnes vzbouřili proti Hospodinu?
#22:17 Což je pro nás maličkostí nepravost spáchaná v Peóru, od níž až dodnes nejsme očištěni a kvůli níž přišla pohroma na Hospodinovu pospolitost?
#22:18 A vy se dnes odvracíte od Hospodina! Proto se stane toto: Dnes se bouříte vy proti Hospodinu a zítra se on rozlítí na celou izraelskou pospolitost.
#22:19 Jestliže země vašeho trvalého vlastnictví je nečistá, přejděte přece do země trvalého vlastnictví Hospodinova, v němž přebývá Hospodinův příbytek, a dejte si přidělit podíl mezi námi. Jen se nebuřte proti Hospodinu a nezatahujte nás do své vzpoury budováním vlastního oltáře mimo oltář Hospodina, našeho Boha.
#22:20 Což se Akán, syn Zerachův, nedopustil zpronevěry věcí oddaných klatbě? Což nedolehlo Hospodinovo rozlícení na celou izraelskou pospolitost? A to způsobil jediný člověk! Což pro svou nepravost nezahynul?“
#22:21 Tu odpověděli Rúbenovci, Gádovci a polovina kmene Manasesova a takto hovořili s představiteli izraelských šiků:
#22:22 „Bůh bohů Hospodin, Bůh bohů Hospodin to ví a ať to zví rovněž Izrael: Jestliže to byla vzpoura či zpronevěra vůči Hospodinu, ať nás dnešního dne nezachrání.
#22:23 Jestliže jsme si zbudovali oltář, abychom se odvrátili od Hospodina, abychom na něm obětovali oběti zápalné a přídavné nebo připravovali při něm hody oběti pokojné, ať zakročí sám Hospodin!
#22:24 Ve skutečnosti jsme tak učinili z obavy, protože jsme si řekli, že se jednou synové vaši zeptají našich: ‚Co máte společného s Hospodinem, Bohem Izraele?
#22:25 Vždyť Hospodin určil za hranici mezi námi a vámi Jordán, synové Rúbenovi a Gádovi. Nemáte podíl v Hospodinu.‘ Tak vaši synové zaviní, že se naši synové přestanou bát Hospodina.
#22:26 Proto jsme si řekli: Dejme se do díla a zbudujme si oltář, ale ne pro zápalnou oběť ani pro obětní hod,
#22:27 nýbrž aby byl svědkem mezi námi a vámi i mezi našimi pokoleními po nás, že máme právo sloužit Hospodinu před jeho tváří svými zápalnými obětmi i hody oběti pokojné. Vaši synové nebudou moci jednou říkat našim synům: ‚Nemáte podíl v Hospodinu.‘
#22:28 Řekli jsme si: Kdyby se stalo, že by tak mluvili v budoucnu s námi a s našimi pokoleními, odpověděli bychom: Hleďte, zde je zpodobení Hospodinova oltáře; ten zřídili naši otcové ne pro zápalnou oběť ani pro obětní hod, nýbrž aby byl svědkem mezi námi a vámi.
#22:29 Jsme daleci toho vzbouřit se proti Hospodinu a odvrátit se dnes od Hospodina vybudováním oltáře pro oběti zápalné, přídavné a obětní hod mimo oltář Hospodina, našeho Boha, který je před jeho příbytkem.“
#22:30 Když uslyšeli kněz Pinchas a předáci pospolitosti i představitelé izraelských šiků, kteří byli s ním, co mluvili Rúbenovci, Gádovci a Manasesovci, líbilo se jim to.
#22:31 I řekl kněz Pinchas, syn Eleazarův, Rúbenovcům, Gádovcům a Manasesovcům: „Dnes jsme poznali, že je Hospodin uprostřed nás. Nezpronevěřili jste se Hospodinu. Vyprostili jste tím Izraelce z Hospodinových rukou.“
#22:32 Pak se kněz Pinchas, syn Eleazarův, s předáky vrátil od Rúbenovců a Gádovců ze země gileádské k Izraelcům do země kenaanské a podal jim zprávu.
#22:33 Izraelcům se to zalíbilo. Dobrořečili Bohu a nemluvili již o tom, že potáhnou do boje proti nim a že zničí zemi, v níž sídlí Rúbenovci s Gádovci.
#22:34 Rúbenovci a Gádovci pak oltář pojmenovali a řekli: „Je mezi námi svědkem, že Hospodin je Bůh.“ 
#23:1 Po mnoha dnech, když Hospodin dal Izraeli odpočinout od všech okolních nepřátel, Jozue, stařec pokročilého věku,
#23:2 svolal celý Izrael, jeho starší a představitele, soudce a správce, a řekl jim: „Jsem stařec pokročilého věku.
#23:3 Sami jste viděli všechno, co učinil Hospodin, váš Bůh, všem těm pronárodům před vašima očima; neboť Hospodin, váš Bůh, bojoval za vás.
#23:4 Pohleďte, přidělil jsem vám tyto zbylé pronárody do dědictví pro vaše kmeny, od Jordánu, se všemi pronárody, které jsem vyhladil, až k Velkému moři, kde slunce zapadá.
#23:5 Hospodin, váš Bůh, je zapudí od vaší tváře a vyžene je před vámi; a vy obsadíte jejich zemi, jak vám přislíbil Hospodin, váš Bůh.
#23:6 A tak buďte zcela rozhodní a zachovávejte a čiňte všechno, co je napsáno v knize Mojžíšova zákona, abyste se od něho neuchýlili napravo ani nalevo.
#23:7 Nesmíte se mísit s těmito pronárody, jejichž zbytky jsou mezi vámi. Jména jejich bohů nepřipomínejte ani skrze ně nepřísahejte; neslužte jim a neklaňte se jim,
#23:8 ale přimkněte se k Hospodinu, svému Bohu, jak jste činili dodnes.
#23:9 Neboť Hospodin před vámi vyhnal velké a mocné pronárody; nikdo před vámi až dodnes neobstál.
#23:10 Jediný z vás bude schopen pronásledovat tisíc mužů, protože Hospodin, váš Bůh, bojuje za vás, jak vám přislíbil.
#23:11 Sami u sebe dbejte úzkostlivě o to, abyste milovali Hospodina, svého Boha.
#23:12 Jestliže však přece odpadnete a přimknete se ke zbytku těch pronárodů, které ještě zbyly mezi vámi, a jestliže se s nimi spřízníte sňatkem a budete se mísit s nimi a oni s vámi,
#23:13 tedy jistotně vězte, že Hospodin, váš Bůh, už před vámi ty pronárody nevyžene. Stanou se vám osidlem a léčkou, karabáčem na vašich bocích a trny v očích, dokud nevymizíte z této dobré země, kterou vám dal Hospodin, váš Bůh.
#23:14 Hle, já nyní odcházím cestou všeho pozemského. Uznejte tedy celým srdcem a celou duší, že nezapadlo ani jedno ze všech dobrých slov, která o vás mluvil Hospodin, váš Bůh. Všechno se vám uskutečnilo, nezapadlo z toho jediné slovo.
#23:15 A tak se i stane, že jako se na vás uskutečnilo všechno dobro, o němž k vám mluvil Hospodin, váš Bůh, tak na vás Hospodin uvede všechno zlo a vyhladí vás z této dobré země, kterou vám Hospodin, váš Bůh, dal.
#23:16 Jestliže přestoupíte smlouvu Hospodina, svého Boha, kterou vám přikázal, a půjdete sloužit jiným bohům a klanět se jim, vzplane proti vám Hospodinův hněv a rychle vymizíte z této dobré země, kterou vám dal.“ 
#24:1 Jozue shromáždil všechny izraelské kmeny do Šekemu. Svolal izraelské starší, představitele, soudce a správce, a postavili se před Bohem.
#24:2 Jozue řekl všemu lidu: „Toto praví Hospodin, Bůh Izraele: Vaši otcové, Terach, otec Abrahamův a otec Náchorův, sídlili odedávna za řekou Eufratem a sloužili jiným bohům.
#24:3 Vzal jsem vašeho otce Abrahama odtamtud za řekou a provedl jsem jej celou kenaanskou zemí. Rozmnožil jsem jeho potomstvo a dal jsem mu Izáka.
#24:4 Izákovi jsem dal Jákoba a Ezaua. Ezauovi jsem dal do vlastnictví Seírské pohoří. Jákob a jeho synové sestoupili do Egypta.
#24:5 Poslal jsem Mojžíše a Árona a porazil jsem Egypt divy, které jsem učinil uprostřed nich. Potom jsem vás vyvedl.
#24:6 A když jsem vyvedl vaše otce z Egypta, přišli jste k moři. Egypťané však vaše otce pronásledovali u Rákosového moře na vozech a koních.
#24:7 I úpěli k Hospodinu a on položil mračno mezi vás a Egypťany a způsobil, že se přes ně přelilo moře a přikrylo je. Na vlastní oči jste viděli, jak jsem naložil s Egyptem. Pak jste pobývali dlouhý čas v poušti.
#24:8 Potom jsem vás uvedl do země Emorejců, kteří sídlili v Zajordání. Bojovali proti vám, ale já jsem vám je vydal do rukou. Obsadili jste jejich zemi a já jsem je před vámi vyhladil.
#24:9 Když se pozdvihl moábský král Balák, syn Sipórův, aby bojoval proti Izraeli, a poslal pro Bileáma, syna Beórova, aby vám zlořečil,
#24:10 nechtěl jsem Bileáma slyšet, opětovně vám musel žehnat. Tak jsem vás vytrhl z jeho rukou.
#24:11 Pak jste přešli Jordán a přišli jste k Jerichu. Občané Jericha, Emorejci, Perizejci, Kenaanci, Chetejci, Girgašejci, Chivejci a Jebúsejci bojovali proti vám, ale já jsem vám je vydal do rukou.
#24:12 Poslal jsem před vámi děsy a ti je před vámi zapudili, totiž dva krále emorejské; nezapudil jsi je ty svým mečem a lukem.
#24:13 Dal jsem vám zemi, na kterou jste nevynaložili žádnou námahu, města, která jste nestavěli, ale sídlíte v nich, vinice a olivoví, které jste nesázeli, a přece z nich jíte.
#24:14 Bojte se tedy Hospodina a služte mu bezvýhradně a věrně. Odstraňte božstva, kterým vaši otcové sloužili za řekou Eufratem a v Egyptě, a služte Hospodinu.
#24:15 Jestliže se vám zdá, že sloužit Hospodinu je zlé, vyvolte si dnes, komu chcete sloužit: zda božstvům, kterým sloužili vaši otcové, když byli za řekou Eufratem, nebo božstvům Emorejců, v jejichž zemi sídlíte. Já a můj dům budeme sloužit Hospodinu.“
#24:16 Lid odpověděl: „Jsme daleci toho, opustit Hospodina a sloužit jiným bohům!
#24:17 Naším Bohem je přece Hospodin. On nás i naše otce vyvedl z egyptské země, z domu otroctví. On činil před našimi zraky ta veliká znamení, opatroval nás na celé cestě, kterou jsme šli, i mezi kdejakým lidem, skrze nějž jsme procházeli.
#24:18 Hospodin zapudil od nás každý lid, i Emorejce sídlící v zemi. Také my budeme sloužit Hospodinu. On je náš Bůh.“
#24:19 Tu řekl Jozue lidu: „Nebudete moci sloužit Hospodinu, neboť on je Bůh svatý. Je to Bůh žárlivý, nepromine vám vaše nevěrnosti a hříchy.
#24:20 Jestliže opustíte Hospodina a budete sloužit cizím bohům a odvrátíte se, zle s vámi naloží a skoncuje s vámi, ač vám předtím učinil mnoho dobrého.“
#24:21 Lid Jozuovi odpověděl: „Nikoli. Budeme sloužit jen Hospodinu!“
#24:22 Nato Jozue vyhlásil lidu: „Budete svědky sami proti sobě, nedodržíte-li své rozhodnutí, že budete sloužit Hospodinu.“ Odpověděli: „Ano, budeme svědky.“
#24:23 Jozue pokračoval: „Odstraňte tedy cizí božstva, která jsou mezi vámi, a přikloňte se srdcem k Hospodinu, Bohu Izraele.“
#24:24 Lid řekl Jozuovi: „Budeme sloužit Hospodinu, svému Bohu, a jeho budeme poslouchat.“
#24:25 I uzavřel Jozue onoho dne v Šekemu smlouvu s lidem a vydal mu nařízení a právní ustanovení.
#24:26 Tato slova zapsal Jozue do knihy Božího zákona. Pak vzal veliký kámen a postavil jej tam pod posvátným stromem, který stál při Hospodinově svatyni.
#24:27 A všemu lidu Jozue řekl: „Hle, tento kámen bude proti nám jako svědek, neboť slyšel všechna slova, která s námi Hospodin mluvil. Bude proti vám jako svědek, abyste nezapírali svého Boha.“
#24:28 Potom Jozue lid rozpustil, každého k jeho dědičnému podílu.
#24:29 Po těch událostech Jozue, syn Núnův, služebník Hospodinův, zemřel ve věku sto deseti let.
#24:30 Pohřbili jej na území jeho dědičného podílu v Timnat-serachu v Efrajimském pohoří na sever od hory Gaaše.
#24:31 Izrael sloužil Hospodinu po celou dobu Jozuovu i po celou dobu starších, kteří Jozua přežili a znali celé Hospodinovo dílo, které pro Izraele vykonal. -
#24:32 Josefovy kosti, přenesené Izraelci z Egypta, pohřbili v Šekemu na dílu pole, který koupil Jákob od synů Šekemova otce Chamóra za sto kesít. Připadlo do dědictví Josefovým synům. -
#24:33 Také Eleazar, syn Áronův, zemřel a pohřbili jej v Gibeji, v městě jeho syna Pinchasa, které mu bylo dáno v pohoří Efrajimském.  

\book{Judges}{Judg}
#1:1 Po Jozuově smrti se synové Izraele dotázali Hospodina: „Kdo z nás má vytáhnout do boje proti Kenaancům první?“
#1:2 Hospodin odpověděl: „První potáhne Juda; do jeho rukou jsem dal zemi.“
#1:3 I vyzval Juda svého bratra Šimeóna: „Vytáhni se mnou do území, které mi bylo přiděleno losem, a budeme bojovat proti Kenaancům. Já pak půjdu zase s tebou do území přiděleného tobě.“ Šimeón s ním tedy šel.
#1:4 Juda vytáhl a Hospodin vydal Kenaance a Perizejce do jejich rukou. Pobili je v Bezeku, deset tisíc mužů.
#1:5 V Bezeku zastihli také Adoní-bezeka a bojovali proti němu a pobili Kenaance a Perizejce.
#1:6 Avšak Adoní-bezek utekl. Proto jej pronásledovali, chytili jej a uťali mu palce na rukou i u nohou.
#1:7 Tu Adoní-bezek doznal: „Sedmdesát králů s uťatými palci na rukou i u nohou sbíralo drobty pod mým stolem. Jak jsem činíval, tak mi Bůh odplatil.“ Odvlekli jej do Jeruzaléma a tam zemřel.
#1:8 Judovci bojovali proti Jeruzalému, dobyli jej, vybili ostřím meče a město vypálili.
#1:9 Potom se vypravili do boje proti Kenaancům, kteří byli usazeni na pohoří, v Negebu a v Přímořské nížině.
#1:10 Tak se vydal Juda proti Kenaancům sídlícím v Chebrónu; Chebrón se předtím jmenoval Kirjat-arba. Pobili Šešaje, Achímana a Talmaje.
#1:11 Odtud se vydal proti obyvatelům Debíru; Debír se předtím jmenoval Kirjat-sefer.
#1:12 Káleb tenkrát prohlásil: „Kdo přepadne Kirjat-sefer a dobude jej, tomu dám za manželku svou dceru Aksu.“
#1:13 Dobyl jej Otníel, syn Kenazův, mladší bratr Kálebův; dal mu tedy svou dceru Aksu za manželku.
#1:14 Když přišla, navedla jej, aby žádal jejího otce o pole. Sesedla z osla. Káleb se jí otázal: „Co si přeješ?“
#1:15 Odpověděla mu: „Obdař mě požehnáním! Když jsi mne provdal do jižního suchopáru, dej mi také vodní zřídla.“ Káleb jí tedy dal zřídla, horní a dolní.
#1:16 Kénijci, synové Mojžíšova tchána, vytáhli z Palmového města spolu s Judovci do Judské stepi jižně od Aradu. Tam se usadili s lidem.
#1:17 Juda však táhl se svým bratrem Šimeónem a pobili Kenaance sídlící v Sefatu. Zničili město jako klaté; proto mu dali jméno Chorma (to je Klatbě propadlé).
#1:18 Juda dobyl také Gázu a její území, Aškalón a jeho území, Ekrón a jeho území.
#1:19 Hospodin byl s Judou. Podrobil si pohoří, avšak nebyl s to podrobit si obyvatele doliny, protože měli železné vozy.
#1:20 Kálebovi dali Chebrón, jak rozhodl Mojžíš; vyhnal odtud tři Anákovce.
#1:21 Pokud však jde o Jebúsejce, obyvatele Jeruzaléma, nebyli Benjamínovci s to si je podrobit; proto až dodnes sídlí Jebúsejec v Jeruzalémě spolu s Benjamínovci.
#1:22 Také dům Josefův vytáhl, a to k Bét-elu, a Hospodin byl s nimi.
#1:23 Dům Josefův provedl v Bét-elu průzkum; město se předtím jmenovalo Lúz.
#1:24 Hlídka spatřila muže vycházejícího z města a vyzvala ho: „Ukaž nám, kudy se dá vniknout do města, a my ti dáme milost.“
#1:25 Ukázal jim tedy, kudy se dá vniknout do města. Město vybili ostřím meče, ale toho muže i všechnu jeho čeleď propustili.
#1:26 Muž se odebral do chetejské země a vystavěl město, kterému dal jméno Lúz; tak se jmenuje až dodnes.
#1:27 Manases však nebyl s to podrobit si Bét-šeán a jeho vesnice ani Taanak a jeho vesnice ani obyvatele Dóru a jeho vesnic ani obyvatele Jibleámu a jeho vesnic ani obyvatele Megida a jeho vesnic, a to umožnilo Kenaancům zůstat v té zemi.
#1:28 Teprve když se Izrael vzmohl, podrobil Kenaance nuceným pracím, ale nebyl s to podrobit si je úplně.
#1:29 Efrajim nebyl s to podrobit si Kenaance sídlící v Gezeru; Kenaanci sídlili v Gezeru uprostřed nich.
#1:30 Zabulón nebyl s to podrobit si obyvatele Kitrónu ani obyvatele Nahalólu; Kenaanci sídlili uprostřed nich, ale časem byli podrobeni nuceným pracím.
#1:31 Ašer nebyl s to podrobit si obyvatele Aka ani obyvatele Sidónu, Achlábu, Akzíbu, Chelby, Afíku a Rechóbu.
#1:32 Ašerovci sídlili uprostřed Kenaanců usedlých v zemi, protože nebyli s to si je podrobit.
#1:33 Neftalí nebyl s to podrobit si obyvatele Bét-šemeše ani Bét-anaty; sídlil uprostřed Kenaanců usedlých v zemi, ale obyvatelé Bét-šemeše a Bét-anaty jím byli časem podrobeni nuceným pracím.
#1:34 Danovce zatlačili Emorejci do hor a nedovolili jim sestoupit do doliny.
#1:35 To umožnilo Emorejcům zůstat v Har-cheresu, v Ajalónu a v Šaalbímu. Když však vzrostla moc Josefova domu, byli podrobeni nuceným pracím.
#1:36 Emorejská hranice vedla od Svahu štírů, od Skaliska nahoru. 
#2:1 I vystoupil posel Hospodinův z Gilgálu do Bokímu a volal: „Vyvedl jsem vás z Egypta a uvedl jsem vás do země, kterou jsem přísežně slíbil vašim otcům. Prohlásil jsem: ‚Svou smlouvu s vámi navěky nezruším.
#2:2 Vy však nesmíte uzavřít žádnou smlouvu s obyvateli této země. Jejich oltáře rozkopáte.‘ Vy jste mě neuposlechli. Čeho jste se to dopustili?
#2:3 Proto jsem také prohlásil: ‚Nevypudím je před vámi. Budou vám osidlem a jejich bohové vám budou léčkou.‘“
#2:4 Když Hospodinův posel promluvil tato slova ke všem Izraelcům, dal se lid do hlasitého pláče.
#2:5 Proto dali tomu místu jméno Bokím (to je Plačící) a obětovali tam Hospodinu.
#2:6 Když Jozue propustil lid, odebrali se Izraelci každý do svého dědictví, aby obsadili zemi.
#2:7 Lid sloužil Hospodinu po všechny dny Jozuovy a po celou dobu života starších, kteří Jozua přežili a viděli celé velké Hospodinovo dílo, které pro Izraele vykonal.
#2:8 I zemřel Hospodinův služebník Jozue, syn Núnův, ve věku sto desíti let.
#2:9 Pochovali ho na jeho dědičném území v Timnat-cheresu v Efrajimském pohoří severně od hory Gaaše.
#2:10 Též celé ono pokolení se odebralo ke svým otcům. Po nich nastoupilo jiné pokolení, které neznalo Hospodina ani jeho dílo, jež pro Izraele vykonal.
#2:11 Izraelci se dopouštěli toho, co je zlé v Hospodinových očích, a sloužili baalům.
#2:12 Opustili Hospodina, Boha svých otců, který je vyvedl z egyptské země. Chodili za jinými bohy, za božstvy těch národů, které byly kolem nich, a klaněli se jim. Tak Hospodina uráželi.
#2:13 Opustili Hospodina a sloužili Baalovi a Aštoretě.
#2:14 Proto Hospodin vzplanul proti Izraeli hněvem a vydal jej do rukou plenitelů a ti jej plenili. Vydal je napospas okolním nepřátelům, že už vůbec před svými nepřáteli nemohli obstát.
#2:15 Pokaždé, když vytrhli do boje, zle na ně dopadla Hospodinova ruka, jak jim to Hospodin prohlásil a přísahou stvrdil. Doléhalo na ně veliké soužení.
#2:16 I povolával Hospodin soudce, aby je vysvobozovali z rukou plenitelů.
#2:17 Avšak ani své soudce neposlouchali, dál smilnili s jinými bohy a klaněli se jim. Brzo sešli z cesty, po níž chodívali jejich otcové v poslušnosti Hospodinových přikázání; vůbec tak nejednali.
#2:18 Kdykoli jim Hospodin povolával soudce, býval se soudcem a vysvobozoval je z rukou nepřátel po všechny soudcovy dny. Hospodin měl totiž s nimi soucit, když sténali pod svými utlačovateli a tyrany.
#2:19 Po soudcově smrti si však opět počínali hůře než jejich otcové, chodili za jinými bohy, sloužili jim a klaněli se jim. Nevzdali se svých způsobů ani svého zatvrzelého počínání.
#2:20 Proto Hospodin vzplanul proti Izraeli hněvem a prohlásil: „Protože tento pronárod přestoupil moji smlouvu, kterou jsem uložil jejich otcům, a neposlouchají mě,
#2:21 ani já už před nimi nevyženu žádný z těch pronárodů, které ponechal Jozue, než zemřel.
#2:22 Jimi budu Izraele zkoušet, zda bude chodit po cestě Hospodinově právě tak bedlivě jako jejich otcové, či nebude.“
#2:23 Hospodin tam proto tyto pronárody nechal a nevyhnal je hned, ani je nevydal do Jozuových rukou. 
#3:1 Tyto pronárody ponechal Hospodin, aby jimi zkoušel Izraele, totiž všechny ty, kteří nepoznali žádné boje o Kenaan,
#3:2 aby je izraelská pokolení poznala a naučila se bojovat, ti totiž, kteří předtím boj vůbec nepoznali:
#3:3 ponechal pět pelištejských knížat a všechny Kenaance i Sidóňany a Chivejce, kteří obývali Libanónské pohoří od hory Baal-chermónu až po cestu do Chamátu.
#3:4 Ti zůstali, aby byl skrze ně Izrael zkoušen, aby se ukázalo, budou-li poslouchat Hospodinovy příkazy, které přikázal jejich otcům skrze Mojžíše.
#3:5 Izraelci tedy sídlili uprostřed Kenaanců, Chetejců, Emorejců, Perizejců, Chivejců a Jebúsejců.
#3:6 Brali si jejich dcery za ženy a své dcery dávali jejich synům a sloužili jejich bohům.
#3:7 Izraelci se dopouštěli toho, co je zlé v Hospodinových očích. Zapomněli na svého Boha Hospodina a sloužili baalům a ašerám.
#3:8 Proto Hospodin vzplanul proti Izraeli hněvem a vydal jej napospas Kúšanovi Rišátajimskému, králi aramského Dvojříčí. Izraelci otročili Kúšanovi Rišátajimskému osm let.
#3:9 I úpěli Izraelci k Hospodinu a Hospodin jim povolal vysvoboditele, aby je vysvobodil, Otníela, syna Kenazova, mladšího bratra Kálebova.
#3:10 Na něm spočinul duch Hospodinův a on se ujal soudu nad Izraelem. Vytrhl do boje a Hospodin mu vydal do rukou aramejského krále Kúšana Rišátajimského. Na Kúšana Rišátajimského mocně dolehla jeho ruka.
#3:11 Země pak žila v míru po čtyřicet let, dokud Otníel, syn Kenazův, nezemřel.
#3:12 Izraelci se dále dopouštěli toho, co je zlé v Hospodinových očích. Proto Hospodin dopustil, aby moábský král Eglón nabyl nad Izraelem vrchu, neboť se dopouštěli toho, co je zlé v Hospodinových očích.
#3:13 Eglón k sobě shromáždil Amónovce a Amáleka, vytáhl a přepadl Izraele. Přitom obsadili Palmové město.
#3:14 Izraelci otročili moábskému králi Eglónovi osmnáct let.
#3:15 I úpěli Izraelci k Hospodinu a Hospodin jim povolal vysvoboditele, Ehúda, syna Benjamínce Géry, muže, který nevládl pravou rukou. Po něm poslali Izraelci moábskému králi Eglónovi obětní dar.
#3:16 Ehúd si zhotovil na píď dlouhou oboustrannou dýku, připásal si ji pod šat k pravému boku
#3:17 a přinesl moábskému králi Eglónovi obětní dar. Eglón byl velice tlustý.
#3:18 Jakmile skončilo předání obětního daru, Ehúd propustil lid, který obětní dar přinesl,
#3:19 a sám se vrátil od model u Posvátného kruhu a řekl: „Mám pro tebe tajné sdělení, králi.“ Král řekl: „Tiše!“ Všichni, kteří ho obklopovali, od něho tedy odešli.
#3:20 Eglón seděl v chladném vrchním pokoji, určeném jenom pro něho. Ehúd k němu přistoupil a řekl: „Mám pro tebe výrok Boží.“ Eglón povstal z křesla.
#3:21 Vtom Ehúd vytrhl levicí od pravého boku dýku a vrazil mu ji do břicha.
#3:22 Za čepelí vnikl i jílec a tuk se za čepelí zavřel, protože mu dýku z břicha nevytáhl. Vyšel ven záchodem
#3:23 a prošel chodbou; dveře vrchního pokoje však za sebou zavřel na zástrčku.
#3:24 Sotva vyšel, přišli Eglónovi služebníci. Když viděli, že dveře vrchního pokoje jsou zavřeny na zástrčku, řekli si: „Jistě vykonává v chladném pokoji svou potřebu.“
#3:25 Načekali se až hanba, nikdo však dveře vrchního pokoje neotvíral. Proto vzali klíč, otevřeli, a hle, jejich pán leží na zemi mrtev.
#3:26 Zatímco váhali, Ehúd unikl, již minul modly a unikl do Seíry.
#3:27 Jakmile přišel, dal troubit po Efrajimském pohoří na polnici. Izraelci vytáhli s ním dolů z pohoří a on jim byl v čele.
#3:28 Vyzval je: „Pospěšte za mnou! Hospodin vám vydal do rukou vaše nepřátele Moábce!“ Sestoupili za ním a dobyli na Moábu jordánské brody. Nikomu nedovolili přejít.
#3:29 Toho času pobili z Moába asi deset tisíc mužů, vesměs významných a udatných; nikdo z nich neunikl.
#3:30 Onoho dne se Moáb musel před Izraelem pokořit. Země žila v míru po osmdesát let.
#3:31 Po něm byl Šamgar, syn Anatův. Pobil Pelištejce, šest set mužů, bodcem na pohánění dobytka. Tak i on vysvobodil Izraele. 
#4:1 Po Ehúdově smrti se Izraelci dále dopouštěli toho, co je zlé v Hospodinových očích.
#4:2 I vydal je Hospodin napospas Jabínovi, králi kenaanskému, který kraloval v Chasóru. Velitelem jeho vojska byl Sísera; ten sídlil v Charóšetu pronárodů.
#4:3 I úpěli Izraelci k Hospodinu, protože Jabín měl devět set železných vozů a po dvacet let Izraelce krutě utlačoval.
#4:4 Toho času v Izraeli soudila prorokyně Debóra, žena Lapidótova.
#4:5 Sedávala pod Debóřinou palmou mezi Rámou a Bét-elem v Efrajimském pohoří a Izraelci za ní přicházeli, aby je soudila.
#4:6 Ta poslala pro Báraka, syna Abínoamova, z neftalíjské Kedeše, a naléhala na něho: „Sám Hospodin, Bůh Izraele, ti přikazuje: ‚Táhni hned na horu Tábor a vezmi s sebou deset tisíc mužů z Neftalíovců a Zabulónovců.
#4:7 Já k tobě přivedu k potoku Kíšonu velitele Jabínova vojska Síseru i jeho vozbu a jeho hlučící dav a dám ti jej do rukou.‘“
#4:8 Bárak jí odpověděl: „Půjdeš-li se mnou, půjdu, nepůjdeš-li se mnou, nepůjdu.“
#4:9 Řekla: „Určitě s tebou půjdu, avšak na cestě, kterou půjdeš, se neproslavíš, Hospodin totiž vydá Síseru do rukou ženy.“ I vstala Debóra a vypravila se s Bárakem do Kedeše.
#4:10 Bárak svolal do Kedeše Zabulóna i Neftalího; táhlo za ním deset tisíc mužů, i Debóra táhla s ním.
#4:11 A Kénijec Cheber se odloučil od Kajina, od potomků Chóbaba, tchána Mojžíšova, a přemístil svůj stan k božišti v Saanajimu u Kedeše.
#4:12 Když ohlásili Síserovi, že Bárak, syn Abínoamův, vystoupil na horu Tábor,
#4:13 svolal celou svou vozbu, devět set železných vozů, i všechen lid, který měl v pohotovosti, z Charóšetu pronárodů k potoku Kíšonu.
#4:14 Debóra vyzvala Báraka: „Připrav se, toto je den, kdy ti Hospodin vydal Síseru do rukou. Hospodin sám vytáhl před tebou.“ I sestoupil Bárak z hory Táboru a za ním deset tisíc mužů.
#4:15 Hospodin ostřím meče uvedl ve zmatek Síseru a všechnu jeho vozbu a celý tábor před Bárakem. Sísera seskočil z vozu a prchal pěšky.
#4:16 Bárak pronásledoval vozbu a tábor až k Charóšetu pronárodů. Celý Síserův tábor padl ostřím meče, nezůstal ani jediný.
#4:17 Sísera prchal pěšky ke stanu Jáely, ženy Kénijce Chebera, neboť mezi chasórským králem Jabínem a domem Kénijce Chebera byl mír.
#4:18 Jáel vyšla Síserovi vstříc a zvala jej: „Uchyl se, můj pane, uchyl se ke mně, neboj se!“ Uchýlil se k ní do stanu a ona ho přikryla houní.
#4:19 Poprosil ji: „Dej mi prosím napít trochu vody, mám žízeň.“ Otevřela měch s mlékem, dala mu napít a přikryla ho.
#4:20 Nato ji požádal: „Stůj u vchodu do stanu, a kdyby někdo přišel a ptal se tě: ‚Je zde někdo?‘ odpověz: ‚Není.‘“
#4:21 I uchopila Jáel, žena Cheberova, stanový kolík, vzala do ruky kladivo, přikradla se k němu a vrazila mu stanový kolík do spánku, že pronikl až do země. On totiž tvrdě spal, protože byl unaven. Tak zemřel.
#4:22 A tu, když Bárak pronásledoval Síseru, vyběhla mu Jáel vstříc a volala na něho: „Pojď, ukážu ti muže, kterého hledáš.“ Vstoupil k ní, a hle, Sísera leží mrtev a v jeho spánku vězí stanový kolík.
#4:23 Onoho dne pokořil Bůh kenaanského krále Jabína před syny Izraele.
#4:24 Ruka Izraelců začala postupně tvrdě doléhat na kenaanského krále Jabína, až i Jabína, krále kenaanského, zničili. 
#5:1 Onoho dne zpívala Debóra a Bárak, syn Abínoamův:
#5:2 Vlas bojovníků v Izraeli volně vlaje, že se dobrovolně sešel lid, dobrořečte Hospodinu!
#5:3 Slyšte, králové, poslouchejte, hodnostáři, já budu pět Hospodinu, já mu budu zpívat, zpívat žalmy Hospodinu, Bohu Izraele.
#5:4 Hospodine, když jsi táhl ze Seíru, když jsi vykročil z Edómského pole, země se třásla, z nebes kanuly krůpěje, ano, z oblaků kanuly krůpěje vod.
#5:5 Hory se zapotácely před Hospodinem, Bohem ze Sínaje, před Hospodinem, Bohem Izraele.
#5:6 Ve dnech Šamgara, syna Anatova, ve dnech Jáeliných byly opuštěny stezky. Kdo se vydávali na cesty, vydávali se po stezkách křivolakých.
#5:7 Opuštěn byl venkov, pusto bylo v Izraeli, až jsem povstala já, Debóra, povstala jsem jako matka v Izraeli.
#5:8 Kdykoli si lid volil nové bohy, rozpoutal se v branách boj; ale ukázal se štít nebo kopí mezi čtyřiceti tisíci v Izraeli?
#5:9 Mé srdce je s těmi, kdo třímají v Izraeli palcát, při dobrovolnících v lidu; dobrořečte Hospodinu!
#5:10 Vy, kteří jezdíte na bělavých oslicích, kdo sedáváte na kobercích, kdo se ubíráte cestou, rozvažujte!
#5:11 Vzdáleni hluku střelců, mezi napajedly, tam ať opěvují spravedlivé činy Hospodina, spravedlivé činy jeho vojevůdce v Izraeli, když lid Hospodinův sestoupil k branám.
#5:12 Procitni, procitni, Debóro, procitni, procitni, píseň pěj! Povstaň, Báraku, odveď své zajatce, synu Abínoamův!
#5:13 Tehdy ten z lidu, kdo vyvázl, pošlapal urozené; Hospodin mi dal pošlapat bohatýry.
#5:14 Z Efrajima ti, kdo z něho vzešli, bojovali s Amálekem, Benjamíne, za tebou táhly tvé zástupy, z Makíra sestoupili ti, kdo třímají palcát, ze Zabulóna ti, kdo nosí velitelskou hůl.
#5:15 Isacharští velmožové šli s Debórou, Isachar hned za Bárakem, vyslán do doliny, šel mu v patách. Avšak u potoků Rúbenových, bylo velké zasedání.
#5:16 Proč jsi zůstal sedět mezi ohradami? Abys naslouchal, jak svolávají stáda? U potoků Rúbenových bylo velké rokování.
#5:17 Gileád si zůstal za Jordánem. A proč Dan pobývá jako host na lodích? Ašer se usadil na pobřeží moře, při svých zálivech si zůstal.
#5:18 Zabulón je lid, který dokázal nasadit svůj život, rovněž Neftalí na výšinách pole.
#5:19 Přišli králové a bojovali, tehdy bojovali kenaanští králové v Taanaku, při Vodách megidských, stříbra se však nezmocnili.
#5:20 Z nebes bojovaly hvězdy, bojovaly ze svých drah se Síserou.
#5:21 Potok Kíšon odplavil ty krále, potok prastarý, ten potok Kíšon. Jen je mocně pošlapej, má duše!
#5:22 Tehdy dupala kopyta koní, stateční cválali ostrým tryskem.
#5:23 Proklejte Meróz, velí Hospodinův posel, uvalte na jeho obyvatele kletbu: Nepřišli na pomoc Hospodinu, na pomoc Hospodinu proti bohatýrům.
#5:24 Požehnána buď nad jiné ženy Jáel, žena Kénijce Chebera, nad jiné ženy ve stanech buď požehnána!
#5:25 Požádal o vodu, poskytla mléko, v koflíku urozených podala smetanu.
#5:26 Rukou chopila stanový kolík, pravicí pádné kladivo, udeřila Síseru, roztříštila mu hlavu, probodla a prorazila jeho spánky.
#5:27 Klesl jí k nohám, padl, zůstal ležet, k nohám jí klesl, padl; tam klesl a padl zabitý.
#5:28 Síserova matka vyhlížela z okna, naříkala za okenní mříží: „Proč tak dlouho nepřijíždí jeho vůz? Proč mešká hřmot jeho vozby?“
#5:29 Odpověděly jí nejmoudřejší z jejích kněžen a ona si teď opakuje jejich slova:
#5:30 „Určitě získali a rozdělují kořist, jednu dvě zajatkyně každému hrdinovi, kořist pro Síseru - pestré šaty, kořist, pestré šaty, jeden dva pestré šátky na krk co kořist.“
#5:31 Tak ať zhynou všichni tvoji nepřátelé, Hospodine! Ale ti, kdo jej milují, budou jako slunce vycházející v plné síle. - Země žila v míru po čtyřicet let. 
#6:1 Synové Izraele se dále dopouštěli toho, co je zlé v Hospodinových očích. Proto je Hospodin vydal na sedm let do rukou Midjánců.
#6:2 Ruka Midjánců na Izraele mocně doléhala. Izraelci si dělali před Midjánci úkryty na horách, v soutěskách, v jeskyních a na nepřístupných vrcholcích.
#6:3 Sotvaže Izrael zasel, přitáhli Midjánci s Amálekem a syny východu a přepadali jej.
#6:4 Položili se proti nim a ničili úrodu země až po cestu do Gázy. Neponechávali v Izraeli k obživě ani ovci ani býka ani osla.
#6:5 Přitáhli vždy se svými stády a stany; přihnali se jako kobylky v takovém množství, že nebylo možno spočítat je ani jejich velbloudy. Přicházeli do země, aby ji ničili.
#6:6 Izrael byl od Midjánců úplně zbídačen. I úpěli Izraelci k Hospodinu.
#6:7 Když Izraelci tak úpěli k Hospodinu kvůli Midjáncům,
#6:8 poslal k nim proroka, aby jim řekl: „Toto praví Hospodin, Bůh Izraele: Já jsem vás přivedl z Egypta, vyvedl jsem vás z domu otroctví.
#6:9 Vyprostil jsem vás z moci Egypta i z rukou všech vašich utlačovatelů. Vypudil jsem je před vámi a jejich zemi jsem dal vám.
#6:10 Řekl jsem vám: Já jsem Hospodin, váš Bůh. Nebojte se bohů Emorejců, v jejichž zemi sídlíte. Ale vy jste mě neuposlechli.“
#6:11 I přišel Hospodinův posel a posadil se v Ofře pod posvátným stromem, který náležel Jóašovi Abíezerskému; jeho syn Gedeón mlátil pšenici v lisu, aby s ní utekl před Midjánci.
#6:12 Hospodinův posel se mu ukázal a oslovil jej: „Hospodin s tebou, udatný bohatýre!“
#6:13 Gedeón mu odpověděl: „Dovol, můj pane, je-li s námi Hospodin, proč nás tohle všechno potkává? Kde jsou všechny jeho podivuhodné činy, o nichž nám vypravovali naši otcové? Říkali: ‚Což nás Hospodin nevyvedl z Egypta?‘ Ale teď nás Hospodin zavrhl a vydal do rukou Midjánců.“
#6:14 Hospodin se k němu obrátil a pravil: „Jdi v této své síle a vysvobodíš Izraele z rukou Midjánců. Hle, já tě posílám.“
#6:15 On mu však namítl: „Dovol prosím, Panovníku, jak bych mohl Izraele vysvobodit? Můj rod je v Manasesovi nejslabší a já jsem v otcovském domě nejnepatrnější.“
#6:16 Ale Hospodin mu řekl: „Protože já budu s tebou, pobiješ Midjánce jako jediného muže.“
#6:17 Tu jej požádal: „Jestliže jsem opravdu našel u tebe milost, dej mi nějaké znamení, že se mnou mluvíš ty sám.
#6:18 Jenom se odtud nevzdaluj, dokud k tobě nepřijdu. Rád bych přinesl dar a předložil ti jej.“ Odpověděl: „Zůstanu, dokud se nevrátíš.“
#6:19 Gedeón odešel a připravil kůzle a nekvašené chleby z éfy mouky. Maso vložil do košíku, vývar nalil do hrnce, vynesl to k němu pod posvátný strom a nabídl mu to.
#6:20 Posel Boží mu poručil: „Vezmi maso a nekvašené chleby a polož je na toto skalisko; vývar vylej.“ On tak učinil.
#6:21 Hospodinův posel se dotkl koncem napřažené hole, kterou měl v ruce, masa a nekvašených chlebů. I vyšlehl ze skály oheň a pohltil maso i nekvašené chleby, zatímco Hospodinův posel mu zmizel z očí.
#6:22 Tu Gedeón shledal, že to byl Hospodinův posel, a zvolal: „Běda mi, Panovníku Hospodine, vždyť jsem viděl Hospodinova posla tváří v tvář!“
#6:23 Hospodin jej však uklidnil: „Pokoj tobě; neboj se, nezemřeš.“
#6:24 I vybudoval tam Gedeón Hospodinu oltář a nazval jej: „Hospodin je pokoj.“ Je v abíezerské Ofře až dodnes.
#6:25 Té noci mu Hospodin poručil: „Vezmi býčka, který patří tvému otci, toho druhého býka, sedmiletého. Zboříš Baalův oltář, který patří tvému otci, a skácíš posvátný kůl, který je u něho.
#6:26 Pak zbuduješ podle řádu na vrcholu tohoto kopce oltář Hospodinu, svému Bohu, vezmeš toho druhého býka a budeš jej obětovat jako zápalnou oběť na dříví z posvátného kůlu, který jsi skácel.“
#6:27 Gedeón přibral deset mužů ze svých služebníků a vykonal, k čemu jej vyzval Hospodin. Protože se však bál domu svého otce a mužů města, nevykonal to ve dne, nýbrž v noci.
#6:28 Za časného jitra mužové města vstali, a hle, Baalův oltář byl rozbořen, posvátný kůl, který byl u něho, byl skácen a druhý býk byl obětován na zbudovaném oltáři.
#6:29 Vyptávali se jeden druhého: „Kdo tohle udělal?“ Pátrali a hledali, až zjistili, že to udělal Gedeón, syn Jóašův.
#6:30 Tu vyzvali mužové města Jóaše: „Vydej svého syna; musí zemřít. Rozbořil Baalův oltář a skácel posvátný kůl, který byl u něho.“
#6:31 Jóaš však odpověděl všem, kteří stáli proti němu: „To vy chcete vést spor za Baala? Copak vy ho zachráníte? Kdo za něj chce vést spor, ať do jitra zemře! Je-li Baal bohem, ať si sám vede svůj spor, jemu přece rozbořil oltář.“
#6:32 Toho dne ho nazval Jerubaalem (to je Odpůrce Baalův); řekl: „Ať si Baal proti němu vede spor, jemu přece rozbořil oltář!“
#6:33 Celý Midján spolu s Amálekem a syny východu se spojili, přešli Jordán a utábořili se v dolině Jizreelu.
#6:34 Avšak Gedeóna vyzbrojil duch Hospodinův a on zatroubil na polnici, a tak svolal k sobě Abíezerovce.
#6:35 Rozeslal posly po celém Manasesovi, takže i ten byl k němu přivolán. Rozeslal také posly k Ašerovi, Zabulónovi a Neftalímu a oni jim vytáhli naproti.
#6:36 Potom Gedeón prosil Boha: „Chceš mou rukou vysvobodit Izraele, jak jsi prohlásil?
#6:37 Hle, rozprostírám na humně ovčí rouno. Bude-li rosa jenom na rouně a všude po zemi bude sucho, poznám, že mou rukou vysvobodíš Izraele, jak jsi řekl.“
#6:38 Tak se také stalo. Nazítří za časného jitra rouno vyždímal a vytlačil z něho plný koflík rosy.
#6:39 Gedeón dále prosil Boha: „Nechť proti mně nevzplane tvůj hněv, když promluvím ještě jednou. Rád bych to s rounem zkusil ještě jednou. Kéž je jenom rouno suché a všude po zemi rosa!“
#6:40 A Bůh to tak té noci učinil. Jenom rouno bylo suché, zatímco všude po zemi byla rosa. 
#7:1 Za časného jitra se Jerubaal, to je Gedeón, a všechen lid, který byl s ním, utábořili u pramene Charódu. Tábor Midjánců byl od něho na sever, v dolině za návrším Móre.
#7:2 I řekl Hospodin Gedeónovi: „Je s tebou příliš mnoho lidu, než abych jim vydal Midjánce do rukou, aby se Izrael vůči mně nevychloubal: ‚Vysvobodil jsem se vlastní rukou.‘
#7:3 Nuže, provolej teď k lidu: Kdo se bojí a třese, ať se z Gileádského pohoří vrátí a vzdálí.“ Vrátilo se dvaadvacet tisíc mužů z lidu, zůstalo jich jen deset tisíc.
#7:4 Hospodin však Gedeónovi řekl: „Ještě je lidu mnoho. Poruč jim, ať sestoupí k vodě, tam ti je vyzkouším. O kom ti řeknu: Půjde s tebou, ten s tebou půjde. Ale nesmí jít s tebou nikdo, o kom ti řeknu: Ten s tebou nepůjde.“
#7:5 Poručil tedy lidu sestoupit k vodě. Hospodin řekl Gedeónovi: „Postavíš zvlášť každého, kdo bude chlemtat vodu jazykem jako pes, a každého, kdo si při pití klekne na kolena.“
#7:6 Těch, kteří chlemtali a podávali si vodu rukou k ústům, bylo celkem tři sta mužů. Všechen ostatní lid klekal při pití vody na kolena.
#7:7 Hospodin řekl Gedeónovi: „Třemi sty muži, kteří chlemtali, vás vysvobodím a vydám ti Midjánce do rukou. Všechen ostatní lid ať odejde, každý do svého domova.“
#7:8 Lid si tedy nabral příděl potravy a svoje polnice a Gedeón propustil všechny izraelské muže, každého k jeho stanu; jen těch tři sta si podržel. Tábor Midjánců ležel pod ním v dolině.
#7:9 Té noci mu Hospodin poručil: „Ihned sestup do tábora nepřátel, neboť jsem ti je vydal do rukou.
#7:10 Jestliže se bojíš sestoupit sám, sestup do tábora se svým mládencem Púrou.
#7:11 Uslyšíš, o čem budou mluvit. Pak jednej rozhodně a vtrhni dolů do tábora.“ Gedeón sestoupil se svým mládencem Púrou k vojenským hlídkám na pokraji tábora.
#7:12 Midjánci s Amálekem a se všemi syny východu leželi totiž v dolině v takovém množství jako kobylky, i jejich velbloudů byl bezpočet, takové množství jako písku na mořském břehu.
#7:13 Gedeón přišel, právě když jeden druhému vyprávěl sen. Povídal: „Považ, jaký jsem to měl sen! Na midjánský tábor se valil pecen ječného chleba; přivalil se ke stanu a vrazil do něho, až padl, úplně jej převrátil, že stan zůstal ležet.“
#7:14 Jeho druh mu odpověděl: „To nemůže znamenat nic jiného než meč Izraelce Gedeóna, syna Jóašova. Bůh mu dal do rukou Midjánce i s celým táborem.“
#7:15 Když Gedeón slyšel vyprávění tohoto snu i jeho výklad, poklonil se. Vrátil se do izraelského tábora a zvolal: „Vstaňte! Hospodin vám vydal do rukou tábor Midjánců!“
#7:16 Těch tři sta mužů rozdělil do tří oddílů, všem jim dal do rukou polnice a prázdné džbány a do džbánů pochodně.
#7:17 Poručil jim: „Sledujte mě a dělejte, co já. Hle, půjdu na okraj tábora, a co udělám, udělejte i vy.
#7:18 Až spolu s ostatními, kteří budou se mnou, zatroubím na polnici, zatroubíte i vy na polnice okolo celého tábora a zvoláte: ‚Za Hospodina a za Gedeóna!‘“
#7:19 Gedeón se sto muži, kteří byli s ním, přišli na okraj tábora na počátku prostřední noční hlídky, právě když střídali stráže. Zatroubili na polnice a rozbili džbány, které měli v rukou.
#7:20 Tři oddíly najednou zatroubily na polnice a rozbily džbány; do levé ruky uchopili pochodně, do pravé polnice, aby troubili, a zvolali: „Meč za Hospodina a za Gedeóna!“
#7:21 Zůstali stát kolem tábora, každý na svém místě. V celém táboře nastal divý spěch, zmateně pokřikovali a dávali se na útěk.
#7:22 Zatímco tři sta mužů troubilo na polnice, Hospodin obrátil v celém táboře meč jednoho proti druhému; tábor se dal na útěk do Bét-šity směrem k Seréře a k ábelmechólskému břehu naproti Tabatu.
#7:23 Tu byli svoláni Izraelci z Neftalího, Ašera a celého Manasesa, aby Midjánce pronásledovali.
#7:24 Také po celém Efrajimském pohoří rozeslal Gedeón posly s výzvou: „Sestupte dolů proti Midjáncům a obsaďte jordánské vody až k Bét-baře.“ Všichni Efrajimci byli přivoláni a obsadili jordánské vody až k Bét-baře.
#7:25 Přitom zajali dva midjánské velmože, Oréba (to je Havrana) a Zéba (to je Vlka). Oréba zabili na Havraní skále a Zéba zabili ve Vlčím lisu, a pronásledovali Midjánce. Hlavu Orébovu a Zébovu přinesli Gedeónovi za Jordán. 
#8:1 Efrajimští muži vytýkali Gedeónovi: „Cos nám to udělal, že jsi nás nepovolal, když jsi táhl do boje proti Midjáncům?“ A ostře se s ním přeli.
#8:2 Ale on jim odpověděl: „Copak jsem teď učinil něco tak velikého jako vy? Což není Efrajimovo paběrkování lepší než Abíezerovo vinobraní?
#8:3 Do vašich rukou vydal Bůh midjánské velmože Oréba a Zéba. Mohl jsem učinit něco tak velikého jako vy?“ Když k nim promluvil toto slovo, jejich pobouření proti němu polevilo.
#8:4 Potom přišel Gedeón k Jordánu a přešel jej spolu se třemi sty muži, kteří byli s ním, unavenými pronásledováním.
#8:5 Vyzval muže ze Sukótu: „Dejte prosím několik bochníků chleba lidu, který jde za mnou. Jsou unaveni. Pronásleduji midjánské krále Zebacha a Salmunu.“
#8:6 Sukótští velmožové mu však odpověděli: „To už je ruka Zebachova a Salmunova v tvých rukou, že máme dát chleba tvému voji?“
#8:7 Gedeón prohlásil: „Za tohle zmrskám vaše těla stepním trním a bodláčím, jen co mi dá Hospodin Zebacha a Salmunu do rukou.“
#8:8 Odtud táhl do Penúelu a promluvil k nim stejně, avšak penúelští mužové mu odpověděli právě tak, jako odpověděli sukótští.
#8:9 Řekl tedy penúelským mužům: „Až se budu vracet v pokoji, rozbořím tuto věž.“
#8:10 Zebach a Salmuna byli i se svým vojskem v Karkóru, asi s patnácti tisíci muži, se všemi, kteří zbyli z celého tábora synů východu; padlo sto dvacet tisíc mužů ozbrojených mečem.
#8:11 Gedeón táhl cestou kočovníků východně od Nóbachu a Jogbohy a přepadl tábor, právě když se tábor cítil v bezpečí.
#8:12 Zebach a Salmuna utíkali a on je pronásledoval. Oba midjánské krále, Zebacha a Salmunu, chytil a celý tábor vyděsil.
#8:13 Když se Gedeón, syn Jóašův, vracel z bitvy naproti Slunečnímu svahu,
#8:14 chytil jednoho mládence ze sukótských mužů a vyslýchal ho. On mu sepsal sukótské velmože a starší, sedmdesát sedm mužů.
#8:15 Potom přišel k sukótským mužům a řekl: „Tu je Zebach a Salmuna, kvůli nimž jste mě potupili slovy: ‚To už je ruka Zebachova a Salmunova v tvých rukou, že máme dát tvým unaveným mužům chleba?‘
#8:16 Vzal na stařešiny města stepní trní a bodláčí; tak dal sukótským mužům poučení.
#8:17 Potom rozbořil penúelskou věž a muže města pobil.
#8:18 Zebacha a Salmuny se otázal: „Co to bylo za muže, které jste povraždili na Táboru?“ Odpověděli: „Byli podobní tobě, každý vypadal jako kralevic.“
#8:19 Nato řekl: „To byli moji bratři, synové mé matky. Jakože živ je Hospodin, kdybyste je byli nechali naživu, nezabil bych vás.“
#8:20 Poručil svému prvorozenému synu Jeterovi: „Vstaň a zabij je!“ Ale mládenec nevytasil meč, protože se bál, byl to ještě chlapec.
#8:21 Tu řekli Zebach a Salmuna: „Vstaň sám a zasaď nám úder, vždyť jaký muž, taková jeho síla.“ Gedeón tedy vstal a zabil Zebacha a Salmunu a pobral měsíčky, které měli jejich velbloudi na krku.
#8:22 Izraelští muži požádali potom Gedeóna: „Buď naším vladařem, ty i tvůj syn i vnuk, protože jsi nás vysvobodil z rukou Midjánců.“
#8:23 Ale Gedeón je odmítl: „Nebudu vaším vladařem, ani můj syn nebude vaším vladařem. Nad vámi bude vládnout Hospodin!“
#8:24 Dále jim Gedeón řekl: „Rád bych vás o něco požádal. Každý mi dejte ze své kořisti jeden nosní kroužek.“ Nepřátelé totiž měli zlaté nosní kroužky, byli to Izmaelci.
#8:25 Odpověděli: „Rádi dáme.“ Rozprostřeli plášť a každý tam hodil nosní kroužek ze své kořisti.
#8:26 Váha zlatých nosních kroužků, které si vyžádal, byla tisíc sedm set šekelů zlata, kromě měsíčkovitých přívěsků, náušnic a purpurových rouch, která měli na sobě midjánští králové, a kromě ozdob, které měli na krku jejich velbloudi.
#8:27 Gedeón z toho udělal efóda vystavil jej ve svém městě, v Ofře. Tam za ním chodil smilnit celý Izrael, takže se stal Gedeónovi i jeho domu léčkou.
#8:28 Midján však byl před Izraelci pokořen a už se nikdy nevzpamatoval. Za dnů Gedeónových žila země v míru po čtyřicet let.
#8:29 Jerubaal, syn Jóašův, pak odešel a usadil se ve svém domě.
#8:30 Gedeón měl sedmdesát synů, kteří vzešli z jeho beder; měl totiž mnoho žen.
#8:31 Také jeho ženina mu v Šekemu porodila syna; jemu dal jméno Abímelek.
#8:32 I zemřel Gedeón, syn Jóašův, v utěšeném stáří a byl pochován v hrobě svého otce Jóaše v abíezerské Ofře.
#8:33 Jakmile však Gedeón zemřel, Izraelci opět chodili smilnit k baalům a Baala smlouvy dokonce prohlásili za svého boha.
#8:34 Nepřipomínali si Hospodina, svého Boha, který je vytrhl z rukou všech jejich okolních nepřátel.
#8:35 Ani domu Jerubaala-Gedeóna neprojevovali vděčnost za všechno dobro, které pro Izraele vykonal. 
#9:1 Abímelek, syn Jerubaalův, odešel do Šekemu, k bratrům své matky a promluvil k nim i k celé čeledi rodu své matky takto:
#9:2 „Předneste všem šekemským občanům: Co je pro vás lepší, aby nad vámi vládlo sedmdesát mužů, samí Jerubaalovci, anebo aby nad vámi vládl jediný muž? Pamatujte, že jsem vaše krev a vaše tělo.“
#9:3 Bratři jeho matky přednesli o něm toto všechno všem šekemským občanům a ti se k Abímelekovi přiklonili, neboť si řekli: „Je to náš bratr.“
#9:4 Potom mu vydali sedmdesát šekelů stříbra z chrámu Baala smlouvy; za ně si Abímelek najal lehkomyslné a bezohledné muže, aby ho provázeli.
#9:5 Vešel do otcovského domu v Ofře a povraždil na jednom kameni své bratry Jerubaalovce, sedmdesát mužů. Zbyl jen Jótam, nejmladší syn Jerubaalův, protože se ukryl.
#9:6 Všichni šekemští občané a všichni z domu Miló se shromáždili a prohlásili Abímeleka za krále při božišti u sloupu, který byl v Šekemu.
#9:7 Když to oznámili Jótamovi, šel a postavil se na vrcholu hory Gerizímu, hlasitě volal a mluvil k nim: „Slyšte mě, šekemští občané, a vás nechť slyší Bůh!
#9:8 Sešly se pospolu stromy, aby si nad sebou pomazaly krále. Vyzvaly olivu: ‚Kraluj nad námi!‘
#9:9 Oliva jim však odpověděla: ‚Mám se vzdát své tučnosti, jíž se uctívají bohové i lidé, a kymácet se nad stromy?‘
#9:10 Stromy pak vyzvaly fík: ‚Pojď nad námi kralovat ty!‘
#9:11 Fík jim však odpověděl: ‚Mám se vzdát své sladkosti a svých výborných plodů a kymácet se nad stromy?‘
#9:12 Stromy pak vyzvaly vinnou révu: ‚Pojď nad námi kralovat ty!‘
#9:13 Vinná réva jim však odpověděla: ‚Mám se vzdát svého moštu, který je k radosti bohům i lidem, a kymácet se nad stromy?‘
#9:14 I vyzvaly všechny stromy trnitý keř: ‚Pojď nad námi kralovat ty!‘
#9:15 A trnitý keř stromům odpověděl: ‚Jestliže mě doopravdy chcete pomazat za krále nad sebou, pojďte se schoulit do mého stínu! Jestliže však ne, vyšlehne z trnitého keře oheň a pozře i libanónské cedry!‘
#9:16 Nuže: Jednali jste věrně a bezelstně, když jste si ustanovili za krále Abímeleka? Jednali jste dobře s Jerubaalem a s jeho domem? Jednali jste s ním podle toho, jak si zasloužil?
#9:17 Můj otec přece za vás bojoval s nasazením vlastního života a vytrhl vás z rukou Midjánců.
#9:18 Vy však jste dnes proti domu mého otce povstali, povraždili jste jeho syny, sedmdesát mužů, na jednom kameni, a ustanovili jste Abímeleka, syna jeho otrokyně, za krále šekemských občanů jen proto, že to je váš bratr.
#9:19 Jestliže jste tohoto dne jednali s Jerubaalem a s jeho domem věrně a bezelstně, radujte se z Abímeleka a také on ať se raduje z vás.
#9:20 Jestliže však nikoli, ať vyšlehne z Abímeleka oheň a pozře šekemské občany i dům Miló, pak ať vyšlehne oheň z šekemských občanů a z domu Miló a pozře Abímeleka.“
#9:21 Poté se dal Jótam na útěk a uprchl před svým bratrem Abímelekem do Beéru, kde se usadil.
#9:22 Abímelek panoval nad Izraelem tři roky.
#9:23 Bůh poslal mezi Abímeleka a šekemské občany zlého ducha a šekemští občané se vůči Abímelekovi zachovali věrolomně.
#9:24 Tak násilí spáchané na sedmdesáti Jerubaalovcích a jejich prolitá krev dopadly na Abímeleka, jejich bratra, který je povraždil, i na šekemské občany za to, že ho podporovali, když vraždil své bratry.
#9:25 Šekemští občané postavili proti němu zálohy na vrcholcích hor a oloupili každého, kdo se ubíral cestou mimo ně. Bylo to hlášeno Abímelekovi.
#9:26 Tenkrát přišel Gaal, syn Ebedův, se svými bratry a dorazili do Šekemu. Šekemští občané k němu pojali důvěru.
#9:27 Vyšli na pole a sbírali na svých vinicích hrozny, lisovali je a uspořádali slavnost vinobraní, pak vešli do domu svého boha, hodovali, pili a zlořečili Abímelekovi.
#9:28 Přitom mluvil Gaal, syn Ebedův: „Kdo je Abímelek ve srovnání s Šekemem, že mu máme sloužit? Což to není syn Jerubaalův? A není u něho dozorcem Zebúl? Služte raději mužům Chamóra, otce Šekemova. Proč bychom my měli sloužit jemu?
#9:29 Kdybych měl v rukou tento lid, Abímeleka bych odstranil.“ Vyzval Abímeleka: „Rozmnož své oddíly a vytáhni!“
#9:30 Jak uslyšel Zebúl, velitel města, řeč Gaala, syna Ebedova, vzplanul hněvem
#9:31 a tajně poslal k Abímelekovi posly se vzkazem: „Hle, do Šekemu přišel Gaal, syn Ebedův, se svými bratry a pobuřují proti tobě město.
#9:32 Povstaň teď v noci spolu s lidem, který je s tebou, a ukryj se v poli do zálohy.
#9:33 Za časného jitra při východu slunce přitrhneš k městu. On i jeho lid vyjdou k tobě a ty mu uděláš, co se tvé ruce naskytne.“
#9:34 Abímelek v noci povstal, i všechen lid, který byl s ním, a rozestavili proti Šekemu do čtyř oddílů zálohy.
#9:35 Gaal, syn Ebedův, vyšel a stoupl si do vrat městské brány, právě když Abímelek se svým lidem povstal ze zálohy.
#9:36 Když Gaal uviděl ten lid, řekl Zebúlovi: „Hle, z hřebenů hor sestupuje nějaký lid!“ Zebúl mu odpověděl: „Vidíš stín hor a připadá ti jako mužové.“
#9:37 Ale Gaal mluvil dál: „Hle, nějaký lid sestupuje z Pupku země a jeden oddíl přichází cestou od božiště mrakopravců.“
#9:38 Tu mu řekl Zebúl: „Kdepak jsou teď tvé řeči? Vždyť jsi říkal: ‚Kdo je Abímelek, že bychom mu měli sloužit?‘ Ano, je to ten lid, který jsi tak podceňoval. Jen vytrhni a bojuj proti němu!“
#9:39 Gaal tedy vyšel v čele šekemských občanů, aby bojoval proti Abímelekovi,
#9:40 ale Abímelek ho zahnal na útěk, a mnoho skolených zůstalo ležet až po vrata brány.
#9:41 Abímelek se usadil v Arumě a Zebúl vypudil Gaala a jeho bratry ze Šekemu.
#9:42 Druhého dne vyšel lid na pole. Hlásili to Abímelekovi.
#9:43 Ten rozdělil lid do tří oddílů a ukryl se do zálohy v poli. Když uviděl, že lid vychází z města, zvedl se proti nim, aby je pobil.
#9:44 Abímelek a oddíly, které byly s ním, přitrhli a postavili se do vrat městské brány, kdežto na všechny, kdo byli na poli, přitrhly dva zbývající oddíly a pobily je.
#9:45 Po celý den bojoval Abímelek proti městu. Město dobyl a povraždil lid, který byl v něm, potom je rozbořil a posypal solí.
#9:46 Jak to uslyšeli všichni občané šekemské věže, vešli do sklepení domu boha smlouvy.
#9:47 Abímelekovi bylo hlášeno, že se tam shromáždili všichni občané šekemské věže.
#9:48 I vystoupil Abímelek se vším svým lidem na horu Salmón, uchopil sekeru, usekl ze stromu větev, zvedl ji a dal si ji na rameno. Lidu, který byl s ním, poručil: „Co jste viděli, že jsem udělal, udělejte rychle jako já!“
#9:49 Všechen lid, každý z nich, usekli větev a šli za Abímelekem. Nakladli je na sklepení a zapálili je nad nimi. Tak zemřeli všichni muži šekemské věže, na tisíc mužů a žen.
#9:50 Potom táhl Abímelek do Tebesu, oblehl jej a dobyl.
#9:51 Uprostřed města byla pevná věž, kam utekli všichni mužové a ženy i všichni občané města. Zavřeli za sebou a vystoupili na střechu věže.
#9:52 Abímelek přišel až k věži a bojoval proti ní. Přiblížil se ke vchodu do věže, aby založil požár.
#9:53 Vtom hodila nějaká žena Abímelekovi na hlavu mlýnský kámen a prorazila mu lebku.
#9:54 Rychle zavolal svého zbrojnoše a poručil mu: „Vytas meč a usmrť mě, ať o mně neříkají: ‚Zabila ho žena!‘“ Mládenec jej tedy probodl a on zemřel.
#9:55 Když izraelští muži uviděli, že Abímelek zemřel, vrátil se každý do svého domova.
#9:56 Tak odplatil Bůh Abímelekovi za zločin, jehož se dopustil proti svému otci, když povraždil svých sedmdesát bratrů.
#9:57 Též všechny zločiny šekemských mužů obrátil Bůh na jejich hlavu. Postihlo je zlořečení Jótama, syna Jerubaalova. 
#10:1 Po Abímelekovi povstal k vysvobozování Izraele Tóla, syn Púy, syna Dódova, muž z pokolení Isacharova. Sídlil v Šamíru v Efrajimském pohoří.
#10:2 Soudil Izraele třiadvacet let. Když zemřel, byl pochován v Šamíru.
#10:3 Po něm povstal Jaír Gileádský. Soudil Izraele dvaadvacet let.
#10:4 Měl třicet synů, kteří jezdili na třiceti oslátkách a měli třicet měst; podle nich se nazývají Jaírovy vesnice až podnes; jsou v gileádské zemi.
#10:5 Když Jaír zemřel, byl pochován v Kamónu.
#10:6 Izraelci se dále dopouštěli toho, co je zlé v Hospodinových očích. Sloužili baalům a aštoretám i bohům aramejským a bohům sidónským, též bohům moábským a bohům Amónovců i bohům Pelištejců. Hospodina opustili a nesloužili mu.
#10:7 Hospodin proto vzplanul proti Izraeli hněvem a vydal jej napospas Pelištejcům a Amónovcům.
#10:8 Ti potírali a týrali Izraelce od toho roku po osmnáct let, i všechny Izraelce za Jordánem v zemi emorejské v Gileádu.
#10:9 Amónovci přešli Jordán, aby bojovali také proti Judovi a Benjamínovi i proti domu Efrajimovu. Na Izraele dolehlo veliké soužení.
#10:10 I úpěli Izraelci k Hospodinu: „Zhřešili jsme proti tobě, opustili jsme svého Boha a sloužili baalům.“
#10:11 Hospodin řekl Izraelcům: „Což jsem vás nevysvobodil od Egypťanů, Emorejců, Amónovců a Pelištejců?
#10:12 Také Sidónci, Amálek a Maónci vás utlačovali, ale úpěli jste ke mně a já jsem vás vysvobozoval z jejich rukou.
#10:13 Vy jste mě však zase opustili a sloužíte jiným bohům, proto vás už nevysvobodím.
#10:14 Jděte si úpět k bohům, které jste si vybrali, ať vás v čas vašeho soužení vysvobodí!“
#10:15 Izraelci Hospodinu odpověděli: „Zhřešili jsme. Nalož s námi, jak sám uznáš za vhodné, jenom nás ještě tentokrát vyprosti!“
#10:16 Odstranili ze svého středu cizí bohy a sloužili Hospodinu. I ustrnul se nad bídou Izraele.
#10:17 Amónovci se svolali a utábořili se v Gileádu. Izraelci se také shromáždili a utábořili se v Mispě.
#10:18 Lid a gileádští velmožové se mezi sebou dohodli: „Ten, kdo zahájí boj proti Amónovcům, bude náčelníkem všech obyvatel Gileádu.“ 
#11:1 Jiftách Gileádský byl udatný bohatýr. Byl synem nevěstky, zplozený Gileádem.
#11:2 Gileádovi porodila syny také vlastní žena. Když její synové dospěli, vypudili Jiftácha. Řekli mu: „Nemáš nárok na dědictví v domě našeho otce, protože jsi syn jiné ženy.“
#11:3 Jiftách před svými bratry uprchl a usadil se v zemi Tóbu. Seběhli se k němu lehkomyslní muži a podnikali s ním výpady.
#11:4 Po nějaké době se Amónovci pustili do boje s Izraelem.
#11:5 Když už Amónovci s Izraelem bojovali, vypravili se gileádští starší pro Jiftácha do země Tóbu.
#11:6 Řekli mu: „Pojď, staň se naším vůdcem a budeme bojovat proti Amónovcům.“
#11:7 Jiftách gileádským starším odpověděl: „Což právě vy jste mě z nenávisti nevypudili z otcovského domu? Jak to, že přicházíte ke mně teď, když jste v soužení?“
#11:8 Gileádští starší Jiftáchovi řekli: „Právě proto se teď obracíme na tebe, abys šel s námi a bojoval proti Amónovcům a byl náčelníkem nás, všech obyvatel Gileádu.“
#11:9 Jiftách gileádským starším odpověděl: „Jestliže mě pohnete k návratu, abych bojoval proti Amónovcům, a Hospodin mi je vydá, budu vskutku vaším náčelníkem?“
#11:10 Gileádští starší se Jiftáchovi zavázali: „Ať Hospodin, který to slyší, je mezi námi svědkem, že všechno uděláme podle tvých slov.“
#11:11 Jiftách tedy šel s gileádskými staršími a lid si jej ustanovil za náčelníka a vůdce. Jiftách pak přednesl celou svou záležitost před Hospodinem v Mispě.
#11:12 Potom vyslal Jiftách posly ke králi Amónovců se vzkazem: „Co máš proti mně, že jsi na mě přitáhl, abys bojoval proti mé zemi?“
#11:13 Král Amónovců Jiftáchovým poslům odpověděl: „Izrael přece při svém tažení z Egypta zabral mou zemi od Arnónu až k Jaboku, ba až k Jordánu. Vydej ji nyní po dobrém.“
#11:14 Jiftách vyslal posly znovu ke králi Amónovců
#11:15 a vzkázal mu: „Toto praví Jiftách: Izrael nezabral zemi Moábců ani zemi Amónovců.
#11:16 Když táhl Izrael z Egypta, ubíral se pouští k Rákosovému moři, až přišel do Kádeše.
#11:17 Tenkrát vyslal Izrael posly k edómskému králi se žádostí: ‚Dovol, abych prošel tvou zemí‘, ale edómský král nechtěl ani slyšet. Poslal také k moábskému králi, ale ani on nesvolil. Proto zůstal Izrael v Kádeši.
#11:18 Potom táhl pouští a obešel edómskou a moábskou zemi, až dospěl na východ od moábské země. Utábořili se na druhém břehu Arnónu, vůbec na moábské území nevstoupili; Arnón tvoří totiž hranici Moábska.
#11:19 Izrael pak poslal posly k emorejskému králi Síchonovi, králi Chešbónu, a požádal ho: ‚Dovol, abychom prošli tvou zemí ke svému cíli!‘
#11:20 Síchon však Izraeli nedůvěřoval, že by jeho územím jen prošel. Proto shromáždil všechen svůj lid, utábořili se v Jahse a zahájil s Izraelem boj.
#11:21 Ale Hospodin, Bůh Izraele, vydal Izraeli do rukou Síchona i všechen jeho lid, takže je pobili. Tak Izrael obsadil celou zemi Emorejců, kteří v té zemi sídlili.
#11:22 Obsadili celé území Emorejců od Arnónu až k Jaboku a od pouště až k Jordánu.
#11:23 Nuže, Hospodin, Bůh Izraele, vyhnal Emorejce před svým izraelským lidem. A ty bys to území chtěl obsadit?
#11:24 Jistě, můžeš obsadit zemi toho, koho ti podrobí tvůj bůh Kemóš. Ale zemi každého, koho před námi vyžene Hospodin, náš Bůh, obsadíme my.
#11:25 Nuže, jsi snad lepší než moábský král Balák, syn Sipórův? Pustil se on s Izraelem do sporu nebo do boje?
#11:26 Izrael sídlí v Chešbónu a jeho vesnicích, v Aróeru a jeho vesnicích i ve všech městech při Arnónu již tři sta let. Proč jste je za tu dobu nezabrali?
#11:27 Pokud jde o mne, já jsem se proti tobě neprohřešil, avšak ty se dopouštíš zlého vůči mně, když se chystáš proti mně bojovat. Nechť dnes Hospodin rozhodne mezi Izraelci a Amónovci jako soudce!“
#11:28 Ale král Amónovců nechtěl slyšet vzkaz, který mu Jiftách poslal.
#11:29 Tu spočinul na Jiftáchovi duch Hospodinův a on táhl Gileádem a Manasesem do Mispy gileádské a z Mispy gileádské táhl proti Amónovcům.
#11:30 Jiftách složil Hospodinu slib: „Vydáš-li mi Amónovce opravdu do rukou,
#11:31 ten, kdo mi vyjde naproti z vrat mého domu, až se budu vracet v pokoji od Amónovců, bude patřit Hospodinu a toho obětuji v zápalnou oběť.“
#11:32 Nato táhl Jiftách do boje proti Amónovcům a Hospodin mu je vydal do rukou.
#11:33 Připravil jim zdrcující porážku mezi Aróerem a cestou do Minítu a až po Ábel-keramím, totiž dvaceti městům. Tak byli Amónovci před syny Izraele pokořeni.
#11:34 Když přicházel Jiftách do Mispy ke svému domu, hle, vychází mu naproti s bubínky a s tancem jeho dcera. Měl jenom tu jedinou, kromě ní neměl syna ani dceru.
#11:35 Jak ji uviděl, roztrhl své roucho a zvolal: „Ach, má dcero, srazilas mě do prachu, uvrhla jsi mě do zkázy! Zavázal jsem se svými ústy Hospodinu a nemohu to vzít zpět.“
#11:36 Ona mu odpověděla: „Můj otče, když ses svými ústy zavázal Hospodinu, učiň se mnou, co jsi vyřkl, za to, co Hospodin pro tebe učinil, abys mohl vykonat pomstu nad svými nepřáteli, nad Amónovci.“
#11:37 Požádala pak svého otce: „Nechť je mi dovoleno toto: Ponech mi dva měsíce. Ráda bych odešla do hor a oplakávala se svými družkami své panenství.“
#11:38 On jí řekl: „Jdi.“ Propustil ji na dva měsíce a ona odešla se svými družkami, aby na horách oplakávala své panenství.
#11:39 Po uplynutí dvou měsíců se vrátila k otci a on splnil slib, který o ní učinil. Muže nepoznala. V Izraeli se pak stalo zvykem,
#11:40 že izraelské dívky vycházívají rok co rok po čtyři dny v roce opěvovat dceru Jiftácha Gileádského. 
#12:1 I svolali se efrajimští muži a táhli k severu. Vyčítali Jiftáchovi: „Proč jsi táhl do boje proti Amónovcům a nezavolal jsi nás, abychom šli s tebou? Zapálíme ti dům nad hlavou!“
#12:2 Jiftách jim odpověděl: „Já i můj lid jsme měli s Amónovci těžký spor. Ač jsem vás úpěnlivě volal, nevysvobodili jste mě z jejich rukou.
#12:3 Když jsem zjistil, že mě nevysvobodíte, nasadil jsem vlastní život a táhl jsem na Amónovce. A Hospodin mi je vydal do rukou. Proč jste tedy dnes na mě vytáhli? Abyste proti mně bojovali?“
#12:4 Jiftách shromáždil všechny gileádské muže, bojoval s Efrajimem a gileádští muži Efrajima pobili. Ti totiž říkali: „Vy Gileáďané jste efrajimští uprchlíci, zmítáte se mezi Efrajimem a Manasesem.“
#12:5 Gileád dobyl na Efrajimovi jordánské brody. Když nyní řekl někdo z efrajimských uprchlíků: „Rád bych se přebrodil“, zeptali se ho gileádští muži: „Jsi Efratejec?“ Jestliže odvětil „Nejsem“,
#12:6 vyzvali ho: „Tak řekni šibbolet!“ On však řekl: „Sibbolet“ a nedokázal to přesně vyslovit. Tu ho popadli a zabili při jordánských brodech. Toho času padlo z Efrajima čtyřicet dva tisíce mužů.
#12:7 Jiftách soudil Izraele po šest let. I zemřel Jiftách Gileádský a byl pochován v jednom z gileádských měst.
#12:8 Po něm soudil Izraele Ibsán z Bét-lechemu.
#12:9 Měl třicet synů. Třicet dcer vyvdal jinam a třicet dívek přivedl svým synům odjinud. Soudil Izraele po sedm let.
#12:10 I zemřel Ibsán a byl pochován v Bét-lechemu.
#12:11 Po něm soudil Izraele Elón Zabulónský. Soudil Izraele po deset let.
#12:12 I zemřel Elón Zabulónský a byl pochován v Ajalónu v zabulónské zemi.
#12:13 Po něm soudil Izraele Abdón, syn Hiléla Pireatónského.
#12:14 Měl čtyřicet synů a třicet vnuků, kteří jezdili na sedmdesáti oslátkách. Soudil Izraele po osm let.
#12:15 I zemřel Abdón, syn Hiléla Pireatónského, a byl pochován v Pireatónu v efrajimské zemi v Amáleckém pohoří. 
#13:1 Izraelci se dále dopouštěli toho, co je zlé v Hospodinových očích. Proto je Hospodin vydal na čtyřicet let do rukou Pelištejců.
#13:2 Byl jeden muž ze Soreje, z danovské čeledi, jménem Manóach. Jeho žena byla neplodná a nerodila.
#13:3 I ukázal se té ženě Hospodinův posel a řekl jí: „Hle, ty jsi neplodná, nerodila jsi, avšak otěhotníš a porodíš syna.
#13:4 Proto se teď měj na pozoru: nepij víno ani opojný nápoj a nejez nic nečistého.
#13:5 Hle, otěhotníš a porodíš syna, ale jeho hlavy se nesmí dotknout břitva; ten chlapec bude od mateřského života Boží zasvěcenec. On začne vysvobozovat Izraele z rukou Pelištejců.“
#13:6 Žena přišla povědět svému muži: „Přišel ke mně muž Boží; vypadal jako Boží posel, tak byl hrozný. Ani jsem se ho nezeptala, odkud je, a on mi své jméno nepověděl.
#13:7 Řekl mi: ‚Hle, otěhotníš a porodíš syna. Nepij teď víno ani opojný nápoj a nejez nic nečistého, protože ten chlapec bude od mateřského života až do dne své smrti Boží zasvěcenec.‘“
#13:8 Manóach prosil Hospodina: „Dovol prosím, Panovníku, nechť k nám znovu přijde muž Boží, kterého jsi poslal, a poučí nás, co máme dělat s chlapcem, který se má narodit.“
#13:9 Bůh Manóacha vyslyšel a Boží posel přišel znovu k té ženě, právě když seděla na poli a její muž Manóach nebyl s ní.
#13:10 Žena rychle běžela povědět o tom svému muži. Řekla mu: „Ukázal se mi ten muž, který ke mně onehdy přišel.“
#13:11 Manóach šel hned za svou ženou, přišel k tomuto muži a otázal se ho: „Jsi ty ten muž, který mluvil k této ženě?“ On odpověděl: „Jsem.“
#13:12 Manóach řekl: „Nuže, až se splní tvá slova, na co je nutno u toho chlapce dbát a co s ním máme činit?“
#13:13 Hospodinův posel Manóachovi odpověděl: „Ať se žena varuje všeho, o čem jsem jí řekl.
#13:14 Ať neokusí ničeho, co pochází z vinné révy. Nesmí pít víno ani opojný nápoj ani jíst něco nečistého. Ať bedlivě dbá na všechno, co jsem jí přikázal.“
#13:15 Manóach řekl Hospodinovu poslu: „Dovol, abychom tě zdrželi a připravili ti kůzle.“
#13:16 Ale Hospodinův posel Manóachovi odpověděl: „I kdybys mě zdržel, nejedl bych z tvého pokrmu. Chceš-li však připravit zápalnou oběť, obětuj ji Hospodinu.“ Manóach totiž nevěděl, že je to posel Hospodinův.
#13:17 Manóach se ještě Hospodinova posla zeptal: „Jaké je tvé jméno? Rádi bychom tě poctili, až se splní tvá slova.“
#13:18 Ale Hospodinův posel mu odvětil: „Proč se ptáš na mé jméno? Je podivuhodné.“
#13:19 Manóach vzal tedy kůzle a oběť přídavnou a obětoval na té skále Hospodinu a ten před Manóachem a jeho ženou učinil podivuhodnou věc:
#13:20 Když totiž vystupoval plamen z oltáře k nebi, vystoupil v plameni z oltáře i Hospodinův posel. Jak to uviděli Manóach a jeho žena, padli tváří k zemi.
#13:21 Potom se už Hospodinův posel Manóachovi a jeho ženě neukázal. Tehdy Manóach poznal, že to byl posel Hospodinův,
#13:22 a řekl své ženě: „Určitě zemřeme, neboť jsme viděli Boha.“
#13:23 Jeho žena mu však odpověděla: „Kdyby nás chtěl Hospodin usmrtit, nepřijal by od nás oběť zápalnou ani přídavnou, nebyl by nám toto všechno ukázal, ani by nám teď neoznámil něco takového.“
#13:24 I porodila ta žena syna a dala mu jméno Samson. Chlapec vyrostl a Hospodin mu žehnal.
#13:25 A duch Hospodinův ho začal ponoukat v Danovském táboře mezi Soreou a Eštaólem. 
#14:1 Samson sestoupil do Timnaty a spatřil v Timnatě ženu z pelištejských dcer.
#14:2 Když se vrátil, sdělil svému otci a matce: „V Timnatě jsem spatřil ženu z pelištejských dcer. Nuže, vezměte mi ji za ženu!“
#14:3 Ale otec s matkou mu bránili: „Což není žádná mezi dcerami tvých bratří a všeho mého lidu, že si jdeš pro ženu mezi pelištejské neobřezance?“ Samson však svému otci odpověděl: „Vezmi pro mne tuto, protože to je v mých očích ta pravá!“
#14:4 Otec s matkou netušili, že je to od Hospodina, že hledá záminku proti Pelištejcům; toho času totiž vládli nad Izraelem Pelištejci.
#14:5 I sestoupil Samson s otcem a matkou do Timnaty. Když přišli k timnatským vinicím, vyskočil náhle proti němu řvoucí mladý lev.
#14:6 Tu se ho zmocnil duch Hospodinův, že jej holýma rukama roztrhl jako kůzle. Otci ani matce nepověděl, co udělal.
#14:7 Pak sestoupil dolů, aby se domluvil s tou ženou; v Samsonových očích to byla ta pravá.
#14:8 Když opět za čas šel, aby si ji vzal, odbočil podívat se na zabitého lva. A hle, ve lví zdechlině bylo včelstvo a med.
#14:9 Nabral si jej do hrsti a cestou jej jedl. Zašel k otci a matce a dal i jim, aby jedli; ale nepověděl jim, že med vybral ze lví zdechliny.
#14:10 I sestoupil jeho otec k té ženě a Samson tam uspořádal hostinu, jak to mládenci dělávali.
#14:11 Jakmile ho uviděli, vybrali třicet družbů, aby byli s ním.
#14:12 Samson jim řekl: „Chci vám dát hádanku. Jestli mi ji dokážete za sedm dní hostiny rozluštit a uhodnete její smysl, dám vám třicet košil a třicatery sváteční šaty.
#14:13 Jestli mi ji však nebudete moci rozluštit, dáte vy mně třicet košil a třicatery sváteční šaty.“ Souhlasili s ním: „Dej nám svou hádanku, ať si ji poslechneme.“
#14:14 Řekl jim tedy: „Ze žrouta vyšel pokrm, ze siláka vyšla sladkost.“
#14:15 Po tři dny nebyli s to hádanku rozluštit. Sedmého dne navedli Samsonovu ženu: „Přemluv svého muže, ať nám hádanku prozradí! Jinak tě i s domem tvého otce upálíme. To jste nás sem pozvali, abyste nás obrali?“
#14:16 I vyčítala ta žena s pláčem Samsonovi: „Chováš vůči mně jenom nenávist, nemáš mě rád! Dals hádanku synům mého lidu a neprozradil jsi mi ji.“ Odpověděl jí: „Ani svému otci a své matce jsem ji neprozradil, a tobě bych ji měl prozradit?“
#14:17 Ale ona na něj s pláčem naléhala po celých sedm dní hostiny, že jí to sedmého dne prozradil. Tolik ho obtěžovala. A ona prozradila hádanku synům svého lidu.
#14:18 Sedmého dne, ještě před západem slunce, mu řekli mužové města: „Co je sladší než med, co je silnější než lev?“ Odpověděl jim: „Kdybyste neorali mou jalovicí, moji hádanku byste neuhodli.“
#14:19 Tu se ho zmocnil duch Hospodinův, sestoupil do Aškalónu a pobil tam třicet mužů. Pobral jim jejich výstroj a dal sváteční šaty těm, kdo rozluštili hádanku. S planoucím hněvem se pak vrátil do otcovského domu.
#14:20 Samsonovu ženu dostal jeden z družbů, který mu byl druhem. 
#15:1 Po nějakém čase, ve dnech, kdy se žala pšenice, navštívil Samson svou ženu a přinesl darem kozlíka. Řekl: „Rád bych vešel do pokojíku ke své ženě.“ Ale její otec mu nedovolil vejít
#15:2 a namítl: „Myslil jsem, že k ní chováš nenávist, tak jsem ji dal tvému družbovi. Což není její mladší sestra hezčí? Může být tvá místo ní!“
#15:3 Ale Samson jim odvětil: „Tentokrát budu vůči Pelištejcům bez viny, když jim provedu něco zlého.“
#15:4 Odešel a pochytal tři sta lišek. Potom vzal pochodně, otočil vždy dvě lišky ocasy k sobě a mezi ně připevnil pochodeň.
#15:5 Pochodně zapálil a lišky pustil do nepožatého obilí Pelištejců. Zapálil jak kopky, tak nepožaté obilí, dokonce i vinice a olivy.
#15:6 Pelištejci se vyptávali: „Kdo tohle udělal?“ Rozkřiklo se: Samson, zeť Timnaťana, za to, že mu vzal ženu a dal ji jeho družbovi. I přitáhli Pelištejci a upálili ji i s jejím otcem.
#15:7 Tu jim Samson řekl: „Když takhle jednáte, neustanu, dokud nad vámi nevykonám pomstu.“
#15:8 Bil je hlava nehlava a způsobil jim zdrcující porážku. Potom sestoupil a usadil se ve skalní strži Étamu.
#15:9 Ale Pelištejci přitáhli a utábořili se v Judsku; rozložili se v Lechí.
#15:10 Judští muži se ptali: „Proč jste na nás přitáhli?“ Odpověděli: „Přitáhli jsme spoutat Samsona a naložit s ním stejně, jako on nakládal s námi.“
#15:11 I sestoupilo tři tisíce judských mužů ke skalní strži Étamu a řekli Samsonovi: „Copak nevíš, že nad námi vládnou Pelištejci? Cos nám to provedl?“ Odvětil jim: „Jak jednali oni se mnou, tak jsem já jednal s nimi.“
#15:12 Řekli mu: „Sestoupili jsme, abychom tě spoutali a vydali do rukou Pelištejců.“ Samson je požádal: „Odpřisáhněte mi, že se na mě nevrhnete vy sami!“
#15:13 Přislíbili mu: „Nevrhneme se na tebe. Jenom tě pevně spoutáme a vydáme do jejich rukou, určitě tě neusmrtíme.“ Spoutali ho dvěma novými provazy a odvedli ho od skály.
#15:14 Když přišel do Lechí, Pelištejci ho přivítali válečným pokřikem. Tu se ho zmocnil duch Hospodinův a provazy na jeho pažích byly jako nitě, které sežehl oheň. Pouta mu na rukou povolila.
#15:15 Našel čerstvou oslí čelist, popadl ji do ruky a pobil jí tisíc mužů.
#15:16 I řekl Samson: „Oslí čelistí jsem pobil hromadu, dvě hromady, oslí čelistí jsem pobil tisíc mužů.“
#15:17 Když domluvil, odhodil čelist. To místo nazval Rámat-lechí (to je Výšina čelisti).
#15:18 Dostal pak nesmírnou žízeň a volal k Hospodinu. Řekl: „Ty jsi způsobil skrze svého služebníka toto velké vysvobození. A teď mám umřít žízní a upadnout do rukou neobřezanců?“
#15:19 I rozpoltil Bůh skalní kotlinu v Lechí a vytryskla z ní voda. Napil se, okřál na duchu a ožil. Proto nazval to místo Pramen volajícího; ten je v Lechí dodnes.
#15:20 Soudil Izraele za dnů Pelištejců po dvacet let. 
#16:1 Pak se Samson odebral do Gázy. Spatřil tam ženu nevěstku a vešel k ní.
#16:2 Obyvatelům Gázy bylo oznámeno: „Přišel sem Samson.“ Obcházeli kolem a číhali na něho celou noc v městské bráně. Po celou noc tiše vyčkávali. Řekli: „Až nastane jitro, zabijeme ho.“
#16:3 Ale Samson ležel jen do půlnoci. O půlnoci vstal, vysadil křídla vrat městské brány s oběma veřejemi, vytrhl je i se závorou, vložil si je na ramena a vynesl je na vrchol hory ležící směrem k Chebrónu.
#16:4 Potom se zamiloval do ženy v Hroznovém úvalu. Jmenovala se Delíla.
#16:5 I přišla k ní pelištejská knížata se žádostí: „Hleď na něm vymámit, v čem spočívá jeho veliká síla, jak bychom ho přemohli a spoutali a zneškodnili. Každý z nás ti dá tisíc sto šekelů stříbra.“
#16:6 Delíla tedy naléhala na Samsona: „Prozraď mi přece, v čem spočívá tvá veliká síla a čím tě spoutat, abys byl zneškodněn?“
#16:7 Samson jí řekl: „Kdyby mě spoutali sedmi syrovými houžvemi, které ještě nevyschly, zeslábl bych a byl bych jako kterýkoli člověk.“
#16:8 Pelištejská knížata jí donesla sedm syrových houžví, které ještě nevyschly, a ona ho jimi spoutala.
#16:9 Přitom u ní v pokojíku na něj číhali. Pak křikla na Samsona: „Samsone, jdou na tebe Pelištejci!“ Tu zpřetrhal houžve, jako se trhá šňůrka z koudele, když ji sežehne oheň, a nevyšlo najevo, v čem spočívá jeho síla.
#16:10 Delíla naléhala na Samsona: „Vidíš, jak jsi mě obelstil a vykládals mi lži. Teď mi však prozraď, čím bys mohl být spoután.“
#16:11 Odpověděl jí: „Kdyby mě pevně spoutali novými provazy, jichž se dosud nepoužilo, zeslábl bych a byl bych jako kterýkoli člověk.“
#16:12 I vzala Delíla nové provazy, spoutala ho jimi a křikla na něho: „Samsone, jdou na tebe Pelištejci!“ Přitom v pokojíku na něj číhali. Tu strhal provazy s paží jako nitě.
#16:13 Delíla vyčítala Samsonovi: „I teď jsi mě obelstil a vykládáš mi lži. Prozraď mi, čím bys mohl být spoután!“ Odpověděl jí: „Kdybys vpletla mých sedm pramenů vlasů do osnovy.“
#16:14 Učinila tak, upevnila kolíkem a křikla na něho: „Samsone, jdou na tebe Pelištejci!“ Procitl ze spánku a vytrhl kolík, vratidlo i osnovu.
#16:15 Zase mu vytýkala: „Jak můžeš říkat: ‚Miluji tě‘, když tvé srdce není při mně! Už třikrát jsi mě obelstil a neprozradils mi, v čem je tvá veliká síla.“
#16:16 Když ho po celé dny obtěžovala svými řečmi a dotírala na něho, že z toho byl až k smrti unaven,
#16:17 otevřel jí své srdce dokořán a řekl jí: „Nikdy se nedotkla mé hlavy břitva, protože jsem od života své matky Boží zasvěcenec. Kdybych byl oholen, má síla by ode mne odstoupila, zeslábl bych a byl bych jako každý člověk.“
#16:18 Delíla viděla, že jí otevřel své srdce dokořán, a poslala pelištejským knížatům vzkaz: „Tentokrát přijďte, neboť mi otevřel své srdce dokořán.“ Pelištejská knížata k ní tedy přišla a přinesla s sebou stříbro.
#16:19 Ona ho uspala na klíně, zavolala jednoho muže a dala oholit sedm pramenů vlasů na jeho hlavě. Tak se stala příčinou jeho ponížení. Jeho síla od něho odstoupila.
#16:20 Křikla: „Samsone, jdou na tebe Pelištejci!“ Procitl ze spánku a pomyslil si: „Dostanu se z toho jako dosud vždycky a pouta setřesu.“ Nevěděl, že Hospodin od něho odstoupil.
#16:21 Pelištejci se ho zmocnili, vypíchli mu oči a odvlekli ho do Gázy, kde ho spoutali dvojitým bronzovým řetězem. Ve vězení musel mlít.
#16:22 Ale vlasy na hlavě mu začaly hned po oholení dorůstat.
#16:23 Pelištejská knížata se shromáždila, aby obětovala velkou oběť svému bohu Dágonovi a aby se oddala radovánkám. Řekli: „Náš bůh nám vydal do rukou Samsona, našeho nepřítele.“
#16:24 Když ho lid viděl, vychvaloval svého boha. Volali: „Náš bůh nám vydal do rukou našeho nepřítele, pustošitele naší země, který mnoho našich skolil.“
#16:25 Rozjařili se a křičeli: „Zavolejte Samsona, ať nám poslouží k nevázaným hrám!“ Zavolali tedy Samsona z vězení, aby si s ním nevázaně pohrávali. Postavili ho mezi sloupy.
#16:26 Samson požádal mládence, který jej vedl za ruku: „Pusť mě, ať mohu ohmatat sloupy, na nichž budova spočívá, a opřít se o ně!“
#16:27 Dům byl plný mužů i žen, byla tam všechna pelištejská knížata. I na střeše bylo na tři tisíce mužů a žen, hodlajících přihlížet nevázaným hrám se Samsonem.
#16:28 I volal Samson k Hospodinu a prosil: „Panovníku Hospodine, rozpomeň se na mne a dej mi prosím jen ještě tentokrát sílu, Bože, abych rázem mohl vykonat na Pelištejcích pomstu za svoje oči!“
#16:29 Pak Samson pevně objal oba prostřední sloupy, na nichž budova spočívala, a vzepřel se proti nim, proti jednomu pravicí a proti druhému levicí.
#16:30 A řekl: „Ať zhynu zároveň s Pelištejci!“ Napnul sílu, a dům se zřítil na knížata i na všechen lid, který byl v něm, takže mrtvých, které usmrtil umíraje, bylo víc než těch, které usmrtil zaživa.
#16:31 I sestoupili jeho bratří a celý jeho dům, vynesli ho, vrátili se a pochovali ho mezi Soreou a Eštaólem v hrobě jeho otce Manóacha. Soudil Izraele po dvacet let. 
#17:1 Byl jeden muž z Efrajimského pohoří jménem Míkajáš.
#17:2 Přiznal se své matce: „Těch tisíc sto šekelů stříbra, které ti byly vzaty, pro něž jsi dokonce vyřkla přede mnou kletbu, to stříbro je u mne, já jsem je vzal.“ Tu jeho matka řekla: „Můj synu, buď požehnán Hospodinu!“
#17:3 Vrátil matce těch tisíc sto šekelů stříbra a matka prohlásila: „To stříbro jsem cele zasvětila Hospodinu. Předávám je tobě, můj synu, abys z něho dal udělat sochu tesanou a litou. Dávám ti je nyní zpátky.“
#17:4 Ale on vrátil stříbro své matce. I vzala dvě stě šekelů stříbra a dala je zlatníkovi; udělal z nich sochu tesanou a litou, která pak byla v Míkově domě.
#17:5 Ten muž, Míka, měl totiž svatyni. Udělal efód a domácí bůžky a pověřil jednoho ze svých synů, aby mu sloužil jako kněz.
#17:6 V těch dnech neměli v Izraeli krále. Každý dělal, co uznal za správné.
#17:7 Byl jeden mládenec z judského Betléma, z Judovy čeledi; byl to lévijec a pobýval tam jako host.
#17:8 Ten muž odešel z města, z judského Betléma, aby pobýval jako host, kde se mu naskytne. Při svém putování došel na Efrajimské pohoří k Míkovu domu.
#17:9 Míka se ho otázal: „Odkud přicházíš?“ Odpověděl mu: „Jsem lévijec z judského Betléma a putuji, abych pobýval jako host, kde se mi naskytne.“
#17:10 Míka mu navrhl: „Zůstaň u mne a budeš mi otcem a knězem. Budu ti za to dávat deset šekelů stříbra ročně, ošacení a stravu.“ Lévijec na to přistoupil.
#17:11 Přivolil zůstat u toho muže. A byl mu mládenec jako jeden z jeho synů.
#17:12 Míka lévijského mládence pověřil, aby mu sloužil jako kněz; zůstal tedy v Míkově domě.
#17:13 Míka si řekl: „Nyní vím, že mi Hospodin bude prokazovat dobro, neboť mám za kněze lévijce.“ 
#18:1 V oněch dnech neměli v Izraeli krále. Kmen Danovců si tehdy hledal dědičný podíl, aby se mohl usadit, protože mu až po onen den nepřipadl mezi izraelskými kmeny žádný dědičný podíl.
#18:2 Danovci tedy vyslali ze své čeledi pět mužů, chrabrých bojovníků, ze svých končin, ze Soreje a Eštaólu, aby jako zvědové prozkoumali zemi. Poručili jim: „Jděte prozkoumat zemi.“ I přišli na Efrajimské pohoří až k Míkovu domu a přenocovali tam.
#18:3 Když byli u Míkova domu, poznali po řeči lévijského mládence. Odbočili tam a zeptali se ho: „Kdo tě sem přivedl? Co tady děláš? Co tu pohledáváš?“
#18:4 Odpověděl jim: „Tak a tak to se mnou vyjednal Míka. Najal mě, abych mu sloužil jako kněz.“
#18:5 Poprosili ho: „Doptej se teď Boha, abychom věděli, zdaří-li se nám cesta, na kterou jsme se vydali.“
#18:6 Kněz jim řekl: „Jděte v pokoji. Nad cestou, na kterou jste se vydali, bdí Hospodin.“
#18:7 Těch pět mužů tedy šlo dál, až přišli do Lajiše. Spatřili město žijící si v bezpečí po způsobu Sidónců a v něm klidný a bezstarostný lid. Nebylo v zemi nikoho, kdo by je potupil, nebylo uchvatitele moci. Od Sidónců byli daleko a s nikým neměli žádnou dohodu.
#18:8 Zvědové se vrátili ke svým bratřím do Soreje a Eštaólu. Bratří se jich zeptali: „Jak jste pořídili?“
#18:9 Odpověděli: „Vzhůru, vytáhněme proti nim! Viděli jsme tu zemi, je velmi dobrá. Vy otálíte? Nerozpakujte se tam vtrhnout a tu zemi obsadit.
#18:10 Přijdete tam k bezstarostnému lidu. Země je na všechny strany otevřená, Bůh vám ji vydal do rukou. Je to místo, kde není nedostatek v ničem, co může země dát.“
#18:11 Tak vyrazilo odtamtud z čeledi Danovců, ze Soreje a Eštaólu, šest set válečně vyzbrojených mužů.
#18:12 Táhli vzhůru a utábořili se v Kirjat-jearímu v Judsku. Proto se to místo nazývá Danovský tábor až dodnes; je za Kirjat-jearímem.
#18:13 Odtud se ubírali přes Efrajimské pohoří, až došli k Míkovu domu.
#18:14 Tu promluvilo těch pět mužů, kteří jako zvědové prošli lajišskou zemi. Řekli svým bratřím: „Zdalipak víte, že v těchto domech je efód a domácí bůžci i socha tesaná a litá? Teď víte, co máte udělat.“
#18:15 Odbočili tam, přišli k domu lévijského mládence, k Míkovu domu, a pozdravili ho přáním pokoje.
#18:16 Oněch šest set válečně vyzbrojených mužů zůstalo stát při vchodu do brány; byli to Danovci.
#18:17 Pak vystoupilo těch pět mužů, kteří jako zvědové prošli zemi, vešli tam a vzali tesanou sochu, efód a domácí bůžky i sochu litou, zatímco kněz stál u vchodu do brány se šesti sty válečně vyzbrojenými muži.
#18:18 Když vešli do Míkova domu a vzali tesanou sochu, efód a domácí bůžky i sochu litou, otázal se jich kněz: „Co to děláte?“
#18:19 Odpověděli mu: „Mlč, buď zticha a pojď s námi; budeš nám otcem a knězem. Je lepší být knězem domu jediného muže, anebo být knězem kmene a čeledi v Izraeli?“
#18:20 Knězi se to zalíbilo, vzal efód a domácí bůžky i tesanou sochu a vstoupil mezi lid.
#18:21 Nato se obrátili k odchodu. Děti, dobytek i cenný majetek umístili dopředu.
#18:22 Byli už daleko od Míkova domu, když byli přivoláni muži z domů, které stály při domě Míkově. Dohonili Danovce
#18:23 a volali na ně. Přiměli je k tomu, že se otočili a otázali se Míky: „Co ti je, že jsi je přivolal?“
#18:24 Odpověděl: „Vzali jste mi bohy, které jsem si udělal, i kněze a odtáhli jste. Co mi ještě zbývá? A to se mě ptáte: ‚Co ti je?‘“
#18:25 Danovci jej okřikli: „Ať už za námi není slyšet tvůj hlas! Jinak se na vás vrhnou rozhořčení muži a budeš smeten i se svým domem.“
#18:26 Potom Danovci pokračovali v cestě. Míka viděl, že jsou silnější než on, obrátil se a vrátil do svého domu.
#18:27 Když tedy vzali, co si dal udělat Míka, i kněze, kterého měl, přitáhli na Lajiš, na klidný a bezstarostný lid. Pobili je ostřím meče a město vypálili.
#18:28 Nebylo nikoho, kdo by je vysvobodil, protože město bylo daleko od Sidónu a s nikým neměli žádnou dohodu. Leželo v dolině u Bét-rechóbu. Danovci pak město opět vystavěli a osídlili.
#18:29 Pojmenovali město Dan podle jména svého praotce Dana, který se narodil Izraelovi; původně se ovšem jmenovalo Lajiš.
#18:30 Danovci si tam postavili tu tesanou sochu a Jónatan, syn Geršóma, syna Mojžíšova, i jeho potomci byli kněžími kmene Danovců až do dne, kdy byli ze země vysídleni.
#18:31 Míkovu tesanou sochu, kterou si dal udělat, tam měli umístěnu po celou dobu, kdy byl dům Boží v Šílu. 
#19:1 V těch dnech, kdy neměli v Izraeli krále, pobýval nějaký lévijec jako host v odlehlých končinách Efrajimského pohoří. Vzal si za ženinu jakousi ženu z judského Betléma.
#19:2 Ale ženina se proti němu provinila smilstvem a odešla od něho do otcovského domu, do judského Betléma. Byla tam po dobu čtyř měsíců.
#19:3 Její muž se za ní vypravil, aby jí domluvil a pohnul ji k návratu. Měl s sebou svého mládence a pár oslů. Ona ho uvedla do otcovského domu. Jakmile ho dívčin otec spatřil, šel mu radostně vstříc.
#19:4 Pak ho jeho tchán, dívčin otec, zdržoval, takže u něho zůstal tři dny. Jedli a pili a nocovali tam.
#19:5 Čtvrtého dne za časného jitra, když se zvedl k odchodu, řekl dívčin otec svému zeti: „Posilni se soustem chleba a potom půjdete.“
#19:6 Tak zůstali a oba spolu jedli a pili. Potom dívčin otec vybídl muže: „Buď tak laskav, zůstaň ještě přes noc a buď dobré mysli.“
#19:7 Když se muž přece zvedl k odchodu, přinutil ho jeho tchán, že tam opět zůstal přes noc.
#19:8 Pátého dne za časného jitra chtěl odejít, ale dívčin otec na něho zase naléhal: „Posilni se prosím!“ Pozdrželi se, až den pokročil, a oba pojedli.
#19:9 Potom se muž se svou ženinou a mládencem zvedl k odchodu. Jeho tchán, dívčin otec, mu však řekl: „Pohleď, den se schyluje k večeru. Zůstaňte přes noc. Den se přece sklání. Přenocuj zde a buď dobré mysli. Zítra za časného jitra se vydáš na cestu a půjdeš ke svému stanu.“
#19:10 Avšak muž nechtěl zůstat přes noc, zvedl se k odchodu a došel se svým párem osedlaných oslů a se svou ženinou až naproti Jebúsu, což je Jeruzalém.
#19:11 Když byli u Jebúsu, den téměř minul. Tu navrhl mládenec svému pánu: „Pojď prosím, uchýlíme se do tohoto jebúsejského města a přenocujeme v něm.“
#19:12 Ale jeho pán mu odvětil: „Neuchýlíme se do cizího města, které nepatří Izraelcům. Půjdeme dál do Gibeje.“
#19:13 A pobídl mládence: „Pojď, abychom se přiblížili k některému z těch míst a mohli přenocovat v Gibeji nebo v Rámě.“
#19:14 Šli tedy dál. Slunce už zapadlo, když byli u benjamínské Gibeje.
#19:15 Uchýlili se tam a vstoupili do Gibeje, aby přenocovali. Když tam lévijec přišel, usadil se na městském prostranství, ale nebyl tu nikdo, kdo by jej přijal na noc do domu.
#19:16 Večer se vracel z práce na poli nějaký starý muž. Pocházel z Efrajimského pohoří a pobýval v Gibeji jako host; mužové toho místa byli Benjamínovci.
#19:17 Rozhlédl se a spatřil na městském prostranství pocestného. I otázal se ten starý muž: „Kam jdeš a odkud přicházíš?“
#19:18 Odvětil mu: „Ubíráme se z judského Betléma do odlehlých končin Efrajimského pohoří, odkud pocházím. Šel jsem do judského Betléma; chodím sloužit do Hospodinova domu. Tady však není nikdo, kdo by mě přijal do domu.
#19:19 A přitom mám jak slámu a obrok pro své osly, tak chléb a víno pro sebe i pro tvou služebnici a pro mládence, který je s tvými služebníky. V ničem nemáme nedostatek.“
#19:20 Starý muž mu pravil: „Pokoj tobě! Klidně mi přenech starost o všechno, co potřebuješ. Jenom nezůstávej přes noc zde na prostranství.“
#19:21 Uvedl ho do svého domu a pro osly připravil krmivo. Umyli si nohy, jedli a pili.
#19:22 Byli dobré mysli, avšak hle, mužové města, muži ničemníci, obklíčili dům, tloukli na dveře a vyzývali toho starého muže, hospodáře: „Vyveď muže, který vešel do tvého domu, ať ho poznáme!“
#19:23 Tu k nim ten hospodář vyšel a domlouval jim: „Ne tak, moji bratři! Nedopustíte se přece něčeho tak zlého! Vždyť tento muž vešel do mého domu. Nesmíte spáchat takovou hanebnost!
#19:24 Tady je má dcerka, panna, a jeho ženina. Hned je vyvedu. Můžete je zneužít a nakládat s nimi, jak se vám zlíbí, ale vůči tomuto muži nesmíte spáchat takovou hanebnost.“
#19:25 Ti muži ho však nechtěli ani slyšet. Proto popadl ten muž svou ženinu a vyvedl jim ji ven. Obcovali s ní a zneužívali ji celou noc až do rána; propustili ji, teprve když vzešla jitřenka.
#19:26 Za ranního rozbřesku se ta žena dovlekla zpět, zhroutila se u vchodu do domu muže, kde byl její pán, a zůstala ležet až do světla.
#19:27 Ráno její pán vstal, otevřel domovní dveře a vyšel, aby se vydal na cestu. Tu spatřil ženu, svou ženinu, zhroucenou u vchodu do domu, s rukama na prahu.
#19:28 Řekl jí: „Vstaň a půjdeme!“ Ale nedostal odpověď. Naložil ji tedy na osla a vydal se na cestu k domovu.
#19:29 Když přišel do svého domu, vzal nůž, uchopil ženinu a rozsekal ji i s kostmi na dvanáct dílů a rozeslal ji po celém izraelském území.
#19:30 Kdokoli to viděl, říkal: „Něco takového se nestalo a nebylo spatřeno ode dne, kdy Izraelci vyšli z egyptské země, až podnes. Uvažujte o tom, poraďte se a vyjádřete se!“ 
#20:1 I vytáhli všichni synové Izraele a shromáždila se celá pospolitost od Danu po Beer-šebu i gileádská země jako jeden muž k Hospodinu do Mispy.
#20:2 Dostavili se vůdcové všeho lidu i všechny izraelské kmeny do shromáždění Božího lidu, čtyři sta tisíc pěšáků ozbrojených meči.
#20:3 Benjamínovci slyšeli, že Izraelci přitáhli vzhůru do Mispy. Izraelci se tázali: „Mluvte, jak došlo k takovému zločinu?“
#20:4 Nato odpověděl ten lévijec, muž zavražděné ženy: „Přišel jsem se svou ženinou přenocovat do benjamínské Gibeje.
#20:5 Gibejští občané však proti mně povstali a obklíčili mě v noci v tom domě. Měli v úmyslu mě zavraždit. A mou ženinu zneužili tak, že zemřela.
#20:6 Tu jsem vzal svou ženinu, rozsekal ji na díly a rozeslal je do všech krajů izraelského dědictví, neboť oni se dopustili v Izraeli mrzké hanebnosti.
#20:7 Vy všichni jste přece synové Izraele. Ujměte se věci a poraďte tady.“
#20:8 Všechen lid povstal jako jeden muž a rozhodl: „Nikdo z nás neodejde do svého stanu, nikdo z nás se neuchýlí do svého domu,
#20:9 a s Gibeou naložíme takto: Potáhneme proti ní, jak určí los.
#20:10 Vybereme ve všech izraelských kmenech deset mužů ze sta, sto z tisíce a tisíc z deseti tisíců, aby opatřovali potravu pro lid, který přitáhne proti benjamínské Gibeji, aby jí odplatil za všechnu hanebnost, které se v Izraeli dopustila.“
#20:11 Tak se shromáždili všichni muži k městu, semknuti jako jeden muž.
#20:12 Izraelské kmeny vyslaly muže do všech Benjamínových čeledí s dotazem: „Co je to za zločin, k němuž u vás došlo?
#20:13 Ihned vydejte ty gibejské ničemy. Na smrt s nimi, ať odstraníme z Izraele zlořád!“ Ale Benjamínovci nechtěli výzvy svých bratří Izraelců uposlechnout.
#20:14 Shromáždili se z měst do Gibeje, aby vytáhli do boje proti Izraelcům.
#20:15 Onoho dne bylo napočteno Benjamínovců z měst dvacet šest tisíc mužů ozbrojených meči, mimo sedm set vybraných mužů napočtených z obyvatelů Gibeje.
#20:16 Ze všeho toho lidu bylo sedm set vybraných mužů leváků; každý z nich vrhal kameny prakem navlas přesně a neminul se.
#20:17 Izraelských mužů, bez Benjamínců, bylo napočteno čtyři sta tisíc mužů ozbrojených meči, samí bojovníci.
#20:18 Zvedli se a táhli vzhůru do Bét-elu, aby se doptali Boha. I ptali se Izraelci: „Kdo z nás má táhnout do boje s Benjamínci první?“ Hospodin odpověděl: „První bude Juda.“
#20:19 Tak se Izraelci ráno zvedli a položili se proti Gibeji.
#20:20 Izraelští muži vytáhli do boje s Benjamínem a seřadili se k bitvě proti Gibeji.
#20:21 Tu z Gibeje vyrazili Benjamínci a vnesli onoho dne do Izraele zkázu; srazili k zemi dvaadvacet tisíc mužů.
#20:22 Ale lid, izraelští muži, se vzchopili a opět se seřadili k boji na místě, na němž se seřadili prvého dne.
#20:23 Předtím však Izraelci táhli vzhůru do Bét-elu a plakali před Hospodinem až do večera; potom se Hospodina doptali: „Máme znovu podstoupit boj se syny svého bratra Benjamína?“ Hospodin odpověděl: „Táhněte proti němu!“
#20:24 Druhého dne se Izraelci utkali s Benjamínci.
#20:25 Ale i druhého dne proti nim vyrazil Benjamín z Gibeje a znovu vnesli mezi Izraelce zkázu; srazili k zemi osmnáct tisíc mužů, ačkoli všichni byli ozbrojeni meči.
#20:26 I táhli všichni Izraelci, totiž všechen lid, vzhůru a přišli do Bét-elu. Plakali a seděli tam před Hospodinem a postili se onoho dne až do večera; přitom přinášeli Hospodinu oběti zápalné a pokojné.
#20:27 Pak se Izraelci doptali Hospodina; byla tam totiž v té době schrána Boží smlouvy
#20:28 a v té době stával před ní Pinchas, syn Eleazara, syna Áronova. Tázali se: „Máme pokračovat dál ve válečném tažení proti synům svého bratra Benjamína, anebo toho máme nechat?“ Hospodin odpověděl: „Vytáhněte, neboť zítra ti je vydám do rukou.“
#20:29 Izrael tedy položil proti Gibeji ze všech stran zálohy.
#20:30 Třetího dne táhli Izraelci vzhůru proti Benjamíncům a seřadili se proti Gibeji jako už dvakrát předtím.
#20:31 Benjamínci vyrazili proti lidu, dali se odlákat od města a začali jako po dvakrát předtím bít a pobíjet některé z lidu. Na silnicích, z nichž jedna vystupuje do Bét-elu a druhá do Gibeje, padlo v poli asi třicet mužů z Izraele.
#20:32 Benjamínci si řekli: „Zase jsme je porazili jako předtím.“ Izraelci totiž řekli: „Utíkejme a odlákejme je od města k silnicím!“
#20:33 Tu se všichni izraelští muži zvedli ze svého místa a seřadili se v Baal-támaře, zatímco izraelská záloha se vyrojila ze svého stanoviště na Gibejské planině.
#20:34 Proti Gibeji nyní nastoupilo deset tisíc mužů vybraných ze všeho Izraele a nastal urputný boj. Benjamínci ovšem nevěděli, že jim hrozí záhuba.
#20:35 I porazil Hospodin Benjamína před Izraelem. Izraelci vnesli onoho dne do Benjamína zkázu; padlo dvacet pět tisíc sto mužů, všichni ozbrojení meči.
#20:36 Benjamínci viděli, že jsou poraženi. Izraelští muži vyklizovali před Benjamínem prostor, protože spoléhali na zálohu, kterou zřídili proti Gibeji.
#20:37 Muži ze zálohy přispěchali a zaútočili na Gibeu. Záloha přitáhla a vybila celé město ostřím meče.
#20:38 Izraelští muži byli domluveni se zálohou, že jim dá znamení dýmem vystupujícím z města.
#20:39 Proto se izraelští muži v bitvě obrátili k Benjamíncům zády. Benjamín začal pobíjet a dobíjet izraelské muže, asi třicet mužů. Řekli si: „Teď už jsme je nadobro porazili jako v první bitvě.“
#20:40 Vtom začal vystupovat z města sloup dýmu. Když se Benjamín ohlédl dozadu, spatřil, jak vystupuje kouř z města k nebi jako při celopalu.
#20:41 Tu se k nim izraelští muži obrátili čelem. Benjamínci se zhrozili, neboť viděli, že jim hrozí záhuba.
#20:42 Dali se před izraelskými muži na ústup cestou k poušti, avšak boji neunikli, a ti, kteří vyrazili z měst, vnášeli mezi ně zkázu.
#20:43 Obklíčili Benjamína, pronásledovali jej bez oddechu až na východ od Gibeje a deptali ho.
#20:44 Tenkrát padlo z Benjamína osmnáct tisíc mužů, samí válečníci.
#20:45 Někteří se obrátili a dali na útěk do pouště ke skalisku Rimónu. Izraelci paběrkovali po silnicích a pochytali pět tisíc mužů. Hnali je až do Gideómu a pobili z nich ještě dva tisíce mužů.
#20:46 Všech padlých z Benjamína bylo onoho dne pětadvacet tisíc mužů ozbrojených meči, samí válečníci.
#20:47 Ale šest set mužů se obrátilo a uteklo do pouště ke skalisku Rimónu; zůstali na skalisku Rimónu čtyři měsíce.
#20:48 Izraelští muži se vrátili k Benjamínovcům a pobili je ostřím meče, pobili všechno, co v městě zůstalo, i dobytek, všechno, nač přišli; také všechna města, do nichž přišli, vypálili. 
#21:1 Izraelští muži přísahali v Mispě takto: „Nikdo z nás nedá svou dceru za ženu Benjamínci.“
#21:2 I přišel lid do Bét-elu a seděli tam před Bohem až do večera za hlasitého nářku a velmi usedavého pláče.
#21:3 Volali: „Hospodine, Bože Izraele, proč se v Izraeli stalo, aby byl dnes odečten od Izraele jeden kmen?“
#21:4 Nazítří za časného jitra tam lid zbudoval oltář a obětovali oběti zápalné a pokojné.
#21:5 Izraelci se tázali: „Kdo se ze všech izraelských kmenů nedostavil do shromáždění k Hospodinu?“ Byla totiž vyslovena veliká přísaha proti tomu, kdo by se nedostavil do Mispy k Hospodinu, že bez milosti propadne smrti.
#21:6 Synové Izraele totiž litovali svého bratra Benjamína a říkali: „Dnes byl odťat od Izraele jeden kmen.
#21:7 Jak to uděláme se zbývajícími Benjamínci, aby dostali ženy? My jsme přece přísahali při Hospodinu, že jim nedáme za ženu žádnou ze svých dcer.“
#21:8 Proto se tázali: „Je někdo z izraelských kmenů, který se nedostavil k Hospodinu do Mispy?“ A hle: Z Jábeše v Gileádu nepřišel do tábora na shromáždění nikdo.
#21:9 Byl přepočítán lid a skutečně tam nebyl nikdo z obyvatelů Jábeše v Gileádu.
#21:10 Pospolitost Izraele tam ihned vyslala dvanáct tisíc mužů, chrabrých bojovníků. A přikázali jim: „Jděte pobít obyvatele Jábeše v Gileádu ostřím meče, i ženy a děti.
#21:11 Toto je úkol, který máte splnit: Vyhubíte jako klaté všechny mužského pohlaví i každou ženu, která poznala muže a obcovala s ním.“
#21:12 Mezi obyvateli Jábeše v Gileádu se našlo čtyři sta dívek, panen, které dosud muže nepoznaly a s mužem neobcovaly. Přivedli je do tábora v Šílu, jež je v kenaanské zemi.
#21:13 Celá pospolitost pak vyslala mluvčí k Benjamínovcům, kteří byli na skalisku Rimónu, aby jim vyhlásili pokoj.
#21:14 Tak se toho času Benjamín vrátil. Dali jim ženy, které ponechali naživu z žen z Jábeše v Gileádu, ale nebylo jich pro ně dost.
#21:15 Lid Benjamína litoval, neboť Hospodin způsobil mezi izraelskými kmeny trhlinu.
#21:16 Stařešinové pospolitosti se tázali: „Co uděláme se zbývajícími, aby dostali ženy, když ženy benjamínské byly vyhubeny?“
#21:17 Řekli: „Vlastnictví získané uprchlíky bude Benjamínovo, aby nebyl vyhlazen kmen z Izraele.
#21:18 My však jim nemůžeme dát za ženy své dcery.“ Izraelci se totiž zapřísahali: „Buď proklet, kdo by dal ženu Benjamínovi!“
#21:19 Pak řekli: „Hle, rok co rok bývá Hospodinova slavnost v Šílu, které je severně od Bét-elu, východně od silnice vystupující z Bét-elu do Šekemu a jižně od Lebóny.“
#21:20 Benjamínovcům přikázali: „Skryjte se ve vinicích jako zálohy.
#21:21 Jak uvidíte, že šíloské dcery vycházejí v průvodu k tanečním rejům, vyrazte z vinic a uchvaťte si každý ženu ze šíloských dcer. Pak odejděte do benjamínské země.
#21:22 Kdyby přišli jejich otcové nebo bratři a chtěli před námi vést spor, řekneme jim: Smilujte se nad nimi kvůli nám. Nevzali jsme v boji ženu pro každého. Vy jste jim je přece nedali, abyste se teď provinili.“
#21:23 Benjamínovci to tak udělali a unesli si ženy podle svého počtu z tančících dívek, které uloupili. Pak odešli a vrátili se do svého dědictví, vystavěli města a usadili se v nich.
#21:24 Toho času se odtamtud rozešli též Izraelci, každý ke svému kmeni a ke své čeledi, každý odešel do svého dědictví.
#21:25 V těch dnech neměli v Izraeli krále. Každý dělal, co uznal za správné.  

\book{Ruth}{Ruth}
#1:1 Za dnů, kdy soudili soudcové, nastal v zemi hlad. Tehdy odešel jeden muž z judského Betléma se svou ženou a dvěma syny, aby pobýval jako host na Moábských polích.
#1:2 Jmenoval se Elímelek, jeho žena Noemi a dva jeho synové Machlón a Kiljón. Byli to Efratejci z judského Betléma. Přišli na Moábská pole a přebývali tam.
#1:3 Ale Noemin muž Elímelek zemřel a ona zůstala s oběma syny sama.
#1:4 Ti se oženili s Moábkami. Jedna se jmenovala Orpa, druhá Rút. Sídlili tam asi deset let.
#1:5 Oba, Machlón i Kiljón, rovněž zemřeli, a tak ta žena zůstala sama, bez dětí i bez muže.
#1:6 Proto se přichystala se svými snachami k návratu z Moábských polí. Uslyšela totiž na Moábských polích, že Hospodin se opět přiklonil ke svému lidu a dal mu chléb.
#1:7 Odešla tedy se svými snachami z místa, kde přebývala. Když se vracely do judské země,
#1:8 vybídla cestou Noemi obě své snachy: „Jděte, vraťte se každá do domu své matky. Nechť vám Hospodin prokáže milosrdenství, jako jste je vy prokazovaly zemřelým i mně.
#1:9 Kéž vám Hospodin dá, abyste každá našla odpočinutí v domě svého muže.“ A políbila je. Rozplakaly se hlasitě
#1:10 a namítaly jí: „Nikoli, vrátíme se s tebou k tvému lidu.“
#1:11 Ale Noemi jim domlouvala: „Jen se vraťte, mé dcery! Proč byste se mnou chodily? Cožpak mohu ještě zrodit syny, aby se stali vašimi muži?
#1:12 Vraťte se, mé dcery, jděte! Vždyť už jsem na vdávání stará. I kdybych si řekla, že mám ještě naději, a kdybych se hned této noci vdala a porodila syny,
#1:13 čekaly byste proto, až by dospěli? Zdráhaly byste se proto vdát? Nikoli, mé dcery. Můj úděl je pro vás příliš trpký: Dolehla na mne Hospodinova ruka.“
#1:14 Tu se rozplakaly ještě hlasitěji. Orpa políbila svou tchyni na rozloučenou, avšak Rút se k ní přimkla.
#1:15 Noemi jí řekla: „Hle, tvá švagrová se vrací ke svému lidu a ke svým bohům. Vrať se také, následuj svou švagrovou!“
#1:16 Ale Rút jí odvětila: „Nenaléhej na mne, abych tě opustila a vrátila se od tebe. Kamkoli půjdeš, půjdu, kdekoli zůstaneš, zůstanu. Tvůj lid bude mým lidem a tvůj Bůh mým Bohem.
#1:17 Kde umřeš ty, umřu i já a tam budu pochována. Ať se mnou Hospodin udělá, co chce! Rozdělí nás od sebe jen smrt.“
#1:18 Když Noemi viděla, že Rút je odhodlána jít s ní, přestala ji přemlouvat.
#1:19 Tak šly obě, až došly do Betléma. Když přišly do Betléma, shluklo se kolem nich celé město. Ženy se ptaly: „Je toto Noemi?“
#1:20 Odvětila jim: „Nenazývejte mě Noemi (to je Rozkošná). Nazývejte mě Mara (to je Trpká), neboť Všemohoucí mi připravil velmi trpký úděl.
#1:21 Odcházela jsem s plnou náručí, ale Hospodin mě přivádí zpět s prázdnou. Jak byste mě mohly nazývat Noemi, když je Hospodin proti mně a když mi Všemohoucí určil zlý úděl?“
#1:22 Tak se Noemi vrátila a s ní se navrátila z Moábských polí její snacha, moábská Rút. Přišly do Betléma, když začínala sklizeň ječmene. 
#2:1 Noemi měla příbuzného z manželovy strany, významného muže z Elímelekovy čeledi. Jmenoval se Bóaz.
#2:2 Moábská Rút řekla Noemi: „Ráda bych šla na pole sbírat klasy za někým, u koho dojdu přízně.“ Noemi jí odpověděla: „Jdi, má dcero.“
#2:3 Šla tedy, přišla na pole a sbírala za ženci klasy. Shodou okolností patřil ten díl pole Bóazovi z Elímelekovy čeledi.
#2:4 Tu přišel z Betléma Bóaz a pozdravil žence: „Hospodin s vámi.“ Odpověděli: „Hospodin ti žehnej.“
#2:5 Bóaz se otázal svého služebníka, který dozíral na žence: „Čí je to dívka?“
#2:6 Služebník, který dozíral na žence, odpověděl: „To je moábská dívka, která se vrátila s Noemi z Moábských polí.
#2:7 Optala se: ‚Mohla bych sbírat a paběrkovat za ženci mezi snopy?‘ Jak přišla, zůstala tu od samého rána až do této chvíle. Jen trochu si tady oddechla.“
#2:8 Bóaz oslovil Rút: „Poslyš, má dcero. Nechoď sbírat na jiné pole a neodcházej odtud. Přidrž se mých děveček.
#2:9 Podívej se vždy, na které části pole budou sklízet, a jdi za nimi. Poručil jsem služebníkům, aby tě neobtěžovali. Budeš-li mít žízeň, jdi k nádobám a napij se vody, kterou služebníci načerpají.“
#2:10 Tu padla na tvář, poklonila se k zemi, a otázala se ho: „Jak to, že jsem u tebe došla přízně, že se mě ujímáš, ačkoli jsem cizinka?“
#2:11 Bóaz jí odpověděl: „Jsem dobře zpraven o všem, co jsi po smrti svého muže učinila pro svou tchyni, že jsi opustila otce a matku i rodnou zemi a odešla jsi k lidu, který jsi dříve neznala.
#2:12 Nechť ti Hospodin odplatí za tvůj skutek. Ať tě bohatě odmění Hospodin, Bůh Izraele, pod jehož křídla ses přišla ukrýt!“
#2:13 Ona řekla: „Kéž bys mi i nadále projevoval přízeň, pane. Potěšil jsi mě, že jsi se svou otrokyní mluvil přívětivě, ačkoli se nemohu rovnat žádné z tvých otrokyň.“
#2:14 Když byl čas k jídlu, řekl jí Bóaz: „Přistup blíž, pojez chleba a namáčej si sousta ve víně.“ Přisedla si k žencům a on jí nabídl pražené zrní. Najedla se dosyta a ještě jí zbylo.
#2:15 Když vstala, aby sbírala dál, přikázal Bóaz svým služebníkům: „Bude-li sbírat i mezi snopy, nevyčítejte jí to.
#2:16 Ano, upouštějte pro ni klasy z hrstí a nechávejte je ležet. Jen ať sbírá, neokřikujte ji.“
#2:17 Sbírala tedy na tom poli až do večera. Pak vymlátila to, co nasbírala, a byla toho asi éfa ječmene.
#2:18 Odnesla si jej do města. Když její tchyně spatřila, co nasbírala, a když Rút vyňala i to, co jí zbylo po nasycení, a dala jí,
#2:19 otázala se: „Kde jsi dnes sbírala? Kde jsi pracovala? Požehnán buď ten, kdo se tě ujal.“ Pověděla tchyni, u koho pracovala: „Muž, u něhož jsem dnes pracovala, se jmenuje Bóaz.“
#2:20 Tu řekla Noemi své snaše: „Požehnán buď od Hospodina, který neodňal své milosrdenství od živých ani od mrtvých.“ A pokračovala: „Ten muž je náš blízký příbuzný, patří k našim zastáncům.“
#2:21 Nato řekla moábská Rút: „Dokonce mě vybídl: Přidrž se mých služebníků, dokud nebudou hotovi se sklizní všeho, co mi patří.“
#2:22 Noemi své snaše Rút přisvědčila: „Dobře, má dcero, že chceš chodit s jeho děvečkami. Aspoň na tebe nebudou na cizím poli dorážet.“
#2:23 Přidržela se tedy Bóazových děveček a sbírala, dokud neskončila sklizeň ječmene a pšenice. A bydlela s tchyní. 
#3:1 Potom jí její tchyně Noemi řekla: „Neměla bych ti, má dcero, vyhledat odpočinutí, aby ti bylo dobře?
#3:2 Hleď, což není Bóaz, s jehož děvečkami jsi byla, náš příbuzný? Právě dnes v noci bude na humně převívat ječmen.
#3:3 Umyj se, potři se mastí, přehoď si plášť a sejdi na humno. Nedej se však tomu muži poznat, dokud nedojí a nedopije.
#3:4 Až si lehne a ty zjistíš, kde leží, půjdeš, odkryješ mu plášť v nohách a lehneš si tam. On ti pak poví, co máš učinit.“
#3:5 Odpověděla jí: „Vykonám všechno, co mi říkáš.“
#3:6 Sešla na humno a udělala všechno, jak jí tchyně přikázala.
#3:7 Bóaz se najedl, napil a byl dobré mysli. Pak si šel lehnout na kraj hromady obilí. Ona se přikradla, odkryla mu plášť v nohách a lehla si.
#3:8 O půlnoci se ten muž vyděsil, trhl sebou a vidí - v nohách mu leží žena.
#3:9 Otázal se: „Kdo jsi?“ Odpověděla: „Jsem Rút, tvá služebnice. Rozprostři nad svou služebnicí křídlo svého pláště, vždyť jsi zastánce.“
#3:10 Nato jí řekl: „Požehnána buď od Hospodina, má dcero. Projevila jsi teď větší oddanost než dříve, že nechodíš za mládenci, ani nuznými ani bohatými.
#3:11 Už se neboj, má dcero! Udělám pro tebe všechno, oč si říkáš. Všechen můj lid v bráně ví, že jsi žena znamenitá.
#3:12 Ano, jsem vskutku váš zastánce. Je však ještě jiný zastánce, bližší příbuzný než já.
#3:13 Zůstaň tu přes noc. Ráno, bude-li on chtít, dobrá, ať se tě zastane; nebude-li ochoten být ti zastáncem, zastanu se tě sám, jakože živ je Hospodin. Spi klidně až do rána.“
#3:14 Tak spala u jeho nohou až do rána. Ale vstala dříve, než by kdo mohl poznat druhého, protože Bóaz řekl: „Jen ať se nikdo nedoví, že přišla na humno žena.“
#3:15 A dodal: „Podej mi loktuši, co máš na sobě, a nastav ji.“ Když ji nastavila, odměřil jí šest měr ječmene a vložil na ni. Pak odešel do města
#3:16 a ona odešla k tchyni. Ta se otázala: „S jakou přicházíš, má dcero?“ I vyprávěla jí všechno, jak s ní ten muž jednal.
#3:17 A dodala: „Těchto šest měr ječmene mi dal on. Řekl totiž: Nesmíš přijít ke své tchyni s prázdnou.“
#3:18 Noemi jí pravila: „Jen vyčkej, má dcero, a poznáš, jak to dopadne. Ten muž si nedá pokoj a dovede tu záležitost ke konci ještě dnes.“ 
#4:1 Bóaz vystoupil k bráně a posadil se tam. Tu šel kolem zastánce, o němž Bóaz mluvil. Vybídl ho: „Člověče, zastav se a posaď se tu.“ A on se zastavil a posadil se.
#4:2 Bóaz pak vybral deset mužů z městských starších a požádal je: „Zasedněte zde.“ A oni zasedli.
#4:3 Zastánci pak řekl: „Noemi, která se vrátila z Moábských polí, chce prodat díl pole, který patřil našemu bratru Elímelekovi.
#4:4 Řekl jsem si, že ti to dám na vědomí a vybídnu tě, abys jej koupil v přítomnosti těch, kteří tu zasedají, před staršími mého lidu. Chceš-li použít svého práva k vykoupení, tedy jej vykup, nechceš-li, oznam mi to. Vím, že kromě tebe není bližšího zastánce. Já jsem po tobě.“ On odpověděl: „Vykoupím jej.“
#4:5 Bóaz řekl: „V den, kdy koupíš od Noemi pole, kupuješ je i od moábské Rút, ženy po zemřelém, se závazkem zachovat jméno zemřelého v jeho dědictví.“
#4:6 Tu řekl zastánce: „Nemohu je vykoupit pro sebe, aniž bych zničil vlastní dědictví. Použij pro sebe mého výkupního práva; já vykoupit nemohu.“
#4:7 V Izraeli tomu bývalo odedávna při vykupování nebo při výměnném obchodu takto: Každé jednání se stvrzovalo tím, že si jeden vyzul střevíc a dal jej druhému. To byl v Izraeli způsob stvrzování.
#4:8 Zastánce tedy řekl Bóazovi: „Kup si to sám.“ A zul si střevíc.
#4:9 Bóaz pak řekl starším a všemu lidu: „Dnes jste svědky, že jsem koupil od Noemi všechno, co patřilo Elímelekovi, i vše, co patřilo Kiljónovi a Machlónovi.
#4:10 Koupí jsem získal za manželku i moábskou Rút, ženu Machlónovu, abych zachoval jméno zemřelého v jeho dědictví. Tak nebude vyhlazeno jméno zemřelého z kruhu jeho bratří ani z brány jeho rodiště. Jste toho dnes svědky.“
#4:11 Všechen lid, který byl v bráně, i starší odpověděli: „Jsme svědky. Kéž dá Hospodin, aby žena, která přichází do tvého domu, byla jako Ráchel a Lea, které obě zbudovaly dům izraelský. Počínej si zdatně v Efratě a zachovej jméno v Betlémě.
#4:12 Nechť je tvůj dům skrze potomstvo, které ti dá Hospodin z této dívky, jako dům Peresa, jehož Támar porodila Judovi.“
#4:13 I vzal si Bóaz Rút a stala se jeho ženou. Vešel k ní a Hospodin jí dopřál, že otěhotněla a porodila syna.
#4:14 Tu řekly ženy Noemi: „Požehnán buď Hospodin, který tě odedneška nenechává bez zastánce, jehož jméno se bude ozývat v Izraeli.
#4:15 On ti vrátí smysl života, bude o tebe ve stáří pečovat. Vždyť jej porodila tvá snacha, která tě tolik miluje. Ta je pro tebe lepší než sedm synů.“
#4:16 Noemi vzala dítě, položila si je na klín a stala se mu chůvou.
#4:17 Sousedky mu daly jméno. Řekly: „Noemi se narodil syn“, a pojmenovaly jej Obéd (to je Ctitel Hospodinův). To je otec Jišaje, otce Davidova.
#4:18 Toto je rodopis Peresův: Peres zplodil Chesróna,
#4:19 Chesrón zplodil Ráma, Rám zplodil Amínadaba,
#4:20 Amínadab zplodil Nachšóna, Nachšón zplodil Salmu,
#4:21 Salmón zplodil Bóaza, Bóaz zplodil Obéda,
#4:22 Obéd zplodil Jišaje a Jišaj zplodil Davida.  

\book{I Samuel}{1Sam}
#1:1 Byl jeden muž z Ramatajim-sófímu, z Efrajimského pohoří; jmenoval se Elkána. Byl to Efratejec, syn Jeróchama, syna Elíhúa, syna Tochúa, syna Súfova.
#1:2 Měl dvě ženy: jedna se jmenovala Chana a druhá Penina. Penina měla děti, Chana děti neměla.
#1:3 Ten muž putoval rok co rok ze svého města, aby se klaněl Hospodinu zástupů a obětoval mu v Šílu. Tam byli Hospodinovými kněžími dva synové Élího, Chofní a Pinchas.
#1:4 Když nastal den, kdy Elkána obětoval, dával své ženě Penině i všem jejím synům a dcerám díly z oběti;
#1:5 Chaně pak dával dvojnásobný díl, protože Chanu miloval; Hospodin však uzavřel její lůno.
#1:6 Její protivnice ji ustavičně urážela, že Hospodin uzavřel její lůno, jen aby jí dráždila.
#1:7 Tak tomu bývalo každého roku. Pokaždé, když putovala do Hospodinova domu, tak ji urážela, že Chana pro pláč ani nejedla.
#1:8 Její muž Elkána ji uklidňoval: „Chano, proč pláčeš? Proč nejíš? Proč jsi tak ztrápená? Což já pro tebe neznamenám víc než deset synů?“
#1:9 Jednou, když v Šílu pojedli a popili, Chana vstala, zatímco kněz Élí seděl na stolci u veřejí Hospodinova chrámu,
#1:10 a v hořkosti duše se modlila k Hospodinu a usedavě plakala.
#1:11 Složila slib. Řekla: „Hospodine zástupů, jestliže opravdu shlédneš na ponížení své služebnice a rozpomeneš se na mne, jestliže na svou služebnici nezapomeneš, ale daruješ své služebnici mužského potomka, daruji jej tobě, Hospodine, na celý život; břitva se jeho hlavy nedotkne.“
#1:12 Když se před Hospodinem tolik modlila, Élí dával pozor na její ústa.
#1:13 Chana hovořila jen v srdci a pouze její rty se pohybovaly, ale její hlas nebylo slyšet, takže ji Élí pokládal za opilou.
#1:14 Řekl jí proto: „Jak dlouho budeš opilá? Zanech už vína!“
#1:15 Ale Chana odpověděla: „Nikoli, můj pane; jsem žena hluboce zarmoucená. Nepila jsem víno ani jiný opojný nápoj, pouze jsem vylévala před Hospodinem svou duši.
#1:16 Nepokládej svou služebnici za ženu ničemnou. Vždyť až dosud jsem mluvila ze své velké beznaděje a žalosti.“
#1:17 Élí odpověděl: „Jdi v pokoji. Bůh Izraele ti dá, zač jsi ho tak naléhavě prosila.“
#1:18 Ona na to řekla: „Kéž tvá služebnice najde u tebe milost!“ Potom ta žena šla svou cestou, pojedla a její tvář už nebyla smutná.
#1:19 Za časného jitra se poklonili před Hospodinem a vraceli se. Když přišli do svého domu do Rámy, Elkána poznal svou ženu Chanu a Hospodin se na ni rozpomenul.
#1:20 Chana otěhotněla, a než uplynul rok, porodila syna a pojmenovala ho Samuel (to je Vyslyšel Bůh). Řekla: „Vždyť jsem si ho vyprosila od Hospodina.“
#1:21 Ten muž Elkána putoval opět s celým svým domem, aby Hospodinu obětoval výroční oběť a splnil svůj slib.
#1:22 Ale Chana s ním neputovala. Řekla svému muži: „Až bude chlapec odstaven, přivedu ho, aby se ukázal před Hospodinem a zůstal tam navždy.“
#1:23 Nato jí její muž Elkána odpověděl: „Učiň, co pokládáš za dobré. Zůstaň, dokud ho neodstavíš. Kéž Hospodin utvrdí své slovo!“ Žena tedy zůstávala doma a kojila svého synka, dokud ho neodstavila.
#1:24 Když ho odstavila, vzala ho s sebou, a s ním tři býčky, jednu éfu bílé mouky a měch vína, a uvedla ho do Hospodinova domu v Šílu. Chlapec byl ještě malý.
#1:25 Porazili býčka a uvedli chlapce k Élímu.
#1:26 Chana řekla: „Dovol, můj pane, při tvém životě, můj pane, já jsem ta žena, která tu stála u tebe a modlila se k Hospodinu.
#1:27 Modlila jsem se za tohoto chlapce a Hospodin mi dal, zač jsem ho tak naléhavě prosila.
#1:28 Vyprosila jsem si ho přece od Hospodina, aby byl jeho po všechny dny, co bude živ. Je vyprošený pro Hospodina.“ I poklonil se tam Hospodinu. 
#2:1 Chana se takto modlila: „Mé srdce jásotem oslavuje Hospodina, můj roh se zvedá dík Hospodinu. Má ústa se otevřela proti nepřátelům, raduji se ze tvé spásy.
#2:2 Nikdo není svatý mimo Hospodina, není nikoho krom tebe, nikdo není skálou jako náš Bůh.
#2:3 Nechte už těch povýšených řečí, urážka ať z úst vám neunikne! Vždyť Hospodin je Bůh vševědoucí, neobstojí před ním lidské činy.
#2:4 Zlomen je luk bohatýrů, ale ti, kdo klesali, jsou opásáni statečností.
#2:5 Sytí se dávají najmout za chléb, hladoví přestali lačnět. Neplodná posedmé rodí, syny obdařená chřadne.
#2:6 Hospodin usmrcuje i obživuje, do podsvětí přivádí a vyvádí též odtud.
#2:7 Hospodin ochuzuje i zbohacuje, ponižuje a též povyšuje.
#2:8 Nuzného pozvedá z prachu, z kalu vytahuje ubožáka; posadí je v kruhu knížat a za dědictví jim dá trůn slávy. Vždyť pilíře země patří Hospodinu, on sám založil svět na nich.
#2:9 On střeží nohy svých věrných, ale svévolníci zajdou ve tmách; svou silou se nikdo neprosadí.
#2:10 Ti, kdo s Hospodinem vedou spor, se zděsí, až on z nebe na ně zaburácí. Hospodin povede při i s dálavami země. Udělí moc svému králi, roh svého pomazaného zvedne.“
#2:11 Elkána odešel do svého domu do Rámy. A chlapec konal službu Hospodinu pod dohledem kněze Élího.
#2:12 Synové Élího byli ničemníci, neznali se k Hospodinu.
#2:13 Uplatňovali vůči lidu tento kněžský řád: Kdykoli někdo připravil obětní hod, přicházel kněžský mládenec s trojzubou vidlicí, právě když se maso vařilo.
#2:14 Vrazil ji do kotle nebo do hrnce, do kotlíku nebo do kotlíku nebo pekáče, a co vidlice zachytila, to si bral kněz pro sebe. Tak to dělávali všem z Izraele, kteří tam do Šíla přicházeli.
#2:15 Dokonce dříve než obrátili tuk v obětní dým, přicházel kněžský mládenec a říkal obětujícímu muži: „Dej knězi maso na pečeni. Nepřijme od tebe maso vařené, ale syrové.“
#2:16 Když mu ten člověk řekl: „Napřed se musí obrátit tuk v obětní dým, pak si vezmi, po čem toužíš“, odpovídal: „Nikoli. Dej to hned. Nedáš-li, vezmu si to násilím.“
#2:17 Hřích těch mládenců byl před Hospodinem nesmírně veliký, protože lidé znevažovali Hospodinovy obětní dary.
#2:18 Ale Samuel konal službu před Hospodinem, mládeneček přepásaný lněným efódem.
#2:19 Jeho matka mu dělávala malou pláštěnku a rok co rok mu ji přinášela, když putovala se svým mužem, aby obětovali výroční oběť.
#2:20 Élí žehnal Elkánovi a jeho manželce. Říkal: „Nechť tě Hospodin zahrne potomstvem z této ženy místo vyprošeného, který byl vyprošen pro Hospodina.“ Pak odcházeli domů.
#2:21 Hospodin navštívil Chanu a ta otěhotněla a porodila tři syny a dvě dcery. Mládeneček Samuel však vyrůstal při Hospodinu.
#2:22 Élí byl již velmi starý. Slyšel o všem, čeho se dopouštěli jeho synové na celém Izraeli, i o tom, že obcovali se ženami konajícími službu u vchodu do stanu setkávání.
#2:23 Říkal jim: „Proč děláte takové věci? Ode všeho lidu slyším o vás samé zlé věci.
#2:24 To nejde, moji synové! Není to dobrá zpráva, kterou slyším; svádíte Hospodinův lid k přestoupením.
#2:25 Jestliže hřeší člověk proti člověku, je rozhodčím nad ním Bůh. Zhřeší-li však člověk proti Hospodinu, kdo nad ním bude rozhodčím?“ Ale oni svého otce neposlouchali. Hospodin tedy rozhodl, že propadnou smrti.
#2:26 Mládeneček Samuel však prospíval a byl oblíben u Hospodina i u lidí.
#2:27 Tu přišel k Élímu muž Boží a řekl mu: „Toto praví Hospodin: Což jsem se nezjevil právě tvému rodu, když byli v Egyptě v područí domu faraónova?
#2:28 Vyvolil jsem si jej ze všech izraelských kmenů za kněze, aby na mém oltáři přinášel zápalné oběti, aby pálil kadidlo a nosil přede mnou efód. Tvému rodu jsem dal všechny ohnivé oběti synů Izraele.
#2:29 Proč pošlapáváte můj obětní hod a dar, které jsem přikázal pro tento příbytek? Ctil jsi své syny více než mne. Ztloustli jste z prvotin všech obětních darů Izraele, mého lidu.
#2:30 Proto slyš výrok Hospodina, Boha Izraele. Prohlásil jsem sice, že tvůj dům a tvůj rod budou přede mnou konat svůj úřad věčně. Ale nyní je toto Hospodinův výrok: Jsem toho dalek! Ty, kdo mě ctí, poctím, ale ti, kdo mnou pohrdají, budou zlehčeni.
#2:31 Hle, přicházejí dny, kdy odetnu rámě tobě i tvému rodu, takže tvůj dům zůstane bez starce.
#2:32 Spatříš soužení, jež dolehne na příbytek při všem dobrém, co Bůh učiní Izraeli. Už nikdy nebude v tvém domě stařec;
#2:33 jen jediného z tvých od svého oltáře nevyhladím. Tvé oči vyhasnou a tvá duše se naplní steskem; všichni, kdo rozmnoží tvůj dům, zemřou v mužném věku.
#2:34 Co přijde na oba tvé syny, Chofního a Pinchasa, bude pro tebe znamením: oba zemřou v týž den.
#2:35 Sobě však ustanovím věrného kněze, který bude konat, co mám na srdci a v mysli. Vybuduji mu trvalý dům, kde bude vykonávat po všechny dny svůj úřad před mým pomazaným.
#2:36 Každý pak, kdo zbude v tvém domě, přijde se mu poklonit kvůli kousku stříbra a bochníku chleba a řekne: ‚Připoj mě prosím k některé skupině kněží, abych s nimi mohl jíst sousto chleba.‘“ 
#3:1 Mládenec Samuel vykonával službu Hospodinovu pod dohledem Élího. V těch dnech bylo Hospodinovo slovo vzácné, prorocké vidění nebylo časté.
#3:2 Jednoho dne ležel Élí na svém místě. Oči mu začaly pohasínat, takže neviděl.
#3:3 Boží kahan ještě nezhasl a Samuel ležel v Hospodinově chrámě, kde byla Boží schrána.
#3:4 Hospodin zavolal na Samuela. On odpověděl: „Tu jsem.“
#3:5 Běžel k Élímu a řekl: „Tu jsem, volal jsi mě.“ On však řekl: „Nevolal jsem, lehni si zase.“ Šel si tedy lehnout.
#3:6 Ale Hospodin zavolal Samuela znovu. Samuel vstal, šel k Élímu a řekl: „Tu jsem, volal jsi mě.“ On však řekl: „Nevolal jsem, můj synu, lehni si zase.“
#3:7 Samuel ještě Hospodina neznal a Hospodinovo slovo mu ještě nebylo zjeveno.
#3:8 A znovu, potřetí, zavolal Hospodin Samuela. On vstal, šel k Élímu a řekl: „Tu jsem, volal jsi mě.“ Tu Élí pochopil, že mládence volá Hospodin.
#3:9 I řekl Élí Samuelovi: „Jdi si lehnout; jestliže tě zavolá, řekneš: Mluv, Hospodine, tvůj služebník slyší.“ Samuel si tedy šel lehnout na své místo.
#3:10 A Hospodin přišel, stanul a zavolal jako předtím: „Samueli, Samueli!“ Samuel odpověděl: „Mluv, tvůj služebník slyší.“
#3:11 Hospodin řekl Samuelovi: „Hle, já učiním v Izraeli něco takového, že bude znít v obou uších každému, kdo o tom uslyší.
#3:12 Onoho dne uvedu na Élího všechno, co jsem ohlásil jeho domu, od začátku až do konce.
#3:13 Oznámil jsem mu, že jeho dům odsuzuji navěky pro nepravost, o které věděl: Jeho synové přivolávají na sebe zlořečení, on však proti nim nezakročil.
#3:14 Proto jsem o Élího domu přísahal: Dům Élího nebude nikdy zbaven viny ani obětním hodem ani obětním darem.“
#3:15 Samuel ležel až do jitra. Pak otevřel dveře Hospodinova domu. Samuel se bál oznámit Élímu to vidění.
#3:16 Élí si však Samuela zavolal a pravil: „Samueli, můj synu!“ On odpověděl: „Tu jsem.“
#3:17 Otázal se: „Co to bylo, o čem s tebou mluvil? Nic prosím přede mnou nezatajuj! Ať s tebou Bůh udělá, co chce, jestliže přede mnou zatajíš něco z toho všeho, o čem s tebou mluvil!“
#3:18 Samuel mu tedy oznámil všechno a nic před ním nezatajil. A on řekl: „On je Hospodin. Ať učiní, co je dobré v jeho očích.“
#3:19 Tak Samuel vyrůstal a Hospodin byl s ním. Nedopustil, aby některé z jeho slov padlo na zem.
#3:20 Celý Izrael od Danu až k Beer-šebě poznal, že Samuel má od Hospodina prorocké pověření.
#3:21 Hospodin se mu dával i nadále vidět v Šílu; Hospodin se totiž v Šílu zjevoval Samuelovi svým slovem. 
#4:1 Samuelovo slovo se týkalo celého Izraele. Izrael vytáhl do boje proti Pelištejcům. Utábořili se u Eben-ezeru (to je u Kamene pomoci); Pelištejci se utábořili v Afeku
#4:2 a seřadili se proti Izraeli. Rozpoutal se boj a Izrael byl od Pelištejců poražen; ti pobili v bitvě na poli asi čtyři tisíce mužů.
#4:3 Když se lid sešel v táboře, řekli izraelští starší: „Proč nás Hospodin dnes před Pelištejci porazil? Vezměme si ze Šíla schránu Hospodinovy smlouvy, ať přijde mezi nás a zachrání nás ze spárů našich nepřátel.“
#4:4 Lid tedy poslal do Šíla a přinesli odtud schránu smlouvy Hospodina zástupů, trůnícího na cherubech. Se schránou Boží smlouvy tam byli oba synové Élího, Chofní a Pinchas.
#4:5 Když schrána Hospodinovy smlouvy přicházela do tábora, spustil všechen Izrael mohutný válečný ryk, až země duněla.
#4:6 Pelištejci uslyšeli válečný ryk a ptali se: „Co znamená tento mohutný válečný ryk v táboře Hebrejů?“ Pak zjistili, že Hospodinova schrána přišla do tábora.
#4:7 Tu se Pelištejci začali bát. Řekli: „Bůh přišel do jejich tábora.“ A naříkali: „Běda nám! Dosud se nic takového nestalo.
#4:8 Běda nám! Kdo nás vysvobodí z rukou těch vznešených bohů? To jsou přece ti bohové, kteří pobili Egypt všelijakými ranami na poušti.
#4:9 Vzchopte se! Vzmužte se, Pelištejci, ať neotročíte Hebrejům, jako oni otročili vám. Vzmužte se a dejte se do boje!“
#4:10 Pelištejci tedy bojovali a Izrael byl poražen; každý utíkal ke svému stanu. Byla to převeliká rána. Z Izraele padlo třicet tisíc pěších.
#4:11 I Boží schrána byla vzata a oba synové Élího, Chofní a Pinchas, zahynuli.
#4:12 Onoho dne nějaký Benjamínec zběhl z bitvy a dorazil do Šíla; měl roztržený šat a na hlavě hlínu.
#4:13 Když přiběhl, Élí právě seděl na stolci a vyhlížel na cestu, neboť se v srdci třásl o Boží schránu. Muž přiběhl, aby v městě oznámil porážku. Celé město se dalo do úpěnlivého křiku.
#4:14 Élí slyšel ten úpěnlivý křik a ptal se: „Co je to za hrozný hluk?“ Muž přispěchal a oznámil to Élímu.
#4:15 Élímu bylo devadesát osm let, oči měl zakalené a neviděl.
#4:16 Muž řekl Élímu: „Přicházím z bitvy. Ano, utekl jsem dnes z bitvy.“ Élí se zeptal: „Co se stalo, můj synu?“
#4:17 Tu posel odpověděl: „Izrael utekl před Pelištejci. Lid utrpěl zdrcující porážku. I oba tvoji synové, Chofní a Pinchas, jsou mrtvi. Boží schrána byla vzata.“
#4:18 Jakmile se zmínil o Boží schráně, spadl Élí nazad ze stolce k pilíři brány, zlomil si vaz a zemřel; byl to muž starý a těžký. Soudil Izraele čtyřicet let.
#4:19 Sotva jeho snacha, manželka Pinchasova, která byla těhotná a blízká porodu, uslyšela zprávu, že Boží schrána byla vzata a že umřel její tchán i muž, sklonila se a porodila, protože ji přepadly křeče.
#4:20 Ve chvíli, kdy umírala, řekly jí ženy stojící u ní: „Neboj se, porodila jsi syna.“ Ona však neodpověděla, byla bez zájmu.
#4:21 Hocha nazvala Í-kábód (to je Je-po-slávě) a řekla: „Sláva se odstěhovala z Izraele.“ To kvůli Boží schráně, že byla vzata, a kvůli svému tchánu a muži.
#4:22 Řekla: „Sláva se odstěhovala z Izraele, vždyť Boží schrána je vzata.“ 
#5:1 Pelištejci vzali Boží schránu a zanesli ji z Eben-ezeru do Ašdódu.
#5:2 Pelištejci tedy Boží schránu vzali, přenesli ji do domu Dágonova a vystavili ji vedle Dágona.
#5:3 Když Ašdóďané nazítří časně vstali, hle, Dágon ležel tváří k zemi před Hospodinovou schránou. I vzali Dágona a dali ho zpět na jeho místo.
#5:4 Nazítří za časného jitra vstali, a hle, Dágon zase ležel tváří k zemi před Hospodinovou schránou; Dágonova hlava a obě jeho ruce ležely uraženy na prahu. Z Dágona zůstal jen trup.
#5:5 Proto až dodnes nešlapou v Ašdódu Dágonovi kněží a nikdo, kdo vstupuje do Dágonova domu, na Dágonův práh.
#5:6 Hospodinova ruka těžce dolehla na Ašdóďany a způsobila mezi nimi spoušť. Ranila je nádory, Ašdód i jeho území.
#5:7 Když ašdódští muži viděli, jak se věci mají, rozhodli: „Schrána Boha Izraele nemůže u nás zůstat. Jeho ruka tvrdě dopadla na nás i na našeho boha Dágona.“
#5:8 Obeslali všechna pelištejská knížata, shromáždili je k sobě a tázali se: „Co uděláme se schránou Boha Izraele?“ Oni odvětili: „Nechť je schrána Boha Izraele dopravena do Gatu.“ I dopravili tam schránu Boha Izraele.
#5:9 Avšak sotva ji tam dopravili, způsobila Hospodinova ruka ve městě nesmírnou hrůzu a ranila muže města, malé i velké; vyrazily se jim nádory.
#5:10 Proto odeslali Boží schránu do Ekrónu. Sotva však přišla Boží schrána do Ekrónu, dali se Ekróňané do úpěnlivého křiku: „Dopravili k nám schránu Boha Izraele, aby umořili nás i náš lid!“
#5:11 Obeslali všechna pelištejská knížata, shromáždili je a rozhodli: „Odešlete schránu Boha Izraele zpět na její místo, ať nás a náš lid neumoří.“ Celé město zachvátila smrtelná hrůza. Boží ruka tam dolehla velmi těžce.
#5:12 Muži, kteří nezemřeli, byli raněni nádory. Volání města o pomoc stoupalo k nebesům. 
#6:1 Po sedm měsíců byla Hospodinova schrána na poli Pelištejců.
#6:2 Pelištejci svolali kněze a věštce a ptali se: „Co máme s Hospodinovou schránou udělat? Sdělte nám, jakým způsobem ji odeslat na její místo.“
#6:3 Ti jim odpověděli: „Jestliže budete posílat schránu Boha Izraele, nesmíte ji poslat s prázdnou. Musíte jí přinést oběť za provinění. Teprve pak budete uzdraveni a bude vám zřejmé, proč se jeho ruka od vás neodvracela.“
#6:4 Dále se tázali: „Jakou oběť za provinění jí máme přinést?“ A oni řekli: „Pět zlatých nádorů a pět zlatých myší podle počtu pelištejských knížat, neboť stejná pohroma postihla všechny, i vaše knížata.
#6:5 Uděláte napodobeniny svých nádorů a napodobeniny svých myší, které přinášejí zemi zkázu, a vzdáte přitom slávu Bohu Izraele. Snad nadlehčí svou ruku, která dolehla na vás, na vaše bohy a na vaši zemi.
#6:6 Proč byste byli neoblomní v srdci, jako byli neoblomní Egypťané a farao? Když proti nim Bůh zasáhl, zdalipak je nepustili, aby šli?
#6:7 Nyní tedy udělejte nový vůz a přiveďte dvě krávy po otelení, na které ještě nebylo vloženo jho, zapřáhněte krávy do vozu, ale jejich telata odveďte od nich zpět domů.
#6:8 Pak vezmete Hospodinovu schránu a dáte ji na vůz; zlaté předměty, které jí přinesete jako oběť za provinění, vložíte do vaku po jejím boku a pustíte ji, ať jede,
#6:9 a uvidíte: Jestliže bude směřovat ke svému území do Bét-šemeše, pak nám způsobil toto veliké zlo on; jestliže nikoli, tedy poznáme, že nás neranila jeho ruka, ale stalo se nám to náhodou.“
#6:10 Mužové tak učinili. Vzali dvě krávy po otelení, zapřáhli je do vozu a jejich telata zadrželi doma;
#6:11 na vůz vložili Hospodinovu schránu, vak a do něho zlaté myši a napodobeniny svých hlíz.
#6:12 Krávy zamířily přímou cestou na Bét-šemeš, šly stále po téže silnici a bučely, neuchýlily se napravo ani nalevo. Pelištejská knížata šla za nimi až k území Bét-šemeše.
#6:13 Bétšemešané právě sklízeli v dolině pšenici. Tu se rozhlédli a spatřili schránu a zaradovali se, že ji vidí.
#6:14 Vůz dojel na pole Jóšuy Bétšemešského a tam se zastavil. Byl tam veliký kámen. I rozštípali dříví vozu a obětovali krávy Hospodinu v zápalnou oběť.
#6:15 Lévijci složili Hospodinovu schránu a vak, který byl při ní a v němž byly zlaté předměty, a položili je na ten veliký kámen. Onoho dne přinesli bétšemešští muži zápalné oběti a připravili obětní hody Hospodinu.
#6:16 Pět pelištejských knížat to vidělo a vrátilo se onoho dne do Ekrónu.
#6:17 Toto jsou zlaté hlízy, které odvedli Pelištejci Hospodinu v oběť za provinění: jedna za Ašdód, jedna za Gázu, jedna za Aškalón, jedna za Gat a jedna za Ekrón.
#6:18 Zlaté myši odpovídaly počtu všech pelištejských měst s pěti knížaty, od opevněného města až po venkovskou obec, až po kámen Velikého truchlení, na který uložili Hospodinovu schránu; ten je na poli Jóšuy Bétšemešského dodnes.
#6:19 Hospodin však ranil bétšemešské muže za to, že se podívali do Hospodinovy schrány. Sedmdesát mužů z lidu ranil, a bylo tu padesát tisíc mužů. Proto lid truchlil, že Hospodin postihl lid zdrcující ranou.
#6:20 Muži z Bét-šemeše se ptali: „Kdo může obstát před Hospodinem, tímto svatým Bohem? A ke komu poputuje od nás?“
#6:21 I poslali posly k obyvatelům Kirjat-jearímu se vzkazem: „Pelištejci vrátili Hospodinovu schránu. Přijďte a odneste si ji.“ 
#7:1 Muži kirjatjearímští přišli a Hospodinovu schránu odnesli. Dopravili ji do domu Abínádabova na pahorek a jeho syna Eleazara posvětili, aby u Hospodinovy schrány držel stráž.
#7:2 Ode dne, kdy schrána spočinula v Kirjat-jearímu, uplynulo mnoho dní, celkem dvacet let. Tu celý izraelský dům zatoužil po Hospodinu.
#7:3 Samuel vyzval celý izraelský dům: „Jestliže se chcete celým srdcem obrátit k Hospodinu, odstraňte ze svého středu cizí bohy i aštarty, upněte se srdcem k Hospodinu a služte jenom jemu. On vás vysvobodí z rukou Pelištejců.“
#7:4 Izraelci tedy odstranili baaly a aštarty a sloužili jenom Hospodinu.
#7:5 Samuel vyhlásil: „Shromážděte celý Izrael do Mispy a já se budu za vás modlit k Hospodinu.“
#7:6 Shromáždili se tedy do Mispy, čerpali vodu a vylévali ji před Hospodinem; také se onoho dne postili a říkali tam: „Zhřešili jsme proti Hospodinu.“ A Samuel soudil Izraelce v Mispě.
#7:7 Když se Pelištejci doslechli, že se Izraelci shromáždili do Mispy, vytáhla pelištejská knížata proti Izraeli. Izraelci to uslyšeli a báli se Pelištejců.
#7:8 Naléhali na Samuela: „Nepřestaň za nás úpěnlivě volat k Hospodinu, našemu Bohu, ať nás zachrání z rukou Pelištejců.“
#7:9 Samuel vzal jedno neodstavené jehňátko a obětoval je Hospodinu v zápalnou oběť, celopal, a úpěnlivě volal k Hospodinu za Izraele. A Hospodin mu odpověděl.
#7:10 Když Samuel obětoval zápalnou oběť, Pelištejci vyrazili k bitvě s Izraelci. Ale Hospodin onoho dne zahřměl mocným hlasem proti Pelištejcům a uvedl je ve zmatek. Byli před Izraelem poraženi.
#7:11 Izraelští mužové vytáhli z Mispy, pronásledovali Pelištejce a pobíjeli je až pod Bét-kar.
#7:12 I vzal Samuel jeden kámen a položil jej mezi Mispu a Šén. Dal mu jméno Eben-ezer (to je Kámen pomoci) a prohlásil: „Až potud nám Hospodin pomáhal.“
#7:13 Pelištejci byli pokořeni a nadále už nepronikali na izraelské území. Hospodinova ruka na Pelištejce doléhala po všechny Samuelovy dny.
#7:14 Města, která Pelištejci Izraeli odňali, od Ekrónu až po Gat, připadla zase Izraeli a Izrael vysvobodil z rukou Pelištejců i jejich pomezí. Mezi Izraelem a Emorejci zavládl pokoj.
#7:15 Samuel byl soudcem Izraele po celý svůj život.
#7:16 Rok co rok se vydával na obchůzku do Bét-elu, do Gilgálu a do Mispy a na všech těchto místech soudil Izraele.
#7:17 Pak se vracel do Rámy, neboť tam byl jeho dům, a tam soudil Izraele; tam také vybudoval Hospodinu oltář. 
#8:1 Když Samuel zestárl, ustanovil Izraeli za soudce své syny.
#8:2 Jméno jeho prvorozeného syna bylo Jóel a jméno druhého Abijáš; byli soudci v Beer-šebě.
#8:3 Jeho synové však nechodili jeho cestou, ale propadli lakotě, brali úplatky a převraceli právo.
#8:4 Všichni izraelští starší se tedy shromáždili a přišli k Samuelovi do Rámy.
#8:5 Řekli mu: „Hle, ty jsi už starý a tvoji synové nechodí tvou cestou. Dosaď nyní nad námi krále, aby nás soudil, jako je tomu u všech národů.“
#8:6 Ale Samuelovi se to nelíbilo, že řekli: „Dej nám krále, aby nás soudil.“ I modlil se Samuel k Hospodinu.
#8:7 A Hospodin Samuelovi odpověděl: „Uposlechni lid ve všem, co od tebe žádají. Vždyť nezavrhli tebe, ale zavrhli mne, abych nad nimi nekraloval.
#8:8 Vším, co dělali ode dne, kdy jsem je vyvedl z Egypta, až dodnes, dokazují, že mě opustili a že slouží jiným bohům; tak jednají i vůči tobě.
#8:9 Teď však je uposlechni, ale důrazně je varuj a oznam jim právo krále, který nad nimi bude kralovat.“
#8:10 Samuel pověděl všechna Hospodinova slova lidu, těm, kteří od něho žádali krále.
#8:11 Pravil: „Toto bude právo krále, který nad vámi bude kralovat: Vezme vám syny a zařadí je ke svému vozatajstvu a jezdectvu, aby běhali před jeho vozem.
#8:12 Ustanoví si velitele nad tisíci a nad sty a další, aby pro něho obstarávali orbu a sklizeň, a další, aby pro něho zhotovovali válečnou výzbroj a výstroj jeho vozů.
#8:13 Také dcery vám vezme za mastičkářky, kuchařky a pekařky.
#8:14 Vezme vám nejlepší pole, vinice a olivové háje a dá je svým služebníkům.
#8:15 Z vašich výmlatů a vinic bude vybírat desátky a bude je dávat svým dvořanům a služebníkům.
#8:16 Vezme vám otroky a otrokyně a nejlepší jinochy i osly, aby pro něho pracovali.
#8:17 Bude vybírat desátky z vašich stád, a stanete se jeho otroky.
#8:18 A přijde den, kdy budete úpět kvůli svému králi, kterého jste si vyvolili, ale Hospodin vám onoho dne neodpoví.“
#8:19 Lid však odmítl Samuela uposlechnout. Prohlásili: „Nikoli. Ať je nad námi král!
#8:20 I my chceme být jako všechny ostatní národy. Náš král nás bude soudit a bude před námi vycházet a povede naše boje.“
#8:21 Samuel všechna slova lidu vyslechl a přednesl je Hospodinu.
#8:22 Hospodin Samuelovi pravil: „Uposlechni je a ustanov jim krále.“ Samuel pak izraelským mužům řekl: „Jděte každý do svého města.“ 
#9:1 Byl muž z Benjamínova pokolení jménem Kíš, syn Abíela, syna Seróra, syna Bekórata, syna Afíacha, syna jednoho Jemínce, statečný bohatýr.
#9:2 Ten měl syna jménem Saul, hezkého mladíka. Mezi Izraelci nebylo hezčího muže nad něj; od ramen vzhůru převyšoval všechen lid.
#9:3 Saulovu otci Kíšovi se ztratily oslice. Kíš proto řekl svému synu Saulovi: „Vezmi s sebou některého z mládenců a vydej se ty oslice hledat.“
#9:4 Prošel tedy Efrajimské pohoří, prošel zemi Šališu, ale nenašli nic. Prošel též zemi Šaalím, a nikde nic. Prošel i zemi Jemíní, a nenašli nic.
#9:5 Když přišli do země Súfu, řekl Saul svému mládenci, který byl s ním: „Pojď, vraťme se, aby můj otec místo o oslice neměl obavy o nás.“
#9:6 On mu řekl: „Hle, v tom městě je muž Boží. A ten muž je vážený. Všechno, co řekne, se skutečně stane. Pojďme tedy tam, snad nám oznámí, kterou cestou se máme dát.“
#9:7 Saul svému mládenci odvětil: „Tak pojďme! Ale co tomu muži přineseme? Vždyť chléb nám v brašně došel. Nemáme dárek, který bychom muži Božímu přinesli. Co máme u sebe?“
#9:8 Nato mládenec Saulovi odpověděl: „Mám u sebe čtvrt šekelu stříbra. Dám jej muži Božímu, a on nám oznámí cestu.“
#9:9 Když se dříve někdo v Izraeli šel dotazovat Boha, říkal: „Nuže, pojďme k vidoucímu.“ Dnešní prorok se dříve nazýval vidoucí.
#9:10 Saul svému mládenci řekl: „Dobře jsi promluvil. Nuže, pojďme.“ A šli do města, kde byl muž Boží.
#9:11 Když stoupali do svahu k městu, zastihli dívky, které šly čerpat vodu. Otázali se jich: „Je zde vidoucí?“
#9:12 „Ano, je,“ odpověděly, „máš ho před sebou. Pospěš si! Právě dnes přišel do města, protože lid má dnes na posvátném návrší obětní hod.
#9:13 Až vejdete do města, zastihnete ho, ještě než vystoupí na posvátné návrší, kde se bude jíst. Lid totiž nebude jíst, dokud on nepřijde. Teprve až on požehná obětnímu hodu, budou pozvaní jíst. Jděte tedy nahoru, dnes ho zastihnete.“
#9:14 Stoupali tedy k městu. Když přicházeli do středu města, zrovna proti nim vycházel Samuel, aby vystoupil na posvátné návrší.
#9:15 Den před Saulovým příchodem odhalil Hospodin Samuelovu uchu toto:
#9:16 „Zítra touto dobou pošlu k tobě muže z benjamínské země. Pomažeš ho za vévodu nad mým izraelským lidem; on můj lid vysvobodí z rukou Pelištejců. Pohlédl jsem na svůj lid, jehož úpěnlivě volání ke mně proniklo.“
#9:17 Když Samuel uviděl Saula, Hospodin ho upozornil: „To je ten muž, o němž jsem ti řekl, že bude spravovat můj lid.“
#9:18 Vtom Saul přistoupil uprostřed brány k Samuelovi a řekl: „Pověz mi prosím, kde je tu dům vidoucího.“
#9:19 Samuel Saulovi odpověděl: „Já jsem vidoucí. Vystup přede mnou na posvátné návrší, dnes budete jíst se mnou. Ráno tě propustím a odpovím ti na vše, co máš na srdci.
#9:20 Pokud jde o oslice, které se ti před třemi dny ztratily, neměj o ně starost, neboť se nalezly. Koho si však Izrael tolik žádá? Zdalipak ne tebe a celý dům tvého otce?“
#9:21 Saul odpověděl: „Což nejsem Benjamínec? Z těch nejmenších izraelských kmenů? A má čeleď je ze všech čeledí Benjamínova kmene nejnepatrnější! Jak mi můžeš říkat něco takového?“
#9:22 Ale Samuel vzal Saula i jeho mládence, uvedl je do hodovní síně a dal jim místo v čele pozvaných. Těch bylo na třicet mužů.
#9:23 Kuchařovi Samuel řekl: „Podej tu část, kterou jsem ti dal a o níž jsem ti řekl: Ulož ji u sebe.“
#9:24 Kuchař pozdvihl kýtu i to, co bylo na ní, a položil vše před Saula. Samuel řekl: „Hle, polož před sebe, co zůstalo, a jez. Bylo to uchováno pro tebe na tento slavnostní okamžik.“ A dodal: „Proto jsem svolal lid.“ Tak jedl onoho dne Saul se Samuelem.
#9:25 Potom sestoupili z posvátného návrší do města a Samuel mluvil se Saulem na střeše.
#9:26 Za časného jitra, když vycházela jitřenka, zavolal Samuel na Saula na střeše: „Vstaň, propustím tě.“ Saul tedy vstal a oba, on i Samuel, vyšli ven.
#9:27 Sestupovali k okraji města. Tu řekl Samuel Saulovi: „Řekni mládenci, ať jde před námi.“ A když je předešel: „A ty se na chvíli zastav, ohlásím ti Boží slovo.“ 
#10:1 Samuel vzal nádobku s olejem, vylil mu jej na hlavu, políbil ho a řekl: „Sám Hospodin tě pomazává za vévodu nad svým dědictvím.
#10:2 Až půjdeš dnes ode mne, zastihneš u Ráchelina hrobu na benjamínském území v Selsachu dva muže; oni ti řeknou: ‚Oslice, které jsi šel hledat, se našly. Tvůj otec již nedbá o to, co je s oslicemi, ale má obavy o vás. Říká: Co mám pro svého syna udělat?‘
#10:3 Odtud budeš postupovat dál, až přijdeš k božišti Táboru; tam tě zastihnou tři muži putující k Bohu do Bét-elu. Jeden ponese tři kůzlata, druhý ponese tři bochníky chleba a třetí ponese měch s vínem.
#10:4 Popřejí ti pokoj a dají ti dva chleby a ty je od nich přijmeš.
#10:5 Potom vstoupíš na Boží pahorek, na kterém jsou pelištejská výsostná znamení. Až tam vejdeš do města, narazíš na hlouček proroků sestupujících z posvátného návrší; před nimi harfa, buben, píšťala a citara, a oni budou v prorockém vytržení.
#10:6 Vtom se tě zmocní duch Hospodinův a upadneš do prorockého vytržení s nimi a změníš se v jiného muže.
#10:7 Až se u tebe tato znamení dostaví, učiň, co se tvé ruce naskytne, neboť Bůh bude s tebou.
#10:8 Potom sestoupíš přede mnou do Gilgálu a já sestoupím k tobě, abych obětoval zápalné oběti a připravil hody oběti pokojné. Sedm dní budeš čekat, než k tobě přijdu. Pak ti dám vědět, co máš udělat.“
#10:9 Sotva se Saul obrátil, aby od Samuela odešel, proměnil mu Bůh srdce v jiné. Onoho dne se dostavila všechna tato znamení.
#10:10 Když přišli na onen pahorek, přicházel mu vstříc hlouček proroků. Tu se ho zmocnil duch Boží a on upadl uprostřed nich do prorockého vytržení.
#10:11 Všichni z lidu, kdo ho znali z dřívějška, když viděli, že prorokuje spolu s proroky, říkali jeden druhému: „Co se to s Kíšovým synem děje? Což také Saul je mezi proroky?“
#10:12 A někdo odtamtud řekl: „Kdo je jejich otec?“ Proto se stalo příslovím: „Což také Saul je mezi proroky?“
#10:13 Když prorocké vytržení pominulo, vstoupil na posvátné návrší.
#10:14 I zeptal se Saulův strýc jeho a mládence: „Kde jste chodili?“ On odpověděl: „Hledali jsme oslice. Když jsme viděli, že nikde nejsou, šli jsme k Samuelovi.“
#10:15 Saulův strýc řekl: „Oznam mi prosím, co vám Samuel pověděl.“
#10:16 Saul svému strýci odpověděl: „Oznámil nám s jistotou, že se oslice našly.“ Ale co Samuel říkal ohledně království, mu neoznámil.
#10:17 Samuel svolal lid k Hospodinu do Mispy.
#10:18 Řekl Izraelcům: „Toto praví Hospodin, Bůh Izraele: Já jsem přivedl Izraele z Egypta; vysvobodil jsem vás z moci Egypta i z moci všech království, která vás utlačovala.
#10:19 Vy však dnes zavrhujete svého Boha, který vás zachraňoval ze všech vašich strastí a úzkostí. Říkáte mu: ‚Ustanov nad námi krále.‘ Tak se teď postavte před Hospodina podle svých kmenů a rodů.“
#10:20 Samuel dal předvést všechny izraelské kmeny. Losem byl označen kmen Benjamín.
#10:21 Dal tedy předvést kmen Benjamín po čeledích. Byla označena čeleď Matrí, v ní byl označen Saul, syn Kíšův. Hledali ho tedy, ale nebyl k nalezení.
#10:22 Tázali se Hospodina dále: „Přijde sem ještě ten muž?“ A Hospodin řekl: „Hle, skrývá se mezi zbrojí.“
#10:23 Běželi ho přivést. Když se postavil doprostřed lidu, převyšoval od ramen vzhůru všechen lid.
#10:24 Samuel řekl všemu lidu: „Hleďte, koho vyvolil Hospodin, není mu rovného ve všem lidu!“ A všechen lid spustil pokřik: „Ať žije král!“
#10:25 Samuel potom promluvil k lidu o právu královském, zapsal je do knihy a uložil před Hospodinem. Nato propustil všechen lid, každého do jeho domu.
#10:26 Také Saul šel domů do Gibeje; a s ním šla družina bojovníků, jejichž srdcí se dotkl Bůh.
#10:27 Ničemové však říkali: „Tenhleten že nás zachrání?“ A pohrdali jím, ani dar mu nepřinesli. On však jako by neslyšel. 
#11:1 Tu přitáhl Náchaš Amónský a utábořil se proti Jábeši v Gileádu. Všichni jábešští muži Náchašovi řekli: „Uzavři s námi smlouvu a budeme tvými otroky.“
#11:2 Náchaš Amónský jim odpověděl: „Uzavřu ji s vámi tím způsobem, že každému z vás vyloupnu pravé oko. Tak uvedu pohanu na celý Izrael.“
#11:3 Jábešští starší mu řekli: „Poskytni nám sedm dní. Rozešleme posly do celého izraelského území. Nebude-li nikoho, kdo by nás zachránil, vyjdeme k tobě.“
#11:4 I přišli poslové do Gibeje Saulovy a přednesli ta slova lidu. Všechen lid se dal do hlasitého pláče.
#11:5 Saul šel právě za dobytkem z pole. Tázal se: „Co je lidu, že pláče?“ A oni mu sdělili slova jábešských mužů.
#11:6 Jakmile Saul ta slova uslyšel, zmocnil se ho duch Boží. Vzplanul nesmírným hněvem.
#11:7 Vzal spřežení dobytka, rozsekal je na kusy a rozeslal po poslech do celého izraelského území se vzkazem: „Tohle se stane s dobytkem každému, kdo nevytáhne za Saulem a Samuelem.“ Na lid padl strach před Hospodinem, i vytáhli jako jeden muž.
#11:8 Saul jim dal nastoupit v Bezeku: Izraelců bylo tři sta tisíc a judských mužů třicet tisíc.
#11:9 Přišlým poslům řekli: „Vyřiďte mužům v Jábeši Gileádském toto: Zítra za slunečního žáru vám přijde vysvobození.“ I přišli poslové a oznámili to jábešským mužům a ti se zaradovali.
#11:10 Jábešští mužové vzkázali Náchašovi: „Zítra vyjdeme k vám. Pak s námi udělejte, co uznáte za dobré.“
#11:11 Nazítří Saul rozestavil lid do tří oddílů a za jitřní hlídky vtrhli doprostřed ležení a pobíjeli Amóna až do denního žáru; ti, kteří zůstali, se rozprchli, ani dva z nich nezůstali pospolu.
#11:12 Tu lid řekl Samuelovi: „Kdo říkal: ‚To Saul má nad námi kralovat?‘ Vydejte ty muže, ať je usmrtíme!“
#11:13 Ale Saul řekl: „Dnešního dne nebude nikdo usmrcen, neboť dnes Hospodin připravil Izraeli vysvobození.“
#11:14 A Samuel vybídl lid: „Nuže, pojďme do Gilgálu, obnovíme tam království.“
#11:15 Všechen lid šel tedy do Gilgálu a před Hospodinem tam v Gilgálu nastolili Saula za krále. Slavili tam před Hospodinem hody oběti pokojné a Saul se tam se všemi izraelskými muži převelice radoval. 
#12:1 Samuel pak řekl celému Izraeli: „Hle, uposlechl jsem vás ve všem, oč jste mě žádali, a ustanovil jsem nad vámi krále.
#12:2 Nyní tedy hle, svůj úřad koná před vámi král. Já jsem zestárl a zešedivěl, i moji synové jsou tu s vámi. Svůj úřad jsem před vámi konal od svého mládí až dodnes.
#12:3 Tu jsem. Vypovídejte proti mně před Hospodinem i před jeho pomazaným, zdali jsem vzal někomu býka či osla, zdali jsem někoho vydíral a někomu křivdil, zdali jsem od někoho vzal úplatek a nad něčím přivřel oči, a já vám to nahradím.“
#12:4 Odvětili: „Nevydíral jsi nás a nekřivdil jsi nám a nic jsi od nikoho nevzal.“
#12:5 Řekl jim: „Hospodin je svědkem proti vám, i jeho pomazaný je dnešního dne svědkem, že jste na mně nic neshledali.“ Lid odvětil: „Je svědkem.“
#12:6 Samuel pak řekl lidu: „Ano, Hospodin, který učinil Mojžíše a Árona a který přivedl vaše otce z egyptské země.
#12:7 Nyní předstupte! Povedu s vámi před Hospodinem soud o všech Hospodinových spravedlivých činech, které konal pro vás a pro vaše otce.
#12:8 Když Jákob přišel do Egypta a vaši otcové úpěli k Hospodinu, Hospodin poslal Mojžíše a Árona a ti vyvedli vaše otce z Egypta a usadili je na tomto místě.
#12:9 Oni však na Hospodina, svého Boha, zapomněli. Proto je vydal napospas Síserovi, veliteli chasórského vojska, i Pelištejcům a moábskému králi; ti bojovali proti nim.
#12:10 Tu úpěli k Hospodinu a říkali: ‚Zhřešili jsme, neboť jsme opustili Hospodina a sloužili jsme baalům a aštartám, ale nyní nás vysvoboď z rukou našich nepřátel a budeme sloužit tobě.‘
#12:11 Hospodin tedy poslal Jerubaala, Bedána, Jiftácha a Samuela a vysvobodil vás z rukou vašich okolních nepřátel, takže jste sídlili v bezpečí.
#12:12 Když jste viděli, že na vás táhne Náchaš, král Amónovců, řekli jste mi: ‚Nikoli, ať nad námi kraluje král‘, ačkoli vaším králem je Hospodin, váš Bůh.
#12:13 Nuže tady je král, kterého jste si zvolili a vyžádali. Ano, Hospodin vám dal krále.
#12:14 Jestliže se budete Hospodina bát, jemu sloužit a poslouchat ho, nebudete-li se vzpírat Hospodinovým rozkazům, obstojíte vy i král, který nad vámi kraluje; půjdete-li ovšem za Hospodinem, svým Bohem.
#12:15 Jestliže však nebudete Hospodina poslouchat a budete se Hospodinovým rozkazům vzpírat, dolehne Hospodinova ruka na vás i na vaše otce.
#12:16 Nyní předstupte! Hleďte na tuto velikou věc, kterou Hospodin učiní před vašima očima.
#12:17 Není právě pšeničná žatva? Budu volat k Hospodinu a on způsobí hromobití a déšť. Tak očividně poznáte, jak velmi zlé je v Hospodinových očích to, čeho jste se dopustili tím, že jste si vyžádali krále.“
#12:18 A tak Samuel volal k Hospodinu a Hospodin onoho dne způsobil hromobití a déšť. A všechen lid se velice bál Hospodina i Samuela.
#12:19 I řekl všechen lid Samuelovi: „Modli se k Hospodinu, svému Bohu, za své služebníky, abychom nezemřeli za to, že jsme ke všem svým hříchům přidali ještě to zlé, že jsme si vyžádali krále.“
#12:20 Nato řekl Samuel lidu: „Nebojte se. Ano, dopustili jste se všeho toho zla, avšak neodchylujte se od Hospodina a služte Hospodinu celým srdcem.
#12:21 Neuchylujte se k nicotám, které neprospějí a nevysvobodí; vždyť jsou jen nicotou.
#12:22 Hospodin pro své veliké jméno svůj lid nezavrhne, vždyť z vás hodlá učinit svůj lid.
#12:23 Jsem dalek toho, abych proti Hospodinu hřešil a přestal se za vás modlit. I nadále vás budu vyučovat dobré a přímé cestě.
#12:24 Jen se bojte Hospodina a věrně mu služte celým srdcem. Hleďte, co velikého s vámi učinil.
#12:25 Jestliže však budete kupit zlo na zlo, budete smeteni jak vy, tak váš král.“ 
#13:1 Uplynul rok Saulova kralování. Když kraloval nad Izraelem druhý rok,
#13:2 vybral si z Izraele tři tisíce mužů; dva tisíce jich bylo se Saulem v Mikmásu v Bételském pohoří, tisíc jich bylo s Jónatanem v Gibeji Benjamínově; zbytek lidu rozpustil, každého k jeho stanům.
#13:3 Jónatan porazil výsostné znamení pelištejské v Gebě. Pelištejci o tom uslyšeli, proto dal Saul troubit na polnici po celé zemi. Řekl: „Ať to Hebrejové slyší!“
#13:4 Tak uslyšel celý Izrael: „Saul porazil výsostné znamení pelištejské.“ Tím vzbudil Izrael u Pelištejců nelibost. Proto byl svolán lid k Saulovi do Gilgálu.
#13:5 Pelištejci se shromáždili, aby bojovali s Izraelem. Měli třicet tisíc vozů, šest tisíc jezdců a lidu takové množství jako písku na mořském břehu. Přitáhli a utábořili se v Mikmásu na východ od Bét-ávenu.
#13:6 Izraelští muži viděli, že se dostali do tísně a že lid je sklíčen. Lid se ukrýval v jeskyních, rozsedlinách, skalních stržích, slujích a jamách.
#13:7 Hebrejové přešli Jordán do země gádské a gileádské, zatímco Saul byl ještě v Gilgálu; všechen lid, který šel za ním, byl zděšen.
#13:8 Čekal sedm dní do chvíle, kterou určil Samuel, ale Samuel do Gilgálu nepřicházel a lid se od Saula rozprchával.
#13:9 Saul tedy řekl: „Přineste ke mně oběť zápalnou a pokojnou.“ A obětoval zápalnou oběť.
#13:10 Sotva skončil obětování zápalné oběti, přišel Samuel. Saul mu vyšel vstříc a pozdravil ho.
#13:11 Samuel se ho otázal: „Cos to udělal?“ Saul odvětil: „Když jsem viděl, že se lid ode mne rozprchává, neboť tys k určenému dni nepřicházel, a že se Pelištejci shromáždili v Mikmásu,
#13:12 řekl jsem si: Pelištejci teď sejdou proti mně do Gilgálu a já jsem si nenaklonil Hospodina. A tak jsem se opovážil zápalnou oběť obětovat sám.“
#13:13 Samuel nato Saulovi řekl: „Počínal sis jako pomatenec. Nedbals příkazu, který ti dal Hospodin, tvůj Bůh. Tak by byl Hospodin upevnil tvé království nad Izraelem navěky.
#13:14 Teď však tvé království neobstojí. Hospodin si vyhledal muže podle svého srdce. Jemu Hospodin přikáže, aby byl vévodou nad jeho lidem, protože ty jsi nedbal toho, co ti Hospodin přikázal.“
#13:15 A Samuel hned vystoupil z Gilgálu do Gibeje Benjamínovy. Saul dal nastoupit lidu, který při něm setrval; bylo jich kolem šesti set mužů.
#13:16 Saul, jeho syn Jónatan i lid, který s nimi setrval, usadili se v Gibeji Benjamínově, kdežto Pelištejci tábořili v Mikmásu.
#13:17 Z pelištejského ležení vyšly tři oddíly záškodníků; jeden oddíl se ubíral směrem k Ofře, k zemi Šúalu,
#13:18 druhý oddíl se ubíral směrem k Bét-chorónu a třetí oddíl se ubíral směrem k území vybíhajícímu nad Seboímským údolím k poušti.
#13:19 V celé izraelské zemi nebylo možno najít kováře. Pelištejci totiž říkali: „Jen ať Hebrejové nevyrábějí meče ani kopí.“
#13:20 Když si kdokoli z Izraele chtěl dát naostřit rádlo, motyku, širočinu či radlici, musel sestoupit k Pelištejcům.
#13:21 Naostření rádla, motyky, trojzubce a širočiny a zakalení ostnu stálo jeden pím.
#13:22 V čas boje tomu bylo tak, že lid, který byl se Saulem a Jónatanem, neměl po ruce meč ani kopí; měl je pouze Saul a jeho syn Jónatan.
#13:23 Postavení Pelištejců se posunulo k průsmyku u Mikmásu. 
#14:1 Jednoho dne vyzval Jónatan, syn Saulův, mládence, svého zbrojnoše: „Pojď, pronikneme k postavení Pelištejců tamhle naproti.“ Svému otci to však neoznámil.
#14:2 Saul se usadil na úpatí pahorku pod granátovníkem v Migrónu. Lidu s ním bylo na šest set mužů.
#14:3 Achijáš, syn Achítúba, bratra Íkábóda, syna Pinchasa, syna Élího, Hospodinova kněze v Šílu, nosil efód. Lid ovšem nevěděl, že Jónatan odešel.
#14:4 V průsmyku, kterým se Jónatan snažil proniknout nad postavení Pelištejců, byl skalní útes z jedné strany i skalní útes z druhé strany; jeden se jmenoval Bóses a druhý Sené.
#14:5 Jeden útes byl jako pilíř na severu naproti Mikmásu, druhý na jihu naproti Gebě.
#14:6 Jónatan tedy vyzval mládence, svého zbrojnoše: „Pojď, pronikneme k postavení těch neobřezanců. Snad pro nás Hospodin něco udělá. Vždyť Hospodinu nemůže nic zabránit, aby zachránil ať skrze mnoho nebo skrze málo.“
#14:7 Zbrojnoš mu odpověděl: „Udělej všechno, co máš na mysli. Jen vpřed! Já budu s tebou podle tvé vůle.“
#14:8 Jónatan rozhodl: „Pronikneme k těm mužům a znenadání se před nimi objevíme.
#14:9 Jestliže nás vyzvou: ‚Nehýbejte se, dokud k vám nedorazíme‘, zůstaneme na místě a nepůjdeme k nim nahoru.
#14:10 Jestliže však řeknou: ‚Pojďte k nám nahoru‘, půjdeme vzhůru, neboť Hospodin nám je vydal do rukou. To bude pro nás znamením.“
#14:11 Když se oba objevili před postavením Pelištejců, Pelištejci řekli: „Hle, Hebrejové vylézají z děr, kam se ukryli.“
#14:12 Muži z toho postavení vyzvali Jónatana a jeho zbrojnoše: „Pojďte k nám nahoru, něco vám povíme!“ Tu řekl Jónatan svému zbrojnoši: „Vzhůru za mnou! Hospodin je vydal Izraeli do rukou!“
#14:13 Jónatan lezl po čtyřech nahoru a za ním jeho zbrojnoš. I padali před Jónatanem a zbrojnoš je za ním pobíjel.
#14:14 To byl první úder, kdy Jónatan se svým zbrojnošem pobili kolem dvaceti mužů na polovičním honu pole.
#14:15 V nepřátelském táboře, na poli i ve všem lidu, nastalo zděšení; také muži z onoho postavení i záškodníci byli zděšeni a země se chvěla. To zděšení způsobil Bůh.
#14:16 I Saulova hlídka v Gibeji Benjamínově viděla, jak hlučící dav zmateně pobíhá sem a tam.
#14:17 Saul rozkázal lidu, který byl s ním: „Dejte nastoupit! Zjistěte, kdo od nás odešel.“ Nastoupili a zjistilo se, že tu není Jónatan a jeho zbrojnoš.
#14:18 Saul poručil Achijášovi: „Přines Boží schránu!“ Boží schrána byla totiž toho času mezi Izraelci.
#14:19 Zatímco Saul ještě mluvil s knězem, hluk v táboře Pelištejců se dále vzmáhal. Proto Saul poručil knězi: „Stáhni ruku zpět!“
#14:20 Saul a všechen lid, který byl s ním, se svolali pokřikem a přišli až k té bitvě. A hle, tam řádil meč jednoho proti druhému, nesmírná hrůza.
#14:21 Také Hebrejové z okolí, kteří předtím byli s Pelištejci a s nimi přitáhli do tábora, přidali se k Izraeli, který byl se Saulem a Jónatanem.
#14:22 Rovněž všichni izraelští muži, kteří se skrývali v pohoří Efrajimském, jakmile uslyšeli, že se Pelištejci dali na útěk, pustili se v té bitvě za nimi.
#14:23 Tak Hospodin zachránil v onen den Izraele. Bitva se přehnala přes Bét-áven.
#14:24 Když se izraelští muži onoho dne zase sešli, Saul zaklínal lid slovy: „Buď proklet muž, který by do večera, dokud se nepomstím svým nepřátelům, pojedl nějaký pokrm.“ Proto nikdo z lidu žádný pokrm neokusil.
#14:25 Celá země chodívala hledat plástve; med býval na povrchu pole.
#14:26 Když lid přišel na plástve, přetékaly medem. Nikdo však nepozdvihl ruku k ústům. Lid se bál té přísahy.
#14:27 Ale Jónatan neslyšel, že jeho otec zavázal lid přísahou. Napřáhl hůl, kterou měl v ruce, ponořil její konec do plástve medu a rukou si ji podal k ústům. Hned se mu rozjasnily oči.
#14:28 Nějaký muž z lidu se ozval a řekl: „Tvůj otec zavázal lid těžkou přísahou: Buď proklet muž, který by dnes pojedl nějaký pokrm. Proto je lid zemdlen.“
#14:29 Jónatan odvětil: „Můj otec přivede do zkázy celou zemi. Jen pohleďte, jak se mi rozjasnily oči, sotva jsem ochutnal trochu toho medu.
#14:30 Což teprve, kdyby se dnes mohl lid pořádně najíst z kořisti svých nepřátel, kterou nalezl! Proto teď není možný další úder proti Pelištejcům.“
#14:31 Onoho dne pobíjeli Pelištejce od Mikmásu až k Ajalónu a lid byl velmi zemdlen.
#14:32 Lid se tedy vrhl na kořist, brali brav i skot i býčky a poráželi je na zemi. Lid jedl maso s krví.
#14:33 Saulovi oznámili: „Hle, lid hřeší proti Hospodinu, když jí s krví.“ On vzkřikl: „Zachovali jste se věrolomně.“ Okamžitě ke mně přivalte veliký kámen.“
#14:34 Dále Saul rozkázal: „Rozejděte se mezi lid s výzvou, ať každý ke mně přivede svého býka nebo svou ovci. Tady je porazíte a sníte, abyste nehřešili proti Hospodinu tím, že byste jedli maso s krví.“ Všichni z lidu, každý přivedl té noci svého býka a poráželi je tam.
#14:35 A Saul vybudoval Hospodinu oltář. Byl to první oltář, který Hospodinu vybudoval.
#14:36 Saul rozhodl: „V noci vyrazíme za Pelištejci. Až do jitřního úsvitu budeme moci mezi nimi loupit. Ani jednoho z nich neušetříme.“ Odvětili: „Udělej vše, co pokládáš za správné.“ Ale kněz podotkl: „Přistupme sem k Bohu.“
#14:37 Saul se doptával Boha: „Mám vyrazit dolů za Pelištejci? Vydáš je Izraeli do rukou?“ Ale on mu onoho dne neodpověděl.
#14:38 Proto Saul poručil: „Přistupte sem, všichni vůdcové lidu! Vyzvěďte a zjistěte, jaký hřích byl dnes spáchán.
#14:39 Jakože je živ Hospodin, zachránce Izraele, i kdyby šlo o mého syna Jónatana, musí zemřít!“ Ale ze všeho lidu mu nikdo neodpověděl.
#14:40 Poručil tedy celému Izraeli: „Vy budete na jedné straně a já se svým synem Jónatanem na straně druhé.“ Lid Saulovi odvětil: „Udělej, co pokládáš za správné.“
#14:41 Saul řekl Hospodinu: „Bože Izraele, ukaž, kdo není bezúhonný.“ Losem byl postižen Jónatan a Saul, kdežto lid z toho vyvázl.
#14:42 Saul znovu poručil: „Vrhněte los mezi mnou a mým synem Jónatanem.“ Tu byl označen Jónatan.
#14:43 Saul vyzval Jónatana: „Přiznej se mi, co jsi spáchal!“ Jónatan mu to pověděl a doznal: „Koncem hole, kterou jsem měl v ruce, jsem ochutnal trochu medu. Jsem připraven zemřít.“
#14:44 Saul řekl: „Ať se mnou Bůh udělá, co chce! Jónatane, ty musíš zemřít!“
#14:45 Avšak lid se proti Saulovi ozval: „Jónatan že by měl zemřít? Vždyť připravil Izraeli takové veliké vysvobození! Buď toho dalek! Jakože je živ Hospodin, ani vlásek z hlavy mu nesmí spadnout na zem. S pomocí Boží jednal dnešního dne.“ tak lid Jónatana vykoupil, že nezemřel.
#14:46 Saul odtáhl od Pelištejců a Pelištejci odešli do svých domovů.
#14:47 Když se Saul chopil kralování nad Izraelem, bojoval proti všem svým okolním nepřátelům, proti Moábovi, Amónovcům a Edómu, proti králům Sóby a proti Pelištejcům. Všude, kam se obrátil, dopouštěl se svévolností.
#14:48 Statečnost však prokázal, když pobil Amáleka a vysvobodil Izraele z rukou těch, kteří ho plenili.
#14:49 Saulovi synové byli Jónatan, Jišví a Malkíšúa. A jména jeho dvou dcer: prvorozená se jmenovala Mérab, mladší pak Míkal.
#14:50 Saulova žena se jmenovala Achínoam, dcera Achímaasova. Jméno velitele jeho vojska bylo Abínér, syn Néra, strýce Saulova.
#14:51 Kíš, Saulův otec, a Nér, otec Abnérův, byli syny Abíelovými.
#14:52 Po všechny dny Saulovy se vedl tuhý boj proti Pelištejcům. Když Saul spatřil nějakého bohatýra a statečného muže, přibral ho k sobě. 
#15:1 Samuel řekl Saulovi: „Hospodin mě poslal, abych tě pomazal za krále nad jeho lidem, nad Izraelem. Nyní tedy poslechni Hospodinova slova.
#15:2 Toto praví Hospodin zástupů: Mám v patrnosti, co učinil Amálek Izraeli; položil se mu do cesty, když vystupoval z Egypta.
#15:3 Nyní jdi a pobij Amáleka; jako klaté zničíte vše, co mu patří. Nebudeš ho šetřit, ale usmrtíš muže i ženu, pachole i kojence, býka i ovci, velblouda i osla!“
#15:4 Saul svolal lid a dal mu nastoupit v Telaímu; bylo to dvě stě tisíc pěších a judských mužů deset tisíc.
#15:5 Pak Saul přitáhl až k Amálekovu městu a v úvalu umístil zálohu.
#15:6 Kénijcům Saul vzkázal: „Odejděte od Amáleka a sestupte dolů, abych vás nesmetl spolu s ním. Vy jste přece prokázali milosrdenství všem Izraelcům, když vystupovali z Egypta.“ Kénijci tedy od Amáleka odešli.
#15:7 A Saul Amáleka pobíjel od Chavíly směrem na Šúr, který je naproti Egyptu.
#15:8 Amáleckého krále Agaga chytil živého, ale ostřím meče vyhubil všechen lid jako klatý.
#15:9 Saul a jeho lid ušetřil Agaga a nejlepší kusy z bravu a skotu, z druhého vrhu, a ze skopců a vůbec ze všeho pěkného, a nechtěli je vyhubit jako klaté. Jako klaté zničili jen to, co bylo podřadné a bezcenné.
#15:10 I stalo se slovo Hospodinovo k Samuelovi:
#15:11 „Lituji, že jsem ustanovil Saula králem, neboť se ode mne odvrátil a mými slovy se neřídí.“ Tu vzplanul Samuel hněvem a po celou noc úpěl k Hospodinu.
#15:12 Za časného jitra šel Samuel vstříc Saulovi. Tu oznámili Samuelovi: „Saul přitáhl na Karmel a postavil si tam znamení své moci. Pak se obrátil, táhl dál a sestoupil do Gilgálu.“
#15:13 Když Samuel přišel k Saulovi, ten ho pozdravil: „Buď požehnán od Hospodina. Řídil jsem se Hospodinovým slovem.“
#15:14 Samuel se však tázal: „A co to bečení ovcí, jež doléhá k mým uším, a to bučení dobytka, které slyším?“
#15:15 Saul vysvětloval: „Přivedli je od Amáleka. Lid ušetřil nejlepší kusy z bravu a skotu, aby je obětoval Hospodinu, tvému Bohu. Ale zbytek jsme vyhubili jako klatý.“
#15:16 Samuel Saula přerušil: „Přestaň! Musím ti oznámit, co ke mně této noci Hospodin promluvil.“ Řekl mu: „Mluv.“
#15:17 Samuel pravil: „Cožpak nejsi vůdcem izraelských kmenů, ačkoli sis dříve připadal nepatrný? Hospodin tě pomazal za krále nad Izraelem.
#15:18 Hospodin tě také poslal na cestu s rozkazem: Jdi a jako klaté vyhubíš ty amálecké hříšníky. Budeš proti nim bojovat, dokud je úplně nezničíš.
#15:19 Proč jsi Hospodina neuposlechl, ale vrhl se na kořist a dopustil se toho, co je v Hospodinových očích zlé?“
#15:20 Saul vysvětloval Samuelovi: „Vždyť jsem Hospodina uposlechl. Šel jsem cestou, na kterou mě Hospodin poslal. Přivedl jsem amáleckého krále Agaga a Amáleka jsem zničil jako klatého.
#15:21 Ale lid si vzal z kořisti, z bravu a skotu, to nejlepší z klatého, aby měl co obětovat v Gilgálu Hospodinu, tvému Bohu.“
#15:22 Tu řekl Samuel: „Líbí se Hospodinu zápalné oběti a obětní hody víc než poslouchat Hospodina? Hle, poslouchat je lepší než obětní hod, pozorně rozvažovat je víc než tuk beranů.
#15:23 Vzdor je jako hříšné věštění a svéhlavost jako kouzla a ctění domácích bůžků. Protože jsi zavrhl Hospodinovo slovo, i on zavrhl tebe jako krále.“
#15:24 Saul doznal Samuelovi: „Zhřešil jsem, neboť jsem přestoupil Hospodinův rozkaz i tvá slova. Bál jsem se lidu, proto jsem je uposlechl.
#15:25 Nyní však sejmi ze mne prosím můj hřích a vrať se se mnou; chci se poklonit Hospodinu.“
#15:26 Samuel však Saula odmítl: „Nevrátím se s tebou. Zavrhl jsi Hospodinovo slovo, proto Hospodin zavrhl tebe, abys nebyl králem nad Izraelem.“
#15:27 Když se Samuel obrátil k odchodu, uchopil Saul cíp jeho pláště a ten se odtrhl.
#15:28 Samuel mu řekl: „Dnes od tebe Hospodin odtrhl izraelské království a dal je tvému bližnímu, lepšímu, než jsi ty.
#15:29 Věčný Bůh Izraele neklame ani nebude litovat; není to přece člověk, aby litoval.“
#15:30 Saul se dožadoval: „Zhřešil jsem. Nyní mi však prosím prokaž poctu před staršími mého lidu i před Izraelem a vrať se se mnou. Chci se poklonit Hospodinu, tvému Bohu.“
#15:31 Samuel se tedy vrátil a šel za Saulem. A Saul se poklonil Hospodinu.
#15:32 Pak Samuel poručil: „Přiveďte ke mně amáleckého krále Agaga!“ Agag šel k němu sebevědomě, neboť si řekl: „Zajisté je odvrácena hořkost smrti.“
#15:33 Samuel řekl: „Jako tvůj meč zbavoval ženy dětí, tak ať je tvá matka bezdětná nad jiné ženy.“ A Samuel rozsekal Agaga na kusy před Hospodinem v Gilgálu.
#15:34 Potom Samuel odešel do Rámy, zatímco Saul vystoupil ke svému domu, do Gibeje Saulovy.
#15:35 A Samuel už nikdy až do dne své smrti Saula nespatřil, avšak pro Saula truchlil. Hospodin litoval, že Saula ustanovil králem nad Izraelem. 
#16:1 Hospodin řekl Samuelovi: „Jak dlouho ještě budeš nad Saulem truchlit? Já jsem ho zavrhl, aby nad Izraelem nekraloval. Naplň svůj roh olejem a jdi, posílám tě k Jišajovi Betlémskému. Vyhlédl jsem si krále mezi jeho syny.“
#16:2 Samuel se zdráhal: „Jak mohu jít? Saul o tom uslyší a zabije mě.“ Hospodin řekl: „Vezmeš s sebou jalovici a řekneš: ‚Přicházím obětovat Hospodinu.‘
#16:3 Na obětní hod pozveš Jišaje a já ti dám vědět, co máš dělat. Pomažeš mi toho, o němž ti povím.“
#16:4 Samuel vykonal, co řekl Hospodin. Když přišel do Betléma, vyděšení starší města mu spěchali vstříc s otázkou: „Přinášíš pokoj?“
#16:5 Odvětil: „Pokoj. Přišel jsem obětovat Hospodinu. Posvěťte se a půjdete se mnou k obětnímu hodu.“ Pak posvětil Jišaje a jeho syny a pozval je k obětnímu hodu.
#16:6 Když se dostavili a on spatřil Elíaba, řekl si: „Jistě tu stojí před Hospodinem jeho pomazaný.“
#16:7 Hospodin však Samuelovi řekl: „Nehleď na jeho vzhled ani na jeho vysokou postavu, neboť já jsem ho zamítl. Nejde o to, nač se dívá člověk. Člověk se dívá na to, co má před očima, Hospodin však hledí na srdce.“
#16:8 Jišaj pak zavolal Abínádaba a předvedl ho před Samuela. On však řekl: „Hospodin nevyvolil ani toho.“
#16:9 Jišaj tedy předvedl Šamu. Řekl: „Hospodin nevyvolil ani toho.“
#16:10 Tak předvedl Jišaj před Samuela svých sedm synů, ale Samuel řekl: „Žádného z nich Hospodin nevyvolil.“
#16:11 A Samuel se Jišaje otázal: „To jsou všichni mládenci?“ Jišaj odvětil: „Ještě zbývá nejmladší, ten však pase stádo.“ Samuel Jišajovi poručil: „Pošli pro něj! Nebudeme stolovat, dokud sem nepřijde.“
#16:12 Poslal tedy pro něj a dal ho přivést. Byl ryšavý, s krásnýma očima a pěkného vzhledu. Tu řekl Hospodin: „Nuže, pomaž ho! To je on.“
#16:13 Samuel tedy vzal roh s olejem a pomazal ho uprostřed jeho bratrů. A duch Hospodinův se Davida zmocňoval od onoho dne i nadále. Samuel pak hned nato odešel do Rámy.
#16:14 Duch Hospodinův odstoupil od Saula a přepadal ho zlý duch od Hospodina.
#16:15 Proto Saulovi jeho služebníci navrhli: „Hle, přepadá tě zlý duch od Boha.
#16:16 Ať náš pán poručí svým služebníkům, kteří jsou před ním, aby vyhledali někoho, kdo umí hrát na citaru. Bude na ni hrát, kdykoli na tebe dolehne zlý duch od Boha, a bude ti dobře.“
#16:17 Saul tedy poručil svým služebníkům: „Vyhlédněte mi někoho, kdo dobře hraje, a přiveďte ho ke mně!“
#16:18 Jeden z družiny nato řekl: „Viděl jsem syna Jišaje Betlémského, ten umí hrát. Přitom je to statečný bohatýr, bojovník i muž hbitý v řeči a pohledný, a je s ním Hospodin.“
#16:19 Saul poslal posly k Jišajovi a poručil: „Pošli ke mně svého syna Davida, který je u stáda!“
#16:20 Jišaj vzal osla, chléb a měch s vínem i jednoho kozlíka a poslal to Saulovi po svém synu Davidovi.
#16:21 David přišel k Saulovi, stával před ním a on si ho velice zamiloval; stal se jeho zbrojnošem.
#16:22 Jišajovi poslal Saul vzkaz: „Ať David zůstane u mne, neboť získal mou přízeň.“
#16:23 Kdykoli na Saula doléhal duch od Boha, bral David citaru a hrál na ni. Saulovi to přinášelo úlevu a bylo mu dobře, zlý duch od něho odstupoval. 
#17:1 Pelištejci shromáždili své šiky k bitvě. Shromáždili se do Sóka, jež patří Judovi, a utábořili se mezi Sókem a Azekou v Efes-damímu.
#17:2 Také Saul a izraelští muži se shromáždili, utábořili se v dolině Posvátného stromu a seřadili se k bitvě proti Pelištejcům.
#17:3 Na hoře z jedné strany stáli Pelištejci, na hoře z druhé strany stál Izrael a mezi nimi bylo údolí.
#17:4 I vycházíval z pelištejských šiků soubojový zápasník jménem Goliáš z Gatu, vysoký šest loket a jednu píď.
#17:5 Na hlavě měl bronzovou přilbu a byl oděn do šupinatého pancíře; váha pancíře byla pět tisíc šekelů bronzu.
#17:6 Na nohou měl bronzové holenice a na ramenou bronzový oštěp.
#17:7 Násada jeho kopí byla jako tkalcovské vratidlo a hrot jeho kopí vážil šest set šekelů železa. Před ním chodíval štítonoš.
#17:8 Goliáš stával a volal na izraelské řady. Říkal jim: „Proč vycházíte a řadíte se k bitvě? Což nejsem já Pelištejec a vy služebníci Saulovi? Vyberte si někoho, ať ke mně sestoupí.
#17:9 Když mě v boji přemůže a zabije mě, budeme vašimi otroky. Avšak jestliže já přemohu jeho a zabiji ho, budete vy našimi otroky a budete nám sloužit.“
#17:10 A Pelištejec dodával: „Dneska jsem potupil izraelské řady. Vydejte mi někoho a budeme spolu bojovat.“
#17:11 Kdykoli Saul a celý Izrael slyšeli tato Pelištejcova slova, děsili se a velice se báli.
#17:12 David byl synem Efratejce, toho z Betléma Judova, jenž se jmenoval Jišaj a měl osm synů. Za dnů Saulových byl ten muž příliš starý, aby mohl jít mezi vojáky,
#17:13 ale šli tři nejstarší Jišajovi synové; odešli se Saulem do války. Jména jeho tří synů, kteří odešli do války, jsou: prvorozený Elíab, druhý po něm Abínádab a třetí Šama.
#17:14 David, ten byl nejmladší. Se Saulem odešli tři nejstarší.
#17:15 David odešel od Saula a vrátil se, aby pásl stádo svého otce v Betlémě.
#17:16 A Pelištejec se předváděl za časného jitra i navečer a stavěl se na odiv po čtyřicet dní.
#17:17 Jišaj vybídl svého syna Davida: „Vezmi pro své bratry éfu praženého zrní a těchto deset chlebů a běž do tábora za svými bratry.
#17:18 Těchto deset homolek sýra doneseš veliteli nad tisíci. Navštívíš své bratry s přáním pokoje a převezmeš od nich vzkazy.
#17:19 Bojují se Saulem a všemi izraelskými muži proti Pelištejcům v dolině Posvátného stromu.“
#17:20 Za časného jitra přenechal David stádo hlídači, naložil si všechno a šel, jak mu Jišaj přikázal. Když přicházel k ležení, vojsko vycházelo, řadilo se a vydávalo válečný pokřik.
#17:21 Izrael i Pelištejci se seřadili, řada proti řadě.
#17:22 David složil své zásoby u strážného nad zásobami a běžel k bojové řadě. Přišel a popřál svým bratrům pokoj.
#17:23 Ještě s nimi mluvil, když tu z řad Pelištejců vystoupil soubojový zápasník jménem Goliáš, Pelištejec z Gatu, a mluvil táž slova jako dříve. David je slyšel.
#17:24 Všichni izraelští muži, sotvaže toho muže uviděli, dali se před ním na útěk a velmi se báli.
#17:25 A nějaký Izraelec řekl: „Viděli jste toho muže, co vystoupil? Vystupuje, aby tupil Izraele. Avšak toho, kdo ho zabije, zahrne král velikým bohatstvím, dá mu svou dceru a jeho rod učiní v Izraeli svobodným“.
#17:26 David se tázal mužů, kteří u něho stáli: „Cože dostane muž, který zabije tohoto Pelištejce a sejme z Izraele potupu? Vždyť kdo je ten neobřezaný Pelištejec, že tupí řady živého Boha?“
#17:27 Lid mu odpověděl stejně: „To a to dostane muž, který ho zabije.“
#17:28 Jeho nejstarší bratr Elíab však slyšel, jak mluvil s muži. Elíab vzplanul proti Davidovi hněvem a okřikl ho: „Proč jsi sem přišel? Komu jsi nechal ten houfeček ovcí na stepi? Však znám tvou drzost i tvé zlé srdce! Přišel jsi sem dolů, aby ses mohl dívat na bitvu.“
#17:29 David se otázal: „Copak jsem udělal? Nejde o královo slovo?“
#17:30 Obrátil se od něho na jiného a tázal se stejně. A lid mu odpověděl stejně jako poprvé.
#17:31 Slova, která David promluvil, se roznesla, hlásili je Saulovi a on ho přijal.
#17:32 David Saulovi řekl: „Člověk nesmí klesat na mysli. Tvůj služebník půjde s tím Pelištejcem bojovat.“
#17:33 Saul Davidovi odvětil: „Nemůžeš jít proti tomu Pelištejci a bojovat s ním. Jsi přece mladíček, kdežto on je bojovník od mládí.“
#17:34 David řekl Saulovi: „Tvůj služebník byl pastýřem ovcí svého otce. Když přišel lev anebo medvěd, aby odnesl ze stáda ovci,
#17:35 hnal jsem se za ním a bil jsem ho a vyrval mu ji z tlamy. Když se proti mně postavil, chytil jsem ho za dolní čelist a bil jsem ho, až jsem ho usmrtil.
#17:36 Tvůj služebník ubil jak lva, tak medvěda. A tomu neobřezanému Pelištejci se povede jako jednomu z nich, protože potupil řady živého Boha.“
#17:37 A David dodal: „Hospodin, který mě vytrhl ze spárů lva a medvěda, ten mě vytrhne i ze spárů tohoto Pelištejce.“ Saul tedy Davidovi řekl: „Jdi; Hospodin buď s tebou!“
#17:38 Poté Saul oblékl Davida do svého odění, na hlavu mu dal bronzovou přilbu a oblékl ho do pancíře.
#17:39 Na jeho odění si David připásal jeho meč a pokusil se chodit, ale nebyl na to zvyklý. David tedy Saulovi řekl: „Nemohu v tom chodit, nejsem zvyklý.“ A svlékl to ze sebe.
#17:40 Vzal si do ruky svou hůl, z potoka vybral pět oblázků, vložil je do své pastýřské torby, do brašny, a s prakem v ruce postupoval proti Pelištejci.
#17:41 Pelištejec se k Davidovi pomalu přibližoval a před ním jeho štítonoš.
#17:42 Pelištejec se podíval, spatřil Davida a pohrdl jím, protože to byl mladíček, ryšavý, krásného vzhledu.
#17:43 Pelištejec na Davida pokřikoval: „Copak jsem pes, že na mě jdeš s holí?“ A Pelištejec zlořečil Davidovi skrze své bohy.
#17:44 Pokřikoval na Davida: „Pojď ke mně, ať vydám tvé tělo nebeskému ptactvu a polnímu zvířectvu.“
#17:45 Ale David Pelištejci odpověděl: „Ty jdeš proti mně s mečem, kopím a oštěpem, já však jdu proti tobě ve jménu Hospodina zástupů, Boha izraelských řad, kterého jsi potupil.
#17:46 Ještě dnes mi tě Hospodin vydá do rukou. Zabiji tě a srazím ti hlavu. Ještě dnes vydám mrtvoly z pelištejského tábora nebeskému ptactvu a zemské zvěři. Celý svět pozná, že při Izraeli stojí Bůh.
#17:47 A celé toto shromáždění pozná, že Hospodin nezachraňuje mečem a kopím. Vždyť boj je Hospodinův. On vás vydá do našich rukou.“
#17:48 Když Pelištejec vykročil a přibližoval se k Davidovi, David rychle vyběhl z řady proti Pelištejci.
#17:49 David sáhl rukou do mošny, vzal odtud kámen, vymrštil jej z praku a zasáhl Pelištejce do čela. Kámen mu prorazil čelo a on se skácel tváří k zemi.
#17:50 Tak zdolal David Pelištejce prakem a kamenem, zasáhl Pelištejce a usmrtil ho, aniž měl v ruce meč.
#17:51 David přiběhl a stanul u Pelištejce. Popadl jeho meč, vytrhl jej z pochvy a usmrtil ho; uťal mu jím hlavu. Když Pelištejci viděli, že jejich hrdina je mrtev, dali se na útěk.
#17:52 Izraelští a judští muži vyskočili, spustili válečný pokřik a pronásledovali Pelištejce až tam, kudy se vstupuje do údolí, a až k branám Ekrónu; i padali ranění Pelištejci cestou od Šaarajimu až ke Gatu a k Ekrónu.
#17:53 Pak Izraelci přestali Pelištejce stíhat a vyplenili jejich tábory.
#17:54 David vzal Pelištejcovu hlavu, přinesl ji do Jeruzaléma a jeho zbroj uložil ve svém stanu.
#17:55 Když Saul viděl Davida, jak jde proti Pelištejci, otázal se vojevůdce Abnéra: „Čí syn je ten mládenec, Abnére?“ Abnér mu odvětil: „Jakože jsi živ, králi, nevím.“
#17:56 Král mu poručil: „Zeptej se, čí syn je ten mladík.“
#17:57 Když se David po vítězství nad Pelištejcem vrátil, uvedl ho Abnér před Saula; Pelištejcovu hlavu měl David v ruce.
#17:58 Saul se ho otázal: „Čí syn jsi, mládenče?“ David odvětil: „Syn tvého služebníka Jišaje Betlémského.“ 
#18:1 I skončil rozhovor se Saulem. Jónatan přilnul celou duší k Davidovi, zamiloval si ho jako sebe sama.
#18:2 Saul ho totiž vzal onoho dne k sobě a nedovolil mu vrátit se do otcovského domu.
#18:3 A Jónatan uzavřel s Davidem smlouvu, neboť ho miloval jako sám sebe.
#18:4 Jónatan svlékl plášť, který měl na sobě, a dal jej Davidovi, též své odění i s mečem, lukem a opaskem.
#18:5 A David podnikal výpravy, a kamkoli ho Saul poslal, měl úspěch. Proto ho Saul ustanovil velitelem nad bojovníky. To se líbilo všemu lidu i Saulovým služebníkům.
#18:6 Tenkrát, když přicházeli, když se David vracel od vítězství nad Pelištejcem, vyšly ze všech izraelských měst zpívající a tančící ženy vstříc králi Saulovi s radostí, s bubínky a loutnami.
#18:7 A křepčící ženy prozpěvovaly: „Saul pobil své tisíce, ale David své desetitisíce.“
#18:8 Saul se velice rozzlobil; taková slova se mu nelíbila. Řekl: „Davidovi přiřkli desetitisíce, a mně jen tisíce. Už mu chybí jen to království.“
#18:9 Od onoho dne hleděl Saul na Davida stále s nedůvěrou.
#18:10 Druhého dne se Saula zmocnil zlý duch od Boha a on uvnitř paláce běsnil jako posedlý. David hrál na citaru jako každý den. Saul měl v ruce kopí.
#18:11 Náhle Saul kopím mrštil. Řekl si: „Přibodnu Davida ke stěně.“ Ale David před ním dvakrát uhnul.
#18:12 Tu se Saul začal Davida bát, neboť s ním byl Hospodin, kdežto od Saula odstoupil.
#18:13 Proto ho Saul odstranil ze své blízkosti a ustanovil ho velitelem nad tisíci. A on vycházel a vcházel před lidem.
#18:14 Na všech výpravách si David vedl úspěšně; Hospodin byl s ním.
#18:15 Když Saul viděl, jak velké má úspěchy, třásl se před ním strachem.
#18:16 Celý Izrael a Juda však Davida miloval, protože on před nimi vycházel a vcházel.
#18:17 Jednou řekl Saul Davidovi: „Tu je má starší dcera Mérab. Dám ti ji za ženu, budeš-li statečně vést Hospodinovy boje.“ Saul si říkal: „Ať to není moje ruka, která ho zasáhne, nýbrž ruka Pelištejců.“
#18:18 David Saulovi odvětil: „Kdo jsem já a co je můj život a má otcovská čeleď v Izraeli, že mám být královým zetěm?“
#18:19 Avšak v době, kdy Saulova dcera Mérab měla být dána Davidovi, byla dána za ženu Adríelovi Mechólatskému.
#18:20 Davida si však zamilovala Saulova dcera Míkal. Oznámili to Saulovi a jemu se to zamlouvalo.
#18:21 Saul totiž řekl: „Dám mu ji, aby mu byla léčkou. A zasáhne ho ruka Pelištejců.“ Davidovi však Saul řekl: „Můžeš se dnes stát mým zetěm prostřednictvím druhé.“
#18:22 A Saul svým služebníkům přikázal: „Namluvte nenápadně Davidovi: Hle, král si tě oblíbil a všichni jeho služebníci tě milují. Nyní se můžeš stát královým zetěm.“
#18:23 Saulovi služebníci to tedy Davidovi přednesli. David namítl: „Myslíte si, že je lehké být zetěm krále? Já jsem člověk chudý a nepatrný.“
#18:24 Služebníci pak Saulovi oznámili: „David mluvil tak a tak.“
#18:25 Saul poručil: „Řekněte Davidovi toto: Král nemá zájem o věno, leč o sto pelištejských předkožek. Nechť je vykonána pomsta nad královými nepřáteli.“ Saul si myslel, že David padne Pelištejcům do rukou.
#18:26 Jeho služebníci oznámili tato slova Davidovi a Davidovi se zamlouvalo stát se královým zetěm. Ještě neuplynuIy stanovené dny,
#18:27 když se David se svými muži vypravil a pobil mezi Pelištejci dvě stě mužů. Jejich předkožky David přinesl; předali je v plném počtu králi. Tak se stal královým zetěm; Saul mu dal svou dceru Míkal za ženu.
#18:28 Saul viděl a poznal, že Hospodin je s Davidem, a jeho dcera Míkal ho milovala.
#18:29 I bál se Saul Davida ještě víc a byl po všechny dny Davidovým nepřítelem.
#18:30 A pelištejští velitelé táhli opět do pole. Ale kdykoli vytáhli, byl David úspěšnější než všichni Saulovi služebníci, takže jeho jméno bylo lidu velmi drahé. 
#19:1 Saul promluvil se svým synem Jónatanem i se všemi svými služebníky o tom, že by měl být David usmrcen. Saulův syn Jónatan si však Davida velice oblíbil.
#19:2 Proto Jónatan Davida varoval: „Můj otec Saul usiluje o to, aby tě usmrtil. Tak se měj prosím ráno na pozoru. Schovej se a zůstaň v úkrytu!
#19:3 Já vyjdu a postavím se vedle svého otce na poli, kde budeš ty, a promluvím o tobě se svým otcem. Když něco zjistím, povím ti to.“
#19:4 A Jónatan se za Davida u svého otce Saula přimlouval. Řekl mu: „Ať se král nedopustí vůči svému služebníku Davidovi hříchu! Vždyť on se proti tobě neprohřešil a jeho činy jsou tobě k užitku.
#19:5 Vlastní život nasadil a zabil Pelištejce. Hospodin tak připravil veliké vysvobození celému Izraeli. Viděl jsi to a radoval ses. Proč by ses měl prohřešit prolitím nevinné krve a Davida bezdůvodně usmrtit?“
#19:6 Saul Jónatana uposlechl. A Saul se zapřisáhl: „Jakože živ je Hospodin, nebude usmrcen.“
#19:7 Jónatan tedy zavolal Davida a pověděl mu všechna tato slova. Nato Jónatan přivedl Davida k Saulovi, aby stával před Saulem jako dříve.
#19:8 Ale vypukla opět válka. David vytáhl do boje proti Pelištejcům a připravil jim zdrcující porážku, že se před ním dali na útěk.
#19:9 Jenže zlý duch od Hospodina zase napadl Saula, když seděl ve svém domě. Měl kopí v ruce a David hrál na citaru.
#19:10 Saul chtěl Davida přibodnout kopím ke stěně. Ten však před Saulem uskočil, takže Saul zabodl kopí do stěny. David té noci utekl a unikl.
#19:11 Saul vyslal k Davidovu domu posly, aby jej střežili a Davida ráno usmrtili. Ale Davidova žena Míkal mu to oznámila. Řekla: „Jestliže se ti nepodaří zachránit se v noci, budeš zítra mrtev.“
#19:12 Míkal spustila Davida oknem a on prchl pryč a unikl.
#19:13 Pak vzala Míkal domácího bůžka, položila na lože, do hlav položila chumáč kozí srsti a přikryla pokrývkou.
#19:14 Když Saul vyslal posly, aby Davida jali, řekla: „Je nemocen.“
#19:15 Saul vyslal posly znovu, aby se na Davida podívali. Řekl: „Přineste ho na loži ke mně, abych ho usmrtil.“
#19:16 Poslové přišli, a hle, na loži domácí bůžek a chumáč kozí srsti v hlavách.
#19:17 Saul se na Míkal osopil: „Proč jsi mě takto obelstila? Pustilas mého nepřítele, aby unikl.“ Míkal Saulovi odpověděla: „On mi vyhrožoval: Pusť mě, jinak tě usmrtím.“
#19:18 David uprchl a unikl. Přišel do Rámy k Samuelovi a oznámil mu všechno, co mu Saul provedl. Pak šel se Samuelem a usadili se v prorockém domě.
#19:19 Tu bylo Saulovi oznámeno: „Hle, David je v prorockém domě v Rámě.“
#19:20 Saul vyslal posly, aby Davida jali. Ti spatřili sbor proroků v prorockém vytržení a Samuela stojícího jako představeného nad nimi. Vtom spočinul duch Boží na Saulových poslech a také oni upadli do prorockého vytržení.
#19:21 Oznámili to Saulovi a on poslal další posly, ale také ti upadli do prorockého vytržení. Poslal posly ještě potřetí, ale i ti upadli do prorockého vytržení.
#19:22 Šel tedy do Rámy sám. Když přišel k té veliké cisterně v Sekú, otázal se: „Kde je tu Samuel a David?“ A dostal odpověď: „V prorockém domě v Rámě.“
#19:23 Šel tam, do prorockého domu v Rámě, ale i na něm spočinul duch Boží, takže dál šel v prorockém vytržení, až přišel do prorockého domu v Rámě.
#19:24 Také on si svlékl šaty a také on upadl před Samuelem do prorockého vytržení. Celý den a celou noc ležel nahý. Proto se říká: „Což také Saul je mezi proroky?“ 
#20:1 David uprchl z prorockého domu v Rámě a přišel si Jónatanovi postěžovat: „Co jsem provedl? Jaký je můj zločin? Jaký je můj hřích vůči tvému otci, že mi ukládá o život?“
#20:2 On mu odvětil: „Toho buď dalek! Nezemřeš. Vždyť můj otec neudělá ani to nejmenší, aniž se mi svěří. Proč by tohle můj otec přede mnou skrýval? Není tomu tak.“
#20:3 David se ještě zapřisáhl a odporoval: „Tvůj otec určitě ví, že jsem získal tvou přízeň. Řekl: ‚Ať o tom neví Jónatan, aby se netrápil.‘ Nicméně, jakože živ je Hospodin a jakože živ jsi ty, od smrti mě dělí jen krůček.“
#20:4 Jónatan se tedy Davida otázal: „Co chceš, abych pro tebe udělal?“
#20:5 David Jónatanovi řekl: „Hle, zítra je novoluní a já mám sedět s králem u jídla. Propusť mě a já se budu až do třetího dne navečer skrývat v poli.
#20:6 Bude-li mě tvůj otec pohřešovat, řekneš: ‚David mě naléhavě prosil, aby si směl odběhnout do svého města Betléma; koná se tam výroční obětní hod pro celou čeleď.‘
#20:7 Jestliže řekne: ‚Dobře‘, může být tvůj služebník klidný, jestliže však ho popadne vztek, věz, že se odhodlal k nejhoršímu.
#20:8 Prokaž proto milosrdenství svému služebníku, neboť jsi vstoupil se svým služebníkem ve smlouvu před Hospodinem. Jestliže jsem spáchal nějaký zločin, usmrť mě ty sám. Proč bys mě teď vodil ke svému otci?“
#20:9 Ale Jónatan odvětil: „Toho buď dalek! Dozvím-li se bezpečně, že se můj otec vůči tobě odhodlal k nejhoršímu, neoznámím ti to snad?“
#20:10 David se Jónatana otázal: „Kdo mi oznámí, když ti tvůj otec odpoví tvrdě?“
#20:11 Jónatan Davida vybídl: „Pojď, vyjdeme na pole.“ Oba vyšli na pole.
#20:12 Tu řekl Jónatan Davidovi: „Při Hospodinu, Bohu Izraele! Když zítra či pozítří v příhodnou dobu od svého otce vyzvím, že je to s Davidem dobré, tedy k tobě nikoho nepošlu a nebudu ti nic vzkazovat.
#20:13 Ať se mnou Hospodin udělá, co chce! Vzkážu ti, rozhodne-li se můj otec vůči tobě pro to nejhorší. Propustím tě a půjdeš v pokoji. Hospodin buď s tebou, jako byl s mým otcem.
#20:14 Avšak i ty mi prokazuj Hospodinovo milosrdenství, pokud budu živ a nezemřu,
#20:15 a nezpřetrhej svazky svého milosrdenství vůči mé rodině nikdy, ani tenkrát, až Hospodin vyhladí Davidovy nepřátele z povrchu země do posledního muže.“
#20:16 I uzavřel Jónatan s Davidovým domem smlouvu: „Ať Hospodin volá Davidovy nepřátele k odpovědnosti.“
#20:17 Nadto zavázal Jónatan Davida přísahou při své lásce k němu, neboť ho z duše miloval.
#20:18 Jónatan mu řekl: „Zítra bude novoluní, a až si všimnou tvého sedadla, budeš pohřešován.
#20:19 Třetího dne rychle sestup a přijď k místu, kde ses skrýval v den jednání, a zůstaň u kamene Ezelu.
#20:20 Já vystřelím vedle něho tři šípy, jako bych je vypustil na nějaký terč.
#20:21 Hned nato pošlu chlapce: ‚Jdi najít šípy.‘ Jestliže chlapci výslovně řeknu: ‚Hle, šípy jsou zde za tebou, seber je‘, pak přijď, neboť to pro tebe bude znamenat pokoj, nic se ti nestane, jakože živ je Hospodin.
#20:22 Jestliže však řeknu mladíkovi: ‚Hle, šípy jsou dál před tebou‘, pak odejdi, neboť tě posílá Hospodin.
#20:23 Ale slovo, které jsme si dali, já a ty, platí. Hospodin je navěky svědkem mezi mnou a tebou!“
#20:24 David se tedy skryl v poli. Když nastalo novoluní, král zasedl k hodu a jedl.
#20:25 Král se posadil tak jako jindy na sedadlo u stěny, ale Jónatan stál; po Saulově boku se posadil Abnér. Povšimli si Davidova místa.
#20:26 Onoho dne o tom Saul nepromluvil, neboť si řekl: „Přihodilo se mu něco, co jej znečistilo, není čistý.“
#20:27 Když však druhého dne po novoluní si opět všimli Davidova místa, otázal se Saul svého syna Jónatana: „Proč Jišajův syn včera ani dnes nepřišel k hodu?“
#20:28 Jónatan Saulovi odpověděl: „David mě naléhavě prosil, aby směl do Betléma.
#20:29 Řekl: ‚Propusť mě, naše čeleď má ve městě obětní hod. Můj bratr mi přikázal přijít. Jestliže jsem tedy získal tvou přízeň, rád bych se uvolnil a podíval se na své bratry.‘ Proto se nedostavil ke královu stolu.“
#20:30 Saul vzplanul proti Jónatanovi hněvem a křičel na něj: „Ty synu poběhlice, což jsem to nevěděl, že si Jišajova syna vyvoluješ ke své hanbě i k hanbě klínu své matky?
#20:31 Po všechny dny života Jišajova syna na zemi neobstojíš ty ani tvé království. Hned pro něho pošli a dej ho přivést ke mně, protože je synem smrti.“
#20:32 Jónatan svému otci Saulovi odpověděl a otázal se ho: „Proč má zemřít? Co provedl?“
#20:33 Avšak Saul po něm mrštil kopím, aby ho proklál. I poznal Jónatan, že jeho otec je odhodlán Davida usmrtit.
#20:34 Jónatan, rozpálen hněvem, vstal od stolu a toho druhého dne po novoluní při hodu nejedl, protože se trápil pro Davida, že ho jeho otec potupil.
#20:35 Za jitra vyšel Jónatan na pole podle dohody s Davidem a byl s ním malý chlapec.
#20:36 Poručil svému chlapci: „Běž najít šípy, které vystřelím.“ Zatímco chlapec běžel, on přes něho vystřelil šíp.
#20:37 Když chlapec přišel k místu, kam Jónatan šíp vystřelil, volal Jónatan za chlapcem: „Není šíp tam dál před tebou?“
#20:38 A Jónatan volal za chlapcem: „Rychle, spěchej, nezastavuj se!“ Jónatanův chlapec posbíral šípy a přišel ke svému pánu.
#20:39 Chlapec nic netušil. Jen Jónatan a David věděli, oč jde.
#20:40 Jónatan pak dal svou zbroj chlapci, kterého měl s sebou, a poručil mu: „Jdi, zanes to do města.“
#20:41 A chlapec šel. Tu povstal David od jižní strany, padl tváří k zemi a třikrát se poklonil. Políbili se a plakali jeden pro druhého, až se David vzchopil.
#20:42 Jónatan řekl Davidovi: „Jdi v pokoji. Co jsme si my dva v Hospodinově jménu přísahali, toho ať je Hospodin navěky svědkem mezi mnou a tebou i mezi mým a tvým potomstvem.“ 
#21:1 David vstal a odešel, zatímco Jónatan se ubíral do města.
#21:2 I přišel David do Nóbu ke knězi Achímelekovi. Achímelek vyděšen vyšel Davidovi vstříc a otázal se ho: „Proč jsi sám a nikdo není s tebou?“
#21:3 David knězi Achímelekovi odvětil: „Mám pověření od krále. Poručil mi: Nikdo ať nezví nic o tom, k čemu tě posílám a čím jsem tě pověřil. Družině jsem uložil, aby čekala na určeném místě.
#21:4 Nyní však mi dej, co máš po ruce, pět chlebů, či co se najde.“
#21:5 Kněz Davidovi odpověděl takto: „Nemám po ruce obyčejný chléb, je tu jenom svatý chléb. Jen jestli se družina vyvarovala styku se ženami.“
#21:6 David odpověděl knězi ujištěním: „Jistě. Žen jsme se nedotkli už předtím. Když jsem odcházel, byla těla členů družiny svatá. Cesta ovšem svatá nebyla, ale dnes bude tímto způsobem posvěcena.“
#21:7 Kněz mu tedy dal svatý chléb, protože tam jiný nebyl, totiž jenom chléb předkladný, který byl od Hospodinovy tváře odložen, aby tam byl položen čerstvý chléb v den, kdy se starý odebíral.
#21:8 V onen den tam však byl jeden ze Saulových služebníků v uzavření před Hospodinem, jménem Dóeg Edómský, nejpřednější ze Saulových pastýřů.
#21:9 David se ještě otázal Achímeleka: „Nemáš zde po ruce nějaké kopí nebo meč? Nevzal jsem s sebou svůj meč ani svou zbroj, protože králova záležitost byla neodkladná.“
#21:10 Kněz řekl: „Je tu meč Pelištejce Goliáše, kterého jsi zabil v dolině Posvátného stromu. Je zavinutý do pláště za efódem. Chceš-li si jej vzít, vezmi. Kromě něho tu žádný jiný není.“ David odvětil: „Není nad něj, dej mi jej.“
#21:11 I povstal David, aby onoho dne prchl před Saulem. Přišel k Akíšovi, králi Gatu.
#21:12 Akíšovi služebníci řekli králi: „Což to není David, král země? Což se o něm nezpívalo při tanečním reji: ‚Saul pobil své tisíce, ale David své desetitisíce‘?“
#21:13 Tato slova utkvěla Davidovi v mysli a velice se bál Akíše, krále Gatu.
#21:14 Proto před nimi změnil své chování, jednal v jejich rukou jako potřeštěný, dělal značky na vrata brány a pouštěl po bradě sliny.
#21:15 Akíš však řekl svým služebníkům: „Víte přece, že to je šílenec, proč ho přivádíte ke mně?
#21:16 Nedostává se mi snad šílenců, že jste přivedli tady toho, aby mě svým šílením obtěžoval? Takový by měl vstoupit do mého domu?“ 
#22:1 David odtud odešel a uchýlil se do jeskyně Adulámu. Když o tom uslyšeli jeho bratři a celý dům jeho otce, sestoupili tam k němu.
#22:2 Také se kolem něho shromáždili všichni utlačovaní, všichni stíhaní věřitelem a všichni, jejichž život byl plný hořkosti. Stal se jejich velitelem. Bylo s ním na čtyři sta mužů.
#22:3 Odtud David odešel do Mispy Moábské. Moábskému králi řekl: „Ať tu prosím zůstanou můj otec a má matka s vámi, dokud nezvím, jak se mnou Bůh naloží.“
#22:4 Nechal je u moábského krále a pobývali u něho po všechny dny, pokud byl David ve skalní skrýši.
#22:5 Prorok Gád pak řekl Davidovi: „Nezůstávej ve skalní skrýši. Odeber se do judské země.“ David šel, až přišel do lesa Cheretu.
#22:6 Saul se doslechl, že se David a jeho muži zase ukázali. Saul tehdy seděl v Gibeji pod tamaryškem na výšině s kopím v ruce a všichni jeho služebníci stáli před ním.
#22:7 Tu řekl svým služebníkům, kteří před ním stáli: „Poslyšte, Benjamínovci! To vám všem dá Jišajův syn pole a vinice? Všechny vás ustanoví za velitele nad tisíci a nad sty?
#22:8 Proto jste se všichni proti mně spikli? Nikdo mi neohlásí, že můj syn uzavřel se synem Jišajovým smlouvu! Nikdo z vás se kvůli mně netrápí a neohlásí mi, že můj syn podporuje mého služebníka, aby mi strojil úklady, jak je tomu právě dnes!“
#22:9 Dóeg Edómský, který byl ustanoven nad Saulovými služebníky, odpověděl: „Viděl jsem syna Jišajova, jak přišel do Nóbu k Achímelekovi, synu Achítúbovu.
#22:10 Ten se kvůli němu doptával Hospodina, dal mu stravu na cestu, i meč Pelištejce Goliáše mu dal.“
#22:11 Král si předvolal kněze Achímeleka, syna Achítúbova, i celý jeho rod, kněžstvo z Nóbu. Všichni se ke králi dostavili.
#22:12 Saul řekl: „Poslyš, synu Achítúbův!“ On odpověděl: „Tu jsem, můj pane.“
#22:13 Saul se na něj obořil: „Proč jste se proti mně spikli, ty a syn Jišajův? Dal jsi mu chléb a meč a doptával ses kvůli němu Boha, aby proti mně povstal a strojil mi úklady, jak je tomu právě dnes.“
#22:14 Achímelek králi odpověděl: „Kdo je mezi všemi tvými služebníky tak věrný jako David? Je zetěm krále, je příslušníkem tvé tělesné stráže a je ve tvém domě vážen.
#22:15 Že bych se teprve dnes kvůli němu doptával Boha, toho jsem dalek. Ať král nepodkládá něco takového svému služebníku ani celému mému rodu! Vždyť tvůj služebník o tom všem vůbec nic neví, ani to nejmenší!“
#22:16 Král se však rozkřikl: „Propadl jsi smrti, Achímeleku, ty i celý tvůj rod.“
#22:17 A král poručil běžcům, kteří stáli před ním: „Obkličte a usmrťte Hospodinovy kněze. I jejich ruka byla s Davidem! Věděli, že prchá, a neohlásili mi to.“ Avšak královi služebníci nechtěli vztáhnout ruku a udeřit na kněze Hospodinovy.
#22:18 Král tedy poručil Dóegovi: „Obklič je ty! Udeř na kněze!“ Dóeg Edómský je obklíčil, udeřil na kněze a usmrtil onoho dne osmdesát pět mužů, nosících lněný efód.
#22:19 Také kněžské město Nób vybil ostřím meče; muže i ženy, pacholata i kojence, býky, osly a ovce pobil ostřím meče.
#22:20 Unikl jen jeden syn Achímeleka, syna Achítúbova, jménem Ebjátar. Uprchl za Davidem.
#22:21 Ebjátar oznámil Davidovi, že Saul Hospodinovy kněze povraždil.
#22:22 David Ebjátarovi řekl: „Věděl jsem onoho dne, že je tam Dóeg Edómský a že to určitě oznámí Saulovi. To jako bych sám napadl všechny příslušníky tvého rodu.
#22:23 Zůstaň se mnou, neboj se. Kdo ukládá o život mně, ukládá o život tobě. U mne jsi pod ochranou.“ 
#23:1 Davidovi oznámili: „Hle, Pelištejci bojují proti Keíle, už plení humna.“
#23:2 David se doptával Hospodina: „Mám jít a pobít tyto Pelištejce?“ Hospodin Davidovi řekl: „Jdi, pobiješ Pelištejce a zachráníš Keílu.“
#23:3 Davidovi muži však namítali: „Už tady v Judsku žijeme ve strachu; natož když potáhneme ke Keíle proti bitevním řadám Pelištejců.“
#23:4 Proto se David doptával Hospodina znovu a Hospodin mu odpověděl. Řekl: „Jen táhni dolů ke Keíle, já ti vydávám Pelištejce do rukou.“
#23:5 David se svými muži tedy odtáhl ke Keíle a bojoval proti Pelištejcům. Zahnal jejich stáda a připravil jim zdrcující porážku. Tak David vysvobodil obyvatele Keíly.
#23:6 Když Ebjátar, syn Achímelekův, uprchl k Davidovi do Keíly, vzal s sebou efód.
#23:7 Saulovi bylo oznámeno, že David přišel do Keíly. Saul řekl: „Bůh ho vydal do mých rukou. Je polapen, vstoupil do města s vraty a závorami.“
#23:8 Saul vyzval k boji všechen lid, aby táhli dolů na Keílu a oblehli Davida a jeho muže.
#23:9 Když David zvěděl, že Saul proti němu chystá to nejhorší, poručil knězi Ebjátarovi: „Přines efód.“
#23:10 A David se ptal: „Hospodine, Bože Izraele, tvůj služebník dostal jistou zprávu, že Saul hodlá přitrhnout ke Keíle a zničit město kvůli mně.
#23:11 Vydají mě občané Keíly do jeho rukou? Sestoupí sem Saul, jak to slyšel tvůj služebník? Hospodine, Bože Izraele, oznam to prosím svému služebníku.“ Hospodin odpověděl: „Sestoupí.“
#23:12 David se tázal: „Vydají občané Keíly mě i mé muže Saulovi do rukou?“ Hospodin odpověděl: „Vydají.“
#23:13 Nato David se svými muži, bylo jich kolem šesti set, vytáhl z Keíly a přecházeli z místa na místo. Když bylo Saulovi oznámeno, že David z Keíly unikl, zanechal tažení.
#23:14 David pobýval v poušti na nepřístupných vrcholcích; pobýval v horách pouště Zífu. Ačkoli po něm Saul po celý čas pátral, Bůh mu ho do rukou nevydal.
#23:15 David viděl, že Saul vytáhl, aby mu ukládal o život, i ukryl se v poušti Zífu v Choreši.
#23:16 Jónatan, syn Saulův, se vypravil a šel za Davidem do Choreše. Ve jménu Božím mu dodal odvahy.
#23:17 Řekl mu: „Neboj se, ruka mého otce Saula tě nenajde. Ty budeš kralovat nad Izraelem a já budu druhým po tobě. Také můj otec Saul to ví.“
#23:18 Oba uzavřeli před Hospodinem smlouvu. David zůstal v Choreši a Jónatan se odebral ke svému domu.
#23:19 Zífejci vystoupili k Saulovi do Gibeje a řekli: „David se přece skrývá u nás na nepřístupných vrcholcích v Choreši, na pahorku Chakíle, na jih od pouště Ješímónu.
#23:20 Nyní tedy, králi, když celou svou duší toužíš táhnout dolů, přitáhni. Vydat ho králi do rukou už bude na nás.“
#23:21 Saul odvětil: „Buďte požehnáni od Hospodina, neboť jste se mnou měli soucit.
#23:22 Jděte ještě vykonat přípravy. Zjistěte a obhlédněte místo, kde se zdržuje a kdo ho tam viděl. Bylo mi řečeno, že si počíná velmi vychytrale.
#23:23 Obhlédněte a zjistěte všechny úkryty, kde se ukrývá. Až to zjistíte, vraťte se ke mně a já potáhnu s vámi. Jestli je v zemi, budu po něm slídit mezi všemi judskými rody.“
#23:24 Vypravili se tedy a šli před Saulem do Zífu. Ale David se svými muži byl v poušti Maónu, v pustině na jih od pouště Ješímónu.
#23:25 Saul se dal se svými muži do pátrání. Když to oznámili Davidovi, táhl dolů ke skalisku a zůstal v poušti Maónu. Saul o tom uslyšel a Davida v poušti Maónu pronásledoval.
#23:26 Saul táhl po jedné straně pohoří, zatímco David se svými muži byl na druhé straně pohoří. David se snažil před Saulem nakvap ustoupit, ale Saul se svými muži Davida a jeho muže obklíčil, aby je pochytali.
#23:27 Vtom přišel k Saulovi posel se zprávou: „Rychle odtáhni, neboť do země vpadli Pelištejci.“
#23:28 Saul tedy přestal pronásledovat Davida a táhl proti Pelištejcům. Proto se to místo nazývá Skalisko rozdělení. 
#24:1 David odtud vystoupil a usadil se na nepřístupných vrcholcích u Én-gedí.
#24:2 Když se Saul vrátil ze stíhání Pelištejců, oznámili mu: „Hle, David je v poušti Én-gedí.“
#24:3 Saul tedy vzal tři tisíce mužů vybraných z celého Izraele a vydal se hledat Davida a jeho muže po Kozorožčích skalách.
#24:4 Došel až k ohradám pro stáda, které byly při cestě; byla tam jeskyně. Saul do ní vstoupil, aby vykonal potřebu. V odlehlém koutu jeskyně však seděl David se svými muži.
#24:5 Tu Davidovi jeho muži řekli: „Toto je den, o němž ti Hospodin řekl: ‚Vydám ti do rukou tvého nepřítele.‘ Můžeš s ním naložit, jak se ti zlíbí.“ David se přikradl a odřízl cíp Saulova pláště.
#24:6 Ale pro to odříznutí cípu Saulova pláště si David potom dělal výčitky.
#24:7 Svým mužům řekl: „Chraň mě Hospodin, abych se dopustil něčeho takového na svém pánu, na Hospodinově pomazaném, a vztáhl na něho ruku. Je to přece Hospodinův pomazaný!“
#24:8 Těmi slovy David své muže zarazil a nedovolil jim povstat proti Saulovi. Saul vstal, vyšel z jeskyně a šel svou cestou.
#24:9 Potom vstal i David, vyšel z jeskyně a volal za Saulem: „Králi, můj pane!“ Saul se ohlédl. David padl na kolena tváří k zemi a klaněl se.
#24:10 David Saulovi řekl: „Proč posloucháš lidské řeči, že David usiluje o tvou zkázu?
#24:11 Hle, na vlastní oči dnes vidíš, že tě dnes v jeskyni vydal Hospodin do mých rukou. Říkali, abych tě zabil. Já jsem tě však ušetřil. Řekl jsem: Na svého pána nevztáhnu ruku. Je to přece Hospodinův pomazaný.
#24:12 Pohleď, můj otče! Pohleď na cíp svého pláště v mé ruce. Odřízl jsem cíp tvého pláště a nezabil jsem tě. Uvaž a pohleď, že na mé ruce nelpí žádné zlo ani nevěrnost. Já jsem se proti tobě neprohřešil, ale ty mi ukládáš o život, chceš mi jej vzít.
#24:13 Ať nás rozsoudí Hospodin. Ať Hospodin vykoná nad tebou pomstu za mne. Má ruka však proti tobě nebude.
#24:14 Jak praví dávné přísloví: ‚Svévole pochází od svévolníků, ale má ruka proti tobě nebude.‘
#24:15 Proti komu vytáhl izraelský král? Koho pronásleduješ? Mrtvého psa, pouhou blechu!
#24:16 Ať Hospodin vede mou při a rozsoudí nás. Ať přihlédne a rozhodne můj spor a svým rozsudkem mě vysvobodí z tvých rukou.“
#24:17 Když David přestal k Saulovi takto mluvit, Saul zvolal: „Což to není tvůj hlas, můj synu Davide?“ A Saul se dal do hlasitého pláče.
#24:18 Pak Davidovi řekl: „Jsi spravedlivější než já. Ty mi odplácíš dobrým, ale já jsem ti odplácel zlým.
#24:19 Dals mi to dnes najevo, když jsi mi prokázal dobro. Hospodin mě vydal do tvých rukou, ale tys mě nezabil.
#24:20 Kdypak někdo najde svého nepřítele a nechá ho odejít v dobrém? Kéž tě Hospodin odmění dobrým za to, cos mi dnes učinil!
#24:21 Věru, teď už vím, že kralovat budeš určitě ty. Izraelské království bude v tvých rukou stálé.
#24:22 A teď mi odpřisáhni při Hospodinu, že nevyhubíš mé potomstvo a že mé jméno z domu mého otce nevyhladíš!“
#24:23 David to Saulovi odpřisáhl. Saul se pak odebral ke svému domu a David se svými muži vystoupil do skalní skrýše. 
#25:1 Když Samuel zemřel, všechen Izrael se shromáždil a oplakával ho. Pochovali ho v jeho domě v Rámě. A David nato sestoupil do Páranské stepi.
#25:2 V Maónu byl muž, který měl hospodářství na Karmelu. Byl velmi zámožný. Měl tři tisíce ovcí a tisíc koz. Právě stříhal na Karmelu ovce.
#25:3 Ten muž se jmenoval Nábal a jeho žena se jmenovala Abígajil. Byla to žena bystrého rozumu a krásné postavy, ale muž byl zatvrzelý a v jednání zlý. Byl to Kálebovec.
#25:4 David se v poušti doslechl, že Nábal stříhá své ovce.
#25:5 I poslal David deset mládenců a řekl jim: „Vystupte na Karmel. Půjdete k Nábalovi a popřejete mu mým jménem pokoj.
#25:6 Řeknete toto: Buď zdráv! Pokoj tobě, pokoj tvému domu, pokoj všemu, co máš.
#25:7 Právě jsem slyšel, že slavíš stříž. Nuže, tvoji pastýři bývali s námi. Neubližovali jsme jim. Nepohřešili nic po všechny dny, co byli na Karmelu.
#25:8 Zeptej se svých mládenců, povědí ti o tom. Kéž moji mládenci u tebe najdou vlídné přijetí! Vždyť přicházíme ve šťastný den. Dej prosím svým otrokům a svému synu Davidovi něco z toho, co máš po ruce.“
#25:9 Davidovi mládenci přišli, vyřídili to vše Nábalovi Davidovým jménem a vyčkávali.
#25:10 Ale Nábal se na Davidovy služebníky rozkřikl: „Kdo je David? Kdo je to Jišajův syn? Dnes přibývá otroků, kteří se odtrhují od svých pánů.
#25:11 Mám snad vzít svůj chléb, svou vodu a zvířata, která jsem porazil pro své střihače, a dát je mužům, o nichž ani nevím, odkud jsou?“
#25:12 Davidovi mládenci se vydali na zpáteční cestu. Vrátili se a všechno mu to oznámili.
#25:13 David svým mužům rozkázal: „Opásejte se každý mečem!“ Každý se tedy opásal mečem. I David se opásal mečem. Za Davidem vytáhlo na čtyři sta mužů a dvě stě jich zůstalo u výstroje.
#25:14 Jeden z mládenců oznámil Nábalově ženě Abígajile: „Hle, David poslal z pouště posly s požehnáním pro našeho pána, ale on se na ně osopil.
#25:15 Ti muži se k nám chovali velice dobře. Neubližovali nám a nic jsme nepohřešili po všechny dny, co jsme se s nimi stýkali, když jsme byli na poli.
#25:16 Byli nám hradbou ve dne v noci po všechny dny, co jsme byli s nimi, když jsme pásli ovce.
#25:17 Teď uvaž a pohleď, co bys měla udělat. Vždyť se na našeho pána a celý jeho dům valí pohroma! A on je takový ničema, že se s ním nedá mluvit.“
#25:18 Abígajil rychle vzala dvě stě chlebů, dva měchy vína, pět připravených ovcí, pět měr praženého zrní, sto sušených hroznů, dvě stě pletenců sušených fíků a naložila to na osly.
#25:19 Svým mládencům řekla: „Jděte napřed, já půjdu za vámi.“ Svému muži Nábalovi neoznámila nic.
#25:20 Když sjížděla na oslu dolů, kryta horou, tu proti ní sestupoval David se svými muži, takže na ně narazila.
#25:21 David si řekl: „Věru, nadarmo jsem střežil ve stepi všechno, co mu patřilo. Ze všeho, co má, nic nepohřešil. A přece se mi odplatil za dobro zlem.
#25:22 Ať Bůh udělá s mými nepřáteli, co chce, jestliže mu ve všem, co má, zanechám do rána jediného močícího na stěnu.“
#25:23 Jakmile Abígajil uviděla Davida, rychle sesedla s osla, padla před Davidem na tvář a poklonila se k zemi.
#25:24 Padla mu k nohám a zvolala: „Má, má je to vina, můj pane. Nechť smí tvá otrokyně k tobě promluvit, vyslechni slova své otrokyně.
#25:25 Kéž můj pán nebere toho ničemu Nábala vážně. Vždyť je takový jako jeho jméno. Jmenuje se Nábal (to je Bloud) a jen blud je v něm. Já, tvá otrokyně, jsem neviděla mládence svého pána, které poslal.
#25:26 Ale nyní, můj pane, jakože živ je Hospodin a jakože živ jsi ty, Hospodin ti zabránil dopustit se krveprolití a pomoci si vlastní rukou. Ať jsou nyní jako Nábal tvoji nepřátelé, kteří vyhledávají zkázu mého pána.
#25:27 Toto požehnání zde přinesla svému pánu tvá služka. Ať je rozdáno družině, která chodí v šlépějích mého pána.
#25:28 Promiň své otrokyni přestoupení. Vždyť Hospodin jistě zbuduje mému pánu trvalý dům. Můj pán vede boje Hospodinovy a po všechny tvé dny se na tobě nenašlo nic zlého.
#25:29 Kdyby někdo povstal, aby tě pronásledoval a ukládal ti o život, ať je život mého pána pojat do svazku živých u Hospodina, tvého Boha. Ale život tvých nepřátel ať vloží do praku a odmrští.
#25:30 Až Hospodin učiní mému pánu všechno to dobré, jež ti přislíbil, a pověří tě, abys byl vévodou nad Izraelem,
#25:31 nebudeš mít újmu ani výčitky, můj pane, že jsi zbytečně prolil krev, aby sis pomohl. Až Hospodin prokáže mému pánu dobro, vzpomeň na svou otrokyni.“
#25:32 David Abígajile odvětil: „Požehnán buď Hospodin, Bůh Izraele, že mi tě dnes poslal vstříc.
#25:33 A požehnán buď tvůj důvtip a požehnána ty sama, žes mě dnes zadržela, abych se nedopustil krveprolití a nepomohl si vlastní rukou.
#25:34 Ale jakože je živ Hospodin, Bůh Izraele, který mi zabránil způsobit ti něco zlého, kdybys mi nebyla rychle vyšla vstříc, nezůstal by Nábalovi do jitřního úsvitu jediný močící na stěnu.“
#25:35 David od ní vzal, co mu přinesla, a řekl jí: „Pokojně vystup do svého domu. Hleď, vyslyšel jsem tě a beru na tebe ohled.“
#25:36 Když Abígajil přišla k Nábalovi, měl právě ve svém domě hody jako nějaký král. Byl dobře naladěn, opilý až přespříliš. Proto mu až do jitřního úsvitu neoznámila ani to nejmenší.
#25:37 Ráno, když Nábal vystřízlivěl, oznámila mu jeho žena, co se událo. Tu ho ranila mrtvice a ztuhl jako kámen.
#25:38 Asi po deseti dnech Hospodin Nábala tvrdě zasáhl, takže zemřel.
#25:39 Když David uslyšel, že Nábal zemřel, řekl: „Požehnán buď Hospodin, že rozhodl spor proti Nábalovi, jenž mě pohaněl, a že překazil svému služebníku zlý čin a že zlobu Nábalovu obrátil Hospodin na jeho hlavu.“ David pak Abígajile poslal vzkaz, že si ji chce vzít za ženu.
#25:40 Davidovi služebníci přišli k Abígajile na Karmel a promluvili k ní: „Posílá nás k tobě David. Chce si tě vzít za ženu.“
#25:41 Hned se poklonila tváří až k zemi a řekla: „Hle, tvoje otrokyně bude služkou, která umývá nohy služebníkům svého pána.“
#25:42 Abígajil se rychle vypravila; jela na oslu a doprovázelo ji pět jejích dívek. Následovala Davidovy posly a stala se jeho ženou.
#25:43 David pojal také Achínoamu z Jizreelu; tak byly obě jeho ženami.
#25:44 Ale Saul dal svou dceru Míkal, ženu Davidovu, Paltímu, synu Lajišovu, který byl z Galímu. 
#26:1 Zífejci přišli říci Saulovi do Gibeje: „David se přece skrývá na pahorku Chakíle naproti poušti Ješímónu.“
#26:2 Saul hned sestoupil do pouště Zífu se třemi tisíci muži vybranými z Izraele, aby v poušti Zífu hledal Davida.
#26:3 Saul se utábořil na pahorku Chakíle, který je při cestě naproti poušti Ješímónu. David, který se usadil v poušti, viděl, že Saul přitáhl do pouště za ním.
#26:4 David totiž vyslal zvědy a zjistil, že Saul opravdu přitáhl.
#26:5 Odebral se tedy na místo, kde se Saul utábořil. David uviděl místo, kde ležel Saul i Abnér, syn Nérův, velitel jeho vojska. Saul spal v ležení a lid tábořil kolem něho.
#26:6 David vyzval Chetejce Achímeleka a Abíšaje, syna Serújina, bratra Jóabova: „Kdo se mnou sestoupí k Saulovi do tábora?“ Abíšaj řekl: „Já s tebou sestoupím.“
#26:7 Tak vnikl David s Abíšajem v noci mezi lid, a hle, Saul leží, spí v ležení a jeho kopí je zabodnuto do země v hlavách lože; Abnér a lid leželi kolem něho.
#26:8 Abíšaj řekl Davidovi: „Bůh ti dnes vydal do rukou tvého nepřítele. Teď dovol, ať ho jedinou ranou přirazím kopím k zemi, druhé rány nebude třeba.“
#26:9 David však Abíšajovi řekl: „Neodpravuj ho! Vždyť kdo vztáhne ruku na Hospodinova pomazaného a zůstane bez trestu?“
#26:10 David dále řekl: „Jakože živ je Hospodin, jistě jej Hospodin zasáhne; buď nadejde jeho den, kdy zemře, anebo odejde do boje a bude smeten.
#26:11 Chraň mě však Hospodin, abych vztáhl ruku na Hospodinova pomazaného. Vezmi tady to kopí, které má v hlavách, i džbánek na vodu a odejděme.“
#26:12 David vzal kopí a džbánek na vodu od hlav Saulova lože a odešli. Nikdo nic neviděl, nikdo nic nevěděl, nikdo se neprobudil, všichni spali; padla na ně mrákota od Hospodina.
#26:13 David pak přešel na protilehlou stranu, postavil se na vrchol hory vpovzdálí, takže mezi nimi byl značný prostor.
#26:14 David zavolal na lid a na Abnéra, syna Nérova: „Jestlipak odpovíš, Abnére?“ Abnér se ozval: „Kdo jsi, že voláš na krále?“
#26:15 David Abnérovi řekl: „Jsi přece muž. Kdo je ti v Izraeli roven? Proč jsi nestřežil krále, svého pána? Někdo z lidu přišel krále, tvého pána, odpravit.
#26:16 Nepočínal sis dobře. Jakože živ je Hospodin, jste syny smrti, protože jste nestřežili svého pána, Hospodinova pomazaného. Podívej se teď, kde je královo kopí a džbánek na vodu, který měl v hlavách!“
#26:17 Saul poznal po hlase Davida. Otázal se: „Je to tvůj hlas, můj synu Davide?“ David řekl: „Je to můj hlas, králi, můj pane.“
#26:18 Dále řekl: „Proč vlastně můj pán pronásleduje svého otroka? Vždyť čeho jsem se dopustil? Co je na mně zlého?
#26:19 Nechť nyní král, můj pán, vyslechne slova svého otroka. Jestli tě proti mně podněcuje Hospodin, nechť přijme vůni obětního daru. Jestli však lidé, ať jsou prokleti před Hospodinem. Vždyť mě dnes zapudili, abych se nemohl podílet na Hospodinově dědictví, jako by řekli: ‚Jdi soužit jiným bohům.‘
#26:20 Kéž má krev nevyteče na zem daleko od Hospodinovy tváře. Vždyť izraelský král vytáhl, aby hledal pouhou blechu, jako by honil po horách koroptev.“
#26:21 Saul na to řekl: „Zhřešil jsem. Vrať se, můj synu Davide. Nic zlého ti už neudělám, protože sis dnes cenil mého života. Počínal jsem si jako pomatenec, převelice jsem chybil.“
#26:22 David odpověděl: „Tu je královo kopí. Ať sem přijde někdo z družiny a odnese je.
#26:23 Hospodin odplatí každému za jeho spravedlnost a věrnost. Hospodin tě dnes vydal do mých rukou, ale já jsem nechtěl na Hospodinova pomazaného vztáhnout ruku.
#26:24 Hle, jaký význam jsem dnes přikládal tvému životu, takový význam ať přikládá Hospodin životu mému a vysvobodí mne z každé úzkosti!“
#26:25 Saul Davidovi pravil: „Buď požehnán, můj synu Davide! Jistě mnoho vykonáš a dokážeš.“ David pak šel svou cestou a Saul se vrátil ke svému místu. 
#27:1 David si řekl: „Kteréhokoli dne mohu být zahuben Saulovou rukou. Nezbývá mi nic lepšího než uniknout do pelištejské země. Saul mě nechá být, přestane mě po celém izraelském území hledat. Tak mu uniknu z rukou.“
#27:2 A David hned přešel spolu se svými šesti sty muži k Akíšovi, synu Maókovu, králi Gatu.
#27:3 David se usadil u Akíše v Gatu, on i jeho muži, každý se svou rodinou, David se svými dvěma ženami, Achínoamou Jizreelskou a Abígajilou Karmelskou, ženou po Nábalovi.
#27:4 Když Saulovi ohlásili, že David uprchl do Gatu, už ho dál nehledal.
#27:5 David řekl Akíšovi: „Jestli jsem získal tvou přízeň, kéž je mi dáno místo v některém z venkovských měst a já se tam usadím. Proč by měl tvůj služebník sídlit s tebou v městě královském?“
#27:6 Akíš mu dal onoho dne Siklag. Proto náleží Siklag judským králům dodnes.
#27:7 Údobí, po které David sídlil v pelištejské krajině, trvalo celkem jeden rok a čtyři měsíce.
#27:8 David se svými muži vycházel a podnikal vpády proti Gešúrejcům, Girzejcům a Amálekovcům; ti všichni byli odedávna obyvateli té země od cesty do Šúru až po egyptskou zemi.
#27:9 David pobíjel obyvatele země, nenechával naživu muže ani ženu, ale bral brav a skot, osly, velbloudy i šaty. S tím se vracel a přicházel k Akíšovi.
#27:10 Když se Akíš ptal: „Kam jste dnes vpadli?“, David odpovídal: „Na jih Judy“ nebo „Na jih Jerachmeelců“ nebo „Na jih Kénijců“.
#27:11 David nenechával naživu muže ani ženu a nevodil je do Gatu. Říkal: „Mohli by na nás vyzradit: Toto dělá David.“ Takto si počínal po celou dobu, co sídlil v pelištejské krajině.
#27:12 Akíš Davidovi důvěřoval. Říkal: „Vzbudil velikou nelibost u svého lidu, Izraele; navždy zůstane mým služebníkem.“ 
#28:1 V oněch dnech shromáždili Pelištejci své válečné šiky, aby bojovali proti Izraeli. Akíš řekl Davidovi: „Jistě víš, že potáhneš se mnou v šiku ty i tvoji muži.“
#28:2 David Akíšovi odvětil: „Však poznáš sám, co tvůj služebník udělá!“ Akíš na to Davidovi řekl: „Ustanovuji tě tedy svým osobním strážcem po všechny dny.“
#28:3 Samuel zemřel a všechen Izrael ho oplakával. Pochovali ho v Rámě, jeho městě. Saul pak vymýtil ze země vyvolávače duchů zemřelých a jasnovidce.
#28:4 Pelištejci se shromáždili, vtrhli do země a utábořili se v Šúnemu. Saul proto shromáždil celý Izrael a utábořili se v pohoří Gilbóa.
#28:5 Když Saul uviděl pelištejský tábor, strachy se celý roztřásl.
#28:6 I doptával se Saul Hospodina, ale Hospodin mu neodpovídal ani skrze sny ani skrze urím ani skrze proroky.
#28:7 Tu řekl Saul svým služebníkům: „Vyhledejte mi ženu, která vyvolává duchy zemřelých. Půjdu k ní a dotážu se jí.“ Jeho služebníci mu řekli: „Taková žena, která vyvolává duchy zemřelých, je v Én-dóru.“
#28:8 Saul se přestrojil, vzal si jiné šaty a šel tam spolu se dvěma muži. Přišel k té ženě v noci a řekl jí: „Věšti mi prostřednictvím ducha zemřelého. Přivolej mi, koho ti řeknu.“
#28:9 Žena mu odpověděla: „Však ty víš, co udělal Saul, že vyhladil ze země vyvolávače duchů zemřelých a jasnovidce. Proč mi strojíš léčku? Chceš mě vydat na smrt?“
#28:10 Ale Saul se jí zapřisáhl při Hospodinu: „Jakože živ je Hospodin, žádný trest tě za to nestihne.“
#28:11 Žena se ho zeptala: „Koho ti mám přivolat?“ Odvětil: „Přivolej mi Samuela.“
#28:12 Když žena Samuela viděla, hlasitě vykřikla a obrátila se na Saula: „Proč jsi mě obelstil? Vždyť ty jsi Saul!“
#28:13 Ale král jí řekl: „Neboj se! Co vidíš?“ Žena Saulovi odvětila: „Vidím božský zjev, jak vystupuje ze země.“
#28:14 Řekl jí: „Jak vypadá?“ Odpověděla: „Vystupuje starý muž, zahalený pláštěm.“ Saul poznal, že to je Samuel, padl na kolena tváří k zemi a klaněl se.
#28:15 Samuel se Saula otázal: „Proč rušíš můj klid? Proč jsi mě dal přivolat?“ Saul řekl: „Jsem ve velkých úzkostech. Bojují proti mně Pelištejci a Bůh ode mne odstoupil. Vůbec mi neodpovídá ani prostřednictvím proroků ani skrze sny. Proto jsem zavolal tebe, abys mi oznámil, co mám dělat.“
#28:16 Samuel odvětil: „Proč se ptáš mne, když Hospodin od tebe odstoupil a stal se tvým protivníkem?
#28:17 Hospodin učinil, co prohlásil skrze mne: Hospodin vytrhl království z tvé ruky a dal je tvému bližnímu, Davidovi.
#28:18 Žes neuposlechl Hospodina a nestal se vykonavatelem jeho planoucího hněvu proti Amálekovi, za to ti dnes Hospodin udělá toto:
#28:19 Spolu s tebou vydá Hospodin do rukou Pelištejců též Izraele. Zítra budeš ty i tvoji synové u mne. Také izraelský tábor vydá Hospodin do rukou Pelištejců.“
#28:20 Saul se náhle skácel, jak byl dlouhý, k zemi; tak velice se polekal Samuelových slov. Nebylo v něm síly, protože nepojedl chleba celý den a celou noc.
#28:21 Žena přistoupila k Saulovi a viděla, jak strašně se zhrozil. Řekla mu: „Hle, tvoje otrokyně tě uposlechla. Dala jsem v sázku svůj život. Poslechla jsem tě v tom, cos mi poručil.
#28:22 Teď zase ty prosím poslechni svou otrokyni. Předložím ti sousto chleba a pojíš, aby ses posílil, vždyť musíš pokračovat v cestě.“
#28:23 Ale on se zdráhal. Říkal: „Nebudu jíst.“ Jeho služebníci spolu s tou ženou však na něho naléhali a on je uposlechl. Povstal ze země a posadil se na lože.
#28:24 Žena měla doma vykrmeného býčka. Rychle ho připravila k hodu, nabrala mouky, zadělala a napekla nekvašených chlebů.
#28:25 Předložila to Saulovi i jeho služebníkům a oni jedli. Pak se zvedli a odešli ještě té noci. 
#29:1 Pelištejci shromáždili všechny své šiky do Afeku, zatímco Izrael tábořil u pramene, který je v Jizreelu.
#29:2 Pelištejská knížata pochodovala se svými setninami a pluky, kdežto David a jeho muži pochodovali s Akíšem jako poslední.
#29:3 Pelištejští velitelé se ptali: „Co s těmito Hebreji?“ Akíš pelištejským velitelům odvětil: „Vždyť je to David, služebník izraelského krále Saula, který je u mne už rok, ba léta. Neshledal jsem na něm nic zlého ode dne, kdy odpadl od Saula, až dodnes.“
#29:4 Ale pelištejští velitelé se na něho rozlítili. Řekli mu: „Pošli toho muže zpět, ať se vrátí tam na své místo, které jsi mu určil. S námi ať do boje netáhne, aby se v boji nestal naším protivníkem. Čímpak by se více svému pánu zalíbil, než hlavami těchto mužů?
#29:5 Cožpak to není David, kterého opěvovali v tanečním reji: ‚Saul pobil své tisíce, ale David své desetitisíce‘?“
#29:6 Akíš si zavolal Davida a řekl mu: „Jakože živ je Hospodin, ty jsi přímý a tvé vycházení a vcházení u mne v táboře se mi líbí. Neshledal jsem na tobě nic zlého ode dne, kdy jsi ke mně přišel, až dodnes. Ale nelíbíš se knížatům.
#29:7 Vrať se tedy pokojně zpátky, aby ses nedopustil něčeho, co pelištejská knížata považují za zlé.“
#29:8 David Akíšovi namítl: „Čeho jsem se dopustil a co zlého jsi shledal na svém služebníku ode dne, kdy jsem se před tebou objevil, až dodnes, že nesmím jít do boje proti nepřátelům krále, svého pána?“
#29:9 Akíš Davidovi odpověděl: „Uznávám, v mých očích jsi dobrý jako Boží posel. Pelištejští velitelé však řekli: ‚Ať netáhne s námi do boje.‘
#29:10 Nyní tedy za časného jitra ty i služebníci tvého pána, kteří přišli s tebou, za časného jitra hned na úsvitě odejděte.“
#29:11 Tak se David spolu se svými muži vydal za časného jitra na cestu zpátky do pelištejské země, zatímco Pelištejci vystupovali do Jizreelu. 
#30:1 David právě dorazil se svými muži třetího dne do Siklagu, poté co Amálekovci vpadli do Negebu, k Siklagu. Dobyli Siklagu a vypálili jej.
#30:2 Ženy, mladé i staré, které v něm byly, zajali. Nikoho neusmrtili. Odvedli je a šli svou cestou.
#30:3 Když přišel David se svými muži k městu, viděl, že je vypáleno a že jejich ženy, synové a dcery jsou zajati.
#30:4 David i všechen lid, který byl s ním, se dali do hlasitého pláče a plakali až do úplného vysílení.
#30:5 Zajaty byly také obě Davidovy ženy, Achínoam Jizreelská a Abígajil, žena po Nábalovi Karmelském.
#30:6 Davidovi bylo velmi úzko. Lid se domlouval, že ho ukamenuje. Všechen lid byl totiž rozhořčen, každý pro své syny a dcery. David však našel posilu v Hospodinu, svém Bohu.
#30:7 David řekl knězi Ebjátarovi, synu Achímelekovu: „Přines mi efód!“ Ebjátar Davidovi efód přinesl.
#30:8 David se doptával Hospodina: „Mám pronásledovat tu hordu? Dostihnu ji?“ On mu řekl: „Pronásleduj! Jistě dostihneš a jistě všechny vysvobodíš.“
#30:9 David tedy vytáhl se šesti sty muži, které měl při sobě, a došli až k Besórskému úvalu, kde někteří zůstali.
#30:10 David pronásledoval nepřítele se čtyřmi sty muži; dvě stě mužů tam zůstalo, protože byli na smrt zemdleni a nemohli už Besórský úval přejít.
#30:11 Na poli našli nějakého Egypťana a vzali ho k Davidovi. Dali mu najíst chleba a napít vody.
#30:12 Dali mu též pletenec sušených fíků a dva sušené hrozny. Jedl a okřál na duchu, neboť tři dny a tři noci nejedl chléb a nenapil se vody.
#30:13 David se ho zeptal: „Čí jsi a odkud jsi?“ On řekl: „Já jsem rodem Egypťan, otrok jednoho Amálekovce. Před třemi dny mě můj pán opustil, protože jsem byl nemocen.
#30:14 Vpadli jsme do Negebu osídleného Keretejci i na území Judy, totiž do Negebu Kálebova, a vypálili jsme Siklag.“
#30:15 David se ho otázal: „Přivedl bys mě k té hordě?“ On řekl: „Přísahej mi při Bohu, že mě nezabiješ a že mě nevydáš mému pánu do rukou, a přivedu tě k té hordě.“
#30:16 Přivedl ho k nim, a hle, byli roztažení po celé zemi, jedli, pili a oslavovali všechnu tu velikou kořist, kterou pobrali v pelištejské a judské zemi.
#30:17 David je pobíjel od ranního rozbřesku až do večera příštího dne. Nikdo z nich neunikl kromě čtyř set mladých mužů, kteří vsedli na velbloudy a ujeli.
#30:18 Tak vysvobodil David všechno, co Amálekovci pobrali; i obě své ženy David vysvobodil.
#30:19 Nikdo z nich nechyběl, mladí ani staří, synové ani dcery, kořist ani cokoli jim pobrali. Všechno David získal zpět.
#30:20 David také pobral všechen brav a skot. Hnali před ním to stádo a říkali: „To je Davidova kořist.“
#30:21 Tak došel David ke dvěma stům mužů, kteří byli na smrt zemdleni a nemohli jít za Davidem a které zanechali v Besórském úvalu. Vyšli vstříc Davidovi a lidu, který byl s ním. David přistoupil k lidu a popřál jim pokoj.
#30:22 Tu se mezi těmi muži, kteří šli s Davidem, ozvali všichni zlí a ničemové: „Protože s námi nešli, nedostanou nic z kořisti, kterou jsme získali. Ať si každý pouze odvede svou ženu a děti a ať jdou!“
#30:23 Ale David řekl: „Nenakládejte tak, moji bratři, s tím, co nám dal Hospodin. Ochraňoval nás a vydal nám do rukou tu hordu, která na nás přitrhla.
#30:24 Kdopak vás v této věci uposlechne? Podíl toho, jenž vyšel do boje, bude stejný jako podíl toho, jenž zůstal u výstroje; dostanou stejný díl.“
#30:25 Tak tomu bylo od onoho dne i nadále. Ustanovil to jako nařízení a řád pro Izraele, platný podnes.
#30:26 Když David dorazil do Siklagu, rozeslal podíly z kořisti starším judským, svým přátelům, se vzkazem: „To je pro vás dar z kořisti získané na Hospodinových nepřátelích.“
#30:27 Poslal jej starším do Bét-elu, do Rámotu v Negebu, do Jatiru,
#30:28 do Aróeru, do Sifmótu, do Eštemoy,
#30:29 do Rákalu, do měst Jerachmeelovců, do měst Kénijců,
#30:30 do Chormy, do Bór-ašanu, do Ataku,
#30:31 do Chebrónu a na všechna místa, kam David se svými muži chodíval. 
#31:1 Pelištejci bojovali proti Izraeli. Izraelští muži před Pelištejci utíkali a padali pobiti v pohoří Gilbóa.
#31:2 Pelištejci se pustili za Saulem a jeho syny. I pobili Pelištejci Saulovy syny Jónatana, Abínádaba a Malkíšúu.
#31:3 Pak zesílil boj proti Saulovi. Objevili ho lukostřelci a těžce ho postřelili.
#31:4 Saul řekl svému zbrojnoši: „Vytas meč a probodni mě jím, než přijdou ti neobřezanci, aby mě neprobodli oni a nezneuctili.“ Zbrojnoš však nechtěl, velmi se bál. Saul tedy uchopil meč a nalehl na něj.
#31:5 Když zbrojnoš viděl, že Saul zemřel, také on nalehl na svůj meč a zemřel s ním.
#31:6 Toho dne spolu zemřeli Saul, jeho tři synové, jeho zbrojnoš i všichni jeho mužové.
#31:7 Když izraelští muži na druhé straně doliny a na druhé straně Jordánu viděli, že izraelští muži utekli a že Saul a jeho synové zemřeli, opouštěli města a utíkali. Přišli Pelištejci a usadili se v nich.
#31:8 Pelištejci přišli druhého dne, aby obrali pobité, a našli přitom Saula a jeho tři syny padlé v pohoří Gilbóa.
#31:9 Uťali mu hlavu, vzali zbroj a poslali kolovat po pelištejské zemi jako radostné poselství pro dům svých modlářských stvůr i pro lid.
#31:10 Jeho zbroj uložili v Aštartině domě a jeho mrtvolu přibili na hradby Bét-šanu.
#31:11 I uslyšeli o něm obyvatelé Jábeše v Gileádu, o tom, co Pelištejci se Saulem provedli.
#31:12 Všichni válečníci se vypravili, šli celou noc a sňali mrtvolu Saulovu i mrtvoly jeho synů z bétšanských hradeb. Když došli do Jábeše, spálili je tam.
#31:13 Pak posbírali jejich kosti a pochovali je v Jábeši pod tamaryškem. Poté se postili sedm dní.  

\book{II Samuel}{2Sam}
#1:1 Po Saulově smrti se David vrátil z vítězné bitvy s Amálekem a zůstal dva dny v Siklagu.
#1:2 Třetího dne přišel nějaký muž ze Saulova tábora s roztrženým šatem a s prstí na hlavě. Přišel k Davidovi, padl na zem a poklonil se.
#1:3 David se ho zeptal: „Odkud jdeš?“ On mu řekl: „Unikl jsem z izraelského tábora.“
#1:4 David se dále tázal: „Pověz, co se stalo?“ On řekl, že lid utekl z bitvy, že také mnoho lidu padlo a našlo smrt, že i Saul a jeho syn Jónatan zahynuli.
#1:5 David se zeptal mládence, který mu to oznámil: „Jak víš, že zemřel Saul a jeho syn Jónatan?“
#1:6 Mládenec, který mu to oznámil, odvětil: „Náhodou jsem se ocitl v pohoří Gilbóa. A hle, Saul se opíral o své kopí a vozy a jezdci už na něj dotírali.
#1:7 Ještě se obrátil, spatřil mě a zavolal na mě. Ozval jsem se: ‚Tu jsem.‘
#1:8 Zeptal se: ‚Kdo jsi?‘ Odvětil jsem mu: ‚Jsem Amálekovec.‘
#1:9 Vyzval mě: ‚Postav se ke mně a usmrť mě, neboť mě svírá smrtelná křeč, ale ještě je ve mně život.‘
#1:10 Postavil jsem se k němu a usmrtil jsem ho. Poznal jsem, že po svém pádu stejně nebude živ. Sňal jsem mu z hlavy královskou čelenku a z paže náramek a přinesl jsem je sem svému pánu.“
#1:11 David uchopil svůj šat a roztrhl jej, stejně tak i všichni muži, kteří byli s ním.
#1:12 Naříkali, plakali a postili se až do večera pro Saula a pro jeho syna Jónatana i pro Hospodinův lid, pro dům izraelský, že padli mečem.
#1:13 David se zeptal mládence, který mu to oznámil: „Odkud jsi?“ On řekl: „Jsem syn Amálekovského bezdomovce.“
#1:14 David se na něj rozkřikl: „Jak to, že ses nebál vztáhnout ruku a zahubit Hospodinova pomazaného?“
#1:15 David zavolal jednoho z družiny a poručil: „Přistup a sraz ho!“ A on ho ubil k smrti.
#1:16 David mu totiž řekl: „Krev, kterou jsi prolil, ať padne na tvou hlavu. Tvá ústa tě usvědčila, když jsi řekl: ‚Já jsem usmrtil Hospodinova pomazaného.‘“
#1:17 I zpíval David nad Saulem a nad jeho synem Jónatanem tento žalozpěv.
#1:18 Vyzval také Judejce, aby se při něm učili zacházet s lukem. Je zapsán v Knize Přímého.
#1:19 „Ozdoba tvá, Izraeli, na tvých návrších skolena leží. Tak padli bohatýři!
#1:20 Neoznamujte to v Gatu, nezvěstujte to v ulicích Aškalónu, ať se neradují dcery pelištejské, dcery neobřezanců ať nejásají.
#1:21 Hory v Gilbóe, rosa ani déšť ať na vás nesestoupí, ať se v oběť nepozdvihuje nic z vašich polí, neboť tam byl pohozen štít bohatýrů, štít Saulův, nepomazán olejem,
#1:22 nýbrž krví skolených a tukem bohatýrů. Jónatanův luk neselhal nikdy, Saulův meč doprázdna nebil.
#1:23 Saul a Jónatan, v životě hodni líbezné lásky, neodloučili se od sebe ani v smrti. Bývali nad orly bystřejší a nad lvy udatnější.
#1:24 Dcery izraelské, plačte pro Saula, který vás odíval karmínem a rozkošemi, jenž zlatou okrasou zdobil váš oděv.
#1:25 Tak padli bohatýři uprostřed boje, Jónatan na tvých návrších skolen!
#1:26 Stýská se mi po tobě, můj bratře Jónatane, byls ke mně pln něhy, tvá láska ke mně byla podivuhodnější nad lásku žen.
#1:27 Tak padli bohatýři, v zmar přišla válečná zbroj!“ 
#2:1 Potom se David doptával Hospodina: „Mám vystoupit do některého judského města?“ Hospodin mu řekl: „Vystup.“ David se tázal: „Kam mám vystoupit?“ On řekl: „Do Chebrónu.“
#2:2 David tam tedy vystoupil i se svými dvěma ženami, s Achínoamou Jizreelskou a s Abígajilou, ženou po Nábalovi Karmelském.
#2:3 Také své muže, kteří byli s ním, přivedl David i s jejich rodinami. Usadili se v městech chebrónských.
#2:4 I přišli judští muži a pomazali tam Davida za krále nad domem judským. Potom bylo Davidovi oznámeno: „Muži z Jábeše v Gileádu, ti pochovali Saula.“
#2:5 David poslal k mužům v Jábeši v Gileádu posly se vzkazem: „Jste Hospodinovi požehnaní. Prokázali jste milosrdenství svému pánu Saulovi tím, že jste ho pochovali.
#2:6 Nechť nyní Hospodin prokáže milosrdenství a věrnost vám. I já vám budu prokazovat dobro za to, co jste učinili.
#2:7 Teď jednejte rozhodně a buďte stateční. Váš pán Saul zemřel; avšak rovněž mne pomazal dům judský za krále nad sebou.“
#2:8 Ale Abnér, syn Nérův, velitel Saulova vojska, vzal Íš-bóšeta, syna Saulova, a přivedl ho do Machanajimu.
#2:9 Ustanovil ho králem Gileádu, Ašúrců a Jizreelu i nad Efrajimem, Benjamínem a celým Izraelem.
#2:10 Íš-bóšetovi, synu Saulovu, bylo čtyřicet let, když začal nad Izraelem kralovat. Kraloval dva roky. Za Davidem stál pouze dům judský.
#2:11 David byl v Chebrónu králem nad domem judským celkem sedm let a šest měsíců.
#2:12 Když Abnér, syn Nérův, vytáhl se služebníky Íš-bóšeta, syna Saulova, z Machanajimu do Gibeónu,
#2:13 vytáhl také Jóab, syn Serújin, a Davidovi služebníci. Narazili na sebe u Gibeónského rybníka. Usadili se u rybníka, jedni z jedné, druzí z druhé strany rybníka.
#2:14 Abnér navrhl Jóabovi: „Ať nastoupí družiny a uspořádají před námi bojové hry.“ Jóab souhlasil: „Ať nastoupí.“
#2:15 Nastoupili tedy proti sobě ve stejném počtu, dvanáct za Benjamína, za Íš-bóšeta, syna Saulova a dvanáct ze služebníků Davidových.
#2:16 Jeden druhého uchopil za hlavu a vrazil mu do boku dýku. Tak padli zároveň. Ono místo bylo nazváno Chelkat-súrím a leží v Gibeónu.
#2:17 Onoho dne se strhl velmi tuhý boj. Abnér s izraelskými muži byl Davidovými služebníky poražen.
#2:18 Byli tam i tři Serújini synové Jóab, Abíšaj a Asáel. Asáel byl hbitých nohou jako gazela na poli.
#2:19 Pronásledoval Abnéra a neodbočoval od něho vpravo ani vlevo.
#2:20 Abnér se k němu obrátil a tázal se: „Ty jsi Asáel?“ On odvětil: „Jsem.“
#2:21 Abnér mu řekl: „Odboč vpravo nebo vlevo, zmocni se někoho z družiny a vezmi si jeho výzbroj.“ Ale Asáel od něho nechtěl uhnout.
#2:22 Abnér domlouval Asáelovi ještě jednou: „Uhni ode mne; proč tě mám srazit k zemi? Jak bych se mohl podívat na tvého bratra Jóaba?“
#2:23 Protože odmítal uhnout, vrazil mu Abnér pod žebra kopí a kopí jím projelo. I padl tam a na místě zemřel. Každý, kdo přijde k místu, kde Asáel padl a zemřel, zůstane stát.
#2:24 Jóab s Abíšajem pronásledovali Abnéra. Když slunce zapadlo, přišli k pahorku Amě naproti Gíachu, směrem ke Gibeónské poušti.
#2:25 Benjamínovci se shromáždili k Abnérovi a v jednom šiku zaujali postavení na vrcholu jednoho pahorku.
#2:26 Abnér zavolal na Jóaba: „Což musí meč požírat nepřetržitě? Nevíš, že nakonec zůstává jen hořkost? Kdy konečně poručíš lidu, aby se vrátil a nehonil své bratry?“
#2:27 Jóab odpověděl: „Jakože živ je Bůh, kdybys byl nepromluvil, byl by se lid stáhl a přestal honit jeden druhého teprve ráno.“
#2:28 Jóab dal zatroubit na polnici, všechen lid se zastavil a Izraele už nepronásledoval. Dál již nebojovali.
#2:29 Abnér a jeho muži šli po celou tu noc pustinou, přešli Jordán, prošli celý Bitrón, až přišli do Machanajimu.
#2:30 Když se Jóab vrátil a přestal Abnéra honit, shromáždil všechen lid. Z Davidových služebníků pohřešili devatenáct mužů a Asáela.
#2:31 Z Benjamínců a z mužů Abnérových zemřelo tři sta šedesát mužů, které pobili Davidovi služebníci.
#2:32 Asáela přenesli a pochovali v hrobě jeho otce v Betlémě. Jóab šel se svými muži po celou noc; svítání je zastihlo v Chebrónu. 
#3:1 Boj mezi domem Saulovým a Davidovým se protahoval, David se stále vzmáhal, zatímco dům Saulův stále upadal.
#3:2 V Chebrónu se Davidovi narodili synové: Jeho prvorozený byl Amnón z Achínoamy Jizreelské,
#3:3 jeho druhý syn byl Kileab z Abígajily, ženy po Nábalovi Karmelském, třetí byl Abšalóm, syn Maaky, dcery gešúrského krále Talmaje,
#3:4 čtvrtý Adonijáš, syn Chagíty, pátý Šefatjáš, syn Abítaly,
#3:5 a šestý Jitreám z Davidovy manželky Egly. Ti se Davidovi narodili v Chebrónu.
#3:6 Avšak boj mezi domem Saulovým a domem Davidovým trval dále. V Saulově domě nabýval převahy Abnér.
#3:7 Saul měl ženinu jménem Rispu, dceru Ajovu. Íš-bóšet vytýkal Abnérovi: „Jak to, že jsi vešel k ženině mého otce?“
#3:8 Pro ta Íš-bóšetova slova Abnér velice vzplanul a vyčítal: „Copak jsem psí hlava, patřím snad k Judovi? Denně prokazuji milosrdenství domu Saula, tvého otce, jeho bratřím a přátelům. Nedopustil jsem, abys padl do Davidových rukou, a ty mi dnes předhazuješ přečin s tou ženou?
#3:9 Ať se mnou Bůh udělá, co chce! Učiním pro Davida, co mu přisáhl Hospodin.
#3:10 Odejmu království Saulovu domu a upevním Davidův trůn nad Izraelem a nad Judou od Danu až do Beer-šeby.“
#3:11 Íš-bóšet nebyl s to ozvat se Abnérovi ani slovem, protože se ho bál.
#3:12 Abnér poslal za sebe k Davidovi posly se slovy: „Komu patří země?“ a s nabídkou: „Uzavři se mnou smlouvu. Budu stát při tobě a obrátím k tobě celý Izrael.“
#3:13 David odvětil: „Dobře, uzavřu s tebou smlouvu. Ale jednu věc od tebe žádám: Až přijdeš, abys spatřil mou tvář, nespatříš ji, dokud nepřivedeš Saulovu dceru Míkal.“
#3:14 A David vyslal k Íš-bóšetovi, synu Saulovu, posly se vzkazem: „Vydej mi mou ženu Míkal, kterou jsem pravoplatně získal za sto pelištejských předkožek!“
#3:15 Í-bóšet ji dal vzít od muže, od Paltíela, syna Lavišova.
#3:16 Její muž ji s pláčem doprovázel až do Bachurímu. Tam mu Abnér řekl: „Vrať se.“ A on se vrátil.
#3:17 Abnér pak promluvil s izraelskými staršími: „Už dávno jste se dožadovali toho, aby králem nad vámi byl David.
#3:18 Nyní se mějte k činu. Hospodin přece o Davidovi prohlásil: ‚Rukou svého služebníka Davida vysvobodím svůj izraelský lid z rukou Pelištejců i z rukou všech jeho nepřátel.‘“
#3:19 A totéž Abnér přednesl Benjamínovi. Potom Abnér odešel, aby v Chebrónu přednesl Davidovi vše, co Izrael a celý Benjamínův dům považují za dobré.
#3:20 Když Abnér s dvaceti muži přišel k Davidovi do Chebrónu, David pro něho a pro muže, kteří byli s ním, uspořádal hostinu.
#3:21 Abnér Davidovi řekl: „Rád bych odešel a shromáždil ke králi, svému pánu, celý Izrael. Uzavřou s tebou smlouvu a ty budeš nade vším kralovat, jak si budeš přát.“ David pak Abnéra propustil a on pokojně odešel.
#3:22 A hle, přišli Davidovi služebníci s Jóabem z výpravy a přinášeli s sebou velikou kořist. Abnér už u Davida v Chebrónu nebyl, neboť ho David propustil a on pokojně odešel.
#3:23 Tu přišel Jóab s veškerým vojskem, které bylo s ním. Jóabovi oznámili, že Abnér, syn Nérův, přišel ke králi a ten že jej propustil a on pokojně odešel.
#3:24 Jóab vstoupil ke králi a optal se: „Co jsi to udělal? Když k tobě Abnér přišel, proč jsi ho nechal odejít?
#3:25 Znáš Abnéra, syna Nérova! Přišel tě oklamat. Chce znát tvé vycházení a vcházení, chce vědět o všem, co konáš.“
#3:26 Nato Jóab od Davida odešel a vyslal za Abnérem posly. Ti ho přivedli od cisterny v Siře zpět. David však o tom nevěděl.
#3:27 Když se Abnér vrátil do Chebrónu, uchýlil se s ním Jóab dovnitř brány, jako by s ním chtěl důvěrně promluvit. Tam mu zasadil ránu pod žebra, a tak zemřel za krev jeho bratra Asáela.
#3:28 Když o tom David později uslyšel, prohlásil: „Já i mé království jsme před Hospodinem navěky nevinni prolitou krví Abnéra, syna Nérova.
#3:29 Ta ať dopadne na hlavu Jóabovu i na celý dům jeho otce. Ať v Jóabově domě nikdy nechybí trpící výtokem ani stižený malomocenstvím ani držící se hůlky ani padlý mečem ani trpící nedostatkem chleba!“
#3:30 Tak Jóab se svým bratrem Abíšajem zabil Abnéra, protože v Gibeónu usmrtil v boji jejich bratra Asáela.
#3:31 I poručil David Jóabovi a všemu lidu, který byl s ním: „Roztrhněte svůj šat, přepásejte se žíněným rouchem a naříkejte nad Abnérem!“ Král David šel za márami.
#3:32 Abnéra pochovali v Chebrónu a král nad Abnérovým hrobem hlasitě plakal; plakal i všechen lid.
#3:33 Král pěl nad Abnérem žalozpěv: „Což musel zemřít Abnér, tak jako umírá bloud?
#3:34 Nebyly spoutány tvé ruce ani do okovů sevřeny tvé nohy. Padl jsi jak ten, kdo rukou bídáků padá.“ A všechen lid nad ním plakal dále.
#3:35 Pak přistoupil všechen lid k Davidovi, aby ho přiměli k jídlu, dokud byl ještě den. David se však zapřisáhl: „Ať se mnou Bůh udělá, co chce, okusím-li před západem slunce chleba či čehokoli jiného.“
#3:36 A všechen lid to pozoroval a líbilo se jim to, tak jako se lidu líbilo všechno, co král činil.
#3:37 Všechen lid a celý Izrael onoho dne poznal, že Abnér, syn Nérův, nebyl usmrcen z popudu krále.
#3:38 Král svým služebníkům řekl: „Nevíte, že dnešního dne padl v Izraeli velitel, a to veliký?
#3:39 A já jsem dnes tak sláb, já pomazaný král. A tihle muži, Serújini synové, jsou tvrdší než já. Ať Hospodin odplatí tomu, kdo páchá zlo, podle zla, jež spáchal.“ 
#4:1 Když uslyšel Saulův syn, že Abnér v Chebrónu zemřel, ochably jeho ruce. I celý Izrael se zhrozil.
#4:2 Saulův syn měl dva muže, velitele houfů; jeden se jmenoval Baana a druhý Rekáb. Byli to synové Rimóna Beerótského z Benjamínovců; Beerót se totiž také počítá k Benjamínovi.
#4:3 Beeróťané však uprchli do Gitajimu, kde pobývají jako hosté až dodnes.
#4:4 Jónatan, syn Saulův, měl syna, který měl zchromené nohy. Bylo mu pět let, když z Jizreelu došla zpráva o Saulovi a Jónatanovi. Jeho chůva jej tehdy vzala a dala se na útěk. Při útěku tak pospíchala, že upadl; proto kulhal. Jmenoval se Mefíbóšet.
#4:5 Synové Rimóna Beerótského, Rekáb a Baana, přišli za denního horka do Íš-bóšetova domu. Ten ležel za poledne na lůžku.
#4:6 Vešli dovnitř do domu jakoby brát pšenici. Zasadili Íš-bóšetovi ránu pod žebra; potom Rekáb a jeho bratr Baana unikli.
#4:7 Když totiž vešli do domu, on ležel na svém lehátku ve své ložnici. Ubili ho k smrti a uřízli mu hlavu; vzali ji a šli celou noc pustinou.
#4:8 Íš-bóšetovu hlavu donesli Davidovi do Chebrónu. Řekli králi: „Tu je hlava Íš-bóšeta, syna tvého nepřítele Saula, který ti ukládal o život. Hospodin dopřál králi, mému pánu, dnešního dne pomstu nad Saulem i nad jeho potomstvem.“
#4:9 David odpověděl Rekábovi a jeho bratru Baanovi, synům Rimóna Beerótského, slovy: „Jakože živ je Hospodin, který vykoupil můj život ze všeho soužení,
#4:10 toho, který mi oznámil: ‚Hle, Saul je mrtev‘, a pokládal se za zvěstovatele radosti, toho jsem v Siklagu jal a popravil. To jsem mu dal za tu zvěst.
#4:11 Jakpak když svévolníci zavraždí spravedlivého v jeho domě a na jeho lůžku? To bych vás teď neměl volat k odpovědnosti za jeho krev? To bych vás teď neměl vymýtit ze země?“
#4:12 David přikázal družině, aby je popravili. Usekli jim ruce i nohy a pověsili je u rybníka v Chebrónu. Hlavu Íš-bóšetovu vzali a pohřbili v Abnérově hrobě v Chebrónu. 
#5:1 Všechny izraelské kmeny přišly k Davidovi do Chebrónu a prohlásily: „Hle, jsme tvá krev a tvé tělo.
#5:2 Už tenkrát, když králem nad námi byl Saul, vyváděl a přiváděl jsi Izraele ty. Tobě Hospodin řekl: ‚Ty budeš pást Izraele, můj lid, ty budeš vévodou nad Izraelem.‘“
#5:3 Přišli i všichni izraelští starší ke králi do Chebrónu a král David s nimi v Chebrónu uzavřel před Hospodinem smlouvu. I pomazali Davida za krále nad Izraelem.
#5:4 Davidovi bylo třicet let, když se stal králem; kraloval čtyřicet let.
#5:5 V Chebrónu kraloval nad Judou sedm let a šest měsíců, v Jeruzalémě kraloval třiatřicet let nad celým Izraelem i nad Judou.
#5:6 Král šel se svými muži na Jeruzalém proti Jebúsejcům, obyvatelům země. Ti Davidovi řekli: „Sem nevstoupíš, leč bys odstranil slepé a kulhaví.“ Prohlásili: „Sem David nevstoupí.“
#5:7 Ale David dobyl skalní pevnost Sijón, to je Město Davidovo.
#5:8 Onoho dne David řekl: „Každý, kdo chce bít Jebúsejce, ať pronikne vodní štolou na ty kulhavé a slepé, které David z duše nenávidí.“ Proto se říká: ‚Slepý a kulhavý nevstoupí do Božího domu.‘
#5:9 David se usadil ve skalní pevnosti a nazval ji Město Davidovo. Vybudoval okruh od Miló až k domu.
#5:10 David se stále více vzmáhal a Hospodin, Bůh zástupů, byl s ním.
#5:11 Týrský král Chíram poslal k Davidovi posly; poslal mu cedrové dřevo, tesaře a kameníky; ti postavili Davidovi dům.
#5:12 David poznal, že jej Hospodin potvrdil za krále nad Izraelem a že kvůli svému izraelskému lidu povznese jeho království.
#5:13 Po svém příchodu z Chebrónu si vzal David další ženiny a ženy z Jeruzaléma. Davidovi se narodili další synové a dcery.
#5:14 Toto jsou jména těch, kteří se mu narodili v Jeruzalémě: Šamúa, Šóbab, Nátan, Šalomoun,
#5:15 Jibchár, Elíšúa, Nefeg a Jafía,
#5:16 Elíšáma, Eljáda a Elífelet.
#5:17 Když Pelištejci uslyšeli, že byl David pomazán za krále nad Izraelem, vytáhli všichni Pelištejci Davida hledat. David o tom uslyšel a sestoupil ke skalní pevnosti.
#5:18 Pelištejci přitáhli a rozložili se v dolině Refájců.
#5:19 David se doptával Hospodina: „Mám proti Pelištejcům vytáhnout? Vydáš mi je do rukou?“ Hospodin Davidovi odpověděl: „Vytáhni. Určitě ti vydám Pelištejce do rukou.“
#5:20 David přišel do Baal-perasímu. Tam je David pobil. Prohlásil: „Hospodin jako prudké vody prolomil přede mnou řady mých nepřátel.“ Proto pojmenovali to místo Baal-perasím (to je Pán průlomů).
#5:21 Pelištejci tam zanechali své modlářské stvůry; David a jeho muži je odnesli.
#5:22 Pelištejci potom vytáhli znovu a rozložili se v dolině Refájců.
#5:23 Když se David doptával Hospodina, on odpověděl: „Netáhni. Obejdi je zezadu a napadni je směrem od balzámovníků.
#5:24 Jakmile uslyšíš v korunách balzámovníků šelest kroků, tehdy si pospěš, neboť tehdy vyjde před tebou Hospodin a pobije tábor Pelištejců.“
#5:25 David vykonal, co mu Hospodin přikázal, a pobíjel Pelištejce od Geby až do Gezeru. 
#6:1 David se znovu se všemi vybranými muži z Izraele, se třiceti tisíci,
#6:2 a s veškerým lidem, který byl s ním, vydal na cestu z Baalímu Judova, aby odtud přivezli Boží schránu, při níž se vzývá Jméno, jméno Hospodina zástupů, trůnícího nad cheruby.
#6:3 Vezli Boží schránu na novém povozu. Vyzvedli ji z Abínádabova domu na pahorku; Uza a Achjó, synové Abínádabovi, řídili ten nový povoz.
#6:4 Vyzvedli ji tedy z Abínádabova domu na pahorku, Uza šel při Boží schráně, Achjó před schránou.
#6:5 David a všechen izraelský dům křepčili před Hospodinem za doprovodu různých nástrojů z cypřišového dřeva, citar, harf, bubínků, chřestítek a cymbálů.
#6:6 Když přišli k Nákonovu humnu, vztáhl Uza ruku k Boží schráně a zachytil ji, protože spřežení vybočilo z cesty.
#6:7 Hospodin vzplanul proti Uzovi hněvem. Bůh ho tam pro neúctu zabil a on tam při Boží schráně zemřel.
#6:8 Též David vzplanul, neboť se Hospodin prudce obořil na Uzu, a proto nazval to místo Peres-uza (to je Uzovo zbořenisko); jmenuje se tak dodnes.
#6:9 V onen den pojala Davida bázeň před Hospodinem. Řekl: „Jak by mohla Hospodinova schrána vejít ke mně?“
#6:10 Proto David nechtěl přenést Hospodinovu schránu k sobě do Města Davidova. David ji dal tedy dopravit do domu Obéd-edóma Gatského.
#6:11 V domě Obéd-edóma Gatského zůstala Hospodinova schrána tři měsíce. Hospodin Obéd-edómovi i celému jeho domu žehnal.
#6:12 Potom králi Davidovi oznámili, že Hospodin pro Boží schránu žehná Obéd-edómovu domu i všemu, co mu patří. David šel s radostí přenést Boží schránu z domu Obéd-edómova do Města Davidova.
#6:13 Když ti, kdo nesli Hospodinovu schránu, ušli šest kroků, obětoval býka a vykrmené dobytče.
#6:14 A David poskakoval před Hospodinem ze vší síly; byl přitom přepásán lněným efódem.
#6:15 David a všechen izraelský dům vystupovali s Hospodinovou schránou za ryčného troubení polnic.
#6:16 Když Hospodinova schrána vstupovala do Města Davidova, Míkal, dcera Saulova, se právě dívala z okna. Viděla krále Davida, jak se točí a vyskakuje před Hospodinem, a v srdci jím pohrdla.
#6:17 Hospodinovu schránu přinesli a umístili ji na příslušném místě uprostřed stanu, který pro ni David postavil. David obětoval před Hospodinem zápalné a pokojné oběti.
#6:18 Když David dokončil obětování zápalných a pokojných obětí, požehnal lidu ve jménu Hospodina zástupů.
#6:19 Pak podělil všechen lid, všechno množství Izraele, muže i ženy, každého jedním bochánkem chleba, jedním datlovým koláčem a jedním koláčem hrozinkovým. Poté se všechen lid rozešel, každý do svého domu.
#6:20 David se vrátil, aby požehnal svému domu. Tu Míkal, dcera Saulova, vyšla Davidovi vstříc se slovy: „Jak se dnes proslavil izraelský král! Pro oči otrokyň svých služebníků se dnes odhaloval jako nějaký blázen.“
#6:21 David Míkal odvětil: „Před Hospodinem, který mě vyvolil místo tvého otce a místo celého jeho domu a ustanovil mě vévodou Hospodinova lidu, Izraele, před Hospodinem jsem tak dováděl.
#6:22 I když budu ještě víc zlehčován než teď a budu docela maličký i ve vlastních očích, budu vážen právě u těch otrokyň, o nichž jsi mluvila.“
#6:23 A Míkal, dcera Saulova, neměla děti až do dne své smrti. 
#7:1 Když král už sídlil ve svém domě a Hospodin mu dopřál klid ode všech jeho okolních nepřátel,
#7:2 tu řekl král proroku Nátanovi: „Hleď, já sídlím v domě cedrovém, a Boží schrána sídlí pod stanovou houní.“
#7:3 Nátan králi odvětil: „Jen udělej vše, co máš na srdci, neboť Hospodin je s tebou.“
#7:4 Ale té noci se stalo slovo Hospodinovo k Nátanovi:
#7:5 „Jdi a řekni mému služebníku Davidovi: Toto praví Hospodin: Ty mi chceš vybudovat dům, abych v něm sídlil?
#7:6 Nesídlil jsem v domě od toho dne, kdy jsem syny Izraele vyvedl z Egypta, až do dne tohoto. Přecházel jsem se stanem a s příbytkem.
#7:7 Ať jsem přecházel se všemi Izraelci kudykoli, zdalipak jsem kdy řekl některému z izraelských vůdců, jemuž jsem přikázal pást Izraele, svůj lid: ‚Proč mi nezbudujete cedrový dům?‘
#7:8 Nyní tedy promluvíš takto k mému služebníku Davidovi: Toto praví Hospodin zástupů: Vzal jsem tě z pastvin od stáda, abys byl vévodou nad mým lidem, nad Izraelem.
#7:9 Byl jsem s tebou, ať jsi šel kamkoli. Vyhladil jsem před tebou všechny tvé nepřátele. Tvé jméno jsem učinil tak veliké, jako je jméno velikánů na zemi.
#7:10 I svému lidu, Izraeli, jsem připravil místo a zasadil jej; tam bude bydlet a už nikdy nebude znepokojován, už jej nebudou ponižovat bídáci jako dřív,
#7:11 ode dne, kdy jsem správou svého izraelského lidu pověřil soudce. Tobě jsem dopřál klid ode všech tvých nepřátel. Hospodin ti oznamuje, že on vybuduje dům tobě.
#7:12 Až se naplní tvé dny a ty ulehneš ke svým otcům, dám po tobě povstat tvému potomku, který vzejde z tvého lůna, a upevním jeho království.
#7:13 Ten vybuduje dům pro mé jméno a já upevním jeho královský trůn navěky.
#7:14 Já mu budu Otcem a on mi bude synem. Když se proviní, budu ho trestat metlou a ranami jako kteréhokoli člověka.
#7:15 Avšak svoje milosrdenství mu neodejmu, jako jsem je odňal Saulovi, kterého jsem před tebou odvrhl.
#7:16 Tvůj dům a tvé království budou před tebou trvat navěky, tvůj trůn bude navěky upevněn.“
#7:17 Nátan k Davidovi promluvil ve smyslu všech těchto slov a celého tohoto vidění.
#7:18 Král David pak vešel, usedl před Hospodinem a řekl: „Co jsem já, Panovníku Hospodine, a co je můj dům, že jsi mě přivedl až sem?
#7:19 A i to bylo v tvých očích málo, Panovníku Hospodine. Dokonce přislibuješ domu služebníka svého dlouhá léta. Takové naučení dopřáváš člověku, Panovníku Hospodine.
#7:20 O čem by ještě mohl David k tobě promluvit? Panovníku Hospodine, ty znáš svého služebníka.
#7:21 Pro své slovo a podle svého srdce jsi učinil celou tuto velikou věc a dal o ní vědět svému služebníku.
#7:22 Vždyť jsi tak veliký, Hospodine Bože. Není žádného jako ty, není Boha kromě tebe, podle toho všeho, co jsme na vlastní uši slyšeli.
#7:23 Kdo je jako tvůj lid, jako Izrael, jediný pronárod na zemi, jejž si Bůh přišel vykoupit jako svůj lid, a tak si učinil jméno! Vykonal jsi pro něj veliké a hrozné věci, pro svou zemi, před svým lidem, který sis vykoupil z Egypta, z pronárodů, z jeho bohů.
#7:24 Pevně jsi podepřel svůj izraelský lid, je tvým lidem navěky. A sám ses jim stal, Hospodine, Bohem.
#7:25 Nyní tedy, Hospodine Bože, potvrď navěky své slovo, které jsi promluvil o svém služebníku a o jeho domě. Učiň, jak jsi promluvil.
#7:26 Ať je navěky veliké tvé jméno, ať se říká: ‚Hospodin zástupů je Bůh nad Izraelem.‘ A dům tvého služebníka Davida ať je před tebou pevný.
#7:27 Neboť ty, Hospodine zástupů, Bože Izraele, jsi svému služebníku ohlásil: ‚Vybuduji ti dům.‘ Proto tvůj služebník našel odvahu modlit se k tobě tuto modlitbu.
#7:28 Ano, Panovníku Hospodine, ty sám jsi Bůh, tvá slova jsou pravda. Přislíbil jsi svému služebníku takové dobrodiní.
#7:29 Nyní tedy požehnej laskavě domu svého služebníka, aby trval před tebou navěky. Vždyť jsi to, Panovníku Hospodine, přislíbil. Tvým požehnáním bude dům tvého služebníka požehnán navěky.“ 
#8:1 Potom David porazil Pelištejce a podrobil si je. Zbavil Pelištejce vlády nad mateřským městem Gatem.
#8:2 Porazil i Moábce a přeměřil je provazcem; rozkázal jim ulehnout na zem, dva provazce jich odměřil k usmrcení, jeden celý k zachování při životě. Tak se Moábci stali Davidovými otroky a odváděli dávky.
#8:3 David porazil také Hadad-ezera, syna Rechóbova, krále Sóby, když táhl, aby se zmocnil řeky Eufratu.
#8:4 David zajal jeho sedm set jezdců a dvacet tisíc pěšáků. David ochromil koně celé vozby a ponechal koně jen ke stu vozům.
#8:5 Hadad-ezerovi, králi Sóby, přišli na pomoc Aramejci z Damašku. David pobil z Aramejců dvaadvacet tisíc mužů.
#8:6 Pak David umístil do damašského Aramu výsostná znamení. Tak se i Aramejci stali Davidovými otroky a odváděli dávky. Hospodin Davida zachraňoval, ať šel kamkoli.
#8:7 David také pobral zlaté štíty, které měli Hadad-ezerovi služebníci, a přinesl je do Jeruzaléma.
#8:8 Z Betachu a z Berótaje, Hadad-ezerových měst, pobral král David velké množství mědi.
#8:9 Když Toí, král Chamátu, uslyšel, že David porazil celé Hadad-ezerovo vojsko,
#8:10 poslal Toí svého syna Jórama ke králi Davidovi, aby mu popřál pokoje a dobrořečil mu za jeho boj a vítězství nad Hadad-ezerem, že jej porazil. Hadad-ezer vedl totiž s Toím války. Jóram s sebou přinesl stříbrné, zlaté a měděné předměty.
#8:11 I ty král David zasvětil Hospodinu; zasvětil je spolu se stříbrem a zlatem ode všech podmaněných pronárodů,
#8:12 od Aramejců, Moábců, Amónovců, Pelištejců, Amáleka, i z kořisti od Hadad-ezera, syna Rechóbova, krále Sóby.
#8:13 Tak si David získal jméno, když se vracel po svém vítězství nad osmnácti tisíci Aramejců v Solném údolí.
#8:14 I v Edómu umístil výsostná znamení, umístil výsostná znamení po celém Edómu. Všichni Edómci se stali Davidovými otroky. Hospodin Davida zachraňoval, ať šel kamkoli.
#8:15 David kraloval nad celým Izraelem a zjednával právo a spravedlnost všemu svému lidu.
#8:16 Jóab, syn Serújin, byl vrchním velitelem, Jóšafat, syn Achílúdův, byl kancléřem.
#8:17 Sádok, syn Achítúbův, a Achímelek, syn Ebjátarův, byli kněžími. Serajáš byl písařem.
#8:18 Pak tu byli Benajáš, syn Jójadův, a Keretejci a Peletejci. Davidovi synové byli kněžími. 
#9:1 David se zeptal: „Zdalipak zůstal ještě někdo ze Saulova domu? Rád bych mu kvůli Jónatanovi prokázal milosrdenství.“
#9:2 K Saulovu domu patřil otrok jménem Síba. Toho předvolali k Davidovi. Král mu řekl: „Ty jsi Síba?“ On odvětil: „Tvůj otrok.“
#9:3 Král se otázal: „Už nezůstal ze Saulova domu nikdo? Rád bych mu prokázal Boží milosrdenství.“ Síba králi odvětil: „Je tu ještě Jónatanův syn, zchromený na nohy.“
#9:4 Král se otázal: „Kde je?“ A Síba králi řekl: „Ten je v Lódebaru, v domě Makíra, syna Amíelova.“
#9:5 Král David ho dal tedy přivést z Lódebaru, z domu Makíra, syna Amíelova.
#9:6 Když Mefíbóšet, syn Jónatana, syna Saulova, přišel k Davidovi, padl tváří k zemi a klaněl se. David řekl: „Mefíbóšete!“ On pravil: „Zde je tvůj otrok.“
#9:7 David mu řekl: „Neboj se. Rád bych ti prokázal milosrdenství kvůli Jónatanovi, tvému otci. Vrátím ti všechna pole tvého děda Saula a budeš každodenně jídat u mého stolu.“
#9:8 On se poklonil a řekl: „Co je tvůj služebník, že se obracíš k mrtvému psu, jako jsem já?“
#9:9 Král předvolal Síbu, sluhu Saulova, a řekl mu: „Všechno, co patřilo Saulovi a celému jeho domu, jsem dal vnukovi tvého pána.
#9:10 Ty mu budeš obdělávat půdu, ty se svými syny a svými otroky budeš dodávat chléb k obživě pro vnuka svého pána. Mefíbóšet, vnuk tvého pána, bude každodenně jídat u mého stolu.“ Síba měl patnáct synů a dvacet otroků.
#9:11 Odpověděl králi: „Vše, co král, můj pán, svému služebníku přikázal, tvůj služebník splní, ačkoli Mefíbóšet může jíst u mého stolu jako jeden z královských synů.“
#9:12 Mefíbóšet měl malého syna jménem Míka. Všichni obyvatelé Síbova domu byli Mefíbóšetovými otroky.
#9:13 A tak sídlil Mefíbóšet v Jeruzalémě, neboť každodenně jídal u králova stolu. Kulhal na obě nohy. 
#10:1 Potom se stalo, že zemřel král Amónovců a po něm se ujal království jeho syn Chanún.
#10:2 I řekl David: „Prokáži milosrdenství Chanúnovi, synu Náchašovu, jako prokázal jeho otec milosrdenství mně.“ David tedy poslal své služebníky, aby ho jejich prostřednictvím potěšil v zármutku nad otcem. Davidovi služebníci přišli do země Amónovců.
#10:3 Amónovští velitelé však svého pána Chanúna podněcovali: „Tobě se zdá, že David chce uctít tvého otce? Že k tobě poslal těšitele? Neposlal David své služebníky k tobě proto, aby prozkoumal město, aby je mohl oblehnout a vyvrátit?“
#10:4 Chanún dal tedy Davidovým služebníkům oholit polovinu brady a uříznout polovinu roucha až k zadkům a tak je propustil.
#10:5 Když to oznámili Davidovi, poslal jim naproti, protože ti muži byli velice zhanobeni. Král jim nařídil: „Zůstaňte v Jerichu, dokud vám brady neobrostou; pak se vrátíte.“
#10:6 Amónovci viděli, že vzbudili Davidovu nelibost. Poslali k Aramejcům z Bét-rechóbu a Aramejcům ze Sóby a najali za mzdu dvacet tisíc pěšáků, od krále z Maaky tisíc mužů a od Íš-tóba dvanáct tisíc mužů.
#10:7 Když to David uslyšel, poslal Jóaba s celým vojskem bohatýrů.
#10:8 Amónovci vytáhli a seřadili se k boji u vchodu do brány. Aramejci ze Sóby a Rechóbu a Íš-tób a Maaka stáli stranou v poli.
#10:9 Když Jóab spatřil, že má proti sobě bitevní řady zpředu i zezadu, vybral si nejlepší ze všech izraelských mladíků a seřadil je proti Aramejcům.
#10:10 Zbytek lidu podřídil svému bratru Abíšajovi a seřadil jej proti Amónovcům.
#10:11 Řekl: „Budou-li mít Aramejci nade mnou převahu, přijdeš mi na pomoc. Budou-li Amónovci mít převahu nad tebou, přijdu ti na pomoc já.
#10:12 Buď rozhodný! Vzchopme se! Za náš lid, za města našeho Boha! Hospodin nechť učiní, co uzná za dobré.“
#10:13 Jóab a lid, který byl s ním, se dali do boje proti Aramejcům; ti před ním utekli.
#10:14 Když Amónovci viděli, že Aramejci utíkají, utekli také oni před Abíšajem a vešli do města. Jóab nechal Amónovce a přišel do Jeruzaléma.
#10:15 Když Aramejci viděli, že jsou Izraelem poraženi, opět se shromáždili.
#10:16 Hadad-ezer vydal rozkaz, aby vytáhli též Aramejci za Řekou. Ti přišli do Chélamu. V jejich čele byl Šóbak, velitel Hadad-ezerova vojska.
#10:17 Oznámili to Davidovi a on shromáždil celý Izrael, překročil Jordán a přišel do Chélamu. Aramejci se proti Davidovi seřadili a bojovali s ním.
#10:18 Aramejci se však dali před Izraelem na útěk. David pobil koně od sedmi set aramejských vozů a čtyřicet tisíc jezdců. Zabil také Šóbaka, velitele vojska; ten na místě zemřel.
#10:19 Když všichni králové, Hadad-ezerovi služebníci, viděli, že jsou Izraelem poraženi, uzavřeli s Izraelem příměří a sloužili mu. Aramejci se pak už báli jít Amónovcům na pomoc. 
#11:1 Na přelomu roku, v době, kdy králové táhnou do boje, poslal David Jóaba a s ním své služebníky i celý Izrael, aby hubili Amónovce a oblehli Rabu. David však zůstal v Jeruzalémě.
#11:2 Jednou k večeru vstal David z lože a procházel se po střeše královského domu. Tu spatřil ze střechy ženu, která se právě omývala. Byla to žena velmi půvabného vzhledu.
#11:3 David si dal zjistit, kdo je ta žena. Ptal se: „Není to Bat-šeba, dcera Elíamova, manželka Chetejce Urijáše?“
#11:4 David pak pro ni poslal posly. Ona k němu přišla a on s ní spal; očistila se totiž od své nečistoty. Potom se vrátila do svého domu.
#11:5 Ta žena však otěhotněla a poslala Davidovi zprávu: „Jsem těhotná.“
#11:6 David vzkázal Jóabovi: „Pošli mi Chetejce Urijáše.“ Jóab tedy poslal Urijáše k Davidovi.
#11:7 Urijáš k němu přišel a David se ho ptal, jak se daří Jóabovi, jak se daří lidu a jak se daří v boji.
#11:8 Pak řekl David Urijášovi: „Zajdi do svého domu a umyj si nohy.“ Urijáš vyšel z královského domu a za ním nesli královský dar.
#11:9 Urijáš však ulehl se všemi služebníky svého pána u vchodu do královského domu a do svého domu nezašel.
#11:10 Když oznámili Davidovi, že Urijáš do svého domu nezašel, otázal se David Urijáše: „Což jsi nepřišel z cesty? Proč jsi tedy nezašel do svého domu?“
#11:11 Urijáš Davidovi odvětil: „Schrána, Izrael i Juda sídlí v stáncích, můj pán Jóab i služebníci mého pána táboří v poli. A já bych měl vstoupit do svého domu, jíst a pít a spát se svou ženou? Jakože jsi živ, jakože živa je tvá duše, něčeho takového se nedopustím!“
#11:12 David Urijášovi řekl: „Pobuď tu ještě dnes, zítra tě propustím.“ Urijáš tedy zůstal v Jeruzalémě toho dne i nazítří.
#11:13 David ho pozval, aby s ním jedl a pil, a opil ho. On však večer vyšel a ulehl na svém loži se služebníky svého pána. Do svého domu nezašel.
#11:14 Ráno David napsal Jóabovi dopis a poslal jej po Urijášovi.
#11:15 V tom dopise psal: „Postavte Urijáše do nejtužšího boje a pak od něho ustupte, ať je zabit a zemře.“
#11:16 Jóab tedy obhlédl město a určil Urijášovi místo, o kterém věděl, že tam jsou vybraní válečníci.
#11:17 Vtom vytrhli mužové města a bojovali s Jóabem. Z lidu padlo několik Davidových služebníků, též Chetejec Urijáš našel smrt.
#11:18 Jóab poslal Davidovi zprávu o všem, co se v té bitvě stalo.
#11:19 Poslovi přikázal: „Vypovíš králi všechno o té bitvě.
#11:20 On ti v návalu královského rozhořčení jistě vytkne: ‚Proč jste bojovali tak blízko města? Což jste nevěděli, že z hradeb střílejí?
#11:21 Kdo zabil Abímeleka, syna Jerúbešetova? Nebyla to žena, která na něho svrhla ze zdi mlýnský kámen, takže umřel v Tebesu? Proč jste se přibližovali k hradbám?‘ Pak řekneš: ‚Také tvůj služebník Chetejec Urijáš je mrtev.‘“
#11:22 Posel šel, přišel k Davidovi a oznámil všechno, s čím ho Jóab poslal.
#11:23 Posel Davidovi řekl: „Ti muži byli v přesile a vytrhli proti nám do pole, ale my jsme na ně dotírali až do vrat brány.
#11:24 Vtom začali z hradeb na tvé služebníky střílet lučištníci a usmrtili několik královských služebníků. Také tvůj služebník Chetejec Urijáš je mrtev.“
#11:25 David řekl poslovi: „Vyřiď Jóabovi: ‚Nepokládej to za nic zlého, vždyť meč požírá tak i onak. Zesil svůj boj proti městu a zboř je.‘ Posilni ho!“
#11:26 Když Urijášova žena uslyšela, že její muž Urijáš je mrtev, naříkala nad svým manželem.
#11:27 Jakmile však smutek pominul, David pro ni poslal, přijal ji do svého domu a ona se stala jeho ženou. Porodila mu pak syna. Ale v očích Hospodinových bylo zlé, co David spáchal. 
#12:1 Hospodin poslal k Davidovi Nátana. Ten k němu přišel a řekl mu: „V jednom městě byli dva muži, jeden boháč a druhý chudák.
#12:2 Boháč měl velmi mnoho bravu a skotu.
#12:3 Chudák neměl nic než jednu malou ovečku, kterou si koupil. Živil ji, rostla u něho spolu s jeho syny, jedla z jeho skývy chleba a pila z jeho poháru, spávala v jeho klíně a on ji měl jako dceru.
#12:4 Tu přišel k bohatému muži návštěvník. Jemu bylo líto vzít nějaký kus ze svého bravu či skotu, aby jej připravil poutníkovi, který k němu přišel. Vzal tedy ovečku toho chudého muže a připravil ji muži, který k němu přišel.“
#12:5 David vzplanul proti tomu muži náramným hněvem. Řekl Nátanovi: „Jakože živ je Hospodin, muž, který tohle spáchal, je synem smrti!
#12:6 A tu ovečku nahradí čtyřnásobně zato, že něco takového spáchal a neměl soucitu.“
#12:7 Nátan Davidovi odvětil: „Ten muž jsi ty! Toto praví Hospodin, Bůh Izraele: Já jsem tě pomazal za krále nad Izraelem a já jsem tě vytrhl ze Saulových rukou.
#12:8 Dal jsem ti dům tvého pána a do tvé náruče ženy tvého pána, dal jsem ti dům Izraelův i Judův, a kdyby ti to bylo málo, přidal bych ti mnohem víc.
#12:9 Proč jsi pohrdl Hospodinovým slovem a dopustil ses toho, co je v jeho očích zlé? Chetejce Urijáše jsi zabil mečem a jeho ženu sis vzal za manželku, zavraždils ho mečem Amónovců.
#12:10 Nyní se už nikdy nevyhne meč tvému domu, protože jsi mnou pohrdl a vzal jsi Chetejci Urijášovi ženu, aby ti byla manželkou.
#12:11 Toto praví Hospodin: Hle, já způsobím, aby proti tobě povstalo zlo z tvého domu. Před tvýma očima vezmu tvé ženy a dám je tomu, kdo je ti blízký; ten bude s tvými ženami spát za bílého dne.
#12:12 A ačkoli tys to spáchal tajně, já tuto věc učiním před celým Izraelem, a to za dne.“
#12:13 David Nátanovi řekl: „Zhřešil jsem proti Hospodinu.“ Nátan Davidovi pravil: „Týž Hospodin hřích s tebe sňal. Nezemřeš.
#12:14 Poněvadž jsi však touto věcí zavinil, aby nepřátelé Hospodina znevažovali, syn, který se ti narodí, musí zemřít.“
#12:15 Nátan pak odešel do svého domu. I zasáhl Hospodin dítě, které Davidovi porodila Urijášova žena, takže těžce onemocnělo.
#12:16 David kvůli chlapci hledal Boha a tvrdě se postil, pak šel domů a ležel přes noc na zemi.
#12:17 Starší jeho domu k němu přistoupili, aby ho zvedli ze země, on však nechtěl, ba ani s nimi chléb nepojedl.
#12:18 Sedmého dne dítě zemřelo. Davidovi služebníci se mu báli oznámit, že dítě zemřelo. Říkali: „Když dítě bylo ještě naživu, domlouvali jsme mu, ale neuposlechl nás. Jak bychom mu teď mohli říci: ‚Dítě zemřelo‘? Provedl by něco zlého.“
#12:19 Když David viděl, že si jeho služebníci šeptají, porozuměl, že dítě zemřelo. David se tedy svých služebníků zeptal: „Dítě zemřelo?“ Odvětili: „Zemřelo.“
#12:20 A David vstal ze země, umyl se, pomazal se, převlékl si oděv, vešel do Hospodinova domu a klaněl se. Pak vstoupil do svého domu a požádal, aby mu předložili chléb, a jedl.
#12:21 Jeho služebníci mu pravili: „Jak můžeš takto jednat? Dokud dítě bylo naživu, postil ses a plakal. Jakmile dítě zemřelo, vstaneš a jíš chléb.“
#12:22 Odvětil: „Dokud dítě ještě žilo, postil jsem se a plakal, neboť jsem si říkal: Kdoví, zda se Hospodin nade mnou neslituje a zda dítě nezůstane naživu.
#12:23 Teď zemřelo. Proč bych se měl postit? Což je mohu ještě přivést zpět? Já půjdu k němu, ale ono se ke mně nevrátí.“
#12:24 Pak David potěšil svou ženu Bat-šebu, vešel k ní a spal s ní. I porodila syna a on mu dal jméno Šalomoun. Toho si Hospodin zamiloval
#12:25 a ohlásil skrze proroka Nátana, že ho mají pojmenovat Jedidjáš (to je Hospodinův miláček) kvůli Hospodinu.
#12:26 Jóab bojoval proti Rabě Amónovců; dobýval královské město.
#12:27 I poslal Jóab posly k Davidovi s hlášením: „Bojoval jsem proti Rabě a dobyl jsem Město vod.
#12:28 Nyní tedy seber zbytek lidu a polož se proti tomu městu a dobuď je, abych nedobyl to město já, aby nad ním nebylo provoláváno mé jméno.“
#12:29 David sebral všechen lid, táhl do Raby, bojoval proti ní a dobyl ji.
#12:30 Sňal z hlavy jejich Krále korunu, jejíž zlato s drahokamem vážilo talent. Bývala pak na hlavě Davidově. Z města odvezl též velké množství kořisti.
#12:31 Lid, který byl v něm, odvedl a postavil k pilám, železným špičákům a železným sekyrám a převedl je do cihelny. Tak naložil se všemi městy Amónovců. Pak se David i všechen lid vrátili do Jeruzaléma. 
#13:1 Potom se přihodilo toto: Davidův syn Abšalóm měl krásnou sestru jménem Támar. Do ní se zamiloval Davidův syn Amnón.
#13:2 Amnón se tak soužil, že až pro svou sestru Támaru onemocněl; byla to panna a Amnónovi připadalo nemožné něco si s ní začít.
#13:3 Amnón však měl přítele jménem Jónadaba, syna Davidova bratra Šimey. Jónadab byl muž velmi protřelý.
#13:4 Ten se ho zeptal: „Proč tak den ode dne chřadneš, královský synu? Nepovíš mi to?“ Amnón mu odvětil: „Miluji Támaru, sestru svého bratra Abšalóma.“
#13:5 Jónadab mu poradil: „Ulehni na lůžko a předstírej nemoc. Tvůj otec se na tebe přijde podívat a ty mu řekneš: ‚Nechť přijde prosím má sestra Támar a posilní mě jídlem. Ale ať dělá ten posilující pokrm před mýma očima, abych se mohl dívat a jíst z jejích rukou.‘“
#13:6 Amnón tedy ulehl a předstíral nemoc. Král se na něho přišel podívat. Amnón králi řekl: „Nechť přijde prosím má sestra Támar a udělá před mýma očima dvě srdíčka, abych se z její ruky posilnil.“
#13:7 David poslal k Támaře do domu vzkaz: „Jdi prosím do domu svého bratra Amnóna a udělej mu posilující pokrm.“
#13:8 Támar tedy šla do domu svého bratra Amnóna. On ležel. Vzala těsto, uhnětla je a před jeho očima udělala srdíčka a připravila je.
#13:9 Pak vzala pánev a vyklopila je před něho, ale on odmítal jíst. Poručil: „Ať jdou všichni pryč!“ Všichni tedy šli pryč.
#13:10 Pak řekl Amnón Támaře: „Přines ten posilující pokrm do pokojíka a já se posilním z tvé ruky.“ Támar vzala srdíčka, která udělala, a přinesla je do pokojíka svému bratru Amnónovi.
#13:11 Když k němu však přistoupila, aby mu dala jíst, uchopil ji a řekl jí: „Pojď, spi se mnou, má sestro!“
#13:12 Odvětila mu: „Ne, můj bratře, neponižuj mne! To se v Izraeli přece nedělá! Nedopouštěj se té hanebnosti!
#13:13 Kam bych se poděla se svou potupou? A ty budeš v Izraeli jako nějaký hanebný bloud. Promluv nyní s králem, on mě tobě neodepře.“
#13:14 On však na to nedal a neposlechl; zmocnil se jí, ponížil ji a spal s ní.
#13:15 Pak ji však Amnón začal převelice nenávidět. Nenávist, kterou k ní pociťoval, byla větší než láska, kterou ji miloval. Amnón jí poručil: „Ihned odejdi!“
#13:16 Řekla mu: „Nemám proč. To, že mě vyháníš, je mnohem větší zlo než předešlé, jehož ses na mně dopustil.“ Ale on ji nechtěl slyšet.
#13:17 Zavolal mládence, který mu posluhoval, a poručil: „Ať ji hned ode mne vyvedou! A zavři za ní dveře.“
#13:18 Měla na sobě pestře tkanou suknici; takové řízy totiž oblékaly královské dcery panny. Jeho sluha ji tedy vyvedl a zavřel za ní dveře.
#13:19 Támar si posypala hlavu prachem a roztrhla pestře tkanou suknici, kterou měla na sobě, položila si ruku na hlavu a odcházela s úpěnlivým nářkem.
#13:20 Její bratr Abšalóm se jí zeptal: „Nebyl s tebou tvůj bratříček Amnón? Ale teď, má sestro, mlč. Je to tvůj bratr, nepřipouštěj si to k srdci.“ Zneuctěná Támar se usídlila v domě svého bratra Abšalóma.
#13:21 Když král David uslyšel o všech těchto věcech, velice vzplanul.
#13:22 Abšalóm už nepromluvil s Amnónem ani v dobrém ani ve zlém. Abšalóm totiž Amnóna nenáviděl, protože ponížil jeho sestru Támaru.
#13:23 Po dvou letech se přihodilo, že Abšalóm slavil stříž ovcí v Baal-chasóru, jenž je nedaleko Efrajimu. Abšalóm pozval všechny královské syny
#13:24 a předstoupil před krále s prosbou: „Hle, tvůj otrok slaví stříž. Nechť jde prosím král a jeho služebníci s tvým otrokem.“
#13:25 Král však Abšalómovi odvětil: „Nikoli, můj synu. Všichni jít nemůžeme, nechceme ti být na obtíž.“ A ač na něho naléhal, nechtěl jít, ale požehnal mu.
#13:26 Abšalóm tedy řekl: „Nemohl by s námi jít můj bratr Amnón?“ Král se ho zeptal: „Proč má s tebou jit?“
#13:27 Ale když Abšalóm na něho naléhal, poslal s ním Amnóna i všechny královské syny.
#13:28 Abšalóm přikázal své družině: „Hleďte, až bude Amnón rozjařen vínem a až vám řeknu: ‚Bijte Amnóna!‘, usmrťte ho a nebojte se. Je to na můj příkaz. Buďte rozhodní a stateční!“
#13:29 Abšalómova družina naložila s Amnónem podle Abšalómova příkazu. Tu se všichni královští synové zvedli, vsedli na mezky a dali se na útěk.
#13:30 Byli ještě na cestě, když Davidovi došla zpráva: „Abšalóm pobil všechny královské syny, nezůstal z nich ani jeden.“
#13:31 Král povstal, roztrhl svá roucha a vrhl se na zem. Všichni jeho služebníci stáli s roztrženými rouchy.
#13:32 Tu se ozval Jónadab, syn Davidova bratra Šimey: „Nechť se můj pán nedomnívá, že usmrtili všechny mládence, syny královské. Mrtev je pouze Amnón. Vždyť Abšalóm to měl v úmyslu ode dne, kdy ponížil jeho sestru Támaru.
#13:33 Nyní však ať si král, můj pán, nepřipouští k srdci takovou věc, že všichni královští synové jsou mrtvi. Vždyť mrtev je pouze Amnón.“
#13:34 Abšalóm uprchl. Když se mládenec na hlídce rozhlížel, spatřil, jak vzadu po cestě po úbočí hory přichází mnoho lidí.
#13:35 I řekl Jónadab králi: „Hle, královští synové přicházejí. Jak tvůj služebník pověděl, tak tomu je.“
#13:36 Jen to dořekl, hned nato přišli královští synové a převelice usedavě plakali; také král a všichni jeho služebníci převelice plakali.
#13:37 Abšalóm uprchl a odešel k Talmajovi, synu Amíchúrovu, králi Gešúru. I truchlil David pro svého syna po všechny ty dny.
#13:38 Abšalóm uprchl a odešel do Gešúru a byl tam tři roky.
#13:39 Král David upustil od tažení proti Abšalómovi, neboť Amnónovu smrt oželel. 
#14:1 Jóab, syn Serújin, poznal, že král je Abšalómovi nakloněn.
#14:2 Poslal tedy Jóab do Tekóje pro moudrou ženu a řekl jí: „Předstírej smutek, oblékni si smuteční šaty a nemaž se olejem. Vydávej se za ženu, která již po mnoho dní drží smutek nad mrtvým.
#14:3 Předstoupíš před krále a promluvíš k němu takto.“ A Jóab ji naučil, co má mluvit.
#14:4 Tekójská žena s králem promluvila. Padla tváří k zemi, klaněla se a zvolala: „Pomoz, králi!“
#14:5 Král se jí zeptal: „Co je ti?“ Ona odvětila: „Ach, jsem ovdovělá žena, muž mi zemřel.
#14:6 Tvá otrokyně měla dva syny. Ti se spolu na poli pohádali a nebylo nikoho, kdo by je od sebe odtrhl. Jeden udeřil druhého a usmrtil ho.
#14:7 A teď celé příbuzenstvo povstalo proti tvé otrokyni a žádají: Vydej nám bratrovraha, ať ho usmrtíme za život jeho bratra, kterého zavraždil. Ať vyhladíme i dědice. Tak chtějí uhasit i tu jiskřičku, která mi zbývá, aby po mém muži nezůstalo na zemi ani jméno ani potomstvo.“
#14:8 Král ženě řekl: „Jdi do svého domu a já o tobě rozhodnu.“
#14:9 Ale tekójská žena králi odvětila: „Králi, můj pane, nechť je ta vina na mně a na domu mého otce, avšak král a jeho trůn ať je bez viny.“
#14:10 Král prohlásil: „Kdo by proti tobě mluvil, toho přiveď ke mně a víckrát se tě nedotkne.“
#14:11 Ona řekla: „Nechť prosím král pamatuje na Hospodina, svého Boha, aby krevní mstitel neusiloval dál o zkázu a aby mého syna nezahladili.“ On prohlásil: „Jakože živ je Hospodin, ani vlas tvého syna nepadne na zem!“
#14:12 Žena odvětila: „Nechť smí tvá otrokyně králi, svému pánu, povědět ještě něco.“ Řekl: „Mluv.“
#14:13 Žena pravila: „Proč ty smýšlíš právě tak proti Božímu lidu? Když král vyslovil toto slovo, jako by obvinil sebe, že neumožnil návrat tomu, kterého zahnal.
#14:14 Vždyť jsme smrtelní, jsme jako po zemi rozlitá voda, která se nedá sebrat. Bůh však nechce život odejmout a velmi se rozmýšlí, aby zapuzeného od sebe zapudil nadobro.
#14:15 Nyní jsem tedy přišla, abych králi, svému pánu, přednesla tuto řeč, protože jsem se bála lidu. Tvá otrokyně si řekla: Promluvím ke králi. Snad král prosbu své služebnice splní.
#14:16 Král zajisté vyslyší a vyprostí svou služebnici ze spárů muže, který usiluje vyhladit mě i s mým synem z Božího dědictví.
#14:17 Tvá otrokyně si řekla: Kéž slovo krále, mého pána, přinese uklidnění, vždyť král, můj pán, je jako Boží posel a rozlišuje dobro a zlo. Hospodin, tvůj Bůh, buď s tebou!“
#14:18 I odpověděl král ženě: „Nezatajuj přede mnou, nač se tě zeptám.“ Žena řekla: „Mluv prosím, králi, můj pane.“
#14:19 Král se zeptal: „Nemá v tom všem ruce Jóab?“ Žena odpověděla: „Jakože je živa tvá duše, králi, můj pane, nelze uhnout napravo ani nalevo od ničeho, co král, můj pán, vyřkne. Tvůj služebník Jóab to byl, kdo mi dal příkaz a kdo tvou otrokyni naučil všemu, co má mluvit.
#14:20 Jen proto, aby se tato záležitost stočila žádoucím směrem, dopustil se tvůj služebník Jóab této věci. Můj pán je však moudrý, tak moudrý jako Boží posel, a pozná vše, co se na zemi děje.“
#14:21 I řekl král Jóabovi: „Hle, již jsem to zařídil. Jdi, přiveď mládence Abšalóma zpátky.“
#14:22 Tu padl Jóab tváří k zemi, klaněl se a dobrořečil králi. Řekl: „Dnes tvůj služebník poznal, že u tebe nalezl přízeň, králi, můj pane, když král prosbu svého služebníka splnil.“
#14:23 Jóab se hned odebral do Gešúru a přivedl Abšalóma do Jeruzaléma.
#14:24 Ale král nařídil: „Ať se uchýlí do svého domu. Moji tvář nespatří.“ Abšalóm se tedy uchýlil do svého domu a královu tvář nespatřil.
#14:25 V celém Izraeli nebyl nikdo pro svou krásu tak vyhlášený jako Abšalóm. Od hlavy k patě nebylo na něm poskvrny.
#14:26 Když si dával ostříhat hlavu - dával se stříhat koncem každého roku, poněvadž ho vlasy tížily tak, že se stříhat musel -, váha vlasů z jeho hlavy byla dvě stě šekelů královské váhy.
#14:27 Abšalómovi se narodili tři synové a jedna dcera jménem Támar. Byla to žena krásného vzhledu.
#14:28 Abšalóm bydlel v Jeruzalémě dva roky, aniž spatřil královu tvář.
#14:29 Pak poslal Abšalóm pro Jóaba, aby ho vyslal ke králi, ale on k němu přijít nechtěl. Poslal pro něho ještě podruhé, ale on zase nechtěl přijít.
#14:30 Proto řekl svým služebníkům: „Hleďte, Jóab má díl pole vedle mého a má tam ječmen. Jděte a vypalte jej.“ Abšalómovi služebníci ten díl vypálili.
#14:31 Hned nato se Jóab vypravil do domu k Abšalómovi a vyčítal mu: „Proč tvoji služebníci vypálili díl, který mi patří?“
#14:32 Abšalóm Jóabovi odpověděl: „Hle, poslal jsem ti vzkaz, abys přišel sem, že tě vyšlu ke králi s dotazem: Proč jsem přišel z Gešúru? Bylo by mi lépe, kdybych tam byl zůstal. Nyní chci spatřit královu tvář. Je-li na mně vina, ať mě vydá na smrt!“
#14:33 Jóab předstoupil před krále a oznámil mu to a král dal Abšalóma zavolat. Ten před krále předstoupil a klaněl se před králem tváří až k zemi. I políbil král Abšalóma. 
#15:1 Později si Abšalóm opatřil vůz, koně a padesát mužů, kteří před ním běhali.
#15:2 Abšalóm za časného jitra stával při cestě u brány. Každého muže, který měl spor a přicházel ke králi k soudu, si Abšalóm zavolal a vyptával se: „Z kterého jsi města?“ Když odpověděl: „Tvůj služebník je z toho a toho izraelského kmene“,
#15:3 Abšalóm mu řekl: „Pohleď, tvá záležitost je dobrá a správná, ale od krále nemůžeš čekat vyslyšení.“
#15:4 A Abšalóm dodával: „Kdybych já byl ustanoven v zemi soudcem, zjednal bych spravedlnost každému, kdo by ke mně se svým sporem a právní záležitostí přišel.“
#15:5 Když se pak někdo přiblížil, aby se mu poklonil, vztáhl ruku, uchopil ho a políbil.
#15:6 Tímto způsobem se Abšalóm choval k celému Izraeli, který přicházel ke králi k soudu. Tak Abšalóm lstivě získával srdce izraelských mužů.
#15:7 Když Abšalóm dovršil čtyřicet let, řekl králi: „Dovol mi jít do Chebrónu splnit slib, který jsem učinil Hospodinu.
#15:8 Tvůj služebník totiž, když bydlel v Gešúru v Aramu, učinil takovýto slib: Dá-li Hospodin, abych se vrátil do Jeruzaléma, budu sloužit Hospodinu.“
#15:9 Král mu řekl: „Jdi v pokoji!“ A on hned odešel do Chebrónu.
#15:10 Abšalóm poslal do všech izraelských kmenů zvědy s rozkazem: „Jakmile uslyšíte polnici, provolejte: Abšalóm se stal v Chebrónu králem!“
#15:11 S Abšalómem šlo z Jeruzaléma dvě stě mužů. Byli pozváni, šli bezelstně, nic netušili.
#15:12 Abšalóm též poslal pro Achítofela Gíloského, rádce Davidova, do jeho města Gíla, zatímco připravoval obětní hody. Spiknutí nabývalo síly a s Abšalómem šlo stále více lidu.
#15:13 K Davidovi přišel muž se zprávou: „Izraelští muži stojí za Abšalómem.“
#15:14 Tu řekl David všem svým služebníkům, kteří s ním byli v Jeruzalémě: „Ihned prchněme, jinak nebudeme moci před Abšalómem uniknout. Pospěšte s odchodem, aby si nepospíšil on a nedostihl nás, aby na nás neuvedl to nejhorší a město nevybil ostřím meče.“
#15:15 Královi služebníci králi odvětili: „Ať král, náš pán, rozhodne jakkoli, my jsme jeho otroky.“
#15:16 Král tedy vyšel a celý jeho dům ho následoval. Král zanechal jen deset ženin, aby střežily dům.
#15:17 Král vyšel a všechen lid ho následoval; zastavili se u posledního domu.
#15:18 Všichni jeho služebníci přecházeli kolem něho, též všichni Keretejci a všichni Peletejci a všichni Gaťané, šest set mužů, kteří za ním přišli pěšky z Gatu, přecházeli před králem.
#15:19 Král oslovil Itaje Gatského: „Proč jdeš s námi i ty? Vrať se a zůstaň s králem. Jsi přece cizinec, vyhoštěný ze svého domova.
#15:20 Včera jsi přišel a dnes bych měl chtít, abys jako štvanec šel s námi? Já jdu, protože jít musím, ty však se vrať se svými bratry zpátky. Milosrdenství a věrnost buď s tebou.“
#15:21 Avšak Itaj králi odpověděl: „Jakože živ je Hospodin a jakože živ je král, můj pán, kdekoli bude král, můj pán, ať mrtev nebo živ, tam bude i jeho otrok.“
#15:22 David tedy Itajovi odvětil: „Pojď, přejdi potok.“ I přešel Itaj Gatský a všichni jeho muži a všechny děti, které byly s ním.
#15:23 Celá země hlasitě plakala, když všechen ten lid přecházel. I král přešel Kidrónský úval a všechen ten lid se ubíral cestou do pouště.
#15:24 A hle, i Sádok a všichni lévijci s ním nesli schránu Boží smlouvy. Postavili Boží schránu a Ebjátar obětoval, až úplně všechen lid z města přešel.
#15:25 Potom král Sádokovi nařídil: „Dones Boží schránu zpět do města. Naleznu-li u Hospodina milost a on mě přivede zpět, dá mi spatřit ji i svůj příbytek.
#15:26 Řekne-li: ‚Nelíbíš se mi‘, ať se mnou naloží, jak uzná za dobré.“
#15:27 Dále král knězi Sádokovi řekl: „Ty jsi přece vidoucí. Vrať se pokojně do města i se svým synem Achímaasem a s Jónatanem, synem Ebjátarovým; oba vaši synové ať jdou s vámi.
#15:28 Hleďte, já se budu zdržovat u brodu v pustině, dokud mi od vás nepřijde nějaká zpráva.“
#15:29 Sádok a Ebjátar donesli tedy Boží schránu zpět do Jeruzaléma a zůstali tam.
#15:30 David pak stoupal po svahu Olivové hory, stoupal a plakal, hlavu měl zahalenou a šel bos. I všechen lid, který byl s ním, stoupal se zahalenou hlavou, stoupal a plakal.
#15:31 David dostal zprávu: „Achítofel je mezi spiklenci s Abšalómem.“ David pravil: „Hospodine, zhať prosím to, k čemu bude Achítofel radit!“
#15:32 Když přišel David na vrchol, aby se tam klaněl Bohu, hle, vyšel mu vstříc Chúšaj Arkijský s roztrženým rouchem a hlínou na hlavě.
#15:33 David mu řekl: „Půjdeš-li dál, budeš mi břemenem.
#15:34 Vrátíš-li se do města a řekneš Abšalómovi: ‚Králi, jsem tvým služebníkem; dříve jsem byl služebníkem tvého otce, nyní budu služebníkem tvým‘, překazíš to, k čemu bude Achítofel radit.
#15:35 Vždyť tam s tebou budou kněží Sádok a Ebjátar. O každém slovu z královského domu, které uslyšíš, podáš zprávu kněžím Sádokovi a Ebjátarovi.
#15:36 A jsou tam s nimi jejich dva synové, Achímaas Sádokův a Jónatan Ebjátarův. Po nich mi vzkážete každé slovo, které uslyšíte.“
#15:37 Davidův přítel Chúšaj vstoupil tedy do města. Také Abšalóm vstoupil do Jeruzaléma. 
#16:1 Sotvaže David přešel přes vrchol, jde mu vstříc Mefíbóšetův sluha Síba se spřežením osedlaných oslů a na nich dvě stě chlebů, sto hrud sušených hroznů, sto košíků letního ovoce a měch vína.
#16:2 Král se Síby zeptal: „K čemu máš tyhle věci?“ Síba odvětil: „Osly pro královský dům k jízdě, chléb s ovocem k jídlu pro družinu a víno, aby se unavený mohl na poušti napít.“
#16:3 Král dodal: „A kde je syn tvého pána?“ Síba králi řekl: „Ten vyčkává v Jeruzalémě. Řekl totiž: ‚Dnes mi izraelský dům vrátí království mého otce.‘“
#16:4 Král Síbovi pravil: „Hle, všechno, co patří Mefíbóšetovi, je tvé!“ Síba odvětil: „Skláním se. Kéž najdu u tebe přízeň, králi, můj pane.“
#16:5 Když král David přicházel k Bachurímu, právě odtamtud vycházel muž z čeledi Saulova domu jménem Šimeí, syn Gérův. Vyšel a zlořečil
#16:6 a házel kamením po Davidovi a po všech služebnících krále Davida, ačkoli po jeho pravici i levici byl veškerý lid a všichni bohatýři.
#16:7 Při svém zlořečení volal Šimeí takto: „Táhni, táhni, vrahu, ničemníku!
#16:8 Na tebe Hospodin obrátil všechnu krev Saulova domu. Na jeho místě stal ses králem, Hospodin však dal království do rukou tvého syna Abšalóma. To máš za všechno to zlo, vždyť jsi vrah!“
#16:9 Tu řekl králi Abíšaj, syn Serújin: „Copak tento mrtvý pes smí zlořečit králi, mému pánu? Nejraději bych šel a srazil mu hlavu.“
#16:10 Ale král odvětil: „Co je vám do mých záležitostí, synové Serújini? On zlořečí, protože mu Hospodin řekl: ‚Zlořeč Davidovi!‘ Kdo se potom smí ptát: ‚Proč to děláš?‘“
#16:11 A David Abíšajovi a všem svým služebníkům řekl: „Hle, můj syn, který vyšel z mého lůna, ukládá mi o život. Oč větší důvod má k tomu ten Benjamínec! Nechte ho, ať zlořečí, neboť mu to nařídil Hospodin.
#16:12 Snad Hospodin pohlédne na mé ponížení, snad mi Hospodin za to jeho dnešní zlořečení odplatí dobrem!“
#16:13 David se svými muži pokračoval v cestě a Šimeí se ubíral podle něho po svahu hory, šel a zlořečil a házel po něm kamení a metal prach.
#16:14 Král a všechen lid, který byl s ním, přišli unavení na místo, kde si oddechli.
#16:15 Abšalóm a všechen lid, izraelští muži, vstoupili do Jeruzaléma; byl s ním i Achítofel.
#16:16 Když Davidův přítel Chúšaj Arkijský přišel k Abšalómovi, pozdravil Abšalóma: „Ať žije král! Ať žije král!“
#16:17 Abšalóm Chúšajovi řekl: „Tak se odvděčuješ svému příteli? Proč jsi nešel se svým přítelem?“
#16:18 Chúšaj Abšalómovi odpověděl: „Nikoli! Koho vyvolil Hospodin a všechen tento lid i všichni izraelští muži, s tím budu a u toho zůstanu.
#16:19 A pak, komu bych měl sloužit? Přece jeho synu. Jako jsem sloužil tvému otci, tak budu sloužit tobě.“
#16:20 Abšalóm řekl Achítofelovi: „Poraďte, co máme dělat.“
#16:21 Achítofel Abšalómovi odvětil: „Vejdi k ženinám svého otce, které ponechal, aby střežily dům. Celý Izrael uslyší, že jsi vzbudil u svého otce nelibost, a všichni, kdo jsou s tebou, začnou jednat rozhodně.“
#16:22 Proto postavili Abšalómovi na střeše stan a Abšalóm před zraky celého Izraele vešel k ženinám svého otce.
#16:23 Radit se s Achítofelem, který byl v těch dnech rádcem, bylo jako doptávat se na božské slovo. Tak tomu bylo s každou Achítofelovou radou, dávanou jak Davidovi, tak Abšalómovi. 
#17:1 Achítofel pravil Abšalómovi: „Dovol, abych vybral dvanáct tisíc mužů a hned této noci se vydal pronásledovat Davida.
#17:2 Dostihnu ho, až bude zemdlený a skleslý, a překvapím ho. Všechen lid, který je s ním, uteče a já osamoceného krále zabiji.
#17:3 Všechen lid obrátím k tobě; vrátí se všichni, až na muže, kterého hledáš. Všechen lid bude mít pokoj.“
#17:4 Abšalómovi i všem izraelským starším se to líbilo.
#17:5 Abšalóm však poručil: „Ať je povolán také Chúšaj Arkijský. Poslechneme si i jeho výrok.“
#17:6 Když Chúšaj přišel k Abšalómovi, Abšalóm mu pravil: „Toto navrhuje Achítofel. Máme jednat podle jeho slova? Ne-li, vyslov se.“
#17:7 Chúšaj Abšalómovi odvětil: „Rada, kterou dal Achítofel, není v tomto případě dobrá.“
#17:8 A Chúšaj pokračoval: „Ty znáš svého otce i jeho muže, že to jsou bohatýři a že jsou rozhořčeni jak osiřelá medvědice v poli. Tvůj otec je bojovník, ten nebude nocovat s lidem.
#17:9 Jistě je už ukryt v nějaké strži nebo na jiném místě. Když někteří z tvých hned na začátku padnou, někdo to uslyší a roznese se: ‚Lid, který jde za Abšalómem, byl poražen!‘
#17:10 Nakonec ztratí odvahu i ten nejstatečnější muž lvího srdce. Celý Izrael přece ví, že tvůj otec je bohatýr a že ti, kteří jsou s ním, jsou muži stateční.
#17:11 Proto ti radím: Nechť se k tobě shromáždí celý Izrael od Danu až k Beer-šebě. Bude jich jak mořského písku. A ty táhni do bitvy osobně.
#17:12 Přitrhneme na jedno z těch míst, kde se on zdržuje, a padneme na něho jako rosa na roli. Nezbude z něho a ze všech mužů, kteří jsou s ním, ani jeden.
#17:13 Stáhne-li se do města, snesou všichni Izraelci k tomu městu provazy. Budeme vláčet jeho znamení až do úvalu, že se z něho nenajde ani oblázek.“
#17:14 Abšalóm a všichni izraelští muži prohlásili: „Rada Chúšaje Arkijského je lepší než rada Achítofelova.“ To Hospodin přikázal překazit dobrou radu Achítofelovu, aby na Abšalóma uvedl zkázu.
#17:15 I řekl Chúšaj kněžím Sádokovi a Ebjátarovi: „Tak a tak radil Achítofel Abšalómovi a izraelským starším, já však jsem poradil tak a tak.
#17:16 Nyní tedy rychle pošlete Davidovi zprávu: ‚Nezůstávej přes noc ve stepních pustinách. Musíš přejít Jordán. Jinak ti, králi, hrozí záhuba, i všemu lidu, který je s tebou.‘“
#17:17 Jónatan a Achímaas stáli u pramene Rogelu. Služebná šla a oznámila jim to. Oni to šli oznámit králi Davidovi; sami se totiž ve městě ukázat nemohli.
#17:18 Přesto je však uviděl mládenec a oznámil to Abšalómovi. Ti dva se rychle vydali na cestu, až přišli do domu nějakého muže v Bachurímu. Ten měl na dvoře studnu a oni se tam do ní spustili.
#17:19 Žena pak vzala přikrývku a rozprostřela ji navrch té studny a na ni rozestřela obilnou drť, takže nebylo nic znát.
#17:20 Abšalómovi služebníci přišli k té ženě do domu a ptali se: „Kde jsou Achímaas a Jónatan?“ Žena jim odvětila: „Přešli tudy přes strouhu.“ Hledali je tedy, ale nenašli a vrátili se do Jeruzaléma.
#17:21 Sotva odešli, ti dva ze studně vystoupili a šli podat zprávu králi Davidovi. Řekli Davidovi: „Vydejte se na cestu a rychle přejděte vodu, neboť Achítofel dal o vás takovouto radu.“
#17:22 David vstal se vším lidem, který byl s ním, a do ranního úsvitu přešli Jordán. Když Jordán přešli, nechyběl ani jeden.
#17:23 Když Achítofel viděl, že jeho rada provedena nebyla, osedlal osla a vydal se na cestu do svého domu ve svém městě. Udělal pořízení o svém domě a oběsil se. Umřel a byl pohřben v hrobě svého otce.
#17:24 David přišel do Machanajimu a Abšalóm přešel Jordán se všemi izraelskými muži.
#17:25 Místo Jóaba ustanovil Abšalóm nad vojskem Amasu. Amasa byl synem muže jménem Jitra Izraelský, který vešel k Abígale, dceři Náchašově, sestře Serúji, matky Jóabovy.
#17:26 Izrael a Abšalóm se utábořili v zemi Gileádu.
#17:27 Když David přišel do Machanajimu, Šóbi, syn Náchašův z Raby Amónovců, a Makír, syn Amíelův z Lódebaru, a Barzilaj Gileádský z Rogelímu
#17:28 přinesli lůžko, misky a hliněné nádobí i pšenici, ječmen a mouku, pražené zrní, boby, čočku a pražmu,
#17:29 i med a máslo, ovce a kravský sýr Davidovi a lidu, který byl s ním, aby pojedli. Řekli si totiž: „Lid je v té poušti hladový, unavený a žíznivý.“ 
#18:1 David dal nastoupit lidu, který byl s ním, a ustanovil nad nimi velitele nad tisíci a nad sty.
#18:2 Potom David podřídil třetinu lidu Jóabovi, třetinu Jóabovu bratru Abíšajovi, synu Serújinu, a třetinu Itaji Gatskému. Lidu král řekl: „Já s vámi potáhnu rovněž.“
#18:3 Lid však prohlásil: „Netáhni. Na nás jim nebude záležet, budeme-li utíkat. Usmrtí-li nás i polovic, nebude jim na nás záležet. Ty však jsi za deset tisíc z nás. Nyní nám lépe pomůžeš, když budeš ve městě.“
#18:4 Král jim odvětil: „Učiním, co pokládáte za dobré.“ Král se postavil k bráně a všechen lid vycházel po stech a tisících.
#18:5 Král přikázal Jóabovi, Abíšajovi a Itajovi: „S ohledem na mne jednejte s mládencem Abšalómem šetrně.“ Všechen lid slyšel, co král všem velitelům ohledně Abšalóma přikázal.
#18:6 I vytáhl lid do pole proti Izraeli. K bitvě došlo v Efrajimském lese.
#18:7 Izraelský lid tam byl Davidovými služebníky poražen. V ten den došlo k velké porážce, bylo dvacet tisíc padlých.
#18:8 Bitva se rozšířila po celé té zemi a les toho dne pohltil více lidu než meč.
#18:9 Abšalóm narazil na Davidovy služebníky, a jak ujížděl na mezku, vběhl mezek pod větvoví velikého posvátného stromu a on v tom stromě uvázl hlavou, takže se ocitl mezi nebem a zemí; mezek, na němž jel, běžel dál.
#18:10 Uviděl to nějaký muž a oznámil to Jóabovi: „Hle, viděl jsem Abšalóma visícího na posvátném stromě.“
#18:11 Jóab řekl muži, který mu to oznámil: „Když jsi ho viděl, proč jsi ho tam nesrazil k zemi? Byl bych ti dal deset šekelů stříbra a jeden pás.“
#18:12 Ale ten muž Jóabovi odvětil: „I kdybych dostal na ruku tisíc šekelů stříbra, nevztáhl bych ruku na královského syna. Na vlastní uši jsme přece slyšeli, co král přikázal tobě, Abíšajovi a Itajovi: ‚Jen mi dejte pozor na mládence Abšalóma!‘
#18:13 Mám být sám proti sobě? Před králem se přece pranic neukryje, ba ty sám by ses proti mně postavil.“
#18:14 Jóab řekl: „Nebudu se tu s tebou zdržovat.“ Popadl tři oštěpy a vrazil je Abšalómovi do srdce; ten byl ještě v objetí toho posvátného stromu živý.
#18:15 Deset Jóabových zbrojnošů Abšalóma obstoupilo a ubili ho k smrti.
#18:16 Pak Jóab zatroubil na polnici a lid přestal Izraele pronásledovat; Jóab v tom lidu zabránil.
#18:17 Abšalóma vzali, hodili ho v lese do veliké strže a navršili na něj převelikou hromadu kamenů. Celý Izrael se rozutekl, každý ke svému stanu.
#18:18 Abšalóm si ještě za svého života postavil v Královské dolině posvátný sloup. Řekl si totiž: „Nemám syna, jenž by zachoval památku mého jména“, a nazval ten posvátný sloup svým jménem. Dodnes se nazývá Ruka Abšalómova.
#18:19 Achímaas, syn Sádokův, řekl Jóabovi: „Dovol mi běžet a zvěstovat králi, že Hospodin mu zjednal právo a vysvobodil ho z rukou jeho nepřátel.“
#18:20 Jóab mu však odvětil: „Dnes bys nebyl nositelem dobré zvěsti. Ohlásíš dobrou zvěst v jiný den. Dnes tu zvěst neohlašuj, neboť králi zemřel syn.“
#18:21 Jóab pak poručil jistému Kúšijci: „Jdi a oznam králi, co jsi viděl.“ Kúšijec se Jóabovi poklonil a dal se v běh.
#18:22 Achímaas, syn Sádokův, řekl Jóabovi ještě jednou: „Buď jak buď, i já poběžím za Kúšijcem.“ Jóab se zeptal: „Proč bys běžel ty, můj synu, když nemáš co dobrého zvěstovat?“
#18:23 „Buď jak buď, poběžím.“ Jóab mu řekl: „Běž.“ Achímaas běžel směrem k jordánskému okrsku a Kúšijce předběhl.
#18:24 David právě seděl mezi dvěma branami. Tu vyšla hlídka na střechu brány u hradby, rozhlédla se a spatřila, že nějaký muž běží sám.
#18:25 Hlídka zvolala a hlásila to králi. Král řekl: „Je-li sám, nese dobrou zvěst.“ Zatímco se přibližoval víc a více,
#18:26 spatřila hlídka druhého běžce. Hlídka zvolala na strážného: „Hle, nějaký muž, a běží sám.“ Král řekl: „Také ten ohlásí dobrou zvěst.“
#18:27 Hlídka hlásila: „Připadá mi, že ten první, co běží, je jako Achímaas, syn Sádokův.“ Král podotkl: „To je dobrý muž a přichází s dobrou zvěstí.“
#18:28 Achímaas hlasitě pozdravil krále: „Pokoj!“ Poklonil se králi tváří k zemi a pokračoval: „Požehnán buď Hospodin, tvůj Bůh, jenž ti vydal muže, kteří pozdvihli ruce proti králi, mému pánu!“
#18:29 Král se ptal: „Je s mládencem Abšalómem vše v pořádku?“ Achímaas odvětil: „Když Jóab posílal králova služebníka, totiž služebníka tvého, viděl jsem veliký shluk, ale nevím nic.“
#18:30 Král řekl: „Odstup a postav se stranou.“ Odstoupil tedy a stál tam.
#18:31 A hle, vešel Kúšijec se slovy: „Krále, mého pána, čeká dobrá zvěst; Hospodin ti dnes zjednal právo a vysvobodil tě z rukou všech, kteří proti tobě povstali.“
#18:32 Ale král se Kúšijce zeptal: „Je s mládencem Abšalómem vše v pořádku?“ Kúšijec odpověděl: „Nechť se stane nepřátelům krále, mého pána, jako tomu mládenci, i všem, kteří proti tobě zlovolně povstávají!“ 
#19:1 Král otřesen vystoupil do přístřešku nad branou a plakal. Šel a naříkal: „Můj synu Abšalóme, můj synu Abšalóme! Kéž bych byl umřel místo tebe, Abšalóme, synu můj, synu můj!“
#19:2 Jóabovi bylo oznámeno: „Hle, král pro Abšalóma pláče a truchlí.“
#19:3 Tak se toho dne proměnilo vysvobození ve smutek všeho lidu, neboť lid slyšel toho dne slova: „Král se trápí pro svého syna.“
#19:4 A lid se toho dne kradl do města, jako se vkrádá lid, který se stydí, že utekl z bitvy.
#19:5 Král si zavinul tvář a velmi hlasitě křičel: „Můj synu Abšalóme, Abšalóme, synu můj, synu můj!“
#19:6 Tu vstoupil ke králi do domu Jóab a řekl: „Dnes jsi způsobil hanbu všem svým služebníkům, kteří dnes zachránili život tvůj i životy tvých synů a tvých dcer i životy tvých žen a životy tvých ženin.
#19:7 Miluješ ty, kdo tě nenávidí, a nenávidíš ty, kdo tě milují. Vždyť jsi dnes dal najevo, že ti nejsou ničím ani velitelé ani služebníci. Dnes jsem poznal, že kdyby zůstal naživu Abšalóm a my všichni byli dnes mrtvi, měl bys to za správné.
#19:8 Vstaň nyní, vyjdi a vlídně promluv ke svým služebníkům! Přísahám při Hospodinu: Nevyjdeš-li, nebude této noci s tebou nocovat ani jeden muž. A bude to pro tebe horší než všechno zlo, které na tebe dolehlo od tvého mládí až do nynějška.“
#19:9 Král tedy vstal a usedl v bráně. Všemu lidu bylo oznámeno: „Hle, král sedí v bráně.“ I vyšel všechen lid před krále. Izrael se pak rozutekl, každý ke svým stanům.
#19:10 Všechen lid ve všech izraelských kmenech se dostal do rozepře: „Král nás vysvobodil ze spárů našich nepřátel, zachránil nás ze spárů Pelištejců a nyní uprchl ze země před Abšalómem.
#19:11 Avšak Abšalóm, jehož jsme nad sebou pomazali za krále, zahynul v bitvě. Proč nyní ohledně králova návratu mlčíte?“
#19:12 Král David vzkázal kněžím Sádokovi a Ebjátarovi: „Promluvte k judským starším: Proč chcete být při návratu krále do jeho domu poslední?“ Řeč celého Izraele se totiž donesla ke králi, do jeho domu.
#19:13 „Jste přece moji bratři, má krev a mé tělo. Proč chcete být při králově návratu poslední?
#19:14 Také Amasovi povězte: Což nejsi ty má krev a mé tělo? Ať se mnou Bůh udělá, co chce, nebudeš-li u mne po všechny dny velitelem vojska místo Jóaba.“
#19:15 Tak si naklonil srdce každého judského muže, že byli jako jeden muž. Poslali králi vzkaz: „Vrať se zpět se všemi svými služebníky!“
#19:16 Král se tedy vracel a přišel až k Jordánu. Juda vstoupil do Gilgálu a šel králi vstříc, aby ho převedl přes Jordán.
#19:17 Tu si pospíšil Benjamínec Šimeí, syn Gérův z Bachurímu, a sestoupil s judskými muži králi Davidovi vstříc.
#19:18 Bylo s ním tisíc mužů z Benjamína, též služebník Saulova domu Síba s patnácti svými syny a dvaceti otroky. Dorazili k Jordánu před králem
#19:19 a přešli brod, aby převedli králův dům a aby se králi zalíbili. Šimeí, syn Gérův, padl před králem, jenž se chystal přejít Jordán,
#19:20 a řekl králi: „Nechť mi to můj pán nepokládá za vinu a nechť nevzpomíná, čím se jeho otrok provinil v ten den, kdy král, můj pán, vycházel z Jeruzaléma; nechť to král nechová v srdci.
#19:21 Vždyť tvůj otrok ví, že zhřešil. Dnes však jsem přišel z celého Josefova domu první, abych sestoupil králi, svému pánu, vstříc.“
#19:22 Tu se ozval Abíšaj, syn Serújin: „Což nebude Šimeí usmrcen za to, že zlořečil Hospodinovu pomazanému?“
#19:23 David však prohlásil: „Co je vám do mých záležitostí, synové Serújini! Zase se mi dnes stavíte na odpor. Dnes by měl být v Izraeli někdo usmrcen? Což nevím, že jsem dnes nad Izraelem opět králem?“
#19:24 Šimeímu král potom řekl: „Nezemřeš.“ A král mu přísahal.
#19:25 Také Mefíbóšet, vnuk Saulův, sešel králi vstříc. Ode dne, co král odešel, až do dne, co se v pokoji vrátil, nepečoval o své nohy ani o svůj vous a nepral si šaty.
#19:26 Když přišel králi vstříc do Jeruzaléma, řekl mu král: „Proč jsi nešel se mnou, Mefíbóšete?“
#19:27 On odpověděl: „Králi, můj pane, můj služebník mě oklamal. Tvůj otrok si totiž řekl: Osedlám si osla, vsednu na něj a pojedu s králem. Tvůj otrok přece kulhá.
#19:28 Ale on tvého otroka u tebe, králi, můj pane, pomluvil. Avšak král, můj pán, je jako Boží posel. Učiň tedy, co uznáš za dobré.
#19:29 Celý dům mého otce je před králem, mým pánem, hoden smrti. Ty však jsi svého otroka posadil mezi ty, kdo jedí z tvého stolu. Jaké spravedlnosti bych se měl u krále ještě dovolávat?“
#19:30 Král mu odpověděl: „Proč ještě vedeš takové řeči? Rozhodl jsem: Ty a Síba si pole rozdělíte.“
#19:31 Mefíbóšet králi odvětil: „Ať si vezme třeba všechno, jen když se král, můj pán, v pokoji vrátil do svého domu.“
#19:32 Též Barzilaj Gileádský sestoupil z Rogelímu, aby s králem přešel Jordán; vydal se přes Jordán s ním.
#19:33 Barzilaj byl velmi starý, bylo mu osmdesát let. Ten krále v jeho vyhnanství opatřoval v Machanajimu potravou, neboť to byl muž velmi zámožný.
#19:34 Král Barzilaje vyzval: „Táhni se mnou a já tě u sebe v Jeruzalémě zaopatřím.“
#19:35 Barzilaj však králi odpověděl: „Kolik let života mi zbývá, že bych měl s králem vystoupit do Jeruzaléma?
#19:36 Je mi teď osmdesát let. Což mohu ještě odlišovat dobré od špatného? Zdalipak si tvůj služebník může ještě pochutnat na jídle a pití? Což ještě mohu naslouchat zpěvákům a zpěvačkám? Proč by tvůj služebník měl být ještě králi, svému pánu, břemenem?
#19:37 Tvůj služebník potáhne s králem jen kousek za Jordán. Proč by mě měl král tolik odměňovat?
#19:38 Nechť se prosím tvůj služebník smí vrátit. Rád bych umřel ve svém městě při hrobu svého otce a své matky. Hle, s králem, mým pánem, potáhne tvůj služebník Kimhám. Pro něho učiň, co uznáš za dobré.“
#19:39 Král řekl: „Ať tedy Kimhám táhne se mnou. Učiním pro něho, co uznáš za dobré. Splním ti vše, co si ode mne budeš přát.“
#19:40 Tak přešel všechen lid Jordán, přešel jej i král. Král Barzilaje políbil a požehnal mu a on se vrátil domů.
#19:41 Král táhl do Gilgálu a Kimhám táhl s ním. Krále doprovázel i všechen judský lid a též polovina lidu izraelského.
#19:42 A hle, všichni izraelští muži přišli ke králi a tázali se krále: „Proč tě naši bratři, muži judští, ukradli? Proč převedli krále i jeho dům přes Jordán a všechny Davidovy muže s ním?“
#19:43 Všichni judští muži mužům izraelským odpověděli: „Král je přece náš příbuzný. Proč se tedy kvůli tomu tolik hněváte? Cožpak jsme z krále kus ujedli? Nebo jsme ho snad unesli?“
#19:44 Izraelští muži odpověděli mužům judským: „Na krále, a to i na Davida, máme desetkrát větší nárok než vy. Proč jste nás tedy znevážili? Což to nebyla naše první starost, aby se náš král vrátil?“ Ale řeč mužů judských byla tvrdší než řeč mužů izraelských. 
#20:1 Tehdy se vyskytl ničemník jménem Šeba, syn Bichrího, benjamínského muže. Ten zatroubil na polnici a zvolal: „My nemáme podíl v Davidovi, nemáme dědictví v synu Jišajovu. Každý ke svým stanům, Izraeli!“
#20:2 Všichni izraelští muži přešli od Davida k Šebovi, synu Bichrího, ale judští muži, od Jordánu až k Jeruzalému, se přimkli ke svému králi.
#20:3 David vešel do svého domu v Jeruzalémě. Král vzal deset žen, byly to ženiny zanechané, aby střežily dům, a dal je do střeženého domu; zaopatřil je, ale nevcházel k nim. Tak byly uzavřeny až do dne své smrti a žily jako vdovy.
#20:4 Král poručil Amasovi: „Do tří dnů mi svolej judské muže. I ty se sem dostav.“
#20:5 Amasa šel Judu svolat, ale promeškal určenou lhůtu.
#20:6 I řekl David Abíšajovi: „Šeba, syn Bichrího, nám teď bude škodit více než Abšalóm. Ihned vezmi služebníky svého pána a pronásleduj ho, aby si nenašel opevněná města a neunikl našim očím.“
#20:7 Za ním vytáhli Jóabovi muži, Keretejci a Peletejci i všichni bohatýři. Vytáhli z Jeruzaléma, aby pronásledovali Šebu, syna Bichrího.
#20:8 Byli právě u velikého kamene, který je v Gibeónu, když k nim přišel Amasa. Jóab byl přepásán přes své odění a šat; měl pás s mečem v pochvě připnutý na bedrech; když vycházel, meč se povysunul.
#20:9 Jóab Amasovi řekl: „Pokoj tobě, můj bratře!“ A pravou rukou vzal Jóab Amasu za bradu, aby ho políbil.
#20:10 Amasa však přitom nedal pozor na meč v Jóabově ruce. Ten ho bodl pod žebra, takže mu vnitřnosti vyhřezly na zem; aniž mu dal druhou ránu, zemřel. Pak Jóab a jeho bratr Abíšaj pronásledovali Šebu, syna Bichrího.
#20:11 Jeden z Jóabovy družiny stál nad Amasou a volal: „Kdo chce jít s Jóabem a kdo je Davidův, za Jóabem!“
#20:12 Amasa se válel v krvi uprostřed silnice. Když ten muž viděl, že se všechen lid zastavuje, odvlekl Amasu ze silnice do pole a hodil přes něj pokrývku; viděl totiž, že každý, kdo k němu přijde, zůstává stát.
#20:13 Jakmile ho odklidil ze silnice, všichni muži táhli za Jóabem, aby pronásledovali Šebu, syna Bichrího. -
#20:14 Ten prošel všemi izraelskými kmeny až do Abelu a Bét-maaky; také všichni Bérejci se shromáždili a šli za ním. -
#20:15 Jóabovi muži přišli a oblehli ho v Abel-bét-maace. Zřídili proti městu násep a ten dosahoval k valu; všechen lid, který byl s Jóabem, se snažil strhnout hradby.
#20:16 Tu jedna moudrá žena zavolala z města: „Slyšte, slyšte! Povězte prosím Jóabovi: Přistup sem a já k tobě promluvím.“
#20:17 Když se k ní přiblížil, žena se ho zeptala: „Ty jsi Jóab?“ Odpověděl: „Jsem.“ Řekla mu: „Vyslechni slova své otrokyně.“ Odvětil: „Poslouchám.“
#20:18 Pokračovala: „Kdysi se říkávalo: Je nutno doptat se v Abelu a přesně tak to provést.
#20:19 Jsme pokojné, věrné izraelské město. A ty usiluješ umořit město, matku v Izraeli. Proč chceš pohltit Hospodinovo dědictví?“
#20:20 Jóab odpověděl: „Toho jsem dalek, toho jsem dalek! Nechci je pohltit ani zničit.
#20:21 Věc se má jinak. Muž z Efrajimského pohoří, jenž se jmenuje Šeba, syn Bichrího, pozdvihl ruku proti králi, proti Davidovi. Vydejte ho, jen jeho samotného, a já od města odtáhnu.“ Žena Jóabovi řekla: „Nuže, bude ti hozena přes hradby jeho hlava.“
#20:22 Pak šla ta žena se svou moudrou radou ke všemu lidu. Šebovi, synu Bichrího, uťali hlavu a hodili ji Jóabovi. Ten zatroubil na polnici a rozešli se od města, každý ke svým stanům. Jóab se vrátil do Jeruzaléma ke králi.
#20:23 Jóab byl nad vším izraelským vojskem. Benajáš, syn Jójadův, byl nad Keretejci a Peletejci.
#20:24 Adorám byl nad nucenými pracemi, Jóšafat, syn Achílúdův, byl kancléřem,
#20:25 Šeja byl písařem, Sádok a Ebjátar kněžími.
#20:26 Též Íra Jaírský byl Davidovým knězem. 
#21:1 Za dnů Davidových byl hlad po tři roky za sebou. David tedy vyhledal Hospodinovu tvář. Hospodin řekl: „Na Saulovi a jeho domu lpí krev, protože dal povraždit Gibeóňany.“
#21:2 Král předvolal Gibeóňany a promluvil k nim. Gibeóňané nejsou z Izraelců, nýbrž ze zbytku Emorejců, jimž se Izraelci zavázali přísahou, ale Saul je usiloval vybít ve svém zanícení pro Izraelce a Judu.
#21:3 David se Gibeóňanů tázal: „Co mám pro vás udělat? Jak dosáhnu smíření, abyste dobrořečili Hospodinovu dědictví?“
#21:4 Gibeóňané mu odvětili: „Co se týče Saula a jeho domu, nejde nám o stříbro a zlato, ani nám nejde o to, aby byl usmrcen kdokoli z Izraele.“ Pravil: „Co řeknete, to pro vás udělám.“
#21:5 Odpověděli králi: „Za toho muže, který nás zamýšlel zničit a vyhladit, aby po nás v celém izraelském území nic nezůstalo,
#21:6 nechť je nám vydáno sedm mužů z jeho synů a my jim zpřerážíme údy v Gibeji Saula, někdejšího Hospodinova vyvoleného, kvůli Hospodinu.“ Král prohlásil: „Vydám je.“
#21:7 Král však ušetřil Mefíbóšeta, syna Jónatana, syna Saulova, pro přísahu ve jménu Hospodinově, která byla mezi nimi, totiž mezi Davidem a Jónatanem, synem Saulovým.
#21:8 Král vzal dva syny Rispy, dcery Ajovy, jež porodila Saulovi, Armóna a Mefíbóšeta, a pět synů Saulovy dcery, Míkaliny sestry, které porodila Adríelovi, synu Barzilaje Mechólatského,
#21:9 a vydal je Gibeóňanům do rukou. Ti jim zpřeráželi údy na hoře před Hospodinem. Tak jich najednou padlo sedm. Byli usmrcení v prvních dnech žně, když se začal žít ječmen.
#21:10 Rispa, dcera Ajova, vzala žíněné roucho a rozprostřela si je na skále; zůstala při nich od začátku žně, dokud je neskropila voda z nebes; nenechala na ně sedat nebeské ptactvo ve dne ani polní zvěř v noci.
#21:11 Davidovi bylo oznámeno, co učinila Rispa, dcera Ajova, ženina Saulova.
#21:12 David šel a vzal kosti Saulovy i kosti jeho syna Jónatana od občanů Jábeše v Gileádu, kteří je potají odnesli z bétšanského prostranství, kde je pověsili Pelištejci v den, kdy porazili Saula v pohoří Gilbóa.
#21:13 Přinesl odtamtud kosti Saulovy a kosti jeho syna Jónatana. Pak posbírali kosti těch, jimž byly zpřeráženy údy,
#21:14 a pochovali je s kostmi Saula a jeho syna Jónatana v zemi Benjamínově v Sele, v hrobě jeho otce Kíše. Vykonali všechno, co král přikázal. Bůh potom přijal prosby za zemi.
#21:15 A došlo znovu k boji Pelištejců s Izraelem. David sestoupil se svými služebníky a bojovali s Pelištejci. Ale David byl už unaven.
#21:16 Tu Jišbí Benób, který byl z rodu obrů, jehož dřevce vážilo tři sta šekelů bronzu a jenž byl opásán novým mečem, prohlásil, že Davida zabije.
#21:17 Avšak Abíšaj, syn Serújin, mu přišel na pomoc a toho Pelištejce ubil k smrti. Davidovi mužové tehdy Davida zapřísáhli: „Víckrát se s námi do boje nevydávej, ať nezhasíš světlo Izraele!“
#21:18 O něco později se opět strhl boj s Pelištejci v Góbu. Tehdy ubil Sibekaj Chúšatský Sáfa, který byl z rodu obrů.
#21:19 Když se znovu strhl boj s Pelištejci v Góbu, Elchánan, syn Jaare Oregíma Betlémského, ubil Goliáše Gatského. Násada jeho kopí byla jako tkalcovské vratidlo.
#21:20 A opět se strhl boj v Gatu. Tam byl obrovitý muž, který měl šest prstů na rukou a šest prstů u nohou, celkem čtyřiadvacet. Ten také pocházel z obrů.
#21:21 Tupil Izraele, a Jónatan, syn Šimeáje, bratra Davidova, ho ubil.
#21:22 Tito čtyři pocházeli z obrů v Gatu. Padli rukou Davida a rukou jeho služebníků. 
#22:1 Slova této písně přednášel David Hospodinu v den, kdy jej Hospodin vysvobodil ze spárů všech jeho nepřátel i ze spárů Saulových.
#22:2 Pravil: „Hospodine, skalní štíte můj, má pevná tvrzi, můj vysvoboditeli,
#22:3 Bože můj, má skálo, utíkám se k tobě, štíte můj a rohu spásy, nedobytný hrade, moje útočiště, zachránce můj, ty mě před násilím zachraňuješ!
#22:4 Když jsem vzýval Hospodina, jemuž patří chvála, byl jsem zachráněn před svými nepřáteli.
#22:5 Ovinuly mě příboje smrti, zachvátily mě dravé proudy Ničemníka,
#22:6 provazy podsvětí se kolem mne stáhly, dostihly mě léčky smrti.
#22:7 V soužení jsem vzýval Hospodina, ke svému Bohu jsem volal. Uslyšel můj hlas ze svého chrámu, mé volání proniklo až k jeho sluchu.
#22:8 Zachvěla se země, roztřásla se, nebesa v základech se hnula, chvěla se před jeho plamenným hněvem.
#22:9 Z chřípí se mu valil dým, z úst sžírající oheň, planoucí řeřavé uhlí.
#22:10 Sklonil nebesa a sestupoval, pod nohama černé mračno.
#22:11 Na cheruba usedl a letěl, ukázal se na perutích větru.
#22:12 Temno učinil stánkem kolem sebe, vířící vodstvo, mračna prachu.
#22:13 Před jeho jasem vzplálo hořící uhlí.
#22:14 Hospodin zaburácel z nebe, Nejvyšší vydal svůj hlas.
#22:15 Vyslal šípy a rozehnal je, blesky je uvedl v zmatek.
#22:16 Tu se objevila koryta moře, základy světa se obnažily, když Hospodin zaútočil, když zadul svým hněvivým dechem.
#22:17 Vztáhl ruku z výše, uchopil mě, vytáhl mě z nesmírného vodstva.
#22:18 Nepříteli mocnému mě vyrval, těm, kdo nenáviděli mě, kdo zdatnější byli.
#22:19 Přepadli mě v den mých běd, ale Hospodin mě podepíral.
#22:20 Učinil mě volným, ubránil mě, protože si mě oblíbil.
#22:21 Hospodin mi odplatil podle mé spravedlnosti, odměnil mě podle čistoty mých rukou,
#22:22 neboť jsem dbal na Hospodinovy cesty, neodvrátil jsem se svévolně od svého Boha.
#22:23 Všechny jeho řády jsem měl na zřeteli, neodbočil jsem od jeho nařízení.
#22:24 Jemu jsem náležel dokonale, varoval se nepravosti.
#22:25 Podle mé spravedlnosti mě Hospodin odměňoval, podle čistoty mé tak, jak jevila se jemu.
#22:26 Ty věrnému osvědčuješ věrnost, muži dokonalému svou dokonalost,
#22:27 ryzímu svou ryzost osvědčuješ, s neupřímným se však pouštíš do zápasu.
#22:28 A lid ponížený zachraňuješ, ale svým pohledem ponižuješ povýšené.
#22:29 Ty jsi moje světlo, Hospodine. Hospodin mi září do mých temnot.
#22:30 S tebou proběhnu i nepřátelskou vřavou, se svým Bohem zdolám hradbu,
#22:31 s Bohem, jehož cesta je tak dokonalá! To, co řekne Hospodin, je protříbené. On je štítem všech, kteří se k němu utíkají.
#22:32 Kdo je Bůh krom Hospodina, kdo je skála, ne-li Bůh náš!
#22:33 Bůh je má záštita pevná a vodí mě cestou dokonalou,
#22:34 on dává mým nohám hbitost laně, na mých posvátných návrších mi dopřává stanout,
#22:35 učí bojovat mé ruce a mé paže napnout bronzový luk.
#22:36 Podal jsi mi štít své spásy, tvá péče mé síly rozmnožila.
#22:37 Dals volnost mým krokům, nohy se mi nepodvrtnou.
#22:38 Budu stíhat nepřátele, vyhladím je. Nevrátím se zpět, dokud je neudolám.
#22:39 Docela je rozdrtím, už nepovstanou, pod nohy mi padnou.
#22:40 Opásals mě statečností k boji; ty, kdo povstávají proti mně, sám srazíš.
#22:41 Obrátil jsi na útěk mé nepřátele, navždy umlčím ty, kdo mě nenávidí.
#22:42 Budou volat o pomoc, a nespasí je nikdo, volat k Hospodinu, ale neodpoví.
#22:43 Roztluču je, budou jako prach země, pošlapu a podupu je jako bláto ulic.
#22:44 Dals mi vyváznout z rozbrojů mého lidu, jako vůdce pronárodů jsi mě střežil. Sloužit bude mi i lid, který jsem neznal.
#22:45 Cizinci se budou vtírat do mé přízně, na slovo mě uposlechnou.
#22:46 Cizinci jak tráva zvadnou, vypotácejí se ze svých hradišť.
#22:47 Živ je Hospodin! Buď požehnána moje skála, vyvýšen buď Bůh, má spásná skála!
#22:48 Bůh, jenž mě pověřil vykonáním pomsty, národy mi podřizuje.
#22:49 Ty mě z rukou mých nepřátel vytrhuješ, pozvedáš mě nad ty, kdo proti mně povstávají, ty mě násilníku vyrveš.
#22:50 Proto ti vzdám, Hospodine, mezi pronárody chválu, budu zpívat žalmy tvému jménu.
#22:51 Velká vítězství dopřává svému králi, prokazuje milosrdenství svému pomazanému, Davidovi a jeho potomstvu, navěky.“ 
#23:1 Toto jsou Davidova poslední slova: „Výrok Davida, syna Jišajova, výrok muže Nejvyšším povýšeného, pomazaného Bohem Jákobovým, líbezného pěvce izraelských žalmů.
#23:2 Hospodinův duch skrze mne mluvil, na mém jazyku byla řeč jeho.
#23:3 Pravil mi Bůh Izraele, skála Izraele ke mně promluvila: Kdo člověku vládne spravedlivě, ten, kdo vládne v Boží bázni,
#23:4 je jak jitřní světlo při východu slunce, jak bez mráčku jitro, od jehož jasu a vláhy se zelená země.
#23:5 I když můj dům před Bohem takový není, ustanovil pro mne věčnou smlouvu ve všem uspořádanou a dodrženou. Je to veškerá má spása a veškeré blaho, odjinud nevzejde nic.
#23:6 Ničemník bude jak trní zavržené všemi, nikdo na ně rukou nesáhne.
#23:7 Ten, kdo by se ho chtěl dotknout, vezme železo, násadu kopí. Budou sežehnuti a spáleni ohněm ve svém sídle.“
#23:8 Toto jsou jména Davidových bohatýrů: Jóšeb Bašebet Tachkemónský, vůdce osádky. Ten s rozkoší vrhal oštěp; naráz jich proklál osm set.
#23:9 Po něm Eleazar, syn Dóda, syna Achóchiova, jeden ze tří bohatýrů, kteří byli s Davidem, když dráždili Pelištejce a ti se tam shromáždili k boji; izraelští muži též přitáhli.
#23:10 Povstal a bil Pelištejce, až se mu ruka unavila a křečovitě svírala meč. Toho dne připravil Hospodin veliké vysvobození. Lid se obrátil za ním a už jen obíral pobité.
#23:11 Po něm byl Šama, syn Ageův, Hararský. Pelištejci se shromáždili do houfu; byl tam díl pole plný čočky. Lid před Pelištejci utíkal.
#23:12 On se postavil doprostřed toho dílu, osvobodil jej a bil Pelištejce. Tak Hospodin připravil veliké vysvobození.
#23:13 Jiní tři ze třiceti vůdců sestoupili a přišli o žních k Davidovi do jeskyně Adulámu. Pelištejský houf ležel v dolině Refájců.
#23:14 David byl tehdy ve skalní skrýši, postavení Pelištejců bylo tenkrát u Betléma.
#23:15 Tu David zatoužil: „Kdo mi dá napít vody z betlémské studny, která je u brány?“
#23:16 Ti tři bohatýři vtrhli do pelištejského tábora, načerpali vodu z betlémské studny, která je u brány, a přinesli ji Davidovi. On však nechtěl pít, nýbrž vykonal jí úlitbu Hospodinu.
#23:17 Řekl: „Hospodine, nechť jsem dalek toho, abych udělal něco takového. Což to není krev mužů, kteří šli s nasazením života?“ Proto se jí nechtěl napít. To vykonali ti tři bohatýři.
#23:18 Také Abíšaj, bratr Jóabův, syn Serújin, byl vůdce tří. I on zamával kopím a proklál jich na tři sta. Mezi těmi třemi byl nejproslulejší.
#23:19 Byl z těch tří nejváženější a byl jejich velitelem, ale oněm třem prvním se nevyrovnal.
#23:20 Benajáš, syn Jójady, syna zdatného muže z Kabseelu, muže mnoha činů, ubil dva moábské reky a sestoupil a ubil v jámě lva; byl tehdy sníh.
#23:21 Ubil též Egypťana hrozného vzezření; Egypťan měl v ruce kopí. Sestoupil k němu s holí, vytrhl Egypťanovi kopí z ruky a jeho vlastním kopím ho zabil.
#23:22 Toto vykonal Benajáš, syn Jójadův. A byl mezi těmi třemi bohatýry proslulý.
#23:23 Z těch třiceti byl nejváženější, ale oněm třem prvním se nevyrovnal. David ho přidělil ke své tělesné stráži.
#23:24 Mezi těmi třiceti byl také Asáel, bratr Jóabův, Elchánan, syn Dóda Betlémského,
#23:25 Šama Charódský, Elíka Charódský,
#23:26 Cheles Paltejský, Íra, syn Íkeše Tekójského,
#23:27 Abíezer Anatótský, Mebúnaj Chúšatský,
#23:28 Salmón Achóchejský, Mahraj Netófský,
#23:29 Cheleb, syn Baany Netófského, Ítaj, syn Ríbaje z Gibeje Benjamínovců,
#23:30 Benajáš Pireatónský, Hidaj z Gaašských úvalů,
#23:31 Abí-albón Arbátský, Azmávet Barchúmský,
#23:32 Eljachba Šaalbónský, ze synů Jášenových Jónatan,
#23:33 Šama Hararský, Achíam, syn Šárara Ararského,
#23:34 Elífelet, syn Achasbaje, syna Maakaťanova, Elíam, syn Achítofela Gíloského,
#23:35 Chesrav Karmelský, Paaraj Arbejský,
#23:36 Jigál, syn Nátana ze Sóby, Bání Gádský,
#23:37 Selek Amónský, Nacharaj Beerótský, zbrojnoš Jóaba, syna Serújina,
#23:38 Íra Jitrejský, Gáreb Jitrejský
#23:39 a Urijáš Chetejský. Všech bylo třicet sedm. 
#24:1 Hospodin znovu vzplanul proti Izraeli hněvem a podnítil Davida proti nim: „Jdi, sečti Izraele a Judu!“
#24:2 Král poručil Jóabovi, veliteli vojska, který byl s ním: „Obejdi všechny izraelské kmeny od Danu až k Beer-šebě. Spočítejte lid, chci znát počet lidu.“
#24:3 Jóab králi namítl: „Nechť Hospodin, tvůj Bůh, zvětší lid jak chce, třeba stokrát. Kéž na vlastní oči to král, můj pán, vidí. Ale proč král, můj pán, o to tak stojí?“
#24:4 Královo rozhodnutí bylo však pro Jóaba a velitele vojska nezvratné. Jóab s veliteli vojska vyšel od krále, aby sečetli ten lid, Izraele.
#24:5 Překročili Jordán a utábořili se v Aróeru, napravo od města, jež je uprostřed Gádského údolí, směrem k Jaezeru.
#24:6 Pak přišli do Gileádu a do země Tachtím-chodší a přišli do Dan-jaanu a do okolí Sidónu.
#24:7 Přišli do pevnosti Sóru a do všech měst chivejských a kenaanských, pak šli na jih Judy k Beer-šebě.
#24:8 Obešli celou zemi a po uplynutí devíti měsíců a dvaceti dnů přišli do Jeruzaléma.
#24:9 Jóab králi odevzdal součet branného lidu: V Izraeli bylo osm set tisíc bojeschopných mužů, schopných tasit meč, a judských mužů bylo pět set tisíc.
#24:10 Ale potom měl David výčitky svědomí, že dal lid sečíst. David volal k Hospodinu: „Velmi jsem zhřešil, že jsem to učinil. Nyní, Hospodine, sejmi prosím ze svého služebníka vinu; počínal jsem si jako velký pomatenec.“
#24:11 Ráno, když David vstal, stalo se slovo Hospodinovo k proroku Gádovi, Davidovu vidoucímu:
#24:12 „Jdi a promluv k Davidovi: Toto praví Hospodin: Předkládám ti trojí; jedno z toho si vyber a tak s tebou naložím.“
#24:13 Gád přišel k Davidovi a oznámil mu: „Má na tebe dolehnout sedm let hladu v tvé zemi? Nebo chceš tři měsíce utíkat před svými protivníky, kteří tě budou pronásledovat? Anebo má řádit ve tvé zemi po tři dny mor? Nuže tedy rozvaž, co mám vyřídit tomu, který mě poslal.“
#24:14 David Gádovi odvětil: „Je mi velmi úzko. Nechť tedy prosím padneme do rukou Hospodinu, neboť jeho slitování je nesmírné, jen ať nepadnu do rukou lidských.“
#24:15 Hospodin tedy dopustil na Izraele mor od toho jitra až do určeného času. I pomřelo z lidu od Danu až k Beer-šebě sedmdesát tisíc mužů.
#24:16 Pak vztáhl anděl svou ruku nad Jeruzalém, aby v něm šířil zkázu. Ale Hospodina pojala nad tím zlem lítost. I řekl andělu, jenž šířil mezi lidem zkázu: „Dost, již přestaň.“ Hospodinův anděl byl právě u humna Aravny Jebúsejského.
#24:17 Když David uviděl anděla bijícího lid, pravil Hospodinu: „Hle, já jsem zhřešil a já jsem se provinil; co však učinily tyto ovce? Buď tedy tvá ruka proti mně a proti domu mého otce!“
#24:18 Téhož dne přišel Gád opět k Davidovi a řekl mu: „Vystup a postav Hospodinu na humně Aravny Jebúsejského oltář.“
#24:19 David vystoupil podle Gádova slova, jak Hospodin přikázal.
#24:20 Aravna vyhlížel a spatřil krále s jeho služebníky, že jdou k němu. I vyšel Aravna a poklonil se králi tváří k zemi.
#24:21 Aravna pravil: „Proč přicházíš, králi, můj pane, k svému otroku?“ David odpověděl: „Abych od tebe koupil humno k vybudování oltáře Hospodinu; zde byla od lidu odvrácena pohroma.“
#24:22 Aravna pravil Davidovi: „Nechť král, můj pán, vezme a obětuje, co uzná za dobré. Tu je dobytek k zápalné oběti a jako dříví smyky a dobytčí jha.“
#24:23 To vše dával král Aravna králi. Aravna ještě králi řekl: „Kéž má v tobě Hospodin, tvůj Bůh, zalíbení.“
#24:24 Král Aravnovi odvětil: „Nikoli, odkoupím to od tebe za kupní cenu. Hospodinu, svému Bohu, nemohu obětovat zápalné oběti darované.“ David tedy koupil to humno i ten skot za padesát šekelů stříbra.
#24:25 I vybudoval tam David Hospodinu oltář a obětoval zápalné a pokojné oběti. Hospodin prosby za zemi přijal a pohroma byla od Izraele odvrácena.  

\book{I Kings}{1Kgs}
#1:1 Král David zestárl a dosáhl vysokého věku. Přikrývali ho pokrývkami, ale nemohl se zahřát.
#1:2 Proto mu jeho služebníci navrhli: „Ať vyhledají králi, našemu pánu, dívku pannu, aby byla králi k službám a opatrovala ho; bude též uléhat po tvém boku a zahřívat krále, našeho pána.“
#1:3 Hledali tedy po celém izraelském území krásnou dívku, až našli Šúnemanku Abíšagu a přivedli ji ke králi.
#1:4 Byla to překrásná dívka; opatrovala krále a obsluhovala ho. Král ji však nepoznal.
#1:5 Adónijáš, syn Chagítin, se vynášel: „Kralovat budu já.“ A opatřil si vůz, koně a padesát mužů, kteří běhali před ním.
#1:6 Otec ho po celý ten čas nezarmoutil otázkou: „Co to děláš?“ Také on byl velmi pěkné postavy; narodil se po Abšalómovi.
#1:7 Dohodl se s Jóabem, synem Serújiným, a s knězem Ebjátarem. Ti Adónijáše podporovali.
#1:8 Avšak kněz Sádok a Benajáš, syn Jójadův, ani prorok Nátan a Šimeí, Reí a Davidovi bohatýři při Adónijášovi nestáli.
#1:9 Adónijáš připravil obětní hod, brav, skot a vykrmený dobytek, u kamene Zócheletu, který je vedle pramene Rogelu, a pozval všechny své bratry, královy syny, i všechny judské muže, královy služebníky.
#1:10 Ale proroka Nátana, Benajáše, bohatýry a svého bratra Šalomouna nepozval.
#1:11 I řekl Nátan Bat-šebě, matce Šalomounově: „Neslyšela jsi, že se Adónijáš, syn Chagítin, stal králem? A David, náš pán, o tom neví.
#1:12 Pojď tedy, poradím ti, jak se zachráníš ty i tvůj syn Šalomoun.
#1:13 Nuže, vejdi ke králi Davidovi a zeptej se ho: ‚Což jsi ty sám, králi, můj pane, nepřísahal své otrokyni: Tvůj syn Šalomoun bude po mně králem, ten dosedne na můj trůn? Proč se tedy stal králem Adónijáš?‘
#1:14 Zatímco tam ještě budeš s králem mluvit, přijdu za tebou a potvrdím tvá slova.“
#1:15 Bat-šeba tedy vešla do pokojíka ke králi. Král totiž byl už velmi starý a Šúnemanka Abíšag ho obsluhovala.
#1:16 Bat-šeba před králem padla na kolena a poklonila se mu. Král se otázal: „Co si přeješ?“
#1:17 Odpověděla mu: „Můj pane, ty jsi své otrokyni přísahal při Hospodinu, svém Bohu: ‚Tvůj syn Šalomoun bude po mně králem, ten dosedne na můj trůn‘.
#1:18 A hle, teď se stal králem Adónijáš, a ty, králi, můj pane, o tom nevíš.
#1:19 Už připravil obětní hod, býky a velké množství vykrmeného dobytka a bravu, a pozval všechny královy syny, kněze Ebjátara a velitele vojska Jóaba, ale tvého služebníka Šalomouna nepozval.
#1:20 K tobě, králi, můj pane, se upírají oči celého Izraele, abys jim oznámil, kdo po tobě dosedne na trůn krále, mého pána.
#1:21 Kdyby se stalo, že by král, můj pán, ulehl ke svým otcům, budeme tu stát já a můj syn Šalomoun jako hříšníci.“
#1:22 Zatímco ještě mluvila s králem, přišel prorok Nátan.
#1:23 Oznámili králi: „Je tu prorok Nátan.“ Vstoupil před krále a klaněl se mu tváří až k zemi.
#1:24 Pak se Nátan otázal: „Králi, můj pane, což jsi řekl: ‚Po mně bude králem Adónijáš, ten dosedne na můj trůn‘?
#1:25 Dnes totiž sestoupil a připravil obětní hod, býky a množství vykrmeného dobytka a bravu, a pozval všechny královy syny i velitele vojska a kněze Ebjátara a oni s ním jedí a pijí a provolávají: ‚Ať žije král Adónijáš!‘
#1:26 Ale mne, tvého služebníka, ani kněze Sádoka a Benajáše, syna Jójadova, ani tvého služebníka Šalomouna nepozval.
#1:27 Pochází tento rozkaz vskutku od krále, mého pána? Neuvědomil jsi svého služebníka o tom, kdo po tobě dosedne na trůn, králi, můj pane.“
#1:28 Král David odpověděl: „Zavolejte mi Bat-šebu.“ Vešla před krále a zůstala před ním stát.
#1:29 Král se zapřisáhl: „Jakože živ je Hospodin, který mě vykoupil z každého soužení,
#1:30 jak jsem ti přísahal při Hospodinu, Bohu Izraele, králem bude po mně tvůj syn Šalomoun, ten dosedne na můj trůn místo mne; učiním tak ještě dnes.“
#1:31 Bat-šeba padla na kolena, poklonila se králi tváří k zemi a řekla: „Ať navěky žije král David, můj pán.“
#1:32 Potom král David nařídil: „Zavolejte mi kněze Sádoka, proroka Nátana a Benajáše, syna Jójadova.“ Když předstoupili před krále,
#1:33 král jim řekl: „Vezměte s sebou služebníky svého pána, posaďte mého syna Šalomouna na mou mezkyni a doveďte ho dolů ke Gichónu.
#1:34 Tam ho kněz Sádok a prorok Nátan pomažou za krále nad Izraelem. Zatroubíte na polnici a budete volat: ‚Ať žije král Šalomoun!‘
#1:35 Pak půjdete za ním vzhůru a on přijde, dosedne na můj trůn a bude králem místo mne. Je to můj příkaz, aby on byl vévodou nad Izraelem i nad Judou.“
#1:36 Benajáš, syn Jójadův, králi odpověděl: „Stane se! Tak rozhodl Hospodin, Bůh krále, mého pána.
#1:37 Jako byl Hospodin při králi, mém pánu, tak ať je při Šalomounovi a ať vyvýší jeho trůn nad trůn krále Davida, mého pána.“
#1:38 Kněz Sádok, prorok Nátan a Benajáš, syn Jójadův, i Keretejci a Peletejci sešli dolů; Šalomouna posadili na mezkyni krále Davida a vedli ho ke Gichónu.
#1:39 Kněz Sádok vzal ze stanu Hospodinova roh s olejem a Šalomouna pomazal. Zatroubili na polnici a všechen lid provolal: „Ať žije král Šalomoun!“
#1:40 Všechen lid pak šel vzhůru za ním, pískal na píšťaly a převelice se radoval; země od toho hluku až pukala.
#1:41 Adónijáš a všichni pozvaní, kteří byli s ním, to uslyšeli, když hostina končila. Jóab zaslechl zvuk polnice a otázal se: „Co je to ve městě za velký halas?“
#1:42 Zatímco ještě mluvil, přišel Jónatan, syn kněze Ebjátara, a Adónijáš mu řekl: „Pojď, ty jsi statečný muž a jistě přinášíš dobrou zvěst.“
#1:43 Ale Jónatan Adónijášovi odpověděl: „Naopak. Král David, náš pán, ustanovil za krále Šalomouna.
#1:44 Král s ním poslal kněze Sádoka, proroka Nátana a Benajáše, syna Jójadova, i Keretejce a Peletejce a posadili ho na královu mezkyni.
#1:45 Kněz Sádok a prorok Nátan ho u Gichónu pomazali za krále, a když šli odtamtud radostně vzhůru, město halasilo. To byl ten zvuk, který jste zaslechli.
#1:46 Šalomoun už dosedl na královský trůn.
#1:47 Královi služebníci přišli rovněž a dobrořečili našemu pánu, králi Davidovi: ‚Ať tvůj Bůh učiní jméno Šalomounovo slavnějším než tvé jméno a ať vyvýší jeho trůn nad tvůj trůn.‘ A král se na svém lůžku poklonil.
#1:48 Poté král řekl: ‚Požehnán buď Hospodin, Bůh Izraele, že nám dal dnes toho, kdo bude sedět na mém trůnu, a že to mé oči vidí.‘“
#1:49 Tu se všichni pozvaní, kteří byli při Adónijášovi, roztřásli, vstali a každý šel svou cestou.
#1:50 I Adónijáš vstal a ze strachu před Šalomounem šel a chytil se rohů oltáře.
#1:51 Šalomounovi oznámili: „Hle, Adónijáš se bojí krále Šalomouna, drží se rohů oltáře a říká: ‚Ať mi dnes král Šalomoun odpřisáhne, že svého služebníka neusmrtí mečem.‘“
#1:52 Šalomoun řekl: „Bude-li se chovat jako statečný muž, ani vlas mu nespadne na zem, bude-li však na něm shledáno něco zlého, zemře.“
#1:53 Král Šalomoun ho dal odvést od oltáře a on přišel a poklonil se králi Šalomounovi. Šalomoun mu řekl: „Jdi do svého domu.“ 
#2:1 Když se přiblížil čas Davidovy smrti, přikázal svému synu Šalomounovi:
#2:2 „Odcházím cestou všeho pozemského. Ty však buď rozhodný a mužný.
#2:3 Dbej na to, co ti svěřil Hospodin, tvůj Bůh: Choď po jeho cestách a dodržuj jeho nařízení a přikázání, jeho práva a svědectví, jak jsou zapsána v zákoně Mojžíšově, a tak budeš mít úspěch ve všem, co budeš konat, ať se obrátíš kamkoli.
#2:4 A Hospodin splní své slovo, které mi dal: ‚Budou-li tvoji synové dbát na svou cestu tak, aby chodili přede mnou věrně celým srdcem a celou duší, nebude z izraelského trůnu vyhlazen následník z tvého rodu.‘
#2:5 Ty také víš, co mi udělal Jóab, syn Serújin, co udělal dvěma velitelům izraelských vojsk, Abnérovi, synu Nérovu, a Amasovi, synu Jeterovu: Zabil je a dopustil se válečného krveprolití v čas pokoje. Krví prolitou jako ve válce si potřísnil opasek na svých bedrech a opánky na svých nohou.
#2:6 Zachovej se podle své moudrosti a nedej mu pokojně sestoupit v šedinách do hrobu.
#2:7 Synům Barzilaje Gileádského však prokážeš milosrdenství. Ať jsou mezi těmi, kdo jedí u tvého stolu, neboť právě tak se zachovali ke mně, když jsem prchal před tvým bratrem Abšalómem.
#2:8 Je tu u tebe i Šimeí, syn Gérův, Benjamínec z Bachurímu, který mi zlořečil strašlivým způsobem v den, kdy jsem odcházel do Machanajimu. Ale potom mi vyšel vstříc dolů k Jordánu a já jsem mu při Hospodinu přísahal: ‚Neusmrtím tě mečem.‘
#2:9 Teď však ho nenechávej bez trestu. Protože jsi moudrý muž, budeš vědět, jak s ním naložit, abys jeho šediny uvedl v krvi do hrobu.“
#2:10 I ulehl David ke svým otcům a byl pohřben v Městě Davidově.
#2:11 David kraloval nad Izraelem po dobu čtyřiceti let. V Chebrónu David kraloval sedm let, v Jeruzalémě třiatřicet let.
#2:12 Šalomoun dosedl na trůn svého otce Davida a jeho království se velmi upevnilo.
#2:13 Adónijáš, syn Chagítin, přišel k Šalomounově matce Bat-šebě. Otázala se: „Přicházíš pokojně?“ Odpověděl: „Pokojně.“
#2:14 A dodal: „Chtěl bych na tobě něco.“ Ona odvětila: „Mluv“.
#2:15 Řekl: „Ty víš, že království patřilo mně a že na mne se obrátil celý Izrael s žádostí, abych kraloval. Ale království připadlo mému bratru. Má je od Hospodina.
#2:16 Nyní mám k tobě jedinou prosbu. Neodmítni mě.“ Ona mu odpověděla: „Mluv.“
#2:17 Řekl tedy: „Řekni prosím králi Šalomounovi, tebe přece neodmítne, aby mi dal za ženu Šúnemanku Abíšagu.“
#2:18 Bat-šeba odpověděla: „Dobře. Já se za tebe u krále přimluvím.“
#2:19 A Bat-šeba šla ke králi Šalomounovi, aby s ním promluvila o Adónijášovi. Král jí vyšel vstříc a poklonil se jí. Pak se posadil na svůj trůn. I králově matce přistavili trůn a ona se posadila po jeho pravici.
#2:20 Pak řekla: „Mám k tobě jednu malou prosbu. Neodmítni mě.“ Král jí odvětil: „Žádej, má matko, tebe neodmítnu.“
#2:21 Řekla: „Ať je Šúnemanka Abíšag dána za ženu tvému bratru Adónijášovi.“
#2:22 Král Šalomoun své matce odpověděl: „Proč žádáš pro Adónijáše Šúnemanku Abíšagu? Žádej pro něho království. Vždyť je to můj bratr, starší než já. Žádej je pro něho a pro kněze Ebjátara a pro Jóaba, syna Serújina!“
#2:23 A král Šalomoun přísahal při Hospodinu: „Ať se mnou Bůh udělá, co chce! Adónijáš vyřkl rozsudek sám nad sebou.
#2:24 Jakože živ je Hospodin, který mě pevně posadil na trůn mého otce Davida a který mi podle svého slova vybudoval dům, ještě dnes Adónijáš zemře!“
#2:25 Král Šalomoun to zařídil prostřednictvím Benajáše, syna Jójadova. Ten ho skolil; tak zemřel.
#2:26 Knězi Ebjátarovi král poručil: „Odejdi do Anatótu ke svým polím, neboť jsi propadl smrti. Dnes tě zabít nedám, protože jsi nosíval schránu Panovníka Hospodina před mým otcem Davidem a protože jsi snášel všechno, co snášel můj otec.“
#2:27 I zapudil Šalomoun Ebjátara, takže přestal být Hospodinovým knězem, a tak se naplnilo Hospodinovo slovo, které vyslovil proti domu Élího v Šílu.
#2:28 Zpráva o tom se dostala až k Jóabovi; Jóab se totiž přiklonil k Adónijášovi, ačkoli k Abšalómovi se nepřiklonil. I utekl Jóab ke stanu Hospodinovu a chytil se rohů oltáře.
#2:29 Králi Šalomounovi oznámili, že Jóab utekl ke stanu Hospodinovu, že je u oltáře. Proto Šalomoun poslal Benajáše, syna Jójadova, s rozkazem: „Jdi a skol ho!“
#2:30 Benajáš tedy přišel ke stanu Hospodinovu a řekl: „Toto praví král: Jdi odtud!“ Ale on odpověděl: „Nikoli, zemřu zde.“ Benajáš tedy vyřídil králi tuto odpověď: „Jóab řekl toto a odpověděl mi takto.“
#2:31 Král mu nařídil: „Učiň, jak řekl. Skol ho! Potom ho pohřbi. Tak sejmeš ze mne a z domu mého otce vinu za krev, kterou Jóab bezdůvodně prolil.
#2:32 Hospodin obrátí jeho krev na jeho hlavu, protože skolil dva muže spravedlivější a lepší, než je sám. Zavraždil je mečem, aniž o tom můj otec David věděl, totiž Abnéra, syna Nérova, velitele vojska izraelského, a Amasu, syna Jeterova, velitele vojska judského.
#2:33 Jejich krev se obrátí na hlavu Jóabovu a na hlavu jeho potomstva navěky. Davidovi a jeho potomstvu, jeho domu i trůnu ať vzejde od Hospodina pokoj až navěky.“
#2:34 Benajáš, syn Jójadův, tedy vyšel a Jóaba skolil a usmrtil. Pohřbili ho pak v jeho domě na poušti.
#2:35 Benajáše, syna Jójadova, ustanovil král místo něho nad vojskem a kněze Sádoka ustanovil král místo Ebjátara.
#2:36 Potom si dal král předvolat Šimeího a nařídil mu: „Postav si dům v Jeruzalémě a usaď se tu. Nikam odtud nevycházej.
#2:37 Buď jist, že v den, kdy vyjdeš a překročíš Kidrónský úval, propadneš smrti. Tvá krev padne na tvou hlavu.“
#2:38 Šimeí králi odpověděl: „Slovo, jež promluvil král, můj pán, je dobré a tvůj služebník se tak zachová.“ Šimeí tedy bydlil v Jeruzalémě po mnoho dnů.
#2:39 Po třech letech uprchli Šimeímu dva otroci k Akíšovi, synu Maakovu, králi gatskému. Šimeímu bylo oznámeno: „Tvoji otroci jsou v Gatu.“
#2:40 Nato Šimeí osedlal osla a jel do Gatu k Akíšovi hledat své otroky. Odjel a své otroky z Gatu přivedl.
#2:41 Šalomounovi však bylo ohlášeno, že Šimeí odešel z Jeruzaléma do Gatu a zase se vrátil.
#2:42 Král si dal Šimeího zavolat a řekl mu: „Což jsem tě nezavázal přísahou při Hospodinu a nedal jsem ti výstrahu: ‚Buď jist, že v den, kdy vyjdeš a půjdeš kamkoli, propadneš smrti‘? A tys mi řekl: ‚Slovo, jež jsem slyšel, je dobré.‘
#2:43 Proč jsi tedy nedbal na Hospodinovu přísahu a na příkaz, který jsem ti dal?“
#2:44 Král dále Šimeímu řekl: „Ty víš o všem tom zlém, co bylo v tvém srdci, co jsi provedl mému otci Davidovi. Hospodin tvou zlobu obrátil na tvoji hlavu.
#2:45 Král Šalomoun však bude požehnaný a trůn Davidův zůstane před Hospodinem pevný navěky.“
#2:46 Pak dal král příkaz Benajášovi, synu Jójadovu, a ten vyšel a skolil ho; tak zemřel. A království se v ruce Šalomounově upevnilo. 
#3:1 Šalomoun se spříznil s faraónem, králem egyptským. Pojal faraónovu dceru za ženu a uvedl ji do Města Davidova, kde pobývala, dokud nedokončil stavbu svého domu a domu Hospodinova a hradeb kolem Jeruzaléma.
#3:2 Lid však obětoval na posvátných návrších, neboť dům pro Hospodinovo jméno nebyl ještě v oněch dnech zbudován.
#3:3 Šalomoun miloval Hospodina. Řídil se nařízeními svého otce Davida, až na to, že obětoval a pálil kadidlo na posvátných návrších.
#3:4 Jednou šel král do Gibeónu, aby tam obětoval; bylo to největší posvátné návrší. Šalomoun tam na oltáři obětoval tisíc zápalných obětí.
#3:5 V Gibeónu se Šalomounovi ukázal v noci ve snu Hospodin. Bůh řekl: „Žádej, co ti mám dát.“
#3:6 Šalomoun odvětil: „Ty jsi prokazoval velké milosrdenství svému služebníku, mému otci Davidovi, a on před tebou chodil věrně, spravedlivě a se srdcem upřímným vůči tobě. Toto velké milosrdenství jsi mu zachoval a dal jsi mu syna, který sedí na jeho trůnu, jak je tomu dnes.
#3:7 Hospodine, můj Bože, ty jsi nyní po mém otci Davidovi ustanovil za krále svého služebníka, ale já jsem příliš mladý, neumím vycházet a vcházet.
#3:8 A tvůj služebník je uprostřed tvého lidu, který jsi vyvolil, lidu tak početného, že nemůže být pro množství počítán ani sečten.
#3:9 Kéž bys tedy dal svému služebníku srdce vnímavé, aby mohl soudit tvůj lid a dovedl rozlišovat mezi dobrem a zlem. Neboť kdo by dokázal soudit tento tvůj lid, jemuž je tak těžko vládnout?“
#3:10 Panovníku se líbilo, že Šalomoun žádal o tuto věc.
#3:11 Bůh mu řekl: „Protože jsi žádal o toto a nežádal jsi pro sebe ani dlouhý věk ani jsi nežádal bohatství, ba ani jsi nežádal bezživotí svých nepřátel, ale žádal jsi pro sebe rozumnost při soudním jednání,
#3:12 hle, učiním podle tvých slov. Dávám ti moudré a rozumné srdce, takže nikdo tobě podobný nebyl před tebou a ani po tobě nepovstane nikdo tobě podobný.
#3:13 A dávám ti i to, oč jsi nežádal, bohatství i slávu, tak aby nebyl nikdo tobě podobný mezi králi po všechny tvé dny.
#3:14 Budeš-li chodit po mých cestách a zachovávat má nařízení a přikázání, tak jako chodil tvůj otec David, prodloužím tvé dny.“
#3:15 Tu se Šalomoun probudil a viděl, že to byl sen. Šel tedy do Jeruzaléma a postavil se před schránu Panovníkovy smlouvy, přinesl zápalné oběti, připravil oběti pokojné a vystrojil všem svým služebníkům hody.
#3:16 Tehdy přišly ke králi dvě ženy nevěstky a postavily se před něj.
#3:17 Jedna z těch žen řekla: „Prosím, můj pane, já a tato žena bydlíme v jednom domě a já jsem u ní v domě porodila.
#3:18 Třetího dne po mém porodu také tato žena porodila. Byly jsme spolu a v tom domě s námi nebyl nikdo cizí, v domě nebyl nikdo kromě nás dvou.
#3:19 Syn této ženy však v noci zemřel, neboť ho zalehla.
#3:20 Proto v noci vstala, a zatímco tvá otrokyně spala, vzala mého syna od mého boku, položila si ho do klína a svého mrtvého syna položila do klína mně.
#3:21 Ráno jsem vstala, abych svého syna nakojila, ale on byl mrtev. Když jsem si ho však zrána pozorně prohlédla, zjistila jsem, že to není můj syn, kterého jsem porodila.“
#3:22 Druhá žena však prohlásila: „Nikoli. Můj syn je ten živý a ten mrtvý je tvůj.“ Ale první trvala na svém: „Ne. Tvůj syn je ten mrtvý, a ten živý je můj.“ A tak se před králem hádaly.
#3:23 Král řekl: „Tato tvrdí: ‚Ten živý je můj syn, a ten mrtvý je tvůj.‘ A tato tvrdí: ‚Ne, tvůj syn je ten mrtvý, a ten živý je můj.‘“
#3:24 Král proto poručil: „Podejte mi meč.“ Přinesli tedy před krále meč.
#3:25 A král nařídil: „Rozetněte to živé dítě ve dví. Jednu polovinu dejte jedné a druhou polovinu druhé.“
#3:26 Tu řekla králi žena, jejíž syn byl ten živý a jíž se srdce svíralo soucitem nad jejím synem: „Prosím, můj pane, dejte to živé novorozeně jí, jen je neusmrcujte!“ Ale druhá řekla: „Ať není ani moje ani tvoje. Rozetněte je!“
#3:27 Tu král rozhodl: „Dejte to živé novorozeně té, která řekla: ‚Neusmrcujte je‘; to je jeho matka.“
#3:28 Když se celý Izrael dozvěděl o rozsudku, který král vynesl, jala je bázeň před králem. Viděli, že je nadán Boží moudrostí k vykonávání soudu. 
#4:1 Tak byl král Šalomoun králem nad celým Izraelem.
#4:2 A toto byli jeho nejvyšší úředníci: Azarjáš, syn Sádokův, byl knězem.
#4:3 Elíchoref a Achijáš, synové Šíšovi, byli písaři, Jóšafat, syn Achílúdův, byl kancléřem,
#4:4 Benajáš, syn Jójadův, byl nad vojskem. Sádok a Ebjátar byli kněžími.
#4:5 Azarjáš, syn Nátanův, byl nad správci, Zabúd, syn Nátanův, byl knězem a královým přítelem.
#4:6 Achíšar byl nad domem a Adóníram, syn Abdův, byl nad nucenými pracemi.
#4:7 Šalomoun měl dále dvanáct správců nad celým Izraelem a ti opatřovali potravu pro krále a jeho dům. Každý měl v roce svůj měsíc, kdy potravu opatřoval.
#4:8 Toto jsou jejich jména: Syn Chúrův v Efrajimském pohoří.
#4:9 Syn Dekerův v Mákasu a v Šaalbímu, Bét-šemeši, Elónu a Bét-chánanu.
#4:10 Syn Chesedův v Arubótu. Jemu podléhalo Sóko a celá země cheferská.
#4:11 Syn Abínádabův spravoval celou pahorkatinu dórskou. Měl za ženu Táfatu, dceru Šalomounovu.
#4:12 Baana, syn Achílúdův, spravoval Taanak a Megido a celý Bét-šeán, který leží vedle Saretanu pod Jizreelem, od Bét-šeánu až k Ábel-mechóle, až za Jokmoám.
#4:13 Syn Geberův v Rámotu v Gileádu. Jemu podléhaly vsi Jaíra, syna Manasesova, který byl v Gileádu, podléhala mu také krajina argóbská, která je v Bášanu, šedesát velkých měst s hradbami a bronzovými závorami.
#4:14 Achínádab, syn Idův, v Machanajimu.
#4:15 Achímaas na území Neftalíově; také on si vzal za ženu Šalomounovu dceru, a to Basematu.
#4:16 Baaná, syn Chúšajův, na území Ašerově a v Alótu.
#4:17 Jóšafat, syn Parúachův, na území Isacharově.
#4:18 Šimeí, syn Élův, na území Benjamínově.
#4:19 Geber, syn Uríův, v zemi gileádské, v zemi emorejského krále Síchona a bášanského krále Óga; v té zemi byl jeden výsostný znak.
#4:20 Juda a Izrael byli tak početní jako písek u moře. Měli co jíst a pít a radovali se. 
#5:1 Šalomoun vládl nad všemi královstvími od řeky Eufratu až k zemi pelištejské a až k hranicím Egypta; odtud přinášeli Šalomounovi dary a sloužili mu po všechny dny jeho života.
#5:2 Na jeden den se u Šalomounova dvora spotřebovalo třicet kórů bílé mouky a šedesát kórů mouky ječné,
#5:3 deset kusů krmného hovězího dobytka a dvacet kusů dobytka z pastvy, sto kusů ovcí a koz, kromě jelenů, gazel, antilop a vykrmené drůbeže.
#5:4 Panoval nad celým územím na západ od Řeky od Tifsachu až ke Gáze, nad všemi králi za Řekou, a měl pokoj ode všech sousedů.
#5:5 A Juda i Izrael bydleli v bezpečí po všechny dny Šalomounovy, každý pod svou vinnou révou a pod svým fíkovníkem, od Danu až po Beer-šebu.
#5:6 Šalomoun měl také ustájeno čtyřicet tisíc koní pro svou vozbu a dvanáct tisíc koní jezdeckých.
#5:7 Jmenovaní správci opatřovali potravu pro krále Šalomouna a pro všechny, kteří měli přístup ke stolu krále Šalomouna, každý ve svém měsíci. Nedopouštěli, aby něco chybělo.
#5:8 Také ječmen a slámu pro koně a lehkonohé oře dodávali na místo, kde právě král byl, každý podle toho, jak mu bylo určeno.
#5:9 Bůh dal Šalomounovi moudrost, převelikou rozumnost a takovou hojnost myšlenek, jako je písku na břehu moře.
#5:10 Šalomounova moudrost převýšila moudrost všech synů dávnověku i všechnu moudrost egyptskou.
#5:11 Byl moudřejší než všichni lidé, než Étan Ezrachejský a Héman, Kalkol a Darda, synové Machólovi. Jeho jméno bylo proslulé mezi všemi okolními pronárody.
#5:12 Vyslovil tři tisíce přísloví a jeho písní bylo tisíc a pět.
#5:13 Dovedl také mluvit o stromech, od cedru, který je na Libanónu, až po yzop, který roste na zdi, a dovedl mluvit i o zvířatech, ptácích, plazech a rybách.
#5:14 Šalomounově moudrosti přicházeli naslouchat ze všech národů i ode všech králů země, kteří se o jeho moudrosti doslechli.
#5:15 Týrský král Chíram poslal k Šalomounovi své služebníky, neboť uslyšel, že ho pomazali za krále namísto jeho otce. Chíram byl totiž přítelem Davidovým po všechna léta.
#5:16 Šalomoun poslal Chíramovi vzkaz:
#5:17 „Ty víš, že můj otec David nemohl vybudovat dům jménu Hospodina, svého Boha, pro boje, které vedl s okolními nepřáteli, dokud mu je Hospodin nepoložil pod nohy.
#5:18 Teď mi však Hospodin, můj Bůh, dal ode všech okolních nepřátel odpočinutí, nehrozí mi protivník ani žádná zlá událost.
#5:19 Proto jsem se rozhodl vybudovat dům jménu Hospodina, svého Boha, podle Hospodinových slov k mému otci Davidovi: ‚Tvůj syn, kterého dosadím po tobě na tvůj trůn, ten vybuduje dům mému jménu.‘
#5:20 Proto přikaž, aby pro mne poráželi libanónské cedry. Moji služebníci budou s tvými a já ti budu dávat mzdu pro tvé služebníky v takové výši, jakou určíš; vždyť víš, že mezi námi není nikoho, kdo by uměl porážet stromy jako Sidóňané.“
#5:21 Když Chíram uslyšel Šalomounova slova, velmi se zaradoval a řekl: „Požehnán buď dnes Hospodin, že dal Davidovi moudrého syna, aby vládl nad tímto početným lidem.“
#5:22 A Chíram poslal Šalomounovi poselství: „Slyšel jsem, co jsi mi vzkázal. Pokud jde o dřevo cedrové a cypřišové, vyhovím každému tvému přání.
#5:23 Moji služebníci je stáhnou z Libanónu dolů k moři a já z nich dám udělat vory; po moři je dopravím až na místo, které mi určíš. Tam je dám rozpojit a ty je dáš odtáhnout. Také ty vyhovíš mému přání a budeš dodávat potravu pro můj dům.“
#5:24 Dodával tedy Chíram Šalomounovi dřevo cedrové a cypřišové, všechno podle jeho přání.
#5:25 Šalomoun dodával Chíramovi dvacet tisíc kórů pšenice k jídlu pro jeho dům a dvacet kórů oleje získaného drcením; to dával Šalomoun Chíramovi rok co rok.
#5:26 Hospodin dal Šalomounovi moudrost, kterou mu přislíbil, a mezi Chíramem a Šalomounem byl pokoj. Oba spolu uzavřeli smlouvu.
#5:27 Král Šalomoun vybral z celého Izraele robotníky v počtu třiceti tisíc mužů.
#5:28 Z nich posílal na Libanón střídavě deset tisíc každý měsíc. Měsíc byli na Libanónu, dva měsíce doma. Nad robotníky byl Adóníram.
#5:29 Šalomoun měl také sedmdesát tisíc nosičů břemen a osmdesát tisíc kameníků v horách,
#5:30 mimo představené Šalomounových správců, kteří byli nad pracemi v počtu tří tisíc tří set a panovali nad lidem konajícím tu práci.
#5:31 Král přikázal, aby se do základů domu dávaly kvádry; proto lámali veliké kusy ušlechtilého kamene.
#5:32 Dělníci Šalomounovi a dělníci Chíramovi a Gebalci je otesávali a připravovali dřevo i kameny k budování domu. 
#6:1 Čtyři sta osmdesát let po vyjití Izraelců z egyptské země, čtvrtého roku svého kralování nad Izraelem, v měsíci zívu, což je druhý měsíc, začal Šalomoun budovat dům Hospodinu.
#6:2 Dům, který král Šalomoun budoval Hospodinu, byl dlouhý šedesát loket, široký dvacet a vysoký třicet loket.
#6:3 Délka předsíně před lodí chrámovou byla dvacet loket podle šířky domu a před domem byla široká deset loket.
#6:4 V domě dal udělat také okna se zužujícím se ostěním.
#6:5 Při zdi domu dal kolem vybudovat přístavbu, při zdech domu kolem lodi chrámové a svatostánku; tak udělal kolem ochozy.
#6:6 Přístavba byla dole široká pět loket; ve střední části šest loket a v třetí části sedm loket; dal totiž udělat na vnější zdi domu kolem římsovité výstupky, aby stropní trámy nezasahovaly do zdí domu.
#6:7 Když byl dům budován, budovali jej z kamene opracovaného již v lomu. V domě nebylo při budování slyšet kladivo ani dláto, vůbec žádné železné nástroje.
#6:8 Vchod ke střednímu ochozu byl na pravé zdi domu; po točitých schodech se vystupovalo na střední ochoz a ze středního do třetího.
#6:9 Tak budoval Šalomoun dům a dokončil jej. Přikryl jej cedrovými trámy a cedrovým obložením.
#6:10 Při celém domě dal vybudovat přístavbu vysokou vždy pět loket a spojil ji s domem cedrovými kmeny.
#6:11 I stalo se slovo Hospodinovo k Šalomounovi:
#6:12 „Pokud jde o tento dům, který buduješ: Budeš-li se řídit mými nařízeními, budeš-li uplatňovat má práva, dodržovat všechna má přikázání a podle nich žít, splním na tobě své slovo, které jsem dal tvému otci Davidovi,
#6:13 budu bydlet uprostřed Izraelců a Izraele, svůj lid, neopustím.“
#6:14 Šalomoun budování domu dokončil.
#6:15 Obložil také stěny domu uvnitř cedrovými deskami; od podlahy domu až ke stropu obložil vnitřek dřevem a podlahu domu položil z cypřišových desek.
#6:16 Dvacet loket od zadní strany domu vybudoval přepážku z cedrových desek od podlahy až ke stropu a uvnitř vybudoval svatostánek, velesvatyni.
#6:17 Dům, to je chrámová loď před velesvatyní, měřil čtyřicet loket.
#6:18 Uvnitř domu bylo všechno z cedru; v něm byly vyřezány kalichy květů a věncoví z květin. Kámen nebylo vidět.
#6:19 V nejvnitřnější části domu zřídil svatostánek, aby tam umístil schránu Hospodinovy smlouvy.
#6:20 Před svatostánkem, který byl dvacet loket dlouhý, dvacet loket široký a dvacet loket vysoký a který obložil lístkovým zlatem, umístil oltář z cedrového dřeva a obložil jej.
#6:21 Uvnitř obložil Šalomoun dům lístkovým zlatem a oddělil zlatými řetězy prostor před svatostánkem obloženým zlatem.
#6:22 Vůbec celý dům obložil zlatem, úplně celý; také oltář, který patřil ke svatostánku, obložil celý zlatem.
#6:23 Pro svatostánek zhotovil dva cheruby z olivového dřeva; jejich výška byla deset loket.
#6:24 Pět loket měřilo jedno křídlo cheruba a pět loket druhé křídlo, od jednoho konce křídla ke druhému bylo deset loket.
#6:25 Také druhý cherub měřil deset loket. Oba cherubové měli stejnou míru a stejný tvar.
#6:26 Výška prvního i druhého cheruba byla deset loket.
#6:27 Cheruby umístil do nejvnitřnější části domu. Rozprostírali křídla tak, že křídlo jednoho se dotýkalo jedné stěny a křídlo druhého se dotýkalo druhé stěny; jejich druhá křídla se uprostřed domu navzájem dotýkala.
#6:28 Také cheruby obložil zlatem.
#6:29 Na všechny stěny přední i zadní části domu dal vyřezat dokola řezby cherubů, palem a věncoví z květin.
#6:30 Rovněž podlahu přední i zadní části domu obložil zlatem.
#6:31 Vchod svatostánku opatřil dveřmi z olivového dřeva; pilíř a veřeje byly pětihranné.
#6:32 Obě křídla dveří byla z olivového dřeva. Na ně dal vyřezat řezby cherubů, palem a věncoví z květin a obložil je zlatem. I na cheruby a palmy nanesl zlato.
#6:33 Podobně dal zhotovit i pro vchod do chrámové lodi veřeje z olivového dřeva, ale čtverhranné,
#6:34 a dvojité dveře z cypřišového dřeva; jedno dveřní křídlo bylo dvojdílné a otáčelo se, též druhé křídlo bylo dvojdílné a otáčelo se.
#6:35 A dal na nich vyřezat cheruby, palmy a věncoví z květin a vyryté pak obložit tepaným zlatem.
#6:36 Vybudoval také zeď vnitřního nádvoří ze tří vrstev kamenných kvádrů a z jedné vrstvy otesaných cedrových trámů.
#6:37 Hospodinův dům byl založen ve čtvrtém roce v měsíci zívu.
#6:38 V jedenáctém roce v měsíci búlu, což je osmý měsíc, král dům dokončil podle všech svých směrnic a pokynů. Stavěl jej sedm let. 
#7:1 Svůj dům stavěl Šalomoun třináct let, než jej zcela dokončil:
#7:2 Vystavěl dům Libanónského lesa, dlouhý sto loket, široký padesát loket a vysoký třicet loket, na čtyřech řadách cedrových sloupů; na sloupech byly cedrové trámy.
#7:3 Byl přikryt cedrovým dřevem spočívajícím na nosnících, které byly na sloupech. Těch bylo čtyřicet pět, po patnácti v jedné řadě.
#7:4 A dal udělat tři řady okenních ostění, takže třikrát bylo okno proti oknu.
#7:5 Všechny vchody s veřejemi měly obdélníková ostění a třikrát bylo okno proti oknu.
#7:6 Udělal také sloupovou síň dlouhou padesát loket a širokou třicet loket. Vpředu byla předsíň se sloupy a schodištěm.
#7:7 Udělal i trůnní síň, kde soudil; soudní síň byla obložena cedrovým dřevem od podlahy ke stropu.
#7:8 Jeho dům, v němž sídlil, byl v jiném nádvoří stranou soudní síně a byl udělán jako ona. Šalomoun udělal stejným způsobem jako onu síň také dům pro dceru faraónovu, kterou pojal za ženu.
#7:9 Všechno uvnitř i zvenčí bylo z ušlechtilého kamene, z kvádrů řezaných na míru pilou, a to od základu až k horním okrajům a od vnějšku až do velkého nádvoří.
#7:10 Základy byly z ušlechtilého kamene, z velkých kamenů, z kamenů o deseti a osmi loktech,
#7:11 a nad tím kvádry z ušlechtilého kamene, řezané na míru, a cedrové dřevo.
#7:12 Kolem bylo velké nádvoří ohražené zdí ze tří vrstev kamenných kvádrů a z jedné vrstvy cedrových trámů po vzoru vnitřního nádvoří domu Hospodinova a předsíně toho domu.
#7:13 Král Šalomoun poslal pro Chírama z Týru.
#7:14 Byl to syn jedné vdovy z pokolení Neftalíova; jeho otec byl Týřan, řemeslník obrábějící měď. Byl naplněn moudrostí, rozumností a poznáním, takže se vyznal v každé práci s mědí. Přišel ke králi Šalomounovi a konal všechnu svěřenou práci.
#7:15 Vytvořil dva bronzové sloupy; jeden sloup byl vysoký osmnáct loket a dal se obepnout šňůrou dvanáct loket dlouhou; stejně tak sloup druhý.
#7:16 Udělal též dvě hlavice, odlité z bronzu, a nasadil je na vrchol sloupů; jedna i druhá hlavice byly vysoké pět loket.
#7:17 Na hlavicích, které byly na vrcholu sloupů, bylo proplétané mřížování s řetízkovými ozdobami. Sedm jich bylo na jedné i na druhé hlavici.
#7:18 Ke sloupům udělal dvě řady granátových jablek kolem každého mřížování přikrývajícího hlavici na vrcholu sloupu; stejně to udělal s druhou hlavicí.
#7:19 Hlavice na vrcholu sloupů u chrámové předsíně byly zakončeny lilií čtyři lokte vysokou.
#7:20 Na obou sloupech bylo nahoře kolem vydutí hlavic připevněno k mřížování dvě stě granátových jablek v řadách, u obou hlavic.
#7:21 Sloupy postavil u předsíně chrámu. Postavil pravý sloup a dal mu jméno Jakín. Pak postavil levý sloup a dal mu jméno Bóaz.
#7:22 Sloupy byly na vrcholku zakončeny lilií. Tak byla dohotovena práce na sloupech.
#7:23 Odlil také moře o průměru deseti loket, okrouhlé, pět loket vysoké; dalo se obepnout měřicí šňůrou dlouhou třicet loket.
#7:24 Pod jeho okrajem je dokola vroubilo věncoví, deset ozdob na jeden loket; lemovaly moře dokola. Věncoví bylo odlito ve dvou řadách spolu s ním.
#7:25 Moře spočívalo na dvanácti býcích; tři byli obráceni na sever, tři na západ, tři na jih a tři na východ. Na nich bylo moře položeno a zadky všech byly obráceny dovnitř.
#7:26 Bylo na dlaň silné a jeho okraj byl udělán jako okraj poháru nebo rozkvetlé lilie. Pojalo dva tisíce batů.
#7:27 Udělal také deset bronzových stojanů; každý stojan byl čtyři lokte dlouhý, čtyři lokte široký a tři lokte vysoký.
#7:28 Stojany byly udělány tak, že jejich kostru tvořily lišty spojované příčkami.
#7:29 Na lištách mezi příčkami byli znázorněni lvi, býci a cherubové a nad příčkami pod lvy a býky se vinuly věnce.
#7:30 Každý stojan měl čtyři bronzová kola s bronzovými osami. Jeho čtyři rohové sloupky vybíhaly v ramena. Ramena byla odlita vždy proti věncům a nesla nádrž.
#7:31 Okraj nádrže vyčníval nad prstenec stojanu o jeden loket. Byl kruhový, přiměřeně udělaný, široký jeden a půl lokte, a také na něm byla odlita výzdoba. Lišty stojanů však tvořily čtyřúhelníky, nikoli kruhy.
#7:32 Pod lištami byla čtyři kola s osami vsazenými do stojanu a každé kolo bylo vysoké jeden a půl lokte.
#7:33 Kola byla stejná jako u válečného vozu; jejich osy, loukotě, paprsky a náboje byly celé lité.
#7:34 Na čtyřech rozích každého stojanu byla čtyři ramena; ramena vycházela ze stojanu.
#7:35 Nahoře měl stojan dokola obrubu půl lokte vysokou a držadla spojená s lištami.
#7:36 Na plochy držadel a lišt vyryl cheruby, lvy a palmy na každé volné místo a věnce kolem.
#7:37 Všech deset stojanů udělal týmž způsobem: všechny byly jednotně odlity, měly jednotnou míru a jednotný tvar.
#7:38 Udělal také deset bronzových nádrží. Každá nádrž měla obsah čtyřicet batůa v průměru měla čtyři lokte. Na každém z deseti stojanů byla jedna nádrž.
#7:39 Pět stojanů postavil při pravé straně domu a pět při levé; moře postavil po pravé straně domu k jihovýchodu.
#7:40 Chíram také udělal kotlíky, lopaty a kropenky. Tak dokončil Chíram veškerou práci, kterou dělal králi Šalomounovi pro Hospodinův dům:
#7:41 dva sloupy a kulovité hlavice na vrchol obou sloupů, dvoje mřížování, aby obě kulovité hlavice na vrcholu sloupů přikrývalo,
#7:42 čtyři sta granátových jablek k obojímu mřížování, po dvou řadách granátových jablek na jedno mřížování, aby obě kulovité hlavice na sloupech přikrývaly,
#7:43 dále deset stojanů a deset nádrží na stojany,
#7:44 jedno moře a dvanáct býků jako podstavce pod moře,
#7:45 i hrnce, lopaty a kropenky a všechny ty předměty, které udělal Chíram králi Šalomounovi pro Hospodinův dům; byly z leštěného bronzu.
#7:46 Král je dal odlévat v jordánském okrsku mezi Sukótem a Saretanem do forem v zemi.
#7:47 Šalomoun upustil od vážení všech předmětů, protože jich bylo převeliké množství; na váhu mědi se nehledělo.
#7:48 Šalomoun tedy udělal všechny předměty, které byly pro Hospodinův dům: zlatý oltář a zlatý stůl, na nějž se kladl předkladný chléb,
#7:49 svícny potažené lístkovým zlatem, pět napravo a pět nalevo před svatostánkem, jejich květy, kahánky a kleště na knoty ze zlata,
#7:50 misky, nože, kropenky, číše a pánve potažené lístkovým zlatem; zlaté byly i stěžeje ke dveřím velesvatyně uvnitř domu a ke dveřím hlavní chrámové lodi.
#7:51 Tak byla ukončena všechna práce, kterou vykonal král Šalomoun pro Hospodinův dům. Šalomoun tam pak vnesl svaté dary svého otce Davida; stříbrné a zlaté i ostatní předměty uložil mezi poklady Hospodinova domu. 
#8:1 Tehdy Šalomoun svolal k sobě do Jeruzaléma shromáždění izraelských starších, všechny představitele dvanácti pokolení a předáky izraelských rodů, aby vynesli schránu Hospodinovy smlouvy z Města Davidova, totiž ze Sijónu.
#8:2 Ke králi Šalomounovi se shromáždili všichni izraelští muži ve svátek v měsíci etanímu, což je sedmý měsíc.
#8:3 Když všichni izraelští starší přišli, zvedli kněží schránu
#8:4 a vynesli nahoru schránu Hospodinovu a stan setkávání i všechny svaté předměty, které byly ve stanu; to vše vynesli kněží a lévijci.
#8:5 Král Šalomoun a s ním celá izraelská pospolitost, která se kolem něho před schránu sešla, obětovali tolik bravu a skotu, že nemohl být pro množství spočítán ani sečten.
#8:6 Kněží vnesli schránu Hospodinovy smlouvy na její místo do svatostánku domu, do velesvatyně, pod křídla cherubů.
#8:7 Cherubové totiž rozprostírali křídla k místu, kde byla schrána, takže cherubové zakrývali shora schránu i její tyče.
#8:8 Tyče však byly tak dlouhé, že jejich konce bylo vidět ze svatyně před svatostánkem, avšak zvenčí viditelné nebyly; jsou tam až dodnes.
#8:9 Ve schráně nebylo nic než dvě kamenné desky, které tam uložil Mojžíš na Chorébu, kde Hospodin uzavřel smlouvu s Izraelci, když vyšli z egyptské země.
#8:10 Když kněží vycházeli ze svatyně, naplnil Hospodinův dům oblak,
#8:11 takže kněží kvůli tomu oblaku nemohli konat službu, neboť Hospodinův dům naplnila Hospodinova sláva.
#8:12 Tehdy Šalomoun řekl: „Hospodin praví, že bude přebývat v mrákotě.
#8:13 Vybudoval jsem ti sídlo, kde budeš přebývat, vznešené obydlí, po všechny věky.“
#8:14 Pak se král obrátil a žehnal celému shromáždění Izraele; celé shromáždění Izraele přitom stálo.
#8:15 Řekl: „Požehnán buď Hospodin, Bůh Izraele, který vlastními ústy mluvil k mému otci Davidovi a vlastní rukou naplnil, co řekl:
#8:16 ‚Ode dne, kdy jsem vyvedl Izraele, svůj lid, z Egypta, nevyvolil jsem v žádném z izraelských kmenů město k vybudování domu, aby tam dlelo mé jméno. Ale vyvolil jsem Davida, aby byl nad Izraelem, mým lidem.‘
#8:17 Můj otec David měl v úmyslu vybudovat dům jménu Hospodina, Boha Izraele.
#8:18 Hospodin však mému otci Davidovi řekl: ‚Máš sice dobrý úmysl vybudovat mému jménu dům,
#8:19 avšak ten dům nezbuduješ ty, nýbrž tvůj syn, který vzejde z tvých beder; ten vybuduje dům mému jménu.‘
#8:20 Hospodin splnil své slovo, které vyřkl. Nastoupil jsem po svém otci Davidovi, dosedl podle Hospodinova slova na izraelský trůn a vybudoval jsem dům jménu Hospodina, Boha Izraele.
#8:21 Připravil jsem tam místo pro schránu, v níž je Hospodinova smlouva, kterou uzavřel s našimi otci, když je vyvedl z egyptské země.“
#8:22 Pak se Šalomoun v přítomnosti celého shromáždění Izraele postavil před Hospodinův oltář, rozprostřel dlaně k nebi
#8:23 a řekl: „Hospodine, Bože Izraele, není Boha tobě podobného nahoře na nebi ani dole na zemi. Ty zachováváš smlouvu a milosrdenství svým služebníkům, kteří chodí před tebou celým srdcem.
#8:24 Ty jsi zachoval svému služebníku, mému otci Davidovi, to, co jsi mu přislíbil. Vlastními ústy jsi přislíbil a vlastní rukou jsi to naplnil, jak je dnes zřejmé.
#8:25 Nyní, Hospodine, Bože Izraele, zachovej svému služebníku, mému otci Davidovi, to, co jsi mu přislíbil slovy: ‚Nebude přede mnou vyhlazen následník z tvého rodu, jenž bude sedět na izraelském trůnu, budou-li ovšem tvoji synové dbát na svou cestu, aby přede mnou chodili, jako jsi chodil přede mnou ty.‘
#8:26 Nyní tedy, Bože Izraele, nechť se prokáže spolehlivost tvého slova, které jsi promluvil ke svému služebníku, mému otci Davidovi.
#8:27 Ale může Bůh opravdu sídlit na zemi, když nebesa, ba ani nebesa nebes tě nemohou pojmout, natož tento dům, který jsem vybudoval?
#8:28 Hospodine, můj Bože, skloň se k modlitbě svého služebníka a k jeho prosbě o smilování a vyslyš lkání a modlitbu, kterou se tvůj služebník před tebou dnes modlí:
#8:29 Ať jsou tvé oči upřeny na tento dům v noci i ve dne, na místo, o kterém jsi řekl, že tam bude dlít tvé jméno. Vyslýchej modlitbu, kterou se bude tvůj služebník modlit obrácen k tomuto místu.
#8:30 Vyslýchej prosbu svého služebníka i Izraele, svého lidu, kterou se budou modlit obráceni k tomuto místu, vyslýchej v místě svého přebývání, v nebesích, vyslýchej a odpouštěj.
#8:31 Jestliže se někdo prohřeší proti svému bližnímu a ten by mu uložil, aby se zaklínal přísahou, a on by tu přísahu složil před tvým oltářem v tomto domě,
#8:32 ty sám v nebesích vyslyš, zasáhni a rozsuď své služebníky; prohlaš svévolníka za svévolného a odplať mu podle jeho cesty, a prohlaš spravedlivého za spravedlivého a odměň jej podle jeho spravedlnosti.
#8:33 Bude-li poražen Izrael, tvůj lid, od nepřítele pro hřích proti tobě, ale navrátí se k tobě, vzdají chválu tvému jménu a budou se k tobě modlit a prosit o smilování v tomto domě,
#8:34 vyslyš v nebesích a odpusť Izraeli, svému lidu, hřích a uveď je zpět do země, kterou jsi dal jejich otcům.
#8:35 Uzavřou-li se nebesa a nebude déšť pro hřích proti tobě, budou-li se modlit obráceni k tomuto místu a vzdávat chválu tvému jménu a odvrátí se od svých hříchů, protože jsi je pokořil,
#8:36 vyslyš v nebesích a odpusť svým služebníkům a Izraeli, svému lidu, hřích; vždyť je vyučuješ dobré cestě, po níž by měli chodit, a dej déšť své zemi, kterou jsi dal svému lidu do dědictví.
#8:37 Bude-li v zemi hlad, bude-li mor, obilná rez či sněť, kobylky nebo jiná havěť, bude-li ho v zemi jeho bran sužovat nepřítel či jakákoli rána a jakákoli nemoc,
#8:38 vyslyš každou modlitbu, každou prosbu, kterou bude mít kterýkoli člověk ze všeho tvého izraelského lidu, každý, kdo pozná ránu svého srdce a rozprostře své dlaně obrácen k tomuto domu.
#8:39 Vyslyš v nebesích, v sídle, kde přebýváš, a odpusť, učiň a odplať každému podle všech jeho cest, neboť znáš jeho srdce - vždyť ty sám jediný znáš srdce všech lidských synů -,
#8:40 aby se tě báli po všechny dny, kdy budou žít na půdě, kterou jsi dal našim otcům.
#8:41 Také přijde-li cizinec, který není z Izraele, tvého lidu, ze vzdálené země kvůli tvému jménu,
#8:42 neboť budou slyšet o tvém velkém jménu a o tvé mocné ruce a o tvé vztažené paži, přijde-li a bude se modlit obrácen k tomuto domu,
#8:43 vyslyš v nebesích, v sídle, kde přebýváš, a učiň vše, oč k tobě ten cizinec bude volat, aby poznaly tvé jméno všechny národy země a bály se tě jako Izrael, tvůj lid, aby poznaly, že se tento dům, který jsem vybudoval, nazývá tvým jménem.
#8:44 Vytáhne-li tvůj lid do boje proti nepříteli po cestě, kterou jej pošleš, a budou-li se modlit k Hospodinu směrem k městu, které jsi vyvolil, a k domu, který jsem vybudoval tvému jménu,
#8:45 vyslyš v nebesích jejich modlitbu a prosbu a zjednej jim právo.
#8:46 Zhřeší-li proti tobě, neboť není člověka, který by nehřešil, a ty se na ně rozhněváš a vydáš je nepříteli, aby je zajali a jaté vedli do nepřátelské země, vzdálené nebo blízké,
#8:47 a oni si to v zemi, do níž byli jako zajatci odvedeni, vezmou k srdci, obrátí se a budou tě v zemi těch, kdo je zajali, prosit o smilování: ‚Zhřešili jsme, provinili jsme se, svévolně si vedli‘,
#8:48 navrátí-li se tedy k tobě celým srdcem a celou duší v zemi svých nepřátel, kteří je odvedli do zajetí a budou se k tobě modlit směrem ke své zemi, kterou jsi dal jejich otcům, k městu, které jsi vyvolil, a k domu, který jsem vybudoval tvému jménu,
#8:49 vyslyš v nebesích, v sídle, kde přebýváš, jejich modlitbu a prosbu a zjednej jim právo;
#8:50 odpusť svému lidu, čím proti tobě zhřešili, i všechna jejich provinění, kterých se proti tobě dopustili. Dej jim najít slitování u těch, kdo je zajali; ať se nad nimi slitují.
#8:51 Vždyť jsou tvým lidem a tvým dědictvím, které jsi vyvedl z Egypta, z pece železné.
#8:52 Nechť jsou tvé oči otevřené k prosbě tvého služebníka i k prosbě Izraele, tvého lidu, a slyš je, kdykoli budou k tobě volat.
#8:53 Ty sám jsi je sobě oddělil za dědictví ze všech národů země, jak jsi prohlásil skrze svého služebníka Mojžíše, když jsi vyvedl naše otce z Egypta, Panovníku Hospodine.“
#8:54 Když Šalomoun dokončil svou modlitbu k Hospodinu, celou tuto modlitbu a prosbu o smilování, vstal od Hospodinova oltáře, kde klečel na kolenou s dlaněmi rozprostřenými k nebesům.
#8:55 Pak vstoje udělil mocným hlasem požehnání celému shromáždění Izraele:
#8:56 „Požehnán buď Hospodin, který podle svého slova dal odpočinutí Izraeli, svému lidu. Nezapadlo ani jedno ze všech dobrých slov, která mluvil skrze svého služebníka Mojžíše.
#8:57 Kéž je Hospodin, náš Bůh, s námi, jako byl s našimi otci! Nechť nás neopouští a neodvrhuje!
#8:58 Nechť nakloní naše srdce k sobě, abychom chodili po všech jeho cestách a dodržovali jeho přikázání, nařízení a práva, která přikázal našim otcům.
#8:59 A nechť jsou tato má slova, kterými jsem prosil o smilování před Hospodinem, blízká Hospodinu, našemu Bohu, ve dne i v noci, aby zjednával den co den právo svému služebníku i právo Izraeli, svému lidu,
#8:60 aby poznaly všechny národy země, že Hospodin je Bůh; není žádný jiný.
#8:61 Vaše srdce buď cele při Hospodinu, našem Bohu, abyste se řídili jeho nařízeními a dodržovali jeho přikázání tak jako dnes.“
#8:62 Král a s ním celý Izrael slavili před Hospodinem obětní hod.
#8:63 Šalomoun obětoval Hospodinu v oběť pokojnou dvacet dva tisíce kusů skotu a sto dvacet tisíc kusů bravu. Tak zasvětili král a všichni Izraelci Hospodinův dům.
#8:64 Onoho dne posvětil král střed nádvoří, které je před Hospodinovým domem, neboť tam přinesl oběť zápalnou, obětní dar a tučné díly pokojných obětí, protože bronzový oltář, který je před Hospodinem, byl příliš malý, než aby pojal zápalnou oběť, obětní dar i tučné díly pokojných obětí.
#8:65 V onen čas slavil Šalomoun a s ním celý Izrael slavnost, velké shromáždění před Hospodinem, naším Bohem, od cesty do Chamátu až k Egyptskému potoku; trvala sedm dní a dalších sedm dní, celkem čtrnáct dní.
#8:66 Osmého dne propustil lid a oni dobrořečili králi a šli do svých stanů radostně a s dobrou myslí pro všechno, co dobrého učinil Hospodin svému služebníku Davidovi a Izraeli, svému lidu. 
#9:1 Když Šalomoun dokončil budování Hospodinova domu i královského domu a všeho, co si s takovým zaujetím přál vykonat,
#9:2 ukázal se Hospodin Šalomounovi podruhé, jako se mu ukázal v Gibeónu.
#9:3 Hospodin mu řekl: „Vyslyšel jsem tvou modlitbu a tvou prosbu, s kterou ses na mne obrátil. Oddělil jsem jako svatý tento dům, který jsi vybudoval, a dal jsem tam spočinout svému jménu navěky. Mé oči i mé srdce tam budou po všechny dny.
#9:4 A co se tebe týče, budeš-li chodit přede mnou v bezúhonnosti srdce a přímosti, jako chodil tvůj otec David, jednat podle všeho, co jsem ti přikázal, dodržovat má nařízení a práva,
#9:5 upevním trůn tvého království nad Izraelem navěky, jak jsem přislíbil tvému otci Davidovi, že nebude z izraelského trůnu vyhlazen následník z tvého rodu.
#9:6 Jestliže se ode mne odvrátíte vy a vaši synové a nebudete dodržovat má přikázání a má nařízení, která jsem vám vydal, a půjdete sloužit jiným bohům a klanět se jim,
#9:7 vyhladím Izraele z povrchu země, kterou jsem jim dal, a zřeknu se domu, který jsem oddělil jako svatý pro své jméno. Izrael se stane pořekadlem a předmětem výsměchu mezi všemi národy.
#9:8 Nad tímto domem, ačkoli byl nejvyšší ze všech, ustrne a usykne každý kolemjdoucí. Bude se říkat: ‚Proč Hospodin takto naložil s touto zemí a s tímto domem?‘
#9:9 A bude se odpovídat: ‚Protože opustili Hospodina, svého Boha, který vyvedl jejich otce z egyptské země, a chytili se jiných bohů, klaněli se jim a sloužili jim. Proto na ně Hospodin uvedl všechno toto zlo.‘“
#9:10 Po uplynutí dvaceti let, během nichž Šalomoun stavěl oba domy, dům Hospodinův a dům královský,
#9:11 a týrský král Chíram podporoval Šalomouna cedrovým a cypřišovým dřevem a zlatem podle každého jeho přání, tehdy dal král Šalomoun Chíramovi dvacet měst v zemi galilejské.
#9:12 Chíram vyjel z Týru, aby se podíval na města, která mu Šalomoun dal, ale nezamlouvala se mu.
#9:13 Řekl tedy: „Cos mi to dal za města, můj bratře!“ A nazval je zemí Kabúl. Tak se nazývá až dodnes.
#9:14 Chíram totiž poslal králi sto dvacet talentů zlata.
#9:15 Důvodem nucených prací, které král Šalomoun ukládal, bylo, aby budoval Hospodinův dům a dům svůj, Miló a jeruzalémské hradby, Chasór, Megido a Gezer.
#9:16 Farao, egyptský král, předtím vytáhl, dobyl Gezer a vypálil jej. Kenaance, kteří v městě bydleli, pobil a město dal věnem své dceři, ženě Šalomounově.
#9:17 Šalomoun vystavěl Gezer a Dolní Bét-chorón
#9:18 v zemi a Baalat a Tadmór v poušti,
#9:19 všechna města pro sklady, které Šalomoun měl, města pro vozbu a města pro koně a vše, co s tak velkým zaujetím budoval v Jeruzalémě i na Libanónu a v celé zemi, v níž vládl.
#9:20 Všechen lid, který zbyl z Emorejců, Chetejců, Perizejců, Chivejců a Jebúsejců, kteří nebyli z Izraele,
#9:21 totiž jejich syny, kteří po nich v zemi zbyli, které Izraelci nedokázali vyhubit jako klaté, podrobil Šalomoun otrockým nuceným pracím, a tak je tomu až dodnes.
#9:22 Z Izraelců však Šalomoun neudělal otrokem nikoho; ti byli bojovníky, jeho služebníky, veliteli, tvořili osádku jeho válečných vozů a byli veliteli jeho vozby a jízdy.
#9:23 Správců představených nad pracemi pro Šalomouna bylo pět set padesát: ti panovali nad lidem konajícím tu práci.
#9:24 Tehdy, když faraónova dcera odešla z Města Davidova do svého domu, který jí Šalomoun vystavěl, začal stavět Miló.
#9:25 Třikrát za rok obětoval Šalomoun oběti zápalné a pokojné na oltáři, který vybudoval Hospodinu, a pálil na něm před Hospodinem kadidlo, když dům dokončil.
#9:26 Král Šalomoun dal také udělat lodě v Esjón-geberu, který je blízko Elótu na břehu Rákosového moře v edómské zemi.
#9:27 Na ty lodě poslal Chíram své služebníky, lodníky znalé moře, aby byli se služebníky Šalomounovými.
#9:28 Dopluli do Ofíru a přivezli odtamtud králi Šalomounovi čtyři sta dvacet talentů zlata. 
#10:1 I královna ze Sáby uslyšela zprávu o tom, co Šalomoun vykonal pro Hospodinovo jméno, a přijela ho vyzkoušet hádankami.
#10:2 Přijela do Jeruzaléma s velmi okázalým doprovodem, s velbloudy nesoucími balzámy, velké množství zlata a drahokamy. Přišla k Šalomounovi a mluvila s ním o všem, co měla na srdci.
#10:3 Šalomoun jí zodpověděl všechny její otázky; ani jedna otázka nebyla pro krále tak tajemná, že by ji nezodpověděl.
#10:4 Když královna ze Sáby viděla všechnu Šalomounovu moudrost a dům, který vybudoval,
#10:5 i jídlo na jeho stole, zasedání jeho hodnostářů, pohotovost jeho služebnictva a jejich oděvy, jeho číšníky i jeho zápalnou oběť, kterou přinášel v Hospodinově domě, zůstala bez dechu.
#10:6 Řekla králi: „Co jsem slyšela ve své zemi o tvém podnikání i o tvé moudrosti, je pravda.
#10:7 Nevěřila jsem těm slovům, dokud jsem nepřišla a nespatřila to na vlastní oči. A to mi nebyla sdělena ani polovina. Překonal jsi moudrostí a blahobytem pověst, kterou jsem slyšela.
#10:8 Blaze tvým mužům, blaze těmto tvým služebníkům, kteří jsou ustavičně v tvých službách a naslouchají tvé moudrosti.
#10:9 Požehnán buď Hospodin, tvůj Bůh, který si tě oblíbil a dosadil tě na izraelský trůn. Protože si Hospodin zamiloval Izraele navěky, ustanovil tě králem, abys zjednával právo a spravedlnost.“
#10:10 Pak dala králi sto dvacet talentů zlata, velmi mnoho balzámu a drahokamy. Už nikdy nebylo přivezeno tolik takového balzámu, jaký dala královna ze Sáby králi Šalomounovi.
#10:11 Také Chíramovy lodě, které dopravovaly z Ofíru zlato, přivezly z Ofíru velmi mnoho almugímového dřeva a drahokamy.
#10:12 Z almugímového dřeva dal král udělat zábradlí pro dům Hospodinův i pro dům královský a citary a harfy pro zpěváky. Takové almugímové dřevo už víckrát nebylo dovezeno a dodnes se nevidí.
#10:13 Král Šalomoun splnil královně ze Sáby každé přání, jež vyslovila. Navíc jí dal dary hodné krále Šalomouna. Pak se odebrala i se svými služebníky do své země.
#10:14 Váha zlata, které bylo přiváženo Šalomounovi za jeden rok, činila šest set šedesát šest talentů,
#10:15 mimo zlato od kupců a ze zisku obchodníků i od všech králů západua místodržitelů země.
#10:16 Král Šalomoun dal udělat dvě stě pavéz z tepaného zlata; na jednu pavézu vynaložil šest set šekelů zlata.
#10:17 Dále dal udělat tři sta štítů z tepaného zlata; na jeden štít vynaložil tři hřivny zlata. Král je pak dal uložit do domu Libanónského lesa.
#10:18 Král dal také udělat veliký trůn ze slonoviny a obložil jej ryzím zlatem.
#10:19 Trůn měl šest stupňů, vzadu nahoře byl zaoblen, po obou stranách sedadla měl opěradla a vedle opěradel stáli dva lvi.
#10:20 Na šesti stupních tam stálo z obou stran dvanáct lvů. Nic takového nebylo zhotoveno v žádném království.
#10:21 Všechny nádoby, ze kterých král Šalomoun pil, byly zlaté a všechny předměty domu Libanónského lesa byly potaženy lístkovým zlatem; nic nebylo ze stříbra, to v Šalomounově době nemělo cenu.
#10:22 Král měl na moři zámořské loďstvo spolu s loděmi Chíramovými. Jednou za tři roky zámořské loďstvo přijíždělo a přiváželo zlato a stříbro, slonovinu, opice a pávy.
#10:23 Král Šalomoun převýšil všechny krále země bohatstvím i moudrostí.
#10:24 Celá země vyhledávala Šalomouna, aby slyšela jeho moudrost, kterou mu Bůh vložil do srdce.
#10:25 Každý přinášel svůj dar: předměty stříbrné i zlaté, pláště, zbroj, balzámy, koně, mezky, a to rok co rok.
#10:26 Tak Šalomoun nashromáždil vozy a jezdecké koně; měl tisíc čtyři sta vozů a dvanáct tisíc jezdeckých koní. Rozmístil je ve městech pro vozbu a u sebe v Jeruzalémě.
#10:27 Král měl v Jeruzalémě stříbra jako kamení a cedrů jako planých fíků, jakých roste mnoho v Přímořské nížině.
#10:28 Koně, které měl Šalomoun, přicházeli z Egypta. Karavany králových překupníků kupovaly stáda koní za pevnou cenu.
#10:29 Vůz se vyvážel z Egypta za šest set šekelů stříbra a kůň za sto padesát. Jejich prostřednictvím se vyváželi všem králům chetejským a aramejským. 
#11:1 Král Šalomoun si zamiloval mnoho žen cizinek, faraónovu dceru, Moábky, Amónky, Edómky, Sidóňanky a Chetejky,
#11:2 z těch pronárodů, o nichž Hospodin Izraelcům řekl: „Nebudete vcházet k nim a oni nebudou vcházet k vám. Jistě by naklonili vaše srdce ke svým bohům.“ Šalomoun k nim přilnul velkou láskou.
#11:3 Měl mnoho žen: sedm set kněžen a tři sta ženin. Jeho ženy odklonily jeho srdce.
#11:4 Když nadešel Šalomounovi čas stáří, jeho ženy odklonily jeho srdce k jiným bohům, takže jeho srdce nebylo cele při Hospodinu, jeho Bohu, jako bylo srdce jeho otce Davida.
#11:5 Šalomoun chodil za božstvem Sidóňanů Aštoretou a za ohyzdnou modlou Amónců Milkómem.
#11:6 Tak se Šalomoun dopouštěl toho, co je zlé v Hospodinových očích, a neoddal se cele Hospodinu jako jeho otec David.
#11:7 Tehdy Šalomoun vystavěl posvátné návrší Kemóšovi, ohyzdné modle Moábců, na hoře, která je naproti Jeruzalému, a Molekovi, ohyzdné modle Amónovců.
#11:8 Totéž udělal pro všechny své ženy cizinky, které pálily kadidlo a obětovaly svým bohům.
#11:9 Hospodin se na Šalomouna rozhněval, že se srdcem odklonil od Hospodina, Boha Izraele, jenž se mu dvakrát ukázal
#11:10 a výslovně mu zakázal chodit za jinými bohy. Ale on nedbal na to, co Hospodin přikázal.
#11:11 Hospodin tedy řekl Šalomounovi: „Protože to s tebou dopadlo tak, že nedodržuješ mou smlouvu a má nařízení, která jsem ti přikázal, zcela jistě odtrhnu království od tebe a dám je tvému služebníku.
#11:12 Avšak za tvého života to neučiním kvůli tvému otci Davidovi. Odtrhnu je až z ruky tvého syna.
#11:13 Celé království však přece neodtrhnu, jeden kmen dám tvému synu kvůli svému služebníku Davidovi a kvůli Jeruzalému, který jsem vyvolil.“
#11:14 Hospodin vzbudil Šalomounovi protivníka v Edómci Hadadovi. Byl z edómského královského rodu.
#11:15 Když totiž David byl v Edómu a Jóab, velitel vojska, vytáhl, aby pohřbil padlé, vybil v Edómu všechny mužského pohlaví.
#11:16 Jóab a celý Izrael tam byli usazeni šest měsíců, dokud v Edómu nebyli vyhlazeni všichni mužského pohlaví.
#11:17 Hadad však uprchl a s ním někteří edómští muži ze služebnictva jeho otce a přišli do Egypta. Hadad byl tenkrát příliš mladý.
#11:18 Odešli z Midjánu a přišli do Páranu. Z Páranu s sebou vzali muže a přišli do Egypta k faraónovi, králi egyptskému. Farao mu dal dům, zaručil mu pokrm a dal mu i zemi.
#11:19 Hadad získal velkou přízeň v očích faraónových, takže mu dal za ženu sestru své manželky, sestru vznešené paní Tachpenésy.
#11:20 Sestra Tachpenésy mu porodila syna Genubata a Tachpenés ho po odstavení vychovávala přímo ve faraónově domě. Tak byl Genubat ve faraónově domě mezi faraónovými syny.
#11:21 V Egyptě Hadad uslyšel, že David ulehl ke svým otcům a že zemřel i velitel vojska Jóab. Proto Hadad řekl faraónovi: „Propusť mne, rád bych šel do své země.“
#11:22 Farao mu řekl: „Co ti u mne chybí, že chceš jít do své země?“ Odvětil: „Nic. Jenom mě propusť!“
#11:23 Bůh vzbudil Šalomounovi též protivníka v Rezónovi, synu Eljádovu, který uprchl svému pánu Hadad-ezerovi, králi Sóby.
#11:24 Shromáždil kolem sebe muže a stal se velitelem hordy, když je David začal vybíjet. Odešli do Damašku; tam se usídlili a v Damašku kralovali.
#11:25 Byl protivníkem Izraele po všechny Šalomounovy dny, spolu s tím zlem, které představoval Hadad. Zprotivil si Izraele, když kraloval nad Aramem.
#11:26 Také Jarobeám, syn Nebatův, služebník Šalomounův, Efratejec ze Seredy, jehož matka jménem Serúa byla vdovou, pozdvihl ruku proti králi.
#11:27 A důvod, proč pozdvihl ruku proti králi, je tento: Šalomoun vystavěl Miló a uzavřel trhlinu města svého otce Davida.
#11:28 Ten muž Jarobeám byl udatný bohatýr. Když Šalomoun mladíka uviděl, jak koná své dílo, ustanovil ho nad veškerou pracovní povinností domu Josefova.
#11:29 V onen čas vyšel jednou Jarobeám z Jeruzaléma. Cestou ho potkal prorok Achijáš Šíloský, zahalený do nového pláště. Byli na poli sami.
#11:30 Achijáš uchopil nový plášť, který měl na sobě, roztrhal jej na dvanáct kusů
#11:31 a řekl Jarobeámovi: „Vezmi si deset kusů! Toto praví Hospodin, Bůh Izraele: ‚Hle, já odtrhnu království z ruky Šalomounovy a tobě dám deset kmenů.
#11:32 Jemu zbude jeden kmen ze všech izraelských kmenů kvůli mému služebníku Davidovi a kvůli Jeruzalému, městu, které jsem vyvolil.
#11:33 To proto, že mě opustili, klaněli se Aštoretě, božstvu Sidóňanů, Kemóšovi, božstvu Moábců, a Milkómovi, božstvu Amónovců. Nechodili po mých cestách, neřídili se tím, co je správné v mých očích, mými nařízeními a právními ustanoveními jeho otec David.
#11:34 Ale celé království mu z ruky nevezmu, neboť jsem ho ustanovil vladykou na celý život kvůli svému služebníku Davidovi, jehož jsem vyvolil a který mé příkazy a má nařízení dodržoval.
#11:35 Vezmu království z ruky jeho syna a dám je tobě, totiž deset kmenů.
#11:36 Jeho synu dám jeden kmen, tak, aby mému služebníku Davidovi zůstalo planoucí světlo po všechny dny přede mnou v Jeruzalémě, v městě, které jsem si vyvolil, abych tam dal spočinout svému jménu.
#11:37 Tebe si však vezmu, abys kraloval nade vším, po čem tvá duše touží. Ty budeš králem nad Izraelem.
#11:38 Budeš-li poslušný ve všem, co ti přikážu, a budeš-li chodit po mých cestách a řídit se tím, co je správné v mých očích, a dodržovat má nařízení a má přikázání, jak to činil můj služebník David, budu s tebou a vybuduji ti trvalý dům, jako jsem vybudoval Davidovi, a dám ti Izraele.
#11:39 Pokořím tím Davidovo potomstvo, ale ne pro všechny dny.‘“
#11:40 Šalomoun usilovaI Jarobeáma usmrtit. Proto Jarobeám uprchl do Egypta k egyptskému králi Šíšakovi a zůstal v Egyptě až do Šalomounovy smrti.
#11:41 O ostatních příbězích Šalomounových, o všem, co konal, i o jeho moudrosti se píše, jak známo, v Knize příběhů Šalomounových.
#11:42 Šalomoun kraloval v Jeruzalémě nad celým Izraelem čtyřicet let.
#11:43 I ulehl Šalomoun ke svým otcům a byl pohřben v městě svého otce Davida. Po něm se stal králem jeho syn Rechabeám. 
#12:1 Rechabeám se odebral do Šekemu. Do Šekemu totiž přišel celý Izrael, aby ho ustanovil králem.
#12:2 Jarobeám, syn Nebatův, se o tom doslechl, když byl ještě v Egyptě, kam uprchl před králem Šalomounem; Jarobeám se totiž usadil v Egyptě.
#12:3 Poslali pro něj a povolali ho zpět. Jarobeám přišel s celým shromážděním Izraele a promluvili k Rechabeámovi:
#12:4 „Tvůj otec nás sevřel tvrdým jhem. Ulehči nám nyní tvrdou službu svého otce a těžké jho, které na nás vložil, a my ti budeme sloužit.“
#12:5 Rechabeám jim řekl: „Jděte a po třech dnech se opět ke mně vraťte.“ A lid se rozešel.
#12:6 Král Rechabeám se radil se starci, kteří byli ve službách jeho otce Šalomouna, dokud ještě žil. Ptal se: „Jak vy radíte? Co mám tomuto lidu odpovědět?“
#12:7 Promluvili k němu takto: „Jestliže se dnes tomuto lidu projevíš jako služebník a posloužíš jim, jestliže jim vyhovíš a dáš jim laskavou odpověď, budou po všechny dny tvými služebníky.“
#12:8 On však nedbal rady starců, kterou mu dávali, a radil se s mladíky, kteří s ním vyrostli a teď byli v jeho službách.
#12:9 Těch se zeptal: „Co radíte vy? Jak máme odpovědět tomuto lidu, který mi řekl: ‚Ulehči nám jho, které na nás vložil tvůj otec‘?“
#12:10 Mladíci, kteří s ním vyrostli, promluvili k němu takto: „Toto řekni tomuto lidu, který k tobě promluvil slovy: ‚Tvůj otec nás obtížil jhem, ty však nám je ulehči.‘ Promluv k nim takto: ‚Můj malík je tlustší než bedra mého otce.
#12:11 Tak tedy můj otec na vás vložil těžké jho? Já k vašemu jhu ještě přidám. Můj otec vás trestal biči, já vás však budu trestat důtkami!‘“
#12:12 Třetího dne přišel Jarobeám a všechen lid k Rechabeámovi, jak král nařídil: „Třetího dne se ke mně navraťte.“
#12:13 Král lidu odpověděl tvrdě a nedbal rady, kterou mu dali starci.
#12:14 Promluvil k nim podle rady mladíků: „Můj otec vás obtížil jhem, já k vašemu jhu ještě přidám. Můj otec vás trestal biči, já vás však budu trestat důtkami.“
#12:15 Král lid nevyslyšel, neboť to bylo řízení Hospodinovo. Tak se splnilo slovo, které Hospodin ohlásil Jarobeámovi, synu Nebatovu, skrze Achijáše Šíloského.
#12:16 Když celý Izrael uviděl, že je král nevyslyšel, dal lid králi tuto odpověď: „Jaký podíl máme v Davidovi? Nemáme dědictví v synu Jišajovu. Ke svým stanům, Izraeli! Nyní pohleď na svůj dům, Davide!“ A Izrael se rozešel ke svým stanům.
#12:17 Proto Rechabeám kraloval pouze nad Izraelci usedlými v judských městech.
#12:18 Král Rechabeám vyslal Adoráma, který byl nad nucenými pracemi, ale celý Izrael ho ukamenoval k smrti. Král Rechabeám byl nucen skočit do vozu a utéci do Jeruzaléma.
#12:19 Tak se vzbouřil Izrael proti domu Davidovu, a to trvá dodnes.
#12:20 Když se celý Izrael doslechl, že se Jarobeám navrátil, poslali pro něj a povolali ho do pospolitosti a ustanovili ho za krále nad celým Izraelem. Při domě Davidově nezůstal nikdo kromě kmene Judova.
#12:21 Když přišel Rechabeám do Jeruzaléma, shromáždil celý dům Judův a kmen Benjamínův, sto osmdesát tisíc vybraných bojovníků, aby bojovali s domem izraelským a vrátili království Rechabeámovi, synu Šalomounovu.
#12:22 I stalo se slovo Boží k Šemajášovi, muži Božímu:
#12:23 „Řekni judskému králi Rechabeámovi, synu Šalomounovu, a celému domu Judovu a Benjamínovi i zbytku lidu:
#12:24 Toto praví Hospodin: ‚Netáhněte a nebojujte proti svým bratřím Izraelcům. Každý ať se vrátí do svého domu, neboť se to stalo z mé vůle.‘“ Uposlechli tedy Hospodinova slova a vrátili se zpět, jak Hospodin nařídil.
#12:25 Jarobeám vystavěl Šekem v Efrajimském pohoří a usadil se v něm. Odtud vyšel a vystavěl Penúel.
#12:26 Pak si Jarobeám v srdci řekl: „Nyní by se mohlo království navrátit k Davidovu domu.
#12:27 Kdyby tento lid chodil slavit obětní hody do Hospodinova domu v Jeruzalémě, obrátilo by se srdce tohoto lidu k jejich pánu, judskému králi Rechabeámovi. Mne by zabili a vrátili by se k judskému králi Rechabeámovi“.
#12:28 Král se poradil a dal udělat dva zlaté býčky a řekl lidu: „Už jste se dost nachodili do Jeruzaléma. Zde jsou tvoji bohové, Izraeli, kteří tě vyvedli z egyptské země!“
#12:29 Jednoho býčka postavil v Bét-elu a druhého dal do Danu.
#12:30 To svádělo lid k hříchu. Lid chodíval za jedním z nich až do Danu.
#12:31 Jarobeám udělal též domy na posvátných návrších a nadělal ze spodiny lidu kněze, kteří nepocházeli z Léviovců.
#12:32 V osmém měsíci, patnáctého dne toho měsíce, zavedl Jarobeám svátek, podobný svátku v Judsku, a vystoupil k oltáři. Tak si počínal v Bét-elu: obětoval býčkům, které dal udělat, a ustanovil v Bét-elu kněze posvátných návrší, která nadělal.
#12:33 Vystoupil k oltáři, který dal udělat v Bét-elu, patnáctý den osmého měsíce, toho měsíce, v kterém si usmyslel zavést svátek pro Izraelce; vystoupil k oltáři pálit kadidlo. 
#13:1 Když Jarobeám stanul u oltáře, aby pálil kadidlo, přišel do Bét-elu z Judska muž Boží s Hospodinovým slovem.
#13:2 Volal proti oltáři na Hospodinův pokyn: „Oltáři, oltáři, toto praví Hospodin: ‚Hle, Davidovu domu se narodí syn jménem Jóšijáš. Ten bude na tobě obětovat kněze posvátných návrší, kteří na tobě pálí kadidlo. Budou se na tobě spalovat i lidské kosti.‘“
#13:3 Téhož dne dal i věštecké znamení: „Toto je věštecké znamení, které ohlásil Hospodin: Hle, oltář se roztrhne a popel z obětí, který je na něm, se rozsype.“
#13:4 Když uslyšel král Jarobeám slovo muže Božího, které zvolal proti oltáři v Bét-elu, vztáhl od oltáře svou ruku a poručil: „Chopte se ho!“ Ale ruka, kterou proti němu vztáhl, strnula, takže ji nemohl přitáhnout k sobě zpět.
#13:5 Oltář se roztrhl a popel z obětí se z oltáře rozsypal podle věšteckého znamení, které muž Boží učinil na Hospodinův pokyn.
#13:6 Král se na muže Božího obrátil se slovy: „Prosím, pros Hospodina, svého Boha, o shovívavost a modli se za mne, abych mohl přitáhnout ruku k sobě zpět.“ Muž Boží prosil Hospodina o shovívavost a král přitáhl ruku k sobě zpět a byla zase jako dřív.
#13:7 Král k muži Božímu promluvil: „Pojď se mnou do domu, posilni se a dám ti dar.“
#13:8 Muž Boží králi odvětil: „I kdybys mi dal polovinu svého domu, nepůjdu s tebou. Na tomto místě nebudu jíst chléb ani pít vodu,
#13:9 neboť tak mi přikázal Hospodin svým slovem: ‚Nebudeš jíst chléb ani pít vodu a nevrátíš se zpět cestou, kterou jsi šel.‘“
#13:10 Muž Boží pak odešel jinou cestou a nevracel se cestou, po níž do Bét-elu přišel.
#13:11 V Bét-elu bydlel jeden starý prorok. Jeho synové přišli a vyprávěli mu o všech činech, které toho dne vykonal muž Boží v Bét-elu. Vyprávěli svému otci i o tom, co promluvil ke králi.
#13:12 Nato se jich otec zeptal: „Kterou cestou odešel?“ Jeho synové viděli, kudy muž Boží, který přišel z Judska, odešel.
#13:13 Poručil tedy synům: „Osedlejte mi osla!“ Osedlali mu osla, on na něj nasedl
#13:14 a jel za mužem Božím. Našel jej, jak sedí pod posvátným stromem. Řekl mu: „Ty jsi ten muž Boží, který přišel z Judska?“ Odpověděl: „Jsem.“
#13:15 Nato mu řekl: „Pojď se mnou do domu a pojez chléb.“
#13:16 Muž Boží pravil: „Nemohu se s tebou vrátit, nepůjdu s tebou. Na tomto místě nebudu jíst chléb ani s tebou pít vodu,
#13:17 neboť tak mi nařídil Hospodin svým slovem: ‚Nebudeš tam jíst chléb ani pít vodu ani se nevrátíš cestou, po níž jsi šel.‘“
#13:18 On mu však řekl: „I já jsem prorok jako ty. Na Hospodinův pokyn ke mně promluvil anděl: ‚Přiveď ho s sebou zpátky do svého domu, ať pojí chléb a napije se vody.‘“ Tak ho obelhal.
#13:19 I vrátil se s ním, jedl chléb v jeho domě a pil vodu.
#13:20 Když ještě seděli za stolem, stalo se slovo Hospodinovo k proroku, který ho přivedl zpět.
#13:21 Zvolal na muže Božího, který přišel z Judska: „Toto praví Hospodin: ‚Poněvadž ses vzepřel rozkazu Hospodinovu a nedodržel jsi příkaz, který ti dal Hospodin, tvůj Bůh,
#13:22 ale vrátil ses a jedls chléb a pils vodu na místě, o němž ti řekl: Nebudeš tam jíst chléb ani pít vodu, nedostane se tvé mrtvé tělo do hrobu tvých otců.‘“
#13:23 Když prorok, kterého přivedl zpět, pojedl a napil se, osedlal mu osla
#13:24 a on odejel. Na cestě ho přepadl lev a usmrtil ho. Jeho mrtvola ležela na cestě a vedle ní stál osel; i lev stál vedle mrtvoly.
#13:25 Tu šli kolem nějací muži a viděli na cestě ležící mrtvolu i lva stojícího vedle té mrtvoly. Přišli do města, v němž bydlel ten starý prorok, a vykládali o tom.
#13:26 Když to uslyšel prorok, který ho přivedl z cesty zpět, řekl: „To je muž Boží, který se vzepřel Hospodinovu rozkazu. Hospodin jej vydal lvu a ten ho roztrhal a usmrtil podle slova, které k němu Hospodin promluvil.“
#13:27 Pak rozkázal svým synům: „Osedlejte mi osla!“ Osedlali mu ho.
#13:28 Odejel a nalezl mrtvolu ležící na cestě; vedle mrtvoly stál osel i lev. Lev nesežral mrtvolu ani neroztrhal osla.
#13:29 Prorok zvedl mrtvolu muže Božího, naložil ji na osla a vezl ji zpět. Ten starý prorok přišel do města, naříkal nad ním a pohřbil ho.
#13:30 Jeho mrtvé tělo uložil do svého hrobu a naříkali nad ním: „Běda, můj bratře!“
#13:31 Po pohřbu řekl svým synům: „Až umřu, pochovejte mě do hrobu, v němž je pohřben muž Boží. Vedle jeho kostí uložte i mé kosti.
#13:32 Jistě se stane to, co na Hospodinův pokyn volal proti oltáři v Bét-elu i proti všem domům na posvátných návrších v samařských městech.“
#13:33 Ani po této události se Jarobeám neodvrátil od své zlé cesty, ale dál dělal kněze posvátných návrší ze spodiny lidu. Kdo chtěl, toho pověřil, aby byl knězem posvátných návrší.
#13:34 Touto věcí se Jarobeámův dům prohřešil, a proto musel být zničen a vyhlazen z povrchu země. 
#14:1 V té době onemocněl Abijáš, syn Jarobeámův.
#14:2 I řekl Jarobeám své ženě: „Vstaň, přestroj se, aby se nepoznalo, že jsi žena Jarobeámova, a jdi do Šíla. Tam je prorok Achijáš, který mi ohlásil, že budu nad tímto lidem králem.
#14:3 Vezmi s sebou deset chlebů, ale rozdrobených, a lahvici medu a jdi k němu. On ti oznámí, co se s chlapcem stane.“
#14:4 Jarobeámova žena tak učinila. Vydala se na cestu do Šíla a vstoupila do Achijášova domu. Achijáš už neviděl, jeho oči byly zastřeny stářím.
#14:5 Ale Hospodin Achijášovi řekl: „Hle, přišla žena Jarobeámova dotázat se tě na slovo o svém synu, který je nemocen. Promluvíš k ní tak a tak. Až přijde, bude se vydávat za někoho jiného.“
#14:6 Sotvaže Achijáš uslyšel její kroky, když vcházela do dveří, řekl: „Vstup, ženo Jarobeámova! Proč se vydáváš za někoho jiného? Jsem k tobě poslán s tvrdým slovem.
#14:7 Jdi, řekni Jarobeámovi: Toto praví Hospodin, Bůh Izraele: ‚Vyvýšil jsem tě zprostředka lidu a učinil jsem tě vévodou nad Izraelem, svým lidem,
#14:8 odtrhl jsem království od domu Davidova a dal jsem je tobě. Ty však nejsi jako můj služebník David, který dodržoval má přikázání a následoval mě celým svým srdcem, takže konal jen to, co je správné v mých očích.
#14:9 Ty sis počínal hůře než všichni, kdo byli před tebou. Odešels a udělal sis jiné bohy, slité modly, a tak jsi mě urážel. Obrátil ses ke mně zády.
#14:10 Proto uvedu na Jarobeámův dům zlo. Vyhladím Jarobeámovi toho, jenž močí na stěnu, a v Izraeli zajatého i zanechaného. Vymetu Jarobeámův dům tak dokonale, jako se vymetá mrva.
#14:11 Kdo zemře Jarobeámovi v městě, toho sežerou psi, kdo zemře na poli, toho sežere nebeské ptactvo.‘ Tak promluvil Hospodin.
#14:12 Ty však vstaň a jdi domů. Až tvá noha vstoupí do města, dítě zemře.
#14:13 Všechen Izrael bude nad ním naříkat a pohřbí je. Jen ono samotné z Jarobeámova rodu přijde do hrobu, protože z Jarobeámova domu jen na něm nalezl Hospodin, Bůh Izraele, něco dobrého.
#14:14 Hospodin vzbudí proti Izraeli krále, který Jarobeámův dům vyhladí. Ještě dnes! Co říkám - hned teď!
#14:15 Hospodin bude bít Izraele, takže se bude klátit jako rákosí ve vodě, vyrve Izraele z dobré země, kterou dal jejich otcům, a rozptýlí je až za Řeku, protože si dělali posvátné kůly a Hospodina uráželi.
#14:16 Vydá Izraele pro Jarobeámovy hříchy, jichž se dopouštěl a jimiž svedl Izraele k hříchu.“
#14:17 Jarobeámova žena se vydala na cestu a přišla do Tirsy. Sotva vstoupila na práh domu, chlapec zemřel.
#14:18 Pohřbili ho a všechen Izrael nad ním naříkal, podle slova, které ohlásil Hospodin skrze svého služebníka, proroka Achijáše.
#14:19 O ostatních příbězích Jarobeámových, jak válčil a kraloval, se píše v Knize letopisů králů izraelských.
#14:20 Jarobeám kraloval po dobu dvaceti dvou let. I ulehl ke svým otcům. Po něm se stal králem jeho syn Nádab.
#14:21 Rechabeám, syn Šalomounův, kraloval v Judsku. Bylo mu jedenačtyřicet let, když začal kralovat, a kraloval sedmnáct let v Jeruzalémě, v městě, které vyvolil Hospodin ze všech izraelských kmenů, aby tam spočinulo jeho jméno. Jeho matka se jmenovala Naama; byla to Amónka.
#14:22 Juda činil to, co je zlé v Hospodinových očích; popouzeli ho k žárlivosti svými hříchy, jichž se dopouštěli ještě víc než jejich otcové.
#14:23 Na každém vysokém pahorku a pod každým zeleným stromem si stavěli i oni posvátná návrší, posvátné sloupy a kůly.
#14:24 V zemi bylo také plno modlářského smilstva; páchali je podle všelijakých ohavností těch pronárodů, které Hospodin před Izraelci vyhnal.
#14:25 V pátém roce vlády krále Rechabeáma vytáhl Šíšak, král egyptský, proti Jeruzalému.
#14:26 Pobral poklady domu Hospodinova i poklady domu královského; pobral všechno. Pobral i všechny zlaté štíty, které pořídil Šalomoun.
#14:27 Místo nich pořídil král Rechabeám štíty bronzové a svěřil je velitelům běžců, kteří střežili vchod do královského domu.
#14:28 Kdykoli král vcházel do Hospodinova domu, běžci je přinášeli a zase odnášeli do místnosti pro běžce.
#14:29 O ostatních příbězích Rechabeámových, o všem, co konal, se píše, jak známo, v Knize letopisů králů judských.
#14:30 Válka mezi Rechabeámem a Jarobeámem trvala po všechna ta léta.
#14:31 I ulehl Rechabeám ke svým otcům a byl pohřben vedle svých otců v Městě Davidově. Jeho matka se jmenovala Naama; byla to Amónka. Po něm se stal králem jeho syn Abijám. 
#15:1 V osmnáctém roce vlády krále Jarobeáma, syna Nebatova, se stal králem nad Judou Abijám.
#15:2 Kraloval v Jeruzalémě tři léta. Jeho matka se jmenovala Maaka; byla to dcera Abíšalómova.
#15:3 Chodil ve všech hříších svého otce, jichž se před ním dopouštěl, a jeho srdce nebylo cele při Hospodinu, jeho Bohu, jako srdce jeho otce Davida.
#15:4 Ale pro Davida mu dal Hospodin, jeho Bůh, v Jeruzalémě planoucí světlo; vzbudil po něm jeho syna a upevnil Jeruzalém.
#15:5 To proto, že David činil to, co je správné v Hospodinových očích, a po celý svůj život se neodchýlil od ničeho, co mu on přikázal, kromě té věci s Chetejcem Urijášem.
#15:6 Válka mezi Rechabeámem a Jarobeámem trvala po všechna léta Abijámova života.
#15:7 O ostatních příbězích Abijámových, o všem, co konal, se píše, jak známo, v Knize letopisů králů judských. Válka mezi Abijámem a Jarobeámem trvala dál.
#15:8 I ulehl Abijám ke svým otcům a pohřbili ho v Městě Davidově. Po něm se stal králem jeho syn Ása.
#15:9 Ve dvacátém roce vlády izraelského krále Jarobeáma se stal králem Ása, král judský.
#15:10 Kraloval v Jeruzalémě jedenačtyřicet let. Jeho matka se jmenovala Maaka; byla to dcera Abíšalómova.
#15:11 Ása činil to, co je správné v Hospodinových očích, jako jeho otec David.
#15:12 Vymýtil ze země ty, kdo se oddávali modlářskému smilstvu, a odstranil všechny hnusné modly, které udělali jeho otcové.
#15:13 Také svou matku Maaku zbavil jejího královského postavení za to, že udělala nestvůrnou modlu pro Ašéru. Ása její nestvůrnou modlu podťal a spálil v Kidrónském úvalu.
#15:14 I když nebyla odstraněna posvátná návrší, přece bylo srdce Ásovo po všechny jeho dny cele při Hospodinu.
#15:15 Svaté dary svého otce i své svaté dary, stříbro, zlato a různé předměty, přinesl do Hospodinova domu.
#15:16 Válka mezi Ásou a Baešou, králem izraelským, trvala po všechny jejich dny.
#15:17 Izraelský král Baeša vytáhl proti Judsku. Vystavěl Rámu, aby zabránil judskému králi Ásovi vycházet a vcházet.
#15:18 Ása vzal všechno stříbro a zlato, které zbylo v pokladech domu Hospodinova, i poklady domu královského a předal je svým služebníkům. Ty pak poslal král Ása k aramejskému králi Ben-hadadovi, synu Tabrimóna, syna Chezjóna, který sídlil v Damašku, se vzkazem:
#15:19 „Máme spolu smlouvu, měli ji i můj a tvůj otec. Zde ti posílám stříbro a zlato jako úplatu. Nuže, zruš svou smlouvu s izraelským králem Baešou, ať ode mne odtáhne.“
#15:20 Ben-hadad krále Ásu uposlechl a poslal velitele se svými vojsky proti izraelským městům. Vybil Ijón, Ábel-bét-maaku, celý Kinerót a celou zemi Neftalí.
#15:21 Když to Baeša uslyšel, přestal stavět Rámu. Sídlil v Tirse.
#15:22 Král Ása vyzval celé Judsko, všechny bez výjimky, aby z Rámy odnášeli kámen a dřevo, z nichž stavěl Baeša. Král Ása z toho vystavěl Gebu Benjamínovu a Mispu.
#15:23 O všech ostatních příbězích Ásových, o všech jeho bohatýrských činech, o všem, co konal, i o městech, která vystavěl, se píše, jak známo, v Knize letopisů králů judských; pouze ve stáří byl nemocen na nohy.
#15:24 I ulehl Ása ke svým otcům a byl pohřben vedle svých otců v městě svého otce Davida. Po něm se stal králem jeho syn Jóšafat.
#15:25 V druhém roce vlády judského krále Ásy se stal králem nad Izraelem Nádab, syn Jarobeámův. Kraloval nad Izraelem dva roky.
#15:26 Dopouštěl se toho, co je zlé v Hospodinových očích. Chodil po cestě svého otce, v jeho hříchu, jímž svedl k hříchu Izraele.
#15:27 Proti němu zosnoval spiknutí Baeša, syn Achijášův, z domu Isacharova. V Gibetónu, který patří Pelištejcům, ho Baeša zabil. Nádab a všechen Izrael totiž Gibetón obléhali.
#15:28 Baeša ho usmrtil v třetím roce vlády judského krále Ásy a kraloval místo něho.
#15:29 Když se ujal kralování, vybil všechen dům Jarobeámův, nezanechal Jarobeámovi nic, co mělo dech; vyhladil jej podle slova Hospodinova, které ohlásil skrze svého služebníka Achijáše Šíloského.
#15:30 To pro Jarobeámovy hříchy, jichž se dopouštěl a jimiž svedl k hříchu Izraele, pro jeho urážky, jimiž urážel Hospodina, Boha Izraele.
#15:31 O ostatních příbězích Nádabových, o všem, co konal, se píše, jak známo, v Knize letopisů králů izraelských.
#15:32 Válka mezi Ásou a izraelským králem Baešou trvala po všechny jejich dny.
#15:33 V třetím roce vlády judského krále Ásy se stal králem nad celým Izraelem Baeša, syn Achijášův. Kraloval v Tirse dvacet čtyři roky.
#15:34 Dopouštěl se toho, co je zlé v Hospodinových očích. Chodil po cestě Jarobeámově, v jeho hříchu, jímž svedl k hříchu Izraele. 
#16:1 I stalo se slovo Hospodinovo k Jehúovi, synu Chananíovu, proti Baešovi:
#16:2 „Ačkoli jsem tě pozvedl z prachu a učinil jsem tě vévodou nad Izraelem, svým lidem, chodil jsi po cestě Jarobeámově a Izraele, můj lid, jsi svedl k hříchu, takže mě svými hříchy uráželi.
#16:3 Hle, já vymetu Baešu i jeho dům a učiním s jeho domem totéž, co s domem Jarobeáma, syna Nebatova.
#16:4 Kdo zemře Baešovi ve městě, toho sežerou psi, a kdo mu zemře na poli, toho sežere nebeské ptactvo.“
#16:5 O ostatních příbězích Baešových, o tom, co konal, i o jeho bohatýrských činech se píše, jak známo, v Knize letopisů králů izraelských.
#16:6 I ulehl Baeša ke svým otcům a byl pohřben v Tirse. Po něm se stal králem jeho syn Éla.
#16:7 Tak skrze proroka Jehúa, syna Chananíova, se stalo slovo Hospodinovo k Baešovi a k jeho domu kvůli všemu, co konal a co je zlé v Hospodinových očích. Urážel ho dílem svých rukou a stal se podobný domu Jarobeámovu, který vybil.
#16:8 V dvacátém šestém roce vlády judského krále Ásy se stal králem Éla, syn Baešův. Kraloval dva roky nad Izraelem v Tirse.
#16:9 Proti němu zosnoval spiknutí jeho služebník Zimrí, velitel poloviny vozby. Éla v Tirse právě popíjel a byl opilý v domě Arsy, správce domu v Tirse.
#16:10 Zimrí vstoupil a ubil ho k smrti; bylo to ve dvacátém sedmém roce vlády judského krále Ásy. A stal se místo něho králem.
#16:11 Když jako král nastoupil na trůn, vybil celý Baešův dům. Nezanechal mu toho, jenž močí na stěnu, ani mstitele ani přítele.
#16:12 Tak Zimrí vyhladil celý Baešův dům podle Hospodinova slova, které Baešovi ohlásil skrze proroka Jehúa,
#16:13 kvůli všem Baešovým hříchům i kvůli hříchům jeho syna Ély, jichž se dopouštěli a jimiž svedli k hříchu Izraele; svými modlářskými přeludy uráželi Hospodina, Boha Izraele.
#16:14 O ostatních příbězích Élových, o všem, co konal, se píše, jak známo, v Knize letopisů králů izraelských.
#16:15 V dvacátém sedmém roce vlády judského krále Ásy se stal králem Zimrí. Kraloval v Tirse sedm dní. Lid tábořil u pelištejského Gibetónu.
#16:16 Tábořící lid uslyšel, že Zimrí zosnoval spiknutí a že dokonce ubil krále. Všechen Izrael tedy v onen den ustanovil v táboře králem nad Izraelem velitele vojska Omrího.
#16:17 Omrí odtáhl se vším Izraelem od Gibetónu a oblehli Tirsu.
#16:18 Když Zimrí viděl, že město bude dobyto, vešel do paláce královského domu, královský dům nad sebou zapálil, a tak zemřel
#16:19 pro své hříchy, které páchal, když se dopouštěl toho, co je zlé v Hospodinových očích, a chodil po cestě Jarobeámově, v jeho hříchu, jímž svedl k hříchu Izraele.
#16:20 O ostatních příbězích Zimrího a o spiknutí, které zosnoval, se píše, jak známo, v Knize letopisů králů izraelských.
#16:21 Tehdy se izraelský lid rozdvojil. Polovina lidu byla při Tibním, synu Gínatovu, a chtěla ho ustanovit králem, polovina byla při Omrím.
#16:22 Lid, který byl při Omrím, přemohl lid, který byl při Tibním, synu Gínatovu. Tibní zahynul a králem se stal Omrí.
#16:23 V třicátém prvním roce vlády judského krále Ásy se stal králem nad Izraelem Omrí. Kraloval dvanáct let, z toho v Tirse šest let.
#16:24 Od Šemera získal horu Šomerón za dva talenty stříbra. Tu horu obestavěl a město, které vystavěl, pojmenoval Šomerón (to je Samaří) podle jména pána hory Šemera.
#16:25 Omrí se dopouštěl toho, co je zlé v Hospodinových očích, horších věcí než všichni, kdo byli před ním.
#16:26 Chodil po všech cestách Jarobeáma, syna Nebatova, v jeho hříchu, jímž svedl k hříchu Izraele; svými modlářskými přeludy uráželi Hospodina, Boha Izraele.
#16:27 O ostatních příbězích Omrího, o tom, co konal, o jeho bohatýrských činech, jež konal, se píše, jak známo, v Knize letopisů králů izraelských.
#16:28 I ulehl ke svým otcům a byl pohřben v Samaří. Po něm se stal králem jeho syn Achab.
#16:29 V třicátém osmém roce vlády judského krále Ásy se stal králem nad Izraelem Achab, syn Omrího. Kraloval nad Izraelem v Samaří dvacet dva roky.
#16:30 Dopouštěl se toho, co je zlé v Hospodinových očích, více než všichni, kdo byli před ním.
#16:31 Bylo mu málo chodit v hříších Jarobeáma, syna Nebatova. Vzal si za ženu Jezábelu, dceru Etbaala, krále Sidóňanů, a chodil sloužit Baalovi a klaněl se mu.
#16:32 Postavil Baalovi oltář v Baalově domě, který vystavěl v Samaří.
#16:33 Achab také udělal posvátný kůl. Tím, čeho se dopouštěl, urážel Hospodina, Boha Izraele, víc než všichni izraelští králové, kteří byli před ním.
#16:34 V jeho dnech vystavěl Chíel Bételský Jericho. Na Abíramovi, svém prvorozeném, položil jeho základ a na Segúbovi, svém nejmladším, postavil jeho vrata podle Hospodinova slova, které ohlásil skrze Jozua, syna Núnova. 
#17:1 Elijáš Tišbejský z přistěhovalců gileádských řekl Achabovi: „Jakože živ je Hospodin, Bůh Izraele, v jehož jsem službách, v těchto letech nebude rosa ani déšť, leč na mé slovo.“
#17:2 I stalo se k němu slovo Hospodinovo:
#17:3 „Jdi odtud a obrať se na východ a skryj se u potoka Kerítu proti Jordánu.
#17:4 Z potoka budeš pít a přikázal jsem havranům, aby tě tam opatřovali potravou.“
#17:5 Šel tedy a učinil, jak řekl Hospodin. Usadil se u potoka Kerítu proti Jordánu.
#17:6 A havrani mu přinášeli chléb i maso ráno a chléb i maso večer a z potoka pil.
#17:7 Uplynula řada dnů a potok vyschl, protože v zemi nenastaly deště.
#17:8 I stalo se k němu slovo Hospodinovo:
#17:9 „Vstaň a jdi do Sarepty, jež je u Sidónu, a usaď se tam. Hle, přikázal jsem tam jedné vdově, aby tě opatřovala potravou.“
#17:10 Vstal tedy a šel do Sarepty. Přišel ke vchodu do města, a hle, jedna vdova tam sbírá dříví. Zavolal na ni: „Naber mi prosím trochu vody do nádoby, abych se napil.“
#17:11 Když ji šla nabrat, zavolal na ni: „Vezmi pro mě prosím s sebou skývu chleba.“
#17:12 Řekla: „Jakože živ je Hospodin, tvůj Bůh, nemám nic upečeno, mám ve džbánu jen hrst mouky a v láhvi trochu oleje. Hle, sbírám trochu dříví. Pak to půjdu připravit pro sebe a svého syna. Najíme se a zemřeme.“
#17:13 Elijáš jí řekl: „Neboj se. Jdi a udělej, co jsi řekla. Jen mi z toho nejdřív připrav malý podpopelný chléb a přines mi jej. Potom připravíš jídlo pro sebe a svého syna,
#17:14 neboť toto praví Hospodin, Bůh Izraele: ‚Mouka ve džbánu neubude a olej v láhvi nedojde až do dne, kdy dá Hospodin zemi déšť.‘“
#17:15 Šla a udělala, jak Elijáš řekl, a měla co jíst po mnoho dní ona i on i její dům.
#17:16 Mouka ve džbánu neubývala a olej v láhvi nedocházel podle slova Hospodinova, které ohlásil skrze Elijáše.
#17:17 Po těchto událostech onemocněl syn té ženy, paní domu. V nemoci se mu přitížilo, ba již přestal dýchat.
#17:18 Tu řekla Elijášovi: „Co ti bylo do mých věcí, muži Boží? Přišel jsi ke mně, abys mi připomněl mou nepravost a mému synu přivodil smrt?“
#17:19 On jí řekl: „Dej mi svého syna!“ Vzal jí ho z klína, vynesl jej do pokojíka na střeše, kde bydlel, a položil ho na své lože.
#17:20 Pak volal k Hospodinu: „Hospodine, můj Bože, cožpak i té vdově, u které jsem hostem, způsobíš zlo a přivodíš jejímu synu smrt?“
#17:21 Třikrát se nad dítětem sklonil a volal k Hospodinu: „Hospodine, můj Bože, ať se prosím vrátí do tohoto dítěte život!“
#17:22 Hospodin Elijášův hlas vyslyšel, do dítěte se navrátil život a ožilo.
#17:23 Elijáš dítě vzal, snesl je z pokojíka na střeše do domu, dal je jeho matce a řekl: „Pohleď, tvůj syn je živ.“
#17:24 Žena Elijášovi odpověděla: „Nyní jsem poznala, že jsi muž Boží a že slovo Hospodinovo v tvých ústech je pravdivé.“ 
#18:1 Po mnoha dnech, třetího roku, se stalo slovo Hospodinovo k Elijášovi: „Jdi a ukaž se Achabovi, chci dát zemi déšť.“
#18:2 Elijáš tedy šel, aby se ukázal Achabovi. V Samaří se rozmohl hlad.
#18:3 Achab si zavolal Obadjáše, správce domu. Obadjáš se velice bál Hospodina.
#18:4 Když dala Jezábel vyhladit Hospodinovy proroky, ujal se Obadjáš sta proroků, schoval je po padesáti mužích v jeskyni a opatřoval je chlebem a vodou.
#18:5 Achab Obadjášovi řekl: „Jdi ke všem vodním pramenům a ke všem potokům v zemi; snad se najde tráva, abychom uživili koně a mezky a nemuseli porážet dobytek.“
#18:6 Rozdělili si zemi, aby ji prošli. Achab šel sám jednou cestou a Obadjáš šel sám jinou cestou.
#18:7 Když byl Obadjáš na cestě, hle, Elijáš mu jde vstříc. Obadjáš ho zpozoroval, padl na tvář a zvolal: „Jsi to ty, Elijáši, můj pane?“
#18:8 Odvětil mu: „Jsem. Jdi a vyřiď svému pánu: Je zde Elijáš.“
#18:9 Obadjáš řekl: „Čím jsem zhřešil, že vydáváš svého služebníka do rukou Achabovi, aby mě usmrtil?
#18:10 Jakože živ je Hospodin, tvůj Bůh, není pronároda ani království, kam by můj pán nebyl poslal, aby tě vyhledal. Když řekli: ‚Není tady‘, museli v tom království nebo pronárodu odpřisáhnout, že tě nenašli.
#18:11 A ty nyní říkáš: ‚Jdi a vyřiď svému pánu: Je zde Elijáš.‘
#18:12 Stane se, že já od tebe odejdu a Hospodinův duch tě odnese nevím kam. Já to půjdu oznámit Achabovi, on tě nenajde a zabije mě. Tvůj služebník se přece od mládí bojí Hospodina!
#18:13 Což mému pánu nebylo oznámeno, co jsem udělal, když Jezábel vraždila Hospodinovy proroky? Sto mužů z Hospodinových proroků jsem schoval po padesáti mužích v jeskyni a opatřoval je chlebem a vodou.
#18:14 A ty nyní říkáš: ‚Jdi a vyřiď svému pánu: Je zde Elijáš.‘ Vždyť mě zabije!“
#18:15 Elijáš mu řekl: „Jakože živ je Hospodin zástupů, v jehož jsem službách, že se dnes před ním ukážu.“
#18:16 Obadjáš šel tedy naproti Achabovi a oznámil mu to. Achab šel Elijášovi naproti.
#18:17 Když Achab uviděl Elijáše, řekl mu: „Jsi to ty, jenž uvádíš do zkázy Izraele?“
#18:18 Ten odvětil: „Izraele neuvádím do zkázy já, ale ty a dům tvého otce tím, že opouštíte Hospodinova přikázání a ty že chodíš za baaly.
#18:19 Ale nyní zařiď, ať se ke mně shromáždí na horu Karmel celý Izrael i čtyři sta padesát Baalových proroků a čtyři sta proroků Ašéřiných, kteří jedí u stolu s Jezábelou.“
#18:20 Achab tedy obeslal všechny Izraelce a shromáždil proroky na horu Karmel.
#18:21 Tu přistoupil Elijáš ke všemu lidu a řekl: „Jak dlouho budete poskakovat na obě strany? Je-li Hospodin Bohem, následujte ho; jestliže Baal, jděte za ním!“ Lid mu neodpověděl ani slovo.
#18:22 Elijáš dále řekl lidu: „Jako Hospodinův prorok zbývám už sám, ale Baalových proroků je čtyři sta padesát.
#18:23 Ať nám dají dva býky. Oni ať si vyberou jednoho býka, ať ho rozsekají na kusy a položí na dříví, ale oheň ať nezakládají. Já udělám totéž s druhým býkem: dám ho na dříví, ale oheň nezaložím.
#18:24 Vzývejte pak jména svých bohů a já budu vzývat jmého Hospodinovo. Bůh, který odpoví ohněm, ten je Bůh.“ Všechen lid odpověděl: „To je správná řeč.“
#18:25 Elijáš vyzval Baalovy proroky: „Vyberte si jednoho býka a připravte ho první, protože vás je víc. Potom vzývejte jména svých bohů, ale oheň nezakládejte!“
#18:26 Vzali tedy býka, kterého jim dal, připravili ho a od rána až do poledne vzývali Baalovo jméno: „Baale, odpověz nám!“ Neozval se však nikdo, nikdo neodpověděl. A poskakovali u zhotoveného oltáře.
#18:27 V poledne se jim Elijáš začal posmívat: „Volejte co nejhlasitěji, vždyť je to bůh! Třeba je zamyšlen nebo má nucení anebo odcestoval. Snad spí, ať se probudí!“
#18:28 Oni volali co nejhlasitěji a zasazovali si podle svého obyčeje rány meči a oštěpy, až je polévala krev.
#18:29 Minulo poledne a oni ještě pokřikovali až do chvíle, kdy se přináší obětní dar. Neozval se však nikdo, nikdo neodpověděl, nikdo tomu nevěnoval pozornost.
#18:30 Tu řekl Elijáš všemu lidu: „Přistupte ke mně!“ Všechen lid k němu přistoupil a on opravil pobořený Hospodinův oltář.
#18:31 Vzal dvanáct kamenů podle počtu kmenů synů Jákoba, k němuž se stalo slovo Hospodinovo, že se bude jmenovat Izrael.
#18:32 Z kamenů vybudoval oltář ve jménu Hospodinově a kolem oltáře vymezil příkopem prostor pro vysetí dvou měr zrní.
#18:33 Pak narovnal dříví, rozsekal býka na kusy a položil na dříví.
#18:34 Nato řekl: „Naplňte čtyři džbány vodou a vylijte ji na zápalnou oběť a na dříví!“ Potom řekl: „Udělejte to ještě jednou.“ Oni to udělali. Znovu řekl: „Udělejte to potřetí!“ Udělali to tedy potřetí.
#18:35 Voda tekla okolo oltáře a naplnila i příkop.
#18:36 Nastal čas, kdy se přináší obětní dar. Prorok Elijáš přistoupil a řekl: „Hospodine, Bože Abrahamův, Izákův a Izraelův, ať se dnes pozná, že ty jsi Bůh v Izraeli a já tvůj služebník a že jsem učinil všechny tyto věci podle tvého slova.
#18:37 Odpověz mi, Hospodine! Odpověz mi, ať pozná tento lid, že ty, Hospodine, jsi Bůh. Ty sám obrať jejich srdce zpět k sobě.“
#18:38 I spadl Hospodinův oheň a pozřel zápalnou oběť i dříví, kameny i prsť, a vodu z příkopu vypil.
#18:39 Když to všechen lid spatřil, padli na tvář a volali: „Jen Hospodin je Bůh! Jen Hospodin je Bůh!“
#18:40 Elijáš jim poručil: „Pochytejte Baalovy proroky! Nikdo z nich ať neunikne!“ Když je pochytali, zavedl je Elijáš dolů k potoku Kíšonu a tam je pobil.
#18:41 Poté řekl Elijáš Achabovi: „Vystup vzhůru, jez a pij. Je slyšet hukot deště!“
#18:42 Achab tedy vystoupil vzhůru, aby jedl a pil. Elijáš mezitím vystoupil na vrchol Karmelu, sehnul se k zemi a vtiskl si tvář mezi kolena.
#18:43 Pak řekl svému mládenci: „Vystup a pohleď směrem k moři.“ On vystoupil, pohleděl a řekl: „Nic tam není.“ Elijáš pravil: „Opakuj to sedmkrát.“
#18:44 Když to bylo posedmé, řekl: „Hle, z moře vystupuje mráček malý jako lidská dlaň.“ Elijáš mu řekl: „Vystup vzhůru a vyřiď Achabovi: ‚Zapřáhni a jeď dolů, ať tě nestihne déšť!‘“
#18:45 A vtom už k tomu došlo. Nebe se zachmuřilo, vítr přihnal mraky a spustil se silný déšť. Achab dal zapřáhnout a jel do Jizreelu.
#18:46 Hospodinova ruka byla s Elijášem. Podkasal si bedra a běžel před Achabem, až doběhl do Jizreelu. 
#19:1 Achab oznámil Jezábele vše, co udělal Elijáš, že pobil všechny Baalovy proroky mečem.
#19:2 Jezábel poslala k Elijášovi posla se slovy: „Ať bohové udělají, co chtějí! Zítra v tento čas naložím s tebou, jako ty jsi naložil s nimi!“
#19:3 Když to Elijáš zjistil, vstal a odešel, aby si zachránil život. Přišel do Beer-šeby v Judsku a tam zanechal svého mládence.
#19:4 Sám šel den cesty pouští, až přišel k jednomu trnitému keři a usedl pod ním; přál si umřít. Řekl: „Už dost, Hospodine, vezmi si můj život, vždyť nejsem lepší než moji otcové.“
#19:5 Pak pod tím keřem ulehl a usnul. Tu se ho dotkl anděl a řekl mu: „Vstaň a jez!“
#19:6 Vzhlédl, a hle, v hlavách podpopelný chléb, pečený na žhavých kamenech, a láhev vody. Pojedl, napil se a opět ulehl.
#19:7 Hospodinův anděl se ho však dotkl podruhé a řekl: „Vstaň a jez, máš před sebou dlouhou cestu!“
#19:8 Vstal, pojedl, napil se a šel v síle onoho pokrmu čtyřicet dní a čtyřicet nocí až k Boží hoře Chorébu.
#19:9 Tam vešel do jeskyně a v ní přenocoval. Tu k němu zaznělo slovo Hospodinovo. Bůh mu řekl: „Co tu chceš, Elijáši?“
#19:10 Odpověděl: „Velice jsem horlil pro Hospodina, Boha zástupů, protože Izraelci opustili tvou smlouvu, tvé oltáře zbořili a tvé proroky povraždili mečem. Zbývám už jen sám, avšak i mně ukládají o život, jak by mě o něj připravili.“
#19:11 Hospodin řekl: „Vyjdi a postav se na hoře před Hospodinem.“ A hle, Hospodin se tudy ubírá. Před Hospodinem veliký a silný vítr rozervávající hory a tříštící skály, ale Hospodin v tom větru nebyl. Po větru zemětřesení, ale Hospodin v tom zemětřesení nebyl.
#19:12 Po zemětřesení oheň, ale Hospodin ani v tom ohni nebyl. Po ohni hlas tichý, jemný.
#19:13 Jakmile jej Elijáš uslyšel, zavinul si tvář pláštěm, vyšel a postavil se u vchodu do jeskyně. Tu mu hlas pravil: „Co tu chceš, Elijáši?“
#19:14 Odpověděl: „Velice jsem horlil pro Hospodina, Boha zástupů, protože Izraelci opustili tvou smlouvu, tvé oltáře zbořili a tvé proroky povraždili mečem. Zbývám už jen sám, avšak i mně ukládají o život, jak by mě o něj připravili.“
#19:15 Hospodin mu řekl: „Jdi, vrať se svou cestou k damašské poušti. Až tam přijdeš, pomažeš Chazaela za krále nad Aramem.
#19:16 Jehúa, syna Nimšího, pomažeš za krále nad Izraelem a Elíšu, syna Šáfatova z Ábel-mechóly, pomažeš za proroka místo sebe.
#19:17 Kdo unikne Chazaelovu meči, toho usmrtí Jehú, a kdo unikne Jehúovu meči, toho usmrtí Elíša.
#19:18 Ale zachovám v Izraeli sedm tisíc, všechny ty, jejichž kolena nepoklekla před Baalem a jejichž ústa ho nepolíbila.“
#19:19 Odešel odtud a našel Elíšu, syna Šáfatova, jak orá. Bylo před ním dvanáct spřežení a on sám při dvanáctém. Elijáš k němu přikročil a hodil na něj svůj plášť.
#19:20 Elíša opustil dobytek, rozběhl se za Elijášem a řekl: „Dovol, ať políbím otce a matku. Pak půjdu za tebou.“ On mu řekl: „Jdi a vrať se! Nezapomeň, co jsem ti učinil.“
#19:21 Obrátil se tedy od něho, vzal spřežení dobytčat a obětoval je. Maso uvařil na dříví z jejich jha a dal je lidu, a ti jedli. Potom vstal, šel za Elijášem a přisluhoval mu. 
#20:1 Ben-hadad, král aramejský, shromáždil celé své vojsko. Bylo s ním dvaatřicet králů s koni a vozy. Vytáhl a oblehl Samaří a dobýval je.
#20:2 Poslal do města k Achabovi, králi izraelskému, posly
#20:3 se vzkazem: „Toto praví Ben-hadad: Mně patří tvé stříbro i tvé zlato; mně patří i tvé ženy i tví nejlepší synové.“
#20:4 Izraelský král odpověděl: „Jak poroučíš, králi, můj pane, jsem tvůj se vším, co mám.“
#20:5 Poslové přišli znovu a řekli: „Toto praví Ben-hadad: Poslal jsem k tobě se vzkazem: Vydej mi své stříbro i své zlato, své ženy i své syny.
#20:6 A proto zítra v tento čas pošlu k tobě své služebníky. Prohledají tvůj dům i domy tvých služebníků a všechno, co je žádoucí tvým očím, zabaví a vezmou.“
#20:7 Tu svolal izraelský král všechny starší země. a řekl: „Uvažte prosím a pohleďte, jaké zlo ten člověk chystá! Poslal ke mně pro mé ženy i pro mé syny, pro mé stříbro i pro mé zlato a já jsem mu nic neodepřel.“
#20:8 Všichni starší i všechen lid mu řekli: „Neposlouchej ho a nepřivoluj mu.“
#20:9 Odpověděl tedy Ben-hadadovým poslům: „Řekněte králi, mému pánu: Všechno, kvůli čemu jsi k svému služebníku poslal poprvé, učiním, ale tuto věc učinit nemohu.“ Poslové odešli a vyřídili odpověď.
#20:10 Nato mu Ben-hadad vzkázal: „Ať se mnou bohové udělají, co chtějí, jestliže zbude ze Samaří tolik prachu, aby každý z lidu, který se mnou přitáhl, mohl jej nabrat do hrsti!“
#20:11 Izraelský král odpověděl: „Vyřiďte mu: Ať se nechlubí ten, kdo se opásává, jako ten, kdo pás odkládá!“
#20:12 Jak Ben-hadad to slovo uslyšel, právě totiž popíjel, on i králové, ve stanech, rozkázal svým služebníkům: „Nastupte!“ A nastoupili proti městu.
#20:13 I přistoupil k izraelskému králi Achabovi jeden prorok a řekl: „Toto praví Hospodin: Vidíš celé toto veliké množství? Ještě dnes ti je dám do rukou a poznáš, že já jsem Hospodin.“
#20:14 Tu řekl Achab: „Skrze koho?“ On odvětil: „Toto praví Hospodin: Skrze družinu velitelů krajů.“ Otázal se: „Kdo svede bitvu?“ Prorok odvětil: „Ty!“
#20:15 Achab dal nastoupit družině velitelů krajů: bylo jich dvě stě třicet dva. Po nich dal nastoupit veškerému lidu, všem Izraelcům; bylo jich sedm tisíc.
#20:16 Vytáhli v poledne. Ben-hadad ve stanech popíjel a opil se, on i dvaatřicet králů, kteří mu přišli na pomoc.
#20:17 Nejprve vytáhla družina velitelů krajů. Ben-hadad vyslal zvědy a ti mu oznámili, že muži vytáhli ze Samaří.
#20:18 Řekl: „Jestliže vyšli kvůli míru, zajměte je živé, a jestliže vyšli k bitvě, zajměte je také živé!“
#20:19 Z města vytáhli tito: družina velitelů krajů a za ní vojsko.
#20:20 Muž udeřil na muže a Aramejci se dali na útěk. Izrael je pronásledoval. Aramejský král Ben-hadad unikl na koni i s jízdou.
#20:21 Také izraelský král vytáhl a pobil koně, rozbil vozy a připravil Aramejcům zdrcující porážku.
#20:22 I přistoupil k izraelskému králi prorok a řekl mu: „Jdi a vzchop se! Uvaž a pohleď, co učiníš, až proti tobě na přelomu roku vytáhne aramejský král.“
#20:23 Aramejskému králi služebníci řekli: „Jejich Bůh je Bohem hor, proto nás přemohli. Ale budeme-li s nimi bojovat na rovině, jistě přemůžeme my je.
#20:24 Učiň tuto věc: Odvolej ty krále z jejich míst a na jejich místa dosaď místodržitele.
#20:25 A pořiď si vojsko, jako bylo to, co padlo, rovněž koně za koně a vozy za vozy, a budeme bojovat proti Izraelcům na rovině a jistě je přemůžeme.“ I uposlechl je a učinil tak.
#20:26 Na přelomu roku vykonal Ben-hadad přehlídku Aramejců a vytáhl k Afeku do boje s Izraelem.
#20:27 Také Izraelci vykonali přehlídku, opatřili si potravu a vytáhli proti nim. Izraelci se před nimi utábořili jako dva houfy koz, kdež to Aramejci zaplavili zemi.
#20:28 Tu přistoupil muž Boží a řekl izraelskému králi: „Toto praví Hospodin: Protože Aramejci řekli: ‚Hospodin je Bohem hor a nikoli Bohem dolin‘, dám celý tento veliký hlučící dav do tvých rukou a poznáte, že já jsem Hospodin.“
#20:29 I tábořili jedni proti druhým sedm dní. Sedmého dne došlo k boji a Izraelci Aramejce pobili, sto tisíc pěšáků v jednom dni.
#20:30 Zbylí utíkali směrem k Afeku. U města se zřítily hradby na dvacet sedm tisíc zbylých mužů. Ben-hadad utekl, dostal se do města a skryl se v nejzazším pokojíku.
#20:31 Jeho služebníci mu řekli: „Slyšeli jsme, že králové izraelského domu jsou králové milosrdní. Vložme si na bedra žíněné roucho a na hlavu provaz a vyjděme k izraelskému králi, snad tě zachová při životě.“
#20:32 Opásali si tedy bedra žíněným rouchem a na hlavu si vložili provaz. Přišli k izraelskému králi a řekli: „Tvůj služebník Ben-hadad vzkazuje: ‚Zachovej mě při životě.‘“ Achab se tázal: „Je ještě živ? Je to můj bratr!“
#20:33 Ti muži to pochopili jako příznivé znamení od něho a rychle řekli: „Ovšem, Ben-hadad je tvůj bratr!“ Achab jim řekl: „Jděte a přiveďte ho.“ Ben-hadad k němu vyšel a on ho vzal k sobě do vozu.
#20:34 Ben-hadad mu řekl: „Města, která můj otec vzal tvému otci, vrátím. Také si můžeš zřídit ulice v Damašku, jako si je zřídil můj otec v Samaří.“ „A já,“ řekl Achab, „na základě této smlouvy tě propustím.“ Uzavřel s ním smlouvu a propustil ho.
#20:35 Jeden muž z prorockých žáků řekl na Hospodinovo slovo svému druhovi: „Zbij mě!“ Ale ten muž ho odmítl zbít.
#20:36 I řekl mu: „Protože jsi neuposlechl Hospodinova hlasu, hle, až půjdeš ode mne, zadáví tě lev.“ Když od něho odešel, přepadl ho lev a zadávil ho.
#20:37 Prorok vyhledal jiného muže a řekl mu: „Zbij mě!“ Ten muž ho důkladně zbil a zranil.
#20:38 Prorok šel a postavil se králi do cesty. Aby ho nepoznali, zakryl si oči páskou.
#20:39 Když jel král mimo, vykřikl na něho: „Tvůj služebník vyšel do boje. Vtom kdosi vybočil, přivedl ke mně muže a řekl: ‚Tohoto muže hlídej; ztratí-li se, dáš svůj život za jeho život nebo zaplatíš talent stříbra.‘
#20:40 Ač tvůj služebník dělal to i ono, zajatec zmizel.“ Izraelský král mu řekl: „Vynesl jsi rozsudek sám nad sebou.“
#20:41 Tu prorok rychle sňal pásku z očí a izraelský král poznal že je to jeden z proroků.
#20:42 Řekl králi: „Toto praví Hospodin: Protože jsi propustil muže propadlého klatbě, dáš svůj život za jeho život a svůj lid za jeho lid.“
#20:43 Nato odjel izraelský král do svého domu rozmrzelý a podrážděný. Tak přijel do Samaří. 
#21:1 Po těchto událostech se stalo toto: Nábot Jizreelský měl vinici v Jizreelu vedle paláce samařského krále Achaba.
#21:2 Achab promluvil s Nábotem: „Dej mi svou vinici, chci z ní mít zelinářskou zahradu, protože je blízko mého domu. Dám ti za ni lepší vinici anebo, chceš-li raději, vyplatím ti její kupní cenu ve stříbře.“
#21:3 Nábot řekl Achabovi: „Chraň mě Hospodin, abych ti dal dědictví po svých otcích.“
#21:4 Achab vstoupil do svého domu rozmrzelý a podrážděný tím, jak s ním Nábot Jizreelský mluvil, když řekl: „Dědictví po svých otcích ti nedám.“ Ulehl na lože, odvrátil tvář, ani chléb nepojedl.
#21:5 Přišla k němu Jezábel, jeho žena, a promluvila k němu: „Čím to je, že je tvůj duch rozmrzelý a ani chleba nejíš?“
#21:6 Odpověděl jí: „Mluvil jsem s Nábotem Jizreelským a řekl jsem mu: ‚Dej mi svou vinici za stříbro nebo, přeješ-li si, dám ti za ni jinou vinici.‘ Ale on mi řekl: ‚Svou vinici ti nedám.‘“
#21:7 Jezábel, jeho žena, mu řekla: „Teď ukážeš svou královskou moc nad Izraelem! Vstaň, pojez chleba a buď dobré mysli. Já sama ti dám vinici Nábota Jizreelského.“
#21:8 Pak napsala Achabovým jménem dopisy, zapečetila jeho pečetí a poslala je starším a šlechticům, těm, kteří byli v jeho městě a bydlili s Nábotem.
#21:9 V dopisech psala: „Vyhlaste půst a posaďte Nábota do čela lidu.
#21:10 Proti němu posaďte dva muže ničemníky a ti ať vydají svědectví, že zlořečil Bohu a králi. Pak ho vyveďte a ukamenujte k smrti.“
#21:11 Mužové jeho města, starší a šlechticové, ti, kteří bydleli v jeho městě, vykonali, co jim Jezábel vzkázala, jak bylo psáno v dopisech, které jim poslala:
#21:12 Vyhlásili půst a posadili Nábota do čela lidu.
#21:13 Pak přišli dva muži ničemníci a posadili se proti němu. Ti ničemníci vydali před lidem proti Nábotovi svědectví, že zlořečil Bohu a králi. I vyvedli ho ven z města a ukamenovali ho k smrti.
#21:14 Poté vzkázali Jezábele: „Nábot byl ukamenován k smrti.“
#21:15 Když Jezábel uslyšela, že Nábot byl ukamenován a zemřel, řekla Achabovi: „Vstaň a zaber vinici Nábota Jizreelského, kterou ti odmítl dát za stříbro. Nábot už nežije, je mrtev.“
#21:16 Jakmile Achab uslyšel, že Nábot je mrtev, vstal, sestoupil do vinice Nábota Jizreelského a zabral ji.
#21:17 I stalo se slovo Hospodinovo k Elijáši Tišbejskému:
#21:18 „Vstaň a jdi vstříc Achabovi, králi izraelskému ze Samaří. Je právě v Nábotově vinici; šel tam, aby ji zabral.
#21:19 Promluv k němu: ‚Toto praví Hospodin: Zavraždil jsi a teď si zabíráš.‘ Ohlas mu: ‚Toto praví Hospodin: Na místě, kde psi chlemtali krev Nábotovu, budou psi chlemtat i tvoji krev.‘“
#21:20 Achab řekl Elijášovi: „Přece jsi mě našel, můj nepříteli?“ On řekl: „Našel, protože ses zaprodal a dopouštíš se toho, co je zlé v Hospodinových očích.
#21:21 Hle, praví Hospodin, uvedu na tebe zlo, vymetu ty, kdo přijdou po tobě, vyhladím Achabovi toho, jenž močí na stěnu, a v Izraeli zajatého i zanechaného.
#21:22 A dopustím na tvůj dům totéž, co na dům Jarobeáma, syna Nebatova, a na dům Baeši, syna Achijášova, za urážku, jíž jsi mě urazil a svedl k hříchu Izraele.“
#21:23 Také o Jezábele promluvil Hospodin: „Jezábelu sežerou na valech Jizreelu psi.
#21:24 Kdo zemře Achabovi ve městě, toho sežerou psi, a kdo zemře na poli, toho sežere nebeské ptactvo.“
#21:25 Nebyl nikdo jako Achab, aby se zaprodal a dopouštěl toho, co je zlé v Hospodinových očích, jak ho k tomu podněcovala Jezábel, jeho žena.
#21:26 Jednal velice ohavně tím, že chodil za hnusnými modlami; páchal všechno to, co Emorejci, které Hospodin před Izraelci vyhnal.
#21:27 Jakmile Achab uslyšel tato slova, roztrhl svůj šat, přehodil přes sebe žíněné roucho, postil se a spával v žíněném rouchu a chodil zkroušeně.
#21:28 I stalo se slovo Hospodinovo k Elijášovi Tišbejskému:
#21:29 „Viděl jsi, že se Achab přede mnou pokořil? Protože se přede mnou pokořil, nedopustím to zlo za jeho dnů, ale uvedu je na jeho dům za dnů jeho syna.“ 
#22:1 Po tři roky nedošlo mezi Aramem a Izraelem k válce.
#22:2 V třetím roce sestoupil Jóšafat, král judský, ke králi izraelskému.
#22:3 Izraelský král řekl svým služebníkům: „Víte, že Rámot v Gileádu patří nám? A my otálíme vzít jej z rukou aramejského krále.“
#22:4 Pak se otázal Jóšafata: „Půjdeš se mnou do války o Rámot v Gileádu?“ Jóšafat izraelskému králi odvětil: „Jsme jedno, já jako ty, můj lid jako tvůj lid, mí koně jako tví koně.“
#22:5 Jóšafat izraelskému králi řekl: „Dotaž se ještě dnes na slovo Hospodinovo.“
#22:6 Izraelský král shromáždil proroky, na čtyři sta mužů, a řekl jim: „Mám jít do války proti Rámotu v Gileádu nebo mám od toho upustit?“ Odpověděli: „Jdi, Panovník jej vydá králi do rukou.“
#22:7 Ale Jóšafat se zeptal: „Cožpak tu už není žádný prorok Hospodinův, abychom se dotázali skrze něho?“
#22:8 Izraelský král Jóšafatovi odpověděl: „Je tu ještě jeden muž, skrze něhož bychom se mohli dotázat Hospodina, ale já ho nenávidím, protože mi neprorokuje nic dobrého, nýbrž jen zlo. Je to Míkajáš, syn Jimlův.“ Jóšafat řekl: „Nechť král tak nemluví!“
#22:9 Izraelský král tedy povolal jednoho dvořana a řekl: „Rychle přiveď Míkajáše, syna Jimlova.“
#22:10 Král izraelský i Jóšafat, král judský, seděli každý na svém trůnu, slavnostně oblečeni, na humně u vchodu do samařské brány a všichni proroci před nimi prorokovali.
#22:11 Sidkijáš, syn Kenaanův, si udělal železné rohy a říkal: „Toto praví Hospodin: ‚Jimi budeš trkat Arama, dokud neskoná.‘“
#22:12 Tak prorokovali všichni proroci: „Vytáhni proti Rámotu v Gileádu. Budeš mít úspěch. Hospodin jej vydá králi do rukou!“
#22:13 Posel, který šel zavolat Míkajáše, mu domlouval: „Hle, slova proroků jedněmi ústy ohlašují králi dobré věci. Ať je tvá řeč jako řeč každého z nich; mluv o dobrých věcech.“
#22:14 Míkajáš odpověděl: „Jakože živ je Hospodin, budu mluvit to, co mi řekne Hospodin.“
#22:15 Když přišel ke králi, král mu řekl: „Míkajáši, máme jít do války proti Rámotu v Gileádu, nebo máme od toho upustit?“ On mu odpověděl: „Vytáhni, budeš mít úspěch. Hospodin jej vydá králi do rukou.“
#22:16 Král ho okřikl: „Kolikrát tě mám zapřísahat, abys mi v Hospodinově jménu nemluvil nic než pravdu?“
#22:17 Míkajáš odpověděl: „Viděl jsem všechen izraelský lid rozptýlený po horách jako ovce, které nemají pastýře. Hospodin řekl: ‚Zůstali bez pánů, ať se každý v pokoji vrátí domů.‘“
#22:18 Izraelský král řekl Jóšafatovi: „Neřekl jsem ti, že mi nebude prorokovat nic dobrého, nýbrž jen zlo?“
#22:19 Ale Míkajáš pokračoval: „Tak tedy slyš slovo Hospodinovo. Viděl jsem Hospodina, sedícího na trůně. Všechen nebeský zástup stál před ním zprava i zleva.
#22:20 Hospodin řekl: ‚Kdo zláká Achaba, aby vytáhl a padl u Rámotu v Gileádu?‘ Ten říkal to a druhý ono.
#22:21 Tu vystoupil jakýsi duch, postavil se před Hospodinem a řekl: ‚Já ho zlákám.‘ Hospodin mu pravil: ‚Čím?‘
#22:22 On odvětil: ‚Vyjdu a stanu se zrádným duchem v ústech všech jeho proroků.‘ Hospodin řekl: ‚Ty ho zlákáš, ty to dokážeš. Jdi a učiň to!‘
#22:23 A nyní, hle, Hospodin dal zrádného ducha do úst všech těchto tvých proroků. Hospodin ti ohlásil zlé věci.“
#22:24 Sidkijáš, syn Kenaanův, přistoupil, uhodil Míkajáše do tváře a řekl: „Kudy přešel Hospodinův duch ode mne, aby mluvil skrze tebe?“
#22:25 Míkajáš řekl: „To uvidíš v onen den, až vejdeš do nejzazšího pokojíku, aby ses ukryl.“
#22:26 Izraelský král nařídil: „Seber Míkajáše a odveď ho k veliteli města Ámonovi a ke královu synu Jóašovi.
#22:27 Vyřídíš: ‚Toto praví král: Tady toho vsaďte do vězení a dávejte mu jen kousek chleba a trochu vody, dokud nepřijdu v pokoji.‘“
#22:28 Ale Míkajáš řekl: „Jestliže se vrátíš zpět v pokoji, nemluvil skrze mne Hospodin.“ A připojil: „Slyšte to, všichni lidé!“
#22:29 Král izraelský i Jóšafat, král judský, vytáhli proti Rámotu v Gileádu.
#22:30 Izraelský král řekl Jóšafatovi, že se přestrojí, až vyjede do boje. Dodal: „Ty ovšem obleč své roucho.“ Tak se izraelský král přestrojil a vyjel do boje.
#22:31 Aramejský král přikázal třiceti dvěma velitelům své vozby: „Nebojujte ani s malým ani s velkým, ale jen se samým izraelským králem!“
#22:32 Když velitelé vozby spatřili Jóšafata, řekli si: „To je jistě izraelský král!“ Odbočili, aby bojovali proti němu. Tu Jóšafat vyrazil válečný pokřik.
#22:33 Když velitelé vozby viděli, že to není izraelský král, odvrátili se od něho.
#22:34 Kdosi však bezděčně napjal luk a zasáhl izraelského krále mezi články pancíře. Král řekl svému vozataji: „Obrať a odvez mě z bojiště, jsem raněn.“
#22:35 Ale boj se toho dne tak vystupňoval, že král musel zůstat na voze proti Aramejcům; večer pak zemřel. Krev z rány vytékala do korby vozu.
#22:36 Při západu slunce se táborem rozlehl pokřik: „Každý do svého města, každý do své země!“
#22:37 Král tedy zemřel a byl dopraven do Samaří. V Samaří krále pohřbili.
#22:38 Když oplachovali vůz v samařském rybníku, chlemtali psi jeho krev a nevěstky se v ní omývaly, podle slova Hospodinova, které ohlásil.
#22:39 O ostatních příbězích Achabových, o všem, co konal, o domě ze slonoviny, který vystavěl, a o všech městech, která vystavěl, se píše, jak známo, v Knize letopisů králů izraelských.
#22:40 I ulehl Achab ke svým otcům. Po něm se stal králem jeho syn Achazjáš.
#22:41 Ve čtvrtém roce vlády izraelského krále Achaba se stal králem nad Judou Jóšafat, syn Ásův.
#22:42 Jóšafatovi bylo třicet pět let, když začal kralovat, a kraloval v Jeruzalémě dvacet pět let. Jeho matka se jmenovala Azúba; byla to dcera Šilchího.
#22:43 Chodil po všech cestách svého otce Ásy. Neodchýlil se od nich, ale činil to, co je správné v Hospodinových očích.
#22:44 Jenom neodstranili posvátná návrší; lid na posvátných návrších ještě obětoval a pálil kadidlo.
#22:45 Jóšafat také žil v pokoji s králem izraelským.
#22:46 O ostatních příbězích Jóšafatových, o jeho bohatýrských činech, jež konal, a jak bojoval, se píše, jak známo, v Knize letopisů králů judských.
#22:47 Zbytek těch, kdo provozovali modlářské smilstvo a zůstali ještě ve dnech jeho otce Ásy, vymýtil ze země.
#22:48 V Edómu tenkrát neměli krále, byl tam královský správce.
#22:49 Jóšafat dal udělat zámořské lodě, aby pluly do Ofíru pro zlato, ale nevyplul; lodě ztroskotaly v Esjón-geberu.
#22:50 Achazjáš, syn Achabův, tehdy řekl Jóšafatovi: „Ať se plaví na lodích s tvými služebníky i moji služebníci!“ Ale Jóšafat nesvolil.
#22:51 I ulehl Jóšafat ke svým otcům a byl pohřben vedle svých otců v městě svého otce Davida. Po něm se stal králem jeho syn Jóram.
#22:52 V sedmnáctém roce vlády judského krále Jóšafata se stal králem nad Izraelem v Samaří Achazjáš, syn Achabův. Kraloval nad Izraelem dva roky.
#22:53 Dopouštěl se toho, co je zlé v Hospodinových očí. Chodil po cestě svého otce a své matky i po cestě Jarobeáma, syna Nebatova, který svedl Izraele k hříchu.
#22:54 Sloužil Baalovi a klaněl se mu a urážel Hospodina, Boha Izraele; všechno konal tak jako jeho otec.  

\book{II Kings}{2Kgs}
#1:1 Po Achabově smrti odpadl Moáb od Izraele.
#1:2 Achazjáš propadl mříží svého pokojíku na střeše v Samaří a churavěl. Vyslal posly a řekl jim: „Jděte a dotažte se Baal-zebúba, boha Ekrónu, zdali z tohoto ochoření vyváznu živ.“
#1:3 Tu promluvil anděl Hospodinův k Elijášovi Tišbejskému: „Vstaň a vyjdi vstříc poslům samařského krále a pověz jim: Což není Bůh v Izraeli, že se jdete dotazovat Baal-zebúba, boha Ekrónu?
#1:4 Proto Hospodin praví toto: Z lože, na něž jsi ulehl, nepovstaneš, ale zcela jistě zemřeš.“ A Elijáš šel.
#1:5 Když se poslové vrátili ke králi, zeptal se jich: „Jak to, že se vracíte?“
#1:6 Odvětili mu: „Jakýsi muž nám vyšel vstříc a řekl nám: ‚Jděte zpět ke králi, který vás vyslal, a vyřiďte mu: Toto praví Hospodin: Což není Bůh v Izraeli, že se obracíš s dotazem na Baal-zebúba, boha Ekrónu? Proto z lože, na nějž jsi ulehl, nepovstaneš, ale zcela jistě zemřeš.‘“
#1:7 Otázal se jich: „Jak vypadal ten muž, který vám vyšel vstříc a mluvil k vám tato slova?“
#1:8 Oni mu řekli: „Byl to muž v chlupatém plášti a bedra měl opásaná koženým pásem.“ Tu řekl: „To byl Elijáš Tišbejský.“
#1:9 Poslal pro něho velitele s padesáti vojáky. Ten k němu vystoupil, a hle, Elijáš seděl na vrcholu hory. Promluvil k němu: „Muži Boží, král rozkazuje: ‚Sestup!‘“
#1:10 Elijáš veliteli těch padesáti odpověděl: „Jsem-li muž Boží, ať sestoupí oheň z nebe a pozře tebe i tvých padesát!“ I sestoupil oheň z nebe a pozřel jej i jeho padesát.
#1:11 Král pro něho poslal opět jiného velitele s padesáti vojáky. Ten promluvil: „Muži Boží, toto praví král: ‚Rychle sestup!‘
#1:12 Elijáš jim odpověděl: „Jsem-li muž Boží, ať sestoupí oheň z nebe a pozře tebe i tvých padesát!“ I sestoupil Boží oheň z nebe a pozřel jej i jeho padesát.
#1:13 Ale král poslal ještě potřetí velitele s padesáti vojáky. Když tento třetí velitel přišel nahoru, poklekl před Elijášem a prosil ho o smilování. Promluvil k němu: „Muži Boží, kéž má v tvých očích život můj i život těchto tvých padesáti služebníků nějakou cenu.
#1:14 Hle, oheň sestoupil z nebe a pozřel oba předešlé velitele i jejich padesát vojáků. Kéž má nyní můj život v tvých očích nějakou cenu!“
#1:15 Tu promluvil k Elijášovi Hospodinův anděl: „Sestup s ním. Neboj se ho!“ Povstal tedy, sestoupil s ním ke králi
#1:16 a promluvil k němu: „Toto praví Hospodin: Že jsi poslal posly s dotazem k Baal-zebúbovi, bohu Ekrónu, jako by nebyl Bůh v Izraeli, aby ses dotázal na jeho slovo, proto z lože, na něž jsi ulehl, nepovstaneš, ale zcela jistě zemřeš.“
#1:17 I zemřel podle Hospodinova slova, které promluvil Elijáš. Po něm se stal králem Jóram, a to v druhém roce vlády judského krále Jórama, syna Jóšafatova. Achazjáš totiž neměl syna.
#1:18 O ostatních příbězích Achazjášových, o tom, co konal, se píše, jak známo, v Knize letopisů králů izraelských. 
#2:1 I stalo se, když Hospodin chtěl vzít Elijáše ve vichru vzhůru do nebe, že Elijáš s Elíšou se právě ubírali z Gilgálu.
#2:2 Elijáš řekl Elíšovi: „Zůstaň zde, protože mě Hospodin posílá do Bét-elu.“ Elíša mu odvětil: „Jakože živ je Hospodin a jakože živ jsi ty, neopustím tě.“ I sestoupili do Bét-elu.
#2:3 Proročtí žáci, kteří byli v Bét-elu, vyšli k Elíšovi a otázali se ho: „Víš, že Hospodin dnes vezme tvého pána od tebe vzhůru?“ Odvětil: „Vím to také. Mlčte!“
#2:4 A Elijáš mu řekl: „Elíšo, zůstaň zde, protože Hospodin mě posílá do Jericha.“ Odvětil: „Jakože živ je Hospodin a jakože živ jsi ty, neopustím tě.“ I přišli do Jericha.
#2:5 Proročtí žáci, kteří byli v Jerichu, přistoupili k Elíšovi a řekli mu: „Víš, že Hospodin dnes vezme tvého pána od tebe vzhůru?“ Odvětil: „Vím to také. Mlčte!“
#2:6 Tu mu řekl Elijáš: „Zůstaň zde, protože Hospodin mě posílá k Jordánu.“ Odvětil: „Jakože živ je Hospodin a jakože živ jsi ty, neopustím tě.“ I šli oba spolu.
#2:7 Padesát mužů z prorockých žáků šlo za nimi a postavilo se naproti nim vpovzdálí; a oni oba stáli u Jordánu.
#2:8 Elijáš vzal svůj plášť, svinul jej, udeřil do vody a ta se rozestoupila, takže oba přešli po suchu.
#2:9 A stalo se, jak přešli, že Elijáš řekl Elíšovi: „Požádej, co mám pro tebe udělat, dříve než budu od tebe vzat.“ Elíša řekl: „Ať je na mně dvojnásobný díl tvého ducha!“
#2:10 Elijáš mu řekl: „Těžkou věc si žádáš. Jestliže mě uvidíš, až budu od tebe brán, stane se ti tak. Jestliže ne, nestane se.“
#2:11 Pak šli dál a rozmlouvali. A hle, ohnivý vůz s ohnivými koni je od sebe odloučil a Elijáš vystupoval ve vichru do nebe.
#2:12 Elíša to viděl a vykřikl: „Můj otče! Můj otče! Vozataji Izraele!“ A pak už ho neviděl. I uchopil své roucho a roztrhl je na dva kusy.
#2:13 Pak zdvihl Elijášův plášť, který z něho spadl, vrátil se a postavil se na břehu Jordánu.
#2:14 Vzal Elijášův plášť, který z něho spadl, udeřil jím do vody a zvolal: „Kde je Hospodin, Bůh Elijášův, i on sám?“ Když udeřil do vody, rozestoupila se a Elíša přešel.
#2:15 Viděli to proročtí žáci z Jericha, kteří byli naproti, a řekli: „Na Elíšovi spočinul duch Elijášův.“ Šli mu vstříc, poklonili se mu až k zemi
#2:16 a řekli mu: „Hle, je tu s tvými služebníky padesát udatných mužů. Ať jdou hledat tvého pána, aby ho Hospodinův duch neodnesl a nemrštil jím na nějakou horu nebo do nějakého údolí.“ Řekl: „Nikam je neposílejte.“
#2:17 Když ho však nutili až do omrzení, řekl: „Pošlete je.“ Oni tedy poslali těch padesát mužů; ti hledali Elijáše tři dny, ale nenašli ho.
#2:18 Pak se vrátili k Elíšovi, který zůstal v Jerichu. Řekl jim: „Což jsem vám neřekl: Nikam nechoďte?“
#2:19 Mužové města řekli Elíšovi: „Hle, v městě se dobře sídlí, jak náš pán vidí, ale voda je zlá, takže země trpí neplodností.“
#2:20 I řekl: „Podejte mi novou mísu a dejte do ní sůl.“ Podali mu ji tedy.
#2:21 Pak vyšel k místu, kde voda vyvěrala, hodil tam sůl a řekl: „Toto praví Hospodin: Uzdravuji tuto vodu, už nikdy odtud nevzejde smrt či neplodnost.“
#2:22 A voda je zdravá až dodnes podle Elíšova slova, které promluvil.
#2:23 Odtud vystoupil do Bét-elu. Když byl na cestě, vyšli z města malí chlapci, pošklebovali se mu a pokřikovali na něj: „Táhni, ty s lysinou, táhni, ty s lysinou!“
#2:24 On se obrátil, podíval se na ně a ve jménu Hospodinově jim zlořečil. Vtom vyběhly z křovin dvě medvědice a roztrhaly z nich čtyřicet dvě děti.
#2:25 Odtud odešel na horu Karmel a odtud se vrátil do Samaří. 
#3:1 V osmnáctém roce vlády judského krále Jóšafata se stal králem nad Izraelem v Samaří Jóram, syn Achabův. Kraloval dvanáct let.
#3:2 Dopouštěl se toho, co je zlé v Hospodinových očích, i když ne tolik jako jeho otec a matka. Odstranil Baalův posvátný sloup, který udělal jeho otec.
#3:3 Avšak lpěl na hříších Jarobeáma, syna Nebatova, jimiž svedl Izraele k hříchu, a neupustil od nich.
#3:4 Méša, král moábský, byl drobopravec. Odváděl izraelskému králi sto tisíc jehňat a sto tisíc beranů s vlnou.
#3:5 Po Achabově smrti odpadl moábský král od krále izraelského.
#3:6 Onoho dne vytáhl král Jóram ze Samaří a povolal do zbraně celý Izrael.
#3:7 Pak vzkázal Jóšafatovi, králi judskému: „Moábský král ode mne odpadl. Půjdeš se mnou do války proti Moábu?“ On řekl: „Potáhnu. Jsme jedno, já jako ty, můj lid jako tvůj lid, mí koně jako tví koně.“
#3:8 A otázal se: „Kterou cestou potáhneme?“ On odvětil: „Směrem k Edómské poušti.“
#3:9 Šli tedy, král izraelský a král judský i král edómský. Po sedmi dnech cesty oklikami, jimiž se ubírali, neměl vodu ani tábor ani dobytek v zadním voji.
#3:10 Izraelský král řekl: „Běda! Hospodin povolal nás, tyto tři krále, aby nás vydal do rukou Moábcům!“
#3:11 Ale Jóšafat se zeptal: „Což tu není žádný prorok Hospodinův, abychom se skrze něho dotázali Hospodina?“ Jeden ze služebníků izraelského krále odpověděl: „Je zde Elíša, syn Šáfatův, který líval vodu na ruce Elijášovy.“
#3:12 Jóšafat řekl: „U něho je Hospodinovo slovo.“ Tak se k němu vypravili, izraelský král a Jóšafat i král edómský.
#3:13 Elíša řekl izraelskému králi: „Co je mi do tvých věcí? Jdi si k prorokům svého otce a k prorokům své matky!“ Izraelský král mu odvětil: „Nikdy! Hospodin povolal nás, tyto tři krále, aby nás vydal do rukou Moábcům.“
#3:14 Elíša nato řekl: „Jakože živ je Hospodin zástupů, v jehož službách stojím, kdybych neměl ohled na Jóšafata, krále judského, tebe bych si ani nevšiml, ani bych se na tebe nepodíval.
#3:15 Ale nyní mi přiveďte hudebníka.“ Když hudebník hrál, byla nad Elíšou Hospodinova ruka.
#3:16 Řekl: „Toto praví Hospodin: Udělej v tomto úvalu prohlubeň vedle prohlubně.
#3:17 Neboť toto praví Hospodin: Nepocítíte vítr a neuvidíte déšť, a přesto se tento úval naplní vodou a budete pít vy i vaše stáda a váš dobytek.
#3:18 Ale to vše je v Hospodinových očích málo. Vydá vám Moábce do rukou.
#3:19 Vybijete každé opevněné město, všechna nejlepší města, vykácíte všechno dobré stromoví a zasypete všechny vodní prameny a každý dobrý díl půdy znehodnotíte kamením.“
#3:20 I stalo se za jitra, když se přináší obětní dar, hle, směrem od Edómu začaly přicházet vody a země se naplnila vodou.
#3:21 Všichni Moábci uslyšeli, že králové proti nim vytáhli do boje. Dali svolat všechny, kdo byli schopní opásat se zbrojí, i starší, a zaujali postavení na pomezí.
#3:22 Za časného jitra, když slunce vycházelo nad vodami, spatřili Moábci naproti vody rudé jako krev.
#3:23 Řekli: „To je krev. Ti králové se pobili, ubili se navzájem. A teď za kořistí, Moábci!“
#3:24 Když se přihnali k izraelskému táboru, Izraelci vyskočili a bili Moábce. Ti se před nimi dali na útěk a Izraelci se hnali za nimi a pobíjeli je,
#3:25 bořili města, každý díl dobré půdy zaházeli úplně kamením, každý vodní pramen zasypali a každý dobrý strom pokáceli, takže zůstaly jen kamenné zdi kírcharešetské, ale i ty obklíčili prakovníci a udeřili na ně.
#3:26 Když moábský král viděl, že boj je nad jeho síly, vzal s sebou sedm set mužů s tasenými meči, aby se probil ke králi edómskému, ale nedokázali to.
#3:27 Jal však jeho prvorozeného syna, který měl kralovat po něm, a obětoval ho v zápalnou oběť na městských hradbách. I postihlo Izraele veliké rozlícení, takže od něho odtáhli a vrátili se do své země. 
#4:1 Jedna z žen prorockých žáků úpěnlivě volala k Elíšovi: „Tvůj služebník, můj muž, je mrtev. Ty víš, že se tvůj služebník bál Hospodina. A teď přišel věřitel, aby si vzal obě mé děti za otroky.“
#4:2 Elíša jí řekl: „Co mohu pro tebe udělat? Pověz mi, co máš doma.“ Odpověděla: „Tvá služebnice nemá v domě nic než baňku oleje.“
#4:3 Řekl jí: „Jdi, vypůjč si venku nádoby ode všech svých sousedů, prázdné nádoby, ale nespokojuj se s málem.
#4:4 Pak jdi domů, zavři za sebou a za svými syny dveře a nalévej do všech těch nádob; plné dávej stranou.“
#4:5 Odešla od něho a zavřela za sebou a za svými syny dveře. Ti jí podávali nádoby a ona nalévala.
#4:6 Když nádoby naplnila, řekla svému synu: „Podej mi další nádobu.“ On jí odvětil: „Už tu žádná nádoba není.“ Tu přestal olej téci.
#4:7 Šla to sdělit muži Božímu. Ten jí řekl: „Jdi prodat olej a vyrovnej svůj dluh. Potom budeš se svými syny žít z toho, co zbude.“
#4:8 Jednoho dne procházel Elíša Šúnemem. Tam byla znamenitá žena. Ta ho přiměla, aby u ní pojedl chléb. Kdykoli pak tudy procházel, zašel tam, aby pojedl chléb.
#4:9 Řekla svému muži: „Hle, vím, že muž Boží, který kolem nás často chodívá, je svatý.
#4:10 Udělejme malý zděný pokojík na střeše a dejme mu tam lůžko, stůl, stoličku a svícen. Kdykoli k nám přijde, může se tam uchýlit.“
#4:11 Jednoho dne tam opět přišel, uchýlil se do pokojíku na střeše a ulehl tam.
#4:12 Pak řekl svému mládenci Géchazímu: „Zavolej tu Šúnemanku!“ Zavolal ji a ona před něj předstoupila.
#4:13 Pravil mu: „Řekni jí: ‚Hle, staráš se o nás se vší péčí. Co bych pro tebe mohl udělat? Mám se za tebe přimluvit u krále nebo u velitele vojska?‘“ Odpověděla: „Bydlím spokojeně uprostřed svého lidu.“
#4:14 I otázal se Elíša: „Co bych pro ni mohl udělat?“ Géchazí odvětil: „Přece něco. Nemá syna a její muž je starý.“
#4:15 Tu řekl Elíša: „Zavolej ji.“ Zavolal ji a ona zůstala stát u vchodu.
#4:16 Řekl: „V jistém čase, po obvyklé době, budeš chovat syna.“ Ona řekla: „Ne, můj pane, muži Boží, nelži své služebnici!“
#4:17 Ale žena počala a v jistém čase, po obvyklé době, porodila syna, jak jí Elíša předpověděl.
#4:18 Když dítě vyrostlo, vyšlo jednoho dne k otci za ženci.
#4:19 Řeklo svému otci: „Má hlava, ach, má hlava!“ On přikázal mládenci: „Dones je matce.“
#4:20 Ten je odnesl a odevzdal matce. Sedělo jí na kolenou až do poledne a zemřelo.
#4:21 Ona vyšla nahoru a uložila je na lůžko muže Božího, zavřela za ním a vyšla.
#4:22 Pak zavolala svého muže a řekla mu: „Pošli mi jednoho z mládenců a jednu oslici. Pospíším k muži Božímu a vrátím se.“
#4:23 On řekl: „Proč jdeš k němu dnes? Není novoluní ani den odpočinku.“ Řekla: „To je v pořádku.“
#4:24 Osedlala oslici a řekla svému mládenci: „Poháněj ji a běž, nezdržuj se kvůli mně v jízdě, leč bych ti řekla.“
#4:25 I jela, až přišla k muži Božímu na horu Karmel. Jak ji muž Boží zdálky spatřil, řekl svému mládenci Géchazímu: „Hle, to je ta Šúnemanka.
#4:26 Běž jí naproti a zeptej se jí: ‚Je vše v pořádku s tebou, s tvým mužem i s tvým dítětem? ‘“ Ona řekla: „V pořádku.“
#4:27 Ale když přišla k muži Božímu na horu, chopila se jeho nohou. Géchazí přistoupil a chtěl ji odstrčit. Muž Boží však řekl: „Nech ji, má hořko v duši. Hospodin mi to zatajil a neoznámil mi to.“
#4:28 Ona řekla: „Což jsem si syna od svého pána vyžádala? Což jsem neříkala, abys mě nešálil?“
#4:29 Elíša řekl Géchazímu: „Opásej si bedra, vezmi si do ruky mou hůl a jdi. Potkáš-li někoho, nezdrav ho, a pozdraví-li tě někdo, neodpovídej. Mou hůl polož chlapci na tvář.“
#4:30 Ale chlapcova matka řekla: „Jakože živ je Hospodin a jakože živa je tvá duše, nepustím se tě!“ I povstal a šel za ní.
#4:31 Géchazí je předešel a položil hůl na chlapcovu tvář, ale ten se neozval a nic nevnímal. Vrátil se tedy vstříc Elíšovi a ohlásil mu: „Chlapec se neprobudil.“
#4:32 Elíša vešel do domu, a hle, mrtvý chlapec byl uložen na jeho lůžku.
#4:33 Vstoupil, zavřel dveře, aby byli sami, a modlil se k Hospodinu.
#4:34 Pak se zdvihl, položil se na dítě, vložil svá ústa na jeho ústa, své oči na jeho oči a své dlaně na jeho dlaně; byl nad ním skloněn, dokud se tělo dítěte nezahřálo.
#4:35 Potom se obrátil a prošel se domem sem a tam. Vrátil se a sklonil se nad chlapcem; ten sedmkrát kýchl a otevřel oči.
#4:36 Elíša zavolal Géchazího a řekl: „Zavolej tu Šúnemanku!“ Zavolal ji. Když k němu přišla, řekl: „Vezmi si svého syna.“
#4:37 Vstoupila, padla mu k nohám a poklonila se až k zemi. Pak si vzala svého syna a odešla.
#4:38 Jednou, když byl v zemi hlad, vrátil se Elíša do Gilgálu. Proročtí žáci seděli před ním. Tu řekl svému mládenci: „Přistav velký hrnec a uvař prorockým žákům polévku.“
#4:39 Jeden z žáků vyšel na pole nasbírat zeliny. Našel polní popínavou rostlinu a sesbíral z ní plný šat polních tykví. I přišel a rozkrájel to do hrnce na polévku, aniž to kdo znal.
#4:40 Pak to nalili těm mužům k jídlu. Když trochu polévky snědli, vzkřikli: „V hrnci je smrt, muži Boží!“ A neodvážili se jíst.
#4:41 Elíša řekl: „Dejte sem mouku.“ Nasypal ji do hrnce a řekl: „Nalévej lidu.“ I jedli a v hrnci už nic zlého nebylo.
#4:42 Potom přišel jakýsi muž z Baal-šališi a přinesl muži Božímu chléb z prvního obilí, dvacet ječných chlebů, a v ranci čerstvé obilí. Elíša řekl: „Dej to lidu, ať jedí.“
#4:43 Ten, který mu přisluhoval, namítl: „Cožpak to mohu předložit stu mužů?“ Ale on odvětil: „Dej to lidu, ať jedí, neboť toto praví Hospodin: ‚Budou jíst a ještě zůstane.‘“
#4:44 Předložil jim to a oni jedli a ještě zůstalo podle Hospodinova slova. 
#5:1 Naamán, velitel vojska aramejského krále, byl u svého pána ve veliké vážnosti a oblibě, protože skrze něho dal Hospodin Aramejcům vítězství. Tento muž, udatný bohatýr, byl postižen malomocenstvím.
#5:2 Jednou vyrazily z Aramu hordy a zajaly v izraelské zemi malé děvčátko. To sloužilo Naamánově ženě.
#5:3 Řeklo své paní: „Kdyby se můj pán dostal k proroku, který je v Samaří, ten by ho jistě malomocenství zbavil.“
#5:4 Naamán to šel oznámit svému pánu: „Tak a tak mluvilo to děvče z izraelské země.“
#5:5 Aramejský král řekl: „Vyprav se tam a já pošlu izraelskému králi dopis.“ I šel. Vzal s sebou deset talentů stříbra a šest tisíc šekelů zlata a desatery sváteční šaty.
#5:6 Izraelskému králi přinesl dopis: „Jakmile ti dojde tento dopis, s nímž jsem ti poslal svého služebníka Naamána, zbav ho malomocenství.“
#5:7 Když izraelský král dopis přečetl, roztrhl své roucho a řekl: „Jsem snad Bůh, abych rozdával smrt nebo život, že ke mně posílá někoho, abych ho zbavil malomocenství? Jen uvažte a pohleďte, že hledá proti mně záminku!“
#5:8 Když Elíša, muž Boží, uslyšel, že izraelský král roztrhl své roucho, vzkázal králi: „Proč jsi roztrhl své roucho? Jen ať přijde ke mně. Pozná, že je v Izraeli prorok.“
#5:9 Naamán tedy přijel se svými koni a s vozem a zastavil u vchodu do Elíšova domu.
#5:10 Elíša mu po poslovi vzkázal: „Jdi, omyj se sedmkrát v Jordánu a tvé tělo bude opět zdravé. Budeš čist.“
#5:11 Ale Naamán se rozlítil a odešel. Řekl: „Hle, říkal jsem si: ‚Zajisté ke mně vyjde, postaví se a bude vzývat jméno Hospodina, svého Boha, bude mávat rukou směrem k posvátnému místu, a tak mě zbaví malomocenství.‘
#5:12 Cožpak nejsou damašské řeky Abána a Parpar lepší než všechny vody izraelské? Cožpak jsem se nemohl omýt v nich, abych byl čist?“ Obrátil se a rozhořčeně odcházel.
#5:13 Ale jeho služebníci přistoupili a domlouvali mu: „Otče, ten prorok ti řekl důležitou věc. Proč bys to neudělal? Přece ti řekl: ‚Omyj se, a budeš čist.‘“
#5:14 On tedy sestoupil a ponořil se sedmkrát do Jordánu podle slova muže Božího. A jeho tělo bylo opět jako tělo malého chlapce. Byl čist.
#5:15 Vrátil se k muži Božímu s celým svým průvodem. Přišel a postavil se před něho a řekl: „Hle, poznal jsem, že není Boha na celé zemi, jenom v Izraeli. A nyní přijmi prosím od svého služebníka projev vděčnosti.“
#5:16 Elíša odvětil: „Jakože živ je Hospodin, v jehož službách stojím, nevezmu nic.“ Třebaže ho nutil, aby si něco vzal, on odmítl.
#5:17 Potom Naamán řekl: „Tedy nic? Kéž je tvému služebníku dáno tolik prsti, kolik unese pár mezků, neboť tvůj služebník už nebude připravovat zápalné oběti ani obětní hody jiným bohům než Hospodinu.
#5:18 Toliko v této věci ať Hospodin tvému služebníku odpustí: Když můj pán vstupuje do domu Rimónova, aby se tam klaněl, a opírá se o mou ruku, i já se v Rimónově domě skláním. Když se tedy v Rimónově domě budu sklánět, ať Hospodin tvému služebníku tuto věc odpustí!“
#5:19 On mu řekl: „Jdi v pokoji.“
#5:20 řekl si Géchazí, mládenec Elíši, muže Božího: „Hle, můj pán ušetřil toho Aramejce Naamána, když nevzal z jeho ruky, co přivezl. Jakože živ je Hospodin, poběžím za ním a něco si od něho vezmu.“
#5:21 A Géchazí se za Naamánem rozběhl. Když Naamán spatřil, že za ním běží, seskočil z vozu, šel mu vstříc a řekl: „Je všechno v pořádku?“
#5:22 Odpověděl: „V pořádku. Můj pán mě posílá se vzkazem: ‚Hle, právě ke mně přišli z Efrajimského pohoří dva mládenci z prorockých žáků. Dej jim talent stříbra a dvoje sváteční šaty!‘“
#5:23 Naamán řekl: „Vezmi si laskavě dva talenty.“ A ještě ho nutil. Zavázal oba talenty stříbra do dvou vaků, k tomu dvoje sváteční roucho, a dal to dvěma svým mládencům, aby to před ním nesli.
#5:24 Když přišel na kopec, vzal si to od nich a doma uložil. Muže propustil a oni odešli.
#5:25 Sám pak vešel a postavil se před svého pána. Elíša se ho zeptal: „Odkudpak, Géchazí?“ Odvětil: „Tvůj služebník nikam nešel.“
#5:26 Elíša mu řekl: „Což jsem v duchu nebyl při tom, když se muž obrátil ze svého vozu a šel ti vstříc? Copak je čas brát stříbro a brát šaty a olivoví či vinice, brav či skot, otroky či otrokyně?
#5:27 Proto Naamánovo malomocenství navěky ulpí na tobě a na tvém potomstvu.“ I vyšel od něho malomocný, bílý jako sníh. 
#6:1 Proročtí žáci řekli Elíšovi: „Hle, místo, kde před tebou sedáváme, je pro nás těsné.
#6:2 Pojďme k Jordánu, vezmeme si odtamtud každý jeden trám a uděláme si místo, kde bychom sedávali.“ Řekl: „Jděte!“
#6:3 Ale jeden z nich řekl: „Buď tak laskav a pojď se svými služebníky.“ Odvětil: „Ano, půjdu.“
#6:4 I šel s nimi. Přišli k Jordánu a začali kácet stromy.
#6:5 Při porážení stromu se jednomu z nich stalo, že mu železná sekera spadla do vody. Vykřikl a zvolal: „Ach, můj pane, je vypůjčená!“
#6:6 Muž Boží se otázal: Kam padla?“ Když mu to místo ukázal, uřízl Elíša dřevo, hodil je tam a železo vyplavalo.
#6:7 Pak řekl: „Přitáhni si je!“ On vztáhl ruku a vzal si je.
#6:8 Kdykoli se aramejský král chystal do boje proti Izraeli, radíval se se svými služebníky: „Můj tábor bude na tom a tom místě.“
#6:9 Muž Boží však izraelskému králi vždycky vzkazoval: „Dej si pozor a netáhni tímto místem, protože tam táboří Aramejci!“
#6:10 Izraelský král posílal zvědy na místo, které mu muž Boží označil a před nímž ho varoval, aby se tam měl na pozoru; stalo se tak ne jednou nebo dvakrát.
#6:11 Ta věc pobouřila mysl aramejského krále; svolal své služebníky a řekl jim: „Proč mi neoznámíte, kdo z našich donáší izraelskému králi?“
#6:12 Jeden z jeho služebníků řekl: „Nikoli, králi, můj pane. Je to prorok Elíša, který je v Izraeli. Ten oznamuje izraelskému králi i ta slova, která vyslovíš ve své ložnici.“
#6:13 On poručil: „Jděte zjistit, kde je, a já ho dám zajmout.“ Oznámili mu: „Hle, je v Dótanu.“
#6:14 Poslal tam koně a vozy a silný oddíl vojska. Přitáhli v noci a oblehli město.
#6:15 Za časného jitra vstal sluha muže Božího, vyšel ven, a hle, vojsko s koni a vozy obkličovalo město. Mládenec Elíšovi řekl: „Běda, můj pane, co teď budeme dělat?“
#6:16 Odvětil: „Neboj se, protože s námi je jich víc než s nimi.“
#6:17 Potom se Elíša modlil: „Hospodine, otevři mu prosím oči, aby viděl!“ Tu Hospodin otevřel mládenci oči a on uviděl horu plnou koní a ohnivých vozů okolo Elíši.
#6:18 Když se k němu Aramejci začali stahovat, modlil se Elíša k Hospodinu: „Raň tento pronárod zaslepeností!“ I ranil je zaslepeností podle Elíšova slova.
#6:19 Elíša jim řekl: „To není ta cesta ani to město. Pojďte za mnou, dovedu vás k muži, kterého hledáte.“ A dovedl je do Samaří.
#6:20 Sotva vstoupili do Samaří, Elíša řekl: „Hospodine, otevři jim oči, ať vidí.“ Hospodin jim otevřel oči a spatřili, že jsou uprostřed Samaří.
#6:21 Když je uviděl izraelský král, řekl Elíšovi: „Můj otče, mám je dát pobít?“
#6:22 Řekl mu: „Nepobíjej. Což jsi je zajal svým mečem a lukem, abys je pobil? Předlož jim chléb a vodu, ať jedí a pijí. Pak ať jdou ke svému pánu.“
#6:23 Připravil jim tedy velkou hostinu. Když se najedli a napili, propustil je a oni odešli ke svému pánu. A aramejské hordy už nikdy nevpadly do izraelské země.
#6:24 Po nějakém čase Ben-hadad, král aramejský, shromáždil celý svůj tábor, vytáhl a oblehl Samaří.
#6:25 V Samaří nastal veliký hlad, neboť je dlouho obléhali, až byla oslí hlava za osmdesát šekelů stříbra a čtvrtka dížky holubího trusu za pět šekelů stříbra.
#6:26 Když izraelský král obcházel hradby, vykřikla na něho jedna žena: „Pomoz, králi, můj pane!“
#6:27 Odpověděl: „Nepomůže-li ti Hospodin, jak bych ti mohl pomoci já? Něčím z humna nebo lisu?“
#6:28 A král se jí zeptal: „Co je ti?“ Odvětila: „Tato žena mi řekla: ‚Dej sem svého syna a dnes jej sníme. Mého syna sníme zítra.‘
#6:29 Tak jsme mého syna uvařily a snědly. Ale když jsem jí druhého dne řekla: ‚Dej sem svého syna a sníme jej‘, ona svého syna ukryla.“
#6:30 Když král uslyšel slova té ženy, roztrhl své roucho. Jak obcházel hradby, lid viděl, že má na těle vespod žíněnou suknici.
#6:31 I řekl: „Ať se mnou Bůh udělá, co chce, jestliže Elíša, syn Šáfatův, dnes nepřijde o hlavu.“
#6:32 Elíša seděl ve svém domě a starší seděli s ním, když k němu poslal jednoho ze svých mužů. Ale ještě než posel k němu vstoupil, řekl prorok starším: „Víte, že ten vrahův syn poslal, aby mi uťali hlavu? Nuže, až bude posel přicházet, zavřete dveře, zapřete ty dveře proti němu. Což nejsou za ním slyšet kroky jeho pána?“
#6:33 Ještě k nim mluvil, když posel k němu sestoupil. A král řekl: „Hle, toto zlo je od Hospodina. Co ještě mohu od Hospodina očekávat?“ 
#7:1 Elíša řekl: „Slyšte slovo Hospodinovo. Toto praví Hospodin: ‚Zítra v tuto dobu bude v samařské bráně míra bílé mouky za šekel a dvě míry ječmene budou za šekel.‘“
#7:2 Nato odpověděl štítonoš, o jehož ruku se král opíral, muži Božímu: „I kdyby Hospodin otevřel nebeské propustě, jakpak se toto slovo uskuteční?“ Prorok řekl: „Hle, uvidíš to na vlastní oči, ale jíst z toho nebudeš.“
#7:3 U vchodu do brány byli čtyři malomocní. Řekli si vespolek: „Nač tu sedíme? Abychom zemřeli?
#7:4 Jestliže řekneme: Vejdeme do města - ve městě je hlad a zemřeme tam. A jestliže zůstaneme sedět zde, také zemřeme. Pojďme tedy, přeběhněme do aramejského tábora. Jestliže nás nechají naživu, budeme žít; jestliže nás usmrtí, zemřeme.“
#7:5 Za setmění se vydali do aramejského tábora. Došli na jeho okraj, a hle, nikdo tam nebyl.
#7:6 Panovník totiž způsobil, že Aramejci uslyšeli ve svém táboře zvuk vozů, zvuk koní, zvuk velikého vojska. Řekli si mezi sebou: „Hle, izraelský král najal proti nám chetejské a egyptské krále a ti na nás táhnou.“
#7:7 Za setmění se dali na útěk. Opustili své stany, koně a osly i tábor tak, jak byl, a utekli, aby si zachránili život.
#7:8 Malomocní přišli na okraj tábora, vešli do jednoho stanu, najedli se a napili. Vynesli odtud stříbro, zlato i roucha, odešli a tajně to uložili. Pak se vrátili, vešli do jiného stanu, vynesli odtud věci, odešli a také je tajně uložili.
#7:9 Potom si vespolek řekli: „Neměli bychom takto jednat. Tento den je dnem radostné zprávy a my jsme zticha. Budeme-li vyčkávat až do ranního úsvitu, stihne nás trest. Pojďme to tedy ohlásit v královském domě.“
#7:10 Přišli k městské bráně a volali na vrátného. Oznámili mu: „Vešli jsme do aramejského tábora, a hle, nikdo tam nebyl, nikdo se neozýval. Byli tam jen přivázaní koně a osli, a stany zůstaly tak, jak byly.“
#7:11 Ten zavolal ostatní vrátné a ti to oznámili uvnitř královského domu.
#7:12 Král v noci vstal a řekl svým služebníkům: „Chci vám oznámit, co nám provedli Aramejci. Vědí, že hladovíme. Vyšli z tábora, aby se ukryli v poli. Řekli si: ‚Vyjdou z města, pochytáme je živé a vtrhneme do města.‘“
#7:13 Jeden z jeho služebníků odpověděl: „Ať vezmou pět koní z těch, kteří ve městě ještě zůstali. Ti budou představovat celé množství Izraele, které zůstává ve městě, anebo celé množství Izraele, které zahyne. Pustíme je a uvidíme.“
#7:14 Vzali dvě spřežení koní a král je pustil směrem k aramejskému táboru. Poručil osádce: „Jeďte se podívat!“
#7:15 Jeli tedy za nimi až k Jordánu. Všude po cestě bylo plno výstroje a výzbroje, kterou Aramejci ve zmatku odhazovali. Poslové se vrátili a oznámili to králi.
#7:16 Pak vytáhl lid a vyloupil aramejský tábor. A tak se stalo, že míra bílé mouky byla za šekel a dvě míry ječmene byly za šekel, podle Hospodinova slova.
#7:17 Král ustanovil štítonoše, o jehož ruku se opíral, správcem brány. Lid ho však v bráně ušlapal k smrti, jak to předpověděl muž Boží tehdy, když k němu král sestoupil.
#7:18 Stalo se, co muž Boží králi předpověděl: „Zítra v tuto dobu budou v samařské bráně dvě míry ječmene za šekel a míra bílé mouky bude za šekel.“
#7:19 Tehdy odpověděl štítonoš muži Božímu: „I kdyby Hospodin otevřel nebeské propustě, jakpak se toto slovo uskuteční?“ On řekl: „Hle, uvidíš to na vlastní oči, ale jíst z toho nebudeš.“
#7:20 Tak se mu též stalo: Lid ho v bráně ušlapal k smrti. 
#8:1 Potom mluvil Elíša k ženě, jejíhož syna vzkřísil: „Vstaň a jdi, ty i tvůj dům, a pobývej kdekoli jako host, protože Hospodin přivolal hlad. Ten již přišel do země a potrvá sedm let.“
#8:2 Žena se zařídila podle slova muže Božího. Odešla se svým domem a po sedm let pobývala jako host v pelištejské zemi.
#8:3 Když sedm let uplynulo, vrátila se ta žena z pelištejské země a šla prosit krále o svůj dům a o své pole.
#8:4 Král právě rozmlouval s Géchazím, mládencem muže Božího. Řekl: „Vyprávěj mi o všech velikých skutcích, které Elíša učinil!“
#8:5 Zrovna když králi vyprávěl, jak vzkřísil mrtvého, ta žena, jejíhož syna vzkřísil, přišla prosit krále o svůj dům a o své pole. Géchazí řekl: „Králi, můj pane, to je ta žena a to je její syn, kterého Elíša vzkřísil.“
#8:6 Král se ženy vyptával a ona mu o tom vyprávěla. Pak o ní král uložil jednomu dvořanovi: „Ať je jí vráceno vše, co jí patří, i všechen výnos pole ode dne, kdy opustila zemi, až do nynějška!“
#8:7 Elíša přišel do Damašku. Ben-hadad, král aramejský, byl nemocen. Oznámili mu: „Přišel sem muž Boží.“
#8:8 Král nařídil Chazaelovi: „Vezmi s sebou dar, jdi muži Božímu vstříc a dotaž se skrze něho Hospodina, zda vyváznu z této nemoci živ.“
#8:9 Chazael mu tedy šel vstříc a vzal s sebou dar, všechno, co bylo v Damašku nejlepšího, náklad pro čtyřicet velbloudů. Přišel a postavil se před něho se slovy: „Posílá mě k tobě tvůj syn Ben-hadad, král aramejský, s dotazem: ‚Vyváznu z této nemoci živ?‘“
#8:10 Elíša mu odvětil: „Jdi, vyřiď mu: ‚Určitě vyvázneš živ‘. Hospodin mi však ukázal, že určitě zemře.“
#8:11 Potom mu znehybněla tvář, muž Boží nadobro strnul a pak se rozplakal.
#8:12 Chazael se otázal: „Proč můj pán pláče?“ Odvětil: „Protože vím, jakého zla se dopustíš vůči Izraelcům: Jejich pevnosti vypálíš ohněm, jejich jinochy povraždíš mečem, jejich pacholátka rozdrtíš a jejich těhotné poroztínáš.“
#8:13 Chazael mu řekl: „Cožpak je tvůj služebník pes, že by se dopustil něčeho tak strašného?“ Elíša řekl: „Hospodin mi ukázal, že budeš králem nad Aramem.“
#8:14 On pak odešel od Elíši a přišel ke svému pánu. Ten se ho otázal: „Co ti řekl Elíša?“ Odvětil: „Řekl mi, že určitě vyvázneš živ.“
#8:15 Ale druhého dne vzal prostěradlo, namočil je do vody a rozprostřel mu je na obličej, a on zemřel. Místo něho se stal králem Chazael.
#8:16 V pátém roce vlády Jórama, syna izraelského krále Achaba, za Jóšafata, krále judského, se stal králem Jóram, syn Jóšafatův, král judský.
#8:17 Bylo mu dvaatřicet let, když začal kralovat, a kraloval v Jeruzalémě osm let.
#8:18 Chodil po cestě králů izraelských, jak to činil dům Achabův; jeho ženou byla totiž dcera Achabova. Dopouštěl se toho, co je zlé v Hospodinových očích.
#8:19 Ale Hospodin nechtěl na Judu uvést zkázu kvůli Davidovi, svému služebníku, protože mu přislíbil, že dá jemu i jeho synům planoucí světlo po všechny dny.
#8:20 Za jeho dnů se Edómci vymanili z područí Judy a ustanovili nad sebou krále.
#8:21 Jóram vytáhl s celou svou vozbou do Sáíru. Vstal v noci a udeřil na Edómce, kteří ho obklíčili, a na velitele vozby; ale lid utekl ke svým stanům.
#8:22 Edóm se tedy vymanil z područí Judy, jak je tomu dodnes. Tehdy, v onen čas, se vymanila i Libna.
#8:23 O ostatních příbězích Jóramových, o všem, co konal, se píše, jak známo, v Knize letopisů králů judských.
#8:24 I ulehl Jóram ke svým otcům a byl pohřben vedle svých otců v Městě Davidově. Po něm se stal králem jeho syn Achazjáš.
#8:25 V dvanáctém roce vlády izraelského krále Jórama, syna Achabova, se stal králem Achazjáš, syn Jóramův, král judský.
#8:26 Achazjášovi bylo dvaadvacet let, když začal kralovat, a kraloval v Jeruzalémě jeden rok. Jeho matka se jmenovala Atalja; byla to dcera Omrího, krále izraelského.
#8:27 Chodil po cestě domu Achabova a dopouštěl se toho, co je zlé v Hospodinových očích, jako dům Achabův. Byl totiž s domem Achabovým spřízněn.
#8:28 Vypravil se s Jóramem, synem Achabovým, do války proti aramejskému králi Chazaelovi, do Rámotu v Gileádu. Ale Aramejci Jórama ranili.
#8:29 Král Jóram se vrátil, aby se léčil v Jizreelu z ran, které mu Aramejci v Rámě zasadili, když válčil proti Chazaelovi, králi aramejskému. Achazjáš, syn Jóramův, král judský, sestoupil do Jizreelu podívat se na Jórama, syna Achabova, protože byl nemocný. 
#9:1 Prorok Elíša zavolal jednoho z prorockých žáků a řekl mu: „Opásej si bedra, do ruky vezmi tuto nádobku oleje a jdi do Rámotu v Gileádu.
#9:2 Až tam přijdeš, poohlédni se po Jehúovi, synu Jóšafata, syna Nimšího. Přijdeš, řekneš, aby vstal ze středu svých bratří, a zavedeš ho do nejzazšího pokojíka.
#9:3 Pak vezmeš nádobku s olejem, vyleješ mu jej na hlavu a řekneš: ‚Toto praví Hospodin: Pomazal jsem tě za krále nad Izraelem.‘ Otevřeš dveře a dáš se bez otálení na útěk.“
#9:4 Mládenec, byl to prorocký mládenec, šel do Rámotu v Gileádu.
#9:5 Přišel, když velitelé vojska právě zasedali. Řekl: „Mám něco pro tebe, veliteli.“ Jehú se tázal: „Pro koho z nás všech?“ Odpověděl: „Pro tebe, veliteli.“
#9:6 Jehú vstal a vešel dovnitř a on mu vylil na hlavu olej. Řekl mu: „Toto praví Hospodin, Bůh Izraele: Pomazal jsem tě za krále nad Hospodinovým lidem, nad Izraelem.
#9:7 Vybiješ dům svého pána Achaba. Tak pomstím prolitou krev svých služebníků proroků i krev všech Hospodinových služebníků, již prolila Jezábel.
#9:8 Celý Achabův dům zahyne. Vyhladím z Achabova domu toho, jenž močí na stěnu, a v Izraeli zajatého i zanechaného.
#9:9 Achabovu domu učiním jako domu Jarobeáma, syna Nebatova, a jako domu Baeši, syna Achijášova.
#9:10 Jezábelu sežerou psi na dílu pole v Jizreelu a nikdo ji nepohřbí.“ Potom otevřel dveře a utekl.
#9:11 Když Jehú vyšel k služebníkům svého pána, tázali se ho: „Je vše v pořádku? Proč k tobě ten ztřeštěnec přišel?“ Řekl jim: „Víte přece, že jaký muž, takové jeho tlachání.“
#9:12 Odpověděli: „Klameš. Pověz nám, co se děje!“ Tu řekl: „Tak a tak ke mně mluvil. Řekl: ‚Toto praví Hospodin: Pomazal jsem tě za krále nad Izraelem.‘“
#9:13 Každý vzal rychle své roucho a položili mu je pod nohy na schody, zatroubili na polnice a provolávali: „Jehú se stal králem!“
#9:14 Tak se spikl Jehú, syn Jóšafata, syna Nimšího, proti Jóramovi. Jóram hájil tenkrát s celým Izraelem Rámot v Gileádu proti aramejskému králi Chazaelovi.
#9:15 Pak se král Jóram vrátil, aby se v Jizreelu léčil z ran, které mu zasadili Aramejci, když bojoval proti aramejskému králi Chazaelovi. Jehú řekl: „Jste-li se mnou, ať se nikdo nepokusí dostat se z města a oznámit to v Jizreelu.“
#9:16 Jehú nasedl na vůz a jel do Jizreelu, neboť tam ležel Jóram a judský král Achazjáš tam sestoupil podívat se na Jórama.
#9:17 Strážný, který stál v Jizreelu na věži, spatřil přijíždět Jehúovu tlupu a volal: „Vidím nějakou tlupu!“ Jóram poručil: „Vyšli jim vstříc jezdce, aby se zeptal, zda je vše v pořádku.“
#9:18 Jezdec jim jel na koni vstříc a řekl: „Toto praví král: ‚Je vše v pořádku?‘“ Jehú na to: „Co je ti do pořádku? Otoč se a zařaď se za mne!“ Strážný hlásil: „Posel dojel až k nim, ale nevrací se.“
#9:19 Jóram poslal na koni druhého jezdce. Ten dojel k nim a řekl: „Toto praví král: ‚Je vše v pořádku?‘“ Jehú odpověděl: „Co je ti do pořádku? Otoč se a zařaď se za mne!“
#9:20 Strážný opět hlásil: „Dojel až k nim, ale nevrací se. Podle jízdy je to Jehú, syn Nimšího, neboť jede ztřeštěně.“
#9:21 Jóram poručil: „Zapřahej!“ Zapřáhli tedy do jeho vozu. I vyjel izraelský král Jóram i judský král Achazjáš, každý na svém voze, a vyjeli vstříc Jehúovi; setkali se s ním na dílu pole Nábota Jizreelského.
#9:22 Když Jóram spatřil Jehúa, zeptal se ho: „Je vše v pořádku, Jehú?“ Jehú odpověděl: Jakýpak pořádek, když trvá smilstvo tvé matky Jezábely a množství jejích čárů?“
#9:23 Jóram obrátil vůz a dal se na útěk. Na Achazjáše zavolal: „Achazjáši, zrada!“
#9:24 Jehú napjal luk a zasáhl Jórama mezi lopatky. Šíp mu pronikl srdcem a on se na voze zhroutil.
#9:25 Jehú řekl jeho štítonosi Bidkarovi: „Seber ho a vyhoď ho na díl pole Nábota Jizreelského. Vzpomeň si: Když jsme spolu jezdívali se spřežením za jeho otcem Achabem, Hospodin nad ním vynesl tento výnos:
#9:26 ‚Což jsem nedávno neviděl prolitou krev Nábota a jeho synů? je výrok Hospodinův. Proto ti dám odplatu na tomto dílu pole, je výrok Hospodinův.‘ Seber ho nyní a vyhoď ho na ten díl pole podle Hospodinova slova.“
#9:27 Judský král Achazjáš to viděl a dal se na útěk směrem na Bét-hagan. Jehú ho pronásledoval a volal: „Také toho ubijte!“ Ranili ho na voze na svahu Gúru u Jibleámu, ale Achazjáš ujel do Megida a tam zemřel.
#9:28 Jeho služebníci ho převezli do Jeruzaléma a pohřbili ho v Městě Davidově v jeho hrobě vedle jeho otce.
#9:29 V jedenáctém roce vlády Jórama, syna Achabova, se stal Achazjáš králem nad Judou.
#9:30 Než přijel Jehú do Jizreelu, Jezábel o všem uslyšela. Namalovala si oči líčidlem, okrášlila si hlavu a vyhlížela z okna.
#9:31 Když Jehú vstoupil do brány, zvolala: „Je vše v pořádku, Zimrí, vrahu svého pána?“
#9:32 Pozdvihl tvář k oknu a zvolal: „Kdo je se mnou? Kdo?“ Dva tři dvořané na něho vyhlédli.
#9:33 Poručil: „Shoďte ji!“ A shodili ji dolů. Její krev vystříkla na stěnu a na koně. Ti ji ušlapali.
#9:34 Pak Jehú vstoupil, jedl a pil. Nařídil: „Postarejte se o tu proklatou a pohřběte ji. Je to dcera královská.“
#9:35 Šli tedy, aby ji pohřbili, ale nenašli z ní nic než lebku, nohy a ruce.
#9:36 Vrátili se a oznámili mu to. Tu řekl: „To je to Hospodinovo slovo, které vyhlásil skrze svého služebníka Elijáše Tišbejského: ‚Na dílu pole v Jizreelu budou psi žrát tělo Jezábely.‘
#9:37 A Jezábelina mrtvola bude na dílu pole v Jizreelu jako hnůj na poli, takže nikdo neřekne: Toto je Jezábel.“ 
#10:1 Achab měl v Samaří sedmdesát synů. Jehú napsal dopisy a poslal je do Samaří, k velitelům Jizreelu, starším a vychovatelům synů Achabových:
#10:2 „Hned, jak vám dojde tento dopis - máte syny svého pána, máte i vozbu a koně, opevněné město a zbroj -,
#10:3 vyhlédněte si ze synů svého pána někoho, kdo je dobrý a vhodný, a dosaďte ho na trůn jeho otce a bojujte za dům svého pána.“
#10:4 Oni se však převelice báli a řekli: „Hle, neobstáli před ním ani dva králové, jak obstojíme my?“
#10:5 Správce domu a správce města i starší a vychovatelé vzkázali tedy Jehúovi: „Jsme tvými služebníky a uděláme všechno, co nám poručíš. Za krále nedosadíme nikoho, udělej, co uznáš za dobré.“
#10:6 Napsal jim druhý dopis: „Jste-li se mnou a chcete-li mě poslouchat, vezměte hlavy všech synů svého pána a přijďte ke mně zítra v tuto dobu do Jizreelu!“ Králových synů bylo sedmdesát, byli u významných lidí města a ti je vychovávali.
#10:7 Jakmile jim dopis došel, chopili se králových synů, všech sedmdesát pobili, jejich hlavy složili do košů a poslali je do Jizreelu.
#10:8 Posel přišel a oznámil mu: „Přinesli hlavy králových synů.“ Jehú nařídil: „Složte je na dvě hromady u vchodu do brány, budou tam až do rána.“
#10:9 Ráno vyšel, zastavil se a řekl všemu lidu: „Vy jste spravedliví. Ano, já jsem se spikl proti svému pánu a zabil jsem ho. Kdo však pobil všechny tyto?
#10:10 Z toho můžete poznat, že nic z Hospodinova slova, které Hospodin promluvil proti Achabovu domu, nepadlo na zem. Hospodin vykonal, o čem mluvil skrze svého služebníka Elijáše.“
#10:11 Jehú pak pobil všechny, kdo zbyli v Jizreelu z Achabova domu, všechny jeho významné lidi, důvěrné přátele i kněze; nenechal vyváznout nikoho.
#10:12 Pak se vydal do Samaří. Šel cestou přes Bét-eked pastýřů.
#10:13 I nalezl Jehú bratry judského krále Achazjáše a otázal se: „Kdo jste?“ Oni odpověděli: „Jsme Achazjášovi bratři. Přišli jsme popřát pokoj synům krále a synům paní.“
#10:14 Jehú rozkázal: „Pochytejte je živé.“ I chytili je živé a pobili u cisterny v Bét-ekedu, dvaačtyřicet mužů. Žádného z nich neponechali.
#10:15 Odtud odešel a setkal se s Jónadabem, synem Rekábovým, který mu vyšel vstříc. Pozdravil ho a otázal se ho: „Myslíš to se mnou upřímně jako já s tebou?“ Jónadab řekl: „Zcela jistě!“ Nato Jehú: „Podej mi ruku.“ I podal mu ruku a vystoupil k němu na vůz.
#10:16 Jehú pravil: „Pojeď se mnou a pohleď na moji horlivost pro Hospodina.“ A jeli spolu v jeho voze.
#10:17 Když přijeli do Samaří, pobil a vyhladil všechny, kdo Achabovi v Samaří zůstali, podle slova Hospodina, které promluvil k Elijášovi.
#10:18 Potom Jehú shromáždil všechen lid a řekl jim: „Achab málo sloužil Baalovi, Jehú mu bude sloužit více.
#10:19 Svolejte nyní ke mně všechny Baalovy proroky, všechny jeho ctitele a všechny jeho kněze, ať žádný nechybí, neboť uspořádám pro Baala veliký obětní hod. Nikdo, kdo by chyběl, nezůstane naživu.“ Jehú jednal úskočně; chtěl vyhubit Baalovy ctitele.
#10:20 Pak Jehú rozkázal: „Připravte Baalovi slavnostní shromáždění.“ Svolali je.
#10:21 Jehú obeslal celý Izrael. Přišli všichni Baalovi ctitelé; nebylo nikoho, kdo by nepřišel. Vstoupili do Baalova domu, takže Baalův dům byl plný, hlava na hlavě.
#10:22 Správci nad šatnou Jehú nařídil: „Vydávej roucha všem Baalovým ctitelům.“ I vydal jim roucha.
#10:23 Pak vstoupil do Baalova domu Jehú a Jónadab, syn Rekábův. Řekl Baalovým ctitelům: „Rozhlédněte se a zjistěte, zda tu s vámi není někdo ze služebníků Hospodinových, zda tu jsou jen ctitelé Baalovi.“
#10:24 I šli připravit obětní hody a zápalné oběti. Jehú pak postavil venku osmdesát mužů a nařídil: „Jestliže někomu unikne někdo z mužů, které jsem vám vydal do rukou, zaplatí to životem.“
#10:25 Když se dokončila příprava zápalné oběti, nařídil Jehú běžcům a štítonošům: „Vyrazte a bijte je. Ať nikdo neujde!“ Pobili je ostřím meče. Běžci a štítonoši je pak vyházeli a pronikli až do svatyňky Baalova domu.
#10:26 Vynesli z Baalova domu posvátné sloupy a spálili je.
#10:27 Vyvrátili Baalův sloup, vyvrátili i Baalův dům a proměnili jej v hnojiště, jak je tomu dodnes.
#10:28 Tak vyhladil Jehú Baala z Izraele.
#10:29 Jehú se však neodvrátil od hříchů Jarobeáma, syna Nebatova, který svedl Izraele k hříchu, od zlatých býčků, kteří byli v Bét-elu a v Danu.
#10:30 Hospodin řekl Jehúovi: „Poněvadž jsi dobře udělal, co je v mých očích správné, poněvadž jsi Achabovu domu učinil všechno tak, jak jsem to měl v úmyslu, tvoji synové do čtvrtého pokolení budou sedět na izraelském trůnu.“
#10:31 Ale Jehú nedbal na to, aby se řídil celým srdcem podle zákona Hospodina, Boha Izraele, neodvrátil se od hříchů Jarobeáma, který svedl Izraele k hříchu.
#10:32 V těch dnech začal Hospodin Izraele oklešťovat. Chazael jej porazil na celém izraelském pomezí
#10:33 od Jordánu na východ slunce, v celém území Gileádu, patřícím Gádovi i Rúbenovi a Manasesovi, od Aróeru, který je v úvalu Arnónu, v Gileádu i Bášanu.
#10:34 O ostatních příbězích Jehúových, o všem, co konal, o všech jeho bohatýrských činech se píše, jak známo, v Knize letopisů králů izraelských.
#10:35 I ulehl Jehú ke svým otcům a pohřbili ho v Samaří. Po něm se stal králem jeho syn Jóachaz.
#10:36 Jehú kraloval nad Izraelem v Samaří dvacet osm let. 
#11:1 Když Atalja, matka Achazjášova, viděla, že její syn zemřel, rozhodla se vyhubit všechno královské potomstvo.
#11:2 Ale Jóšeba, dcera krále Jórama, sestra Achazjášova, vzala Jóaše, syna Achazjášova, a unesla ho zprostřed královských synů, kteří měli být usmrceni, a ukryla ho i jeho kojnou v pokojíku s lůžky. Skryli ho před Ataljou, takže nebyl usmrcen.
#11:3 Schovával se u ní v Hospodinově domě po šest let, zatímco v zemi kralovala Atalja.
#11:4 V sedmém roce obeslal Jójada velitele setnin, tělesnou stráž a běžce, uvedl je k sobě do Hospodinova domu a uzavřel s nimi smlouvu. V Hospodinově domě je zavázal přísahou a pak jim ukázal králova syna.
#11:5 Přikázal jim: „Uděláte tuto věc: Třetina z vás, kteří přicházíte v den odpočinku a jste na stráži u královského domu,
#11:6 i třetina od brány Súru a třetina od brány za domem běžců budete střídavě na stráži u domu.
#11:7 Dvě skupiny z vás, všichni, kteří v den odpočinku odcházíte, zůstanete na stráži u Hospodinova domu při králi.
#11:8 Obklopíte krále, každý se zbraní v ruce. Kdo by chtěl vniknout do vašich oddílů, bude usmrcen. Buďte s králem při jeho vycházení a vcházení.“
#11:9 Velitelé setnin učinili všechno, co jim kněz Jójada přikázal. Každý vzal své muže, kteří šli v den odpočinku do služby, i ty, kteří v den odpočinku ze služby odcházeli, a přišli ke knězi Jójadovi.
#11:10 Kněz vydal velitelům setnin kopí a štíty krále Davida, které byly v Hospodinově domě.
#11:11 Běžci se rozestavili každý se svou zbraní v ruce od pravé strany domu až k levé straně domu, při oltáři i při domě, aby byli kolem krále.
#11:12 Jójada vyvedl králova syna, vložil na něj královskou čelenku a předali mu Hospodinovo svědectví. Dosadili ho za krále a pomazali ho, tleskali rukama a provolávali: „Ať žije král!“
#11:13 Když Atalja uslyšela hlas lidu, který se sběhl, přišla k lidu do Hospodinova domu.
#11:14 Podívala se, a hle, král stojí podle řádu na svém stanovišti, velitelé a trubači u krále, všechen lid země se raduje a troubí na trubky. Atalja roztrhla své roucho a vykřikla: „Spiknutí! Spiknutí!“
#11:15 Kněz Jójada vydal příkaz velitelům setnin, ustanoveným nad vojskem. Řekl jim: „Vyveďte ji středem oddílů. Kdo by šel za ní, ať zemře mečem!“ Kněz totiž řekl: „Ať není usmrcena v domě Hospodinově.“
#11:16 Přinutili ji, aby vstoupila na cestu, kudy vjíždějí koně do královského domu, a tam byla usmrcena.
#11:17 Jójada pak uzavřel smlouvu mezi Hospodinem a králem i lidem, že budou lidem Hospodinovým, i smlouvu mezi králem a lidem.
#11:18 Všechen lid země přišel k Baalovu domu a strhli jej; jeho oltáře i jeho obrazy nadobro roztříštili a Baalova kněze Matána zabili před oltáři. Kněz pak ustanovil nad Hospodinovým domem dohled.
#11:19 Potom vzal velitele setnin, tělesnou stráž a běžce i všechen lid země a vedli krále dolů z Hospodinova domu; branou běžců přišli do královského domu, kde zasedl na královský trůn.
#11:20 Všechen lid země se radoval a v městě nastal klid. - Atalju usmrtili mečem v královském domě. 
#12:1 Jóašovi bylo sedm let, když začal kralovat.
#12:2 V sedmém roce vlády Jehúovy se stal králem a kraloval v Jeruzalémě čtyřicet let. Jeho matka se jmenovala Sibja a byla z Beer-šeby.
#12:3 Jóaš činil to, co je správné v Hospodinových očích, po všechny své dny, poněvadž ho vyučoval kněz Jójada.
#12:4 Avšak posvátná návrší neodstranili, lid dále obětoval a pálil na posvátných návrších kadidlo.
#12:5 Jóaš řekl kněžím: „Všechno stříbro ze svatých darů, které bude přineseno do Hospodinova domu, totiž stříbro, jímž se běžně platí, stříbro za osoby podle kněžského ocenění, všechno stříbro, které někdo dobrovolně přinese do Hospodinova domu,
#12:6 ať kněží berou k sobě, každý od lidí svého obvodu, a z toho se bude opravovat, co je na domě poškozeno, všechny vzniklé škody.“
#12:7 Ale když do dvacátého třetího roku vlády krále Jóaše kněží poškozené části domu ještě neopravili,
#12:8 předvolal král Jóaš kněze Jójadu a ostatní kněze a vytkl jim: „Proč neopravujete poškozené části domu? Přestanete brát stříbro od lidí svého obvodu, protože je určeno na opravu poškozených částí domu.“
#12:9 Kněží svolili, že nebudou brát od lidu stříbro, ale také že nebudou opravovat poškozené části domu.
#12:10 Kněz Jójada vzal jednu truhlu, vydlabal v jejím víku otvor a dal ji vedle oltáře, napravo od místa, kudy se vchází do Hospodinova domu, a tam dávali kněží, strážci prahu, všechno stříbro přinášené do Hospodinova domu.
#12:11 Když viděli, že je v truhle hodně stříbra, přicházeli královský písař i velekněz a zavázali do měšců spočítané stříbro, které se v Hospodinově domě nalezlo.
#12:12 Pak dávali odpočítané stříbro těm, kdo pracovali v Hospodinově domě jako dohlížitelé, a ti je vydávali tesařům a stavebním dělníkům pracujícím na Hospodinově domě,
#12:13 zedníkům a kameníkům, i na opatřování dřeva a tesaného kamene k opravě poškozených částí Hospodinova domu a na všechno ostatní, co bylo třeba na opravu domu vynaložit.
#12:14 Avšak stříbrné misky, kleště na knoty, kropenky, trubky, veškeré nádoby zlaté a stříbrné pro Hospodinův dům nebyly pořízeny ze stříbra přinášeného do Hospodinova domu,
#12:15 poněvadž je dávali těm, kdo pracovali jako dohlížitelé na opravách Hospodinova domu.
#12:16 Od mužů, kterým dávali stříbro do rukou, aby je vydávali těm, kdo pracovali, nepožadovali vyúčtování, poněvadž ti jednali poctivě.
#12:17 Stříbro z obětí za vinu a stříbro z obětí za hřích nebylo přinášeno do Hospodinova domu; náleželo kněžím.
#12:18 Tehdy vytáhl aramejský král Chazael do boje proti Gatu a dobyl jej. Když se Chazael chystal táhnout na Jeruzalém,
#12:19 Jóaš, král judský, vzal všechny svaté dary, které oddělili jako svaté Jóšafat a Jóram i Achazjáš, jeho otcové, králové judští, i svoje svaté dary, též všechno zlato, které se nacházelo v pokladech Hospodinova domu i domu královského, a poslal vše Chazaelovi, králi aramejskému, a ten od Jeruzaléma odtáhl.
#12:20 O ostatních příbězích Jóašových, o všem, co konal, se píše, jak známo, v Knize letopisů králů judských.
#12:21 Jeho služebníci povstali a zosnovali spiknutí a ubili Jóaše v domě Miló, když sestupoval do Sily.
#12:22 Ubili ho k smrti jeho služebníci Józabad, syn Šimeátin, a Józabad, syn Šómerův. Pohřbili ho vedle jeho otců v Městě Davidově. Po něm se stal králem jeho syn Amasjáš. 
#13:1 V třináctém roce vlády judského krále Jóaše, syna Achazjášova, se stal králem nad Izraelem v Samaři Jóachaz, syn Jehúův. Kraloval sedmnáct roků.
#13:2 Dopouštěl se toho, co je zlé v Hospodinových očích, a chodil v hříších Jarobeáma, syna Nebatova, který svedl Izraele k hříchu; neupustil od nich.
#13:3 Proto vzplanul Hospodin proti Izraeli hněvem. Vydal je po všechen ten čas do rukou aramejského krále Chazaela a do rukou Ben-hadada, syna Chazaelova.
#13:4 Jóachaz však prosil Hospodina o shovívavost a Hospodin ho vyslyšel. Viděl totiž, jak je Izrael utlačován, že jej utlačuje král aramejský.
#13:5 Hospodin dal Izraeli zachránce, takže se z aramejského područí vymanili. Izraelci zase bydlili ve svých stanech jako dříve.
#13:6 Jen od hříchů domu Jarobeáma, který svedl Izraele k hříchu, neupustili, ale chodili v nich. V Samaří zůstal stát dokonce posvátný kůl.
#13:7 Hospodin neponechal Jóachazovi z válečného lidu více než padesát jezdců a deset vozů a deset tisíc pěších. Aramejský král je totiž zničil tak, že byli jako prach při mlácení.
#13:8 O ostatních příbězích Jóachazových, o všem, co konal, i o jeho bohatýrských činech se píše, jak známo, v Knize letopisů králů izraelských.
#13:9 I ulehl Jóachaz ke svým otcům a pohřbili ho v Samaří. Po něm se stal králem jeho syn Jóaš.
#13:10 V třicátém sedmém roce vlády judského krále Jóaše se stal králem nad Izraelem v Samaří Jóaš, syn Jóachazův. Kraloval šestnáct let.
#13:11 Dopouštěl se toho, co je zlé v Hospodinových očích. Neupustil od žádných hříchů Jarobeáma, syna Nebatova, který svedl Izraele k hříchu, nýbrž chodil v nich.
#13:12 O ostatních příbězích Jóašových, o všem, co konal, i o jeho bohatýrských činech, jak bojoval s Amasjášem, králem judským, se píše, jak známo, v Knize letopisů králů izraelských.
#13:13 I ulehl Jóaš ke svým otcům a na jeho trůn dosedl Jarobeám. Jóaš byl pohřben v Samaří vedle králů izraelských.
#13:14 Elíša těžce onemocněl a umíral. Tehdy k němu sestoupil izraelský král Jóaš, rozplakal se nad ním a zvolal: „Můj otče, můj otče! Vozataji Izraele!“
#13:15 Elíša mu řekl: „Vezmi luk a šípy.“ Přinesl k němu tedy luk a šípy.
#13:16 Řekl izraelskému králi: „Napni rukou luk.“ I napjal jej rukou. Elíša položil své ruce na ruce královy.
#13:17 A nařídil: „Otevři okno na východ!“ Když je otevřel, Elíša řekl: „Střel!“ Střelil a on řekl: „Šíp záchrany Hospodinovy, šíp záchrany proti Aramovi. Úplně pobiješ Arama v Afeku.“
#13:18 Potom Elíša řekl: „Vezmi šípy.“ Vzal je tedy. Řekl izraelskému králi: „Střílej k zemi.“ Vystřelil třikrát třikrát a přestal.
#13:19 Tu se na něho muž Boží rozlítil. Vykřikl: „Měls vystřelit pětkrát nebo šestkrát, pak bys pobil Arama úplně! Nyní pobiješ Arama jen třikrát.“
#13:20 I zemřel Elíša a pohřbili ho. Nastávajícího roku vpadly do země hordy Moábců.
#13:21 Když Izraelci pohřbívali jednoho muže, spatřili takovou hordu. I hodili toho muže do hrobu Elíšova. Jakmile muž přišel do styku s Elíšovými kostmi, ožil a postavil se na nohy.
#13:22 Chazael, král aramejský, utlačoval Izraele po všechny dny Jóachazovy vlády.
#13:23 Ale Hospodin jim byl milostiv. Slitovával se nad nimi a obracel se k nim pro svou smlouvu s Abrahamem, Izákem a Jákobem. Nechtěl je zničit a až do nynějška je od své tváře neodvrhl.
#13:24 I umřel Chazael, král aramejský. Po něm se stal králem jeho syn Ben-hadad.
#13:25 A Jóaš, syn Jóachazův, získal zase z rukou Ben- hadada, syna Chazaelova, města, která zabral ve válce Chazael jeho otci Jóachazovi. Jóaš ho třikrát porazil a získal izraelská města nazpět. 
#14:1 V druhém roce vlády izraelského krále Jóaše, syna Jóachazova, se stal králem Amasjáš, syn Jóašův, král judský.
#14:2 Bylo mu pětadvacet let, když začal kralovat, a kraloval v Jeruzalémě dvacet devět let. Jeho matka se jmenovala Jóadan a byla z Jeruzaléma.
#14:3 Činil to, co je správné v Hospodinových očích, ne však jako David, jeho otec. Všechno činil tak, jak to činil jeho otec Jóaš.
#14:4 Avšak posvátná návrší neodstranili, lid dále obětoval a pálil na posvátných návrších kadidlo.
#14:5 Jakmile bylo království pevně v jeho rukou, pobil ze svých služebníků ty, kteří ubili krále, jeho otce.
#14:6 Ale syny vrahů neusmrtil, neboť je napsáno v knize Mojžíšova zákona, že Hospodin přikázal: „Nebudou usmrcováni otcové kvůli synům a synové nebudou usmrcováni kvůli otcům, nýbrž každý bude usmrcen za svůj hřích.“
#14:7 Pobil Edómce v Solném údolí, deset tisíc mužů, a v bitvě se zmocnil Sely a přejmenoval ji na Jokteel, jak je tomu dodnes.
#14:8 Tehdy poslal Amasjáš posly k izraelskému králi Jóašovi, synu Jóachaza, syna Jehúova, se vzkazem: „Pojď, utkáme se.“
#14:9 Jóaš, král izraelský, poslal Amasjášovi, králi judskému, odpověď: „Na Libanónu vzkázalo trní libanónskému cedru: ‚Dej svou dceru za ženu mému synovi.‘ I přešlo tudy libanónské polní zvíře a to trní rozšlapalo.
#14:10 Vítězně jsi pobil Edómce, proto se tvé srdce tak vypíná. Užívej si slávy, ale seď doma! Proč si zahráváš se zlem? Abys padl ty i Juda s tebou?“
#14:11 Ale Amasjáš neposlechl. Proto Jóaš, král izraelský, vytáhl a utkali se, on a Amasjáš, král judský, u Bét-šemeše, jenž patřil Judovi.
#14:12 Juda byl před tváří Izraele poražen; každý utíkal ke svému stanu.
#14:13 Judského krále Amasjáše, syna Jóaše, syna Achazjášova, izraelský král Jóaš v Bét-šemeši zajal. Pak přitáhl do Jeruzaléma a prolomil jeruzalémské hradby od Efrajimské brány až k bráně Nárožní v délce čtyř set loket.
#14:14 Pobral všechno zlato a stříbro a všechno náčiní, které se nacházelo v Hospodinově domě a mezi poklady domu královského, i rukojmí a vrátil se do Samaří.
#14:15 O ostatních příbězích Jóašových, o tom, co konal, i o jeho bohatýrských činech, jak bojoval s Amasjášem, králem judským, se píše, jak známo, v Knize letopisů králů izraelských.
#14:16 I ulehl Jóaš ke svým otcům a byl pohřben v Samaří vedle izraelských králů. Po něm se stal králem jeho syn Jarobéam.
#14:17 Amasjáš, syn Jóašův, král judský, žil po smrti izraelského krále Jóaše, syna Jóachazova, ještě patnáct let.
#14:18 O ostatních příbězích Amasjášových se píše, jak známo, v Knize letopisů králů judských.
#14:19 Když proti němu v Jeruzalémě zosnovali spiknutí, utekl do Lakíše. Ale poslali za ním do Lakíše vrahy a usmrtili ho tam.
#14:20 Potom ho převezli na koních a byl pohřben v Jeruzalémě v Městě Davidově vedle svých otců.
#14:21 Všechen judský lid pak vzal Azarjáše, kterému bylo šestnáct let, a dosadili ho za krále po jeho otci Amasjášovi.
#14:22 On vystavěl Élat a navrátil jej Judovi, poté co král Amasjáš ulehl ke svým otcům.
#14:23 V patnáctém roce vlády judského krále Amasjáše, syna Jóašova, se stal králem v Samaří Jarobeám, syn Jóašův, král izraelský. Kraloval jedenačtyřicet roků.
#14:24 Dopouštěl se toho, co je zlé v Hospodinových očích. Neupustil od žádných hříchů Jarobeáma, syna Nebatova, který svedl Izraele k hříchu.
#14:25 Znovu získal pomezí Izraele od cesty do Chamátu až k Pustému moři podle slova Hospodina, Boha Izraele, které promluvil skrze svého služebníka Jonáše, syna Amítajova, proroka z Gat-chéferu.
#14:26 Hospodin totiž viděl přetrpké pokoření Izraele, jak zajatého, tak zanechaného; nebylo nikoho, kdo by Izraeli pomohl.
#14:27 Hospodin se přece nerozhodl vymazat jméno Izraele zpod nebes. Zachránil je skrze Jarobeáma, syna Jóašova.
#14:28 O ostatních příbězích Jarobeámových, o všem, co konal, i o jeho bohatýrských činech, jak bojoval a jak znovu získal Izraeli Damašek a judský Chamát, se píše, jak známo, v Knize letopisů králů izraelských.
#14:29 I ulehl Jarobeám ke svým otcům, ke králům izraelským. Po něm se stal králem jeho syn Zekarjáš. 
#15:1 V dvacátém sedmém roce vlády izraelského krále Jarobeáma se stal králem Azarjáš, syn Amasjášův, král judský.
#15:2 Bylo mu šestnáct let, když začal kralovat, a kraloval v Jeruzalémě dvaapadesát let. Jeho matka se jmenovala Jekolja a byla z Jeruzaléma.
#15:3 Činil to, co je správné v Hospodinových očích, zcela jak to činil jeho otec Amasjáš.
#15:4 Avšak posvátná návrší neodstranili, lid dále obětoval a pálil na posvátných návrších kadidlo.
#15:5 Hospodin krále ranil, takže byl malomocný až do dne své smrti; bydlel v odděleném domě. Králův syn Jótam byl správcem domu a soudil lid země.
#15:6 O ostatních příbězích Azarjášových, o všem, co konal, se píše, jak známo, v Knize letopisů králů judských.
#15:7 I ulehl Azarjáš ke svým otcům a pohřbili ho vedle jeho otců v Městě Davidově. Po něm kraloval jeho syn Jótam.
#15:8 V třicátém osmém roce vlády judského krále Azarjáše se stal králem nad Izraelem v Samaří Zekarjáš, syn Jarobeámův. Kraloval šest měsíců.
#15:9 Dopouštěl se toho, co je zlé v Hospodinových očích, jak se toho dopouštěli jeho otcové. Neupustil od hříchů Jarobeáma, syna Nebatova, který svedl Izraele k hříchu.
#15:10 Proti němu se spikl Šalúm, syn Jábešův. Ubil ho před lidem k smrti a stal se místo něho králem.
#15:11 O ostatních příbězích Zekarjášových se píše v Knize letopisů králů izraelských.
#15:12 Takové bylo Hospodinovo slovo, které promluvil k Jehúovi: „Tvoji synové do čtvrtého pokolení budou sedět na izraelském trůnu.“ Tak se i stalo.
#15:13 V třicátém devátém roce vlády juského krále Uzijáše se stal králem Šalům, syn Jábešův. Kraloval v Samaří jeden měsíc.
#15:14 Potom přitáhl Menachém, syn Gadíův, z Tirsy. Přišel do Samaří, ubil v Samaří Šalúna, syna Jábešova, k smrti a stal se místo něho králem.
#15:15 O ostatních příbězích Šalúmových i o spiknutí, které zosnoval, se píše v Knize letopisů králů izraelských.
#15:16 Tehdy Menachém vybil Tipsach a všechno, co v něm bylo, i jeho území už od Tirsy; vybil je, poněvadž mu neotevřelo brány. Všechny těhotné v něm poroztínal.
#15:17 V třicátém devátém roce vlády judského krále Azarjáše se stal králem Menachém, syn Gadíův. Kraloval nad Izraelem v Samaří deset let.
#15:18 Dopouštěl se toho, co je zlé v Hospodinových očích. Po všechny dny své vlády neupustil od hříchů Jarobeáma, syna Nebatova, který svedl Izraele k hříchu.
#15:19 Do země vtrhl Púl, král asyrský. Menachém dal Púlovi tisíc talentů stříbra, aby si zajistil jeho pomoc a upevnil království ve svých rukou.
#15:20 Menachém vymáhal to stříbro na Izraeli, na všech bohatýrských válečnících, aby je mohl odvést asyrskému králi, padesát šekelů stříbra na jednoho muže. Asyrský král odtáhl a nezůstal tam v zemi.
#15:21 O ostatních příbězích Menachémových, o všem, co konal, se píše, jak známo, v Knize letopisů králů izraelských.
#15:22 I ulehl Menachém ke svým otcům. Po něm se stal králem jeho syn Pekachjáš.
#15:23 V padesátém roce vlády judského krále Azarjáše se stal králem nad Izraelem v Samaří Pekachjáš, syn Menachémův. Kraloval dva roky.
#15:24 Dopouštěl se toho, co je zlé v Hospodinových očích. Neupustil od hříchů Jarobeáma, syna Nebatova, který svedl Izraele k hříchu.
#15:25 Tu se proti němu spikl jeho štítonoš Pekach, syn Remaljášův, a ubil ho v Samaří v paláci královského domu, též Argóba a Arjéha. Bylo s ním padesát mužů, Gileádovců. Usmrtil ho a stal se místo něho králem.
#15:26 O ostatních příbězích Pekachjášových, o všem, co konal, se píše v Knize letopisů králů izraelských.
#15:27 V padesátém druhém roce vlády judského krále Azarjáše se stal králem nad Izraelem v Samaří Pekach, syn Remaljášův. Kraloval dvacet let.
#15:28 Dopouštěl se toho, co je zlé v Hospodinových očích. Neupustil od hříchů Jarobeáma, syna Nebatova, který svedl Izraele k hříchu.
#15:29 Za dnů Pekacha, krále izraelského, vtrhl do země Tiglat-pileser, král asyrský. Zabral Ijón, Ábel-bét-maaku, Janóach, Kedeš, Chasór, Gileád a Galileu, celou zemi Neftalí, a vystěhoval je do Asýrie.
#15:30 Hóšea, syn Élův, zosnoval spiknutí proti Pekachovi, synu Remaljášovu. Ubil ho k smrti a stal se místo něho králem v dvacátém roce vlády Jótama, syna Uzijášova.
#15:31 O ostatních příbězích Pekachových, o všem, co konal, se píše v Knize letopisů králů izraelských.
#15:32 V druhém roce vlády izraelského krále Pekacha, syna Remaljášova, se stal králem Jótam, syn Uzijášův, král judský.
#15:33 Bylo mu dvacet pět let, když začal kralovat, a kraloval v Jeruzalémě šestnáct let. Jeho matka se jmenovala Jerúša; byla to dcera Sádokova.
#15:34 Činil to, co je správné v Hospodinových očích, zcela jak to činil jeho otec Uzijáš.
#15:35 Avšak posvátná návrší neodstranili, lid dále obětoval a pálil na posvátných návrších kadidlo. Vystavěl Horní bránu Hospodinova domu.
#15:36 O ostatních příbězích Jótamových, o všem, co konal, se píše, jak známo, v Knize letopisů králů judských.
#15:37 V oněch dnech začal Hospodin posílat proti Judovi Resína, krále aramejského, a Pekacha, syna Remaljášova.
#15:38 I ulehl Jótam ke svým otcům a byl pohřben u svých otců v městě svého otce Davida. Po něm se stal králem jeho syn Achaz. 
#15:39 
#16:1 V sedmnáctém roce vlády Pekacha, syna Remaljášova, se stal králem Achaz, syn Jótamův, král judský.
#16:2 Achazovi bylo dvacet let, když začal kralovat, a kraloval v Jeruzalémě šestnáct let. Nečinil, co je správné v očích Hospodina, jeho Boha, jako činil jeho otec David.
#16:3 Chodil po cestě králů izraelských. Dokonce dal svého syna provést ohněm podle ohavností pronárodů, které Hospodin před Izraelci vyhnal.
#16:4 Obětoval a pálil kadidlo na posvátných návrších a na pahorcích a pod každým zeleným stromem.
#16:5 Tehdy vytáhli Resín, král aramejský, a Pekach, syn Remaljášův, král izraelský, do boje proti Jeruzalému. Oblehli Achaza, ale nic proti němu v boji nezmohli.
#16:6 Toho času získal Resín, král aramejský, nazpět Élat pro Aram a vypudil Judejce z Élatu. Do Élatu vstoupili Edómci a sídlí tam až dodnes.
#16:7 Proto poslal Achaz posly k Tiglat-pileserovi, králi asyrskému, s prosbou: „Jsem tvůj otrok a tvůj syn. Vytáhni a vysvoboď mě ze spárů krále aramejského a ze spárů krále izraelského, kteří proti mně povstali.“
#16:8 Achaz vzal stříbro a zlato, které se nacházelo v Hospodinově domě a mezi poklady domu královského, a poslal je králi asyrskému darem.
#16:9 Asyrský král jej vyslyšel. I vytáhl asyrský král proti Damašku, zmocnil se ho a vystěhoval jej do Kíru. Resína dal usmrtit.
#16:10 Král Achaz se odebral do Damašku, aby se tam setkal s Tiglat-pileserem, králem asyrským. Když uviděl oltář, který byl v Damašku, poslal král Achaz knězi Urijášovi nákres oltáře a plán všech prací.
#16:11 Kněz Urijáš postavil tedy oltář přesně tak, jak vzkázal král Achaz z Damašku. Kněz Urijáš tak učinil ještě před příchodem krále Achaza z Damašku.
#16:12 Když král z Damašku přišel a viděl oltář, přistoupil k oltáři a obětoval na něm.
#16:13 Zapálil na něm svou zápalnou oběť a oběť přídavnou, vykonal svou úlitbu a pokropil oltář krví z obětí pokojných.
#16:14 Bronzový oltář, který byl před Hospodinem, dal přemístit z místa před Hospodinovým domem, z místa mezi novým oltářem a Hospodinovým domem, a umístil jej stranou nového oltáře, k severu.
#16:15 Potom přikázal král Achaz knězi Urijášovi: „Na velkém oltáři budeš zapalovat jitřní oběti zápalné a přídavnou oběť večerní, ale i zápalnou oběť královu i jeho oběť přídavnou; dále zápalnou oběť všeho lidu země a jejich oběť přídavnou a jejich úlitby. A budeš jej kropit veškerou krví zápalné oběti a veškerou krví obětního hodu. Bronzový oltář mi bude sloužit k drobopravectví.“
#16:16 Kněz Urijáš učinil všechno přesně podle příkazu krále Achaza.
#16:17 Potom král Achaz odsekal lišty podstavců a sňal z nich nádrže. Dal též sejmout moře z bronzových býků, kteří byli pod ním, a dal je na kamenné dláždění.
#16:18 Dal též pozměnit u Hospodinova domu krytou chodbu dne odpočinku, kterou postavili při domě, i zvláštní královský vchod kvůli králi asyrskému.
#16:19 O ostatních příbězích Achazových, o tom, co konal, se píše, jak známo, v Knize letopisů králů judských.
#16:20 I ulehl Achaz ke svým otcům a byl pohřben vedle svých otců v Městě Davidově. Po něm se stal králem jeho syn Chizkijáš. 
#17:1 Ve dvanáctém roce vlády judského krále Achaza se stal králem nad Izraelem v Samaří Hóšea, syn Élův. Kraloval devět let.
#17:2 Dopouštěl se toho, co je zlé v Hospodinových očích, ne však jako izraelští králové, kteří byli před ním.
#17:3 Proti němu vytáhl Šalmaneser, král asyrský, a Hóšea se stal jeho služebníkem a musel mu odvádět dávky.
#17:4 Potom odhalil asyrský král Hóšeovo spiknutí, že totiž poslal posly k Sóovi, králi egyptskému, a asyrskému králi už nepřinášel pravidelné roční dávky. Proto jej asyrský král spoutal a vsadil do žaláře.
#17:5 Asyrský král táhl celou zemí, přitáhl až k Samaří a obléhal je tři léta.
#17:6 V devátém roce Hóšeovy vlády dobyl asyrský král Samaří. Vystěhoval Izraele do Asýrie a usadil jej v Chalachu a Chabóru při řece Gózanu a v městech médských.
#17:7 Stalo se to proto, že synové Izraele hřešili proti Hospodinu, svému Bohu, který je vyvedl z egyptské země z područí faraóna, krále egyptského. Oni se však báli bohů jiných
#17:8 a řídili se zvyklostmi pronárodů, které Hospodin před Izraelci vyhnal, i ustanoveními, která vydali králové izraelští.
#17:9 Izraelci podnikali věci, které nemohly obstát před Hospodinem, jejich Bohem: nastavěli si posvátná návrší ve všech svých městech, jak při hlídkové věži, tak v opevněném městě,
#17:10 postavili si posvátné sloupy a kůly na kdejakém vysokém pahorku a pod kdejakým zeleným stromem
#17:11 a pálili tam na všech posvátných návrších kadidlo jako pronárody, které Hospodin před nimi vystěhoval. Dopouštěli se zlých věcí, a tak uráželi Hospodina.
#17:12 Sloužili hnusným modlám, ačkoliv jim Hospodin řekl: „Nic takového nedělejte.“
#17:13 Hospodin varoval Izraele i Judu skrze každého proroka a každého vidoucího: „Odvraťte se od svých zlých cest, dbejte na mé příkazy a na má nařízení podle celého zákona, který jsem přikázal vašim otcům a který jsem vám poslal skrze své služebníky proroky.“
#17:14 Neuposlechli, ale byli tvrdošíjní jako jejich otcové, kteří nevěřili Hospodinu, svému Bohu.
#17:15 Zavrhli jeho nařízení i jeho smlouvu, kterou uzavřel s jejich otci, i jeho výstrahy, jimiž je varoval, a chodili za modlářským přeludem a přeludem se stali, následovali pronárody, které byly kolem nich, ačkoli jim Hospodin přikázal, aby byly kolem nich, ačkoli jim Hospodin přikázal, aby nejednali jako ony.
#17:16 Ale oni opustili všechna přikázání Hospodina, svého Boha, a odlili si sochu, dva býčky, udělali si posvátný kůl, klaněli se veškerému nebeskému zástupu a sloužili Baalovi.
#17:17 Prováděli své syny a dcery ohněm, obírali se věštěním a hadačstvím a propůjčovali se k činům, které jsou zlé v Hospodinových očích, a tak ho uráželi.
#17:18 Hospodin se na Izraele velice rozhněval a odvrhl jej od své tváře, nezůstal nikdo, jenom kmen Juda.
#17:19 Ale ani Juda nedbal na příkazy Hospodina, svého Boha. Řídil se zvyklostmi, které zavedl Izrael.
#17:20 Proto Hospodin všechno potomstvo Izraele zavrhl, pokořil je a vydal je do rukou plenitelů, až je vůbec od sebe odmrštil.
#17:21 Když odtrhl Izraele od domu Davidova, dosadili za krále Jarobeáma, syna Nebatova. Jarobeám odvrátil Izraele od Hospodina a svedl jej k velikému hříchu.
#17:22 A Izraelci chodili ve všech hříších Jarobeámových, kterých se dopustil, a neupustili od nich.
#17:23 Proto Hospodin odvrhl Izraele od své tváře, jak mu předpovídal skrze všechny své služebníky proroky. Vystěhoval Izraele z jeho země do Asýrie, jak je tomu dodnes.
#17:24 Asyrský král přivedl lid z Babylóna, Kútu, Avy, Chamátu a Sefarvajimu a usadil jej v samařských městech místo Izraelců. Dostali do vlastnictví Samařsko a sídlili v jeho městech.
#17:25 Tam se na začátku svého pobytu Hospodina nebáli; proto na ně Hospodin poslal lvy, kteří je dávili.
#17:26 Tu řekli asyrskému králi: „Pronárody, které jsi přestěhoval a usadil v samařských městech, neznají řád Boha té země. Proto na ně poslal lvy a ti jim přinášejí smrt, protože neznají řád Boha té země.“
#17:27 Asyrský král přikázal: „Doveďte tam jednoho z kněží, které jste odtud vystěhovali. Ať lidé jdou a bydlí tam a on ať je učí řádu Boha té země!“
#17:28 Tak přišel jeden z kněží, které vystěhovali ze Samaří, a usadil se v Bét-elu. Učil je, jak by se měli bát Hospodina.
#17:29 Každý pronárod si udělal svého boha a postavil jej v domě na posvátných návrších, která udělali Samařané. Každý pronárod to udělal ve svém městě, v němž bydlel.
#17:30 Muži babylónští udělali Sukót-benóta, muži z Kútu udělali Nergala, muži z Chamátu udělali Ašímu.
#17:31 Avíjci udělali Nibchaza a Tartaka, Sefarvajci spalovali své syny v ohni Adramelekovi a Anamelekovi, božstvům sefarvajským.
#17:32 Báli se sice i Hospodina, ale dělali si kněze posvátných návrší ze spodiny lidu. Ti za ně obětovali v domě posvátných návrší.
#17:33 Báli se Hospodina, ale sloužili rovněž svým božstvům podle obyčeje těch pronárodů, z nichž byli přestěhováni.
#17:34 Až dodnes jednají podle dřívějších obyčejů; Hospodina se nebojí a nejednají podle příslušných nařízení a příslušného řádu, podle zákona a podle přikázání, jež vydal Hospodin synům Jákoba, jemuž dal jméno Izrael.
#17:35 S těmi Hospodin uzavřel smlouvu a přikázal jim: „Nebudete se bát jiných bohů, nebudete se jim klanět ani jim sloužit ani jim obětovat.
#17:36 Jenom Hospodina, který vás vyvedl z egyptské země s velikou silou a se vztaženou paží, se budete bát, jemu se budete klanět a jemu obětovat.
#17:37 Budete dbát na nařízení a řády, na zákon i přikázání, které pro vás napsal, a plnit je po všechny dny. Nebudete se bát jiných bohů.
#17:38 Nezapomínejte na smlouvu, kterou jsem s vámi uzavřel, a jiných bohů se nebojte.
#17:39 Budete-li se bát Hospodina, svého Boha, on vás vysvobodí z rukou všech vašich nepřátel.“
#17:40 Ale oni neuposlechli, nýbrž jednali podle svého dřívějšího obyčeje.
#17:41 A tak se ty pronárody sice bály Hospodina, ale přitom sloužily svým tesaným modlám i se svými syny a vnuky. Jednají dodnes tak, jak jednali jejich otcové. 
#18:1 V třetím roce vlády izraelského krále Hóšey, syna Élova, se stal králem Chizkijáš, syn Achazův, král judský.
#18:2 Bylo mu dvacet pět let, když začal kralovat, a kraloval v Jeruzalémě dvacet devět let. Jeho matka se jmenovala Abí; byla to dcera Zekarjášova.
#18:3 Činil to, co je správné v Hospodinových očích, zcela jak to činil jeho otec David.
#18:4 Odstranil posvátná návrší, rozbil posvátné sloupy, skácel posvátný kůl, na kusy roztloukl bronzového hada, kterého udělal Mojžíš a jemuž až do oněch dnů Izraelci pálili kadidlo; nazvali jej Nechuštán.
#18:5 Doufal v Hospodina, Boha Izraele. Po něm už nebyl mezi všemi judskými králi žádný jemu podobný, ani mezi těmi, kteří byli před ním.
#18:6 Přimkl se k Hospodinu, neodstoupil od něho, dbal na jeho přikázání, jak je Hospodin vydal Mojžíšovi.
#18:7 A Hospodin byl s ním. Ve všem, co podnikl, měl úspěch. Vzbouřil se i proti králi asyrskému a nesloužil mu.
#18:8 Vybil Pelištejce až po Gázu, dobyl jejich území, jak hlídkovou věž, tak opevněné město.
#18:9 Ve čtvrtém roce vlády krále Chizkijáše, což byl sedmý rok vlády izraelského krále Hóšey, syna Élova, vytáhl Šalmaneser, král asyrský, proti Samaří a oblehl je.
#18:10 Dobyli je koncem třetího roku. V šestém roce Chizkijášovy vlády, což byl devátý rok vlády izraelského krále Hóšey, bylo Samaří dobyto.
#18:11 Asyrský král vystěhoval Izraele do Asýrie a usadil jej v Chalachu a v Chabóru při řece Gózanu a v městech médských.
#18:12 To proto, že neposlouchali Hospodina, svého Boha, a přestupovali jeho smlouvu. Neposlouchali a nečinili nic z toho, co přikázal Mojžíš, služebník Hospodinův.
#18:13 V čtrnáctém roce vlády krále Chizkijáše vytáhl Sancheríb, král asyrský, proti všem opevněným městům judským a zmocnil se jich.
#18:14 Proto poslal Chizkijáš, král judský, poselství asyrskému králi do Lakíše: „Prohřešil jsem se, odejdi ode mne. Cokoli na mne vložíš, ponesu.“ I uložil král asyrský judskému králi Chizkijášovi tři sta talentů stříbra a třicet talentů zlata.
#18:15 Chizkijáš tedy odevzdal všechno stříbro, které se nacházelo v Hospodinově domě a mezi poklady domu královského.
#18:16 V onen čas odsekal Chizkijáš zlato z dveří Hospodinova chrámu i z pažení dveří, které kdysi dal on, judský král, sám potáhnout, a vydal je asyrskému králi.
#18:17 Asyrský král poslal nejvyššího velitele a nejvyššího dvořana a nejvyššího číšníka z Lakíše ke králi Chizkijášovi do Jeruzaléma se silným vojskem. Vytáhli a přišli k Jeruzalému. Vytáhli, přišli a zastavili se u strouhy Horního rybníka, který je u silnice k Valchárovu poli.
#18:18 Zavolali na krále. Vyšel k nim Eljakím, syn Chilkijášův, který byl správcem domu, Šebna, písař, a Jóach, syn Asafův, kancléř.
#18:19 Nejvyšší číšník na ně zavolal: „Povězte Chizkijášovi: Toto praví velkokrál, král asyrský: Na co vlastně spoléháš?
#18:20 Říkáš: ‚Pouhé slovo přinese radu i zmužilost k boji.‘ Nuže, na koho spoléháš, že se proti mně bouříš?
#18:21 Hle, spoléháš se teď na Egypt, na tu nalomenou třtinovou hůl. Kdokoli se o ni opře, tomu pronikne dlaní a propíchne ji. Takový je farao, král egyptský, vůči všem, kteří na něho spoléhají.
#18:22 Řeknete snad: ‚Spoléháme na Hospodina, svého Boha‘, ale je známo, že Chizkijáš odstranil jeho posvátná návrší a jeho oltáře a že Judovi a Jeruzalému poručil: ‚Pouze před tímto oltářem v Jeruzalémě se budete klanět.‘
#18:23 Vsaď se tedy nyní s mým pánem, asyrským králem. Dám ti dva tisíce koní, dokážeš-li k nim sehnat jezdce.
#18:24 Jak bys mohl odrazit jediného místodržitele z nejmenších služebníků mého pána, i když se spoléháš na Egypt pro jeho vozbu a jízdu!
#18:25 Cožpak jsem vytáhl bez Hospodina proti tomuto místu, abych je zničil? Hospodin mi nařídil: Vytáhni proti této zemi a znič ji!“
#18:26 Eljakím, syn Chilkijášův, a Šebna a Jóach odpověděli nejvyššímu číšníkovi: „Mluv raději ke svým služebníkům aramejsky, vždyť rozumíme. Nemluv s námi judsky, aby to neslyšel lid, který je na hradbách.“
#18:27 Ale nejvyšší číšník jim odvětil: „Zdalipak mě můj pán poslal, abych mluvil tato slova k tvému pánovi a k tobě? Zdali ne k mužům, kteří obsadili hradby a budou jíst svá lejna a pít svou moč spolu s vámi?“
#18:28 Nejvyšší číšník se postavil a volal judsky co nejhlasitěji. Křičel: „Slyšte slovo velkokrále, krále asyrského!
#18:29 Toto praví král: Ať vás Chizkijáš nepodvádí, protože vás nedokáže vysvobodit z jeho rukou!
#18:30 Ať vás Chizkijáš nevede k spoléhání na Hospodina slovy: ‚Hospodin nás určitě vysvobodí a toto město nebude vydáno do rukou asyrského krále!‘
#18:31 Neposlouchejte Chizkijáše. Toto praví král asyrský: Sjednejte se mnou dohodu a vyjděte ke mně. Každý z vás bude jíst ze své vinné révy a ze svého fíkovníku a pít vodu ze své cisterny,
#18:32 dokud nepřijdu a nevezmu vás do země stejné, jako je země vaše, do země obilí a moštu, do země chleba a vinic, do země oliv, oleje a medu. Budete žít, nezemřete. Neposlouchejte Chizkijáše. Jen vás podněcuje slovy: ‚Hospodin nás vysvobodí.‘
#18:33 Zdali někdo z bohů pronárodů vysvobodil svou zemi z rukou asyrského krále?
#18:34 Kde byli bohové Chamátu a Arpádu? Kde byli bohové Sefarvajimu, Heny a Ivy? Což vysvobodili Samaří z mých rukou?
#18:35 Který ze všech bohů těch zemí vysvobodil svou zemi z mých rukou? Že by Hospodin vysvobodil z mých rukou Jeruzalém?“
#18:36 Lid mlčel. Ani slůvkem mu neodpovídal. Králův příkaz totiž zněl: „Neodpovídejte mu.“
#18:37 Eljakím, syn Chilkijášův, správce domu, Šebna, písař, a Jóach, syn Asafův, kancléř, přišli k Chizkijášovi s roztrženým rouchem a oznámili mu slova nejvyššího číšníka. 
#19:1 Když to král Chizkijáš uslyšel, roztrhl své roucho, zahalil se žíněnou suknicí a vešel do Hospodinova domu.
#19:2 Poslal Eljakíma, který byl správcem domu, písaře Šebnu a starší z kněží zahalené žíněnými suknicemi k proroku Izajášovi, synu Amósovu.
#19:3 Měli mu vyřídit: „Toto praví Chizkijáš: Tento den je den soužení, trestání a ponižování; plod je připraven vyjít z lůna, ale rodička nemá sílu k porodu.
#19:4 Kéž Hospodin, tvůj Bůh, slyší všechna slova nejvyššího číšníka, kterého poslal jeho pán, král asyrský, aby haněl Boha živého. Kéž jej Hospodin, tvůj Bůh, potrestá za ta slova, která slyšel. Pozdvihni hlas k modlitbě za pozůstatek lidu, který tu je.“
#19:5 Služebníci krále Chizkijáše přišli k Izajášovi.
#19:6 Izajáš jim řekl: „Vyřiďte svému pánu: Toto praví Hospodin: Neboj se těch slov, která jsi slyšel, jimiž mě hanobili sluhové asyrského krále.
#19:7 Hle, uvedu do něho ducha, uslyší zprávu a vrátí se do své země. V jeho zemi jej nechám padnout mečem.“
#19:8 Když se nejvyšší číšník vracel, uslyšel, že asyrský král odtáhl od Lakíše. Zastihl ho, jak bojuje proti Libně.
#19:9 Asyrský král totiž uslyšel o Tirhákovi, králi kúšském: „Hle, vytáhl proti tobě do boje!“ Poslal znovu k Chizkijášovi posly se vzkazem:
#19:10 „Toto vyřiďte Chizkijášovi, králi judskému: Ať tě nepodvede tvůj Bůh, na něhož spoléháš, že Jeruzalém nebude vydán do rukou asyrského krále.
#19:11 Hle, slyšel jsi o tom, co učinili králové asyrští všem zemím, že je zničili jako klaté. A ty bys byl vysvobozen?
#19:12 Zda bohové těch pronárodů, jimž přinesli zkázu moji otcové, mohli vysvobodit Gózan, Cháran, Resef a syny Edenu, kteří byli v Telasáru?
#19:13 Kde je král Chamátu a král Arpádu a král města Sefarvajimu, Heny a Ivy?“
#19:14 Chizkijáš vzal dopisy z ruky poslů, přečetl je a pak vstoupil do Hospodinova domu a rozložil je před Hospodinem.
#19:15 Chizkijáš se před Hospodinem modlil: „Hospodine, Bože Izraele, který sídlíš nad cheruby, ty sám jsi Bůh nade všemi královstvími země. Ty jsi učinil nebesa i zemi.
#19:16 Nakloň, Hospodine, své ucho a slyš, otevři, Hospodine, své oči a viz! Slyš slova Sancheríbova, která vzkázal, aby haněl Boha živého.
#19:17 Opravdu, Hospodine, králové asyrští zničili pronárody i jejich země.
#19:18 Jejich bohy vydali ohni, protože to nejsou bohové, nýbrž dílo lidských rukou, dřevo a kámen, proto je zničili.
#19:19 Ale teď Hospodine, Bože náš, zachraň nás prosím z jeho rukou, ať poznají všechna království země, že ty, Hospodine, jsi Bůh, ty sám!“
#19:20 I vzkázal Izajáš, syn Amósův, Chizkijášovi: „Toto praví Hospodin, Bůh Izraele: Slyšel jsem, když ses ke mně modlil kvůli Sancheríbovi, králi asyrskému.
#19:21 Toto je slovo, které o něm promluvil Hospodin: Pohrdá tebou, vysmívá se ti panna, dcera sijónská! Potřásá nad tebou hlavou dcera jeruzalémská.
#19:22 Koho jsi haněl a hanobil? Proti komu jsi povýšil hlas a oči pyšně vzhůru zvedl? Proti Svatému Izraele.
#19:23 Svými posly jsi pohaněl Panovníka, když jsi řekl: ‚Se svou nespočetnou vozbou jsem vytáhl do horských výšin na libanónské stráně. Pokácím tam statné cedry, skvělé cypřiše, vstoupím do jeho nejzazších koutů, do křovin a hájů.
#19:24 Já jsem vykopal studny a napil se cizí vody, svými chodidly jsem vysušil všechny průplavy Egypta.‘
#19:25 Což jsi neslyšel, že zdávna jsem to připravoval, už za dnů dávnověkých to chystal? Nyní to uskutečním. V hromady sutin se změní opevněná města.
#19:26 Jejich bezmocní obyvatelé jsou vyděšeni a zostuzeni, jsou jako bylina polní, zelenající se býlí, tráva na střechách, rez v obilí nepožatém.
#19:27 Vím o tobě, ať sedíš či vycházíš a vcházíš, jak proti mně běsníš.
#19:28 Protože proti mně tak běsníš a tvá drzost stoupá do mých uší, provleču ti chřípím kruh a do úst vložím uzdu. Odvedu tě cestou, po níž jsi přišel.
#19:29 Toto ti bude znamením: V tomto roce budete jíst, co vyroste samo, i druhý rok to, co samo vzejde, ale třetí rok sejte a sklízejte, vysazujte vinice a jezte jejich plody.
#19:30 Ti z Judova domu, kteří vyváznou a zůstanou, opět se zakoření a vydají ovoce.
#19:31 Z Jeruzaléma vyjde pozůstatek lidu a z hory Sijónu ti, kdo vyvázli. Horlivost Hospodina zástupů to učiní!“
#19:32 Proto praví Hospodin o králi asyrském toto: „Nevejde do tohoto města. Ani šíp tam nevstřelí, se štíty proti němu nenastoupí, násep proti němu nenavrší.
#19:33 Cestou, kterou přišel, se zase vrátí, do tohoto města nevejde, je výrok Hospodinův.
#19:34 Budu štítem tomuto městu, zachráním je kvůli sobě a kvůli Davidovi, svému služebníku.“
#19:35 Stalo se pak té noci, že vyšel Hospodinův anděl a pobil v asyrském táboře sto osmdesát pět tisíc. Za časného jitra, hle, všichni byli mrtví, všude mrtvá těla.
#19:36 Sancheríb, král asyrský, odtáhl pryč a vrátil se do Ninive a usadil se tam.
#19:37 Když se klaněl v chrámě svého boha Nisroka, Adramelek a Sareser, jeho synové, ho zabili mečem a unikli do země Araratu. Po něm se stal králem jeho syn Esarchadón. 
#20:1 V oněch dnech Chizkijáš smrtelně onemocněl. Přišel k němu prorok Izajáš, syn Amósův, a řekl mu: „Toto praví Hospodin: Udělej pořízení o svém domě, protože zemřeš, nebudeš žít.“
#20:2 Chizkijáš se otočil tváří ke zdi a takto se k Hospodinu modlil:
#20:3 „Ach, Hospodine, rozpomeň se prosím, že jsem chodil před tebou opravdově a se srdcem nerozděleným a že jsem činil, co je dobré v tvých očích.“ A Chizkijáš se dal do velikého pláče.
#20:4 Izajáš ještě nevyšel z vnitřního dvora, když se k němu stalo slovo Hospodinovo:
#20:5 „Vrať se a vyřiď Chizkijášovi, vévodovi mého lidu: Toto praví Hospodin, Bůh Davida, tvého otce: Vyslyšel jsem tvou modlitbu, viděl jsem tvé slzy. Hle, uzdravím tě. Třetího dne vstoupíš do Hospodinova domu.
#20:6 Přidám k tvým dnům patnáct let. Vytrhnu tebe i toto město ze spárů asyrského krále. Budu štítem tomuto městu kvůli sobě a kvůli Davidovi, svému služebníku.“
#20:7 Izajáš poručil: „Vezměte suché fíky.“ Vzali je a přiložili na vřed a Chizkijáš zůstal naživu.
#20:8 Chizkijáš se otázal Izajáše: „Co bude znamením, že mě Hospodin uzdraví a že třetího dne vstoupím do Hospodinova domu?“
#20:9 Izajáš mu odpověděl: „Toto ti bude znamením od Hospodina, že Hospodin splní slovo, jež promluvil: Má stín postoupit o deset stupňů nebo se má o deset stupňů vrátit?“
#20:10 Chizkijáš řekl: „Snadněji se stín o deset stupňů nachýlí, než aby se vrátil o deset stupňů nazpět.“
#20:11 Prorok Izajáš tedy volal k Hospodinu. A on vrátil stín na stupních, po nichž sestoupil, na stupních Achazových, o deset stupňů nazpět.
#20:12 V ten čas poslal Beródak-baladán, syn Baladánův, král babylónský, Chizkijášovy dopisy a dar, neboť uslyšel, že Chizkijáš onemocněl.
#20:13 Chizkijáš vyslechl posly a ukázal jim celou svou klenotnici, stříbro a zlato, různé balzámy, výborný olej i svou zbrojnici a všechno, co se nacházelo mezi jeho poklady. Nebylo nic, co by jim Chizkijáš ve svém domě a v celém svém vladařství nebyl ukázal.
#20:14 Prorok Izajáš přišel ke králi Chizkijášovi a zeptal se ho: „Co říkali ti muži a odkud k tobě přišli?“ Chizkijáš odpověděl: „Přišli z daleké země, z Babylóna.“
#20:15 Zeptal se: „Co viděli v tvém domě?“ Chizkijáš odvětil: „Viděli všechno, co je v mém domě. Nebylo nic, co bych jim nebyl ze svých pokladů ukázal.“
#20:16 I řekl Izajáš Chizkijášovi: „Slyš slovo Hospodinovo:
#20:17 Hle, přijdou dny a bude odneseno do Babylóna všechno, co je v tvém domě, poklady, které nahromadili tvoji otcové až do tohoto dne. Nic tu nezbude, praví Hospodin.
#20:18 I některé tvé syny, kteří z tebe vzejdou, které zplodíš, vezmou a stanou se kleštěnci v paláci krále babylónského.“
#20:19 Chizkijáš na to Izajášovi řekl: „Dobré je slovo Hospodinovo, které jsi mluvil.“ A dodal: „Jen když bude za mých dnů opravdový pokoj.“
#20:20 O ostatních příbězích Chizkijášových, o všech jeho bohatýrských činech a jak udělal rybník a vodovod, jímž přivedl vodu do města, se píše, jak známo, v Knize letopisů králů judských.
#20:21 I ulehl Chizkijáš ke svým otcům. Po něm se stal králem jeho syn Menaše. 
#21:1 Menašemu bylo dvanáct let, když začal kralovat, a kraloval v Jeruzalémě padesát pět let. Jeho matka se jmenovala Chefsíbah.
#21:2 Dopouštěl se toho, co je zlé v Hospodinových očích, podle ohavností pronárodů, které Hospodin před Izraelci vyhnal.
#21:3 Znovu vybudoval posvátná návrší, která jeho otec Chizkijáš zničil, nastavěl oltáře Baalovi a udělal posvátný kůl, jak to činil Achab, král izraelský. Klaněl se veškerému nebeskému zástupu a sloužil mu.
#21:4 Zbudoval oltáře v Hospodinově domě, o němž Hospodin řekl: „V Jeruzalémě dám spočinout svému jménu.“
#21:5 Na obou nádvořích Hospodinova domu nastavěl oltáře veškerému nebeskému zástupu.
#21:6 Svého syna provedl ohněm, věštil z oblaků a obíral se hadačstvím, ustanovil vyvolávače duchů a jasnovidce; dopouštěl se mnohého, co je zlé v očích Hospodina, a tak ho urážel.
#21:7 Tesanou modlu Ašéry, kterou udělal, umístil do domu, o němž řekl Hospodin Davidovi a jeho synu Šalomounovi: „V tomto domě a v Jeruzalémě, který jsem vyvolil ze všech izraelských kmenů, dám navěky spočinout svému jménu.
#21:8 Už nikdy nedopustím, aby noha Izraele byla vyhnána z půdy, kterou jsem dal jejich otcům, jen když budou bedlivě činit všechno, jak jsem jim přikázal, podle celého zákona, který jim přikázal můj služebník Mojžíš.“
#21:9 Neposlechli. Menaše je svedl, že se dopouštěli horších věcí než pronárody, které Hospodin před Izraelem vyhladil.
#21:10 Ale Hospodin mluvil skrze své služebníky proroky:
#21:11 „Protože se judský král Menaše dopouští těchto ohavností, horších, než jakých se dopouštěli před ním Emorejci, že svedl k hříchu svými hnusnými modlami též Judu,
#21:12 toto praví Hospodin, Bůh Izraele: Hle, já uvedu zlo na Jeruzalém a na Judu. Každému, kdo o tom uslyší, bude znít v obou uších.
#21:13 Nad Jeruzalémem natáhnu měřicí šňůru jako nad Samařím a spustím olovnici jako na dům Achabův. Vydrhnu Jeruzalém, jako se vydrhne mísa, vydrhne se a překlopí.
#21:14 Pozůstatek svého dědictví zavrhnu, vydám je do rukou jeho nepřátel, v lup a plen všem jeho nepřátelům,
#21:15 protože se dopouštějí toho, co je zlé v mých očích, a urážejí mě ode dne, kdy vyšli jejich otcové z Egypta, až dodnes.“
#21:16 Také nevinné krve prolil Menaše velice mnoho, až jí naplnil Jeruzalém od jednoho konce k druhému. Navíc tu byl jeho hřích, kterým svedl k hříchu Judu, takže se dopouštěl toho, co je zlé v Hospodinových očích.
#21:17 O ostatních příbězích Menašeho, o všem, co konal, i o jeho hříchu, jímž zhřešil, se píše, jak známo, v Knize letopisů králů judských.
#21:18 I ulehl Menaše ke svým otcům a byl pohřben v zahradě svého domu, v zahradě Uzově. Po něm se stal králem jeho syn Amón.
#21:19 Amónovi bylo dvaadvacet let, když začal kralovat, a kraloval v Jeruzalémě dva roky. Jeho matka se jmenovala Mešulemet; byla to dcera Charúsova z Jotby.
#21:20 Dopouštěl se toho, co je zlé v Hospodinových očích, jako se toho dopouštěl jeho otec Menaše.
#21:21 Chodil po všech cestách, po nichž chodil jeho otec, sloužil hnusným modlám, jimž sloužil jeho otec, a klaněl se jim.
#21:22 Opustil Hospodina, Boha svých otců, a nechodil po cestě Hospodinově.
#21:23 Amónovi služebníci se proti němu spikli a usmrtili krále v jeho domě.
#21:24 Ale lid země pobil ty, kteří se proti králi Amónovi spikli. Lid země pak dosadil místo něho za krále jeho syna Jóšijáše.
#21:25 O ostatních příbězích Amónových, o tom, co konal, se píše, jak známo, v Knize letopisů králů judských.
#21:26 Pohřbili ho v jeho hrobě v zahradě Uzově. Po něm se stal králem jeho syn Jóšijáš. 
#22:1 Jóšijášovi bylo osm let, když začal kralovat, a kraloval v Jeruzalémě jedenatřicet let. Jeho matka se jmenovala Jedída; byla to dcera Adajáše z Boskatu.
#22:2 Činil to, co je správné v Hospodinových očích, chodil po všech cestách svého otce Davida a neuchyloval se napravo ani nalevo.
#22:3 V osmnáctém roce vlády Jóšijášovy poslal král písaře Šáfana, syna Asaljáše, syna Mešulámova, do Hospodinova domu. Řekl mu:
#22:4 „Vystup k veleknězi Chilkijášovi, ať připraví stříbro, které bylo doneseno do Hospodinova domu a které od lidu vybrali strážci prahu.
#22:5 Vydá se těm, kdo pracují jako dohlížitelé v Hospodinově domě, a oni je budou dávat těm, kdo pracují při Hospodinově domě na opravách poškozené části domu:
#22:6 řemeslníkům, stavebním dělníkům a zedníkům, i k nákupu dřeva a tesaného kamene k opravě domu.
#22:7 Přitom není třeba dělat vyúčtování za stříbro, které se jim vydá, protože jednají poctivě.“
#22:8 Velekněz Chilkijáš řekl písaři Šáfanovi: „Nalezl jsem v Hospodinově domě knihu Zákona.“ Chilkijáš dal tu knihu Šáfanovi a on ji četl.
#22:9 Poté písař Šáfan vstoupil ke králi a podal králi hlášení. Řekl: „Tvoji služebníci vyzvedli stříbro, které se nacházelo v domě, a vydali je těm, kdo pracují jako dohlížitelé v Hospodinově domě.“
#22:10 Dále písař Šáfan králi oznámil: „Kněz Chilkijáš mi předal knihu.“ A Šáfan ji před králem četl.
#22:11 Když král uslyšel slova knihy Zákona, roztrhl své roucho.
#22:12 Potom král přikázal knězi Chilkijášovi, Achíkamovi, synu Šáfanovu, Akbórovi, synu Míkajášovu, písaři Šáfanovi a Asajášovi, královskému služebníku:
#22:13 „Jděte se dotázat Hospodina ohledně mne i lidu a celého Judska, pokud jde o slova této nalezené knihy. Vždyť je proti nám rozníceno veliké Hospodinovo rozhořčení za to, že naši otcové neposlouchali slova té knihy a nejednali podle toho všeho, co je v ní o nás napsáno.“
#22:14 Kněz Chilkijáš, Achíkam, Akbór, Šáfan a Asajáš se odebrali k prorokyni Chuldě, manželce Šalúma, syna Tikvy, syna Charchasova, strážce rouch. Bydlela v Jeruzalémě v Novém Městě. Mluvili s ní.
#22:15 Odvětila jim: „Toto praví Hospodin, Bůh Izraele: Vyřiďte muži, který vás ke mně poslal:
#22:16 Toto praví Hospodin: Hle, uvedu zlo na toto místo a na jeho obyvatele podle všech slov té knihy, kterou četl judský král.
#22:17 Protože mě opustili a jiným bohům pálili kadidlo, a tak mě uráželi vším tím, co svýma rukama udělali, roznítilo se mé rozhořčení na toto místo a neuhasne.
#22:18 A králi judskému, který vás poslal dotázat se Hospodina, vyřiďte: Toto praví Hospodin, Bůh Izraele: Pokud jde o slova, která jsi slyšel:
#22:19 Protože tvé srdce zjihlo a pokořil ses před Hospodinem, když jsi uslyšel, co jsem mluvil proti tomuto místu a proti jeho obyvatelům, že tu bude spoušť a zlořečení, protože jsi roztrhl své roucho a přede mnou plakal, vyslyšel jsem tě, je výrok Hospodinův.
#22:20 Proto tě připojím ke tvým otcům, budeš uložen do svého hrobu v pokoji a tvé oči nespatří nic z toho zla, které uvedu na toto místo.“ I podali toto hlášení králi. 
#23:1 Král obeslal všechny starší z Judy a z Jeruzaléma, aby se k němu shromáždili.
#23:2 Potom vystoupil král do Hospodinova domu a s ním všichni judští muži a všichni obyvatelé Jeruzaléma, i kněží a proroci, veškerý lid, malí i velicí. I předčítal jim všechna slova Knihy smlouvy, nalezené v Hospodinově domě.
#23:3 Poté se král postavil na své stanoviště a uzavřel před Hospodinem smlouvu, že budou následovat Hospodina a zachovávat jeho příkazy, svědectví a nařízení z celého srdce a z celé duše a plnit slova této smlouvy, jak jsou napsána v této knize. Všechen lid se za smlouvu postavil.
#23:4 Nato král přikázal veleknězi Chilkijášovi a jeho kněžským zástupcům i strážcům prahu, aby vynesli z Hospodinova chrámu všechny předměty zhotovené pro Baala, Ašéru a veškerý nebeský zástup. Venku za Jeruzalémem na polích kidróských je spálili a prach z nich odnesli do Bét-elu.
#23:5 Zakázal činnost žrecům, které dosadili judští králové a kteří pálili kadidlo na posvátných návrších judských měst a v okolí Jeruzaléma, i těm, kteří pálili kadidlo Baalovi, slunci, měsíci, souhvězdím a veškerému nebeskému zástupu.
#23:6 Dal vynést z Hospodinova domu posvátný kůl ven za Jeruzalém do Kidrónského úvalu. V Kidrónském úvalu jej spálil, rozdrtil na prach a prach z něho rozházel po hrobech prostého lidu.
#23:7 Zbořil domečky zasvěcenců bohyně lásky, které byly v prostoru Hospodinova domu, v nichž ženy tkaly stany pro Ašéru.
#23:8 Přivedl všechny kněze z judských měst a poskvrnil posvátná návrší, na nichž kněží pálili kadidlo, od Geby až po Beer-šebu. Zbořil posvátná návrší u bran, i to, které bylo u vchodu do brány Jóšuy, velitele města, nalevo, vchází-li se do městské brány.
#23:9 Kněží těchto posvátných návrší nesměli vystupovat k oltáři Hospodinovu v Jeruzalémě, směli ovšem jíst nekvašené chleby se svými bratřími.
#23:10 Poskvrnil i Tófet v Údolí syna Hinómova, aby už nikdo neprovedl svého syna nebo dceru ohněm k poctě Molekově.
#23:11 Odstranil koně, které judští králové darovali k poctě slunce, od vchodu do Hospodinova domu až po síň kleštěnce Netan-meleka, která byla ve sloupořadí; sluneční vozy spálil ohněm.
#23:12 Král také zbořil oltáře, které byly na střeše, v Achazově pokojíku, a které udělali judští králové, i oltáře, které udělal na obou nádvořích Hospodinova domu Menaše; pospíšil si odstranit je odtud a prach z nich dal rozházet v Kidrónském úvalu.
#23:13 Též posvátná návrší naproti Jeruzalému, jižně od Hory zkázy, jež vybudoval Šalomoun, král izraelský, pro Aštoretu, ohyzdnou modlu Sidóňanů, pro Kemóše, ohyzdnou modlu Moábců, a pro Milkóma, ohavnou modlu Amónovců, král poskvrnil.
#23:14 Posvátné sloupy roztříštil, posvátné kůly skácel a místa po nich naplnil lidskými kostmi.
#23:15 Také oltář v Bét-elu, posvátné návrší, které udělal Jarobeám, syn Nebatův, jenž svedl Izraele k hříchu, také tento oltář i s posvátným návrším zbořil; posvátné návrší spálil a rozdrtil na prach a spálil i posvátný kůl.
#23:16 Pak se Jóšijáš ohlédl a spatřil hroby, které byly na té hoře. Dal z těch hrobů vyzvednout kosti a spálil je na oltáři. Tak jej poskvrnil podle Hospodinova slova, které provolal muž Boží, jenž tyto události předpověděl.
#23:17 Král se otázal: „Co je to za náhrobek, který vidím?“ Mužové města mu odvětili: „To je hrob muže Božího, který přišel z Judy a předpověděl tyto události, to, co jsi učinil s bételským oltářem.“
#23:18 Poručil: „Nechte ho v klidu. Ať nikdo nehýbá jeho kostmi.“ Ušetřili tedy jeho kosti i kosti proroka, který přišel ze Samaří.
#23:19 Všechny domy na posvátných návrších v samařských městech, které udělali izraelští králové a jimiž uráželi Hospodina, Jóšijáš odstranil; naložil s nimi stejným způsobem, jako to učinil v Bét-elu.
#23:20 Všechny kněze posvátných návrší, kteří tam byli, obětoval na oltářích, na nichž spaloval lidské kosti. Potom se vrátil do Jeruzaléma.
#23:21 Král vydal rozkaz veškerému lidu: „Slavte hod beránka Hospodinu, svému Bohu, jak je psáno v této Knize smlouvy.“
#23:22 Takový hod beránka nebyl slaven ode dnů soudců, kteří soudili Izraele, ba ani po všechny dny králů izraelských a králů judských.
#23:23 Až v osmnáctém roce vlády krále Jóšijáše se v Jeruzalémě slavil takový Hospodinův hod beránka.
#23:24 Jóšijáš také vymýtil vyvolavače duchů zemřelých, jasnovidce, domácí bůžky, hnusné modly i všelijaké ohyzdné bůžky, které bylo možno vidět v zemi judské a v Jeruzalémě. Tak se naplnila slova zákona napsaná v knize, kterou nalezl kněz Chilkijáš v Hospodinově domě.
#23:25 Nebyl mu podoben žádný král před ním, který by se obrátil k Hospodinu celým svým srdcem a celou svou duší a celou svou silou a činil vše podle zákona Mojžíšova. A ani po něm nepovstal žádný jemu podobný.
#23:26 Avšak Hospodin neupustil od svého velikého planoucího hněvu, kterým vzplanul proti Judovi pro všechny urážky, jimiž ho urážel Menaše.
#23:27 Hospodin řekl: „Odvrhnu od sebe i Judu, jako jsem odvrhl Izraele. Zavrhnu toto město, které jsem vyvolil, Jeruzalém, i tento dům, o němž jsem prohlásil: Zde bude dlít mé jméno.“
#23:28 O ostatních příbězích Jóšijášových, o všem, co konal, se píše, jak známo, v Knize letopisů králů judských.
#23:29 Za jeho dnů táhl farao Néko, král egyptský, na pomoc králi asyrskému k řece Eufratu. Král Jóšijáš vyšel proti němu. Farao jej usmrtil v Megidu, sotva ho spatřil.
#23:30 Jeho služebníci jej mrtvého odvezli z Megida, přivezli do Jeruzaléma a pohřbili ho v jeho hrobě. Lid země vzal Jóšijášova syna Jóachaza, pomazali ho a dosadili za krále po jeho otci.
#23:31 Jóachazovi bylo třiadvacet let, když začal kralovat, a kraloval v Jeruzalémě tři měsíce. Jeho matka se jmenovala Chamútal; byla to dcera Jirmejáše z Libny.
#23:32 Dopouštěl se toho, co je zlé v Hospodinových očích, zcela tak, jak se toho dopouštěli jeho otcové.
#23:33 Farao Néko jej spoutal u Ribly v zemi Chamátu, aby už nekraloval v Jeruzalémě, a uvalil na zemi poplatek sto talentů stříbra a talent zlata.
#23:34 Farao Néko dosadil za krále Eljakíma, syna Jóšijášova, místo jeho otce Jóšijáše a změnil mu jméno na Jójakím. Jóachaza odvlekl do Egypta a on tam zemřel.
#23:35 Jójakím musel faraónovi odvádět stříbro a zlato. Proto vyměřil zemi daně, aby mohl odvádět příslušný obnos podle faraónova rozkazu. Podle svého výměru vymáhal stříbro a zlato na každém, i na lidu země, aby je mohl odvádět faraónovi Nékovi.
#23:36 Jójakímovi bylo dvacet pět let, když začal kralovat, a kraloval v Jeruzalémě jedenáct let. Jeho matka se jmenovala Zebúda; byla to dcera Pedajáše z Rúmy.
#23:37 Dopouštěl se toho, co je zlé v Hospodinových očích, zcela tak, jak se toho dopouštěli jeho otcové. 
#24:1 Za jeho dnů přitáhl Nebúkadnesar, král babylónský, a Jójakím byl po tři roky jeho služebníkem. Ale potom se proti němu vzbouřil.
#24:2 Hospodin na něho poslal hordy Kaldejců, hordy Aramejců, hordy Moábců a hordy Amónovců. Poslal je na Judu, aby jej zničil podle slova, které vyřkl Hospodin skrze své služebníky proroky.
#24:3 Ano, Judu postihlo to, co vyšlo z úst Hospodinových. Hospodin jej odehnal od své tváře za Menašeovy hříchy, za všechno, čeho se dopustil,
#24:4 i za nevinnou krev, kterou prolil a jíž naplnil Jeruzalém. To nehodlal Hospodin odpustit.
#24:5 O ostatních příbězích Jójakímových, o všem, co konal, se píše, jak známo, v Knize letopisů králů judských.
#24:6 I ulehl Jójakím ke svým otcům. Po něm se stal králem jeho syn Jójakín.
#24:7 A egyptský král už nikdy ze své země nevytáhl, protože babylónský král zabral od Egyptského potoka až k řece Eufratu všechno, co předtím patřilo králi egyptskému.
#24:8 Jójakínovi bylo osmnáct let, když začal kralovat, a kraloval v Jeruzalémě tři měsíce. Jeho matka se jmenovala Nechušta; byla to dcera Elnátanova z Jeruzaléma.
#24:9 Dopouštěl se toho, co je zlé v Hospodinových očích, zcela tak, jak se toho dopouštěl jeho otec.
#24:10 V té době přitáhli služebníci Nebúkadnesara, krále babylónského, k Jeruzalému. Město bylo obleženo.
#24:11 Potom dorazil Nebúkadnesar, král babylónský, k městu, které jeho služebníci obléhali.
#24:12 Jójakín, král judský, vyšel ke králi babylónskému, on i jeho matka, jeho služebníci, velitelé i dvořané. Král babylónský jej vzal do zajetí v osmém roce svého kralování.
#24:13 Odvezl odtud všechny poklady Hospodinova domu i poklady domu královského. Osekal všechny zlaté předměty, které dal zhotovit Šalomoun, král izraelský, pro Hospodinův chrám, podle slova Hospodinova.
#24:14 Přestěhoval celý Jeruzalém, všechny velitele, všechny udatné bohatýry, deset tisíc přesídlenců, všechny tesaře a kováře; kromě chudiny z lidu země tam nikdo nezůstal.
#24:15 Přestěhoval do Babylóna Jójakína i královu matku, královy ženy, jeho dvořany a přední v zemi, odvedl je jako přesídlence z Jeruzaléma do Babylóna.
#24:16 Odvedl i všechny válečníky, sedm tisíc mužů, a tesaře a kováře, tisíc mužů, vesměs muže schopné boje. Král babylónský je přivedl do Babylóna jako přesídlence.
#24:17 A Matanjáše, strýce Jójakínova, dosadil král babylónský za krále místo něho. Změnil mu jméno na Sidkijáš.
#24:18 Sidkijášovi bylo jedenadvacet let, když začal kralovat, a kraloval v Jeruzalémě jedenáct let. Jeho matka se jmenovala Chamútal; byla to dcera Jirmejáše z Libny.
#24:19 Dopouštěl se toho, co je zlé v Hospodinových očích, zcela tak, jak se toho dopouštěl Jójakím.
#24:20 Bylo to proto, že Hospodin svým hněvem stíhal Jeruzalém i Judu, až je od své tváře i zavrhl. Sidkijáš se vzbouřil proti babylónskému králi. 
#25:1 V devátém roce Sidkijášova kralování, v desátém měsíci, desátého dne toho měsíce, přitáhl Nebúkadnesar, král babylónský, s celým svým vojskem proti Jeruzalému, položil se proti němu a zbudovali dokola obléhací val.
#25:2 Město bylo obleženo do jedenáctého roku vlády krále Sidkijáše.
#25:3 Devátého dne čtvrtého měsíce, když už tvrdě doléhal na město hlad a lid země neměl co jíst,
#25:4 byly hradby města prolomeny. Všichni bojovníci uprchli v noci branou mezi hradbami u královské zahrady. Kolem celého města byli Kaldejci. Král se dal cestou k pustině.
#25:5 Kaldejské vojsko krále pronásledovalo a dostihlo ho na Jerišských pustinách. Celé jeho vojsko se od něho rozprchlo.
#25:6 Krále chytili a přivedli jej k babylónskému králi do Ribly, kde nad ním vynesli rozsudek.
#25:7 Sidkijášovy syny před jeho očima popravili; Sidkijáše oslepil a spoutal ho bronzovými řetězy a odvedl ho do Babylóna.
#25:8 Sedmého dne pátého měsíce v devatenáctém roce vlády Nebúkadnesara, krále babylónského, přitáhl do Jeruzaléma Nebúzaradán, velitel tělesné stráže, služebník babylónského krále.
#25:9 Vypálil Hospodinův dům i dům královský a všechny domy v Jeruzalémě; všechny význačné domy vypálil ohněm.
#25:10 Celé kaldejské vojsko, které bylo s velitelem tělesné stráže, zbořilo hradby kolem Jeruzaléma.
#25:11 Zbytek lidu, který zůstal v městě, a přeběhlíky, kteří přeběhli k babylónskému králi, ten zbývající houf, Nebúzaradán, velitel tělesné stráže, přestěhoval.
#25:12 Část chudiny země ponechal velitel tělesné stráže jako vinaře a rolníky.
#25:13 Bronzové sloupy, které byly u Hospodinova domu, podstavce i bronzové moře, které bylo v Hospodinově domě, Kaldejci rozbili a měď z nich odvezli do Babylóna.
#25:14 Pobrali i hrnce, lopatky, kleště na knoty a pánvičky a všechno bronzové náčiní potřebné k bohoslužbě.
#25:15 Také kadidelnice a kropenky, jak zlaté, tak stříbrné, velitel tělesné stráže pobral.
#25:16 Bronzu ze všech těch předmětů, ze dvou sloupů, jednoho moře a podstavců, které dal zhotovit Šalomoun pro Hospodinův dům, bylo tolik, že se nedal ani zvážit.
#25:17 Jeden sloup byl osmnáct loket vysoký a na něm byla bronzová hlavice tři lokte vysoká a kolem hlavice mřížování a granátová jablka. To vše bylo z bronzu. Právě tak vypadal druhý sloup včetně mřížování.
#25:18 Velitel tělesné stráže vzal Serajáše, hlavního kněze, i Sefanjáše, druhého kněze, a tři strážce prahu.
#25:19 Z města vzal jednoho dvořana, který dohlížel na bojovníky, a pět mužů z těch, kdo směli patřit na královu tvář, kteří byli v městě, i písaře, velitele vojska, jenž povolával lid země do služby, a šedesát mužů z lidu země, kteří byli ve městě.
#25:20 Velitel tělesné stráže Nebúzaradán je vzal a dovedl k babylónskému králi do Ribly.
#25:21 Babylónský král je dal v Rible v zemi chamátské popravit. Tak byl Juda přestěhován ze své země.
#25:22 Nad lidem, který byl ponechán v judské zemi, který tam ponechal Nebúkadnesar, král babylónský, ustanovil správcem Gedaljáše, syna Achíkama, syna Šáfanova.
#25:23 Když uslyšeli všichni velitelé vojsk i jejich mužstvo, že babylónský král ustanovil správcem Gedaljáše, přišli ke Gedaljášovi do Mispy. Byl to Jišmael, syn Netanjášův, Jóchanan, syn Karéachův, Serajáš, syn Tachumeta Netófské ho, a Jaazanjáš, syn Maakaťanův, každý se svým mužstvem.
#25:24 Gedaljáš je i jejich mužstvo přísežně ujistil: „Nebojte se stát se kaldejskými služebníky. Zůstaňte v zemi, služte babylónskému králi a povede se vám dobře.“
#25:25 Avšak v sedmém měsíci přišel Jišmael, syn Netanjáše, syna Elíšamova, z královského potomstva, s deseti muži a ubili Gedaljáše k smrti, i Judejce, kteří byli s ním v Mispě.
#25:26 Pak se všechen lid, malí i velicí, i velitelé vojsk sebrali a přitáhli do Egypta, protože se báli Kaldejců.
#25:27 Ve třicátém sedmém roce po přestěhování Jójakína, krále judského, dvacátého sedmého dne dvanáctého měsíce, udělil babylónský král Evíl- merodak v tom roce, kdy začal kralovat, milost Jójakínovi, králi judskému, a propustil ho z vězení.
#25:28 Mluvil s ním vlídně a jeho křeslo dal postavit výše než křesla králů, kteří byli u něho v Babylóně.
#25:29 Změnil také jeho vězeňský šat a on pak po všechny dny svého života jídal každodenně před ním chléb.
#25:30 Vše, co potřeboval, mu bylo každodenně králem poskytováno, den co den, po všechny dny jeho života.  

\book{I Chronicles}{1Chr}
#1:1 Adam, Šét, Enóš,
#1:2 Kénan, Mahalalel, Jered,
#1:3 Henoch, Metúšelach, Lámech,
#1:4 Noe, Šém, Chám a Jefet.
#1:5 Synové Jefetovi: Gomer a Mágog a Mádaj, Jávan a Túbal, Mešek a Tíras.
#1:6 Synové Gomerovi: Aškenaz, Dífat a Togarma.
#1:7 Synové Jávanovi: Elíša a Taršíš, Kitejci a Ródanci.
#1:8 Synové Chámovi: Kúš a Misrajim, Pút a Kenaan.
#1:9 Synové Kúšovi: Seba, Chavila a Sabta, Raema a Sabteka. Synové Raemovi: Šeba a Dedán.
#1:10 Kúš pak zplodil Nimroda; ten se stal na zemi prvním bohatýrem.
#1:11 Misrajim zplodil Lúďany a Anámce, Lehábany a Naftúchany
#1:12 i Patrúsany a Kaslúchany - z nich vzešli Pelištejci - a Kaftórce.
#1:13 Kenaan zplodil Sidóna, svého prvorozeného, a Chéta,
#1:14 Jebúsejce, Emorejce a Girgašejce
#1:15 též Chivejce, Arkejce a Síňana,
#1:16 Arváďana, Semárce a Chamáťana.
#1:17 Synové Šémovi: Élam a Ašúr, Arpakšád a Lúd a Aram, Ús a Chůl, Geter a Mešek.
#1:18 Arpakšád zplodil Šelacha a Šelach zplodil Hebera.
#1:19 Heberovi se narodili dva synové, jméno jednoho bylo Peleg (to je Rozčlenění), neboť za jeho dnů byla zeně rozčleněna, a jméno jeho bratra bylo Joktán.
#1:20 Joktán pak zplodil Almódada a Šelefa, Chasarmáveta a Jeracha,
#1:21 Hadórama, Úzala a Diklu,
#1:22 Ébala, Abímaela a Šebu,
#1:23 Ofíra, Chavílu a Jóbaba; ti všichni jsou synové Joktánovi.
#1:24 Šém, Arpakšád, Šelach,
#1:25 Heber, Peleg, Reú,
#1:26 Serúg, Náchor, Terach,
#1:27 Abram, to je Abraham.
#1:28 Synové Abrahamovi: Izák a Izmael.
#1:29 Toto jsou jejich rodopisy: Izmaelův prvorozený Nebajót, dále Kédar a Abdeel a Mibsám,
#1:30 Mišma a Dúma a Masa, Chadad a Téma,
#1:31 Jetúr, Náfiš a Kedma. To jsou synové Izmaelovi.
#1:32 Synové Abrahamovy ženiny Ketúry: ta porodila Zimrána a Jokšána, Medána a Midjána, Jišbáka a Šúacha. Synové Jokšánovi: Šeba a Dedán.
#1:33 Synové Midjánovi: Éfa a Éfer a Chanók, Abída a Eldáa; ti všichni jsou synové Ketúřini.
#1:34 Abraham zplodil Izáka. Synové Izákovi: Ezau a Izrael.
#1:35 Synové Ezauovi: Elífaz, Reúel a Jeúš, Jaelam a Kórach.
#1:36 Synové Elífazovi: Téman, Ómar, Sefí a Gátam, Kénaz a Timna a Amálek.
#1:37 Synové Reúelovi: Nachat, Zerach, Šama a Miza.
#1:38 Synové Seírovi: Lótan a Šóbal, Sibeón a Ana a Dišón, Eser a Díšan.
#1:39 Synové Lótanovi: Chorí a Hómam; Lótanova sestra byla Timna.
#1:40 Synové Šóbalovi: Alján a Manachat a Ébal, Šefí a Ónam. Synové Sibeónovi: Aja a Ana.
#1:41 Synové Anovi: Dišón. Synové Dišónovi: Chamrán a Ešbán, Jitrán a Keran.
#1:42 Synové Eserovi: Bilhán a Zaavan a Jaakan. Synové Díšanovi: Ús a Aran.
#1:43 Toto jsou králové, kteří kralovali v edómské zemi, dříve než kraloval král synům izraelským: Bela, syn Beórův, a jeho město se jmenovalo Dinhaba.
#1:44 Když Bela zemřel, stal se po něm králem Jóbab, syn Zerachův, z Bosry.
#1:45 Když zemřel Jóbab, stal se po něm králem Chušam z témanské země.
#1:46 Když zemřel Chušam, stal se po něm králem Hadad, syn Bedadův, který porazil Midjána na Moábském poli. Jeho město se jmenovalo Avít.
#1:47 Když zemřel Hadad, stal se po něm králem Samla z Masreky.
#1:48 Když zemřel Samla, stal se po něm králem Šaúl z Rechobótu nad Řekou.
#1:49 Když zemřel Šaúl, stal se po něm králem Baal-chanan, syn Akbórův.
#1:50 Když zemřel Baal-chanan, stal se po něm králem Hadad. Jeho město se jmenovalo Paí a jméno jeho ženy bylo Mehetabel; byla to dcera Matredy, vnučka Mé-zahabova.
#1:51 Když Hadad zemřel, stali se pohlaváry Edómu: pohlavár Timna, pohlavár Alja, pohlavár Jetet,
#1:52 pohlavár Oholíbama, pohlavár Ela, pohlavár Pínon,
#1:53 pohlavár Kenaz, pohlavár Téman, pohlavár Mibsár,
#1:54 pohlavár Magdíel a pohlavár Iram. To byli edómští pohlaváři. 
#2:1 Toto jsou synové Izraelovi: Rúben, Šimeón, Lévi a Juda, Isachar a Zabulón,
#2:2 Dan, Josef a Benjamín, Neftalí, Gád a Ašer.
#2:3 Synové Judovi: Ér, Ónan a Šela; tito tři se mu narodili z dcery Šúovy, Kenaanky. Judův prvorozený Ér však byl v očích Hospodinových zlý, a proto jej Hospodin usmrtil.
#2:4 Judova snacha Támar mu porodila Peresa a Zeracha. Všech Judových synů bylo pět.
#2:5 Synové Peresovi: Chesrón a Chámul.
#2:6 Synové Zerachovi: Zimrí, Étan a Héman, Kalkol a Dára, celkem pět.
#2:7 Synové Karmího: Akár (to je Zkáza), který přivedl na Izraele zkázu tím, že se dopustil zpronevěry při provádění klatby.
#2:8 Synové Étanovi: Azarjáš.
#2:9 Synové Chesrónovi, kteří se mu narodili: Jerachmeel, Rám a Kelúbaj.
#2:10 Rám zplodil Amínadaba a Amínadab zplodil Nachšóna, předáka Judovců.
#2:11 Nachšón zplodil Salmu a Salma zplodil Bóaza.
#2:12 Bóaz zplodil Obéda a Obéd zplodil Jišaje.
#2:13 Jišaj zplodil prvorozeného Elíaba, druhého Amínadaba, třetího Šimeu,
#2:14 čtvrtého Netaneela, pátého Radaje,
#2:15 šestého Osema, sedmého Davida
#2:16 a jejich sestry Serúju a Abígajilu. Synové Serújini: Abšaj, Jóab a Asáel; tito tři.
#2:17 Abígajil porodila Amasu; otec Amasův byl Jeter Izmaelský.
#2:18 Káleb, syn Chesrónův, zplodil tyto syny s manželkou Azúbou a s Jeriótou: Ješera, Šóbaba a Ardóna.
#2:19 Když Azúba zemřela, vzal si Káleb Efratu a ta mu porodila Chúra.
#2:20 Chúr zplodil Urího a Urí zplodil Besaleela.
#2:21 Potom Chesrón vešel k dceři Makíra, otce Gileádova, a vzal si ji; bylo mu šedesát let, když mu porodila Segúba.
#2:22 Segúb zplodil Jaíra; ten měl třiadvacet měst v gileádské zemi.
#2:23 Gešúr a Aram však jim vzali Jaírovy vesnice, Kenat a jeho osady, šedesát měst. Ti všichni jsou synové Makíra, otce Gileádova.
#2:24 Po Chesrónově smrti vešel Káleb k Efratě. Chesrónova manželka byla Abíja; porodila mu Ašchúra, otce Tekóje.
#2:25 Synové Jerachmeela, Chesrónova prvorozeného, byli: prvorozený Rám, pak Búna a Oren a Osen a Achijáš.
#2:26 Jerachmeel měl ještě druhou manželku, jménem Atáru; ta byla matkou Ónamovou.
#2:27 Synové Ráma, prvorozeného Jerachmeelova, byli: Maas, Jamín a Eker.
#2:28 Synové Ónamovi byli: Šamaj a Jáda. Synové Šamajovi: Nádab a Abíšur.
#2:29 Jméno Abíšurovy manželky bylo Abíchajil; ta mu porodila Achbána a Molída.
#2:30 Synové Nádabovi: Seled a Apajim. Seled zemřel bez synů.
#2:31 Synové Apajimovi: Jiší. Synové Jišího: Šešan. Synové Šešanovi: Achlaj.
#2:32 Synové Jády, Šamajova bratra: Jeter a Jónatan; Jeter zemřel bez synů.
#2:33 Synové Jónatanovi: Pelet a Záza. To byli synové Jerachmeelovi.
#2:34 Šešan neměl syny, jenom dcery. Měl však egyptského otroka, který se jmenoval Jarcha.
#2:35 I dal Šešan svému otroku Jarchovi za manželku svou dceru a ta mu porodila Ataje.
#2:36 Ataj zplodil Nátana a Nátan zplodil Zábada.
#2:37 Zábad zplodil Eflála a Eflál zplodil Obéda.
#2:38 Obéd zplodil Jehúa a Jehú zplodil Azarjáše.
#2:39 Azarjáš zplodil Chelesa a Cheles zplodil Eleásu.
#2:40 Eleása zplodil Sismaje a Sismaj zplodil Šalúma.
#2:41 Šalúm zplodil Jekamjáše a Jekamjáš zplodil Elíšamu.
#2:42 Synové Káleba, bratra Jerachmeelova: Méša, jeho prvorozený, ten byl otcem Zífa, a synové Maréši, otce Chebrónova.
#2:43 Synové Chebrónovi: Kórach, Tapúach, Rekem a Šema.
#2:44 Šema zplodil Rachama, otce Jorkoamova, a Rékem zplodil Šamaje.
#2:45 Syn Šamajův: Maón; Maón byl otec Bét-surův.
#2:46 Kálebova ženina Éfa porodila Chárana, Mósu a Gázeza. Cháran zplodil Gázeza.
#2:47 Synové Johdajovi: Regem, Jótam a Géšan, Pelet, Éfa a Šaaf.
#2:48 Kálebova ženina Maaka porodila Šebera a Tirchanu.
#2:49 Porodila též Šaafa, otce Madmany, Ševu, otce Makbeny i otce Gibeje; Kálebova dcera byla Aksa.
#2:50 To byli synové Kálebovi. Syn Chúra, prvorozeného Efratina: Šóbal, otec Kirjat-jearímu,
#2:51 Salma, otec Betléma, Cháref, otec Bét-gáderu.
#2:52 Syny Šóbala, otce Kirjat-jearímu, byli Haróe, polovina Menúchoťanů
#2:53 a čeledi kirjatjearímské: Jitrejci, Pútejci, Šumatejci a Mišraejci, z kterých vzešli Soreatejci a Eštaólci.
#2:54 Synové Salmovi: Betlémané a Netófané, Atróťané, Bét-joábané a polovina Manachaťanů a Sorejci.
#2:55 Čeledi sóferské, obývající Jaebes: Tireaťané, Šimeaťané, Súkaťané; to jsou Kinejci, pocházející z Chamata, otce Bét-rekabova. 
#3:1 Toto jsou synové Davidovi, kteří se mu narodili v Chebrónu: prvorozený Amnón z Achínóamy Jizreelské, druhý Daníjel z Abígajily Karmelské,
#3:2 třetí Abšalóm, syn Maaky, dcery gešúrského krále Talmaje, čtvrtý Adoníjáš, syn Chagíty,
#3:3 pátý Šefatjáš z Abítaly, šestý Jitreám z jeho manželky Egly.
#3:4 Šest se mu jich narodilo v Chebrónu, kde kraloval sedm roků a šest měsíců, třicet tři roky pak kraloval v Jeruzalémě.
#3:5 A tito se mu narodili v Jeruzalémě: Simea, Sóbab, Nátan a Šalomoun, tito čtyři z Bat-šúy, dcery Amíelovy.
#3:6 Též Jibchár, Elíšáma a Elífelet,
#3:7 Nógah, Nefeg a Jafía,
#3:8 Elíšáma, Eljáda a Elífelet, těchto devět.
#3:9 Ti všichni jsou synové Davidovi kromě synů ženin; Támar byla jejich sestra.
#3:10 Syn Šalomounův byl Rechabeám a jeho syn byl Abijáš, jeho syn Ása, jeho syn Jóšafat,
#3:11 jeho syn Jóram, jeho syn Achazjáš, jeho syn Jóaš,
#3:12 jeho syn Amasjáš, jeho syn Azarjáš, jeho syn Jótam,
#3:13 jeho syn Achaz, jeho syn Chizkijáš, jeho syn Menaše,
#3:14 jeho syn Amón, jeho syn Jóšijáš.
#3:15 Synové Jóšijášovi: prvorozený Jóchanan, druhý Jójakím, třetí Sidkijáš, čtvrtý Šalúm.
#3:16 Synové Jójakímovi: jeho syn Jekonjáš, jeho syn Sidkijáš.
#3:17 Synové vězně Jekonjáše: jeho syn Šealtíel,
#3:18 dále Malkíram, Pedajáš a Šenasar, Jekamjáš, Hóšama a Nedabjáš.
#3:19 Synové Pedajášovi: Zerubábel a Šimeí. Zerubábelův syn byl Mešulám a Chananjáš a jejich sestra byla Šelomít.
#3:20 Dále Chašúba, Óhel, Berekjáš, Chasadjáš, Júšab-chesed, těchto pět.
#3:21 Syn Chananjášúv: Pelatjáš a Ješajáš, synové Refajášovi, synové Arnánovi, synové Obadjášovi, synové Šekanjášovi.
#3:22 Synové Šekanjášovi: Šemajáš. Synové Šemajášovi: Chatúš, Jigál a Baríach, Nearjáš a Šafat, těchto šest.
#3:23 Syn Nearjášův: Eljóenaj, Chizkijáš a Azríkam, tito tři.
#3:24 Synové Eljóenajovi: Hódavjáš, Eljašíb a Pelajáš, Akúb a Jóchanan, Delajáš a Ananí, těchto sedm. 
#4:1 Synové Judovi: Peres, Chesrón a Karmí, Chúr a Šóbal.
#4:2 Reajáš, syn Šóbalův, zplodil Jachata a Jachat zplodil Achúmaje a Lahada. To jsou čeledi soreatejské.
#4:3 Z otce Étama jsou tito: Jizreel, Jišma a Jidbáš; jejich sestra se jmenovala Haslelponí.
#4:4 Penúel byl otec Gedóra a Ezer otec Chušího. To jsou synové Chúra, prvorozeného Efratina, otce Betléma.
#4:5 Ašchúr, otec Tekóje, měl dvě manželky: Cheleu a Naaru.
#4:6 Naara mu porodila Achuzama, Chefera, Témního a Achaštarího. To jsou synové Naařini.
#4:7 Synové Cheleini: Seret, Jesochar a Etnán.
#4:8 Kós zplodil Anúba a Sóbebu a čeledi Acharchela, syna Harumova.
#4:9 Jaebes byl váženější než jeho bratři; jeho matka ho pojmenovala Jaebes se slovy: „Porodila jsem ho s trápením.“
#4:10 Jaebes vzýval Boha Izraele takto: „Kéž bys mi dal požehnání a rozmnožil mé území! Kéž by tvá ruka byla se mnou a zbavila mě zlého, aby na mne nedolehlo trápení!“ A Bůh splnil, oč žádal.
#4:11 Kelúb, bratr Šúchův, zplodil Mechíra; ten je otec Eštónův.
#4:12 Eštón zplodil Bét-ráfu, Paseacha a Techinu, otce Ír-nachaše. To jsou mužové Réky.
#4:13 Synové Kenazovi: Otníel a Serajáš. Synové Otníelovi: Chatat.
#4:14 Meónótaj zplodil Ofru a Serajáš zplodil Jóaba, otce Gé-charašímu (to je Údolí řemeslníků); byli tam totiž řemeslníci.
#4:15 Synové Káleba, syna Jefunova: Irú, Éla a Naam. Synové Élovi: Kenaz.
#4:16 Synové Jehalelélovi: Zíf a Zífa, Tírja a Asarel.
#4:17 Syn Ezrův: Jeter a Mered, Efer a Jalón. Jeter zplodil Mirjama a Šamaje a Jišbacha, otce Eštemóova.
#4:18 Jeho manželka Jehúdija porodila Jereda, otce Gedóru, a Chebera, otce Sóka, a Jekutíela, otce Zanóachu. To jsou synové Bitji, dcery faraónovy, kterou pojal Mered.
#4:19 Synové Hódijášovy manželky, sestry Nachama, otce Keílova: Hagarmí a Eštemóa Maakatský.
#4:20 Synové Šímónovi: Amnón a Rina, Ben-chanan a Tilón. Synové Jišího: Zóchet a Ben-zóchet.
#4:21 Synové Šély, syna Judova: Ér, otec Léky, a Laeda, otec Maréši, a čeledi pracující v dílně na bělostné plátno v Bét-ašbéji,
#4:22 Jokím a mužové Kozeby a Jóaš a Sáraf, kteří opanovali Moába, a Jašúbí-lechem. To jsou starobylé příběhy.
#4:23 Byli to hrnčíři, sadaři a vinaři, sídlili tam u krále v jeho službách.
#4:24 Synové Šimeónovi: Nemúel a Jamín, Jaríb, Zerach a Šaúl,
#4:25 jeho syn Šalúm, jeho syn Mibsám, jeho syn Mišma.
#4:26 Synové Mišmovi: jeho syn Chamúel, jeho syn Zakúr, jeho syn Šimeí.
#4:27 Šimeí měl šestnáct synů a šest dcer. Jeho bratři neměli mnoho synů; celá jejich čeleď se nerozmnožila tak jako Judovci.
#4:28 Sídlili v Beer-šebě, Móladě a Chasar-šúalu,
#4:29 v Bilze, v Esemu a v Tóladu,
#4:30 v Bétúelu, Chormě a Siklagu,
#4:31 v Bét-markabótu, v Chasar-súsimu, v Bét-bireji a v Šaarajimu. To byla jejich města až do Davidova kralování.
#4:32 Jejich dvorce byly v Étamu a Ajinu, v Rimónu, v Tokenu a v Ašanu, v pěti městech.
#4:33 Všechny jejich dvorce, které ležely kolem těchto měst až k Baalu, byly jejich sídlišti; byly zaznamenány v rodovém seznamu.
#4:34 Dále to byl Mešóbab a Jamlék a Jóša, syn Amasjášův
#4:35 Jóel a Jehú, syn Jóšibjáše, syna Serajáše, syna Asíelova,
#4:36 a Eljóenaj, Jaakoba a Ješóchajáš, Asajáš a Adíel, Jesímiel a Benajáš
#4:37 a Zíza, syn Šifího, syna Alóna, syna Jedajáše, syna Šimrího, syna Šemajášova.
#4:38 Tito jmenovitě uvedení byli předáky ve svých čeledích; jejich otcovský rod se mnohonásobně rozmohl.
#4:39 Proto šli směrem ke Gedóru na východ od údolí, aby hledali pastvu pro svůj brav.
#4:40 Našli pastvu tučnou a dobrou a zemi na všechny strany otevřenou, poklidnou a pokojnou; shledali, že tamní předešlí obyvatelé pocházejí z Cháma.
#4:41 Tito jmenovitě zapsaní přišli za dnů judského krále Jechizkijáše a pobořili jejich stany i příbytky, které tam nalezli, zničili je jako klaté, jak je tomu dodnes. Usadili se tam místo nich, protože tam byla pastva pro jejich brav.
#4:42 Dále z nich, totiž ze Šimeónovců, odešlo pět set mužů do pohoří Seíru; v jejich čele byli Pelatjáš, Nearjáš, Refajáš a Uzíel, synové Jišeího.
#4:43 Pobili pozůstatek Amálekovců, kteří kdysi vyvázli; tam sídlí dodnes. 
#5:1 Synové Rúbena, prvorozeného syna Izraelova; byl sice prvorozený, ale když poskvrnil lože svého otce, bylo jeho prvorozenství přeneseno na syny Josefa, syna Izraelova, takže zůstal bez zápisu prvorozenství do rodového seznamu.
#5:2 Převahu mezi svými bratry měl Juda; z něho měl vzejít vévoda, ale prvorozenství náleželo Josefovi.
#5:3 Synové Rúbena, prvorozeného syna Izraelova: Chanók a Palú, Chesrón a Karmí.
#5:4 Synové Jóelovi: jeho syn Šemajáš, jeho syn Góg, jeho syn Šimeí,
#5:5 jeho syn Míka, jeho syn Reajáš, jeho syn Baal
#5:6 a jeho syn Beera, kterého dal přestěhovat asyrský král Tilgat-pilneser; byl rúbenským předákem.
#5:7 Jeho bratři podle čeledí, zapsaní v rodových seznamech podle svých rodopisů: náčelník Jeíel a Zekarjáš
#5:8 a Bela, syn Azaza, syna Šemy, syna Jóelova; ten sídlil v Aróeru a až k Nebó a Baal-meónu;
#5:9 na východě pak sídlil až k okraji stepi při řece Eufratu; jejich stáda se totiž v gileádské zemi rozmnožila.
#5:10 Za dnů Saulových vedli válku s Hagrejci, kteří jim padli do rukou; pak se usadili v jejich stanech po celé východní straně Gileádu.
#5:11 Gádovci sídlili naproti nim v bášanské zemi až k Salce.
#5:12 Jóel byl náčelník a Šafam druhý po něm, pak Jaenaj a Šafat v Bášanu.
#5:13 Jejich bratři po otcovských rodech: Míkael, Mešulám a Šeba, Jóraj a Jaekán, Zía a Eber, těchto sedm.
#5:14 To byli synové Abíchajila, syna Chúrího, syna Jaróacha, syna Gileáda, syna Míkaela, syna Ješíšaje, syna Jachdóa, syna Búzova.
#5:15 Achí, syn Abdíela, syna Gúního, byl náčelník otcovského rodu.
#5:16 Sídlili v Gileádu, v Bášanu a jeho vesnicích a na všech pastvinách šáronských, kam až vybíhají.
#5:17 Ti všichni byli zapsáni do rodových seznamů za dnů judského krále Jótama a za dnů izraelského krále Jarobeáma.
#5:18 Rúbenovců, Gádovců a polovice Manasesova kmene, udatných mužů nosících štít a meč a lučištníků vycvičených k boji, schopných vycházet do boje, bylo čtyřicet čtyři tisíce sedm set šedesát.
#5:19 Vedli boj s Hagrejci, s Jetúrem, s Nafíšem a s Nódabem.
#5:20 Dostalo se jim proti nim pomoci, takže jim byli vydáni do rukou Hagarejci i všichni jejich spojenci; úpěli v boji k Bohu a on jejich prosby přijal, protože v něho doufali.
#5:21 Odvedli jejich stáda: padesát tisíc velbloudů, dvě stě padesát tisíc kusů bravu a dva tisíce oslů i sto tisíc osob.
#5:22 Padlých, skolených mečem, bylo mnoho, neboť ten boj byl od Boha. Sídlili tam pak místo nich až do zajetí.
#5:23 Synové poloviny kmene Manasesova sídlili v zemi od Bášánu k Baal-chermónu, totiž k Seníru a Chermónskému pohoří; bylo jich mnoho.
#5:24 Toto byli náčelníci otcovského rodu: Efer, Jišeí, Elíel a Azríel, Jirmejáš a Hódavjáš a Jachdíel, stateční bohatýři, mužové proslulí, náčelníci otcovských rodů.
#5:25 Avšak zpronevěřili se Bohu svých otců a smilnili s božstvy národů země, které Bůh před nimi vyhladil.
#5:26 Bůh Izraele tedy probudil ducha asyrského krále Púla, totiž ducha asyrského krále Tilgat-pilnesera; ten je dal přestěhovat, totiž Rúbenovce, Gádovce a polovinu kmene Manasesova, a přivedl je k Chalachu, Chabóru a Háře a k řece Gózanu, kde jsou dodnes.
#5:27 Synové Léviho: Geršón, Kehat a Merarí.
#5:28 Synové Kehatovi: Amrám, Jishár, Chebrón a Uzíel.
#5:29 Synové Amrámovi: Áron a Mojžíš; a Mirjam. Synové Áronovi: Nádab a Abíhú, Eleazar a Ítamar.
#5:30 Eleazar zplodil Pinchasa, Pinchas zplodil Abíšúu,
#5:31 Abíšúa zplodil Bukího, Bukí zplodil Uzího,
#5:32 Uzí zplodil Zerachjáše, Zerachjáš zplodil Merajóta,
#5:33 Merajót zplodil Amarjáše, Amarjáš zplodil Achítúba,
#5:34 Achítúb zplodil Sádoka, Sádok zplodil Achímaasa,
#5:35 Achímaas zplodil Azarjáše, Azarjáš zplodil Jóchanana,
#5:36 Jóchanan zplodil Azarjáše; to je ten, který sloužil jako kněz v domě, jejž vybudoval Šalomoun v Jeruzalémě.
#5:37 Azarjáš zplodil Amarjáše, Amarjáš zplodil Achítúba,
#5:38 Achítúb zplodil Sádoka, Sádok zplodil Šalúma,
#5:39 Šalúm zplodil Chilkijáše, Chilkijáš zplodil Azarjáše,
#5:40 Azarjáš zplodil Serajáše a Serajáš zplodil Jósadaka;
#5:41 Jósadak pak šel do zajetí, když Hospodin prostřednictvím Nebúkadnesara dal přestěhovat Judu a Jeruzalém. 
#6:1 Synové Léviho: Geršóm, Kehat a Merarí.
#6:2 Toto jsou jména synů Geršómových: Libní a Šimeí.
#6:3 Synové Kehatovi: Amrám, Jishár, Chebrón a Uzíel.
#6:4 Synové Merarího: Machlí a Muší. To jsou čeledi Léviho podle otcovských rodů.
#6:5 Ke Geršómovi patří: jeho syn Libní, jeho syn Jachat, jeho syn Zima,
#6:6 jeho syn Jóach, jeho syn Idó, jeho syn Zerach, jeho syn Jeotraj.
#6:7 Synové Kehatovi: jeho syn Amínadab, jeho syn Kórach, jeho syn Asír,
#6:8 jeho syn Elkána, jeho syn Ebjásaf, jeho syn Asír,
#6:9 jeho syn Tachat, jeho syn Uríel, jeho syn Uzijáš, jeho syn Šaúl.
#6:10 A synové Elkánovi: Amasaj a Achímót.
#6:11 Elkána: Synové Elkánovi: jeho syn Sófaj, jeho syn Nachat,
#6:12 jeho syn Elíab, jeho syn Jerócham, jeho syn Elkána.
#6:13 A synové Samuelovi: prvorozený Vašní a Abijáš.
#6:14 Synové Merarího: Machlí, jeho syn Libní, jeho syn Šimeí, jeho syn Uza,
#6:15 jeho syn Šimea, jeho syn Chagijáš, jeho syn Asajáš.
#6:16 Toto jsou ti, které ustanovil David, aby zpívali v domě Hospodinově od chvíle, kdy tam spočinula schrána.
#6:17 Přisluhovali zpěvem před příbytkem stanu setkávání až do doby, kdy Šalomoun vybudoval v Jeruzalémě dům Hospodinův. Ve své službě stáli podle svého pořádku.
#6:18 Ti, kteří tu stáli, a jejich synové: Ze synů Kehatových: Héman, zpěvák, syn Jóela, syna Samuela,
#6:19 syna Elkány, syna Jeróchama, syna Elíela, syna Tóacha,
#6:20 syna Súfa, syna Elkány, syna Machata, syna Amasaje,
#6:21 syna Elkány, syna Jóela, syna Azarjáše, syna Sefanjáše,
#6:22 syna Tachata, syna Asíra, syna Ebjásafa, syna Kóracha,
#6:23 syna Jishára, syna Kehata, syna Léviho, syna Izraelova.
#6:24 A jeho bratr Asaf, stojící mu po pravici, Asaf, syn Berekjáše, syna Šimey,
#6:25 syna Míkaela, syna Baasejáše, syna Malkijáše,
#6:26 syna Etního, syna Zeracha, syna Adajáše,
#6:27 syna Étana, syna Zimy, syna Šimeího,
#6:28 syna Jachata, syna Geršóma, syna Léviho.
#6:29 A jejich bratří, synové Merarího, kteří jim stáli po levici: Étan, syn Kišího, syna Abdího, syna Malúka,
#6:30 syna Chašabjáše, syna Amasjáše, syna Chilkijáše,
#6:31 syna Amsího, syna Baního, syna Šemera,
#6:32 syna Machlího, syna Mušího, syna Merarího, syna Léviho.
#6:33 A jejich bratří Lévijci byli pověřeni veškerou službou při příbytku Božího domu.
#6:34 Áron pak a jeho synové obraceli v dým, co bylo přinášeno na oltář pro zápalné oběti a na oltář pro kadidlo při veškerém díle ve velesvatyni, a vykonávali smírčí obřady za Izraele podle všeho toho, co přikázal Mojžíš, Boží služebník.
#6:35 Toto jsou synové Áronovi: jeho syn Eleazar, jeho syn Pinchas, jeho syn Abíšua,
#6:36 jeho syn Bukí, jeho syn Uzí, jeho syn Zerachjáš,
#6:37 jeho syn Merajót, jeho syn Amarjáš, jeho syn Achítúb,
#6:38 jeho syn Sádok, jeho syn Achímaas.
#6:39 A toto jsou jejich sídliště při jejich hradištích na jejich území, sídliště pro syny Áronovy, pro čeleď kehatskou, neboť jim připadl první los.
#6:40 Dali jim Chebrón v zemi judské s okolními pastvinami.
#6:41 Avšak pole u města a dvorce dali Kálebovi, synu Jefunovu.
#6:42 Synům Áronovým dali útočištná města Chebrón, Libnu s pastvinami, Jatír, Eštemóu s pastvinami,
#6:43 Chílez s pastvinami, Debír s pastvinami,
#6:44 Ašán s pastvinami a Bét-šemeš s pastvinami.
#6:45 Od pokolení Benjamínova dostali Gebu s pastvinami, Alemet s pastvinami a Anatót s pastvinami. Všech jejich měst pro jejich čeledi bylo třináct.
#6:46 Zbývající Kehatovci podle čeledí svého pokolení dostali losem deset měst od poloviny pokolení Manasesova.
#6:47 Geršómovci dostali pro své čeledi třináct měst od pokolení Isacharova, od pokolení Ašerova, od pokolení Neftalíova a od pokolení Manasesova v Bášanu.
#6:48 Meraríovci dostali losem pro své čeledi dvanáct měst od pokolení Rúbenova, od pokolení Gádova a od pokolení Zabulónova.
#6:49 Izraelci dali Léviovcům města s pastvinami.
#6:50 Od pokolení Judovců, od pokolení Šimeónovců a od pokolení Benjamínovců jim losem přidělili tato města, která uvedli jmenovitě.
#6:51 Některé z čeledí Kehatovců dostaly města od pokolení Efrajimova na jeho území.
#6:52 Přidělili jim útočištná města Šekem s pastvinami v pohoří Efrajimském, Gezer s pastvinami,
#6:53 Jokmeám s pastvinami, Bét-chorón s pastvinami,
#6:54 Ajalón s pastvinami a Gat-rimón s pastvinami.
#6:55 Od poloviny pokolení Manasesova dostaly zbývající čeledi Kehatovců Anér s pastvinami a Bileám s pastvinami.
#6:56 Geršómovci podle čeledí dostali od poloviny pokolení Manasesova Gólan v Bášanu s pastvinami a Aštarót s pastvinami.
#6:57 Od pokolení Isacharova Kedeš s pastvinami, Dobrat s pastvinami,
#6:58 Rámot s pastvinami a Aném s pastvinami.
#6:59 Od pokolení Ašerova Mášal s pastvinami, Abdón s pastvinami,
#6:60 Chúkok s pastvinami a Rechób s pastvinami.
#6:61 Od pokolení Neftalího Kedeš v Galileji s pastvinami, Chamón s pastvinami a Kirjatajim s pastvinami.
#6:62 Zbývající Meraríovci dostali od pokolení Zabulónova Rimónó s pastvinami, Tábor s pastvinami,
#6:63 na druhé straně Jordánu pak, na východ od Jordánu, proti Jerichu, od pokolení Rúbenova Beser ve stepi s pastvinami, Jahsu s pastvinami,
#6:64 Kedemót s pastvinami a Mefaat s pastvinami.
#6:65 A od pokolení Gádova Rámot v Gileádu s pastvinami, Machanajim s pastvinami,
#6:66 Chešbón s pastvinami a Jaezer s pastvinami. 
#7:1 K synům Isacharovým patřil Tóla a Púa, Jašúb a Šimrón, tito čtyři.
#7:2 Synové Tólovi: Uzí, Refajáš, Jeríel, Jachmaj, Jibsám a Šemúel, náčelníci otcovských rodů Tólových, stateční bohatýři, podle svých rodopisů; jejich počet za dnů Davidových činil dvacet dva tisíce šest set.
#7:3 Synové Uzího: Jizrachjáš. Synové Jizrachjášovi: Míkael, Obadjáš, Jóel, Jišijáš, těch pět, všichni náčelníci.
#7:4 Při nich bylo třicet šest tisíc mužů ve vojenských bojových houfech, podle svých rodopisů po otcovských rodech; měli totiž mnoho žen a synů.
#7:5 A jejich bratří, statečných bohatýrů, po všech čeledech Isacharových bylo osmdesát sedm tisíc; všichni byli zapsáni do seznamů rodů.
#7:6 Benjamín: Bela, Beker a Jedíael, tito tři.
#7:7 A synové Belovi: Esbón, Uzí, Uzíel, Jerimót a Írí, těchto pět, náčelníci otcovských rodů, stateční bohatýři; do seznamu rodů jich bylo zapsáno dvacet dva tisíce třicet čtyři.
#7:8 Synové Bekerovi: Zemíra, Jóaš, Elíezer, Eljóenaj, Omrí, Jeremót, Abijáš, Anatót a Alemet; tito všichni jsou synové Bekerovi.
#7:9 Do seznamu rodů, podle svých rodopisů byli zapsáni jako náčelníci otcovských rodů, stateční bohatýři, celkem dvacet tisíc dvě stě.
#7:10 Synové Jedíaelovi: Bilhán. A synové Bilhánovi: Jeúš, Binjamín, Ehúd, Kenaana, Zétan, Taršíš a Achíšachar.
#7:11 Tito všichni jsou synové Jedíaelovi, podle náčelníků rodů, stateční bohatýři, sedmnáct tisíc dvě stě těch, kteří byli schopní k vojenské službě a vycházet do boje.
#7:12 Synové Írovi: Šupím a Chupím. Synové Achérovi: Chuším.
#7:13 Synové Neftalího: Jachasíel, Gúní, Jeser a Šalúm, synové Bilhy.
#7:14 Synové Manasesovi: Asríel, kterého mu porodila jeho žena; jeho aramejská ženina porodila Makíra, otce Gileádova.
#7:15 A Makír vzal ženu pro Chupíma a Šupíma; jméno jeho sestry bylo Maaka. Jméno druhorozeného bylo Selofchad; a Selofchad měl jen dcery.
#7:16 Maaka, žena Makírova, porodila syna a pojmenovala ho Pereš; jeho bratr se jmenoval Šereš; a jeho synové: Úlam a Rekem.
#7:17 Synové Úlamovi: Bedán. To jsou synové Gileáda, syna Makírova, syna Manasesova.
#7:18 Jeho sestra Moleket porodila Íšhóda, Abíezera a Machlu.
#7:19 A synové Šemídovi byli Achján, Šekem, Likchí a Aníam.
#7:20 Synové Efrajimovi: Šútelach, jeho syn Bered, jeho syn Tachat, jeho syn Eleada, jeho syn Tachat,
#7:21 jeho syn Zábad, jeho syn Šútelach, Ezer a Elead. Ale povraždili je domorodci, muži z Gatu, neboť přitáhli, aby zabrali jejich stáda.
#7:22 Po mnoho dnů truchlil jejich otec Efrajim a jeho bratři přišli, aby ho potěšili.
#7:23 Pak vešel ke své ženě, ta otěhotněla a porodila syna; dal mu jméno Bería (to je V neštěstí), neboť byla v jeho domě v neštěstí.
#7:24 Jeho dcerou byla Šeera; ta vybudovala Bét-chorón Dolní a Horní a Uzen-šeeru.
#7:25 A jeho syn byl Refach a Rešef, jeho syn Telach, jeho syn Tachan,
#7:26 jeho syn Laedán, jeho syn Amíhud, jeho syn Elíšama,
#7:27 jeho syn Nón, jeho syn Jozue.
#7:28 Jejich trvalé vlastnictví a sídliště: Bét-el s vesnicemi, na východě Naarán, na západě Gezer s vesnicemi, Šekem s vesnicemi, až do Aje s vesnicemi.
#7:29 V rukou synů Manasesových byl Bét-šeán s vesnicemi, Taenak s vesnicemi, Megido s vesnicemi, Dór s vesnicemi. V těch místech sídlili synové Josefa, syna Izraelova.
#7:30 Synové Ašerovi: Jimna, Jišva, Jišví, Bería a jejich sestra Serach.
#7:31 Synové Beríovi: Cheber a Malkíel; to je otec Birzavitův.
#7:32 Cheber zplodil Jafleta, Šómera, Chótama a jejich sestru Šúu.
#7:33 Synové Jafletovi: Pásak, Bimhál a Ašvát; to jsou synové Jafletovi.
#7:34 Synové Šemerovi: Achí a Rohga, Jechuba a Arám.
#7:35 Synové Helema, jeho bratra: Sófach, Jimna, Šéleš a Ámal.
#7:36 Synové Sófachovi: Súach, Charnefer, Šúal, Berí a Jimra,
#7:37 Beser, Hód, Šama, Šilša, Jitrán a Beéra.
#7:38 Synové Jeterovi: Jefune, Pispa a Ara.
#7:39 Synové Ulovi: Árach, Chaníel a Risja.
#7:40 Ti všichni jsou synové Ašerovi, náčelníci otcovských rodů, zkušení stateční bohatýři, náčelníci předáků. Počet zapsaných do seznamu rodů pro vojenskou službu a do boje činil dvacet šest tisíc mužů. 
#8:1 Benjamín zplodil Belu, svého prvorozeného, Abšela jako druhého, Achracha jako třetího,
#8:2 Nóchu jako čtvrtého a Ráfu jako pátého.
#8:3 Synové Belovi byli: Adár, Géra, Abíhúd,
#8:4 Abíšúa, Naamán, Achóach,
#8:5 Géra, Šefúfan a Chúram.
#8:6 To jsou synové Echúdovi, náčelníci rodů obyvatel Geby, kteří je přestěhovali do Manachatu,
#8:7 totiž Naamán, Achijáš a Géra; ten je přestěhoval; též zplodil Uzu a Achíchuda.
#8:8 Šacharajim zplodil syny na Moábském poli poté, co propustil své ženiny Chúšimu a Baaru;
#8:9 se svou ženou Chodeší zplodil Jóbaba, Sibju, Méšu, Malkáma,
#8:10 Jeúsa, Sekejáše a Mirmu; to jsou jeho synové, náčelníci rodů.
#8:11 S Chúšimou zplodil Abítúba a Elpaala.
#8:12 Synové Elpaalovi: Eber, Mišeám, Šemed - ten vybudoval Óno a Lód s vesnicemi -,
#8:13 Bería a Šema; to jsou náčelníci rodů obyvatel Ajalónu; ti zahnali na útěk obyvatele Gatu.
#8:14 Dále Achjó, Šášak a Jeremót,
#8:15 Zebadjáš, Arad a Eder,
#8:16 Míkael, Jišpa a Jócha, synové Beríovi,
#8:17 Zebadjáš, Mešulám, Chizkí a Cheber,
#8:18 Jišmeraj, Jizlía a Jóbab, synové Elpaalovi,
#8:19 Jakím, Zikrí a Zabdí,
#8:20 Elíenaj, Siletaj a Elíel,
#8:21 Adajáš, Berejáš a Šimrat, synové Šimeího,
#8:22 Jišpán, Eber a Elíel,
#8:23 Ebdón, Zikrí a Chánan,
#8:24 Chananjáš, Élam a Antótijáš,
#8:25 Jifdejáš a Penúel, synové Šášakovi,
#8:26 Šamšeraj, Šecharjáš a Ataljáš,
#8:27 Jaarešjáš, Elijáš a Zikrí, synové Jerochámovi.
#8:28 To jsou náčelníci rodů podle rodopisů, náčelníci, kteří sídlili v Jeruzalémě.
#8:29 V Gibeónu pak sídlili: Otec Gibeónu, jeho žena se jmenovala Maaka
#8:30 a jeho prvorozený syn Abdón, pak Súr, Kíš, Baal a Nádab,
#8:31 Gedór, Achjó a Zeker.
#8:32 Miklót zplodil Šimeu. Také ti sídlili naproti svým bratřím v Jeruzalémě, se svými bratry.
#8:33 Nér zplodil Kíše, Kíš zplodil Saula, Saul zplodil Jónatana, Malkíšúu, Abínádaba a Ešbaala.
#8:34 Synem Jónatanovým byl Meríb-baal. Meríb-baal zplodil Míku.
#8:35 Synové Míkovi: Pítón, Melek, Taeréa a Achaz.
#8:36 Achaz zplodil Jójadu, Jójada zplodil Alemeta, Azmáveta a Zimrího, Zimrí zplodil Mósu.
#8:37 Mósa zplodil Bineu; jeho syn byl Ráfa, jeho synem Eleása a jeho synem Ásel.
#8:38 Ásel měl šest synů. Toto jsou jejich jména: Azríkam, Bokrú, Jišmáel, Šearjáš, Obadjáš a Chánan. Ti všichni jsou synové Áselovi.
#8:39 Synové jeho bratra Ešeka: Úlam, jeho prvorozený, druhý Jeúš a třetí Elífelet.
#8:40 Synové Úlamovi byli stateční bohatýři, lučištníci; měli mnoho synů a vnuků, celkem sto padesát. Ti všichni pocházeli ze synů Benjamínových. 
#9:1 Všichni z Izraele byli zapsáni do seznamu rodů. Zde jsou ti, kdo byli zapsáni do Knihy králů izraelských. Juda pro svou zpronevěru byl přestěhován do Babylóna.
#9:2 Kdo dříve sídlili v městech svého trvalého vlastnictví, Izrael, kněží, levité a chrámoví nevolníci,
#9:3 pocházející z Judovců, Benjamínovců, Efrajimovců a Manasesovců, sídlili nyní v Jeruzalémě:
#9:4 Útaj, syn Amíhuda, syna Omrího, syna Imrího, syna Báního ze synů Peresa, syna Judova.
#9:5 Ze Šíloanů Asajáš, prvorozený, a jeho synové.
#9:6 Ze synů Zerachových Jeúel a jejich bratří, celkem šest set devadesát.
#9:7 Ze synů Benjamínových Salú, syn Mešuláma, syna Hódavjáše, syna Hasenuova,
#9:8 a Jibnejáš, syn Jerochámův, a Éla, syn Uzího, syna Mikrího, a Mešulám, syn Šefatjáše, syna Reúela, syna Jibnijášova,
#9:9 a jejich bratří podle rodopisů, celkem devět set padesát šest. Ti všichni muži byli náčelníci otcovských rodů.
#9:10 Z kněží: Jedajáš, Jójaríb a Jakín,
#9:11 Azarjáš, syn Chilkijáše, syna Mešuláma, syna Sádoka, syna Merajóta, syna Achítúba, představeného Božího domu.
#9:12 Adajáš, syn Jerocháma, syna Pašchúra, syna Malkijášova, Maesaj, syn Adíela, syna Jachzéry, syna Mešuláma, syna Mešilemíta, syna Imerova,
#9:13 a jejich bratří, náčelníci otcovských rodů, tisíc sedm set šedesát statečných bohatýrů pro služebné dílo Božího domu.
#9:14 Z lévijců: Šemajáš, syn Chašúba, syna Azríkama, syna Chašabjáše ze synů Merarího.
#9:15 Bakbakar, Chereš a Gálal, Matanjáš, syn Míky, syna Zikrího, syna Asafova.
#9:16 Obadjáš, syn Šemajáše, syna Gálala, syna Jedútúnova, a Berekjáš, syn Ásy, syna Elkánova, bydlící ve dvorcích nétofských.
#9:17 Vrátní: Šalúm, Akúb, Talmón a Achíman a jejich bratří; Šalúm byl náčelníkem
#9:18 a je až dosud při východní královské bráně. To jsou vrátní z táborů Léviovců.
#9:19 Šalúm, syn Kóreho, syna Ebjásafa, syna Kórachova, a jeho bratří z rodu Kórachovců byli správci služebného díla a střežili prahy stanu; jejich otcové byli správci Hospodinova tábora a střežili vstup.
#9:20 A Pinchas, syn Eleazarův, byl dříve jejich představeným; Hospodin byl s ním.
#9:21 Zekarjáš, syn Mešelemjášův, byl vrátným při vchodu do stanu setkávání.
#9:22 Všech prověřených vrátných prahů bylo dvě stě dvanáct. Byli zapsáni do seznamu rodů ve svých dvorcích. David a Samuel, vidoucí, je ustanovili k stálé službě.
#9:23 Oni a jejich synové byli na stráži u bran domu Hospodinova, domu stanu.
#9:24 Vrátní byli na čtyřech stranách: na východě, na západě, na severu a na jihu.
#9:25 A jejich bratří měli přicházet ze svých dvorců vždy ve stanovené době, aby spolu s nimi sloužili po sedm dnů.
#9:26 Tito byli ustanoveni k stálé službě: čtyři bohatýrští vrátní; byli to lévijci, byli správci komor a pokladů Božího domu.
#9:27 Jiní ponocovali kolem Božího domu, neboť na nich bylo držet stráž a ráno co ráno otvírat.
#9:28 Další měli na starosti služebné náčiní, v určitém počtu je přinášeli dovnitř a vynášeli ven.
#9:29 Někteří měli pod dohledem náčiní a svaté předměty, bílou mouku, víno a olej, kadidlo a vonné látky.
#9:30 Někteří kněžští synové mísili masti z vonných látek.
#9:31 A Matitjáš z lévijců, prvorozený Kórachovce Šalúma, byl ustanoven k stálé službě u pánví.
#9:32 Někteří Kehatovci, někteří z jejich bratří, měli na starosti rovnání předkladných chlebů, měli je připravovat vždycky v den odpočinku.
#9:33 Toto jsou zpěváci, náčelníci lévijských rodů; dleli v komorách, uvolněni od jiných prací. Ve dne i v noci byli totiž v pohotovosti ke svému dílu.
#9:34 Toto jsou náčelníci lévijských rodů podle rodopisů, náčelníci, kteří sídlili v Jeruzalémě.
#9:35 V Gibeónu sídlili: otec Gibeónu Jeíel, jeho žena se jmenovala Maaka
#9:36 a jeho prvorozený syn Abdón, pak Súr, Kíš, Baal, Nér a Nádab,
#9:37 Gedór a Achjó, Zekarjáš a Miklót.
#9:38 Miklót pak zplodil Šimeama. Také ti sídlili naproti svým bratřím v Jeruzalémě se svými bratry.
#9:39 Nér zplodil Kíše a Kíš zplodil Saula. Saul zplodil Jónatana a Malkíšúu a Abínádaba a Eš-baala.
#9:40 Synem Jónatanovým byl Meríb-baal. Meríb-baal zplodil Míku.
#9:41 Synové Míkovi: Pitón, Melek a Tachréa.
#9:42 Achaz pak zplodil Jaeru, Jaera zplodil Alemeta, Azmáveta a Zimrího, Zimrí zplodil Mósu.
#9:43 Mósa zplodil Bineu; jeho synem byl Refajáš, jeho synem byl Eleása, jeho synem byl Ásel.
#9:44 Ásel měl šest synů. Toto jsou jejich jména: Azríkam, Bokrú, Jišmáel, Šearjáš, Obadjáš a Chánan. Toto jsou synové Áselovi. 
#10:1 Pelištejci bojovali proti Izraeli. Izraelští muži před Pelištejci utíkali a padali pobiti v pohoří Gilbóa.
#10:2 Pelištejci se pustili za Saulem a za jeho syny. I pobili Pelištejci Saulovy syny Jónatana, Abínádaba a Malkíšúu.
#10:3 Pak zesílil boj proti Saulovi. Objevili ho lukostřelci a postřelili ho.
#10:4 Saul řekl svému zbrojnoši: „Vytas meč a probodni mě jím, než přijdou ti neobřezanci, aby mě nezneuctili.“ Zbrojnoš však nechtěl, velmi se bál. Saul tedy uchopil meč a nalehl na něj.
#10:5 Když zbrojnoš viděl, že Saul zemřel, také on nalehl na meč a zemřel.
#10:6 Tak zemřeli Saul, jeho tři synové a celý jeho dům, zemřeli zároveň.
#10:7 Když všichni izraelští muži, kteří byli v dolině, viděli, že bojovníci utekli a že Saul i jeho synové zemřeli, opouštěli svá města a utíkali. Pelištejci přišli a usadili se v nich.
#10:8 Pelištejci přišli druhého dne, aby obrali pobité, a našli Saula a jeho syny padlé v pohoří Gilbóa.
#10:9 Obrali ho a odnesli jeho hlavu a jeho zbroj a poslali kolovat po pelištejské zemi jako radostné poselství pro své modlářské stvůry i pro lid.
#10:10 Jeho zbroj uložili v domě svých bohů a jeho lebku přibili v domě Dágonově.
#10:11 Všichni z Jábeše v Gileádu uslyšeli o všem, co se Saulem provedli Pelištejci.
#10:12 Všichni válečníci se vypravili, odnesli mrtvé tělo Saulovo i mrtvá těla jeho synů, přinesli je do Jábeše a pochovali jejich kosti pod posvátným stromem v Jábeši. Poté se postili sedm dní.
#10:13 Tak zemřel Saul pro svoji zpronevěru; zpronevěřil se Hospodinu, protože nedbal na Hospodinovo slovo. Dokonce se dotazoval věštího ducha.
#10:14 Hospodina se nedotazoval, proto jej Hospodin usmrtil a převedl království na Davida, syna Jišajova. 
#11:1 Celý Izrael se shromáždil k Davidovi do Chebrónu. Řekli: „Hle, jsme tvá krev a tvé tělo.
#11:2 Už tenkrát, když byl králem Saul, vyváděl a přiváděl jsi Izraele ty. Tobě Hospodin, tvůj Bůh, řekl: ‚Ty budeš pást Izraele, můj lid, ty budeš vévodou nad mým izraelským lidem.‘“
#11:3 Pak přišli všichni izraelští starší ke králi do Chebrónu a David s nimi v Chebrónu uzavřel před Hospodinem smlouvu. I pomazali Davida za krále nad Izraelem podle Hospodinova slova skrze Samuela.
#11:4 David šel s celým Izraelem na Jeruzalém, to je na Jebús; obyvatelé té země byli Jebúsejci.
#11:5 Obyvatelé Jebúsu Davidovi řekli: „Sem nevstoupíš.“ Ale David dobyl skalní pevnost Sijón, to je Město Davidovo.
#11:6 David řekl: „Kdo udeří na Jebúsejce první, stane se vůdcem a velitelem.“ První pronikl nahoru Jóab, syn Serújin, a tak se stal vůdcem.
#11:7 David se usadil ve skalní pevnosti; proto ji nazvali Město Davidovo.
#11:8 Pak vybudoval město kolem dokola počínaje od Miló. Ostatní části města opravil Jóab.
#11:9 David se stále více vzmáhal a Hospodin zástupů byl s ním.
#11:10 Toto byli vůdci Davidových bohatýrů, kteří byli při něm jako mocná opora jeho království spolu s celým Izraelem; ustanovili jej králem podle Hospodinova slova o Izraeli.
#11:11 Toto je výčet Davidových bohatýrů: Jášobeám, syn Chachmóního, vůdce osádek. Ten zamával kopím a naráz jich proklál na tři sta.
#11:12 Po něm Eleazar, syn Dóda Achóchijského; to byl jeden z těch tří bohatýrů.
#11:13 Ten byl s Davidem v Pas-damímu, když se tam Pelištejci shromáždili k boji. Byl tam díl pole, celý osetý ječmenem. Lid utíkal před Pelištejci.
#11:14 Bohatýři se postavili doprostřed toho dílu pole, osvobodili jej a Pelištejce pobili. Hospodin způsobil veliké vysvobození.
#11:15 Tři z těch třiceti vůdců sestoupili na skálu k Davidovi do jeskyně Adulámu. Pelištejský tábor ležel v dolině Refájců.
#11:16 David byl tehdy ve skalní skrýši a výsostný znak Pelištejců byl tenkrát u Betléma.
#11:17 Tu David zatoužil: „Kdo mi dá napít vody z betlémské studny, která je u brány?“
#11:18 Ti tři vtrhli do pelištejského tábora, načerpali vodu z betlémské studny, která je u brány, a přinesli ji Davidovi. David ji však nechtěl pít, nýbrž vykonal jí úlitbu Hospodinu.
#11:19 Řekl: „Můj Bože, nechť jsem dalek toho, abych udělal něco takového. Což mohu pít krev těch mužů? Nasadili svůj život. Přinesli mi ji s nasazením života.“ Proto se jí nechtěl napít. To vykonali ti tři bohatýři.
#11:20 Také Abšaj, bratr Jóabův, byl vůdce tří. I on zamával kopím a proklál jich na tři sta. Mezi těmi třemi byl nejproslulejší.
#11:21 Z těch tří byl váženější než druzí dva a byl jejich velitelem, ale oněm třem prvním se nevyrovnal.
#11:22 Benajáš, syn Jójady, syna zdatného muže z Kabseelu, muže mnoha činů, ubil dva moábské reky a sestoupil a ubil v jámě lva; byl tehdy sníh.
#11:23 Ubil též Egypťana vysokého pět loket; Egypťan měl v ruce kopí jako tkalcovské vratidlo. Sestoupil k němu s holí, vytrhl Egypťanovi kopí z ruky a jeho vlastním kopím ho zabil.
#11:24 Toto vykonal Benajáš, syn Jójadův. A byl mezi těmi třemi bohatýry proslulý.
#11:25 Z těch třiceti byl nejváženější, ale oněm třem prvním se nevyrovnal. David ho přidělil ke své tělesné stráži.
#11:26 Následují stateční bohatýři Asáel, bratr Jóabův, Elchánan, syn Dóda z Betléma,
#11:27 Šamót Harórský, Cheles Pelónský,
#11:28 Íra, syn Íkeše Tekójského, Abíezer Anatótský,
#11:29 Sibekaj Chúšatský, Ílaj Achóchejský,
#11:30 Mahraj Netófský, Cheled, syn Baany Netófského,
#11:31 Ítaj, syn Ríbaje z Gibeje Benjamínovců, Benajáš Pireatónský,
#11:32 Chúraj z úvalů Gaašských, Abíel Arbátský,
#11:33 Azmávet Bacharúmský, Eljachba Šaalbónský.
#11:34 Proslulí byli Jónatan Gizónský, syn Šagéa Hararského,
#11:35 Achíam, syn Sakara Hararského, Elífal, syn Úrův,
#11:36 Chefer Mekeratský, Achijáš Pelónský,
#11:37 Chesró Karmelský, Naaraj, syn Ezbajův,
#11:38 Jóel, bratr Nátanův, Mibchár, syn Hagrího,
#11:39 Selek Amónský, Nachraj Berótský, zbrojnoš Jóaba, syna Serújina,
#11:40 Íra Jitrejský, Gáreb Jitrejský,
#11:41 Urijáš Chetejský, Zábad, syn Achlájův,
#11:42 Adína, syn Šízy Rúbenského, vůdce Rúbenců, s nimiž bylo oněch třicet,
#11:43 dále Chánan, syn Maakův, a Jóšafat Mitnejský,
#11:44 Uzijáš Ašterátský, Šama a Jeíel, synové Chótama Aróerského,
#11:45 Jedíael, syn Šimrího, a jeho bratr Jócha Tisejský,
#11:46 Elíel Machavímský, Jeríbaj a Jóšavjáš, synové Elnaamovi, a Jitma Moábský,
#11:47 Elíel a Obéd a Jaasíel Mesóbajští. 
#12:1 Toto jsou bohatýři, kteří přišli za Davidem do Siklagu, kde se zdržoval před Saulem, synem Kíšovým, jeho spolubojovníci.
#12:2 Byli vyzbrojeni lukem, bojovali pravou i levou rukou, kamením i šípy z luku. Pocházeli z Benjamína, ze Saulových bratří.
#12:3 Vůdcem byl Achíezer, pak Jóaš, synové Šemaje Gibeatského, Jezíel a Pelet, synové Azmávetovi, Beraka a Jehú Anatótský,
#12:4 Jišmajáš Gibeónský, jeden z třiceti bohatýrů a nad třiceti,
#12:5 Jirmejáš a Jachazíel, Jóchanan a Józabad Gederatský,
#12:6 Eleúzaj a Jerímót, Baaljáš, Šemarjáš a Šefatjáš Charífský,
#12:7 Elkána a Jišijáš, Azarel, Jóezer a Jášobeám, Korchíjci,
#12:8 Jóela a Zebadjáš, synové Jerocháma z Gedóru.
#12:9 Z Gádovců přešli k Davidovi do skalní skrýše na poušť udatní bohatýři, schopní bojovníci vyzbrojení štítem a oštěpem, s tváří lvů a hbitostí gazel na horách.
#12:10 Vůdcem byl Ezer, druhý byl Obadjáš, třetí Elíab,
#12:11 čtvrtý Mišmana, pátý Jirmejáš,
#12:12 šestý Ataj, sedmý Elíel,
#12:13 osmý Jóchanan, devátý Elzabad,
#12:14 desátý Jirmejáš, jedenáctý Makbanaj.
#12:15 To byli Gádovci, vůdcové vojska, ti mladší nad setninou, ti starší nad plukem.
#12:16 To jsou ti, kteří překročili Jordán v prvním měsíci, když se rozvodnil a vylil ze svých břehů, kteří rozehnali nepřátele po všech dolinách na východ i na západ.
#12:17 Také jiní z Benjamínovců a Judejců přišli k Davidovi do skalní skrýše.
#12:18 David jim vyšel vstříc a takto je oslovil: „Přicházíte-li ke mně v pokoji, abyste mi pomohli, ze srdce rád se s vámi spojím. Jestliže mě však chcete vyzradit mým protivníkům, i když na mých rukou nelpí násilí, ať to Bůh našich otců vidí a ztrestá.“
#12:19 Tu vyzbrojil duch Boží Amasaje, vůdce osádek. Řekl: „Jsme s tebou, Davide, jsme při tobě, synu Jišajův. Pokoj, pokoj tobě, pokoj i tvým pomocníkům, vždyť tvůj Bůh ti pomáhá.“ David je přijal a ustanovil je vůdci oddílů.
#12:20 Také někteří z Manasea odpadli k Davidovi, když vytáhl s Pelištejci do boje proti Saulovi. Nepomáhali však Pelištejcům, protože pelištejská knížata se usnesla poslat Davida pryč s odůvodněním: „Za cenu našich hlav odpadne ke svém pánu Saulovi.“
#12:21 Když se ubíral do Siklagu, odpadli k němu z Manasesa Adnach, Józabad, Jedíael, Míkael, Józabad, Elíhu a Siltaj, vůdci šiků z Manasesa.
#12:22 Ti pomohli Davidovi proti nepřátelské hordě. Všichni to byli udatní bohatýři. Stali se ve vojsku veliteli.
#12:23 Den ze dne přicházely Davidovi posily, takže jeho tábor vzrostl jako tábor Boží.
#12:24 Toto je výčet oddílů vyzbrojených k vojenské službě, těch, kteří přišli k Davidovi do Chebrónu, aby na něho podle Hospodinova rozkazu přenesli Saulovo království.
#12:25 Judovců ozbrojených štítem a oštěpem šest tisíc osm set, vyzbrojených k vojenské službě;
#12:26 ze Šimeónovců sedm tisíc jedno sto bohatýrů udatných v boji;
#12:27 z Léviovců čtyři tisíce šest set;
#12:28 Jójada, vévoda Áronovců, měl s sebou tři tisíce sedm set mužů.
#12:29 Mládenec Sádok byl udatný bohatýr. Z domu jeho otce bylo dvaadvacet velitelů.
#12:30 Z Benjamínovců, Saulových bratří, tři tisíce. Většina jich totiž dosud vykonávala strážní službu domu Saulovu.
#12:31 Z Efrajimovců dvacet tisíc osm set udatných bohatýrů, proslulých ve svých otcovských rodech.
#12:32 Z poloviny pokolení Manasesova osmnáct tisíc, kteří byli uvedeni jménem a určeni k tomu, aby Davida ustanovili králem.
#12:33 Z Isacharovců přišli znalci časů, kteří poučovali Izraele, co má dělat. Jejich vůdců bylo dvě stě; všichni jejich bratří si řídili jejich pokyny.
#12:34 Ze Zabulóna vytáhlo vojsko připravené k boji se všemi válečnými zbraněmi, padesát tisíc mužů odhodlaných bez váhání nastoupit.
#12:35 Z Neftalího tisíc velitelů a s nimi třicet sedm tisíc mužů ozbrojených štítem a kopím.
#12:36 Z Dana dvacet osm tisíc šest set mužů připravených k boji.
#12:37 Z Ašera vytáhlo vojsko čtyřicet tisíc mužů připravených k boji.
#12:38 Ze Zajordání z Rúbena, Gáda a z poloviny kmene Manasesova sto dvacet tisíc mužů v plné zbroji do boje.
#12:39 Ti všichni byli bojovníci odhodlaní nastoupit do bitevní řady; přišli v pokoji do Chebrónu, aby ustanovili Davida králem nad celým Izraelem. Též všechen ostatní Izrael hodlal jednomyslně Davida ustanovit králem.
#12:40 Pobyli tam s Davidem tři dny. Jedli a pili, co jim jejich bratří připravili.
#12:41 A jejich sousedé až od Isachara, Zabulóna a Neftalího přiváželi na oslech, velbloudech, na mezcích i dobytčatech chléb, mouku, pletence sušených fíků, sušené hrozny, víno a olej a přiváděli množství skotu a bravu, neboť v Izraeli zavládla radost. 
#13:1 David se radil s veliteli nad tisíci a nad sty, s každým vojevůdcem.
#13:2 David řekl celému shromáždění Izraele: „Pokládáte-li to za vhodné a za pokyn od Hospodina, našeho Boha, pošleme vzkaz svým ostatním bratřím do všech izraelských území a současně i kněžím a lévijcům do jejich měst a na jejich pastviny, aby se k nám shromáždili.
#13:3 Přeneseme k nám schránu našeho Boha; za dnů Saulových jsme ji nevyhledávali.“
#13:4 Celé shromáždění s tím souhlasilo, všechen lid ten návrh pokládal za správný.
#13:5 Proto David svolal celý Izrael, od ramene řeky Egyptské až k cestě do Chamátu, aby přenesli Boží schránu z Kirjat-jearímu.
#13:6 Pak David a celý Izrael táhli do Baaly u Kirjat-jearímu v Judsku, aby odtud vynesli schránu Boha Hospodina, který sídlí nad cheruby, jehož jméno je vzýváno.
#13:7 Vezli Boží schránu na novém povozu z domu Abínádabova. Povoz řídili Uza a Achjó.
#13:8 David a celý Izrael bujaře křepčili před Bohem a zpívali za doprovodu citar, harf, bubínku, cymbálů a pozounů.
#13:9 Když přišli ke Kídonovu humnu, napřáhl Uza ruku, aby uchopil schránu, protože spřežení vybočilo z cesty.
#13:10 Hospodin vzplanul proti Uzovi hněvem a zabil ho, protože napřáhl ruku na schránu. Zemřel tam před Bohem.
#13:11 Též David vzplanul, že se Hospodin prudce obořil na Uzu, a proto nazval to místo Peres-uza (to je Uzovo zbořenisko); jmenuje se tak dodnes.
#13:12 V onen den pojala Davida bázeň před Bohem. Řekl: „Jak bych mohl vnést Boží schránu až k sobě?“
#13:13 Proto David Boží schránu nepřenesl k sobě do Města Davidova, nýbrž ji dal dopravit do domu Obéd-edóma Gatského.
#13:14 Boží schrána zůstala při domě Obéd-edómově; byla v jeho domě po tři měsíce. Hospodin požehnal Obéd-edómovu domu i všemu, co mu patřilo. 
#14:1 Týrský král Chíram poslal k Davidovi posly; poslal mu cedrové dřevo, zedníky a tesaře, aby mu postavili dům.
#14:2 David poznal, že jej Hospodin potvrdil za krále nad Izraelem, že jeho království bude kvůli jeho izraelskému lidu velice povzneseno.
#14:3 David si v Jeruzalémě vzal další ženy a zplodil další syny a dcery.
#14:4 Toto jsou jména těch, kteří se mu narodili v Jeruzalémě: Šamúa a Šóbab, Nátan a Šalomoun,
#14:5 Jibchár, Elíšua a Elpelet,
#14:6 Nógah, Nefeg a Jafía,
#14:7 Elíšama, Beeljáda a Elífelet.
#14:8 Když Pelištejci uslyšeli, že David byl pomazán za krále nad celým Izraelem, vytáhli všichni Pelištejci Davida hledat. David o tom uslyšel a vytáhl proti nim.
#14:9 Pelištejci přitáhli a vpadli do doliny Refájců.
#14:10 David se doptával Boha: „Mám proti Pelištejcům vytáhnout? Vydáš mi je do rukou?“ Hospodin mu odpověděl: „Vytáhni, vydám ti je do rukou!“
#14:11 Vystoupili tedy do Baal-perasímu a tam je David pobil. David prohlásil: „Bůh jako prudké vody prolomil mým prostřednictvím řady mých nepřátel.“ Proto pojmenovali to místo Baal-perasím (to je Pán průlomu).
#14:12 Pelištejci tam zanechali své bohy a David je dal spálit ohněm.
#14:13 Pelištejci potom znovu vpadli do doliny.
#14:14 David se opět doptával Boha. Bůh mu odpověděl: „Netáhni za nimi. Obejdi je a napadni je směrem od balzámovníků.
#14:15 Jakmile uslyšíš v korunách bazámovníků šelest kroků, vyrazíš do boje, neboť Bůh vyjde před tebou a pobije tábor Pelištejců.“
#14:16 David vykonal, co mu Bůh přikázal. Pobili tábor Pelištejců od Gibeónu až do Gezeru.
#14:17 David se stal proslulým ve všech zemích. Hospodin způsobil, že strach z něho dolehl na všechny národy. 
#15:1 David nastavěl v Městě Davidově domy pro sebe, připravil i místo pro Boží schránu a postavil pro ni stan.
#15:2 Tehdy David řekl: „Boží schránu nesmí nést nikdo kromě lévijců, protože je vyvolil Hospodin, aby nosili Hospodinovu schránu a vždycky u ní přisluhovali.“
#15:3 David svolal celý Izrael do Jeruzaléma, aby vynesli Hospodinovu schránu na místo, které pro ni připravil.
#15:4 David tedy shromáždil Áronovce a lévijce:
#15:5 Z Kehatovců předáka Uríela a sto dvacet jeho bratří.
#15:6 Z Meraríovců předáka Asajáše a dvě stě dvacet jeho bratří.
#15:7 Z Geršómovců předáka Jóela a sto třicet jeho bratří.
#15:8 Z Elísáfanovců předáka Šemajáše a dvě stě jeho bratří.
#15:9 Z Chebrónovců předáka Elíela a osmdesát jeho bratří.
#15:10 Z Uzíelovců předáka Amínadaba a sto dvanáct jeho bratří.
#15:11 David povolal kněze Sádoka a Ebjátara a lévijce Uríela, Asajáše, Jóela, Šemajáše, Elíela a Amínadaba.
#15:12 Nařídil jim: „Vy jste představitelé lévijských rodů. Posvěťte se spolu se svými bratry. Vynesete schránu Hospodina, Boha Izraele, na místo, které jsem pro ni připravil.
#15:13 Že jste při tom ponejprv nebyli, Hospodin, náš Bůh, se prudce na nás obořil, protože jsme se ho nedotázali podle řádu.“
#15:14 Kněží a lévijci se tedy posvětili a vynesli schránu Hospodina, Boha Izraele.
#15:15 Léviovci nesli Boží schránu na ramenou na sochorech, jak podle Hospodinova slova přikázal Mojžíš.
#15:16 David také lévijským předákům nařídil, aby ustanovili své bratry zpěváky, kteří by radostně hlaholili na hudební nástroje, harfy, citary a zvučné cymbály.
#15:17 I ustanovili lévijci Hémana, syna Jóelova, a z jeho bratří Asafa, syna Berekjášova, a z Meraríovců, jejich bratří, Étana, syna Kúšajášova.
#15:18 A na střídání s nimi jejich bratry Zakarjáše, Bena, Jaazíela, Šemíramóta, Jechíela, Uního, Elíaba, Benajáše, Maasejáše, Matitjáše, Elíflehúa, Miknejáše, Obéd-edóma a Jeíela, vrátné.
#15:19 Zpěváci Héman, Asaf a Étan zvučně hráli na měděné cymbály.
#15:20 Zekarjáš, Azíel, Šemíramót, Jechíel, Uní, Elíab, Maasejáš a Benajáš doprovázeli vysoký zpěv na harfy,
#15:21 Matitjáš, Elíflehú, Miknejáš, Obéd-edóm, Jeíel a Azazjáš doprovázeli hluboký zpěv na citary.
#15:22 Kenanjáš, předák lévijců, určených k přenášení schrány, byl pověřen dozorem při přenášení; vyznal se v těchto věcech.
#15:23 Berekjáš a Elkána byli při schráně vrátnými.
#15:24 Kněží Šebanjáš, Jóšafat, Netaneel, Amasaj, Zekarjáš, Benajáš a Elíezer troubili před Boží schránou na pozouny. Obéd-edóm a Jechijáš byli také vrátnými při schráně.
#15:25 David, izraelští starší a velitelé nad tisíci radostně vystupovali se schránou Hospodinovy smlouvy z Obéd-edómova domu.
#15:26 Protože Bůh prokázal svou pomoc lévijcům, kteří nesli schránu Hospodinovy smlouvy, obětovali sedm býčků a sedm beranů.
#15:27 David byl oděn pláštěnkou z bělostného plátna, stejně tak všichni lévijci, kteří nesli schránu, zpěváci a Kenanjáš, předák zpěváků při přenášení. David měl na sobě též lněný efód.
#15:28 Celý Izrael vystupoval se schránou Hospodinovy smlouvy za ryčného troubení polnic a za zvuku pozounů, cymbálů, harf a citar.
#15:29 Když schrána Hospodinovy smlouvy vstupovala do Města Davidova, Míkal, dcera Saulova, se právě dívala z okna. Viděla krále Davida, jak poskakuje a křepčí, a v srdci jím pohrdla. 
#16:1 Boží schránu přinesli a umístili ji uprostřed stanu, který pro ni David postavil. I přinášeli před Bohem zápalné a pokojné oběti.
#16:2 Když David dokončil obětování zápalných a pokojných obětí, požehnal lidu v Hospodinově jménu.
#16:3 Pak podělil každého z Izraele, každého muže i ženu, bochníčkem chleba a datlovým a hrozinkovým koláčem.
#16:4 Potom určil lévijce, kteří by přisluhovali u Hospodinovy schrány a připomínali Hospodina, Boha Izraele, vzdávali mu chválu a oslavovali jej.
#16:5 Asaf byl představeným, jeho zástupcem byl Zekarjáš. Jeíel, Šemíramót, Jechíel, Matitjáš, Elíab, Benajáš, Obéd-edóm a Jeíel hráli na harfy a citary, Asaf na zvučné cymbály.
#16:6 Kněží Benajáš a Jachazíel hráli každodenně na pozouny před schránou Boží smlouvy.
#16:7 Tehdy onoho dne nařídil David poprvé, aby Asaf a jeho bratří vzdávali Hospodinu chválu:
#16:8 Chválu vzdejte Hospodinu a vzývejte jeho jméno, uvádějte národům ve známost jeho skutky,
#16:9 zpívejte mu, pějte žalmy, přemýšlejte o všech jeho divech,
#16:10 honoste se jeho svatým jménem, ať se raduje srdce těch, kteří hledají Hospodina!
#16:11 Dotazujte se na vůli Hospodinovu a jeho moc, jeho tvář hledejte ustavičně.
#16:12 Připomínejte divy, jež vykonal, jeho zázraky a rozsudky jeho úst,
#16:13 potomkové Izraele, jeho služebníka, Jákobovi synové, jeho vyvolení!
#16:14 Je to Hospodin, náš Bůh, kdo soudí celou zemi.
#16:15 Věčně pamatujte na jeho smlouvu, na slovo, jež přikázal tisícům pokolení.
#16:16 Uzavřel ji s Abrahamem, přísahou ji stvrdil Izákovi,
#16:17 stanovil ji Jákobovi jako nařízení, Izraeli jako smlouvu věčnou:
#16:18 „Dám ti kenaanskou zemi za dědičný úděl!“
#16:19 Na počet vás byla malá hrstka, byli jste tam hosty.
#16:20 Putovali od jednoho pronároda ke druhému, z jednoho království dál k jinému lidu.
#16:21 On však nedovolil nikomu, aby je utlačoval, káral kvůli nim i krále:
#16:22 „Nesahejte na mé pomazané, ublížit mým prorokům se chraňte!“
#16:23 Zpívej Hospodinu, celá země! Zvěstujte den ze dne jeho spásu,
#16:24 vypravujte mezi pronárody o jeho slávě, mezi všemi národy o jeho divech,
#16:25 neboť Hospodin je veliký, nejvyšší chvály hodný, budí bázeň, je nad všechny bohy.
#16:26 Všechna božstva národů jsou bůžci, ale Hospodin učinil nebe.
#16:27 Před jeho tváří velebná důstojnost, moc a potěšení na svatém místě jeho.
#16:28 Lidské čeledi, přiznejte Hospodinu, přiznejte Hospodinu slávu a moc,
#16:29 přiznejte Hospodinu slávu jeho jména! Přineste dar, vstupte před něj, v nádheře svatyně se klaňte Hospodinu!
#16:30 Svíjej se před ním, celá země! Pevně je založen svět, nic jím neotřese.
#16:31 Nebesa se zaradují, rozjásá se země, mezi národy se bude říkat: „Hospodin kraluje!“
#16:32 Moře i s tím, co je v něm, se rozburácí, pole zazní jásotem, i všechno, co je na něm.
#16:33 Tehdy zaplesají stromy v lese vstříc Hospodinu, že přichází soudit zemi.
#16:34 Chválu vzdejte Hospodinu, protože je dobrý, jeho milosrdenství je věčné.
#16:35 Rcete: Zachraň nás, ó Bože, naše spáso, shromáždi a vytrhni nás z pronárodů; tvému svatému jménu budeme vzdávat chválu, budeme tě chválit chvalozpěvem.
#16:36 Požehnán buď Hospodin, Bůh Izraele, od věků až na věky!“ A všechen lid ať řekne: „Amen. Chvalte Hospodina!“
#16:37 Tam před schránou Hospodinovy smlouvy zanechal David Asafa a jeho bratry, aby každodenně, podle denního rozvrhu, před schránou přisluhovali.
#16:38 Obéd-edóma a jeho šedesát osm bratří i Obéd-edóma, syna Jedítúnova, a Chósu zanechal jako vrátné.
#16:39 Kněze Sádoka a jeho bratry kněze zanechal před Hospodinovým příbytkem na posvátném návrší v Gibeónu,
#16:40 aby každodenně zrána i zvečera obětovali Hospodinu na oltáři zápalné oběti, a to podle všeho toho, co je zapsáno v Hospodinově zákoně, který vydal Izraeli.
#16:41 S nimi byli Héman a Jedútún a ostatní čistí, kteří byli uvedeni jmény, aby vzdávali chválu Hospodinu, neboť jeho milosrdenství je věčné.
#16:42 Héman a Jedútún měli s sebou pozouny, zvučné cymbály a nástroje k Božímu zpěvu. Jedútúnovci byli u brány.
#16:43 I odešel všechen lid, každý do svého domu. A David se obrátil ke svému domu, aby mu udělil požehnání. 
#17:1 Když David už sídlil ve svém domě, řekl proroku Nátanovi: „Hle, já sídlím v domě cedrovém a schrána Hospodinovy smlouvy je pod stanovými houněmi.“
#17:2 Nátan Davidovi odvětil: „Udělej vše, co máš na srdci, neboť Bůh je s tebou.“
#17:3 Ale té noci se stalo Boží slovo k Nátanovi:
#17:4 „Jdi a řekni mému služebníku Davidovi: Toto praví Hospodin: Ty mi nebudeš budovat dům, v němž bych sídlil.
#17:5 Nesídlil jsem v domě od toho dne, kdy jsem vyvedl Izraele, až do dne tohoto. Putoval jsem se stanem, ale příbytek jsem neměl.
#17:6 Ať jsem přecházel s celým Izraelem kudykoli, zdalipak jsem kdy řekl některému z izraelských soudců, jemuž jsem přikázal pást můj lid: ‚Proč mi nezbudujete cedrový dům?‘
#17:7 Nyní tedy promluvíš takto k mému služebníku Davidovi: Toto praví Hospodin zástupů: Vzal jsem tě z pastvin od stáda, abys byl vévodou nad mým lidem, nad Izraelem.
#17:8 Byl jsem s tebou, ať jsi šel kamkoli. Vyhladil jsem před tebou všechny tvé nepřátele. Tvé jméno jsem učinil tak veliké, jako je jméno velikánů na zemi.
#17:9 I svému lidu, Izraeli, jsem připravil místo a zasadil jej; tam bude bydlet a už nikdy nebude znepokojován, už jej nebudou vysávat bídáci jako dřív.
#17:10 Ode dnů, kdy jsem správou svého izraelského lidu pověřil soudce, podrobil jsem ti všechny nepřátele. Oznamuji ti, že Hospodin vybuduje dům tobě.
#17:11 Až se naplní tvé dny a ty odejdeš ke svým otcům, dám po tobě povstat tvému potomku z tvých synů a upevním jeho království.
#17:12 Ten mi vybuduje dům a já upevním jeho trůn navěky.
#17:13 Já mu budu Otcem a on mi bude synem. Svoje milosrdenství mu neodejmu, jako jsem odňal tomu, který byl před tebou.
#17:14 Ustanovím jej ve svém domě a ve svém království navěky. Jeho trůn bude navěky upevněn.“
#17:15 Nátan k Davidovi promluvil ve smyslu všech těchto slov a celého tohoto vidění.
#17:16 Král David pak vešel, usedl před Hospodinem a řekl: „Co jsem, Hospodine Bože, a co je můj dům, že jsi mě přivedl až sem?
#17:17 A i to bylo v tvých očích málo, Bože. Dokonce přislibuješ domu svého služebníka dlouhá léta. Shlédl jsi na mě, jako bych byl člověk vysoké hodnosti, Hospodine Bože.
#17:18 Co může David ještě dodat k slávě, kterou jsi dal svému služebníku? Vždyť ty znáš svého služebníka.
#17:19 Hospodine, kvůli svému služebníku a podle svého srdce jsi učinil celou tuto velikou věc a dal o všech těchto velikých věcech vědět.
#17:20 Hospodine, není žádného jako ty, není Boha kromě tebe, podle toho všeho, co jsme na vlastní uši slyšeli.
#17:21 Kdo je jako tvůj izraelský lid, jediný pronárod na zemi, jejž si Bůh přišel vykoupit za lid; učinil sis jméno velikými a hroznými činy, když jsi vypudil pronárody před svým lidem, který sis vykoupil z Egypta.
#17:22 Ty jsi stanovil, aby tvůj izraelský lid byl tvým lidem navěky, a sám ses jim stal, Hospodine, Bohem.
#17:23 Nyní tedy, Hospodine Bože, ať se navěky věrné ukáže tvé slovo, které jsi promluvil o svém služebníku a o jeho domu. Učiň, jak jsi promluvil.
#17:24 Ať se navěky osvědčí jako věrné a veliké tvé jméno, ať se říká: Hospodin zástupů, Bůh Izraele, je Izraeli Bohem. A dům tvého služebníka Davida ať je před tebou pevný.
#17:25 Neboť ty, Bože, jsi svému služebníku ohlásil, že mu vybuduješ dům. Proto tvůj služebník našel odvahu modlit se před tebou.
#17:26 Ano, Hospodine, ty sám jsi Bůh. Přislíbil jsi svému služebníku takové dobrodiní.
#17:27 Nyní tedy požehnej laskavě domu svého služebníka, aby trval před tebou navěky. Dáš-li mu, Hospodine, požehnání, bude požehnán navěky!“ 
#18:1 Potom David porazil Pelištejce a podrobil si je a vyrval Gat a jeho vesnice z rukou Pelištejců.
#18:2 Porazil i Moábce. Moábci se stali Davidovými otroky a odváděli dávky.
#18:3 David také porazil u Chamátu Hadad-ezera, krále Sóby, když táhl, aby nad řekou Eufratem vztyčil znamení své moci.
#18:4 David zajal tisíc jeho koní k vozům a sedm tisíc jezdců a dvacet tisíc pěšáků. Všechny koně k vozům David ochromil, zanechal koně jen ku stu vozům.
#18:5 Hadad-ezerovi, králi Sóby, přišli na pomoc Aramejci z Damašku. David pobil z Aramejců dvaadvacet tisíc mužů.
#18:6 Pak David umístil do damašského Aramu výsostná znamení. Tak se i Aramejci stali Davidovými otroky a odváděli dávky. Hospodin Davida zachraňoval, ať šel kamkoli.
#18:7 David také pobral zlaté štíty, které na sobě měli Hadad-ezerovi služebníci, a přinesl je do Jeruzaléma.
#18:8 Z Tibchatu a Kúnu, Hadad-ezerových měst, pobral David velké množství bronzu; z něho udělal Šalomoun bronzové moře a sloupy i bronzové nádoby.
#18:9 Když Toú, král Chamátu, uslyšel, že David porazil celé vojsko Hadad-ezera, krále Sóby,
#18:10 poslal svého syna Hadórama ke králi Davidovi, aby mu popřál pokoje a dobrořečil mu za jeho boj a vítězství nad Hadad-ezerem, že jej porazil; Hadad-ezer vedl totiž s Toúem války. Hadóram přinesl všelijaké předměty zlaté, stříbrné a měděné.
#18:11 I ty král David zasvětil Hospodinu se stříbrem a zlatem, které vzal od všech pronárodů, od Edómců, Moábců, Amónovců, Pelištejců a Amáleka.
#18:12 Abšaj, syn Serújin, pobil z Edómců v Solném údolí osmnáct tisíc mužů.
#18:13 V Edómu umístil David výsostná znamení. Všichni Edómci se stali Davidovými otroky. Hospodin Davida zachraňoval, ať šel kamkoli.
#18:14 David kraloval nad celým Izraelem a prosazoval právo a spravedlnost pro všechen svůj lid.
#18:15 Jóab, syn Serújin, byl vrchním velitelem, Jóšafat, syn Achílúdův, byl kancléřem.
#18:16 Sádok, syn Achítúbův, a Abímelek, syn Ebjátarův, byli kněžími. Šavša byl písařem.
#18:17 Benajáš, syn Jójadův, byl nad Keretejci a Peletejci. Davidovi synové byli pobočníky krále. 
#19:1 Potom se stalo, že zemřel král Amónovců Náchaš a po něm se ujal království jeho syn.
#19:2 I řekl David: „Prokáži milosrdenství Chanúnovi, synu Náchašovu. Vždyť jeho otec prokazoval milosrdenství mně.“ David tedy poslal posly, aby ho potěšil v zármutku nad otcem. Davidovi služebníci přišli k Chanúnovi do země Amónovců, aby ho potěšili.
#19:3 Amónovští velitelé však Chanúna podněcovali: „Tobě se zdá, že David chce uctít tvého otce? Že k tobě poslal těšitele? Nepřišli k tobě jeho služebníci spíše proto, aby prozkoumali a rozvrátili zemi?“
#19:4 Chanún dal tedy Davidovy služebníky oholit, dal uříznout jejich roucho do poloviny, až do rozkroku, a tak je propustil.
#19:5 I šli. Když to o těch mužích oznámili Davidovi, poslal jim naproti, protože ti muži byli velice zhanobeni. Král nařídil: „Zůstaňte v Jerichu, dokud vám brady neobrostou; pak se vrátíte.“
#19:6 Amónovci viděli, že zavinili Davidovu nelibost. Proto Chanún s Amónovci poslal tisíc talentů stříbra, aby si najali koně k vozům a jezdce od Aramejců z Dvojříčí, Aramejců z Maaky a ze Sóby.
#19:7 Najali si třicet dva tisíce koní k vozům od krále z Maaky a jeho lid. Ti přitáhli a utábořili se proti Médebě. Také Amónovci se shromáždili ze svých měst a přitáhli do boje.
#19:8 Když to David uslyšel, poslal Jóaba s celým vojem bohatýrů.
#19:9 Amónovci vytáhli a seřadili se k boji u vchodu do města. Králové, kteří přitáhli, stáli stranou v poli.
#19:10 Když Jóab spatřil, že má proti sobě bitevní řady vpředu i vzadu, vybral si nejlepší ze všech izraelských mladíků a seřadil je proti Aramejcům.
#19:11 Zbytek lidu podřídil svému bratru Abšajovi. I seřadili se proti Amónocům.
#19:12 Řekl: „Budou-li mít Aramejci nade mnou převahu, přijdeš mi na pomoc. Budou-li mít Amónovci převahu nad tebou, přijdu ti na pomoc já.
#19:13 Buď rozhodný! Vzchopme se! Za náš lid, za města našeho boha! Hospodin nechť učiní, co uzná za dobré.“
#19:14 Jóab a lid, který byl s ním, se dali do boje proti Aramejcům; ti před ním utekli.
#19:15 Když Amónovci viděli, že Aramejci utíkají, utekli také oni před jeho bratrem Abšajem a vešli do města. Jóab pak přišel do Jeruzaléma.
#19:16 Když Aramejci viděli, že jsou Izraelem poraženi, poslali posly, aby přivedli Aramejce, kteří byli za Řekou. V jejich čele byl Šófak, velitel Hadad-ezerova vojska.
#19:17 Oznámili to Davidovi a on shromáždil celý Izrael, překročil Jordán, přitáhl k nim a seřadil se proti nim. David se seřadil k boji proti Aramejcům a ti s ním bojovali.
#19:18 Aramejci se však dali před Izraelem na útěk. David pobil koně od sedmi tisíc aramejských vozů a čtyřicet tisíc pěšáků, usmrtil též Šófaka, velitele vojska.
#19:19 Když Hadad-ezerovi služebníci viděli, že jsou Izraelem poraženi, uzavřeli s Davidem příměří a sloužili mu. Aramejci pak už nikdy nechtěli jít Amónovcům na pomoc. 
#20:1 Na přelomu roku, v době, kdy králové táhnou do boje, vedl Jóab vojenskou výpravu, hubil zemi Amónovců a oblehl Rabu. David však zůstal v Jeruzalémě. Jóab udeřil na Rabu a rozbořil ji.
#20:2 David sňal z hlavy jejich Krále korunu. Shledal, že váží talent zlata; byl do ní vsazen drahokam. Bývala pak na hlavě Davidově. Z města odvezl též velké množství kořisti.
#20:3 Lid, který byl v něm, odvedl a přidělil k pilám, železným špičákům a sekyrám. Tak David naložil se všemi městy Amónovců. Pak se David i všechen lid vrátili do Jeruzaléma.
#20:4 O něco později došlo k boji s Pelištejci v Gezeru. Tehdy ubil Sibekaj Chúšatský Sipaje z rodu obrů; ti byli pokořeni.
#20:5 Když se znovu strhl boj s Pelištejci, Elchánan, syn Jaírův, ubil Lachmího, bratra Goliáše Gatského. Násada jeho kopí byla jako tkalcovské vratidlo.
#20:6 A opět se strhl boj v Gatu. Tam byl obrovitý muž, který měl po šesti prstech, celkem čtyřiadvacet prstů. Ten také pocházel z obrů.
#20:7 Když tupil Izraele, ubil ho Jónatan, syn Šimey, Davidova bratra.
#20:8 Ti pocházeli z gatských obrů. Padli do rukou Davida a rukou jeho služebníků. 
#21:1 Proti Izraeli povstal satan a podnítil Davida, aby sečetl Izraele.
#21:2 David poručil Jóabovi a předákům lidu: „Jděte, spočítejte Izraele od Beer-šeby až k Danu a podejte mi zprávu, chci znát jejich počet.“
#21:3 Jóab králi namítl: „Nechť Hospodin zvětší svůj lid třeba stokrát. Což nejsou ti všichni, králi, můj pane, služebníky mého pána? Proč to chce můj pán zjistit? Proč se má Izrael provinit?“
#21:4 Královo rozhodnutí bylo však pro Jóaba nezvratné. Jóab vyšel a prošel celý Izrael. Pak přišel do Jeruzaléma.
#21:5 Jóab odevzdal Davidovi celkový součet lidu: Všeho Izraele bylo jeden milión a sto tisíc mužů schopných tasit meč a Judy čtyři sta sedmdesát tisíc mužů schopných tasit meč.
#21:6 Léviho a Benjamína do nich Jóab nezapočítal, neboť královo rozhodnutí pokládal za ohavnost.
#21:7 Byla to zlá věc i v očích Božích, proto Bůh Izraele ranil.
#21:8 David volal k Bohu: „Velmi jsem zhřešil, že jsem učinil tuto věc. Nyní přenes prosím vinu svého služebníka, neboť jsem si počínal jako velký pomatenec.“
#21:9 Hospodin promluvil ke Gádovi, Davidovu vidoucímu:
#21:10 „Jdi a promluv k Davidovi: Toto praví Hospodin: Chystám na tebe trojí. Jedno z toho si vyber a já tak s tebou naložím.“
#21:11 Gád přišel k Davidovi a řekl mu: „Toto praví Hospodin: Zvol si:
#21:12 Tři roky hladu, nebo tři měsíce být ničen protivníky a stíhán mečem nepřátel, anebo po tři dny Hospodinův meč, totiž mor v zemi, a Hospodinův anděl bude šířit zkázu po celém izraelském území. Nuže tedy, co mám vyřídit tomu, který mě poslal?“
#21:13 David Gádovi odvětil: „Je mi velmi úzko. Nechť prosím upadnu do rukou Hospodinu, neboť jeho slitování je přenesmírné, jen ať nepadnu do rukou lidských.“
#21:14 Hospodin tedy dopustil na Izraele mor. I padlo z Izraele sedmdesát tisíc mužů.
#21:15 Bůh vyslal k Jeruzalému anděla, aby v něm šířil zkázu. Ale když šířil zkázu, Hospodin shlédl a pojala ho lítost nad tím zlem. Řekl andělu, jenž šířil zkázu: „Dost! Již přestaň!“ Hospodinův anděl stál právě u humna Ornána Jebúsejského.
#21:16 David se rozhlédl a uviděl Hospodinova anděla, jak stojí mezi zemí a nebem. V ruce měl tasený meč, napřažený na Jeruzalém. David a starší, zahalení v žíněná roucha, padli tváří k zemi.
#21:17 David volal k Bohu: „Což jsem sčítání lidu nenařídil já sám? Já sám jsem zhřešil, já sám jsem se dopustil takového zla! Co však učinily tyto ovce? Hospodine, můj Bože, buď tedy tvá ruka proti mně a proti domu mého otce, jen ať ta pohroma nestíhá tvůj lid!“
#21:18 Hospodinův anděl nařídil Gádovi, aby Davidovi řekl: „Ať David vystoupí a vystaví Hospodinu oltář na humně Ornána Jebúsejského.“
#21:19 David vystoupil podle Gádova slova, jež promluvil v Hospodinově jménu.
#21:20 Ornán se obrátil a spatřil toho anděla; jeho čtyři synové, kteří byli s ním, se skrývali a Ornán mlátil pšenici.
#21:21 Když David došel k Ornánovi, Ornán vzhlédl a uviděl Davida. Vyšel z humna a dvakrát se Davidovi poklonil tváří k zemi.
#21:22 David Ornánovi řekl: „Přenech mi to místo, své humno. Chci na něm vybudovat Hospodinu oltář. Dej mi je za plnou cenu stříbra, ať je od lidu odvrácena pohroma.“
#21:23 Ornán Davidovi odvětil: „Vezmi si je. Ať král, můj pán, udělá, co uzná za dobré. Hle, přidám dobytek k zápalným obětem a smyky jako dříví, též pšenici jako obětní dar. To vše dám.“
#21:24 Ale král David Ornánovi řekl: „Nikoli. Odkoupím to od tebe za plnou cenu stříbra. Vždyť nemohu přinést Hospodinu, co je tvé, nemohu obětovat zápalnou oběť darovanou.“
#21:25 David tedy dal Ornánovi za to místo obnos šesti set šekelů zlata.
#21:26 I vybudoval tam David Hospodinu oltář, obětoval zápalné a pokojné oběti a vzýval Hospodina. Ten mu odpověděl ohněm seslaným z nebe na oltář pro zápalné oběti.
#21:27 Hospodin pak poručil andělu, aby schoval meč do pochvy.
#21:28 V té době, kdy David zjistil, že jej Hospodin na humně Ornána Jebúsejského vyslyšel, začal tam obětovat.
#21:29 Ale Hospodinův příbytek, který zhotovil Mojžíš na poušti, i oltář pro zápalné oběti byly v té době na posvátném návrší v Gibeónu.
#21:30 Tam David nemohl chodit, aby se dotazoval Hospodina, neboť byl mečem Hospodinova anděla přestrašen. 
#22:1 David prohlásil: „Zde bude dům Hospodina Boha, zde bude mít Izrael oltář k zápalným obětem.“
#22:2 David poručil soustředit ty, kdo pobývali v izraelské zemi jako hosté, a udělal z nich lamače, kteří připravovali kvádry pro stavbu Božího domu.
#22:3 David také připravil mnoho železa na čepy pro křídla bran a na skoby a tak mnoho mědi, že se ani nedala zvážit.
#22:4 Také cedrového dřeva bezpočet; to množství cedrového dřeva přiváželi Davidovi Sidóňané a Týřané.
#22:5 David řekl: „Můj syn Šalomoun je ještě mladíček útlého věku. Má-li vybudovat Hospodinův dům do mohutné výše, aby proslul nádherou po všech zemích, musím pro to konat přípravy.“ Proto David ještě před svou smrtí vykonal mnoho příprav.
#22:6 Pak zavolal svého syna Šalomouna a přikázal mu, aby Hospodinu, Bohu Izraele, vybudoval dům.
#22:7 David Šalomounovi řekl: „Můj synu, já sám jsem měl v úmyslu vybudovat dům pro jméno Hospodina, svého Boha.
#22:8 Stalo se však ke mně slovo Hospodinovo: ‚Prolil jsi mnoho krve, vedl jsi velké války. Nebudeš budovat dům pro mé jméno, protože jsi na zemi přede mnou prolil mnoho krve.
#22:9 Hle, narodí se ti syn. Ten bude mužem odpočinutí. Jemu dám odpočinout od všech jeho okolních nepřátel. Vždyť jeho jméno bude Šalomoun (to je Pokojný). Za jeho dnů poskytnu Izraeli pokoj a mír.
#22:10 On vybuduje dům pro mé jméno. On se stane mým synem a já mu budu Otcem. Jeho královský trůn nad Izraelem upevním navěky.‘
#22:11 Nuže, můj synu, nechť je s tebou Hospodin. Se zdarem vybuduješ dům Hospodina, svého Boha, jak to o tobě vyřkl.
#22:12 Kéž ti Hospodin dá prozíravost a rozumnost, až tě ustanoví nad Izraelem, abys dbal na zákon Hospodina, svého Boha.
#22:13 Jen tehdy budeš mít zdar, budeš-li bedlivě plnit nařízení a řády, které Hospodin přikázal Mojžíšovi pro Izraele. Buď rozhodný a udatný! Neboj se a neděs!
#22:14 Hle, i ve svém trápení jsem pro Hospodinův dům připravil sto tisíc talentů zlata, milión talentů stříbra, mědi a železa tolik, že se ani nedá zvážit. Připravil jsem též dřevo a kámen. A ty k tomu ještě přidáš.
#22:15 S tebou bude při díle mnoho dělníků, lamačů, kameníků a tesařů, každý, kdo je dovedný v nějakém díle.
#22:16 Zlata, stříbra, mědi a železa je bezpočet. Dej se do práce a nechť je s tebou Hospodin!“
#22:17 David přikázal všem izraelským předákům, aby jeho synu Šalomounovi byli nápomocni:
#22:18 „Což není Hospodin, váš Bůh, s vámi? Dal vám odpočinout ode všech okolních nepřátel. Vydal mi do rukou obyvatele země. Země je podmaněna Hospodinu a jeho lidu.
#22:19 Nyní se tedy ze srdce a z duše dotazujte Hospodina, svého Boha, a dejte se do budování svatyně Boha Hospodina, abyste mohli vnést schránu Hospodinovy smlouvy a svaté Boží náčiní do domu, který bude pro Hospodinovo jméno vybudován.“ 
#23:1 Když David zestárl a byl sytý dnů, ustanovil za krále nad Izraelem svého syna Šalomouna.
#23:2 Shromáždil všechny izraelské velitele, kněze a lévijce.
#23:3 Lévijci byli sečteni od třicetiletých výše; seznam jednotlivců činil třicet osm tisíc mužů.
#23:4 Z nich bylo čtyřiadvacet tisíc pověřeno dohledem nad dílem domu Hospodinova, šest tisíc bylo správci a soudci,
#23:5 čtyři tisíce vrátnými a čtyři tisíce oslavovaly Hospodina na nástroje, které dal k oslavování Hospodina udělat.
#23:6 David je rozdělil do tříd
#23:7 podle Geršóna, Kehata a Merarího, synů Léviho. Ke Geršónovi patří Laedán a Šimeí.
#23:8 Synové Laedánovi byli tři: Přední byl Jechíel, pak Zétam a Jóel.
#23:9 Synové Šimeího: Šelómít, Chazíel a Háran. Ti tři byli v Laedánových rodech přední.
#23:10 Synové Šimeího: Jáchat, Zina, Jeúš a Bería. To byli čtyři synové Šimeího.
#23:11 Jáchat byl přední, Zína druhý; Jeúš a Bería neměli mnoho synů, byli tedy v jednom rodovém soupisu.
#23:12 Synové Kehatovi byli čtyři: Amrám, Jishár, Chebrón a Uzíel.
#23:13 Synové Amrámovi: Áron a Mojžíš. Áron byl oddělen, aby střežil svatost velesvatyně, on i jeho synové, aby pálil před Hospodinem kadidlo a přisluhoval mu a navěky v jeho jménu udílel požehnání.
#23:14 Mojžíš byl muž Boží. Jeho synové byli jmenováni s kmenem Léviho.
#23:15 Synové Mojžíšovi byli Geršóm a Elíezer.
#23:16 Synové Geršómovi: přední Šebúel.
#23:17 Synové Elíezerovi: přední Rechabjáš. Elíezer neměl jiné syny, zato synové Rechabjášovi se velmi rozmnožili.
#23:18 Synové Jishárovi: přední Šelómít.
#23:19 Synové Chebrónovi: přední Jeriáš, druhý Amarjáš, třetí Jachazíel, čtvrtý Jekameám.
#23:20 Synové Uzíelovi: přední Míka, druhý Jišijáš.
#23:21 Synové Merarího: Machlí a Múši. Synové Machlího: Eleazar a Kíš.
#23:22 Eleazar zemřel, aniž měl syny; měl dcery. Vzali si je jejich bratranci, synové Kíšovi.
#23:23 Synové Múšiho byli tři: Machlí, Éder a Jeremót.
#23:24 To jsou Léviovci podle svých otcovských rodů, přední v rodech, povolaní do služby a zapsaní do jmenného seznamu podle jednotlivců, konající služebné dílo pro Hospodinův dům, a to od dvacetiletých výše.
#23:25 David totiž řekl: „Hospodin, Bůh Izraele, dal svému lidu odpočinutí. Navěky bude přebývat v Jeruzalémě.
#23:26 Lévijci už tedy nebudou nosit příbytek a všechno náčiní potřebné k jeho obsluze.“
#23:27 Proto podle posledního Davidova rozkazu byli do seznamu zahrnuti Léviovci od dvacetiletých výše.
#23:28 Byli dáni k ruce Áronovým synům, aby sloužili v Hospodinově domě, na jeho nádvořích a v komorách, aby dbali na čistotu všeho svatého a konali v Božím domě službu;
#23:29 aby měli na starosti rovnání předkladných chlebů, bílou mouku pro přídavné oběti, oplatky nekvašených chlebů, pánve a mísy, různé odměrky a míry;
#23:30 aby nastoupili každé ráno k vzdávání chval a oslavování Hospodina, a právě tak i večer;
#23:31 aby před Hospodinem ustavičně přisluhovali při obětování zápalných obětí Hospodinu ve dnech odpočinku, o novoluních a v určených časech, a to v tom počtu, jak určuje jejich řád.
#23:32 Konali strážní službu při stanu setkávání, strážní službu při svatyni a strážní službu u svých bratří, synů Áronových, při službě v domě Hospodinově. 
#24:1 Třídy synů Áronových: Synové Áronovi byli Nádab, Abíhú, Eleazar a Ítamar.
#24:2 Nádab a Abíhú zemřeli dříve než jejich otec a neměli syny. Jako kněží sloužili Eleazar a Ítamar.
#24:3 David i Sádok ze synů Eleazarových a Achímelek ze synů Ítamarových je rozdělili do služeb podle jejich povolání.
#24:4 Ukázalo se, že synové Eleazarovi jsou co do počtu předních mužů četnější než synové Ítamarovi. Podle toho je rozdělili: předních mužů ze synů Eleazarových bylo podle otcovských rodů šestnáct, synů Ítamarových podle otcovských rodů osm.
#24:5 Rozdělili je navzájem losem, neboť předáci svatyně a předáci Boží měli být ze synů Eleazarových a ze synů Ítamarových.
#24:6 Písař z Léviho rodu Šemajáš, syn Netaneelův, sepsal lévijce před králem a předáky, před knězem Sádokem a Achímelekem, synem Ebjátarovým, a předními z otcovských rodů kněžských a lévijských. Jeden rod byl určen Eleazarovi a ještě jeden, jiný byl určen Ítamarovi.
#24:7 První los padl na Jójaríba, druhý na Jedajáše,
#24:8 třetí na Chárima, čtvrtý na Seórima,
#24:9 pátý na Malkijáše, šestý na Mijámina,
#24:10 sedmý na Kósa, osmý na Abijáše,
#24:11 devátý na Jéšuu, desátý na Šekanjáše,
#24:12 jedenáctý na Eljašíba, dvanáctý na Jákima,
#24:13 třináctý na Chupu, čtrnáctý na Ješebába,
#24:14 patnáctý na Bilgu, šestnáctý na Iméra,
#24:15 sedmnáctý na Chezíra, osmnáctý na Pisesa,
#24:16 devatenáctý na Petachjáše, dvacátý na Jechezkéla,
#24:17 jedenadvacátý na Jakína, dvaadvacátý na Gamúla,
#24:18 třiadvacátý na Delajáše, čtyřiadvacátý na Maazjáše.
#24:19 Toto jsou povolaní do služby, aby vstupovali do Hospodinova domu podle svého řádu za dozoru svého otce Árona, jak mu jej přikázal Hospodin, Bůh Izraele.
#24:20 Ostatní Léviovci: ze synů Amrámových Šúbael, ze synů Šúbaelových Jechdejáš.
#24:21 Za Rechabjáše: přední ze synů Rechabjášových Jišijáš.
#24:22 Za Jishára: Šelomót, ze synů Šelomótových Jáchat.
#24:23 Synové Jerijášovi: Amarjáš byl druhý, Jachazíel třetí, Jekameám čtvrtý.
#24:24 Synové Uzíelovi: Míka. Ze synů Míkových: Šámir.
#24:25 Bratr Míkův byl Jišijáš. Ze synů Jišijášových Zekarjáš.
#24:26 Synové Merarího: Machlí a Múši, synové jeho syna Jaazijáše.
#24:27 Synové Merarího: za Jaazijáše jeho syn a Šóham, Zákur a Ibrí.
#24:28 Za Machlího Eleazar, ale ten neměl syny.
#24:29 Za Kíše synové Kíšovi, Jerachmeel.
#24:30 Synové Múšiho: Machlí, Éder a Jerímót. To jsou Léviovci podle svých otcovských rodů.
#24:31 Ty rovněž vylosovali, předního stejně jako nejmladšího bratra, aby byli po boku svým bratřím, synům Áronovým, před králem Davidem i před Sádokem a Achímelekem a předními muži z otcovských rodů kněžských a lévijských. 
#25:1 David a velitelé vojska přidělili též službu synům Asafovým, Hémanovým a Jedútúnovým, aby vyhlašovali proroctví při citaře, harfě a při cymbálech. Seznam mužů, konajících služebné dílo:
#25:2 Ze synů Asafových Zákur, Josef, Netanjáš a Asaréla. Synové Asafovi byli k ruce Asafovi, když z králova pověření vyhlašoval proroctví.
#25:3 Za Jedútúna šest Jedútúnových synů: Gedaljáš, Serí, Ješajáš, Chašabjáš a Matitjáš; byli k ruce svému otci Jedútúnovi, jenž vyhlašoval proroctví při citaře k chvále a oslavě Hospodina.
#25:4 Za Hémana synové Hémanovi: Bukijáš, Matanjáš, Uzíel, Šebúel, Jerímót, Chananjáš, Chananí, Elíata, Gidaltí, Rómamtí-ezer, Jošbekáša, Malótí, Hótir a Machazíót.
#25:5 Ti všichni jsou synové Hémana, králova vidoucího, podle Božích slov o pozdvižení rohu. Bůh dal Hémanovi čtrnáct synů a tři dcery.
#25:6 Ti všichni byli k ruce svému otci při zpěvu v Hospodinově domě s cymbály, haframi a citarami, ke službě v domě Božím z králova pověření, k ruce Asafovi, Jedútúnovi a Hémanovi.
#25:7 Jejich počet spolu s jejich bratry vyučenými Hospodinovu zpěvu byl dvě stě osmdesát osm, samých mistrů.
#25:8 Losy pro strážní službu vrhli jak malému, tak velikému, mistru i učedníkovi.
#25:9 První z Asafova rodu byl vylosován Josef, druhý Gedaljáš; ten a jeho bratří a synové, celkem dvanáct mužů.
#25:10 Třetí Zákur, jeho synové a bratří, celkem dvanáct.
#25:11 Čtvrtý Jisrí, jeho synové a bratří, celkem dvanáct.
#25:12 Pátý Netanjáš, jeho synové a bratří, celkem dvanáct.
#25:13 Šestý Bukijáš, jeho synové a bratří, celkem dvanáct.
#25:14 Sedmý Jesaréla, jeho synové a bratří, celkem dvanáct.
#25:15 Osmý Ješajáš, jeho synové a bratří, celkem dvanáct.
#25:16 Devátý Matanjáš, jeho synové a bratří, celkem dvanáct.
#25:17 Desátý Šimeí, jeho synové a bratří, celkem dvanáct.
#25:18 Jedenáctý Azarel, jeho synové a bratří, celkem dvanáct.
#25:19 Dvanáctý Chašabjáš, jeho synové a bratří, celkem dvanáct.
#25:20 Podle třináctého losu Šúbael, jeho synové a bratří, celkem dvanáct.
#25:21 Podle čtrnáctého Matitjáš, jeho synové a bratří, celkem dvanáct.
#25:22 Podle patnáctého Jeremót, jeho synové a bratří, celkem dvanáct.
#25:23 Podle šestnáctého Chananjáš, jeho synové a bratří, celkem dvanáct.
#25:24 Podle sedmnáctého Jošbekáša, jeho synové a bratří, celkem dvanáct.
#25:25 Podle osmnáctého Chananí, jeho synové a bratří, celkem dvanáct.
#25:26 Podle devatenáctého Malótí, jeho synové a bratří, celkem dvanáct.
#25:27 Podle dvacátého Elijata, jeho synové a bratří, celkem dvanáct.
#25:28 Podle dvacátého prvého Hótir, jeho synové a bratří, celkem dvanáct.
#25:29 Podle dvacátého druhého Gidaltí, jeho synové a bratří, celkem dvanáct.
#25:30 Podle dvacátého třetího Machaziót, jeho synové a bratří, celkem dvanáct.
#25:31 Podle dvacátého čtvrtého Rómamtí-ezer, jeho synové a bratří, celkem dvanáct. 
#26:1 Oddíly vrátných; z Kórachovců: Mešelemjáš, syn Kóreho ze synů Asafových.
#26:2 Synové Mešelemjášovi: prvorozený Zekarjáš, druhý Jedíael, třetí Zebadjáš, čtvrtý Jatníel,
#26:3 pátý Élam, šestý Jóchanan, sedmý Eljóenaj.
#26:4 Synové Obéd-edómovi: prvorozený Šemajáš, druhý Józabad, třetí Jóach, čtvrtý Sákar, pátý Netaneel,
#26:5 šestý Amíel, sedmý Jisakar, osmý Peúletaj, jemuž Bůh požehnal.
#26:6 Jeho synu Šemajášovi se narodili synové, kteří se stali vladaři svého rodu, neboť to byli udatní bohatýři.
#26:7 Synové Šemajášoi: Otní a jeho bratři Refael, Obéd a Elzábad, ti stateční; Elíhú a Semakjáš.
#26:8 Ti všichni pocházeli ze synů Obéd-edómových; jejich synové a bratři byli stateční muži, schopní k službě, celkem dvaašedesát Obéd-edómovců.
#26:9 Též Mešelemjáš měl osmnáct synů a bratří, statečných mužů.
#26:10 Synové Chósy ze synů Merarího: Šimrí byl přední, ačkoli nebyl prvorozený, ale jeho otec ho ustanovil za předního,
#26:11 druhý byl Chilkijáš, třetí Tebaljáš, čtvrtý Zekarjáš; všech synů a bratří Chósových bylo třináct.
#26:12 To jsou oddíly vrátných, přední z mužů, kteří drželi stráže po boku svých bratří při službě v Hospodinově domě.
#26:13 Jak o malém, tak o velkém losovali podle jejich otcovského rodu a podle jednotlivých bran.
#26:14 Los východní brány padl Šelemjášovi; jeho syn Zekarjáš byl prozíravý rádce. Losovali a na něj padl los severní brány.
#26:15 Na Obéd-edóma pak jižní a na jeho syny los zásobárny.
#26:16 Na Šupíma a Chósa západní brána, která vede k silnici vzhůru; stráž stála vedle stráže.
#26:17 Na východní straně bylo šest lévijců, na severní straně denně čtyři, na jižní straně denně čtyři, při zásobárně po dvou.
#26:18 V sloupořadí na západní straně čtyři, na silnici u sloupořadí dva.
#26:19 To jsou oddíly vrátných z Kórachovců a Meraríovců.
#26:20 Další lévijci: Achijáš byl nad sklady Božího domu i nad sklady svatých věcí.
#26:21 Synové Laedánovi, Geršónovci, přední z rodu Laedánova: Laedán Geršónský měl Jechíelího.
#26:22 Synové Jechíelího Zétam a jeho bratr Jóel byli nad sklady Hospodinova domu.
#26:23 Amrámci, Jishárci, Chebrónci a Ozíelci:
#26:24 Šebúel, syn Geršóma, syna Mojžíšova, se stal představeným nad sklady.
#26:25 Jeho bratr Elíezer měl syny Rechabjáše, Ješajáše, Jórama, Zikrího a Šelomóta.
#26:26 Ten Šelomót a jeho bratři byli nad všemi sklady svatých věcí, které oddělil jako svaté král David a přední z rodů, velitelé nad tisíci a nad sty a jiní velitelé vojska.
#26:27 Z válečné kořisti to oddělili jako svaté ku podpoře Hospodinova domu.
#26:28 Všechno, co oddělil jako svaté vidoucí Samuel, Saul, syn Kíšův, Abnér, syn Nérův, Jóab, syn Serújin, cokoli kdo oddělil jako svaté, vkládal do rukou Šelomóta a jeho bratří.
#26:29 Z Jishárců Kenanjáš a jeho synové byli určeni pro práci mimo chrám, za správce a soudce nad Izraelem.
#26:30 Z Chebrónců Chašabjáš a jeho bratří, tisíc sedm set statečných, byli určeni k dozoru nad Izraelem od Zajordání na západ pro všechno dílo Hospodinovo a pro královskou službu.
#26:31 U Chebrónců byl Jerijáš přední z Chebrónců, jejich čeledí a rodů. Ve čtyřicátém roce Davidova kralování byli vyhledáváni a mezi nimi nalezeni stateční bohatýři v gileádském Jaezeru.
#26:32 Jeho statečních bratří bylo dva tisíce sedm set předních osob z rodů. Král David je ustanovil nad Rúbenovci, Gádovci a nad polovinou kmene Manases ve všech věcech Božích i královských. 
#27:1 Izraelci podle příslušného seznamu, přední z rodů, velitelé nad tisíci a nad sty a správci, kteří byli ve službách krále pro různé úkoly oddílů, jak měly nastupovat a končit službu, měsíc po měsíci, po všechny měsíce v roce. Každý oddíl měl dvacet čtyři tisíce mužů.
#27:2 Nad prvním oddílem pro první měsíc byl Jášobeám, syn Zabdíelův; jeho oddíl měl dvacet čtyři tisíce mužů.
#27:3 Ze synů Peresových byl předním ze všech velitelů vojska pro první měsíc.
#27:4 Nad oddílem pro druhý měsíc byl Dódaj Achóchijský se svým oddílem a vévoda Miklót; jeho oddíl měl dvacet čtyři tisíce mužů.
#27:5 Velitelem třetího voje pro třetí měsíc byl Benajáš, syn předního kněze Jójady; jeho oddíl měl dvacet čtyři tisíce mužů.
#27:6 Týž Benajáš byl jedním ze třiceti bohatýrů a byl nad těmi třiceti. V jeho oddílu byl jeho syn Amízábad.
#27:7 Čtvrtý pro čtvrtý měsíc byl Asáel, bratr Jóabův, a po něm jeho syn Zebadjáš; jeho oddíl měl dvacet čtyři tisíce mužů.
#27:8 Pátý pro pátý měsíc měl velitele Šamhúta Jizrašského; jeho oddíl měl dvacet čtyři tisíce mužů.
#27:9 Šestý pro šestý měsíc byl Íra, syn Íleše Tekójského; jeho oddíl měl dvacet čtyři tisíce mužů.
#27:10 Sedmý pro sedmý měsíc byl Cheles Pelónský z Efrajimovců; jeho oddíl měl dvacet čtyři tisíce mužů.
#27:11 Osmý pro osmý měsíc byl Sibekaj Chúšatský ze Zerachejců; jeho oddíl měl dvacet čtyři tisíce mužů.
#27:12 Devátý pro devátý měsíc byl Abíezer Anatótský z Benjamínců; jeho oddíl měl dvacet čtyři tisíce mužů.
#27:13 Desátý pro desátý měsíc byl Mahraj Netófský ze Zerachejců; jeho oddíl měl dvacet čtyři tisíce mužů.
#27:14 Jedenáctý pro jedenáctý měsíc byl Benajáš Pireatónský z Efrajimovců; jeho oddíl měl dvacet čtyři tisíce mužů.
#27:15 Dvanáctý pro dvanáctý měsíc byl Cheldaj Netófský z Otníela; jeho oddíl měl dvacet čtyři tisíce mužů.
#27:16 Izraelským kmenům veleli: Rúbenovi vévoda Elíezer, syn Zikríův, Šimeónovi Šefatjáš, syn Maakův,
#27:17 Lévimu Chašabjáš, syn Kemúelův, Áronovi Sádok,
#27:18 Judovi Elíhú z bratří Davidových, Isacharovi Omrí, syn Míkaelův,
#27:19 Zabulónovi Jišmajáš, syn Obadjášův, Neftalímu Jerímót, syn Azríelův,
#27:20 Efrajimovcům Hóšea, syn Azazjášův, polovině kmene Manases Jóel, syn Pedajášův,
#27:21 druhé polovině Manasesa v Gileádu Jidó, syn Zekarjášův, Benjamínovi Jaasíel, syn Abnérův,
#27:22 Danovi Azarel, syn Jeróchamův. To byli velitelé izraelských kmenů.
#27:23 David ani nepořizoval součet od dvacetiletých níže, neboť Hospodin řekl, že Izraele rozmnoží jako nebeské hvězdy.
#27:24 Jóab, syn Serújin, sice začal se sčítáním, ale nedokončil je. Kvůli tomu totiž postihlo Izraele Hospodinovo rozlícení. Proto se jejich součet ani do součtu v letopisech krále Davida nedostal.
#27:25 Nad královskými sklady byl Azmávet, syn Adíelův, nad sklady na polích, ve městech, vesnicích a věžích Jónatan, syn Uzijášův.
#27:26 Nad těmi, kdo pracovali na polích, nad rolníky byl Ezrí, syn Kelúbův,
#27:27 nad vinicemi Šimeí Rámatský a nad sklady vína na vinicích Zabdí Šifmejský,
#27:28 nad olivami a fíkovníky v nížině Baal-chanan Gederský a nad sklady oleje Jóaš,
#27:29 nad skotem, který se pásl v Šáronu, Šitraj Šáronský a nad skotem v dolinách Šáfat, syn Adlajův,
#27:30 nad velbloudy Obíl Izmaelský, nad oslicemi Jechdejáš Meronótský,
#27:31 nad ovcemi a kozami Jazíz Hagrejský. Ti všichni byli předními správci jmění krále Davida.
#27:32 Jónatan, Davidův strýc, rádce, mistr a písař spolu s Jechíelem, synem Chakmóního, byli s královými syny.
#27:33 Achítofel byl královým rádcem a Chúšaj Arkíjský královým přítelem;
#27:34 po Achítofelovi Jójada, syn Benajášův, a Ebjátar. Velitelem králova vojska byl Jóab. 
#28:1 David shromáždil do Jeruzaléma všechny izraelské předáky, předáky kmenů a předáky oddílů, kteří byli v králových službách, též velitele nad tisíci a nad sty i správce jmění a stád krále a jeho synů spolu s dvořany a bohatýry, každého statečného bohatýra.
#28:2 Král David povstal a řekl: „Slyšte mě, moji bratří a můj lide! Já sám jsem měl v úmyslu vybudovat dům odpočinutí pro schránu Hospodinovy smlouvy, pro podnož nohou našeho Boha. Už jsem začal se stavebními přípravami.
#28:3 Ale Bůh mi řekl: ‚Ty nemůžeš vybudovat dům pro mé jméno, neboť jsi vedl mnoho bojů a prolil jsi mnoho krve.‘
#28:4 Hospodin, Bůh Izraele, si vyvolil z celého domu mého otce mne, abych byl navěky králem nad Izraelem. Z Judy si totiž vyvolil vévodu a z domu Judova dům mého otce a mezi syny mého otce našel zalíbení ve mně. Tak mě ustanovil králem nad celým Izraelem.
#28:5 Ze všech mých synů, a Hospodin mi dal synů mnoho, si vyvolil mého syna Šalomouna, aby usedl na Hospodinův královský trůn nad Izraelem.
#28:6 Řekl mi: ‚Tvůj syn Šalomoun, ten vybuduje můj dům a má nádvoří, neboť jsem si ho vyvolil za syna. Já mu budu Otcem.
#28:7 Jeho království upevním navěky, jestliže bude rozhodně dodržovat mé příkazy a právní ustanovení tak jako dnes.‘
#28:8 A nyní před zraky celého Izraele, Hospodinova shromáždění, a před sluchem našeho Boha: Bedlivě se vždy dotazujte na všechny příkazy Hospodina, svého Boha, abyste si udrželi tu dobrou zemi a mohli ji předat svým synům do dědictví navěky.“
#28:9 „Ty pak, můj synu Šalomoune, poznávej Boha svého otce a služ mu celým srdcem a ochotnou myslí. Hospodin zkoumá srdce všech, postřehne každý výtvor mysli. Budeš-li se ho dotazovat, dá se ti najít, jestliže ho opustíš, odvrhne tě navždy.
#28:10 Nyní hleď, Hospodin tě vyvolil, abys mu vybudoval dům, svatyni. Buď rozhodný a jednej!“
#28:11 David předal svému synu Šalomounovi plán předsíně, jednotlivých částí domu, jeho pokladnic a přístřešků, jeho vestavěných pokojíků i domu pro příkrov schrány,
#28:12 též plán všeho, jak si to představoval, nádvoří Hospodinova domu a všech okolních komor, skladů při božím domě a skladů pro svaté věci,
#28:13 seznam tříd kněží a lévijců, rozpis veškerého služebného díla při Hospodinově domě, soupis všeho náčiní k službě při Hospodinově domě.
#28:14 Pro veškeré zlaté náčiní k různým službám předal zlato podle váhy, pro veškeré stříbrné náčiní, pro veškeré náčiní k různým službám stříbro podle váhy,
#28:15 pro zlaté svícny a zlaté kahánky na nich zlato podle váhy svícnu a kahánků, pro stříbrné svícny stříbro podle váhy svícnu a kahánků, na každý svícen zvlášť podle potřeby;
#28:16 dále zlato podle váhy pro stoly na rovnání předkladných chlebů, na každý stůl zvlášť, též stříbro na stříbrné stoly,
#28:17 na vidlice, kropenky i konve zlato čisté a zlato podle váhy na koflíky, na každý koflík zvlášť, též stříbro podle váhy na stříbrné koflíky, na každý koflík zvlášť,
#28:18 na kadidlový oltář přetavené zlato podle váhy; dále plán vozu a zlatých cherubů, rozprostírajících křídla a zastírajících schránu Hospodinovy smlouvy.
#28:19 „To vše bylo zapsáno rukou Hospodinovou, abych mohl prozíravě připravit plán veškerého díla.“
#28:20 David řekl svému synu Šalomounovi: „Buď rozhodný a udatný, jednej! Neboj se a neděs! Hospodin Bůh, můj Bůh, bude s tebou. Nenechá tě klesnout a neopustí tě, dokud veškeré dílo prací na Hospodinově domě nedokončíš.
#28:21 A tu jsou třídy kněží a lévijců pro veškerou práci při Božím domě. S tebou pak budou předáci a všechen lid při veškerém díle, aby se všechna práce konala dobrovolně a moudře a podle každého tvého slova.“ 
#29:1 Potom řekl král David celému shromáždění: „Mého syna Šalomouna si vyvolil Bůh jako jediného, ač je mladíček útlého věku a dílo je obrovské. Nejde totiž o hrad pro člověka, ale pro Boha Hospodina.
#29:2 Pro dům svého Boha jsem připravil, co jsem mohl: zlato pro zlaté náčiní, stříbro pro stříbrné, měď pro měděné, železo pro železné, dřevo pro dřevěné, kameny karneolové pro vsazování, kameny pestře zbarvené a různé drahokamy i množství mramoru.
#29:3 A jelikož mám v domě svého Boha zalíbení, přidávám pro dům svého Boha i ze svého vlastního jmění zlato a stříbro navíc k tomu všemu, co jsem připravil pro svatyni:
#29:4 tři tisíce talentů ofírského zlata, sedm tisíc talentů přetaveného stříbra k obložení stěn domu,
#29:5 pro různé věci ze zlata a pro různé věci ze stříbra, pro veškeré dílo vyráběné řemeslníky. Kdo by chtěl dnes něco ze svého dobrovolně obětovat pro Hospodina?“
#29:6 Předáci rodů izraelských kmenů, velitelé nad tisíci a nad sty i královští úředníci přinášeli dobrovolné dary.
#29:7 Na dílo Božího domu odevzdali pět tisíc talentů zlata a deset tisíc darejků, deset tisíc talentů stříbra a osmnáct tisíc talentů mědi a sto tisíc talentů železa.
#29:8 Kdo pak měl nějaké drahé kamení, odevzdal je na poklad Hospodinova domu do rukou Jechíela Geršónského.
#29:9 Lid se radoval z toho, co bylo dobrovolně darováno, že z celého srdce se odevzdávaly dobrovolné dary Hospodinu. Také král David se převelice radoval.
#29:10 I dobrořečil David Hospodinu před zraky celého shromáždění. Řekl: „Požehnán jsi, Hospodine, Bože Izraele, našeho otce, od věků až na věky.
#29:11 Hospodine, tvá je velikost a bohatýrská síla, skvělost, stálost a velebnost, všechno, co je na nebi a na zemi, je tvé. Hospodine, tvé je království, ty jsi vyvýšen nade vším jako hlava.
#29:12 Bohatství a sláva pocházejí od tebe, ty panuješ nade vším, máš v rukou moc a bohatýrskou sílu, vyvýšení a utvrzení všeho je v tvých rukou.
#29:13 Nyní, Bože náš, vzdáváme ti chválu a oslavujeme tvé skvělé jméno.
#29:14 Vždyť co jsem já a co je můj lid, že máme možnost takto přinášet dobrovolné dary? Od tebe pochází všechno. Dáváme ti, co jsme přijali z tvých rukou.
#29:15 My jsme před tebou jen hosté a příchozí jako všichni naši otcové. Naše dny na zemi jsou jako stín a naděje není.
#29:16 Hospodine, Bože náš, všechno toto množství, jež jsme připravili, abychom vybudovali dům tobě, tvému svatému jménu, pochází z tvé ruky. Tobě patří všechno.
#29:17 Poznal jsem, můj Bože, že ty zkoumáš srdce a že máš zalíbení v přímosti. Přinesl jsem všechny tyto dobrovolné dary z přímého srdce. A nyní vidím, jak radostně ti přináší dobrovolné dary tvůj lid, který je zde.
#29:18 Hospodine, Bože Abrahama, Izáka a Izraele, našich otců, zachovej navěky to, co si tvůj lid ve svém srdci předsevzal, a jejich srdce si připrav pro sebe.
#29:19 Dej, ať můj syn Šalomoun z celého srdce dbá na tvé příkazy, na tvá svědectví a na tvá nařízení, aby vykonal všechno, co je třeba k vybudování hradu, k němuž jsem vykonal přípravy.“
#29:20 Potom David vyzval celé shromáždění: „Dobrořečte Hospodinu, svému Bohu!“ A celé shromáždění dobrořečilo Hospodinu, Bohu svých otců, padli na kolena a klaněli se Hospodinu i králi.
#29:21 Druhého dne připravili Hospodinu oběti. Obětovali Hospodinu oběti zápalné: tisíc býků, tisíc beranů a tisíc beránků s příslušnými úlitbami, množství obětí za celého Izraele.
#29:22 Onoho dne jedli a pili před Hospodinem s velikou radostí. Potom vyhlásili podruhé Šalomouna, syna Davidova, králem a pomazali jej Hospodinu za vévodu a Sádoka za kněze.
#29:23 I dosedl Šalomoun na Hospodinův trůn jako král místo svého otce Davida a provázel ho zdar a celý Izrael ho poslouchal;
#29:24 též všichni předáci a bohatýři, také všichni synové Davida se králi Šalomounovi podrobili.
#29:25 Hospodin Šalomouna před zraky celého Izraele nesmírně vyvýšil. Dal mu takovou královskou velebnost, jakou před ním neměl žádný král nad Izraelem.
#29:26 David, syn Jišajův, kraloval nad celým Izraelem.
#29:27 Kraloval nad Izraelem čtyřicet let, sedm let kraloval v Chebrónu a třicet tři léta kraloval v Jeruzalémě.
#29:28 I zemřel v utěšeném stáří nasycen dny, bohatstvím a slávou. Po něm kraloval jeho syn Šalomoun.
#29:29 Příběhy krále Davida jsou od začátku až do konce zapsány v příbězích Samuela, vidoucího, v příbězích proroka Nátana a v příbězích Gáda, jenž měl dar vidění,
#29:30 spolu se vším, co se týká jeho království a bohatýrských činů i všeho, co se s ním a s Izraelem i se všemi královstvími okolních zemí v oněch časech událo.  

\book{II Chronicles}{2Chr}
#1:1 Šalomoun, syn Davidův, pevně vládl svému království a Hospodin, jeho Bůh, byl s ním a nesmírně jej vyvýšil.
#1:2 Šalomoun vydal rozkaz celému Izraeli, velitelům nad tisíci a nad sty, soudcům a všem předákům z celého Izraele, totiž představitelům rodů,
#1:3 a ubírali se, Šalomoun s celým shromážděním, na posvátné návrší v Gibeónu; tam byl Boží stan setkávání, který na poušti zhotovil Mojžíš, služebník Hospodinův.
#1:4 Boží schránu dal ovšem David přenést z Kirjat-jearímu na místo, které pro ni připravil v Jeruzalémě, kde pro ni postavil stan.
#1:5 Avšak bronzový oltář, který zhotovil Besaleel, syn Urího, syna Chúrova, umístil před Hospodinovým příbytkem a tam se ho Šalomoun se shromážděním dotazoval.
#1:6 Šalomoun tam před Hospodinem obětoval na bronzovém oltáři, který byl u stanu setkávání, obětoval na něm tisíc zápalných obětí.
#1:7 Té noci se Šalomounovi ukázal Bůh a řekl mu: „Žádej, co ti mám dát.“
#1:8 Šalomoun Bohu odpověděl: „Ty jsi prokazoval velké milosrdenství mému otci Davidovi a mne jsi po něm ustanovil za krále.
#1:9 Nyní, Hospodine Bože, nechť se prokáže spolehlivost tvého slova daného mému otci Davidovi. Ty jsi mě přece ustanovil za krále nad lidem tak početným, jako je prach země.
#1:10 Dej mi tedy moudrost a umění, abych dovedl před tímto lidem vycházet a vcházet. Vždyť kdo by mohl soudit tento tvůj lid, jenž je tak četný?“
#1:11 Bůh Šalomounovi řekl: „Protože máš na srdci toto a nežádal jsi bohatství, skvosty ani slávu, ani bezživotí těch, kdo tě nenávidí, ba nežádal jsi ani dlouhý věk, ale požádal jsi pro sebe o moudrost a umění, jak soudit můj lid, nad nímž jsem tě ustanovil za krále,
#1:12 budou ti dány moudrost a umění, ale dám ti i bohatství, skvosty a slávu, takže se ti nevyrovná žádný z králů, kteří byli před tebou, ani z těch, co přijdou po tobě.“
#1:13 Od stanu setkávání na posvátném návrší, které je v Gibeónu, přišel Šalomoun do Jeruzaléma a ujal se království nad Izraelem.
#1:14 Šalomoun nashromáždil vozy a jezdecké koně; měl tisíc čtyři sta vozů a dvanáct tisíc jezdeckých koní. Rozmístil je ve městech pro vozbu a u sebe v Jeruzalémě.
#1:15 Král měl v Jeruzalémě stříbra a zlata jako kamení a cedrů jako planých fíků, jakých roste mnoho v Přímořské nížině.
#1:16 Koně, které měl Šalomoun, přicházeli z Egypta. Karavany králových překupníků kupovaly stáda koní za pevnou cenu.
#1:17 Při návratu přiváželi z Egypta vůz za šest set šekelů stříbra a koně za sto padesát. Jejich prostřednictvím se tak vyváželi všem králům chetejským a aramejským.
#1:18 Šalomoun se rozhodl vybudovat dům Hospodinovu jménu a sobě dům královský. 
#2:1 Šalomoun stanovil počet sedmdesáti tisíc nosičů břemen, osmdesáti tisíc kameníků v horách a nad nimi tři tisíce šest set dozorců.
#2:2 Šalomoun poslal Chúramovi, králi týrskému vzkaz: „Jako jsi jednal s mým otcem Davidem a posílal mu cedry, aby si vybudoval pro sebe dům k bydlení, tak jednej i se mnou.
#2:3 Hle, buduji dům jménu Hospodina, svého Boha. Chci mu jej oddělit jako svatý, aby se tam před ním pálilo kadidlo z vonných látek, pravidelně se rovnaly chleby a ráno i večer, též ve dnech odpočinku, o novoluních a při slavnostech Hospodina, našeho Boha, aby se přinášely zápalné oběti za Izraele navěky.
#2:4 Dům, který buduji, bude veliký, neboť náš Bůh je větší než všichni bohové.
#2:5 Kdo však má tolik síly, aby jemu mohl vybudovat dům? Vždyť jej nemohou pojmout nebesa, ani nebesa nebes. A kdo jsem já, abych mu budoval dům? Leda abych před ním pálil kadidlo.
#2:6 Nyní mi pošli odborníka, který by uměl pracovat se zlatem, stříbrem, mědí a železem, s látkou nachovou, karmínovou a purpurově fialovou a uměl by vyřezávat řezby, aby pracoval s odborníky, kteří jsou u mne v Judsku a v Jeruzalémě, které zjednal můj otec David.
#2:7 Pošli mi též z Libanónu dřevo cedrové, cypřišové a algumímové. Vím totiž, že tvoji služebníci dovedou kácet libanónské stromy. Moji služebníci tam budou spolu s tvými služebníky
#2:8 a připraví mi dosti dřeva, neboť dům, který buduji, má být veliký a obdivuhodný.
#2:9 Hle, pro drvoštěpy kácející stromy, pro tvé služebníky, budu dodávat dvacet tisíc kórů drcené pšenice, dvacet tisíc kórů ječmene, dvacet tisíc batů vína a dvacet tisíc batů oleje.“
#2:10 Chúram, král týrský, poslal Šalomounovi písemnou odpověď: „Protože Hospodin miluje svůj lid, dal mu za krále tebe.“
#2:11 Chúram řekl dále: „Požehnán buď Hospodin, Bůh Izraele, on učinil nebesa i zemi, on dal králi Davidovi syna moudrého, prozíravého a důvtipného, který vybuduje dům Hospodinu a sobě dům královský.
#2:12 Nyní ti tedy posílám důvtipného odborníka Chúrama,
#2:13 syna jedné ženy z dcer danských, jehož otec byl Týřan; ten umí pracovat se zlatem a stříbrem, s mědí, železem, kamenem a dřevem, s látkami nachovými, purpurově fialovými, s bělostným plátnem a látkami karmínovými, dovede vyřezávat všelijaké řezby a vše dovedně vyřešit. Ať je mu dovoleno pracovat s tvými odborníky i s odborníky mého pána, tvého otce Davida.
#2:14 Nyní ať jen posílá můj pán svým služebníkům, jak řekl, pšenici a ječmen, olej a víno.
#2:15 My budeme na Libanónu kácet stromy podle veškeré tvé potřeby a budeme ti je dovážet po moři jako vory do Jafy; odtud si je pak dopravíš do Jeruzaléma.“
#2:16 Šalomoun po sčítání, které provedl jeho otec David, spočítal všechny muže, kteří pobývali v izraelské zemi jako hosté; bylo jich sto padesát tři tisíce šest set.
#2:17 Z nich udělal sedmdesát tisíc nosičů břemen, osmdesát tisíc kameníků v horách a tři tisíce šest set dozorců, aby přidržovali lid k práci. 
#3:1 Šalomoun začal budovat v Jeruzalémě Hospodinův dům na hoře Mórija, kde se Hospodin ukázal jeho otci Davidovi, na místě, které připravil David, na humně Ornána Jebúsejského.
#3:2 S budováním začal druhého dne druhého měsíce ve čtvrtém roce svého kralování.
#3:3 Šalomounem byly stanoveny pro stavbu Božího domu tyto rozměry základů: délka v loktech podle staré míry byla šedesát loket, šířka dvacet loket.
#3:4 Předsíň, která byla vpředu po celé šíři domu, byla dlouhá dvacet loket a vysoká sto dvacet loket; uvnitř ji obložili čistým zlatem.
#3:5 Velkou síň obložil cypřišovým dřevem a to pokryl výborným zlatem, na němž dal zhotovit palmy a řetízkové ozdoby.
#3:6 Pro zvýšení lesku obložil síň drahokamy; zlato bylo zlato parvajimské.
#3:7 Zlatem pokryl síň, trámy, prahy, stěny a dveře a do stěn vyryl cheruby.
#3:8 Pak udělal síň velesvatyně; vpředu byla po celé šíři domu, dlouhá byla dvacet loket a široká též dvacet loket. Pokryl ji šesti sty talenty výborného zlata.
#3:9 Váha hřebů činila padesát šekelů zlata; též vrchní pokojíky pokryl zlatem.
#3:10 Pro síň velesvatyně dal umně zhotovit dva cheruby a dal je pozlatit.
#3:11 Délka křídel těch cherubů byla celkem dvacet loket; jedno křídlo dlouhé pět loket se dotýkalo stěny síně a druhé křídlo dlouhé pět loket se dotýkalo křídla druhého cheruba.
#3:12 Křídlo jednoho cheruba dlouhé pět loket se dotýkalo stěny síně, kdežto druhé křídlo dlouhé pět loket bylo spojeno s křídlem cheruba druhého.
#3:13 Roztažená křídla těchto cherubů byla dvacet loket dlouhá; stáli na nohou tváří k síni.
#3:14 Zhotovil také oponu z látky pupurově fialové, nachové a karmínové a z bělostného plátna a na ni dal udělat cheruby.
#3:15 Před domem udělal dva sloupy vysoké třicet pět loket, hlavice na vrcholu byla vysoká pět loket.
#3:16 Zhotovil i řetízkové ozdoby jako pro svatostánek a dal je na hlavice sloupů; udělal též sto granátových jablek a zavěsil je na řetízkové ozdoby.
#3:17 Sloupy postavil před chrám, jeden z pravé a druhý z levé strany; pravý pojmenoval Jakín a levý pojmenoval Bóaz. 
#4:1 Dále zhotovil bronzový oltář dvacet loket dlouhý, dvacet loket široký a deset loket vysoký.
#4:2 Odlil též moře o průměru deseti loket, okrouhlé, pět loket vysoké; dalo se obepnout měřící šňůrou dlouhou třicet loket.
#4:3 Pod okrajem je dokola vroubily ozdoby v podobě býků, deset na jeden loket; lemovaly moře dokola. Dvě řady býků byly odlity spolu s mořem.
#4:4 Moře spočívalo na dvanácti býcích; tři byli obráceni na sever, tři na západ, tři na jih a tři na východ. Na nich bylo moře položeno a zadky všech byly obráceny dovnitř.
#4:5 Bylo na dlaň silné a jeho okraj byl udělán jako okraj poháru nebo rozkvetlé lilie. Pojalo tři tisíce batů.
#4:6 Udělal též deset nádrží, pět jich dal napravo, pět nalevo; byly určeny k omývání. Oplachovali v nich to, co patřilo k zápalným obětem. Moře bylo určeno k umývání kněží.
#4:7 Udělal též deset zlatých svícnů podle předpisu a dal je do chrámu, pět napravo a pět nalevo.
#4:8 Udělal též deset stolů a postavil je do chrámu, pět napravo a pět nalevo; udělal též sto zlatých kropenek.
#4:9 Udělal též nádvoří pro kněze a veliký dvůr, též dveře do dvora; jejich křídla obložil bronzem.
#4:10 Moře postavil po pravé straně domu k jihovýchodu.
#4:11 Chúram udělal též hrnce, lopaty a kropenky. Tak dokončil Chúram práci, kterou dělal králi Šalomounovi pro Boží dům:
#4:12 dva sloupy a kulovité hlavice na vrcholu obou sloupů, dvoje mřížování, aby obě kulovité hlavice na vrcholu sloupů přikrývalo;
#4:13 čtyři sta granátových jablek k obojímu mřížování, po dvou řadách granátových jablek na jedno mřížování, aby obě kulovité hlavice na sloupech přikrývalo;
#4:14 dále udělal stojany a na stojany udělal nádrže,
#4:15 jedno moře a pod ně dvanáct býků jako podstavce,
#4:16 i hrnce, lopaty, vidlice a všechny příslušné předměty udělal Chúram králi Šalomounovi pro Hospodinův dům; byly z leštěného bronzu.
#4:17 Král je dal odlévat v jordánském okrsku mezi Sukótem a Seredou do forem v zemi.
#4:18 Šalomoun udělal všech těchto předmětů převeliké množství; na váhu mědi se nehledělo.
#4:19 Šalomoun také udělal všechny předměty, které byly pro Boží dům: zlatý oltář a stoly pro předkladné chleby,
#4:20 svícny a jejich kahánky potažené lístkovým zlatem, aby hořely před svatostánkem podle předpisu,
#4:21 dále květ, kahánky a kleště na knoty ze zlata; bylo to nejlepší zlato;
#4:22 též nože, kropenky, číše a pánve potažené lístkovým zlatem. Udělal i vchod do domu; vnitřní dveře do velesvatyně a dveře hlavní lodi chrámové byly ze zlata. 
#5:1 Tak byla ukončena všechna práce, kterou vykonal Šalomoun pro Hospodinův dům. Šalomoun tam pak vnesl svaté dary svého otce Davida; stříbrné a zlaté i ostatní předměty uložil mezi poklady Božího domu.
#5:2 Tehdy Šalomoun svolal do Jeruzaléma shromáždění izraelských starších, všechny představitele dvanácti pokolení a předáky izraelských rodů, aby vynesli schránu Hospodinovy smlouvy z Města Davidova, totiž ze Sijónu.
#5:3 Ke králi se shromáždili všichni izraelští muži ve svátek, to je v sedmém měsíci.
#5:4 Když všichni izraelští starší přišli, zvedli lévijci schránu
#5:5 a vynesli nahoru schránu a stan setkávání i všechny svaté předměty, které byly ve stanu; to vše vynesli lévijští kněží.
#5:6 Král Šalomoun a celá izraelská pospolitost, která se kolem něho před schránou sešla, obětovali tolik bravu a skotu, že nemohl být pro množství spočítán ani sečten.
#5:7 Kněží vnesli schránu Hospodinovy smlouvy na její místo do svatostánku domu, do velesvatyně, pod křídla cherubů.
#5:8 Cherubové rozprostírali křídla nad místem, kde byla schrána, takže cherubové zakrývali shora schránu i její tyče.
#5:9 Tyče však byly tak dlouhé, že jejich konce bylo vidět u schrány před svatostánkem, avšak zvenčí viditelné nebyly; je to tam až dodnes.
#5:10 Ve schráně nebylo nic než dvě desky, které tam dal Mojžíš na Chorébu, kde Hospodin uzavřel smlouvu s Izraelci, když vyšli z Egypta.
#5:11 Když kněží vycházeli ze svatyně, všichni přítomní kněží se posvětili bez ohledu na svou třídu.
#5:12 Všichni lévijší zpěváci, patřící k Asafovi, Hémanovi a Jedútúnovi, se syny a bratry, oblečeni v roucho z bělostného plátna, stáli s cymbály, harfami a citarami na východ od oltáře a s nimi bylo sto dvacet kněží troubících na pozouny.
#5:13 Trubači a zpěváci hráli a zpívali zároveň, aby Hospodinova chvála a oslava zněla jednohlasně; když se začalo hrát na pozouny a cymbály a jiné hudební nástroje, chválili Hospodina, protože je dobrý a že jeho milosrdenství je věčné. I naplnil oblak ten dům, dům Hospodinův,
#5:14 takže kněží kvůli tomu oblaku nemohli konat službu, neboť Boží dům naplnila Hospodinova sláva. 
#6:1 Tehdy Šalomoun řekl: „Hospodin praví, že bude přebývat v mrákotě.
#6:2 Vybudoval jsem ti sídlo, kde budeš přebývat, vznešené obydlí, po všechny věky.“
#6:3 Pak se král obrátil a žehnal celému shromáždění Izraele; celé shromáždění Izraele přitom stálo.
#6:4 Řekl: „Požehnán buď Hospodin, Bůh Izraele, který vlastními ústy mluvil k mému otci Davidovi a vlastní rukou naplnil, co řekl:
#6:5 ‚Ode dne, kdy jsem svůj lid vyvedl z egyptské země, nevyvolil jsem v žádném z izraelských kmenů město k vybudování domu, aby tam dlelo mé jméno, ani jsem nevyvolil nikoho, aby byl vévodou nad Izraelem, mým lidem.
#6:6 Ale vyvolil jsem Jeruzalém, aby tam dlelo mé jméno, a vyvolil jsem Davida, aby byl nad Izraelem, mým lidem.‘
#6:7 Můj otec David měl v úmyslu vybudovat dům jménu Hospodina, Boha Izraele.
#6:8 Hospodin však mému otci Davidovi řekl: ‚Máš sice dobrý úmysl vybudovat mému jménu dům,
#6:9 avšak ten dům nezbuduješ ty, nýbrž tvůj syn, který vzejde z tvých beder; ten vybuduje dům mému jménu.‘
#6:10 Hospodin splnil své slovo, které vyřkl. Nastoupil jsem po svém otci Davidovi, dosedl podle Hospodinova slova na izraelský trůn a vybudoval jsem dům jménu Hospodina, Boha Izraele.
#6:11 Uložil jsem tam schránu, v níž je Hospodinova smlouva, kterou uzavřel se syny Izraele.“
#6:12 Pak se Šalomoun v přítomnosti celého shromáždění Izraele postavil před Hospodinův oltář a rozprostřel dlaně.
#6:13 Šalomoun totiž udělal bronzový stupeň a dal jej doprostřed nádvoří; byl pět loket dlouhý, pět loket široký a tři lokte vysoký. Postavil se na něj před celým shromážděním Izraele, poklekl na kolena, dlaně rozprostřel k nebi
#6:14 a řekl: „Hospodine, Bože Izraele, není Boha tobě podobného na nebi ani na zemi. Ty zachováváš smlouvu a milosrdenství svým služebníkům, kteří chodí před tebou celým srdcem.
#6:15 Ty jsi zachoval svému služebníku, mému otci Davidovi, co jsi mu přislíbil. Vlastními ústy jsi přislíbil a vlastní rukou jsi to naplnil, jak je dnes zřejmé.
#6:16 Nyní, Hospodine, Bože Izraele, zachovej svému služebníku, mému otci Davidovi, to, co jsi mu přislíbil slovy: ‚Nebude přede mnou vyhlazen následník z tvého rodu, jenž bude sedět na izraelském trůnu, budou-li ovšem tvoji synové dbát na svou cestu a řídit se mým zákonem, jako ses jím řídil přede mnou ty.‘
#6:17 Nyní tedy, Hospodine, Bože Izraele, nechť se prokáže spolehlivost tvého slova, které jsi promluvil ke svému služebníku Davidovi.
#6:18 Ale může Bůh opravdu sídlit s člověkem na zemi, když nebesa, ba ani nebesa nebes tě nemohou pojmout, natož tento dům, který jsem vybudoval?
#6:19 Hospodine, můj Bože, skloň se k modlitbě svého služebníka a k jeho prosbě o smilování a vyslyš lkání a modlitbu, kterou se tvůj služebník před tebou modlí.
#6:20 Ať jsou tvé oči upřeny na tento dům ve dne i v noci, na místo, o kterém jsi řekl, že tam dáš spočinout svému jménu. Vyslýchej modlitbu, kterou se tvůj služebník bude modlit obrácen k tomuto místu.
#6:21 Vyslýchej prosby svého služebníka i Izraele, svého lidu, které se budou modlit obráceni k tomuto místu, vyslýchej z místa svého přebývání, z nebes, vyslýchej a odpouštěj.
#6:22 Jestliže se někdo prohřeší proti svému bližnímu a ten by mu uložil, aby se zaklínal přísahou, a on by tu přísahu složil před tvým oltářem v tomto domě,
#6:23 ty sám vyslyš z nebes, zasáhni a rozsuď své služebníky; odplať svévolníkovi, uval na jeho hlavu, čeho se dopustil, a prohlaš spravedlivého za spravedlivého a odměň jej podle jeho spravedlnosti.
#6:24 Bude-li poražen Izrael, tvůj lid, od nepřítele pro hřích proti tobě, ale navrátí se, vzdají chválu tvému jménu a budou se před tebou modlit a prosit o smilování v tomto domě,
#6:25 vyslyš z nebes a odpusť Izraeli, svému lidu, hřích a uveď je zpět do země, kterou jsi dal jim a jejich otcům.
#6:26 Uzavřou-li se nebesa a nebude déšť pro hřích proti tobě, a budou-li se modlit obráceni k tomuto místu a vzdávat chválu tvému jménu a odvrátí se od svých hříchů, protože jsi je pokořil,
#6:27 vyslyš z nebes a odpusť svým služebníkům a Izraeli, svému lidu, hřích, vždyť je vyučuješ dobré cestě, po níž by měli chodit, a dej déšť své zemi, kterou jsi dal svému lidu do dědictví.
#6:28 Bude-li v zemi hlad, bude-li mor, obilná rez či sněť, kobylky nebo jiná havěť, budou-li ho v zemi jeho bran sužovat nepřátelé či jakákoli rána a jakákoli nemoc,
#6:29 vyslyš každou modlitbu, každou prosbu, kterou bude mít kterýkoli člověk ze všeho tvého izraelského lidu, každý, kdo pozná svou ránu a svou bolest a rozprostře své dlaně obrácen k tomuto domu.
#6:30 Vyslyš z nebes, ze sídla, kde přebýváš, a odpusť, odplať každému podle všech jeho cest, neboť znáš jeho srdce - vždyť ty sám jediný znáš přece srdce lidských synů -,
#6:31 aby se tě báli, aby chodili po tvých cestách po všechny dny, kdy budou žít na půdě, kterou jsi dal našim otcům.
#6:32 Také přijde-li cizinec, který není z Izraele, tvého lidu, ze vzdálené země kvůli tvému velikému jménu, tvé mocné ruce a tvé vztažené paži, přijdou-li a budou se modlit obrácení k tomuto domu,
#6:33 vyslyš z nebes, ze sídla, kde přebýváš, a učiň vše, oč k tobě ten cizinec bude volat, aby poznaly tvé jméno všechny národy země a bály se tě jako Izrael, tvůj lid, aby poznaly, že se tento dům, který jsem vybudoval, nazývá tvým jménem.
#6:34 Vytáhne-li tvůj lid do boje proti nepřátelům po cestě, kterou je pošleš, a budou-li se modlit k tobě směrem k tomuto městu, které jsi vyvolil a k domu, který jsem vybudoval tvému jménu,
#6:35 vyslyš z nebes jejich modlitbu a prosbu a zjednej jim právo.
#6:36 Zhřeší-li proti tobě, neboť není člověka, který by nehřešil, a ty se na ně rozhněváš a vydáš je nepříteli,aby je zajali a jaté vedli do země vzdálené nebo blízké,
#6:37 a oni si to v zemi, do níž byli jako zajatci odvedeni, vezmou k srdci, obrátí se a budou tě v zemi svého zajetí prosit o smilování: ‚Zhřešili jsme, provinili jsme se, svévolně si vedli‘,
#6:38 navrátí-li se tedy k tobě celým srdcem a celou duší v zemi svého zajetí, v zemi těch, kdo je odvedli do zajetí, a budou se modlit směrem ke své zemi, kterou jsi dal jejich otcům, k městu, které jsi vyvolil, a k domu, který jsem vybudoval tvému jménu,
#6:39 vyslyš z nebes, ze sídla, kde přebýváš, jejich modlitbu a prosby a zjednej jim právo; odpusť svému lidu, čím proti tobě zhřešili.
#6:40 Nyní, můj Bože, nechť jsou prosím tvé oči otevřené a uši ochotné slyšet modlitbu na tomto místě.
#6:41 Nyní tedy povstaň, Hospodine Bože, k místu svého odpočinku, ty sám i schrána tvé moci. Tvoji kněží ať obléknou spásu, Hospodine Bože, ať se tvoji věrní radují z dobrých věcí.
#6:42 Hospodine Bože, neodmítej pomazaného svého, pamatuj na milosrdenství prokazované svému služebníku Davidovi.“ 
#7:1 Když Šalomoun dokončil svou modlitbu, sestoupil z nebe oheň a pohltil zápalnou oběť i díly obětního hodu a dům naplnila Hospodinova sláva.
#7:2 Kněží nemohli do Hospodinova domu vstoupit, neboť Hospodinova sláva naplnila Hospodinův dům.
#7:3 A všichni Izraelci viděli sestupující oheň i Hospodinovu slávu nad domem, klesli na dlažbu tváří k zemi, klaněli se a vzdávali chválu Hospodinu, protože je dobrý a že jeho milosrdenství je věčné.
#7:4 A král i všechen lid slavili před Hospodinem obětní hod.
#7:5 Král Šalomoun obětoval k obětnímu hodu dvacet dva tisíce kusů skotu a sto dvacet tisíc kusů bravu. Tak zasvětil král s veškerým lidem Boží dům.
#7:6 Kněží stáli na svých místech, též lévijci s hudebními nástroji k oslavě Hospodina; ty dal zhotovit král David, aby jimi vzdávali Hospodinu chválu, protože jeho milosrdenství je věčné. Chválili Hospodina, jak je pověřil David. Kněží naproti nim troubili a celý Izrael stál.
#7:7 Šalomoun posvětil střed nádvoří, které je před Hospodinovým domem, neboť tam přinesl zápalné oběti a tučné díly pokojných obětí, protože bronzový oltář, který dal Šalomoun zhotovit, nemohl pojmout zápalnou a přídavnou oběť a tučné díly.
#7:8 V onen čas slavil Šalomoun a s ním celý Izrael slavnost po sedm dní, převeliké shromáždění od cesty do Chamátu až k Egyptskému potoku.
#7:9 Osmého dne konali slavnostní shromáždění; zasvěcení oltáře slavili sedm dní, též slavnost trvala sedm dní.
#7:10 Dvacátého třetího dne sedmého měsíce lid propustil; šli do svých stanů radostně a s dobrou myslí pro to, co dobrého učinil Hospodin Davidovi a Šalomounovi i Izraeli, svému lidu.
#7:11 Tak dokončil Šalomoun zdárně dům Hospodinův i dům královský a všechno, co si předsevzal udělat v domě Hospodinově i ve svém domě.
#7:12 Tu se v noci ukázal Šalomounovi Hospodin a řekl mu: „Vyslyšel jsem tvou modlitbu a vyvolil jsem si toto místo za dům pro oběti.
#7:13 Uzavřu-li nebesa, takže nebude deště, přikážu-li kobylkám, aby hubily zemi, pošlu-li na svůj lid mor,
#7:14 a můj lid, který se nazývá mým jménem, se pokoří a bude se modlit a vyhledávat mě a odvrátí se od svých zlých cest, tehdy je vyslyším z nebes, odpustím jim jejich hřích a uzdravím jejich zemi.
#7:15 Mé oči budou otevřené a mé uši ochotné slyšel modlitbu na tomto místě.
#7:16 Nyní jsem tento dům vyvolil a oddělil jako svatý, aby tam navěky dlelo mé jméno; mé oči a mé srdce tam budou po všechny dny.
#7:17 A co se tebe týče, budeš-li chodit přede mnou, jako chodil tvůj otec David, jednat podle všeho, co jsem ti přikázal, a dodržovat má nařízení a práva,
#7:18 upevním trůn tvého kralování podle smlouvy uzavřené s tvým otcem Davidem, že z tvého rodu nebude vyhlazen vládce Izraele.
#7:19 Jestliže se však odvrátíte a opustíte má nařízení a má přikázání, která jsem vám vydal, a půjdete sloužit jiným bohům a klanět se jim,
#7:20 vytrhnu vás ze své země, kterou jsem vám dal, a zavrhnu od své tváře tento dům, který jsem oddělil jako svatý pro své jméno, a vydám jej za pořekadlo a předmět výsměchu mezi všemi národy.
#7:21 Nad tímto domem, který byl nejvyšší ze všech, ustrne každý kolemjdoucí a řekne: ‚Proč Hospodin takto naložil s touto zemí a s tímto domem?‘
#7:22 A bude se odpovídat: ‚Protože opustili Hospodina, Boha svých otců, který je vyvedl z egyptské země, a chytili se jiných bohů, klaněli se jim a sloužili jim. Proto na ně uvedl všechno toto zlo.‘“ 
#8:1 Po uplynutí dvaceti let, během nichž Šalomoun stavěl dům Hospodinův i dům svůj,
#8:2 vystavěl i města, která mu dal Chúram, a usadil tam Izraelce.
#8:3 Pak táhl Šalomoun do Chamát-sóby a zmocnil se jí.
#8:4 Vybudoval též Tadmór v poušti a všechna města skladů v Chamátu.
#8:5 Pak vybudoval Bét-chorón Horní a Bét-chorón Dolní, opevněná města s hradbami, vraty a závorami,
#8:6 i Baalat a všechna města pro sklady. Ta patřila Šalomounovi, též všechna města pro vozbu a města pro koně a vše, co Šalomoun s tak velkým zaujetím budoval v Jeruzalémě i na Libanónu a v celé zemi, v níž vládl.
#8:7 Všechen lid, který zbyl z Chetejců, Emorejců, Perizejců, Chivejců a Jebúsejců, kteří nebyli z Izraele,
#8:8 totiž z jejich synů ty, kteří po nich v zemi zbyli, které Izraelci zcela nevyhubili, podrobil Šalomoun nuceným pracím, a tak je tomu až dodnes.
#8:9 Z Izraelců však Šalomoun neudělal otroky na svém díle; ti byli bojovníky a veliteli jeho osádek, veliteli jeho vozby a jízdy.
#8:10 Správců představených nad pracemi pro krále Šalomouna bylo dvě stě padesát, ti panovali nad lidem.
#8:11 Faraónovu dceru vyvedl Šalomoun z Města Davidova do domu, který jí vystavěl. Řekl si totiž: „Moje žena nemůže bydlet v domě Davida, krále izraelského, neboť místa, kam vstoupila Hospodinova schrána, jsou svatá.“
#8:12 Tehdy začal Šalomoun obětovat Hospodinu zápalné oběti na Hospodinově oltáři, který vybudoval před chrámovou předsíní,
#8:13 podle každodenního pořádku, podle Mojžíšova příkazu pro přinášení obětí ve dnech odpočinku, o novoluních a při slavnostech, třikrát za rok, ve svátek nekvašených chlebů, ve svátek týdnů a ve svátek stánků.
#8:14 Podle rozhodnutí svého otce Davida postavil kněze k službě podle jejich tříd a lévijce k jejich povinnostem, aby chválili Boha a konali před kněžími službu podle každodenního pořádku, též vrátné roztřídil do oddílů k jednotlivým branám ve smyslu příkazu Davida, muže Božího.
#8:15 Co se týká kněží a lévijců, neuchýlili se od králova příkazu v žádné věci, ani ve správě pokladů.
#8:16 Tak bylo dohotoveno celé Šalomounovo dílo, ode dne, kdy byly položeny základy domu Hospodinova, až do jeho dokončení, kdy byl Hospodinův dům dostavěn.
#8:17 Tehdy táhl Šalomoun do Esjón-geberu a k Elótu při mořském pobřeží v edómské zemi.
#8:18 Chúram mu poslal po svých služebnících lodě a služebníky znalé moře. Dopluli se Šalomounovými služebníky až do Ofíru a přivezli odtamtud králi Šalomounovi čtyři sta padesát talentů zlata. 
#9:1 I královna ze Sáby uslyšela zprávu o Šalomounovi a přijela do Jeruzaléma vyzkoušet ho hádankami. Přijela s velmi okázalým doprovodem, s velbloudy nesoucími balzámy, velké množství zlata a drahokamy. Přišla k Šalomounovi a mluvila s ním o všem, co měla na srdci.
#9:2 Šalomoun jí zodpověděl všechny její otázky; ani jedna otázka nebyla pro Šalomouna tak tajemná, že by ji nezodpověděl.
#9:3 Když královna ze Sáby viděla Šalomounovu moudrost a dům, který vybudoval,
#9:4 i jídlo na jeho stole, zasedání jeho hodnostářů, pohotovost jeho služebnictva a jejich oděvy, jeho číšníky a jejich oděvy, i jeho vyvýšenou chodbu, kudy vcházel do Hospodinova domu, zůstala bez dechu.
#9:5 Řekla králi: „Co jsem slyšela ve své zemi o tvém podnikání i o tvé moudrosti, je pravda.
#9:6 Nevěřila jsem těm slovům, dokud jsem nepřišla a nespatřila to na vlastní oči. A to mi nebyla sdělena ani polovina z množství tvé moudrosti. Překonal jsi pověst, kterou jsem slyšela.
#9:7 Blaze tvým mužům, blaze těmto tvým služebníkům, kteří jsou ustavičně v tvých službách a naslouchají tvé moudrosti.
#9:8 Požehnán buď Hospodin, tvůj Bůh, který si tě oblíbil a dosadil tě na svůj trůn, abys byl králem před Hospodinem, svým Bohem. Protože si tvůj Bůh zamiloval Izraele, aby trval navěky, ustanovil tě nad ním králem, abys zjednával právo a spravedlnost.“
#9:9 Pak dala králi sto dvacet talentů zlata, velmi mnoho balzámů a drahokamy. Už nikdy tu nebyl takový balzám, jaký dala královna ze Sáby králi Šalomounovi.
#9:10 Také služebníci Chúramovi a služebníci Šalomounovi, kteří dopravovali zlato z Ofíru, přivezli algumímové dřevo a drahokamy.
#9:11 Z algumímového dřeva dal král udělat ochozy pro dům Hospodinův i pro dům královský, i citary a harfy pro zpěváky. Nic takového nikdo v zemi judské předtím nespatřil.
#9:12 Král Šalomoun splnil královně ze Sáby každé přání, jež vyslovila; dal jí víc, než co ona přivezla králi. Pak se odebrala i se svými služebníky do své země.
#9:13 Váha zlata, které bylo přiváženo Šalomounovi za jeden rok, činila šest set šedesát šest talentů
#9:14 mimo zlato, jež přinášeli kupci a obchodníci; i všichni arabští králové a místodržitelé země přinášeli Šalomounovi zlato a stříbro.
#9:15 Král Šalomoun dal udělat dvě stě pavéz z tepaného zlata; na jednu pavézu vynaložil šest set šekelů tepaného zlata.
#9:16 Dále dal udělat tři sta štítů z tepaného zlata; na jeden štít vynaložil tři sta šekelů zlata. Král je pak dal uložit do domu Libanónského lesa.
#9:17 Král dal také udělat veliký trůn ze slonoviny a obložil jej čistým zlatem.
#9:18 Trůn měl šest stupňů a podnoží ze zlata, spojené s trůnem. Po obou stranách sedadla byla u trůnu opěradla a vedle opěradel stáli dva lvi.
#9:19 Na šesti stupních tam stálo z obou stran dvanáct lvů. Nic takového nebylo zhotoveno v žádném království.
#9:20 Všechny nádoby, z kterých král Šalomoun pil, byly zlaté a všechny předměty domu Libanónského lesa byly potaženy lístkovým zlatem; nic nebylo ze stříbra; to v Šalomounově době nemělo cenu.
#9:21 Král měl lodě, plující do zámoří s Chúramovými služebníky. Zámořské lodě přijížděly jednou za tři roky a přivážely zlato a stříbro, slonovinu, opice a pávy.
#9:22 Král Šalomoun tak převýšil všechny krále země bohatstvím i moudrostí.
#9:23 Všichni králové země vyhledávali Šalomouna, aby slyšeli jeho moudrost, kterou mu Bůh vložil do srdce.
#9:24 Každý přinášel svůj dar: předměty stříbrné a zlaté, pláště, zbroj, balzámy, koně, mezky, a to rok co rok.
#9:25 Šalomoun měl také ustájeno čtyři tisíce koní k vozům a dvanáct tisíc jezdeckých koní. Rozmístil je ve městech pro vozbu a u sebe v Jeruzalémě.
#9:26 Tak se stal vladařem nad všemi králi od Řeky až k zemi pelištejské a až k hranicím Egypta.
#9:27 Král měl v Jeruzalémě stříbra jako kamení a cedrů jako planých fíků, jakých roste mnoho v Přímořské nížině.
#9:28 Šalomounovi přiváděli koně z Egypta i z ostatních zemí.
#9:29 O ostatních příbězích Šalomounových, prvních i posledních, se píše, jak známo, v příbězích proroka Nátana, v proroctví Achijáše Šíloského a v prorockých viděních vidoucího Jeeda proti Jarobeámovi, synu Nebatovu.
#9:30 Šalomoun kraloval v Jeruzalémě nad celým Izraelem čtyřicet let.
#9:31 I ulehl Šalomoun ke svým otcům a pohřbili ho v městě jeho otce Davida. Po něm se stal králem jeho syn Rechabeám. 
#10:1 Rechabeám se odebral do Šekemu. Do Šekemu totiž přišel celý Izrael, aby ho ustanovil králem.
#10:2 Jarobeám, syn Nebatův, se o tom doslechl v Egyptě, kam uprchl před králem Šalomounem, a vrátil se z Egypta.
#10:3 Poslali pro něj a povolali ho zpět. Jarobeám přišel s celým Izraelem a promluvili k Rechabeámovi:
#10:4 „Tvůj otec nás sevřel tvrdým jhem. Ulehči nám nyní tvrdou službu svého otce a těžké jho, které na nás vložil, a my ti budeme sloužit.“
#10:5 Rechabeám jim řekl: „Po třech dnech se opět ke mně vraťte.“ A lid se rozešel.
#10:6 Král Rechabeám se radil se starci, kteří byli v službách jeho otce Šalomouna, dokud ještě žil. Ptal se: „Jak vy radíte? Co mám tomuto lidu odpovědět?“
#10:7 Promluvili k němu takto: „Jestliže budeš k tomuto lidu dobrý a přívětivý, jestliže jim dáš laskavou odpověď, budou po všechny dny tvými služebníky.“
#10:8 On však nedbal rady starců, kterou mu dávali, a radil se s mladíky, kteří s ním vyrostli a teď byli v jeho službách.
#10:9 Těch se zeptal: „Co radíte vy? Jak máme odpovědět tomuto lidu, který mi řekl: ‚Ulehči nám jho, které na nás vložil tvůj otec‘?“
#10:10 Mladíci, kteří s ním vyrostli, k němu promluvili takto: „Toto řekni lidu, který k tobě promluvil slovy: ‚Tvůj otec nás obtížil jhem, ty však nám je ulehči.‘ Promluv k nim takto: ‚Můj malík je tlustší než bedra mého otce.
#10:11 Tak tedy můj otec na vás vložil těžké jho? Já k vašemu jhu ještě přidám. Můj otec vás trestal biči, já však vás budu trestat důtkami.‘“
#10:12 Třetího dne přišel Jarobeám a všechen lid k Rechabeámovi, jak král nařídil: „Třetího dne se ke mně navraťte.“
#10:13 Král lidu odpověděl tvrdě, král Rechabeám nedbal rady starců.
#10:14 Promluvil k nim podle rady mladíků: „Obtížím vás jhem a ještě k němu přidám. Můj otec vás trestal biči, já však vás budu trestat důtkami!“
#10:15 Král lid nevyslyšel, neboť to bylo řízení Boží. Hospodin tak splnil slovo, které ohlásil Jarobeámovi, synu Nebatovu, skrze Achijáše Šíloského.
#10:16 Když celý Izrael uviděl, že je král nevyslyšel, dal lid králi tuto odpověď: „Jaký podíl máme v Davidovi? Nemáme dědictví v synu Jišajovu. Každý ke svým stanům, Izraeli! Nyní pohleď na svůj dům, Davide!“ A celý Izrael se rozešel ke svým stanům.
#10:17 Proto Rechabeám kraloval pouze na Izraelci usedlými v judských městech.
#10:18 Král Rechabeám vyslal Hadoráma, který byl nad nucenými pracemi, ale Izraelci ho ukamenovali k smrti. Král Rechabeám byl nucen skočit do vozu a utéci do Jeruzaléma.
#10:19 Tak se vzbouřil Izrael proti domu Davidovu, a to trvá dodnes. 
#11:1 Když přišel Rechabeám do Jeruzaléma, shromáždil dům Judův a Benjamínův, sto osmdesát tisíc vybraných bojovníků, aby bojovali s Izraelem a vrátili království Rechabeámovi.
#11:2 I stalo se slovo Hospodinovo k Šemajášovi, muži Božímu:
#11:3 „Řekni judskému králi Rechabeámovi, synu Šalomounovu, a celému Izraeli na území Judy a Benjamína:
#11:4 Toto praví Hospodin: Netáhněte a nebojujte proti svým bratřím. Každý ať se vrátí do svého domu, neboť se to stalo z mé vůle.“ Uposlechli tedy Hospodinových slov a vrátili se z výpravy proti Jarobeámovi zpět.
#11:5 Rechabeám se usídlil v Jeruzalémě a vystavěl v Judsku opevněná města.
#11:6 Vystavěl Betlém, Étam a Tekóu,
#11:7 Bét-súr, Sóko a Adulám,
#11:8 Gat, Maréšu a Zíf,
#11:9 Adórajim, Lakíš a Azéku,
#11:10 Soreu, Ajalón a Chebrón; to jsou opevněná města v Judovi a Benjamínovi.
#11:11 Zesílil pevnosti, dal jim vévody i zásoby potravy, oleje a vína
#11:12 a do každého jednotlivého města pavézy a oštěpy. Nesmírně je posílil. Patřil mu Juda a Benjamín.
#11:13 Kněží a lévijci z celého Izraele ze všech území se k němu přidali.
#11:14 Lévijci opustili své pastviny i své državy a odešli do Judska a do Jeruzaléma, protože je Jarobeám se svými syny vyloučil, aby nesloužili Hospodinu jako kněží.
#11:15 Sám si ustanovil kněze pro posvátná návrší, pro běsy a pro býčky, které udělal.
#11:16 Za lévijci pak odcházeli ze všech izraelských kmenů ti, kteří si předsevzali hledat Hospodina, Boha Izraele. Přicházeli do Jeruzaléma obětovat Hospodinu, Bohu svých otců.
#11:17 Posilovali judské království a po tři léta dodávali odvahu Rechabeámovi, synu Šalomounovu. Po tři léta totiž chodili cestou Davida a Šalomouna.
#11:18 Rechabeám si vzal za ženu Machalatu, dceru Jerímóta, syna Davidova, a Abíhajilu, dceru Elíaba, syna Jišajova.
#11:19 Porodila mu syny: Jeúše, Šemarjáše a Zahama.
#11:20 Po ní si vzal Maaku, dceru Abšalómovu. Ta mu porodila Abijáše, Ataje, Zízu a Šelomíta.
#11:21 Nade všechny své ženy a ženiny miloval Rechabeám Maaku, dceru Abšalómovu. Měl osmnáct žen a šedesát ženin a zplodil dvacet osm synů a šedesát dcer.
#11:22 Do čela postavil Rechabeám Abijáše, syna Maaky, jako vévodu mezi jeho bratry, neboť ho chtěl dosadit za krále.
#11:23 Ostatní své syny po úvaze rozdělil do všech území Judy a Benjamína po všech opevněných městech a bohatě je zaopatřil. Vyžádal jim množství žen. 
#12:1 Když se Rechabeámo království upevnilo a utvrdilo, opustil Hospodinův zákon a celý Izrael s ním.
#12:2 V pátém roce vlády krále Rechabeáma vytáhl Šíšak, král egyptský, proti Jeruzalému, protože se zpronevěřili Hospodinu,
#12:3 s tisícem a dvěma sty vozů a se šedesáti tisíci jezdců. Lid, který s ním přišel z Egypta, Lúbijce, Sukejce a Kúšijce, nebylo možno spočítat.
#12:4 Dobyl opevněná města, která byla v Judsku, a přitáhl až k Jeruzalému.
#12:5 Tu přišel prorok Šemajáš k Rechabeámovi a judským velitelům, kteří se před Šíšakem stáhli do Jeruzaléma. Řekl jim: „Toto praví Hospodin: ‚Vy jste opustili mne a já teď opouštím vás a ponechám vás v moci Šíšakově.‘“
#12:6 Izraelští velitelé i král se pokořili a řekli: „Hospodin je spravedlivý.“
#12:7 Když Hospodin viděl, že se pokořili, stalo se slovo Hospodinovo k Šemajášovi: „Pokořili se, neuvedu na ně zkázu, ale dám jim brzy možnost úniku; mé rozhořčení na Jeruzalém nebude vylito skrze Šíšaka.
#12:8 Stanou se však jeho otroky, aby poznali, co znamená sloužit mně a co otročit královstvím zemí.“
#12:9 Šíšak, král egyptský, vytáhl proti Jeruzalému a pobral poklady domu Hospodinova i poklady domu královského; pobral všechno. Pobral i zlaté štíty, které pořídil Šalomoun.
#12:10 Místo nich pořídil král Rechabeám štíty bronzové a svěřil je velitelům běžců, kteří střežili vchod do královského domu.
#12:11 Kdykoli král vcházel do Hospodinova domu, přicházeli běžci a přinášeli je a zase je odnášeli do místnosti pro běžce.
#12:12 Že se tak pokořil, Hospodinův hněv se od něho odvrátil a nezničil ho úplně, neboť i v Judsku bylo leccos dobrého.
#12:13 Král Rechabeám pevně vládl v Jeruzalémě. Bylo mu jedenačtyřicet let, když začal kralovat, a kraloval sedmnáct let v Jeruzalémě, v městě, které Hospodin vyvolil ze všech izraelských kmenů, aby tam spočinulo jeho jméno. Jeho matka se jmenovala Naama; byla to Amónka.
#12:14 Dopouštěl se toho, co je zlé, neboť se nedotazoval Hospodina upřímným srdcem.
#12:15 O příbězích Rechabeámových, prvních i posledních se píše, jak známo, v příbězích proroka Šemajáše a vidoucího Ida, kde se uvádí rodokmen. Mezi Rechabeámem a Jarobeámem byly po všechna ta léta války.
#12:16 I ulehl Rechabeám ke svým otcům a byl pohřben v Městě Davidově. Po něm se stal králem jeho syn Abijáš. 
#13:1 V osmnáctém roce vlády krále Jarobeáma se stal králem nad Judou Abijáš.
#13:2 Kraloval v Jeruzalémě tři léta. Jeho matka se jmenovala Míkaja; byla to dcera Uríelova z Gibeje. Válka mezi Abijášem a Jarobeámem trvala dál.
#13:3 Abijáš vypravil do boje vojsko, válečné bohatýry, čtyři sta tisíc vybraných mužů. Jarobeám seřadil do boje proti němu osm set tisíc vybraných mužů, udatných bohatýrů.
#13:4 Abijáš stál na hoře Semarajimu, která je v Efrajimském pohoří, a odtud volal: „Slyšte mě, Jarobeáme i celý Izraeli!
#13:5 Což nevíte, že Hospodin, Bůh Izrale, dal navěky království nad Izraelem Davidovi? Jemu a jeho synům je potvrdil smlouvou soli.
#13:6 Avšak Jarobeám, syn Nebatův, služebník Šalomouna, syna Davidova, povstal a vzbouřil se proti svému pánu.
#13:7 Shromáždili se k němu muži bohaprázdní a ničemové a rozhodně se postavili proti Šalomounovu synovi Rechabeámovi. Rechabeám byl mladý a zbabělý a nedokázal jim čelit.
#13:8 A teď si říkáte, že chcete čelit Hospodinovu království, jež je v rukou synů Davidových? Je vás ohromné množství a máte při sobě zlaté býčky, které vám udělal Jarobeám jako bohy.
#13:9 Ale což jste nevyhnali Hospodinovy kněze Áronovce a lévijce? Nenadělali jste si kněží jako národy jiných zemí? Kdokoli přišel, mohl být za mladého býčka a sedm beranů pověřen, aby byl knězem neboha.
#13:10 Naším Bohem je Hospodin; my jsme ho neopustili. Kněží, synové Áronovi, i lévijci řádně slouží Hospodinu.
#13:11 Ráno co ráno a večer co večer obracejí zápalné oběti v obětní dým, pálí vonná koření, rovnají chléb na stůl z čistého zlata, starají se o zlatý svícen a jeho kahánky, aby hořely večer co večer. My držíme stráž před Hospodinem, naším Bohem. Vy jste ho však opustili.
#13:12 Hle, nám je v čele Bůh. Jeho kněží hlaholí zvučnými trubkami proti vám. Izraelci, nebojujte proti Hospodinu, Bohu svých otců, neboť nebudete mít zdar!“
#13:13 Jarobeám provedl se zálohou obchvat a přitáhl na ně zezadu. Tak byli před Judou a záloha vzadu za ním.
#13:14 Když se Judejci rozhlédli, viděli, že je čeká boj vpředu i vzadu. Úpěnlivě volali k Hospodinu a kněží troubili na trubky.
#13:15 Pak judští muži spustili válečný pokřik. Za pokřiku judských mužů porazil Bůh Jarobeáma a celý Izrael před Abijášem a Judou.
#13:16 Izraelci se před Judou dali na útěk; Bůh jim je vydal do rukou.
#13:17 Abijáš a jeho lid jim způsobili velikou porážku, takže z Izraele padlo a bylo skoleno pět set tisíc vybraných mužů.
#13:18 V té době byli Izraelci pokořeni a Judejci se vzmohli, neboť se opírali o Hospodina, Boha svých otců.
#13:19 Abijáš pronásledoval Jarobeáma a dobyl na něm města Bét-el s vesnicemi, Ješánu s vesnicemi a Efrón s vesnicemi.
#13:20 Za dnů Abijášových se moc Jarobeámova už nepozvedla. Hospodin jej porazil, takže zemřel.
#13:21 Ale Abijáš vládl pevně. Pojal čtrnáct žen a zplodil dvaadvacet synů a šestnáct dcer.
#13:22 O ostatních příbězích Abijášových, o jeho činech i slovech se píše ve výkladu proroka Ida.
#13:23 I ulehl Abijáš ke svým otcům a pohřbili ho v Městě Davidově. Po něm se stal králem jeho syn Ása. Za jeho dnů žila země v míru deset let. 
#14:1 Ása činil to, co bylo dobré a správné v očích Hospodina, jeho Boha.
#14:2 Odstranil cizí oltáře a posvátná návrší, roztříštil posvátné sloupy a pokácel posvátné kůly.
#14:3 Judovi poručil, aby se dotazoval Hospodina, Boha svých otců, a plnil zákon a přikázání.
#14:4 Ze všech judských měst odstranil posvátná návrší a kadidlové oltáříky. Za něho žilo království v míru.
#14:5 V Judsku stavěl opevněná města, neboť země žila v míru. Žádná válka se v těch letech proti němu nevedla. Hospodin mu dopřál klidu.
#14:6 Ása řekl Judovi: „Vystavějme tato města, obežeňme je hradbami s věžemi, vraty a závorami, dokud je naše země v klidu. Dotazovali jsme se Hospodina, našeho Boha, dotazovali jsme se ho a on nám dopřál klid na všech stranách.“ Stavěli tedy a dařilo se jim to.
#14:7 Ása měl vojsko, tři sta tisíc mužů z Judy, nosících pavézy a oštěpy, dvě stě osmdesát tisíc z Benjamína, kteří nosili štíty a stříleli z luku. Všichni byli udatní bohatýři.
#14:8 Proti nim vytáhl Kúšijec Zerach se statisícovým vojskem a třemi sty vozy; přitáhl až k Maréše.
#14:9 Ása vytáhl proti němu. Seřadili se k bitvě v údolí Sefaty u Maréši.
#14:10 Ása volal k Hospodinu, svému Bohu. Řekl: „Hospodine, jen ty jsi s to pomoci, postaví-li se mocný proti bezmocnému. Pomoz nám, Hospodine, náš Bože, neboť ty jsi naše opora. V tvém jménu jsme přitáhli proti tomuto davu. Hospodine, ty jsi náš Bůh, proti tobě člověk nic nezmůže!“
#14:11 I porazil Hospodin Kúšijce před Ásou a před Judou. Kúšijci se dali na útěk.
#14:12 Ása se svým lidem je pronásledoval až ke Geraru. Kúšijci padli, žádný z nich nevyvázl živ; byli rozdrceni před Hospodinem a před jeho táborem. Judejci odnesli velké množství kořisti.
#14:13 Vybili všechna města v okolí Geraru, na něž dolehl strach z Hospodina. Vyloupili všechna města a bylo v nich velké množství lupu.
#14:14 Vybili také stany chovatelů stád a pobrali mnoho ovcí a velbloudů. Pak se vrátili do Jeruzaléma. 
#15:1 Na Azarjášovi, synu Ódedovu, spočinul duch Boží.
#15:2 Vyšel před Ásu a řekl mu: „Poslyš mě, Áso i celý Judo a Benjamíne. Hospodin bude s vámi, když vy budete s ním. Budete-li se dotazovat na jeho slovo, dá se vám nalézt. Jestliže ho opustíte, opustí vás.
#15:3 Mnoho let býval Izrael bez pravého Boha, bez kněze - učitele a bez zákona.
#15:4 Když se však ve svém soužení obraceli k Hospodinu, Bohu Izraele, a hledali ho, dával se jim nalézt.
#15:5 V oněch dobách nebylo možno pokojně vycházet a vcházet, neboť mezi všemi obyvateli zemí vládly obrovské zmatky.
#15:6 Docházelo ke srážkám mezi pronárodem a pronárodem, mezi městem a městem, neboť Bůh je uvedl ve zmatek všelijakým soužením.
#15:7 Jednejte tedy rozhodně, nechť vaše ruce neochabnou, neboť vaše jednání bude odměněno!“
#15:8 Když Ása uslyšel tato slova a proroctví proroka Azarjáše, syna Ódedova, vzchopil se a vymýtil ohyzdné modly z celé země judské i z Benjamína a z měst, která dobyl v Efrajimském pohoří, a obnovil Hospodinův oltář před Hospodinovou předsíní.
#15:9 Pak shromáždil celého Judu i Benjamína a z kmene Efrajimova, Manasesova a Šimeónova ty, kteří u nich pobývali jako hosté; přešlo k němu totiž mnoho lidí z Izraele, když viděli, že je s ním Hospodin, jeho Bůh.
#15:10 Třetího měsíce, v patnáctém roce Ásova kralování, se shromáždili do Jeruzaléma.
#15:11 Z kořisti, kterou přihnali, obětovali onoho dne Hospodinu sedm set kusů hovězího dobytka a sem tisíc ovcí a koz.
#15:12 Zavázali se smlouvou, že se budou dotazovat na slovo Hospodina, Boha svých otců, celým svým srdcem a celou svou duší.
#15:13 A každý, kdo by se nedotazoval Hospodina, Boha Izraele, propadne smrti, ať malý nebo velký, ať muž nebo žena.
#15:14 Přísahali Hospodinu mocným hlasem a nadšeným hlaholem za zvuku trubek a polnic.
#15:15 Z té přísahy se radoval celý Juda. Přísahali celým srdcem, že ho budou hledat s veškerým zanícením, aby se jim dal najít. A Hospodin jim dopřál klid na všech stranách.
#15:16 Také svou matku Maaku zbavil král Ása jejího královského postavení, za to, že udělala nestvůrnou modlu pro Ašéru. Ása její nestvůrnou modlu podťal, rozdrtil na padrť a spálil v Kidrónském úvalu.
#15:17 I když z Izraele nebyla odstraněna posvátná návrší, přece srdce Ásovo bylo po všechny jeho dny cele při Hospodinu.
#15:18 Svaté dary svého otce i své svaté dary, stříbro, zlato a různé předměty, přinesl do Božího domu.
#15:19 Až do pětatřicátého roku Ásova kralování nebyla žádná válka. 
#16:1 V třicátém šestém roce Ásova kralování vytáhl Baeša, král izraelský, proti Judsku. Vystavěl Rámu, aby zabránil judskému králi Ásovi vycházet a vcházet.
#16:2 Ása tehdy vynesl z pokladů domu Hospodinova a domu královského zlato a stříbro a poslal je Ben-hadadovi, králi aramejskému, který sídlil v Damašku, se vzkazem:
#16:3 „Máme spolu smlouvu. Měli ji i můj otec a tvůj otec. Zde ti posílám stříbro a zlato. Nuže, zruš svou smlouvu s izraelským králem Baešou, ať ode mne odtáhne.“
#16:4 Ben-hadad krále Ásu uposlechl a poslal velitele se svými vojsky proti izraelským městům. Vybili Ijón, Dan, Ábel-majim a všechna města skladů v kmeni Neftalí.
#16:5 Když to Baeša uslyšel, přestal stavět Rámu a své dílo přerušil.
#16:6 Král Ása sebral celé Judsko, aby z Rámy odnášeli kámen a dřevo, z nichž stavěl Baeša, a vystavěl z toho Gebu a Mispu.
#16:7 V té době přišel k judskému králi Ásovi Chananí, vidoucí, a řekl mu: „Poněvadž jsi hledal oporu u krále aramejského a neopíráš se o Hospodina, svého Boha, vymkla se ti vojska krále aramejského z ruky.
#16:8 Což neměli Kúšijci a Lúbijci početné vojsko, obrovské množství vozů a jezdců? Když jsi však hledal oporu u Hospodina, dal ti je do rukou.
#16:9 Neboť oči Hospodinovy obzírají celou zemi, takže se mohou vzchopit ti, jejichž srdce je cele při něm. Ale v této věci jsi jednal jako pomatenec, a od nynějska budeš muset válčit.“
#16:10 Ása se na vidoucího rozlítil a dal ho vsadit do žaláře; tak ho ta věc rozzuřila. V oné době Ása ukřivdil mnohým z lidu.
#16:11 Hle, o příbězích Ásových, prvních i posledních, se píše dále v Knize králů judských a izraelských.
#16:12 V třicátém devátém roce svého kralování onemocněl Ása přetěžkou nemocí nohou. Ale ani ve své nemoci se nedotazoval Hospodina, nýbrž vyhledal jen lékaře.
#16:13 I ulehl Ása ke svým otcům a zemřel v jednačtyřicátém roce svého kralování.
#16:14 Pohřbili ho v jeho hrobce, kterou si dal vytesat v Městě Davidově; uložili ho na lůžko plné balzámů a dovedně připravených mastí a zapálili mu převeliký oheň. 
#17:1 Místo Ásy se stal králem jeho syn Jóšafat. Ten pevně vládl nad Izraelem.
#17:2 Do všech opevněných judských měst dal vojsko a na judském území i v efrajimských městech, která dobyl jeho otec Ása, umístil výsostné znaky.
#17:3 Hospodin byl s Jóšafatem, neboť chodil po cestách, jimiž chodíval dříve jeho otec David, a nedotazoval se baalů.
#17:4 Dotazoval se Boha svého otce a řídil se jeho přikázáními, ne jak to činil Izrael.
#17:5 Hospodin upevnil království v jeho ruce a celý Juda dával Jóšafatovi dary, takže měl mnoho bohatství i slávy.
#17:6 Jeho srdce bylo naplněno hrdostí, že chodí po cestách Hospodinových; proto také odstranil z Judy posvátná návrší a posvátné kůly.
#17:7 V třetím roce svého kralování vyslal své správce Ben-chajila, Obadjáše, Zekarjáše, Netaneela a Míkajáše, aby učili v judských městech.
#17:8 S nim poslal lévijce: Šemajáše, Netanjáše, Zebadjáše, Asáela, Šemiramóta, Jónatana, Adónijáše, Tobijáše a Tób-adónijáše, lévijce, a s nimi kněze Elíšámu a Jórama.
#17:9 Ti učili v Judsku. Měli s sebou Knihu zákona Hospodinova, obcházeli všechna judská města a vyučovali lid.
#17:10 Strach z Hospodina dolehl na všechna království těch zemí, které byly kolem Judy, takže proti Jóšafatovi nebojovala.
#17:11 Někteří z Pelištejců přinášeli Jóšafatovi dary a stříbro jako daň; také Arabové mu přiváděli brav: sedm tisíc sedm set beranů a sedm tisíc sedm set kozlů.
#17:12 Jóšafat se stával stále mocnějším. V Judsku vystavěl hradiska a města pro sklady.
#17:13 V judských městech podnikal velká díla. V Jeruzalémě měl bojovníky, udatné bohatýry.
#17:14 Toto jsou povolaní do služby podle jednotlivých otcovských rodů: Juda měl velitele nad tisíci: velitele Adnu s třemi sty tisíci udatnými bohatýry,
#17:15 vedle něho velitele Jóchanana s dvěma sty osmdesáti tisíci muži,
#17:16 dále Amasjáše, syna Zikrího, dobrovolníka Hospodinova, s dvěma sty tisíci udatnými bohatýry;
#17:17 z Benjamína byl udatný bohatýr Eljáda s dvěma sty tisíci těch, kteří byli vyzbrojeni luky a štíty,
#17:18 dále Józabad se sto osmdesáti tisíci vyzbrojenými vojáky.
#17:19 Ti byli ve službách samého krále, kromě těch, které král umístil do opevněných měst v celém Judsku. 
#18:1 Jóšafat měl nesmírné bohatství a slávu. Spříznil se s Achabem.
#18:2 Po několika letech sestoupil k Achabovi do Samaří. Achab pro něho i pro lid, který byl s ním dal porazit mnoho ovcí a koz i hovězího dobytka a pak se ho snažil zlákat k výpravě proti Rámotu v Gileádu.
#18:3 Achab, král izraelský, se otázal Jóšafata, krále judského: „Půjdeš se mnou proti Rámotu v Gileádu?“ On mu odvětil: „Ve válce jsme jedno, já jako ty, můj lid jako tvůj lid.“
#18:4 Nato Jóšafat izraelskému králi řekl: „Dotaž se ještě dnes na slovo Hospodinovo.“
#18:5 Izraelský král shromáždil proroky, čtyři sta mužů, a řekl jim: „Máme jít do války proti Rámotu v Gileádu, nebo mám od toho upustit?“ Odpověděli: „Jdi, Bůh jej vydá králi do rukou.“
#18:6 Ale Jóšafat se zeptal: „Cožpak tu už není žádný prorok Hospodinův, abychom se dotázali skrze něho?“
#18:7 Izraelský král Jóšafatovi odpověděl: „Je tu ještě jeden muž, skrze něhož bychom se mohli dotázat Hospodina, ale já ho nenávidím, protože mi neprorokuje nic dobrého, nýbrž po všechny dny jen zlo. Je to Míkajáš, syn Jimlův.“ Jóšafat řekl: „Nechť král tak nemluví!“
#18:8 Izraelský král tedy povolal jednoho dvořana a řekl: „Rychle přiveď Míkajáše, syna Jimlova.“
#18:9 Král izraelský i Jóšafat, král judský, seděli každý na svém trůnu slavnostně oblečeni, seděli na humně u vchodu do samařské brány a všichni proroci před nimi prorokovali.
#18:10 Sidkijáš, syn Kenaanův, si udělal železné rohy a říkal: „Toto praví Hospodin: Jimi budeš trkat Arama, dokud neskoná.“
#18:11 Tak prorokovali všichni proroci: „Vytáhni proti Rámotu v Gileádu. Budeš mít úspěch. Hospodin jej vydá králi do rukou!“
#18:12 Posel, který šel zavolat Míkajáše, mu domlouval: „Hle, slova proroků jedněmi ústy ohlašují králi dobré věci. Ať je tvá řeč jako řeč každého z nich, mluv o dobrých věcech!“
#18:13 Míkajáš odpověděl: „Jakože živ je Hospodin, budu mluvit to, co řekne můj Bůh.“
#18:14 Když přišel ke králi, král mu řekl: „Míko, máme jít do války proti Rámotu v Gileádu, nebo mám od toho upustit?“ On odpověděl: „Vytáhněte, budete mít úspěch. Budou vám vydáni do rukou.“
#18:15 Král ho okřikl: „Kolikrát tě mám zapřísahat, abys mi v Hospodinově jménu nemluvil nic než pravdu?“
#18:16 Míkajáš odpověděl: „Viděl jsem všechen Izrael rozptýlený po horách jako ovce, které nemají pastýře. - Hospodin řekl: ‚Zůstali bez pánů, ať se každý v pokoji vrátí domů.‘“
#18:17 Izraelský král řekl Jóšafatovi: „Neřekl jsem ti, že mi nebude prorokovat nic dobrého, nýbrž jen zlo?“
#18:18 Ale Míkajáš pokračoval: „Tak tedy slyš slovo Hospodinovo. Viděl jsem Hospodina sedícího na trůně. Všechen nebeský zástup stál po jeho pravici a levici.
#18:19 Hospodin řekl: ‚Kdo zláká izraelského krále Achaba, aby vytáhl a padl u Rámotu v Gileádu?‘ Ten říkal to a druhý ono.
#18:20 Tu vystoupil jakýsi duch, postavil se před Hospodina a řekl: ‚Já ho zlákám.‘ Hospodin mu pravil: ‚Čím?‘
#18:21 On odvětil: ‚Vyjdu a stanu se zrádným duchem v ústech všech jeho proroků.‘ Hospodin řekl: ‚Ty ho zlákáš, ty to dokážeš. Jdi a učiň to!‘
#18:22 A nyní, hle, Hospodin dal zrádného ducha do úst všech těchto tvých proroků. Hospodin ti ohlásil zlé věci.“
#18:23 Sidkijáš, syn Kenaanův, přistoupil, uhodil Míkajáše do tváře a řekl: „Jakou cestou přešel Hospodinův duch ode mne, aby mluvil skrze tebe?“
#18:24 Míkajáš mu řekl: „To uvidíš v onen den, až vejdeš do nejzazšího pokojíku, aby ses ukryl.“
#18:25 Izraelský král nařídil: „Seberte Míkajáše a odveďte ho k veliteli města Amónovi a ke královu synu Jóašovi.
#18:26 Vyřídíte: ‚Toto praví král: Tady toho vsaďte do vězení a dávejte mu jen kousek chleba a trochu vody, dokud se nenavrátím v pokoji!‘“
#18:27 Ale Míkajáš řekl: „Jestliže se vrátíš zpět v pokoji, nemluvil skrze mne Hospodin.“ A připojil: „Slyšte to, všichni lidé!“
#18:28 Král izraelský i Jóšafat, král judský, vytáhli proti Rámotu v Gileádu.
#18:29 Izraelský král řekl Jóšafatovi, že se přestrojí, až vyjede do boje. Dodal: „Ty ovšem obleč své roucho.“ Tak se izraelský král přestrojil a vyjeli do boje.
#18:30 Aramejský král přikázal velitelům své vozby: „Nebojujte ani s malým ani s velkým, ale jen se samým izraelským králem!“
#18:31 Když velitelé vozby spatřili Jóšafata, řekli si: „To je izraelský král!“ Obklopili ho, aby bojovali proti němu. Tu Jóšafat vyrazil válečný pokřik a Hospodin mu pomohl; tak je Bůh od něho odlákal.
#18:32 Když velitelé vozby viděli, že to není izraelský král, odvrátili se od něho.
#18:33 Kdosi však bezděčně napjal luk a zasáhl izraelského krále mezi články pancíře. Král řekl svému vozataji: „Obrať a odvez mě z bojiště, jsem raněn.“
#18:34 Ale boj se toho dne tak vystupňoval, že izraelský král musel zůstat na voze proti Aramejcům až do večera a v čas západu slunce zemřel. 
#19:1 Judský král Jóšafat se v pokoji vrátil do svého domu v Jeruzalémě.
#19:2 Tu proti němu vyšel vidoucí Jehú, syn Chananího, a řekl králi Jóšafatovi: „Budeš pomáhat svévolnému a milovat ty, kdo nenávidí Hospodina? Kvůli tomu se na tebe Hospodin rozhněval.
#19:3 Přece však se na tobě našlo i něco dobrého: Odstranil jsi ze země posvátné kůly a dotazoval ses Boha upřímným srdcem.“
#19:4 Jóšafat sídlil v Jeruzalémě. Pak zase vycházel mezi lid od Beer-šeby až k Efrajimskému pohoří a navracel je k Hospodinu, Bohu jejich otců.
#19:5 V zemi ustanovil soudce, ve všech opevněných městech judských, v každém z nich.
#19:6 Soudcům nařídil: „Dbejte na to, co děláte, vždyť nevykonáváte soud lidský, ale soud Hospodinův. On je s vámi, když vynášíte rozsudek.
#19:7 Ať je nyní nad vámi strach z Hospodina; konejte všechno bedlivě. U Hospodina, našeho Boha, neobstojí žádné bezpráví ani stranění osobám ani úplatkářství.“
#19:8 Také v Jeruzalémě ustanovil Jóšafat některé lévijce, kněze a představené izraelských rodů, aby vykonávali Hospodinův soud a rozhodovali ve sporech těch, kteří by se s nimi obraceli do Jeruzaléma.
#19:9 Král jim přikázal: „Konejte všechno v Hospodinově bázni, věrně a z celého srdce,
#19:10 a to při každém sporu, se kterým by k vám přišli vaši bratří bydlící ve svých městech; ať to bude spor o krevní zločin nebo o zákon a přikázání či o nařízení a řády. Varujte je, aby se neproviňovali proti Hospodinu, ať vás a vaše bratry nepostihne jeho rozlícení. Tak čiňte a budete bez viny!
#19:11 Hle, nejvyšší kněz Amarjáš bude nad vámi ve všem, co se týká Božích věcí, a Zebadjáš, syn Jišmaelův, vévoda domu Judova, bude nad vámi ve všem, co se týká záležitostí královských. A jako úředníci tu budou s vámi lévijci. Buďte rozhodní a jednejte. A Hospodin bude s tím, kdo je dobrý.“ 
#20:1 Potom se stalo, že vytáhli Moábci a Amónovci a s nimi někteří z amónských spojenců do války proti Jóšafatovi.
#20:2 Tu přišli a oznámili Jóšafatovi: „Vytáhlo proti tobě veliké množství z krajin za mořem, od Aramejců. Hle, jsou v Chasesón-támaru, to jest v Én-gedí.“
#20:3 Jóšafat se dotázal s bázní Hospodina. Vyhlásil také po celém Judsku půst.
#20:4 Judejci se shromáždili, aby hledali pomoc od Hospodina. Také ze všech judských měst přišli hledat Hospodina.
#20:5 Jóšafat stanul ve shromáždění Judejců a obyvatel Jeruzaléma v Hospodinově domě před novým nádvořím
#20:6 a řekl: „Hospodine, Bože našich otců, cožpak nejsi ty Bůh na nebi, který vládne nade všemi královstvími pronárodů? Máš v rukou moc a bohatýrskou sílu, nikdo se ti nemůže zpěčovat.
#20:7 Což jsi to nebyl ty, náš Bože, jenž jsi vyhnal obyvatele této země před Izraelem, svým lidem? Dal jsi ji navěky potomkům Abrahama, svého přítele.
#20:8 Usídlili se v ní a vybudovali ti v ní svatyni tvému jménu. Řekli:
#20:9 ‚Jestliže na nás přijde něco zlého, meč soudu, mor či hlad, postavíme se před tento dům a před tebe, protože v tomto domě dlí tvé jméno, a ve svém soužení budeme k tobě volat, a ty nás vyslyšíš a zachráníš.‘
#20:10 Nyní pohleď na Amónovce a Moábce i na horu Seír, kudy jsi Izraeli nedovolil projít, když přicházel z egyptské země; uhnul od nich a nevyhladil je.
#20:11 Hle, jak nám odplácejí. Přitáhli nás vyhnat z tvého vlastnictví, které jsi nám určil.
#20:12 Bože náš, což je nebudeš soudit? Nemáme sílu proti tomuto velikému množství, které táhne proti nám. Nevíme, co máme dělat, proto vzhlížíme k tobě.“
#20:13 Všichni Judejci stáli před Hospodinem i se svými dětmi, ženami a syny.
#20:14 Tu sestoupil uprostřed shromáždění duch Hospodinův na Jachzíela, syna Zekarjáše, syna Benajáše, syna Jeíela, syna Matanjášova, lévijce ze synů Asafových,
#20:15 a ten pravil: „Pozorně naslouchejte, všichni Judejci a obyvatelé Jeruzaléma, i ty, králi Jóšafate. Toto vám praví Hospodin: ‚Nebojte se a neděste se tohoto velikého množství.‘ Boj není váš, ale Boží.
#20:16 Zítra proti nim sestupte, až budou vystupovat do Svahu květů. Potkáte je na konci úvalu směrem k poušti Jerúelu.
#20:17 Vy přitom bojovat nemusíte. Postavte se, Judejci a obyvatelé Jeruzaléma, stůjte a uvidíte, jak vás Hospodin zachrání. Nebojte se a neděste se a zítra proti nim vytáhněte! Hospodin bude s vámi.“
#20:18 Nato Jóšafat padl na kolena tváří k zemi a všichni Judejci i obyvatelé Jeruzaléma padli před Hospodinem a klaněli se mu.
#20:19 Potom povstali lévijci z Kehatovců a Kórachovců, aby chválili Hospodina, Boha Izraele, hlasem velice mocným.
#20:20 Za časného jitra vytáhli na tekójskou poušť. Když vycházeli, Jóšafat se postavil a řekl: „Slyšte mě, Judejci i obyvatelé Jeruzaléma! Věřte v Hospodina, svého Boha, a budete nepohnutelní; věřte jeho prorokům a bude vás provázet zdar.“
#20:21 Po poradě s lidem rozestavil před Hospodinem zpěváky, aby chválili jeho velebnou svatost. Šli před ozbrojenci a provolávali: „Chválu vzdejte Hospodinu, jeho milosrdenství je věčné.“
#20:22 V ten čas, kdy se dali do jásotu a chval, vyslal Hospodin zálohy proti Amónovcům a Moábcům i proti hoře Seíru, proti těm, kteří přitáhli na Judejce, a byli poraženi.
#20:23 Amónovci a Moábci se totiž postavili proti obyvatelům hory Seíru, aby je vyhladili jako klaté. Když pak skoncovali s obyvateli Seíru, pomohli jedni druhým do zkázy.
#20:24 Když vstoupili Judejci na výšinu s výhledem na poušť, a obrátili se směrem k tomu shluku, spatřili jen mrtvá těla ležící na zemi; nikdo nevyvázl.
#20:25 Pak přišel Jóšafat a jeho lid posbírat kořist. Našli tam u mrtvých těl množství majetku a vzácné výzbroje; získali pro sebe tolik, že to nemohli unést. Kořist sbírali tři dny, tak byla hojná.
#20:26 Čtvrtého dne se shromáždili v dolině Beraka (to je Dobrořečení), protože tam dobrořečili Hospodinu. Proto se to místo nazývá Dolina dobrořečení až dodnes.
#20:27 Potom se všichni judští i jeruzalémští s Jóšafatem v čele vraceli s radostí do Jeruzaléma, protože je Hospodin naplnil radostí z vítězství nad nepřáteli.
#20:28 Přišli s harfami, citarami a trubkami do Jeruzaléma, do Hospodinova domu.
#20:29 Strach před Bohem dolehl na království všech zemí, když slyšela, jak Hospodin bojoval proti nepřátelům Izraele.
#20:30 Jóšafatovo království žilo v míru a jeho Bůh mu dopřál klid na všech stranách.
#20:31 Jóšafat se stal králem nad Judou. Bylo mu třicet pět let, když začal kralovat, a kraloval v Jeruzalémě dvacet pět let. Jeho matka se jmenovala Azúba; byla to dcera Šilchího.
#20:32 Chodil po cestě svého otce Ásy. Neodchýlil se od ní, ale činil to, co je správné v Hospodinových očích.
#20:33 Jenom neodstranili posvátná návrší; lid ještě nestál srdcem cele při Bohu svých otců.
#20:34 O ostatních příbězích Jóšafatových, prvních i posledních, se dále píše v Příbězích Jehúa, syna Chananího, jež byly pojaty do Knihy králů izraelských.
#20:35 Ale potom se Jóšafat, král judský, spolčil s Achazjášem, králem izraelským, a ten ho svedl k svévolným činům.
#20:36 Spolčil se s ním, aby mohl udělat lodě k plavbě do zámoří. Ty lodě dělali v Esjón-geberu.
#20:37 Elíezer, syn Dódavahův z Maréši, však prorokoval a řekl Jóšafatovi: „Že jsi se spolčil s Achazjášem, Hospodin přerušil, co jsi udělal.“ Lodě ztroskotaly a nemohly plout do zámoří. 
#21:1 I ulehl Jóšafat ke svým otcům a byl pohřben vedle svých otců v Městě Davidově. Po něm se stal králem jeho syn Jóram.
#21:2 Jóram měl bratry, další syny Jóšafatovy: Azarjáše a Jechíela, Zekarjáše a Azarjáše, Míkaela a Šefatjáše. Ti všichni jsou synové Jóšafata, krále izraelského.
#21:3 Jejich otec jim dal mnoho darů, stříbro, zlato a jiné výtečné věci, též opevněná města v Judsku; království dal Jóramovi, protože ten byl prvorozený.
#21:4 Když se Jóram ujal království svého otce a pevně vládl, vyvraždil mečem všechny své bratry i některé izraelské velitele.
#21:5 Jóramovi bylo dvaatřicet let, když začal kralovat, a kraloval v Jeruzalémě osm let.
#21:6 Chodil po cestě králů izraelských, jak to činil dům Achabův; jeho ženou byla totiž dcera Achabova. Dopouštěl se toho, co je zlé v Hospodinových očích.
#21:7 Ale Hospodin nechtěl na dům Davidův uvést zkázu, a to kvůli smlouvě, kterou s Davidem uzavřel, jak mu přislíbil, že dá jemu i jeho synům planoucí světlo po všechny dny.
#21:8 Za jeho dnů se Edómci vymanili z područí Judy a ustanovili nad sebou krále.
#21:9 Jóram vytáhl se svými veliteli i s celou vozbou. Vstal v noci a udeřil na Edómce, kteří ho obklíčili, a na velitele vozby.
#21:10 Edóm se však vymanil z područí Judy, jak je tomu dodnes. Tehdy, v onen čas, se vymanila z jeho područí i Libna, protože opustil Hospodina, Boha svých otců.
#21:11 Dělal také na judských horách posvátná návrší a sváděl k modloslužbě obyvatele Jeruzaléma a zaváděl Judu na scestí.
#21:12 Byl mu doručen list od proroka Elijáše: „Toto praví Hospodin, Bůh Davida, tvého otce: ‚Za to, že jsi nechodil cestami svého otce Jóšafata ani cestami Ásy, krále judského,
#21:13 ale chodil jsi po cestě králů izraelských a sváděl jsi k modloslužbě Judu a obyvatele Jeruzaléma - jako sváděl k modloslužbě dům Achabův -, a dokonce jsi vyvraždil své bratry, dům svého otce, kteří byli lepší než ty,
#21:14 hle, Hospodin tvrdě napadne velikou pohromou tvůj lid, tvé syny i ženy i všechno tvé jmění.
#21:15 Těžce onemocníš střevní chorobou, až ti vyhřeznou střeva pro nemoc zhoršující se den ze dne.‘“
#21:16 Hospodin popudil proti Jóramovi Pelištejce a Araby, kteří sousedili s Kúšijci.
#21:17 Ti přitáhli, vtrhli do Judska a odvlekli do zajetí s veškerým jměním, které našli v královském domě, i jeho syny a ženy; nezůstal mu žádný syn než Jóachaz, nejmladší ze synů.
#21:18 Po tom všem jej Hospodin tvrdě napadl nevyléčitelnou střevní chorobou.
#21:19 Po určité době, asi do dvou let, vyhřezla mu v důsledku jeho choroby střeva a on za strašných bolestí zemřel. Lid pro něho nespaloval vonné látky jako pro jeho otce.
#21:20 Bylo mu dvaatřicet let, když začal kralovat a kraloval v Jeruzalémě osm let. Odešel a nikomu se nezastesklo. Pohřbili ho v Městě Davidově, avšak nikoli v hrobech královských. 
#22:1 Po Jóramovi dosadili obyvatelé Jeruzaléma za krále jeho nejmladšího syna Achazjáše; všechny jeho starší bratry totiž vyvraždila horda, která vnikla do tábora s Araby. Králem se tedy stal Achazjáš, syn Jórama, krále Judského.
#22:2 Achazjášovi bylo dvaačtyřicet let, když začal kralovat, a kraloval v Jeruzalémě jeden rok. Jeho matka se jmenovala Atalja; byla to dcera Omrího.
#22:3 I on chodil po cestách domu Achabova, neboť jeho matka mu radila k svévolnostem.
#22:4 Dopouštěl se toho, co je zlé v Hospodinových očích, jako dům Achabův, neboť ti mu po smrti jeho otce byli rádci k jeho vlastní zkáze.
#22:5 Na jejich radu také táhl s izraelským králem Jóramem, synem Achabovým, do války proti aramejskému králi Chazaelovi do Rámotu v Gileádu. Ale Aramejci Jórama ranili.
#22:6 Vrátil se, aby se léčil v Jizreelu z ran, které mu zasadili v Rámě, když válčil proti Chazaelovi, králi aramejskému. Azarjáš, syn Jóramův, král judský, sestoupil do Jizreelu podívat se na Jórama, syna Achabova, protože byl nemocný.
#22:7 Bylo to od Boha, že šel Achazjáš k Jóramovi; proto byl zničen. Po svém příchodu vyjel s Jóramem k Jehúovi, synu Nimšího, kterého Hospodin pomazal, aby vyhladil Achabův dům.
#22:8 Když Jehú konal soud nad Achabovým domem, našel judské velitele i syny bratrů Achazjášových, kteří byli v Achazjášových službách, a vyvraždil je.
#22:9 Hledal i Achazjáše. Chytili ho, když se ukrýval v Samaří. Přivedli ho k Jehúovi a usmrtili. Pak ho pohřbili, neboť si řekli: „Je to syn Jóšafata, který se celým svým srdcem dotazoval na Hospodinovo slovo.“ Z domu Achazjášova nebyl nikdo způsobilý kralovat.
#22:10 Když Atalja, matka Achazjášova, viděla, že její syn zemřel, rozhodla se vyhubit všechno královské potomstvo Judova domu.
#22:11 Ale Jóšabat, dcera králova, vzala Jóaše, syna Achazjášova, a unesla ho zprostřed královských synů, kteří měli být usmrceni. Dala ho s jeho kojnou do pokojíka s lůžky. A tak ho Jóšabat, dcera krále Jórama, žena kněze Jójady, protože to byla Achazjášova sestra, skryla před Ataljou, a ta ho neusmrtila.
#22:12 Schovával se u nich v Božím domě po šest let, zatímco v zemi kralovala Atalja. 
#23:1 V sedmém roce se Jójada vzchopil a vzal s sebou velitele setnin: Azarjáše; syna Jerochámova, Jišmáela, syna Jóchananova, Azarjáše, syna Óbedova, Maasejáše, syna Adajášova, a Elíšafata, syna Zikrího, aby uzavřeli smlouvu.
#23:2 Obcházeli Judsko a shromáždili ze všech judských měst lévijce i představitele izraelských otcovských rodů a přišli do Jeruzaléma.
#23:3 Celé shromáždění uzavřelo v Božím domě smlouvu s králem. Jójada jim řekl: „Hle, králův syn! Bude kralovat, jak mluvil Hospodin o Davidových synech.
#23:4 Uděláte tuto věc: Třetina z vás, kteří přicházíte v den odpočinku, kněží a lévijci i strážci prahů, bude zde,
#23:5 třetina bude v královském domě, třetina v Základové bráně a všechen lid na nádvořích Hospodinova domu.
#23:6 Nikdo ať nevchází do Hospodinova domu, jen kněží a službu konající lévijci; ti mohou vejít, poněvadž jsou posvěceni. Všechen lid bude na stráži u Hospodinova domu.
#23:7 Lévijci obklopí krále, každý se zbraní v ruce. Kdo by vešel do domu, bude usmrcen. Buďte s králem při jeho vycházení a vcházení.“
#23:8 Lévijci i celý Juda učinili vše, co jim kněz Jójada přikázal. Každý vzal své muže, ty, kteří šli v den odpočinku do služby, i ty, kteří v den odpočinku ze služby odcházeli; kněz Jójada totiž žádný oddíl ze služby nepropustil.
#23:9 Kněz Jójada vydal velitelům setnin kopí a štíty i štíty krále Davida, které byly v Božím domě.
#23:10 Postavil všechen lid, každého s oštěpem v ruce, od pravé strany domu k levé straně domu, při oltáři i při domě, aby byli kolem krále.
#23:11 Králova syna vyvedli, vložili na něj královskou čelenku a předali mu Hospodinovo svědectví. Dosadili ho za krále, Jójada a jeho synové ho pomazali a provolávali: „Ať žije král!“
#23:12 Když Atalja uslyšela hlas lidu, který se sběhl a oslavoval krále, přišla k lidu do Hospodinova domu.
#23:13 Podívala se, a hle, král stojí na svém stanovišti při vchodu, velitelé a trubači u krále, všechen lid země se raduje a troubí na trubky, jsou tu zpěváci s hudebními nástroji i hlasatelé oslavující krále. Atalja roztrhla své roucho a zvolala: „Spiknutí! Spiknutí!“
#23:14 Kněz Jójada poručil, aby vyšli velitelé setnin, ustanovení nad vojskem. Řekl jim: „Vyveďte ji středem oddílů. Kdo by šel za ní, bude usmrcen mečem.“ Kněz totiž řekl: „neusmrcujte ji v domě Hospodinově!“
#23:15 Přinutili ji, aby šla ke vchodu Koňské brány u královského domu, a tam ji usmrtili.
#23:16 Jójada pak uzavřel smlouvu mezi sebou, vším lidem a králem, že budou lidem Hospodinovým.
#23:17 Všechen lid přišel k Baalovu domu a strhli jej; jeho oltáře i jeho obrazy roztříštili a Baalova kněze Matána zabili před oltáři.
#23:18 Nad Hospodinovým domem ustanovil Jójada dohled lévijských kněží, jak je David rozdělil pro službu domu Hospodinova. Přinášeli Hospodinu zápalné oběti s radostí a zpěvem, jak je psáno v zákoně Mojžíšově, podle pokynů Davidových.
#23:19 K branám Hospodinova domu postavil vrátné, aby nemohl vstoupit nikdo, kdo byl jakkoli nečistý.
#23:20 Potom vzal velitele setnin, urozené a mocné z lidu i všechen lid země a vedl krále dolů z Hospodinova domu. Středem Horní brány přišli do královského domu a posadili krále na královský trůn.
#23:21 Všechen lid země se radoval a v městě nastal klid. - Atalju usmrtili mečem. 
#24:1 Jóašovi bylo sedm let, když začal kralovat, a kraloval v Jeruzalémě čtyřicet let. Jeho matka se jmenovala Sibja a byla z Beer-šeby.
#24:2 Jóaš činil to, co je správné v Hospodinových očích, po všechny dny kněze Jójady.
#24:3 Jójada mu opatřil dvě ženy; i zplodil syny a dcery.
#24:4 Potom si Jóaš předsevzal obnovit Hospodinův dům.
#24:5 Shromáždil kněze a lévijce a rozkázal jim: „Vyjděte do judských měst a vybírejte ode všeho Izraele každoročně stříbro na opravu domu vašeho Boha. A pospěšte si s tou věcí.“ Lévijci však nepospíchali.
#24:6 Král předvolal Jójadu, jejich hlavu, a otázal se ho: „Proč ses nestaral, aby lévijci přinášeli z Judska a Jeruzaléma poplatek pro stan svědectví, uložený Izraeli Mojžíšem, služebníkem Hospodinovým, a celým shromážděním?
#24:7 Vždyť ta svévolná Atalja a její synové násilím vnikli do Božího domu a všechny svaté předměty Hospodinova domu věnovali baalům.“
#24:8 Král poručil, aby zhotovili truhlu, kterou dali do brány Hospodinova domu při její vnější straně.
#24:9 V Judsku i v Jeruzalémě vyhlásili, aby byl Hospodinu přinesen poplatek uložený Izraeli na poušti Mojžíšem, služebníkem Božím.
#24:10 Všichni velitelé i všechen lid s radostí přinášeli poplatky a házeli je do truhly, až byla plná;
#24:11 potom byla truhla předána prostřednictvím lévijců královské správě. Když viděli, že je v truhle hodně stříbra, přišel králův písař a dohlížitel hlavního kněze, truhlu vyprázdnili a odnesli ji zpátky na její místo. Tak to dělali den co den a sebrali množství stříbra.
#24:12 Král i Jójada je vydávali tomu, kdo pracoval na služebném díle při Hospodinově domě. Byli tu dělníci, kameníci a jiní řemeslníci, kteří obnovovali Hospodinův dům, i ti, kteří opracovávali železo a měď na opravu Hospodinova domu.
#24:13 Kdo konali to dílo, dali se do práce a dílo jim rostlo pod rukama. Uvedli Boží dům do původního stavu a zpevnili jej.
#24:14 Když byli hotovi, přinesli před krále a Jójadu stříbro, jež zůstalo. Ten z něho dal udělat pro Hospodinův dům různé nádoby, nádoby pro službu i pro obětování, misky, nádoby zlaté a stříbrné. V Hospodinově domě přinášeli pravidelně oběti po všechny Jójadovy dny.
#24:15 Jójada byl stár a sytý dnů, když zemřel. Když umíral, bylo mu sto třicet let.
#24:16 Pohřbili ho v Městě Davidově s králi, neboť konal dobro v Izraeli i před Bohem a v jeho domě.
#24:17 Po Jójadově smrti přišli judští velmožové, poklonili se před králem a král je vyslechl.
#24:18 Opustili dům Hospodina, Boha svých otců, a začali sloužit posvátným kůlům a modlářským stvůrám. Pro toto provinění postihlo Judu a Jeruzalém Boží rozlícení.
#24:19 Posílal k nim proroky, aby je přivedli zpět k Hospodinu. Ti je varovali, ale oni neposlouchali.
#24:20 Duch Boží vyzbrojil Zekarjáše, syna kněze Jójady. Postavil se proti lidu a řekl jim: „Toto praví Bůh: Proč přestupujete Hospodinovy příkazy? Nepotkáte se se zdarem. Poněvadž jste opustili Hospodina, opustí on vás.“
#24:21 Spikli se proti němu a na králův příkaz ho na nádvoří Hospodinova domu ukamenovali.
#24:22 Král Jóaš nebyl pamětliv milosrdenství, které mu prokázal Jójada, jenž mu byl otcem. Zavraždil jeho syna, který umíraje řekl: „Ať to Hospodin vidí a volá k odpovědnosti.“
#24:23 Na přelomu roku vytáhlo proti Jóašovi aramejské vojsko, vtrhlo do Judska a Jeruzaléma, vyhladilo v lidu všechny jeho velitele a celou kořist poslali králi do Damašku.
#24:24 Ačkoliv aramejské vojsko přitáhlo v malém počtu mužů, Hospodin jim vydal do rukou mnohem větší vojsko, poněvadž Juda opustil Hospodina, Boha svých otců. Tak vykonali na Jóašovi soud.
#24:25 Když od něho odtáhli a opustili ho těžce nemocného, spikli se proti němu jeho služebníci pro prolitou krev synů kněze Jójady a zavraždili ho na jeho lůžku; tak zemřel. Pohřbili ho v Městě Davidově, ale nepohřbili ho v královských hrobech.
#24:26 Spikli se proti němu tito: Zábad, syn Amónky Šimeáty, a Józabad, syn Moábky Šimríty.
#24:27 O jeho synech, o množství jím uložených daní i o znovuzřízení domu Božího se dále píše ve výkladu Knihy králů. Po něm se stal králem jeho syn Amasjáš. 
#25:1 Amasjáš se stal králem v pětadvaceti letech a kraloval v Jeruzalémě dvacet devět let. Jeho matka se jmenovala Jóadan a byla z Jeruzaléma.
#25:2 Činil to, co je správné v Hospodinových očích, ne však celým srdcem.
#25:3 Jakmile bylo království pevně v jeho rukou, zavraždil ze svých služebníků ty, kteří ubili krále, jeho otce.
#25:4 Ale jejich syny neusmrtil, neboť je napsáno v Zákoně, v Knize Mojžíšově, že Hospodin přikázal: „Otcové nezemřou za syny a synové nezemřou za otce, nýbrž každý zemře za svůj hřích.“
#25:5 Amasjáš shromáždil Judu a ustanovil pro celého Judu a Benjamína podle otcovských rodů velitele nad tisíci a velitele nad sty. Dvacetileté a starší sečetl a shledal, že jich je tři sta tisíc vybraných mužů, schopných vycházet do boje, zacházet s oštěpem a pavézou.
#25:6 Z Izraele najal sto tisíc udatných bohatýrů za sto talentů stříbra.
#25:7 Tu k němu přišel muž Boží a řekl: „Králi, ať s tebou netáhne vojenský zástup Izraele, poněvadž Hospodin není s Izraelem, s nikým z Efrajimovců.
#25:8 Ale táhni sám a pusť se rozhodně do boje, jinak Bůh přivodí před tváří nepřítele tvůj pád. Bůh má dost síly, aby pomohl i přivodil pád.“
#25:9 Amasjáš řekl muži Božímu: „Co však udělat s tím stem talentů, které jsem dal houfu z Izraele?“ Muž Boží odpověděl: „Hospodin ti může dát mnohokrát víc než tohle.“
#25:10 Amasjáš je tedy oddělil, totiž houf, který k němu přišel z Efrajimu, aby šli zpět do svých domovů. Ti vzplanuli proti Judovi velikým hněvem a velice rozhněváni se vraceli do svých domovů.
#25:11 Amasjáš se vzchopil, vedl svůj lid, přitáhl do Solného údolí a pobil Seírovce, deset tisíc mužů.
#25:12 Z těch, co zůstali naživu, zajali Judovci deset tisíc, dovedli je na vrchol skály a z vrcholku té skály je svrhli; všichni se roztříštili.
#25:13 Mezitím lidé z houfu, který Amasjáš přiměl k návratu, aby s ním netáhl do bitvy, vpadli do judských měst od Samaří až do Bét-chorónu. Pobili v nich tři tisíce mužů a nabrali mnoho loupeže.
#25:14 Když Amasjáš přitáhl po vítězství nad Edómci, přinesl bohy Seírovců a přijal je za bohy; klaněl se před nimi a pálil jim kadidlo.
#25:15 Hospodin vzplanul proti Amasjášovi hněvem a poslal k němu proroka. Ten mu řekl: „Proč se dotazuješ bohů lidu, kteří svůj lid z tvých rukou nevysvobodili?“
#25:16 Když k němu takto mluvil, řekl mu Amasjáš: „Což tě udělali královým poradcem? Přestaň už, proč tě mají ubít?“ Prorok tedy přestal, jen dodal: „Poznal jsem, že Bůh se rozhodl uvalit na tebe zkázu, protože jsi to učinil a mou radu neposloucháš!“
#25:17 Judský král Amasjáš se poradil a vzkázal izraelskému králi Jóašovi, synu Jóachaza, syna Jehúova: „Pojď, utkáme se!“
#25:18 Jóaš, král izraelský, poslal Amasjášovi, králi judskému, odpověď: „Na Libanónu vzkázalo trní libanónskému cedru: ‚Dej svou dceru za ženu mému synovi.‘ Vtom tudy přešlo libanónské polní zvíře a to trní rozšlapalo.
#25:19 Řekl sis: ‚Hle, pobil jsem Edómce.‘ Proto se tvé srdce tak vypíná, toužíš po slávě. Teď však seď doma. Proč si zahráváš se zlem? Abys padl ty i Juda s tebou?“
#25:20 Ale Amasjáš neposlechl. Bylo to od Boha, že je vydá do rukou Jóašovi, protože se dotazovali bohů Edómu.
#25:21 Jóaš, král izraelský, vytáhl a utkali se, on a Amasjáš, král judský, u Bét-šemeše, jenž patřil Judovi.
#25:22 Juda byl před tváří Izraele poražen; každý utíkal ke svému stanu.
#25:23 Judského krále Amasjáše, syna Jóaše, syna Jóachazova, izraelský král Jóaš v Bét-šemeši zajal a přivlekl ho do Jeruzaléma. Prolomil jeruzalémské hradby od Efrajimské brány až k bráně Nárožní v délce čtyř set loket.
#25:24 Pobral všechno zlato a stříbro a všechno náčiní, které se nacházelo v Božím domě u Obéd-edóma, i poklady domu královského a rukojmí a vrátil se do Samaří.
#25:25 Amasjáš, syn Jóašův, král judský, žil po smrti izraelského krále Jóaše, syna Jóachazova, ještě patnáct let.
#25:26 O ostatních příbězích Amasjášových, prvních i posledních, se dále píše v Knize králů judských a izraelských.
#25:27 Od chvíle, kdy se Amasjáš odvrátil do Hospodina, osnovali proti němu v Jeruzalémě spiknutí. Utekl do Lakíše. Ale poslali za ním do Lakíše vrahy a usmrtili ho tam.
#25:28 Potom ho převezli na koních a pohřbili ho vedle jeho otců v městě Judově. 
#26:1 Všechen judský lid vzal Uzijáše, kterému bylo šestnáct let, a dosadil ho za krále po jeho otci Amasjášovi.
#26:2 On vystavěl Elót a navrátil jej Judovi, poté co král Amasjáš ulehl ke svým otcům.
#26:3 Uzijášovi bylo šestnáct let, když začal kralovat, a kraloval v Jeruzalémě dvaapadesát let. Jeho matka se jmenovala Jekolja a byla z Jeruzaléma.
#26:4 Činil to, co je správné v Hospodinových očích, zcela jak to činil jeho otec Amasjáš.
#26:5 Dotazoval se Boha za dnů Zekarjáše, jenž rozuměl Božímu vidění. Ve dnech, kdy se dotazoval Hospodina, provázel ho Bůh zdarem.
#26:6 Vytáhl a válčil s Pelištejci a strhl hradby města Gatu, hradby Jabne a hradby Ašdódu. Vystavěl města kolem Ašdódu na území Pelištejců.
#26:7 Bůh mu pomáhal proti Pelištejcům, proti Arabům sídlícím v Gúr-baalu a Meúnejcům.
#26:8 Amónci odevzdávali Uzijášovi dary. Jeho jméno proniklo až k branám Egypta, neboť velice upevnil svou moc.
#26:9 Uzijáš vystavěl v Jeruzalémě věže, a to nad Nárožní branou, nad Údolní branou a při rohu hradeb, a opevnil je.
#26:10 Vystavěl věže i ve stepi a vyhloubil tam mnoho cisteren. Měl mnoho stád v Přímořské nížině a na rovině, oráče a vinaře na horách a na vinohradech. Miloval totiž půdu.
#26:11 Uzijáš měl rovněž vojsko vycvičené k boji, schopné vycházet po houfech do boje podle seznamu povolaných do služby, pořízeného písařem Jeíelem a dozorcem Maasejášem pod dohledem Chananjáše z králových velitelů.
#26:12 Celkový počet představitelů otcovských rodů byl dva tisíce šest set udatných bohatýrů.
#26:13 Pod jejich dohledem byly vojenské oddíly čítající tři sta sedm tisíc pět set mužů velké vojenské síly, vycvičených k boji, aby pomáhali králi proti nepříteli.
#26:14 Uzijáš opatřil pro všechny oddíly štíty, oštěpy, přilby, pancíře, luky a praky na vrhání kamenů.
#26:15 V Jeruzalémě zhotovil důmyslně vymyšlené válečné stroje; ty byly na věžích a cimbuřích k vrhání střel a velkých kamenů. Jeho jméno se rozneslo do daleka, neboť se mu dostalo podivuhodné pomoci, takže upevnil svou moc.
#26:16 Jakmile svou moc upevnil, jeho srdce se stalo domýšlivým, až se úplně zkazil a zpronevěřil Hospodinu, svému Bohu. Vstoupil do Hospodinova chrámu, aby pálil kadidlo na kadidlovém oltáři.
#26:17 Tu za ním vstoupil kněz Azarjáš a s ním osmdesát Hospodinových kněží, statečných mužů.
#26:18 Ti se postavili proti králi Uzijášovi a řekli mu: „Tobě, Uzijáši, nepřísluší pálit kadidlo Hospodinu, neboť to je záležitost kněží, synů Áronových, posvěcených k tomu, aby pálili kadidlo. Odejdi ze svatyně, poněvadž ses zpronevěřil. To ti u Hospodina Boha k slávě nebude.“
#26:19 Uzijáš se rozběsnil. V ruce měl kadidelnici, aby pálil kadidlo. Jakmile se vůči kněžím rozběsnil, vyrazilo se mu na čele malomocenství před očima kněží v Hospodinově domě u kadidlového oltáře.
#26:20 Azarjáš, hlavní kněz, i ostatní kněží se k němu obrátili, a hle, byl na čele malomocný. S hrůzou ho odtud vykázali. I on sám se snažil spěšně vyjít, neboť Hospodin ho ranil.
#26:21 Král Uzijáš byl malomocný až do své smrti. Jako malomocný bydlel v odděleném domě, poněvadž byl vyobcován z Hospodinova domu. Jeho syn Jótam byl správcem královského domu a soudil lid země.
#26:22 O ostatních příbězích Uzijášových, prvních i posledních, píše Izajáš, syn Amósův, prorok.
#26:23 I ulehl Uzijáš ke svým otcům a pohřbili ho vedle jeho otců na poli u pohřebiště králů. Řekli totiž: „Byl malomocný.“ Po něm kraloval jeho syn Jótam. 
#27:1 Jótamovi bylo dvacet pět let, když začal kralovat, a kraloval v Jeruzalémě šestnáct let. Jeho matka se jmenovala Jerúša; byla to dcera Sádokova.
#27:2 Činil to, co je správné v Hospodinových očích, zcela jak to činil jeho otec Uzijáš. Avšak do Hospodinova chrámu nevstoupil. Lid dál propadal zkáze.
#27:3 Jótam vystavěl Horní bránu Hospodinova domu; také mnoho stavěl na hradbě Ófelu.
#27:4 V judském pohoří vystavěl města a v lesnatých krajích stavěl hrady a věže.
#27:5 Válčil s králem Amónovců a přemohl je. Amónovci mu dodali toho roku sto talentů stříbra a deset tisíc kórů pšenice a deset tisíc ječmene. To mu odváděli Amónovci i ve druhém a třetím roce.
#27:6 Jótam pevně vládl, poněvadž setrvával na svých cestách před Hospodinem, svým Bohem.
#27:7 O ostatních příbězích Jótamových, o všech jeho válkách a cestách, se dále píše v Knize králů izraelských a judských.
#27:8 Bylo mu dvacet pět let, když začal kralovat, a kraloval v Jeruzalémě šestnáct let.
#27:9 I ulehl Jótam ke svým otcům a pohřbili ho v Městě Davidově. Po něm se stal králem jeho syn Achaz. 
#28:1 Achazovi bylo dvacet let, když začal kralovat, a kraloval v Jeruzalémě šestnáct let. Nečinil, co je správné v Hospodinových očích, jako činil jeho otec David.
#28:2 Chodil po cestě králů izraelských. Dokonce odlil sochy baalů.
#28:3 Pálil kadidlo v Údolí syna Hinómova a spaloval své syny ohněm podle ohavností pronárodů, které Hospodin před Izraelci vyhnal.
#28:4 Obětoval a pálil kadidlo na posvátných návrších a na pahorcích a pod každým zeleným stromem.
#28:5 Proto ho vydal Hospodin, jeho Bůh, do rukou aramejského krále. Aramejci ho porazili a zajali z jeho vojska velký počet zajatců a odvlekli je do Damašku. Byl vydán též do rukou krále izraelského, který mu způsobil velikou porážku.
#28:6 Pekach, syn Remaljášův, povraždil v Judsku v jediném dni sto dvacet tisíc mužů, samé statečné muže; opustili Hospodina, Boha svých otců.
#28:7 Zikrí, efrajimský bohatýr, zavraždil králova syna Maasejáše, Azríkama, představeného domu, a Elkánu, který byl druhý po králi.
#28:8 Izraelci zajali u svých bratří dvě stě tisíc žen, synů a dcer a uloupili u nich obrovskou kořist. Kořist odvezli do Samaří.
#28:9 Tam však byl Hospodinův prorok jménem Obéd. Ten vyšel naproti oddílu přicházejícímu do Samaří. Řekl jim: „Hle, Hospodin, Bůh vašich otců, vydal ve svém rozhořčení Judu do vašich rukou. Ale vy jste vraždili tak zběsile, že to volá až do nebes.
#28:10 A teď jste si usmyslili, že Judejce a obyvatele Jeruzaléma ujařmíte jako své otroky a otrokyně. Neproviňujete se tím právě vy sami vůči Hospodinu, svému Bohu?
#28:11 Slyšte mě nyní a propusťte zajatce, které jste u svých bratří zajali. Vždyť Hospodin proti vám plane hněvem.“
#28:12 Tu povstali někteří z efrajimských představitelů, Azarjáš, syn Jóchananův, Berekjáš, syn Mešilemótův, Jechizkijáš, syn Šalúmův, a Amasa, syn Chadlajův, proti těm, kteří přišli z válečného tažení,
#28:13 a řekli jim: „Nepřivádějte sem ty zajatce! Chcete, abychom se provinili proti Hospodinu a připojili k našim hříchům a proviněním další? Vždyť našich vin je beztak mnoho a hněv Boží plane proti Izraeli.“
#28:14 Ozbrojenci se tedy v přítomnosti velitelů a celého shromáždění vzdali zajatců i lupu.
#28:15 Tu povstali muži uvedení jménem, ujali se zajatců a ze získané kořisti oblékli všechny nahé. Oblékli je a obuli, dali jim najíst a napít, ošetřili je a všechny, kdo únavou klesali, dopravili na oslech a převedli do Jericha, Palmového města, k jejich bratřím. Potom se vrátili do Samaří.
#28:16 V ten čas poslal král Achaz ke králům asyrským prosbu, aby mu poskytli pomoc.
#28:17 Edómci totiž znovu přitáhli a některé v Judsku pobili nebo zajali.
#28:18 Také Pelištejci vpadli do měst v Přímořské nížině a v jižním Judsku a dobyli Bét-šemeš, Ajalón, Gederót, Sóko s jeho vesnicemi, Timnu s jejími vesnicemi, Gimzo s jeho vesnicemi a usídlili se tam.
#28:19 Hospodin totiž pokořoval Judu kvůli Achazovi, králi izraelskému, neboť si počínal v Judsku bezohledně a zpronevěřoval se Hospodinu.
#28:20 I přitáhl k němu Tiglat-pileser, král asyrský, ale jeho moc neupevnil, jen ho sužoval.
#28:21 Ačkoliv Achaz vydrancoval Hospodinův dům i dům královský a domy velmožů a dal všechno asyrskému králi, nepomohlo mu to.
#28:22 I v čase soužení se dále zpronevěřoval Hospodinu. Takový byl král Achaz.
#28:23 Obětoval damašským bohům, kteří způsobili jeho porážku. Řekl: „Bohové aramejských králů, ti jim pomáhají; budu jim obětovat a pomohou i mně.“ Ale oni přivodili pád jemu i celému Izraeli.
#28:24 Achaz tedy sebral nádoby Božího domu, osekal v Božím domě zlaté předměty, dveře Hospodinova domu uzavřel a udělal si v Jeruzalémě oltáře na každém nároží.
#28:25 V každém judském městě udělal posvátná návrší, aby tam pálil kadidlo jiným bohům; tím urážel Hospodina, Boha svých otců.
#28:26 O ostatních jeho příbězích, o všech jeho cestách, prvních i posledních, se dále píše v Knize králů judských a izraelských.
#28:27 I ulehl Achaz ke svým otcům a pohřbili ho v městě, v Jeruzalémě. Ale nevnesli ho do hrobů králů izraelských. Po něm se stal králem jeho syn Chizjijáš. 
#29:1 Chizkijášovi bylo dvacet pět let, když začal kralovat, a kraloval v Jeruzalémě dvacet devět let. Jeho matka se jmenovala Abija; byla to dcera Zekarjášova.
#29:2 Činil to, co je správné v Hospodinových očích, zcela jak to činil jeho otec David.
#29:3 V prvním měsíci prvního roku svého kralování otevřel dveře Hospodinova domu a opravil je.
#29:4 Přivedl kněze a lévijce a shromáždil je na východní straně prostranství.
#29:5 Řekl jim: „Slyšte mě, lévijci! Teď se posvěťte, posvěťte i dům Hospodina, Boha svých otců, a vyneste ze svatyně, co je nečisté.
#29:6 Naši otcové se zpronevěřili, činili to, co je zlé v očích Hospodina, našeho Boha, a opustili ho, odvrátili svou tvář od Hospodinova příbytku, obrátili se k němu zády.
#29:7 Dokonce zavřeli dveře předsíně, zhasili kahánky, nepálili kadidlo a neobětovali ve svatyni Boha Izraele zápalné oběti.
#29:8 Proto Judu a Jeruzalém postihlo Hospodinovo rozlícení a učinil je obrazem hrůzy a předmětem úděsu a pošklebků, jak to vidíte na vlastní oči.
#29:9 Právě proto naši otcové padli mečem, naši synové a naše dcery i naše ženy upadli do zajetí.
#29:10 Teď však mám v úmyslu uzavřít smlouvu s Hospodinem, Bohem Izraele, aby od nás odvrátil svůj planoucí hněv.
#29:11 Moji synové, nebuďte teď liknaví, vždyť vás Hospodin vyvolil, abyste stáli v jeho službách, byli jeho sluhy a pálili mu kadidlo.“
#29:12 Lévijci povstali: Z Kehatovců Machat, syn Amasajův, a Jóel, syn Azarjášův, z Meraríovců Kíš, syn Abdíův, a Azarjáš, syn Jehalelelův; z Geršónovců Jóach, syn Zimův, a Eden, syn Jóachův.
#29:13 Z Elísáfanovců Šimrí a Jeíel, z Asafovců Zekarjáš a Matanjáš,
#29:14 z Hémanovců Jechíel a Šimeí a z Jedútúnovců Šemajáš a Uzíel.
#29:15 Ti shromáždili své bratry, posvětili se podle králova rozkazu na základě Hospodinových slov přišli očistit Hospodinův dům.
#29:16 Kněží vešli dovnitř, do Hospodinova domu, aby jej očistili. Vynesli všechnu nečistotu, kterou našli v Hospodinově chrámu, do nádvoří Hospodinova domu a lévijci to od nich přejímali a vynášeli ven do Kidrónského úvalu.
#29:17 S posvěcováním začali první den prvního měsíce a osmý den téhož měsíce vešli do Hospodinovy předsíně; Hospodinův dům posvěcovali dalších osm dní; šestnáctý den prvního měsíce skončili.
#29:18 Přišli dovnitř ke králi Chizkijášovi a hlásili: „Očistili jsme celý Hospodinův dům, oltář pro zápalné oběti s veškerým příslušenstvím i stůl pro předkladné chleby s veškerým příslušenstvím.
#29:19 Všechny předměty, které Achaz za svého kralování ve své zpronevěře odvrhl, jsme dali na místo a oddělili jako svaté. Hle, jsou před Hospodinovým oltářem.“
#29:20 Král Chizkijáš shromáždil za časného jitra velmože města a vystoupil k Hospodinovu domu.
#29:21 Přivedli sedm býčků, sedm beranů, sedm beránků a sedm kozlů jako oběť za hřích: za království, za svatyni a za Judu. Kněžím Áronovcům nařídil obětovat zápalnou oběť na Hospodinově oltáři.
#29:22 Poráželi skot a kněží chytali krev a kropili směrem k oltáři, poráželi berany a krví kropili směrem k oltáři, poráželi beránky a krví kropili směrem k oltáři.
#29:23 Pak přivedli před krále a před shromáždění kozly jako oběť za hřích. Položili na ně ruce,
#29:24 kněží je porazili a jejich krví očistili oltář od hříchu. Tak vykonali smírčí obřady za celý Izrael, neboť král celému Izraeli nařídil obětovat oběť zápalnou i oběť za hřích.
#29:25 Postavil v Hospodinově domě lévijce s cymbály, harfami a citarami podle příkazů Davida a Gáda, králova vidoucího, a Nátana, proroka. Ten příkaz totiž vydal Hospodin skrze své proroky.
#29:26 I stáli lévijci s Davidovými nástroji a kněží s trubkami.
#29:27 Chizkijáš nařídil, aby na oltáři obětovali zápalné oběti. Ve chvíli, kdy se začalo se zápalnou obětí, začal i zpěv k poctě Hospodinu za doprovodu trubek a nástrojů Davida, krále izraelského.
#29:28 Celé shromáždění se klanělo, zněl zpěv a hlaholily trubky. To všechno trvalo, dokud neskončila zápalná oběť.
#29:29 Když obětování skončilo, poklekl král se všemi, kteří byli s ním, a klaněli se.
#29:30 Král Chizkijáš s velmoži nařídil lévijcům, aby chválili Hospodina slovy Davida a Asafa, vidoucího. Chválili ho plni radosti, padli na kolena a klaněli se.
#29:31 Potom opět Chizkijáš nařídil: „Protože jste uvedeni v kněžský úřad pro Hospodina, přistupte a přineste do Hospodinova domu dary k obětním hodům a oběti děkovné.“ Shromáždění přineslo dary k obětním hodům a oběti děkovné a každý, kdo měl k tomu ochotné srdce, i oběti zápalné.
#29:32 Počet zápalných obětí, které shromáždění přineslo, činil sedmdesát kusů skotu, sto beranů a dvě stě beránků, vše pro zápalnou oběť Hospodinu.
#29:33 Jiných svatých darů bylo šest set kusů skotu a tři tisíce kusů bravu.
#29:34 Ale kněží bylo málo, takže nestačili ze všech zápalných obětí stahovat kůže. Proto jim pomáhali jejich bratří lévijci, dokud nebylo dílo skončeno a dokud se neposvětili všichni kněží; lévijci se totiž posvěcovali pohotověji než kněží.
#29:35 A také zápalných obětí z tučných dílů obětí pokojných a úliteb k zápalné oběti bylo velmi mnoho. Tak byla obnovena služba v Hospodinově domě.
#29:36 Chizkijáš se vším lidem se radoval z toho, co Bůh lidu připravil, neboť vše se sběhlo nečekaně. 
#30:1 Potom obeslal Chizkijáš celý Izrael a Judu a napsal listy Efrajimovi a Manasesovi, aby přišli do Hospodinova domu v Jeruzalémě a slavili hod beránka Hospodinu, Bohu Izraele.
#30:2 Král se dohodl se svými velmoži i s celým shromážděním v Jeruzalémě, že budou hod beránka slavit v druhém měsíci.
#30:3 Nemohli jej totiž slavit v určený čas, protože se neposvětilo dost kněží a lid se nestačil shromáždit do Jeruzaléma.
#30:4 Král i celé shromáždění to pokládali za správné.
#30:5 Usnesli se, že dají v celém Izraeli, od Beer-šeby až do Danu, provolat, aby všichni přišli do Jeruzaléma slavit hod beránka Hospodinu, Bohu Izraele, neboť mnozí jej neslavili tak, jak bylo předepsáno.
#30:6 Běžci prošli s listy od krále a od jeho velmožů celým Izraelem a Judou a podle králova příkazu provolávali: „Izraelci, navraťte se k Hospodinu, Bohu Abrahama, Izáka a Izraele, a on se navrátí k těm pozůstalým z vás, kteří vyvázli z moci asyrských králů.
#30:7 Nebuďte jako vaši otcové a jako vaši bratří, kteří se Hospodinu, Bohu svých otců, zpronevěřili. Proto dopustil, že vzbuzují úžas, jak sami vidíte.
#30:8 Nyní nebuďte tvrdošíjní jako vaši otcové, poddejte se Hospodinu, přijďte k jeho svatyni, kterou navěky oddělil jako svatou. Služte Hospodinu, svému Bohu, a jeho planoucí hněv se od vás odvrátí.
#30:9 Když se obrátíte k Hospodinu, dojdou vaši bratří a vaši synové slitování u těch, kteří je zajali, a navrátí se do této země. Vždyť Hospodin, váš Bůh, je milostivý a slitovný a svou tvář neodvrátí od vás, když se k němu navrátíte.“
#30:10 Běžci procházeli zemí Efrajimovou a Manasesovou od města k městu až do Zabulónu. Všude se jim posmívali a zesměšňovali je.
#30:11 Jen někteří z kmene Ašerova, Manasesova a Zabulónova přišli v pokoře do Jeruzaléma.
#30:12 Avšak v Judsku způsobil Bůh svou mocí, že jim dal jednomyslnost, takže plnili příkaz krále a velmožů podle Hospodinova slova.
#30:13 Do Jeruzaléma se shromáždilo množství lidu, aby v druhém měsíci slavili svátek nekvašených chlebů; shromáždění bylo velmi početné.
#30:14 Začali odstraňovat oltáře, které byly v Jeruzalémě; odstranili i všechna vykuřovadla a vhodili je do Kidrónského úvalu.
#30:15 Čtrnáctého dne druhého měsíce zabíjeli velikonočního beránka. Kněží a lévijci se se zahanbením posvětili a přinesli do Hospodinova domu zápalné oběti.
#30:16 Stáli na svých stanovištích podle příslušných řádů, podle zákona Mojžíše, muže Božího. Kněží kropili krví, kterou brali od lévijců.
#30:17 Protože ve shromáždění bylo mnoho těch, kdo se neposvětili, lévijci, kteří zabíjeli velikonoční beránky, dbali, aby každého, kdo nebyl čistý, posvětili Hospodinu.
#30:18 Velká část lidu, mnozí z kmenů Efrajimova, Manasesova, Isacharova a Zabulónova, se totiž neočistili. Jedli velikonočního beránka, ale ne tak, jak bylo předepsáno. Chizkijáš se za ně modlil: „Dobrotivý Hospodin nechť zprostí viny
#30:19 každého, kdo se upřímným srdcem dotazoval na slovo Boha Hospodina, Boha svých otců, i když neprošli očišťováním ve svatyni.“
#30:20 Hospodin Chizkijáše vyslyšel a lid uzdravil.
#30:21 I slavili Izraelci, kteří se sešli do Jeruzaléma, slavnost nekvašených chlebů s velikou radostí po sedm dní a lévijci i kněží chválili Hospodina den co den, chválili na nástrojích Hospodinovu moc.
#30:22 Chizkijáš promluvil k srdci všech lévijců, kteří s mocným zaujetím sloužili Hospodinu. Připravovali hody pokojných obětí, po sedm dní slavnostního shromáždění jedli a pěli chválu Hospodinu, Bohu svých otců.
#30:23 I usneslo se celé shromáždění, že budou pokračovat ve slavnosti ještě dalších sedm dní. A radostně pokračovali po sedm dní.
#30:24 Chizkijáš, král judský, věnoval pro shromáždění tisíc býčků a sedm tisíc ovcí. Také velmožové věnovali pro shromáždění tisíc býčků a deset tisíc ovcí. Množství kněží se posvětilo.
#30:25 Celé shromáždění judské se radovalo, kněží a lévijci i celé shromáždění, všichni, kteří přišli z Izraele, i ti, kteří dříve přišli ze země izraelské a žili pohostinu v Judsku.
#30:26 V Jeruzalémě byla veliká radost, jaká v Jeruzalémě nebyla ode dnů izraelského krále Šalomouna, syna Davidova.
#30:27 Potom lévijští kněží povstali a lidu udělili požehnání. Jejich hlas byl vyslyšen a jejich modlitba vešla až do jeho svatého příbytku, do nebe. 
#31:1 Když to všechno skončilo, všechen Izrael, ti, kteří se sešli, vyšli do judských měst a roztříštili posvátné sloupy, pokáceli posvátné kůly a pobořili posvátná návrší a oltáře v celém Judsku a Benjamínsku i na území Efrajimově a Manasesově, a to dokonale. Pak se všichni Izraelci vrátili do svých měst, každý do svého území.
#31:2 Chizkijáš zařadil kněze a lévijce do jednotlivých tříd, každého podle přidělené kněžské a lévijské služby, jednak při obětech zápalných a pokojných, jednak při přisluhování, děkování a chválení v branách Hospodinových táborů.
#31:3 Král přispěl ze svého majetku na zápalné oběti, aby byly obětovány ráno i večer, na zápalné oběti ve dnech odpočinku, o novoluních a při slavnostech, jak je předepsáno v Hospodinově zákoně.
#31:4 Také nařídil lidu, obyvatelům Jeruzaléma, aby dávali dary kněžím a lévijcům, aby byli v Hospodinově zákoně pevní.
#31:5 Jak se ta výzva rozšířila, Izraelci přinášeli množství prvotin obilí, moštu, čerstvého oleje, medu i všeho, co se urodilo na poli. Též desátky ze všeho přinášeli v hojnosti.
#31:6 Také Izraelci a Judejci, kteří bydleli v městech judských, odváděli desátky ze skotu a bravu i desátky ze svatých darů, zasvěcených Hospodinu, jejich Bohu. Přinášeli a dávali je na hromady.
#31:7 Začali vše klást na hromady třetího měsíce a skončili sedmého měsíce.
#31:8 Když Chizkijáš a velmožové přišli a spatřili hromady, dobrořečili Hospodinu i Izraeli, jeho lidu.
#31:9 Chizkijáš se kněží a lévijců vyptával na ty hromady.
#31:10 Azarjáš, hlavní kněz z domu Sádokova, mu odpověděl: „Od chvíle, kdy začali přinášet oběť pozdvihování do Hospodinova domu, je jídla do sytosti a ještě hodně zůstává, neboť Hospodin svému lidu požehnal. Tak mnoho toho zůstalo.“
#31:11 Chizkijáš proto nařídil připravit v Hospodinově domě komory. Připravili je
#31:12 a pak tam věrně vnášeli oběť pozdvihování, desátek i svaté dary. Nad nimi jako představený byl lévijec Kónanjáš a jeho bratr Šimeí byl jeho zástupce.
#31:13 Jechíel, Azazjáš, Nachat, Asáel, Jerimót, Józabad, Elíel, Jismakjáš, Machat a Benajáš byli dohlížiteli pod velením Kónanjáše a jeho bratra Šimeího podle ustanovení krále Chizkijáše a Azarjáše, představeného domu Božího.
#31:14 Lévijec Kóre, syn Jimny, vrátný u východní brány, byl nad dobrovolnými Božími dary, aby vydával, co pocházelo z Hospodinovy oběti pozdvihování a z velesvatých darů.
#31:15 Jemu k ruce byli Eden, Minjamín, Jéšua, Šemajáš, Amarjáš a Šekanjáš v městech kněžských, aby svědomitě z toho vydávali svým bratřím podle tříd, jak velkému tak malému.
#31:16 Kromě toho i těm mužského pohlaví od tříletých výše, kteří byli v seznamech, každému, kdo přicházel do Hospodinova domu plnit denní úkol svých služebních povinností podle jednotlivých tříd.
#31:17 Kněžské seznamy se pořizovaly podle otcovských rodů, rovněž lévijské, od dvacetiletých výše podle jejich povinností v jednotlivých třídách.
#31:18 Seznamy platily i pro všechny jejich děti, ženy, syny a dcery, pro celé shromáždění, pokud se věrně posvětili, aby byli svatí.
#31:19 Kněží Áronovci, žijící z polí a pastvin svých měst, měli v každém městě jménem uvedené muže, kteří vydávali podíly každému z kněží a každému z lévijců, kdo byl v seznamu.
#31:20 Tak to učinil Chizkijáš v celém Judsku. Činil, co bylo dobré, správné a pravdivé před Hospodinem, jeho Bohem.
#31:21 Při celém, díle, které začal pro službu v domě Božím a pro zákon a přikázání, dotazoval se svého Boha; činil je celým svým srdcem a dílo se mu dařilo. 
#32:1 Po těchto událostech a skutcích věrnosti přitáhl Sancheríb, král asyrský. Přitáhl do Judska, oblehl opevněná města a nařídil, aby mu prolomili hradby.
#32:2 Když uviděl Chizkijáš, že Sancheríb přitáhl a že se chystá do boje proti Jeruzalému,
#32:3 dohodl se s velmoži a bohatýry, že zasypou vodní prameny, které byly vně za městem, a ti mu poskytli pomoc.
#32:4 Shromáždilo se množství lidu a zasypali všechny prameny i potok, který protékal středem země. Řekli: „Proč mají asyrští králové najít tolik vody, až přitáhnou?“
#32:5 Král vládl pevně. Dal se do přestavby celé hradební zdi, která byla samá trhlina, a vyhnal ji až po věže. Zvenčí vystavěl další hradební zeď. Opravil též Miló v Městě Davidově a pořídil množství oštěpů a štítů.
#32:6 Nad lidem ustanovil vojenské velitele a shromáždil je k sobě na prostranství v bráně města. Promluvil jim k srdci:
#32:7 „Buďte rozhodní a udatní, nebojte se a neděste se asyrského krále ani toho hlučícího davu, který je s ním. S námi je někdo větší než s ním.
#32:8 S ním je paže lidská, ale s námi je Hospodin, náš Bůh, aby nám pomohl a vedl naše boje.“ Lid se spolehl na slova Chizkijáše, krále judského.
#32:9 Potom Sancheríb, král asyrský, když s celou svou brannou mocí stál proti Lakíši, poslal své služebníky do Jeruzaléma k Chizkijášovi, králi judskému, a k celému Judovi, který byl v Jeruzalémě, se vzkazem:
#32:10 „Toto praví Sancheríb, král asyrský: Na co spoléháte, že chcete zůstat v obleženém Jeruzalémě?
#32:11 Nesvedl vás Chizkijáš? Umoří vás hladem a žízní, když říká: ‚Hospodin, náš Bůh, nás vysvobodí ze spárů asyrského krále.‘
#32:12 Cožpak ten Chizkijáš neodstranil posvátná návrší a oltáře a nenařídil Judovi a Jeruzalému: ‚Jen před jediným oltářem se budete klanět a na něm budete pálit kadidlo‘?
#32:13 Nevíte, co jsem udělal já a moji otcové všem národům zemí? Cožpak dokázali bohové pronárodů těch zemí vysvobodit svou zemi z mých rukou?
#32:14 Který ze všech bohů těch pronárodů, které moji otcové vyhubili jako klaté, dokázal svůj lid vysvobodit z mých rukou? Že by váš Bůh dokázal vysvobodit z mých rukou vás?
#32:15 Ať vás Chizkijáš nepodvádí a ať vás nesvádí; nevěřte mu! Žádný bůh žádného pronároda ani království nedokázal vysvobodit svůj lid z mých rukou ani z rukou mých otců. Ani vaši bohové vás z mých rukou nevysvobodí.“
#32:16 A ještě dál mluvili jeho služebníci proti Hospodinu Bohu a proti jeho služebníku Chizkijášovi.
#32:17 Psal také dopisy, jimiž haněl Hospodina, Boha Izraele. Říkal o něm: „Jako nevysvobodili bohové pronárodů těch zemí svůj lid z mých rukou, tak nevysvobodí ani Bůh Chizkijášův svůj lid z mých rukou.“
#32:18 Volali zplna hrdla judsky na jeruzalémský lid, který byl na hradbách, aby na ně padl strach a hrůza, aby dobyli město.
#32:19 Mluvili o Bohu Jeruzaléma jako o bozích národů země, díle lidských rukou.
#32:20 Proto se král Chizkijáš a prorok Izajáš, syn Amósův, modlili a úpěnlivě volali k nebesům.
#32:21 Hospodin poslal anděla a zahubil v táboře asyrského krále všechny bohatýry, vévody a velitele. Král musel s ostudou odtáhnout zpět do své země. Když vešel do domu svých bohů, ti, kteří vzešli z jeho lůna, jej tam srazili mečem.
#32:22 Hospodin zachránil Chizkijáše a obyvatele Jeruzaléma z rukou Sancheríba, krále asyrského, i z rukou ostatních nepřátel a zabezpečil je ze všech stran.
#32:23 Mnozí přinášeli do Jeruzaléma dary Hospodinu i vzácné dary Chizkijášovi, králi judskému. Od té chvíle byl v očích všech pronárodů vyvýšený.
#32:24 V oněch dnech Chizkijáš smrtelně onemocněl. Modlil se k Hospodinu a ten k němu promluvil a dal mu zázračné znamení.
#32:25 Ale Chizkijáš za prokázané dobrodiní nebyl vděčný, jeho srdce se stalo domýšlivým. Proto jej i Judu a Jeruzalém postihlo Hospodinovo rozlícení.
#32:26 Pak se Chizkijáš za pýchu svého srdce pokořil, on i obyvatelé Jeruzaléma, a Hospodinův hněv na ně za dnů Chizkijášových nedolehl.
#32:27 Chizkijáš měl bohatství a převelikou slávu. Nahromadil poklady stříbra, zlata, drahokamů, balzámů, štítů i všelijakých vzácných předmětů.
#32:28 Měl sklady pro úrodu obilí, pro mošt a čerstvý olej, i stáje pro dobytek všeho druhu a ve stájích stáda dobytka.
#32:29 Nastavěl si města a měl mnoho bravu i skotu, neboť mu Bůh dal velice mnoho jmění.
#32:30 Chizkijáš zasypal horní vody Gichónu a svedl je spodem přímo na západ do Města Davidova. Všechno Chizkijášovo počínání provázel zdar.
#32:31 Jenom tehdy, když k němu byli posláni vyslanci knížat babylónských, aby se dotazovali na zázračné znamení, které se stalo v té zemi, jej Bůh opustil. Tak jej podrobil zkoušce, aby poznal všechno, co je v jeho srdci.
#32:32 O ostatních příbězích Chizkijášových i o jeho zbožných činech se dále píše ve Vidění proroka Izajáše, syna Amósova, zapsaném v Knize králů judských a izraelských.
#32:33 I ulehl Chizkijáš ke svým otcům a pohřbili ho ve svahu při hrobech Davidovců. Celý Juda a obyvatelé Jeruzaléma mu při jeho smrti prokázali poctu. Po něm se stal králem jeho syn Menaše. 
#33:1 Menašemu bylo dvanáct let, když začal kralovat, a kraloval v Jeruzalémě padesát pět let.
#33:2 Dopouštěl se toho, co je zlé v Hospodinových očích, podle ohavností pronárodů, které Hospodin před Izraelci vyhnal.
#33:3 Znovu vybudoval posvátná návrší, která jeho otec Chizkijáš zbořil, nastavěl oltáře baalům a udělal posvátné kůly a klaněl se veškerému nebeskému zástupu a sloužil mu.
#33:4 Zbudoval oltáře v Hospodinově domě, o němž Hospodin řekl: „V Jeruzalémě bude navěky dlít mé jméno.“
#33:5 Na obou nádvořích Hospodinova domu nastavěl oltáře veškerému nebeskému zástupu.
#33:6 Své syny provedl v Údolí syna Hinómova ohněm, věštil z oblaků a obíral se hadačstvím a čarováním, ustanovil vyvolavače duchů a jasnovidce; dopouštěl se mnohého, co je zlé v očích Hospodina, a tak ho urážel.
#33:7 Tesanou sochu modly, kterou udělal, umístil do Božího domu, o němž Bůh řekl Davidovi a jeho synu Šalomounovi: „V tomto domě a v Jeruzalémě, který jsem vyvolil ze všech izraelských kmenů, dám navěky spočinout svému jménu.
#33:8 Už nikdy nedopustím, aby noha Izraele musela odejít z půdy, kterou jsem připravil pro vaše otce, jen když budou bedlivě činit všechno, jak jsem jim přikázal, podle celého zákona, nařízení a řádů vydaných skrze Mojžíše.“
#33:9 Menaše však Judu a obyvatele Jeruzaléma svedl, že se dopouštěli horších věcí než pronárody, které Hospodin před Izraelci vyhladil.
#33:10 Hospodin mluvil k Menašemu a jeho lidu, ale ti mu nevěnovali pozornost.
#33:11 Proto na ně Hospodin přivedl velitele vojska asyrského krále. Menašeho lapili, provlékli mu chřípím hák, spoutali ho bronzovými řetězy a odvedli do Babylóna.
#33:12 V nouzi prosil Hospodina, svého Boha, o shovívavost a hluboce se před Bohem svých otců pokořil.
#33:13 Modlil se k němu a on přijal a vyslyšel jeho prosbu a přivedl jej zpět do Jeruzaléma, aby dál kraloval. Tak poznal Menaše, že jenom Hospodin je Bůh.
#33:14 Potom vystavěl vnější hradební zeď Města Davidova v úvalu západně od Gíchónu směrem k Rybné bráně a dál kolem Ófelu a zvedl ji velmi vysoko. Ve všech opevněných městech v Judsku ustanovil velitele vojsk.
#33:15 Odstranil z Hospodinova domu všechny cizí bohy a sochy, též všechny oltáře, které nastavěl na hoře Hospodinova domu a v Jeruzalémě; vyházel je z města ven.
#33:16 Opravil Hospodinův oltář a obětoval na něm oběti pokojné a oběť díků. Nařídil Judovi, aby sloužil Hospodinu, Bohu Izraele.
#33:17 Lid však stále obětoval na posvátných návrších, ale jen Hospodinu, svému Bohu.
#33:18 O ostatních příbězích Menašeho, o jeho modlitbě k Bohu i o slovech vidoucích, kteří k němu mluvili jménem Hospodina, Boha Izraele, se dále píše v příbězích králů izraelských.
#33:19 O jeho modlitbě i o tom, jak Bůh jeho prosby přijal, o každém jeho hříchu a o jeho zpronevěře, o místech, kde stavěl posvátná návrší a postavil posvátné kůly a tesané modly, to před svým pokořením, se dále píše v příbězích Chózajových.
#33:20 I ulehl Menaše ke svým otcům a pohřbili ho u jeho domu. Po něm se stal králem jeho syn Amón.
#33:21 Amónovi bylo dvaadvacet let, když začal kralovat, a kraloval v Jeruzalémě dva roky.
#33:22 Dopouštěl se toho, co je zlé v Hospodinových očích, jako se toho dopouštěl jeho otec Menaše. Amón obětoval všem tesaným modlám, které udělal jeho otec a sloužil jim.
#33:23 Ale před Hospodinem se nepokořil, jako se pokořil jeho otec Menaše. Naopak, Amón vinu jen rozmnožoval.
#33:24 Jeho služebníci se proti němu spikli a usmrtili ho v jeho domě.
#33:25 Ale lid země pobil všechny, kteří se proti králi Amónovi spikli. Lid země pak dosadil místo něho za krále jeho syna Jóšijáše. 
#34:1 Jóšijášovi bylo osm let, když začal kralovat, a kraloval v Jeruzalémě jedenatřicet let.
#34:2 Činil to, co je správné v Hospodinových očích, chodil po cestách svého otce Davida a neuchyloval se napravo ani nalevo.
#34:3 V osmém roce svého kralování, ještě jako mladík, se začal dotazovat na slovo Boha svého otce Davida a ve dvanáctém roce začal očišťovat Judu a Jeruzalém od posvátných návrší, posvátných kůlů a tesaných i litých model.
#34:4 V jeho přítomnosti strhli oltáře baalů; oltáříky pro vykuřování kadidlem, které byly nahoře na nich, pokácel posvátné kůly, tesané i lité modly roztříštil napadrť a rozházel po hrobech těch, kdo jim obětovali.
#34:5 Kosti kněží spálil na jejich oltářích. Tak provedl očistu Judy a Jeruzaléma.
#34:6 V městech kmene Manasesova, Efrajimova a Šimeónova až po Neftalího a v okolních pustinách
#34:7 zbořil oltáře; posvátné kůly a tesané modly rozdrtil napadrť, po celém území izraelském pokácel všechny oltáříky pro vykuřování kadidlem. Potom se vrátil do Jeruzaléma.
#34:8 V osmnáctém roce svého kralování, když očistil zemi i dům, poslal Šáfana, syna Asaljášova, Masejáše, velitele města, a kancléře Jóacha, syna Jóachazova, aby opravili dům Hospodina, jeho Boha.
#34:9 Přišli k veleknězi Chilkijášovi, aby mu odevzdali stříbro, které bylo doneseno do Božího domu, které lévijci, strážci prahu, vybrali od kmene Manasesova a Efrajimova, ode všech, kdo zůstali z Izraele, od celého Judy a Benjamína i od obyvatelů Jeruzaléma.
#34:10 Ti je vydali těm, kdo pracovali jako dohlížitelé v Hospodinově domě, aby je vydávali dělníkům, kteří pracovali při Hospodinově domě na opravách poškozené části domu.
#34:11 Dávali je řemeslníkům a stavebním dělníkům k nákupu tesaného kamene a dřeva k vazbám a trámům budov, které judští králové nechali zpustnout.
#34:12 Muži konali dílo poctivě. Nad nimi byli dohlížiteli Jachat a Obadjáš, lévijci z Meraríovců, dále Zekarjáš a Mešulám z Kehatovců, aby je řídili. Všichni lévijci uměli hrát na hudební nástroje.
#34:13 Nad nosiči a těmi, kdo řídili všechny dělníky při různých pracích, byli lévijští písaři, dozorci a vrátní.
#34:14 Když se vynášelo stříbro, které bylo doneseno do Hospodinova domu, nalezl kněz Chilkijáš knihu Hospodinova zákona, vydaného skrze Mojžíše.
#34:15 Chilkijáš ohlásil písaři Šáfanovi: „Nalezl jsem v Hospodinově domě knihu Zákona.“ Chilkijáš dal tu knihu Šáfanovi.
#34:16 Šáfan knihu přinesl králi a podal králi o tom hlášení: „Tvoji služebníci vykonali vše, co jim bylo uloženo.
#34:17 Stříbro, které se nacházelo v Hospodinově domě, vyzvedli a vydali dohlížitelům a dělníkům.“
#34:18 Dále písař Šáfan králi oznámil: „Kněz Chilkijáš mi předal knihu.“ A Šáfan ji před králem četl.
#34:19 Když král uslyšel slova Zákona, roztrhl své roucho.
#34:20 Potom král přikázal Chilkijášovi a Achíkamovi, synu Šáfanovu, Abdónovi, synu Míkovu, písaři Šáfanovi a Asajášovi, královskému služebníku:
#34:21 „Jděte se dotázal Hospodina ohledně mne i lidu, který zůstal v Izraeli a Judsku, pokud jde o slova této nalezené knihy. Vždyť je na nás vylito velké Hospodinovo rozhořčení za to, že naši otcové nedbali na Hospodinovo slovo a nejednali podle toho všeho, co je v této knize napsáno.“
#34:22 Chilkijáš s těmi, kdo byli u krále, se odebral k prorokyni Chuldě, manželce Šalúma, syna Tokhata, syna Chasrova, strážce rouch. Bydlela v Jeruzalémě v Novém Městě. Mluvili s ní o těch věcech.
#34:23 Odvětila jim: „Toto praví Hospodin, Bůh Izraele: Vyřiďte muži, který vás ke mně poslal:
#34:24 Toto praví Hospodin: ‚Hle, uvedu zlo na toto místo a na jeho obyvatele, podle všech kleteb zapsaných v té knize, kterou četli před judským králem.
#34:25 Protože mne opustili a jiným bohům pálili kadidlo, a tak mě uráželi vším tím, co svýma rukama udělali, vylilo se mé rozhořčení na toto místo a neuhasne.‘
#34:26 A králi judskému, který vás poslal dotázat se Hospodina, vyřiďte: Toto praví Hospodin, Bůh Izraele: ‚Pokud jde o slova, která jsi slyšel:
#34:27 Protože tvé srdce zjihlo a pokořil ses před Bohem, když jsi uslyšel, co mluvil proti tomuto místu a proti jeho obyvatelům, protože ses přede mnou pokořil a roztrhl jsi své roucho a přede mnou plakal, vyslyšel jsem tě, je výrok Hospodinův.
#34:28 Proto tě připojím k tvým otcům, budeš uložen do svého hrobu v pokoji a tvé oči nespatří nic z toho zla, které uvedu na toto místo a na jeho obyvatele.‘“ I podali toto hlášení králi.
#34:29 Král obeslal a shromáždil všechny starší z Judy a z Jeruzaléma.
#34:30 Potom vystoupil král do Hospodinova domu i všichni judští muži a obyvatelé Jeruzaléma, kněží a lévijci, veškerý lid, velcí i malí. I předčítal jim všechna slova Knihy smlouvy, nalezené v Hospodinově domě.
#34:31 Poté se král postavil na své stanoviště a uzavřel před Hospodinem smlouvu, že budou následovat Hospodina a zachovávat jeho příkazy, svědectví a nařízení z celého srdce a z celé duše a plnit slova smlouvy, jak jsou napsána v této knize.
#34:32 Zavázal k tomu všechny, kdo byli v Jeruzalémě a v Benjamínovi. Obyvatelé Jeruzaléma tedy jednali podle smlouvy Boha, Boha svých otců.
#34:33 Jóšijáš odstranil všechny ohavnosti ze všech zemí, které patřily Izraelcům. Nařídil všem, kdo byli v Izraeli, aby sloužili Hospodinu, svému Bohu. Neodchýlili se od Hospodina, Boha svých otců, po všechny jeho dny. 
#35:1 Potom slavil Jóšijáš v Jeruzalémě hod beránka Hospodinu. Čtrnáctého dne prvního měsíce zabíjeli velikonočního beránka.
#35:2 Zavázal kněze, aby plnili své povinnosti, a povzbudil je pro službu v Hospodinově domě.
#35:3 Lévijcům, posvěceným pro Hospodina, kteří měli poučovat všechen Izrael, nařídil: „Svatou schránu dejte do domu, který vybudoval izraelský král Šalomoun, syn Davidův. Už ji nebudete nosit na ramenou. Teď služte Hospodinu, svému Bohu, a Izraeli, jeho lidu.
#35:4 Ustavte se podle svých otcovských rodů, podle příslušných tříd, podle výnosu Davida, krále izraelského, a ustanovení jeho syna Šalomouna.
#35:5 Postavte se ve svatyni podle rozčlenění otcovských rodů svých bratří z lidu, aby odpovídalo rozdělení lévijců podle rodů.
#35:6 Zabijte velikonočního beránka, posvěťte se a buďte připraveni posloužit svým bratřím, a tak splnit Hospodinovo slovo, které vydal skrze Mojžíše.“
#35:7 Jóšijáš věnoval lidu ovce, beránky a kůzlata, to vše k velikonočním hodům pro každého přítomného, celkem třicet tisíc kusů, k tomu tři tisíce kusů skotu; to vše z královského majetku.
#35:8 Jeho velmožové věnovali dobrovolně dary lidu, kněžím a lévijcům: Chilkijáš, Zekarjáš a Jechíel, představení domu Božího, dali kněžím k velikonočním hodům dva tisíce šest set kusů bravu a tři sta kusů skotu.
#35:9 Kónanjáš a jeho bratří Šemajáš a Netaneel, dále Chašabjáš, Jeíel a Józabad, předáci lévijců, věnovali lévijcům k velikonočním hodům pět tisíc kusů bravu a pět set kusů skotu.
#35:10 Když byla bohoslužba připravena, stoupli si kněží podle králova příkazu na svá místa a lévijci se seřadili podle svých tříd.
#35:11 Zabíjeli velikonoční beránky. Kněží kropili krví, kterou brali z jejich rukou, a lévijci stahovali kůže.
#35:12 Oddělovali části pro zápalnou oběť, když dávali příděly lidu podle rozčlenění otcovských rodů, a přinášeli je Hospodinu darem, jak je předepsáno v Knize Mojžíšově; právě tak si počínali při skotu.
#35:13 Připravili velikonočního beránka na ohni podle řádu, ale co bylo posvěceno, vařili v hrncích, kotlících a mísách a spěšně roznášeli všemu lidu.
#35:14 Potom jej připravili také pro sebe a pro kněze. Zatímco kněží Áronovci obětovali zápalné oběti a tuk až do noci, lévijci připravovali jídlo pro sebe a pro kněze Áronovce.
#35:15 Také zpěváci Asafovci stáli podle Davidova příkazu na svých místech. Asaf, Héman a Jedútún, vidoucí králův, též vrátní jednotlivých bran se ze své služby nevzdálili, protože jejich bratří lévijci připravovali jídlo i pro ně.
#35:16 Tak byla v onen den uspořádána celá Hospodinova bohoslužba; slavil se hod beránka a obětovaly se na Hospodinově oltáři zápalné oběti podle příkazu krále Jóšijáše.
#35:17 Izraelci, přítomní toho času na hodu beránka, slavili ještě po sedm dní svátek nekvašených chlebů.
#35:18 V Izraeli nebyl takový hod beránka slaven ode dnů proroka Samuela; ani žádný z izraelských králů neslavil takový hod beránka, jaký slavil Jóšijáš, kněží, lévijci a celý Juda i přítomní Izraelci a obyvatelé Jeruzaléma.
#35:19 Bylo to v osmnáctém roce Jóšijášova kralování, kdy se slavil tento hod beránka.
#35:20 Po tom všem, když Jóšijáš dal do pořádku Hospodinův dům, táhl Néko, král egyptský, aby bojoval u Karkemíše na Eufratu. Jóšijáš proti němu vytáhl.
#35:21 Néko k němu poslal posly se vzkazem: „Co je ti do mých věcí, judský králi? Dnes netáhnu proti tobě, bojuji s jiným domem. Bůh řekl, že si mám pospíšit. Ustup Bohu, který je se mnou, ať tě neuvrhne do zkázy.“
#35:22 Ale Jóšijáš mu neuhnul, chopil se příležitosti bojovat s ním. Neposlechl Nékových slov, která byla z úst Božích. Přitáhl na pláň u Megida, aby tu bojoval.
#35:23 Krále Jóšijáše zasáhli střelci. Král řekl svým služebníkům: „Odneste mě, protože jsem těžce raněn.“
#35:24 Služebníci ho přenesli z válečného vozu, aby ho na jiném voze, který měl s sebou, odvezli a dopravili do Jeruzaléma. Zemřel a byl pohřben v hrobech svých otců. Celý Juda a Jeruzalém nad Jóšijášem truchlili.
#35:25 Jeremjáš složil na Jóšijáše žalozpěv. Všichni zpěváci a zpěvačky opěvují Jóšijáše ve svých žalozpěvech až podnes. To se stalo v Izraeli zvykem a dále se o tom píše v Žalozpěvech.
#35:26 O ostatních příbězích Jóšijášových i o jeho zbožných činech, odpovídajících tomu, co je psáno v zákoně Hospodinově,
#35:27 o jeho příbězích prvních i posledních, se dále píše v Knize králů izraelských a judských. 
#36:1 Lid země vzal Jóšijášova syna Jóachaza a dosadil ho za krále v Jeruzalémě po jeho otci.
#36:2 Jóachazovi bylo třiadvacet let, když začal kralovat, a kraloval v Jeruzalémě tři měsíce.
#36:3 Egyptský král jej v Jeruzalémě sesadil a zemi uložil poplatek sto talentů stříbra a talent zlata.
#36:4 Za krále nad Judou a Jeruzalémem dosadil jeho bratra Eljakíma a změnil mu jméno na Jójakím. Jóachaza, jeho bratra, dal Néko odvléci do Egypta.
#36:5 Jójakímovi bylo dvacet pět let, když začal kralovat, a kraloval v Jeruzalémě jedenáct let. Dopouštěl se toho, co je zlé v očích Hospodina, jeho Boha.
#36:6 Přitáhl na něj Nebúkadnesar, král babylónský, spoutal ho bronzovými řetězy a dopravil ho do Babylóna.
#36:7 Část předmětů Hospodinova domu odvezl Nebúkadnesar do Babylóna a dal je do svého babylónského chrámu.
#36:8 O ostatních příbězích Jójakímových i o ohavnostech, jichž se dopouštěl, i o tom, co jej postihlo, se dále píše v Knize králů izraelských a judských. Po něm se stal králem jeho syn Jójakín.
#36:9 Jójakínovi bylo osm let, když začal kralovat, a kraloval v Jeruzalémě tři měsíce a deset dní. Dopouštěl se toho, co je zlé v Hospodinových očích.
#36:10 Na přelomu roku ho dal král Nebúkadnesar přivést do Babylóna a s ním i vzácné předměty Hospodinova domu. Za krále nad Judou a Jeruzalémem dosadil jeho bratra Sidkijáše.
#36:11 Sidkijášovi bylo jedenadvacet let, když začal kralovat, a kraloval v Jeruzalémě jedenáct let.
#36:12 Dopouštěl se toho, co je zlé v očích Hospodina, jeho Boha. Nepokořil se před prorokem Jeremjášem a před výrokem Hospodinovým.
#36:13 Dokonce se vzbouřil proti králi Nebúkadnesarovi, který jej zavázal při Bohu přísahou, byl tvrdošíjný a troufalého srdce a nenavrátil se k Hospodinu, Bohu Izraele.
#36:14 Rovněž i všichni kněžští předáci a lid se mnohokráte zpronevěřili, jednali podle kdejakých ohavností pronárodů a poskvrnili Hospodinův dům, který si oddělil jako svatý v Jeruzalémě.
#36:15 Hospodin, Bůh jejich otců, k nim nepřetržitě posílal své posly, protože měl soucit se svým lidem i se svým příbytkem.
#36:16 Oni však Boží posly zesměšňovali, pohrdali jeho slovy a jeho proroky prohlašovali za podvodníky, takže Hospodinovo rozhořčení dolehlo na jeho lid a nebyl, kdo by ho zhojil.
#36:17 Přivedl na ně kaldejského krále, jenž mečem vyvraždil ve svatyni jejich mladé muže, neměl soucit s mladými muži ani s pannami, ani se starci a kmety; Bůh mu všechny vydal do rukou.
#36:18 Také všechny předměty Hospodinova domu, velké i malé, poklady Hospodinova domu a poklady královy a jeho velmožů, to vše odvezl do Babylóna.
#36:19 Dům Boží spálili, jeruzalémské hradby strhli, všechny jeho paláce vypálili a všechno vzácné zařízení bylo zničeno.
#36:20 Pozůstatek lidu, jenž unikl meči, přestěhoval do Babylóna. Stali se otroky, sloužili jemu a jeho synům až do doby, kdy se kralování ujali Peršané.
#36:21 Tak se naplnilo slovo Hospodinovo, které mluvil ústy Jeremjáše: „Dokud si země nevynahradí své dny odpočinku, bude odpočívat po všechny dny, co bude zpustošena, až se vyplní sedmdesát let.“
#36:22 V prvním roce vlády Kýra, krále perského, se splnilo slovo Hospodinovo, které mluvil ústy Jeremjáše. Hospodin vzbudil ducha perského krále Kýra, že dal po celém svém království rozhlásit a také zapsat:
#36:23 „Toto praví Kýros, král perský: ‚Hospodin, Bůh nebes, mi dal všechna království země. Pověřil mě, abych mu vybudoval dům v Jeruzalémě, který je v Judsku. Kdokoli z vás, ze všeho jeho lidu - Hospodin, jeho Bůh, buď s ním - se může vydat na cestu.‘“  

\book{Ezra}{Ezra}
#1:1 V prvním roce vlády Kýra, krále perského, se splnilo slovo Hospodinovo, které mluvil ústy Jeremjáše. Hospodin vzbudil ducha perského krále Kýra, že dal po celém svém království rozhlásit a také zapsat:
#1:2 „Toto praví Kýros, král perský: ‚Hospodin, Bůh nebes, mi dal všechna království země. Pověřil mě, abych mu vybudoval dům v Jeruzalémě, který je v Judsku.
#1:3 Kdokoli z vás, ze všeho jeho lidu - Bůh buď s ním - se může vydat na cestu do Jeruzaléma, který je v Judsku, a stavět dům Hospodina, Boha Izraele, toho Boha, který je v Jeruzalémě.
#1:4 Každého, kdo zůstal na kterémkoli místě jako host, nechť podpoří místní obyvatelé stříbrem a zlatem, majetkem a dobytkem spolu s dobrovolnými obětmi pro Boží dům v Jeruzalémě.‘“
#1:5 Tu se vydali na cestu představitelé judských a benjamínských rodů, kněží a levité, všichni, jejichž ducha probudil Bůh, aby stavěli Hospodinův dům v Jeruzalémě.
#1:6 Celé okolí je podpořilo stříbrnými nádobami a zlatem, majetkem a dobytkem i vzácnými dary kromě toho, co obětovali dobrovolně.
#1:7 Král Kýros také vydal předměty Hospodinova domu, které Nebúkadnesar odnesl z Jeruzaléma a dal do domu svého boha.
#1:8 Perský král Kýros je předal správci pokladu Mitredatovi, aby je sečtené odevzdal judskému předáku Šéšbasarovi.
#1:9 Toto je jejich soupis: třicet zlatých obětních misek, tisíc stříbrných obětních misek, dvacet devět jiných misek,
#1:10 třicet zlatých koflíků, čtyři sta deset náhradních stříbrných koflíků, tisíc jiných nádob.
#1:11 Všech zlatých a stříbrných nádob bylo pět tisíc čtyři sta. To všechno přinesl Šéšbasar, když byli přivedeni přesídlenci z Babylóna do Jeruzaléma. 
#2:1 Toto jsou příslušníci judského kraje, kteří přišli ze zajetí, přesídlenci, které do Babylóna přesídlil babylónský král Nebúkadnesar. Navrátili se do Jeruzaléma a do Judska, každý do svého města.
#2:2 Přišli se Zerubábelem, Jéšuou, Nechemjášem, Serajášem, Reelajášem, Mordokajem, Bilšánem, Misparem, Bigvajem, Rechúmem a Baanou. Soupis mužů izraelského lidu:
#2:3 Synů Pareóšových dva tisíce sto sedmdesát dva;
#2:4 synů Šefatjášových tři sta sedmdesát dva;
#2:5 synů Arachových sedm set sedmdesát pět;
#2:6 synů Pachat-moábových, totiž synů Jéšuových a Jóabových, dva tisíce osm set dvanáct;
#2:7 synů Élamových tisíc dvě stě padesát čtyři;
#2:8 synů Zatúových devět set čtyřicet pět;
#2:9 synů Zakajových sedm set šedesát;
#2:10 synů Baního šest set čtyřicet dva;
#2:11 synů Bebajových šest set dvacet tři;
#2:12 synů Azgadových tisíc dvě stě dvacet dva;
#2:13 synů Adoníkamových šest set šedesát šest;
#2:14 synů Bigvajových dva tisíce padesát šest;
#2:15 synů Adínových čtyři sta padesát čtyři;
#2:16 synů Aterových, totiž Jechizkijášových, devadesát osm;
#2:17 synů Besajových tři sta dvacet tři;
#2:18 synů Jórových sto dvanáct;
#2:19 synů Chašumových dvě stě dvacet tři;
#2:20 synů Gibarových devadesát pět;
#2:21 synů betlémských sto dvacet tři;
#2:22 mužů netófských padesát šest;
#2:23 mužů anatótských sto dvacet osm;
#2:24 synů azmávetských čtyřicet dva;
#2:25 synů kirjátarimských, kefírských a beerótských sedm set čtyřicet tři;
#2:26 synů rámských a gebských šest set dvacet jeden;
#2:27 mužů mikmásských sto dvacet dva;
#2:28 mužů bételských a ajských dvě stě dvacet tři;
#2:29 synů neboských padesát dva;
#2:30 synů magbíšských sto padesát šest;
#2:31 synů druhého Élama tisíc dvě stě padesát čtyři;
#2:32 synů Charimových tři sta dvacet;
#2:33 synů lódských, chadídských a ónoských sedm set dvacet pět;
#2:34 synů jerišských tři sta čtyřicet pět;
#2:35 synů Senáových tři tisíce šest set třicet.
#2:36 Kněží: synů Jedajášových z domu Jéšuova devět set sedmdesát tři;
#2:37 synů Imerových tisíc padesát dva;
#2:38 synů Pašchúrových tisíc dvě stě čtyřicet sedm;
#2:39 synů Charimových tisíc sedmnáct.
#2:40 Levité: synů Jéšuových a Kadmíelových, totiž synů Hódavjášových, sedmdesát čtyři.
#2:41 Zpěváci: synů Asafových sto dvacet osm.
#2:42 Synové vrátných: synů Šalúmových, synů Aterových, synů Talmónových, synů Akúbových, synů Chatítových, synů Šobajových, všech sto třicet devět.
#2:43 Chrámoví nevolníci: Synové Síchovi, synové Chasúfovi, synové Tabaótovi,
#2:44 synové Kérosovi, synové Síahovi, synové Padónovi,
#2:45 synové Lebánovi, synové Chagábovi, synové Akúbovi,
#2:46 synové Chágabovi, synové Šamlajovi, synové Chananovi,
#2:47 synové Gidélovi, synové Gacharovi, synové Reajášovi,
#2:48 synové Resínovi, synové Nekódovi, synové Gazamovi,
#2:49 synové Uzovi, synové Paséachovi, synové Besajovi,
#2:50 synové Asnovi, synové Meúnejců, synové Nefúsejců,
#2:51 synové Bakbúkovi, synové Chakúfovi, synové Charchúrovi,
#2:52 synové Baslútovi, synové Mechídovi, synové Charšovi,
#2:53 synové Barkósovi, synové Síserovi, synové Tamachovi,
#2:54 synové Nesíachovi a synové Chatífovi. -
#2:55 Synové služebníků Šalomounových: synové Sótajovi, synové Sóferetovi, synové Perúdovi,
#2:56 synové Jaelovi, synové Darkónovi, synové Gidélovi,
#2:57 synové Šefatjášovi, synové Chatílovi, synové Pokereta Sebajimského, synové Amíovi,
#2:58 všech chrámových nevolníků a synů Šalomounových služebníků tři sta devadesát dva.
#2:59 Tito vyšli z Tel-melachu, z Tel-charši, z Kerúb-adánu a Imeru, ale nemohli prokázat, že jejich otcovský rod a původ je z Izraele:
#2:60 synů Delajášových, synů Tóbijášových a synů Nekódových šest set padesát dva.
#2:61 A z kněžských příslušníků synové Chobajášovi, synové Kósovi a synové Barzilaje, který si vzal za ženu jednu z dcer Barzilaje Gileádského a je nazýván jejich jménem.
#2:62 Ti hledali svůj rodokmen v seznamu rodů, ale marně; proto byli vyloučeni z kněžství jako nezpůsobilí.
#2:63 Místodržící jim zapověděl jíst ze svatých přídělů kněžských, pokud nebude ustanoven kněz pro posvátné losy urím a tumím.
#2:64 Celé shromáždění dohromady čítalo čtyřicet dva tisíce tři sta šedesát duší,
#2:65 mimo jejich otroky a otrokyně, jejichž bylo sedm tisíc tři sta třicet sedm. Měli také dvě stě zpěváků a zpěvaček.
#2:66 Koní bylo sedm set třicet šest, mezků dvě stě čtyřicet pět,
#2:67 velbloudů čtyři sta třicet pět, oslů šest tisíc sedm set dvacet.
#2:68 Někteří představitelé rodů přinesli po svém příchodu k domu Hospodinovu do Jeruzaléma dobrovolné dary pro Boží dům, aby jej mohli postavit na jeho základech.
#2:69 Podle svých možností dali na chrámový poklad pro to dílo šedesát jeden tisíc zlatých darejků a pět tisíc hřiven stříbra a sto kněžských suknic.
#2:70 I usadili se kněží, levité a někteří z lidu i zpěváci, vrátní a chrámoví nevolníci ve svých městech; všechen Izrael se usadil ve svých městech. 
#3:1 Když nastal sedmý měsíc a Izraelci již byli v městech, shromáždil se lid jednomyslně do Jeruzaléma.
#3:2 Jéšua, syn Jósadakův, s bratry kněžími a Zerubábel, syn Šealtíelův, se svými bratry začali budovat oltář Bohu Izraele, aby na něm mohli obětovat zápalné oběti, jak je předepsáno v zákoně Mojžíše, muže Božího.
#3:3 Zřídili oltář na jeho původních základech, neboť žili ve strachu z národů zemí, a obětovali na něm zápalné oběti Hospodinu, zápal jitří a večerní.
#3:4 Potom slavili svátek stánků, jak je předepsáno. Každý den přinášeli zápalné oběti v počtu stanoveném pro onen den.
#3:5 Vedle nich také každodenní zápalnou oběť, oběť při každém novoluní a při všech slanostech zasvěcených Hospodinu, mimo vše, co kdo Hospodinu obětoval dobrovolně.
#3:6 V první den sedmého měsíce začali obětovat zápalné oběti Hospodinu, i když ještě nebyly položeny základy Hospodinova chrámu.
#3:7 Dali peníze kameníkům a tesařům, potravu, nápoj a olej Sidóňanům a Týřanům, aby dopravovali cedrové dříví z Libanónu k moři do Jafy podle povolení, které jim dal perský král Kýros.
#3:8 Druhého roku druhého měsíce po svém příchodu k Božímu domu do Jeruzaléma začali Zerubábel, syn Šealtíelův, a Jéšua, syn Jósadakův, s ostatními bratry kněžími a levity, se všemi, kteří přišli ze zajetí do Jeruzaléma, ustanovovat levity starší dvaceti let, aby řídili dílo Hospodinova domu.
#3:9 Tak nastoupili společně Jéšua se syny a bratry, Kadmíel se syny, synové Judovi, také synové Chenadadovi se syny a bratry levity, aby řídili postup práce při díle Božího domu.
#3:10 Když stavitelé Hospodinova chrámu kladli základy, ustanovili kněze s pozouny, oblečené do svátečních rouch, a levity, syny Asafovy, s cymbály, aby chválili Hospodina podle řádu izraelského krále Davida.
#3:11 Opěvovali Hospodina a vzdávali mu chválu a čest, že je dobrý, že jeho milosrdenství nad Izraelem je věčné. A všechen lid mohutným hlaholem chválil Hospodina, že byly položeny základy Hospodinova domu.
#3:12 Mnozí z kněží a levitů i představitelů rodu, starci, kteří viděli dřívější dům, dali se do hlasitého pláče, když byly kladeny základy tohoto domu před jejich očima. Mnozí však radostně hlaholili, až se to rozléhalo.
#3:13 Radostný hlahol nebylo možno rozeznat od pláče lidu. Lid totiž hlaholil tak mohutným hlaholem, že se to rozléhalo dodaleka. 
#4:1 Když Judovi a Benjamínovi protivníci uslyšeli, že synové přesídlenců budují chrám Hospodinu, Bohu Izraele,
#4:2 dostavili se k Zerubábelovi a k představitelům rodů s návrhem: „Budeme stavět s vámi, neboť se dotazujeme vašeho Boha stejně jako vy. Obětujeme mu ode dnů asyrského krále Esar-chadóna, který nás sem přesídlil.“
#4:3 Ale Zerubábel a Jéšua i ostatní představitelé izraelských rodů jim odpověděli: „Vy s námi nemůžete stavět dům našeho Boha. Hospodinu, Bohu Izraele, budeme stavět my sami, jak nám přikázal král Kýros, král perský!“
#4:4 Lid země bral odvahu judskému lidu a pohrůžkami jej odrazoval od stavby.
#4:5 Podpláceli proti nim rádce, aby rušili jejich plány, po všechny dny perského krále Kýra až do kralování perského krále Dareia.
#4:6 I za kralování Xerxova, na začátku jeho kralování, sepsali na obyvatele Judska a Jeruzaléma žalobu.
#4:7 Také za dnů Artaxerxa psali Bišlám, Mitredat a Tabel s ostatními svými druhy perskému králi Artaxerxovi; list byl napsán aramejským písmem i jazykem.
#4:8 Kancléř Rechům a písař Šimšaj napsali králi Artaxerxovi následující dopis proti Jeruzalému,
#4:9 totiž Rechúm, kancléř, a Šimšaj, písař, s ostatními svými druhy, soudcové, vládní zmocněnci, správní a berní úředníci, lidé z Uruku, z Babylóna, Elamci z Šúšanu
#4:10 a ostatní národy, které veliký a slavný Asenapar zajal a přesídlil do měst samařských a do ostatního Zaeufratí. Nuže,
#4:11 toto je opis dopisu, který mu poslali: „Králi Artaxerxovi,tvoji otroci ze Zaeufratí. Nuže:
#4:12 Známo buď králi, že židé, kteří odešli od tebe, přišli sem do Jeruzaléma a stavějí toto odbojné a protivící se město. Chtějí dokončit hradby, už spojují základy.
#4:13 Nuže, známo buď králi, bude-li toto město vystavěno a jeho hradby dokončeny, že už nebudou odvádět daně, dávky z úrod ani jiné poplatky, takže královská pokladna utrpí škodu.
#4:14 Protože však my okoušíme dobrodiní paláce, nenáleží se, abychom přihlíželi, jak je král odírán; proto posíláme králi tuto zprávu.
#4:15 Nechť se hledá v knize zápisů tvých otců. Nalezneš v knize zápisů a dozvíš se, že toto město je odbojné a působilo škody králům i krajům. Odedávna v něm docházelo ke vzpourám. Proto bylo to město zpustošeno.
#4:16 Oznamujeme králi: Bude-li město vystavěno a jeho hradby dokončeny, nezůstane ti v Zaeufratí žádný podíl!“
#4:17 Král poslal výnos: „Rechúmovi, kancléři, a Šimšajovi, písaři, i ostatním jejich druhům, kteří bydlí v Samaří a v ostatním Zaeufratí: Pokoj! Nuže,
#4:18 list, který jste nám poslali, byl mi slovo za slovem přečten.
#4:19 Vydal jsem rozkaz, aby se hledalo, a shledalo se, že to město odedávna povstává proti králům a že v něm docházelo k odboji a ke vzpouře.
#4:20 V Jeruzalémě byli i mocní králové a panovali nad celým Zaeufratím a vybírali daně, dávky z úrod a jiné poplatky.
#4:21 Nuže, vydejte rozkaz, ať je zabráněno těm mužům dostavět město, dokud nevydám příslušný rozkaz.
#4:22 Buďte opatrní, nejednejte v této věci nedbale, aby nevznikla královskému dvoru veliká škoda!“
#4:23 Jakmile Rechúm, písař Šimšaj a jejich druhové přečetli opis listu krále Artaxerxa, rychle přitáhli do Jeruzaléma k židům a zabránili jim ve stavbě násilím a mocí.
#4:24 Tak byla také zastavena práce na Božím domě v Jeruzalémě a zůstala zastavena až do druhého roku kralování perského krále Dareia. 
#5:1 Proroci Ageus a Zacharjáš, syn Idův, prorokovali židům v Judsku a v Jeruzalémě ve jménu Boha Izraele a napomínali je.
#5:2 Tu se vzchopili Zerubábel, syn Šealtíelův, a Jéšua, syn Jósadakův, a začali budovat Boží dům v Jeruzalémě. A ti Boží proroci při nich stáli a posilovali je.
#5:3 Tehdy k nim přišel Tatenaj, místodržitel zaeufratský, a Šetar-bóznaj se svými druhy a ptali se jich: „Kdo vám dal rozkaz stavět tento dům a obnovovat tuto svatyni?“
#5:4 Odpověděli jsme jim tedy a sdělili jména mužů, kteří budovu stavěli.
#5:5 A oko Boží bdělo nad židovskými staršími, takže jim nebránili stavět, dokud by nedošlo hlášení Dareiovi a nebyl od něho doručen list v této záležitosti.
#5:6 Opis dopisu, který poslal Tatenaj, místodržitel zaeufratský, a Šetar-bóznaj a jeho druhové, úředníci ze Zaeufratí, králi Dareiovi.
#5:7 Poslali mu zprávu, v níž bylo napsáno: „Králi Dareiovi nerušený pokoj!
#5:8 Známo buď králi, že jsme přišli do judského kraje k domu velikého Boha. Ten je stavěn z kamenných kvádrů a do zdí se kladou trámy. Práce se koná svědomitě a dílo roste pod rukama.
#5:9 Ptali jsme se těch starších, když jsme s nimi mluvili: Kdo vám dal rozkaz stavět tento dům a obnovovat tuto svatyni?
#5:10 Také jsme se zeptali na jejich jména, abychom ti je mohli oznámit. Zapsali jsme jména mužů stojících v čele.
#5:11 Oni nám podali tuto zprávu: ‚Jsme služebníci Boha nebes i země a stavíme tento dům, který už byl postaven dříve před mnoha lety. Stavěl a dokončil jej veliký izraelský král.
#5:12 Když však naši otcové rozhněvali Boha nebes, vydal je do rukou Kaldejce Nebúkadnesara, krále babylónského. Ten zbořil tento dům a lid přesídlil do Babylóna.
#5:13 Ale v prvním roce Kýra, krále nad Babylónem, vydal král Kýros rozkaz ke stavbě tohoto Božího domu.
#5:14 Také nádoby z Božího domu, zlaté a stříbrné, které Nebúkadnesar odnesl z jeruzalémského chrámu a vnesl do chrámu babylónského, dal vynést král Kýros z babylónského chrámu a rozkázal je vydat muži jménem Šéšbasar, kterého ustanovil místodržitelem.
#5:15 Poručil mu: Vezmi tyto nádoby a jdi, slož je v jeruzalémském chrámě. Boží dům buď postaven na původním místě!‘
#5:16 Onen Šéšbasar přišel položit základy Božího domu v Jeruzalémě. Od té doby se buduje až dosud a ještě není dokončen.
#5:17 Nyní tedy, uzná-li král za dobré, ať se hledá v královských archivech tam v Babylóně, zda je tomu tak, že král Kýros vydal rozkaz ke stavbě tohoto Božího domu v Jeruzalémě. Nechť je nám sdělena králova vůle v této záležitosti.“ 
#6:1 Král Dareios tedy vydal rozkaz, aby se hledalo v archivu mezi poklady složenými kdesi v Babylóně.
#6:2 Konečně našli v médské krajině v pevnosti Achmetě jeden svitek, v němž stálo: „Zápis:
#6:3 V prvním roce svého kralování vydal král Kýros rozkaz o domě Božím v Jeruzalémě: Tento dům buď zase postaven na místě, kde se obětují oběti. Ať je od základů opraven. Jeho výška budiž šedesát loket, jeho šířka také šedesát loket.
#6:4 Tři vrstvy budou vždy z kamenných kvádrů a jedna vrstva z nových trámů. Náklad bude uhrazen z královské pokladny.
#6:5 Také zlaté a stříbrné nádoby z Božího domu, které odnesl Nebúkadnesar z jeruzalémského chrámu a donesl do Babylóna, ať jsou vráceny, ať přijdou do jeruzalémského chrámu na původní místo a jsou složeny v Božím domě.“
#6:6 „Proto, Tatenaji, místodržiteli zaeufratský, a Šetar-bóznaji se svými druhy, úředníci, kteří jste v Zaeufratí: Nevměšujte se do té záležitosti
#6:7 a nerušte práci na tom Božím domě! Judský místodržitel s židovskými staršími ať stavějí dům Boží na jeho původním místě.
#6:8 Vydal jsem rozkaz, co máte udělat pro židovské starší, pokud jde o stavbu Božího domu. Z královských prostředků, z daní ze Zaeufratí, buď svědomitě a bez průtahu proplácen těm mužům potřebný obnos.
#6:9 Vše nezbytné, býci, berani a beránci k zápalným obětem pro Boha nebes, pšenice, sůl, víno a olej, ať jsou vydávány jeruzalémským kněžím podle jejich požadavků na každý den bez nedbalosti,
#6:10 aby mohli přinášet vonné dary, libé Bohu nebes, a modlit se za život krále a jeho synů.
#6:11 Vydal jsem též rozkaz: Každému, kdo by přestoupil tento výnos, buď vyrván trám z jeho domu a on na něm pověšen a přibit; a jeho dům ať je učiněn hnojištěm.
#6:12 Bůh, který chce, aby tam přebývalo jeho jméno, ať zničí každého krále i národ, který by vztáhl ruku a přestoupil tento výnos a zbořil by Boží dům v Jeruzalémě. Já, Dareios, jsem vydal tento rozkaz. Ať je svědomitě prováděn!“
#6:13 Tatenaj, místodržitel zaeufratský, a Šetar-bóznaj se svými druhy jednali svědomitě podle pokynů krále Dareia.
#6:14 Židovští starší stavěli a dílo se jim dařilo, jak prorokovali proroci Ageus a Zacharjáš, syn Idův. Stavbu dokončili podle rozkazu Boha Izraele a podle rozkazu Kýra, Dareia a Artaxerxa, králů perských.
#6:15 Dům byl dostavěn třetího dne měsíce adaru a v šestém roce kralování krále Dareia.
#6:16 I slavili Izraelci, kněží, levité i ostatní synové přesídlenců s radostí posvěcení domu Božího.
#6:17 Při posvěcení Božího domu obětovali sto býků, dvě stě beranů, čtyři sta beránků a jako oběť za hřích celého Izraele dvanáct kozlů, podle počtu izraelských kmenů.
#6:18 Ustanovili kněze podle tříd i levity podle skupin, aby vykonávali v Jeruzalémě bohoslužbu, jak je psáno v Knize Mojžíšově.
#6:19 Potom čtrnáctého dne prvního měsíce slavili synové přesídlenců hod beránka.
#6:20 Kněží i levité se do jednoho očistili. Všichni byli čistí. Zabíjeli velikonočního beránka pro všechny syny přesídlenců, pro své bratry kněze i pro sebe.
#6:21 I jedli Izraelci, kteří se vrátili z přesídlení, i každý, kdo se k nim připojil a oddělil od nečistoty pohanů v zemi, aby se dotazoval Hospodina, Boha Izraele.
#6:22 Po sedm dní slavili s radostí slavnost nekvašených chlebů, protože Hospodin jim způsobil radost, když jim naklonil srdce asyrského krále, aby je podpořil při díle Božího domu, domu Boha Izraele. 
#7:1 Po těchto událostech, za kralování perského krále Artaxerxa, přišel Ezdráš, syn Serajáše, syna Azarjáše, syna Chilkijáše,
#7:2 syna Šalúma, syna Sádoka, syna Achítúba,
#7:3 syna Amarjáše, syna Azarjáše, syna Merajóta,
#7:4 syna Zerachjáše, syna Uzího, syna Bukího,
#7:5 syna Abíšúy, syna Pinchasa, syna Eleazara, syna Árona, nejvyššího kněze.
#7:6 Tento Ezdráš vyšel z Babylóna. Byl to znalec Zákona, zběhlý v zákoně Mojžíšově, který vydal Hospodin, Bůh Izraele. Král mu dal vše, oč žádal, protože nad ním byla ruka Hospodina, jeho Boha.
#7:7 S ním se vydali do Jeruzaléma i někteří Izraelci, kněží a levité, zpěváci, vrátní a chrámoví nevolníci v sedmém roce krále Artaxerxa.
#7:8 Přišel do Jeruzaléma pátého měsíce téhož sedmého roku jeho kralování.
#7:9 Na první den prvního měsíce byl stanoven nástup cesty z Babylóna a prvního dne pátého měsíce přišel do Jeruzaléma, protože nad ním byla dobrotivá ruka jeho Boha.
#7:10 Ezdráš bádal s upřímným srdcem v Hospodinově zákoně, jednal podle něho a vyučoval v Izraeli nařízením a právům.
#7:11 Toto je opis listu, který vydal král Artaxerxes knězi Ezdrášovi, znalci Zákona, znalému znění příkazů Hospodinových a jeho nařízení Izraeli:
#7:12 „Artaxerxes, král králů knězi Ezdrášovi, znalci zákona Boha nebes: Rozhodnutí. Nuže:
#7:13 Vydal jsem rozkaz, že každý z izraelského lidu, z jeho kněží a levitů v mém království, kdo touží jít do Jeruzaléma, může jít s tebou,
#7:14 protože jsi vyslán králem a jeho sedmi rádci, abys dohlédl na Judsko a na Jeruzalém podle zákona svého Boha, který máš v ruce.
#7:15 Doneseš tam stříbro a zlato, které král a jeho rádcové dobrovolně obětovali Bohu Izraele, jehož příbytek je v Jeruzalémě,
#7:16 i všechno stříbro a zlato, které dostaneš po celé babylónské krajině spolu s dobrovolnými obětmi lidu a kněží, kteří je budou dobrovolně obětovat pro dům svého Boha v Jeruzalémě.
#7:17 Za to stříbro svědomitě nakoupíš býky, berany a beránky s příslušnými obětními dary a úlitbami a budeš je obětovat na oltáři v domě vašeho Boha v Jeruzalémě.
#7:18 Se zbytkem stříbra a zlata učiňte, co ty a tvoji bratří budete pokládat za dobré, v souladu s vůlí vašeho Boha.
#7:19 Předměty, které ti jsou předány pro službu v domě tvého Boha, odevzdej před Bohem v Jeruzalémě.
#7:20 Ostatní své výdaje za potřeby pro dům tvého Boha uhradíš z královské pokladny.
#7:21 Já, král Artaxerxes, jsem vydal rozkaz pro všechny strážce pokladu v Zaeufratí: Všechno, oč vás požádá kněz Ezdráš, znalec zákona Boha nebes, ať je svědomitě vykonáno,
#7:22 až do jednoho sta talentů stříbra, do sta kórů pšenice, do sta batů vína a sta batů oleje, a soli bez omezení.
#7:23 Všechno, co rozkáže Bůh nebes, ať se přesně vykoná pro dům Boha nebes, aby se nerozlítil na královo království a na jeho syny.
#7:24 Buď vám také známo, že žádnému knězi ani levitovi, zpěvákovi, vrátnému, chrámovému nevolníkovi a služebníku Božího domu se nesmějí vyměřit daně, dávky z úrod a jiné poplatky.
#7:25 Ty, Ezdráši, podle moudrosti svého Boha, která je ti svěřena, ustanovíš soudce a rozhodčí, kteří budou rozhodovat pře všeho lidu v Zaeufratí, totiž všech, kteří se znají k zákonům tvého Boha. A kdo je neznají, ty budete vyučovat.
#7:26 Nad každým, kdo nebude svědomitě zachovávat zákon tvého Boha a zákon království, bude vynesen rozsudek smrti nebo vyhnanství, peněžité pokuty nebo vězení.“
#7:27 Požehnán buď Hospodin, Bůh našich otců, že vložil do srdce králi, aby obnovil lesk Hospodinova domu v Jeruzalémě,
#7:28 a že mi příznivě naklonil krále a jeho rádce i všechny významné královy velmože. Vzchopil jsem se tedy, protože nade mnou byla ruka Hospodina, mého Boha, a shromáždil jsem představitele Izraele, aby šli se mnou. 
#8:1 Toto jsou představitelé svých rodů a seznam rodů těch, kteří se mnou vyšli z Babylóna za kralování krále Artaxerxa:
#8:2 Z Pinchasových synů Geršóm, z Ítamarových synů Daníel, z Davidových synů Chatúš.
#8:3 Z Šekanjášových synů..., z Pareóšových synů Zekarjáš; spolu s ním bylo zapsáno do seznamu rodů sto padesát mužů.
#8:4 Z Pachat-moábových synů Eljóenaj, syn Zerachjášův, a s ním dvě stě mužů.
#8:5 Z Šekanjášových synů Jachíezelův syn a s ním tři sta mužů.
#8:6 Z Adínových synů Ebed, syn Jónatanův, a s ním padesát mužů.
#8:7 Z Élamových synů Ješajáš, syn Ataljášův, a s ním sedmdesát mužů.
#8:8 Z Šefatjášových synů Zebadjáš, syn Míkaelův, a s ním osmdesát mužů.
#8:9 Z Jóabových synů Óbadjáš, syn Jechíelův, a s ním dvě stě osmnáct mužů.
#8:10 Z Šelómítových synů syn Jósifjášův a s ním sto šedesát mužů.
#8:11 Z Bébajových synů Zekarjáš, syn Bébajův, a s ním dvacet osm mužů.
#8:12 Z Azgádových synů nejmladší syn Jóchanan a s ním sto deset mužů.
#8:13 Z Adoníkamových synů ti poslední. Jejich jména jsou: Elípelet, Jeíel a Šemajáš a s nimi šedesát mužů.
#8:14 Z Bigvajových synů Útaj a Zabúd a s ním sedmdesát mužů.
#8:15 Shromáždil jsem je u řeky, která vtéká do Ahavy. Tam jsme tábořili tři dny. Pátral jsem mezi lidem a kněžími, ale nenašel jsem tam ani jednoho z levitů.
#8:16 Proto jsem poslal pro představitele, pro Elíezera, Aríela, Šemajáše, Elnatana, Jaríba, druhého Elnatana, Nátana, Zekarjáše a Mešulama, i pro učeného Jójaríba a Elnatana.
#8:17 Dal jsem jim vzkaz pro Ida, představitele místa Kasifji; naučil jsem je, co mají mluvit k Idovi a k jeho bratřím, chrámovým nevolníkům v místě Kasifji, aby nám přivedli sluhy pro dům našeho Boha.
#8:18 Dobrotivá ruka našeho Boha byla nad námi, takže nám přivedli prozíravého muže ze synů Machlího, syna Léviho, syna Izraelova, a to Šerebjáše s jeho syny a bratry, celkem osmnáct;
#8:19 dále Chašabjáše a s ním Ješajáše ze synů Merarího s jeho bratry a jejich syny, celkem dvacet.
#8:20 A chrámových nevolníků, které dal David a velmožové k službě levitům, bylo dvě stě dvacet. Ti všichni byli uvedeni jménem.
#8:21 Tam u řeky Ahavy jsem vyhlásil půst, bychom se před svým Bohem pokořili a vyprosili si u něho pro sebe, pro své dítky a i pro všechen svůj majetek přímou cestu.
#8:22 Ostýchal jsem se totiž žádat od krále ozbrojený doprovod a jízdu, aby nám na cestě pomáhali proti nepříteli; řekli jsme králi: „Dobrotivá ruka našeho Boha je nade všemi, kdo ho hledají, ale jeho moc a hněv je proti všem, kdo ho opouštějí.“
#8:23 Proto jsme se postili a prosili jsme v této věci svého Boha, a on naše prosby přijal.
#8:24 Pak jsem oddělil dvanáct kněžských předáků; k nim Šerebjáše, Chašabjáše a s nimi deset jejich bratří.
#8:25 Odvážil jsem jim stříbro a zlato i předměty jako oběť pozdvihování pro dům našeho Boha, to, co obětoval král se svými rádci a jeho velmožové i všechen Izrael, který tam byl.
#8:26 Odvážil a předal jsem jim šest set padesát talentů stříbra, stříbrných nádob za sto talentů, zlata sto talentů,
#8:27 dvacet zlatých koflíků po tisíci darejcích a dva předměty z výborného nazlátlého bronzu, žádoucí jako zlato.
#8:28 Přitom jsem jim řekl: „Jste svatí Hospodinu, i tyto předměty jsou svaté. Toto stříbro a zlato je dobrovolný obětní dar pro Hospodina, Boha vašich otců.
#8:29 Bedlivě to opatrujte, dokud to neodvážíte před předáky kněží a levitů i předáky izraelských rodů v Jeruzalémě v komorách Hospodinova domu.“
#8:30 Kněží a levité převzali odvážené stříbro a zlato i předměty, aby je dopravili do Jeruzaléma, do domu našeho Boha.
#8:31 Dvanáctý den prvního měsíce jsme se vydali od řeky Ahavy na cestu do Jeruzaléma a ruka našeho Boha byla nad námi. Vysvobozovala nás z rukou nepřátel a těch, kteří nám, na cestě strojili úklady.
#8:32 Přišli jsme do Jeruzaléma a strávili jsme tam tři dny.
#8:33 Čtvrtého dne bylo odváženo stříbro, zlato i předměty v domě našeho Boha a předáno knězi Meremótovi, synu Úrijášovu, za přítomnosti Eleazara, syna Pinchasova, a za přítomnosti levitů Józabada, syna Jéšuova, a Nóadjáše, syna Binújova,
#8:34 všechno podle počtu a váhy. Hned také byla zapsána celková váha.
#8:35 Synové přesídlenců, kteří přišli ze zajetí, přinesli zápalné oběti Bohu Izraele: dvanáct býků za celý Izrael, devadesát šest beranů, sedmdesát sedm beránků, dvanáct kozlů v oběť za hřích, všechno jako zápalnou oběť pro Hospodina.
#8:36 Královská nařízení odevzdali královským satrapům a místodržitelům v Zaeufratí a ti podporovali lid i Boží dům. 
#9:1 Když byly vyřízeny tyto záležitosti, přistoupili ke mně předáci a řekli: „Izraelský lid, ani kněží a levité, se neoddělili od národů zemí, páchají ohavnosti Kenaanců, Chetejců, Perizejců, Jebúsejců, Amónců, Moábců, Egypťanů a Emorejců.
#9:2 Berou sobě i svým synům za ženy jejich dcery. Tak se mísí svaté símě s národy zemí. Předáci a představenstvo v této zpronevěře vedou.“
#9:3 Když jsem tu zprávu uslyšel, roztrhl jsem svůj šat i řízu, rval jsem si vlasy i vousy a seděl jsem jako omráčený.
#9:4 Tu se ke mně shromáždili všichni, kdo se třásli před slovy Boha Izraele pro zpronevěru přesídlenců. Ale já jsem seděl jako omráčený až do večerní oběti.
#9:5 Při večerní oběti jsem se probral ze své trýzně a s roztrženým šatem i řízou jsem klesl na kolena a rozprostřel ruce k Hospodinu, svému Bohu.
#9:6 Volal jsem: „Můj Bože, stydím se a hanbím pozdvihnout k tobě svou tvář, Bože můj. Naše nepravosti se nám rozmohly až nad hlavu, naše provinění vzrostlo až k nebi.
#9:7 Ode dnů svých otců až dodnes jsme se velice proviňovali. Pro své nepravosti jsme byli spolu se svými králi a kněžími vydáni do rukou králů zemí, pod meč, do zajetí, v loupež i k veřejnému zahanbení, jak je tomu i dnes.
#9:8 Je to nyní malý okamžik, co se nám dostalo smilování Hospodina, našeho Boha, který nám zachoval hrstku těch, kdo vyvázli, a dal nám zakotvit na svém svatém místě. Náš Bůh nám dopřál, že se naše oči rozjasnily, dal nám pookřát v našem otroctví.
#9:9 Byli jsme otroky, ale náš Bůh nás v našem otroctví neopustil. Příznivě nám naklonil perské krále a dal nám pookřát. Mohli jsme obnovit dům svého Boha a vystavět, co bylo v troskách. Byl naší záštitou v Judsku i v Jeruzalémě.
#9:10 Ale nyní, co máme povědět po tom všem, Bože náš? Opustili jsme tvé příkazy,
#9:11 které jsi vydal prostřednictvím svých služebníků proroků. Prohlásil jsi: ‚Země, kterou jdete obsadit, je země znečištěná nečistotou národů zemí, jejich ohavnostmi, kterými ji ve své nečistotě naplnili od jednoho konce po druhý.
#9:12 Proto nedávejte své dcery jejich synům a jejich dcery neberte pro své syny! Neusilujte nikdy o pokoj s nimi ani o dobrodiní od nich! Buďte rozhodní a budete užívat dobrých darů země a navěky ji podrobíte pro své syny.‘
#9:13 Po tom všem, co na nás přišlo pro naše zlé skutky a pro naši velikou vinu, jsi nás ty, Bože náš, nepotrestal za naše nepravosti, jak jsme zasloužili, ale dal jsi nám takto vyváznout.
#9:14 Což budeme zase porušovat tvé příkazy a znovu se příznit s těmi ohavnými národy? Což by ses na nás nerozhněval, takže bys nás úplně zničil, že by žádný nezůstal a nevyvázl?
#9:15 Hospodine, Bože Izraele, ty jsi spravedlivý, proto nás dodnes zůstala hrstka těch, kdo vyvázli. Jsme tu teď před tebou se svou vinou, přestože pro takové věci nelze před tebou obstát.“ 
#10:1 Zatímco se Ezdráš modlil, vyznával hříchy a v pláči se před Božím domem vrhl k zemi, shromáždilo se k němu z Izraele převeliké shromáždění mužů, žen i dětí a lid se dal do usedavého pláče.
#10:2 Tu promluvil Šekanjáš, syn Jechíelův, ze synů Élamových, a řekl Ezdrášovi: „Zpronevěřili jsme se svému Bohu, když jsme si přivedli ženy cizinky z národů země. Přesto je ještě pro Izraele naděje.
#10:3 Uzavřeme nyní smlouvu se svým Bohem, že propustíme všechny ženy cizinky i jejich děti podle rozhodnutí Panovníka i těch, kdo se třesou před příkazem našeho Boha. Ať se stane podle Zákona!
#10:4 Vstaň! To bude tvůj úkol. My jsme s tebou. Buď rozhodný a jednej!“
#10:5 Ezdráš vstal a vyzval předáky kněží, levitů i všeho Izraele k přísaze, že budou jednat podle toho slova. A oni přísahali.
#10:6 Pak odešel Ezdráš od Božího domu a šel do komory Jóchanana, syna Eljašíbova. Vešel tam, nejedl chleba a nepil vody a truchlil nad zpronevěrou přesídlenců.
#10:7 Dali rozhlásit všem synům přesídlenců v Judsku a Jeruzalémě, aby se shromáždili do Jeruzaléma.
#10:8 Nedostaví-li se někdo do tří dnů podle rozhodnutí předáků a starších, všechen jeho majetek propadne klatbě a on bude vyloučen ze shromáždění přesídlenců.
#10:9 K třetímu dni se shromáždili do Jeruzaléma všichni muži judští a benjamínští. Bylo to dvacátého dne devátého měsíce. Všechen lid se usadil na prostranství u Božího domu. Chvěli se kvůli té věci i kvůli dešťům.
#10:10 Tu povstal kněz Ezdráš a řekl jim: „Zpronevěřili jste se, že jste si přivedli ženy cizinky. Rozmnožili jste provinění Izraele.
#10:11 Nyní vzdejte chválu Hospodinu, Bohu svých otců, a jednejte podle jeho vůle. Oddělte se od národů země a od žen cizinek!“
#10:12 Celé shromáždění hlasitě odpovědělo: „Ano, budeme jednat podle tvého slova.
#10:13 Avšak lidu je mnoho a je doba dešťů. Nemůžeme stát venku. To není záležitost, která se vyřídí za den nebo dva, vždyť tohoto přestupku se dopustilo mnoho z nás.
#10:14 Ať naši předáci zastupují celé shromáždění. Každý, kdo si v našich městech přivedl ženu cizinku, přijde v určený čas a s ním starší příslušného města a jeho soudcové. Je třeba, abychom od sebe odvrátili planoucí hněv svého Boha a odstranili tuto věc.“
#10:15 Jen Jónatan, syn Asáelův, a Jachzejáš, syn Tikvův, se postavili proti tomu, a Mešulám a levita Šabetaj je podporovali.
#10:16 Ale synové přesídlenců to učinili tak, že kněz Ezdráš si oddělil muže, představitele rodů, všechny jmenovitě podle jejich otcovských rodů. Zasedli prvního dne desátého měsíce, aby věc vyšetřili.
#10:17 Do prvního dne prvního měsíce dokončili šetření u všech mužů, kteří si přivedli ženy cizinky.
#10:18 Ukázalo se, že z kněžských synů si přivedli ženy cizinky tito: ze synů Jéšuy, syna Jósadakova, a z jeho bratří: Maasejáš, Elíezer, Jaríb a Gedaljáš.
#10:19 Podáním ruky slíbili, že své ženy propustí. Za své provinění obětovali berana.
#10:20 Ze synů Imerových: Chananí a Zebadjáš.
#10:21 Ze synů Charimových: Maasejáš, Elijáš, Šemajáš, Jechíel a Uzijáš.
#10:22 Z Pašchúrových synů: Eljóenaj, Maasejáš, Jišmael, Netanel, Józabad a Eleasa.
#10:23 Z levitů: Józabad, Šimeí, Kelajáš čili Kelíta, Petachjáš, Júda a Elíezer.
#10:24 Ze zpěváků: Eljašíb. Z vrátných: Šalum, Telem a Úri.
#10:25 Z ostatního Izraele: Z Pareóšových synů: Ramjáš, Jizijáš, Malkijáš, Mijamín, Eleazar, Malkijáš a Benajáš.
#10:26 Z Élamových synů: Matanjáš, Zekarjáš, Jechíel, Abdí, Jeremót a Elijáš.
#10:27 Ze Zatúových synů: Eljóenaj, Eljašíb, Matanjáš, Jeremót, Zabad a Azíza.
#10:28 Z Bebajových synů: Jóchanan, Chananjáš, Zabaj, Atlaj.
#10:29 Z Baníových synů: Mešulam, Malúk, Adajáš, Jašúb, Šeal, Jeramót.
#10:30 Z Pachat-moábových synů: Adna, Kelal, Benajáš, Maasejáš, Matanjáš, Besalel, Binúj a Menaše.
#10:31 Dále Charimovi synové: Elíezer, Jišijáš, Malkijáš, Šemajáš, Šimeón,
#10:32 Binjamín, Malúk, Šemarjáš.
#10:33 Z Chašúmových synů: „Matenaj, Matata, Zabad, Elípelet, Jeremaj, Menaše, Šimeí.
#10:34 Z Baníových synů: Maadaj, Amram a Úel,
#10:35 Benajáš, Bedjáš, Keluhí,
#10:36 Vanjáš, Meremót, Eljašíb,
#10:37 Matanjáš, Matenaj, Jaasaj,
#10:38 Baní, Binúj, Šimeí,
#10:39 Šelemjáš, Nátan, Adajáš,
#10:40 Maknadbaj, Šašaj, Šaraj,
#10:41 Azarel, Šelemjáš, Šemarjáš,
#10:42 Šalúm, Amarjáš, Jósef.
#10:43 Z Nébových synů: Jeíel, Matitjáš, Zabad, Zebína, Jadaj, Jóel, Benajáš.
#10:44 Ti všichni si vzali ženy cizinky. Byly mezi nimi i ženy, s nimiž už zplodili syny.  

\book{Nehemiah}{Neh}
#1:1 Příběhy Nehemjáše, syna Chakaljášova. V měsíci kislevu dvacátého roku, když jsem byl na hradě v Šúšanu,
#1:2 přišel Chananí, jeden z mých bratrů, s muži z Judska. Zeptal jsem se jich na Judejce, na ty zbylé, kteří ušli zajetí, a na Jeruzalém.
#1:3 Řekli mi: „Ti zbylí, kteří zůstali tam v té krajině a nebyly zajati, jsou vystaveni veliké zlobě a potupě. Jeruzalémské hradby jsou pobořeny a brány zničeny ohněm.“
#1:4 Když jsem slyšel ta slova, usedl jsem a plakal. Truchlil jsem několik dní, postil jsem se před Bohem nebes a modlil se k němu.
#1:5 Říkal jsem: „Ach, Hospodine, Bože nebes, Bože veliký a hrozný, který zachováváš smlouvu a jsi milosrdný k těm, kdo tě milují a zachovávají tvá přikázání.
#1:6 Kéž je tvé ucho ochotné slyšet a oči tvé pohotové vidět, abys vyslyšel modlitbu svého služebníka, kterou se před tebou modlím ustavičně dnem i nocí za syny Izraele, tvoje služebníky. Vyznávám hříchy synů izraelských, kterých jsme se proti tobě dopustili; hřešili jsme i já a dům mého otce.
#1:7 Počínali jsme si vůči tobě hanebně, nezachovávali jsme přikázání, řády a práva, které jsi vydal svému služebníku Mojžíšovi.
#1:8 Rozpomeň se prosím na slovo, které jsi přikázal svému služebníku Mojžíšovi: ‚Zpronevěříte-li se, já sám vás rozptýlím mezi národy.
#1:9 Když se však obrátíte ke mně a budete má přikázání zachovávat a podle nich jednat, pak i kdyby někteří z vás byly zahnáni až na okraj světa, i odtamtud je shromáždím a přivedu na místo, které jsem vyvolil, aby tam přebývalo mé jméno.‘
#1:10 Vždyť to jsou tvoji služebníci, tvůj lid, který jsi vykoupil svou velikou mocí a silnou rukou.
#1:11 Ach Panovníku, kéž je tvé ucho ochotné vyslyšet modlitbu tvého služebníka i modlitbu tvých služebníků, kteří usilují žít v bázni tvého jména. A dopřej dnes zdaru svému služebníku a dej mu najít u onoho muže slitování.“ Byl jsem totiž královským číšníkem. 
#2:1 V měsíci nísanu dvacátého roku Artaxerxova kralování, když před králem bylo víno, vzal jsem víno a podal je králi. Nikdy jsem před ním nebýval ztrápený.
#2:2 Tu mi král řekl: „Proč vypadáš tak ztrápeně? Vždyť nejsi nemocný! Bezpochyby se něčím trápíš.“ Velmi jsem se ulekl
#2:3 a řekl jsem králi: „Ať žije král na věky! Jak bych neměl vypadat ztrápeně, když město, kde jsou hroby mých otců, leží v troskách a jeho brány jsou zničeny ohněm!“
#2:4 Král mi na to řekl: „Co si tedy přeješ?“ Pomodlil jsem se k Bohu nebes
#2:5 a odpověděl jsem králi: „Jestliže to král uzná za vhodné a jestliže tvůj služebník došel tvého zalíbení, propusť mě do Judska, kde jsou hroby mých otců, bych město znovu vystavěl.“
#2:6 Král, vedle něhož seděla královna, se mě zeptal: „Jak dlouho potrvá tvá cesta a kdy se vrátíš?“ Králi se zalíbilo mě propustit, jakmile jsem udal určitý čas.
#2:7 Řekl jsem králi: „Jestliže to král uzná za vhodné, nechť jsou mi dány doporučující listy pro místodržitele v Zaeufratí, aby mi poskytli doprovod, dokud nepřijdu do Judska,
#2:8 a také list pro správce královských obor Asafa, aby mi dodal dříví na trámy k branám hradu při domě Božím, na městské hradby i na dům, k němuž se mám vydat.“ Král mi je dal, neboť dobrotivá ruka mého Boha byla nade mnou.
#2:9 Když jsem přišel k místodržitelům v Zaeufratí, dal jsem jim královské doporučující listy. Král také se mnou poslal velitele vojska a jezdce.
#2:10 Jakmile to uslyšel Sanbalat Chorónský a Tóbijáš, ten amónský otrok, popadla je strašná zlost, že přišel někdo, komu jde o dobro synů Izraele.
#2:11 Tak jsem přišel do Jeruzaléma. Po třech dnech tamního pobytu
#2:12 jsem vstal v noci spolu s několika muži, ale neoznámil jsem nikomu, co mi můj Bůh vložil do srdce, abych udělal pro Jeruzalém. Neměl jsem s sebou žádné zvíře kromě toho, na němž jsem jel.
#2:13 Vyjel jsem v noci Údolní branou směrem k Dračí studni a k Hojné bráně. Přitom jsem si pozorně všímal pobořených jeruzalémských hradeb a bran zničených ohněm.
#2:14 Pak jsem pokračoval k Studničné bráně a ke královskému rybníku; zde však nebylo místo pro zvíře, aby se mnou prošlo.
#2:15 Ubíral jsem se tedy v noci vzhůru úvalem a pozorně jsem si všímal hradeb. Pak jsem se obrátil, vjel jsem Údolní branou a vrátil se zpět.
#2:16 Představenstvo nevědělo o tom, kde jsem chodil a co dělám; dosud jsem totiž nic neoznámil Judejcům, ani kněžím ani šlechtě ani představenstvu ani ostatním, kteří pracovali na tom díle.
#2:17 Teď jsem jim řekl: „Sami vidíte, jak zle jsme postiženi. Jeruzalém je v troskách, jeho brány jsou zničeny ohněm. Pojďte a stavějme jeruzalémské hradby. A nebudeme nadále v potupě.“
#2:18 Sdělil jsem jim, jak dobrotivá ruka mého Boha byla nade mnou, také i slova, jež mi řekl král. Odpověděli: „Nuže, dejme se do stavby!“ I vzchopili se k dobrému dílu.
#2:19 Když o tom uslyšel Sanbalat Chorónský, Tóbijáš, ten amónský otrok, a Gešem Arabský, vysmívali se nám. S pohrdáním nám říkali: „Do čeho se to pouštíte? Chcete se bouřit proti králi?“
#2:20 Nato jsem jim odpověděl a řekl: „Sám Bůh nebes způsobí, že se nám to podaří. My jsme jeho služebníci. Dali jsme se do stavby. Vy však nemáte žádný podíl ani právo ani památku v Jeruzalémě.“ 
#3:1 I dal se velekněz Eljašíb se svými bratry kněžími do stavby Ovčí brány. Když vsadili vrata, posvětili ji, posvětili ji až po věž Meá, a stavěli dál až k věži Chananeelu.
#3:2 Vedle něho stavěli muži z Jericha; vedle něho také stavěl Imrího syn Zakúr.
#3:3 Rybnou bránu stavěli synové Senáovi. Vyztužili ji trámy a vsadili vrata se zámky a závorami.
#3:4 Vedle nich opravoval Meremót, syn Úrijáše, syna Kósova; vedle nich také opravoval Mešulám, syn Berekjáše, syna Mešézabelova; vedle nich také opravoval Baanův syn Sádok.
#3:5 Vedle nich opravovali Tekójští; jejich vznešení však nesklonili šíji k službě pro svého Pána.
#3:6 Starou bránu opravovali Paséachův syn Jójada a Besódjášův syn Mešulám. Vyztužili ji trámy a vsadili vrata se zámky a závorami.
#3:7 Vedle nich opravoval Melatjáš Gibeónský a Jadón Meronótský, též muži z Gibeónu a z Mispy, jež patří pod správu zaeufratského místodržitele.
#3:8 Vedle něho opravoval Charhajášův syn Uzíel, jeden ze zlatníků, a vedle něho opravoval Chananjáš, jeden z mastičkářů; ustali až u jeruzalémské široké hradby.
#3:9 Vedle nich opravoval Chúrův syn Refajáš, správce poloviny jeruzalémského obvodu.
#3:10 Vedle nich také opravoval Charúmafův syn Jedajáš hned naproti svému domu. Vedle něho opravoval Chašabnejášův syn Chatúš.
#3:11 Další úsek spolu s Pecnou věží opravoval Charimův syn Malkijáš a Pachat-moábův syn Chašúb.
#3:12 Vedle něho spolu se svými dcerami opravoval Lochešův syn Šalúm, správce druhé poloviny jeruzalémského obvodu.
#3:13 Údolní bránu opravoval Chanún a obyvatelé Zanóachu. Vystavěli ji a vsadili vrata se zámky a závorami; opravili tisíc loket hradeb až po Hnojnou bránu.
#3:14 Hnojnou bránu opravoval Rekábův syn Malkijáš, správce obvodu bétkeremského; vystavěl ji a vsadil vrata se zámky a závorami.
#3:15 Studničnou bránu opravoval Kol-chozův syn Šalúm, správce obvodu Mispy; ten ji vystavěl, zastřešil a vsadil vrata se zámky a závorami; opravil také hradbu při vodojemu u královské zahrady až po schody vedoucí dolů z Města Davidova.
#3:16 Za ním opravoval Azbúkův syn Nechemjáš, správce poloviny obvodu bétsúrského, až naproti hrobům Davidovým, po umělou vodní nádrž a dům bohatýrů.
#3:17 Za ním také opravovali levité za vedení Baníova syna Rechúma. Vedle něho opravoval Chašabjáš, správce poloviny obvodu keílského, za svůj obvod.
#3:18 Za ním opravovali jejich bratří za vedení Chenadadova syna Bavaje, správce druhé poloviny obvodu keílského.
#3:19 Vedle něho také opravoval Jéšuův syn Ezer, správce Mispy, další úsek naproti přístupu ke zbrojnici v rohu hradby.
#3:20 Za ním horlivě opravoval Zabajův syn Barúk další úsek od rohu až ke vchodu do domu nejvyššího kněze Eljašíba.
#3:21 Za ním opravoval Meremót, syn Úrijáše, syna Kósova, další úsek od vchodu do domu Eljašíbova až na konec Eljašíbova domu.
#3:22 Za ním opravovali kněží, muži z jordánského okrsku.
#3:23 Za ním dále naproti svým domům, opravovali Binjamín a Chašúb; za ním, vedle svého domu, opravoval Azarjáš, syn Maasejáše, syna Ananejášova.
#3:24 Za ním opravoval Chenadadův syn Binúj další úsek od Azarjášova domu až po roh hradby, po nároží.
#3:25 Úzajův syn Pálal opravoval naproti rohu a vysoké věži vyčnívající nad domem královým na nádvoří pro stráže; za ním Pareóšův syn Pedajáš.
#3:26 Chrámoví nevolníci sídlili na Ófelu a opravovali proto na východě naproti Vodní bráně a tamní vyčnívající věži.
#3:27 Za ním opravovali Tekójští další úsek naproti vyčnívající vysoké věži až po zeď na Ófelu.
#3:28 Od Koňské brány dál opravovali kněží, každý naproti svému domu.
#3:29 Za nimi opravoval naproti svému domu Imerův syn Sádok a za ním opravoval Šekanjášův syn Šemajáš, strážce Východní brány.
#3:30 Za ním opravoval další úsek Šelemjášův syn Chananjáš a Salafův šestý syn Chanún; za ním opravoval Berekjášův syn Mešulám naproti své komůrce.
#3:31 Za ním opravoval Malkijáš, jeden ze zlatníků, až po dům chrámových nevolníků a obchodníků, naproti Strážné bráně, k přístřešku na nároží.
#3:32 Mezi přístřeškem na nároží a Ovčí branou opravovali zlatníci a obchodníci.
#3:33 Jakmile uslyšel Sanbalat, že stavíme hradby, vzplanul a velice se rozzlobil. Vysmíval se Judejcům
#3:34 a mluvil před svými bratry a před vojskem samařským takto: „Do čeho se to pouštějí ti židovští ubožáci? To je máme nechat? Budou snad obětovat a dokončí to ještě dnes? Oživí snad z hromad prachu kameny přepálené ohněm?“
#3:35 Ale Tóbijáš amónský stál vedle něho a řekl: „Jen ať si stavějí! Až přiběhne liška, protrhne jejich kamennou hradbu.“
#3:36 „Slyš, Bože náš, že jsme v opovržení. Obrať jejich hanění na jejich hlavu! Vydej je, aby se stali kořistí a zajatci cizí země.
#3:37 Nepřikrývej jejich nepravost, ať není jejich hřích před tebou smazán, neboť podněcují proti těm, kdo stavějí.“
#3:38 Stavěli jsme hradby dále, byly již zpoloviny hotovy, protože lid byl srdcem při práci. 
#4:1 Jakmile uslyšeli Sanbalat a Tóbijáš, Arabové, Amónci a Ašdóďané, že obnova jeruzalémských hradeb pokračuje, že se trhliny opět uzavírají, velmi vzplanuli.
#4:2 Vzájemně se smluvili, že vytáhnou proti Jeruzalému do boje a že v něm vyvolají zmatek.
#4:3 Modlili jsme se ke svému Bohu a stavěli proti nim na obranu stráže ve dne v noci.
#4:4 Juda říkal: „Síla nosičů je podlomena, všude jsou hromady sutin, nejsme schopni hradby dostavět.“
#4:5 Naši protivníci říkali: „Nic nepoznají a nic nepostřehnout, dokud mezi ně nevtrhneme. Pobijeme je a to dílo překazíme.“
#4:6 Avšak Judejci, kteří sídlili v jejich blízkosti, přicházeli a varovali nás znovu a znovu: „Ze všech míst, kam se obrátíte, jdou proti nám!“
#4:7 Proto jsem postavil stráž pod svatým místem při hradbách na nechráněných úsecích. Postavil jsem tam lid podle čeledí s meči, kopími a luky.
#4:8 Když jsem všechno přehlédl, šel jsem a řekl šlechticům, představenstvu a ostatnímu lidu: „Nebojte se jich, pamatujte na Panovníka velikého a budícího bázeň! Bojujte za své bratry, syny a dcery, ženy a domy!“
#4:9 Jakmile naši nepřátelé uslyšeli, že je nám vše známo, pochopili, že Bůh překazil jejich záměr. My všichni jsme se vrátili na hradby, každý ke svému dílu.
#4:10 Avšak od onoho dne pracovala na díle jen polovina mých lidí, zatímco druhá polovina byla ozbrojena kopími, štíty, luky a pancíři. Velmožové stáli při celém domě judském.
#4:11 Ti, kdo stavěli hradby, nosiči břemen i nakladači, pracovali jednou rukou na díle a v druhé drželi zbraň.
#4:12 Všichni, kdo stavěli, měli meč připásaný na bedrech a tak stavěli, a vedle mne stál trubač.
#4:13 Řekl jsem šlechticům, představenstvu a ostatnímu lidu: „Dílo je veliké a rozsáhlé a my jsme rozděleni po hradbách daleko od sebe.
#4:14 Uslyšíte-li z některého místa zvuk polnice, shromážděte se tam k nám. Náš Bůh bude bojovat za nás.“
#4:15 A pokračovali jsme v díle; polovina mužů držela kopí od svítání až do soumraku.
#4:16 V té době jsem také řekl lidu: „Každý se svým služebníkem nocujte uprostřed Jeruzaléma; v noci nás budou střežit a ve dne pracovat.“
#4:17 Nikdo z nás si nesvlékal šat, ani já ani moji bratři ani moji lidé ani strážní, které jsme měli při sobě... 
#5:1 Strhl se pak veliký křik lidu, zvláště žen, proti jejich bratřím Judejcům.
#5:2 Jedni říkali: „Je nás mnoho s našimi syny a dcerami, musíme shánět obilí, abychom měli co jíst a zůstali naživu.“
#5:3 Jiní říkali: „Svá pole, vinice a domy dáváme do zástavy, abychom sehnali obilí v tom hladu.“
#5:4 Jiní opět říkali: „Musili jsme si vypůjčit na svá pole a vinice peníze na daň pro krále.
#5:5 A přece jsme se svými bratry jedno tělo, naše děti jsou jako jejich děti. Jenže my musíme své syny a dcery prodávat do otroctví, některé z našich dcer se již otrokyněmi staly, a nemůžeme nic proti tomu dělat. Naše pole a vinice patří jiným.“
#5:6 Když jsem slyšel jejich křik a tato slova, vzplanul jsem hněvem.
#5:7 Rozhodl jsem se, že proti šlechticům a představenstvu povedu při. Vytkl jsem jim: „Půjčujete svým bratřím na lichvářský úrok“, a svolal jsem proti nim velké shromáždění.
#5:8 Řekl jsem jim: „My jsme vykoupili, pokud nám bylo možno, své bratry Judejce, prodané pohanům. Naproti tomu vy své bratry prodáváte, aby byli prodáváni nám.“ Zmlkli a nenašli slov.
#5:9 Pokračoval jsem: „Co děláte, není dobré. Což nemáte žít v bázni před naším Bohem a uvarovat se pohanění ze strany pohanů, našich nepřátel?
#5:10 Také já, moji bratří a moji lidé, bychom od nich mohli požadovat úrok z peněz a obilí. Upustíme však od toho dluhu.
#5:11 Vraťte jim ještě dnes jejich pole, vinice, olivoví a domy i příslušnou částku peněz, obilí, moštu a čerstvého oleje, za něž jste jim půjčku poskytli.“
#5:12 Odpověděli: „Vrátíme a nic od nich nebudeme požadovat. Uděláme to tak, jak jsi řekl.“ Svolal jsem tedy kněze a zavázal lid přísahou, že toto slovo dodrží.
#5:13 Také jsem vytřepal záhyby svého roucha a řekl jsem: „Právě tak nechť vytřepe Bůh každého muže, který toto slovo nedodrží, z jeho domu a vlastnictví; tak buď vytřepán a vyprázdněn.“ Nato celé shromáždění odpovědělo: „Amen.“ I chválili Hospodina, a lid to slovo dodržel.
#5:14 Také ode dne, kdy mě král ustanovil místodržitelem v zemi judské, od dvacátého roku až do třicátého druhého roku krále Artaxerxa, po dobu dvanácti let, jsem ani já ani moji bratři nebrali místodržitelské příjmy.
#5:15 Dřívější místodržitelé, moji předchůdci, zatěžovali totiž lid tím, že na něm vymáhali mimo pokrm a víno čtyřicet šekelů stříbra, rovněž jejich lidé vykořisťovali lid. Já však jsem tak nejednal z bázně před Bohem.
#5:16 Nadto jsem se účastnil oprav hradeb. Pole jsme neskupovali, ale všichni moji lidé tam byli zapojeni do díla.
#5:17 Judejci a představenstvo, sto padesát osob, mimo ty, kteří k nám přicházeli z okolních pohanů, sedali u mého stolu.
#5:18 Bylo třeba každý den připravovat jednoho býka a šest vybraných ovcí. Dále byla pro mě připravována drůbež a jednou za deset dní množství různého vína. Přitom jsem neuplatňoval nárok na místodržitelské příjmy, protože na tomto lidu už těžce spočívala služba.
#5:19 „Pamatuj na mne, Bože můj, v dobrém, na všechno, co jsem pro tento lid učinil.“ 
#6:1 Když se doneslo Sanbalatovi, Tóbijášovi, Gešemovi Arabskému a ostatním našim nepřátelům, že jsem dostavěl hradby, takže v nich nezbylo trhliny - ovšem do té doby jsem ještě nevsadil vrata do bran -,
#6:2 poslal ke mně Sanbalat a Gešem se vzkazem: „Přijď, sejdeme se spolu v některé vsi na pláni Ónu.“ Zamýšleli však provést mi něco zlého.
#6:3 Poslal jsem k nim tedy posly se vzkazem: „Dělám veliké dílo, proto nemohu odejít. Mělo by se to dílo snad přerušit tím, že bych je opustil a sestoupil k vám?“
#6:4 Stejný vzkaz mi poslali čtyřikrát a já jsem jim pokaždé stejně odpověděl.
#6:5 Stejným způsobem poslal ke mně Sanbalat po páté svého služebníka, a to s otevřeným dopisem.
#6:6 Bylo v něm napsáno: „Proslýchá se mezi pohany, i Gašmu to říká, že ty a Judejci pomýšlíte na vzpouru, že proto stavíš hradby, a ty sám prý se chceš stát jejich králem.
#6:7 Také jsi prý ustanovil proroky, aby o tobě v Jeruzalémě provolávali: ‚To je král judský.‘ Tato slova se jistě donesou ke králi. Proto přijď, poradíme se spolu.“
#6:8 Odpověděl jsem mu: „Nic z toho, co říkáš, se nestalo. Ty sám sis to vymyslel.“
#6:9 Ti všichni nás chtěli zastrašit. Říkali si: „Upustí od díla a ono se neuskuteční.“ „Proto nyní, Bože, posilni mé ruce!“
#6:10 Vstoupil jsem do domu Šemajáše, syna Delajáše, syna Mehétabelova, který žil odloučeně. Řekl: „Sejděme se v domě Božím, uvnitř chrámu; musíme však zavřít chrámová vrata, protože se tě chystají zavraždit; v noci přijdou, aby tě zavraždili.“
#6:11 Odpověděl jsem: „Člověk jako já má prchat? Což může někdo jako já vstoupit do chrámu a zůstat naživu? Nevstoupím.“
#6:12 Zjistil jsem totiž, že ho Bůh neposlal. To proroctví mluvil proti mně, protože ho najali Tóbijáš a Sanbalat.
#6:13 Byl najat proto, abych ze strachu tak jednal a zhřešil. Chtěli toho využít, aby pošpinili mé jméno a potupili mě.
#6:14 „Pamatuj, můj Bože, na Tóbijáše a na Sanbalata, na tyto jejich skutky, a také na prorokyni Nóadju a na ostatní proroky, kteří mě zastrašovali.“
#6:15 Hradby byly úplně dostavěny dvacátého pátého dne měsíce elúlu, za dvaapadesát dní.
#6:16 Jak o tom uslyšeli všichni naši nepřátelé a spatřili to všichni pohané z okolí, velice zmalomyslněli. Poznali totiž, že toto dílo bylo vykonáno s pomocí našeho Boha.
#6:17 V oněch dnech judští šlechtici často posílali dopisy Tóbijášovi a od Tóbijáše přicházely dopisy jim.
#6:18 Mnozí v Judsku s ním totiž byli spjati přísahou, neboť byl zetěm Arachova syna Šekanjáše a jeho syn Jóchanan si vzal dceru Berekjášova syna Mešuláma.
#6:19 Také ho přede mnou vychvalovali a má slova donášeli jemu. A Tóbijáš posílal dopisy, aby mě zastrašoval. 
#7:1 Když byly dostavěny hradby, vsadil jsem vrata. Byli také ustanoveni vrátní, zpěváci a levité.
#7:2 Pak jsem dal ohledně Jeruzaléma příkaz svému bratru Chananímu a správci hradu Chananjášovi, který byl mužem věrnějším a bohabojnějším než mnozí jiní.
#7:3 Řekl jsem jim: „Jeruzalémské brány nesmějí být otvírány dříve, než slunce začne hřát; a když strážci uzavřou vrata, vy zajistěte závory. Stráže budou stavěny z obyvatelů Jeruzaléma, každý bude mít své stanoviště naproti svému domu.“
#7:4 Město bylo rozlehlé a veliké, ale lidu v něm bylo málo a domy nebyly dostavěny.
#7:5 Tu mi můj Bůh vložil do srdce, abych shromáždil šlechtice, představenstvo i lid, aby byli zapsáni do seznamu rodů. Našel jsem písemný záznam o rodu těch, kteří přišli dříve, a našel jsem v něm napsáno:
#7:6 Toto jsou příslušníci judského kraje, kteří přišli ze zajetí, přesídlenci, které přesídlil babylónský král Nebúkadnesar. Navrátili se do Jeruzaléma a do Judska, každý do svého města.
#7:7 Přišli se Zerubábelem, Jéšuou, Nechemjášem, Azarjášem, Raamjášem, Nachamaním, Mordokajem, Bilšánem, Misperetem, Bigvajem, Nechúmem a Baanou. Soupis mužů izraelského lidu:
#7:8 Synů Pareóšových dva tisíce sto sedmdesát dva;
#7:9 synů Šefatjášových tři sta sedmdesát dva;
#7:10 synů Arachových šest set padesát dva;
#7:11 synů Pachat-moábových, totiž synů Jéšuových a Jóabových, dva tisíce osm set osmnáct;
#7:12 synů Élamových dvanáct set padesát čtyři;
#7:13 synů Zatúových osm set čtyřicet pět;
#7:14 synů Zakajových sedm set šedesát;
#7:15 synů Binújových šest set čtyřicet osm;
#7:16 synů Bebajových šest set dvacet osm;
#7:17 synů Azgadových dva tisíce tři sta dvacet dva;
#7:18 synů Adoníkamových šest set šedesát sedm;
#7:19 synů Bigvajových dva tisíce šedesát sedm;
#7:20 synů Adínových šest set padesát pět;
#7:21 synů Aterových, totiž Chizkijášových, devadesát osm;
#7:22 synů Chašumových tři sta dvacet osm;
#7:23 synů Besajových tři sta dvacet čtyři;
#7:24 synů Charífových sto dvanáct;
#7:25 synů gibeónských devadesát pět;
#7:26 mužů betlémských a netófských sto osmdesát osm;
#7:27 mužů anatótských sto dvacet osm;
#7:28 mužů bétazmávetských čtyřicet dva;
#7:29 mužů kirjatjearímských, kefírských a beerótských sedm set čtyřicet tři;
#7:30 mužů rámských a gebských šest set dvacet jeden;
#7:31 mužů mikmáských sto dvacet dva;
#7:32 mužů bételských a ajských sto dvacet tři;
#7:33 mužů z druhého Neba padesát dva;
#7:34 synů druhého Élama dvanáct sed padesát čtyři;
#7:35 synů Charimových tři sta dvacet;
#7:36 synů jerišských tři sta čtyřicet pět;
#7:37 synů lódských, chadídských a ónoských sedm set dvacet jeden;
#7:38 synů Seáových tři tisíce devět set třicet.
#7:39 Kněží: synů Jedajášových z domu Jéšuova devět set sedmdesát tři;
#7:40 synů Imerových tisíc padesát dva;
#7:41 synů Pašchúrových dvanáct set čtyřicet sedm;
#7:42 synů Charimových tisíc sedmnáct.
#7:43 Levité: synů Jéšuových, totiž Kadmíelových, totiž synů Hódevových, sedmdesát čtyři.
#7:44 Zpěváci: synů Asafových sto čtyřicet osm.
#7:45 Vrátní: synů Šalúmových, synů Aterových, synů Talmónových, synů Akúbových, synů Chatítových a synů Šobajových sto třicet osm.
#7:46 Chrámoví nevolníci: synové Síchovi, synové Chasúfovi, synové Tabaótovi,
#7:47 synové Kérosovi, synové Síaovi, synové Padónovi,
#7:48 synové Lebánovi, synové Chagábovi, synové Šalmajovi,
#7:49 synové Chananovi, synové Gidélovi, synové Gacharovi,
#7:50 synové Reajášovi, synové Resínovi, synové Nekódovi,
#7:51 synové Gazamovi, synové Uzovi, synové Paséachovi,
#7:52 synové Besajovi, synové Meúnejců, synové Nefíšesejců,
#7:53 synové Bakbúkovi, synové Chakúfovi, synové Charchúrovi,
#7:54 synové Baslítovi, synové Mechídovi, synové Charšovi,
#7:55 synové Barkósovi, synové Síserovi, synové Tamachovi,
#7:56 synové Nesíachovi a synové Chatífovi.
#7:57 Synové služebníků Šalomounových: synové Sótajovi, synové Sóferetovi, synové Perídovi,
#7:58 synové Jaalovi, synové Darkónovi, synové Gidélovi,
#7:59 synové Šefatjášovi, synové Chatílovi, synové Pokereta Sebajimského a synové Amónovi.
#7:60 Všech chrámových nevolníků a synů Šalomounových služebníků tři sta devadesát dva.
#7:61 Tito vyšli z Tel-melachu, z Tel-charši, z Kerúb-adónu a Imeru, ale nemohli prokázat, že jejich otcovský rod a původ je z Izraele:
#7:62 synů Delajášových, synů Tóbijášových a synů Nekódových šest set čtyřicet dva.
#7:63 A z kněží synové Chobajášovi, synové Kósovi a synové Barzilaje, který si vzal za ženu jednu z dcer Barzilaje Gileádského a je nazýván jejich jménem.
#7:64 Ti hledali svůj rodokmen v seznamu rodů, ale marně; proto byli vyloučeni z kněžství jako nezpůsobilí.
#7:65 Místodržící jim zapověděl jíst ze svatých přídělů kněžských, pokud nebude ustanoven kněz pro posvátné losy urím a tumím.
#7:66 Celé shromáždění dohromady čítalo čtyřicet dva tisíce tři sta šedesát duší,
#7:67 mimo jejich otroky a otrokyně, jichž bylo sedm tisíc tři sta třicet sedm. Měli také dvě stě čtyřicet pět zpěváků a zpěvaček.
#7:68 Velbloudů bylo čtyři sta třicet pět, oslů šest tisíc sedm set dvacet.
#7:69 Z představitelů rodů někteří přispěli na dílo. Místodržící dal na chrámový poklad tisíc zlatých darejků, padesát kropenek, pět set třicet kněžských suknic.
#7:70 Představitelé rodů dali na chrámový poklad pro potřeby díla dvacet tisíc zlatých darejků a dva tisíce dvě stě hřiven stříbra.
#7:71 A toho, co dal ostatní lid, bylo dvacet tisíc zlatých darejků, dva tisíce hřiven stříbra a šedesát sedm kněžských suknic.
#7:72 I usadili se kněží, levité, vrátní, zpěváci i někteří z lidu a chrámoví nevolníci i všechen Izrael ve svých městech. Když nastal sedmý měsíc, byli již Izraelci ve svých městech. 
#7:73 
#8:1 Všechen lid se shromáždil jednomyslně na prostranství před Vodní branou a vyzvali znalce Zákona Ezdráše, aby přinesl knihu Mojžíšova zákona, který vydal Hospodin Izraeli.
#8:2 Kněz Ezdráš tedy přinesl Zákon před shromáždění mužů i žen, všech, kteří byli schopni rozumět, aby jej slyšeli, prvního dne měsíce sedmého.
#8:3 Četl z něho na prostranství před Vodní branou od svítání až do poledne mužům i ženám, těm, kteří byli schopni rozumět. Všechen lid pozorně naslouchal slovům z knihy Zákona.
#8:4 Znalec zákona Ezdráš stál na dřevěném lešení, které k tomu účelu zhotovili; po pravici vedle něho stál Matitjáš, Šema, Anajáš, Úrijáš, Chilkijáš a Maasejáš, po levici pak Pedajáš, Míšael, Malkijáš, Chašum, Chašbadána, Zekarjáš a Mešulám.
#8:5 I otevřel Ezdráš knihu před zraky všeho lidu, stál totiž výše než ostatní lid. Když ji otevřel, všechen lid povstal.
#8:6 Ezdráš dobrořečil Hospodinu, Bohu velikému, a všechen lid odpověděl s pozdviženýma rukama: „Amen, amen.“ Padli na kolena a klaněli se Hospodinu tváří až k zemi.
#8:7 Nato Jéšua, Baní, Šerebjáš, Jamín, Akúb, Šabetaj, Hódijáš, Maasejáš, Kelíta, Azarjáš, Józabad, Chanan a Pelajáš, levité, vysvětlovali lidu Zákon, zatímco lid stál na svém místě.
#8:8 Četli z knihy Božího zákona po oddílech a vykládali smysl, aby lid rozuměl tomu, co četli.
#8:9 Nehemjáš, který byl místodržícím, a kněz Ezdráš, znalec Zákona, a levité, kteří vysvětlovali lidu Zákon, řekli všem lidu: „Dnešní den je svatý Hospodinu, vašemu Bohu. Netruchlete a neplačte.“ Všechen lid totiž plakal, když slyšel slova Zákona.
#8:10 Dále jim řekl: „Jděte, jezte tučná jídla a pijte sladké nápoje a posílejte dárky těm, kdo nemají nic připraveno. Dnešní den je zajisté svatý našemu Pánu. Netrapte se! Radost z Hospodina bude vaší záštitou.“
#8:11 Také levité utěšovali všechen lid: „Utište se. Dnešní den je svatý, netrapte se!“
#8:12 I rozešel se všechen lid, aby jedli a pili a posílali dárky; uspořádali velmi radostnou slavnost, protože porozuměli slovům, která jim byla zvěstována.
#8:13 Na druhý den se pak shromáždili představitelé rodů ze všeho lidu, kněží a levité ke znalci Zákona Ezdrášovi, aby si vyložili slova Zákona.
#8:14 Našli totiž zapsáno v Zákoně, jejž Hospodin vydal skrze Mojžíše, že Izraelci mají při slavnosti v sedmém měsíci sídlit ve stáncích
#8:15 a vyhlásit a rozhlásit ve všech svých městech a v Jeruzalémě: „Vyjděte na hory, přineste větve oliv šlechtěných i planých i myrtové a palmové a jiných listnatých stromů, abyste mohli udělat stánky, jak je psáno.“
#8:16 Lid vyšel, přinesli větve a udělali stánky, každý na své střeše nebo na dvorku i na nádvořích Božího domu, na prostranství u Vodní brány a na prostranství u Efrajimovy brány.
#8:17 Celé shromáždění těch, kdo se vrátili ze zajetí, si udělalo stánky a sídlili v nich, což Izraelci nedělali od dob Jozua, syna Núnova, až do onoho dne. Byla z toho převeliká radost.
#8:18 Z knihy Božího zákona se četlo den co den, od prvního až do posledního dne. Po sedm dní slavili slavnost. Osmého dne bylo slavnostní shromáždění, jak je ustanoveno. 
#9:1 Dvacátého čtvrtého dne toho měsíce se Izraelci shromáždili k postu v žíněných rouchách a s prstí na hlavě.
#9:2 Izraelovi potomci se oddělili ode všech cizinců. Stáli a vyznávali své hříchy i nepravosti svých otců.
#9:3 Povstávali ze svých míst, když četli čtyřikrát za den z knihy Zákona Hospodina, svého Boha, a čtyřikrát se vyznávali a klaněli Hospodinu, svému Bohu.
#9:4 Na výstupek pro levity se postavili Jéšua a Baní, Kadmíel, Šebanjáš, Buní, Šerebjáš, Baní, Kenaní a velmi hlasitě úpěli k Hospodinu, svému Bohu.
#9:5 Pak vyzvali levité Jéšua a Kadmíel, Baní, Chašabnejáš, Šerebjáš, Hódijáš, Šebanjáš a Petachjáš lid: „Povstaňte, dobrořečte Hospodinu, svému Bohu, po všechny věky. Ať dobrořečí tvému slavnému jménu, vyvýšenému nad každé dobrořečení a chválu.
#9:6 Ty, Hospodine, jsi ten jediný, ty jsi učinil nebe, nebesa nebes a všechen jejich zástup, zemi i vše, co je na ní, moře i vše, co je v nich. Sám to všechno zachováváš při životě a nebeské zástupy se ti klanějí.
#9:7 Ty, Hospodine, jsi Bůh; ty jsi vyvolil Abrama a vyvedls jej z Ur-kasdímu a dals mu jméno Abraham.
#9:8 Shledal jsi, že jeho srdce je ti věrné, a uzavřel jsi s ním smlouvu, že zemi Kennanců, Chetejců, Emorejců a Perizejců, Jebúsejců a Girgašijců dáš jeho potomstvu. Dostál jsi svému slovu, neboť jsi spravedlivý.
#9:9 Viděl jsi utrpení našich otců v Egyptě a slyšel jsi jejich úpění u Rákosového moře.
#9:10 I učinil jsi znamení a zázraky proti faraónovi a proti všem jeho služebníkům i proti všemu lidu jeho země, protože jsi poznal, jak se nad nimi zpupně vyvyšovali. Tak sis učinil jméno, jaké máš až dodnes.
#9:11 Rozpoltil jsi před nimi moře a oni prošli mořem po suchu; ale jejich pronásledovatele jsi uvrhl do hlubin jako kámen do dravých vod.
#9:12 Ve dne jsi je vodil sloupem oblakovým, v noci sloupem ohnivým, aby jim osvětloval cestu, po níž by šli.
#9:13 Sestoupil jsi na horu Sínaj a mluvil jsi s nimi z nebe, vydal jsi jim přímé řády a spolehlivá naučení, dobrá nařízení a přikázání.
#9:14 Učinil jsi jim známý svůj svatý den odpočinku, příkazy, nařízení a zákon jsi jim vydal skrze svého služebníka Mojžíše.
#9:15 Dal jsi jim chléb z nebe, když hladověli, vyvedl jsi jim vodu ze skaliska, když žíznili. A řekl jsi jim, aby šli obsadit zemi, o které jsi s pozvednutím ruky přisáhl, že jim ji dáš.
#9:16 Ale oni, naši otcové, se zpupně vyvyšovali, byli tvrdošíjní a neposlouchali tvé příkazy.
#9:17 Odmítli poslouchat a ani si nevzpomněli na divuplné skutky, které jsi pro ně činil. Byli tvrdošíjní a vzali si vzdorně do hlavy, že se vrátí do svého otroctví. Ty jsi však Bůh ochotný odpouštět, milostivý a soucitný, shovívavý a nesmírně milosrdný, proto jsi je neopustil.
#9:18 Dokonce si odlili sochu býčka a řekli: ‚To je tvůj Bůh, který tě vyvedl z Egypta‘. Dopustili se strašného rouhání.
#9:19 Avšak ty jsi je v nesmírném slitování na poušti neopustil. Ve dne od nich neodcházel sloup oblakový, který je vodil po cestě, a v noci sloup ohnivý, který jim osvětloval cestu, po níž by šli.
#9:20 Dával jsi svého dobrého ducha, aby je poučoval. Neodnímal jsi jim od úst svou manu a dával jsi jim vodu, když žíznili.
#9:21 Po čtyřicet let ses o ně na poušti staral, nic jim nescházelo, pláště jim nezvetšely a nohy jim neotekly.
#9:22 Potom jsi jim dal království a národy, přidělil jsi jim každý kout. Obsadili zemi Síchonovu, totiž zemi krále Chešbónského, i zemi bášanského krále Óga.
#9:23 Rozmnožil jsi jejich syny jako hvězdy nebeské, uvedls je do země, o níž jsi řekl jejich otcům, aby ji šli obsadit.
#9:24 A synové do té země vešli a obsadili ji. Pokořil jsi před nimi obyvatele země, Kenaance, a vydals jim je do rukou, i jejich krále a národy země, aby s nimi naložili podle své vůle.
#9:25 Dobyli opevněná města a žírnou zemi a obsadili domy plné všelijakých dobrých věcí, vytesané nádrže, vinice, olivoví a množství ovocného stromoví. Jedli a nasytili se, ztučněli a žili v rozkoši z tvé veliké dobroty.
#9:26 Ale začali být vzpurní a bouřili se proti tobě. Tvůj zákon zavrhli a vraždili tvé proroky, kteří je varovali, aby je obrátili zpět k tobě. Dopustili se strašného rouhání.
#9:27 Proto jsi je vydal do rukou jejich protivníků, aby je sužovali. Když však v dobách svého soužení úpěli k tobě, tys je z nebe vyslýchal a z nesmírného slitování jsi jim dával vysvoboditele, aby je vysvobozovali z rukou jejich protivníků.
#9:28 Ale jen si oddechli, zase páchali, co je před tebou zlé. Proto jsi je nechal napospas jejich nepřátelům, aby nad nimi panovali. Když však opět k tobě úpěli, tys je z nebe vyslýchal a ve svém slitování je nesčetněkrát zachraňoval.
#9:29 Dával jsi jim výstrahu, abys je přivedl zpět ke svému zákonu. Oni se však zpupně vyvyšovali, neposlouchali tvé příkazy a prohřešovali se proti tvým řádům, které dávají život tomu, kdo se jimi řídí. Umíněně se obraceli zády, byli tvrdošíjní a neposlouchali.
#9:30 Měl jsi s nimi trpělivost po mnohá léta. Skrze své proroky jsi jim dával výstrahu svým duchem, ale neposlouchali. Proto jsi je vydal do rukou národů cizích zemí.
#9:31 Ale z nesmírného slitování jsi s nimi neskoncoval a neopustil jsi je, protože jsi Bůh milostivý a soucitný.
#9:32 Nyní tedy, náš Bože, Bože veliký, mocný a budící bázeň, který zachováváš smlouvu a milosrdenství, nepokládej za málo všechny ty útrapy, které dolehly na nás, na naše krále, předáky, kněze a proroky, na naše otce a na všechen tvůj lid od doby asyrských králů až dodnes.
#9:33 Ty jsi spravedlivý ve všem, co na nás přišlo, neboť jsi jednal věrně, kdežto my jsme si počínali svévolně.
#9:34 Naši králové, předáci, kněží a otcové se neřídili tvým zákonem, nedbali na tvá přikázání a na tvá svědectví, kterými jsi je varoval.
#9:35 Ve svém království, přes tvoji mnohou dobrotu, kterou jsi jim prokazoval, v širé a žírné zemi, kterou jsi jim dal, nesloužili tobě a neodvrátili se od svých zlých skutků.
#9:36 Proto jsme dnes otroky. V té zemi, kterou jsi dal našim otcům, aby jedli její plody a dobroty, jsme otroky.
#9:37 Její bohatá úroda připadá králům, které jsi nad námi ustanovil pro naše hříchy; vládnou nad našimi těly i nad naším dobytkem, jak se jim zlíbí. Jsme ve velikém soužení.“ 
#10:1 Vzhledem k tomu všemu uzavíráme a sepisujeme pevnou dohodu zpečetěnou našimi předáky, levity a kněžími.
#10:2 Pečetě připojili místodržící Nehemjáš, syn Chakaljášův, a Sidkijáš,
#10:3 Serajáš, Azarjáš, Jirmejáš,
#10:4 Pašchúr, Amarjáš, Malkijáš,
#10:5 Chatúš, Šebanjáš, Malúk,
#10:6 Charim, Meremót, Obadjáš,
#10:7 Daníel, Gintón, Barúk,
#10:8 Mešulam, Abijáš, Mijamín,
#10:9 Maazjáš, Bilgaj a Šemajáš. To jsou kněží.
#10:10 Levité: Azanjášův syn Jéšua, Binúj ze synů Chénadadových, Kadmíel
#10:11 a jejich bratří Šebanjáš, Hódijáš, Kelíta, Pelajáš, Chanan,
#10:12 Míka, Rechób, Chašabjáš,
#10:13 Zakúr, Šerebjáš, Šebanjáš,
#10:14 Hódijáš, Baní a Benínu.
#10:15 Představitelé lidu: Pareóš, Pachat-moáb, Élam, Zatú, Baní,
#10:16 Buní, Azgád, Bebaj,
#10:17 Adonijáš, Bigvaj, Adín,
#10:18 Ater, Chizkijáš, Azúr,
#10:19 Hódijáš, Chašum, Besaj,
#10:20 Charíf, Anatót, Nébaj,
#10:21 Magpíaš, Mešulám, Chezír,
#10:22 Mešézabel, Sádok, Jadúa,
#10:23 Pelatjáš, Chanan, Anajáš,
#10:24 Hóšea, Chananjáš, Chašúb,
#10:25 Lócheš, Pilchá, Šobek,
#10:26 Rechúm, Chašabna, Maasejáš
#10:27 a Achijáš, Chanan, Anan,
#10:28 Malúk, Charim a Baana.
#10:29 Ostatní lid, kněží, levité, vrátní, zpěváci, chrámoví nevolníci a všichni, kdo se oddělili od národů zemí k Božímu zákonu, také jejich ženy, synové a dcery, každý, kdo to byl schopen pochopit,
#10:30 připojovali se ke svým vznešeným bratřím. Přicházeli a přísežně se zaklínali, že budou žít podle Božího zákona, který byl vydán skrze Božího služebníka Mojžíše, a že budou zachovávat a plnil všechna přikázání Hospodina, našeho Pána, jeho práva a jeho nařízení.
#10:31 Prohlásili: „Nebudeme dávat své dcery národům země a jejich dcery nebudeme brát pro své syny.
#10:32 Když budou přinášet národy země v den odpočinku k prodeji různé zboží a obilí, v den odpočinku, totiž ve svatý den, je od nich nebudeme brát. Sedmého léta necháme pole ladem a nebudeme vymáhat dluhy.
#10:33 Budeme se podřizovat příkazům a odevzdávání třetiny šekelu ročně pro službu v domě našeho Boha,
#10:34 na předkladné chleby, na pravidelný obětní dar, na pravidelnou oběť zápalnou ve dnech odpočinku a o novoluních, na slavnosti, na svaté dávky, na smírčí oběti za hřích za Izraele, vůbec na všechno dílo domu našeho Boha.
#10:35 Losem jsme rozvrhli dodávky dříví mezi kněze, levity a lid, jak je mají přinášet do domu našeho Boha podle našich otcovských rodů ve stanovené časy rok co rok, aby bylo zapalováno na oltáři Hospodina, našeho Boha, jak je psáno v Zákoně.
#10:36 Také budeme přinášet prvotiny ze svých rolí a provotiny všeho ovoce z každého stromu rok co rok do Hospodinova domu.
#10:37 I prvorozené ze svých synů a ze svého dobytka, jak je psáno v Zákoně, prvorozené ze svého skotu a bravu, budeme přinášet do domu svého Boha kněžím, kteří konají službu v domě našeho Boha.
#10:38 Také první část z obilní tluče, z našich obětních dávek, z ovoce každého stromu, z moštu a z čerstvého oleje budeme přinášet kněžím do komor při domě našeho Boha a desátky z naší půdy budeme odvádět levitům; levité sami ať vybírají desátky ve všech městech, která s námi slouží Hospodinu.
#10:39 Kněz z rodu Áronova bude s levity, když budou vybírat desátky. Levité předají desátý díl z desátků do domu našeho Boha, do komor ve skladišti.
#10:40 Do těch komor totiž budou přinášet Izraelci i levitští příslušníci obětní dávky z obilí, z moštu a z čerstvého oleje. Tam je nádobí svatyně i kněží konající službu i vrátní a zpěváci. Dům svého Boha neopustíme.“ 
#11:1 Předáci lidu se usadili v Jeruzalémě. Ostatní lid losoval, aby vybral vždy jednoho z deseti, který by se usadil ve svatém městě Jeruzalémě; devět dalších zůstalo v jiných městech.
#11:2 Lid žehnal všem mužům, kteří se dobrovolně rozhodli, že se usadí v Jeruzalémě.
#11:3 Toto jsou představitelé krajin, kteří se usadili v Jeruzalémě a v judských městech. Každý se usadil ve svém trvalém vlastnictví ve svém městě, Izraelci, kněží, levité, chrámoví nevolníci a synové služebníků Šalomounových.
#11:4 V Jeruzalémě se usadili někteří Judovci a Benjamínovci. - Z Judovců: Atajáš, syn Uzijáše, syna Zekarjáše, syna Amarjáše, syna Šefatjáše, syna Mahalalelova ze synů Peresových.
#11:5 Dále Maasejáš, syn Báruka, syna Kol-choza, syna Chazajáše, syna Adajáše, syna Jójaríba, syna Zekarjáše, příslušníka šíloského.
#11:6 Všech synů Peresových sídlících v Jeruzalémě bylo čtyři sta osmdesát šest bojeschopných mužů.
#11:7 A toto byli Benjamínovci: Salu, syn Mešuláma, syna Jóeda, syna Pedajáše, syna Kólajáše, syna Maasejáše, syna Ítiela, syna Ješajášova.
#11:8 Mimo něho Gabaj-salaj, celkem devět set dvacet osm.
#11:9 Zikríův syn Jóel byl nad nimi dohlížitelem a Senúův syn Juda byl jako druhý nad městem.
#11:10 Z kněží: Jójaríbův syn Jedajáš, Jakín,
#11:11 Serajáš, syn Chilkijáše, syna Mešulama, syna Sádoka, syna Merajóta, syna Achítúbova, představený domu Božího,
#11:12 a jejich bratří, kteří konali službu v domě, celkem osm set dvacet dva. Dále Adajáš, syn Jerochama, syna Pelaljáše, syna Amsího, syna Zekarjáše, syna Pašchúra, syna Malkijášova,
#11:13 a jeho bratří, představitelé rodů, celkem dvě stě čtyřicet dva. Dále Amašsaj, syn Azarela, syna Achazaje, syna Mešilemóta, syna Imerova,
#11:14 a jejich bratří, udatní muži, celkem sto dvacet osm. Dohlížitelem nad nimi byl Zabdíel, syn velikých.
#11:15 Z levitů: Šemajáš, syn Chašúba, syna Azríkama, syna Chašabjáše, syna Búníova,
#11:16 a Šabetaj a Józabad z předních levitů; ti dohlíželi na službu vně domu Božího.
#11:17 Dále Matanjáš, syn Míky, syna Zabdího, syna Asafova, přední modlitebník začínající chvalozpěvy; a jako druhý z jeho bratří Bakbukjáš. Dále Abda, syn Šamúy, syna Galala, syna Jedútúnova.
#11:18 Všech levitů bylo ve svatém městě dvě stě osmdesát čtyři.
#11:19 Vrátní: Akúb, Talmón a jejich bratří. Strážců bran bylo sto sedmdesát dva. -
#11:20 Ostatní Izrael, kněží a levité, byli ve všech městech judských, každý ve svém dědictví.
#11:21 Chrámoví nevolníci se usadili na Ófelu; Sícha a Gišpa byli nad chrámovými nevolníky.
#11:22 Dohlížitelem levitů v Jeruzalémě byl Uzí, syn Baního, syna Chašabjáše, syna Matanjáše, syna Míkova, ze synů Asafových, kteří zpívali při službách v Božím domě.
#11:23 Platil totiž pro ně příkaz krále Davida, aby den co den věrně zpívali.
#11:24 Petachjáš pak, syn Mešézabelův, ze synů Zeracha, syna Júdova, byl králi k ruce, aby vyřizoval všechny záležitosti lidu.
#11:25 Z Judovců se usadili někteří při dvorcích na svých polích v Kirjat-arbě a v jejích vesnicích, v Díbonu a jeho vesnicích, v Jekabseelu a jeho dvorcích
#11:26 a v Ješúe, v Móladě a v Bét-peletu,
#11:27 v Chasar-šúalu a v Beer-šebě a jejích vesnicích,
#11:28 v Siklagu a v Mekóně a jejích vesnicích,
#11:29 v Én-rimónu, v Soreji a v Jarmútu.
#11:30 Osídlili také Zanóach, Adulám a jejich dvorce, Lakíš a jeho polnosti, Azeku a její vesnice, takže přebývali od Beer-šeby až po Hinómské údolí.
#11:31 Benjamínovci z Geby osídlili Mikmás, Aju, Bét-el a jeho vesnice,
#11:32 Anatót, Nób, Ananju,
#11:33 Chasór, Rámu, Gitajim,
#11:34 Chadíd, Seboím, Nebalat,
#11:35 Lód a Óno a údolí řemeslníků.
#11:36 Některé oddíly levitů v Judsku byly přemístěny k Benjamínovi. 
#12:1 Toto jsou kněží a levité, kteří se vrátili spolu se Zerubábelem, synem Šealtíelovým, a s Jéšuou: Serajáš, Jirmejáš, Ezra,
#12:2 Amarjáš, Malúk, Chatúš,
#12:3 Šekanjáš, Rechum, Meremót,
#12:4 Idó, Ginetoj, Abijáš,
#12:5 Mijamín, Maadjáš, Bilga,
#12:6 Šemajáš, Jójaríb, Jedajáš,
#12:7 Salú, Amók, Chilkijáš a Jedajáš. To jsou přední kněží se svými bratry za dnů Jéšuových.
#12:8 A levité: Jéšua, Binúj, Kadmíel, Šerebjáš, Juda a Matanjáš; ten se svými bratry řídil sborový zpěv;
#12:9 jejich bratří Bakbukjáš a Uni stáli naproti nim při vykonávání strážní služby.
#12:10 Jéšua zplodil Jójakíma; Jójakím zplodil Eljašíba; Eljašíb Jójadu.
#12:11 Jójada zplodil Jónatana; Jónatan zplodil Jadúu.
#12:12 Za dnů Jójakímových byli kněžími tito představitelé rodů: rodu Serajášova Merajáš, Jirmejášova Chananjáš,
#12:13 Ezrova Mešulam, Amarjášova Jóchanan,
#12:14 Malúkova Jónatan, Šebanjášova Josef,
#12:15 Charimova Adna, Merajótova Chelkaj,
#12:16 Idova Zekarjáš, Ginetónova Mešulam,
#12:17 Abijášova Zikrí, Minjamínova a Móadjášova Piltaj,
#12:18 Bilgova Šamúa, Šemajášova Jónatan,
#12:19 Jójaríbova Matenaj, Jedajášova Uzí,
#12:20 Salajova Kalaj, Amókova Eber,
#12:21 Chilkijášova Chašabjáš, Jedajášova Netanel.
#12:22 Za dnů Eljašíbových, Jójadových, Jóchananových a Jadúových byli zapsáni představitelé rodů levitských i kněží až do kralování perského Dareia.
#12:23 Levitští příslušníci, představitelé rodů, byli zapsáni do Knihy letopisů až do dnů Eljašíbova syna Jóchanana.
#12:24 Přední levité: Chašabjáš, Šerebjáš a Kadmíelův syn Jéšua a jejich bratří, kteří stáli naproti nim, aby podle příkazu Davida, muže Božího, vzdávali chválu a čest Hospodinu, střídavě sbor po sboru.
#12:25 Matanjáš a Bakjbukjáš, Obadjáš, Mešulam, Talmón, Akúb byli strážci bran; střežili skladiště u bran.
#12:26 Ti byli za dnů Jójakíma, syna Jéšuy, syna Jósadakova, a za dnů místodržitele Nehemjáše a kněze Ezdráše, znalce Zákona.
#12:27 K posvěcení jeruzalémských hradeb vyhledali a přivedli do Jeruzaléma levity ze všech jejich míst, aby uspořádali radostnou slavnost posvěcení spojenou s díkůvzdáním a zpěvem za doprovodu cymbálů, harf a citar.
#12:28 Tu se shromáždili příslušníci pěveckých sborů z okrsku kolem Jeruzaléma a z netófských dvorců,
#12:29 z Bét-gilgálu a z polností Geby a Azmávetu; zpěváci si totiž vystavěli dvorce okolo Jeruzaléma.
#12:30 Kněží a levité se očistili, pak očistili lid i brány a hradby.
#12:31 Přikázal jsem judským velmožům, aby vystoupili na hradby, a postavil jsem dva velké děkovné sbory. Jeden šel doprava po hradbách k bráně Hnojné.
#12:32 Za nimi šel Hóšajáš s polovinou judských velmožů
#12:33 a Azarjáš, Erza, Mešulam,
#12:34 Juda, Binjamín, Šemajáš a Jirmeáš.
#12:35 Z kněžstva tu byli s trubkami Zekarjáš, syn Jónatana, syna Šemajáše, syna Matanjáše, syna Míkajáše, syna Zakúra, syna Asafova,
#12:36 a jeho bratří Šemajáš a Azarel, Milalaj, Gilalaj, Maaj, Netanel, Juda a Chananí s hudebními nástroji Davida, muže Božího, a před nimi znalec Zákona Ezdráš.
#12:37 U Studničné brány, která byla naproti nim, vystoupili po schodech Města Davidova a pokračovali přístupem na hradby při Davidově paláci až k Vodní bráně na východě.
#12:38 Druhý děkovný sbor se jim ubíral naproti. Za ním jsem šel já s druhou polovinou lidu po hradbách, podél Pecné věže až k šiřoké hradbě,
#12:39 podél Efrajimovy brány, Staré brány a Rybné brány při věži Chananeelu a věži Meá až k Ovčí bráně. Zůstali jsme stát ve Strážní bráně.
#12:40 Oba děkovné sbory se zastavili na nádvoří Božího domu, i já s polovinou představenstva,
#12:41 kněží Eljakím, Maasejáš, Minjamín, Míkajáš, Eljóenaj, Zekarjáš, Chananjáš s trubkami,
#12:42 Maasejáš, Šemajáš, Eleazar, Uzí, Jóchanan, Malkijáš, Élam a Ezer. Zpěváci zvučně zpívali za řízení Jizrachjáše.
#12:43 V onen den také obětovali veliké oběti a radovali se, neboť jim Bůh připravil převelikou radost. I ženy a děti se radovaly, takže se radostné veselí Jeruzaléma rozléhalo dodaleka.
#12:44 V onen den byli ustanoveni někteří za správce komor pro sklady, obětní dávky, prvotiny a desátky, aby v nich uskladňovali částky z městských polností určené Zákonem kněžím a levitům. Juda se totiž radoval z kněží a levitů, připravených
#12:45 držet stráž svého Boha a dbát na očišťování. I zpěváci a vrátní konali službu podle nařízení Davida a jeho syna Šalomouna.
#12:46 Odedávna totiž, už za dnů Davidových a Asafových, vzdávali přední zpěváci zpěvem Bohu chválu a čest.
#12:47 Proto za dnů Zerubábelových i za dnů Nehemjášových odváděl celý Izrael příslušné částky na každý den pro zpěváky a vrátné; odděloval je jako svaté dávky pro levity a levité z nich oddělovali svaté dávky pro Áronovce. 
#13:1 V onen den bylo lidu předčítáno z Knihy Mojžíšovy a našlo se v ní napsáno, že Amónec ani Moábec nesmí nikdy vstoupit do Božího shromáždění.
#13:2 Nevyšli totiž synům Izraele vstříc s chlebem a vodou, ale najali proti nim Bileáma, aby jim zlořečil; náš Bůh však obrátil zlořečení v požehnání.
#13:3 Když uslyšeli ten zákon, rozhodli se vyloučit z Izraele všechny přimíšence.
#13:4 Někdy předtím kněz Eljašíb, určený za správce komor domu našeho Boha, spřízněný s Tóbijášem,
#13:5 uvolnil pro něho velkou komoru, kam byly dříve dávány obětní dary, kadidlo, nádobí a desátky z obilí, moštu a čerstvého oleje, určené levitům, zpěvákům a vrátným, a obětní dávky pro kněze.
#13:6 Když se to vše stalo, nebyl jsem v Jeruzalémě. V třicátém druhém roce vlády babylónského krále Artaxerxa jsem se totiž odebral ke králi. Po nějaké době jsem si vyžádal od krále propuštění.
#13:7 Jak jsem přišel do Jeruzaléma, hned jsem porozuměl, jakého zla se dopustil Eljašíb kvůli Tóbijášovi tím, že pro něho uvolnil komoru v nádvořích Božího domu.
#13:8 Velice mě to pohoršilo. Dal jsem všechny věci Tóbijášova domu vyházet z komory ven.
#13:9 Rozkázal jsem, aby komory vyčistily, a znovu jsem do nich dal věci Božího domu, obětní dary a kadidlo.
#13:10 Dozvěděl jsem se také, že levitům nebyly dávány částky, takže se všichni rozprchli ke svým polím, levité i zpěváci konající službu.
#13:11 Měl jsem o to spor s představenstvem a ptal jsem se: „Proč je Boží dům opuštěn?“ Shromáždil jsem levity a dosadil je na jejich místa.
#13:12 A všechen Juda přinášel do skladů desátky z obilí, moštu a čerstvého oleje.
#13:13 Za skladníky jsem určil kněze Šelemjáše, znalce Zákona Sádoka a z levitů Pedajáše; jim k ruce byl Chanan, syn Zakúra, syna Matanjášova; ti byli pokládáni za spolehlivé a měli za úkol vydávat příděly svým bratřím. -
#13:14 „Pamatuj na mne, Bože můj, kvůli tomu a nevymaž mou oddanost, kterou jsem prokázal domu svého Boha a jeho službám!“
#13:15 V oněch dnech jsem v Judsku viděl, že někteří lisují hrozny v den odpočinku, že přinášejí pytle obilí a nakládají na osly, a také že přinášejí do Jeruzaléma víno, hrozny a fíky a všelijaká břemena v den odpočinku. Varoval jsem je toho dne, kdy prodávali potraviny.
#13:16 I Týřané tu byli usazení; přinášeli ryby a všelijaké zboží a prodávali v den odpočinku Judovcům, a to přímo v Jeruzalémě.
#13:17 Měl jsem o to spor s judskými šlechtici a ptal jsem se jich: „Jakého zla se to dopouštíte? Znesvěcujete den odpočinku!
#13:18 Což právě tak nejednali vaši otcové? Proto náš Bůh uvedl všechno toto zlé na nás a na toto město. A vy jej zase popouzíte k hněvu proti Izraeli tím, že znesvěcujete den odpočinku.“
#13:19 Když přede dnem odpočinku padl na jeruzalémské brány stín, rozkázal jsem, aby byla zavřena vrata a aby je neotvírali dříve než po dni odpočinku. Některé ze svých služebníků jsem postavil k branám, aby nikdo nenosil břemena v den odpočinku.
#13:20 Když kupci a prodavači všelijakého zboží nocovali jednou i dvakrát vně Jeruzaléma,
#13:21 varoval jsem je a vytýkal jsem jim: „Proč nocujete před hradbami? Bude-li se to opakovat, vztáhnu na vás ruku.“ Od té doby v den odpočinku nepřicházeli.
#13:22 Pak jsem přikázal levitům, aby se očistili a přišli střežit brány, aby byl svěcen den odpočinku. - „Pamatuj na mne, Bože můj, také kvůli tomu a měj se mnou soucit podle svého hojného milosrdenství!“
#13:23 V oněch dnech jsem také viděl, že si židé brali ašdódské, amónské a moábské ženy.
#13:24 Jejich děti pak mluvily zpoloviny ašdódsky, ale nedovedly mluvit židovsky. Hovořily jazykem toho či onoho lidu.
#13:25 Měl jsem s nimi o to spor a zlořečil jsem jim, některé muže jsem bil a rval za vousy a zapřisáhl jsem je při Bohu: „Nedávejte své dcery jejich synům a neberte jejich dcery svým synům a sobě!
#13:26 Což se právě tím neprohřešil izraelský král Šalomoun? Mezi mnohými národy nebylo krále jemu podobného. Byl milován Bohem, který jej určil za krále nad celým Izraelem. A přece i jeho svedly ženy cizinky ke hříchu.
#13:27 Máme snad mlčet, když vás slyšíme, že pácháte všechno to veliké zlo, že se zpronevěřujete našemu Bohu a berete si ženy cizinky?“
#13:28 Jeden syn nejvyššího kněze Jójady, syna Eljašíbova, byl zetěm Sanbalata Chorónského. Toto jsem od sebe odehnal. -
#13:29 „Pamatuj na ně, Bože můj, že poskvrnili kněžství i smlouvu kněžskou a levitskou!“ -
#13:30 Tak jsem je očistil od všeho cizího. Stanovil jsem povinnosti kněží a levitů, každému podle jeho díla,
#13:31 i pro přinášení dříví v určených lhůtách, i pro prvotiny. - „Pamatuj na mne, Bože můj, v dobrém.“  

\book{Esther}{Esth}
#1:1 Stalo se to za dnů Achašveróšových; to byl Achašveróš, který kraloval nad sto dvaceti sedmi krajinami od Indie až po Kúš.
#1:2 Za oněch dnů, kdy král Achašveróš seděl na svém královském trůnu na hradě v Šúšanu,
#1:3 v třetím roce svého kralování uspořádal hodokvas pro všechny své velmože a služebníky. Vojenští činitelé perští a médští, šlechta a správcové krajin byli shromážděni před ním.
#1:4 Po mnoho dní, totiž po sto osmdesát dnů, přitom vystavoval na odiv bohatství své královské slávy a skvostnou nádheru své velikosti.
#1:5 Když pak ty dny skončily, uspořádal král pro všechen lid, co byl na hradě v Šúšanu, od největšího po nejmenšího, sedmidenní hodokvas na nádvoří v zahradě královského paláce.
#1:6 Sněhobílé tkaniny lněné a bavlněné i látky purpurově fialové byly zavěšeny na bělostných a šarlatových provazech, provlečených stříbrnými kruhy na mramorových sloupech. Zlatá a stříbrná lehátka stála na mozaikové dlažbě z alabastru a mramoru, z perleti a drahokamů.
#1:7 Nápoje se podávaly ve zlatých nádobách, z nichž každá byla jiná. A vína královského bylo mnoho, jak se na krále sluší.
#1:8 Na králův pokyn se popíjelo dle libosti. Tak totiž uložil král všem hodnostářům svého domu, aby každému vyhověli podle přání.
#1:9 Také královna Vašti pořádala hostinu pro ženy v královském domě krále Achašveróše.
#1:10 Sedmého dne, když se král rozjařil vínem, nařídil Mehúmanovi, Bizetovi, Charbónovi, Bigtovi, Abagtovi, Zetarovi a Karkasovi, sedmi dvořanům, obsluhujícím krále Achašveróše,
#1:11 aby před něho přivedli královnu Vašti s královskou korunou. Chtěl národům a velmožům ukázat její krásu; byla totiž půvabného vzhledu.
#1:12 Ale královna Vašti se zdráhala přijít, jak jí vzkázal král po dvořanech. Král se velice rozlítil a vzplanul hněvem.
#1:13 Obrátil se na mudrce znalé časů, neboť tak se královy záležitosti předkládaly všem znalcům zákonů a práv světských i náboženských.
#1:14 Nejblíže mu byli Karšena, Šetar, Admata, Taršíš, Meres, Marsena a Memúkan, sedm perských a médských velmožů, kteří směli hledět na královu tvář a v království zaujímali první místo.
#1:15 Otázal se: „Co se má podle zákona stát s královnou Vašti za to, že neučinila, co jí po dvořanech vzkázal král Achašveróš?“
#1:16 Tu řekl Memúkan před králem a velmoži: ‚Královna Vašti se provinila nejen proti králi, ale i proti všem velmožům a proti všem národům, které žijí ve všech krajinách krále Achašveróše.
#1:17 Zpráva o královnině činu se donese jistě ke všem ženám a uvede jejich manžely v jejich očích v nevážnost, až se bude říkat: „Král Achašveróš nařídil, aby před něho přivedli královnu Vašti, a ona nepřišla!‘
#1:18 Ještě dnes stejně odpovědí kněžny perské a médské, až uslyší o královnině činu, všem královským velmožům a bude z toho příliš nevážnosti a mrzutosti.
#1:19 Uzná-li to král za vhodné, ať vydá královské nařízené, které se zapíše mezi zákony perské a médské a nebude přestupováno, že Vašti už nesmí předstoupit před krále Achašveróše; a její královskou hodnost ať dá král druhé, lepší než je ona.
#1:20 Až se vyhláška, kterou král vydá, rozhlásí po celém jeho královstí - a to je obrovské -, všechny ženy budou prokazovat úctu svým manželům od největšího až po nejmenšího.“
#1:21 Králi i velmožům se ten návrh zalíbil a král učinil podle slov Memúkanových.
#1:22 Poslal dopisy do všech královských krajin, všude do všech krajin jejich písmem a všude do všech národů v jejich jazyku, aby každý muž vládl ve svém domě a aby mluvil řečí svého lidu. 
#2:1 Po těchto událostech, když rozhořčení krále Achašveróše opadlo, vzpomněl si na Vašti, na to, co učinila, i na to, jak o ní bylo rozhodnuto.
#2:2 Tu řekli královští panoši, kteří ho obsluhovali: „Nechť jsou vyhledány pro krále dívky, panny půvabného vzhledu!
#2:3 Král nechť ustanoví ve všech krajinách svého království dohližitele, aby shromáždili všechny dívky, panny půvabného vzhledu, na hrad v Šúšanu do ženského domu pod dohled královského kleštěnce Hegaje, strážce žen, kde se jim poskytne náležitá péče.
#2:4 Ta dívka, která se králi zalíbí, stane se královnou místo Vašti.“ Králi se návrh zalíbil a učinil tak.
#2:5 Na hradě v Šúšanu byl jeden žid jménem Mordokaj, syn Jaíra, syna Šimeího, syna Kíšova, Benjamínec.
#2:6 Byl přestěhován z Jeruzaléma s přesídlenci, kteří byli přestěhování spolu s judským králem Jekonjášem, kterého přestěhoval babylónský král Nebúkadnesar.
#2:7 Vychovával Hadasu neboli Esteru, dceru svého strýce, protože neměla otce ani matku. Byla to dívka krásné postavy a půvabného vzhledu. Po smrti jejího otce a její matky ji Mordokaj přijal za dceru.
#2:8 Když se rozhlásil králův výrok, totiž jeho zákon, a mnoho dívek bylo shromážděno na hradě v Šúšanu pod dohled Hegajův, byla vzata i Ester do královského domu pod dohled Hegaje, strážce žen.
#2:9 Dívka se mu zalíbila a získala jeho náklonnost. Neprodleně jí poskytl náležitou péči i příděly jídla. Přidělil jí též sedm vyhlédnutých dívek z královského domu a přemístil ji i její dívky do nejlepší části ženského domu.
#2:10 Ester neoznámila nic o svém lidu a o svém původu, protože jí Mordokaj přikázal, aby to neoznamovala.
#2:11 Mordokaj se denně procházel před nádvořím ženského domu, aby se dozvěděl, jak se Esteře daří a co se s ní děje.
#2:12 Na každou dívku přicházela řada, aby vešla ke králi Achašveróšovi, po uplynutí dvanácti měsíců, podle zákona pro ženy. Tak dlouho totiž trvaly přípravy: šest měsíců mazání myrhovým olejem a šest měsíců balzámy s ostatní náležitou ženskou péčí.
#2:13 Pak vcházela dívka ke králi. Vše, oč si řekla, jí bylo dáno, když měla z ženského domu vejít do domu králova.
#2:14 Za večera vešla a za jitra se vracela do druhého ženského domu pod dohled královského kleštěnce Šaašgaze, strážce ženin. Ke králi už nevešla, ledaže by si ji král oblíbil; pak byla zavolána jmenovitě.
#2:15 Když přišla řada na Esteru, dceru Abíchajila, strýce Mordokajova, kterou Mordokaj přijal za dceru, aby vešla ke králi, nežádala nic, než co řekl královský kleštěnec Hegaj, strážce žen. A Ester získala přízeň všech, kdo ji spatřili.
#2:16 Tak byla Ester vzata ke králi Achašveróšovi do jeho královského domu desátého měsíce, to je měsíce tebetu, v sedmém roce jeho kralování.
#2:17 Král si Esteru zamiloval nade všechny ženy; získala jeho přízeň a náklonnost nade všechny panny. Na hlavu jí vložil královskou korunu a ustanovil ji královnou místo Vašti.
#2:18 Potom král uspořádal veliký hodokvas pro všechny své velmože a služebníky, hostinu k Esteřině poctě. Poskytl také krajinám úlevy a udílel pocty, jak se sluší na krále.
#2:19 Když se podruhé shromažďovaly panny, seděl Mordokaj v královské bráně.
#2:20 Ester dosud nic neoznámila o svém původu a o svém lidu, jak jí Mordokaj přikázal. Řídila se jeho slovy, jako když byla u něho a on ji vychovával.-
#2:21 V oněch dnech, kdy Mordokaj seděl v královské bráně, rozlítili se Bigtan a Tereš, dva královi dvořané ze strážců prahu, a chtěli vztáhnout ruku na krále Achašveróše.
#2:22 Ale Mordokaj se o té věci dozvěděl, oznámil to královně Esteře a Ester to řekla Mordokajovým jménem králi.
#2:23 Záležitost se vyšetřila, vina byla prokázána a oba byli pověšeni na kůl. Pak to bylo zapsáno před králem do Knihy letopisů. 
#3:1 Po těch událostech povýšil král Achašveróš Hamana, syna Hamedatova, Agagovce, a povznesl jej. Postavil jeho křeslo nad křesla všech velmožů, které měl při sobě.
#3:2 Všichni královi služebníci, kteří zasedali v královské bráně, klekali a klaněli se před Hamanem, neboť tak přikázal král. Ale Mordokaj neklekal a neklaněl se.
#3:3 Královi služebníci, kteří zasedali v královské bráně, se Mordokaje tázali: „Proč přestupuješ králův příkaz?“
#3:4 Říkali mu to den co den, ale on jako by je neslyšel. Tak to oznámili Hamanovi, aby viděli, obstojí-li Mordokajovo odůvodnění. Oznámil jim totiž, že je žid.
#3:5 Když si Haman všiml, že Mordokaj před ním nekleká a že se neklaní, byl pln rozhořčení.
#3:6 Bylo mu však příliš málo vztáhnout ruku na Mordokaje samotného, protože mu oznámili, z jakého lidu Mordokaj pochází. Proto Haman usiloval vyhladit v celém Achašveróšově království všechny židy, lid Mordokajův.
#3:7 V prvním měsíci, to je v měsíci nísanu, v dvanáctém roce vlády krále Achašveróše vrhali před Hamanem púr, to znamená los, pro jednotlivé dny a měsíce až po dvanáctý, to je měsíc adar.
#3:8 Tu řekl Haman králi Achašveróšovi: „Je tu jakýsi lid, roztroušený a oddělený mezi národy po všech krajinách tvého království. Jejich zákony jsou odchylné od zákonů všech národů a zákony královskými se neřídí. Král z toho nemá žádný prospěch, když je trpí.
#3:9 Uzná-li král za vhodné, nechť dá písemný příkaz, aby byli zahubeni. Odvážím do rukou úředníků deset tisíc talentů stříbra, aby je uložili do královských pokladů.“
#3:10 Král sňal z ruky svůj pečetní prsten a dal jej Agagovci Hamanovi, synu Hamedatovu, protivníkovi židů.
#3:11 Potom řekl král Hamanovi: „Je ti dáno to stříbro i ten lid; nalož s ním, jak uznáš za dobré.“
#3:12 Třináctého dne prvního měsíce byli povoláni královští písaři a bylo zapsáno vše, co přikázal Haman královským satrapům a místodržitelům v jednotlivých krajinách a velmožům z jednotlivých národů, pro každou krajinu jejím písmem a pro každý národ jeho jazykem. Psalo se jménem krále Achašveróše a pečetilo se královým prstenem.
#3:13 Dopisy byly poslány po rychlých poslech do všech králových krajin, aby vyhladili, povraždili a zahubili všechny židy od mládence po starce, děti i ženy, v jednom dni, třináctého dne dvanáctého měsíce - to je měsíc adar; a kořist po nich aby si vzali jako lup.
#3:14 Opis spisu měl být vydán jako zákon všude ve všech krajinách a zveřejněn všem národům, aby byly přichystány na ten den.
#3:15 Spěšně vyjeli rychlí poslové, jak král uložil, sotvaže byl ten zákon na hradě v Šúšanu vydán. Král a Haman pak zasedli k pití, zatímco v městě Šúšanu zavládl rozruch. 
#4:1 Mordokaj se dozvěděl o všem, co se stalo. Roztrhl svůj šat, oblékl žíněné roucho a posypal si hlavu popelem. Vyšel doprostřed města a převelice hořekoval.
#4:2 Tak došel až před královskou bránu; do královské brány totiž nebylo dovoleno vejít v žíněném rouchu.
#4:3 A všude, v každé krajině, kamkoli se dostal králův výrok, totiž jeho zákon, konali židé veliké smuteční obřady s postem, pláčem a naříkáním; mnohým se staly lůžkem žíněné roucho a popel.
#4:4 Tu přišly Esteřiny dívky a její kleštěnci a oznámili jí to. Královnu sevřela nesmírná úzkost. Poslala šaty, aby Mordokaje oblékli a sňali z něho žíněné roucho, ale on je nepřijal.
#4:5 Ester tedy zavolala Hatáka z královských dvořanů, který byl ustanoven osobně pro ni, a přikázala mu, aby od Mordokaje vyzvěděl, co se děje a proč si tak počíná.
#4:6 Haták vyšel k Mordokajovi na městské prostranství před královskou branou.
#4:7 Mordokaj mu oznámil všechno, co ho potkalo, i obnos stříbra, které Haman slíbil odvážit do královských pokladů za židy, aby byli zahubeni.
#4:8 Dal mu také opis vydaného zákona o jejich vyhlazení, který byl vydán v Šúšanu, aby jej ukázal Esteře a oznámil jí to, též aby jí vyřídil příkaz, ať vejde ke králi prosit ho o milost a přimluvit se u něho za svůj lid.
#4:9 Haták přišel a oznámil Esteře Mordokajův vzkaz.
#4:10 Ester však řekla Hatákovi a přikázala mu vyřídit Mordokajovi:
#4:11 „Všichni královi služebníci i lid králových krajin vědí, že pro každého muže i ženu, kteří by bez pozvání vešli do vnitřního nádvoří ke králi, platí jediný zákon - usmrtit! Jen ten, k němuž král vztáhne zlaté žezlo, zůstane naživu. Já jsem už třicet dní nebyla zavolána, abych vešla ke králi.“
#4:12 Esteřin vzkaz oznámili Mordokajovi.
#4:13 Mordokaj však vzkázal Esteře: „Nedomnívej se, že v domě králově vyvázneš životem, jediná ze všech židů.
#4:14 Budeš-li v tuto chvíli skutečně mlčet, úleva a osvobození přijde židům odjinud, ale ty a dům tvého otce zahynete. Kdo ví, zda jsi nedosáhla královské hodnosti právě pro chvíli, jako je tato.“
#4:15 Ester dala odpovědět Mordokajovi:
#4:16 „Jdi, shromažď všechny židy, kteří jsou v Šúšanu, a postěte se za mne. Nejezte a nepijte po tři dny, v noci ani ve dne. Také já a mé dívky se budeme takto postit. Potom půjdu ke králi, třebaže to není podle zákona. Mám-li zahynout, zahynu.“
#4:17 Mordokaj odešel a učinil všechno, co mu Ester přikázala. 
#5:1 Třetího dne si Ester oblékla královské roucho a stanula na vnitřním nádvoří králova domu proti královu domu. Král seděl na svém královském trůnu v královském domě proti vchodu do domu.
#5:2 Jakmile spatřil královnu Esteru stojící na nádvoří, získala jeho přízeň. Král vztáhl k Esteře zlaté žezlo, které držel v ruce. Ester se přiblížila a dotkla se hlavice žezla.
#5:3 Tu jí král řekl: „Co ti je, královno Ester? Jaká je tvá žádost? Až do poloviny království ti bude dáno.“
#5:4 Ester odpověděla: „Uzná-li král za vhodné, nechť král s Hamanem přijde dnes na hostinu, kterou jsem pro něho připravila.“
#5:5 Král řekl: „Rychle jděte pro Hamana, ať se splní Esteřino přání!“ I přišel král s Hamanem na hostinu, kterou Ester připravila.
#5:6 Když popíjeli víno, řekl král Esteře: „Jaká je tvá prosba? Bude ti splněna. Jaká je tvá žádost? Až do poloviny království ti bude vyhověno.“
#5:7 Ester odpověděla: „Toto je má prosba a žádost:
#5:8 Jestliže jsem získala královu přízeň a jestliže král uzná za vhodné splnit mou prosbu a vyhovět mé žádosti, nechť opět přijde král s Hamanem na hostinu, kterou pro ně připravím; zítra pak učiním, jak praví král.“
#5:9 Onoho dne vyšel Haman rozradostněný a rozjařený. Avšak když spatřil v královské bráně Mordokaje, že nepovstal ani se před ním netřásl strachem, byl naplněn Haman proti Mordokajovi rozhořčením.
#5:10 Ale ovládl se a šel domů. Pak poslal pro své oblíbence a pro svou ženu Zereš.
#5:11 Vypočítával jim slávu svého bohatství a množství svých synů i všechno, čím ho král povýšil a povznesl nad velmože a královy služebníky.
#5:12 A pokračoval: „Kromě toho nepozvala královna Ester na hostinu, kterou připravila, nikoho jiného než mne spolu s králem, a také na zítřek jsem k ní pozván spolu s králem.
#5:13 Ale z toho všeho nic nemám, dokud se musím dívat na toho žida Mordokaje, jak sedí v královské bráně.“
#5:14 Tu mu řekla jeho žena Zereš i všichni jeho oblíbenci: „Ať udělají kůl vysoký padesát loket, a ráno řekni králi, aby naň Mordokaje pověsili. Jdi na hostinu s králem radostně.“ Hamanovi se ta řeč líbila a dal udělat kůl. 
#6:1 Oné noci spánek od krále prchal. Poručil tedy, aby přinesli Knihu letopisů památných událostí a ty byly králi předčítány.
#6:2 Našel se zápis, jak Mordokaj oznámil na Bigtanu a Tereše, dva královy dvořany ze strážců prahu, že chtěli na krále Achašveróše vztáhnout ruku.
#6:3 Král se otázal: „Jaká pocta a jaké povýšení byly za to Mordokajovi uděleny?“ Královští panoši, kteří ho obsluhovali, odvětili: „Nedostal vůbec nic.“
#6:4 Tu se král otázal: „Kdo je na nádvoří?“ Mezitím přišel na vnější nádvoří králova domu Haman promluvit si s králem, aby dal pověsit Mordokaje na kůl, který pro něho připravil.
#6:5 Panošové řekli králi: „To je Haman, stojí na nádvoří.“ A král řek: „Ať vstoupí!“
#6:6 Haman vstoupil a král se ho otázal: „Co se má stát s mužem, kterého chce král poctít?“ Haman si v duchu řekl: „Koho jiného než mne by chtěl král poctít?“
#6:7 A odvětil králi: „Muži, kterého chce král poctít,
#6:8 ať přinesou královské roucho, které oblékal král, a přivedou koně, na němž jezdil král, a na jeho hlavu ať vloží královskou korunu.
#6:9 Roucho a kůň ať jsou odevzdány do rukou některého z královských velmožů a šlechticů, oni ať obléknou muže, kterého chce král poctít, ať ho na koni provedou po městském prostranství a před ním provolávají: Tak se jedná s mužem, kterého chce král poctít!“
#6:10 Nato řekl král Hamanovi: „Pospěš, vezmi roucho a koně, jak jsi pověděl, a učiň tak židu Mordokajovi, který sedí v královské bráně. Nezanedbej ani slova z toho, co jsi pověděl!“
#6:11 Haman tedy vzal roucho a koně, oblékl Mordokaje a provedl jej na koni po městském prostranství a provolával před ním: „Tak se jedná s mužem, kterého chce král poctít!“
#6:12 Mordokaj se vrátil do královské brány, zatímco Haman spěchal domů smuten a se zakrytou hlavou.
#6:13 Vypravoval své ženě Zereši a všem svým oblíbencům všecko, co ho potkalo. I řekli mu jeho mudrci a jeho žena: „Je-li Mordokaj, který je počátkem tvého pádu, původu židovského, nic proti němu nezmůžeš, ale podlehneš mu.“
#6:14 Ještě s ním rozmlouvali, když dorazili královi dvořané a spěšně odvedli Hamana na hostinu, kterou připravila Ester. 
#7:1 I přišel král s Hamanem opět, aby popíjeli ve společnosti královny Estery.
#7:2 Také tohoto druhého dne, když popíjeli víno, otázal se král Estery: „Jaká je tvá prosba, královno Ester? Bude ti splněna. Jaká je tvá žádost? Až do poloviny království bude ti vyhověno.“
#7:3 Královna Ester odpověděla: Jestliže jsem získala tvou přízeň, králi, a uzná-li král za vhodné, nechť je mi na mou prosbu darován můj život a na mou žádost můj lid.
#7:4 Vždyť jsme byli prodáni, já i můj lid, aby nás vyhladili, povraždili a zahubili. Kdybychom byli prodáni jen za otroky a otrokyně, mlčela bych, neboť takové soužení by nebylo hodno, aby se jím král obtěžoval.“
#7:5 Zeptal se král Achašveróš, zeptal se královny Estery: „Kdo je to a kde je ten, který se opovážil udělat něco takového?“
#7:6 Ester odpověděla: „Tím protivníkem a nepřítelem je ten zloduch Haman.“ Haman zůstal před králem a královnou ochromen.
#7:7 Rozhořčený král povstal od popíjení vína a odešel do zahrady u paláce. Haman přistoupil, aby si od královny Estery vyprosil život, neboť viděl, že na něj dopadne králova zloba.
#7:8 Když se král vracel ze zahrady u paláce do domu, kde popíjeli víno, padl Haman právě na pohovku, na níž byla Ester. Tu se král rozkřikl: „To chceš dokonce u mne v domě učinit královně násilí?“ Sotva to slovo z králových úst vyšlo, zakryli Hamanovi tvář.
#7:9 Charbóna, jeden z kleštěnců, prohodil před králem: „Vždyť je tu také kůl, jejž připravil Haman pro Mordokaje, který promluvil ve prospěch králův. Stojí u Hamanova domu, vysoký padesát loket.“ A král řekl: „Pověste ho na něj!“
#7:10 Pověsili tedy Hamana na kůl, který postavil pro Mordokaje, a královo rozhořčení se uklidnilo. 
#8:1 Onoho dne dal král Achašveróš královně Esteře dům Hamana, protivníka židů. A Mordokajovi se dostalo přístupu před krále, protože Ester oznámila, čím jí je.
#8:2 Král sňal svůj pečetní prsten, který dal odebrat Hamanovi, a dal jej Mordokajovi. A Ester ustanovila Mordokaje nad Hamanovým domem.
#8:3 Ester se však znovu přimlouvala před králem, padla mu k nohám a s pláčem ho prosila o milost, aby odvrátil pohromu, chystanou Agagovcem Hamanem, a to, co zamýšlel proti židům.
#8:4 Král vztáhl k Esteře zlaté žezlo; tu Ester vstala, postavila se před krále
#8:5 a řekla: „Jestliže král uzná za vhodné a já jsem získala jeho přízeň, jestliže to král považuje za přijatelné a má ve mně zalíbení, nechť je napsáno, aby vrátili ty dopisy obsahující záměr Agagovce Hamana, syna Hamedatova, v nichž dal napsat, aby ve všech královských krajinách zahubili židy.
#8:6 Neboť jak bych mohla přihlížet pohromě, která má postihnout můj lid? Jak bych mohla přihlížet záhubě svého rodu?“
#8:7 Král Achašveróš královně Esteře a židu Mordokajovi odvětil: „Hle, dům Hamanův jsem dal Esteře a jeho samého pověsili na kůl za to, že vztáhl na židy ruku.
#8:8 Vy sami teď napište židům, jak uznáte za vhodné, královým jménem, a zapečeťte královým prstenem. Ale spis napsaný jednou královým jménem a zapečetěný královým prstenem nelze vzít zpět.“
#8:9 I byli tehdy, třiadvacátého dne třetího měsíce, to je měsíce sívánu, povoláni královští písaři a bylo napsáno všechno, jak to přikázal Mordokaj, židům a satrapům, místodržitelům a správcům krajin od Indie až po Kúš, celkem sto dvaceti sedmi krajinám. Pro každou krajinu jejím písmem a pro každý národ jeho jazykem, i židům jejich písmem a jazykem.
#8:10 Napsal dopisy jménem krále Achašveróše, zapečetil královým prstenem a rozeslal je po rychlých jízdních poslech, kteří jezdili na říšských lehkonohých ořích, chovaných v hřebčincích.
#8:11 Král dal židům všude po všech městech právo, aby se shromáždili a postavili na obranu svých životů, aby vyhladili, povraždili a zahubili všechnu válečnou moc národa a krajiny těch, kteří by je napadli, i s dětmi a ženami, a kořist po nich aby si vzali jako lup,
#8:12 a to v jediném dni ve všech krajinách krále Achašveróše, třináctého dne dvanáctého měsíce, to je měsíce adaru.
#8:13 Opis spisu ať je vydán jako zákon všude ve všech krajinách a zveřejněn všem národům, aby toho dne byli židé přichystáni vykonat pomstu nad svými nepřáteli.
#8:14 Rychlí poslové, kteří jezdili na říšských lehkonohých ořích, spěšně vyrazili pobízeni královým rozkazem, sotvaže byl ten zákon ha hradě v Šúšanu vydán.
#8:15 Mordokaj vyšel od krále v královském rouchu z purpurově fialové látky a sněhobílého plátna, s velkou zlatou korunou a pláštěm z bělostného plátna a šarlatu. Město Šúšan jásalo a radovalo se.
#8:16 Židům vzešlo světlo a radost, veselí a pocta.
#8:17 Také všude ve všech krajinách a všude ve všech městech, kamkoli se dostal králův výrok, totiž jeho zákon, nastaly židům dny radosti a veselí, hostin a pohody. A mnozí z národů země se připojovali k židům, neboť na ně padl strach ze židů. 
#9:1 Dvanáctého měsíce, to je měsíce adaru, v jeho třináctý den, kdy králův výrok, totiž jeho zákon, měl být proveden, v den, který chtivě vyhlíželi nepřátelé židů, aby se jich zmocnili, došlo k zvratu: naopak židé se zmocnili těch, kdo je nenáviděli.
#9:2 Židé se shromáždili ve svých městech ve všech krajinách krále Achašveróše, aby vztáhli ruku na ty, kdo jim chystali pohromu. Nikdo před nimi neobstál, neboť strach z nich padl na všechny národy.
#9:3 Všichni správcové krajin a satrapové, místodržitelé a královští úředníci podporovali židy, protože na ně padl strach z Mordokaje.
#9:4 Mordokaj měl totiž v králově domě veliký vliv a šla o něm pověst po všech krajinách, protože ten muž Mordokaj získával vliv stále větší.
#9:5 Židé pobili všechny své nepřátele; bili je mečem a pobíjeli, až do vyhubení. Naložili s těmi, kdo je nenáviděli, jak se jim zlíbilo.
#9:6 Na hradě v Šúšanu židé zabili a zahubili pět set mužů.
#9:7 I Paršandatu a Dalfóna a Aspatu
#9:8 a Póratu a Adalju a Arídatu
#9:9 a Parmaštu a Arísaje a Aridaje a Vajzatu,
#9:10 deset synů Hamedatova syna Hamana, protivníka židů, zabili, ale nevztáhli ruce po lupu.
#9:11 Toho dne se donesl ke králi počet zabitých na hradě v Šúšanu.
#9:12 Král řekl královně Esteře: „Na hradě v Šúšanu židé zabili a zahubili pět set mužů a všech deset synů Hamanových. Co asi učinili v ostatních královských krajinách? Jaká je tvá prosba? Bude ti splněna. Jaká je tvá žádost? Bude ti vyhověno.“
#9:13 Ester odvětila: „Uzná-li král za vhodné, ať je židům v Šúšanu dovoleno také zítra jednat podle dnes platného zákona, a deset synů Hamanových ať pověsí na kůl.“
#9:14 A král souhlasil, aby se tak stalo. V Šúšanu byl vydán zákon a deset Hamanových synů pověsili.
#9:15 Židé v Šúšanu se shromáždili také čtrnáctého dne měsíce adaru a zabili v Šúšanu tři sta mužů, ale nevztáhli ruce po lupu.
#9:16 I ostatní židé v krajinách králových se shromáždili, aby se postavili na obranu svých životů a zajistili si klid od svých nepřátel, a z těch, kdo je nenáviděli, zabili sedmdesát pět tisíc mužů, ale nevztáhli ruce po lupu.
#9:17 Bylo to třináctého dne měsíce adaru a čtrnáctého dne byl klid. Ten den učinili dnem radostného hodování.
#9:18 V Šúšanu se židé shromáždili třináctého a čtrnáctého dne téhož měsíce a patnáctého dne nastal klid. Ten den učinili dnem radostného hodování.
#9:19 Proto rozptýlení židé, sídlící ve venkovských městech, slaví čtrnáctého dne měsíce adaru radostné hodování a den pohody a jeden druhému posílá dárky.
#9:20 Mordokaj pak tyto události sepsal a poslal dopisy všem židům, blízkým i dalekým, ve všech krajinách krále Achašveróše.
#9:21 Uložil jim, aby každým rokem slavili čtrnáctého a patnáctého dne měsíce adaru
#9:22 památku na dny, v nichž si židé odpočinuli od svých nepřátel, a na měsíc, který jim přinesl zvrat, místo starosti radost, místo smutku den pohody; aby je slavili jako dny radostného hodování a aby posílal jeden druhému dárky a chudým dary.
#9:23 A židé přijali za obyčej, co začali dělat a co jim Mordokaj napsal:
#9:24 že Agagovec Haman, syn Hamedatův, protivník všech židů, osnoval záměr proti židům, jak by je zahubil, že házel púr, totiž los, aby je vyděsil a zahubil.
#9:25 Ale když ona vešla před krále, nařídil král, dokonce písemně, obrátit jeho zlý záměr, který osnoval proti židům, na jeho hlavu, a pověsili jej i jeho syny na kůl.
#9:26 Proto nazvali ty dny púrím podle slova púr. Proto kvůli všemu, co bylo v té listině řečeno, co přitom viděli a co je postihlo,
#9:27 ustanovili židé a přijali za obyčej pro sebe a pro své potomstvo i pro všechny, kdo se k nim připojí, že neopomenou slavit tyto dva dny pravidelně každého roku a v určený čas, jak je psáno.
#9:28 Tyto dny budou připomínány a slaveny v každém pokolení a v každé čeledi, v každé krajině a v každém městě; tyto dny púrímu nebudou mezi židy opomíjeny a památka na ně v jejich potomstvu nezanikne.
#9:29 Také královna Ester, dcera Abíchajilova, psala s židem Mordokajem ještě jednou, aby dodala váhy ustavující listině o purímu.
#9:30 Dopisy poslali všem židům do všech sto dvaceti sedmi krajin Achašveróšova království, slova pokoje a pravdy,
#9:31 aby dodržovali v určený čas tyto dny púrímu, jak jim stanovil žid Mordokaj a královna Ester; stejně stanovili pro sebe i pro své potomstvo zachovávat jejich posty a nářky.
#9:32 Esteřiným výrokem bylo stanoveno zachovávat púrím a bylo to zapsáno do knihy. 
#10:1 Král Achašveróš podrobil zemi i mořské ostrovy nuceným pracím.
#10:2 Všechny jeho význačné a bohatýrské činy i přesné údaje o velikém vlivu Mordokaje, kterého král povýšil, jsou, jak známo, zapsány v Knize letopisů králů médských a perských.
#10:3 Neboť žid Mordokaj byl první po králi Achašveróšovi; byl veliký u židů a oblíbený u množství svých bratří, pečoval o dobro svého lidu a usiloval o pokoj veškerého svého potomstva.  

\book{Job}{Job}
#1:1 Byl muž v zemi Úsu jménem Jób; byl to muž bezúhonný a přímý, bál se Boha a vystříhal se zlého.
#1:2 Narodilo se mu sedm synů a tři dcery.
#1:3 Jeho stáda čítala sem tisíc ovcí, tři tisíce velbloudů, pět set spřežení skotu a pět set oslic. Měl také velmi mnoho služebnictva. Ten muž předčil všechny syny dávnověku.
#1:4 Jeho synové strojívali doma hodokvasy, každý ve svůj den, a zvali i své tři sestry, aby s nimi hodovaly a pily.
#1:5 Když uplynuly dny hodokvasu, Jób pro ně posílal a posvěcoval Je. Za časného jitra obětoval oběti zápalné za každého z nich; říkal si totiž: „Možná, že moji synové zhřešili a zlořečili v srdci Bohu.“ Tak činil Jób po všechny dny.
#1:6 Nastal pak den, kdy přišli synové Boží, aby předstoupili před Hospodina; přišel mezi ně i satan.
#1:7 Hospodin se satana zeptal: „Odkud přicházíš?“ Satan Hospodinu odpověděl: „Procházel jsem zemi křížem krážem.“
#1:8 Hospodin se satana zeptal: „Zdalipak sis všiml mého služebníka Jóba? Nemá na zemi sobě rovného. Je to muž bezúhonný a přímý, bojí se Boha a vystříhá se zlého.“
#1:9 Satan však Hospodinu odpověděl: „Cožpak se Jób bojí Boha bezdůvodně?
#1:10 Vždyť jsi ho ze všech stran ohradil, rovněž jeho dům a všechno, co má. Dílu jeho rukou žehnáš a jeho stáda se na zemi rozmohla.
#1:11 Ale jen vztáhni ruku a zasáhni všechno, co má, hned ti bude do očí zlořečit.“
#1:12 Hospodin na to satanovi odvětil: „Nuže, měj si moc nade vším, co mu patří, pouze na něho ruku nevztahuj.“ A satan od Hospodina odešel.
#1:13 Nastal pak den, kdy Jóbovi synové a dcery hodovali a pili víno v domě svého prvorozeného bratra.
#1:14 Tu přišel k Jóbovi posel a řekl: „Právě orali s dobytkem a při něm se popásaly oslice.
#1:15 Vtom přitrhli Šebovci, pobrali je a čeleď pobili ostřím meče. Unikl jsem jenom já a oznamuji ti to.“
#1:16 Ještě nedomluvil, když přišel další a řekl: „Z nebe spadl Boží oheň, zachvátil ovce a čeleď pozřel. Unikl jsem jenom já a oznamuji ti to.“
#1:17 Ještě nedomluvil, když přišel další a řekl: „Kaldejci rozdělení do tří houfů napadli velbloudy, pobrali je a čeleď pobili ostřím meče. Unikl jsem jenom já a oznamuji ti to.“
#1:18 Ještě nedomluvil, když přišel další a řekl: „Tvoji synové a dcery hodovali a pili víno v domě svého prvorozeného bratra.
#1:19 Vtom se zvedl od pouště silný vítr a opřel se ze všech čtyř stran do domu. Ten se na mladé lidi zřítil a oni zahynuli. Unikl jsem jenom já a oznamuji ti to.“
#1:20 Tu Jób povstal, roztrhl svou řízu a oholil si hlavu. Potom padl k zemi, klaněl se
#1:21 a pravil: „Z života své matky jsem vyšel nahý, nahý se tam vrátím. Hospodin dal, Hospodin vzal; jméno Hospodinovo buď požehnáno.“
#1:22 Při tom všem se Jób nijak neprohřešil a neřekl proti Bohu nic nepatřičného. 
#2:1 A nastal opět den, kdy synové Boží přišli, aby předstoupili před Hospodina; přišel mezi ně i satan, aby i on předstoupil před Hospodina.
#2:2 Hospodin se satana zeptal: „Odkud přicházíš?“ Satan Hospodinu odpověděl: „Procházel jsem zemi křížem krážem.“
#2:3 Hospodin se satana zeptal: „Zdalipak sis všiml mého služebníka Jóba? Nemá na zemi sobě rovného. Je to muž bezúhonný a přímý, bojí se Boha a vystříhá se zlého. Ve své bezúhonnosti setrvává dosud, ačkoli jsi mě proti němu podnítil, abych ho bezdůvodně mořil.“
#2:4 Satan však Hospodinu odpověděl: „Kůži za kůži! Za sebe samého dá člověk všechno, co má.
#2:5 Ale jen vztáhni ruku a dotkni se jeho kostí a jeho masa, hned ti bude do očí zlořečit.“
#2:6 Hospodin na to satanovi odvětil: „Nuže, měj si ho v moci, avšak ušetři jeho život.“
#2:7 A satan od Hospodina odešel a ranil Jóba od hlavy k patě ošklivými vředy.
#2:8 Jób vzal střep, aby se mohl škrábat, a posadil se do popela.
#2:9 Jeho žena mu však řekla: „Ještě se držíš své bezúhonnosti? Zlořeč Bohu a zemři.“
#2:10 Ale on jí odpověděl: „Mluvíš jako nějaká bláhová žena. To máme od Boha přijímat jenom dobro, kdežto věci zlé přijímat nebudeme?“ Při tom všem se Jób svými rty neprohřešil.
#2:11 O všem tom zlém, co Jóba potkalo, se doslechli jeho tři přátelé a přišli každý ze svého místa: Elífaz Témanský, Bildad Šúchský a Sófar Naamatský. Dohodli se spolu, že mu půjdou projevit soustrast a potěšit ho.
#2:12 Rozhlíželi se po něm už zdaleka, ale nemohli ho poznat. Propukli v hlasitý pláč, roztrhli své řízy a rozhazovali nad hlavou k nebi prach.
#2:13 Seděli potom spolu s ním na zemi po sedm dní a nocí a slova k němu žádný nepromluvil, neboť viděli, že jeho bolest je nesmírná. 
#3:1 Pak otevřel Jób ústa a zlořečil svému dni.
#3:2 Jób mluvil takto:
#3:3 „Ať zanikne den, kdy jsem se zrodil, noc, kdy bylo řečeno: ‚Je počat muž.‘
#3:4 Ať se onen den stane temnotou, shůry Bůh ať po něm nepátrá, svítání ať se nad ním nezaskví.
#3:5 Temnota a šerá smrt ať jsou jeho zastánci, ať se na něj snese temné mračno, zatmění dne ať na něj náhle padne.
#3:6 A tu noc, tu mrákota ať vezme, ať se netěší, že je mezi dny roku, do počtu měsíců ať se nedostane.
#3:7 Ta noc ať je neplodná, žádné plesání ať do ní nepronikne.
#3:8 Ať ji zatratí, kdo zaklínají den, ti, kdo dovedou vyburcovat livjátana.
#3:9 Hvězdy ať se zatmí, nežli začne svítat, ať nevzejde světlo, když je bude očekávat, aby nespatřila řasy zory,
#3:10 neboť neuzavřela život mé matky, neskryla trápení před mým zrakem.
#3:11 Proč jsem nezemřel hned v lůně, nezahynul, sotvaže jsem vyšel ze života matky?
#3:12 Proč jsem byl brán na kolena a nač kojen z prsů?
#3:13 Ležel bych teď v klidu, spal bych, došel odpočinku
#3:14 spolu s králi a zemskými rádci, jimž z toho, co zbudovali, zbyly trosky,
#3:15 nebo s velmoži, co měli plno zlata a domy si naplnili stříbrem,
#3:16 nebo jako zahrabaný potrat - nebyl bych tu, jako nedonošený plod, který nespatřil světlo.
#3:17 Svévolníci přestanou tam bouřit, zemdlení tam dojdou odpočinku,
#3:18 vězňové jsou rovněž bez starostí, neslyší křik poháněče,
#3:19 malý i velký jsou si tam rovni, otrok je tam svobodný, bez pána.
#3:20 Proč dopřává Bůh bědnému světlo, život těm, kdo mají v duši hořkost,
#3:21 kdo toužebně čekají na smrt - a ona nejde, ač ji vyhledávají víc než skryté poklady,
#3:22 těm, kdo radostí by jásali a veselili se, že našli hrob?
#3:23 A proč muži, kterému je cesta skryta, ji Bůh zatarasil?
#3:24 Místo abych pojedl, jen vzdychám, nářek ze mne tryská jako voda;
#3:25 čeho jsem se tolik strachoval, to mě postihlo, dolehlo na mě to, čeho jsem se lekal.
#3:26 Nepoznal jsem klidu ani míru ani odpočinutí - a přišla bouře.“ 
#4:1 Na to navázal Elífaz Témanský slovy:
#4:2 „Neponeseš těžce, zkusí-li to někdo s tebou mluvit? Kdo se však dokáže zdržet domluv?
#4:3 Hle, tys napomínal mnohé, ruce ochablé jsi posiloval,
#4:4 tvé domluvy pozvedaly klopýtajícího, podlomená kolena jsi utvrzoval.
#4:5 Teď došlo na tebe a těžce to neseš, sotva tě to zasáhlo, hned naplněn jsi hrůzou.
#4:6 Nedůvěřuješ už ve svou bohabojnost? Nedává ti naději tvůj bezúhonný život?
#4:7 Jen se rozpomeň, kdo z nevinných kdy zhynul? Kde upadli přímí do záhuby?
#4:8 Pokud jsem já viděl, jen ti, kdo se obírají ničemnostmi, ti, kdo rozsívají trápení, je také sklidí.
#4:9 Hynou Božím dechem, když zavane jeho hněv, je s nimi konec.
#4:10 Lev řve, kňučí mladý lvíček, lvíčatům jsou zuby vyraženy.
#4:11 Bez úlovku hyne lev a lví mláďata se rozeběhnou.
#4:12 Cosi se ke mně přikradlo, mé ucho zachytilo šelest;
#4:13 při přemítání o nočních viděních, když na lidi se snáší mrákota,
#4:14 přepadl mě strach a třásl jsem se, všechny kosti se mi strachem chvěly,
#4:15 když jakýsi duch mě míjel, chlupy se mi zježily po těle.
#4:16 Stanul - ale jeho zjev jsem nerozeznal, jen podoba jakási stanula před mým zrakem a v tichu jsem slyšel hlas:
#4:17 ‚Což je člověk spravedlivější než Bůh, čistší muž než jeho Učinitel?‘
#4:18 Nemůže-li věřit vlastním služebníkům, shledává-li omylnost i na andělech,
#4:19 tím spíš na těch, kteří přebývají ve hliněných domech a svým základem tkví v prachu; ty rozmáčkne snadnějši než mola.
#4:20 Než se setká ráno s večerem, už budou rozdrceni, nežli si to uvědomí, navždy zhynou.
#4:21 Bývá s nimi vytrženo i jejich stanové lano; umírají, ale ne v moudrosti. 
#5:1 Jen si volej, odpoví ti někdo? Na koho ze svatých se obrátíš?
#5:2 Pošetilce zabíjí vztek, žárlivost usmrcuje prostoduché.
#5:3 Viděl jsem, jak pošetilec zakořenil, vím však, že jeho příbytek propadne zatracení,
#5:4 Jeho synům záchrana se vzdálí, v bráně budou zdeptáni, nevysvobodí je nikdo.
#5:5 Hladový sní jeho sklizeň, i z trní si ji vezme, žíznivý po jeho majetku baží.
#5:6 Ničemnost přec nevzchází z prachu, trápení neklíčí z půdy,
#5:7 Člověk je však zrozen pro trápení a jiskry, aby létaly vzhůru.
#5:8 Spíše bych se dotazoval Boha, svoji záležitost předložil bych Bohu,
#5:9 který dělá věci veliké a nevyzpytatelné, nesčíslné divy:
#5:10 Dává mi déšť, polím sesílá vláhu,
#5:11 ponížené staví na vysoké místo, zarmoucení docházejí spásy;
#5:12 chytrákům však hatí plány, aby jejich ruce neprovedly to, čeho jsou schopni;
#5:13 moudré jejich chytrostí dovede lapit, takže záměr potměšilců nadobro se zvrtne:
#5:14 ve dne s temnotou se střetávají, v pravé poledne tápou jak v noci;
#5:15 ubožáka od meče zachraňuje, z jejich úst a z jejich pevné ruky;
#5:16 a tak nuzný má naději, ale podlost musí zavřít ústa.
#5:17 Věru, blaze člověku, jehož Bůh trestá; kázeň Všemocného neodmítej.
#5:18 On působí bolest, ale též obváže rány, co rozdrtí, vyléčí svou rukou.
#5:19 Z šesti soužení tě vysvobodí, v sedmi nezasáhne tě nic zlého,
#5:20 vykoupí tě ze smrti v čas hladu a za války z moci meče,
#5:21 před bičem jazyka budeš ukryt, nebudeš se bát, až přijde zhouba,
#5:22 vysměješ se zhoubě, hladomoru; a neboj se zemské zvěře,
#5:23 vždyť budeš mít smlouvu s kamením na poli a polní zvěř bude žít pokojně s tebou,
#5:24 shledáš, že je pokoj ve tvém stanu, dohlédneš-li na svůj příbytek, neshledáš hříchu,
#5:25 shledáš, že tvé potomstvo je četné, tvoji potomci že jsou jak bylina země,
#5:26 do hrobu sestoupíš ve zralosti, jako se sváží požaté obilí ve svůj čas.
#5:27 Hle, toto jsme vyzpytovali, tak je tomu. Poslechni a sám to poznáš.“ 
#6:1 Jób na to odpověděl:
#6:2 „Kéž by bylo dobře zváženo mé hoře a mé neštěstí na vážky přiloženo!
#6:3 Věru, těžší je než mořský písek, že se mi až slova pletou,
#6:4 neboť ve mně vězí střely Všemocného, můj duch se napájí jejich jedem, seřadily se proti mně hrůzy Boží.
#6:5 Hýká snad divoký osel, když má mladou trávu, bučí snad býk nad svou pící?
#6:6 Což lze bez soli jíst něco mdlého? Má nějakou chuť vaječný bílek?
#6:7 Štítím se dotýkat toho, co by můj chléb znečistilo.
#6:8 Kéž přijde, oč žádám, a kéž Bůh dá, v co naději skládám,
#6:9 aby mě Bůh ráčil rozmáčknout jak mola, pohnout rukou, odlomit mě z kmene.
#6:10 Bylo by to pro mne potěšením, navzdor nelítostným bolestem bych poskakoval, neboť slova Svatého jsem nezatajil.
#6:11 Kde naberu sílu, abych to přečkal? Kdy to skončí, abych to vydržel?
#6:12 Je snad z kamene má síla a mé tělo z bronzu?
#6:13 Cožpak mi pomoci není? Záchrana je mi odepřena?
#6:14 Kdo své milosrdenství bližnímu odepírá, ten opouští bázeň Všemocného.
#6:15 Mí bratři jsou věrolomní, nestálí jak potok, jak koryta potoků, které se vytrácejí,
#6:16 jsou kalné od ledu, když sníh nad nimi taje,
#6:17 v čas léta se vypařují, jeho žárem mizejí ze svého místa,
#6:18 jen stružkami jejich tok se vine, plynou v pustotu a zanikají.
#6:19 Vyhlížely je karavany z Témy, s nadějí k nim hleděly výpravy ze Šeby,
#6:20 za své doufání však musely se stydět, přišli k nim a zklamaly se.
#6:21 I vy jste teď, jako byste nebyly, při tom děsném pohledu vás jala bázeň.
#6:22 Řekl jsem snad: ‚Dejte mi‘ či: ‚Plaťte za mne vlastním zbožím‘
#6:23 nebo: ‚Zachraňte mě z rukou protivníka‘ či snad: ‚Vykupte mě z rukou ukrutníků‘ ?
#6:24 Poučte mě a já zmlknu, vysvětlete mi, v čem jsem chybil.
#6:25 Přímá slova mohou zjitřit ránu, a co sledujete, že mi stále domlouváte?
#6:26 Chcete mě snad kárat za má slova? Cožpak mluví do větru ten, kdo si zoufá?
#6:27 Věru, metáte los o sirotka, svého druha jste ochotni prodat.
#6:28 Buďte tak laskavi a obraťte se ke mně, což bych vám mohl do očí lhát?
#6:29 Zadržte, ať nespáchá se podlost, zadržte, ať je tady ještě moje spravedlnost!
#6:30 Což může být na mém jazyku nějaká podlost? Nepozná mé patro to, co vede do neštěstí? 
#7:1 Zdali není člověk na zemi podroben v službu, nejsou jeho dny jako dny nádeníka?
#7:2 Jako baží otrok po stínu a jak nádeník čeká na výdělek,
#7:3 tak se mi dostaly dědictvím daremné měsíce, noci plné trápení se staly mým údělem.
#7:4 Když uléhám, ptám se: ‚Kdy už vstanu?‘ a pak zase: ‚Kdy se snese večer?‘ Syt jsem toho, na lůžku se převalovat do rozbřesku.
#7:5 Mé tělo je obaleno červy a strupy plnými prachu, kůže mi puká a mokvá.
#7:6 Rychleji než tkalcův člunek uběhly mé dny, skončily v naprosté beznaději.
#7:7 Bože, pomni, že můj život uprchne jak vítr a nic dobrého už nikdy nespatří mé oči.
#7:8 Neuzří mě oko, jež mě vídávalo, budou-li mě tvoje oči hledat, nebudu tu.
#7:9 Oblak se rozplyne, zmizí; stejně kdo sestoupí do podsvětí, už nevystoupí,
#7:10 nevrátí se nikdy zpět do svého domu, neobjeví se už na svém místě.
#7:11 A tak bránit nemohu svým ústům, mluví ze mne úzkost mého ducha, lká ze mne hořkost mé duše.
#7:12 Jsem snad moře nebo dračí netvor, že proti mně stavíš stráž?
#7:13 Řeknu-li si: ‚Potěší mě moje lože, mé lůžko mi ulehčí v mém lkání‘,
#7:14 děsíš mě skrze sny a přepadáš mě viděními,
#7:15 že bych spíše volil zardoušení, spíše smrt než kruté trápení.
#7:16 Život se mi zprotivil, nechci žít věčně, už mě nech, mé dny jsou pouhý vánek.
#7:17 Co je člověk, že mu přikládáš význam, že se jím zabýváš v srdci,
#7:18 že na něj dohlížíš každého rána a každou chvíli ho zkoušíš?
#7:19 Proč svůj zrak ode mne neodvrátíš, nenecháš mě ani slinu polknout?
#7:20 Zhřešil-li jsem, co mám podle tebe dělat, hlídači lidí? Proč sis mě vzal za terč, až jsem se stal břemenem sám sobě?
#7:21 Proč mi přestupek můj nepromineš, nesejmeš ze mne mou vinu? Již uléhám do prachu a až mě budeš za úsvitu hledat, nebudu již.“ 
#8:1 Na to navázal Bildad Šúchský slovy:
#8:2 „Chceš takhle rozprávět ještě dlouho? Slova tvých úst jsou jako prudký vítr.
#8:3 Což Bůh křiví právo, Všemocný snad překrucuje spravedlnost?
#8:4 Jestliže tví synové se proti němu prohřešili, vydal jsi v moc jejich nevěrnosti.
#8:5 Budeš-li však za úsvitu hledat Boha a o milost prosit Všemocného,
#8:6 budeš-li ryzí a přímý, jistě bude nad tebou bdít a obnoví tvůj příbytek pro tvou spravedlnost.
#8:7 Pakli toho, cos měl prve, bylo málo, převelice vzroste, co budeš mít potom.
#8:8 Jen se zeptej předešlého pokolení, a co vyzkoumali jejich otcové, buď hotov slyšet.
#8:9 Jsme tu jenom od včerejška, nic jsme nepoznali, naše dny jsou na zemi jen stínem.
#8:10 Oni tě však poučí, řeknou ti všichno, a ze svého srdce pronesou řeč.
#8:11 Cožpak roste rákos, kde není bažina? Může bez vody vzrůst sítí?
#8:12 Ještě raší, posekat je nelze, a schne dříve než ostatní tráva.
#8:13 Tak je tomu se stezkami všech, kteří na Boha zapomněli, naděje rouhače přijde vniveč;
#8:14 ve své důvěřivé jistotě se zklame, jeho doufání - toť pavučina.
#8:15 Opře se o svůj dům, ale ten neobstojí, podrží se ho, on však nebude stát.
#8:16 Na slunci je plný mízy, jeho výhonek přerůstá ze zahrady,
#8:17 jeho kořeny se proplétají kamennými valy, lze jej vidět i v kamenných domech.
#8:18 Je-li však vyhlazen ze svého místa, ono se ho zřekne: ‚Nikdy jsem tě nevidělo.‘
#8:19 Hle, takové jsou radosti jeho cesty; z jeho prachu vyraší hned jiný.
#8:20 Ovšem, bezúhonného Bůh nezavrhne ani ruku zlovolníků neposílí.
#8:21 Jistě naplní tvá ústa smíchem a hlaholem tvé rty.
#8:22 Kdo tě nenávidí, budou oblečeni v hanbu, po stanu svévolných nezbude nic.“ 
#9:1 Jób na to odpověděl:
#9:2 „Vskutku vím, je tomu tak, což může člověk být před Bohem spravedlivý?
#9:3 Kdo by s ním chtěl vésti spor, z tisíce otázek jedinou nezodpoví.
#9:4 On má srdce moudré a nesmírnou sílu, což dojde pokoje ten, kdo se mu vzepře?
#9:5 On přenáší hory, než by se kdo nadál, převrací je v hněvu;
#9:6 zemí pohne z místa, až se její sloupy chvějí.
#9:7 Slunci rozkáže - a nesmí vzejít, zapečeťuje i hvězdy,
#9:8 sám nebesa roztahuje, kráčí po hřebenech mořských vln,
#9:9 on udělal souhvězdí Lva, Orióna i Plejády a souhvězdí jižní.
#9:10 Dělá věci veliké a nevyzpytatelné, nesčíslné divy.
#9:11 Jde-li mimo mne, nevidím ho, míjí-li mě, ani ho nepostřehnu.
#9:12 Jestliže co uchvátí, kdo ho donutí to vrátit, kdopak se ho zeptá: ‚Co to děláš?‘
#9:13 Bůh, ten hněv svůj neodvrací, sami pomocníci Netvora se před ním musí shrbit.
#9:14 Jak bych mu já tedy mohl odpovídat? Jak bych před ním volil svoje slova?
#9:15 Jemu neuměl bych odpovědět, i kdybych byl spravedlivý; svého Soudce jenom o milost bych prosil.
#9:16 A kdybych i zavolal, aby mi odpověděl, nevěřím, že přál by sluchu mému hlasu.
#9:17 Vždyť mě zachvacuje vichrem, bezdůvodně rozmnožuje moje rány,
#9:18 ani oddechnout mi nedá a jen hořkostmi mě sytí.
#9:19 Má-li kdo nesmírnou sílu, má ji on, a co se týče soudu, kdopak jiný mě předvolá?
#9:20 I kdybych byl spravedlivý, za svévolníka mě prohlásí má ústa, a kdybych byl bezúhonný, prohlásí mě za křivého.
#9:21 Jsem bezúhonný. Nic na sebe nevím. Protiví se mi už život.
#9:22 Je to jedno, proto říkám: On skoncuje s bezúhonným jako se svévolníkem.
#9:23 A když bičem náhle usmrcuje, ze zoufalství nevinných si činí posměch.
#9:24 Země byla vydána v moc svévolníka a on přikrývá tvář jejich soudců; když ne on, kdo tedy?
#9:25 Mé dny byly rychlejší než spěšný posel, uprchly a neužily dobra,
#9:26 prolétly jak rákosové čluny, jako orel na kořist se vrhající.
#9:27 Řeknu-li si: Zapomenu na své lkání, smutku zanechám a pookřeji,
#9:28 hned se všeho toho trápení zas lekám, neboť vím, že trest mi nepromineš.
#9:29 Jestliže jsem si svévolně vedl, co se budu namáhat pro nějaký přelud?
#9:30 I kdybych se umyl sněhem, dlaně si očistil louhem,
#9:31 přece bys mě vnořil do takové jámy, že by si mě hnusil i můj šat.
#9:32 On přec není jako já, abych mu odpovídal, abychom v soud vešli spolu.
#9:33 Není mezi námi rozhodčího, jenž by vložil ruku na nás oba.
#9:34 Kéž by odňal ode mne svou hůl a nepřepadal mě jak postrach.
#9:35 Mluvil bych a nebál se ho, ale v mém případě tomu tak není. 
#10:1 Život mě omrzel, dám teď volný průchod svému lkání, budu mluvit v hořkosti své duše.
#10:2 Řeknu Bohu: Za svévolníka mě nepokládej, dej mi vědět, proč vedeš spor se mnou.
#10:3 K čemu je ti dobré, že mě týráš? Zprotivil se ti výtvor tvých rukou, že dáváš zářit záměrům svévolníků?
#10:4 Cožpak máš tělesné oči, což se díváš stejně jako člověk?
#10:5 Jsou tvoje dny jako dny člověka, léta tvá jako dny muže,
#10:6 že vyhledáváš můj přečin a že pátráš po mém hříchu?
#10:7 Vždyť víš, že svévolník nejsem, a že nikdo nevysvobodí z tvé ruky.
#10:8 Tvé ruce mě ztvárnily a udělaly se vším všudy, a teď najednou mě hubíš.
#10:9 Prosím, upamatuj se, že jsi mě učinil jako hlínu a že mě obracíš v prach.
#10:10 Což jsi mě nenalil do nádoby jako mléko a jako sýr nenechal srazit?
#10:11 Přioděl jsi mě kůží a masem, propletls mě šlachami a kostmi,
#10:12 nakládal jsi se mnou milosrdně, dals mi život a tvůj dohled střežil mého ducha.
#10:13 Ale ve svém srdci ukryls toto - bylo to tvým úmyslem, to vím - :
#10:14 že proti mě budeš ve střehu, jestliže zhřeším, že nenecháš bez trestu můj přečin.
#10:15 Běda mně, kdybych byl svévolně jednal! Ač jsem spravedlivý, hlavu nepozvedám, hanbou přesycen vidím své pokoření.
#10:16 Kdybych se pozvedal, jak lev bys mě honil, svoji divuplnou moc bys opět na mně zjevil.
#10:17 Stavěl bys proti mně nové svědky, stupňoval své roztrpčení na mě, vystřídaly by se u mne celé voje.
#10:18 Proč jsi mě vyvedl z matčina lůna! Kéž bych byl zhynul a žádné oko mě nespatřilo.
#10:19 Byl bych, jako bych nikdy nebyl, byl bych nesen ze života matky k hrobu.
#10:20 Což není mých dnů tak málo? Kéž by toho nechal a odstoupil ode mne, abych trochu okřál,
#10:21 dřív než půjdu tam, odkud návratu není, do země temnot a šeré smrti,
#10:22 do země temné jak mračno, do šera smrti, kde není řádu, kde záblesk svítání je jako mračno.“ 
#11:1 Na to navázal Sófar Naamatský slovy:
#11:2 „Má takové množství slov zůstat bez odpovědi? Má mluvka být v právu?
#11:3 Mohou lidé k tomu, co povídáš, mlčet? Máš ty se dál vysmívat a nikdo tě neusadí?
#11:4 Říkáš: ‚Co jsem zastával, je ryzí, jsem před tebou čistý.‘
#11:5 Jen kdyby Bůh promluvil a otevřel rty proti tobě,
#11:6 prozradil by ti taje moudrosti: dvojnásobný trest k záchraně vede. Věz, že Bůh chce zapomenout na tvé nepravosti.
#11:7 Dokážeš vystihnout Boha či obsáhnout dokonalost Všemocného,
#11:8 jež nebesa převyšuje? Co chceš dělat? Hlubší je než podsvětí. Co o tom víš?
#11:9 Její míra je delší než země, širší nežli moře.
#11:10 Chce-li změnit, uzavřít, svolat, kdo ho odvrátí?
#11:11 Ano, on zná falešníky, vidí ničemnosti - a srozuměn není.
#11:12 Může tupec dostat rozum? Narodí se hříbě divokého osla jako člověk?
#11:13 Jestliže teď napravíš své srdce a vztáhneš své ruce k Bohu,
#11:14 jestliže dáš ruce pryč od ničemností, nepřipustíš, aby ve tvém stanu přebývala podlost,
#11:15 tedy pozdvihneš tvář bez poskvrny, budeš jak odlitý z bronzu, nepocítíš bázně,
#11:16 zapomeneš na trápení, bude ve tvých vzpomínkách jak voda, která uplynula.
#11:17 Nadejde ti věk jasnější nad poledne, chmury obrátí se v jitro.
#11:18 Doufej, naděje ti kyne, pohleď, budeš uléhat v bezpečí.
#11:19 Budeš odpočívat a nikdo tě nevyděsí, získat tebe budou si přát mnozí,
#11:20 kdežto svévolníkům vypoví zrak, ztratí útočiště, jejich nadějí je: vydechnout duši.“ 
#12:1 Jób na to odpověděl:
#12:2 „Vy jste vskutku ten pravý lid, s vámi vymře moudrost!
#12:3 Ale já mám také rozum jako vy, nejsem zpozdilejší než vy; kdopak tohle neví?
#12:4 Jsem k posměchu i vlastnímu příteli, jemuž Bůh odpoví, když volá; v posměchu je spravedlivý, bezúhonný.
#12:5 Bezstarostný smýšlí o zániku pohrdavě, sám stojí pevně, když nohy jiných vrávorají!
#12:6 Stany zhoubců zůstávají nerušené, v bezpečí jsou ti, kdo popouzejí Boha, i ten, kdo chce Boha mít v své moci.
#12:7 Avšak dobytka se zeptej, poučí tě, nebeského ptactva, ono ti to poví,
#12:8 poučí tě i křoviska země, mořské ryby vyprávět ti budou.
#12:9 Kdo z nich všech by nevěděl, že ruka Hospodinova to učinila
#12:10 a že v jeho ruce je život všeho, co žije, duch každého lidského tvora.
#12:11 Zda nezkouší ucho slova jako patro ochutnává pokrm?
#12:12 Což jen u kmetů je moudrost a rozumnost pouze v dlouhém věku?
#12:13 Moudrost, ta je u Boha, i bohatýrská síla, u něho je rozvaha i rozum.
#12:14 Co rozboří, nikdo nezbuduje, zavře dveře za někým a otevřít je nelze.
#12:15 Když zadrží vody, přijde sucho, když je vypustí, pak podvracejí zemi.
#12:16 U něho je moc i pohotová pomoc, patří mu, kdo chybuje, i ten, kdo svádí.
#12:17 Rádce odvádí vysvlečené, ze soudců činí ztřeštěnce,
#12:18 pouta králů rozvazuje, pásem ovazuje jejich bedra,
#12:19 kněze odvádí vysvlečené, vyvrací prastaré rody,
#12:20 odnímá řeč spolehlivým, starcům bere soudnost,
#12:21 opovržením zahrne urozené, uvolňuje hráze řečišť,
#12:22 odkrývá hlubiny temnot a na světlo vyvádí, co je v šeru smrti,
#12:23 pronárodům dává vzrůst i zánik, rozprostírá pronárody i odvádí je,
#12:24 bere rozum náčelníkům lidu země, zavádí je do bezcestných pustot,
#12:25 aby tápali v temnotě beze světla, ano, zavádí je jako opilého. 
#13:1 Hle, to všechno spatřilo mé oko, mé ucho to vyslechlo, rozumím tomu.
#13:2 Co víte vy, to vím také, nejsem zpozdilejší než vy.
#13:3 Ano, budu mluvit se Všemocným, obhájit se chci před Bohem.
#13:4 Ale vy jste šiřitelé klamu, lékaři k ničemu, vy všichni.
#13:5 Kéž byste konečně zmlkli, bylo by to od vás moudré.
#13:6 Slyšte, jak se budu hájit, důvody mých rtů sledujte s pozorností.
#13:7 To v zájmu Boha mluvíte podlost, mluvíte lest kvůli němu?
#13:8 Chcete se zastávat Boha nebo vést spor místo něho?
#13:9 Bude-li vás zkoumat, dopadnete dobře? Či ho chcete obloudit, jako lze obloudit člověka?
#13:10 Tvrdě vás potrestá, budete-li jednat pokoutně a stranit.
#13:11 Což vás neohromí jeho vznešenost, nepadne na vás strach z něho?
#13:12 Vaše připomínky jsou pořekadla z popela, čím se oháníte, je pouhá hlína.
#13:13 Zmlkněte přede mnou, ať mohu mluvit, pak ať se přese mne přežene cokoli.
#13:14 Chci nasadit svou kůži, dát svůj život v sázku.
#13:15 I kdyby mě zabil a já už neměl co očekávat, přece bych chtěl před ním obhájit své cesty.
#13:16 Vždyť on je má spása. Rouhač k němu nemá přístup.
#13:17 Slyšte, slyšte mou řeč, dopřejte sluchu tomu, co vám sdělím.
#13:18 Hleďte, předkládám svou při, vím, že budu uznán spravedlivým.
#13:19 Kdo chce se mnou vésti spor? Budu-li mlčet, zhynu.
#13:20 Jen dvojí mi, Bože, nečiň, a nebudu se před tebou skrývat:
#13:21 Vzdal ode mne svoji ruku a strach z tebe ať mě nepřepadá.
#13:22 Zavolej, a já se ozvu, nebo budu mluvit já, a odpovíš mi.
#13:23 Kolik je mých nepravostí a mých hříchů? Dej mi poznat mou nevěrnost a hřích.
#13:24 Proč skrýváš svou tvář a pokladáš mě za svého nepřítele?
#13:25 Chceš postrašit odvátý list, honit suché stéblo?
#13:26 Věru, znamenáš si na mě trpké věci, přičítáš mi nepravosti mého mládí,
#13:27 svíráš do klády mé nohy, dáváš pozor na všechny mé stezky, zanášíš si každou šlápotu mých nohou.
#13:28 Člověk se rozpadá jako něco zetlelého, jako přikrývka rozežraná moly. 
#14:1 Člověk narozený z ženy má krátký věk, avšak nepokoje do sytosti.
#14:2 Jako květ vzejde a zvadne, prchá jako stín a neobstojí.
#14:3 Přesto na něj upíráš svůj zrak a přivádíš mě na soud s tebou.
#14:4 Kdo dokáže, aby čisté vzešlo z nečistého? Vůbec nikdo.
#14:5 Jestliže jsou stanoveny jeho dny, počet jeho měsíců je ve tvé moci, nepřekročí cíl, jejž jsi mu vytkl.
#14:6 Odvrať od něho své oči, ať si pooddechne jako nádeník, který je rád, že má den za sebou.
#14:7 Stromu zbývá aspoň naděje, že i když podťat, začne rašit znovu a že jeho výhonky růst nepřestanou,
#14:8 byť odumřel jeho kořen v zemi a na prach ztrouchnivěl jeho pařez.
#14:9 Jak ucítí vodu, pučí znovu, rozvětví se jako mladý stromek.
#14:10 Zemře-li muž, rozpadne se. Zhyne-li člověk, kam se poděl?
#14:11 Z moře se vytratí vody, řeka opadne a vyschne;
#14:12 člověk ulehne a nepovstane a dokud nebesa budou, neprocitne, ze spánku se neprobudí.
#14:13 Kéž bys mě skryl v podsvětí a schoval mě, než pomine tvůj hněv, stanovil mi lhůtu a pamatoval na mě.
#14:14 Kdyby mohl ožít muž, jenž zemřel, čekal bych po všechny dny své služby, až budu vystřídán.
#14:15 Zavolal bys a já bych se ozval, až se ti zasteskne po díle tvých rukou.
#14:16 A kdybys pak počítal mé kroky, nedbal bys už mého hříchu,
#14:17 zapečetěna by byla do uzlíku má nevěrnost, moji nepravost bys zastřel.
#14:18 Také hora se rozpadne a zřítí, i skála se pohne z místa;
#14:19 voda kameny omílá, svým proudem odplaví prach země; tak ničíš naději člověka.
#14:20 Dotíráš na něho vytrvale, dokud neodejde, měníš jeho tvář a vyhostíš ho.
#14:21 Neví, jsou-li jeho synové ve cti, není mu známo, jsou-li v nevážnosti.
#14:22 Tělo bolestmi ho souží, sám nad sebou truchlí.“ 
#15:1 Na to navázal Elífaz Témanský slovy:
#15:2 „Může moudrý člověk hlásat tak naduté vědomosti, naplnit si břicho větrem od východu?
#15:3 Obhajovat se slovem, jež k ničemu není, řečmi, které neprospějí?
#15:4 Ty sám porušuješ bázeň Boží, rozjímat před Bohem znemožňuješ.
#15:5 Tvá ústa jsou zajedno s tvou nepravostí, jazyk chytrácký jsi zvolil.
#15:6 Ne já, nýbrž tvá ústa tě usvědčují ze svévole, tvoje rty vypovídají proti tobě.
#15:7 Jsi snad zrozen jako první z lidí, přišels na svět dříve než pahorky?
#15:8 Vyslechls důvěrný rozhovor Boží, že strhuješ jen na sebe moudrost?
#15:9 Co víš, abychom to nevěděli, čemu rozumíš, a nám to není známé?
#15:10 Šedivý i kmet jsou mezi námi, věkem ctihodnější než tvůj otec.
#15:11 Boží útěchy jsou tobě málo, když láskyplně s tebou mluví?
#15:12 Co tě připravuje o rozum? Proč blýskáš očima?
#15:13 Svým duchem se stavíš proti Bohu, vypouštíš z úst nehorázná slova.
#15:14 Co je člověk? Je snad bez poskvrny? Což může být spravedlivý, kdo se zrodil z ženy?
#15:15 Hle, on ani na své svaté nedá a nebesa nejsou v jeho očích bez poskvrny,
#15:16 což teprve ohavný a zvrhlý člověk, který pije podlost jako vodu?
#15:17 Poslouchej mě, co ti sdělím. Budu ti vyprávět, co jsem uzřel,
#15:18 co hlásali moudří, co jim netajili jejich otcové,
#15:19 ti, jimž jediným byla dána země, takže cizák mezi nimi nepřecházel:
#15:20 Svévolník se úzkostlivě chvěje po všechny dny, hrůzovládci počet jeho let je ukryt.
#15:21 Jeho uši slyší zvuk strašlivých zvěstí, v čas pokoje přijde na něj zhoubce.
#15:22 Nevěří, že by se z temnot vrátil - je vyhlédnut pro meč.
#15:23 Poplašeně bloudí, že není chléb, ví, že je mu připraven den temnot.
#15:24 Soužení a úzkosti ho přepadají, dotírají na něho jako král k útoku připravený,
#15:25 neboť vztáhl ruku proti Bohu, počínal si vyzývavě proti Všemocnému,
#15:26 rozběhl se proti němu se skloněnou šíjí, pod hustým krunýřem svých štítů.
#15:27 Svoji tvář zahalil tukem, ztučněl na slabinách.
#15:28 Usadil se ve zničených městech, v neobývatelných domech, hrozících zřícením.
#15:29 Nezůstane bohatý a jeho statek neobstojí, jeho majetek se na zemi nerozmůže.
#15:30 Nevyjde z temnot, jeho výhonek sežehne plamen; sám zajde dechem svých úst.
#15:31 Ač nevěří v šalebnost, bude jí zaveden, výměnou sklidí zase jen faleš.
#15:32 Než vyprší jeho den, plné odměny se dočká, jeho ratolest se nebude zelenat,
#15:33 bude jako vinná réva, když shazuje nedozrálé hrozny, nebo jak oliva shazující květy.
#15:34 Spolek rouhačů zůstane neplodný, oheň zhltá stany úplatkářů;
#15:35 plodí trápení a rodí ničemnosti, jejich lůno připravuje lest.“ 
#16:1 Jób na to odpověděl:
#16:2 „Slyšel jsem už mnoho podobného, těšíte mě všichni jen trápením.
#16:3 Kdypak skončí to mluvení do větru? Co tě rozjitřuje, že tak odpovídáš?
#16:4 Také bych mohl mluvit jako vy, kdybyste vy byli na mém místě, ohánět se proti vám slovy, potřásat nad vámi hlavou.
#16:5 Já bych vám však svými ústy dodával odvahu, svými rty bych šetrně projevil soustrast.
#16:6 Promluvím-li, nezůstanu ušetřen bolesti, když od toho upustím, co ztratím?
#16:7 Naplňuje mě teď, Bože, malomyslností, že jsi zpustošil celou mou pospolitost.
#16:8 Zasáhl jsi mě, mám na to svědka, je jím má vychrtlost, ta mi to do očí dokazuje.
#16:9 Jeho hněv mě rozsápal, zanevřel na mě, skřípe na mě svými zuby; můj protivník zaostřuje na mě svůj zrak.
#16:10 Dokořán na mně otvírají ústa, potupně mě políčkují a všichni se na mě hrnou.
#16:11 Bůh mně dal v plen padouchovi, napospas mě vydal spárům svévolníků.
#16:12 Žil jsem poklidně, on však mnou zacloumal, uchopil mě za šíji a roztříštil mě; učinil mě svým terčem.
#16:13 Obklíčili mě jeho střelci, nelítostně roztíná mé ledví, moji žluč vylévá na zem.
#16:14 Drásá mě, jsem celý rozdrásaný, doráží na mě jako bohatýr.
#16:15 Na svou zjizvenou kůži jsem si vzal žínici, svůj roh jsem do prachu sklonil.
#16:16 Tvář mi opuchla od pláče, na má víčka padlo šero smrti,
#16:17 ač násilí na mých rukou nelpí a má modlitba je ryzí.
#16:18 Země, krev mou nepřikrývej, můj křik ať nenajde místa klidu!
#16:19 Ale nyní, hle, mám svědka na nebesích, můj přímluvce je na výšinách.
#16:20 Ať se mi přátelé posmívají, moje oko hledí v slzách k Bohu.
#16:21 Kéž je muži dáno hájit se před Bohem, lidskému synu před jeho bližním.
#16:22 Vždyť až přejde počet mých let, půjdu stezkou, z níž se nenavrátím. 
#17:1 Na duchu jsem zlomen, mé dny dohasly, jen hrob mi zbývá.
#17:2 Co posměšků zakouším, stále mě napadají, oka nezamhouřím.
#17:3 Slož u sebe za mě záruku, kdo jiný by se zaručil rukoudáním?
#17:4 Jejich srdce prozíravosti jsi zbavil, a proto je nevyvýšíš.
#17:5 Přátelům se pochlebuje, vlastním synům vypovídá zrak.
#17:6 Učinil mě pořekadlem lidu, tím, na nějž se plivá.
#17:7 Můj zrak pohasl hořem, všechny mé údy jsou už jen stín.
#17:8 Poctiví nad tím žasnou, nevinný je pobouřen rouhačem.
#17:9 Spravedlivý se však přidrží své cesty, kdo má čisté ruce, bude ještě odvážnější.
#17:10 Vy všichni, obraťte se a pak přijďte, moudrého však mezi vámi nenacházím.
#17:11 Mé dny pomíjejí, moje záměry se hatí, přání mého srdce ztroskotala.
#17:12 Noc vydávají za den, mluví o světle, kde temnota je blízko.
#17:13 I kdybych měl naději, podsvětí bude mým domem, lože si ustelu ve tmách.
#17:14 Jámě řeknu: ‚Tys můj otec‘, červům: ‚Matko má, má sestro.‘
#17:15 Kde mám jakou naději a splnění mé naděje kdo spatří?
#17:16 Závory podsvětí zapadnou, až se spolu do prachu uložíme.“ 
#18:1 Na to navázal Bildad Šúchský slovy:
#18:2 „Jak dlouho ještě povedete tyhle řeči? Rozvažte to a pak budem mluvit.
#18:3 Proč jsme ceněni jak dobytek, jsme snad nečistí ve vašich očích?
#18:4 Ty, který sám sebe v hněvu rozsápáváš, kvůli tobě má být opuštěna země, má se skála přemístit ze svého místa?
#18:5 Avšak světlo svévolníka zhasne, plamen jeho ohně nezazáří,
#18:6 světlo v jeho stanu ztemní, zhasne nad ním jeho kahan.
#18:7 Těsno bude jeho rázným krokům, vlastní plány přivedou ho k pádu,
#18:8 nohama se zaplete do sítě, prochází se po pletivu nad pastí,
#18:9 za patu se chytí do osidla, zadrhne se kolem něho smyčka.
#18:10 Na zemi je ukryt na něj provaz, nástraha na něho na pěšině.
#18:11 Ze všech stran ho přepadají hrůzy, ženou se mu v patách.
#18:12 Ať vychrtne jeho síla, bědy ať mu připraví pád!
#18:13 Ať po kusech sžírá jeho kůži, ať mu Kníže smrti pozře údy.
#18:14 Vytržen z bezpečí svého stanu musí kráčet v náruč Krále hrůzy.
#18:15 V jeho stanu se zabydlí, co mu nepatřilo, po jeho příbytku bude roztroušena síra.
#18:16 Zdola mu uschnou kořeny a svrchu mu uvadnou větve.
#18:17 Jeho památka vymizí ze země, nezůstane po něm nikde jméno.
#18:18 Vyženou ho ze světla do temnot, zapudí ho pryč ze světa.
#18:19 Nezůstane mu nástupce a následník v jeho lidu, z místa, kde pobývá, nikdo nevyvázne.
#18:20 Nad jeho dnem strnou děsem na západě, na východě se jich zmocní hrůza.
#18:21 Tak to dopadne s příbytky bídáků, s místem, kde neznali Boha.“ 
#19:1 Jób na to odpověděl:
#19:2 „Jak dlouho ještě mě budete trápit a mučit svými řečmi?
#19:3 Nejméně desetkrát už jste mi utrhali na cti. Vy se nestydíte se mnou tak nestoudně jednat?
#19:4 I kdybych opravdu chybil, mé pomýlení zůstane na mně.
#19:5 Chcete se opravdu nade mne vynášet a tupit mě svými domluvami?
#19:6 Uznejte přece, že mi Bůh křivdí, zatáhl kolem mne loveckou síť.
#19:7 Úpím-li pro násilí, zůstávám bez odpovědi, o pomoc volám a zastání není.
#19:8 Mou cestu zahradil zdí, že nemohu projít, mé stezky obestřel temnem.
#19:9 Mou slávu ze mne svlékl a sňal korunu z mé hlavy.
#19:10 Ze všech stran mě boří, abych zašel, vyvrátil mou naději jako strom.
#19:11 Rozpálil se na mě hněvem, považuje mě za svého protivníka.
#19:12 Společně přitáhly jeho houfy, navršily proti mně svou cestu a táboří kolem mého stanu.
#19:13 Mé bratry ode mne vzdálil, moji známí se mi odcizili,
#19:14 moji příbuzní mě opustili, kdo se ke mně znali, zapomněli na mě.
#19:15 Hosté mého domu i mé služky mě pokládají za cizího, v jejich očích jsem cizozemec.
#19:16 Když zavolám na otroka, neodpoví, svými ústy se ho musím doprošovat.
#19:17 Můj dech se oškliví i mé ženě, vlastním dětem páchnu.
#19:18 I padouši se mě štítí, když chci povstat, spílají mi,
#19:19 všichni moji důvěrní přátelé si mě hnusí, ti, které jsem miloval,se ke mně obracejí zády.
#19:20 Jsem vyzáblý, kost a kůže, vyvázl jsem jenom s kůží kolem zubů.
#19:21 Smilujte se, smilujte se nade mnou, přátelé moji, neboť se mě dotkla ruka Boží.
#19:22 Proč mě pronásledujete jako Bůh a nemůžete se nasytit mého masa?
#19:23 Kéž by byly mé řeči sepsány, vyznačeny jako nápis
#19:24 rydlem železným a olovem, do skály trvale vytesány!
#19:25 Já vím, že můj Vykupitel je živ a jako poslední se postaví nad prachem.
#19:26 A kdyby mi i kůži sedřeli, ač zbaven masa, uzřím Boha,
#19:27 já ho uzřím, pro mne tu bude, mé oči ho uvidí, ne někdo cizí, mé ledví po tom prahne v mé nitru.
#19:28 Říkáte: ‚Jak ho chytit?‘ Ve mně prý je příčina všeho.
#19:29 Lekejte se meče. Mečem na nepravosti je rozhořčení. Poznáte soud Všemocného.“ 
#20:1 Na to navázal Sófar Naamatský slovy:
#20:2 „Jak tak o tom přemítám, musím odporovat, proto abych si pospíšil.
#20:3 Důtku, jež mě uráží, vyslechnout musím, ale můj rozumný duch mi velí odpovědět.
#20:4 Což nevíš, že je to tak odjakživa, od chvíle, co byl člověk postaven na zemi,
#20:5 že plesání svévolníků trvá krátce a radost rouhačů jen okamžik?
#20:6 I když ve své povznesenosti se vypíná až k nebi a jeho hlava se dotýká oblaků,
#20:7 navždy zanikne tak jako jeho výkal, a kdo jej vídali, řeknou: ‚Kde je?‘
#20:8 Odlétne jak sen a nenajdou ho, bude zaplašen jako vidění noční.
#20:9 Oko, jež ho spatřovalo, už ho nezahlédne, nikdo ho na jeho místě neuvidí.
#20:10 Jeho synové se budou chtít zalíbit nuzným, vlastníma rukama vrátí, co svou silou získal.
#20:11 I když jeho kosti jsou plny mladistvé svěžesti, i ta s ním ulehne do prachu.
#20:12 Ať si mu zloba připadá v ústech sladká a přechovává ji pod jazykem,
#20:13 kochá se v ní, nevzdá se jí, zadržuje ji na svém patře,
#20:14 jeho pokrm se mu v útrobách změní, stane se mu ve vnitřnostech zmijím jedem.
#20:15 Nahltal se majetku, ale vyzvrátí jej, Bůh mu jej vyžene z břicha.
#20:16 Bude sát zmijí jed, hadí jazyk ho zabije.
#20:17 Nebude se dívat na tekoucí vody, na řeky, potoky medu a mléka.
#20:18 Výtěžek vrátí, nezhltne jej, nad ziskem z obchodu nebude jásat.
#20:19 Odstrkoval a opouštěl nuzné, uchvátil dům, ač jej nestavěl.
#20:20 V nitru však nepoznal klidu, to, po čem dychtil, ho nezachrání.
#20:21 Před jeho žravostí nic neobstálo, proto jeho blahobyt nebude trvalý.
#20:22 Při nejhojnějším nadbytku mu bude úzko, dopadnou na něho ruce všech bědných.
#20:23 Jen ať si plní své břicho, Bůh na něho sešle svůj planoucí hněv, jeho útroby zasáhne sprškou šípů.
#20:24 Bude prchat před železnou zbrojí a postřelí ho bronzový luk.
#20:25 Střela pronikne mu zády, blesk jeho žlučí; zajde obklopen přízraky.
#20:26 Všechna temnota je pro něj uchována, pohltí ho oheň, který nerozdmýchal člověk, zle dopadne v jeho stanu i ten, kdo vyvázl.
#20:27 Nebesa zjeví jeho nepravost a země povstane proti němu.
#20:28 Víno z jeho domu se nepřestěhuje jinam, rozplyne se v den Božího hněvu.
#20:29 To je úděl svévolného člověka, určený Bohem, Bohem mu přiřčené dědictví. 
#21:1 Jób na to odpověděl:
#21:2 „Poslouchejte dobře mou řeč, potěšíte mě tím.
#21:3 Strpte, abych též promluvil; až domluvím, posmíveje se.
#21:4 Týká se mé lkání jen člověka? Což nemám důvod být netrpělivý?
#21:5 Obraťte se ke mně, užasnete; dejte si na ústa ruku:
#21:6 Když se rozpomínám, jsem naplněn hrůzou, mého těla se zmocňuje zděšení.
#21:7 Proč naživu zůstávají svévolníci? Dožijí se vysokého věku, rozmohou se, kupí statky,
#21:8 své potomstvo mají pevně kolem sebe, své potomky mají před očima,
#21:9 v jejich domech je pokoj beze strachu, Boží hůl na ně nedopadá.
#21:10 Jejich býk nebývá připouštěn nadarmo, jejich kráva se otelí a nezmetá.
#21:11 Vypouštějí jako stádo své nezvedence a jejich děti skotačí.
#21:12 Hulákají při bubínku a citaře a radují se za zvuku flétny.
#21:13 Tráví své dny v pohodě a do podsvětí sestupují v mžiku.
#21:14 Bohu říkají: ‚Jdi pryč od nás, nechcem o tvých cestách vědět.
#21:15 Kdo je Všemocný, že mu máme sloužit, co nám prospěje obracet se na něj?‘
#21:16 Avšak svůj blahobyt nemají v svých rukou. Nechť jsou mi vzdáleny záměry svévolníků.
#21:17 Jak náhle hasne kahan svévolníků a přicházejí na ně bědy! Bůh jim dá za úděl útrapy v svém hněvu,
#21:18 jsou jak sláma ve větru a jako plevy uchvácené vichrem.
#21:19 Bůh prý uchovává jejich ničemnosti jejich synům. Kéž splatí přímo jemu, aby to poznal!
#21:20 Kéž na vlastní oči spatří svou bídu, vypije si kalich rozhořčení Všemocného.
#21:21 Zajímá ho, co se stane s jeho domem po něm, když je počet jeho měsíců odměřen?
#21:22 Bude někdo učit Boha vědomostem? On soudí přece i vysoko postavené.
#21:23 Jeden zemře v plném květu, zcela bezstarostný, klidný,
#21:24 jeho dížka je plná mléka a jeho kosti jsou prosáklé morkem.
#21:25 Druhý zemře s hořkou duší, aniž okusil co dobrého.
#21:26 Společně ulehnou do prachu a pokryjí je červi.
#21:27 Hle, já vím, co zamýšlíte, vaším záměrem je způsobit mi příkoří.
#21:28 Říkáte zajisté: ‚Kde je dům urozeného? Kde je stan, v němž přebývali svévolníci?‘
#21:29 Což jste se neptali mimojdoucích, neporozuměli jste jejich znamením?
#21:30 V den běd bude zlý ušetřen, bude odnesen v den hrozné prchlivosti.
#21:31 Kdo mu vytkne jeho cestu, kdo mu odplatí za to, co napáchal?
#21:32 Až bude odnesen do hrobky, nad jeho rovem se bude bdít.
#21:33 Hroudy v jámě budou mu lehké; potáhnou se za ním všichni lidé, bude jich před ním bezpočet.
#21:34 Jak mě chcete těšit takovými přeludy? Vaše odpovědi - samá věrolomnost.“ 
#22:1 Na to navázal Elífaz Témanský slovy:
#22:2 „Znamená před Bohem něco muž? Věru, prozíravý dbá na svoje činy.
#22:3 Což se Všemocný zajímá o tvou spravedlnost, jsou mu tvoje bezúhonné cesty ziskem?
#22:4 Trestá tě snad za to, že se ho bojíš, za to s tebou vchází v soud?
#22:5 Tvá zloba je velká, tvá nepravost nekonečná.
#22:6 Bezdůvodně jsi bral zástavu od bratří, svlékal jsi z nich šat a nechával je nahé,
#22:7 znaveného neosvěžils vodou, hladovému odepřel jsi chleba.
#22:8 Země patří tomu, kdo má pevnou paži, bude na ní sídlit ten, koho Bůh milostivě přijal.
#22:9 Tys však s prázdnou posílal pryč vdovy, paže sirotků jsi drtil.
#22:10 Proto jsou kolem tebe osidla, náhlý strach tě plní hrůzou,
#22:11 pro tmu nevidíš a spousty vod tě přikrývají.
#22:12 Což není Bůh vysoko nad nebesy? Pohleď vzhůru na hvězdy, jak jsou vyvýšeny.
#22:13 A ty říkáš: ‚Copak Bůh ví, skrze temné mrákoty co může soudit?
#22:14 Oblaka ho zahalují, takže nevidí; prochází se kdesi po obvodu nebes.‘
#22:15 Chceš se držet stezky dávnověku, po níž chodívají mužové propadlí ničemnostem,
#22:16 kteří byli zasaženi, ačkoli čas nenadešel, proudem zatopeni do základů?
#22:17 Říkávali Bohu: ‚Jdi pryč od nás.‘ Co jiného jim měl Všemocný udělat?
#22:18 Jejich domy naplňoval blahobytem. Nechť jsou mi vzdáleny záměry svévolníků!
#22:19 Spravedliví to uvidí a zaradují se, nevinný se jim vysměje:
#22:20 ‚Zničeni jsou ti, kdo na nás útočili, a co po nich zbylo, pozřel oheň.‘
#22:21 Měj se k Bohu důvěrněji, ať užiješ pokoje. Vzejde ti z toho jen užitek.
#22:22 Přijímej z jeho úst naučení, vkládej si do srdce jeho slova.
#22:23 Vrátíš-li se k Všemocnému, budeš vybudován, vzdálíš-li od svého stanu podlost.
#22:24 Odlož do prachu svá zlatá zrnka, mezi potoční skaliska ofirské zlato.
#22:25 Pak bude Všemocný sám tvým zlatem, bude ti hromadou stříbra.
#22:26 Ve Všemocném najdeš svoje blaho, budeš pozvedat svoji tvář k Bohu.
#22:27 Budeš-li ho prosit, vyslyší tě, a ty budeš plnit svoje sliby.
#22:28 Rozhodneš-li se pro něco, stane se tak, a na tvých cestách bude zářit světlo.
#22:29 Až budou jiní poníženi, řekneš: ‚Ó pýcho!‘ Bůh zachrání ty, kdo klopí oči.
#22:30 Zachrání toho, kdo není bez viny; bude zachráněn pro čistotu tvých rukou.“ 
#23:1 Jób na to odpověděl:
#23:2 „Také dnes zní moje lkání vzpurně, vzdychám pod rukou, která mě tíží.
#23:3 Kéž bych věděl, kde ho najdu. Vydal bych se k jeho sídlu,
#23:4 předložil bych mu svou při a plno důkazů by podala má ústa.
#23:5 Chtěl bych vědět, jakými slovy by odpověděl, porozumět tomu, co mi řekne.
#23:6 Ukáže svou velkou moc, až povede spor se mnou? Nikoli, jistě by mi dopřál sluchu.
#23:7 Jako přímý bych se před ním obhajoval, navždy unikl bych svému soudci.
#23:8 Půjdu-li vpřed, není nikde, jestliže zpět, též ho nepostřehnu,
#23:9 jestliže něco učiní vlevo, neuzřím ho, skryje-li se vpravo, neuvídím to.
#23:10 Zato on zná moji cestu; ať mě zkouší, vyjdu jako zlato.
#23:11 Má noha se přidržela jeho kroků, držel jsem se jeho cesty, neodhýlil jsem se,
#23:12 od příkazů jeho rtů jsem neodstoupil, řeči jeho úst jsem střežil víc než vlastní cíle.
#23:13 Rozhodne-li se k čemu, kdo to zvrátí? Udělá, co se mu zachce.
#23:14 Jistě splní, co mi určil, má k tomu dost moci.
#23:15 Proto se ho hrozím, chci mu porozumět, ale mám z něho strach.
#23:16 Bůh naplnil úzkostí mé srdce, Všemocný mě naplnil hrůzou.
#23:17 Jen proto jsem v temnotách provždy neumlkl, že se přede mnou zahalil mračnem. 
#24:1 Když nejsou před Všemocným skryty časy, proč ti, kdo ho znají, jeho dny nepostřehnou?
#24:2 Mocní posouvají mezníky, pasou uchvácené stádo,
#24:3 odvádějí osla sirotkům a vdově berou býka do zástavy,
#24:4 ubožáky odstrkují z cesty; musejí se skrývat všichni utištění v zemi.
#24:5 Ti jsou jak divocí oslové v poušti: vycházejí za svou prací, za úsvitu hledají si pokrm, v pustině hledají chléb pro omladinu.
#24:6 Sklízejí z pole, jež není jejich, oberou vinici svévolníka.
#24:7 Nocují nazí, protože nemají oděv ani přikrývku v chladu;
#24:8 prudkými horskými dešti promočeni nemají útočiště, ke skále se tisknou.
#24:9 Tamti uchvacují sirotka od prsu, vymáhají zástavu od utištěného.
#24:10 Tito chodí nazí, bez oděvu, hladoví snášejí cizím snopy;
#24:11 mezi jejich zídkami lisují olej, šlapou ve vinných lisech a mají žízeň,
#24:12 ston umírajících zaznívá z města, prokláni volají o pomoc. Avšak tyto nepatřičnosti Bůh nepůsobí.
#24:13 Tamti se bouří proti světlu, neznají se k Božím cestám, nesetrvávají na jeho stezkách.
#24:14 Za světla povstává vrah a vraždí utištěného ubožáka, i v noci se plíží jako zloděj.
#24:15 Též oko cizoložníka se drží v přítmí, říká: ‚Nikdo mě nespatří‘, tvář si zahaluje rouškou.
#24:16 Za tmy se do domů vloupávají, za dne se zamykají, o světle nechtějí vědět.
#24:17 Jim všem je jitro šerem smrti, s hrůzami šeré smrti se znají.
#24:18 Takového rychle vezme voda, jeho úděl na zemi je zlořečený, neobrací se na cestu do vinic.
#24:19 Suchopár a žár pohlcuje sněhové vody a podsvětí ty, kdo hřeší.
#24:20 Ať na něho zapomene i matčino lůno, ať si na něm pochutnají červi; nebude ho nikdy vzpomenuto. Podlost bude roztříštěna jako dřevo.
#24:21 Odírá neplodnou, která nerodí, a vdově neprokáže dobrodiní.
#24:22 I vznešené zachovává Bůh svou mocí; povstane-li, nikdo si není životem jist.
#24:23 Dopřává člověku bezpečí a on se má oč opřít, ale oči má upřené na jejich cesty.
#24:24 Nakrátko jsou povýšeni a už zanikají, sehnuti jsou jako ti, co hynou, vadnou jako vršky klasů.
#24:25 Není tomu tak? Kdo obviní mě ze lži? Kdo moji řeč za nic nemá?“ 
#25:1 Na to navázal Bildad Šúchský slovy:
#25:2 „Vladařství a strach jsou v ruce toho, jenž na svých výšinách působí pokoj.
#25:3 Je možno jeho houfy sečíst? Nad kým nevzchází jeho světlo?
#25:4 Což může člověk být před Bohem spravedlivý a čistý ten, kdo se zrodil z ženy?
#25:5 Jestliže ani měsíc nesvítí jasně, nejsou-li ani hvězdy čisté před jeho zrakem,
#25:6 což teprve lidský červ, lidský syn, červíček pouhý!“ 
#26:1 Jób na to odpověděl:
#26:2 „Jak pomůžeš tomu, komu chybí síla? Jak zachráníš vysílené rámě?
#26:3 Jak poradíš tam, kde chybí moudrost? Jak seznámíš s tím, co skýtá hojnou pomoc?
#26:4 Komu povídáš ta slova? Čí to dech vychází z tebe?
#26:5 Stíny zemřelých se úzkostně chvějí, i vody dole a co v nich přebývá.
#26:6 Podsvětí je před ním obnaženo, říše zkázy nepřikryta.
#26:7 On roztáhl sever nad pustotou, nad nicotou zavěsil zemi,
#26:8 vody zabaluje do oblaků, mračno pod nimi se neprotrhne;
#26:9 svůj trůn zahaluje, rozprostírá nad ním oblak.
#26:10 Na vodní hladině vykroužil obzor, kde končí světlo i tma.
#26:11 Sloupy nebes se chvějí a trnou, když okřikne vody;
#26:12 svou mocí vzdouvá moře, svou rozumností Netvora zdolal.
#26:13 Jeho duch dal nebesům velkolepost, jeho ruka proklála útočného hada.
#26:14 Hle, to je jen část jeho cest; zaslechli jsme o něm pouhý šelest, kdo může porozumět hřímání jeho bohatýrské síly?“ 
#27:1 Jób pak pokračoval v pronášení svých průpovědí takto:
#27:2 „Jakože živ je Bůh, on upírá mi právo, Všemocný naplnil hořkostí mou duši.
#27:3 Ale dokud budu dýchat, dokud Boží dech bude v mých chřípích,
#27:4 mé rty nevysloví podlost a můj jazyk nebude hovořit lstivě.
#27:5 Jsem dalek toho prohlásit vás za spravedlivé, dokud nezhynu, své bezúhonnosti se nevzdám,
#27:6 setrvám ve spravedlnosti a neochabnu, srdce nebude mě hanět za žádný můj den.
#27:7 Můj nepřítel však ať je na tom jako svévolník, a ten, kdo proti mně povstává, jako bídák.
#27:8 Jakou naději na zisk má rouhač, vezme-li mu Bůh život?
#27:9 Bude Bůh poslouchat jeho křik, až bude v tísni?
#27:10 Najde blaho ve Všemocném, bude přivolávat Boha v každém čase?
#27:11 Chci vás poučit o Boží moci, netajit, jak tomu je s Všemocným.
#27:12 Hle, sami jste to všichni uzřeli, proč se tedy oddáváte přeludům?
#27:13 Toto je před Bohem úděl člověka svévolného, dědictví ukrutníků, jež dostanou od Všemocného:
#27:14 Rozmnoží-li se jejich synové, přijde na ně meč, jejich potomci se nenasytí chlebem;
#27:15 ty co vyváznou, ty pohřbí smrt, jejich vdovy nebudou je oplakávat.
#27:16 Kdyby někdo nakupil stříbra jak prachu a navršil oděvů jak hlíny,
#27:17 co navrší, to oblékne spravedlivý a stříbro připadne nevinnému.
#27:18 Svůj dům postavil jako mol, jako chatrč, kterou si udělal hlídač.
#27:19 Boháč ulehne a už se nesebere, než rozevře oči, nebude tu.
#27:20 Hrůzy ho dostihnou jako příval vod, vichřice ho zachvátí v noci,
#27:21 odnese ho východní vítr a bude pryč, vichr jej odvane z jeho místa.
#27:22 Tím ho Bůh postihne nelítostně, před jeho rukou bude marně prchat.
#27:23 Budou nad ním tleskat rukama a ušklíbnou se nad místem, kde býval.“ 
#28:1 Stříbro má své naleziště a zlato místo, kde se čistí,
#28:2 železo se získává z prachu, z rudy se taví měď.
#28:3 Člověk překonává tmu, prozkoumává v říši šeré smrti temný kámen do každého koutku.
#28:4 Proráží šachtu daleko od místa, kde přebývá. Zapomenuti, bez půdy pod nohama, na laně se houpou a kývají, vzdáleni lidem.
#28:5 Země, z níž vzchází chléb, je vespod zpřevracena jakoby ohněm;
#28:6 v jejím kamení je ložisko safírů, jsou v něm i zlatá zrnka.
#28:7 Dravý pták tam nezná stezku, oko luňáka ji nezahlédne,
#28:8 mláďě šelmy po ní nešlapalo, lev po ní nevleče kořist.
#28:9 Člověk vztáhl ruku po křemeni, hory zpřevracel až do základů,
#28:10 do skal vytesal štoly, jeho oko spatřilo kdejaký skvost,
#28:11 zamezil prosakování proudících vod, a co se tají v zemi, vynáší na světlo.
#28:12 Ale moudrost, kde se najde? Kde je místo rozumnosti?
#28:13 Člověk nezná její cenu, v zemi živých se nenajde.
#28:14 Propastná tůň praví: ‚Ve mně není‘,moře říká: ‚Já ji nemám‘.
#28:15 Nelze ji získat za lístkové zlato, její hodnota se nevyváží stříbrem,
#28:16 nemůže být zaplacena ofírským zlatem, vzácným karneolem či safírem.
#28:17 Nedá se srovnat se zlatem či se sklem ani směnit za věci z ryzího zlata,
#28:18 natož za korál a křišťál; moudrost má větší cenu než perly.
#28:19 Nedá se srovnat s kúšským topasem, nedá se zaplatit nejčistším zlatem.
#28:20 Odkud tedy přichází moudrost? Kde je místo rozumnosti?
#28:21 Je utajena před očima všeho živého, zahalena i před nebeským ptactvem.
#28:22 Říše zkázy a smrt říkají: ‚Pouze jsme zaslechly pověst o ní.‘
#28:23 Jenom Bůh rozumí její cestě, on zná také její místo,
#28:24 neboť on dohlédne až do končin země, vidí vše, co je pod nebem.
#28:25 Když větru udělil prudkost a vody odměrkou změřil,
#28:26 když dešti stanovil cíl a cestu bouřnému mračnu,
#28:27 hned tehdy ji viděl a vyprávěl o ní, učinil ji nepohnutelnou a také ji prozkoumal
#28:28 a řekl člověku: ‚Hle, bát se Panovníka, to je moudrost, vystříhat se zlého, toť rozumnost.‘ 
#29:1 Jób pak pokračoval v pronášení svých průpovědí takto:
#29:2 „Kéž by mi bylo jako za předešlých měsíců, jako za dnů, kdy mě Bůh střežil,
#29:3 kdy jeho kahan mi nad hlavou zářil a já jsem temnotou šel v jeho světle,
#29:4 jako se mi vedlo za dnů mé svěžesti, kdy můj stan byl místem důvěrného rozhovoru s Bohem,
#29:5 kdy ještě Všemocný byl při mně a kolem mne moje čeleď,
#29:6 kdy se mé nohy koupaly ve smetaně a skála mi vylévala potoky oleje.
#29:7 Když jsem procházel branou vzhůru k městu, abych na náměstí zaujal své místo,
#29:8 mladíci, jak mě viděli, ztichli, kmeti povstávali a zůstali stát,
#29:9 velmoži se vystříhali řečí, kladli si na ústa ruku,
#29:10 hlas vévodů tichl, jazyk jim přilnul k patru.
#29:11 Čí ucho o mně slyšelo, ten mi blahořečil, a oko, které mě vidělo, svědčilo pro mě,
#29:12 že jsem utištěného zachránil, když volal o pomoc, i sirotka, který neměl, kdo by mu pomohl.
#29:13 Žehnání hynoucího se snášelo na mne a srdce vdovy jsem pohnul k plesání.
#29:14 Oblékal jsem spravedlnost, to byl můj oděv; jak říza a turban bylo mi právo.
#29:15 Slepému jsem byl okem a kulhavému nohou,
#29:16 ubožákům jsem byl otcem, spor neznámých jsem rozsuzoval,
#29:17 bídákovi jsem však zvyrážel tesáky, ze zubů mu vyrval kořist.
#29:18 Říkal jsem: ‚Zahynu se svým hnízdem a rozmnožím své dny jako Fénix.
#29:19 Můj kořen se rozloží při vodách, na mém větvoví bude nocovat rosa.
#29:20 Moje sláva se mi bude obnovovat a můj luk v mé ruce bude stále pružný.‘
#29:21 Poslouchali mě a na mě čekávali, umlkali při mé radě;
#29:22 po mém slovu už nic neměnili, má řeč na mě kanula jak rosa.
#29:23 Čekávali na mě jako na déšť, otvírali ústa jak po jarním dešti.
#29:24 Usmíval jsem se na ně, když ztráceli víru, a oni neodmítali světlo mé tváře.
#29:25 Když jsem volil cestu k nim, seděl jsem v čele, bydlel jsem jako král mezi houfy, jako ten, kdo těší truchlivé. 
#30:1 Avšak nyní se mi vysmívají ti, kteří jsou mladší než já, jejich otců jsem si vážil tak málo, že jsem je nestavěl ani ke svým ovčáckým psům.
#30:2 K čemu je mi síla jejich rukou, když zhynula jejich svěžest?
#30:3 Nouzí a hladověním vyčerpáni ohlodávají suchopár od včerejška pustý, zpustošený,
#30:4 trhají lebedu mezi křovím a kořen kručinky jim je chlebem.
#30:5 Byli vypuzeni z obce, pokřikovali za nimi jako za zlodějem;
#30:6 musejí bydlet ve srázných úvalech, v podzemních slujích a pod útesy,
#30:7 hýkají v křoví, zalezlí pod kopřivami,
#30:8 bloudové, bezejmenní, ze země vymrskaní.
#30:9 A teď jsem jim předmětem popěvků a řečí;
#30:10 hnusí si mě a vzdalují se ode mne, nestydí se mi do tváře plivat.
#30:11 Bůh rozvázal mé stanové lano, pokořil mě, a oni všechny zábrany odhodili.
#30:12 Vyvstávají z pravé strany jako roj, podrážejí mi nohy, navršili proti mně cesty svých běd,
#30:13 rozkopali moji stezku, usilují o můj pád a nebrání jim nikdo.
#30:14 Jakoby širokou průrvou přicházejí, valí se uprostřed trosek.
#30:15 Proti mně se obrátily hrůzy, jako vítr pronásledují mě za šlechetnost, jako oblak odplynula moje spása.
#30:16 A nyní je vylita má duše ve mně, chopily se mě dny pokoření.
#30:17 V noci mě bodá v kostech, hlodá mě bolest neutuchající.
#30:18 Pro úpornou bolest se změnil můj oděv, sevřela mě jako pás suknici;
#30:19 jsem smeten do bláta, podoben prachu a popelu.
#30:20 Volám k tobě o pomoc, a ty mi neodpovídáš, stojím tu, měj pro mě pochopení.
#30:21 Změnil ses mi v krutého protivníka, strojíš mi úklady svou mocnou rukou.
#30:22 Unášíš mě větrem jako na voze, až mě opouštějí smysly.
#30:23 Vím, že mě předáš smrti, přivedeš do domu, kde se setká všechno živé.
#30:24 Nikdo nepodá ruku do sutin, kdyby z nich hynoucí o pomoc volal.
#30:25 Neplakal jsem snad za krušných dnů, nestaral jsem se o ubožáka?
#30:26 Dobro jsem s nadějí očekával, a přišlo zlo, čekal jsem na světlo, a přišlo mračno.
#30:27 Mé útroby neklidně vřou, mám před sebou dny utrpení.
#30:28 Chodím zarmoucen, ačkoli nepálí slunko, povstávám ve shromáždění a volám o pomoc.
#30:29 Stal jsem se bratrem šakalů, druhem pštrosů.
#30:30 Kůže na mně zčernala, kosti mám horkostí rozpálené.
#30:31 Má citara zní při truchlení, moje flétna při hlasu plačících.“ 
#31:1 „Uzavřel jsem smlouvu se svýma očima, jak bych se tedy směl ohlížet za pannou?
#31:2 Jaký je úděl od Boha shůry, dědictví od Všemocného z výšin?
#31:3 Zdali ne bědy pro bídáka a neštěstí pro pachatele ničemností?
#31:4 Nevidí snad Bůh mé cesty a nepočítá všechny mé kroky?
#31:5 Jestliže jsem licoměrně jednal a moje noha spěchala za lstí,
#31:6 ať mě Bůh zváží na vážkách spravedlnosti a pozná mou bezúhonnost.
#31:7 Jestliže se můj krok odchýlil z cesty a mé srdce si šlo za mýma očima nebo poskvrna ulpěla na mých dlaních,
#31:8 ať jiný sní, co zaseju, a moji potomci ať jsou vykořeněni.
#31:9 Jestliže se mé srdce dalo zlákat cizí ženou a já jsem strojil nástrahy u dveří svého druha,
#31:10 ať moje žena jinému mele a jiní ať se sklánějí nad ní.
#31:11 Byla by to přece mrzkost, nepravost hodná odsouzení,
#31:12 byl by to oheň, jenž zžírá v říši zkázy, který do kořenů zničí všechnu mou úrodu.
#31:13 Jestliže jsem porušil právo svého otroka či otrokyně, když se mnou vedli spor,
#31:14 co bych měl udělat, až Bůh povstane, co mu odvětím, až mě bude stíhat?
#31:15 Což ten, který mě učinil v mateřském životě, neučinil i jeho, nebyl to týž, který nás připravil v lůně?
#31:16 Jestliže jsem odmítl nuzákovo přání, oči vdovy nechal hasnout v beznaději
#31:17 a sám pojídal své sousto, aniž z něho též sirotek jedl -
#31:18 ten přec od mládí už se mnou rostl jak u otce, také vdovu jsem od života své matky vodil -,
#31:19 jestliže jsem viděl hynoucího bez oděvu, že se ubožák čím přikrýt nemá,
#31:20 jestliže mi jeho bedra nežehnala za to, že se ohřívá vlnou mých jehňat,
#31:21 jestliže má ruka hrozila sirotkovi, ač jsem v bráně viděl, že mu mohu pomoci,
#31:22 ať mi odpadne lopatka od ramene a paže se mi vylomí z kloubu.
#31:23 Propadl bych jistě strachu, bědám Božím, na nic bych se nevzmohl před jeho vznešeností.
#31:24 Jestliže jsem svou důvěru složil v zlato, třpytivému zlatu říkal: ‚V tebe doufám‘,
#31:25 jestliže jsem se radoval, že jsem tak zámožný, že má ruka nabyla úctyhodného jmění,
#31:26 jestliže jsem viděl světlo sluneční, jak září, a měsíc, jak skvostně pluje po obloze,
#31:27 a mé srdce potají se dalo zlákat, abych polibek svých úst jim rukou poslal,
#31:28 i to by byla nepravost hodná odsouzení, neboť bych tím zapřel Boha shůry.
#31:29 Jestliže jsem se zaradoval ze zániku toho, kdo mě nenáviděl, a byl rozjařen, že potkalo ho něco zlého -
#31:30 vždyť jsem nedal svým ústům zhřešit, nevyprošoval jsem si pro něho kletbu,
#31:31 lidé z mého stanu si nestěžovali: ‚Kéž by nám dal trochu masa, nejsme syti‘,
#31:32 cizinec nezůstával přes noc venku, pocestnému jsem otvíral dveře -,
#31:33 jestliže jsem po lidsku své přestoupení skrýval, nepravost svou tajně uložil v své hrudi,
#31:34 že bych se byl lekal početného davu a že bych se opovržení lidských čeledí děsil, pak ať zmlknu a nevyjdu ze dveří.
#31:35 Kéž bych měl někoho, kdo by mě vyslyšel! Zde je mé znamení. Ať mi Všemocný odpoví. Můj odpůrce sepsal zápis.
#31:36 Klidně si jej vložím na rameno, ovinu si jej jak věnec.
#31:37 O každém svém kroku mu povím, jako vévoda přistoupím k němu.
#31:38 Jestliže má půda proti mně křičela a její brázdy přitom plakaly,
#31:39 jestliže jsem její sílu spotřeboval bez placení a její vlastníky odbýval,
#31:40 ať místo pšenice vzejde trní, místo ječmene býlí.“ Zde končí Jóbova slova. 
#32:1 Tu přestali oni tři muži Jóbovi odpovídat, protože byl ve vlastních očích spravedlivý.
#32:2 Zato vzplanul hněvem Elíhú, syn Berakeela Búzského z čeledi Rámovy; jeho hněv vzplanul proti Jóbovi, protože se pokládal za spravedlivějšího než Bůh.
#32:3 Jeho hněv vzplanul i proti jeho třem nepřátelům, protože nenašli správnou odpověď a Jóba prohlašovali za svévolníka.
#32:4 Elíhú se slovy k Jóbovi vyčkával, protože ostatní byli věkem starší než on.
#32:5 Elíhú však viděl, že v ústech oněch tří mužů není vhodná odpověď, proto vzplanul hněvem.
#32:6 Na to tedy navázal Elíhů, syn Berakeela Búzského, slovy: „Co se týče let, jsem mladší, kdežto vy jste kmeti, proto jsem se ostýchal a obával vám sdělit, co vím.
#32:7 Řekl jsem si: Ať promluví léta, ti, kteří mají let mnoho, ať s moudrostí seznamují.
#32:8 Avšak je to duch člověku daný, dech Všemocného, jenž lidi činí rozumnými.
#32:9 Nejsou vždycky moudří ti, kdo mají mnoho let, starci nemusejí vždy rozumět právu.
#32:10 Proto říkám: Poslyšte mě. Rovněž já chci sdělit, co vím.
#32:11 Hle, čekal jsem na vaše slova, naslouchal jsem vaší rozumnosti, až co svými slovy vystihnete.
#32:12 Snažil jsem se vám porozumět, avšak Jóba nikdo neusvědčil, žádný z vás neodpověděl na jeho řeči.
#32:13 Neříkejte: ‚My jsme našli moudrost; Bůh ho odvane jak plevu, a ne člověk.‘
#32:14 Jób svá slova nezaměřil proti mně, nebudu mu odpovídat vašimi řečmi. -
#32:15 Jsou zděšeni, už neodpovídají, slova jim už došla.
#32:16 Čekám, ale oni nepromluví, stojí tu a neodpovídají.
#32:17 Přispěji tedy já svým dílem, rovněž já chci sdělit, co vím.
#32:18 Jsem naplněn slovy, těsno je duchu v mém nitru.
#32:19 Hle, mé nitro je jako víno, které nemá průduch, jako nové měchy, jež pukají.
#32:20 Musím mluvit, aby se mi ulevilo; otevřu rty a odpovím.
#32:21 Nebudu nikomu stranit, nechci nikomu lichotit;
#32:22 nevím, co je lichocení, to by mě můj Učinitel brzy smetl. 
#33:1 Ty, Jóbe, slyš dobře mou řeč, všem mým slovům dopřej sluchu.
#33:2 Pohleď, otevírám ústa, pod mým patrem promlouvá můj jazyk.
#33:3 Má řeč tryská z upřímného srdce, mé rty mluví o poznání vytříbeném.
#33:4 Boží duch mě učinil, dech Všemocného mě oživil.
#33:5 Můžeš-li, odpovídej, předlož mi svou při, postav se!
#33:6 Hle, já budu tvými ústy před Bohem, i já jsem jen kus hlíny.
#33:7 Hle, nemusí tě přepadat strach ze mne, ani moje naléhání tebe tížit.
#33:8 Řekl jsi to přede mnou a já jsem tvá slova slyšel:
#33:9 ‚Ryzí jsem, prost přestoupení, jsem bez úhony, bez nepravosti.
#33:10 Hle, Bůh proti mně záminky hledá, pokládá mě za nepřítele,
#33:11 sevřel do klády mé nohy a střeží všechny mé stezky.‘
#33:12 Hle, odpovídám ti: Zde nejsi v právu, neboť Bůh je větší než člověk.
#33:13 Proč s ním vedeš spor? Že svými slovy neodpovídá?
#33:14 Bůh přece promluví jednou i podruhé, a člověk to nepostřehne.
#33:15 Ve snu, ve vidění nočním, když na lidi padá mrákota, v dřímotě na lůžku,
#33:16 tehdy otvírá lidem ucho a zpečeťuje varování, jež jim dal,
#33:17 aby člověka odvedl od toho, co páchá, aby muže chránil před vypínavostí,
#33:18 aby jeho duši ušetřil před jámou a jeho život aby nezašel hubící střelou.
#33:19 Bolestmi je kárán na svém loži, stálým svárem v kostech,
#33:20 takže jeho život si oškliví pokrm a jeho duše vytoužené jídlo.
#33:21 Jeho tělo se očividně ztrácí, vystupují kosti, které nebývalo vidět.
#33:22 Jeho duše se blíží jámě a jeho život jisté smrti.
#33:23 Bude-li s ním anděl, tlumočník, jeden z tisíců, a poví-li, že je to přímý člověk,
#33:24 tehdy Bůh se nad ním smiluje a řekne: ‚Vykup ho, ať do jámy nesestoupí; mám za něho výkupné.‘
#33:25 Jeho tělo bude svěžejší než v mládí, vrátí se do dnů dospívání.
#33:26 Bude-li prosit Boha a on v něm najde zalíbení, ukáže mu svou tvář za hlaholu polnic a člověku vrátí jeho spravedlnost.
#33:27 Bude zpívat před lidmi a řekne: ‚Hřešil jsem a převracel, co bylo přímé, ale nebylo mi odplaceno stejným.‘
#33:28 Bůh vykoupil jeho duši, aby neodešla do jámy, jeho život smí hledět na světlo.
#33:29 To vše učiní Bůh pro muže dvakrát i třikrát,
#33:30 aby odvrátil jeho duši od jámy a přivedl ke světlu, do světla živých.
#33:31 Poslouchej mě, Jóbe, pozorně, mlč, já budu mluvit.
#33:32 Bude-li co říci, odpověz mi, promluv, chci pro tebe spravedlnost.
#33:33 Nebudeš-li mít co říci, poslouchej mě, mlč a já ti zprostředkuji moudrost.“ 
#34:1 Na to navázal Elíhú slovy:
#34:2 „Vy moudří, slyšte mé řeči, vy vědoucí, naslouchejte mi.
#34:3 Ucho přece přezkušuje řeči, patro vychutnává pokrm.
#34:4 Volme si, co je právo, ať zvíme mezi sebou, co je dobré.
#34:5 Jób řekl: ‚Jsem spravedlivý, a Bůh upírá mi právo.
#34:6 Mám lhát? Proti svému právu mluvit? Zasáhl mě zkázonosný šíp, avšak ne pro přestoupení.‘
#34:7 Který muž je jako Jób? Prý pije výsměch jako vodu,
#34:8 spolčuje se s pachateli ničemností, obcuje se svévolnými lidmi;
#34:9 řekl: ‚Muži není k ničemu mít zalíbení v Bohu.‘
#34:10 Proto mě slyšte, rozumní mužové: Daleko je Bůh od svévole, Všemocný od bezpráví.
#34:11 Podle toho, co člověk udělá, odplatí jemu, s každým nakládá podle jeho stezky.
#34:12 Opravdu, Bůh nejedná svévolně, Všemocný právo nepokřiví.
#34:13 Kdo mu dal pod dohled zemi a kdo před něho položil celý svět?
#34:14 Kdyby měl na mysli jenom sebe a svého ducha i dech k sobě zpět stáhl,
#34:15 tu by všechno tvorstvo rázem vyhynulo, člověk by se obrátil v prach.
#34:16 Máš-li tedy rozum, poslyš toto, dopřej sluchu tomu, co říkám.
#34:17 Což může panovat ten, kdo nenávidí právo? Nebo Úctyhodného, jenž je spravedlivý, prohlásíš za svévolníka?
#34:18 Je snad možno říci králi: ‚Ty ničemo‘ nebo urozeným: ‚Svévolníku‘?
#34:19 Bůh nestraní velmožům, nedá přednost váženému před nuzákem, protože všichni jsou dílem jeho rukou.
#34:20 Umírají v okamžení, třeba o půlnoci. Lid se zachvěje, když odcházejí; i vznešený je odvolán, aniž kdo hne rukou.
#34:21 Na cesty každého jsou upřeny jeho oči, on každý krok vidí.
#34:22 Není temnoty ani šera smrti, kde by se mohli skrýt pachatelé ničemností.
#34:23 Na nikoho nevkládá nic navíc, že by se musel s Bohem soudit.
#34:24 Ctihodné bije, nemusí nic vyšetřovat, jiné staví na ta místa;
#34:25 je obeznámen s jejich dílem, podvrátí je za noc a jsou rozdrceni.
#34:26 Potrestá je jako svévolníky, každý na to bude hledět,
#34:27 za to, že se odchýlili pryč od něho a neměli uznání pro žádnou jeho cestu,
#34:28 takže se až k němu donesl křik nuzných a slyšel křik utištěných.
#34:29 Když zůstane klidný, kdo ho smí prohlásit za svévolníka? Když skrývá tvář, kdo ho může spatřit? A přece bdí nad pronárody i jednotlivci,
#34:30 aby nekraloval člověk - rouhač a nestal se svůdcem lidu.
#34:31 Sluší se doznat Bohu: ‚Snáším svůj trest, nebudu už pohoršlivě jednat.
#34:32 Poučuj mě o tom, na co nestačí můj pohled. Spáchal-li jsem podlost, už tak neučiním.‘
#34:33 Má snad odplácet dle tvého, když ty jeho názor za nic nemáš? Ty bys to měl rozhodnout, ne já. Pověz, co víš!
#34:34 Rozumní mužové mi přisvědčí a moudrý muž mi bude naslouchat:
#34:35 ‚Jób nemluví uváženě, jeho slova nejsou prozíravá.
#34:36 Je třeba jednou provždy Jóba přezkoušet, protože odpovídá jako mužové propadlí ničemnostem.
#34:37 Vždyť ke svému hříchu přidává nevěrnost, mezi námi si výsměšně tleská a má mnoho řečí proti Bohu.‘“ 
#35:1 Na to navázal Elíhú slovy:
#35:2 „Myslíš si, že je to podle práva říkat: ‚Moje spravedlnost převyšuje Boží?‘
#35:3 Říkat: ‚Co z toho máš? Co mi prospěje, nebudu-li hřešit?‘
#35:4 Odpovím ti několika slovy, i tvým přátelům, kteří jsou s tebou.
#35:5 Pohleď na nebe a viz, popatř na mraky vysoko nad sebou.
#35:6 Jestliže jsi zhřešil, čeho jsi dosáhl proti němu? Bude-li tvých přestoupení sebevíc, co mu tím uděláš?
#35:7 Jestliže jsi jednal spravedlivě, co jsi mu dal, co přijal z tvé ruky?
#35:8 Jen člověka jako ty zasáhne tvá svévole, lidského syna tvá spravedlnost.
#35:9 Lidé křičí pro množství útisku, volají o pomoc pro tvrdou paži mocných,
#35:10 a nikdo se nezeptá: ‚Kde je Bůh, můj Učinitel, který dává člověku i v noci prozpěvovat,
#35:11 vyučuje nás a nikoli zvířata zemská a činí nás moudřejší nad nebeské ptactvo?‘
#35:12 Potom jim ovšem neodpovídá, když křičí pro pýchu zlovolných.
#35:13 Bůh přece nedá sluchu falši, Všemocný na to ani nepohlédne,
#35:14 zvláště když o něm prohlašuješ: ‚Nedívá se na to‘. Tvá pře je před ním, jen na něho čekej!
#35:15 Jestliže tedy nestíhá hněvem a nestará se příliš o zpupnost,
#35:16 Jób si otevírá ústa do větru a neuváženě vede zbytečné řeči.“ 
#36:1 Elíhú dále pokračoval slovy:
#36:2 „Měj se mnou trochu trpělivosti a sdělím ti, že ještě mám o Bohu co říci.
#36:3 Začnu se svým věděním zdaleka a přiznám svému Učiniteli spravedlnost,
#36:4 neboť opravdu má slova nejsou klam; před tebou je člověk dokonalých vědomostí.
#36:5 Hle, Bůh je úctyhodný, nezavrhne bez příčiny, je úctyhodný silou i srdcem.
#36:6 Nenechá naživu svévolníka, ale utištěným dopomůže k právu.
#36:7 Od spravedlivého neodvrací zrak; s králi je na trůn dosazuje navždy a jsou vyvýšeni.
#36:8 Jsou-li však spoutáni řetězy, chyceni do provazů utrpení,
#36:9 oznamuje jim, čeho se dopustili, že se rozbujela jejich přestoupení,
#36:10 otvírá jim ucho pro napomenutí a vyzývá je, aby se odvrátili do ničemností.
#36:11 Poslechnou-li, budou-li mu sloužit, dokončí své dny v pohodě a v rozkoších svá léta.
#36:12 Neposlechnou-li, sejdou hubící střelou a zahynou pro svou nerozvážnost.
#36:13 Kdo jsou rouhavého srdce, přispívají k hněvu, nevolají o pomoc, ani když je spoutá.
#36:14 Proto zemře jejich duše v mládí a jejich život skončí se zasvěcenci smilstva.
#36:15 Utištěného zachrání skrze jeho utrpení, otevře mu ucho skrze útlak.
#36:16 To on tě odládá z jícnu soužení; abys měl kolem volný prostor, bez stísněnosti, a poklidný stůl plný tučnosti.
#36:17 Chceš-li se svévolně přít, pře tě přivede před soud.
#36:18 Ať tě rozhořčení nezláká k tleskání a velké výkupné ať tě nesvede z cesty.
#36:19 Vytrhne tě volání o pomoc, když se soužíš a napínáš všechnu svou sílu?
#36:20 Nedychti po noci, která odvede národy z jejich míst.
#36:21 Bedlivě dbej, aby ses neobracel k ničemnosti a nevolil ji raději než utrpení.
#36:22 Hle, jak nedostupný je Bůh ve své síle! Kdo je učitel podobný jemu?
#36:23 Kdo smí dohlížet na jeho cestu? Kdo smí říci: ‚Jednáš podle‘?
#36:24 Pamatuj, že máš vyvyšovat jeho dílo, jež lidé opěvují.
#36:25 Všichni lidé je vyhlížejí, avšak člověk přihlíží jen zdáli.
#36:26 Hle, Bůh je vyvýšený, nad naše poznání, počet jeho let je nevyzpytatelný.
#36:27 Přitahuje vodní kapky, padají jako déšť a v jeho záplavu se mění,
#36:28 když se z mračen valí proudy a hojně skrápějí člověka.
#36:29 Rozumí někdo rozprostření mraků, hromobití v jeho stánku?
#36:30 Hle, nad tím rozprostírá světlo a přikrývá kořeny moře.
#36:31 Povede při s lidmi ten, jenž dává hojný pokrm.
#36:32 Dlaněmi se chápe světla blesků, přikazuje jim zasáhnout útočníka.
#36:33 Ohlašuje ho jeho hromový hlahol, dokonce i stádo ohlašuje jeho příchod. 
#37:1 Mé srdce se z toho chvěje, chtělo by vyskočit ze svého místa.
#37:2 Naslouchejte bedlivě burácení jeho hlasu, rachotu, který mu vychází z úst.
#37:3 Nechá jej dunět pod celým nebem a jeho světlo září až k okrajům země.
#37:4 Za ním řve jeho hlas, on hřmí svým důstojným hlasem a nezadržuje blesky, když se jeho hlas rozléhá.
#37:5 Podivuhodně hřmí Bůh svým hlasem, dělá veliké věci, nad naše poznání:
#37:6 sněhu velí: ‚Padej na zem‘ a dešti: ‚Ať prší‘, a déšť padá v mocných proudech.
#37:7 Na ruku každého člověka klade svoji pečeť, aby všichni lidé, které učinil, nabyli poznání:
#37:8 Zvěř vchází do doupat a přebývá v svých peleších;
#37:9 vichřice vyráží ze své komnaty a severák přináší chlad;
#37:10 svým dechem vyvolává Bůh led, že zamrzají hladiny vod;
#37:11 oblak obtěžkává vláhou, svým bleskem rozhání mračno,
#37:12 a to se točí a převrací, jak on určí, a vše, co mu přikáže, vykoná na tváři okrsku země;
#37:13 užívá ho jako metly nebo k dobru své země, jako důkazu milosrdenství.
#37:14 Dopřej tomu sluchu, Jóbe, postůj, zda pochopíš Boží divy.
#37:15 Víš snad, co jimi Bůh míní, když ozáří oblak svým bleskem?
#37:16 Víš něco o tom, jak plují oblaka, a vůbec něco o divech Vševědoucího?
#37:17 Ty, jemuž horkem žhne šat, když země znehybní pod jižním větrem,
#37:18 dovedl bys jako on vzklenout oblohu pro mraky, pevnou jak lité zrcadlo?
#37:19 Seznam nás s tím, co mu máme říci. Nic nedokážeme pro temnotu.
#37:20 Ohlásí mu někdo, že hodlám mluvit? Řekne si někdo o to, aby byl zničen?
#37:21 Nelze se dívat do světla blesků tehdy, když září v mracích; až se přežene vítr, pročistí je.
#37:22 Od severu přichází zlatavá záře, třpyt kolem Boha budící hrůzu;
#37:23 avšak Všemocného nenajdeme. Je vznešený v síle, ale soudem a přísnou spravedlností nechce pokořovat.
#37:24 Ať se ho proto lidé bojí; nehledí na žádného, kdo spoléhá na své moudré srdce.“ 
#38:1 Na to odpověděl Jóbovi ze smrště Hospodin slovy:
#38:2 „Kdo to zatemňuje úradek Boží neuváženými slovy?
#38:3 Nuže, opásej si bedra jako muž, budu se tě ptát a poučíš mě.
#38:4 Kde jsi byl, když jsem zakládal zemi? Pověz, víš-li něco rozumného o tom.
#38:5 Víš, kdo stanovil její rozměry, kdo nad ní natáhl měřící šňůru?
#38:6 Do čeho jsou zapuštěny její podstavce, kdo kladl její úhelný kámen,
#38:7 zatímco jitřní hvězdy společně plesaly a všichni synové Boží propukli v hlahol?
#38:8 Kdo sevřel moře vraty, když se valilo z lůna země,
#38:9 když jsem mu určil za oděv mračno a za plénku temný mrak,
#38:10 když jsem mu stanovil meze, položil závory a vrata
#38:11 a řekl: ‚Až sem smíš přijít, ale ne dál; zde se složí tvé nespoutané vlnobití!‘
#38:12 Zdali jsi ty někdy za svých dnů dal příkaz jitru a vykázal jitřence její místo,
#38:13 aby se chopila okrajů země a svévolníci byli z ní vytřeseni?
#38:14 Země se mění jak hlína pod pečetidlem, ale oni tu stojí, ač jsou jen šat.
#38:15 Svévolníkům bude odepřeno světlo a zvednutá paže bude přeražena.
#38:16 Přišel jsi až ke zřídlu moře, procházel ses po dně propastné tůně?
#38:17 Byly ti odkryty brány smrti, brány šeré smrti jsi spatřil?
#38:18 Postřehls celou šíři země? Pověz, znáš-li to všechno.
#38:19 Kde je cesta k obydlí světla? Kde má své místo temnota?
#38:20 Můžeš je vykázat do jejich hranic? Máš ponětí o stezkách k jejich domu?
#38:21 Tušil jsi, že se jednou narodíš a jak velký bude počet tvých dnů?
#38:22 Přišel jsi někdy ke skladům sněhu, spatřil jsi sklady krupobití,
#38:23 které si šetřím pro časy soužení, pro den války a boje?
#38:24 Kde je cesta k místu, kde se dělí světlo? Odkud se žene na zemi východní vítr?
#38:25 Kdo vyryl koryta povodni a bouřnému mračnu cestu,
#38:26 aby pršelo na liduprázdnou zemi, na poušť, v níž člověka není,
#38:27 aby se napojila místa pustá a zpustošená a vzešla mladá tráva?
#38:28 Má snad déšť otce? Kdo zplodil krůpěje rosy?
#38:29 Z čího lůna vyšel led? Kdo rodí nebeské jíní?
#38:30 Vody tuhnou na kámen, zamrzá hladina propastné tůně.
#38:31 Dovedeš spoutat mihotavý třpyt Plejád nebo rozvázat pouta Orióna?
#38:32 Vyvedeš hvězdy zvířetníku v pravý čas a povedeš souhvězdí Lva s jeho mladými?
#38:33 Víš, jaké jsou řády nebes? Ty jsi je ustanovil, aby dozírala na zemi?
#38:34 Pozvedneš svůj hlas k oblaku, aby tě přikryla spousta vod?
#38:35 Posíláš snad pro blesky, aby přišly a ohlásily se ti: ‚Tu jsme‘?
#38:36 Kdo dal ibisovi moudrost a kouhoutovi rozum?
#38:37 Kdo je tak moudrý, aby sečetl mraky? Kdo složí nebeské měchy,
#38:38 když prach se v slitinu spojí a hroudy k sobě přilnou?
#38:39 Můžeš ulovit pro lva kořist a uspokojit lačnost lvíčat,
#38:40 když se krčí v peleších, číhají ve svém doupěti v houští?
#38:41 Kdo opatří úlovek krkavci, když jeho mláďata volají k Bohu a bloudí bez potravy? 
#39:1 Znáš čas vrhu skalních kozorožců? Opatrovals laň, když rodí?
#39:2 Počítáš měsíce, kdy jsou březí, a znáš čas jejich vrhu,
#39:3 jak se skloní, vrhnou svá mláďata, svých bolestí pozbývají?
#39:4 Jak jejich mláďata sílí, volně si rostou, odběhnou a už se k nim nevracejí?
#39:5 Kdo vypustil divokého osla na svobodu, kdo uvolnil řemení stepnímu oslu,
#39:6 jemuž jsem za domov vykázal pustinu, solnou pláň za příbytek?
#39:7 Posmívá se městskému hluku, neposlouchá povyk poháněče,
#39:8 na horách slídí po pastvě, pídí se po zeleni.
#39:9 Bude ti chtít jednorožeč sloužit či nocovat u tvého krmelce?
#39:10 Připoutáš jednorožce provazem k brázdě, bude snad za tebou doliny vláčet?
#39:11 Důvěřuješ mu, protože má tak velkou sílu? Ponecháš mu výtěžek své práce?
#39:12 Věřil bys mu, že tvé zrno sveze a shromáždí na tvůj mlat?
#39:13 Pštrosice mává křídlem, není to však peruť s brky čápa;
#39:14 svá vejce ponechává na zemi a zahřívá je v prachu,
#39:15 zapomíná, že je noha může rozšlápnout a polní zvěř zdupat.
#39:16 Se svými mláďaty zachází tvrdě, jako by nebyla její, nemá strach, že bude její námaha marná.
#39:17 Bůh totiž odepřel dát jí moudrost, nedal jí ani díl rozumnosti.
#39:18 Když se však vyplašena vymrští, je jí k smíchu kůň i s jezdcem.
#39:19 Dal jsi snad koni bohatýrskou sílu, přioděl jsi jeho šíji hřívou?
#39:20 Docílíš, aby poskakoval jako luční kobylka? Jeho vznešené frkání vzbuzuje strach,
#39:21 v dolině hrabe nohama, rozjařen silou, má-li vytáhnout proti ozbrojencům,
#39:22 vysmívá se strachu, neděsí se, před mečem se neobrací.
#39:23 Toulec nad ním chřestí, blýská se kopí a oštěp,
#39:24 s burácením se řítí po zemi, až se chvěje, nedá na nic než na zvuk polnice,
#39:25 zařehtá, kdykoli polnice zazní, zdaleka větří bitvu, povely velitelů a válečný ryk.
#39:26 Řídí se snad sokol tvým rozumem, když vzlétne, rozprostře křídla k jihu?
#39:27 Což se na tvůj rozkaz vznese orel, aby si vysoko udělal hnízdo?
#39:28 Přebývá a nocuje na skále, na skalním útesu, nepřístupném místě.
#39:29 Odtud vyhlíží pokrm, dodaleka hledí jeho oči.
#39:30 Jeho mláďata střebají krev, a kde jsou skolení, tam je i on.“ 
#40:1 Hospodin dále řekl Jóbovi toto:
#40:2 „Smí se člověk přít se Všemocným? Smí ho kárat? Ten, kdo Boha obvinil, ať odpovídá.“
#40:3 Jób na to Hospodinu odpověděl:
#40:4 „Co ti odpovím, když jsem tak bezvýznamný! Kladu si na ústa ruku.
#40:5 Jednou jsem už promluvil a nevím co odpovědět, ba i podruhé, ale nemohu pokračovat.“
#40:6 Na to odpověděl Jóbovi ze smrště Hospodin slovy:
#40:7 „Nuže, opásej si bedra jako muž, budu se tě ptát a poučíš mě.
#40:8 Chceš vskutku rušit můj soud, prohlásit mě za svévolníka a sám zůstat spravedlivý?
#40:9 Zdalipak máš paži jako Bůh a jako on hřímáš svým hlasem?
#40:10 Ozdob se tedy důstojností a vyvýšeností, oblékni si velebnost a vznešenost.
#40:11 Vylej všechnu prchlivost svého hněvu, pohleď na každého pyšného a sniž ho,
#40:12 pohleď na každého pyšného a zkruš ho, sraz na místě svévolníky,
#40:13 ukryj je všechny společně do prachu, jejich tvář ovaž pro úkryt v zemi.
#40:14 Potom ti vzdám chválu i já, že tvá pravice tě zachránila.
#40:15 Pohleď jen na behemóta, i jeho jsem učinil jako tebe; on jako dobytče žere trávu.
#40:16 Pohleď, jakou má sílu v bedrech, jak mocné jsou svaly jeho břicha.
#40:17 Napřímí ocas jako cedr, šlachy jeho stehen jsou propletené,
#40:18 jeho kosti jsou bronzové válce, jeho hnáty jako železný sochor.
#40:19 On byl na počátku Božích cest; jen jeho Učinitel může na něj s mečem.
#40:20 Pastvu mu poskytují hory, kde dovádí všeliká zvěř polní,
#40:21 uléhá pod lotosem, skryt ve třtině a bahnu.
#40:22 Lotos jej zastírá svým stínem, potoční topoly ho obklopují.
#40:23 A hle, vzedme-li se řeka, neustoupí, důvěřuje si, i když se mu Jordán do tlamy valí.
#40:24 Kdo se mu postaví do očí a provleče mu chřípím smyčku?
#40:25 Vytáhneš udicí livjátana a zkrotíš provazem jeho jazyk?
#40:26 Vložíš mu do chřípí sítěnou houžev, probodneš mu čelist hákem?
#40:27 Bude se tě doprošovat o smilování a pokorně s tebou mluvit?
#40:28 Uzavře snad s tebou smlouvu, abys jej vzal provždy za otroka?
#40:29 Můžeš si s ním pohrávat jak s ptáčkem? Uvážeš ho pro své děvečky?
#40:30 Budou o něj společníci smlouvat, rozkouskují si ho kupčíci?
#40:31 Propíchneš mu kůži bodci, jeho hlavu rybářskými harpunami?
#40:32 Zkus na něho vložit ruku! Pomysli na boj, a neuděláš to. 
#41:1 Hle, čekat na něho je ošidné, při pohledu na něho se člověk hroutí.
#41:2 Není odvážlivce, který by ho dráždil. Kdo potom obstojí přede mnou?
#41:3 Kdo mi napřed něco dal, abych mu to splatil? Pod celým nebem všechno je mé.
#41:4 Nemohu mlčet o jeho údech, nemluvit o jeho bohatýrské síle a jeho výborné stavbě těla.
#41:5 Kdo odkryl jeho oděv a přistoupil k němu s udidly dvojitými?
#41:6 Kdo otevřel vrata jeho tlamy? Strach jde z jeho zubů.
#41:7 Jeho hřbet je jako řady štítů těsně uzavřených pod pečetí,
#41:8 dotýkají se jedny druhých, vítr mezi ně nevnikne;
#41:9 jeden přilétá k druhému, jsou pevně sevřeny, nelze je oddělit.
#41:10 Jeho kýcháním se rozžehává světlo, jeho oči jsou jak řasy jitřenky,
#41:11 z úst mu vycházejí pochodně, unikají ohnivé jiskry,
#41:12 z nozder mu vystupuje kouř jak z hrnce nad ohněm z rákosí rozdmýchaným,
#41:13 jeho dech rozžhavuje uhlí, z tlamy mu šlehá plamen,
#41:14 v jeho šíji dříme síla, před ním se každý zkroušeně chvěje,
#41:15 laloky mu přiléhají k tělu jak ulité, netřesou se,
#41:16 jeho srdce je slité, tvrdé jako kámen, slité jako spodní žernov.
#41:17 Když se zvedne, sami bohové se leknou, při tom hrozném třesku jsou bez sebe strachy,
#41:18 meč, jenž by ho zasáhl, tu ránu nevydrží, ani kopí, vržená střela či hrot šípu.
#41:19 Železo má za slámu, bronz za trouchnivé dřevo,
#41:20 šíp z luku ho nezažene na útěk, kameny z praku se před ním mění v stébla slámy,
#41:21 kyj má jen za slaměné stéblo, posmívá se, když přiletí chvějící se oštěp.
#41:22 Vespod má ostny podobné střepům, vleče se blátem jako smyk na obilí,
#41:23 způsobuje, že to v hlubině vře jako v hrnci, moře je pro něj jak kelímek na masti,
#41:24 nechává za sebou světélkující dráhu, až se zdá, že propastná tůň zšedivěla.
#41:25 Na prachu země mu není podobného, kdo by byl prost všeho děsu.
#41:26 Na všechno vysoké pohlíží svrchu, nad všemi šelmami je králem.“ 
#42:1 Jób na to Hospodinu odpověděl:
#42:2 „Uznávám, že všechno můžeš a že žádný záměr tobě není neproveditelný.
#42:3 Kdo smí nerozvážně zatemňovat úradek Boží? Ano, hlásal jsem, čemu jsem nerozuměl. Jsou to věci pro mě příliš divuplné, které neznám.
#42:4 Rač mě vyslyšet a nech mě mluvit; budu se tě ptát a poučíš mě.
#42:5 Jen z doslechu o tobě jsem slýchal, teď však jsem tě spatřil vlastním okem.
#42:6 Proto odvolávám a lituji všeho v prachu a popelu.“
#42:7 Když Hospodin k Jóbovi domluvil tato slova, řekl Elífazovi Témanskému: „Můj hněv plane proti tobě a oběma tvým přátelům, protože jste o mně nemluvili náležitě jako můj služebník Jób.
#42:8 Vezměte tedy za sebe sedm býčků a sedm beranů a jděte k mému služebníku Jóbovi. Obětujte za sebe zápalnou oběť a můj služebník Jób ať se za vás modlí. Já ho přijmu milostivě a nepotrestám vás za vaše poblouzení, že jste o mně nemluvili náležitě jako můj služebník Jób.“
#42:9 I šli Elífaz Témanský, Bildad Šúchský a Sófar Naamatský učinit, co jim Hospodin nařídil. A Hospodin Jóba milostně přijal.
#42:10 Hospodin také změnil Jóbův úděl, poté co se modlil za své přátele, a dal mu všeho dvojnásob, než míval.
#42:11 I přišli k němu všichni jeho bratři a všechny sestry a všichni jeho dřívější známí, aby s ním v jeho domě pojedli chléb. Projevovali mu soustrast a snažili se ho potěšit po všem tom zlu, jež na něj Hospodin uvedl, a každý mu daroval po kesítě a zlatém kroužku.
#42:12 A Hospodin Jóbovi žehnal ke konci více než na začátku, takže měl čtrnáct tisíc ovcí a koz, šest tisíc velbloudů, tisíc spřežení skotu a tisíc oslic.
#42:13 Měl také sedm synů a tři dcery.
#42:14 Jednu nazval Jemíma, druhou Kesía a třetí Keren-ha-púk.
#42:15 Po celé zemi se nenašly tak krásné ženy, jako byly dcery Jóbovy. Otec jim dal dědictví jako jejich bratrům.
#42:16 Jób potom žil ještě sto čtyřicet let a viděl své syny i syny svých synů do čtvrtého pokolení.
#42:17 I zemřel Jób stár a sytý dnů.  

\book{Psalms}{Ps}
#1:1 Blaze muži, který se neřídí radami svévolníků, který nestojí na cestě hříšných, který nesedává s posměvači,
#1:2 nýbrž si oblíbil Hospodinův zákon, nad jeho zákonem rozjímá ve dne i v noci.
#1:3 Je jako strom zasazený u tekoucí vody, který dává své ovoce v pravý čas, jemuž listí neuvadá. Vše, co podnikne, se zdaří.
#1:4 Se svévolníky je tomu jinak: jsou jak plevy hnané větrem.
#1:5 Na soudu svévolní neobstojí, ani hříšní v shromáždění spravedlivých.
#1:6 Hospodin zná cestu spravedlivých, ale cesta svévolníků vede do záhuby. 
#2:1 Proč se pronárody bouří, proč národy kují marné plány?
#2:2 Srocují se králové země, vládcové se spolu umlouvají proti Hospodinu a pomazanému jeho:
#2:3 „Zpřetrháme jejich pouta, jejich provazy pryč odhodíme.“
#2:4 Ten, jenž trůní v nebesích, se směje, Panovníkovi jsou k smíchu.
#2:5 Jednou k nim promluví v hněvu, ve svém rozlícení je naplní děsem:
#2:6 „Já jsem ustanovil svého krále na Sijónu, na své svaté hoře!“
#2:7 Přednesu Hospodinovo rozhodnutí. On mi řekl: „Ty jsi můj syn, já jsem tě dnes zplodil.
#2:8 Požádej, a národy ti předám do dědictví, v trvalé vlastnictví i dálavy země.
#2:9 Rozdrtíš je železnou holí, rozbiješ je jak nádobu z hlíny.“
#2:10 Nuže, králové, mějte rozum, dejte na výstrahu, soudcové země!
#2:11 Služte Hospodinu s bázní a jásejte s chvěním.
#2:12 Líbejte syna, ať se nerozhněvá, ať na cestě nezhynete, jestliže jen málo vzplane hněvem. Blaze všem, kteří se k němu utíkají! 
#3:1 Žalm Davidův, když prchal před svým synem Abšalómem.
#3:2 Hospodine, jak mnoho je těch, kteří mě souží, mnoho je těch, kdo proti mně povstávají!
#3:3 Mnoho je těch, kteří o mně prohlašují: „Ten u Boha spásu nenalezne.“
#3:4 Ze všech stran jsi mi však, Hospodine, štítem, tys má sláva, ty mi pozvedáš hlavu.
#3:5 Pozvedám hlas k Hospodinu, a on ze své svaté hory mi už odpovídá.
#3:6 Ulehnu, usnu a probudím se, neboť Hospodin mě podepírá.
#3:7 Nebojím se davu desetitisíců, kteří kolem proti mně se kladou.
#3:8 Povstaň, Hospodine, zachraň mě, můj Bože! Rozbiješ čelisti všem mým nepřátelům, svévolníkům zvyrážíš zuby.
#3:9 V Hospodinu je spása. S tvým lidem je tvoje požehnání! 
#4:1 Pro předního zpěváka za doprovodu strunných nástrojů. Žalm Davidův.
#4:2 Když volám, odpověz mi, Bože mé spravedlnosti! V soužení mi zjednáš volnost. Smiluj se nade mnou, vyslyš mou modlitbu!
#4:3 „Urození, dlouho ještě bude tupena má sláva? Milujete marnost, vyhledáváte lež.
#4:4 Vězte, Hospodin pro svého věrného koná divy. Hospodin slyší, když k němu volám.
#4:5 Jste tím pobouřeni, nezhřešte však. Přemítejte o tom v srdci na loži a buďte zticha.
#4:6 Oběť spravedlnosti mu obětujte, důvěřujte Hospodinu.“
#4:7 Mnozí říkají: „Kdo nám dá užít dobrých věcí?“ Avšak nad námi ať vzejde jas tvé tváře, Hospodine!
#4:8 Mému srdci dáváš větší radost, než mívají oni z hojných žní a vinobraní.
#4:9 Pokojně uléhám, pokojně spím, neboť ty sám, Hospodine, v bezpečí mi dáváš bydlet. 
#5:1 Pro předního zpěváka ke hře na flétnu. Žalm Davidův.
#5:2 Hospodine, přej sluchu mým slovům, měj porozumění pro mou zneklidněnou mysl.
#5:3 Pozornost mi věnuj, když o pomoc volám, můj Králi, můj Bože, vždyť se modlím k tobě!
#5:4 Hospodine, ty můj hlas uslyšíš zrána, ráno ti připravím oběť a budu čekat.
#5:5 Ty nejsi bůh, který má zálibu ve svévoli, zlý nemůže být u tebe hostem,
#5:6 tobě nesmějí na oči potřeštěnci, nenávidíš všechny, kdo páchají ničemnosti,
#5:7 do záhuby vrháš ty, kdo lživě mluví. Kdo prolévá krev a jedná lstivě, toho má Hospodin v ohavnosti.
#5:8 Já však pro tvé hojné milosrdenství smím přicházet do tvého domu, smím se klanět před tvým svatým chrámem ve tvé bázni.
#5:9 Hospodine, ve své spravedlnosti mě veď navzdory těm, kdo proti mně sočí, svou cestu přede mnou učiň přímou.
#5:10 Vždyť na jejich ústa není spolehnutí, jejich nitro je zdroj zhouby, jejich hrdlo je hrob otevřený, na jazyku samé úlisnosti.
#5:11 Odhal jejich vinu, Bože, ať doplatí na své plány, zavrhni je pro tak četné jejich nevěrnosti, vždyť vzdorují tobě!
#5:12 Ať se zaradují všichni, kdo se k tobě utíkají; věčně budou plesat, když je budeš chránit, jásotem tě budou oslavovat ti, kdo milují tvé jméno.
#5:13 Spravedlivému, Hospodine, žehnáš, jako pavézou ho obklopuješ přízní. 
#6:1 Pro předního zpěváka za doprovodu osmistrunných nástrojů. Žalm Davidův.
#6:2 Nekárej mě, Hospodine, ve svém hněvu, netrestej mě ve svém rozhořčení!
#6:3 Smiluj se nade mnou, Hospodine, chřadnu, Hospodine, uzdrav mě, mé kosti trnou děsem.
#6:4 Má duše je tolik vyděšená, a ty, Hospodine, dokdy budeš váhat?
#6:5 Vrať se, Hospodine, braň mě, pro své milosrdenství mě zachraň!
#6:6 Mezi mrtvými tě nebude nic připomínat; což ti v podsvětí vzdá někdo chválu?
#6:7 Vyčerpán jsem nářkem, každé noci smáčím svou podušku pláčem, skrápím slzami své lože.
#6:8 Zrak mi slábne hořem, kalí se mi vinou všech mých protivníků.
#6:9 Pryč ode mne všichni, kdo pácháte ničemnosti! Hospodin můj hlasitý pláč slyší,
#6:10 Hospodin slyší mou prosbu, moji modlitbu Hospodin přijme.
#6:11 Hanba a velký děs padnou na všechny mé nepřátele. Pryč odtáhnou v náhlém zahanbení. 
#7:1 Tklivá píseň Davidova, kterou zpíval Hospodinu kvůli Benjamínci Kúšovi.
#7:2 Hospodine, Bože můj, k tobě se utíkám, zachraň mě a vytrhni z rukou všech, kdo mě stíhají,
#7:3 aby mě jako lev nerozsápal ten, který odvléká, a kdo by vytrhl, není.
#7:4 Hospodine, Bože můj, jestliže jsem něco spáchal, jestliže lpí bezpráví na mých dlaních,
#7:5 jestliže jsem odplácel zlem tomu, který mi přál pokoj, nebo z prázdných ohledů jsem šetřil protivníka,
#7:6 ať nepřítel stíhá moji duši, chytí a zašlape do země můj život, ať uvede v prach mou slávu.
#7:7 Povstaň, Hospodine, ve svém hněvu, proti zuřivosti protivníků mých se zvedni, bděle při mně stůj na soudu, k němuž jsi dal příkaz.
#7:8 Pospolitost národů až kolem tebe stane, k soudu nad nimi se navrať na výšinu.
#7:9 Sám Hospodin povede při s lidmi. Dopomoz mi, Hospodine, k právu, podle mé spravedlnosti a bezúhonnosti.
#7:10 Kéž je konec zlobě svévolníků. Dej, ať spravedlivý stojí pevně, Bože spravedlivý, jenž zkoumáš srdce a ledví!
#7:11 Štítem je mi Bůh, on zachraňuje ty, kdo mají přímé srdce.
#7:12 Bůh je spravedlivý soudce; každý den má Bůh proč vzplanout hněvem.
#7:13 Což si znovu člověk nebrousí svůj meč? Napíná svůj luk a míří,
#7:14 smrtonosnou zbraň si chystá, šípy ohnivé si připravuje.
#7:15 Počíná-li ničemnosti, zplodí trápení a zrodí křivdy.
#7:16 Kope jámu, vyhloubí ji, spadne však do pasti, kterou chystá.
#7:17 To, čím jiné trápí, jemu se na hlavu vrátí, jeho násilnictví na lebku mu padne.
#7:18 Budu vzdávat chválu Hospodinu, že je spravedlivý, budu zpívat žalmy jménu Hospodina, Boha nejvyššího. 
#8:1 Pro předního zpěváka podle gatského způsobu. Žalm Davidův.
#8:2 Hospodine, Pane náš, jak vznešené je tvoje jméno po vší zemi! Svou velebnost vyvýšil jsi nad nebesa.
#8:3 Ústy nemluvňat a kojenců jsi vybudoval mocný val proti svým protivníkům a zastavil nepřítele planoucího pomstou.
#8:4 Vidím tvá nebesa, dílo tvých prstů, měsíc a hvězdy, jež jsi tam upevnil:
#8:5 Co je člověk, že na něho pamatuješ, syn člověka, že se ho ujímáš?
#8:6 Jen maličko jsi ho omezil, že není roven Bohu, korunuješ ho slávou a důstojností.
#8:7 Svěřuješ mu vládu nad dílem svých rukou, všechno pod nohy mu kladeš:
#8:8 všechen brav a skot a také polní zvířata
#8:9 a ptactvo nebeské a mořské ryby, i netvora, který se prohání po mořských stezkách.
#8:10 Hospodine, Pane náš, jak vznešené je tvoje jméno po vší zemi! 
#9:1 Pro předního zpěváka. Podle mút-labben. Žalm Davidův.
#9:2 Hospodine, celým svým srdcem ti vzdávám chválu, o všech divuplných činech tvých chci vypravovat.
#9:3 Budu se radovat, jásotem tě oslavovat, tvému jménu, Nejvyšší, pět žalmy.
#9:4 Moji nepřátelé obrátí se nazpět, upadnou a zhynou před tvou tváří,
#9:5 protožes mi zjednal právo, obhájils mě, usedl jsi na trůn, soudce spravedlivý.
#9:6 Okřikl jsi pronárody, do záhuby svrhl svévolníka, jejich jméno zahladil jsi navěky a navždy.
#9:7 Po nepříteli zůstaly provždy jen trosky. Vyvrátil jsi jejich města, i památka na ně zašla.
#9:8 Hospodin však bude trůnit věčně, soudnou stolici má připravenu.
#9:9 Rozsoudí svět spravedlivě, povede při národů dle práva.
#9:10 Hospodin je zdeptanému nedobytným hradem, v dobách soužení je hradem nedobytným.
#9:11 V tebe nechť doufají, kdo znají tvé jméno. Vždyť ty, kdo se dotazují po tvé vůli, neopouštíš, Hospodine.
#9:12 Pějte žalmy Hospodinu, který na Sijónu trůní, oznamujte mezi lidmi jeho skutky.
#9:13 Za prolitou krev k odpovědnosti volá, pamatuje na ni, na úpění ponížených nezapomněl.
#9:14 Hospodine, smiluj se nade mnou, pohleď, jak mě ponižují ti, kteří mě nenávidí, smiluj se ty, jenž mě zvedáš z bran smrti!
#9:15 O všech chvályhodných činech tvých chci vypravovat v branách sijónské dcery, budu jásat nad tím, žes mě spasil.
#9:16 Pronárody klesly do pasti, již udělaly, v síti, kterou nastražily, uvízla jim noha.
#9:17 Hospodin tu zřejmě zjednal právo: svévolník uvízl v léčce přichystané vlastní rukou. - Higgájón.
#9:18 Do podsvětí se navrátí svévolníci, všechny pronárody, jež na Boha zapomněly.
#9:19 Ubožák však nikdy neupadne v zapomnění; naděje ponížených nikdy neztroskotá.
#9:20 Povstaň, Hospodine, ať si člověk nezakládá na své moci, ať už pronárody stanou před tvým soudem!
#9:21 Hospodine, zdrť je strachem, ať si pronárody uvědomí, že jsou jenom lidé. 
#10:1 Hospodine, proč jen stojíš v dáli, v dobách soužení se skrýváš?
#10:2 Ponížený je pro zpupnost svévolníka v jednom ohni. Ať se zapletou v těch piklech, které vymýšlejí!
#10:3 Když svévolník chválí, pak jen pro své choutky, když chamtivec dobrořečí, znevažuje Hospodina.
#10:4 Svévolník ve zpupném hněvu říká: „Bůh nic nevypátrá, Bůh tu není.“ Odtud všechny jeho pikle.
#10:5 Úspěšné jsou jeho cesty v každé době, vysoko jsou od něho tvé soudy, soptí proti všem svým protivníkům.
#10:6 Říkává si v srdci: „Mnou nic neotřese. Z pokolení do pokolení mě nepotká nic zlého.“
#10:7 V ústech má jen samou kletbu, lest a útisk, na jazyku trápení a ničemnosti.
#10:8 Na nádvořích číhá na čekané, v skrýších vraždí nevinného, jeho oči pasou po bezbranném.
#10:9 Číhá v úkrytu jak v houští lev, číhá na poníženého, aby se ho zmocnil, zmocnil se ho, do sítě ho vtáhl.
#10:10 Plíží se a krčí a bezbranní upadají do jeho spárů.
#10:11 Říkává si v srdci: „Bůh vše zapomíná, skryl svou tvář, nikdy nic neuvidí.“
#10:12 Povstaň, Hospodine Bože, pozvedni svou ruku, nezapomeň na ponižované!
#10:13 Smí svévolník znevažovat Boha, říkat si v srdci: „Ty nic nevypátráš“?
#10:14 Ty však vidíš trápení a hoře. Shlédneš a vše vezmeš do svých rukou. Bezbranný se spoléhá na tebe, sirotkovi poskytuješ pomoc.
#10:15 Přeraz paži svévolníka, paži zlého. Jeho svévoli vypátráš, nic ti neunikne.
#10:16 Hospodin je Králem navěky a navždy, pronárody zmizí z jeho země.
#10:17 Tužbu pokorných vyslýcháš, Hospodine, jejich srdce posiluješ, máš pozorné ucho,
#10:18 zjednáš právo sirotkovi, právo zdeptanému; člověk na zemi už nebude vzbuzovat strach. 
#11:1 Pro předního zpěváka, Davidův. Utíkám se k Hospodinu. Pročpak mi říkáte: „Jen si lítej, ptáče, po té vaší hoře!
#11:2 Hle, jak svévolníci luk už napínají, šípy na tětivu kladou, aby skláli ve tmách ty, kdo mají přímé srdce.
#11:3 Když se všechno od základů hroutí, co dokáže spravedlivý?“
#11:4 Hospodin je ve svém svatém chrámu, Hospodin má trůn svůj na nebesích. Jeho oči hledí, jeho pohled zkoumá lidské syny.
#11:5 Hospodin zkoumá spravedlivého i svévolníka; toho, kdo miluje násilí, z té duše nenávidí.
#11:6 Sešle na svévolníky déšť žhavého uhlí a hořící síry, jejich údělem se stane žhoucí vichr.
#11:7 Hospodin je spravedlivý, miluje vše, co je spravedlivé; jeho tvář pohlíží na přímého. 
#12:1 Pro předního zpěváka za doprovodu osmistrunného nástroje, žalm Davidův.
#12:2 Hospodine, pomoz! Se zbožným je konec, berou za své věrní mezi lidmi.
#12:3 Jeden druhého svou řečí šálí, mluví úlisnými rty a obojakým srdcem.
#12:4 Kéž Hospodin zcela vymýtí ty úlisné rty, jazyk, co se velikášsky chvástá,
#12:5 ty, kdo říkají: „Náš jazyk převahu nám zaručuje, máme přece ústa. Kdo je naším Pánem?“
#12:6 „Pro útlak ponížených, pro sténání ubožáků teď povstanu,“ praví Hospodin, „daruji spásu tomu, proti němuž svévolník soptí.“
#12:7 Co vysloví Hospodin, jsou slova ryzí, stříbro přetavené do kadlubu v zemi, sedmkráte protříbené.
#12:8 Ty je, Hospodine, budeš střežit, navěky nás budeš chránit před tím pokolením;
#12:9 kolem obcházejí svévolníci a vzmáhá se mezi lidmi neurvalost. 
#13:1 Pro předního zpěváka, žalm Davidův.
#13:2 Dlouho ještě, Hospodine, na mě ani nevzpomeneš? Dlouho ještě chceš mi svou tvář skrývat?
#13:3 Dlouho ještě musím sám u sebe hledat rady, strastmi se den ze dne v srdci soužit? Dlouho ještě se bude můj nepřítel proti mně vyvyšovat?
#13:4 Shlédni, Hospodine, Bože můj, a odpověz mi! Rozjasni mé oči, ať neusnu spánkem smrti,
#13:5 ať nepřítel neřekne: „Zdolal jsem ho!“ Moji protivníci budou jásat, zhroutím-li se.
#13:6 Já v tvé milosrdenství však doufám, moje srdce jásá nad tvou spásou. Budu zpívat Hospodinu, neboť se mě zastal. 
#14:1 Pro předního zpěváka, Davidův. Bloud si v srdci říká: „Bůh tu není.“ Všichni kazí, zohavují, na co sáhnou, nikdo nic dobrého neudělá.
#14:2 Hospodin na lidi pohlíží z nebe, chce vidět, má-li kdo rozum, dotazuje-li se po Boží vůli.
#14:3 Zpronevěřili se všichni, zvrhli se do jednoho, nikdo nic dobrého neudělá, naprosto nikdo.
#14:4 Což nevědí všichni, kdo páchají ničemnosti, kdo jedí můj lid, jako by jedli chleba, ti, kdo Hospodina nevzývají,
#14:5 že se jednou třást budou strachem? S pokolením spravedlivého je Bůh.
#14:6 Radu poníženého ostouzíte, ale Hospodin je jeho útočiště.
#14:7 Kéž už přijde Izraeli ze Sijónu spása! Až Hospodin změní úděl svého lidu, bude Jákob jásat, Izrael se zaraduje. 
#15:1 Žalm Davidův. Hospodine, kdo smí pobývat v tvém stanu, kdo smí bydlet na tvé svaté hoře?
#15:2 Ten, kdo žije bezúhonně, ten, kdo jedná spravedlivě, ten, kdo ze srdce zastává pravdu,
#15:3 nemá pomlouvačný jazyk, druhému nedělá nic zlého, na svého druha nekydá hanu,
#15:4 pohrdá tím, kdo je hoden zavržení, váží si těch, kdo se bojí Hospodina, nemění, co odpřisáhl, byť i ke své škodě,
#15:5 nepůjčuje na lichvářský úrok, nedá se podplatit proti nevinnému. Ten, kdo takto jedná, nikdy se nezhroutí. 
#16:1 Pamětní zápis; Davidův. Ochraňuj mě, Bože, utíkám se k tobě!
#16:2 Pravím Hospodinu: „Ty jsi, Panovníku, moje dobro, nad tebe není.“
#16:3 Svatým, těm, kteří jsou v této zemi, pravím, těm vznešeným, jež jsem si oblíbil nade všechno:
#16:4 „Útrapy si rozmnožují, kdo běhají za jinými bohy. Ani trochu krve v úlitbu jim nedám, jejich jméno nepřejde mi přes rty.“
#16:5 Hospodin je podíl mně určený, je můj kalich; můj los držíš pevně, Hospodine.
#16:6 Měřicí provazce mi padly v kraji blaha, moje dědictví je velkolepé!
#16:7 Dobrořečím Hospodinu, on mi radí, i v noci mě moje ledví napomíná.
#16:8 Hospodina stále před oči si stavím, je mi po pravici, nic mnou neotřese.
#16:9 Proto se mé srdce raduje a moje sláva jásá, v bezpečí přebývá i mé tělo,
#16:10 neboť v moci podsvětí mě neponecháš, nedopustíš, aby se tvůj věrný octl v jámě.
#16:11 Stezku života mi dáváš poznat; vrcholem radosti je být s tebou, ve tvé pravici je neskonalé blaho. 
#17:1 Modlitba; Davidova. Hospodine, slyš při spravedlivou, věnuj pozornost mému bědování, dopřej sluchu mé modlitbě z bezelstných rtů.
#17:2 Ty sám vynes nade mnou rozsudek; tvoje oči vidí, kde je právo.
#17:3 Zkoumal jsi mé srdce, dozíral jsi v noci, tříbil jsi mě, nic ti neuniklo, ani úmysl, jenž nepřešel mi přes rty.
#17:4 Pokud jde o lidské činy, slovu tvých rtů věren vyvaroval jsem se stezek rozvratníka.
#17:5 Drž mé kroky ve svých stopách, aby moje nohy neuklouzly.
#17:6 K tobě volám a ty odpovíš mi, Bože, nakloň ke mně ucho, slyš, co říkám.
#17:7 Ukaž divy svého milosrdenství, spasiteli těch, kdo před útočníky se k pravici tvé utíkají.
#17:8 Ochraňuj mě jako zřítelnici oka, skryj mě ve stínu svých křídel
#17:9 před svévolníky, kteří zahubit mě chtějí, úhlavními nepřáteli, když mě obkličují.
#17:10 Tukem obrostlo jim srdce, jejich ústa mluví zpupně.
#17:11 Kruhem svírají mě při každičkém kroku, slídí zrakem, jak mě dostat k zemi.
#17:12 Můj nepřítel podobá se lvu, jenž lační po kořisti, je jak lvíče, které číhá vskrytu.
#17:13 Povstaň, Hospodine, předejdi jej, sraz ho! Kéž mi dá tvůj meč vyváznout z moci svévolníka
#17:14 a tvá ruka, Hospodine, z moci lidí, lidí věku tohoto, jejichž podílem je pouze život. Ty jim ze svých zásob plníš břicho, sytí se i synové a zbytek nechávají nemluvňatům.
#17:15 Já však ve spravedlnosti uzřím tvoji tvář, až procitnu, budu se sytit tvým zjevem. 
#18:1 Pro předního zpěváka. Žalm Hospodinova služebníka Davida, který přednášel slova této písně Hospodinu v den, kdy jej Hospodin vysvobodil ze spárů všech jeho nepřátel i z rukou Saulových.
#18:2 Pravil: Miluji tě vroucně, Hospodine, moje sílo.
#18:3 Hospodine, skalní štíte můj, má pevná tvrzi, vysvoboditeli, Bože můj, má skálo, utíkám se k tobě, štíte můj a rohu spásy, nedobytný hrade!
#18:4 Když jsem vzýval Hospodina, jemuž patří chvála, byl jsem zachráněn před svými nepřáteli.
#18:5 Ovinuly mě provazy smrti, zachvátily mě dravé proudy Ničemníka,
#18:6 provazy podsvětí se kolem mne stáhly, dostihly mě léčky smrti.
#18:7 V soužení jsem vzýval Hospodina, k svému Bohu o pomoc jsem volal. Uslyšel můj hlas ze svého chrámu, mé volání proniklo až k jeho sluchu.
#18:8 Země se zachvěla, roztřásla se, hory v základech se hnuly, chvěly se před jeho plamenným hněvem.
#18:9 Z chřípí se mu valil dým, z úst sžírající oheň, planoucí řeřavé uhlí.
#18:10 Sklonil nebesa a sestupoval, pod nohama černé mračno.
#18:11 Na cheruba usedl a letěl a vznášel se na perutích větru.
#18:12 Temno učinil svou skrýší, stánkem kolem sebe, temné vodstvo, mračna prachu.
#18:13 Mračna se hnala před jeho jasem, krupobití a hořící uhlí.
#18:14 Hospodin na nebi zaburácel, Nejvyšší vydal svůj hlas, krupobití a hořící uhlí.
#18:15 Vyslal své šípy a rozehnal mračna, množstvím blesků je uvedl v zmatek.
#18:16 Tu se objevila koryta vod, základy světa se obnažily, když jsi, Hospodine, zaútočil, když jsi zadul svým hněvivým dechem.
#18:17 Vztáhl ruku z výše, uchopil mě, vytáhl mě z nesmírného vodstva.
#18:18 Nepříteli mocnému mě vyrval, těm, kdo nenáviděli mě, kdo zdatnější byli.
#18:19 Přepadli mě v den mých běd, ale Hospodin mě podepíral,
#18:20 učinil mě volným; ubránil mě, protože si mě oblíbil.
#18:21 Hospodin mi odplatil podle mé spravedlnosti, odměnil mě podle čistoty mých rukou,
#18:22 neboť jsem dbal na Hospodinovy cesty, neodvrátil jsem se svévolně od svého Boha.
#18:23 Všechny jeho řády jsem měl na zřeteli, neodvrhl jeho nařízení.
#18:24 Jeho jsem se dokonale držel, varoval se nepravosti.
#18:25 Podle mé spravedlnosti mě Hospodin odměňoval, podle čistoty mých rukou, tak jak jevila se jemu.
#18:26 Ty věrnému osvědčuješ věrnost, muži dokonalému svou dokonalost,
#18:27 ryzímu svou ryzost osvědčuješ, s neupřímným se však pouštíš do zápasu.
#18:28 Ty lid ponížený zachraňuješ, ale povýšené nutíš sklopit oči.
#18:29 Ty mi rozsvěcuješ světlo, Hospodine. Můj Bůh září do mých temnot.
#18:30 S tebou proběhnu i nepřátelskou vřavou, se svým Bohem zdolám hradbu,
#18:31 s Bohem, jehož cesta je tak dokonalá! To, co řekne Hospodin, je protříbené. On je štítem všech, kteří se k němu utíkají.
#18:32 Kdo je Bůh krom Hospodina, kdo je skála, ne-li Bůh náš!
#18:33 Bůh, který mě opásává statečností a mou cestu činí dokonalou,
#18:34 ten dává mým nohám hbitost laně, na mých posvátných návrších mi dopřává stanout,
#18:35 učí bojovat mé ruce a mé paže napnout bronzový luk.
#18:36 Podal jsi mi štít své spásy, tvoje pravice mě podepírá, tvá mírnost mé síly rozmnožila.
#18:37 Dals volnost mým krokům, nohy se mi nepodvrtnou.
#18:38 Budu stíhat nepřátele, dopadnu je. Nevrátím se zpět, dokud je neudolám.
#18:39 Rozdrtím je, že už nepovstanou, pod nohy mi padnou.
#18:40 Opásals mě statečností k boji; ty, kdo povstávají proti mně, sám srazíš.
#18:41 Obrátil jsi na útěk mé nepřátele, navždy umlčím ty, kdo mě nenávidí.
#18:42 Budou volat o pomoc, a nespasí je nikdo, volat k Hospodinu, ale neodpoví.
#18:43 Roztluču je, budou jako prach ve větru, smetu je jak bláto z ulic.
#18:44 Dals mi vyváznout z rozbrojů lidu, vůdcem pronárodů jsi mě ustanovil. Sloužit bude mi i lid, který jsem neznal.
#18:45 Na slovo mě uposlechnou, cizinci se budou vtírat do mé přízně.
#18:46 Cizinci jak tráva zvadnou, vypotácejí se ze svých hradišť.
#18:47 Živ je Hospodin! Buď požehnána moje skála, vyvýšen buď Bůh, má spása!
#18:48 Bůh, jenž mě pověřil vykonáním pomsty, národy mi podrobuje.
#18:49 Vysvoboditeli od nepřátel, pozvedáš mě nad ty, kdo proti mně povstávají, ty mě násilníku vyrveš.
#18:50 Proto ti vzdám, Hospodine, mezi pronárody chválu, budu zpívat žalmy tvému jménu.
#18:51 Velká vítězství dopřává svému králi, prokazuje milosrdenství svému pomazanému, Davidovi, a jeho potomstvu navěky! 
#19:1 Pro předního zpěváka. Žalm Davidův.
#19:2 Nebesa vypravují o Boží slávě, obloha hovoří o díle jeho rukou.
#19:3 Svoji řeč předává jeden den druhému, noc noci sděluje poznatky.
#19:4 Není to řeč lidská, nejsou to slova, takový hlas od nich nelze slyšet.
#19:5 Jejich tón zvučí celičkou zemí, zní jejich hovor po širém světě. Bůh slunci na nebi postavil stan.
#19:6 Ono jak ženich z komnaty vyjde, vesele jako rek, když běží k cíli.
#19:7 Vychází na jednom okraji nebes, probíhá obloukem k druhému konci a nic se neskryje před jeho žárem.
#19:8 Hospodinův zákon je dokonalý, udržuje při životě. Hospodinovo svědectví je pravdivé, nezkušený jím zmoudří.
#19:9 Hospodinova ustanovení jsou přímá, jsou pro radost v srdci. Hospodinovo přikázání je ryzí, dává očím světlo.
#19:10 Hospodinova bázeň je čistá, obstojí navždy. Hospodinovy řády jsou pravda, jsou nejvýš spravedlivé,
#19:11 nad zlato vzácnější, nad množství ryzího zlata, sladší než med, než včelí med z plástve.
#19:12 Jsou poučením i pro tvého služebníka, když na ně dbá, má odměnu hojnou.
#19:13 Kdo může rozpoznat bludy? Zprosť mě i vin, jež jsou mi skryty.
#19:14 Též před opovážlivci chraň svého služebníka, nedopusť, aby mi vládli. Pak budu bez vady a shledán čistý, prost množství nevěrností.
#19:15 Kéž se ti líbí řeč mých úst i to, o čem rozjímám v srdci, Hospodine, má skálo, vykupiteli můj! 
#20:1 Pro předního zpěváka. Žalm Davidův.
#20:2 Kéž ti v den soužení Hospodin odpoví, kéž je ti hradem jméno Boha Jákobova!
#20:3 Kéž ti sešle pomoc ze svatyně, kéž tě podepírá ze Sijónu!
#20:4 Kéž má na paměti všechny tvé obětní dary, kéž tvou oběť zápalnou rád přijme.
#20:5 Kéž ti dá, po čem tvé srdce touží, kéž splní každý tvůj záměr!
#20:6 Budeme plesat nad tvým vítězstvím, vztyčíme praporce ve jménu svého Boha. Kéž splní Hospodin všechna tvá přání!
#20:7 Nyní vím, že Hospodin dá svému pomazanému vítězství, on mu odpoví ze svého svatého nebe bohatýrskými činy své vítězné pravice.
#20:8 Jedni se honosí vozy, jiní koňmi, ale my připomínáme jméno Hospodina, svého Boha.
#20:9 Oni klesali, až padli, my jsme povstali a přetrváme.
#20:10 „Hospodine, pomoz!“ Král ať nám odpoví v den, kdy budem volat. 
#21:1 Pro předního zpěváka. Žalm Davidův.
#21:2 Hospodine, král se raduje z tvé moci, nad tvým vítězstvím vděčně jásá.
#21:3 Splnil jsi mu touhu jeho srdce, prosbě jeho rtů jsi neodepřel.
#21:4 Vyšels mu vstříc štědrým požehnáním, na hlavu mu kladeš korunu z ryzího zlata.
#21:5 O život tě prosil; daroval jsi mu jej do nejdelších časů, navěky a navždy.
#21:6 Dík tvému vítězství je velká jeho sláva, obestřel jsi ho velebnou důstojností.
#21:7 Učinils ho navždy požehnáním, oblažuješ ho radostí z tvé přítomnosti.
#21:8 Vždyť král doufá v Hospodina; pro milosrdenství Nejvyššího jím nic neotřese.
#21:9 Tvá ruka si najde všechny nepřátele, tvá pravice nalezne ty, kdo tě nenávidí.
#21:10 Rozpálíš je jako pec v čas, kdy se objevíš, Hospodine. Ve svém hněvu je Bůh pohltí, pozře je oheň.
#21:11 Vyhubíš ze země jejich plod a jejich potomstvo mezi lidmi.
#21:12 Ačkoli ti chystali zlý konec a vymýšleli pikle, nepořídí.
#21:13 Položíš je na lopatky, napneš proti nim tětivy svých luků.
#21:14 Pozvedni se, Hospodine, ve své moci! Budem zpívat a pět žalmy o tvé bohatýrské síle. 
#22:1 Pro předního zpěváka. Podle „Laně za ranních červánků„. Žalm Davidův.
#22:2 Bože můj, Bože můj, proč jsi mě opustil? Daleko spása má, ač o pomoc volám.
#22:3 Bože můj, volám ve dne, a neodpovídáš, nemohu se ztišit ani v noci.
#22:4 Ty jsi ten Svatý, jenž trůní obklopen chválami Izraele.
#22:5 Otcové naši doufali v tebe, doufali, tys jim dal vyváznout.
#22:6 Úpěli k tobě a unikli zmaru, doufali v tebe a nebyli zahanbeni.
#22:7 Já však jsem červ a ne člověk, potupa lidství, povrhel lidu.
#22:8 Všem, kdo mě vidí, jsem jenom pro smích, šklebí se na mě, potřásají hlavou:
#22:9 „Svěř to Hospodinu!“ „Ať mu dá vyváznout, ať ho vysvobodí, když si ho oblíbil!“
#22:10 Ty jsi mě vyvedl z života matky, chovals mě v bezpečí u jejích prsou.
#22:11 Na tebe jsem odkázán už z lůna, z života mé matky ty jsi můj Bůh.
#22:12 Nebuď mi vzdálen, blízko je soužení, na pomoc nikoho nemám!
#22:13 Množství býků mě kruhem svírá, bášanští tuři mě obstoupili.
#22:14 Rozevírají na mě tlamu jak řvoucí lev, když trhá kořist.
#22:15 Rozlévám se jako voda, všechny kosti se mi uvolňují, jako vosk je mé srdce, rozplynulo se v mém nitru.
#22:16 Jako střep vyschla má síla, jazyk mi přisedl k patru. Vrháš mě do prachu smrti!
#22:17 Smečka psů mě kruhem svírá, zlovolná tlupa mě obkličuje; sápou se jako lev na mé ruce a nohy,
#22:18 mohu si spočítat všechny své kosti. Pasou se na mně svým zrakem.
#22:19 Dělí se o mé roucho, losují o můj oděv.
#22:20 Nebuď mi vzdálen, Hospodine, má sílo, pospěš mi na pomoc!
#22:21 Vysvoboď mou duši od meče, chraň jediné, co mám, před psí tlapou;
#22:22 zachraň mě ze lví tlamy, před rohy jednorožců! - A tys mi odpověděl.
#22:23 O tvém jménu budu vyprávět svým bratřím, ve shromáždění tě budu chválit.
#22:24 Kdo se bojíte Hospodina, chvalte ho! Ctěte ho, všichni potomci Jákobovi, celé Izraelovo potomstvo, žij před ním v bázni!
#22:25 Nepohrdl poníženým, v opovržení ho neměl. Když trpěl příkoří, neukryl před ním svou tvář, slyšel, když k němu o pomoc volal.
#22:26 Od tebe vzejde mi chvála ve velikém shromáždění. Své sliby splním před těmi, kdo se ho bojí.
#22:27 Pokorní budou jíst dosyta, budou chválit Hospodina ti, kdo se na jeho vůli dotazují. Vaše srdce bude žít navždy.
#22:28 Rozpomenou se a navrátí se k Hospodinu všechny dálavy země. Tobě se budou klanět všechny čeledi pronárodů.
#22:29 Vždyť Hospodinu náleží kralovat, i nad pronárody vládnout.
#22:30 Všichni tuční v zemi, ti, kdo jedli a kdo se klaněli, všichni, kteří sestupují v prach, musí před ním padnout na kolena; a jejich duše si život nezachová.
#22:31 Potomstvo bude mu sloužit. O Panovníku budou vyprávět dalšímu pokolení,
#22:32 to přijde a bude hlásat jeho spravedlnost lidu, který se zrodí: „To učinil on!“ 
#23:1 Žalm Davidův. Hospodin je můj pastýř, nebudu mít nedostatek.
#23:2 Dopřává mi odpočívat na travnatých nivách, vodí mě na klidná místa u vod,
#23:3 naživu mě udržuje, stezkou spravedlnosti mě vede pro své jméno.
#23:4 I když půjdu roklí šeré smrti, nebudu se bát ničeho zlého, vždyť se mnou jsi ty. Tvoje berla a tvá hůl mě potěšují.
#23:5 Prostíráš mi stůl před zraky protivníků, hlavu mi olejem potíráš, kalich mi po okraj plníš.
#23:6 Ano, dobrota a milosrdenství provázet mě budou všemi dny mého žití. Do Hospodinova domu se budu vracet do nejdelších časů. 
#24:1 Davidův; žalm. Hospodinova je země se vším, co je na ní, svět i ti, kdo na něm sídlí.
#24:2 To on základ na mořích jí kladl, pevně ji usadil nad vodními proudy.
#24:3 Kdo vystoupí na Hospodinovu horu? A kdo stanout smí na jeho svatém místě?
#24:4 Ten, kdo má čisté ruce a srdce ryzí, ten, kdo nezneužije mou duši, ten, kdo nepřísahá lstivě.
#24:5 Ten dojde požehnání od Hospodina, spravedlnosti od Boha, své spásy.
#24:6 To je pokolení těch, kdo se na jeho vůli dotazují. Ti, kdo hledají tvou tvář - toť Jákob.
#24:7 Brány, zvedněte výše svá nadpraží, výše se zvedněte, vchody věčné, ať může vejít Král slávy.
#24:8 Kdo to je Král slávy? Hospodin, mocný bohatýr, Hospodin, bohatýr v boji.
#24:9 Brány, zvedněte výše svá nadpraží, výše je zvedněte, vchody věčné, ať může vejít Král slávy.
#24:10 Kdo to je Král slávy? Hospodin zástupů, on je Král slávy! 
#25:1 Davidův. K tobě, Hospodine, pozvedám svou duši,
#25:2 v tebe doufám, Bože můj, kéž nejsem zahanben, ať nade mnou moji nepřátelé nejásají.
#25:3 Ano, nebude zahanben, kdo skládá naději v tebe, zahanbeni budou věrolomní, vyjdou s prázdnou.
#25:4 Dej mi poznat svoje cesty, Hospodine, uč mě chodit po svých stezkách.
#25:5 Veď mě cestou své pravdy a vyučuj mě, vždyť jsi Bůh, má spása, každodenně skládám svou naději v tebe.
#25:6 Hospodine, pamatuj na svoje slitování, na své milosrdenství, které je od věčnosti.
#25:7 Nepřipomínej si hříchy mého mládí, moje nevěrnosti, pamatuj na mě se svým milosrdenstvím pro svou dobrotivost, Hospodine.
#25:8 Hospodin je dobrotivý, přímý, proto ukazuje hříšným cestu.
#25:9 On pokorné vede cestou práva, on pokorné učí chodit po své cestě.
#25:10 Všechny stezky Hospodinovy jsou milosrdenství a věrnost pro ty, kteří dodržují jeho smlouvu a svědectví.
#25:11 Pro své jméno, Hospodine, odpusť mi mou nepravost, je velká.
#25:12 Jak je tomu s mužem, jenž se bojí Hospodina? Ukáže mu cestu, kterou si má zvolit.
#25:13 Jeho duše se uhostí v dobru, jeho potomstvo obdrží zemi.
#25:14 Hospodinovo tajemství patří těm, kdo se ho bojí, ve známost jim uvádí svou smlouvu.
#25:15 Stále upírám své oči k Hospodinu, on vyprostí ze sítě mé nohy.
#25:16 Obrať ke mně svou tvář, smiluj se nade mnou, jsem tak sám, tak ponížený.
#25:17 Mému srdci přibývá soužení. Vyveď mě z úzkostí.
#25:18 Pohleď na mé pokoření, na moje trápení, sejmi ze mne všechny hříchy.
#25:19 Pohleď, jak mnoho je mých nepřátel, jak zavile mě nenávidí.
#25:20 Ochraňuj můj život, vysvoboď mě, ať nejsem zahanben, vždyť se utíkám k tobě.
#25:21 Bezúhonnost a přímost mě chrání, svou naději skládám v tebe.
#25:22 Bože, vykup Izraele ze všeho soužení! 
#26:1 Davidův. Dopomoz mi, Hospodine, k právu. Žil jsem bezúhonně, neochvějně doufal v Hospodina.
#26:2 Hospodine, zkoumej mě a podrob zkoušce, protřib moje ledví a mé srdce!
#26:3 Tvoje milosrdenství mám před očima, řídím se tvou pravdou.
#26:4 Nesedávám s šalebníky, nescházím se s potměšilci.
#26:5 Schůzky zlovolníků nenávidím, mezi svévolníky nezasednu.
#26:6 Umývám si ruce v nevinnosti, při tvém oltáři se držím, Hospodine,
#26:7 aby bylo slyšet mé hlasité díkůvzdání, abych vyprávěl o všech tvých divech.
#26:8 Hospodine, zamiloval jsem si dům, v němž bydlíš, místo, kde má příbytek tvá sláva.
#26:9 Nesmeť spolu s hříšníky mou duši a můj život s těmi, kdo krev prolévají,
#26:10 kterým na rukou lpí mrzkost, jejichž pravice je plná úplatků.
#26:11 Já přec žiji bezúhonně, vykup mě, smiluj se nade mnou!
#26:12 Moje noha stojí na rovině, v shromážděních budu dobrořečit Hospodinu. 
#27:1 Davidův. Hospodin je světlo mé a moje spása, koho bych se bál? Hospodin je záštita mého života, z koho bych měl strach?
#27:2 Když se na mě vrhli zlovolníci, aby pozřeli mé tělo, protivníci, moji nepřátelé, sami klopýtli a padli.
#27:3 Kdyby se proti mně položilo vojsko, mé srdce nepocítí bázeň, kdyby proti mně i bitva vzplála, přece budu doufat.
#27:4 O jedno jsem prosil Hospodina a jen o to budu usilovat: abych v domě Hospodinově směl bydlet po všechny dny, co živ budu, abych patřil na Hospodinovu vlídnost a zpytoval jeho vůli v chrámu.
#27:5 On mě ve zlý den schová ve svém stánku, ukryje mě v skrýši svého stanu, na skálu mě zvedne.
#27:6 Již teď zvedám hlavu nad své nepřátele kolem. V jeho stanu budu obětovat své oběti za hlaholu polnic, budu zpívat, prozpěvovat žalmy Hospodinu.
#27:7 Hospodine, slyš můj hlas, když volám, smiluj se nade mnou, odpověz mi!
#27:8 Mé srdce si opakuje tvoji výzvu: „Hledejte mou tvář.“ Hospodine, tvář tvou hledám.
#27:9 Svoji tvář přede mnou neukrývej, v hněvu nezamítej svého služebníka. Ty jsi byl má pomoc, neodvrhuj mě a neopouštěj, Bože, moje spáso.
#27:10 I kdyby mě opustil můj otec, moje matka, Hospodin se mě vždy ujme.
#27:11 Hospodine, ukaž mi svou cestu, veď mě rovnou stezkou navzdory těm, kdo proti mně sočí.
#27:12 Nevydávej mě zvůli mých protivníků! Zvedli se proti mně křiví svědkové, i ten, z něhož násilí čiší.
#27:13 Jak bych nevěřil, že budu hledět na Hospodinovu dobrotivost v zemi živých!
#27:14 Naději slož v Hospodina. Buď rozhodný, buď udatného srdce, naději slož v Hospodina! 
#28:1 Davidův. K tobě, Hospodine, volám, nebuď ke mně hluchý, skálo moje. Neozveš-li se mi, budu podoben těm, kteří sestupují v jámu.
#28:2 Vyslyš moje prosby, k tobě o pomoc teď volám, pozvedám své ruce k svatostánku tvé svatyně.
#28:3 Mezi lidi svévolné mě nezařazuj, mezi pachatele ničemností. S bližním rozmlouvají o pokoji, ale v srdci mají zlobu.
#28:4 Nalož s nimi podle jejich skutků, podle jejich zlého počínání. Nalož s nimi podle díla jejich rukou, podle zásluhy jim odplať.
#28:5 Když jim nezáleží na Hospodinových skutcích a na díle jeho rukou, rozmetá je a víc nezbuduje.
#28:6 Požehnán buď Hospodin, že vyslyšel mé prosby!
#28:7 Hospodin je síla má a štít můj, mé srdce v něj doufá. Pomoci jsem došel, proto v srdci jásám, svou písní mu budu vzdávat chválu.
#28:8 Hospodin je síla svého lidu, spásná záštita svého pomazaného.
#28:9 Zachraň svůj lid a žehnej dědictví svému, buď mu pastýřem a nes jej na ramenou věčně! 
#29:1 Žalm Davidův. Přiznejte Hospodinu, synové Boží, přiznejte Hospodinu slávu a sílu.
#29:2 Přiznejte Hospodinu slávu jeho jména, v nádheře svatyně klanějte se Hospodinu.
#29:3 Hospodinův hlas burácí nad vodami, zahřímal Bůh slávy, Hospodin nad mocným vodstvem.
#29:4 Hospodinův hlas je plný moci, Hospodinův hlas je plný důstojnosti.
#29:5 Hospodinův hlas poráží cedry, Hospodin poráží cedry libanónské.
#29:6 Nutí poskakovat Libanón jak býčka, Sirjón jako mládě jednorožce.
#29:7 Hospodinův hlas křeše plameny ohně.
#29:8 Hospodinův hlas nutí poušť svíjet se v křeči, Hospodin nutí svíjet se v křeči poušť Kádeš.
#29:9 Hospodinův hlas nutí laně k porodu, sloupává z lesních stromů kůru a vše v jeho chrámu volá: „Sláva!“
#29:10 Hospodin trůnil nad potopou, Hospodin bude trůnit jako král navěky.
#29:11 Hospodin dává svému lidu sílu, Hospodin žehná svůj lid pokojem. 
#30:1 Žalm. Píseň při posvěcení Hospodinova domu. Davidův.
#30:2 Hospodine, tebe vyvyšuji, neboť jsi mě vytáhl z hlubin, nedopřáls mým nepřátelům, aby se nade mnou radovali.
#30:3 Hospodine, Bože můj, k tobě jsem volal o pomoc a uzdravils mě.
#30:4 Vyvedl jsi mě z podsvětí, Hospodine, zachovals mě při životě, abych nesestoupil v jámu.
#30:5 Pějte žalmy Hospodinu, jeho věrní, vzdejte chválu tomu, co připomíná jeho svatost,
#30:6 neboť jeho hněv je na okamžik, jeho přízeň však na celý život. Večer se uhostí pláč, a ráno všechno plesá.
#30:7 V dobách pohody jsem řekl: Mnou nikdy nic neotřese.
#30:8 Hospodine, svou přízní jsi mocnou učinil mou horu. Ukryls tvář a zděsil jsem se.
#30:9 K tobě, Hospodine, volám, tebe, Panovníku, prosím:
#30:10 Jaký užitek vzejde z mé krve, sestoupím-li v jámu? Což ti prach vzdá chválu, bude hlásat tvoji věrnost?
#30:11 Hospodine, slyš a smiluj se nade mnou, buď mi pomocníkem, Hospodine.
#30:12 Můj nářek jsi změnil v taneční rej, vysvlékls mě z žíněného roucha, opásal jsi mě radostí,
#30:13 aby má sláva ti pěla žalmy a již neumlkla. Hospodine, Bože můj, budu ti vzdávat chválu věčně. 
#31:1 Pro předního zpěváka. Žalm Davidův.
#31:2 Hospodine, utíkám se k tobě, kéž nejsem na věky zahanben; pomoz mi vyváznout pro svou spravedlnost!
#31:3 Skloň ke mně své ucho, pospěš, vysvoboď mě, buď mi skálou záštitnou, buď opevněným domem pro mou spásu.
#31:4 Tys můj skalní štít a pevná tvrz má, veď mě pro své jméno a doveď mě k cíli.
#31:5 Vyvleč mě z té sítě, již mi nastražili, vždyť jsi záštita má.
#31:6 Svého ducha kladu do tvých rukou, vykoupils mě, Hospodine, Bože věrný.
#31:7 Nenávidím ty, kdo se drží šalebných přeludů, já spoléhám na Hospodina.
#31:8 Z tvého milosrdenství se budu radovat a budu jásat, že jsi shlédl na mé pokoření. Vždyť ty víš, co sužuje mou duši.
#31:9 Nevydal jsi mě do rukou nepřítele, dopřáls volnosti mým nohám.
#31:10 Hospodine, smiluj se, vždyť se tak soužím, zrak mi slábne hořem, moje duše i mé tělo chřadnou.
#31:11 V strastech pomíjí můj život, moje léta v nářku, pro mou nepravost mi ubývá sil a mé kosti slábnou.
#31:12 Potupen jsem všemi protivníky, a sousedy nejvíc; známí ze mne mají strach, spatří-li mě venku, vyhnou se mi.
#31:13 Sešel jsem jim z mysli jako mrtvý, jako rozbitá nádoba.
#31:14 Z mnoha stran pomluvy slyším. Kolkolem děs! Smlouvají se na mě, kují pikle, chtějí mi vzít život.
#31:15 Já však, Hospodine, důvěřuji tobě, pravím: „Ty jsi můj Bůh,
#31:16 moje budoucnost je ve tvých rukou.“ Vysvoboď mě z rukou nepřátel a těch, kdo pronásledují mě.
#31:17 Rozjasni tvář nad svým služebníkem, ve svém milosrdenství mě zachraň.
#31:18 Hospodine, kéž nejsem zahanben, když tě volám; ať jsou zahanbeni svévolníci, ať v podsvětí zmlknou.
#31:19 Ať oněmějí zrádné rty, jež mluví proti spravedlivému tak urážlivě, zpupně, s pohrdáním.
#31:20 Jak nesmírná je tvoje dobrotivost, kterou jsi uchoval těm, kdo se tě bojí, a prokázal těm, kteří se k tobě utíkají, před zraky všech lidí.
#31:21 Ukrýváš je u sebe v své skrýši před srocením, před jazyky svárlivými schováváš je v stánku.
#31:22 Požehnán buď Hospodin, že mi prokázal divy svého milosrdenství v nepřístupném městě!
#31:23 A já jsem si ukvapeně řekl: „Jsem zapuzen, nechceš mě už vidět.“ Avšak vyslyšel jsi moje prosby, když jsem k tobě o pomoc volal.
#31:24 Milujte Hospodina, všichni jeho zbožní, Hospodin své věrné opatruje, odplácí však plnou měrou těm, kteří si vedou zpupně.
#31:25 Buďte rozhodní a buďte udatného srdce, všichni, kdo čekáte na Hospodina! 
#32:1 Davidův; poučující. Blaze tomu, z něhož je nevěrnost sňata, jehož hřích je přikryt.
#32:2 Blaze člověku, jemuž Hospodin nepravost nepočítá, v jehož duchu není záludnosti.
#32:3 Mlčel jsem a moje kosti chřadly, celé dny jsem pronaříkal.
#32:4 Ve dne v noci na mně těžce ležela tvá ruka, vysýchal mně morek jako v letním žáru.
#32:5 Svůj hřích jsem před tebou přiznal, svoji nepravost jsem nezakrýval, řekl jsem: „Vyznám se Hospodinu ze své nevěrnosti.“ A ty jsi ze mne sňal nepravost, hřích můj.
#32:6 Proto ať se každý věrný k tobě modlí v čas, kdy lze tě ještě nalézt. I kdyby se vzdulo mocné vodstvo, k němu nedosáhne.
#32:7 Tys má skrýše, ty mě chráníš před soužením, nad tím, že jsem vyvázl, zaplesá všechno kolem.
#32:8 Dám ti prozíravost, ukážu ti cestu, kterou půjdeš, budu ti radit, spočine na tobě mé oko.
#32:9 Nebuďte jako kůň či mezek bez rozumu: toho zdobí ohlávka a uzda na zkrocení, jinak ho u sebe neudržíš.
#32:10 Mnoho bolestí postihne svévolníka, toho však, kdo doufá v Hospodina, obklopuje milosrdenství.
#32:11 Radujte se z Hospodina a jásejte, spravedliví, plesejte všichni, kdo máte přímé srdce! 
#33:1 Zaplesejte, spravedliví, Hospodinu, přímým lidem sluší se ho chválit.
#33:2 Hospodinu vzdejte chválu při citaře, zpívejte mu žalmy s harfou o deseti strunách.
#33:3 Zpívejte mu novou píseň, hrejte dobře za hlaholu polnic.
#33:4 Neboť slovo Hospodinovo je přímé, v každém svém díle je věrný.
#33:5 Miluje spravedlnost a právo, Hospodinova milosrdenství je plná země.
#33:6 Nebesa byla učiněna Hospodinovým slovem, dechem jeho úst pak všechen jejich zástup.
#33:7 Jako hrází drží pohromadě mořské vody, vodstva propastí uložil v zásobnicích.
#33:8 Boj se Hospodina, celá země, všichni obyvatelé světa, žijte v jeho bázni!
#33:9 Co on řekl, to se stalo, jak přikázal, tak vše stojí.
#33:10 Záměry národů Hospodin maří, lidem úmysly hatí.
#33:11 Záměry Hospodinovy platí věčně, úmysly jeho srdce po všechna pokolení.
#33:12 Blaze národu, jemuž je Hospodin Bohem, lidu, jejž si zvolil za dědictví.
#33:13 Hospodin se dívá z nebe, vidí všechny lidské syny,
#33:14 ze svého pevného trůnu shlíží na všechny, kdo obývají zemi.
#33:15 On utvořil srdce každého z nich, on též rozumí všem jejich skutkům.
#33:16 Král se nezachrání velkým vojskem, rek se nevyprostí velkou zmužilostí.
#33:17 Selže kůň, k záchraně nepostačí, velkou silou svou k úniku nepomůže.
#33:18 Ale oko Hospodinovo bdí nad těmi, kdo se ho bojí, nad těmi, kdo čekají na jeho milosrdenství,
#33:19 aby ze smrti je vysvobodil, naživu je zachoval v čas hladu.
#33:20 Naše duše s touhou vzhlíží k Hospodinu, on je naše pomoc, náš štít.
#33:21 Z něho se raduje naše srdce, my doufáme v jeho svaté jméno.
#33:22 Tvoje milosrdenství buď, Hospodine, s námi; na tebe s důvěrou čekáme! 
#34:1 Davidův, když změnil své chování před abímelekem, a ten ho vypudil a on odešel.
#34:2 Dobrořečit budu Hospodinu v každém čase, z úst mi bude znít vždy jeho chvála.
#34:3 O Hospodinu mluv s chloubou, moje duše, ať to slyší pokorní a radují se.
#34:4 Velebte Hospodina se mnou, vyvyšujme spolu jeho jméno.
#34:5 Dotázal jsem se Hospodina a odpověděl mi, vysvobodil mě od všeho, čeho jsem se lekal.
#34:6 Kdo na něho budou hledět, rozzáří se, rdít se nemusejí.
#34:7 Tento ponížený volal a Hospodin slyšel, ve všem soužení byl jeho spása.
#34:8 Hospodinův anděl se položí táborem kolem těch, kteří se bojí Boha, a bude je bránit.
#34:9 Okuste a uzříte, že Hospodin je dobrý. Blaze muži, který se utíká k němu.
#34:10 Žijte v Hospodinově bázni, jeho svatí, vždyť kdo se ho bojí, nemají nedostatek.
#34:11 I lvíčata strádají a hladovějí, ale nic dobrého nechybí těm, kdo se dotazují Hospodina.
#34:12 Pojďte, synové, poslyšte mě, vyučím vás Hospodinově bázni.
#34:13 Jak je tomu s mužem, který si oblíbil život, jenž miluje dny, v nichž by užil dobra?
#34:14 Střeží před zlobou svůj jazyk a své rty před záludnými řečmi.
#34:15 Vyhýbá se zlu a koná dobro, vyhledává pokoj a snaží se o něj.
#34:16 Oči Hospodinovy jsou obráceny k spravedlivým, když volají o pomoc, on nakloní své ucho.
#34:17 Hospodinova tvář se však obrací proti pachatelům zla, vymýtí ze země každou památku po nich.
#34:18 Ty, kdo úpěli, Hospodin slyšel, ze všech soužení je vysvobodil.
#34:19 Hospodin je blízko těm, kdo jsou zkrušeni v srdci, zachraňuje lidi, jejichž duch je zdeptán.
#34:20 Mnoho zla doléhá na spravedlivého, Hospodin ho však ze všeho vysvobodí.
#34:21 Ochraňuje všechny jeho kosti, nebude mu zlomena ani jedna.
#34:22 Svévolníku připraví smrt jeho zloba, a kdo nenávidí spravedlivého, ponese vinu.
#34:23 Hospodin vykoupí duše svých služebníků, nikdo z těch, kteří se k němu utíkají, vinu neponese. 
#35:1 Davidův. Hospodine, veď spor s těmi, kdo vedou spor se mnou, bojuj proti těm, kdo proti mně boj vedou.
#35:2 Chop se pavézy a štítu, na pomoc mi povstaň,
#35:3 rozmáchni se kopím a sekerou proti těm, kdo mě pronásledují, a řekni mi: „Jsem tvoje spása.“
#35:4 Ať se hanbí, ať se stydí ti, kdo o život mi ukládají. Ať táhnou zpět, ať se zardí ti, kdo zlo mi strojí.
#35:5 Ať jsou jako plevy hnané větrem, až na ně udeří anděl Hospodinův,
#35:6 ať je jejich cesta ztemnělá a kluzká, až je Hospodinův anděl bude stíhat.
#35:7 Bez důvodů síť mi nastražili, bez důvodů na mne vykopali jámu.
#35:8 Ať na něj přikvačí znenadání zkáza, do sítě, již nastražil, ať sám se chytí, do zkázy ať padne.
#35:9 Má duše však bude jásat Hospodinu, bude se veselit z jeho spásy.
#35:10 Každá kost ve mně řekne: „Hospodine, kdo je tobě roven? Poníženého ty vysvobozuješ z moci silnějšího, poníženého ubožáka od uchvatitele.“
#35:11 Zlovolní svědkové povstávají, ptají se mne na to, o čem nevím.
#35:12 Za dobro mi odplácejí zlobou, stojím tu jak osiřelý.
#35:13 Já, když byli nemocni, jsem chodil v rouchu žíněném a pokořoval jsem se postem. Ale moje modlitba se mi do klína vrátí.
#35:14 Choval jsem se, jako by byl postižen můj druh či bratr, byl jsem sehnut, sklíčen smutkem jak truchlící matka.
#35:15 Avšak při mém pádu s radostí se shlukli; shlukli se proti mně, aniž jsem co tušil, sami zbiti drásali mě, nedali si pokoj.
#35:16 Rouhali se, šklebili se škodolibě a cenili na mě zuby.
#35:17 Jak dlouho chceš přihlížet, Panovníku? Zachovej mou duši před zkázou, kterou mi přichystali, jediné, co mám, chraň před lvíčaty.
#35:18 Vzdám ti chválu ve velikém shromáždění a budu tě chválit před početným lidem.
#35:19 Ať se nade mnou mí zrádní nepřátelé neradují, ať nemhouří oči, kdo mě bez důvodu nenávidí.
#35:20 To, co řeknou, není ku pokoji, nýbrž proti mírumilovným v zemi; myslí jen na záludnosti.
#35:21 Otevírají si na mě ústa, říkají: „Dobře ti tak! Už to vidíme na vlastní oči.“
#35:22 Hospodine, ty vidíš a nejsi hluchý, Panovníku, nebuď mi vzdálen!
#35:23 Probuď se, postav se za mé právo, za mou při, můj Bože, Panovníku.
#35:24 Hospodine, Bože můj, podle své spravedlnosti mi zjednej právo, ať se nade mnou neradují.
#35:25 Ať si v srdci neříkají: „Dobře mu tak, to jsme chtěli!“ Ať neřeknou: „Zhltli jsme ho.“
#35:26 Ať se zardí hanbou ti, kdo se radují ze zla, které mě postihlo. Ať poleje stud a hanba ty, kdo se nade mne tolik vypínají.
#35:27 Ať však plesají a radují se, kdo mi přejí spravedlnost, ať říkají stále: „Hospodin je velký, přeje pokoj svému služebníku.“
#35:28 O tvé spravedlnosti bude pak hovořit můj jazyk, každý den tě bude chválit. 
#36:1 Pro předního zpěváka. Pro Hospodinova služebníka, Davidův.
#36:2 Tak zní vzpurný výrok svévolníka: „Nemám v srdci místo pro strach z Boha.“ A to do očí mu říká.
#36:3 Lichotí si ve svých očích, a tak bude shledán vinným, hodným nenávisti.
#36:4 Slova jeho úst jsou ničemná a lstivá, přestal jednat rozumně a dobře.
#36:5 Vymýšlí si na svém lůžku ničemnosti, postavil se na nedobrou cestu, neštítí se zlého.
#36:6 Tvoje milosrdenství, Hospodine, sahá až k nebi, tvoje věrnost se dotýká mraků.
#36:7 Tvoje spravedlnost je jak mocné horstvo, propastná tůň nezměrná jsou tvoje soudy; zachraňuješ lidi i dobytek, Hospodine.
#36:8 Jak vzácný skvost je tvé milosrdenství, Bože! Lidé se utíkají do stínu tvých křídel.
#36:9 Osvěžují se tím nejtučnějším z tvého domu, z potoka svých rozkoší jim napít dáváš.
#36:10 U tebe je pramen žití, když ty jsi nám světlem, spatřujeme světlo.
#36:11 Uchovej své milosrdenství těm, kdo tě znají, a svou spravedlnost těm, kdo mají přímé srdce.
#36:12 Kéž na mě nevtrhne zpupná noha, ruka svévolných kéž ze mne neučiní štvance!
#36:13 Ano, pachatelé ničemností padli, jsou sraženi, nejsou schopni povstat. 
#37:1 Davidův. Nevzrušuj se kvůli zlovolníkům, nezáviď těm, kdo jednají podle.
#37:2 Uvadají rychle jako tráva, jak zelené býlí zvadnou.
#37:3 Doufej v Hospodina, konej dobro, v zemi přebývej a zachovávej věrnost.
#37:4 Hledej blaho v Hospodinu, dá ti vše, oč požádá tvé srdce.
#37:5 Svou cestu svěř Hospodinu, doufej v něho, on sám bude jednat.
#37:6 Dá, že tvoje spravedlnost zazáří jak světlo, jako polední jas tvoje právo.
#37:7 Ztiš se před Hospodinem a čekej na něj. Nevzrušuj se kvůli tomu, kdo jde úspěšně svou cestou, nad tím, kdo strojí pikle.
#37:8 Odlož hněv a zanech rozhořčení, nevzrušuj se, ať se nedopustíš zlého,
#37:9 neboť zlovolníci budou vymýceni, ale kdo naději skládá v Hospodina, obdrží zemi.
#37:10 Ještě maličko a bude po svévolníkovi, všimneš-li si jeho místa, bude prázdné.
#37:11 Ale pokorní obdrží zemi a bude je blažit dokonalý pokoj.
#37:12 Proti spravedlivému strojí svévolník pikle, skřípe proti němu svými zuby.
#37:13 Je však Panovníkovi jen k smíchu, vždyť on vidí, že jeho den přijde.
#37:14 Svévolníci tasí meč, luk napínají, chtějí srazit poníženého a ubožáka, zabít ty, kdo chodí přímou cestou.
#37:15 Jejich meč však jim do srdce vnikne, jejich luky budou zpřeráženy.
#37:16 Lepší je to málo, co má spravedlivý, než bohatství mnoha svévolníků,
#37:17 neboť svévolníkům budou přeraženy paže, kdežto spravedlivé Hospodin vždy podepírá.
#37:18 Hospodinu jsou známy dny bezúhonných, jejich dědictví potrvá věčně.
#37:19 V čase zlém nebudou zahanbeni, najedí se dosyta i za dnů hladu.
#37:20 Avšak svévolníci zhynou, nepřátelé Hospodinovi se vytratí jak půvab lučin, vytratí se v dýmu.
#37:21 Svévolník si půjčuje a k splácení se nemá, spravedlivý se však slituje a dává.
#37:22 Požehnaní, ti obdrží zemi, zlořečení budou vymýceni.
#37:23 Hospodin činí krok muže pevným, našel zalíbení v jeho cestě.
#37:24 I kdyby klesal, nikdy neupadne, neboť Hospodin podpírá jeho ruku.
#37:25 Od své mladosti, a jsem už starý, jsem neviděl, že by byl opuštěn spravedlivý, nebo že by jeho potomci žebrali o chléb.
#37:26 Slituje se kdykoli a vypomůže půjčkou, také jeho potomci jsou požehnáním.
#37:27 Odstup od zla, konej dobro, navěky pak budeš bydlet v zemi,
#37:28 neboť Hospodin miluje právo a své věrné neopouští. Pod jeho ochranou budou věčně, kdežto plémě svévolníků bude vymýceno.
#37:29 Spravedliví obdrží zemi a budou v ní bydlet navždy.
#37:30 Ústa spravedlivého pronášejí moudrost, jeho jazyk vyhlašuje právo.
#37:31 Má v svém srdci Boží zákon, jeho kroky nezakolísají.
#37:32 Svévolník však číhá na spravedlivého, chce ho uštvat k smrti.
#37:33 Ale Hospodin ho v jeho ruce neponechá, nedopustí, aby svévolně byl souzen.
#37:34 Slož naději v Hospodina, drž se jeho cesty; povýší tě a obdržíš zemi a spatříš, jak budou svévolníci vymýceni.
#37:35 Viděl jsem krutého svévolníka: Rozpínal se jako bujné křoví;
#37:36 odešel a není, hledal jsem ho, nebyl k nalezení.
#37:37 Přidržuj se bezúhonného, hleď na přímého; pokojný muž bude mít potomstvo.
#37:38 Zato vzpurní budou do jednoho vyhlazeni, potomstvo svévolníků bude vymýceno.
#37:39 Hospodin je spása spravedlivých, záštitou v čas soužení jim bývá.
#37:40 Hospodin jim pomáhá a vyváznout jim dává, dá jim z moci svévolníků vyváznout a zachrání je, protože se k němu utíkají. 
#38:1 Žalm Davidův, k připamatování.
#38:2 Nekárej mě, Hospodine, ve svém rozlícení, netrestej mě ve svém rozhořčení,
#38:3 neboť na mě dopadly tvé šípy, těžce na mě dopadla tvá ruka.
#38:4 Pro tvůj hrozný hněv už není na mém těle zdravé místo, pro můj hřích pokoje nemá jediná kost ve mně.
#38:5 Nad hlavu mi přerostly mé nepravosti, jako těžké břemeno mě tíží.
#38:6 Páchnou, hnisají mé rány pro mou pošetilost.
#38:7 Zhrouceně, až k zemi sehnut celé dny se sklíčen smutkem vláčím.
#38:8 Mé ledví je v jednom ohni, nezbylo nic zdravého v mém těle.
#38:9 Jsem ochromen, zcela zdeptán, křičím a mé srdce sténá.
#38:10 Před sebou máš, Panovníku, všechny moje tužby a můj nářek utajen ti není.
#38:11 Selhává mi srdce, opouští mě síla a mým očím hasne světlo.
#38:12 Kdo mě milovali, moji druzi, odtáhli se pro mou ránu, moji nejbližší opodál stojí.
#38:13 Ti, kdo mi o život ukládají, nastražili léčku, kdo mi chtějí škodit, vedou zhoubné řeči, po celé dny vymýšlejí záludnosti.
#38:14 Já však neslyším, jsem jako hluchý, jako němý, ani ústa neotevřu.
#38:15 Stal se ze mne člověk, který neslyší a neodmlouvá,
#38:16 neboť na tebe jen, Hospodine, čekám, Panovníku, Bože můj, kéž bys odpověděl!
#38:17 Pravím: Ať nemají ze mne radost, budou se nade mne vypínat, uklouzne-li mi noha.
#38:18 Můj pád je už blízko, stále mám před sebou svoji bolest.
#38:19 Přiznávám se ke své nepravosti a svého hříchu se lekám.
#38:20 Moji nepřátelé životem jen kypí, mnoho je těch, kdo mě zrádně nenávidí.
#38:21 Ti, kdo odplácejí za dobrotu zlobou, za to, že se snažím o dobro, mě osočují.
#38:22 Hospodine, ty mě neopouštěj, nevzdaluj se ode mne, můj Bože,
#38:23 na pomoc mi pospěš, Panovníku, moje spáso! 
#39:1 Pro předního zpěváka, pro Jedútúna. Žalm Davidův.
#39:2 Řekl jsem si: Budu dbát na svoje cesty, abych se jazykem neprohřešil. Budu držet na uzdě svá ústa, dokud budu před sebou mít svévolníka.
#39:3 Byl jsem zticha, mlčel jsem jak němý, ale nic se nezlepšilo, má bolest se rozjitřila.
#39:4 Srdce pálilo mě v hrudi, v zneklidněné mysli mi plál oheň. Promluvil jsem svým jazykem takto:
#39:5 Hospodine, dej mi poznat, kdy přijde můj konec a kolik dnů je mi vyměřeno, ať vím, kdy ze světa sejdu.
#39:6 Hle, jen na píď odměřils mi dnů a jako nic je před tebou můj věk. Člověk je jen vánek pouhý, i kdyby stál pevně.
#39:7 Každý žitím putuje jak přelud, hluku nadělá, ten vánek pouhý, kupí majetek a neví, kdo to shrábne.
#39:8 A tak jakou mám naději, Panovníku? Moje očekávání se upíná jen k tobě.
#39:9 Vysvoboď mě ode všech mých nevěrností, nedopouštěj, aby bloud mě tupil!
#39:10 Již jsem němý, neotevřu ústa, vždyť je to tvé dílo.
#39:11 Odejmi už ode mne svou ránu, hynu pod tvou pádnou rukou!
#39:12 Když někoho za nepravost napomínáš tresty, rozkládáš jak mol to, po čem dychtil. Člověk je jen vánek.
#39:13 Hospodine, vyslyš mou modlitbu, přej mi sluchu, když o pomoc volám, nebuď k mému pláči hluchý. Vždyť jsem u tebe jen hostem, příchozím, jako všichni otcové moji.
#39:14 Odvrať ode mne svůj pohled, abych okřál, dřív než odejdu a nebudu již. 
#40:1 Pro předního zpěváka. Davidův, žalm.
#40:2 Všechnu naději jsem složil v Hospodina. On se ke mně sklonil, slyšel mě, když o pomoc jsem volal.
#40:3 Vytáhl mě z jámy zmaru, z tůně bahna, postavil mé nohy na skálu, dopřál mi bezpečně kráčet
#40:4 a do úst mi vložil novou píseň, chvalozpěv našemu Bohu. Uvidí to mnozí a pojme je bázeň, budou doufat v Hospodina.
#40:5 Blaze muži, který doufá v Hospodina, k obludám se neobrací, ani k těm, kteří se uchylují ke lži.
#40:6 Mnoho divů jsi už vykonal, můj Bože, Hospodine, ve tvých úmyslech s námi se ti nevyrovná nikdo. Chci je rozhlašovat, mluvit o nich, je jich tolik, že je vypovědět nelze.
#40:7 Obětní hod ani oběť přídavnou sis nepřál, nýbrž protesals mi uši; nežádals oběť zápalnou ani oběť za hřích.
#40:8 Tu jsem řekl: Hle, přicházím, jak ve svitku knihy o mně stojí psáno.
#40:9 Plnit, Bože můj, tvou vůli je mým přáním, tvůj zákon mám ve svém nitru.
#40:10 Zvěstoval jsem spravedlnost ve velikém shromáždění. Že v tom svým rtům nezbraňuji, víš sám, Hospodine.
#40:11 Spravedlnost tvou jsem ve svém srdci neskryl, hovořil jsem o tvé pravdě a tvé spáse. Nezatajil jsem tvé milosrdenství a věrnost velikému shromáždění.
#40:12 Ty mně, Hospodine, neodepřeš svoje slitování. Kéž mě stále opatruje tvoje milosrdenství a věrnost!
#40:13 Tolik zla mě obklopilo, že mu není počtu. Postihly mě moje nepravosti, že nemohu ani vzhlédnout. Je jich víc než vlasů na mé hlavě, odvahu jsem pozbyl.
#40:14 Hospodine, rač mě vysvobodit, Hospodine, na pomoc mi pospěš!
#40:15 Ať se zardí hanbou všichni, kteří mi o život ukládají a vzít mi jej chtějí, ať táhnou zpět, ať se stydí ti, kdo mi zlo přejí.
#40:16 Za svou hanebnost ať vzbuzují úděs ti, kdo mi říkají: „Dobře ti tak!“
#40:17 Ať jsou veselí a radují se z tebe všichni, kteří tě hledají; ať říkají stále: „Hospodin je velký“ ti, kdo milují tvou spásu.
#40:18 Ač jsem ponížený ubožák, Panovník přec na mě myslí. Tys má pomoc, vysvoboditel můj, neotálej už, můj Bože! 
#41:1 Pro předního zpěváka. Žalm Davidův.
#41:2 Blaze tomu, kdo má pochopení pro nuzného, Hospodin ho ve zlý den zachrání.
#41:3 Hospodin ho bude ochraňovat, zachová mu život, bude mu na zemi blaze. Zvůli nepřátel ho nevydávej!
#41:4 Hospodin ho podepře na loži v jeho mdlobách. V nemoci mu změníš celé lůžko!
#41:5 Pravím: Hospodine, smiluj se nade mnou, uzdrav mě, neboť jsem proti tobě zhřešil.
#41:6 Nepřátelé o mně škodolibě mluví: „Kdy už zemře? Kdy zanikne jeho jméno?“
#41:7 Přijde-li se kdo podívat, šalebně mluví, spřádá v srdci ničemnosti, vyjde ven a mluví.
#41:8 Všichni, kdo mě nenávidí, šeptají si o mě, zlo mi strojí:
#41:9 „Ať na něj dolehne slovo Ničemníka; ulehl, už nepovstane.“
#41:10 I ten, s nímž jsem žil v pokoji a jemuž jsem důvěřoval, ten, jenž můj chléb jedl, vypíná se nade mne a zvedá patu.
#41:11 Ty však, Hospodine, smiluj se a dej mi povstat, abych jim to splatil.
#41:12 Že sis mě oblíbil, poznám z toho, že nade mnou nezazní ryk nepřítele.
#41:13 Ujal ses mne pro mou bezúhonnost, postavils mě navěky před svou tvář.
#41:14 Požehnán buď Hospodin, Bůh Izraele, od věků na věky. Amen. Amen. 
#42:1 Pro předního zpěváka. Poučující, pro Kórachovce.
#42:2 Jako laň dychtí po bystré vodě, tak dychtí duše má po tobě, Bože!
#42:3 Po Bohu žízním, po živém Bohu. Kdy se smím ukázat před Boží tváří?
#42:4 Slzy jsou chléb můj ve dne i v noci, když se mne každý den ptají: „Kde je tvůj Bůh?“
#42:5 Vzpomínám na to a duši vylévám v sobě, jak jsem se v čele zástupu brával k Božímu domu, jak zvučně plesal a vzdával chválu hlučící dav, když slavil svátek.
#42:6 Proč se tak trpce rmoutíš, má duše, proč ve mně úzkostně sténáš? Na Boha čekej, opět mu budu vzdávat chválu, jen jemu, své spáse.
#42:7 Můj Bože, duše se ve mně tak trpce rmoutí, proto mé vzpomínky za tebou spějí z krajiny jordánské, z chermónských končin, od hory Miseáru.
#42:8 Propastná tůně na tůni volá v hukotu peřejí tvých, všechny tvé příboje, tvá vlnobití se přese mne valí.
#42:9 Kéž ve dne přikáže Hospodin milosrdenství svému a v noci své písni být se mnou! Modlím se k Bohu života mého,
#42:10 promlouvám k Bohu, své skále: „Proč na mě zapomínáš, proč musím chodit zármutkem sklíčen v sevření nepřítele?“
#42:11 Smrtelnou ranou mým kostem jsou protivníci, kteří mě tupí, když se mě každý den ptají: „Kde je tvůj Bůh?“
#42:12 Proč se tak trpce rmoutíš, má duše, proč ve mně úzkostně sténáš? Na Boha čekej, opět mu budu vzdávat chválu, jemu, své spáse. On je můj Bůh. 
#43:1 Dopomoz mi, Bože, k právu, ujmi se mého sporu, dej mi vyváznout před bezbožným pronárodem, před člověkem záludným a podlým!
#43:2 Tys přece moje záštita, Bože. Proč zanevřel jsi na mě, proč stále chodit mám zármutkem sklíčen v sevření nepřítele?
#43:3 Sešli své světlo a svoji věrnost; ty ať mě vedou, ty ať mě přivedou k tvé svaté hoře, k příbytku tvému,
#43:4 a já tam přistoupím k božímu oltáři, k Bohu, zdroji své jásavé radosti, a hrou na citaru ti budu vzdávat chválu, Bože, můj Bože!
#43:5 Proč se tak trpce rmoutíš, má duše, proč ve mně úzkostně sténáš? Na Boha čekej, opět mu budu vzdávat chválu, jemu, své spáse. On je můj Bůh. 
#44:1 Pro předního zpěváka; pro Kórachovce, poučující.
#44:2 Bože, na vlastní uši jsme slýchali vyprávění svých otců o činu, který jsi vykonal za jejich dnů, za dnů dávných.
#44:3 Pronárody jsi vyhnal svou rukou, je však jsi zasadil jako révu; rozdrtils národy, je však jsi propustil na svobodu.
#44:4 Nezmocnili se země svým mečem, vítězství jim nedobyla jejich paže, nýbrž tvá pravice, tvoje paže, a světlo tvé tváře, neboť jsi v nich našel zalíbení.
#44:5 Jenom ty jsi můj Král, Bože! Rozhodni, a Jákob bude spasen.
#44:6 S tebou jsme nabrali na rohy své protivníky, útočníky podupali jsme v tvém jménu.
#44:7 Proto na svůj luk se nespoléhám, ani meč mě nezachrání;
#44:8 před protivníky jen tys nás spasil, a ty, kdo nás nenáviděli, jsi zahanbil.
#44:9 Po všechny dny byl Bůh naší chloubou, tvému jménu chceme vzdávat chválu věčně.
#44:10 Teď jsi však na nás zanevřel a musíme se stydět, s našimi zástupy do boje nevycházíš.
#44:11 Před protivníkem nás nutíš ustupovat, oloupili nás ti, kdo nás nenávidí.
#44:12 Vydáváš nás jako ovce na porážku, rozptýlils nás mezi pronárody.
#44:13 Lacino jsi prodal svůj lid, žádný zisk jsi z toho neměl.
#44:14 Dovoluješ sousedům nás tupit, svému okolí jsme pro smích, pro pošklebky.
#44:15 Pronárodům učinils nás pořekadlem, národy nad námi potřásají hlavou.
#44:16 Neustále mám před sebou svoje zostuzení, tváře mi pokrývá hanba,
#44:17 když slyším, jak se utrhač rouhá, když mě souží mstivost nepřítele.
#44:18 To všechno nás postihlo, ač na tebe jsme nezapomínali, nezradili jsme tvou smlouvu.
#44:19 Naše srdce se neodklonilo jinam, naše kroky neodbočily z tvé stezky,
#44:20 i když jsi nás deptal v kraji draků, když jsi nás zahalil v šero smrti.
#44:21 Kdybychom na jméno svého Boha zapomněli, k bohu cizímu své ruce rozepjali,
#44:22 což by to Bůh neodhalil? Zná přece tajnosti srdce.
#44:23 Kvůli tobě jsme vražděni denně, mají nás za ovce na zabití.
#44:24 Vzbuď se, proč spíš, Panovníku? Procitni a nezanevři na nás provždy!
#44:25 Proč skrýváš svou tvář? Proč na naše pokoření, na náš útisk zapomínáš?
#44:26 Naše duše leží v prachu, naše hruď je přitištěna k zemi.
#44:27 Na pomoc nám povstaň, vykup nás pro svoje milosrdenství! 
#45:1 Pro předního zpěváka, podle „Lilií„. Pro Kórachovce; poučující, píseň lásky.
#45:2 Slavnostní řeč mi ze srdce tryská, své dílo přednesu králi; jazyk můj - hbitého písaře rydlo:
#45:3 Ty nejkrásnější ze synů lidských, z tvých rtů se line milost, proto ti Bůh navěky žehná!
#45:4 Boky si opásej, bohatýre, mečem, svou velebnou důstojností,
#45:5 se zdarem do boje důstojně vyjeď za pravdu, mírnost a spravedlnost, svou pravicí dokážeš činy, jež vzbudí bázeň.
#45:6 Máš ostré šípy, národy padnou ti k nohám, zasáhneš srdce nepřátel, králi!
#45:7 Tvůj trůn, ó božský, bude stát věčně a navždy, tvým žezlem královským je žezlo práva.
#45:8 Miluješ spravedlnost, nenávidíš zvůli; proto tě, božský, pomazal Bůh tvůj olejem veselí nad tvoje druhy.
#45:9 Celé tvé roucho myrhou, aloe, kasií voní, z paláců zdobených slonovou kostí hra strun zní pro radost tobě.
#45:10 Královské dcery se skvějí v tvých skvostech, královna ve zlatě z Ofíru ti stojí po pravici.
#45:11 Slyš, dcero, pohleď, nakloň své ucho, zapomeň na svůj lid, na otcův dům,
#45:12 neboť král zatouží po tvé kráse; je to tvůj pán a jemu se klaněj.
#45:13 I dcera Týru svůj dar ti nese, naklonit si tě chtějí boháči z lidu.
#45:14 Královská dcera v celé své slávě již čeká uvnitř, svůj šat má protkaný zlatem.
#45:15 V zářivém oděvu ji vedou králi a za ní její panenské družky, k tobě je uvádějí.
#45:16 Průvod se ubírá v radostném jásotu, vstupuje v královský palác.
#45:17 Namísto otců budeš mít syny, učiníš je velmoži po celé zemi.
#45:18 Tvé jméno budu připomínat po všechna pokolení; proto ti národy budou vzdávat chválu navěky a navždy. 
#46:1 Pro předního zpěváka; pro Kórachovce, píseň vysokým hlasem.
#46:2 Bůh je naše útočiště, naše síla, pomoc v soužení vždy velmi osvědčená.
#46:3 Proto se bát nebudeme, byť se převrátila země a základy hor se pohnuly v srdci moří.
#46:4 Ať si jejich vody hučí, ať se pění, ať se hory pro jejich zpupnost třesou!
#46:5 Řeka svými toky oblažuje město Boží, svatyni příbytku Nejvyššího.
#46:6 Nepohne se, Bůh je v jeho středu, Bůh mu pomáhá hned při rozbřesku jitra.
#46:7 Pronárody hlučí, království se hroutí, jen vydá hlas a země se rozplývá.
#46:8 Hospodin zástupů je s námi, Bůh Jákobův, hrad náš nedobytný.
#46:9 Pojďte jen, pohleďte na Hospodinovy skutky, jak úžasné činy v zemi koná!
#46:10 Činí přítrž válkám až do končin země, tříští luky, láme kopí, spaluje v ohni válečné vozy.
#46:11 „Dost už! Uznejte, že já jsem Bůh. Budu vyvyšován mezi pronárody, vyvyšován v zemi.“
#46:12 Hospodin zástupů je s námi, Bůh Jákobův, hrad náš nedobytný. 
#47:1 Pro předního zpěváka; pro Kórachovce, žalm.
#47:2 Lidé všech národů, tleskejte v dlaně, hlaholte Bohu, plesejte zvučně.
#47:3 Hospodin, Nejvyšší, vzbuzuje bázeň, on je Král velký nad celou zemí.
#47:4 Podrobí nám lidská společenství, národy k nohám nám složí.
#47:5 Vybral nám za náš dědičný podíl chloubu Jákoba, jehož si zamiloval.
#47:6 Bůh vystoupil vzhůru za hlaholu, Hospodin za zvuku polnic.
#47:7 Zpívejte žalmy Bohu, zpívejte žalmy, zpívejte žalmy našemu Králi, zpívejte žalmy!
#47:8 Vždyť Bůh je Král nad celou zemí, zpívejte žalmy k poučení.
#47:9 Bůh kraluje nad všemi pronárody, Bůh sedí na svém svatém trůnu.
#47:10 Knížata lidských společenství se shromáždila, jsou lidem Abrahamova Boha. Vždyť štíty země náležejí Bohu nesmírně vyvýšenému. 
#48:1 Žalmová píseň, pro Kórachovce.
#48:2 Veliký je Hospodin, nejvyšší chvály hodný ve městě našeho Boha, na své svaté hoře.
#48:3 Krásně se vypíná k potěše celé země Sijónská hora, ten nejzazší Sever, sídlo velkého Krále.
#48:4 Bůh v jeho palácích proslul jako hrad nedobytný.
#48:5 Hle, když se králové umluvili a přitáhli spolu,
#48:6 sotva je spatřili, strnuli, děsem se rozutekli.
#48:7 Zachvátilo je tam úzkostné chvění, náhlá křeč jako tu, která rodí.
#48:8 Větrem od východu tříštíš taršíšské lodě.
#48:9 O čem jsme slýchali, to jsme uviděli ve městě Hospodina zástupů, ve městě našeho Boha: až navěky je Bůh činí pevným.
#48:10 Na mysli nám tane tvé milosrdenství, Bože, zde, uprostřed tvého chrámu.
#48:11 Jak tvé jméno, Bože, tak i tvoje chvála zní až do končin země. Tvá pravice je plná spravedlnosti.
#48:12 Raduje se hora Sijón, jásají judské dcery nad tím, jak soudíš.
#48:13 Obejděte Sijón, vydejte se kolem spočítat jeho věže.
#48:14 Pozorně si povšimněte valů, jeho paláce si prohlédněte a příštímu pokolení vyprávějte:
#48:15 „Tento Bůh je Bůh náš navěky a navždy; on sám nás povede věčně.“ 
#49:1 Pro předního zpěváka. Pro Kórachovce, žalm.
#49:2 Slyšte to, všichni lidé, všichni obyvatelé světa, naslouchejte,
#49:3 ať jste rodu prostého anebo urození, boháči i ubožáci.
#49:4 Moje ústa budou vyhlašovat moudrost, mé srdce bude rozjímat o rozumnosti.
#49:5 Nakloním své ucho ku příslovím, při citaře předložím svou hádanku i výklad.
#49:6 Proč bych se bál ve zlých dnech, když mě obklopují zvrhlí záškodníci,
#49:7 kteří spoléhají na své jmění a svým velkým bohatstvím se chlubí?
#49:8 Nikdo nevykoupí ani bratra, není schopen vyplatit Bohu sám sebe.
#49:9 Výkupné za lidský život je tak velké, že se každý musí provždy zříci toho,
#49:10 že by natrvalo, neustále žil a nedočkal se zkázy.
#49:11 Vždyť vidí, že umírají moudří, hynou stejně jako hlupák nebo tupec a své jmění zanechají jiným.
#49:12 Ti však myslí, že tu jejich domy budou věčně, jejich příbytky po všechna pokolení, svými jmény nazývají role.
#49:13 Ale člověk, byť byl ve cti, nemusí ani noc přečkat; podobá se zvířatům, jež zajdou.
#49:14 To je cesta těch, kdo bláznovství se drží; za nimi jdou ti, kdo si libují v jejich řečech.
#49:15 Ženou se jak ovce do podsvětí, sama smrt je pase. Zrána je pošlapou lidé přímí, a co vytvořili, zhltne podsvětí, trvání to nemá.
#49:16 Avšak mne Bůh ze spárů podsvětí vykoupí, on mě přijme!
#49:17 Jen se neboj, bohatne-li někdo a množí-li slávu svého domu;
#49:18 zemře, nic nevezme s sebou, jeho sláva za ním nesestoupí.
#49:19 I když zaživa si dobrořečil: „Chválí tě, že sis to zařídil tak dobře“,
#49:20 musí se přidat k pokolení svých otců, kteří nikdy nezahlédnou světlo.
#49:21 Člověk, byť byl ve cti, nemusí mít rozum, podobá se zvířatům, jež zajdou. 
#50:1 Žalm pro Asafa. Bůh sám, Bůh Hospodin promluvil a volá zemi od slunce východu až po západ.
#50:2 Ze Sijónu, místa dokonalé krásy, zaskvěl se Bůh,
#50:3 přichází Bůh náš a nehodlá mlčet. Před ním jde oheň sžírající, vichřice běsní kolem něho.
#50:4 Nebesa shůry i zemi volá, povede při se svým lidem.
#50:5 „Shromážděte mi mé věrné, ty, kdo při oběti přijali mou smlouvu!“
#50:6 Nebesa hlásají jeho spravedlnost, Bůh sám bude soudcem.
#50:7 „Slyš, můj lide, budu mluvit, Izraeli, svědčím proti tobě, já jsem Bůh, tvůj Bůh jsem.
#50:8 Má žaloba se netýká tvých obětí, tvé zápaly mám před sebou stále.
#50:9 Nevezmu si býčka z tvého domu, kozly ze tvých ohrad.
#50:10 Všechna lesní zvěř mi patří i dobytek na tisíci horách,
#50:11 v horách vím o každém ptáku, polní havěť též mám kolem sebe.
#50:12 Kdybych měl hlad, neřeknu si tobě, mně patří svět se vším, co je na něm.
#50:13 Jídám snad já maso z tura nebo napájím se kozlí krví?
#50:14 Přines Bohu oběť díků a plň svoje sliby Nejvyššímu!
#50:15 Až mě potom budeš v den soužení volat, já tě ubráním a ty mě budeš oslavovat.“
#50:16 Ale svévolníkovi Bůh praví: „Nač odříkáváš má nařízení, proč si bereš do úst moji smlouvu?
#50:17 Ty přec nenávidíš kázeň, ty má slova za sebe jen házíš.
#50:18 Spatříš zloděje a běžíš k němu, s cizoložníky máš podíl.
#50:19 Ústa propůjčuješ k zlému, svůj jazyk jsi spřáhl se lstí.
#50:20 Usedneš a mluvíš proti bratru, kydáš hanu na syna své matky.
#50:21 To jsi dělával, a já jsem mlčel. Domníval ses, že jsem jako ty. Vznáším proti tobě obžalobu.
#50:22 Pochopte to, vy, kdo na Boha jste zapomněli, ať vás nerozsápu a nikdo vás nevytrhne.
#50:23 Mne oslaví ten, kdo přinese oběť díků, ten, kdo jde mou cestou; tomu dám zakusit Boží spásu.“ 
#51:1 Pro předního zpěváka. Žalm Davidův,
#51:2 když k němu přišel prorok Nátan, protože vešel k Bat-šebě.
#51:3 Smiluj se nade mnou, Bože, pro milosrdenství svoje, pro své velké slitování zahlaď moje nevěrnosti,
#51:4 moji nepravost smyj ze mne dokonale, očisť mě od mého hříchu!
#51:5 Doznávám se ke svým nevěrnostem, svůj hřích mám před sebou stále.
#51:6 Proti tobě samému jsem zhřešil, spáchal jsem, co je zlé ve tvých očích. A tak se ukážeš spravedlivý v tom, co vyřkneš, ryzí ve svém soudu.
#51:7 Ano, zrodil jsem se v nepravosti, v hříchu mě počala matka.
#51:8 Ano, v opravdovosti máš zalíbení, dáváš mi poznávat tajuplnou moudrost.
#51:9 Zbav mě hříchu, očisť yzopem a budu čistý, umyj mě, budu bělejší nad sníh.
#51:10 Dej, ať slyším veselí a radost, ať jásají kosti, jež jsi zdeptal.
#51:11 Odvrať svou tvář od mých hříchů, zahlaď všechny moje nepravosti.
#51:12 Stvoř mi, Bože, čisté srdce, obnov v mém nitru pevného ducha.
#51:13 Jen mě neodvrhuj od své tváře, ducha svého svatého mi neber!
#51:14 Dej, ať se zas veselím z tvé spásy, podepři mě duchem oddanosti.
#51:15 Budu učit nevěrné tvým cestám a hříšníci navrátí se k tobě.
#51:16 Vysvoboď mě, abych nebyl vinen krví, Bože, Bože, moje spáso, ať plesá můj jazyk pro tvou spravedlnost.
#51:17 Panovníku, otevři mé rty, ať má ústa hlásají tvou chválu.
#51:18 Oběť, kterou bych dal, se ti nezalíbí, na zápalných obětech ti nezáleží.
#51:19 Zkroušený duch, to je oběť Bohu. Srdcem zkroušeným a zdeptaným ty, Bože, nepohrdáš!
#51:20 Prokaž dobro Sijónu svou přízní, vybuduj jeruzalémské hradby.
#51:21 Pak se ti zalíbí oběti spravedlnosti, zápaly a celopaly, pak ti budou na oltáři obětovat býčky. 
#52:1 Pro předního zpěváka. Poučující, Davidův,
#52:2 když přišel Edómec Dóeg a oznámil Saulovi: „David vešel do Achímelekova domu.“
#52:3 Proč se chlubíš zlem, ty bohatýre? Boží milosrdenství po všechny dny trvá!
#52:4 Jen zhoubu tvůj jazyk splétá, máš ho ostrý jako břitvu, pleticháři.
#52:5 Zlo miluješ víc než dobro, klam více než spravedlivé slovo.
#52:6 Miluješ každé podvratné slovo, jazyku lstivý.
#52:7 Proto Bůh tě smete provždy, popadne tě, ze stanu tě vyrve, vykoření tě ze země živých.
#52:8 Až to spatří spravedliví, zmocní se jich bázeň a posměšně o něm řeknou:
#52:9 „Ano, byl to muž, jenž nepokládal za záštitu Boha, ve své veliké bohatství doufal, zakládal si na své zhoubné moci.“
#52:10 Ale já jsem jako zelenající se oliva v domě Božím. Doufám v Boží milosrdenství navěky a navždy.
#52:11 Navěky ti budu vzdávat chválu, neboť jsi to způsobil ty. Naději jsem složil ve tvé jméno, je tak dobré ke tvým věrným. 
#53:1 Pro předního zpěváka, při tanečním reji. Poučující, Davidův.
#53:2 Bloud si v srdci říká: „Bůh tu není.“ Všichni kazí a bezprávně kdeco zohavují, nikdo nic dobrého neudělá.
#53:3 Bůh na lidi pohlíží z nebe, chce vidět, má-li kdo rozum, dotazuje-li se po Boží vůli.
#53:4 Odpadli však všichni, zvrhli se do jednoho, nikdo nic dobrého neudělá, naprosto nikdo.
#53:5 Což nevědí ti, kdo páchají ničemnosti, kdo jedí můj lid, jako by jedli chleba, ti, kdo Boha nevzývají,
#53:6 že se jednou třást budou strachem, strachem, jaký ještě nebyl? Kosti toho, kdo tě oblehl, Bůh rozmetá, zahanbíš je, protože je Bůh zavrhl.
#53:7 Kéž už přijde Izraeli ze Sijónu spása! Až Bůh změní úděl svého lidu, bude Jákob jásat, Izrael se zaraduje. 
#54:1 Pro předního zpěváka za doprovodu strunných nástrojů. Poučující, Davidův,
#54:2 když přišli Zifejci a řekli Saulovi: „David se přece skrývá u nás.“
#54:3 Bože, pro své jméno mě zachraň, ujmi se mé pře svou bohatýrskou silou.
#54:4 Slyš moji modlitbu, Bože, naslouchej slovům mých úst.
#54:5 Povstali proti mně cizáci, ukrutníci o život mi ukládají, na Boha neberou zřetel.
#54:6 Avšak Bůh mi poskytuje pomoc, Panovník je s těmi, kdo mě podpírají.
#54:7 Kéž to zlo odrazí na ty, kdo proti mně sočí! Ve své věrnosti je umlč.
#54:8 Ochotně ti budu obětovat, vzdávat chválu, Hospodine, tvému jménu, protože je dobré.
#54:9 Ze všeho soužení jsi mě vysvobodil, pád nepřátel spatřilo mé oko. 
#55:1 Pro předního zpěváka za doprovodu strunných nástrojů. Poučující, Davidův.
#55:2 Dopřej, Bože, modlitbě mé sluchu, neskrývej se před mou prosbou.
#55:3 Věnuj mi pozornost, odpověz mi, lkám a sténám, zmateně se toulám,
#55:4 neboť nepřítel mi spílá, svévolník mě tísní, chtěli by mě zlomit ničemnostmi, štvou proti mně plni vzteku.
#55:5 Srdce se mi v hrudi svírá, přepadly mě hrůzy smrti,
#55:6 padá na mě strach a chvění, zděšení mě zachvátilo.
#55:7 Pravím: Kéž bych měl křídla jako holubice, uletěl bych, usadil se jinde.
#55:8 Ano, daleko bych letěl, pobýval bych v poušti,
#55:9 spěchal do bezpečí před náporem větru, před vichřicí.
#55:10 Ve zmatek je uveď, Panovníku, jazyky jim rozdvoj, v městě vidím násilí a sváry,
#55:11 ve dne v noci po hradbách se plíží, ničemnost a trápení jsou v jeho středu,
#55:12 v jeho středu běsní zhouba, týrání a lest se nehnou z ulic.
#55:13 Není to nepřítel, kdo mě tupí, to bych přece snesl; nade mne se nevypíná ten, který mě nenávidí, před tím bych se ukryl,
#55:14 jsi to však ty, člověk jako já, můj druh a přítel!
#55:15 Ten nejlepší vztah nás pojil, za bouřného jásotu jsme chodívali do Božího domu.
#55:16 Ať je překvapí smrt, ať zaživa sejdou do podsvětí! Jen zloba je v jejich doupatech, je v jejich středu.
#55:17 A já volám k Bohu, Hospodin mě spasí.
#55:18 Večer, zrána, za poledne lkám a sténám. - Vyslyšel můj hlas!
#55:19 On vykoupí mě z války, daruje mi pokoj, ač jich proti mně je tolik.
#55:20 Slyší Bůh a pokoří je, on, jenž odedávna trůní, když se změnit nechtějí a nebojí se Boha.
#55:21 Leckdo vztáhne ruku na ty, kteří v pokoji s ním žijí, jeho smlouvu znesvěcuje.
#55:22 Lichotek má plná ústa, ale v srdci válku. Hladší nad olej jsou jeho slova, ale jsou to vytasené meče.
#55:23 Na Hospodina slož svoji starost, postará se o tebe a nedopustí, aby se kdy spravedlivý zhroutil.
#55:24 Ty je, Bože, srázíš v jámu zkázy. Ti, kdo prolévají krev a stavějí vše na lsti, nedožijí ani poloviny svých dnů. Já však doufám v tebe! 
#56:1 Pro předního zpěváka, podle „Němé holubice dálek„. Pamětní zápis, Davidův, když se ho v Gatu zmocnili Pelištejci.
#56:2 Smiluj se nade mnou, Bože, neboť po mně šlape člověk, válečník mě denně utiskuje.
#56:3 Denně po mně šlapou ti, kteří proti mně sočí, těch, kdo zpupně válčí proti mně, je mnoho.
#56:4 Přichází den strachu, já však doufám v tebe.
#56:5 V Boha, jehož slovo chválím, v Boha doufám, nebojím se, co mi může udělat tělo?
#56:6 Denně překrucují moje slova, vymýšlejí na mě kdeco zlého.
#56:7 Srocují se, číhají a stopují mě, o mé bezživotí usilují.
#56:8 Vyváznout je necháš za ty ničemnosti? Takové lidi sraz, Bože, svým hněvem!
#56:9 O mém vyhnanství si vedeš záznam; ukládej si do měchu mé slzy. Což je v svých záznamech nemáš?
#56:10 Jednou se mí nepřátelé obrátí zpět, v den, kdy provolám: „Teď vím, že Bůh je při mně!“
#56:11 V Boha, jehož slovo chválím, v Hospodina, jehož slovo chválím,
#56:12 v Boha doufám, nebojím se, co mi může udělat člověk?
#56:13 Bože, sliby tobě dané splním, přinesu ti oběť chvály,
#56:14 neboť jsi mě vysvobodil z jisté smrti. Což jsi neušetřil moje nohy podvrtnutí, abych před Bohem směl chodit dál ve světle živých? 
#57:1 Pro předního zpěváka, jako „Nevyhlazuj!“ Davidův, pamětní zápis, když uprchl před Saulem do jeskyně.
#57:2 Smiluj se nade mnou, Bože, smiluj se nade mnou, k tobě se utíká moje duše. Utíkám se do stínu tvých křídel, dokud nepomine zhoubné nebezpečí.
#57:3 Volám k Bohu Nejvyššímu, k Bohu, jenž za mě dokončí zápas.
#57:4 Sešle pomoc z nebe, zachrání mě, potupí toho, kdo po mně šlape. Bůh sešle své milosrdenství a věrnost.
#57:5 Mezi lvy být musím, s těmi uléhám, kdo srší ohněm, s lidmi, jejichž zuby jsou kopí a šípy, jejichž jazyk je ostrý meč.
#57:6 Povznes se až nad nebesa, Bože, ať nad celou zemí vzejde tvoje sláva!
#57:7 Nastražili síť mým krokům, člověk mě chtěl zlomit; vykopali na mě jámu, spadli do ní sami.
#57:8 Mé srdce je připraveno, Bože, mé srdce je připraveno, budu zpívat, prozpěvovat žalmy.
#57:9 Probuď se, má slávo! Probuď se už, citaro a harfo, ať jitřenku vzbudím.
#57:10 Panovníku, chci ti mezi lidmi vzdávat chválu, mezi národy ti budu zpívat žalmy;
#57:11 vždyť tvé milosrdenství až k nebi sahá, až do mraků tvoje věrnost.
#57:12 Povznes se až nad nebesa, Bože, a nad celou zemí ať je tvoje sláva! 
#58:1 Pro předního zpěváka, jako: „Nevyhlazuj!“ Davidův, pamětní zápis.
#58:2 Opravdu svou němotou hlásáte spravedlnost? Lidé, soudíte dle práva?
#58:3 V zemi jednáte, jak podlé srdce velí, razíte si cestu násilím svých rukou.
#58:4 Svévolníci, ti se odrodili hned v matčině lůně, z mateřského života se lháři dali bludnou cestou.
#58:5 Mají jedu jako hadi, jsou jak hluchá zmije, když si zacpe uši,
#58:6 aby neslyšela kouzelnický šepot zaklínače zkušeného v zaklínání.
#58:7 Bože, vylámej jim zuby v ústech, vyraz tesáky těch lvíčat, Hospodine!
#58:8 Ať se rozplynou jak tratící se vody, ať se tomu, kdo napne luk, ztupí šípy.
#58:9 Ať se ztratí jako slizký plž, jako potracený plod, který neuzřel slunce.
#58:10 Dříve než co pochopí, už trním pod hrnci vám budou, smete je vichr, ať svěží či zprahlé.
#58:11 Radovat se bude spravedlivý, až uzří tu pomstu, omyje si nohy v krvi svévolníka.
#58:12 A lidé si řeknou: „Spravedlivý ovoce se dočkal. Ano, Bůh to je, kdo na zemi soud koná.“ 
#59:1 Pro předního zpěváka, jako: „Nevyhlazuj!“ Davidův, pamětní zápis, když Saul dal střežit dům, aby ho usmrtil.
#59:2 Vysvoboď mě od mých nepřátel, můj Bože, buď mi hradem proti útočníkům.
#59:3 Vysvoboď mě od těch, kdo páchají ničemnosti, zachraň mě před těmi, kdo prolévají krev.
#59:4 Hle, jak nástrahy mi strojí, mocní srocují se proti mně, ne snad pro nevěrnost nebo můj hřích, Hospodine,
#59:5 sbíhají se, aby zakročili, ne však kvůli nepravosti. Vzbuď se, pojď mi vstříc a pohleď!
#59:6 Hospodine, Bože zástupů, ty jsi Bůh Izraele. Procitni a ztrestej všechny pronárody, neměj slitování s nikým, kdo by věrolomně páchal ničemnosti.
#59:7 K večeru se navracejí, skučí jako psi a pobíhají kolem města.
#59:8 Hle, co chrlí jejich ústa! Mezi jejich rty jsou meče, že prý: „Kdo to slyší?“
#59:9 Ty se jim však, Hospodine, směješ, všechny pronárody jsou ti k smíchu.
#59:10 Moje sílo, budu se tě držet! Bůh je přece hrad můj nedobytný.
#59:11 Bůh můj milosrdný jde přede mnou, Bůh mi dá, že spatřím pád těch, kdo proti mně sočí.
#59:12 Jen je nepobíjej, aby můj lid nezapomněl; rozežeň je mocí svou a sraz je, ty jsi náš štít, Panovníku!
#59:13 Hříchem svých úst, slovem svých rtů, vlastní pýchou ať jsou polapeni za kletbu a za lež, které vyřkli.
#59:14 Skoncuj v rozhořčení, skoncuj s nimi, ať poznají, že Bůh vládne v Jákobovi i v dálavách země!
#59:15 K večeru se navracejí, skučí jako psi a pobíhají kolem města.
#59:16 Za potravou sem a tam se honí, když se nenasytí, zůstávají přes noc.
#59:17 Já však budu zpívat o tvé síle, nad tvým milosrdenstvím hned zrána budu plesat. Vždyť ses mi stal nedobytným hradem, útočištěm v den soužení mého.
#59:18 Moje sílo, o tobě chci žalmy zpívat. Bůh je přece hrad můj nedobytný, Bůh můj milosrdný. 
#60:1 Pro předního zpěváka, podle „Lilie svědectví„. Pamětní zápis, Davidův, k vyučování,
#60:2 když válčil s Aramejci z Dvojříčí a s Aramejci ze Sóby; po návratu pobil Joáb Edómce v Solném údolí, dvanáct tisíc mužů.
#60:3 Zanevřels na nás, Bože, prolomils naše řady, stíhal nás hněv tvůj. Navrať se k nám!
#60:4 Zemí jsi otřásl, rozštěpil jsi ji; scel její trhliny, neboť se hroutí!
#60:5 Vlastnímu lidu dals okusit tvrdost, dals nám pít víno, až závrať nás jímá.
#60:6 Dopustils, aby ti, kdo se tě bojí, korouhev k ústupu před lukem zvedli.
#60:7 Aby tvoji milí byli zachováni, pomoz svou pravicí, odpověz nám!
#60:8 Bůh ve své svatyni promluvil: „S jásotem rozdělím Šekem, rozměřím dolinu Sukót.
#60:9 Mně patří Gileád, mně patří Manases, Efrajim, přilba mé hlavy, Juda, můj palcát.
#60:10 Moáb je mé umývadlo, na Edóm hodím svůj střevíc. Spusť, Pelišteo, proti mně válečný ryk!“
#60:11 Kdože mě uvede do nepřístupného města? Kdo mě dovedl až do Edómu?
#60:12 Což ne, Bože, ty, jenž zanevřel jsi na nás? Což s našimi zástupy bys nevytáhl, Bože?
#60:13 Před protivníkem buď naše pomoc, je šalebné čekat spásu od člověka.
#60:14 S Bohem statečně si povedeme, on rozšlape naše protivníky. 
#61:1 Pro předního zpěváka, za doprovodu strunného nástroje. Davidův.
#61:2 Slyš, Bože, mé bědování, mé modlitbě pozornost věnuj,
#61:3 z končin země k tobě volám se sklíčeným srdcem. Doveď mě na skálu, je pro mě strmá,
#61:4 ty jsi moje útočiště a pevná věž proti nepříteli.
#61:5 Chci navěky pobývat v tvém stanu, utéci se do skrýše tvých křídel.
#61:6 Ty jsi, Bože, slyšel moje sliby. Do vlastnictví dals mi ty, kteří se bojí tvého jména.
#61:7 Přidávej dny králi, ať jsou jeho léta jako celá pokolení,
#61:8 ať před Bohem trůní věčně. Připrav milosrdenství a věrnost, ať ho opatrují!
#61:9 A tak budu tvému jménu navždy zpívat žalmy, den co den budu plnit své sliby. 
#62:1 Pro předního zpěváka, podle Jedútúna. Žalm Davidův.
#62:2 Jen v Bohu se ztiší duše má, od něho vzejde mi spása.
#62:3 Jen on je má skála, má spása, můj nedobytný hrad, mnou nikdy nic neotřese.
#62:4 Jak dlouho budete napadat člověka? Zabitím hrozíte všichni jak stěna nahnutá, jako zeď podkopaná.
#62:5 Stále se radí, jak strhnout ho z výše, ve lhaní zálibu našli, žehnají ústy, zlořečí v nitru.
#62:6 Jen zmlkni před Bohem, duše má, vždyť on mi naději vlévá.
#62:7 Jen on je má skála, má spása, můj nedobytný hrad, mnou nic neotřese.
#62:8 Má spása a sláva je v Bohu, on je má mocná skála, v Bohu mám útočiště.
#62:9 Lide, v každý čas v něho doufej, vylévej před ním své srdce! Bůh je naše útočiště.
#62:10 Lidé jsou jen vánek, urození jsou jen lživé zdání. Na váze stoupají vzhůru, dohromady jsou lehčí než vánek.
#62:11 Neslibujte si nic od útisku, nedejte se šálit tím, že něco uchvátíte, k jmění, byť i přibývalo, neupněte srdce.
#62:12 Bůh promluvil jednou, dvojí věc jsem slyšel: Bohu patří moc,
#62:13 i milosrdenství je, Panovníku, tvoje. Ty každému splatíš podle jeho skutků. 
#63:1 Žalm Davidův, když byl v Judské poušti.
#63:2 Bože, tys Bůh můj! Hledám tě za úsvitu, má duše po tobě žízní. Mé tělo touhou po tobě hyne ve vyschlé, prahnoucí, bezvodé zemi.
#63:3 Proto tě vyhlížím ve svatyni, chci spatřit tvoji sílu a slávu;
#63:4 tvé milosrdenství je lepší než život, mé rty tě chválí zpěvem.
#63:5 Proto ti žehnám po celý život, v tvém jménu pozvedám dlaně.
#63:6 Má duše se sytí nejtučnější stravou, moje rty plesají, má ústa zpívají chválu.
#63:7 Když si tě na lůžku připomínám, o tobě rozjímám za nočních hlídek,
#63:8 že jsi mou pomocí býval, ve stínu křídel tvých plesám.
#63:9 Má duše přilnula k tobě, tvá pravice mě pevně drží.
#63:10 Ti, kdo mi chystají zkázu a o život ukládají, sestoupí v nejhlubší útroby země;
#63:11 vydáni napospas meči za kořist šakalům padnou.
#63:12 Král se však bude radovat; Bůh bude chloubou všech, kdo přísahají při něm, a budou zacpána ústa těm, kdo zrádně mluví. 
#64:1 Pro předního zpěváka. Žalm Davidův.
#64:2 Vyslyš, Bože, moje lkání, chraň mě, ať mohu žít beze strachu z nepřítele,
#64:3 skryj mě, když se zlovolníci tajně radí, když se bouří, kdo páchají ničemnosti,
#64:4 kdo si nabrousili jazyky jak meče, namířili šípy jedovatých řečí,
#64:5 aby zákeřně stříleli na bezúhonného! Znenadání na něj vystřelí bez jakékoli bázně.
#64:6 Utvrzují se v zlém rozhodnutí, počítají, jak by nastražili léčky, říkají si: „Kdo nás může spatřit?
#64:7 Ať si kdo chce slídí po podlostech, nastrojili jsme to dokonale, ať si slídil slídí, hluboké je nitro člověka i jeho srdce.“
#64:8 Bůh však znenadání na ně vystřelí šíp; budou samá rána.
#64:9 Jejich jazyk je sklátí, každý, kdo je spatří, se jim vyhne.
#64:10 Všichni lidé se budou bát, budou rozhlašovat Boží skutek a pochopí jeho dílo.
#64:11 Spravedlivý se bude radovat z Hospodina a utíkat se k němu; on bude chloubou všech, kdo mají přímé srdce. 
#65:1 Pro předního zpěváka, žalm. Davidův, píseň.
#65:2 Ztišením se sluší tebe chválit, Bože, na Sijónu, plnit tobě sliby.
#65:3 K tobě, jenž modlitby slyšíš, přichází veškeré tvorstvo.
#65:4 Přemohly mě nepravosti, našich nevěrností jenom ty nás zprostíš.
#65:5 Blaze tomu, koho vyvolíš a přijmeš, aby směl pobývat ve tvých nádvořích. Tam se budem sytit dary tvého domu, tvého svatého chrámu.
#65:6 Ve své spravedlnosti nám odpovídáš činy budícími bázeň, Bože, naše spáso, naděje všech končin země i zámořských dálek,
#65:7 jenž jsi upevnil svou mocí hory, opásán bohatýrskou silou,
#65:8 jenž konejšíš hukot moří, hukot jejich vlnobití i vřavu národů.
#65:9 Z tvých znamení jímá bázeň obyvatele všech končin, tam, kde jitro nastává i kde se snáší večer, všechno naplňuješ plesem.
#65:10 Navštěvuješ zemi, hojností ji zahrnuješ, velmi bohatou ji činíš. Boží potok je naplněn vodou, pečuješ jim o obilí, ano, máš o zemi péči:
#65:11 zavlažuješ brázdy, kypříš hroudy, vydatnými prškami ji činíš vláčnou, žehnáš tomu, co z ní raší.
#65:12 Ty svou dobrotou celý rok korunuješ, ve tvých stopách kane tučnost,
#65:13 kane na pastviny v stepi, a pahorky jásotem se opásaly.
#65:14 Louky se oděly stády, doliny se halí obilím, zvučí hlaholem a zpěvem. 
#66:1 Pro předního zpěváka, žalmová píseň. Hlahol Bohu, celá země!
#66:2 Pějte žalmy k slávě jeho jména, jeho chválu šiřte chvalozpěvem.
#66:3 Řekněte Bohu: Jakou bázeň vzbuzují tvé činy! Pro tvoji nesmírnou moc se vtírají v tvou přízeň i tví nepřátelé.
#66:4 Ať se ti klaní celá země a zpívá ti žalmy, ať zpívá žalmy tvému jménu.
#66:5 Pojďte, pohleďte na Boží skutky, tím, co koná mezi lidskými syny, vzbuzuje bázeň.
#66:6 Moře obrátil v souš, řeku přešli suchou nohou; radujme se tady z něho!
#66:7 Věčně vládne svou bohatýrskou silou, pronárody sleduje svým zrakem. Odbojníci ať se nevyvyšují!
#66:8 Dobrořečte, národy, našemu Bohu, zvučně rozhlašujte jeho chválu.
#66:9 Zachoval nás při životě, nedopustil, aby nám uklouzly nohy.
#66:10 Ano, zkoušel jsi nás, Bože, protříbil jsi nás, jako se tříbí stříbro:
#66:11 zavedls nás do lovecké sítě, těžké břemeno jsi nám na bedra vložil.
#66:12 Dopustils, že člověk nám po hlavách jezdil, šli jsme ohněm, vodou, vyvedl jsi nás však a dal hojnost všeho.
#66:13 Vstoupím se zápalnou obětí do tvého domu, splním ti své sliby,
#66:14 jež moje rty vyslovily, jež v soužení vyřkla moje ústa.
#66:15 V oběť zápalnou ti přinesu to nejtučnější, s obětním dýmem z beranů připravím skot i kozly.
#66:16 Pojďte, slyšte, všichni bohabojní, budu vám vyprávět, co mi Bůh prokázal.
#66:17 Svými ústy jsem volával k němu, svým jazykem jsem ho vyvyšoval.
#66:18 Kdybych se snad upnul srdcem k ničemnosti, byl by mě Panovník nevyslyšel.
#66:19 Bůh však slyšel, mé modlitbě věnoval pozornost.
#66:20 Požehnán buď Bůh, že mou modlitbu nezamítl a své milosrdenství mi neodepřel. 
#67:1 Pro předního zpěváka za doprovodu strunných nástrojů. Zpívaný žalm.
#67:2 Kéž je nám Bůh milostiv a dá nám požehnání, kéž nad námi rozjasní svou tvář!
#67:3 Ať je známa na zemi tvá cesta, mezi všemi pronárody tvoje spása!
#67:4 Kéž ti, Bože, lidé vzdají chválu, kéž ti vzdají chválu všichni lidé!
#67:5 Ať se národy radují, ať zaplesají, neboť soudíš lidi podle práva, spravuješ národy země.
#67:6 Kéž ti, Bože, lidé vzdají chválu, kéž ti vzdají chválu všichni lidé!
#67:7 Země vydala své plody, Bůh nám žehná, Bůh náš.
#67:8 Bůh nám dává svoje požehnání. Nechť se ho bojí všechny dálavy země! 
#68:1 Pro předního zpěváka, Davidův. Zpívaný žalm.
#68:2 Povstane Bůh, a rozprchnou se jeho nepřátelé, na útěk se před ním dají, kdo ho nenávidí.
#68:3 Odvaneš je jako dým. Tak jako taje vosk před žárem ohně, tak zhynou před Bohem svévolníci.
#68:4 Spravedliví se však zaradují a jásotem budou oslavovat Boha, budou se radostně veselit.
#68:5 Prozpěvujte Bohu, pějte žalmy jeho jménu, upravujte cestu tomu, který jede pustinami. Hospodin je jeho jméno, jásotem ho oslavujte.
#68:6 Otec sirotků, obhájce vdov je Bůh v obydlí svém svatém.
#68:7 Bůh, jenž osamělé usazuje v domě, vyvádí vězně k šťastnému životu; odbojníci však zůstanou ve vyprahlé zemi.
#68:8 Bože, když jsi táhl v čele svého lidu, když jsi kráčel pustým krajem,
#68:9 třásla se země, kanuly krůpěje z nebe před Bohem ze Sínaje, před Bohem, Bohem Izraele.
#68:10 Štědrým přívalem jsi skrápěl své dědictví, Bože, když se země vysílila, pečoval jsi o ni.
#68:11 Tvoje stádce se v ní usadilo. Bože, ve své dobrotě máš péči o poníženého!
#68:12 Panovník pronáší výrok. Nesmírný je zástup žen, jež radostnou zvěst nesou:
#68:13 „Pádí králové i se zástupy, pádí, hospodyně rozdělují kořist.
#68:14 Zůstanete ležet mezi ohradami? Jak se blyští stříbrem potažená křídla holubice, perutě zlatavě zelenavé!
#68:15 Už jí Všemohoucí rozehnal ty krále! To tys přikryl sněhem šerý Salmón.“
#68:16 Horou bohů je hora Bášan, hora Bášan je hora strmých štítů.
#68:17 Proč, vy strmé horské štíty, úkosem shlížíte na horu, na níž se usadit zatoužil Bůh? Tam bude Hospodin přebývat natrvalo.
#68:18 Božích vozů jsou desetitisíce, tisíce tisíců. Panovník je mezi nimi, ze Sínaje táhne do svatyně.
#68:19 Vystoupil jsi na výšinu, ty, kdo byli v zajetí, jsi vedl, mnohé z lidí přijals darem, odbojníci však museli zůstat v poušti, Hospodine, Bože!
#68:20 Požehnán buď Panovník, den ze dne za nás nosí břímě. Bůh je naše spása.
#68:21 Bůh je Bohem, jenž nás zachraňuje. Je to on, Panovník Hospodin, kdo vyvádí z tenat smrti.
#68:22 Ano, Bůh rozdrtí nepřátelům hlavu, témě tomu, který se ve vinách brodí.
#68:23 Panovník řekl: „Vyvedu z Bášanu, vyvedl jsem i z hlubin moře
#68:24 a noha tvá je rozdrtí v krvi, ať jazyk tvých psů má z nepřátel podíl.“
#68:25 Všichni spatřili tvůj průvod, Bože, průvod mého Boha, mého Krále, do svatyně.
#68:26 Napřed šli zpěváci, za nimi hudebníci, uprostřed s bubínky dívky.
#68:27 V shromážděních dobrořečte Bohu, Hospodinu, vy z pramene Izraele!
#68:28 Tady je Benjamín, nejmladší, jenž se stal nad nimi pánem, nad veliteli Judy a jeho oddíly, nad veliteli Zabulóna, veliteli Neftalího.
#68:29 Byl to příkaz tvého Boha. Nechť se tvá moc, Bože, mocně prokazuje v tom, co pro nás konáš
#68:30 ze svého chrámu nad Jeruzalémem. Králové ti budou přinášet dary.
#68:31 Oboř se na tu divou zvěř v rákosí, na stádo turů mezi býčky národů, na toho, který se vrhá do bláta kvůli úlomku stříbra. Rozpraš národy válkychtivé!
#68:32 Ať přijdou egyptští velmožové, ať Kúš vztáhne ruce vstříc Bohu.
#68:33 Království země, zpívejte Bohu, zapějte žalmy Panovníku,
#68:34 tomu, jenž jezdí po nebi, po nebi odvěkém. Hle, vydal hlas, hlas plný moci.
#68:35 Uznejte Boží moc! Jeho vznešenost se klene nad Izraelem, jeho moc do mraků strmí.
#68:36 Bůh ze tvých svatyň vzbuzuje bázeň, Bůh Izraele. On dává moc a udatnost lidu. Požehnán buď Bůh! 
#69:1 Pro předního zpěváka, podle „Lilií„. Davidův.
#69:2 Zachraň mě, můj Bože, vody mi pronikly k duši!
#69:3 V bahně hlubiny se topím, není na čem stanout, do hlubokých vod se nořím, dravý proud mě vleče.
#69:4 Volám do umdlení, hrdlo zanícené, Boha vyhlížím, až zrak mi vypovídá.
#69:5 Víc než vlasů na mé hlavě je těch, kdo mě bez důvodů nenávidí; zdatní jsou, kdo umlčet mě chtějí, zrádní nepřátelé! Co jsem neuchvátil, mám teď vracet.
#69:6 Ty znáš, Bože, moji pošetilost, čím jsem se kdy provinil, ti není skryto.
#69:7 Kvůli mně však ať se nemusejí hanbit, kdo složili svou naději v tebe, Panovníku, Hospodine zástupů, ať kvůli mně se nemusejí stydět ti, kdo tě hledají, Bože Izraele!
#69:8 Snáším potupu přec kvůli tobě, stud pokryl mé tváře.
#69:9 Za cizího mě mají mí bratři, cizincem jsem pro syny své matky,
#69:10 neboť horlivost o tvůj dům mě strávila, tupení těch, kdo tě tupí, padlo na mne.
#69:11 Postil jsem se, vyplakal si oči, ale v potupu se mi to obrátilo.
#69:12 Oblékl jsem si žíněné roucho a stal se pro ně pořekadlem.
#69:13 O mně klevetí, kdo sedávají v bráně, pijani si o mně popěvují.
#69:14 Ale má modlitba spěje, Hospodine, k tobě, je čas přízně, Bože nejvýš milosrdný, odpověz mi, věrný dárce spásy,
#69:15 vysvoboď mě z bahna, ať se neutopím! Kéž jsem vysvobozen z rukou těch, kteří mě nenávidí, z hlubokých vod.
#69:16 Ať mě neodvleče dravý vodní proud, ať hlubina mě nepohltí, ať nade mnou studnice nezavře ústa!
#69:17 Odpověz mi, Hospodine, vždyť tvé milosrdenství je dobrotivé, pro své velké slitování shlédni na mne,
#69:18 neskrývej tvář před svým služebníkem, když se soužím, pospěš, odpověz mi,
#69:19 buď mi blízko, zastaň se mne, vykup mě kvůli mým nepřátelům!
#69:20 Ty víš, jak jsem tupen, ostouzen a haněn, máš před sebou všechny moje protivníky.
#69:21 Srdce potupou mi puká, jsem jak ochrnulý. Na soucit jsem čekal, ale marně; na ty, kdo by potěšili - nenašel se nikdo.
#69:22 Do jídla mi dali žluč, když jsem žíznil, dali mi pít ocet.
#69:23 Jejich stůl se jim staň osidlem a těm, kdo jsou s nimi zajedno, buď léčkou.
#69:24 Dej, ať se jim zatmí v očích, aby neviděli, jejich bedra ustavičně zbavuj síly.
#69:25 Vylej na ně svůj hrozný hněv, tvůj planoucí hněv ať postihne je.
#69:26 Jejich hradiště ať zpustne, v jejich stanech ať nebydlí nikdo.
#69:27 Neboť toho, kterého jsi ty zbil, pronásledovali, vypravují o bolesti těch, které jsi proklál.
#69:28 Sečti jejich nepravosti, ať před tebou nejsou spravedliví.
#69:29 Nechť jsou vymazáni z knihy živých, nezapsáni mezi spravedlivé.
#69:30 Jsem ponížen, bolestí se soužím, avšak hradem je mi, Bože, tvoje spása.
#69:31 Písní budu chválit jméno Boží, velebit je budu díkůvzdáním.
#69:32 Hospodin to přijme raději než býka, býčka s paznehty a rohy.
#69:33 Pokorní to spatří a budou mít radost; kdo se dotazujete na Boží vůli, okřeje v srdci.
#69:34 Vždyť Hospodin ubožáky slyší, nepohrdá svými, když jsou uvězněni.
#69:35 Ať ho chválí nebesa i země, moře se vším, co se v nich hemží,
#69:36 neboť Bůh zachrání Sijón a zbuduje judská města; budou tam sídlit a je mít ve vlastnictví.
#69:37 Potomstvo jeho služebníků je bude dědičně držet; budou v nich bydlet ti, kdo milují jeho jméno. 
#70:1 Pro předního zpěváka. Davidův, k připamatování.
#70:2 Bože, vysvoboď mě, Hospodine, na pomoc mi pospěš!
#70:3 Ať se zardí hanbou ti, kdo mi o život ukládají, ať táhnou zpět, ať se stydí ti, kdo mi zlo přejí.
#70:4 Za svou hanebnost ať musí couvnout ti, kdo mi říkají: „Dobře ti tak!“
#70:5 Ať jsou veselí a radují se z tebe všichni, kteří tě hledají; ať říkají stále: „Bůh je velký“ ti, kdo milují tvou spásu.
#70:6 Ponížený jsem a ubohý, Bože, pospěš ke mně! Tys má pomoc, vysvoboditel můj, neotálej, Hospodine! 
#71:1 Utíkám se k tobě, Hospodine, kéž nikdy nejsem zahanben!
#71:2 Pro svou spravedlnost mě vysvoboď, pomoz mi vyváznout, skloň ke mně své ucho, buď mou spásou.
#71:3 Buď mi skalním příbytkem a budu se tam uchylovat stále. Rozhodls o mé záchraně, tys můj skalní štít, moje pevná tvrz!
#71:4 Bože můj, pomoz mi vyváznout z moci svévolníka, z rukou bídáka a násilníka.
#71:5 Ty jsi přece má naděje, Panovníku Hospodine, v tebe už od mládí doufám.
#71:6 Na tebe jsem odkázán už ze života matky, oddělil sis mě v matčině nitru, chvalozpěv můj o tobě bude znít stále.
#71:7 Za zázrak mě mnozí měli, tys byl moje mocné útočiště.
#71:8 Z mých úst plně zní tvá chvála, den co den tě oslavuji.
#71:9 Nezamítej mě v čas stáří, neopouštěj mě, když pozbývám sil.
#71:10 Nepřátelé se už na mě domlouvají, ti, kdo mě hlídají, společně se radí,
#71:11 prohlašují: „Bůh ho opustil, pusťte se za ním, chopte se ho, nevysvobodí ho nikdo.“
#71:12 Nevzdaluj se ode mne, můj Bože, Bože, na pomoc mi pospěš!
#71:13 Ať zahynou v hanbě, kdo mě osočují, potupa a stud ať halí ty, kdo zlo mi strojí.
#71:14 Já však budu vytrvale čekat a vždy víc a nade vše tě budu chválit.
#71:15 O tvé spravedlnosti budou má ústa vypravovat, každého dne svědčit o tvé spáse, a přece nestačím všechno vypovědět.
#71:16 Přicházím s bohatýrskými činy Panovníka Hospodina; tvoji spravedlnost, jenom tvoji, budu připomínat.
#71:17 Od mladosti, Bože, jsi mě vyučoval a já dosud oznamuji tvoje divy.
#71:18 Ani ve stáří a šedinách mě, Bože, neopouštěj, dokud neseznámím toto pokolení se skutky své paže a s tvou bohatýrskou silou všechny, kteří přijdou.
#71:19 Bože, tvoje spravedlnost až k výšinám sahá, vykonals veliké věci. Bože, kdo je tobě roven!
#71:20 Ty jsi mi dal zakusit četná zlá soužení a zase mi život vracíš a z propasti země přivádíš mě nazpět.
#71:21 Ty mě stále činíš větším, útěchou mě zahrnuješ.
#71:22 A já strunným nástrojem ti budu vzdávat chválu, Bože můj, za tvoji věrnost, s citarou ti budu zpívat žalmy, Svatý Izraele.
#71:23 Ať plesají mé rty, když ti zpívám žalmy, i má duše tebou vykoupená.
#71:24 A můj jazyk bude o tvé spravedlnosti hovořit každodenně, budou se rdít hanbou ti, kdo mi strojili zlé věci. 
#72:1 Šalomounův. Bože, předej své soudy králi a svou spravedlnost královskému synu,
#72:2 aby obhajoval tvůj lid spravedlivě a tvé ponížené podle práva.
#72:3 Hory přinesou lidu pokoj a pahorky spravedlnost.
#72:4 Zjedná právo poníženým z lidu, dá zvítězit synům ubožáka, zdeptá utlačovatele.
#72:5 Budou se tě bát, dokud bude slunce, dokud potrvá měsíc, po všechna pokolení.
#72:6 Sestoupí jak déšť na posečenou louku, jako vláha svlažující zemi.
#72:7 V jeho dnech rozkvete spravedlivý, bude hojný pokoj, dokud nezanikne měsíc.
#72:8 A panovat bude od moře až k moři, od Řeky do dálav země.
#72:9 Divá sběř se před ním bude krčit, prach budou lízat jeho nepřátelé.
#72:10 Králové Taršíše a ostrovů přinesou dary, budou odvádět daň králové Šeby a Seby.
#72:11 Všichni králové se mu budou klanět, všechny národy mu budou sloužit.
#72:12 Vysvobodí ubožáka, jenž volá o pomoc, poníženého, jenž nemá pomocníka.
#72:13 Bude mít soucit s nuzným ubožákem, ubohým zachrání život.
#72:14 Vykoupí je z útisku a od násilí, jejich krev mu bude drahocená.
#72:15 Ať je živ! Budou mu dávat zlato z Šeby, ustavičně se za něj modlit, žehnat mu neustále.
#72:16 Země bude oplývat obilím, jež se bude vlnit i po vrcholech hor, jeho klasy budou jako libanónské cedry, obyvatelé měst pokvetou jak bylina země.
#72:17 Jeho jméno bude věčné, dokud potrvá slunce, poroste jeho jméno. Budou si jím žehnat, všechny národy ho budou blahoslavit.
#72:18 Buď požehnán Bůh Hospodin, Bůh Izraele; jedině on koná divy.
#72:19 Buď navěky požehnáno jeho slavné jméno, celou zemi nechť naplní jeho sláva! Amen, amen.
#72:20 Končí modlitby Davida, syna Jišajova. 
#73:1 Žalm. Pro Asafa. Jak je Bůh dobrý k Izraeli, k těm, kdo jsou čistého srdce!
#73:2 Avšak moje nohy málem odbočily, moje kroky téměř sešly z cesty,
#73:3 neboť jsem záviděl potřeštěncům, když jsem viděl svévolné, jak pokojně si žijí.
#73:4 Smrt je do okovů ještě nesevřela, jejich tělo kypí,
#73:5 nevědí, co je to lidské plahočení, nebývají postiženi jako jiní lidé.
#73:6 Jejich náhrdelníkem je zpupnost, násilnictví šatem, do něhož se halí.
#73:7 Jejich oko vystupuje z tuku, provaluje se smýšlení srdce.
#73:8 Vysmívají se a mluví zlomyslně, povýšenou řečí utiskují druhé.
#73:9 Do úst nebesa si berou, jazykem prosmýčí zemi.
#73:10 A lid se za nimi hrne lokat vodu plným douškem.
#73:11 Říkávají: „Což se to Bůh dozví? Cožpak to Nejvyšší pozná?“
#73:12 Ano, to jsou svévolníci: bez starostí věčně kupí jmění.
#73:13 Tedy zbytečně jsem si uchoval ryzí srdce a dlaně omýval nevinností?
#73:14 Každý den se na mě sypou rány, každé ráno bývám trestán.
#73:15 Kdybych řekl: „Budu mluvit jako oni“, věrolomně bych opustil pokolení tvých synů.
#73:16 Přemýšlel jsem, jak se v tom všem vyznat, nesnadné se mi to zdálo.
#73:17 Teprv když jsem vstoupil do svatyně Boží, pochopil jsem, jaký vezmou konec.
#73:18 Věru, stavíš je na kluzké cesty, do zkázy je srazíš.
#73:19 Jaký úděs náhle vzbudí, hrůzou se obrátí vniveč do jednoho.
#73:20 Jako snem po procitnutí, Panovníku, pohrdneš jejich přeludem, až se vzbudíš.
#73:21 Když bylo mé srdce roztrpčené, když se jitřilo mé ledví,
#73:22 byl jsem tupec, nic jsem neznal, jak dobytče jsem před tebou býval.
#73:23 Já však chci být ustavičně s tebou, uchopils mě za pravici,
#73:24 povedeš mě podle svého rozhodnutí a pak do slávy mě přijmeš.
#73:25 Koho bych měl na nebesích? A na zemi v nikom kromě tebe nemám zalíbení.
#73:26 Ač mé tělo i mé srdce chřadne, Bůh bude navěky skála mého srdce a můj podíl.
#73:27 Hle, ti, kdo se tobě vzdálí, zhynou. Ty umlčíš každého, kdo poruší ti věrnost.
#73:28 Mně však v Boží blízkosti je dobře, v Panovníku Hospodinu mám své útočiště, proto vyprávím o všech tvých činech. 
#74:1 Poučující, pro Asafa. Bože, zanevřel jsi na nás natrvalo? Tvůj hněv dýmá proti ovcím, které paseš.
#74:2 Rozpomeň se na svou pospolitost, kterou jsi před věky získal, na svůj dědičný kmen, jejž sis vykoupil, na horu Sijón, kde bydlíš.
#74:3 Zaměř kroky vzhůru k trvající spoušti, ve svatyni nepřítel vše zničil.
#74:4 Ve tvém shromáždění zněl řev protivníků, svá vítězná znamení tu postavili.
#74:5 Ví se, že tak, jako když se zvedá širočina vzhůru v spleti stromů,
#74:6 nyní otloukali řezby ve svatyni sekerou a mlatem.
#74:7 Tvou svatyni podpálili a příbytek tvého jména znesvětili, strhli k zemi.
#74:8 V srdci si řekli: „Jejich rod úplně vyhubíme!“ Vypálili všechna místa Božích shromáždění v zemi.
#74:9 Svá znamení nevidíme, proroka už není, neví nikdo z nás, co přijde.
#74:10 Jak dlouho se, Bože, smí protivník rouhat? Smí nepřítel znevažovat trvale tvé jméno?
#74:11 Proč stahuješ ruku nazpět? Zvedni svou pravici z klína, skoncuj s nimi!
#74:12 Bůh je můj Král odedávna, on uprostřed země koná spásné činy.
#74:13 Svou mocí jsi rozkymácel moře, drakům na vodách roztříštils hlavy,
#74:14 rozdrtil jsi hlavy livjátana, dals ho sežrat hordě divé sběře,
#74:15 rozpoltil jsi skálu, vytryskl pramen i potok, vysušil jsi mocné říční toky.
#74:16 Tobě patří den, i noc je tvoje, tys upevnil světlo noci i slunce,
#74:17 ty sám jsi vytyčil veškerá pomezí země, vytvořils léto i zimu.
#74:18 Hospodine, rozpomeň se na rouhání nepřítele, na zbloudilý lid, jenž znevážil tvé jméno.
#74:19 Nevydávej život své hrdličky dravci, na život svých ponížených nikdy nezapomeň,
#74:20 přihlédni ke smlouvě! Plno doupat násilí je v temných koutech země.
#74:21 Kéž zdeptaný není znovu tupen, ponížený ubožák ať chválí tvoje jméno!
#74:22 Povstaň, Bože, a své pře se ujmi, rozpomeň se, že ti bloud utrhá denně.
#74:23 Nezapomeň na křik protivníků, na ten ustavičně stoupající hukot útočníků! 
#75:1 Pro předního zpěváka, jako: „Nevyhlazuj!“ Žalm; pro Asafa, píseň.
#75:2 Vzdáváme ti chválu, Bože, vzdáváme ti chválu! Tvé jméno je blízko, vypráví se o tvých divech.
#75:3 „Já určím tu chvíli, kdy vykonám soud dle práva.
#75:4 Země se rozplyne se všemi, kdo na ní sídlí; já jsem to, kdo dává pevnost jejím sloupům.“
#75:5 Potřeštěncům říkám: Nechte ztřeštěností! Říkám svévolníkům: Nezvedejte rohy!
#75:6 Nezvedejte svoje rohy vzhůru, nemluvte s tak drzou šíjí!
#75:7 Nikdo od východu, nikdo od západu, nikdo od hornaté pouště,
#75:8 jenom Bůh je soudce: jednoho poníží, druhého povýší.
#75:9 Hospodin má v ruce kalich: víno kvasí, je plné příměsků, z něho nalévá a vypijí je i s kaly až do dna všichni svévolníci země.
#75:10 A já to budu navěky hlásat, budu zpívat žalmy Jákobovu Bohu.
#75:11 A všem svévolníkům srazím rohy, rohy spravedlivého se zvednou. 
#76:1 Pro předního zpěváka. Za doprovodu strunných nástrojů. Žalm pro Asafa. Píseň.
#76:2 V judské zemi je znám Bůh, má velké jméno v Izraeli.
#76:3 V Šálemu je jeho stánek, jeho obydlí je na Sijónu.
#76:4 Roztříštil tam střely ohnivé i s lukem, štíty, meče, válku.
#76:5 Vznešený, jak jasně záříš nad horami rozsápaných!
#76:6 Zarputilci za kořist ti padli, dřímají svůj sen; všem válečníkům ochrnuly ruce.
#76:7 Pohrozils, Jákobův Bože, a mrákota obestřela válečné vozy i koně.
#76:8 Ty vzbuzuješ bázeň. Kdo před tebou obstojí, když vzplaneš hněvem?
#76:9 Svou při jsi vyhlásil z nebe. Země žila hned v bázni a v míru,
#76:10 když Bůh povstal k soudu pro záchranu všech pokorných v zemi.
#76:11 Rozhořčením stíháš lidi a je ti to k chvále, po svém rozhořčení opášeš se pozůstatkem lidu.
#76:12 Učiňte slib a splňte jej Hospodinu, svému Bohu! Všichni kolem něho ať přinesou poctu tomu, jenž vzbuzuje bázeň,
#76:13 tomu, který krotí ducha vojevůdců a vzbuzuje bázeň v králích země. 
#77:1 Pro předního zpěváka, podle Jedútúna. Pro Asafa, žalm.
#77:2 Úpěnlivě volám k Bohu, volám k Bohu, kéž mi přeje sluchu!
#77:3 V den svého soužení se dotazuji Panovníka, v noci k němu vztahuji své ruce bez umdlení, má duše se utěšit nedá.
#77:4 Připomínám si Boha a sténám, přemítám a jsem na duchu skleslý.
#77:5 Nutíš moje oči k bdění, jsem vzrušen a nejsem schopen slova.
#77:6 Myslím na dny dávnověké, na pradávná léta.
#77:7 V noci si připomínám, jak jsem na struny hrával, v srdci přemítám a duch můj hloubá:
#77:8 Zatvrdil se Panovník na věky věků? Nikdy už svou přízeň neprojeví?
#77:9 Je snad jeho milosrdenství pryč natrvalo? Nebude už mluvit v dalších pokoleních?
#77:10 Což Bůh zapomněl na smilování? V hněvu uzavřel zdroj svého slitování?
#77:11 Pravím: V tom tkví moje bolest, že se změnila pravice Nejvyššího.
#77:12 Skutky Hospodinovy si připomínám, připomínám si onen tvůj div z dávnověku.
#77:13 Rozjímám o všech tvých činech a přemítám o tvých skutcích.
#77:14 Bože, tvá cesta je svatá. Který bůh je velký jako Bůh náš?
#77:15 Ty jsi Bůh, jenž činí divy! Svoji moc jsi dal národům poznat.
#77:16 Svůj lid jsi vykoupil vlastní paží, syny Jákobovy, Josefovy.
#77:17 Spatřily tě vody, Bože, vody tě spatřily a svíjely se v křeči, ba i tůně propastné se hnuly.
#77:18 Z oblaků se lily proudy vod, hrom duněl v mračnech, rozletěly se tvé šípy.
#77:19 Přivalilo se tvé hromobití, nad světem se rozsvítily blesky, země se pohybovala a třásla.
#77:20 Tvá cesta šla mořem, množstvím vod tvá stezka, aniž bylo znát tvé stopy.
#77:21 Svůj lid jsi vedl jak stádce rukou Mojžíše a Árona. 
#78:1 Poučující. Pro Asafa. Lide můj, naslouchej učení mému, k slovům úst mých nakloň ucho,
#78:2 otevřu svá ústa ku přísloví, předložím hádanky dávnověké.
#78:3 Co jsme slýchali a o čem víme, to, co nám otcové vyprávěli,
#78:4 nebudeme tajit jejich synům. Budeme vyprávět budoucímu pokolení o Hospodinových chvályhodných činech, o mocných skutcích a divech, jež konal.
#78:5 Stanovil svědectví v Jákobovi, vydal zákon v Izraeli a přikázal našim otcům, aby s tím seznamovali své syny,
#78:6 aby o tom vědělo budoucí pokolení, synové, kteří se zrodí. Ti to budou dále vyprávět svým synům,
#78:7 aby složili důvěru v Boha a na Boží skutky nezapomínali, aby zachovávali vždy jeho přikázání,
#78:8 a nebyli jako jejich otcové, to umíněné, vzdorné pokolení, pokolení nestálého srdce, jehož duch nezůstal věrný Bohu.
#78:9 Lukem vyzbrojeni Efrajimci otočili se v den bitvy.
#78:10 Nedbali na smlouvu Boží, podle jeho zákona žít odepřeli,
#78:11 zapomněli na jeho skutky, na divy, jež jim dal shlédnout.
#78:12 On před jejich otci činil divy v zemi egyptské na Sóanském poli.
#78:13 Moře rozpoltil a převedl je, jako hráz postavil vody.
#78:14 Ve dne oblakem je vodil, po celou noc září ohně.
#78:15 Rozpoltil na poušti skály, dal jim pít hojně jak z propastných tůní,
#78:16 bystřiny vyvedl ze skalního štítu, nechal plynout vodstva jako řeky.
#78:17 Hřešili však proti němu opět. Vzdorovali v tom vyprahlém kraji Nejvyššímu,
#78:18 srdcem pokoušeli Boha, chtěli stravu podle vlastní vůle.
#78:19 A reptali proti Bohu: „Může Bůh prostřít stůl v poušti?
#78:20 Do skály sic udeřil a potoky vod tekly proudem. Může však dát také chleba nebo opatřit pro svůj lid maso?“
#78:21 Hospodin se rozlítil, když to vše slyšel. Oheň prudce vzplanul proti Jákobovi, hněv vyšlehl proti Izraeli,
#78:22 neboť nevěřili Bohu, nedoufali v jeho spásu.
#78:23 Přesto vydal příkaz mračnům shůry, zotevíral vrata nebes
#78:24 a jako déšť spouštěl na ně manu, aby jedli, nebeské obilí jim dával.
#78:25 Člověk jísti směl chléb mocných, stravu sesílal jim do sytosti.
#78:26 Dal na nebi vanout východnímu větru, jižní vítr přivedl svou mocí
#78:27 a déšť masa na ně spustil, tolik jako prachu, pernatého ptactva jak mořského písku,
#78:28 nechával je padat rovnou do tábora okolo příbytků jejich.
#78:29 Najedli se víc než do sytosti, dopřál jim to, čeho byli chtiví.
#78:30 Nepřešla je chtivost. Měli ještě pokrm v ústech,
#78:31 když proti nim vyšlehl hněv Boží. Zahubil jejich velmože, izraelské junáky v prach srazil.
#78:32 Přes to vše hřešili dále, nevěřili jeho divům.
#78:33 Tu učinil konec jejich dnům, že přešly jako vánek, jejich léta přeťal náhlým děsem.
#78:34 Kdykoli je hubil, dotazovali se po jeho vůli, za úsvitu hledávali opět Boha,
#78:35 připomínali si, že Bůh býval jejich skála, že Bůh nejvyšší byl jejich vykupitel.
#78:36 Ale klamali ho svými ústy, jazykem mu lhali,
#78:37 v svém srdci nestáli při něm, nezůstali věrni jeho smlouvě.
#78:38 Ale on se slitovával, zprošťoval je nepravostí, nevydal je zkáze, často odvrátil svůj hněv a nedal zcela procitnout svému rozhořčení,
#78:39 pamatoval, že jsou jenom tělo, vítr, který zavane a už se nevrací.
#78:40 Kolikrát mu v poušti vzdorovali, jakou trýzeň působili mu v tom pustém kraji!
#78:41 Zas a zase pokoušeli Boha, činili výčitky Svatému Izraele.
#78:42 Nepřipomínali jeho sílu, den, kdy je vykoupil z moci protivníka,
#78:43 když činil svá znamení v Egyptě a své zázraky na Sóanském poli.
#78:44 Ramena řek i bystřiny proměnil v krev, takže se nemohli napít.
#78:45 Seslal na ně mouchy, aby je sžíraly, žáby, které jim přinesly zkázu.
#78:46 Co se urodilo, vydal hmyzu, kobylkám plod jejich práce.
#78:47 Jejich révu zničil krupobitím a smokvoně mrazem.
#78:48 Jejich dobytek vydal v plen krupobití a ohnivým střelám jejich stáda.
#78:49 Tak je stíhal svým planoucím hněvem, hněvem hrozným, prchlivostí, souženími, řadou poslů zkázy.
#78:50 Dával průchod svému hněvu, jejich duše neušetřil smrti, vydal moru v plen vše živé.
#78:51 Všechno prvorozené v Egyptě pobil, prvotiny plodivé síly v Chámových stanech.
#78:52 Ale svůj lid vedl jako ovce, jako stádo je převedl pouští,
#78:53 vodil je bezpečně, strach mít nemuseli, jejich nepřátele pohltilo moře.
#78:54 Dovedl je až k území své svatyně, k této hoře, kterou pravicí svou získal.
#78:55 Vypudil před nimi pronárody, jejich zem rozměřil na dědictví, v jejich stanech ubytoval izraelské kmeny.
#78:56 Ale pokoušeli ho a vzdorovali Bohu nejvyššímu a nedbali na svědectví jeho,
#78:57 odpadali, byli věrolomní jak otcové jejich, selhávali jako záludný luk.
#78:58 Na posvátných návrších ho uráželi, svými modlami ho podnítili k žárlivosti.
#78:59 Bůh se rozlítil, když to vše slyšel, velmi si zprotivil Izraele:
#78:60 opovrhl svým příbytkem v Šílu, stanem, v němž přebýval mezi lidmi.
#78:61 Vydal do zajetí svoji sílu, svoji čest do rukou protivníka
#78:62 a svůj lid vydal v plen meči, proti svému dědictví vzplál prchlivostí.
#78:63 Jejich junáky sežehl oheň, jejich panny svatebních chval nepoznaly.
#78:64 Jejich kněží padli mečem, jejich vdovy neplakaly.
#78:65 A pak se Panovník probudil jak ze sna, jak bohatýr rozjařený vínem,
#78:66 zezadu bil svoje protivníky, uvedl na ně potupu věčnou.
#78:67 Zprotivil si Josefův stan, Efrajimův kmen si nevyvolil,
#78:68 zvolil si kmen Judův, horu Sijón, tu si zamiloval.
#78:69 Svou svatyni vybudoval jak výšiny nebes, navěky ji založil jak zemi.
#78:70 Davida si zvolil, svého služebníka, povolal ho ze salaší,
#78:71 od ovcí s jehňaty ho přivedl, aby pásl Jákoba, jeho lid, Izraele, jeho dědičný díl.
#78:72 A pásl je s bezúhonným srdcem, rukou zkušenou je vodil. 
#79:1 Žalm pro Asafa. Bože, vtrhly pronárody do dědictví tvého, tvůj svatý chrám poskvrnily, Jeruzalém obrátily v hromady sutin!
#79:2 Mrtvoly tvých služebníků daly za potravu nebeskému ptactvu, těla tobě věrných zemské zvěři.
#79:3 Prolévaly jejich krev okolo Jeruzaléma jak vodu, a nebyl, kdo by je pohřbil.
#79:4 Tupení svých sousedů jsme vystaveni, svému okolí jsme pro smích, pro pošklebky.
#79:5 Dlouho ještě, Hospodine? Chceš se pořád hněvat? Bude tvoje rozhorlení planout jako oheň?
#79:6 Vylej svoje rozhořčení na pronárody, jež neznají se k tobě, na království, která nevzývají tvoje jméno;
#79:7 vždyť pozřely Jákoba a zpustošily jeho nivy.
#79:8 Nepřipomínej nám staré nepravosti, pospěš nám vstříc se svým slitováním, jsme naprosto vyčerpáni.
#79:9 Bože, naše spáso, pomoz nám pro slávu svého jména, vysvoboď nás, zprosť nás hříchů pro své jméno!
#79:10 Proč by měly pronárody říkat: „Kde je ten jejich Bůh?“ Kéž by pronárody poznaly před naším zrakem pomstu za prolitou krev tvých služebníků.
#79:11 Kéž k tobě pronikne sténání vězňů! Mocnou paží svou zachovej syny smrti!
#79:12 Našim sousedům vrať sedminásobně do klína potupu, jíž potupili tebe, Panovníku.
#79:13 My, tvůj lid - ovce, jež paseš, budeme ti navěky, po všechna pokolení, vzdávat chválu a vyprávět o tvých chvályhodných činech. 
#80:1 Pro předního zpěváka. Podle „Lilií„. Svědectví pro Asafa. Žalm.
#80:2 Naslouchej, pastýři Izraele, ty, jenž Josefa jak ovce vodíš, zaskvěj se, jenž trůníš nad cheruby, před kmeny Efrajim, Benjamín, Manases!
#80:3 Vzbuď svou bohatýrskou sílu a přijď zachránit nás.
#80:4 Bože, obnov nás, ukaž jasnou tvář a budem zachráněni.
#80:5 Hospodine, Bože zástupů, jak dlouho budeš dýmat hněvem při modlitbách svého lidu?
#80:6 Chlebem slzí jsi je krmil, pít jsi jim dal slzy plnou měrou.
#80:7 Pro sousedy jsme předmětem sváru, nepřátelům pro smích.
#80:8 Bože zástupů, obnov nás, ukaž jasnou tvář a budem zachráněni!
#80:9 Vinnou révu z Egypta jsi vyňal, vypudil jsi pronárody a ji jsi zasadil.
#80:10 Připravil jsi pro ni všechno, zapustila kořeny a rozrostla se v zemi.
#80:11 Hory přikryla svým stínem, její ratolesti jsou jak Boží cedry,
#80:12 rozložila výhonky až k moři a své úponky až k Řece.
#80:13 Proč pobořils její zídky? Aby trhali z ní všichni, kdo jdou kolem?
#80:14 Rozrývá ji kanec z lesa a spásá ji polní havěť.
#80:15 Bože zástupů, navrať se, shlédni z nebe, popatř, ujmi se té révy,
#80:16 kmene, který pravice tvá zasadila, letorostu, jejž sis vypěstoval.
#80:17 Spálena je ohněm, porubána, zachmuřil jsi tvář a hynou.
#80:18 Nad mužem své pravice drž svoji ruku, nad synem člověka, letorostem, jejž sis vypěstoval.
#80:19 Nikdy se tě nespustíme, zachovej nám život, ať můžeme vzývat tvoje jméno.
#80:20 Hospodine, Bože zástupů, obnov nás, ukaž jasnou tvář a budem zachráněni! 
#81:1 Pro předního zpěváka, podle gatského způsobu. Pro Asafa.
#81:2 Plesejte Bohu, naší síle, hlaholte Bohu Jákobovu!
#81:3 Prozpěvujte žalmy, bijte v buben, hrejte na líbeznou citaru a harfu,
#81:4 při novoluní zatrubte na polnice, při úplňku v den našeho svátku.
#81:5 Takové je nařízení v Izraeli a řád Jákobova Boha.
#81:6 Svědectví uložil v Josefovi, když vytáhl na egyptskou zemi, kde jsem slýchal řeč, kterou jsem neznal.
#81:7 Jeho záda břemene jsem zbavil, i koš odložily jeho ruce.
#81:8 V soužení jsi volal a já jsem tě bránil, odpověděl jsem ti skryt v rachotu hromu, tam při Vodách sváru jsem tě zkoušel.
#81:9 Slyš, můj lide, svědčím proti tobě, kéž bys mě poslechl, Izraeli.
#81:10 Nesmíš mít cizího boha, nebudeš se klanět cizáckému bohu!
#81:11 Já jsem Hospodin, tvůj Bůh, já jsem tě přivedl z egyptské země. Otevři svá ústa, naplním je.
#81:12 Neuposlechl mě lid můj, Izrael mi povolný být nechtěl.
#81:13 Nechal jsem je tedy být s tím zarputilým srdcem, ať si jdou za svými plány.
#81:14 Kdyby mě však můj lid uposlechl, kdyby Izrael mými cestami chodil,
#81:15 brzy bych pokořil jeho nepřátele, obrátil bych ruku proti jeho protivníkům.
#81:16 Ti, kdo Hospodina nenávidí, vtírali by se do jeho přízně, věčně by trval čas jeho lidu,
#81:17 bělí pšeničnou by ho sám krmil. Medem ze skály tě budu sytit! 
#82:1 Žalm pro Asafa. V shromáždění bohů postavil se Bůh, vykoná soud mezi bohy:
#82:2 „Dlouho ještě chcete soudit proti právu, stranit svévolníkům?
#82:3 Dopomozte nuznému a sirotkovi k právu, poníženému a chudému zjednejte spravedlnost,
#82:4 pomozte vyváznout nuznému ubožáku, svévolným ho vytrhněte z rukou!“
#82:5 Nic nevědí, nic nechápou, chodí v tmách, a celá země v základech se hroutí.
#82:6 „Ač jsem řekl: Jste bohové, všichni jste synové Nejvyššího
#82:7 zemřete též jako jiní lidé, padnete tak jako každý vladař.“
#82:8 Bože, povstaň, rozsuď zemi, dědičně ti patří všechny pronárody! 
#83:1 Píseň. Žalm pro Asafa.
#83:2 Bože, nebuď zticha, nemlč, Bože, nezůstávej v klidu!
#83:3 Hleď, jak tvoji nepřátelé hlučí, kdo tě nenávidí, pozvedají hlavu,
#83:4 kují proti tvému lidu tajné plány, radí se proti těm, které skrýváš:
#83:5 „Pojďme,“ praví, „vyhlaďme je, ať národem nejsou, ať se nikdy jména Izraele ani nevzpomene!“
#83:6 Tak se spolu svorně uradili, uzavřeli proti tobě smlouvu
#83:7 stany edómské a Izmaelci, Hagrejci i Moáb,
#83:8 Gebal, Amón s Amálekem, Pelištea, ti, kdo sídlí v Týru,
#83:9 Ašúr se k nim také přidal. Paží synů Lotových se stali.
#83:10 Nalož s nimi jako s Midjánem, jako se Síserou, jako s Jabínem v Kíšonském úvalu:
#83:11 u Én-dóru byli vyhlazeni, mrvou na roli se stali.
#83:12 Nalož s jejich knížaty jak s Orébem a Zébem, s všemi jejich vojevůdci jako se Zebachem, se Salmunou,
#83:13 kteří řekli: „Zaberme si Boží nivy!“
#83:14 Bože můj, dej, ať jsou jako chmýří, jako stéblo unášené větrem!
#83:15 Jako požár, když stravuje lesy, jako plameny, když sežehují hory,
#83:16 tak je postihni svou bouří, svou smrští je vyděs.
#83:17 Pohanou jim pokryj líce, aby hledali tvé jméno, Hospodine.
#83:18 Ať jsou zahanbeni, vyděšeni navždy, ať se rdí a hynou,
#83:19 ať poznají, že ty jediný, jenž Hospodin máš jméno, jsi ten nejvyšší nad celou zemí! 
#84:1 Pro předního zpěváka. Podle gatského způsobu. Pro Kórachovce, žalm.
#84:2 Jak je tvůj příbytek milý, Hospodine zástupů!
#84:3 Má duše zmírá steskem po Hospodinových nádvořích, mé srdce i mé tělo plesají vstříc živému Bohu!
#84:4 Vždyť i vrabec přístřeší si najde, vlaštovka si staví hnízdo u tvých oltářů, aby svá mláďata zde uložila, Hospodine zástupů, můj Králi a můj Bože!
#84:5 Blaze těm, kdo bydlí ve tvém domě, mohou tě zde vždycky chválit.
#84:6 Blaze člověku, jenž sílu hledá v tobě, těm, kteří se vydávají na pouť.
#84:7 Když Dolinou balzámovníků se ubírají, učiní ji prameništěm, včasný déšť ji halí požehnáním.
#84:8 Pokračují stále s novou silou, objeví se před svým Bohem na Sijónu.
#84:9 Hospodine, Bože zástupů, slyš modlitbu mou, naslouchej, Jákobův Bože!
#84:10 Ty jsi štít náš, Bože, pohleď, na tvář svého pomazaného rač shlédnout.
#84:11 Den v tvých nádvořích je lepší než tisíce jinde; raději chci stát před prahem domu svého Boha, než prodlévat v stanech svévolnosti,
#84:12 vždyť Hospodin Bůh je štít a slunce, Hospodin je dárce milosti a slávy, žádné dobro neodepře těm, kdo žijí bezúhonně.
#84:13 Hospodine zástupů, blaze člověku, jenž doufá v tebe! 
#85:1 Pro předního zpěváka. Pro Kórachovce, žalm.
#85:2 Hospodine, projevoval jsi své zemi přízeň, úděl Jákobův jsi změnil.
#85:3 Snímal jsi ze svého lidu nepravosti, přikrýval jsi všechny jeho hříchy.
#85:4 Svou prchlivost odkládal jsi zcela, přestával jsi planout hněvem.
#85:5 Navrať se k nám, Bože, naše spáso, učiň konec svému rozlícení!
#85:6 Chceš se na nás hněvat věčně? Stíhat hněvem všechna pokolení?
#85:7 Což nám nenavrátíš život, aby se tvůj lid z tebe radoval?
#85:8 Ukaž nám, Hospodine, své milosrdenství, uděl nám svou spásu!
#85:9 Vyslechnu, co promluví Bůh Hospodin, zajisté vyhlásí pokoj pro svůj lid, své věrné, jenom ať se k své hlouposti nevracejí!
#85:10 Ano, jeho spása je blízko těm, kdo se ho bojí, v naší zemi bude přebývat sláva.
#85:11 Setkají se milosrdenství a věrnost, spravedlnost s pokojem si dají políbení.
#85:12 Ze země vyraší pravda, z nebe bude shlížet spravedlnost.
#85:13 Hospodin dopřeje dobrých časů, svou úrodu vydá naše země.
#85:14 Před ním půjde spravedlnost a on bude kráčet její cestou. 
#86:1 Modlitba Davidova. Hospodine, nakloň ucho, odpověz mi, jsem tak ponížený, zubožený.
#86:2 Ochraňuj mě, jsem tvůj věrný, spas, můj Bože, svého služebníka, který v tebe doufá.
#86:3 Smiluj se nade mnou, Panovníku, po celé dny k tobě volám.
#86:4 Vlej do duše svého služebníka radost, k tobě, Panovníku, pozvedám svou duši,
#86:5 neboť ty jsi, Panovníku, dobrý a nabízíš odpuštění; ke všem, kdo tě volají, jsi nejvýš milosrdný.
#86:6 Dopřej, Hospodine, mé modlitbě sluchu, věnuj pozornost mým prosbám.
#86:7 V den svého soužení volám k tobě a ty mi odpovíš.
#86:8 Panovníku, není ti rovného mezi bohy a tvým činům se nic nevyrovná.
#86:9 Všechny pronárody, tvoje dílo, se ti přijdou klanět, Panovníku, budou oslavovat tvoje jméno,
#86:10 protože jsi veliký a konáš divy; jedině ty jsi Bůh.
#86:11 Hospodine, ukaž mi svou cestu, budu žít podle tvé pravdy, soustřeď mou mysl na bázeň tvého jména.
#86:12 Celým srdcem, Panovníku, Bože můj, ti budu vzdávat chválu, tvoje jméno věčně oslavovat,
#86:13 vždyť tvé velké milosrdenství je se mnou; z nejhlubšího podsvětí jsi vytrhl mou duši.
#86:14 Bože, povstávají proti mně opovážlivci, o život mi ukládá smečka ukrutníků, na tebe se neohlíží.
#86:15 Ty však, Panovníku, jsi Bůh slitovný a milostivý, shovívavý, nejvýš milosrdný, věrný.
#86:16 Shlédni na mne, smiluj se nade mnou, dej svou sílu svému služebníku, vítězství synu své služebnice!
#86:17 Ukaž na mně dobré znamení, ať to vidí, kdo mě mají v nenávisti, ať jsou zahanbeni, že jsi, Hospodine, moje pomoc, moje potěšení! 
#87:1 Pro Kórachovce, zpívaný žalm. Na posvátných horách založil svůj chrám.
#87:2 Hospodin miluje brány Sijónu nad všechny příbytky Jákobovy.
#87:3 Slavné věci hlásány jsou v tobě, město Boží!
#87:4 Připomenu Rahaba i Bábel jakožto ty, kdo mě znají, také Pelišteu a Týr s Kúšem, každý z nich se zrodil tady.
#87:5 O Sijónu řeknou: V něm se zrodil ten i onen, a sám Nejvyšší jej činí pevným.
#87:6 Při soupisu lidských společenství Hospodin si zaznamená: Každý z nich se zrodil tady.
#87:7 A dají se do zpěvu a tance: Všechna spása pramení mi z tebe! 
#88:1 Žalmová píseň, pro Kórachovce. Pro předního zpěváka, pro zpěv při tanečním reji. Poučující, pro Hémana Ezrachejského.
#88:2 Hospodine, Bože, má spáso, ve dne i v noci před tebou úpím;
#88:3 kéž vstoupí moje modlitba k tobě, nakloň ucho k mému bědování!
#88:4 Samým zlem se sytí moje duše, k podsvětí spěje můj život.
#88:5 Jsem počítán k těm, co sestupují v jámu, jsem jako muž, který pozbyl síly.
#88:6 Jsem odložen mezi mrtvé jako skolení, co leží v hrobech, na které už ani nevzpomeneš, kteří jsou zbaveni ochrany tvých rukou.
#88:7 Odložils mě do nejhlubší jámy, do temnoty hlubin.
#88:8 Leží na mně tíha tvého rozhořčení, všemi svými příboji mě trýzníš.
#88:9 Známé jsi mi vzdálil, zhnusils mě jim, jsem uvězněn, nemám východiska.
#88:10 Oči mám zkalené utrpením, po celé dny jsem tě, Hospodine, volal, vztahoval jsem k tobě dlaně.
#88:11 Což pro mrtvé budeš konat svoje divy? Povstanou snad stíny a vzdají ti chválu?
#88:12 Což se o tvém milosrdenství vypráví v hrobě? O tvé věrnosti v říši zkázy?
#88:13 Což jsou známy v temnotě tvé divy, tvoje spravedlnost v zemi zapomnění?
#88:14 Hospodine, pomoz, k tobě volám, ráno má modlitba přichází k tobě.
#88:15 Hospodine, proč jsi na mě zanevřel a svou tvář přede mnou skrýváš?
#88:16 Jsem ponížen a od mládí hynu, snáším tvé hrůzy, nevím si rady.
#88:17 Přehnala se přese mne výheň tvého hněvu, děsuplné rány tvé mě umlčely,
#88:18 ze všech stran se na mě denně hrnou jako vody, všechny najednou mě obkličují.
#88:19 Přítele a druha jsi mi vzdálil, jenom temnoty se ke mně znají! 
#89:1 Poučující, pro Étana Ezrachejského.
#89:2 O Hospodinově milosrdenství chci zpívat věčně, svými ústy v známost uvádět tvou věrnost po všechna pokolení.
#89:3 Pravím: Tvoje milosrdenství je zbudováno navěky, v nebesích jsi pevně založil svou věrnost,
#89:4 řekls: „Uzavřel jsem smlouvu se svým vyvoleným, přísahal jsem Davidovi, svému služebníku:
#89:5 Dám, že tvé potomstvo bude navěky stát pevně, já jsem zbudoval tvůj trůn pro všechna pokolení.“
#89:6 Nebesa ti, Hospodine, vzdávají za tvůj div chválu, shromáždění svatých velebí tvou věrnost.
#89:7 Vždyť kdo v oblacích se může Hospodinu rovnat? Kdo z Božích synů se podobá Hospodinu?
#89:8 Strašný je Bůh v kruhu svatých, nesmírný, budící bázeň ve všech kolem sebe.
#89:9 Hospodine, Bože zástupů, kdo je jako ty? Jsi nepřemožitelný, Hospodine, a tvá věrnost tě provází všudy.
#89:10 Umíš zvládnout zpupné moře, zkrotit jeho vzduté vlny.
#89:11 Netvora jsi zdeptal, jako bys ho proklál, mocnou paží jsi rozprášil svoje nepřátele.
#89:12 Tvá jsou nebesa, tvá je i země, založil jsi svět a všechno, co je na něm.
#89:13 Ty jsi stvořil jih i sever, i Tábor a Chermón plesají v tvém jménu.
#89:14 V své paži máš bohatýrskou sílu, tvá ruka je mocná, tvá pravice pozdvižená.
#89:15 Spravedlnost a právo jsou pilíře tvého trůnu, před tebou jde milosrdenství a věrnost.
#89:16 Blaze lidu, který zná vítězný hlahol, který chodí ve světle tvé tváře, Hospodine.
#89:17 Budou jásat každý den v tvém jménu, pozvedá je tvoje spravedlnost,
#89:18 neboť ty jsi leskem jejich moci; tvou přízní se náš roh zvedá.
#89:19 Štít náš patří Hospodinu, náš král Svatému Izraele.
#89:20 Ve vidění promluvil jsi jednou ke svým věrným. Řekls: „Poskytl jsem bohatýru pomoc, povýšil jsem vybraného z lidu.
#89:21 Nalezl jsem Davida, svého služebníka, pomazal jsem ho svým olejem svatým.
#89:22 Pevně bude moje ruka při něm, udatným ho učiní má paže.
#89:23 Nepřítel ho nepřekvapí a bídák ho nepokoří.
#89:24 Před ním potřu jeho protivníky, porazím ty, kdo ho nenávidí.
#89:25 Bude s ním mé milosrdenství a věrnost, jeho roh se pozvedne v mém jménu.
#89:26 I na moře vložím jeho ruku a jeho pravici na řeky.
#89:27 On mě bude vzývat: Tys můj Otec, můj Bůh, moje spásná skála
#89:28 a já ho učiním prvorozeným, nejvyšším nad králi země.
#89:29 Svoje milosrdenství mu zachovám věčně, věrně dodržím svou smlouvu.
#89:30 Jeho potomstvu dám trvat navěky, jeho trůnu po všechny dny nebes.
#89:31 Opustí-li jeho synové můj zákon, nebudou-li žít podle mých řádů,
#89:32 budou-li má nařízení znesvěcovat, nebudou-li dbát mých přikázání,
#89:33 ztrestám jejich nevěrnosti metlou a ranami jejich nepravosti,
#89:34 ale svoje milosrdenství mu neodejmu, nezradím svou věrnost,
#89:35 neznesvětím svoji smlouvu, nezměním, co splynulo mi ze rtů.
#89:36 Jednou jsem přísahal na svou svatost. Cožpak bych lhal Davidovi?
#89:37 Jeho potomstvo potrvá navěky, i jeho trůn přede mnou bude jako slunce,
#89:38 pevně bude stát navěky jako měsíc, věrný svědek nad oblaky.“
#89:39 Zanevřels však na něj, zavrhls ho, na svého pomazaného ses rozlítil.
#89:40 Zrušil jsi smlouvu se svým služebníkem, jeho čelenku jsi znesvětil, srazil k zemi.
#89:41 Všechny jeho zdi jsi zbořil, jeho pevnosti obrátils v trosky.
#89:42 Plení ho všichni, kdo táhnou kolem, tupení svých sousedů je vydán.
#89:43 Pozvedls pravici jeho protivníků, učinils všem jeho nepřátelům radost.
#89:44 Otupil jsi ostří jeho meče, nedal jsi mu obstát v boji.
#89:45 Posvátné čistoty jsi ho zbavil, jeho trůn jsi skácel na zem.
#89:46 Zkrátil jsi dny jeho mládí, pokryl jsi ho hanbou.
#89:47 Dlouho ještě, Hospodine? Skryl ses natrvalo? Bude tvoje rozhořčení sálat jako oheň?
#89:48 Pamatuj, co jsem. Vždyť co je lidský věk? Což šalebně jsi stvořil všechny lidi?
#89:49 Který muž kdy žil a nedočkal se smrti? Z moci podsvětí kdo zachránil se?
#89:50 Kde zůstalo někdejší tvé milosrdenství, ó Panovníku, jež jsi odpřisáhl Davidovi na svou věrnost?
#89:51 Panovníku, připomeň si potupu svých služebníků, kterou od všech možných národů v svém klíně nosím.
#89:52 Tvoji nepřátelé, Hospodine, potupili, potupili každou stopu po tvém pomazaném!
#89:53 Požehnán buď Hospodin navěky. Amen, amen. 
#90:1 Modlitba Mojžíše, muže Božího. Panovníku, u tebe jsme měli domov v každém pokolení!
#90:2 Než se hory zrodily, než vznikl svět a země, od věků na věky jsi ty, Bože.
#90:3 Ty člověka v prach obracíš, pravíš: „Zpět, synové Adamovi!“
#90:4 Tisíc let je ve tvých očích jako včerejšek, jenž minul, jako jedna noční hlídka.
#90:5 Jako povodeň je smeteš, prchnou jako spánek, jsou jak tráva, která odkvétá hned ráno:
#90:6 zrána rozkvete a už odkvétá, večer uvadne a uschne.
#90:7 Pro tvůj hněv spějeme k svému konci zděšeni tvým rozhořčením.
#90:8 Před sebe si kladeš naše nepravosti, do světla své tváře naše tajné hříchy.
#90:9 Pro tvou prchlivost naše dny pomíjejí a jako vzdech doznívají naše léta.
#90:10 Počet našich let je sedmdesát roků, jsme-li při síle, pak osmdesát, a mohou se pyšnit leda trápením a ničemnostmi; kvapem uplynou a v letu odcházíme.
#90:11 Kdo zná sílu tvého hněvu, tvou prchlivost, jak by se tě nebál?
#90:12 Nauč nás počítat naše dny, ať získáme moudrost srdce.
#90:13 Vrať se, Hospodine! Ještě dlouho se chceš hněvat? Měj se svými služebníky soucit,
#90:14 nasyť nás svým milosrdenstvím hned ráno a po všechny dny se budeme radovat a plesat.
#90:15 Tolik radosti nám dopřej, kolik bylo dnů, v nichž jsi nás pokořoval, a let, v nichž se nám zle vedlo.
#90:16 Nechť se na tvých služebnících ukáže tvé dílo a tvá důstojnost na jejich synech!
#90:17 Vlídnost Panovníka, Boha našeho, buď s námi. Upevni nám dílo našich rukou, dílo našich rukou učiň pevným! 
#91:1 Kdo v úkrytu Nejvyššího bydlí, přečká noc ve stínu Všemocného.
#91:2 Říkám o Hospodinu: „Mé útočiště, má pevná tvrz je můj Bůh, v nějž doufám.“
#91:3 Vysvobodí tě z osidla lovce, ze zhoubného moru.
#91:4 Přikryje tě svými perutěmi, pod jeho křídly máš útočiště; pavézou a krytem je ti jeho věrnost.
#91:5 Nelekej se hrůzy noci ani šípu, který létá ve dne,
#91:6 moru, jenž se plíží temnotami, nákazy, jež šíří zhoubu za poledne.
#91:7 Byť jich po tvém boku padlo tisíc, byť i deset tisíc tobě po pravici, tebe nestihne nic takového.
#91:8 Na vlastní oči to spatříš, uzříš odplatu, jež stihne svévolníky.
#91:9 Máš-li útočiště v Hospodinu, u Nejvyššího svůj domov,
#91:10 nestane se ti nic zlého, pohroma se k tvému stanu nepřiblíží.
#91:11 On svým andělům vydal o tobě příkaz, aby tě chránili na všech tvých cestách.
#91:12 Na rukou tě budou nosit, aby sis o kámen nohu neporanil;
#91:13 po lvu a po zmiji šlapat budeš, pošlapeš lvíče i draka.
#91:14 Dám mu vyváznout, neboť je mi oddán, budu jeho hradem, on zná moje jméno.
#91:15 Až mě bude volat, odpovím mu, v soužení s ním budu, ubráním ho, obdařím ho slávou,
#91:16 dlouhých let dopřeji mu do sytosti, ukáži mu svoji spásu. 
#92:1 Zpívaný žalm. Píseň ke dni odpočinku.
#92:2 Jak dobré je vzdávat Hospodinu chválu, tvému jménu, Nejvyšší, pět žalmy,
#92:3 hlásat zrána tvoje milosrdenství a v noci tvou věrnost
#92:4 při nástroji o deseti strunách, s harfou, při hře na citaru.
#92:5 Hospodine, svými skutky působíš mi radost, plesám nad činy tvých rukou:
#92:6 Tvoje činy, Hospodine, jsou tak velkolepé, tvoje záměry jsou přehluboké!
#92:7 Tupec o tom neví, hlupák tomu nerozumí.
#92:8 Svévolníci bují jako plevel, všichni pachatelé ničemností rozkvétají, aby byli navždy vyhlazeni.
#92:9 Ty však, Hospodine, jsi navěky vyvýšený.
#92:10 Ano, tvoji nepřátelé, Hospodine, ano, tvoji nepřátelé zhynou, všichni pachatelé ničemností budou rozprášeni.
#92:11 Můj roh jsi však vyvýšil jako roh jednorožce, olej nejčistší jsi na mne vylil.
#92:12 Moje oko shlíží na ty, kdo proti mně sočí, moje uši slyší o zlovolných útočnících.
#92:13 Spravedlivý roste jako palma, rozrůstá se jako libanónský cedr.
#92:14 Ti, kdo v domě Hospodinově jsou zasazeni, kdo rostou v nádvořích našeho Boha,
#92:15 ještě v šedinách ponesou plody, zůstanou statní a svěží,
#92:16 aby hlásali, že Hospodin je přímý, skála má, a podlosti v něm není! 
#93:1 Hospodin kraluje! Oděl se důstojností. Oděl se Hospodin, opásal se mocí. Pevně je založen svět, nic jím neotřese.
#93:2 Tvůj trůn pevně stojí odedávna. Ty jsi od věčnosti.
#93:3 Zvedají řeky, Hospodine, zvedají řeky svůj hlas, zvedají řeky své vlnobití.
#93:4 Nad hukot mohutných vodstev, nad vznosné příboje mořské je vznešenější Hospodin na výšině.
#93:5 Tvá svědectví jsou naprosto věrná. A tvému domu přísluší svatost do nejdelších časů, Hospodine! 
#94:1 Bože mstiteli, Hospodine, Bože mstiteli, zaskvěj se!
#94:2 Povznes se výše, ty soudce země, dej odplatu pyšným.
#94:3 Dlouho ještě svévolníci, Hospodine, dlouho ještě svévolníci budou jásat?
#94:4 Chrlí drzé řeči, chvástají se všichni pachatelé ničemností
#94:5 a deptají tvůj lid, Hospodine, činí příkoří dědictví tvému,
#94:6 vraždí vdovu a bezdomovce, sirotky zabíjejí.
#94:7 Říkají: „Hospodin nevidí, Bůh Jákobův to nepostřehne.“
#94:8 Přijdete na to, vy tupci z lidu, hlupáci, pochopíte to někdy?
#94:9 Neslyší snad ten, jenž učinil ucho? Nedívá se snad ten, jenž vytvořil oko?
#94:10 Neumí snad trestat ten, jenž kárá pronárody, ten, jenž učí člověka, co by měl vědět?
#94:11 Hospodin zná smýšlení lidí; jsou pouhý vánek.
#94:12 Blaze muži, jehož, Hospodine, káráš, jehož svým zákonem vyučuješ:
#94:13 Dopřeješ mu klidu ve zlých dnech, zatímco se bude kopat jáma svévolníku.
#94:14 Vždyť Hospodin lid svůj neodvrhne, své dědictví neopustí.
#94:15 Na soudu opět zavládne spravedlnost, půjdou za ní všichni, kteří mají přímé srdce.
#94:16 Kdo se mne zastane proti zlovolníkům? Kdo se za mne postaví proti pachatelům ničemností?
#94:17 Kdyby mi Hospodin nepomáhal, zakrátko bych bydlel v říši ticha.
#94:18 Řeknu-li: „Už ujíždí mi noha“, podepře mě tvé milosrdenství, Hospodine.
#94:19 Když v mém nitru roste neklid, naplní mě útěcha tvá potěšením.
#94:20 Může být tvým spojencem trůn zhouby, který proti právu jen trápení plodí?
#94:21 Napadají duši spravedlivou, nevinnou krev viní ze svévole.
#94:22 Ale Hospodin je můj hrad nedobytný, můj Bůh je má útočištná skála.
#94:23 Obrátí proti nim jejich ničemnosti, umlčí je jejich vlastní zlobou, umlčí je Hospodin, Bůh náš! 
#95:1 Pojďte, zaplesejme Hospodinu, oslavujme hlaholem skálu své spásy,
#95:2 vstupme před jeho tvář s díkůvzdáním, oslavujme ho hlaholem žalmů!
#95:3 Hospodin je velký Bůh, je velký Král nad všemi bohy.
#95:4 On má v svých rukou hlubiny země, temena hor patří jemu.
#95:5 Jeho je moře, on sám je učinil, souš vytvořily jeho ruce.
#95:6 Přistupte, klaňme se, klekněme, skloňme kolena před Hospodinem, který nás učinil.
#95:7 On je náš Bůh, my lid, jejž on pase, ovce, jež vodí svou rukou. Uslyšíte-li dnes jeho hlas,
#95:8 nezatvrzujte svá srdce jako při sváru v Meribě, jako v den pokušení na poušti v Masse,
#95:9 kde mě vaši otcové pokoušeli, kde mě chtěli zkoušet, i když viděli mé činy.
#95:10 Po čtyřicet let mi bylo na obtíž to pokolení. Řekl jsem si: Je to lid bloudící srdcem, k mým cestám se nezná.
#95:11 Proto jsem se v hněvu zapřisáhl: Nevejdou do mého odpočinutí! 
#96:1 Zpívejte Hospodinu píseň novou, zpívej Hospodinu, celá země!
#96:2 Zpívejte Hospodinu, dobrořečte jeho jménu, zvěstujte den ze dne jeho spásu,
#96:3 vypravujte mezi pronárody o jeho slávě, mezi všemi národy o jeho divech,
#96:4 neboť veliký je Hospodin, nejvyšší chvály hodný, budí bázeň, je nad všechny bohy.
#96:5 Všechna božstva národů jsou bůžci, ale Hospodin učinil nebe.
#96:6 Před jeho tváří velebná důstojnost, moc a lesk v svatyni jeho.
#96:7 Lidské čeledi, přiznejte Hospodinu, přiznejte Hospodinu slávu a moc,
#96:8 přiznejte Hospodinu slávu jeho jména! Přineste dar, vstupte do nádvoří jeho,
#96:9 v nádheře svatyně se klaňte Hospodinu! Svíjej se před ním, celá země!
#96:10 Říkejte mezi pronárody: Hospodin kraluje! Pevně je založen svět, nic jím neotřese. On povede při národů podle práva.
#96:11 Nebesa se zaradují, rozjásá se země, moře i s tím, co je v něm, se rozburácí,
#96:12 pole zazní jásotem, i všechno, co je na něm. Tehdy zaplesají všechny stromy v lese
#96:13 vstříc Hospodinu, že přichází, že přichází soudit zemi. On bude soudit svět spravedlivě a národy podle své pravdy. 
#97:1 Hospodin kraluje! Zajásej, země, raduj se, ostrovů množství!
#97:2 Oblak a mrákota jsou kolem něho, spravedlnost a právo jsou pilíře jeho trůnu.
#97:3 Žene se před ním oheň, kolkolem sežehne jeho protivníky.
#97:4 Nad světem planou světla jeho blesků, země to vidí a svíjí se v křeči.
#97:5 Hory se před Hospodinem jako vosk taví, před Pánem veškeré země.
#97:6 Nebesa hlásají jeho spravedlnost a všechny národy vidí jeho slávu.
#97:7 Budou zahanbeni všichni, kteří slouží modlám, kdo se chlubí svými bůžky; jemu se všichni bohové klanějí.
#97:8 Sijón to slyší a raduje se, jásají dcery judské nad tvými soudy, Hospodine.
#97:9 Vždyť ty, Hospodine, jsi nejvyšší nad celou zemí, neskonale převyšuješ všechny bohy.
#97:10 Vy, kdo milujete Hospodina, mějte v nenávisti zlo, on střeží duše svých věrných, svévolníkům je z rukou vytrhuje.
#97:11 Pro spravedlivého je zaseto světlo, radost pro ty, kteří mají přímé srdce.
#97:12 Radujte se, spravedliví, z Hospodina, vzdejte chválu tomu, co připomíná jeho svatost! 
#98:1 Žalm. Zpívejte Hospodinu píseň novou, neboť učinil podivuhodné věci, zvítězil svou pravicí, svou svatou paží!
#98:2 Hospodin dal poznat svoji spásu, zjevil před očima pronárodů svoji spravedlnost,
#98:3 na své milosrdenství se rozpomenul, na svou věrnost domu Izraele. Spatřily všechny dálavy země spásu našeho Boha.
#98:4 Hlahol Hospodinu, celá země, dejte se do plesu, pějte žalmy,
#98:5 pějte Hospodinu žalmy při citaře, při citaře nechť zazvučí žalmy;
#98:6 s doprovodem trub a polnic hlaholte před Hospodinem Králem!
#98:7 Ať se moře s tím, co je v něm, rozburácí, svět i ti, kdo na něm sídlí,
#98:8 dlaněmi nechť zatleskají řeky, s nimi ať plesají hory
#98:9 vstříc Hospodinu, že přichází, aby soudil zemi. On bude soudit svět spravedlivě a národy podle práva. 
#99:1 Hospodin kraluje! Národy trnou. Trůní na cherubech! Země se zmítá.
#99:2 Velký je Hospodin na Sijónu, nad všechny národy vyvýšený.
#99:3 Nechť vzdávají chválu tvému jménu, velkému a budícímu bázeň - je svaté!
#99:4 Silou Krále je láska k právu. Ty jsi určil právní řády; právo, spravedlnost v Jákobovi ty sám vykonáváš.
#99:5 Vyvyšujte Hospodina, našeho Boha, klanějte se před podnožím jeho nohou - je svaté!
#99:6 Mojžíš a Áron byli z jeho kněží, Samuel z těch, kdo vzývali jeho jméno; k Hospodinu volali a on jim odpovídal.
#99:7 Promlouval k nim z oblačného sloupu; dbali na jeho svědectví a na nařízení, která jim dal.
#99:8 Hospodine, Bože náš, ty jsi jim odpovídal, byl jsi jim Bohem, jenž promíjí, i když jsi jejich skutky stíhal pomstou.
#99:9 Vyvyšujte Hospodina, našeho Boha, klanějte se směrem k jeho svaté hoře, neboť Hospodin, náš Bůh, je svatý! 
#100:1 Žalm k díkůvzdání. Hlahol Hospodinu, celá země!
#100:2 Radostně služ Hospodinu! Vstupte před jeho tvář s plesem!
#100:3 Vězte, Hospodin je Bůh, on nás učinil, a ne my sami sebe, jsme jeho lid, ovce, které pase.
#100:4 Vstupte do jeho bran s díkůvzdáním, do nádvoří jeho s chvalozpěvem! Vzdávejte mu chválu, dobrořečte jeho jménu,
#100:5 neboť Hospodin je dobrý, jeho milosrdenství je věčné, jeho věrnost do všech pokolení! 
#101:1 Davidův. Žalm. O milosrdenství a soudu chci zpívat, tobě, Hospodine, prozpěvovat žalmy.
#101:2 Obezřetně půjdu bezúhonnou cestou. Kdy už ke mně přijdeš? Budu žít v bezúhonnosti srdce ve svém domě:
#101:3 Nebude mi vzorem, co ničemník páchá. Nenávidím, co dělají odpadlíci, nepropadnu tomu.
#101:4 Ať mi je vzdálena neupřímnost srdce, nechci mít nic se zlem.
#101:5 Kdo pomlouvá bližního, toho umlčím. Kdo má pyšné oči a naduté srdce, toho nestrpím.
#101:6 Vyhlédnu si v zemi věrné lidi, aby se mnou přebývali. Kdo jde bezúhonnou cestou, ten bude v mých službách.
#101:7 Nesmí přebývat v mém domě, kdo záludně jedná. Kdo proradně mluví, na oči mi nesmí.
#101:8 Každé ráno budu umlčovat všechny svévolníky v zemi, a tak vymýtím z Hospodinova města všechny pachatele ničemností. 
#102:1 Modlitba pro poníženého, když je sklíčen a vylévá před Hospodinem své lkání.
#102:2 Hospodine, vyslyš mou modlitbu, kéž k tobě pronikne moje volání!
#102:3 Neukrývej přede mnou tvář v den soužení mého, nakloň ke mně ucho, v den, kdy volám, pospěš, odpověz mi!
#102:4 Mé dny se v dým obracejí, mé kosti jsou rozpálené jak ohniště.
#102:5 Jak zlomená bylina schne moje srdce, i svůj chléb jíst zapomínám;
#102:6 od samého naříkání jsem vyzáblý na kost.
#102:7 Podobám se pelikánu v poušti, jsem jak sova v rozvalinách,
#102:8 probdím celé noci, jsem jak ptáče, jež na střeše osamělo.
#102:9 Celé dny mě moji nepřátelé tupí, klnou mi a za potřeštěnce mě mají,
#102:10 popel jím jak chleba, nápoj slzami si ředím
#102:11 pro tvůj hrozný hněv a pro tvé rozlícení; tys mě vyzvedl a srazil.
#102:12 Mé dny jsou jak stín, který se prodlužuje, usychám jako bylina.
#102:13 Ty však, Hospodine, ty zůstáváš věčně, budeš připomínán ve všech pokoleních.
#102:14 Ty povstaneš, slituješ se nad Sijónem, je čas smilovat se nad ním, nastala ta chvíle!
#102:15 Vždyť tvým služebníkům je v něm milý každý kámen a nad jeho sutinami je přepadá lítost.
#102:16 A budou se bát Hospodinova jména pronárody, tvé slávy všichni králové země,
#102:17 protože Hospodin vybuduje Sijón, ukáže se ve své slávě
#102:18 a k modlitbě bezmocných se skloní, jejich modlitbami nepohrdne.
#102:19 Pro budoucí pokolení je to psáno, aby lid, jenž bude stvořen, chválil Hospodina,
#102:20 že pohleděl ze svých svatých výšin, že Hospodin shlédl z nebe na zem,
#102:21 vyslyšel sténání vězňů, osvobodil syny smrti,
#102:22 aby na Sijónu vyprávěli o Hospodinově jménu, aby v Jeruzalémě šířili jeho chválu,
#102:23 až se tam shromáždí v jedno národy a království a budou sloužit Hospodinu.
#102:24 Na té cestě pokořil mou sílu a moje dny zkrátil.
#102:25 Pravím: Bože můj, uprostřed mých dnů mě odtud neber! z pokolení do pokolení půjdou tvá léta.
#102:26 Dávno jsi založil zemi, i nebesa jsou dílo tvých rukou.
#102:27 Ta zaniknou, a ty budeš trvat, všechno zvetší jako roucho, vyměníš je jako šat a vše se změní.
#102:28 Ale ty jsi stále týž a bez konce jsou tvoje léta.
#102:29 Budou zde přebývat synové tvých služebníků, jejich potomstvo bude před tebou stát pevně. 
#103:1 Davidův. Dobrořeč, má duše, Hospodinu, celé nitro mé, jeho svatému jménu!
#103:2 Dobrořeč, má duše, Hospodinu, nezapomínej na žádné jeho dobrodiní!
#103:3 On ti odpouští všechny nepravosti, ze všech nemocí tě uzdravuje,
#103:4 vykupuje ze zkázy tvůj život, věnčí tě svým milosrdenstvím a slitováním,
#103:5 po celý tvůj věk tě sytí dobrem, tvé mládí se obnovuje jako mládí orla.
#103:6 Hospodin zjednává spravedlnost a právo všem utlačeným.
#103:7 Dal poznat své cesty Mojžíšovi, synům Izraele svoje skutky.
#103:8 Hospodin je slitovný a milostivý, shovívavý, nejvýš milosrdný;
#103:9 nepovede pořád spory, nebude se hněvat věčně.
#103:10 Nenakládá s námi podle našich hříchů, neodplácí nám dle našich nepravostí.
#103:11 Jak vysoko nad zemí je nebe, tak mohutně se klene jeho milosrdenství nad těmi, kdo se ho bojí;
#103:12 jak je vzdálen východ od západu, tak od nás vzdaluje naše nevěrnosti.
#103:13 Jako se nad syny slitovává otec, slitovává se Hospodin nad těmi, kdo se ho bojí.
#103:14 On ví, že jsme jen stvoření, pamatuje, že jsme prach.
#103:15 Člověk, jehož dny jsou jako tráva, rozkvétá jak polní kvítí;
#103:16 sotva ho ovane vítr, už tu není, už se neobjeví na svém místě.
#103:17 Avšak Hospodinovo milosrdenství je od věků na věky s těmi, kteří se ho bojí, jeho spravedlnost i se syny synů,
#103:18 s těmi, kteří dodržují jeho smlouvu, kteří pamatují na jeho ustanovení a plní je.
#103:19 Hospodin si postavil trůn na nebesích, všemu vládne svou královskou mocí.
#103:20 Dobrořečte Hospodinu, jeho andělé, vy silní bohatýři, kteří plníte, co řekne, vždy poslušni jeho slova!
#103:21 Dobrořečte Hospodinu, všechny jeho zástupy, vy, kdo jste v jeho službách a plníte jeho vůli!
#103:22 Dobrořečte Hospodinu, všechna jeho díla, na všech místech jeho vlády. Dobrořeč, má duše, Hospodinu! 
#104:1 Dobrořeč, má duše, Hospodinu! Hospodine, Bože můj, jsi neskonale velký, oděl ses velebnou důstojností.
#104:2 Halíš se světlem jak pláštěm, rozpínáš nebesa jako stanovou plachtu.
#104:3 Mezi vodami si kleneš síně, z mračen si vůz činíš a vznášíš se na perutích větru.
#104:4 Z vichrů si činíš své posly, z ohnivých plamenů sluhy.
#104:5 Zemi jsi založil na pilířích, aby se nehnula navěky a navždy.
#104:6 Propastnou tůň jsi přikryl jako šatem. Nad horami stály vody;
#104:7 pohrozils a na útěk se daly, rozutekly se před tvým hromovým hlasem.
#104:8 Když vystoupila horstva, klesly do údolí, do míst, která jsi jim určil.
#104:9 Mez, kterou jsi stanovil, už nepřekročí, nepřikryjí znovu zemi.
#104:10 Prameny vysíláš do potoků, které mezi horami se vinou.
#104:11 Napájejí veškerou zvěř polí, divocí osli tu hasí žízeň.
#104:12 Při nich přebývá nebeské ptactvo, ozývá se v ratolestech.
#104:13 Ze svých síní zavlažuješ hory, země se sytí ovocem tvého díla.
#104:14 Dáváš růst trávě pro dobytek i rostlinám, aby je pěstoval člověk, a tak si ze země dobýval chléb.
#104:15 Dáváš víno pro radost lidskému srdci, až se tvář leskne víc než olej; chléb dodá lidskému srdci síly.
#104:16 Hospodinovy stromy se sytí vláhou, libanónské cedry, které on zasadil.
#104:17 A tam hnízdí ptactvo, na cypřiších má domov čáp.
#104:18 Horské štíty patří kozorožcům, skaliska jsou útočištěm pro damany.
#104:19 Učinil jsi měsíc k určování času, slunce ví, kdy k západu se schýlit.
#104:20 Přivádíš tmu, noc se snese, celý les se hemží zvěří;
#104:21 lvíčata řvou po kořisti, na Bohu se dožadují stravy.
#104:22 Slunce vychází a stahují se, v doupatech se ukládají k odpočinku.
#104:23 Člověk vyjde za svou prací a koná službu až do večera.
#104:24 Jak nesčetná jsou tvá díla, Hospodine! Všechno jsi učinil moudře; země je plná tvých tvorů.
#104:25 Tu je veliké a širé moře: hemží se v něm nespočetných živočichů maličkých i velkých,
#104:26 plují po něm lodě. Vytvořil jsi livjátana, aby v něm dováděl.
#104:27 A to vše s nadějí vzhlíží k tobě, že jim dáš v pravý čas pokrm;
#104:28 rozdáváš jim a oni si berou, otevřeš ruku a nasytí se dobrým.
#104:29 Skryješ-li tvář, propadají děsu, odejmeš-li jejich ducha, hynou, v prach se navracejí.
#104:30 Sesíláš-li svého ducha, jsou stvořeni znovu, a tak obnovuješ tvářnost země.
#104:31 Hospodinova sláva potrvá věčně! Hospodin se bude radovat ze svého díla.
#104:32 Shlédne na zemi a ta se třese, dotkne se hor a kouří se z nich.
#104:33 Budu zpívat Hospodinu po celý svůj život, svému Bohu zpívat žalmy, dokud budu.
#104:34 Kéž mu je příjemné moje přemítání! Hospodin je moje radost.
#104:35 Kéž hříšníci vymizí ze země, kéž svévolníci nejsou! Dobrořeč, má duše, Hospodinu! Haleluja. 
#105:1 Chválu vzdejte Hospodinu a vzývejte jeho jméno, uvádějte národům ve známost jeho skutky,
#105:2 zpívejte mu, pějte žalmy, přemýšlejte o všech jeho divech,
#105:3 honoste se jeho svatým jménem, ať se zaraduje srdce těch, kteří hledají Hospodina!
#105:4 Dotazujte se na vůli Hospodinovu a jeho moc, jeho tvář hledejte ustavičně.
#105:5 Připomínejte divy, jež vykonal, jeho zázraky a rozsudky jeho úst,
#105:6 potomkové Abrahama, jeho služebníka, Jákobovi synové, jeho vyvolení!
#105:7 Je to Hospodin, náš Bůh, kdo soudí celou zemi.
#105:8 Věčně pamatuje na svou smlouvu, na slovo, jež přikázal tisícům pokolení.
#105:9 Uzavřeli ji s Abrahamem, přísahou ji stvrdil Izákovi,
#105:10 stanovil ji Jákobovi jako nařízení, Izraeli jako smlouvu věčnou:
#105:11 „Dám ti kenaanskou zemi za dědičný úděl!“
#105:12 Na počet jich byla malá hrstka, byli tam jen hosty.
#105:13 Putovali od jednoho pronároda ke druhému, z jednoho království dál k jinému lidu.
#105:14 On však nedovolil nikomu, aby je utlačoval, káral kvůli nim i krále:
#105:15 „Nesahejte na mé pomazané, ublížit mým prorokům se chraňte!“
#105:16 Než přivolal na zemi hlad, než každou hůl chleba zlomil,
#105:17 vyslal už před nimi muže, Josefa, který byl prodán do otroctví.
#105:18 Sevřeli mu nohy do okovů, do želez se dostal,
#105:19 až do chvíle, kdy došlo na jeho slovo, když řeč Hospodinova ho protříbila.
#105:20 Poslal pro něj král a pout ho zbavil, vládce národů ho osvobodil.
#105:21 Učinil ho pánem svého domu, vládcem veškerého svého jmění,
#105:22 aby jeho velmože k sobě připoutal a moudrosti učil jeho starce.
#105:23 Pak přišel Izrael do Egypta, v zemi Chámově byl Jákob hostem.
#105:24 Hospodin velice rozplodil svůj lid, dopřál mu, aby zdatností předčil protivníky,
#105:25 jejichž srdce změnil, takže začali jeho lid nenávidět a záludně jednat s jeho služebníky.
#105:26 Poslal k nim Mojžíše, svého služebníka, s Áronem, jehož si zvolil.
#105:27 Jeho znamení jim předváděli, zázraky v Chámově zemi.
#105:28 Seslal temnotu a zatmělo se, a nikdo se neodvážil vzepřít jeho slovu.
#105:29 Jejich vody proměnil v krev, ryby nechal leknout.
#105:30 V zemi se jim vyrojila žabí havěť, nalezla i do královských komnat.
#105:31 Rozkázal a přiletěly mouchy a na celé jejich území komáři.
#105:32 Přívaly dešťů jim změnil v krupobití, ohnivými plameny bil jejich zemi.
#105:33 Potloukl jim vinice a fíkovníky, v jejich území polámal stromy.
#105:34 Rozkázal a snesly se kobylky, nesčetné roje žravého hmyzu.
#105:35 Sežraly jim vše, co v zemi rostlo, sežraly jim plody polí.
#105:36 Všechno prvorozené jim v zemi pobil, každou prvotinu jejich plodné síly.
#105:37 Ale své vyvedl se stříbrem a zlatem, nikdo z jejich kmenů neklopýtl.
#105:38 Egypt se radoval, že už táhnou, neboť strach z nich na něj padl.
#105:39 Jako závěs rozestíral oblak, ohněm svítíval jim v noci.
#105:40 Žádali a přihnal jim křepelky, chlebem nebeským je sytil.
#105:41 Otevřel skálu a vody tekly proudem, valily se jako řeka vyprahlými kraji.
#105:42 Neboť pamatoval na své svaté slovo a na Abrahama, svého služebníka.
#105:43 Vyvedl svůj lid - a veselili se, svoje vyvolené - a plesali.
#105:44 Daroval jim země pronárodů, výsledek námahy národů obdrželi,
#105:45 aby dbali na jeho nařízení a zachovávali jeho zákony. Haleluja. 
#106:1 Haleluja. Chválu vzdejte Hospodinu, protože je dobrý, jeho milosrdenství je věčné.
#106:2 Kdo vylíčí bohatýrské činy Hospodina, kdo rozhlásí všechnu chválu o něm?
#106:3 Blaze těm, kteří se drží práva, tomu, kdo si v každém čase vede spravedlivě.
#106:4 Hospodine, rozpomeň se na mě pro přízeň, jíž svůj lid zahrnuješ, navštiv mě svou spásou,
#106:5 abych směl spatřit dobro tvých vyvolených, abych se radoval radostí národa tvého, abych společně s tvým dědictvím o tobě s chloubou mluvil.
#106:6 Zhřešili jsme už se svými otci, provinili jsme se, svévolně si vedli.
#106:7 Naši otcové v Egyptě nepochopili tvé divy, tvé hojné milosrdenství si nepřipomínali, vzepřeli se při moři, při moři Rákosovém.
#106:8 On však je zachránil pro své jméno, aby v známost uvedl svou bohatýrskou sílu.
#106:9 Obořil se na Rákosové moře a vyschlo, propastnými tůněmi je vedl jako pouští.
#106:10 Zachránil je z rukou toho, kdo je nenáviděl, vykoupil je z rukou nepřítele.
#106:11 Jejich protivníky přikryly vody, nezůstal z nich ani jeden.
#106:12 Tehdy uvěřili jeho slovům, do zpěvu se dali k jeho chvále.
#106:13 Rychle však na jeho činy zapomněli, nečekali trpělivě na jeho pokyn.
#106:14 Chtivostí se dali strhnout v poušti, pokoušeli Boha v pustém kraji.
#106:15 On jim splnil jejich prosbu, ale stihl je pak úbytěmi.
#106:16 V táboře žárlili na Mojžíše, na Árona, jenž byl Hospodinův svatý;
#106:17 tu se rozevřela země, Dátana pohltila a přikryla Abírámův spolek.
#106:18 Proti jejich spolku vyšlehl oheň, svévolníky sežehl plamen.
#106:19 Na Chorébu udělali býčka, klaněli se slité modle,
#106:20 zaměnili svoji Slávu za podobu býka, býložravce.
#106:21 Zapomněli na Boha, svou spásu, který v Egyptě konal tak velké věci,
#106:22 v zemi Chámově úžasné divy, bázeň vzbuzující činy u Rákosového moře.
#106:23 Už vyhlásil jejich vyhlazení, ale Mojžíš, jeho vyvolený, postavil se před ním do trhliny a odvrátil jeho zkázonosné rozhořčení.
#106:24 Přežádoucí zem si zprotivili, nevěřili jeho slovu,
#106:25 žehrali v svých stanech, Hospodina neposlechli.
#106:26 Pozvedl k přísaze proti nim svou ruku, že je v té poušti nechá padnout,
#106:27 že jejich símě rozhodí mezi pronárody, že je rozpráší do všech zemí.
#106:28 Pak se spřáhli s Baal-peórem, jedli při obětních hodech k poctě mrtvých model.
#106:29 Svým jednáním uráželi Hospodina, proto je postihla pohroma.
#106:30 Povstal Pinchas k vykonání soudu, pohroma se zastavila;
#106:31 bylo mu to připočteno jako spravedlnost až navěky, do všech pokolení.
#106:32 U Meribských vod ho rozlítili, kvůli nim zle pochodil i Mojžíš,
#106:33 neboť se vzepřeli jeho duchu, jeho rty pronesly nerozvážnost.
#106:34 Nevyhladili národy, o nichž Hospodin mluvil,
#106:35 smísili se s pronárody, učili se dělat to, co ony.
#106:36 Sloužili jejich modlářským stvůrám a ty se jim staly léčkou;
#106:37 obětovali své syny a své dcery běsům.
#106:38 Nevinnou krev prolévali, krev svých synů a dcer, které obětovali modlářským stvůrám Kenaanu; proléváním krve zhanobili zemi.
#106:39 Tak se poskvrnili svými činy, svým jednáním porušili věrnost.
#106:40 Hospodin vzplál hněvem proti svému lidu, svoje dědictví si zhnusil,
#106:41 vydal je do rukou pronárodů, vládli jim ti, kdo je měli v nenávisti.
#106:42 Utiskovali je jejich nepřátelé, byli jimi pokořeni, dostali se do područí.
#106:43 Mnohokrát je vysvobodil; svými nápady však vzpírali se opět, ale na svou nepravost jen dopláceli.
#106:44 On jejich soužení viděl, slyšel jejich bědování,
#106:45 kvůli nim si připomínal svoji smlouvu, ve svém velkém milosrdenství měl s nimi soucit.
#106:46 Dal jim dojít slitování u všech, kteří je odvlekli do zajetí.
#106:47 Hospodine, zachraň nás, náš Bože, shromáždi nás z pronárodů, tvému svatému jménu budeme vzdávat chválu, budeme tě chválit chvalozpěvem.
#106:48 Požehnán buď Hospodin, Bůh Izraele, od věků až na věky! A všechen lid ať řekne: „Amen.“ Haleluja. 
#107:1 Chválu vzdejte Hospodinu, protože je dobrý, jeho milosrdenství je věčné!
#107:2 Tak ať řeknou ti, kdo byli Hospodinem vykoupeni, ti, které vykoupil z rukou protivníka,
#107:3 které shromáždil ze všech zemí, od východu, od západu, severu i moře.
#107:4 Bloudili pouští, cestou pustin, město sídla Božího však nenalezli.
#107:5 Žíznili a hladověli, byli v duši skleslí.
#107:6 A když ve svém soužení úpěli k Hospodinu, vytrhl je z tísně:
#107:7 sám je vedl přímou cestou, aby došli k městu jeho sídla.
#107:8 Ti ať vzdají Hospodinu chválu za milosrdenství a za divy, jež pro lidi koná:
#107:9 dosyta dal najíst lačným, hladovým dal plno dobrých věcí.
#107:10 Seděli v temnotách šeré smrti, v železných poutech a v ponížení,
#107:11 neboť se vzepřeli tomu, co řekl Bůh, znevážili úradek Nejvyššího.
#107:12 Trápením pokořil jejich srdce, klesali, a nikde žádná pomoc.
#107:13 A když ve svém soužení úpěli k Hospodinu, zachránil je z tísně:
#107:14 vyvedl je z temnot šeré smrti, sám zpřetrhal jejich pouta.
#107:15 Ti ať vzdají Hospodinu chválu za milosrdenství a za divy, jež pro lidi koná:
#107:16 rozrazil bronzová vrata, železné závory zlomil.
#107:17 Pošetilci pro svou cestu nevěrnosti, pro své nepravosti byli pokořeni.
#107:18 Každý pokrm se jim hnusil, dospěli až k branám smrti.
#107:19 A když ve svém soužení úpěli k Hospodinu, zachránil je z tísně:
#107:20 seslal slovo své a uzdravil je, zachránil je z jámy.
#107:21 Ti ať vzdají Hospodinu chválu za milosrdenství a za divy, jež pro lidi koná,
#107:22 ať mu obětují oběť díků, ať s plesáním vypravují o všech jeho skutcích.
#107:23 Ti, kteří se vydávají na lodích na moře, kdo konají dílo na nesmírných vodách,
#107:24 spatřili Hospodinovy skutky, jeho divy na hlubině.
#107:25 Poručil a povstal bouřný vichr, jenž do výše zvedl vlnobití.
#107:26 Vznášeli se k nebi, řítili se do propastných tůní, ztráceli v té spoušti hlavu.
#107:27 V závrati jak opilí se potáceli, s celou svou moudrostí byli v koncích.
#107:28 A když ve svém soužení úpěli k Hospodinu, vyvedl je z tísně:
#107:29 utišil tu bouři, ztichlo vlnobití.
#107:30 Zaradovali se, když se uklidnilo, on pak je dovedl do přístavu, jak si přáli.
#107:31 Ti ať vzdají Hospodinu chválu za milosrdenství a za divy, jež pro lidi koná,
#107:32 ať ho vyvyšují v shromáždění lidu, v zasedání starších ať ho chválí!
#107:33 Řeky mění v poušť a vodní zřídla v suchopáry,
#107:34 v solné pláně žírnou zemi pro zlobu těch, kdo v ní sídlí.
#107:35 Poušť v jezero mění a zem vyprahlou ve vodní zřídla.
#107:36 Tam usadil ty, kdo hladověli, zbudovali město, sídlo Boží.
#107:37 Pole oseli, vinice vysázeli, sklidili úrodu hojnou.
#107:38 Žehnal jim a velmi se rozrostli, ani dobytka jim neubylo.
#107:39 Jich však ubývalo, ohýbali se pod tíhou zla a strastí,
#107:40 když je ten, jenž může vylít opovržení i na knížata, zavedl do bezcestných pustot.
#107:41 Avšak ubožáku se stal v ponížení hradem a čeledi jeho lidu rozmnožil jak ovce.
#107:42 Přímí lidé to vidí a radují se, ale každá podlost musí zavřít ústa.
#107:43 Kdo je moudrý, ať dbá těchto věcí a Hospodinovu milosrdenství ať hledí porozumět! 
#108:1 Píseň, žalm Davidův.
#108:2 Mé srdce je připraveno, Bože, budu zpívat, prozpěvovat žalmy, rovněž moje sláva!
#108:3 Probuď se už, citaro a harfo, ať jitřenku vzbudím.
#108:4 Hospodine, chci ti mezi lidmi vzdávat chválu, mezi národy ti budu zpívat žalmy;
#108:5 vždyť tvé milosrdenství nad nebe sahá, až do mraků tvoje věrnost.
#108:6 Povznes se až nad nebesa, Bože, a nad celou zemí ať je tvoje sláva!
#108:7 Aby tvoji milí byli zachováni, pomoz svou pravicí, odpověz mi!
#108:8 Bůh ve své svatyni promluvil: „S jásotem rozdělím Šekem, rozměřím dolinu Sukót.
#108:9 Mně patří Gileád, mně patří Manases, Efrajim, přilba mé hlavy, Juda, můj palcát.
#108:10 Moáb je mé umývadlo, na Edóm hodím svůj střevíc, proti Pelišteji válečný ryk spustím.“
#108:11 Kdože mě uvede do nepřístupného města? Kdo mě dovedl až do Edómu?
#108:12 Což ne, Bože, ty, jenž zanevřel jsi na nás? Což s našimi zástupy bys nevytáhl, Bože?
#108:13 Před protivníkem buď naše pomoc, je šalebné čekat spásu od člověka.
#108:14 S Bohem statečně si povedeme, on rozšlape naše protivníky. 
#109:1 Pro předního zpěváka, žalm Davidův. Bože, má chválo, nestav se hluchým,
#109:2 když se na mě rozevřela ústa svévolná a lstivá! Zrádným jazykem mě napadají,
#109:3 slovy nenávistnými mě zasypali, bojují proti mě bez důvodu.
#109:4 Osočují mě za moji lásku, zatímco se modlím.
#109:5 Za dobro mě zavalují zlobou, za mou lásku nenávistí.
#109:6 Postav proti němu svévolníka, po jeho pravici žalobce ať stane.
#109:7 Ať dopadne jako svévolník, až bude souzen; jeho modlitba ať je mu počítána za hřích.
#109:8 Dny ať jsou mu ukráceny, jeho pověření ať převezme jiný.
#109:9 Jeho synové ať sirotky se stanou, jeho žena vdovou.
#109:10 Jeho synové ať toulají se po žebrotě, ze svých rozvalin ať chodí prosit.
#109:11 Na všechno, co má, ať políčí si lichvář, co vytěžil, cizáci ať loupí.
#109:12 Ať nemá nikoho, kdo by mu nadále prokázal milosrdenství, kdo by se smiloval nad sirotky po něm.
#109:13 Jeho potomstvo buď vymýceno, jeho jméno smazáno buď v příštím pokolení.
#109:14 Ať Hospodin pamatuje na nepravost jeho otců a hřích jeho matky vymazán ať není.
#109:15 Ať je má Hospodin ustavičně před očima, ať vymýtí ze země památku po nich
#109:16 za to, že nepamatoval na milosrdenství, ale pronásledoval člověka poníženého a ubohého, chtěl usmrtit zkrušeného v srdci.
#109:17 Miloval zlořečení, ať ho postihne! O požehnání nestál, ať se ho vzdálí!
#109:18 Zlořečení oblékal jak šaty; ať mu pronikne nitrem jak voda, ať mu prostoupí kosti jak olej.
#109:19 Ať mu je jako roucho, kterým se halí, opaskem, kterým se přepásává ustavičně.
#109:20 To ať si vyslouží od Hospodina ti, kteří mě osočují, kteří proti mně zlovolně mluví.
#109:21 Ty však, Panovníku Hospodine, ukaž na mně pro své jméno, jak je tvé milosrdenství dobrotivé, vysvoboď mě!
#109:22 Jsem ponížený ubožák, v nitru mám zraněné srdce.
#109:23 Odcházím jako stín, který se prodlužuje, jako luční kobylka jsem smeten.
#109:24 V kolenou se podlamuji postem, bez oleje chátrá moje tělo.
#109:25 Jsem jim jenom pro potupu, jak mě vidí, potřásají hlavou.
#109:26 Pomoz mi, můj Bože, Hospodine, podle svého milosrdenství mě zachraň,
#109:27 aby poznali, že tvá ruka to byla, žes to učinil ty, Hospodine.
#109:28 Jen ať zlořečí, ale ty žehnej! Když povstali, ať je stihne hanba, a tvůj služebník se zaraduje.
#109:29 Stud ať poleje ty, kdo mě osočují, hanbou ať se zahalí jak pláštěm.
#109:30 Moje ústa vzdají Hospodinu velkou chválu, mezi mnohými ho budu chválit,
#109:31 neboť stanul po pravici ubožáku, aby ho zachránil před jeho soudci. 
#110:1 Davidův, žalm. Výrok Hospodinův mému pánu: „Zasedni po mé pravici, já ti položím tvé nepřátele za podnoží k nohám.“
#110:2 Hospodin vztáhne žezlo tvé moci ze Sijónu. Panuj uprostřed svých nepřátel!
#110:3 Tvůj lid přijde dobrovolně v den, kdy pohotovost svoláš; v nádheře svatyně jak rosa z lůna úsvitu se objeví tvé mužstvo.
#110:4 Hospodin přísahal a nebude želet: Ty jsi kněz navěky podle Malkísedekova řádu.
#110:5 Panovník ti bude po pravici, rozdrtí v den svého hněvu nepřátelské krále.
#110:6 Bude soudit pronárody - plno mrtvol všude -, on rozdrtí hlavu velké země. Cestou z potoka pít bude, proto vztyčí hlavu. 
#110:7 
#111:1 Haleluja. Chválu vzdávám Hospodinu celým srdcem, v kruhu přímých, v shromáždění.
#111:2 Činy Hospodinovy jsou velké, vyhledávané všemi, kdo zálibu v nich našli.
#111:3 Velebné a důstojné je jeho dílo, jeho spravedlnost trvá navždy.
#111:4 On zajistil památku svým divům; Hospodin je milostivý, plný slitování.
#111:5 Dal potravu těm, kdo se ho bojí, navěky je pamětliv své smlouvy.
#111:6 Svému lidu ohlásil své mocné činy, že mu dá dědictví pronárodů.
#111:7 Činy jeho rukou jsou pravda a právo, všechna jeho ustanovení jsou věrná,
#111:8 spolehlivá navěky a navždy, pravdou a přímostí vytvořená.
#111:9 Seslal svému lidu vykoupení, ustanovil navěky svou smlouvu; svaté, bázeň budící je jeho jméno.
#111:10 Počátek moudrosti je bát se Hospodina; velice jsou prozíraví všichni, kdo tak činí. Jeho chvála trvá navždy! 
#112:1 Haleluja. Blaze muži, jenž se bojí Hospodina, jenž velikou zálibu má v jeho přikázáních!
#112:2 Jeho potomci se stanou bohatýry v zemi, pokolení přímých bude požehnáno.
#112:3 Jmění, bohatství má ve svém domě, jeho spravedlnost trvá navždy.
#112:4 Ve tmách vzchází přímým světlo; Bůh je milostivý, plný slitování, spravedlivý.
#112:5 Dobře bývá muži, jenž se smiluje a půjčí a své věci spravuje dle práva:
#112:6 nezhroutí se nikdy, spravedlivý zůstane v paměti věčně.
#112:7 Nemusí se bát zlé zprávy, jeho srdce pevně doufá v Hospodina.
#112:8 Jeho srdce má oporu v Bohu, nebojí se, jednou spatří pád svých protivníků.
#112:9 Rozděluje, dává ubožákům, jeho spravedlnost trvá navždy, jeho roh se zvedne v slávě.
#112:10 Svévolník na to zlostně hledí, skřípe zuby a odvahu ztrácí; choutky svévolníků přijdou vniveč. 
#113:1 Haleluja. Chvalte, Hospodinovi služebníci, chvalte jméno Hospodina!
#113:2 Jméno Hospodinovo buď požehnáno nyní i navěky.
#113:3 Od východu slunce až na západ chváleno buď jméno Hospodina.
#113:4 Hospodin je vyvýšen nad všemi národy, nad nebesa strmí jeho sláva.
#113:5 Kdo je jako Hospodin, náš Bůh, jenž tak vysoko trůní?
#113:6 Sestupuje níže, aby viděl na nebesa a na zemi.
#113:7 Nuzného pozvedá z prachu, z kalu vytahuje ubožáka
#113:8 a pak ho posadí vedle knížat, vedle knížat svého lidu.
#113:9 Neplodnou usazuje v domě jako šťastnou matku synů. Haleluja. 
#114:1 Když vyšel Izrael z Egypta, Jákobův dům z lidu temné řeči,
#114:2 stal se Juda Boží svatyní, Izrael Božím vladařstvím.
#114:3 Moře to vidělo a dalo se na útěk, Jordán se nazpět obrátil,
#114:4 Hory poskakovaly jako berani a pahorky jako jehňata.
#114:5 Moře, co je ti, že utíkáš, Jordáne, že se zpět obracíš?
#114:6 Hory, proč poskakujete jako berani, a vy, pahorky, jako jehňata?
#114:7 Chvěj se, země, před Pánem, před Bohem Jákobovým!
#114:8 On proměňuje skálu v jezero, křemen v prameny vod. 
#115:1 Ne nás, Hospodine, ne nás, ale svoje jméno oslav pro své milosrdenství a pro svou věrnost!
#115:2 Proč by měly pronárody říkat: „Kde je ten jejich Bůh?“
#115:3 Náš Bůh je v nebesích a všechno, co chce, koná.
#115:4 Jejich modly jsou stříbro a zlato, dílo lidských rukou.
#115:5 Mají ústa, a nemluví, mají oči, a nevidí,
#115:6 mají uši, a neslyší, mají nosy, a necítí,
#115:7 rukama nemohou hmatat, nohama nemohou chodit, z hrdla nevydají hlásku.
#115:8 Jim jsou podobni ti, kdo je zhotovují, každý, kdo v ně doufá.
#115:9 Izraeli, doufej v Hospodina, je tvou pomocí a štítem.
#115:10 Áronův dome, doufej v Hospodina, je tvou pomocí a štítem.
#115:11 Vy, kdo se bojíte Hospodina, doufejte v Hospodina, je vám pomocí a štítem.
#115:12 Hospodin na nás pamatuje, on nám žehná: žehná domu Izraele, žehná domu Áronovu,
#115:13 žehná těm, kteří se bojí Hospodina, jak malým, tak velkým.
#115:14 Hospodin ať vás rozmnoží, vás i vaše syny!
#115:15 Jste Hospodinovi požehnaní; on učinil nebesa i zemi.
#115:16 Nebesa, ta patří Hospodinu, zemi dal však lidem.
#115:17 Mrtví nechválí už Hospodina, nikdo z těch, kdo sestupují v říši ticha.
#115:18 Avšak my budeme Hospodinu dobrořečit nyní i navěky. Haleluja. 
#116:1 Hospodina miluji; on slyší můj hlas, moje prosby,
#116:2 sklonil ke mně ucho. Po všechny své dny chci k němu volat.
#116:3 Ovinuly mě provazy smrti, přepadly mě úzkosti podsvětí; nacházím jen soužení a strasti.
#116:4 Vzývám však Hospodinovo jméno: Hospodine, prosím, zachraň mi život!
#116:5 Hospodin je milostivý, spravedlivý, náš Bůh se slitovává.
#116:6 Hospodin je ochránce nezkušených: byl jsem vyčerpán, a dopřál mi zvítězit.
#116:7 Můžeš opět odpočinout, moje duše, neboť Hospodin se tě zastal.
#116:8 Ubránils mě před smrtí, mé oko před slzami, moje nohy před zvrtnutím.
#116:9 Před Hospodinem smím dále chodit v zemi živých.
#116:10 Uvěřil jsem, proto mluvím; byl jsem velmi pokořený.
#116:11 Ukvapeně jsem si říkal: Každý člověk je lhář.
#116:12 Jak se mám odvděčit Hospodinu, že se mne tolikrát zastal?
#116:13 Zvednu kalich spásy a budu vzývat Hospodinovo jméno.
#116:14 Svoje sliby Hospodinu splním před veškerým jeho lidem.
#116:15 Velkou cenu má v Hospodinových očích oddanost jeho věrných až k smrti.
#116:16 Hospodine, prosím, já jsem tvůj služebník, služebník tvůj, syn tvé služebnice. Ty jsi mi rozvázal pouta.
#116:17 Tobě obětuji oběť díků a budu vzývat Hospodinovo jméno.
#116:18 Svoje sliby Hospodinu splním před veškerým jeho lidem,
#116:19 v nádvořích Hospodinova domu, Jeruzaléme, v tvém středu. Haleluja. 
#117:1 Chvalte Hospodina, všechny národy, všichni lidé, chvalte ho zpěvem,
#117:2 neboť se nad námi mohutně klene jeho milosrdenství. Hospodinova věrnost je věčná! Haleluja. 
#118:1 Chválu vzdejte Hospodinu, protože je dobrý, jeho milosrdenství je věčné!
#118:2 Ať vyzná Izrael: Jeho milosrdenství je věčné!
#118:3 Ať vyzná Áronův dům: Jeho milosrdenství je věčné!
#118:4 Ať vyznají ti, kteří se bojí Hospodina: Jeho milosrdenství je věčné!
#118:5 V soužení jsem volal Hospodina, Hospodin mi odpověděl, daroval mi volnost.
#118:6 Hospodin je při mně, nebojím se. Co by mi mohl udělat člověk?
#118:7 Hospodin je při mně, mezi mými pomocníky, spatřím pád těch, kdo mě nenávidí.
#118:8 Lépe utíkat se k Hospodinu, než doufat v člověka.
#118:9 Lépe utíkat se k Hospodinu, než doufat v knížata.
#118:10 Všechny pronárody mě obklíčily; odrazil jsem je v Hospodinově jménu.
#118:11 Oblehly mě, ano, obklíčily; odrazil jsem je v Hospodinově jménu.
#118:12 Oblehly mě jako vosy; zhasly jak planoucí trní, odrazil jsem je v Hospodinově jménu.
#118:13 Udeřil jsi na mě tvrdě, abych padl. Hospodin je moje pomoc,
#118:14 Hospodin je síla má i moje píseň; stal se mou spásou.
#118:15 Ze stanů spravedlivých zní plesání nad spásou. Hospodinova pravice koná mocné činy!
#118:16 Hospodinova pravice se vyvýšila, Hospodinova pravice koná mocné činy!
#118:17 Nezemřu, budu žít, budu vypravovat o Hospodinových činech.
#118:18 Hospodin mě přísně trestal, ale nevydal mě smrti.
#118:19 Brány spravedlnosti mi otevřete, vejdu jimi vzdávat chválu Hospodinu.
#118:20 Toto je Hospodinova brána, skrze ni vcházejí spravedliví.
#118:21 Tobě vzdávám chválu, žes mi odpověděl; stal ses mou spásou.
#118:22 Kámen, jejž zavrhli stavitelé, stal se kamenem úhelným.
#118:23 Stalo se tak skrze Hospodina, tento div se udál před našimi zraky.
#118:24 Toto je den, který učinil Hospodin, jásejme a radujme se z něho.
#118:25 Prosím, Hospodine, pomoz! Prosím, Hospodine, dopřej zdaru!
#118:26 Požehnaný, jenž přichází v Hospodinově jménu. Žehnáme vám z Hospodinova domu.
#118:27 Hospodin je Bůh, dává nám světlo. Slavte svátek ratolestí, k rohům oltáře se ubírejte.
#118:28 Ty jsi můj Bůh, tobě vzdávám chválu, vyvyšuji tě, můj Bože.
#118:29 Chválu vzdejte Hospodinu, protože je dobrý, jeho milosrdenství je věčné! 
#119:1 Blaze těm, kdo vedou bezúhonný život, těm, kdo žijí tak, jak učí Hospodinův zákon.
#119:2 Blaze těm, kdo zachovávají jeho svědectví, těm, kdo se na jeho vůli dotazují celým srdcem.
#119:3 Ti podlosti nepáchají, jeho cestami se berou.
#119:4 Ty jsi vydal svá ustanovení, aby se přesně dodržovala.
#119:5 Kéž jsou moje cesty pevně zaměřeny k dodržování tvých nařízení.
#119:6 Nebudu zahanben tehdy, budu-li brát zřetel na všechna tvá přikázání.
#119:7 Z přímého srdce ti vzdávám chválu, že se smím učit tvým spravedlivým soudům.
#119:8 Chci tvá nařízení dodržovat, jen mě nikdy neopouštěj!
#119:9 Jak si mladík udrží svou stezku čistou? Musí se vždy držet tvého slova.
#119:10 Dotazuji se na tvoji vůli celým srdcem, nedej, abych zbloudil od tvých přikázání.
#119:11 Tvou řeč uchovávám v srdci, nechci proti tobě hřešit.
#119:12 Požehnán buď, Hospodine, vyučuj mě v tom, co nařizuješ.
#119:13 Mé rty budou vypravovat o všech soudech tvých úst.
#119:14 Veselím se z cesty tvých svědectví více než ze všeho jmění.
#119:15 O tvých ustanoveních přemýšlím, na zřeteli mám tvé stezky.
#119:16 Nařízení tvá jsou pro mne potěšením, nezapomínám na tvé slovo.
#119:17 Zastávej se svého služebníka a budu žít, chci se držet tvého slova.
#119:18 Otevři mi oči, ať mám na zřeteli divy ze Zákona tvého.
#119:19 Jsem na zemi jenom hostem, neukrývej přede mnou svá přikázání.
#119:20 V duši se stravuji touhou po tvých soudech v každé době.
#119:21 Oboříš se na opovážlivce, na proklatce, kteří pobloudili od tvých přikázání.
#119:22 Sejmi ze mne potupu a pohrdání, neboť tvá svědectví zachovávám.
#119:23 I když vládci zasednou a umluví se na mě, tvůj služebník bude přemýšlet o tvých nařízeních,
#119:24 tvá svědectví budou nadále mým potěšením, jsou to moji rádci.
#119:25 Do prachu je přitisknuta moje duše, zachovej mi život podle svého slova.
#119:26 Vyprávěl jsem o svých cestách - tys mi odpověděl, vyučuj mě v tom, co nařizuješ.
#119:27 Dej mi porozumět cestě svých ustanovení, chci přemýšlet o tvých divuplných činech.
#119:28 Moje duše se rozplývá žalem, pozvedni mě podle svého slova.
#119:29 Odvrať ode mne cesty klamu, podle svého Zákona se nade mnou smiluj.
#119:30 Zvolil jsem si cestu věrnosti, stavím si před oči tvé soudy.
#119:31 Přimkl jsem se k tvým svědectvím, nedej, Hospodine, abych byl zahanben!
#119:32 Poběžím cestou tvých přikázání, dal jsi mému srdci volnost.
#119:33 Ukaž mi cestu svých nařízení, Hospodine, důsledně ji budu zachovávat.
#119:34 Dej mi rozum a budu tvůj Zákon zachovávat, budu se ho držet celým srdcem.
#119:35 Veď mě po cestě svých přikázání, oblíbil jsem si ji.
#119:36 Ke svým svědectvím nakloň mé srdce, nikoli k zištnosti.
#119:37 Odvracej mé oči, ať nehledí na šalebnost, na své cestě mi zachovej život.
#119:38 Splň své slovo svému služebníku, jenž žije v tvé bázni.
#119:39 Odvrať ode mne potupu, které se lekám, tvé soudy jsou dobrotivé.
#119:40 Po tvých ustanoveních tak toužím, svou spravedlností mi zachovej život.
#119:41 Kéž na mě sestoupí tvé milosrdenství, Hospodine, i tvá spása, jak jsi řekl,
#119:42 a já dám odpověď tomu, kdo mě tupí, neboť doufám ve tvé slovo.
#119:43 Nikdy nezbavuj má ústa slova pravdy, neboť čekám na tvůj soud.
#119:44 Tvého Zákona se budu držet ustavičně, navěky a navždy.
#119:45 Volně budu chodit, na tvá ustanovení se dotazuji.
#119:46 O tvých svědectvích před králi budu mluvit a nebudu zahanben.
#119:47 Tvá přikázání jsou pro mne potěšením, já jsem si je zamiloval.
#119:48 Dlaně vztahuji k tvým přikázáním, já jsem si je zamiloval, chci přemýšlet o tvých nařízeních.
#119:49 Pamatuj na slovo dané svému služebníku, jímž jsi ve mně očekávání vzbudil.
#119:50 Je mi útěchou v mém pokoření, že tvá řeč mi zachová život.
#119:51 Opovážlivci se mi velice posmívají, já se od tvého Zákona neuchýlím.
#119:52 Připomínám si tvé dávné soudy, Hospodine, v tom útěchu najdu.
#119:53 Sežehuje mě zuřivost svévolníků, těch, kdo opustili tvůj Zákon.
#119:54 Nařízení tvá si zpívám jako žalmy v domě, kde jsem jenom hostem.
#119:55 V noci si připomínám tvé jméno, Hospodine, tvého Zákona se držím.
#119:56 Mně připadl tento úkol: ustanovení tvá zachovávat.
#119:57 Pravím: Hospodine, tys můj úděl, držím se tvých slov.
#119:58 O shovívavost tě prosím celým srdcem, smiluj se nade mnou, jak jsi řekl.
#119:59 Uvažuji o svých cestách, k tvým svědectvím obracím své nohy.
#119:60 Pospíchám a neotálím tvé příkazy dodržovat.
#119:61 Ovinují mě provazy svévolníků, nezapomínám však na tvůj Zákon.
#119:62 Vstávám o půlnoci, abych ti vzdal chválu za tvé spravedlivé soudy.
#119:63 Jsem druhem všech, kteří se tě bojí a tvých ustanovení se drží.
#119:64 Tvého milosrdenství je, Hospodine, plná země, vyučuj mě v tom, co nařizuješ.
#119:65 Prokázal jsi dobro svému služebníku podle svého slova, Hospodine.
#119:66 Nauč mě okoušet a znát, co je dobré, já tvým přikázáním věřím.
#119:67 Dokud jsem se nepokořil, bloudíval jsem, nyní dodržuji, co jsi řekl.
#119:68 Jsi dobrý a prokazuješ dobro, vyučuj mě v tom, co nařizuješ.
#119:69 Opovážlivci mě mrzce pošpinili, já však zachovávám ustanovení tvá celým srdcem.
#119:70 Bezcitné je jejich tučné srdce, mně však je tvůj Zákon potěšením.
#119:71 Byl jsem pokořen a bylo mi to k dobru, naučil jsem se tvým nařízením.
#119:72 Zákon tvých úst je mi dražší než tisíce hřiven zlata nebo stříbra.
#119:73 Ruce tvé mě učinily pevným, dej mi rozum, ať se naučím tvým přikázáním.
#119:74 Ti, kdo se tě bojí, spatří mě a budou se radovat, neboť čekám na tvé slovo.
#119:75 Hospodine, vím, že tvé soudy jsou spravedlivé, pokořils mě pravdou.
#119:76 Kéž se projeví tvé milosrdenství a potěší mě, jak jsi svému služebníku řekl.
#119:77 Kéž mě zahrne tvé slitování a budu žít, tvůj Zákon je pro mne potěšením.
#119:78 Opovážlivce ať stihne hanba za to, že mi křivdí, já o tvých ustanoveních přemýšlím.
#119:79 Navrátí se ke mně ti, kdo se tě bojí, ti, kdo tvá svědectví znají.
#119:80 Kéž je mé srdce bezúhonné podle tvých nařízení, ať nejsem zahanben.
#119:81 Moje duše chřadne touhou po tvé spáse, čekám na tvé slovo.
#119:82 Zrak mi slábne, vyhlížím tvé slovo, kdy už mě potěšíš?
#119:83 Vedlo se mi jako měchu v kouři, na tvá nařízení jsem však nezapomněl.
#119:84 Kolikpak dnů zbývá tvému služebníku? Kdy nad těmi, kdo mě pronásledují, vykonáš soud?
#119:85 Opovážlivci mi kopou jámy bez ohledu na tvůj Zákon.
#119:86 Všechna tvá přikázání jsou pravda. Pronásledují mě neprávem, pomoz!
#119:87 Bezmála už se mnou v zemi skoncovali, já však tvá ustanovení neopustím.
#119:88 Podle svého milosrdenství mi zachovej život, svědectví tvých úst se budu držet.
#119:89 Věčně, Hospodine, stojí pevně v nebesích tvé slovo.
#119:90 Z pokolení do pokolení trvá tvá věrnost, upevnil jsi zemi a ta stojí.
#119:91 Podle tvých soudů vše stojí dodnes, to všechno jsou tvoji služebníci.
#119:92 Kdyby mi tvůj Zákon nebyl potěšením, dávno bych v svém pokoření zhynul.
#119:93 Na tvá ustanovení nezapomenu nikdy, protože mi jimi zachováváš život.
#119:94 Tobě patřím, buď má spása, na tvá ustanovení se dotazuji.
#119:95 Svévolníci na mě čekají, aby mě zahubili, já se snažím porozumět tvým svědectvím.
#119:96 Vidím, že vše spěje k svému konci, přikázání tvé má nekonečný prostor.
#119:97 Jak jsem si tvůj Zákon zamiloval! Každý den o něm přemýšlím.
#119:98 Nad nepřátele mě činí moudřejším tvá přikázání, navěky jsou moje.
#119:99 Jsem prozíravější než všichni moji učitelé, neboť přemýšlím o tvých svědectvích.
#119:100 Rozumu jsem nabyl víc než starci, neboť tvá ustanovení zachovávám.
#119:101 Před každou zlou stezkou jsem zdržel své kroky, držel jsem se tvého slova.
#119:102 Neodchyluji se od tvých soudů, neboť ty mi ukazuješ cestu.
#119:103 Jak lahodnou chuť má, co ty říkáš! Sladší než med je to pro má ústa.
#119:104 Z tvých ustanovení jsem nabyl rozumnosti, proto nenávidím každou stezku klamu.
#119:105 Světlem pro mé nohy je tvé slovo, osvěcuje moji stezku.
#119:106 Co jsem přísahal, to splním, držím se tvých spravedlivých soudů.
#119:107 Byl jsem velice pokořen, Hospodine, podle svého slova mi zachovej život.
#119:108 Kéž se ti líbí, Hospodine, dobrovolné oběti mých úst, nauč mě svým soudům.
#119:109 Ustavičně dávám v sázku život, nezapomínám však na tvůj Zákon.
#119:110 Svévolníci osidlo mi nastražili, avšak od tvých ustanovení jsem nepobloudil.
#119:111 Tvá svědectví budou mým dědictvím věčně, z nich se veselí mé srdce.
#119:112 V srdci jsem se rozhodl plnit tvá nařízení navěky a do důsledků.
#119:113 Nenávidím obojetné lidi, miluji tvůj Zákon.
#119:114 Tys má skrýš a štít můj, čekám na tvé slovo.
#119:115 Pryč ode mne, zlovolníci! Budu zachovávat přikázání svého Boha.
#119:116 Podpírej mě, jak jsi řekl, a budu žít, v mých nadějích mě nezahanbuj.
#119:117 Buď mi oporou a budu spasen, stále budu hledět na tvá nařízení.
#119:118 Zhrdáš všemi, kdo zbloudili od tvých nařízení, protože záludně klamou.
#119:119 Všechny svévolníky v zemi odklízíš jak strusku, proto jsem si zamiloval tvá svědectví.
#119:120 Strachem před tebou se chvěje mé tělo, bojím se tvých soudů.
#119:121 Vykonávám soud a spravedlnost, neponechej mě těm, kdo mě utlačují.
#119:122 Zasaď se za svého služebníka k jeho dobru, aby mě opovážlivci neutlačovali.
#119:123 Zrak mi slábne, vyhlížím tvou spásu, výrok tvé spravedlnosti.
#119:124 Nalož se svým služebníkem podle svého milosrdenství, vyučuj mě v tom, co nařizuješ.
#119:125 Jsem tvůj služebník, učiň mě rozumným, abych poznal tvá svědectví.
#119:126 Hospodine, je čas jednat, tvůj Zákon se porušuje.
#119:127 Ano, miluji tvá přikázání víc než zlato, víc než zlato ryzí.
#119:128 Všechna ustanovení chci ve všem správně plnit, nenávidím každou stezku klamu.
#119:129 Tvá svědectví jsou divuplná, proto je má duše zachovává.
#119:130 Kam tvá slova proniknou, tam vzchází světlo, nezkušení nabývají rozumnosti.
#119:131 Dychtivě otvírám ústa, toužím po tvých přikázáních.
#119:132 Shlédni na mne, smiluj se nade mnou podle toho, jak soudíváš ty, kdo milují tvé jméno.
#119:133 Upevni mé kroky tím, cos řekl, dej, ať žádná ničemnost mě neovládne.
#119:134 Vykup mě z útlaku lidí, chci se tvých ustanovení držet.
#119:135 Svou jasnou tvář ukaž svému služebníku, vyučuj mě v tom, co nařizuješ.
#119:136 Proudem se mi řinou slzy z očí, že se nedodržuje tvůj Zákon.
#119:137 Ty jsi, Hospodine, spravedlivý, přímý ve svých soudech.
#119:138 Přikázal jsi, aby tvá svědectví byla spravedlnost a naprostá pravda.
#119:139 Horlivost mě sžírá, moji protivníci zapomněli na tvá slova.
#119:140 Co jsi řekl, je důkladně protříbené, tvůj služebník si to zamiloval.
#119:141 Já jsem nepatrný, pohrdaný člověk, avšak na tvá ustanovení jsem nezapomněl.
#119:142 Věčně spravedlivá je tvá spravedlnost, tvůj Zákon je pravda.
#119:143 Soužení a úzkost na mě doléhají, přikázání tvá jsou pro mne potěšením.
#119:144 Spravedlnost tvých svědectví je věčná, dej mi rozum a budu žít.
#119:145 Celým srdcem volám: Odpověz mi, Hospodine, chci tvá nařízení zachovávat.
#119:146 Volám k tobě, buď má spása, chci se tvých svědectví držet.
#119:147 Dřív než začne svítat, na pomoc tě volám, čekám na tvé slovo.
#119:148 Mé oči se budí dřív než noční hlídky a přemýšlím o tom, co jsi řekl.
#119:149 Podle svého milosrdenství mě vyslyš, Hospodine, podle svého soudu mi zachovej život.
#119:150 Blíží se už, kdo se ženou za mrzkostmi, vzdalují se od Zákona tvého.
#119:151 Ty jsi, Hospodine, blízko, všechna tvá přikázání jsou pravda.
#119:152 Dávno je mi z tvých svědectví známo, žes jim dal navěky pevný základ.
#119:153 Pohleď na mé pokoření, braň mě, vždyť na tvůj Zákon nezapomínám.
#119:154 Ujmi se mé pře a zastaň se mne, zachovej mi život, jak jsi řekl.
#119:155 Svévolníkům je vzdálena spása, protože se nedotazují na tvá nařízení.
#119:156 Nesmírné je, Hospodine, tvoje slitování, podle svých soudů mi zachovej život.
#119:157 Mnoho je těch, kdo mě pronásledují a souží, já se však od tvých svědectví neodchýlím.
#119:158 Na věrolomné se dívám s ošklivostí, nedrží se toho, co jsi řekl.
#119:159 Hleď, jak jsem si tvá ustanovení zamiloval, Hospodine, podle svého milosrdenství mi zachovej život.
#119:160 To hlavní v tvém slovu je pravda, každý soud tvé spravedlnosti je věčný.
#119:161 Bez důvodů mě pronásledují velmožové, mé srdce má strach jen z tvého slova.
#119:162 Veselím se z toho, co jsi řekl, jako ten, kdo našel velkou kořist.
#119:163 Nenávidím klam, hnusí se mi, miluji tvůj Zákon.
#119:164 Chválívám tě sedmkráte za den za tvé spravedlivé soudy.
#119:165 Hojný pokoj mají ti, kdo milují tvůj Zákon, o nic neklopýtnou.
#119:166 S nadějí vyhlížím tvoji spásu, Hospodine, a tvá přikázání plním.
#119:167 Má duše se drží tvých svědectví, velice jsem si je zamiloval.
#119:168 Tvých ustanovení a tvých svědectví se držím, máš před sebou všechny moje cesty.
#119:169 Kéž mé bědování dolehne až k tobě, Hospodine, dej mi rozum podle svého slova.
#119:170 Kéž má prosba dojde k tobě, vysvoboď mě, jak jsi řekl.
#119:171 Chvalozpěv ať vytryskne mi ze rtů, neboť mě svým nařízením učíš.
#119:172 Ať můj jazyk opěvuje, co jsi řekl, všechna tvá přikázání jsou spravedlivá.
#119:173 Na pomoc mi podej svoji ruku, tvá ustanovení jsem si zvolil.
#119:174 Hospodine, toužím po tvé spáse, tvůj Zákon je pro mne potěšením.
#119:175 Kéž má duše žije a může tě chválit; kéž mi pomáhají tvoje soudy.
#119:176 Bloudím jako zatoulané jehně, hledej svého služebníka, vždyť jsem na tvá přikázání nezapomněl! 
#120:1 Poutní píseň. K Hospodinu v soužení jsem volal, on mi odpověděl.
#120:2 Hospodine, vysvoboď mě od zrádných rtů, od jazyka záludného.
#120:3 Co ti patří, co tě stihne, záludný jazyku?
#120:4 Ostré šípy bohatýra, žhoucí kručinkové uhle.
#120:5 Běda mi, musím dlít jako host v Mešeku, pobývat při stanech Kédarců!
#120:6 Pobývám už dlouho u toho, kdo nenávidí pokoj.
#120:7 Promluvím-li o pokoji, oni odpovědí válkou. 
#121:1 Píseň k pouti. Pozvedám své oči k horám: Odkud mi přijde pomoc?
#121:2 Pomoc mi přichází od Hospodina, on učinil nebesa i zemi.
#121:3 Nedopustí, aby uklouzla tvá noha, nedříme ten, jenž tě chrání.
#121:4 Ano, nedříme a nespí ten, jenž chrání Izraele.
#121:5 Hospodin je tvůj ochránce, Hospodin je ti stínem po pravici.
#121:6 Ve dne tě nezasáhne slunce ani za noci měsíc.
#121:7 Hospodin tě chrání ode všeho zlého, on chrání tvůj život.
#121:8 Hospodin bude chránit tvé vycházení a vcházení nyní i navěky. 
#122:1 Poutní píseň, Davidova. Zaradoval jsem se, když mi řekli: Půjdem do Hospodinova domu!
#122:2 A naše nohy již stojí ve tvých branách, Jeruzaléme.
#122:3 Jeruzalém je zbudován jako město semknuté v jediný celek.
#122:4 Tam nahoru vystupují kmeny, Hospodinovy to kmeny, Izraeli na svědectví, vzdát Hospodinovu jménu chválu.
#122:5 Tam jsou postaveny soudné stolce, stolce Davidova domu.
#122:6 Vyprošujte Jeruzalému pokoj: Kéž v klidu žijí ti, kdo tě milují!
#122:7 Kéž je na tvých valech pokoj, kéž se tvé paláce těší klidu!
#122:8 Pro své bratry, pro své druhy vyhlašuji: „Budiž v tobě pokoj!“
#122:9 Pro dům Hospodina, našeho Boha, usiluji o tvé dobro. 
#123:1 Poutní píseň. Pozvedám své oči k tobě, jenž v nebesích trůníš.
#123:2 Hle, jak oči služebníků k rukám jejich pánů, jako oči služebnice k rukám její paní, tak vzhlížejí naše oči k Hospodinu, našemu Bohu, dokud se nad námi nesmiluje.
#123:3 Smiluj se nad námi, Hospodine, smiluj se nad námi! Dosyta jsme zakusili pohrdání.
#123:4 Naše duše už dosyta zakusila posměchu sebejistých a pohrdání pyšných. 
#124:1 Poutní píseň, Davidova. Kdyby sám Hospodin nebyl při nás - Izrael ať řekne -,
#124:2 kdyby sám Hospodin nebyl při nás, když proti nám povstali lidé,
#124:3 zaživa by nás tehdy zhltli v hněvu, jímž proti nám vzpláli.
#124:4 Tehdy by nás zatopily vody, dravý proud by se přes nás valil,
#124:5 tehdy by se přes nás převalily vzduté vody.
#124:6 Požehnán buď Hospodin, že za kořist nás nedal jejich zubům!
#124:7 Unikli jsme jako ptáče z osidla lovců. Osidlo je protrženo, unikli jsme!
#124:8 Naše pomoc je ve jménu Hospodina; on učinil nebesa i zemi. 
#125:1 Poutní píseň. Kdo doufají v Hospodina, jsou jak hora Sijón: nepohne se, strmí věčně.
#125:2 Kolem Jeruzaléma jsou hory, kolem svého lidu je Hospodin, nyní i navěky.
#125:3 Nesetrvá žezlo zvůle na údělu spravedlivých, aby snad nevztáhli spravedliví své ruce k podlostem.
#125:4 Hospodine, prokaž dobro dobrým, těm, kdo mají přímé srdce!
#125:5 Ale ty, kdo uhýbají na své křivolaké stezky, zažene Hospodin s pachateli ničemností. Pokoj s Izraelem! 
#126:1 Poutní píseň. Když Hospodin úděl Sijónu změnil, bylo nám jak ve snu.
#126:2 Tehdy naše ústa naplnil smích a náš jazyk plesal. Tehdy se říkalo mezi pronárody: „Hospodin s nimi učinil velké věci.“
#126:3 Hospodin s námi učinil velké věci, radovali jsme se.
#126:4 Hospodine, změň náš úděl, jako měníš potoky na jihu země!
#126:5 Ti, kdo v slzách sejí, s plesáním budou sklízet.
#126:6 S pláčem nyní chodí, kdo rozsévá, s plesáním však přijde, až ponese snopy. 
#127:1 Poutní píseň, Šalomounova. Nestaví-li dům Hospodin, nadarmo se namáhají stavitelé. Nestřeží-li město Hospodin, nadarmo bdí strážný.
#127:2 Nadarmo časně vstáváte, dlouho vysedáváte a jíte chléb trápení, zatímco Bůh dopřává svému milému spánek.
#127:3 Hle, synové jsou dědictví od Hospodina, mzdou od něho plod lůna.
#127:4 Čím jsou šípy v ruce bohatýra, tím jsou synové zplození v mládí.
#127:5 Blaze muži, který jimi naplnil svůj toulec! Nebudou zahanbeni, až budou v bráně jednat s nepřáteli. 
#128:1 Poutní píseň. Blaze každému, kdo se bojí Hospodina, kdo chodí po jeho cestách!
#128:2 Co rukama vytěžíš, budeš i jíst. Blaze tobě, bude s tebou dobře.
#128:3 Tvá žena bude jak plodná réva uvnitř tvého domu, tvoji synové jak olivové ratolesti kolem tvého stolu.
#128:4 Hle, jak bývá požehnáno muži, jenž se bojí Hospodina.
#128:5 Hospodin ať požehná ti ze Sijónu, abys viděl dobro Jeruzaléma po všechny dny svého žití,
#128:6 abys uviděl syny svých synů. Pokoj s Izraelem! 
#129:1 Poutní píseň. Jak často mě už od mládí sužovali - Izrael ať řekne -,
#129:2 jak často mě už od mládí sužovali, ale nepřemohli.
#129:3 Po zádech mi orali oráči, vyorali dlouhé brázdy.
#129:4 Hospodin je spravedlivý, postraňky těch svévolníků přeťal.
#129:5 S hanbou zpět ať táhnou všichni, kteří nenávidí Sijón!
#129:6 Ať jsou jak tráva na střechách, ta uschne dřív, než je vytrhána.
#129:7 Jí si ten, kdo žne, dlaň nenaplní, ani náruč ten, jenž sbírá snopy.
#129:8 Neřeknou jim ti, kdo půjdou kolem: „Požehnání Hospodinovo buď s vámi! Žehnáme vám v Hospodinově jménu.“ 
#130:1 Poutní píseň. Z hlubin bezedných tě volám, Hospodine,
#130:2 Panovníku, vyslyš můj hlas! Kéž tvé ucho pozorně vyslechne moje prosby.
#130:3 Budeš-li mít, Hospodine, na zřeteli nepravosti, kdo obstojí, Panovníku?
#130:4 Ale u tebe je odpuštění; tak vzbuzuješ bázeň.
#130:5 Skládám naději v Hospodina, má duše v něho naději skládá, čekám na jeho slovo.
#130:6 Má duše vyhlíží Panovníka víc než strážní jitro, když drží stráž k jitru.
#130:7 Čekej, Izraeli, na Hospodina! U Hospodina je milosrdenství, hojné je u něho vykoupení,
#130:8 on vykoupí Izraele ze všech jeho nepravostí. 
#131:1 Poutní píseň, Davidova. Nemám, Hospodine, domýšlivé srdce ani povýšený pohled. Neženu se za velkými věcmi, za divy, jež nevystihnu,
#131:2 nýbrž chovám se klidně a tiše. Jako odstavené dítě u své matky, jako odstavené dítě je ve mně má duše.
#131:3 Čekej, Izraeli, na Hospodina nyní i navěky. 
#132:1 Poutní píseň. Hospodine, rozpomeň se na Davida, na veškerou jeho usilovnou péči,
#132:2 jak se zapřisáhl Hospodinu, zavázal se slibem Přesilnému Jákobovu:
#132:3 „Nevejdu do stanu svého domu, nevstoupím na rohož svého lože,
#132:4 očím nedopřeji spánku ani víčkům podřímnutí,
#132:5 dokud nenaleznu Hospodinu místo, příbytek Přesilnému Jákobovu!“
#132:6 A hle, v Efratě jsme o ní uslyšeli, našli jsme ji na Jaarských polích.
#132:7 Vstupme do jeho příbytku, klanějme se před podnožím jeho nohou.
#132:8 Povstaň, Hospodine, k místu svého odpočinku, ty sám i schrána tvé moci!
#132:9 Tvoji kněží ať obléknou spravedlnost, tvoji věrní ať plesají.
#132:10 Pro Davida, svého služebníka, neodmítej svého pomazaného.
#132:11 Hospodin přísahal Davidovi na svou věrnost - nevezme to nazpět: „Toho, jenž vzejde z tvých beder, dosadím po tobě na trůn.
#132:12 Dodrží-li tvoji synové mou smlouvu i toto mé svědectví, jemuž je budu učit, navždy budou též jejich synové sedat na tvém trůnu.“
#132:13 Hospodin si totiž zvolil Sijón, zatoužil jej mít za sídlo:
#132:14 „To je místo mého odpočinku navždy, usídlím se tady, neboť po něm toužím.
#132:15 Jeho stravě budu hojně žehnat, jeho ubožáky budu sytit chlebem.
#132:16 Jeho kněžím dám za oděv spásu, jeho věrní budou zvučně plesat.
#132:17 Zde dám pučet Davidovu rohu, svému pomazanému budu pečovat o planoucí světlo.
#132:18 Jeho nepřátelům dám za oděv hanbu, ale na něm se bude jeho čelenka třpytit.“ 
#133:1 Poutní píseň, Davidova. Jaké dobro, jaké blaho tam, kde bratří bydlí svorně!
#133:2 Jako výborný olej na hlavě, jenž kane na vous, na vous Áronovi, kane mu na výstřih roucha.
#133:3 Jak chermónská rosa, která kane na sijónské hory. Tam udílí Hospodin své požehnání, život navěky. 
#134:1 Poutní píseň. Dobrořečte Hospodinu, všichni Hospodinovi služebníci, kteří stojíte v Hospodinově domě za nočního času.
#134:2 Pozvedejte ruce ke svatyni, dobrořečte Hospodinu!
#134:3 Hospodin ti žehnej ze Sijónu; on učinil nebesa i zemi. 
#135:1 Haleluja. Chvalte Hospodinovo jméno, chvalte je, Hospodinovi služebníci,
#135:2 kteří stojíte v Hospodinově domě, v nádvořích domu našeho Boha.
#135:3 Chvalte Hospodina, neboť Hospodin je dobrý, pějte žalmy jeho jménu, neboť je líbezné.
#135:4 Hospodin si vyvolil Jákoba, za zvláštní vlastnictví přijal Izraele.
#135:5 Já přec vím, že Hospodin je velký, náš Pán je nade všechny bohy.
#135:6 Všechno, co Hospodin chce, to činí na nebesích i na zemi, v mořích i ve všech propastných tůních.
#135:7 Přivádí mlhu od končin země, déšť provází blesky, ze svých zásobnic vyvádí vítr.
#135:8 Prvorozené Egypta pobil od člověka po dobytče.
#135:9 Znamení a zázraky sesílal na tebe, Egypte, na faraóna, na všechny jeho služebníky.
#135:10 Pobil mnohé pronárody a zahubil smělé krále,
#135:11 krále Emorejců Síchona a bášanského krále Óga i všechna království Kenaanu.
#135:12 Jejich země předal do dědictví, do dědictví Izraeli, svému lidu.
#135:13 Hospodine, tvé jméno je věčné, budeš připomínán ve všech pokoleních, Hospodine.
#135:14 Hospodin svůj lid obhájí, bude mít se svými služebníky soucit.
#135:15 Modly pronárodů jsou stříbro a zlato, dílo lidských rukou.
#135:16 Mají ústa, a nemluví, mají oči, a nevidí.
#135:17 Mají uši, a nic nezaslechnou, v jejich ústech není dechu.
#135:18 Jim podobni jsou ti, kdo je zhotovují, každý, kdo v ně doufá.
#135:19 Dome Izraelův, dobrořeč Hospodinu, dome Áronův, dobrořeč Hospodinu.
#135:20 Dome Léviův, dobrořeč Hospodinu, vy, kdo se bojíte Hospodina, dobrořečte Hospodinu.
#135:21 Požehnán buď Hospodin ze Sijónu, ten, který přebývá v Jeruzalémě. Haleluja. 
#136:1 Chválu vzdejte Hospodinu, protože je dobrý, jeho milosrdenství je věčné.
#136:2 Chválu vzdejte Bohu bohů, jeho milosrdenství je věčné.
#136:3 Chválu vzdejte Pánu pánů, jeho milosrdenství je věčné.
#136:4 Jedině on koná velké divy, jeho milosrdenství je věčné.
#136:5 Nebesa učinil důmyslně, jeho milosrdenství je věčné.
#136:6 Zemi na vodách překlenul oblohou, jeho milosrdenství je věčné.
#136:7 Učinil veliká světla, jeho milosrdenství je věčné.
#136:8 Slunce, aby vládlo ve dne, jeho milosrdenství je věčné.
#136:9 Měsíc a hvězdy, aby vládly v noci, jeho milosrdenství je věčné.
#136:10 Ranil Egypt v jeho prvorozených, jeho milosrdenství je věčné.
#136:11 A vyvedl Izraele z jeho středu, jeho milosrdenství je věčné.
#136:12 Pevnou rukou a vztaženou paží, jeho milosrdenství je věčné.
#136:13 On rozdělil Rákosové moře ve dví, jeho milosrdenství je věčné.
#136:14 A převedl Izraele jeho středem, jeho milosrdenství je věčné.
#136:15 Faraóna s vojskem smetl do Rákosového moře, jeho milosrdenství je věčné.
#136:16 Svůj lid vodil pouští, jeho milosrdenství je věčné.
#136:17 Pobil velké krále, jeho milosrdenství je věčné.
#136:18 Zahubil vznešené krále, jeho milosrdenství je věčné.
#136:19 Krále Emorejců Síchona, jeho milosrdenství je věčné.
#136:20 A bášanského krále Óga, jeho milosrdenství je věčné.
#136:21 Jejich země předal do dědictví, jeho milosrdenství je věčné.
#136:22 Do dědictví Izraele, svého služebníka, jeho milosrdenství je věčné.
#136:23 Když jsme byli poníženi, rozpomněl se na nás, jeho milosrdenství je věčné.
#136:24 A vyrval nás našim protivníkům, jeho milosrdenství je věčné.
#136:25 Veškerému tvorstvu dává pokrm, jeho milosrdenství je věčné.
#136:26 Chválu vzdejte Bohu nebes, jeho milosrdenství je věčné. 
#137:1 U řek babylónských, tam jsme sedávali s pláčem ve vzpomínkách na Sijón.
#137:2 Své citary jsme v té zemi zavěsili na topoly,
#137:3 když nás ti, kdo nás odvlekli, vybízeli tam ke zpěvu, trýznitelé k radovánkám: „Zazpívejte nám některý ze sijónských zpěvů!“
#137:4 Jak bychom však mohli zpívat píseň Hospodinovu v té cizí zemi?
#137:5 Jestli, Jeruzaléme, na tebe zapomenu, ať mi má pravice sloužit zapomene.
#137:6 Ať mi jazyk přilne k patru, nebudu-li si tě připomínat, nebudu-li Jeruzalém považovat za svou svrchovanou radost.
#137:7 Připomeň synům Edómu, Hospodine, den Jeruzaléma, jak volali: „Bořte! Bořte do základů!“
#137:8 Záhubě propadlá babylónská dcero, blaze tomu, kdo ti odplatí za skutky spáchané na nás.
#137:9 Blaze tomu, kdo tvá nemluvňata uchopí a roztříští o skálu. 
#138:1 Davidův. Celým svým srdcem ti vzdávám chválu, přímo před bohy ti zpívám žalmy,
#138:2 klaním se ti před tvým svatým chrámem, tvému jménu vzdávám chválu za tvé milosrdenství a za tvou věrnost; svou řeč jsi vyvýšil nad každé své jméno.
#138:3 Odpověděl jsi mi v den, kdy jsem tě volal, dodal jsi mé duši sílu.
#138:4 Hospodine, všichni králové země ti vzdají chválu, až uslyší, co jsi vyřkl.
#138:5 Budou zpívat o Hospodinových cestách, neboť sláva Hospodinova je velká.
#138:6 Hospodin je vyvýšený, ale hledí na poníženého, zdálky pozná domýšlivce.
#138:7 I když jsem v soužení, ty mi zachováš život, vztáhneš ruku proti hněvu mých nepřátel a tvá pravice mě spasí.
#138:8 Hospodin za mě dokončí zápas. Hospodine, tvoje milosrdenství je věčné, neopouštěj dílo vlastních rukou! 
#139:1 Pro předního zpěváka. Davidův, žalm. Hospodine, zkoumáš mě a znáš mě.
#139:2 Víš o mně, ať sedím nebo vstanu, zdálky je ti jasné, co chci dělat.
#139:3 Sleduješ mou stezku i místo, kde ležím, všechny moje cesty jsou ti známy.
#139:4 Ještě nemám slovo na jazyku, a ty, Hospodine, víš už všechno.
#139:5 Sevřel jsi mě zezadu i zpředu, svou dlaň jsi položil na mě.
#139:6 Nad mé chápání jsou tyto divy, jsou nedostupné, nestačím na to.
#139:7 Kam odejdu před tvým duchem, kam uprchnu před tvou tváří?
#139:8 Zamířím-li k nebi, jsi tam, a když si ustelu v podsvětí, také tam budeš.
#139:9 I kdybych vzlétl na křídlech jitřní záře, chtěl přebývat při nejzazším moři,
#139:10 tvoje ruka mě tam doprovodí, tvá pravice se mě chopí.
#139:11 Kdybych řekl: Snad mě přikryje tma, i noc kolem mne se stane světlem.
#139:12 Žádná tma pro tebe není temná: noc jako den svítí, temnota je jako světlo.
#139:13 Tys to byl, kdo utvořil mé ledví, v životě mé matky jsi mě utkal.
#139:14 Tobě vzdávám chválu za činy, jež budí bázeň: podivuhodně jsem utvořen, obdivuhodné jsou tvé skutky, toho jsem si plně vědom.
#139:15 Tobě nezůstala skryta jediná z mých kostí, když jsem byl v skrytosti tvořen a hněten v nejhlubších útrobách země.
#139:16 Tvé oči mě viděly v zárodku, všechno bylo zapsáno v tvé knize: dny tak, jak se vytvářely, dřív než jediný z nich nastal.
#139:17 Jak si vážím divů, které konáš, Bože! Nesmírný je jejich počet,
#139:18 sčetl bych je, ale je jich víc než písku. Sotva procitnu, jsem s tebou.
#139:19 Kéž bys, Bože, skolil svévolníka. Pryč ode mne, vy, kdo proléváte krev!
#139:20 Dovolávají se tě při svých pletichách, zneužívají tvé jméno tvoji protivníci.
#139:21 Nemám nenávidět, Hospodine, ty, kdo nenávidí tebe? S odporem pohlížet na ty, kdo se proti tobě zvedli?
#139:22 Nenávidím je, rozhodně nenávidím, jsou to také moji nepřátelé.
#139:23 Bože, zkoumej mě, ty znáš mé srdce, zkoušej mě, ty znáš můj neklid,
#139:24 hleď, zda jsem nesešel na cestu trápení, a po cestě věčnosti mě veď! 
#140:1 Pro předního zpěváka. Žalm Davidův.
#140:2 Vysvoboď mě, Hospodine, od člověka zlého, chraň mě proti násilníku,
#140:3 proti těm, kdo mají v srdci zlé úmysly a den ze dne se srocují k bojům,
#140:4 jazyky si ostří jako hadi, za rty mají jed jak zmije.
#140:5 Ochraňuj mě, Hospodine, před rukama svévolníka, chraň mě proti násilníku, proti těm, kdo zamýšlejí podrazit mi nohy.
#140:6 Pyšní osidlo mi nastražili, síť z provazů na cestu rozestřeli, nachystali na mě léčky.
#140:7 Pravím Hospodinu: Ty jsi můj Bůh! Přej sluchu mým prosbám, Hospodine.
#140:8 Hospodine, Panovníku, moje mocná spáso, hlavu kryješ mi v den bitvy.
#140:9 Nepřivoluj, Hospodine, k choutkám svévolníka, nedej, aby vyšly jeho plány, ať se nevypíná.
#140:10 Na hlavu těch, kdo mě obkličují, ať padne trápení, na němž se jejich rty umluvily.
#140:11 Ať na ně dopadne žhavé uhlí, ať je Bůh do ohně srazí, do vířivých proudů, aby nepovstali.
#140:12 Pomlouvač se v zemi neudrží, zlovolného násilníka stihne náhlá zkáza.
#140:13 Vím, že Hospodin obhájí poníženého, že ubožákům zjedná právo.
#140:14 Ano, spravedliví vzdají tvému jménu chválu, přímí budou bydlet před tvou tváří! 
#141:1 Žalm Davidův. Hospodine, k tobě volám, pospěš ke mně, dopřej sluchu mému hlasu, když tě volám!
#141:2 Jako kadidlo ať míří má modlitba k tobě, pozdvižení mých rukou jak večerní oběť.
#141:3 Hospodine, postav stráž k mým ústům, přede dveře mých rtů hlídku,
#141:4 nedej, aby se mé srdce přiklonilo ke zlu, ať se nedopustím svévolnosti s muži, kteří pášou ničemnosti; jejich vlídnosti okoušet nechci.
#141:5 Spravedlivý ať mě třeba bije, pokládám to za milosrdenství, ať mě trestá, je to pro mou hlavu olej, má hlava to neodmítne, v jejich neštěstí se za ně ještě budu modlit.
#141:6 Ať jsou jejich soudci svrženi na úbočí skály, ale spravedliví ať slyší, že vlídná jsou má slova.
#141:7 Jako když se v zemi všechno seká a poltí, k jícnu podsvětí se sypou naše kosti.
#141:8 K tobě, Panovníku Hospodine, pozvedám své oči, utíkám se k tobě, nevydej mě smrti,
#141:9 ochraňuj mě před osidlem na mě políčeným, před léčkami pachatelů ničemností!
#141:10 Svévolníci ať se chytí do svých tenat, kdežto já ať projdu! 
#142:1 Poučující, Davidův. Modlitba, když byl v jeskyni.
#142:2 Volám k Hospodinu, úpím, volám k Hospodinu, prosím,
#142:3 před ním vylévám své lkání, o svém soužení mu vypovídám.
#142:4 Jsem na duchu skleslý, ale ty znáš moji stezku! Na cestě, jíž kráčím, osidlo mi nastražili.
#142:5 Pohleď napravo a uzříš: není tu nikoho, kdo by se ke mně znal, nemám kam utéci, není tu, kdo by měl o mě péči.
#142:6 Úpím k tobě, Hospodine, pravím: Tys mé útočiště, tys můj podíl v zemi živých!
#142:7 Věnuj pozornost mému bědování, jsem zcela vyčerpán. Vysvoboď mě od pronásledovatelů, jsou zdatnější než já.
#142:8 Vyveď mě ze žaláře, abych vzdával chválu tvému jménu. Obstoupí mě spravedliví, ty se mě zastaneš. 
#143:1 Žalm Davidův. Hospodine, slyš mou modlitbu, přej sluchu mým prosbám, odpověz mi pro svou pravdu, pro svou spravedlnost.
#143:2 Nevcházej v soud se svým služebníkem, vždyť před tebou nikdo z živých není spravedlivý.
#143:3 Nepřítel mě pronásleduje, můj život tiskne k zemi, do temnot mě vsadil jak od věků mrtvé.
#143:4 Jsem skleslý na duchu, srdce mi usedá v nitru.
#143:5 Připomínám si dávné dny, rozjímám o všech tvých skutcích, přemýšlím o činech tvých rukou,
#143:6 rozpínám své ruce k tobě, má duše po tobě žízní jak vyprahlá země.
#143:7 Pospěš, odpověz mi, Hospodine, můj duch dokonává, neukrývej přede mnou svou tvář, nebo se budu podobat těm, kdo sestupují v jámu.
#143:8 Ohlas mi zrána své milosrdenství, neboť doufám v tebe. Dej mi poznat cestu, po níž mám jít, neboť k tobě pozvedám svou duši.
#143:9 Vysvoboď mě od nepřátel, Hospodine, u tebe si hledám úkryt.
#143:10 Nauč mě plnit tvou vůli, vždyť jsi můj Bůh. Kéž mě tvůj dobrotivý duch vede po rovné zemi!
#143:11 Pro své jméno mi zachovej, Hospodine, život, ve své spravedlnosti mě vyveď ze soužení,
#143:12 ve svém milosrdenství umlč mé nepřátele, přiveď nazmar všechny moje protivníky, vždyť jsem tvůj služebník! 
#144:1 Davidův. Požehnán buď Hospodin, má skála, který učí bojovat mé ruce a mé prsty válčit!
#144:2 Moje milosrdenství a moje pevná tvrz, můj nedobytný hrad, můj vysvoboditel, můj štít, k němuž se utíkám, on mi můj lid podmaňuje.
#144:3 Hospodine, co je člověk, že ho bereš na vědomí, co syn člověka, že na něj myslíš?
#144:4 Člověk se podobá vánku, jeho dny jsou jak stín pomíjivé.
#144:5 Hospodine, nakloň nebesa a sestup! Dotkni se hor a bude se z nich kouřit.
#144:6 Udeř bleskem, rozptyl nepřátele, vypusť své šípy a uveď je v zmatek,
#144:7 vztáhni ruku z výše, vyprosti mě a vysvoboď z nesmírného vodstva, z rukou cizozemců!
#144:8 Jejich ústa mluví šalebně, jejich pravice je pravice zrádná!
#144:9 Bože, chci ti zpívat novou píseň, s harfou o deseti strunách budu ti pět žalmy.
#144:10 Ty, jenž dáváš spásu králům, jenž Davida, svého služebníka, vyprošťuješ od zhoubného meče,
#144:11 vyprosti mě a vysvoboď z rukou cizozemců! Jejich ústa mluví šalebně, jejich pravice je pravice zrádná.
#144:12 Kéž jsou naši synové jak štěpy, krásně urostlí v svém mládí. Naše dcery ať jsou jako sloupy vytesané podle chrámového vzoru.
#144:13 Naše sýpky ať jsou plné, ať skýtají hojnost všeho. Našich ovcí ať je na tisíce, desetitisíce všude vůkol,
#144:14 náš skot ať je březí. Ať nás nepostihne vpád a odvlékání, ať nezazní žalostný křik na ulicích.
#144:15 Blaze lidu, jemuž se tak daří. Blaze lidu, jehož Bohem je Hospodin! 
#145:1 Chvalozpěv Davidův. Chci tě vyvyšovat, Bože můj a Králi, tvému jménu dobrořečit navěky a navždy.
#145:2 Po všechny dny ti chci dobrořečit a tvé jméno chválit navěky a navždy.
#145:3 Veliký je Hospodin, nejvyšší chvály hodný, jeho velikost nelze vyzpytovat.
#145:4 Všechna pokolení chválí tvoje skutky zpěvem, hlásají tvé bohatýrské činy.
#145:5 Tvoje velebnost je důstojná a slavná, chci přemýšlet o tvých divuplných dílech.
#145:6 Všichni budou mluvit o tvých mocných, bázeň vzbuzujících skutcích, i já budu vypravovat o tvé velikosti.
#145:7 Budou šířit vše, co připomíná tvoji velkou dobrotivost, budou jásat, jak jsi spravedlivý.
#145:8 Hospodin je milostivý, plný slitování, shovívavý a nesmírně milosrdný.
#145:9 Hospodin je ke všem dobrotivý, nade vším, co učinil, se slitovává.
#145:10 Kéž ti vzdají chválu, Hospodine, veškeré tvé skutky, kéž ti tvoji věrní dobrořečí,
#145:11 ať hovoří o tvém slavném kralování, ať promluví o tvé bohatýrské síle,
#145:12 aby lidem uváděli ve známost tvé bohatýrské činy a slávu a důstojnost království tvého.
#145:13 Tvoje království je království všech věků, tvoje vláda přečká všechna pokolení. Ve všech svých slovech je Hospodin věrný, je svatý ve všech svých skutcích.
#145:14 Hospodin podpírá všechny klesající a všechny sehnuté napřimuje.
#145:15 Oči všech s nadějí vzhlížejí k tobě a ty jim v pravý čas dáváš pokrm,
#145:16 otvíráš svou ruku a ve své přízni sytíš všechno, co žije.
#145:17 Hospodin je spravedlivý ve všech svých cestách, milosrdný ve všech svých skutcích.
#145:18 Hospodin je blízko všem, kteří volají k němu, všem, kdo ho volají opravdově.
#145:19 Vyplňuje přání těch, kdo se ho bojí, slyší, když volají o pomoc, a zachrání je.
#145:20 Všechny, kdo ho milují, Hospodin ochraňuje, ale všechny svévolníky vyhlazuje.
#145:21 Z mých úst zazní chvála Hospodinu, jeho přesvatému jménu bude dobrořečit všechno tvorstvo navěky a navždy. 
#146:1 Haleluja. Chval, duše má, Hospodina!
#146:2 Hospodina budu chválit po celý svůj život, svému Bohu zpívat žalmy, co živ budu.
#146:3 Nedoufejte v knížata, v člověka, u něhož záchrany není.
#146:4 Jeho duch odchází, on se vrací do své země, tím dnem berou za své jeho plány.
#146:5 Blaze tomu, kdo má ku pomoci Boha Jákobova, kdo s nadějí vzhlíží k Hospodinu, svému Bohu,
#146:6 jenž učinil nebesa i zemi s mořem a vším, co k nim patří, jenž navěky zachovává věrnost.
#146:7 Utištěným dopomáhá k právu, hladovým chléb dává. Hospodin osvobozuje vězně.
#146:8 Hospodin otvírá oči slepým, Hospodin sehnuté napřimuje, Hospodin miluje spravedlivé.
#146:9 Hospodin ochraňuje ty, kdo jsou bez domova, ujímá se sirotka i vdovy, svévolným však mate cestu.
#146:10 Hospodin bude kralovat věčně, Bůh tvůj, Sijóne, po všechna pokolení. Haleluja. 
#147:1 Haleluja. Je tak dobré Bohu našemu pět žalmy, rozkošná a líbezná je chvála.
#147:2 Hospodin buduje Jeruzalém, shromažďuje rozehnané z Izraele,
#147:3 uzdravuje ty, kdo jsou zkrušeni v srdci, jejich rány obvazuje.
#147:4 On určuje počet hvězd, on každou vyvolává jménem.
#147:5 Velký je náš Pán, je velmi mocný, jeho myšlení obsáhnout nelze.
#147:6 Hospodin se ujímá pokorných, svévolníky snižuje až k zemi.
#147:7 Pějte Hospodinu píseň díků, zpívejte našemu Bohu žalmy při citaře,
#147:8 tomu, který zahaluje nebe mračny, který připravuje zemi deště, který dává na horách růst trávě,
#147:9 tomu, který zvířatům potravu dává, i krkavčím mláďatům, když křičí.
#147:10 Netěší ho síla koně, nemá zalíbení v svalech muže.
#147:11 Hospodin má zalíbení v těch, kdo se ho bojí, v těch, kdo čekají na jeho milosrdenství.
#147:12 Jeruzaléme, chval zpěvem Hospodina, chval, Sijóne, svého Boha!
#147:13 On upevnil závory v tvých branách, požehnal tvým synům v tobě.
#147:14 Na tvém území ti zjednal pokoj, bělí pšeničnou tě sytí.
#147:15 Na zem vysílá svůj výrok, rychle běží jeho slovo.
#147:16 Dává sníh jak vlnu, sype jíní jako popel.
#147:17 Rozhazuje kroupy jako sousta chleba, kdo odolá jeho mrazu?
#147:18 Sešle slovo své a taje, káže vát větru a plynou vody.
#147:19 Oznámil své slovo Jákobovi, nařízení svá a soudy Izraeli.
#147:20 Tak žádnému z pronárodů neučinil, jeho soudy nepoznaly. Haleluja. 
#148:1 Haleluja. Chvalte Hospodina z nebes, chvalte ho ve výšinách!
#148:2 Chvalte ho, všichni jeho andělé, chvalte ho, všechny jeho zástupy.
#148:3 Chvalte ho, slunce s měsícem, chvalte ho, všechny jasné hvězdy.
#148:4 Chvalte ho, nebesa nebes, rovněž vody nad nebesy,
#148:5 chvalte Hospodinovo jméno! Vždyť on přikázal, a bylo to stvořeno,
#148:6 on tomu dal povstat navěky a navždy, nařízení, které vydal, nepomine.
#148:7 Ze země ať chválí Hospodina netvoři a všechny propastné tůně,
#148:8 oheň, krupobití, sníh i mlha, bouřný vichr, který plní jeho slovo,
#148:9 horstva a všechny pahorky, ovocné stromy a všechny cedry,
#148:10 zvěř a všechna dobytčata, plazi, okřídlené ptactvo,
#148:11 králové země a všechny národy, vladaři a všichni soudci země,
#148:12 jinoši i panny, starci i mladí.
#148:13 Ať chválí Hospodinovo jméno, pouze jeho jméno je vyvýšené, jeho velebnost je nad zemí i nebem.
#148:14 Vyvýšil roh svého lidu k chvále všech svých věrných, synů Izraele, lidu, který je mu blízký. Haleluja. 
#149:1 Haleluja. Zpívejte Hospodinu píseň novou, jeho chválu v shromáždění věrných!
#149:2 Ať se Izrael raduje ze svého Tvůrce, ať synové Sijónu jásají nad svým Králem,
#149:3 ať tanečním rejem chválí jeho jméno, ať mu pějí žalmy při bubnu a při citaře.
#149:4 Hospodin má ve svém lidu zalíbení, pokorné oslaví spásou.
#149:5 Věrní ať jásají v slávě, ať plesají na svých ložích.
#149:6 Ať svým hrdlem vyvyšují Boha, s dvojsečným mečem v svých rukou,
#149:7 aby nad pronárody konali pomstu, tresty na národech,
#149:8 aby spoutali řetězy jejich krále, železnými okovy ty, kdo jsou u nich ve cti,
#149:9 aby na nich vykonali soud, jak o tom psáno. Je to čest pro všechny jeho věrné. Haleluja. 
#150:1 Haleluja. Chvalte Boha v jeho svatyni, chvalte ho i na obloze, již sklenul svou mocí,
#150:2 chvalte ho za jeho bohatýrské činy, chvalte ho pro jeho nesmírnou velikost!
#150:3 Chvalte ho zvukem polnice, chvalte ho harfou a citarou,
#150:4 chvalte ho bubnem a tancem, chvalte ho strunami a flétnou,
#150:5 chvalte ho zvučnými cymbály, chvalte ho cymbály dunivými!
#150:6 Všechno, co má dech, ať chválí Hospodina! Haleluja.  

\book{Proverbs}{Prov}
#1:1 Přísloví Šalomouna, syna Davidova, krále Izraele,
#1:2 jak poznat moudrost a kázeň, jak pochopit výroky rozumnosti,
#1:3 jak si prozíravě osvojit kázeň, spravedlnost, právo a přímost,
#1:4 aby prostoduší byli obdařeni chytrostí, mladík poznáním a důvtipem.
#1:5 Bude-li naslouchat moudrý, přibude mu znalostí, a rozumný získá schopnost
#1:6 porozumět přísloví a jinotaji, slovům mudrců i jejich hádankám.
#1:7 Počátek poznání je bázeň před Hospodinem, moudrostí a kázní pohrdají pošetilci.
#1:8 Můj synu, poslouchej otcovo kárání a matčiným poučováním neopovrhuj.
#1:9 Budou ti půvabným věncem na hlavě a náhrdelníkem na tvém hrdle.
#1:10 Můj synu, kdyby tě lákali hříšníci, nepřivoluj!
#1:11 Kdyby tě přemlouvali: „Pojď s námi, vražedné úklady nastrojíme, počíháme si na nevinného bez důvodu,
#1:12 pohltíme je, jako podsvětí pohlcuje živé, bezúhonní budou jako ti, kdo sestupují do jámy,
#1:13 přijdeme si na rozličný drahocenný majetek a domy kořistí si naplníme,
#1:14 spoj svůj úděl s námi, budeme mít všichni jeden měšec“ -
#1:15 můj synu, nechoď s nimi jejich cestou, zdržuj svou nohu od jejich stezky,
#1:16 neboť jejich nohy běží za zlem, pospíchají prolévat krev.
#1:17 Je zbytečné sypat pod síť, když to každý okřídlenec vidí.
#1:18 Oni však strojí ty vražedné úklady proti sobě, číhají na vlastní duši.
#1:19 Takové jsou stezky všech, kdo za ziskem se honí; ten stojí své pány život.
#1:20 Moudrost pronikavě volá na ulici, na náměstích vydává svůj hlas.
#1:21 Volá na nároží plném hluku, pronáší své výroky u vchodů do městských bran:
#1:22 „Dokdy budete, vy prostoduší, milovat prostoduchost, dokdy posměvači budou mít zálibu v posmívání, hlupáci poznání nenávidět?
#1:23 Obraťte se, když vám domlouvám. Hle, nechám na vás proudit svého ducha, uvedu vám ve známost svá slova:
#1:24 Protože jsem volala, a vy jste odmítali, ruce jsem vztahovala, a nikdo na to nedbal,
#1:25 každé mé radě jste se vyhýbali, nedali jste na mé domlouvání,
#1:26 i já se budu smát, až budete v bídě, budu se vysmívat, až na vás přijde strach,
#1:27 až na vás přijde strach jako ničivá bouře a vaše bída se přižene jako vichřice, až na vás přijde soužení a tíseň.
#1:28 Tehdy mě budou volat, a neodpovím, budou mě hledat za úsvitu, a nenaleznou,
#1:29 protože měli poznání v nenávisti a bázeň před Hospodinem si nezvolili,
#1:30 nedali na mou radu, každou mou domluvu znevážili.
#1:31 Budou jíst plody své cesty, přesytí se svými plány.
#1:32 Prostoduché zavraždí jejich odmítavost, hlupáky zahubí jejich netečnost.
#1:33 Ale kdo mě poslouchá, v bezpečí bude bydlet a žít klidně, beze strachu z něčeho zlého.“ 
#2:1 Můj synu, jestliže přijmeš mé výroky, uchováš-li mé příkazy ve svém nitru,
#2:2 abys věnoval pozornost moudrosti a naklonil své srdce k rozumnosti,
#2:3 jestliže přivoláš porozumění a hlasitě zavoláš na rozumnost,
#2:4 budeš-li ji hledat jako stříbro a pátrat po ní jako po skrytých pokladech,
#2:5 tehdy pochopíš, co je bázeň před Hospodinem, a dojdeš k poznání Boha.
#2:6 Neboť moudrost dává Hospodin, poznání i rozumnost pochází z jeho úst.
#2:7 Pro přímé má pohotovou pomoc, je štítem těm, kdo žijí bezúhonně,
#2:8 chrání stezky práva a střeží cestu svých věrných.
#2:9 Tehdy porozumíš spravedlnosti, právu a přímosti, všemu, co zanechává dobré stopy.
#2:10 Neboť moudrost vejde do tvého srdce a poznání oblaží tvou duši.
#2:11 Tvou stráží stane se důvtip, rozumnost tě bude chránit.
#2:12 Ochrání tě před zlou cestou, před každým, kdo proradně mluví,
#2:13 před těmi, kdo opouštějí přímé stezky a chodí po temných cestách,
#2:14 kdo se radují, když páchají zlo, jásají nad zhoubnými proradnostmi,
#2:15 jejichž stezky jsou křivolaké; jsou neupřímní na každém kroku.
#2:16 Moudrost tě ochrání před cizí ženou, před cizinkou, která se lísá svými řečmi,
#2:17 která opouští druha svého mládí, a na smlouvu svého Boha zapomíná.
#2:18 Propadla smrti i se svým domem, její stopy směřují do říše stínů.
#2:19 Nikdo z těch, kdo k ní vcházejí, se nevrátí, stezek života nedosáhne.
#2:20 Kéž bys jen chodil po cestě dobrých a bedlivě dbal na stezky spravedlivých.
#2:21 Neboť zemi budou obývat přímí a zůstanou v ní bezúhonní.
#2:22 Ale svévolníci budou ze země vyťati, věrolomní budou z ní vyrváni. 
#3:1 Můj synu, na mé učení nezapomínej, ať tvé srdce příkazy mé dodržuje.
#3:2 Prodlouží ti dny a léta života a přidají ti pokoj.
#3:3 Ať tě neopouští milosrdenství a věrnost! Přivaž si je na hrdlo, napiš je na tabulku svého srdce.
#3:4 Tak najdeš milost a uznání v očích Božích i lidských.
#3:5 Důvěřuj Hospodinu celým srdcem, na svoji rozumnost nespoléhej.
#3:6 Poznávej ho na všech svých cestách, on sám napřímí tvé stezky.
#3:7 Nebuď moudrý sám u sebe, boj se Hospodina, od zlého se odvrať.
#3:8 To dá tvému tělu zdraví a svěžest tvým kostem.
#3:9 Cti Hospodina ze svého majetku i prvotinami z celé své úrody!
#3:10 Bohatě se naplní tvé sýpky, moštem budou přetékat tvé kádě.
#3:11 Neodvrhuj, můj synu, Hospodinovo kárání, neprotiv se jeho domlouvání.
#3:12 Vždyť Hospodin domlouvá tomu, koho miluje, jako otec synu, v němž nalezl zalíbení.
#3:13 Blaze člověku, jenž našel moudrost, člověku, jenž došel rozumnosti.
#3:14 Nabýt jí je lepší nežli nabýt stříbra, její výnos je nad ryzí zlato.
#3:15 Je drahocennější než perly, nevyrovnají se jí žádné tvé skvosty.
#3:16 V její pravici je dlouhověkost, v její levici bohatství a čest.
#3:17 Její cesty vedou k blaženosti, všechny její stezky ku pokoji.
#3:18 Stromem života je těm, kdo se jí chopí, blaze těm, kdo se jí drží.
#3:19 Hospodin moudrostí založil zemi, nebesa upevnil rozumností,
#3:20 propastné tůně se jeho věděním rozpoltily a mraky vydaly krůpěje rosy.
#3:21 Můj synu, ať toto ti nesejde z očí. Zachovej si pohotovost a důvtip.
#3:22 To dá tvé duši život a milost tvému hrdlu.
#3:23 Tehdy půjdeš bezpečně svou cestou, tvá noha se neporaní.
#3:24 Ulehneš-li, nebudeš se strachovat, ulehneš a příjemný bude tvůj spánek.
#3:25 Neboj se náhlého strachu, až přijde spoušť na svévolníky.
#3:26 Hospodin bude po tvém boku, bude střežit před pastí tvou nohu.
#3:27 Neodpírej dobrodiní těm, kteří je potřebují, je-li v tvé moci je prokázat.
#3:28 Neříkej svému bližnímu: „Jdi a přijď zase a dám ti to zítra“, když to máš s sebou.
#3:29 Nechystej nic zlého na svého bližního, který s tebou důvěřivě bydlí.
#3:30 S nikým neměj spory bez důvodu, jestliže se proti tobě nedopustil zlého.
#3:31 Nezáviď násilníkovi, žádnou z jeho cest si nezvol.
#3:32 Neboť Hospodin má neupřímného v ohavnosti, s přímými však je v důvěrném obecenství.
#3:33 Hospodinovo prokletí spočívá na domě svévolníka, jeho požehnání na obydlí spravedlivých.
#3:34 Vysmívá se posměvačům, pokorným však dává milost.
#3:35 Čest připadne moudrým do dědictví, hlupáci si však odnesou hanbu. 
#4:1 Synové, slyšte otcovské kárání, pozornost věnujte poznávání rozumnosti.
#4:2 Vždyť jsem vám předal výborné znalosti, moje učení neopouštějte:
#4:3 Když jsem byl jako syn u svého otce, útlý jedináček při své matce,
#4:4 on mě vyučoval, říkával mi: „Drž se celým srdcem mých slov, dbej na mé příkazy a budeš živ.
#4:5 Získej moudrost, získej rozumnost. Na výroky mých úst nezapomeň, neodchyl se od nich.
#4:6 Neopouštěj ji, bude tě střežit. Miluj ji, bude tě chránit.
#4:7 Počátek moudrosti je: Snaž se získat moudrost, za všechno své jmění získej rozumnost.
#4:8 Obklop se jí, vyvýší tě; oslaví tě, když ji obejmeš.
#4:9 Vloží ti na hlavu půvabný věnec, ozdobnou korunu ti předá.
#4:10 Slyš, můj synu, přijmi mé výroky, živ budeš mnoho let.
#4:11 Cestě moudrosti jsem tě učil, vodil tě stopami přímosti.
#4:12 Půjdeš-li jimi, kroky tvé neuvíznou, jestliže poběžíš, neklopýtneš.
#4:13 Napomenutí se chop a neochabuj, dodržuj je, to je tvůj život.
#4:14 Nevcházej na stezku svévolníků, cestou zlých se neubírej.
#4:15 Vyhni se jí, nechoď po ní, odstup od ní a jdi dál.
#4:16 Neusnou, když nespáchají něco zlého, o spánek jsou připraveni, když někoho nepřivedou k pádu.
#4:17 Chlebem svévole se živí, vínem násilí se napájejí.
#4:18 Stezka spravedlivých je jak jasné světlo, které svítí stále víc, až je tu den.
#4:19 Cesta svévolníků je však jako soumrak, nevědí, o co klopýtnou.
#4:20 Můj synu, věnuj pozornost mým slovům, k mým výrokům nakloň ucho.
#4:21 Ať nesejdou ti z očí, střez je v hloubi srdce.
#4:22 Dají život těm, kteří je nalézají, a zdraví celému jejich tělu.
#4:23 Především střez a chraň své srdce, vždyť z něho vychází život.
#4:24 Odvracej svá ústa od falše a od svých rtů vzdal neupřímnost.
#4:25 Tvé oči ať hledí rovně, zpříma před sebe se dívej.
#4:26 Sleduj stopy svých nohou, všechny tvé cesty ať jsou pevné.
#4:27 Napravo ani nalevo se neuchyluj, odvrať od zlého svou nohu. 
#5:1 Můj synu, věnuj pozornost mé moudrosti, nakloň své ucho k mé rozumnosti,
#5:2 dbej na obezřetné rady, poznání ať zachovávají tvé rty.
#5:3 Ze rtů cizí ženy sice kape med, a její jazyk je hladší než olej,
#5:4 nakonec je však hořká jako pelyněk, ostrá jak dvojsečný meč.
#5:5 Její nohy sestupují k smrti, její kroky uvíznou v podsvětí.
#5:6 Nevysleduješ stezku života, její stopy se motají, nevíš kam.
#5:7 Proto, synové, poslyšte mě, neodvracejte se od výroků mých úst.
#5:8 Ať jde tvá cesta daleko od ní, nepřibližuj se ke dveřím jejího domu,
#5:9 ať nevydáš svou důstojnost jiným, ukrutníkovi svá léta,
#5:10 ať se tvou silou nesytí cizáci a ovocem tvého trápení dům cizí.
#5:11 Nakonec budeš skučet, až celé tvé tělo zchátrá.
#5:12 Řekneš: „Jak jsem mohl nenávidět napomínání? Jak mohlo mé srdce znevažovat domlouvání?
#5:13 Svoje vychovatele jsem neposlouchal, nenaslouchal jsem svým učitelům.
#5:14 Málem bych byl propadl nejhoršímu uprostřed shromáždění a pospolitosti.“
#5:15 Pij vodu z vlastní nádrže, tu, jež vyvěrá z tvé studnice.
#5:16 Mají se tvé prameny roztékat ven do široka jako vodní toky?
#5:17 Tobě mají patřit, tobě jedinému, a ne cizím spolu s tebou.
#5:18 Buď požehnán tvůj zdroj, raduj se z ženy svého mládí,
#5:19 z milované laně, z líbezné srny; její prsy ať tě vždycky opojují, kochej se v jejím milování ustavičně.
#5:20 Proč by ses kochal, můj synu, v cizačce, proč bys v náručí cizinku svíral?
#5:21 Cesty člověka jsou Hospodinu zřejmé, on sleduje všechny jeho stopy.
#5:22 Svévolníka polapí jeho zločiny, bude spoután provazy svého hříchu.
#5:23 Nedbal na napomenutí, proto zemře; bloudí pro svou velkou pošetilost. 
#6:1 Můj synu, jestliže ses zaručil za svého druha nebo se zavázal rukoudáním za cizáka
#6:2 a zapletl se výroky svých úst, a výroky svých úst se chytil,
#6:3 udělej, můj synu, toto: Hleď se vyprostit. Dostal ses do rukou svého druha. Jdi, vrhni se do bláta a naléhej na svého druha.
#6:4 Nedopřej svým očím spánku, ani zdřímnout nedávej svým víčkům.
#6:5 Jak gazela vytrhni se z rukou, jako ptáče z rukou čihařových.
#6:6 Jdi k mravenci, lenochu, dívej se, jak žije, ať zmoudříš.
#6:7 Ač nemá žádného vůdce, dozorce či vládce,
#6:8 opatřuje si v létě pokrm, o žních sklízí svou potravu.
#6:9 Jak dlouho, lenochu, budeš ležet? Kdy se probudíš ze svého spánku?
#6:10 Trochu si pospíš, trochu zdřímneš, trochu složíš ruce v klín a poležíš si
#6:11 a tvá chudoba přijde jak pobuda a tvá nouze jako ozbrojenec.
#6:12 Ničemný člověk, muž propadlý ničemnostem, má plná ústa falše,
#6:13 mrká očima, nohama cosi naznačuje, svými prsty ukazuje.
#6:14 V srdci má proradnost, osnuje zlo v každém čase, vyvolává sváry.
#6:15 Proto náhlá pohroma ho stihne, bude nenadále rozdrcen a nezhojí ho nikdo.
#6:16 Těchto šest věcí Hospodin nenávidí a sedmá je mu ohavností:
#6:17 přezíravé oči, zrádný jazyk, ruce, které prolévají nevinnou krev,
#6:18 srdce osnující ničemné plány, nohy rychle spěchající za zlem,
#6:19 křivý svědek, který šíří lži, a ten, kdo vyvolává mezi bratry sváry.
#6:20 Dodržuj, můj synu, otcovy příkazy, a matčiným poučováním neopovrhuj.
#6:21 Přivaž si je natrvalo k srdci, oviň si je kolem hrdla.
#6:22 Povedou tě, kamkoli půjdeš, když budeš ležet, budou tě střežit, procitneš a budou s tebou rozmlouvat.
#6:23 Vždyť příkaz je světlem a vyučování osvěcuje, domluvy a kárání jsou cesty k životu:
#6:24 Budou tě střežit před špatnou ženou, před úlisným jazykem cizinky.
#6:25 Nedychti v srdci po její kráse, ať tě svými řasami neuchvátí!
#6:26 Nevěstce zaplatíš bochníčkem chleba, žena jiného však loví drahou duši.
#6:27 Může si kdo shrnout do klína oheň a nespálit si šaty?
#6:28 Což může někdo chodit po žhavém uhlí, a nepopálit si nohy?
#6:29 Tak dopadne ten, kdo vchází k ženě svého druha; nezůstane bez trestu, kdo se jí dotkne.
#6:30 Nepohrdá se zlodějem, že kradl, aby se nasytil, když měl hlad.
#6:31 Je-li však přistižen, nahradí to sedmeronásobně, dá všechen majetek svého domu.
#6:32 Kdo s ženou cizoloží, nemá rozum, k vlastní zkáze to činí.
#6:33 Sklidí jen rány a hanbu a jeho potupa nebude smazána.
#6:34 Neboť žárlivost rozpálí muže, ten bude v den pomsty nelítostný.
#6:35 Nepřijme žádný dar na usmířenou, nepovolí, i kdybys sebevíc úplatků dával. 
#7:1 Můj synu, dbej na mé výroky, chovej mé příkazy ve svém nitru.
#7:2 Dbej na mé příkazy a budeš živ, střez moje učení jak zřítelnici oka.
#7:3 Přivaž si je k prstům, napiš je na tabulku svého srdce.
#7:4 Moudrosti řekni: „Jsi moje sestra“, rozumnost nazvi svou příbuznou,
#7:5 aby tě střežila před cizí ženou, před cizinkou, která lichotí svými řečmi.
#7:6 Jednou jsem vyhlížel mříží z okna svého domu
#7:7 a díval jsem se na prostoduché; pozoroval jsem mezi těmi synky mladíka, který neměl rozum.
#7:8 Přecházel ulici kolem jejího nároží, vykročil směrem k jejímu domu
#7:9 na sklonku dne, za soumraku, pod záštitou temnoty noční.
#7:10 A hle, žena mu jde vstříc v nevěstčím úboru se záludným srdcem.
#7:11 Je halasná, dotěrná, její nohy nemají doma stání.
#7:12 Hned je na ulici, hned na náměstí, na každém nároží úklady strojí.
#7:13 Uchopí jej, políbí ho, s nestoudnou tváří mu řekne:
#7:14 „Vystrojila jsem pokojné obětní hody, vyplnila jsem dnes svoje sliby.
#7:15 Proto jsem ti vyšla vstříc a za úsvitu jsem tě hledala, až jsem tě našla.
#7:16 Prostřela jsem na své lehátko přehozy, pestrá egyptská prostěradla.
#7:17 Navoněla jsem své lůžko myrhou, aloe a skořicí.
#7:18 Pojď, opájejme se laskáním až do jitra, potěšme se milováním.
#7:19 Muž není doma, odešel na dalekou cestu.
#7:20 Váček s penězi vzal s sebou, vrátí se domů až v den úplňku.“
#7:21 Naklonila si ho mnohým přemlouváním, svými úlisnými rty ho svedla.
#7:22 Hned šel za ní jako vůl na porážku, jako pošetilec v poutech k potrestání,
#7:23 než mu šíp rozetne játra; spěchá do osidla jako ptáče, neví, že mu jde o život.
#7:24 Nyní tedy, synové, slyšte mě, věnujte pozornost výrokům mých úst.
#7:25 Ať tvé srdce nesejde na její cesty, na její pěšiny se nedej zavést.
#7:26 Mnohé už sklála, přivedla k pádu, všichni, i ti nejzdatnější, byli od ní zavražděni.
#7:27 Její dům - toť cesty do podsvětí, vedoucí do komor smrti. 
#8:1 Cožpak moudrost nevolá, nevydává rozumnost svůj hlas?
#8:2 Na nejvyšším místě, nad cestou, na křižovatce stojí.
#8:3 Při branách, kudy se chodí do města, u vchodu pronikavě volá:
#8:4 „Na vás, muži, volám, můj hlas je určen synům lidským.
#8:5 Prostoduší, pochopte, v čem je chytrost, hlupáci, pochopte, v čem je rozum!
#8:6 Slyšte! Vyhlašuji, co je směrodatné, otevírám rty a plyne z nich přímost.
#8:7 O pravdě hovoří můj jazyk, mým rtům se svévole hnusí.
#8:8 Všechny výroky mých úst jsou spravedlivé, není v nich nic potměšilého či falešného.
#8:9 Všechny jsou správné pro toho, kdo porozumí, přímé těm, kdo naleznou poznání.
#8:10 Přijměte mé napomenutí, nikoli stříbro, spíše poznání, než převýborné ryzí zlato.
#8:11 Moudrost je lepší než perly, nevyrovnají se jí žádné skvosty.
#8:12 Já, Moudrost, bydlím s chytrostí, nalézám obezřetné poznání.
#8:13 Bázeň před Hospodinem znamená nenávidět zlo; nenávidím povýšenost, pýchu, cestu zlou, proradná ústa.
#8:14 U mne je rada i pohotová pomoc, jsem rozumnost, u mne je bohatýrská síla.
#8:15 Skrze mne kralují králové a vydávají spravedlivá nařízení vládci,
#8:16 skrze mne velí velitelé a všichni urození soudí spravedlivě.
#8:17 Já miluji ty, kdo milují mne, a kdo mě za úsvitu hledají, naleznou mne.
#8:18 Bohatství a čest jsou u mne, ustavičný dostatek i spravedlnost.
#8:19 Mé ovoce je lepší než ryzí a čisté zlato, má úroda je nad výborné stříbro.
#8:20 Chodím stezkou spravedlnosti a pěšinami práva,
#8:21 abych dala dědictví těm, kdo mě milují. Já naplním jejich pokladnice.
#8:22 Hospodin mě vlastnil jako počátek své cesty, dříve než co konal odedávna.
#8:23 Od věků jsem ustanovena, od počátku, od pravěku země.
#8:24 Ještě nebyly propastné tůně, když jsem se zrodila, ještě nebyly prameny vodami obtěžkány.
#8:25 Když ještě byly hory ponořeny, před pahorky jsem se narodila.
#8:26 Ještě než učinil zemi a všechno kolem a první hroudy pevniny,
#8:27 když upevňoval nebesa, byla jsem při tom, když vymezoval obzor nad propastnou tůní,
#8:28 když seshora zavěšoval mračna, když sílily prameny propastné tůně,
#8:29 když kladl moři jeho meze, aby vody nevystupovaly z břehů, když vymezoval základy země,
#8:30 byla jsem mu věrně po boku, byla jsem jeho potěšením den ze dne a radostně si před ním hrála v každý čas.
#8:31 Hraji si na jeho pevné zemi; mým potěšením je být s lidskými syny.
#8:32 Nyní tedy, synové, slyšte mě: Blaze těm, kdo dbají na mé cesty.
#8:33 Slyšte napomenutí a buďte moudří; nevyhýbejte se tomu!
#8:34 Blaze člověku, který mě poslouchá, bdí u mých dveří den ze dne a střeží veřeje mého vchodu.
#8:35 Vždyť ten, kdo mě nalézá, nalezl život a došel u Hospodina zalíbení.
#8:36 Kdo hřeší proti mně, činí násilí své duši; všichni, kdo mě nenávidí, milují smrt. 
#9:1 Moudrost si vystavěla dům, vytesala sedm sloupů.
#9:2 Porazila dobytče, smísila víno a prostřela svůj stůl.
#9:3 Vyslala své dívky, volá na vrcholu městských výšin:
#9:4 „Kdo je prostoduchý, ať se sem uchýlí!“ Toho, kdo nemá rozum, zve:
#9:5 „Pojďte, jezte můj chléb a pijte víno, které jsem smísila,
#9:6 nechte prostoduchosti a budete živi, kráčejte cestou rozumnosti!“
#9:7 Kdo napomíná posměvače, dojde pohany, kdo domlouvá svévolníku, poskvrní se!
#9:8 Nedomlouvej posměvači, aby tě nezačal nenávidět. Domlouvej moudrému a bude tě milovat.
#9:9 Moudrému dej a bude ještě moudřejší, pouč spravedlivého a přibude mu znalostí.
#9:10 Začátek moudrosti je bázeň před Hospodinem a poznat Svatého je rozumnost.
#9:11 Neboť skrze mne se rozhojní tvé dny, přibude ti let života.
#9:12 Jsi-li moudrý, k svému prospěchu jsi moudrý, jsi-li posměvač, sám na to doplatíš.
#9:13 Paní Hloupost je halasná, prostoduchá, vůbec nic nezná.
#9:14 Sedá u vchodu do svého domu, na křesle na městských výšinách,
#9:15 a volá na mimojdoucí, kteří jdou po přímých stezkách:
#9:16 „Kdo je prostoduchý, ať odbočí sem!“ Toho, kdo nemá rozum, zve:
#9:17 „Kradená voda je sladká a pokoutný chléb blaží.“
#9:18 Ale on neví, že je tam říše stínů a že v hlubinách podsvětí jsou ti, jež pozvala. 
#10:1 Přísloví Šalomounova. Syn moudrý dělá radost otci, kdežto syn hloupý působí žal matce.
#10:2 Neprávem nabyté poklady neprospějí, kdežto spravedlnost vysvobodí od smrti.
#10:3 Hospodin nedopustí, aby hladověl spravedlivý, kdežto choutkám svévolníků činí přítrž.
#10:4 Zchudne, kdo pracuje zahálčivou dlaní, kdežto pilné jejich ruka obohatí.
#10:5 Kdo v létě sklízí, je syn prozíravý, kdo prospí žně, je pro ostudu.
#10:6 Mnohé požehnání spočine na hlavě spravedlivého, kdežto v ústech svévolníků se skrývá násilí.
#10:7 Památka spravedlivého bude k požehnání, kdežto jméno svévolníků zpráchniví.
#10:8 Kdo je moudrého srdce, přijímá příkazy, kdežto kdo je pošetilých rtů, ten padne.
#10:9 Kdo žije bezúhonně, žije bezpečně, kdežto kdo chodí křivolakými cestami, bude odhalen.
#10:10 Kdo mrká okem, působí trápení, a kdo je pošetilých rtů, ten padne.
#10:11 Ústa spravedlivého jsou zdrojem života, kdežto v ústech svévolníků se skrývá násilí.
#10:12 Nenávist vyvolává sváry, kdežto láska přikrývá všechna přestoupení.
#10:13 Na rtech rozumného se nalézá moudrost, kdežto hůl dopadne na hřbet toho, kdo je bez rozumu.
#10:14 Moudří uchovávají poznání, kdežto ústa pošetilce přinášejí zkázu.
#10:15 Majetek bohatého je jeho pevnou tvrzí, chudoba nuzných je jejich zkázou.
#10:16 Výdělek spravedlivého slouží k životu, výtěžek svévolníkův k hříchu.
#10:17 Ukazuje stezku k životu, kdo dbá na napomenutí, kdežto kdo si domluv nevšímá, zavádí na scestí.
#10:18 Kdo skrývá nenávist za zrádné rty, i ten, kdo šíří pomluvy, je hlupák.
#10:19 Mnohomluvnost nezůstává bez přestoupení, kdežto kdo krotí své rty, je prozíravý.
#10:20 Jazyk spravedlivého je výborné stříbro, srdce svévolníků nestojí za nic.
#10:21 Rty spravedlivého připravují pastvu mnohým, ale pošetilci umírají na to, že jsou bez rozumu.
#10:22 Hospodinovo požehnání obohacuje a trápení s sebou nepřináší.
#10:23 Radostnou hrou je pro hlupáka mrzké jednání, kdežto pro rozumného muže moudrost.
#10:24 Čeho se leká svévolník, to na něj přijde, kdežto touha spravedlivých se splní.
#10:25 Když se přižene vichřice, je po svévolníkovi, kdežto spravedlivý má základ věčný.
#10:26 Jako ocet zubům a kouř očím, tak je lenoch těm, kteří ho posílají.
#10:27 Bázeň před Hospodinem přidává dnů, kdežto svévolníkům se léta zkrátí.
#10:28 Očekávání spravedlivých je radostné, kdežto naděje svévolníků přijde vniveč.
#10:29 Cesta Hospodinova je záštitou bezúhonnému, kdežto zkázou pro ty, kdo páchají ničemnosti.
#10:30 Spravedlivý se nikdy nezhroutí, kdežto svévolníci zemi nezabydlí.
#10:31 Ústa spravedlivého plodí moudrost, kdežto jazyk proradný bude vyťat.
#10:32 Rty spravedlivého vědí, v čem má Bůh zalíbení, kdežto ústa svévolníků znají jen proradnost. 
#11:1 Falešné váhy jsou Hospodinu ohavností, kdežto v přesném závaží má zalíbení.
#11:2 Za zpupností přichází hanba, kdežto s umírněnými je moudrost.
#11:3 Přímé vede bezúhonnost, kdežto věrolomné zahubí pokřivenost.
#11:4 V den prchlivosti neprospěje majetek, kdežto spravedlnost vysvobodí od smrti.
#11:5 Spravedlnost napřimuje bezúhonnému cestu, kdežto svévolník svou svévolí padne.
#11:6 Spravedlnost přímé vysvobodí, kdežto věrolomní se lapí do svých choutek.
#11:7 Když zemře člověk svévolný, naděje přijde vniveč, vniveč přijde očekávání ničemníků.
#11:8 Spravedlivý bývá zachován, je-li v soužení, kdežto svévolník se dostane na jeho místo.
#11:9 Rouhač ústy uvádí druha do zkázy, kdežto spravedliví budou zachováni věděním.
#11:10 Když je dobře spravedlivým, město jásá, když zhynou svévolníci, plesá.
#11:11 Žehnáním přímých se město pozvedá, kdežto ústy svévolníků se boří.
#11:12 Kdo je bez rozumu, pohrdá svým druhem, kdežto muž rozumný mlčí.
#11:13 Utrhač, kudy chodí, roznáší důvěrnosti, kdežto kdo je duchem věrný, ukryje to.
#11:14 Pro nerozvážné vedení padá lid, kdežto v množství rádců je záchrana.
#11:15 Velmi zle dopadne ten, kdo se zaručil za cizího, kdežto kdo zaručování nenávidí, je v bezpečí.
#11:16 Ušlechtilá žena se drží cti, kdežto ukrutníci se drží bohatství.
#11:17 Muž milosrdný činí dobře i sám sobě, kdežto nelítostný drásá i vlastní tělo.
#11:18 Svévolníka výdělek z jeho práce zklame, kdežto kdo rozsévá spravedlnost, má mzdu jistou.
#11:19 Tak spravedlnost vede k životu, kdežto kdo se žene za zlem, spěje k smrti.
#11:20 Hospodin má v ohavnosti lidi falešného srdce, kdežto zalíbení má v těch, jejichž cesta je bezúhonná.
#11:21 Zlý zaručeně nezůstane bez trestu, kdežto potomstvo spravedlivých bude ušetřeno.
#11:22 Zlatý kroužek na rypáku vepře je žena krásná, ale svéhlavá a rozmarná.
#11:23 Touhou spravedlivých je jen dobro, kdežto nadějí svévolníků je prosadit se zuřivostí.
#11:24 Někdo rozdává a přibývá mu stále, kdežto ten, kdo je skoupý, mívá nedostatek.
#11:25 Duše štědrá bude nasycena tukem, a kdo občerstvuje, bude též občerstven.
#11:26 Na toho, kdo zadržuje obilí, láteří národ, kdežto žehnání se snáší na hlavu toho, kdo je prodává.
#11:27 Kdo usilovně hledá dobro, hledá Boží zalíbení, kdežto kdo se pídí po zlu, toho zlo postihne.
#11:28 Kdo doufá ve své bohatství, padne, kdežto spravedliví budou rašit jako listí.
#11:29 Kdo rozvrací svůj dům, zdědí vítr, a pošetilec bude otrokem toho, kdo je moudrého srdce.
#11:30 Ovoce spravedlivého je jako strom života, a kdo se ujímá duší, je moudrý.
#11:31 Hle, spravedlivý dochází na zemi odplaty, tím spíše svévolník a hříšník. 
#12:1 Kdo miluje napomenutí, miluje poznání, kdežto kdo domluvy nenávidí, je tupec.
#12:2 Dobrý dochází u Hospodina zalíbení, kdežto pleticháře prohlašuje Hospodin za svévolníka.
#12:3 Svévolí se nikdo nezajistí, kořen spravedlivých se nezviklá.
#12:4 Žena statečná je korunou svého manžela, kdežto ostudná mu je jako kostižer v kostech.
#12:5 Spravedliví přemýšlejí o právu, svévolníci rozvažují o lsti.
#12:6 Slova svévolníků jsou vražedné úklady, kdežto přímé jejich ústa vysvobodí.
#12:7 Svévolníci budou podvráceni a nebudou již, kdežto dům spravedlivých obstojí.
#12:8 Muž bude chválen pro svá prozíravá ústa, kdežto ten, kdo má zvrácené srdce, upadne v pohrdání.
#12:9 Lépe je být zlehčován a mít otroka, než pokládat se za slavného a nemít chleba.
#12:10 Spravedlivý cítí i se svým dobytkem, kdežto nitro svévolníků je nelítostné.
#12:11 Kdo obdělává svou půdu, nasytí se chlebem, kdežto kdo následuje povaleče, nemá rozum.
#12:12 Svévolník dychtivě loví kdejaké zlo, kdežto kořen spravedlivých vydává dobro.
#12:13 Zlovolník se chytí do přestupků svých rtů, kdežto spravedlivý vyvázne ze soužení.
#12:14 Ovocem svých úst se každý dobře nasytí, skutek rukou se člověku vrátí.
#12:15 Pošetilci zdá se, že je jeho cesta přímá, kdežto kdo poslouchá rady, je moudrý.
#12:16 U pošetilce se jeho hoře pozná týž den, kdežto chytrý přikryje hanbu.
#12:17 Kdo šíří pravdu, hlásá spravedlnost, kdežto křivý svědek lest.
#12:18 Někdo tlachá, jako by probodával mečem, kdežto jazyk moudrých hojí.
#12:19 Pravdivé rty se zajistí navždy, kdežto jazyk zrádný na okamžik.
#12:20 Lest mají v srdci ti, kdo osnují zlo, kdežto radost ti, kdo radí ku pokoji.
#12:21 Žádná ničemnost nezasáhne spravedlivého, kdežto svévolníci okusí zlého vrchovatě.
#12:22 Zrádné rty jsou Hospodinu ohavností, kdežto zalíbení má v těch, kdo prosazují pravdu.
#12:23 Chytrý člověk poznání skrývá, kdežto srdce hlupáků pošetilosti provolává.
#12:24 Ruka pilných bude vládnout, kdežto zahálka vede do poroby.
#12:25 Obavy lidské srdce tíží, ale dobré slovo vrací radost.
#12:26 Spravedlivý prozkoumává svému příteli cestu, kdežto svévolníky jejich cesta zavede.
#12:27 Zahálčivý nebude si péci úlovek, kdežto vzácným jměním pro člověka je píle.
#12:28 Na stezce spravedlnosti je život, tato cesta nesměřuje k smrti. 
#13:1 Syn otcovským káráním zmoudří, kdežto posměvač pohrůžky neposlouchá.
#13:2 Ovoce svých úst se každý dobře nají, duše věrolomných okusí násilí.
#13:3 Kdo hlídá svá ústa, střeží svůj život, kdo se pošklebuje, toho stihne zkáza.
#13:4 Lenoch jen touží a ničeho nedosáhne, kdežto pilní se nasytí tukem.
#13:5 Spravedlivý nenávidí jakýkoli klam, kdežto svévolník vzbuzuje nelibost a hnus.
#13:6 Spravedlnost chrání toho, kdo žije bezúhonně, kdežto svévole vyvrací hříšníka.
#13:7 Leckdo se vydává za bohatého, ačkoli nic nemá, někdo se vydává za chudého, a má velké jmění.
#13:8 Bohatstvím se člověk může vyplatit, kdežto chudý vyhrožování neslýchá.
#13:9 Světlo spravedlivých radostně září, kdežto svévolníkům svítilna hasne.
#13:10 Ze zpupnosti vzniká jen hádka, kdežto u těch, kdo si dají poradit, je moudrost.
#13:11 Jmění snadno nabyté se zmenšuje, kdežto kdo shromažďuje vlastní prací, tomu přibývá.
#13:12 Dlouhým čekáním zemdlívá srdce, kdežto splněná touha je stromem života.
#13:13 Kdo pohrdá slovem, přivádí se do záhuby, kdežto kdo se bojí přikázání, dochází odplaty.
#13:14 Učení moudrého je zdrojem života, pomůže uniknout léčkám smrti.
#13:15 Prozíravost zjednává přízeň, kdežto věrolomní vytrvávají na své cestě.
#13:16 Kdo je chytrý, počíná si podle poznání, kdežto hlupák roztrušuje pošetilost.
#13:17 Svévolný posel propadne zkáze, kdežto věrný vyslanec přináší zdraví.
#13:18 Chudoba a hanba stihnou toho, kdo se vyhýbá trestu, kdežto kdo dbá na domluvu, bude vážený.
#13:19 Když se naplní touha, je sladko v duši, kdežto hlupákům se hnusí odvrátit se od zlého.
#13:20 Kdo chodívá s moudrými, stane se moudrým, kdežto tomu, kdo se přátelí s hlupáky, se povede zle.
#13:21 Hříšníky stíhá zlo, kdežto spravedlivým je odplatou dobro.
#13:22 Dobrý zanechá dědictví vnukům, kdežto jmění hříšníka bývá uchováno pro spravedlivého.
#13:23 Úhor dává chudým mnoho pokrmu, ale bývá ničen bezprávím.
#13:24 Kdo šetří hůl, nenávidí svého syna, kdežto kdo jej miluje, trestá ho včas.
#13:25 Spravedlivý se nají dosyta, kdežto břicho svévolníků trpí nedostatkem. 
#14:1 Moudrá žena buduje svůj dům, kdežto pošetilá jej vlastníma rukama boří.
#14:2 Kdo se bojí Hospodina, chodí přímo, kdežto kdo jím pohrdá, chodí cestou neupřímnosti.
#14:3 Z úst pošetilce vyrůstá povýšenost, kdežto moudré jejich rty střeží.
#14:4 Kde není skot, je čistý žlab, kdežto hojná úroda bývá, kde jsou silní býci.
#14:5 Spolehlivý svědek nelže, kdežto křivý svědek šíří lži.
#14:6 Posměvač hledá moudrost, ale marně, kdežto rozumný má poznání usnadněné.
#14:7 Jdi z cesty muži hloupému, neboť na jeho rtech poznání nenalezneš.
#14:8 Moudrost chytrého je v tom, že rozumí své cestě, kdežto pošetilost hlupáků je v záludnosti.
#14:9 Oběť za vinu je pošetilcům směšná, kdežto u přímých nalézá zalíbení.
#14:10 Jen srdce zná hořkost vlastní duše, ani do jeho radosti se nikdo cizí nemůže vmísit.
#14:11 Dům svévolníků bude vyhlazen, kdežto stánek přímých bude vzkvétat.
#14:12 Někdy se člověku zdá cesta přímá, ale nakonec přivede k smrti.
#14:13 Také při smíchu bolívá srdce a na konci radosti bývá žal.
#14:14 Odpadlík se sytí svými cestami, kdežto člověk dobrý tím, co je mu dáno.
#14:15 Prostoduchý kdečemu důvěřuje, kdežto chytrý promýšlí své kroky.
#14:16 Moudrý se bojí a odvrací se od zlého, kdežto hlupák se vypíná a cítí se v bezpečí.
#14:17 Nedočkavý se dopouští pošetilosti a pletichář je nenáviděn.
#14:18 Prostoduší dědí pošetilost, kdežto chytří jsou korunováni poznáním.
#14:19 Zlí se skloní před dobrými a svévolníci u bran spravedlivého.
#14:20 Chudého nemá rád ani jeho druh, kdežto mnoho je těch, kdo milují bohatého.
#14:21 Kdo pohrdá svým druhem, hřeší, kdežto blaze tomu, kdo se slitovává nad utištěnými.
#14:22 Což nebloudí ti, kdo osnují zlé věci? Kdežto milosrdenství a věrnost provází ty, kdo chystají dobro.
#14:23 Každé trápení je k užitku, ale pouhé mluvení vede k nedostatku.
#14:24 Korunou moudrých je jejich bohatství, ale pošetilost hlupáků zůstává pošetilostí.
#14:25 Pravdivý svědek druhé vysvobodí, kdežto lstivý šíří lži.
#14:26 V bázni před Hospodinem má člověk pevné bezpečí a útočiště pro své syny.
#14:27 Bázeň před Hospodinem je zdroj života, pomůže uniknout léčkám smrti.
#14:28 V množství lidu spočívá důstojnost krále, kdežto úbytek národa je zkáza pro vládce.
#14:29 Shovívavý oplývá rozumností, kdežto ukvapený vystavuje na odiv pošetilost.
#14:30 Mírné srdce je tělu k životu, kdežto žárlivost je jako kostižer.
#14:31 Kdo utiskuje nuzného, tupí toho, kdo jej učinil, kdežto ctí ho ten, kdo se nad ubožákem slitovává.
#14:32 Svévolník bude sražen zlem, jež páchá, kdežto spravedlivý má útočiště i při smrti.
#14:33 V srdci rozumného spočívá moudrost, kdežto co je v nitru hlupáků, se pozná.
#14:34 Spravedlnost vyvyšuje pronárody, kdežto hřích je národům pro potupu.
#14:35 Král má zalíbení v prozíravém služebníku, kdežto jeho prchlivost dolehne na toho, kdo jedná ostudně. 
#15:1 Vlídná odpověď odvrací rozhořčení, kdežto slovo, které ubližuje, popouzí k hněvu.
#15:2 Jazyk moudrých přináší dobré poznání, kdežto ústa hlupáků chrlí pošetilost.
#15:3 Oči Hospodinovy jsou na každém místě, pozorně sledují zlé i dobré.
#15:4 Mírný jazyk je stromem života, kdežto pokřivený rozkládá ducha.
#15:5 Pošetilec znevažuje otcovské kárání, kdežto kdo na domluvy dbá, jedná chytře.
#15:6 Dům spravedlivého je rozsáhlou klenotnicí, kdežto z výtěžku svévolníkova vzejde rozvrat.
#15:7 Rty moudrých rozsívají poznání, ale ne tak srdce hlupáků.
#15:8 Hospodinu je ohavností oběť svévolníků, kdežto v modlitbě přímých má zalíbení.
#15:9 Hospodinu je ohavností cesta svévolníka, ale miluje toho, kdo následuje spravedlnost.
#15:10 Tvrdý trest postihne toho, kdo opouští stezku, kdo domluvy nenávidí, zemře.
#15:11 I podsvětí, říše zkázy, je na očích Hospodinu, tím spíše srdce synů lidských.
#15:12 Posměvač nemiluje toho, kdo mu domlouvá, k moudrým nechodí.
#15:13 Radostným srdcem zkrásní tvář, kdežto trápení srdce ubíjí ducha.
#15:14 Rozumné srdce hledá poznání, kdežto ústa hlupáků se honí za pošetilostí.
#15:15 Všechny dny utištěného jsou zlé, kdežto kdo je dobré mysli, má hody každodenně.
#15:16 Je lépe mít málo a bát se Hospodina, než mít velký poklad a s ním neklid.
#15:17 Lepší je jídlo ze zeleniny a k tomu láska, než z vykrmeného býka a s tím nenávist.
#15:18 Vznětlivý muž rozněcuje sváry, kdežto shovívavý spory uklidňuje.
#15:19 Cesta lenocha je zarostlá trním, kdežto stezka přímých je upravená.
#15:20 Syn moudrý dělá radost otci, kdežto člověk hloupý pohrdá svou matkou.
#15:21 Z pošetilosti se raduje, kdo nemá rozum, kdežto rozumný muž chodí přímo.
#15:22 Plány selžou bez společné porady, kdežto při množství rádců se uskuteční.
#15:23 Člověk má radost, když může dát odpověď, jak je dobré slovo v pravý čas!
#15:24 Stezka života vede prozíravého vzhůru, aby unikl podsvětí dole.
#15:25 Hospodin strhne dům pyšných, kdežto mezníky vdovy vytyčuje.
#15:26 Zlé plány jsou Hospodinu ohavností, kdežto vlídná řeč je čistá.
#15:27 Kdo se žene za ziskem, rozvrací svůj dům, kdežto kdo o dary nestojí, bude živ.
#15:28 Srdce spravedlivého rozjímá, co odpovědět, kdežto ústa svévolníků chrlí samou zlobu.
#15:29 Hospodin je daleko od svévolníků, kdežto modlitbu spravedlivých vyslýchá.
#15:30 Zářivý pohled vlévá do srdce radost, dobrá zpráva vzpružuje kosti.
#15:31 Ucho, které poslouchá životodárné domluvy, bude přebývat mezi moudrými.
#15:32 Kdo se vyhýbá trestu, neváží si vlastního života, kdežto kdo domluvy poslouchá, získává rozum.
#15:33 Bázeň před Hospodinem napomíná k moudrosti, slávu předchází pokora. 
#16:1 Člověku je dáno pořádat, co má na srdci, ale na Hospodinu záleží, co odpoví jazyk.
#16:2 Člověku se všechny jeho cesty zdají ryzí, ale pohnutky zpytuje Hospodin.
#16:3 Svěř Hospodinu své počínání a tvé plány budou zajištěny.
#16:4 Hospodin učinil vše k svému cíli, i svévolníka pro zlý den.
#16:5 Hospodin má každého domýšlivce v ohavnosti, zaručeně nezůstane bez trestu.
#16:6 Milosrdenstvím a věrností se usmiřuje provinění a bázeň před Hospodinem odvrací od zlého.
#16:7 Líbí-li se Hospodinu cesty člověka, vede ku pokoji s ním i jeho nepřátele.
#16:8 Lepší je maličko se spravedlností než mnoho výtěžků s bezprávím.
#16:9 Člověk uvažuje v srdci o své cestě, ale jeho kroky řídí Hospodin.
#16:10 Na rtech králových je božský výrok, při soudu se jeho ústa nezpronevěří.
#16:11 Vahadla i správné misky patří Hospodinu, všechna závaží jsou jeho dílem.
#16:12 Králům se hnusí svévolně jednat, vždyť spravedlnost upevňuje trůn.
#16:13 Králové mají zalíbení ve spravedlivých rtech, a toho, kdo mluví přímo, milují.
#16:14 Královo rozhořčení je poselstvo smrti, ale moudrý muž je usmíří.
#16:15 V jasné tváři králově je život, jeho přízeň je jak oblak s jarním deštěm.
#16:16 Získat moudrost je lepší než ryzí zlato a získat rozumnost je výbornější než stříbro.
#16:17 Přímí se na své dráze odvracejí od zlého; střeží svůj život, kdo hlídá svou cestu.
#16:18 Pýcha předchází pád, domýšlivost klopýtnutí.
#16:19 Je lépe být poníženého ducha s pokornými, než se dělit o kořist s pyšnými.
#16:20 Kdo je prozíravý ve slovu, nalézá dobro, blaze tomu, kdo doufá v Hospodina.
#16:21 Kdo je moudrého srdce, je nazýván rozumným; lahodná řeč přidává znalostí.
#16:22 Prozíravost je zdrojem života těm, kdo ji mají, ale kárat pošetilce je pošetilost.
#16:23 Srdce moudrého dává jeho ústům prozíravost a na jeho rty přidává znalosti.
#16:24 Pláství medu je řeč vlídná, lahodou duši a uzdravením kostem.
#16:25 Někdy se člověku zdá cesta přímá, ale nakonec přivede k smrti.
#16:26 Ten, kdo se plahočí, plahočí se pro sebe, popohánějí ho vlastní ústa.
#16:27 Ničema vyhrabává zlo, na jeho rtech jako by byl spalující oheň.
#16:28 Proradný člověk vyvolává sváry a klevetník rozlučuje důvěrné přátele.
#16:29 Násilník svého bližního láká a svádí ho na nedobrou cestu.
#16:30 Kdo přimhuřuje oči, myslí na proradnost, kdo svírá rty, už dokonal zlo.
#16:31 Šediny jsou ozdobnou korunou, lze je nalézt na cestě spravedlnosti.
#16:32 Lepší je shovívavý než bohatýr, a kdo ovládá sebe, je nad dobyvatele města.
#16:33 Los se vytahuje ze záňadří, ale každé rozhodnutí je od Hospodina. 
#17:1 Lepší suchá skýva a k tomu klid než dům plný obětních hodů a spory.
#17:2 Prozíravý otrok panuje nad synem, který dělá ostudu, a mezi bratry mu připadne dědictví.
#17:3 Na stříbro kelímek, na zlato pec; srdce však Hospodin prozkoumává.
#17:4 Zlé je věnovat pozornost ničemným rtům, klamné dopřávat sluchu zkázonosnému jazyku.
#17:5 Kdo se posmívá chudému, tupí toho, kdo jej učinil, kdo má radost z běd, ten bez trestu nezůstane.
#17:6 Korunou starců jsou vnuci, ozdobou synů otcové.
#17:7 Nesluší bloudovi mnoho řečí, tím méně urozenému řeči lživé.
#17:8 Jako vzácný kámen je úplatek v očích dárce, kamkoli se obrátí, má úspěch.
#17:9 Kdo stojí o lásku, přikrývá přestoupení, ale kdo je přetřásá, rozlučuje důvěrné přátele.
#17:10 Pohrůžka zapůsobí na rozumného víc než na hlupáka sto ran.
#17:11 Jen o vzpouru usiluje zlý, ale bude proti němu vyslán nelítostný posel.
#17:12 Raději potkat medvědici zbavenou mláďat než hlupáka s jeho pošetilostí.
#17:13 Tomu, kdo za dobré odplácí zlým, zlo se nehne z domu.
#17:14 Protrhne vodní hráz, kdo začne svár, přestaň dřív, než propukne spor.
#17:15 Prohlásit svévolníka za spravedlivého a spravedlivého za svévolníka, to obojí je Hospodinu ohavností.
#17:16 K čemu peníze v rukou hlupáka? Chce koupit moudrost? Vždyť nemá rozum!
#17:17 V každičkém čase miluje přítel, zrodil se bratrem pro doby soužení.
#17:18 Člověk bez rozumu dává ruku a svému druhovi se zaručuje.
#17:19 Má rád přestoupení, kdo má rád hádky, kdo si zvyšuje vchod, říká si o zkázu.
#17:20 Kdo má falešné srdce, nenajde dobro, kdo má proradný jazyk, upadne do zla.
#17:21 Kdo zplodil hlupáka, sobě k žalu ho zplodil, otec blouda nemůže mít radost.
#17:22 Radostné srdce hojí rány, kdežto ubitý duch vysušuje kosti.
#17:23 Svévolník po straně přijímá úplatek, tak převrací stezky práva.
#17:24 Na tváři rozumného se zračí moudrost, ale oči hlupáka těkají po končinách země.
#17:25 Hloupý syn působí svému otci hoře a své rodičce hořkost.
#17:26 Pokutovat spravedlivého není dobré, natož urozené bít pro přímost.
#17:27 Kdo se krotí v řeči, má správné poznání, kdo je klidného ducha, je rozumný člověk.
#17:28 I pošetilec, když mlčí, může být pokládán za moudrého, nechá-li rty zavřené, za rozumného. 
#18:1 Za svými choutkami jde, kdo se zříká druhých, pohotově rozpoutává sváry.
#18:2 Hlupák si nelibuje v rozumnosti, nýbrž v obnažování svého srdce.
#18:3 Kam vejde svévolník, vchází opovržení a potupa s hanbou.
#18:4 Hluboké vody jsou slova z úst muže, potok plný vody, zdroj moudrosti.
#18:5 Není dobré nadržovat svévolníkovi a odstrčit na soudu spravedlivého.
#18:6 Hlupáka zavedou jeho rty do sporu, jeho ústa volají po výprasku.
#18:7 Ústa přinesou hlupákovi zkázu, jeho rty jsou léčkou jeho duši.
#18:8 Řeči klevetníkovy jsou jak pamlsky, sestoupí až do nejvnitřnějších útrob.
#18:9 Ten, kdo při své práci otálí, je bratrem zhoubce.
#18:10 Pevná věž je Hospodinovo jméno, k němu se uteče spravedlivý jak do hradu.
#18:11 Pevnou tvrzí je boháčovi jeho majetek, jeví se mu jako nedostupná hradba.
#18:12 Srdce člověka bývá před pádem zpupné, kdežto slávu předchází pokora.
#18:13 Odpoví-li kdo dřív, než vyslechl, toť pošetilost a hanba pro něj.
#18:14 Mužný duch snáší nemoc, ale ducha ubitého kdo unese?
#18:15 Rozumné srdce získává poznání, ucho moudrých je vyhledává.
#18:16 Dar otvírá člověku dveře, uvádí ho i k velmožům.
#18:17 Kdo při sporu mluví první, jeví se spravedlivý, pak přijde druhá strana a podrobí ho zkoušce.
#18:18 Losování činí přítrž svárům, odtrhne od sebe i zarputilce.
#18:19 Zhrzený bratr je nepřístupnější než pevná tvrz a sváry jsou jako závora paláce.
#18:20 Nitro člověka se sytí ovocem úst, sytí je úroda rtů.
#18:21 V moci jazyka je život i smrt, kdo ho rád používá, nají se jeho plodů.
#18:22 Kdo našel ženu, našel dobro a došel u Hospodina zalíbení.
#18:23 Chudák prosí o smilování, ale boháč odpovídá tvrdě.
#18:24 Přátelit se s kdekým je ke škodě; kdo však miluje, přilne víc než bratr. 
#19:1 Lepší je chudák žijící bezúhonně než falešník a k tomu hlupák.
#19:2 Bez poznání nemůže být nikdo dobrý, kdo je zbrklý, hřeší.
#19:3 Pošetilostí si člověk podvrací cestu, ale jeho srdce má zlost na Hospodina.
#19:4 Majetek zjednává víc a víc přátel, nuzáka se i přítel zřekne.
#19:5 Křivý svědek nezůstane bez trestu, neunikne ten, kdo šíří lži.
#19:6 Urozenému pochlebují mnozí, se štědrým mužem se přátelí kdekdo.
#19:7 Chudáka nenávidí všichni jeho bratři, tím spíš se mu vzdalují jeho přátelé. Kdo se honí za slovy, nemá z toho nic.
#19:8 Kdo získal rozum, má rád svůj život, kdo dbá na rozumnost, najde dobro.
#19:9 Křivý svědek nezůstane bez trestu, zahyne ten, kdo šíří lži.
#19:10 Nepřísluší blahobyt hlupákovi, natož otroku vláda nad knížaty.
#19:11 Prozíravost činí člověka shovívavým, promíjet přestupky je jeho ozdobou.
#19:12 Jak řev mladého lva je králova zlost, jak rosa na bylinu jeho přízeň.
#19:13 Neštěstím pro otce je syn hlupák; svárlivá žena jak neustálé zatékání vody.
#19:14 Dům a majetek lze zdědit po otcích, ale prozíravá žena je od Hospodina.
#19:15 Lenost uvede do mrákot; zahálčivá duše bude hladovět.
#19:16 Kdo zachovává přikázání, střeží svůj život, zemře, kdo jeho cestami zhrdá.
#19:17 Hospodinu půjčuje, kdo se nad nuzným smilovává, on mu odplatí jeho dobročinnost.
#19:18 Trestej syna, dokud je naděje, a nechtěj mu přivodit smrt!
#19:19 Velká vznětlivost volá po pokutě, odpustíš-li, budeš muset přidat.
#19:20 Poslechni radu, přijmi i trest, abys byl napříště moudrý.
#19:21 Člověk má v srdci mnoho plánů, ale úradek Hospodinův obstojí.
#19:22 Na člověku se žádá, aby byl milosrdný; chudák je na tom lépe než lhář.
#19:23 Bázeň před Hospodinem vede k životu; nasycen přečkáš noc a nic zlého tě nepostihne.
#19:24 Lenoch sáhne rukou do mísy, ale k ústům už ji nevrátí.
#19:25 Bij posměvače a prostoduchý se stane chytřejším, domluv rozumnému a porozumí poznání.
#19:26 Kdo týrá otce a vyhání matku, je syn hanebný a hnusný.
#19:27 Přestaneš-li, synu, poslouchat kárání, od výroků poznání zbloudíš.
#19:28 Ničemný svědek se posmívá právu a ústa svévolníků hltají ničemnosti.
#19:29 Na posměvače jsou schystány soudy, na hřbet hlupáků výprask. 
#20:1 Víno je posměvač, opojný nápoj je křikloun; kdo se v něm kochá, ten moudrý není.
#20:2 Král budí hrůzu, jako když zařve mladý lev; kdo ho rozlítí, hřeší sám proti sobě.
#20:3 Je ctí pro muže upustit od sporu, kdejaký pošetilec jej však rozpoutává.
#20:4 Na podzim lenoch neorá, o žních bude žebrat, ale nic nebude.
#20:5 Jak hluboké vody je rada v srdci muže, muž rozumný z ní čerpá.
#20:6 Mnoho lidí rozhlašuje své milosrdenství, ale muže spolehlivého kdo najde?
#20:7 Spravedlivý žije bezúhonně; blaze bude po něm jeho synům!
#20:8 Král sedí na soudném stolci a očima převívá vše, co je zlé.
#20:9 Kdo může říci: „Zachoval jsem si ryzí srdce, jsem čistý, bez hříchu“?
#20:10 Dvojí závaží a dvojí míra, obojí je Hospodinu ohavností.
#20:11 Už na chlapci se z jeho způsobů pozná, zda bude v jednání ryzí a přímý.
#20:12 Ucho, které slyší, a oko, které vidí, obojí učinil Hospodin.
#20:13 Nemiluj spánek, ať nezchudneš, nech oči otevřené a nasytíš se chlebem.
#20:14 „Špatné, špatné“, říká kupující, ale jen poodejde, už se tím chlubí.
#20:15 Někdo má zlato a množství perel, ale nejdrahocennější skvost jsou rty plné poznání.
#20:16 Šaty zadrž tomu, kdo se zaručuje za cizáka, a když za cizinku, vezmi od něho zástavu.
#20:17 Někomu je lahodný chléb klamu, ale nakonec má plná ústa štěrku.
#20:18 Úmysly se zajišťují poradou, boj veď s rozvahou.
#20:19 Utrhač, kudy chodí, vynáší důvěrnosti, nezaplétej se tedy s mluvkou.
#20:20 Kdo zlořečí otci a matce, tomu zhasne jeho světlo v nejhlubší tmě.
#20:21 Dědictví, na jehož počátku je nečestnost, nebývá nakonec požehnané.
#20:22 Neříkej: „Odplatím za zlo!“ Čekej na Hospodina a on tě zachrání.
#20:23 Dvojí závaží je Hospodinu ohavností, falešné váhy nejsou nic dobrého.
#20:24 Kroky muže určuje Hospodin; jak by mohl člověk rozumět své cestě?
#20:25 Člověku je léčkou nerozvážně říci: „Je to zasvěceno“, a po slibu si to rozmýšlet.
#20:26 Moudrý král převívá svévolníky, dá je přejet koly mlátícího vozu.
#20:27 Lidský duch je světlo od Hospodina; to propátrá všechny nejvnitřnější útroby.
#20:28 Milosrdenství a věrnost chrání krále, milosrdenstvím podepře svůj trůn.
#20:29 Ozdobou jinochů je jejich síla, důstojností starců jsou šediny.
#20:30 Jizvy a modřiny vydrhnou špatnost a rány pročistí nejvnitřnější útroby. 
#21:1 Královo srdce je v Hospodinových rukou jako vodní toky; nakloní je, kam se mu zlíbí.
#21:2 Člověku se všechny jeho cesty zdají přímé, ale srdce zpytuje Hospodin.
#21:3 Prosazovat spravedlnost a právo je před Hospodinem výbornější než oběť.
#21:4 Pýcha očí a nadutost srdce, ač jsou svévolníkům světlem, jsou hříchem.
#21:5 Plány jsou pilnému k užitku, ale každý, kdo se ukvapuje, bude mít nedostatek.
#21:6 Poklady dobývané zrádným jazykem jsou jen odvátý přelud těch, kdo vyhledávají smrt.
#21:7 Svévolníky zachvátí zhouba, kterou rozpoutali, neboť odmítali zjednat právo.
#21:8 Klikatá je cesta muže proradného, ryzí člověk v jednání je přímý.
#21:9 Lépe je bydlet na střeše v koutku než se svárlivou ženou ve společném domě.
#21:10 Svévolník je chtivý zlého, v jeho očích nenalezne slitování ani přítel.
#21:11 Když pokutují posměvače, prostoduchý zmoudří, když je k prozíravosti veden moudrý, nabývá poznání.
#21:12 Spravedlivý prozíravě vede v patrnosti dům svévolníka; svévolníky vyvrací pro jejich zlobu.
#21:13 Kdo před křikem nuzného si zacpe uši, bude také volat, a odpověď nedostane.
#21:14 Tajný dar tlumí hněv a postranní úplatek prudké rozhořčení.
#21:15 Prosazovat právo je radostí spravedlivému, ale zkázou pachatelům ničemnosti.
#21:16 Člověk zbloudivší z cesty prozíravosti spočine v shromáždění říše stínů.
#21:17 Nedostatek pozná, kdo miluje radovánky, kdo miluje víno a olej, nezbohatne.
#21:18 Výkupným za spravedlivého bude svévolník, za přímé lidi věrolomník.
#21:19 Lépe je bydlet v zemi pusté než se ženou svárlivou a zlostnou.
#21:20 Žádoucí poklad a olej jsou v obydlí moudrého, kdežto hloupý člověk je prohýří.
#21:21 Kdo jde za spravedlností a milosrdenstvím, najde život, spravedlnost a slávu.
#21:22 Moudrý vstoupí do města bohatýrů a srazí baštu, na niž spoléhali.
#21:23 Kdo střeží svá ústa a jazyk, střeží svou duši před soužením.
#21:24 Pyšný opovážlivec jménem posměvač jedná bezmezně zpupně.
#21:25 Lenocha usmrtí choutky, neboť jeho ruce odmítají práci;
#21:26 choutky ho stravují neustále, ale spravedlivý dává a nešetří.
#21:27 Oběť svévolníků je ohavností, tím spíše, když se přináší s mrzkým záměrem.
#21:28 Lživý svědek zahyne, ale muž, který vypoví, co slyšel, bude mít poslední slovo.
#21:29 Svévolník vystupuje s nestoudnou tváří, přímý jde však rovnou cestou.
#21:30 Žádná moudrost, žádná rozumnost, žádný úradek nic nesvedou proti Hospodinu.
#21:31 Kůň je strojen pro den boje, ale vítězství je u Hospodina. 
#22:1 Výborné jméno je nad hojné bohatství, lepší než stříbro a zlato je přízeň.
#22:2 Boháč a chudák se střetávají, Hospodin učinil oba.
#22:3 Chytrý vidí nebezpečí a ukryje se, kdežto prostoduší půjdou dál a doplatí na to.
#22:4 Pokoru doprovází bázeň před Hospodinem, bohatství, sláva a život.
#22:5 Háky a osidla jsou na cestě falešníka; kdo střeží svůj život, vzdaluje se od nich.
#22:6 Zasvěť už chlapce do jeho cesty, neodchýlí se od ní, ani když zestárne.
#22:7 Boháč panuje nad chudáky, dlužník se stává otrokem věřitele.
#22:8 Kdo rozsévá bezpráví, sklidí ničemnost, hůl jeho prchlivosti vezme za své.
#22:9 Kdo hledí vlídným okem, bude požehnán, neboť dal ze svého chleba nuznému.
#22:10 Vyžeň posměvače a odejde i svár, ustane pře i hanba.
#22:11 Kdo miluje čistotu srdce a má ušlechtilé rty, tomu bude přítelem i král.
#22:12 Hospodin dohlíží na pravé poznání, ale slova věrolomného vyvrací.
#22:13 Lenoch říká: „Venku je lev! Na náměstí by mě zadávil.“
#22:14 Ústa cizaček jsou hluboká jáma; na koho se Hospodin rozhněvá, ten tam spadne.
#22:15 Vězí-li v srdci chlapce pošetilost, trestající hůl ji od něho vzdálí.
#22:16 Kdo utiskuje nuzného, aby mu přibylo, nebo kdo dává bohatému, bude mít nedostatek.
#22:17 Nakloň své ucho a slyš slova moudrých, zaměř srdce k tomu, co jsem poznal.
#22:18 Bude ti k blahu, budeš-li je ve svém nitru zachovávat, budou rovněž pohotově na tvých rtech.
#22:19 Abys na Hospodina spoléhal, poučím tě dnes - ano, tebe.
#22:20 Zdali jsem ti již dříve nenapsal rady a poznání,
#22:21 abych tě poučil o spolehlivé jistotě slov pravdy, abys mohl pravdivě odpovědět tomu, kdo tě poslal?
#22:22 Neodírej nuzného, vždyť nic nemá, po utištěném nešlapej v bráně.
#22:23 Neboť jejich spor povede Hospodin, uchvátí život jejich uchvatitelům.
#22:24 Nepřátel se s hněvivým člověkem a s mužem vznětlivým se nestýkej,
#22:25 ať se nepřizpůsobíš jeho stezkám a nenastražíš léčku své duši.
#22:26 Nebuď mezi těmi, kdo dají druhému ruku, kdo se zaručují za půjčku,
#22:27 nemáš-li čím splatit; proč ti má někdo vzít pod tebou lůžko?
#22:28 Nepřenášej dávné mezníky, které zasadili tvoji otcové.
#22:29 Viděls muže, který je zběhlý v svém díle? Před králi bude stávat; nebude stávat před bezvýznamnými. 
#23:1 Jestliže zasedneš k jídlu s vladařem, dobře rozvaž, co je před tebou.
#23:2 Nasadíš si nůž na hrdlo, budeš-li nenasytný.
#23:3 Nedychti po jeho pochoutkách, je to ošidný pokrm.
#23:4 Neštvi se za bohatstvím, z vlastního rozumu toho zanech.
#23:5 Jen letmo na ně pohlédneš, už není! Vždycky si opatří křídla, jak orel odlétne k nebi.
#23:6 Nejez pokrm nepřejícího, nedychti po jeho pochoutkách!
#23:7 Vždyť je to duše vypočítavá. Říká ti: „Jez a pij“, ale jeho srdce s tebou není.
#23:8 Snědené sousto zvrátíš a pokazíš svá vlídná slova.
#23:9 Nedomlouvej hlupákovi, neboť pohrdne tvou prozíravou řečí.
#23:10 Neposunuj dávné mezníky a nevstupuj na pole sirotků,
#23:11 neboť jejich zastánce je mocný, on povede jejich spor proti tobě.
#23:12 Ber si k srdci napomenutí a pozorně naslouchej výrokům poznání.
#23:13 Nepřipravuj chlapce o trest! Nezemře, když mu nabiješ holí.
#23:14 Nabiješ mu holí a jeho život vysvobodíš od podsvětí.
#23:15 Můj synu, bude-li tvé srdce moudré, zaraduje se i moje vlastní srdce.
#23:16 Zajásá mé ledví, když tvé rty budou mluvit, co je správné.
#23:17 Ať tvé srdce nezávidí hříšníkům, ale ať horlí pro bázeň před Hospodinem po všechny dny.
#23:18 Však to budoucnost ukáže! Tvá naděje nebude zmařena.
#23:19 Ty, můj synu, poslouchej a zmoudříš, veď své srdce touto cestou:
#23:20 Nebývej mezi pijany vína, mezi žrouty masa,
#23:21 vždyť pijan a žrout přijdou na mizinu, dřímota je oblékne v cáry.
#23:22 Poslouchej otce, on tě zplodil, a matkou nepohrdej, když zestárla.
#23:23 Pravdu získej a nekupči moudrostí, kázní a rozumností.
#23:24 Otec spravedlivého velice jásá, zplodil moudrého a raduje se z něho.
#23:25 Ať se raduje tvůj otec i tvá matka, tvoje rodička ať jásá.
#23:26 Můj synu, dej mi své srdce, ať si tvé oči oblíbí mé cesty.
#23:27 Vždyť nevěstka je hluboká jáma, těsná studna je cizinka.
#23:28 Číhá jak zákeřník a zaviní, že přibývá mezi lidmi věrolomných.
#23:29 Komu zbude „Ach“ a komu „Běda“? Komu sváry? Komu plané řeči? Komu zbytečné modřiny? Komu zkalený zrak?
#23:30 Těm, kdo se zdržují u vína, kdo chodí okoušet kořeněný nápoj.
#23:31 Nehleď na víno, jak se rdí, jak jiskří v poháru. Vklouzne hladce
#23:32 a nakonec uštkne jako had a štípne jako zmije.
#23:33 Tvé oči budou hledět na nepřístojnosti, z tvého srdce budou vycházet proradné řeči.
#23:34 Bude ti, jako bys ležel v srdci moře, jako bys ležel s rozbitou hlavou.
#23:35 „Zbili mě a nic mě nebolí, ztloukli mě a nevím o tom. Až procitnu, vyhledám to zas a zase.“ 
#24:1 Nezáviď zlým lidem a nedychti být s nimi,
#24:2 neboť jejich srdce rozjímá, jak připravit zhoubu, a jejich rty mluví, aby potrápily.
#24:3 Moudrostí se dům buduje, rozumností se zajišťuje;
#24:4 kde je poznání, tam se komory naplňují vším drahocenným a příjemným majetkem.
#24:5 Moudrý muž je mocný, a kdo má poznání, upevňuje svou sílu.
#24:6 Boj veď s rozvahou, ve množství rádců je vítězství.
#24:7 Příliš vysoko je moudrost pro pošetilého, ten v bráně neotevře ústa.
#24:8 Kdo přemýšlí, jak páchat zlo, toho nazvou pletichářem.
#24:9 Vymýšlet pošetilost je hřích, posměvač se lidem hnusí.
#24:10 Budeš-li v čas soužení liknavý, budeš se svou silou v úzkých.
#24:11 Vysvoboď ty, kdo jsou vlečeni na smrt; což se neujmeš těch, kdo se potácejí na popravu?
#24:12 Řekneš-li: „My jsme to nevěděli“, což ten, který zpytuje nitro, tomu nerozumí? Ten, který chrání tvůj život, to neví? On odplatí člověku podle jeho činů.
#24:13 Můj synu, jez med, je dobrý, plástev medu je tvému patru sladká.
#24:14 Právě tak poznávej moudrost pro svou duši. Když ji najdeš, máš budoucnost, tvá naděje nebude zmařena.
#24:15 Svévolníku, nestroj úklady obydlí spravedlivého a nepleň místo, kde on odpočívá!
#24:16 Spravedlivý, i když sedmkrát padne, zase povstane, svévolníci zaklopýtnou a zle končí.
#24:17 Neraduj se z pádu svého nepřítele, nejásej nad jeho klopýtnutím ani v srdci,
#24:18 nebo to Hospodin uvidí a bude to zlé v jeho očích a odvrátí od něho svůj hněv.
#24:19 Nerozčiluj se kvůli zlovolníkům, nezáviď svévolníkům.
#24:20 Zlý žádnou budoucnost nemá, svévolníkům zhasne světlo.
#24:21 Můj synu, boj se Hospodina a krále a nezaplétej se s lidmi vrtkavými;
#24:22 kdo ví, kdy se na ně náhle snese zničující pohroma od obou?
#24:23 I toto jsou výroky moudrých: Stranit osobám při soudu není dobré.
#24:24 Kdo řekne svévolníkovi: „Jsi spravedlivý“, na toho bude láteřit lid, na toho zanevřou národy.
#24:25 Ti, kdo mu domluví, dojdou blaha, a budou požehnáni vším dobrým.
#24:26 Na rty líbá ten, kdo dává správné odpovědi.
#24:27 Zajisti své dílo venku, starej se o ně na svém poli a potom si postavíš i dům.
#24:28 Nebuď bezdůvodně svědkem proti bližnímu! Chceš svými rty klamat?
#24:29 Neříkej: „Jak jednal se mnou, tak budu jednat s ním, odplatím každému podle jeho činů“.
#24:30 Šel jsem kolem pole muže lenivého, kolem vinice člověka bez rozumu,
#24:31 a hle, byla celá zarostlá plevelem, celý její povrch pokrývaly kopřivy a její kamenná zeď byla pobořena.
#24:32 Když jsem na to hleděl, vzal jsem si to k srdci, přijal jako napomenutí to, co jsem viděl:
#24:33 Trochu si pospíš, trochu zdřímneš, trochu složíš ruce v klín a poležíš si
#24:34 a tvá chudoba přijde jak pobuda a tvá nouze jako ozbrojenec. 
#25:1 Toto jsou rovněž přísloví Šalomounova, která sebrali mužové judského krále Chizkijáše.
#25:2 Sláva Boží je věc ukrýt, sláva králů je věc prozkoumat.
#25:3 Výšiny nebes, hlubiny země a srdce králů nelze prozkoumat.
#25:4 Odstraní-li se ze stříbra struska, výrobek se zlatníkovi povede;
#25:5 odstraní-li se svévolník z blízkosti krále, jeho trůn bude upevněn spravedlností.
#25:6 Před králem se nevypínej a na místo velmožů se nestav.
#25:7 Lépe bude, řekne-li ti: „Vystup sem“, než když tě poníží před urozeným, jak na vlastní oči vídáš.
#25:8 Nezačínej unáhleně spor; jinak co si nakonec počneš, až tě tvůj bližní zahanbí?
#25:9 Veď svůj spor se svým bližním, ale nevyzraď tajemství jiného,
#25:10 jinak tě bude tupit, kdo o tom uslyší, a nepřestanou tě pomlouvat.
#25:11 Jako zlatá jablka se stříbrnými ozdobami je vhodně pronesené slovo.
#25:12 Zlatý nosní kroužek či náhrdelník z třpytivého zlata je kárající mudrc slyšícímu uchu.
#25:13 Jako chladný sníh o žních je spolehlivý vyslanec těm, kdo ho poslali. Občerství duši svého pána.
#25:14 Oblaka s větrem, ale bez deště, to je muž, který klame slibováním darů.
#25:15 Vůdce se dá přemluvit trpělivostí a měkký jazyk láme kosti.
#25:16 Najdeš-li med, jez s mírou, jinak se jím přesytíš a zvrátíš jej.
#25:17 Choď do domu svého bližního jen zřídka, jinak se tě přesytí a bude tě nenávidět.
#25:18 Palcát a meč a naostřený šíp je ten, kdo vydává proti svému bližnímu křivé svědectví.
#25:19 Jako vykotlaný zub a kulhavá noha je spoléhat se na věrolomného v den soužení.
#25:20 Svlékat šaty v chladný den či nalévat do louhu ocet je zpívat písně srdci sklíčenému.
#25:21 Hladoví-li ten, kdo tě nenávidí, nasyť jej chlebem, žízní-li, napoj ho vodou,
#25:22 tím shrneš řeřavé uhlí na jeho hlavu a Hospodin ti odplatí.
#25:23 Severní vítr přihání déšť a hněvivý obličej pokoutní řeči.
#25:24 Lépe je bydlet na střeše v koutku než se svárlivou ženou ve společném domě.
#25:25 Jak chladná voda znavené duši je dobrá zpráva z daleké země.
#25:26 Zkalený pramen a zkažená studánka je spravedlivý kolísající před svévolníkem.
#25:27 Není dobré jíst příliš mnoho medu a není slavné zkoumat slávu druhých.
#25:28 Město se strženými hradbami je muž, který se neovládá. 
#26:1 Jako sníh v létě a déšť ve žni, tak se nehodí k hlupákovi sláva.
#26:2 Vrabec přeletuje, vlaštovka poletuje, bezdůvodné zlořečení nezasáhne.
#26:3 Na koně bič, na osla uzdu, na hřbet hlupáků hůl.
#26:4 Neodpovídej hlupákovi podle jeho pošetilosti, abys nebyl jako on.
#26:5 Odpověz hlupákovi podle jeho pošetilosti, aby se sám sobě nezdál moudrý.
#26:6 Nohy si mrzačí, zakusí příkoří, kdo posílá vzkaz po hlupákovi.
#26:7 Slabé jsou nohy chromého i přísloví v ústech hlupáků.
#26:8 Jako oblázek vložený do praku je pocta vzdaná hlupákovi.
#26:9 Jako trn v ruce opilého, tak přísloví v ústech hlupáků.
#26:10 Jako střelec, který chce všechno zasáhnout, je ten, kdo najímá hlupáka a kdejaké tuláky.
#26:11 Jako se pes vrací ke svému zvratku, tak hlupák opakuje svou pošetilost.
#26:12 Uvidíš-li muže, který si připadá moudrý, věz, že hlupák má víc naděje než on.
#26:13 Lenoch říká: „Na cestě je lvíče, v ulicích je lev.“
#26:14 Dveře se otáčejí ve svém čepu a lenoch na svém loži.
#26:15 Lenoch sáhne rukou do mísy, ale je mu zatěžko vrátit ji k ústům.
#26:16 Lenoch si připadá moudřejší než sedm zkušeně odpovídajících.
#26:17 Chytá psa za uši, kdo se rozlítí ve sporu, který se ho netýká.
#26:18 Jako pomatený, který střílí ohnivé šípy a smrtící střely,
#26:19 tak jedná muž, který obelstí bližního a řekne: „Já jsem jen žertoval.“
#26:20 Není-li už dřevo, uhasne oheň, není-li klevetník, utichne svár.
#26:21 Uhlí do výhně, dříví na oheň - tím je svárlivý muž pro vzplanutí sporu.
#26:22 Řeči klevetníkovy jsou jak pamlsky, sestoupí až do nejvnitřnějších útrob.
#26:23 Stříbrná poleva na hliněném střepu jsou planoucí rty, ale zlé srdce.
#26:24 Na rtech má přetvářku, kdo nenávidí, ve svém nitru chová lest.
#26:25 Mluví-li přívětivě, nevěř mu. Vždyť v jeho srdci je sedmerá ohavnost;
#26:26 nenávist může podvodně zakrýt, ale ve shromáždění bude jeho zloba odhalena.
#26:27 Kdo kope jámu, padne do ní, a kdo valí balvan, na toho se zvrátí.
#26:28 Zrádný jazyk nenávidí ty, na které dotírá, úlisná ústa přivodí pád. 
#27:1 Nechlub se zítřejším dnem, vždyť nevíš, co den zrodí.
#27:2 Ať tě chválí cizí a ne tvá vlastní ústa, cizinec a ne tvoje rty.
#27:3 Těžký je kámen, písek má váhu, ale hoře, jež působí pošetilec, je těžší než obojí.
#27:4 Rozhořčení je kruté, hněv je jak povodeň, ale před žárlivostí kdo obstojí?
#27:5 Lepší jsou zjevná kárání než skrývaná láska.
#27:6 Věrně jsou míněny šlehy od milujícího, ale záludné jsou polibky nenávidícího.
#27:7 Sytý šlape i po plástvi medu, kdežto hladovému je každá hořkost sladká.
#27:8 Jako pták vyplašený z hnízda je muž, který prchá ze svého místa.
#27:9 Olej a kadidlo jsou pro radost srdci, přítel je sladší než chtění vlastní duše.
#27:10 Svého přítele a přítele svého otce neopouštěj, ale v den svých běd nechoď ani do domu vlastního bratra. Lepší je blízký soused než vzdálený bratr.
#27:11 Buď moudrý, můj synu, dělej radost mému srdci, abych mohl odpovědět tomu, kdo mě tupí.
#27:12 Chytrý vidí nebezpečí a skryje se, prostoduší jdou dál a doplatí na to.
#27:13 Šaty zadrž tomu, kdo se zaručuje za cizáka, a když za cizinku, vezmi od něho zástavu.
#27:14 Kdo za časného jitra příliš hlasitě dobrořečí svému bližnímu, tomu se to bude počítat za zlořečení.
#27:15 Neustálé zatékání vody v době dešťů a svárlivá žena jsou totéž;
#27:16 kdo ji chce zvládnout, chce zvládnout vítr, pravicí chytá olej.
#27:17 Železo se ostří železem a jeden ostří tvář druhého.
#27:18 Kdo hlídá fíkovník, bude jíst jeho plody, a kdo střeží svého pána, získá vážnost.
#27:19 Jako se na vodě zrcadlí tvář, tak srdce člověka na člověku.
#27:20 Podsvětí, říše zkázy, se nenasytí; nenasytí se ani oči člověka.
#27:21 Na stříbro kelímek, na zlato pec; a muž prověřuje ústa, která ho chválí.
#27:22 Kdybys pošetilce roztloukl v hmoždíři na padrť tloukem, pošetilost z něho nedostaneš.
#27:23 Dobře si všímej, jak vypadá tvůj brav, starej se pečlivě o svá stáda,
#27:24 vždyť žádná klenotnice není věčně plná, ani královská čelenka nepřetrvá všechna pokolení.
#27:25 Tráva se poseká, ukáže se otava, sklízí se píce z hor,
#27:26 z beránků máš oděv, za kozly koupíš pole,
#27:27 máš dostatek kozího mléka k své obživě i k obživě své rodiny, máš živobytí pro své děvečky. 
#28:1 Svévolníci utíkají, i když je nikdo nepronásleduje, ale spravedliví žijí v bezpečí jako lvíče.
#28:2 Když je v zemi nevěrnost, bývá v ní mnoho velmožů, ale člověk rozumný a znalý udrží řád.
#28:3 Chudý muž, který utiskuje nuzné, je jako déšť, který odplavuje půdu a nedává chléb.
#28:4 Ti, kteří opouštějí Zákon, vychvalují svévolníka, ale ti, kteří jej zachovávají, se jim stavějí na odpor.
#28:5 Lidé zlí nerozumějí právu, kdežto ti, kdo hledají Hospodina, rozumějí všemu.
#28:6 Lepší je chudák žijící bezúhonně než falešný obojetník, byť bohatý.
#28:7 Rozumný syn dodržuje Zákon, ale kdo se přátelí se žrouty, dělá svému otci hanbu.
#28:8 Kdo rozmnožuje svůj statek lichvou a úrokem, shromažďuje jej pro toho, kdo se smilovává nad nuznými.
#28:9 Odvrací-li se někdo od slyšení Zákona, i jeho modlitba je ohavností.
#28:10 Kdo svede přímé na špatnou cestu, padne sám do jámy, kterou přichystal, ale bezúhonní nabudou dobrého dědictví.
#28:11 Boháč se zdá sám sobě moudrý, ale nuzný, který má rozum, ho prohlédne.
#28:12 Když jásají spravedliví, je to nádherné, ale když se pozdvihují svévolníci, člověk se ukrývá.
#28:13 Kdo kryje svá přestoupení, nebude mít zdar, ale kdo je vyznává a opouští, dojde slitování.
#28:14 Blaze člověku, který se stále bojí Boha, ale kdo zatvrdí své srdce, upadne do neštěstí.
#28:15 Řvoucí lev a sápající se medvěd je svévolný vladař nad nuzným lidem.
#28:16 Vévoda bez rozumu znamená mnoho útisku. Kdo nenávidí zištnost, bude dlouho živ.
#28:17 Člověk obtížený vraždou spěje k jámě; nikdo ho nezadržuj!
#28:18 Kdo žije bezúhonně, bude zachráněn, ale falešný obojetník rázem padne.
#28:19 Kdo obdělává svou půdu, nasytí se chlebem, kdežto kdo následuje povaleče, nasytí se chudobou.
#28:20 Věrný muž má mnoho požehnání, ale kdo se žene za bohatstvím, nezůstane bez trestu.
#28:21 Stranit někomu není dobré. Leckdo se proviní pro skývu chleba.
#28:22 Závistivec se honí za majetkem a neví, že na něj přijde nedostatek.
#28:23 Kdo domlouvá člověku, dojde později vděku spíše než ten, kdo má úlisný jazyk.
#28:24 Kdo odírá otce a matku a říká: „To není žádný přestupek“, je společníkem zhoubce.
#28:25 Nadutec podněcuje spor, ale kdo spoléhá na Hospodina, bude nasycen tukem.
#28:26 Kdo spoléhá na svůj rozum, je hlupák, ale kdo žije moudře, unikne zlému.
#28:27 Kdo dává chudému, nebude mít nedostatek, ale kdo dělá, že nevidí, sklidí hojnost kleteb.
#28:28 Když se pozdvihují svévolníci, člověk se skrývá, když však mizí, rozhojní se spravedliví. 
#29:1 Kdo zatvrdí šíji proti domluvám, bude nenadále rozdrcen a nezhojí ho nikdo.
#29:2 Když přibývá spravedlivých, lid se raduje, panuje-li svévolník, lid vzdychá.
#29:3 Kdo miluje moudrost, působí radost svému otci, ale kdo se přátelí s nevěstkami, mrhá majetek.
#29:4 Král zajistí zemi právem, ale kdo ji vydírá dávkami, pustoší ji.
#29:5 Muž, který lichotí bližnímu, rozprostírá síť jeho krokům.
#29:6 V přestoupení zlého člověka je léčka, ale spravedlivý plesá a raduje se.
#29:7 Spravedlivý zná při nemajetných, svévolník pro takové poznání nemá porozumění.
#29:8 Posměvači pobuřují město, kdežto moudří hněv odvracejí.
#29:9 Když se moudrý soudí s pošetilcem, ten se rozčiluje a posmívá bez ustání.
#29:10 Krvežíznivci nenávidí bezúhonného, kdežto přímí usilují zachránit mu život.
#29:11 Hlupák soptí, co mu dech stačí, ale moudrý se vždycky ovládne.
#29:12 Když vládce věnuje pozornost klamnému slovu, všichni jeho sluhové se stanou svévolníky.
#29:13 Chudý a utlačovatel se střetávají, ale oběma dal světlo očí Hospodin.
#29:14 Jestliže král soudí nuzné podle pravdy, jeho trůn bude upevněn provždy.
#29:15 Hůl a domluva dávají moudrost, ale bezuzdný mladík dělá ostudu své matce.
#29:16 Rozhojní se přestupky, kde se rozhojňují svévolníci, ale spravedliví spatří jejich pád.
#29:17 Trestej svého syna, připraví ti odpočinek, zpříjemní ti život.
#29:18 Není-li žádného vidění, lid pustne, ale blaze tomu, kdo zachovává Zákon.
#29:19 Otroka nelze ukáznit slovy; chápe sice, ale odezva žádná.
#29:20 Spatříš-li muže ukvapeného v řeči, věz, že hlupák má víc naděje než on.
#29:21 Když někdo otroka od mládí hýčká, bude mít nakonec příživníka.
#29:22 Hněvivý člověk podnítí svár a vznětlivý napáchá mnoho přestupků.
#29:23 Člověka poníží jeho povýšenost, kdežto kdo je poníženého ducha, dojde slávy.
#29:24 Kdo se dělí se zlodějem, nenávidí vlastní život: slyší kletbu, a nic neprozradí.
#29:25 Kdo se třese před lidmi, ten klade sobě léčku, kdo však doufá v Hospodina, má v něm svůj hrad.
#29:26 Mnozí usilují naklonit si vladaře, ale soudcem všech je Hospodin.
#29:27 Spravedlivým se hnusí podlý člověk, svévolníkovi se hnusí, kdo jde přímou cestou. 
#30:1 Slova Agúra, syna Jákeova. Výnos. Výrok toho muže k Itíelovi, k Itíelovi a Ukalovi.
#30:2 Já jsem nejtupější z mužů a lidskou rozumnost nemám.
#30:3 Moudrosti jsem se neučil ani jsem si neosvojil poznání Svatého.
#30:4 Kdo vystoupil do nebe i sestoupil? Kdo si nabral vítr do hrstí? Kdo svázal vody do pláště? Kdo vytyčil všechny dálavy země? Jaké je jeho jméno a jaké je jméno jeho syna? Vždyť je znáš.
#30:5 Všechna Boží řeč je protříbená, on je štítem těch, kteří se k němu utíkají.
#30:6 K jeho slovům nic nepřidávej, jinak tě potrestá a budeš shledán lhářem.
#30:7 O dvě věci tě prosím; neodpírej mi je, dříve než umřu:
#30:8 Vzdal ode mne šálení a lživé slovo, nedávej mi chudobu ani bohatství! Opatřuj mě chlebem podle mé potřeby,
#30:9 tak abych přesycen neselhal a neřekl: „Kdo je Hospodin?“ ani abych z chudoby nekradl a nezneuctil jméno svého Boha.
#30:10 Nepomlouvej otroka před jeho pánem, aby ti nezlořečil a ty bys pykal za svou vinu.
#30:11 Je pokolení, které zlořečí svému otci a své matce nežehná,
#30:12 pokolení, které se pokládá za čisté, ale není umyto od své špíny,
#30:13 pokolení, jež zvysoka hledí a přezíravě zvedá víčka,
#30:14 pokolení, jehož zuby jsou meče a jehož špičáky jsou dýky, aby požíralo v zemi utištěné a ubožáky mezi lidmi.
#30:15 Upír má dvě dcery: „Dej! Dej!“ - Tyto tři věci se nenasytí a čtyři neřeknou: „Dost“:
#30:16 podsvětí a neplodné lůno, země, která se nenasytí vodou, a oheň, jenž neřekne: „Dost!“
#30:17 Oko, které se vysmívá otci a pohrdá poslušností matky, vyklovou havrani od potoka, nebo je sezobou supí mláďata.
#30:18 Tyto tři věci mám za podivuhodné a čtyři nemohu pochopit:
#30:19 cestu orla po nebi, cestu hada po skále, cestu lodi v srdci moře a cestu muže při dívce.
#30:20 Stejně je tomu s cestou cizoložné ženy: pojí, otře si ústa a řekne: „Nedopustila jsem se ničemnosti.“
#30:21 Pod třemi věcmi se chvěje země a čtyři nemůže unést:
#30:22 otroka, který kraluje, blouda, který se přesytí chlebem,
#30:23 vdanou ženu, která není milována, a služku, která vyžene svou paní.
#30:24 Tito čtyři jsou na zemi nejmenší, a přece znamenití mudrci:
#30:25 mravenci, národ bez síly, a přece si zajišťuje v létě pokrm;
#30:26 damani, národ bez moci, a přece si staví obydlí v skalách;
#30:27 kobylky, které nemají krále, a přece všechny vytáhnou válečně seřazeny;
#30:28 ještěrka, kterou můžeš vzít do ruky, a přece bývá v královských palácích.
#30:29 Tito tři pěkně vykračují a čtyři pěkně pochodují:
#30:30 lev, bohatýr mezi zvířaty, před nikým neustoupí;
#30:31 oř silných beder nebo kozel a král se svým válečným lidem.
#30:32 Jsi bloud, vynášíš-li se; máš-li nějaký záměr, ruku na ústa!
#30:33 Stloukáním smetany vznikne máslo, tlak na nos přivodí krvácení, tlak hněvu vyvolá spor. 
#31:1 Slova krále Lemúela, výnos, jímž ho napomínala jeho matka:
#31:2 Co mám říci, můj synu? Co, synu života mého? Co, synu mých slibů?
#31:3 Nevydávej své síly ženám a svoje cesty tomu, co ničí krále.
#31:4 Nehodí se králům, Lemóeli, nehodí se králům být pijany vína a vládcům toužit po opojném nápoji,
#31:5 aby nikdo z nich v opilosti nezapomněl na Boží nařízení a nepřevrátil při nikoho z utištěných.
#31:6 Dejte opojný nápoj hynoucímu a víno těm, kterým je hořko,
#31:7 ať se napije a zapomene na svou chudobu a na své plahočení již nevzpomíná.
#31:8 Otevři svá ústa za němého, za právo všech postižených,
#31:9 ústa otevři, suď spravedlivě a zastaň se utištěného a ubožáka.
#31:10 Ženu statečnou kdo nalezne? Je daleko cennější než perly.
#31:11 Srdce jejího muže na ni spoléhá a nepostrádá kořist.
#31:12 Prokazuje mu jen dobro a žádné zlo po celý svůj život.
#31:13 Stará se o vlnu a o len, pracuje s chutí vlastníma rukama.
#31:14 Podobna obchodním lodím zdaleka přiváží svůj chléb.
#31:15 Ještě za noci vstává dát potravu svému domu a příkazy služkám.
#31:16 Vyhlédne si pole a získá je, z ovoce svých rukou vysází vinici.
#31:17 Bedra si opáše silou a posílí své paže.
#31:18 Okusí, jak je dobré její podnikání. Její svítilna nehasne ani v noci.
#31:19 Vztahuje ruce po přeslenu, svými prsty se chápe vřetena.
#31:20 Dlaň má otevřenou pro utištěného a ruce vztahuje k ubožáku.
#31:21 Nebojí se o svůj dům, když sněží, celý její dům je oblečen do dvojího šatu.
#31:22 Zhotovuje si přikrývky. Z jemného plátna a šarlatu je její oděv.
#31:23 Uznáván je v branách její manžel, když zasedá se staršími země.
#31:24 Zhotovuje plátno na prodej a pásy dodává kupci.
#31:25 Síla a důstojnost je jejím šatem, s úsměvem hledí vstříc příštím dnům.
#31:26 Její ústa promlouvají moudře, na jazyku mívá vlídné naučení.
#31:27 Pozorně sleduje chod svého domu a chleba lenosti nejí.
#31:28 Její synové povstávají a blahořečí jí, též její manžel ji chválí:
#31:29 „Statečně si vedly mnohé dcery, ale ty je všechny předčíš.“
#31:30 Klamavá je líbeznost, pomíjivá krása; žena, jež se bojí Hospodina, dojde chvály.
#31:31 Dejte jí z ovoce jejích rukou, ať ji chválí v branách její činy!  

\book{Ecclesiastes}{Eccl}
#1:1 Slova Kazatele, syna Davidova, krále v Jeruzalémě.
#1:2 Pomíjivost, samá pomíjivost, řekl Kazatel, pomíjivost, samá pomíjivost, všechno pomíjí.
#1:3 Jaký užitek má člověk ze všeho svého pachtění, z toho, jak se pod sluncem pachtí?
#1:4 Pokolení odchází, pokolení přichází, ale země stále trvá.
#1:5 Slunce vychází, slunce zapadá a dychtivě tíhne k místu, odkud opět vzejde.
#1:6 Vítr spěje k jihu, stáčí se k severu, točí se, točí, spěje dál, až se zas oklikou vrátí.
#1:7 Všechny řeky spějí do moře, a moře se nepřeplní; do místa, z něhož vytékají, se zase vracejí k novému koloběhu.
#1:8 Všechny věci jsou tak únavné, že se to ani nedá vypovědět; nenasytí se oko viděním, nenaplní se ucho slyšením.
#1:9 Co se dálo, bude se dít zase, a co se dělalo, bude se znovu dělat; pod sluncem není nic nového.
#1:10 Je něco, o čem lze říci: Hleď, to je cosi nového? I to bylo v dávných dobách, které byly před námi.
#1:11 Nelze podržet v paměti věci minulé; a ani budoucí, které nastanou, nezůstanou v paměti těch, kteří budou potom.
#1:12 Já, Kazatel, jsem byl králem nad Izraelem v Jeruzalémě.
#1:13 Předsevzal jsem si, že moudře propátrám a prozkoumám vše, co se pod nebem děje. Úmornou lopotu vložil Bůh lidským synům, a tak se lopotí.
#1:14 Viděl jsem všechno, co se pod sluncem děje, a hle, to vše je pomíjivost a honba za větrem.
#1:15 Co je pokřivené, nelze napřímit, a čeho se nedostává, nelze spočítat.
#1:16 Rozmlouval jsem se svým srdcem: Hle, nabyl jsem větší a hojnější moudrosti než ti všichni, kdo byli v Jeruzalémě přede mnou, mé srdce nabylo mnoho moudrosti a vědění.
#1:17 Předsevzal jsem si, že poznám moudrost, poznám i ztřeštěnost a pomatenost. Poznal jsem však, že i to je honička za větrem,
#1:18 neboť kde je mnoho moudrosti, ji i mnoho hoře, a čím víc vědění, tím víc bolesti. 
#2:1 Řekl jsem si v srdci: Nuže, teď to zkusím s radovánkami a popřeji si dobrého. A hle, také to je pomíjivost!
#2:2 O smíchu jsem řekl: Je to ztřeštěnost, o radosti pak: Co to provádíš?
#2:3 Usmyslel jsem si, že své tělo osvěžím vínem, ačkoli mé srdce tíhne k moudrosti, a že se chopím té pomatenosti, dokud nezjistím, co dobrého pod nebem mají lidští synové z toho, co konají ve vyměřených dnech svého života.
#2:4 Podnikal jsem velkolepá díla, postavil jsem si domy, vysázel vinice,
#2:5 založil si zahrady a sady a v nich vysadil kdejaké ovocné stromoví,
#2:6 zřídil jsem si i vodní nádrže pro zavlažování lesních porostů.
#2:7 Nakoupil jsem si otroků a otrokyň a měl jsem i doma narozenou čeleď, stád skotu a bravu jsem měl víc než všichni, kdo byli v Jeruzalémě přede mnou.
#2:8 Nahromadil jsem si také stříbro a zlato a zabral i vlastnictví králů a krajin; opatřil jsem si zpěváky a zpěvačky i rozkoše synů lidských, milostnice.
#2:9 Stal jsem se velikým a předčil jsem všechny, kteří byli v Jeruzalémě přede mnou; nadto při mně stála má moudrost.
#2:10 V ničem, co si žádali mé oči, jsem jim nebránil, svému srdci jsem neodepřel žádnou radost a mé srdce se zaradovalo ze všeho, za čím jsem se pachtil, a to byl můj podíl ze všeho mého pachtění.
#2:11 I pohlédl jsem na všechno, co bylo mýma rukama vykonáno, na své klopotné pachtění, a hle, všechno je pomíjivost a honba za větrem; a žádný užitek z toto pod sluncem není.
#2:12 I pohlédl jsem a viděl moudrost i ztřeštěnost a pomatenost. Jaký člověk nastoupí po králi, jehož si ustanovili před tím?
#2:13 Shledal jsem, že moudrost nese více užitku než pomatenost, jako světlo dává větší užitek než tma.
#2:14 Moudrý má v hlavě oči, kdežto hlupák chodí ve tmě. Avšak i to jsem poznal, že týž úděl potká oba dva.
#2:15 V srdci jsem si řekl: Co potkává hlupáka, potká i mne. Nač jsem tedy byl tak nadmíru moudrý? A v srdci jsem usoudil, že i to je pomíjivost.
#2:16 Vždyť po moudrém ani po hlupákovi nezůstane památka navěky; všechno, co bylo, bude v příštích dnech zapomenuto. Moudrý umírá stejně jako hlupák.
#2:17 Pojal jsem nenávist k životu, zošklivilo se mi, co se pod sluncem děje. To vše je pomíjivost a honba za větrem.
#2:18 Pojal jsem v nenávist všechno své klopotné pachtění pod sluncem, neboť jeho plody zanechám člověku, který bude po mně.
#2:19 Kdo ví, bude-li moudrý nebo pomatený, přesto se však zmocní všeho, čeho jsem se při své moudrosti pod sluncem klopotně dopachtil. Také to je pomíjivost.
#2:20 Odvrátil jsem se od toho se srdcem plným zoufalství nad vším svým klopotným pachtěním pod sluncem.
#2:21 Některý člověk se pachtí moudře, dovedně a prospěšně, ale musí svůj podíl předat člověku, který se s tím nepachtil. Také to je pomíjivost a prašpatná věc.
#2:22 Vždyť co má člověk z veškerého svého pachtění, z honičky za žádostmi svého srdce, z toho, jak se pod sluncem pachtí?
#2:23 Všechny jeho dny jsou samá bolest a jeho lopota je plná hoře; jeho srdce mu nedá spát ani v noci. I to je pomíjivost.
#2:24 Není pro člověka dobré jíst a pít a při svém pachtění se aspoň pomět? Shledal jsem, že i to je z Boží ruky.
#2:25 Vždyť kdo jídal a dopřál si víc než já?
#2:26 Bůh dal člověku, který je mu milý, dává moudrost a poznání i radost. Hříšníka však nechá lopotit se, shánět a hromadit věci, které nakonec musí předat tomu, kdo se zalíbí Bohu. Také to je pomíjivost a honba za větrem. 
#3:1 Všechno má určenou chvíli a veškeré dění pod nebem svůj čas:
#3:2 Je čas rození i čas umírání, čas sázet i čas trhat;
#3:3 je čas zabíjet i čas léčit, čas bořit i čas budovat;
#3:4 je čas plakat i čas smát se, čas truchlit i čas poskakovat;
#3:5 je čas kameny rozhazovat i čas kameny sbírat, čas objímat i čas objímání zanechat;
#3:6 je čas hledat i čas ztrácet, čas opatrovat i čas odhazovat;
#3:7 je čas roztrhávat i čas sešívat, čas mlčet i čas mluvit;
#3:8 je čas milovat i čas nenávidět, čas boje i čas pokoje.
#3:9 Jaký užitek má ten, kdo pracuje, ze všeho svého pachtění?
#3:10 Viděl jsem lopotu, kterou Bůh uložil lidským synům, a tak se lopotí.
#3:11 On všechno učinil krásně a v pravý čas, lidem dal do srdce i touhu po věčnosti, jenže člověk nevystihne začátek ani konec díla, jež Bůh koná.
#3:12 Poznal jsem, že není pro něho nic lepšího, než se radovat a konat v životě dobro.
#3:13 A tak je tomu s každým člověkem: to, že jí a pije a okusí při veškerém svém pachtění dobrých věcí, je dar Boží.
#3:14 Poznal jsem, že vše, co činí Bůh, zůstává navěky; nic k tomu nelze přidat ani z toho ubrat. A Bůh to učinil, aby lidé žili v bázni před ním.
#3:15 Co se děje, bylo odedávna, a co bude, i to bylo; a Bůh vyhledává, co zašlo.
#3:16 Dále jsem pod sluncem viděl: na místě práva - svévole, na místě spravedlnosti - svévole.
#3:17 Řekl jsem si v srdci: Spravedlivého i svévolníka bude soudit Bůh, nastane čas soudu nad vším děním, nad vším, co se koná.
#3:18 Řekl jsem si v srdci: To se stane kvůli synům lidským, aby je Bůh tříbil, aby nahlédli, že je to s nimi jako se zvířaty.
#3:19 Vždyť úděl synů lidských a úděl zvířat je stejný: Jedni jako druzí umírají, jejich duch je stejný, člověk nemá žádnou přednost před zvířaty, neboť všechno pomíjí.
#3:20 Vše spěje k jednomu místu, všechno vzniklo z prachu a vše se v prach navrací.
#3:21 Kdo ví, zda duch lidských synů stoupá vzhůru a duch zvířat sestupuje dolů k zemi?
#3:22 Shledal jsem, že není nic lepšího, než když se člověk raduje z toho, co koná, neboť to je jeho podíl. Kdo mu dá nahlédnout, co se stane po něm? 
#4:1 Znovu jsem pohleděl na všechen útisk, který se pod sluncem děje. Hle, slzy utiskovaných, a oni jsou bez utěšitele, jejich utiskovatelé mají v rukou moc, a oni jsou bez utěšitele.
#4:2 Vychvaloval jsem mrtvé, kteří dávno zemřeli, více nežli živé, kteří ještě žijí.
#4:3 Ale nad oboje líp je na tom ten, kdo ještě není a nevidí zlo, které se pod sluncem děje.
#4:4 Viděl jsem též všechno pachtění i vše, co prospěšného se koná, a jak přitom jeden na druhého žárlí. Také to je pomíjivost a honba za větrem.
#4:5 Hlupák skládá ruce v klín a užírá se.
#4:6 Lepší na dlaň odpočinku než hrstě plné pachtění a honby za větrem.
#4:7 A znovu jsem pohleděl, jak všechno pod sluncem je pomíjivé.
#4:8 Někdo je sám a nikoho nemá, ani syna ani bratra, a všechno jeho pachtění nebere konce, jeho oko se bohatství nenasytí. „Pro koho se pachtím a sám se připravuji o pohodlí?“ Také to je pomíjivost a úmorná lopota.
#4:9 Lépe dvěma než jednomu, mají dobrou mzdu ze svého pachtění.
#4:10 Upadne-li jeden, druh jej zvedne. Běda samotnému, který upadne; pak nemá nikoho, kdo by ho zvedl.
#4:11 Také leží-li dva pospolu, je jim teplo; jak se má však zahřát jeden?
#4:12 Přepadnou-li jednoho, postaví se proti nim oba. A nit trojitá se teprv nepřetrhne!
#4:13 Lepší nuzný moudrý chlapec než starý král, ale hloupý, který nedovede přijmout poučení.
#4:14 Vyšel z vězení a stal se králem, narodil se jako chuďas a došel království.
#4:15 Shledal jsem, že všichni živí, ti, kdo chodí pod sluncem, byli při tom druhém, při chlapci, jenž nastoupil po něm.
#4:16 Nesčetný byl všechen lid, jemuž byl v čele, ale radovat se z něho nebude už příští pokolení. Věru i to je pomíjivost a honba za větrem.
#4:17 Dej si pozor na každý krok, když jdeš do Božího domu. Pohotovější buď k slyšení než k přinášení obětí jak hlupáci; ti ani nevědí, že činí něco zlého. 
#5:1 Ústa spěšně neotvírej, neukvapuj se v srdci, když máš pronést slovo před Bohem; vždyť Bůh je v nebi a ty na zemi, tak ať jsou nemnohá tvá slova.
#5:2 Po mnohé lopotě přichází sen; hlupák se ozývá mnoha slovy.
#5:3 Ty, když se zavážeš Bohu slibem, splň jej bez meškání, neboť v hlupácích nemá Bůh zalíbení. Co slíbíš, to splň!
#5:4 Lépe je, když neslibuješ, než když slíbíš a neplníš.
#5:5 Nedovol svým ústům, aby svedla ke hříchu tvé tělo, neříkej před Božím poslem: „To byl omyl.“ Proč se má Bůh rozlítit pro to, cos řekl, a zničit dílo tvých rukou?
#5:6 Kde mnoho snů, tam samá pomíjivost, samá prázdná slova. Ty se však boj Boha!
#5:7 Uvidíš-li, že je v kraji chudých utiskován, že je znásilňováno právo a spravedlnost, nediv se té zvůli; vyšší hlídá vysokého a nad nimi jsou ještě vyšší.
#5:8 Ale při tom všem vydává země užitek, vždyť pole slouží i králi.
#5:9 Kdo miluje peníze, peněz se nenasytí, kdo miluje hojnost, nemá nikdy dosti. Také to je pomíjivost.
#5:10 Když se rozmnožuje jmění, množí se i příživníci. Jaký prospěch z toho mívá vlastník? Ledaže se na to může dívat.
#5:11 Sladký je spánek toho, kdo pracuje, ať jí málo nebo mnoho, ale boháčovi nedopřeje spánku sytost.
#5:12 Je zlý neduh, který jsem pod sluncem viděl: vlastníkovi je ke zlému bohatství, jež střeží.
#5:13 Po úmorné lopotě může o bohatství přijít a syn, jehož zplodil, stojí s prázdnou rukou.
#5:14 Jako vyšel z života své matky, nahý zase odchází, jak přišel, a za svoje pachtění si nic neodnese, ani co by se do ruky vešlo.
#5:15 A také to je zlý neduh: Každý odejde, jak přišel; jaký užitek má z toho, že se pachtil a honil vítr?
#5:16 Nadto po všechny dny jídal ve tmě, a takového hoře, nemoci a hněvu!
#5:17 Hle, co jsem shledal: Je dobré a pěkné, aby člověk jedl a pil a měl se dobře při veškerém klopotném pachtění pod sluncem v časných dnech života, které mu dal Bůh, neboť to je jeho podíl.
#5:18 Tak je tomu s každým člověkem; to, že mu Bůh dal bohatství a poklady i možnost užívat jich, brát svůj podíl a radovat se při svém pachtění, je dar Boží.
#5:19 Ten totiž příliš nemyslí na dny svého života, protože Bůh oblažuje jeho srdce radostí. 
#6:1 Je zlo, které jsem pod sluncem viděl, a je mezi lidmi časté:
#6:2 Někomu Bůh dává bohatství a poklady i slávu, takže nepostrádá pro sebe nic z toho všeho, po čem touží. Bůh však mu nedá možnost toho užívat a má z toho užitek někdo cizí. To je pomíjivost a zlý neduh.
#6:3 Kdyby někdo zplodil synů sto a byl živ mnoho let a bylo sebevíc dnů jeho věku, pokud dobra neužil a nebyl řádně pohřben, pravím, že mrtvý plod je na tom lépe než on.
#6:4 Vešel do pomíjivosti, zapadl v temnotách a temnotou je přikryto jeho jméno;
#6:5 ani slunce nespatřil, nic nepoznal a má klidu víc než onen člověk.
#6:6 A kdyby žil dvakrát tisíc let a dobra neokusil, zdali oba neodejdou do stejného místa?
#6:7 Všechno lidské pachtění je pro ústa, duše ukojena není.
#6:8 A jakou má přednost moudrý před hlupákem? Co má z života utištěný, i kdyby žít uměl?
#6:9 Lepší je, co vidí oči, než za čím se žene duše; i to je pomíjivost a honba za větrem.
#6:10 Cokoli vzniklo, dávno dostalo své jméno. A je známo, že ten, jenž byl nazván Adam (to je Člověk), nemůže se přít s tím, který má převahu nad ním.
#6:11 Čím více slov, tím více pomíjivosti. A jaký užitek má z toho člověk?
#6:12 Kdo může vědět, co je člověku v životě k dobru, v časných dnech jeho pomíjivého žití, jež jako stín mu plynou? Kdo oznámí člověku, co pod sluncem nastane po něm? 
#7:1 Dobré jméno je nad výborný olej a den smrti nad den narození.
#7:2 Lépe je jít do domu truchlení než vejít do domu hodování, neboť tam je zřejmé, jak každý člověk skončí, a živý si to může vzít k srdci.
#7:3 Lepší je hoře než smích, neboť zachmuřelá tvář prospívá srdci.
#7:4 Srdce moudrých je v domě truchlení, ale srdce hlupáků v domě radovánek.
#7:5 Lépe je slyšet důtku od moudrého než poslouchat opěvování od hlupáků,
#7:6 neboť jako praskot trní pod hrncem je smích hlupáka. I to je pomíjivost.
#7:7 Útlak i z moudrého udělá ztřeštěnce a dar ničí srdce.
#7:8 Lepší je zakončení věci než její počátek, lepší je trpělivost nežli povýšenost.
#7:9 Nepropukej v náhlé hořekování, neboť hořekování si hoví v klíně hlupáků.
#7:10 Neříkej: „Čím to je, že dny dřívější byly lepší než tyto?“ To není moudré, ptáš-li se na to.
#7:11 Dobrá je moudrost, jakož i dědictví, užitečná těm, kdo vidí slunce;
#7:12 být ve stínu moudrosti je jako být ve stínu peněz, avšak užitečnější je poznat moudrost: ta zachovává život tomu, kdo ji má.
#7:13 Pohleď na Boží dílo! Kdo může narovnat, co on zkřivil?
#7:14 V den dobrý užívej dobra a v den zlý si uvědom, že ten i onen učinil Bůh proto, aby se člověk nedozvěděl, co bude po něm.
#7:15 To všechno jsem viděl ve dnech své pomíjivosti: Spravedlivý hyne i při své spravedlnosti a svévolník dlouho žije i při své zlobě.
#7:16 Nebuď příliš spravedlivý ani nadmíru moudrý; proč by ses měl zničit?
#7:17 Nebuď příliš svévolný a nebuď jako pomatenec; proč bys umíral, než vyprší tvůj čas?
#7:18 Bude dobře, když se tohoto přidržíš a ono nespustíš ze zřetele; vždyť kdo se bojí Boha, ujde obojímu.
#7:19 Moudrost dává moudrému více síly, než jakou má deset mocných v městě.
#7:20 Není na zemi člověka spravedlivého, aby konal dobro a nehřešil.
#7:21 Nevšímej si všech řečí, které se vedou, abys neuslyšel, jak tě zlehčuje tvůj otrok.
#7:22 Sám v srdci víš, žes i ty častokrát zlehčoval druhé.
#7:23 To všechno jsem chtěl vyzkoušet moudrostí; řekl jsem si: Budu moudrý. Ale moudrost se ode mne vzdalovala.
#7:24 Daleko je to, co se stalo, hluboko, přehluboko; kdo to nalezne?
#7:25 Zaměřil jsem se cele na to, abych poznal a prozkoumal a vyhledal moudrost a smysl všeho, abych poznal i hloupou svévoli a ztřeštěnou pomatenost.
#7:26 A přicházím na to, že trpčí než smrt je žena, je-li léčkou, je-li její srdce síť a ruce pouta. Kdo je Bohu milý, unikne jí, hříšník jí však bude lapen.
#7:27 Hleď, na to jsem přišel, řekl Kazatel, když jsem porovnával jedno s druhým, abych se dopídil smyslu,
#7:28 jejž jsem stále hledal a nenalézal: Mezi tisíci jsem našel jednoho člověka, ale ženu jsem mezi těmi všemi nenalezl.
#7:29 Hleď, jenom na to jsem přišel, že Bůh sice učinil člověka přímého, ten však vyhledává samé smyšlenky. 
#8:1 Kdo se vyrovná moudrému a kdo zná výklad věcí? Moudrost prosvítí člověku tvář, i tvrdost jeho tváře se změní.
#8:2 Pravím: Dbej na králův rozkaz, a to kvůli přísaze Boží.
#8:3 Neodcházej náhle od něho, nestůj při zlé věci; neboť on může učinit všechno, co se mu zlíbí.
#8:4 Vždyť královo slovo má moc a kdo mu řekne: „Co to děláš?“
#8:5 Kdo dbá na příkazy, neokusí nic zlého, srdce moudrého ví, že nastane čas soudu.
#8:6 Nad každým děním nastane čas soudu; na člověku je mnoho zlého.
#8:7 Nikdo neví, co nastane; kdo oznámí člověku, jak bude?
#8:8 Není člověka, jenž by měl svého ducha v moci, nezadrží jej, nemá moc nade dnem smrti; z toho boje není východiska, svévole nezachrání toho, kdo ji páchá.
#8:9 To všechno jsem viděl, když jsem si předsevzal zabývat se vším, co se pod sluncem děje v čase, kdy má člověk moc nad člověkem a působí mu zlo.
#8:10 Hned nato jsem pohleděl na pohřbené svévolníky; přicházívali a odcházeli ze svatého místa a ve městě se zapomínalo, jak jednali. Také to je pomíjivost.
#8:11 Poněvadž nad zločinem není hned vykonán rozsudek, tíhne srdce lidských synů k páchání zla.
#8:12 Přestože hříšník páchá zlo stokrát a lhůta se mu prodlužuje, já vím, že dobře bude těm, kdo se bojí Boha, těm, kdo se bojí jeho tváře.
#8:13 Avšak svévolníkovi se dobře nepovede, život se mu ani o den neprodlouží, bude jako stín, protože nemá bázeň před Boží tváří.
#8:14 Další pomíjivost, která se na zemi vyskytuje: Jsou spravedliví, které jako by poznamenalo dílo svévolníků, a jsou svévolníci, které jako by poznamenalo dílo spravedlivých. Řekl jsem si, že i to je pomíjivost.
#8:15 I vychvaloval jsem radost, protože pro člověka není pod sluncem nic lepšího než jíst a pít a radovat se. To ho provází při jeho pachtění ve dnech života, které mu Bůh pod sluncem dopřál.
#8:16 Jakmile jsem si předsevzal poznat moudrost, viděl jsem, co je na zemi lopoty; ve dne ani v noci člověk neokusí spánku.
#8:17 Spatřil jsem též, že veškeré dílo Boží, dílo, které se pod sluncem koná, není člověk schopen postihnout; ať se při tom hledání pachtí sebevíc, všechno nepostihne. Ani moudrý, řekne-li, že zná to či ono, není schopen všechno postihnout. 
#9:1 O tom všem jsem uvažoval a ve všem tom jsem zjistil, že spravedliví a moudří i jejich práce jsou v ruce Boží. Člověk neví, co milovat ani co nenávidět, cíl všeho je před ním.
#9:2 Všechno je u všech stejné: stejný úděl má spravedlivý i svévolný, dobrý a čistý i nečistý, obětující i neobětující; dobrý je na tom jako hříšník, přísahající jako ten, kdo se přísahat bojí.
#9:3 Na všem, co se pod sluncem děje, je zlé to, že všichni mají stejný úděl a že srdce lidských synů je naplněno zlobou; po celý svůj život mají v srdci samé ztřeštěnosti a pak se odeberou k mrtvým.
#9:4 Kdo tedy bude vyvolen? Všichni, kdo žijí, mají naději. Vždyť živý pes je na tom lépe než mrtvý lev.
#9:5 Živí totiž vědí, že zemrou, mrtví nevědí zhola nic a nečeká je žádná odměna, jejich památka je zapomenuta.
#9:6 Jak jejich láska, tak jejich nenávist i jejich horlení dávno zanikly a nikdy se již nebudou podílet na ničem, co se pod sluncem děje.
#9:7 Jdi, jez svůj chléb s radostí a popíjej své víno s dobrou myslí, neboť Bůh již dávno našel zalíbení ve tvém díle.
#9:8 Tvé šaty ať jsou v každé době bílé a tvá hlava ať nepostrádá vonný olej.
#9:9 Užívej života se ženou, kterou sis zamiloval, po všechny dny svého pomíjivého života. To ti je pod sluncem dáno po všechny dny tvé pomíjivosti, to je tvůj podíl v životě při tvém klopotném pachtění pod sluncem.
#9:10 Všechno, co máš vykonat, konej podle svých sil, neboť není díla ani myšlenky ani poznání ani moudrosti v říši mrtvých, kam odejdeš.
#9:11 Opět jsem pod sluncem viděl, že běh nezávisí na snaze hbitých ani boj na bohatýrech ani chléb na moudrých ani bohatství na rozumných ani přízeň na těch, kdo mají poznání, ale jak kdy každému z nich přeje čas a příležitost.
#9:12 Vždyť člověk ani nezná svůj čas. Je jako ryby, které se chytají do zlé sítě, a jako ptáci chytaní do osidla. Jako na ně, tak i na lidské syny bývá políčeno ve zlý čas, který je náhle přepadá.
#9:13 Také jsem spatřil pod sluncem tuto moudrost a jevila se mi veliká:
#9:14 Bylo malé město a v něm hrstka mužů. Tu přitáhl na ně velký král, obklíčil je a zbudoval proti němu mohutné náspy.
#9:15 Našel se pak v něm nuzný moudrý muž, který by byl to město svou moudrostí zachránil, ale nikdo si na toho nuzného muže ani nevzpomněl.
#9:16 Řekl jsem si: Lepší je moudrost než síla, třebaže modrostí toho nuzáka pohrdli a jeho slova nebyla slyšena.
#9:17 Slova moudrých v klidu vyslechnutá jsou lepší než křik toho, který panuje nad hlupáky.
#9:18 Moudrost je lepší než válečné zbraně, mnoho dobrého však zničí jediný hříšník. 
#10:1 Mrtvé mouchy způsobí, že mastičkářův olej páchne a kvasí; stejně působí špetka pomatenosti, je-li ceněna víc než moudrost a čest.
#10:2 Moudrý má srdce na pravém místě, hlupák na nepravém.
#10:3 Pomatenec, i když jde po cestě, je bez rozumu, o každém však říká: „To je pomatenec!“
#10:4 Zvedne-li se proti tobě vladařova nevole, své místo neopouštěj, mírnost může zabránit velikým hříchům.
#10:5 Zlo, jež jsem pod sluncem vídal, je omyl, k němuž dochází u mocnáře:
#10:6 Na vysoké místo se dosadí pomatenec, kdežto schopní zůstávají sedět dole.
#10:7 Viděl jsem otroky na ořích, kdežto knížata jak otroky chodit pěšky.
#10:8 Kdo kope jámu, spadne do ní, toho, kdo strhává zeď, uštkne had.
#10:9 Kdo láme kámen, přichází k úrazu, kdo štípe dříví, je při tom ohrožen.
#10:10 Ztupí-li se sekera a nenaostří-li se znovu, je nutno víc napnout síly. Užitečná a prospěšná je moudrost.
#10:11 Uštkne-li had, dříve než byl zaříkán, zaklínač pranic neprospěje.
#10:12 Slova z úst moudrému příjemně plynou, kdežto hlupáka vlastní rty pohlcují.
#10:13 Sotvaže promluví, prozradí pomatenost, a když domluví, zlou ztřeštěnost.
#10:14 Pomatenec má plno řečí. Co nastane, člověk neví, a kdo mu oznámí, co bude po něm?
#10:15 Hlupáci se unavují pachtěním, nevědí ani, kudy jít do města.
#10:16 Běda tobě, země, je-li tvým králem chlapec a tvá knížata hodují z jitra!
#10:17 Blaze tobě, země, je-li tvůj král vznešeného rodu a tvá knížata jídají v náležitý čas pro posilnění, a ne pro opilství.
#10:18 Pro lenošení sesouvá se krov, pro nečinnost rukou zatéká do domu.
#10:19 K obveselení se strojí pokrm a radost životu dává víno; peníze vyřeší všechno.
#10:20 Králi nezlořeč ani v mysli, boháčovi nezlořeč ani v pokojíku, kde uléháš, neboť nebeské ptactvo roznese ten hlas, okřídlenec vyzradí každé slovo. 
#11:1 Pouštěj svůj chléb po vodě, po mnoha dnech se s ním shledáš.
#11:2 Rozdej svůj díl mezi sedm, ba i osm, nevíš, co zlého se na zemi stane.
#11:3 Jestliže mraky se naplní, spustí se na zemi déšť; jestliže strom padne k jihu či na sever, zůstane na místě, kam padl.
#11:4 Kdo příliš dá na vítr, nebude sít, kdo hledí na mraky, nebude sklízet.
#11:5 Jako nevíš, jaká je cesta větru, jak v životě těhotné vznikají kosti, tak neznáš dílo Boha, který to všechno koná.
#11:6 Rozsévej své símě zrána, nedopřej svým rukám klidu do večera, neboť nevíš, zda se zdaří to či ono, či zda obojí je stejně dobré.
#11:7 Sladké je světlo a vidět slunce je milé očím.
#11:8 I když se člověk dožije mnoha let, ať se z nich raduje ze všech, ale na dny temnoty ať pamatuje, že jich bude mnoho. Cokoli přijde, je pomíjivost.
#11:9 Raduj se, jinochu, ze svého mládí, užívej pohody ve svém jinošství a jdi si cestami svého srdce, za vidinou svých očí. Věz však, že tě za to všechno Bůh postaví před soud.
#11:10 Hoře si ze srdce vykliď a drž si od těla zlo, vždyť mládí a úsvit jsou pomíjivé. 
#12:1 Pamatuj na svého Stvořitele ve dnech svého jinošství, než nastanou zlé dny a než se dostaví léta, o kterých řekneš: „Nemám v nich zalíbení“;
#12:2 než se zatmí slunce a světlo, měsíc, hvězdy, a vrátí se po dešti mraky.
#12:3 V ten den se začnou třást strážcové domu a mužové zdatní se zkřiví a mlečky nechají práce a bude jich málo, a ty, kdo hledí z oken, obestře temnota,
#12:4 a zavrou se dveře do ulice a ztiší se hlas mlýnku a vstávat se bude za šveholu ptactva a všechny zpěvy budou znít přidušeně.
#12:5 A člověk se bude bát výšek a úrazů na cestě; a rozkvete mandloň a těžce se povleče kobylka a kapara ztratí účinnost. Člověk se vydá do svého věčného domu a ulicí budou obcházet ti, kdo naříkají nad mrtvými.
#12:6 Pamatuj na svého Stvořitele, než se přetrhne stříbrný provaz a rozbije se mísa zlatá a džbán se roztříští nad zřídlem a kolo u studny se zláme.
#12:7 A prach se vrátí do země, kde byl, a duch se vrátí k Bohu, který jej dal.
#12:8 Pomíjivost, samá pomíjivost, řekl Kazatel, všechno pomíjí.
#12:9 Zbývá jen dodat, že Kazatel byl moudrý, lid stále poznání vyučoval; přemýšlel a bádal a složil přísloví mnohá.
#12:10 Kazatel se snažil najít výstižná slova; tak bylo zapsáno, co je pravé, slova pravdy.
#12:11 Slova moudrých jsou jako bodce, jako vbité hřeby jsou slova sběratelů, pastýřem jediným proslovená.
#12:12 Nadto pak zbývá, synu můj, říci: Přijmi poučení! Spisování mnoha knih nebere konce a mnohé hloubání unaví tělo.
#12:13 Závěr všeho, co jsi slyšel: Boha se boj a jeho přikázání zachovávej; na tom u člověka všechno závisí.
#12:14 Veškeré dílo Bůh postaví před soud, i vše, co je utajeno, ať dobré či zlé.  

\book{Song of Solomon}{Song}
#1:1 Nejkrásnější z písní Šalomounových.
#1:2 Kéž políbí mě polibkem svých úst! Vždyť lepší je tvé laskání než víno.
#1:3 Příjemně voní tvé oleje, nejčistší olej - tvé jméno. Proto tě dívky milují.
#1:4 Táhni mne za sebou! Dáme se v běh. Král uvedl mě do svých komnat. Budeme jásat, radovat se z tebe, připomínat tvé laskání, opojnější než víno. Právem tě všichni milují.
#1:5 Černá jsem, a přece půvabná, jeruzalémské dcery, jak stany Kédarců, jako stanové houně Šalomounovy.
#1:6 Nehleďte na mne, že jsem až dočerna opálená, že mě tak ožehlo slunce. To synové mé matky se proti mně rozohnili: uložili mi vinice hlídat, neuhlídala jsem však vinici vlastní.
#1:7 Pověz mi ty, kterého tolik miluji, kde budeš pást, kde necháš odpočívat stáda za poledne! Proč musím být jako zahalená poběhlice při stádech tvých druhů?
#1:8 „Jestliže to sama nevíš, nejkrásnější z žen, vyjdi po šlépějích stád a kůzlátka svá pas u pastýřských kolib.“
#1:9 Ke klisně vozu faraónova jsem tě připodobnil, má přítelkyně.
#1:10 Půvabné jsou tvé tváře přívěsky ozdobené, tvé hrdlo ovinuté šňůrou perel.
#1:11 Přívěsky zlaté ti uděláme, poseté stříbrem.
#1:12 Pokud je při stole král, vydává nard můj svou vůni.
#1:13 Voničkou myrhy je pro mne můj milý, spočívá na mých prsou.
#1:14 Hroznem henny je pro mne můj milý v éngedských vinicích.
#1:15 Jak jsi krásná, přítelkyně moje, jak jsi krásná, oči tvé jsou holubice.
#1:16 Jak jsi krásný, milý můj, jsi líbezný! A naše lůžko samá zeleň.
#1:17 Trámoví našeho domu je z cedrů, deštění cypřišové. 
#2:1 Jsem kvítek šáronský, lilie v dolinách.
#2:2 Jako lilie mezi trním, tak má přítelkyně mezi dcerami.
#2:3 Jako jabloň mezi lesními stromy, tak můj milý mezi syny. Usedla jsem žádostivě v jeho stínu, jeho ovoce mi sládne na rtech.
#2:4 On mě uvedl do domu vína, jeho prapor nade mnou je láska.
#2:5 Občerstvěte mě koláči hroznovými, osvěžte mě jablky, neboť jsem nemocna láskou.
#2:6 Jeho levice je pod mou hlavou, jeho pravice mě objímá.
#2:7 Zapřísahám vás, jeruzalémské dcery, při gazelách a při polních laních: nebuďte a nezburcujte lásku, dokud nebude chtít sama.
#2:8 Hlas mého milého! Hle, právě přichází, hory přeskakuje, přenáší se přes pahorky.
#2:9 Gazele se podobá můj milý nebo kolouškovi. Hle, právě stojí za naší zídkou, nahlíží do oken, dívá se mřížováním.
#2:10 Můj milý se ozval, řekl mi: „Vstaň, má přítelkyně, krásko má, a pojď!
#2:11 Hle, zima pominula, lijavce přešly, jsou tytam.
#2:12 Po zemi se objevují květy, nadešel čas prořezávat révu, hlas hrdličky je slyšet v naší zemi.
#2:13 Fíkovník nasadil první plody, voní kvítky vinné révy. Vstaň, má přítelkyně, krásko má, a pojď!“
#2:14 Holubičko moje v rozsedlinách skály, v úkrytu nad strží, dopřej mi zahlédnout tvou tvář, dovol mi hlas tvůj slyšet. Jak lahodný je tvůj hlas! Jak půvabnou máš tvář!
#2:15 „Lišky nám schytejte, lištičky malé, plenící vinice, vinice naše, když kvetou!“
#2:16 Můj milý je můj a já jsem jeho, on pase v liliích.
#2:17 Než zavane den a stíny dají se v běh, přiběhni, milý můj, podoben gazele či kolouškovi na Béterských horách. 
#3:1 Noc co noc hledala jsem na svém lůžku toho, kterého tolik miluji. Hledala jsem ho, a nenalezla.
#3:2 Teď vstanu a obejdu město, ulice, náměstí, vyhledám toho, kterého tolik miluji. Hledala jsem ho, a nenalezla.
#3:3 Našli mě strážci obcházející město: „Toho, kterého tolik miluji, jste tu neviděli?“
#3:4 Potom, jen co jsem od nich odešla, hned jsem nalezla toho, kterého tolik miluji. Uchopila jsem ho a už ho nepustím, dokud ho nepřivedu do domu své matky, do pokojíku té, jež mě počala.
#3:5 Zapřísahám vás, jeruzalémské dcery, při gazelách a při polních laních: nebuďte a nezburcujte lásku, dokud nebude chtít sama.
#3:6 „Kdo je ta, jež vystupuje z pouště jako sloup dýmu, ovanuta vůní kadidlovou z myrhy a z nejjemnějšího koření kupeckého?“
#3:7 Hle, jeho lože - lože Šalomouna, šedesát bohatýrů okolo stojí, bohatýrů z Izraele.
#3:8 Všichni drží v rukou meče, vycvičeni k boji, každý po boku má meč proti nočnímu děsu.
#3:9 Nosítka král si zhotovil, král Šalomoun, ze stromů libanónských.
#3:10 Sloupky k nim zhotovil stříbrné, opěradlo zlaté, sedadlo purpurové. Vnitřek je obložen láskou jeruzalémských dcer.
#3:11 Vyjděte jen a pohleďte, sijónské dcery, na krále Šalomouna, na korunu, jíž ho korunovala jeho matka v den jeho svatby, v den, kdy jeho srdce naplnila radost. 
#4:1 Jak jsi krásná, přítelkyně moje, jak jsi krásná, oči tvé jsou holubice pod závojem, vlasy tvé jsou jako stáda koz, které se hrnou z hory Gileádu.
#4:2 Zuby tvé jsou jako stádo ovcí před střiháním, jež z brodiště vystupují, a každá z nich vrhne po dvou, žádná z nich neplodná nezůstane.
#4:3 Jako karmínová šňůrka jsou tvé rty, ústa tvá půvabu plná. Jak rozpuklé granátové jablko jsou tvoje skráně pod závojem.
#4:4 Tvé hrdlo je jak Davidova věž z vrstev kamene zbudovaná, tisíc na ní zavěšeno štítů, samých pavéz bohatýrů.
#4:5 Dva prsy tvé jsou jak dva koloušci, dvojčátka gazelí, která se v liliích pasou.
#4:6 Než zavane den a stíny dají se v běh, vydám se k myrhové hoře, k pahorku kadidlovému.
#4:7 Celá jsi krásná, přítelkyně moje, poskvrny na tobě není.
#4:8 Se mnou z Libanónu, nevěsto má, se mnou z Libanónu půjdeš. Rozhlédneš se z vrcholu Amány, z vrcholu Seníru a Chermónu, ze lvích doupat, z hor leopardů.
#4:9 Učarovala jsi mi, sestro má, nevěsto, učarovala jsi mi jediným pohledem svých očí, jediným článkem svého náhrdelníku.
#4:10 Oč krásnější je tvé laskání, sestro má, nevěsto, oč lepší je tvé laskání než víno. Vůně tvých olejů nad všechny balzámy.
#4:11 Ze rtů ti kane strdí, má nevěsto, pod tvým jazykem je med a mléko, a vůně tvých šatů je jak vůně Libanónu.
#4:12 Zahrada uzavřená jsi, sestro má, nevěsto, uzavřený val, zapečetěný pramen.
#4:13 Vydáváš vůni jako sad s jablky granátovými, s výtečným ovocem, hennou i nardem,
#4:14 s nardem a šafránem, puškvorcem, skořicí, se vším kadidlovým stromovím, myrhou a aloe, se všemi balzámy nejlepšími.
#4:15 Jsi pramen zahradní, studna vody živé, bystřina z Libanónu.
#4:16 Probuď se, vánku severní, přijď, vánku jižní, ať voní moje zahrádka, ať její balzámy proudí jak bystřiny, ať přijde do své zahrady můj milý a jí výtečné ovoce její. 
#5:1 Do zahrady své jsem přišel, sestro má, nevěsto, sbíral jsem svou myrhu a svůj balzám, z plástve jsem jedl svůj med, pil víno své a mléko. Jezte, přátelé, a pijte, opájejte se laskáním.
#5:2 Spím, ale srdce mé bdí. Slyš, milý můj klepe: „Otevři mi, sestro má, přítelkyně má, holubice má, má bezúhonná, vždyť mám hlavu plnou rosy, v kadeřích krůpěje noční.“
#5:3 Svlékla jsem šaty, mám je zas oblékat? Umyla jsem si nohy, mám si je zašpinit?“
#5:4 Můj milý prostrčil otvorem ruku a celé mé nitro ze zachvělo před ním.
#5:5 Vstala jsem otevřít milému svému. Z rukou mi kanula myrha, myrha stékala z mých prstů na rukojeť zástrčky.
#5:6 Než jsem však milému otevřela, můj milý odbočil jinam. Život ze mne prchal, když ke mně mluvil. Hledala jsem ho, a nenalezla, volala jsem ho, a neodpověděl mi.
#5:7 Našli mě strážci obcházející město, zbili mě, zranili mě, přehoz mi strhli strážci hradeb.
#5:8 Zapřísahám vás, jeruzalémské dcery, jestliže najdete mého milého, co mu sdělíte? Že jsem nemocna láskou.
#5:9 Jaký je tvůj milý, že je nad Miláčka, ty nejkrásnější z žen? Jaký je tvůj milý, že je nad Miláčka, že nás tak zapřísaháš?
#5:10 Můj milý je běloskvoucí i červený, významnější nad tisíce jiných.
#5:11 Jeho hlava je třpytivé zlato ryzí, jeho kadeře jsou trsy palmových plodů, černé jako havran.
#5:12 Jeho oči jsou jako holubi nad potůčky vod, v mléce se koupou, podobné vsazeným drahokamům.
#5:13 Jeho líce jsou jak balzámový záhon, schránky kořenných vůní, jeho rty jsou lilie, z nichž kane tekutá myrha.
#5:14 Jeho ruce jsou válce zlaté taršíšem posázené, jeho břicho je mistrné dílo ze slonoviny safíry vykládané.
#5:15 Jeho stehna jsou sloupy z bílého mramoru, spočívající na patkách z ryzího zlata. Vzhled má jak Libanón, je ztepilý jak cedr.
#5:16 Patro jeho úst je přesladké, on sám je přežádoucí skvost. Takový je milý můj, takový je můj přítel, jeruzalémské dcery. 
#6:1 Kam odešel tvůj milý, ty nejkrásnější z žen? Kam se obrátil tvůj milý? Budeme ho s tebou hledat.
#6:2 Můj milý sestoupil do své zahrady k záhonům balzámovým, aby v zahradách pásl a trhal lilie.
#6:3 Já jsem svého milého a můj milý je můj, on pase v liliích.
#6:4 Krásná jsi, přítelkyně má, jak Tirsa, půvabná jak Jeruzalém, strašná jako vojsko pod praporci.
#6:5 Odvrať ode mne své oči, vždyť mě uhranuly! Vlasy tvé jsou jako stáda koz, které se hrnou z Gileádu.
#6:6 Zuby tvé jsou jako stádo březích ovcí, jež z brodiště vystupují a každá z nich vrhne po dvou, žádná z nich neplodná nezůstane.
#6:7 Jak rozpuklé granátové jablko jsou tvoje skráně pod závojem.
#6:8 Byť tu bylo šedesát královen a osmdesát ženin a dívek bez počtu,
#6:9 ona jediná je holubice moje, moje bezúhonná, jedinečná ze své matky, přečistá z té, jež ji porodila. Spatřily ji dcery, blahoslavily ji, královny i ženiny jí vzdaly chválu.
#6:10 „Kdo je ta, jež jak Jitřenka shlíží, krásná jako Luna, čistá jako žhoucí Slunce, strašná jako vojsko pod praporci?“
#6:11 Sestoupil jsem do zahrady ořechové podívat se na poupátka do údolí, podívat se, zda už pučí vinná réva, zda rozkvetly granátové stromy.
#6:12 Sám nevím, jak jsem se dostal do Amínádíbových vozů. 
#7:1 „Obrať se, obrať se, Šulamítko, obrať se, obrať se, chceme tě vidět.“ Co na Šulamítce uvidíte? Že tančí táborový tanec!
#7:2 Jak krásné jsou tvé nohy v opánkách, knížecí dcero! Křivky tvých boků jsou jako náhrdelníky, dílo umělcových rukou.
#7:3 Tvůj pupek je pěkně vykroužená mísa, kéž nechybí v ní vonné víno! Tvé břicho je stoh pšeničný, obrostlý liliemi.
#7:4 Dva prsy tvé jsou jak dva koloušci, dvojčátka gazelí.
#7:5 Tvé hrdlo je jak věž ze slonoviny, tvé oči - rybníky v Chešbónu u brány Batrabímské. Tvůj nos je jak libanónská věž, zkoumavě hledící k Damašku.
#7:6 Tvá hlava se tyčí jako Karmel, vrkoče tvé hlavy jsou jak purpur. Král je těmi kadeřemi spoután.
#7:7 Jak krásná, jak líbezná jsi, lásko, při hrách milostných!
#7:8 Postavou se podobáš palmě a svými prsy hroznům datlí.
#7:9 Řekl jsem: „Vystoupím na palmu, abych se zmocnil plodů.“ Tvé prsy ať jsou hrozny révovými, dech tvého chřípí ať jablky voní,
#7:10 patro tvých úst ať je jako nejlepší víno. Jedině pro mého milého stéká, plyne i ve spánku ze rtů.
#7:11 Já jsem svého milého, on dychtí jen po mně.
#7:12 Pojď, můj milý, vyjděme na pole, přenocujeme v keřích henny.
#7:13 Časně zrána půjdeme do vinic, pohledíme, zda pučí vinná réva, zda její květ se rozvil, zda rozkvetly granátové stromy. Tam tě zahrnu laskáním.
#7:14 Voní jablíčka lásky a všechny výtečné plody nad našimi dveřmi, nové i staré. Schovala jsem je pro tebe, můj milý. 
#8:1 Kéž bys byl jako můj bratr, který sál z prsů mé matky! Až bych tě nalezla někde venku, políbila bych tě a nikdo by mnou pohrdat nesměl.
#8:2 Odvedla bych si tě, uvedla tě do domu své matky a tam bys mě poučoval. Dala bych ti pít kořenné víno, šťávu ze svých granátových jablek.
#8:3 Jeho levice je pod mou hlavou, jeho pravice mě objímá.
#8:4 Zapřísahám vás, jeruzalémské dcery, nebuďte a nezburcujte lásku, dokud nebude chtít sama.
#8:5 Kdo je ta, jež vystupuje z pouště, opřena o svého milého? Zburcovala jsem tě pod jabloní, kde tě počala tvá matka, kde tě počala ta, jež tě porodila.
#8:6 Polož si mě na srdce jako pečeť, jako pečeť na své rámě. Vždyť silná jako smrt je láska, neúprosná jako hrob žárlivost lásky. Žár její - žár ohně, plamen Hospodinův.
#8:7 Lásku neuhasí ani velké vody a řeky ji nezaplaví. Kdyby za lásku chtěl někdo dávat všechno jmění svého domu, sklidil by jen pohrdání.
#8:8 „Maličkou máme sestru, prsy ještě nemá. Co s tou sestrou uděláme v den, kdy o ni přijdou smlouvat?
#8:9 Jestliže je hradbou, stříbrné cimbuří na ní postavíme, jestliže je dveřmi, zahradíme je cedrovou deskou.“
#8:10 „Já jsem hradba, mé prsy jsou jako věže.“ Tehdy stala jsem se v jeho očích tou, která nalézá pokoj.
#8:11 Vinici měl Šalomoun v Baal-hamónu; svěřil tu vinici hlídačům; za její ovoce každý mu přinést musí tisíc šekelů stříbra.
#8:12 „Má vinice patří mně, jen mně samotnému. Měj si ten tisíc, Šalomoune, a dvě stě pro ty, kdo hlídají ovoce její.
#8:13 Ty, která prodléváš v zahradách, kde druhové sledují hlas tvůj, ozvi se mi!“
#8:14 „Uprchni, milý můj, podoben gazele či kolouškovi na balzámových horách.“  

\book{Isaiah}{Isa}
#1:1 Vidění Izajáše, syna Amósova, které viděl o Judsku a Jeruzalému za dnů Uzijáše, Jótama, Achaza a Chizkijáše, králů judských.
#1:2 Slyšte, nebesa, naslouchej, země, tak promluvil Hospodin: „Syny jsem vychoval, pečoval o ně, ale vzepřeli se mi.
#1:3 Vůl zná svého hospodáře, osel jesle svého pána, mne však Izrael nezná, můj lid je nechápavý.“
#1:4 Ach, pronárode hříšný, lide obtížený vinou, potomstvo zlovolníků, synové šířící zkázu! Opustili Hospodina, Svatého, Boha Izraele, znevážili, odcizili se mu.
#1:5 Nač vás ještě bít? Jste jen umíněnější. Hlava je celá chorá a celé srdce zemdlené.
#1:6 Od hlavy až k patě nic zdravého není. Samá modřina a jizva i čerstvá rána, nejsou vymačkány ani obvázány ani ošetřeny olejem.
#1:7 Vaše země je zpustošená, vaše města vypálená ohněm, vaši půdu vám před očima vyžírají cizáci; je zpustošená, cizáky podvrácená.
#1:8 Dcera sijónská zůstala jako chatrč na vinici, jako budka v okurkovém poli, jako obležené město.
#1:9 A kdyby Hospodin zástupů nám neponechal hrstku těch, kdo přežili, byli bychom jako Sodoma, podobni Gomoře bychom byli.
#1:10 Slyšte slovo Hospodinovo, sodomští náčelníci, naslouchejte naučení našeho Boha, lide gomorský.
#1:11 „K čemu je mi množství vašich obětních hodů, praví Hospodin. Přesytil jsem se zápalných obětí beranů i tuku vykrmených dobytčat, nemám zájem o krev býčků, beránků a kozlů.
#1:12 Že se mi chodíte ukazovat! Kdo po vás chce, abyste šlapali má nádvoří?
#1:13 Nepřinášejte už šalebné obětní dary, kouř kadidla je mi ohavností, i novoluní, dny odpočinku a svolaná shromáždění; ničemnost a slavnostní shromáždění, to nemohu vystát.
#1:14 Z duše nenávidím vaše novoluní a slavnosti, jsou mi jen na obtíž, jsem vyčerpán, když je musím snášet.
#1:15 Když rozprostíráte své dlaně, zakrývám si před vámi oči. Ať se modlíte sebevíc, neslyším. Vaše ruce jsou celé od krve.
#1:16 Omyjte se, očisťte se, odkliďte mi své zlé skutky z očí, přestaňte páchat zlo.
#1:17 Učte se činit dobro. Hledejte právo, zakročte proti násilníku, dopomozte k právu sirotkovi, ujímejte se pře vdovy.
#1:18 Pojďte, projednejme to spolu, praví Hospodin. I kdyby vaše hříchy byly jako šarlat, zbělejí jako sníh, i kdyby byly rudé jako purpur, budou bílé jako vlna.
#1:19 Budete-li povolní a poslechnete, budete požívat dobrých darů země.
#1:20 Když však odmítnete a budete se zpěčovat, bude vás požírat meč.“ Tak promluvila Hospodinova ústa.
#1:21 Běda! Nevěstkou se stalo město věrné! Bývalo plné práva, přebývala v něm spravedlnost, teď jsou v něm vrahové.
#1:22 Tvé stříbro se stalo struskou, tvé víno je smíšené s vodou,
#1:23 tvoji velmoži jsou umíněnci, spolčenci zlodějů. Kdekdo miluje úplatek a rád bere dary; sirotkovi právo nezjednají a pře vdovy se k nim nedostane.
#1:24 Takový je výrok Pána, Hospodina zástupů, Přesilného, Boha Izraele: „Ach, jak se potěším na svých protivnících, vykonám pomstu na svých nepřátelích!
#1:25 Obrátím na tebe svou ruku, vytavím tvou strusku, jako louhem odloučím všechny tvé přimíšeniny.
#1:26 A vrátím ti soudce, jako byli na počátku, tvoje rádce, jako byli zprvu; pak tě budou nazývat městem spravedlnosti, městem věrným.“
#1:27 Sijón bude vykoupen soudem a ti v něm, kdo se obrátí, spravedlností.
#1:28 Zkáza nevěrných a hříšníků však bude stejná, zajdou ti, kdo opustili Hospodina.
#1:29 Budou zahanbeni pro mohutné stromy, po nichž jste dychtili, budete se stydět za zahrady, které jste si zvolili.
#1:30 Budete jako posvátný strom, jemuž vadne listí, jako zahrada, v níž není vody.
#1:31 Nejzdatnější bude koudelí a jeho skutky jiskrou; vzplane spolu obojí a nikdo to neuhasí. 
#2:1 Slovo o Judsku a Jeruzalému, jež ve vidění přijal Izajáš, syn Amósův.
#2:2 I stane se v posledních dnech, že se hora Hospodinova domu bude tyčit nad vrcholy hor, bude povznesena nad pahorky a budou k ní proudit všechny pronárody.
#2:3 Mnohé národy půjdou a budou se pobízet: „Pojďte, vystupme na horu Hospodinovu, do domu Boha Jákobova. Bude nás učit svým cestám a my po jeho stezkách budeme chodit.“ Ze Sijónu vyjde zákon, slovo Hospodinovo z Jeruzaléma.
#2:4 On bude soudit pronárody, on ztrestá národy mnohé. I překují své meče na radlice svá kopí na vinařské nože. Pronárod nepozdvihne meč proti pronárodu, nebudou se již cvičit v boji.
#2:5 Nuže, dome Jákobův, choďme v Hospodinově světle!
#2:6 Odmrštil jsi svůj lid, Jákobův dům, protože jsou plni východního pohanství, věští z mraků jako Pelištejci, s dětmi cizáků si podávají ruce.
#2:7 Jeho země je plná stříbra a zlata, jeho poklady jsou nepřeberné, jeho země je plná koní a jeho vozům není konce.
#2:8 Jeho země je plná bůžků, klanějí se dílu svých rukou, tomu, co vyrobili svými prsty.
#2:9 Člověk se hrbí, muž se ponižuje, a ty jim nepromineš.
#2:10 Zalez do skal, schovej se v prachu ze strachu před Hospodinem, před jeho velebnou důstojností!
#2:11 Ponížen bude zpupný pohled člověka, sehnuta bude lidská povýšenost; v onen den bude vyvýšen jedině Hospodin.
#2:12 Neboť den Hospodina zástupů přijde na každou pýchu a povýšenost, na všechno, co se povznáší - to bude sníženo -,
#2:13 na všechny cedry Libanónu, vysoké a vznosné, na všechny bášanské duby,
#2:14 na všechny vysoké hory a na všechny pahorky vyvýšené,
#2:15 na každou vypínající se věž a na každou strmou hradbu,
#2:16 na všechny zámořské lodě i na veškerou okázalost.
#2:17 Sehnuta bude zpupnost člověka, ponížena bude lidská povýšenost, v onen den bude vyvýšen jedině Hospodin.
#2:18 Bůžkové nadobro vymizejí.
#2:19 Lidé zalezou do jeskyň v skalách a do škvír v prachu země ze strachu před Hospodinem, před jeho velebnou důstojností, až povstane, aby nahnal zemi strach.
#2:20 V onen den člověk předhodí potkanům a netopýrům své bůžky stříbrné i bůžky zlaté, které mu vyrobili, aby se jim klaněl.
#2:21 Zaleze do skalních rozsedlin a do strží ve skaliskách ze strachu před Hospodinem, před jeho velebnou důstojností, až povstane, aby nahnal zemi strach.
#2:22 Přestaňte už s člověkem, který nemá než dech svého chřípí. Jakoupak má cenu? 
#3:1 Hle, Pán, Hospodin zástupů, již odnímá Jeruzalému a Judsku oporu i podpěru, všechnu oporu chleba i všechnu oporu vody,
#3:2 bohatýra i bojovníka, soudce i proroka, věštce i starce,
#3:3 velitele i vznešeného, rádce i zručného řemeslníka a zkušeného zaříkávače.
#3:4 Za velitele dám jim chlapce, vládnout jim budou výrostkové.
#3:5 V lidu bude popohánět jeden druhého, druh druha, chlapec zaútočí na starého, bezectný na váženého.
#3:6 Člověk uchopí svého bratra z otcovského domu a řekne: „Máš plášť, buď naším náčelníkem! Měj si tyto rozvaliny v moci.“
#3:7 On se však v onen den ohradí: „Nebudu ranhojičem, nemám doma ani chléb ani plášť. Nečiňte mě náčelníkem lidu!“
#3:8 Jeruzalém klopýtl a Judsko padlo; jejich jazyk a jejich skutky byly proti Hospodinu a do očí vzdorovali jeho slávě.
#3:9 Nestoudnost jejich tváře proti nim svědčí, jako Sodoma svůj hřích vystavují a neskrývají. Běda jim, připravili si zlou odplatu.
#3:10 Řekněte: „Spravedlivému bude dobře, bude jíst ovoce svých skutků.“
#3:11 Běda, zle bude svévolníku, tomu se dostane odplaty za to, co páchal.
#3:12 A můj lid - pacholata jsou jeho poháněči a vládnou mu ženy. Můj lide, ti, kdo řídí tvé kroky, jsou svůdci a matou cestu tvého putování.
#3:13 Hospodin se postavil, povede spor, stojí, aby vedl při s národy.
#3:14 Hospodin zahajuje soud se staršími svého lidu a s velmoži jeho: „Spásali jste vinici, máte ve svých domech, oč jste obrali utištěného.
#3:15 Jak to, že deptáte můj lid a drtíte tvář utištěných?“ je výrok Panovníka, Hospodina zástupů.
#3:16 I řekl Hospodin: „Dcery sijónské se vypínají, chodí s pozdviženou šíjí, svůdně hledí, cupitavě chodí, na nohou jim chřestí nákotníky.“
#3:17 Panovník proto zachvátí témě dcer sijónských prašivinou, Hospodin obnaží jejich spánky.
#3:18 V onen den odejme Panovník okrasné nákotníky, náčelky a půlměsíčky,
#3:19 náušnice, náramky a závojíčky,
#3:20 čelenky a řetízky, stužky, talismany a amulety,
#3:21 prsteny a nosní kroužky,
#3:22 slavnostní roucha a přehozy, šátky a váčky,
#3:23 zrcátka i košilky, turbany a řízy.
#3:24 A stane se, že bude namísto balzámu hnis a místo pásu smyčka, místo pracných účesů lysina, místo skvostného roucha žíněný pás a vypálené znamení na místě krásy.
#3:25 Tvoji muži, Sijóne, padnou mečem a bohatýři v boji.
#3:26 Rmoutit se a truchlit budou brány Sijónu; o vše připravený bude sedět na zemi. 
#4:1 Sedm žen se chytí jednoho muže v onen den. Řeknou: „Budeme jíst vlastní chléb a odívat se vlastními šaty, jenom ať se jmenujeme tvým jménem, zbav nás naší potupy.“
#4:2 V onen den bude výhonek Hospodinův chloubou a slávou, plod země důstojností a okrasou pro ty z Izraele, kdo vyvázli.
#4:3 Kdo zůstane na Sijónu a kdo zbude v Jeruzalémě, bude zván svatým, každý zapsaný k životu v Jeruzalémě.
#4:4 A Panovník smyje špínu dcer sijónských a odplaví krev ze středu Jeruzaléma duchem soudu, duchem žhoucím.
#4:5 A nad každým místem hory sijónské i nad každým jejím shromážděním stvoří Hospodin ve dne oblak, totiž kouř, a plameny ohnivé záře v noci: to bude baldachýn nad veškerou slávou,
#4:6 stánek, který dá stín před denním horkem, útočiště a úkryt před přívalem i deštěm. 
#5:1 Zazpívám svému milému píseň mého milého o jeho vinici: „Můj milý měl vinici na úrodném svahu.
#5:2 Zkypřil ji, kameny z ní vybral a vysadil ušlechtilou révu. Uprostřed ní vystavěl věž i lis v ní vytesal a čekal, že vydá hrozny; ona však vydala odporná pláňata.
#5:3 Teď tedy, obyvateli Jeruzaléma a muži judský, rozhodněte spor mezi mnou a mou vinicí.
#5:4 Co se mělo pro mou vinici ještě udělat a já pro ni neudělal? Když jsem očekával, že vydá hrozny, jak to, že vydala odporná pláňata?
#5:5 Nyní vás tedy poučím, co se svou vinicí udělám: Odstraním její ohrazení a přijde vniveč, pobořím její zídky a bude pošlapána.
#5:6 Udělám z ní spoušť, nebude už prořezána ani okopána a vzejde bodláčí a křoví, mrakům zakážu zkrápět ji deštěm.“
#5:7 Vinice Hospodina zástupů je dům izraelský a muži judští sadbou, z níž měl potěšení. Čekal právo, avšak hle, bezpráví, spravedlnost, a hle, jen úpění.
#5:8 Běda těm, kteří připojují dům k domu a slučují pole s polem, takže nezbývá žádné místo, jako byste jen vy byli usedlíci v zemi.
#5:9 V uších mi zní slovo Hospodina zástupů: „Ano, zpustnou četné domy, i velké a pěkné budou liduprázdné.
#5:10 Deset jiter vinice vydá jediný bat a chómer osevu vydá jedinou éfu.“
#5:11 Běda těm, kteří za časného jitra táhnou za opojným nápojem a až do setmění je rozpaluje víno.
#5:12 Citara a harfa, bubínek a píšťala a víno jsou na jejich pitkách, ale na dílo Hospodinovo neberou zřetel a dílo jeho rukou nevidí.
#5:13 Proto bude můj lid vystěhován, neboť nemá poznání. I nejváženější budou hladovět, jeho hlučící dav bude prahnout žízní.
#5:14 Podsvětí roztáhne svůj chřtán a dokořán rozevře svou tlamu. Do něho sestoupí jeho důstojnost i jeho hlučící dav, jeho hukot a jásot.
#5:15 Sehnut bude člověk a ponížen muž, oči zpupné budou poníženy.
#5:16 Hospodin zástupů se však vyvýší soudem a svatý Bůh se prokáže svatým spravedlností.
#5:17 Jako na své pastvině se tam budou pást beránci, a kde zůstaly trosky po zbohatlících, budou jíst bezdomovci.
#5:18 Běda tě, kdo v provazech šalby vlečou nepravost, jako by v postraňcích táhli povoz hříchu,
#5:19 a říkají: „Nechť urychlí a uspíší své dílo, ať je uvidíme; nechť se přiblíží a uskuteční úradek Svatého Boha Izraele, ať jej poznáme.“
#5:20 Běda těm, kdo říkají zlu dobro a dobru zlo, kdo vydávají tmu za světlo a světlo za tmu, kdo vydávají hořké za sladké a sladké za hořké!
#5:21 Běda těm, kdo jsou moudří ve vlastních očích a rozumní sami před sebou.
#5:22 Běda bohatýrům zdatným v pití vína, mužům udatným v míchání opojných nápojů,
#5:23 těm, kdo za úplatek ospravedlňují svévolníka a spravedlivým upírají spravedlnost.
#5:24 Proto jako jazyky ohně stravují strniště a suchá tráva mizí v plamenech, tak ztrouchniví jejich kořen a jejich květ odlétne jako prach, neboť odvrhli zákon Hospodina zástupů a řečí Svatého, Boha Izraele, pohrdli.
#5:25 Proto plane Hospodin hněvem proti svému lidu; až napřáhne na něj ruku a udeří jej, budou se chvět hory, mrtvol bude jako smetí na ulicích. Tím vším se jeho hněv neodvrátí a jeho ruka zůstane napřažená.
#5:26 I vztyčí korouhev k dalekému pronárodu, hvizdem jej přivolá od končin země. Hle, přijde rychle a hbitě.
#5:27 Nikdo v něm nebude znavený, nikdo nebude klopýtat, nikdo nezdřímne, nikdo nebude spát, nikomu se nerozváže pás na bedrech, nikomu se nepřetrhne řemínek u opánků.
#5:28 Jeho střely budou naostřené, všechny jeho luky připravené, kopyta jeho koní budou jako křemen a kola jeho vozů jako vichr.
#5:29 Bude řvát jako lvice, řvát bude jako lvíčata, zavrčí, popadne úlovek, odvleče a nevyrve mu jej nikdo.
#5:30 V onen den se nad ním bude rozléhat jekot jakoby jekot moře. Pohlédne-li kdo na zemi, hle, skličující temnota; světlo ztemnělo jejím šerem. 
#6:1 Toho roku, kdy zemřel král Uzijáš, spatřil jsem Panovníka. Seděl na vysokém a vznosném trůnu a lem jeho roucha naplňoval chrám.
#6:2 Nad ním stáli serafové: každý z nich měl po šesti křídlech, dvěma si zastíral tvář, dvěma si zakrýval nohy a dvěma se nadnášel.
#6:3 Volali jeden k druhému: „Svatý, svatý, svatý je Hospodin zástupů, celá země je plná jeho slávy.“
#6:4 Od hlasu volajícího se pohnuly podvaly prahů a dům se naplnil dýmem.
#6:5 I řekl jsem: „Běda mi, jsem ztracen. Jsem člověk nečistých rtů a mezi lidem nečistých rtů bydlím, a spatřil jsem na vlastní oči Krále, Hospodina zástupů.“
#6:6 Tu ke mně přiletěl jeden ze serafů. V ruce měl žhavý uhlík, který vzal kleštěmi z oltáře,
#6:7 dotkl se mých úst a řekl: „Hle, toto se dotklo tvých rtů, tvá vina je odňata a tvůj hřích je usmířen.“
#6:8 Vtom jsem uslyšel hlas Panovníka: „Koho pošlu a kdo nám půjde?“ I řekl jsem: „Hle, zde jsem, pošli mne!“
#6:9 Odpověděl: „Jdi a řekni tomuto lidu: ‚Poslouchejte a poslouchejte, nic nepochopíte, dívejte se, dívejte, nic nepoznáte.‘
#6:10 Srdce toho lidu obal tukem, zacpi mu uši, zalep mu oči, aby očima neviděl, ušima neslyšel, srdcem nepochopil, neobrátil se a nebyl uzdraven.“
#6:11 Tu jsem řekl: „Dlouho ještě, Panovníku?“ On odpověděl: „Dokud nezpustnou města a nebudou bez obyvatel a domy bez lidí a role se nepromění v spoušť.“
#6:12 Hospodin odvede ty lidi daleko a v zemi bude mnoho míst opuštěno.
#6:13 Zůstane-li v ní jen desetina, i ta bude zajata a přijde vniveč jako posvátný strom, jako dub, po jehož skácení zbude jen pahýl. Tento pahýl bude símě svaté. 
#7:1 Za dnů Achaza, syna Jótama, syna Uzijášova, krále judského, vytáhl Resín, král aramejský, a s ním Pekach, syn Remaljášův, král izraelský, do války proti Jeruzalému, ale nic proti němu v boji nesvedli.
#7:2 Domu Davidovu bylo oznámeno: „Aram táboří v Efrajimsku.“ I zachvělo se srdce královo i srdce jeho lidu, jako se chvějí lesní stromy ve větru.
#7:3 Hospodin řekl Izajášovi: „Vyjdi Achazovi naproti ty a tvůj syn Šearjašúb na konec strouhy z Horního rybníka, na silnici k Poli valchářovu.
#7:4 Řekni mu: Zachovej klid a neboj se, neklesej na mysli kvůli těm dvěma čadícím oharkům, že hněvem plane Aramejec Resín a syn Remaljášův.
#7:5 Nedbej, že Aram i Efrajim a syn Remaljášův se dohodli proti tobě na zlé věci:
#7:6 ‚Vytáhneme proti Judovi, vyděsíme jej a dobudeme jej pro sebe a uděláme v něm králem Tabealova syna.‘“
#7:7 Toto praví Panovník Hospodin: „Nedojde k tomu a nestane se to.
#7:8 Hlavou Aramu je Damašek a hlavou Damašku Resín. Do šedesáti pěti let ztroskotá Efrajim, takže nebude lidem;
#7:9 hlavou Efrajimu je Samaří a hlavou Samaří syn Remaljášův. Nebudete-li stálí ve víře, neobstojíte!“
#7:10 Hospodin promluvil znovu k Achazovi:
#7:11 „Vyžádej si znamení od Hospodina, svého Boha, buď dole z hlubin nebo nahoře z výšin.“
#7:12 Achaz odpověděl: „Nechci žádat a nebudu pokoušet Hospodina.“
#7:13 I řekl Izajáš: „Slyšte, dome Davidův! Což je vám málo zkoušet trpělivost lidí, že chcete zkoušet i trpělivost mého Boha?
#7:14 Proto vám dá znamení sám Panovník: Hle, dívka počne a porodí syna a dá mu jméno Immanuel (to je S námi Bůh).
#7:15 Bude jíst smetanu a med, aby dovedl zavrhnout zlé a volit dobré.
#7:16 Ještě než bude chlapec umět zavrhnout zlé a volit dobré, bude opuštěna země, z jejíchž obou králů máš hrůzu.
#7:17 Na tebe, na tvůj lid a na dům tvého otce Hospodin v těch dnech, jakých nebylo ode dne, kdy Efrajim odpadl od Judy, přivede asyrského krále.“
#7:18 V onen den hvizdem přivolá Hospodin mouchy z delty řeky egyptské a včely z asyrské země.
#7:19 Přiletí a všechny se snesou do roklí v úvalech, do skalních rozsedlin, na všechny houštiny a na všechny mokřiny.
#7:20 V onen den Panovník břitvou, kterou si najme za Řekou, totiž asyrským králem, dočista oholí každou hlavu i chlupy na nohou a ostříhá bradu.
#7:21 V onen den zůstane člověku jedna kravka a dvě ovce.
#7:22 Nadojí však tolik mléka, že bude jíst smetanu, neboť smetanu a med bude jíst každý, kdo bude zanechán v zemi.
#7:23 V onen den na místech, kde nyní vyrůstá tisíc révových keřů za tisíc šekelů stříbra, bude jen bodláčí a křoví.
#7:24 Jen s šípy a lukem bude možno tudy projít, neboť bodláčím a křovím zaroste celá země.
#7:25 A na žádnou z horských strání, které byly obdělávány motykou, nebudeš moci vstoupit pro samé bodláčí a křoví. Budou výběhem pro býky a drahami pro ovce. 
#8:1 Hospodin mi řekl: „Vezmi si velkou tabulku a napiš na ni běžným způsobem: ‚Rychle za kořistí spěchá lupič‘.“
#8:2 I vzal jsem si spolehlivé svědky: kněze Urijáše a Zekarjáše, syna Jeberekjášova.
#8:3 Přiblížil jsem se k prorokyni, a ona počala a porodila syna. Hospodin mi řekl: „Dej mu jméno: ‚Rychle za kořistí spěchá lupič‘.
#8:4 Neboť dříve než bude chlapec umět volat ‚otče‘ a ‚matko‘, bude odneseno bohatství Damašku a kořist ze Samaří před krále asyrského.“
#8:5 A dál ještě mluvil Hospodin ke mně takto:
#8:6 „Protože tento lid zavrhl šíloašské vody poklidně tekoucí a veselí se s Resínem a synem Remaljášovým,
#8:7 hle, proto Panovník na ně uvede vody Řeky, dravé a mnohé, krále asyrského s celou jeho slávou. I vystoupí ze všech svých řečišť a bude se valit přes všechny své břehy.
#8:8 Zabočí k Judovi, zaplaví jej a bude se valit dál, bude mu sahat až k hrdlu. Jeho rozpjatá křídla vyplní prostor tvé země, ó Immanueli!“
#8:9 Běsněte si, národy, zděsíte se, naslouchejte, všechny daleké země, přepásejte se, zděsíte se, přepásejte se, zděsíte se.
#8:10 Dohodněte se, dohoda bude zmařena, mluvte si, ani slovo neobstojí, neboť s námi je Bůh - Immanuel!
#8:11 Toto mi praví Hospodin, když mě pevně uchopil svou rukou a varoval mě, abych nechodil cestou tohoto lidu:
#8:12 „Neříkejte zrada všemu, čemu říká zrada tento lid, a čeho se bojí, toho vy se nebojte a nestrachujte.“
#8:13 Dosvědčujte svatost Hospodina zástupů! Jeho se bojte a strachujte.
#8:14 Vám bude svatyní, ale oběma domům izraelským kamenem úrazu a skálou, o kterou budou klopýtat, bude osidlem a léčkou obyvatelům Jeruzaléma.
#8:15 Mnozí z nich klopýtnou, padnou a roztříští se, uvíznou v léčce a chytí se.
#8:16 „Zavaž svědectví a zapečeť zákon mezi mými učedníky.“
#8:17 Budu očekávat Hospodina, ačkoli skryl tvář před domem Jákobovým, s nadějí budu na něho čekat.
#8:18 Hle, já a děti, které mi dal Hospodin, jsme v Izraeli znameními a zázraky od Hospodina zástupů, který přebývá na hoře Sijónu.
#8:19 Řeknou vám: „Dotazujte se duchů zemřelých a jasnovidců, kteří sípají a mumlají.“ Což se lid nemá dotazovat svého Boha? Na živé se má ptát mrtvých?
#8:20 K zákonu a svědectví! Což oni neříkají takové slovo, že mu z něho nevzejde jitřní záře?
#8:21 Lid bude procházet zemí zatvrzelý a hladový. A protože bude mít hlad, rozlítí se a bude zlořečit svému králi i Bohu s tváří pozvednutou vzhůru.
#8:22 Podívá se k zemi, a hle, jen soužení a temnota, skličující ponurost, šerý soumrak, do něhož bude zahnán.
#8:23 Avšak tato sklíčená země ponurá nezůstane. Jako zprvu byla země Zabulón a země Neftalí zlehčena, tak nakonec bude přivedena ke cti s Přímořím a Zajordáním i Galileou pronárodů. 
#9:1 Lid, který chodí v temnotách, uvidí velké světlo; nad těmi, kdo sídlí v zemi šeré smrti, zazáří světlo.
#9:2 Rozmnožil jsi národ, rozhojnil jsi jeho radost; budou se před tebou radovat, jako se radují ve žních, tak jako jásají ti, kdo se dělí o kořist.
#9:3 Neboť jho jeho břemene a hůl na jeho záda i prut jeho poháněče zlomíš jako v den Midjánu.
#9:4 Pak každá bota obouvaná do válečné vřavy a každý plášť vyválený v prolité krvi budou k spálení, budou potravou ohně.
#9:5 Neboť se nám narodí dítě, bude nám dán syn, na jehož rameni spočine vláda a bude mu dáno jméno: „Divuplný rádce, Božský bohatýr, Otec věčnosti, Vládce pokoje.“
#9:6 Jeho vladařství se rozšíří a pokoj bez konce spočine na trůně Davidově a na jeho království. Upevní a podepře je právem a spravedlností od toho času až navěky. Horlivost Hospodina zástupů to učiní.
#9:7 Slovo poslal Panovník proti Jákobovi a dopadlo na Izraele.
#9:8 Všechen lid o tom zvěděl, Efrajim i obyvatelé Samaří. V pýše a velikášství srdce však říkají:
#9:9 „Co bylo z cihel, spadlo, vystavíme to z kvádrů; smokvoně byly vykáceny, nahradíme je cedry.“
#9:10 I pozvedl Hospodin proti nim protivníky - Resína - a popudil proti nim jejich nepřátele:
#9:11 z východu Arama, ze západu Pelištejce, aby plnými ústy požírali Izraele. Tím vším se jeho hněv neodvrátil a jeho ruka zůstává napřažena.
#9:12 Avšak lid se neobrátil k tomu, který ho bil, Hospodina zástupů se nedotazovali.
#9:13 Proto Hospodin odsekl od Izraele hlavu i ocas, palmovou ratolest i sítinu v jediný den.
#9:14 Stařec a vznešený jsou ona hlava a prorok učící klam je onen ocas.
#9:15 Vůdcové tohoto lidu se stali svůdci a ti, kdo se dají vést, jsou ztraceni.
#9:16 Proto neměl Panovník radost z jeho jinochů a nad jeho sirotky a vdovami se neslitoval, protože všichni jsou to rouhači a zlovolníci, bludně mluví každá ústa. Tím vším se jeho hněv neodvrátil a jeho ruka zůstává napřažena.
#9:17 Avšak jejich svévole hoří jako oheň, požírá bodláčí a křoví, zapaluje houštiny lesa a kouř se vířivě vznáší.
#9:18 Země se zatměla pro prchlivost Hospodina zástupů a lid se stal potravou ohně; nikdo nemá soucit ani s bratrem.
#9:19 Hltá, co je napravo, a zůstává hladový, požírá nalevo, a sytý není, požírá maso své vlastní paže,
#9:20 Manases Efrajima, Efrajim Manasesa a na Judu se vrhají společně. Tím vším se jeho hněv neodvrátil a jeho ruka zůstává napřažena. 
#10:1 Běda těm, kdo nařizují ničemná nařízení těm, kdo předpisují plahočení,
#10:2 nuzným odnímají možnost obhajoby, utištěné mého lidu zbavují práva; jejich kořistí jsou vdovy a sirotky olupují.
#10:3 Co si počnete v den navštívení, až se z dálky přižene zkáza? Ke komu se utečete o pomoc, kde zanecháte svoji slávu?
#10:4 Nezbude než sklonit se mezi vězně či zůstat ležet mezi zabitými. Tím vším se jeho hněv neodvrátil a jeho ruka zůstává napřažena.
#10:5 „Běda Asýrii, metle mého hněvu, té holi, skrze kterou projevím svůj hrozný hněv.
#10:6 Posílám ji na rouhavý pronárod, proti lidu, proti němuž plane má prchlivost; dávám jí příkaz, aby kořistila a kořistila, loupila a loupila, aby jej šlapala jako hlínu na ulicích.
#10:7 Ona si to však představit nedovede, neuvažuje o tom v svém srdci, myslí jen na to, jak by vyhubila, jak by vymýtila nemálo pronárodů.
#10:8 Říká: ‚Což nejsou mí velitelé zároveň králi?
#10:9 Což nedopadlo Kalno jako Karkemíš? Nebo Chamát jako Arpád a Samaří jako Damašek?‘
#10:10 Jako sáhla moje ruka po královstvích bůžků - a jejich tesané modly byly nad jeruzalémské i nad samařské -,
#10:11 cožpak s Jeruzalémem a jeho modlářskými stvůrami nenaložím, jako jsem naložil se Samařím a jeho bůžky?“
#10:12 Až Panovník přivede ke konci celé své dílo na hoře Sijónu a v Jeruzalémě, řekne: „Ztrestám asyrského krále pro ovoce jeho velikášského srdce, i to, co je v jeho povýšených očích proslavené,
#10:13 neboť řekl: ‚Vykonal jsem to silou své ruky a svou moudrostí - vždyť se v tom vyznám. Zrušil jsem hranice národů, co nakupily, to jsem vydrancoval a jako býk jsem svrhl vládce z trůnu.
#10:14 Jako hnízdo nalezla má ruka bohatství národů; jako se vybírají opuštěná vejce, tak jsem vybral celou zemi. Nikdo ani křídlem nezamával, nikdo zobák neotevřel a nepípl.‘
#10:15 Což se smí sekera holedbat nad toho, kdo jí seká? A pila povyšovat se nad toho, kdo jí řeže? Jako by metla mohla švihat toho, kdo ji zvedá, nebo hůl se vynášet, že není dřevo.“
#10:16 Proto pošle Pán, Hospodin zástupů, na asyrské vypasence úbytě a pod asyrskou slávou to zapraská praskotem ohně.
#10:17 Světlo Izraelovo se stane ohněm a jeho Svatý plamenem. Ten bude hořet a pozře její křoví a bodláčí v jediný den.
#10:18 I slávu jejího lesa a sadu stráví se vším všudy, jako choroba užírá chorého.
#10:19 Stromů v jejím lese zůstane tak málo, že je spočítá i chlapec.
#10:20 V onen den se pozůstatek Izraele s uprchlíky domu Jákobova nebude už opírat o toho, který ho bije, ale bude se věrně opírat o Hospodina, Svatého, Boha Izraele.
#10:21 Pozůstatek se vrátí, pozůstatek Jákoba, k Bohu - bohatýru.
#10:22 Neboť i kdyby bylo tvého lidu, Izraeli, jak mořského písku, vrátí se k němu jen pozůstatek. Je rozhodnut skoncovat, jako záplava se valí spravedlnost.
#10:23 Ano, Panovník, Hospodin zástupů, se rozhodl učinit konec, a to po celé zemi.
#10:24 Proto Panovník, Hospodin zástupů, praví toto: „Můj lide, který bydlíš na Sijónu, neboj se Asýrie. Bije tě metlou a zvedá na tebe hůl jako kdysi Egypt.
#10:25 Ještě jen kratičký čas a můj hrozný hněv skončí; je však můj hněv zničí.“
#10:26 Hospodin zástupů nad ní zamává bičem, jako když bil Midjána na Havraní skále, a zdvihne svou hůl na moře jako kdysi na Egypt.
#10:27 V onen den ti spadne z ramen její břímě a její jho z šíje, jho se z tebe sesmekne jak po oleji.
#10:28 Už táhne na Aját, prošel Migrónem a zbroj si v Michmásu složil.
#10:29 Prošli průsmykem. Řekli: „Přenocujeme v Gebě.“ Ráma se třásla, Saulova Gibea se dala na útěk.
#10:30 Pronikavě křič, galímská dcero, Lajšo, napjatě poslouchej, odpověz, Anatóte.
#10:31 V Madméně je poplach, gebímští obyvatelé hledají záštitu.
#10:32 Ještě dnes stane v Nóbu a rukou mávne proti hoře sijónské dcery, proti pahorku jeruzalémskému.
#10:33 Hle, Pán, Hospodin zástupů, hrozivou silou klestí větvoví, pyšně se tyčící stromy jsou porubány, i ty nejvyšší jsou poníženy.
#10:34 Železem bude vysekáno lesní houští, Libanón padne rukou Vznešeného. 
#11:1 I vzejde proutek z pařezu Jišajova a výhonek z jeho kořenů vydá ovoce.
#11:2 Na něm spočine duch Hospodinův: duch moudrosti a rozumnosti, duch rady a bohatýrské síly, duch poznání a bázně Hospodinovy.
#11:3 Bázní Hospodinovou bude prodchnut. Nebude soudit podle toho, co vidí oči, nebude rozhodovat podle toho, co slyší uši,
#11:4 nýbrž bude soudit nuzné spravedlivě, o pokorných v zemi bude rozhodovat podle práva. Žezlem svých úst bude bít zemi, dechem svých rtů usmrtí svévolníka.
#11:5 Jeho bedra budou opásána spravedlností a jeho boky přepásá věrnost.
#11:6 Vlk bude pobývat s beránkem, levhart s kůzletem odpočívat. Tele a lvíče i žírný dobytek budou spolu a malý hoch je bude vodit.
#11:7 Kráva se bude popásat s medvědicí, jejich mláďata budou odpočívat spolu, lev jako dobytče bude žrát slámu.
#11:8 Kojenec si bude hrát nad děrou zmije, bazilišku do doupěte sáhne ručkou odstavené dítě.
#11:9 Nikdo už nebude páchat zlo a šířit zkázu na celé mé svaté hoře, neboť zemi naplní poznání Hospodina, jako vody pokrývají moře.
#11:10 V onen den budou pronárody vyhledávat kořen Jišajův, vztyčený jako korouhev národům, a místo jeho odpočinutí bude slavné.
#11:11 V onen den ještě podruhé vztáhne Panovník ruku, aby získal pozůstatek svého lidu, který zůstal v Asýrii, Egyptě, Patrósu, Kúši, Élamu, Šineáru, Chamátu a na ostrovech mořských.
#11:12 Povznese korouhev k pronárodům a posbírá rozehnané z Izraele a rozptýlené Judejce shromáždí ze čtyř stran země.
#11:13 Ustane žárlivost Efrajimova, budou vyhlazeni Judovi protivníci, Efrajim nebude žárlit na Judu a Juda nebude osočovat Efrajima,
#11:14 nýbrž společně se vrhnou na úbočí Pelištejců k moři, i syny východu oloupí spolu, vztáhnou svoji ruku na Edóm a Moáb, poslouchat je budou i synové Amónovi.
#11:15 Klatbou stihne Hospodin záliv Egyptského moře, zamává rukou proti Řece v prudkém větru, rozštěpí ji v sedm ramen, takže ji bude možno přejít v opáncích.
#11:16 Tak vznikne silnice pro pozůstatek jeho lidu, jenž zůstal v Asýrii, jako tomu bylo v den, kdy Izrael vystupoval z egyptské země. 
#12:1 V onen den řekneš: „Vzdávám tobě chválu, Hospodine! Rozhněval ses na mě, tvůj hněv se však odvrátil a potěšils mě.
#12:2 Hle, Bůh je má spása, doufám a jsem beze strachu. Hospodin, jen Hospodin je má záštita a píseň, stal se mou spásou.“
#12:3 S veselím budete čerpat vodu z pramenů spásy.
#12:4 V onen den řeknete: „Chválu vzdejte Hospodinu a vzývejte jeho jméno! Uvádějte národům ve známost jeho skutky. Připomínejte, že jeho jméno je vyvýšené.
#12:5 Zpívejte Hospodinu, neboť vykonal důstojné činy, ať o tom zví celá země!“
#12:6 Jásej a plesej, ty, která bydlíš na Sijónu, neboť Veliký je ve tvém středu - Svatý Izraele. 
#13:1 Výnos o Babylónu, který přijal ve vidění Izajáš, syn Amósův:
#13:2 „Vztyčte korouhev na lysé hoře, vzkřikněte na ně, zamávejte rukou, ať vejdou branami urozených.
#13:3 Vydal jsem příkaz svým posvěceným, povolal jsem k vykonání svého hněvu i své bohatýry, jásající nad mou důstojností.
#13:4 Slyš hluk na horách jakoby početného lidu, slyš hukot království, shromážděných pronárodů: Hospodin zástupů sbírá vojsko k boji.
#13:5 Z daleké země přicházejí, až od končin nebes, přichází Hospodin se zbrojí hrozného hněvu, aby pohubil celou zemi.
#13:6 Kvilte, blízko je den Hospodinův, přijde od Všemocného jak zhouba.
#13:7 Proto každá ruka ochabne a každý člověk odvahu ztratí.
#13:8 Budou plni hrůzy, zachvátí je bolesti a křeče, jako rodička se budou svíjet. Strnule bude zírat jeden na druhého, ve tváři plamenem vzplanou.
#13:9 Hle, přichází den Hospodinův, nelítostný, s prchlivostí a planoucím hněvem, aby zemi změnil v spoušť a vymýtil z ní její hříchy.
#13:10 Nebeské hvězdy a souhvězdí se nezatřpytí svým světlem, slunce se při svém východu zatmí, měsíc svým světlem nezazáří.
#13:11 A budu stíhat zlobu světa a nepravost svévolníků. Učiním přítrž pýše opovážlivců, ponížím troufalost ukrutníků.
#13:12 Způsobím, že člověk bude vzácnější než ryzí zlato, člověk bude nad zlato z Ofíru.
#13:13 Otřesu totiž nebesy a země se pohne ze svého místa prchlivostí Hospodina zástupů v den jeho planoucího hněvu.
#13:14 Jako vyplašená laň a jako stádo, jež neshromažďuje nikdo, tak se obrátí každý ke svému lidu, každý uteče do své země.
#13:15 Každý, koho najdou, bude skolen, každý, koho chytí, padne mečem.
#13:16 Jejich nemluvňata budou rozdrcena před jejich zraky, jejich domy budou vypleněny, jejich ženy zneuctěny.
#13:17 Hle, já podnítím proti vám Médy, kteří si neváží stříbra, nemají zálibu v zlatě.
#13:18 Svými luky rozdrtí chlapce, nad plodem života se neslitují, s vašimi syny nebudou mít soucit.
#13:19 Babylón, skvost mezi královstvími, pyšná okrasa Kaldejců, dopadne jako Sodoma a Gomora, vyvrácené Bohem.
#13:20 Už nikdy nebude osídlen, nebude obydlen po všechna pokolení. Žádný Arab si tam nepostaví stan, pastýři tam nenechají odpočívat stáda,
#13:21 nýbrž divá sběř tam bude odpočívat, v jejich domech bude plno výrů, přebývat tam budou pštrosi, běsové tam budou poskakovat.
#13:22 V jeho palácích budou výt hyeny a v chrámech rozkoše šakalové. Jeho čas se přiblížil, dny nebudou mu prodlouženy.“ 
#14:1 Hospodin se slituje nad Jákobem a znovu vyvolí Izraele. Dá jim odpočinout v jejich zemi, přidruží se k nim i bezdomovec a připojí se k domu Jákobovu.
#14:2 Národy je totiž samy dovedou na jejich místo a Izraelův dům je dostane do dědictví v Hospodinově zemi jako otroky a otrokyně. Tak zajmou ty, kdo je prve zajali, a budou panovat nad svými poháněči.
#14:3 V den, kdy ti dá Hospodin odpočinout od tvého trápení a nepokoje, od tvrdé otročiny, jíž jsi byl zotročen,
#14:4 proneseš o babylónském králi pořekadlo: „Jak pominul poháněč, jak pominula knuta!“
#14:5 Hospodin polámal hůl svévolníků, žezlo vládců,
#14:6 jež zběsile bilo národy ranami bez ustání, hněvivě vládlo pronárodům nezkrotným pronásledováním.
#14:7 V míru odpočala celá země, všechno zvučně plesá.
#14:8 I cypřiše se radují nad tebou i libanónské cedry od chvíle, kdy jsi padl: „Už nevstane ten, kdo nás kácel!“
#14:9 I nejhlubší podsvětí je kvůli tobě rozrušeno, čeká na tvůj příchod, probudilo kvůli tobě přízraky, všech vojevůdců země. Všechny krále pronárodů přimělo vstát ze svých trůnů.
#14:10 Ti všichni se ozvou a řeknou tobě: „Také jsi jako my pozbyl síly? Už jsi nám podoben!
#14:11 Do podsvětí byla svržena tvá pýcha, hlučný zvuk tvých harf. Máš ustláno na hnilobě, přikrývku máš z červů.“
#14:12 Jak jsi spadl z nebe, třpytivá hvězdo, jitřenky synu! Jak jsi sražen k zemi, zotročovateli pronárodů!
#14:13 A v srdci sis říkal: „Vystoupím na nebesa, vyvýším svůj trůn nad Boží hvězdy, zasednu na Hoře setkávání na nejzazším Severu.
#14:14 Vystoupím na posvátná návrší oblaků, s Nejvyšším se budu měřit.“
#14:15 Teď jsi svržen do podsvětí, do nejhlubší jámy!
#14:16 Kdo tě uvidí, budou se dívat a divit: „To je ten muž, který zneklidňoval zemi a otřásal královstvími?
#14:17 Svět měnil v poušť, bořil jeho města, vězně nikdy nepropouštěl domů.“
#14:18 Všichni králové pronárodů, všichni leží slavně ve svých hrobkách,
#14:19 ale tebe pohodili mimo hrob jak ohavný výhonek; jsi jako šatem přikryt zavražděnými porubanými mečem, svrženými do kamenité jámy, jsi jak pošlapaná mršina,
#14:20 nebudeš připojen k nim v hrobě. Vždyť jsi ničil vlastní zemi, vlastní lid jsi vraždil. O potomstvu zlovolníků nebude navěky zmínky.
#14:21 „Připravte jeho synům jatky pro nepravost otců! Nepovstanou, nezmocní se země a svět nezaplní městy.“
#14:22 „Já povstanu proti nim, je výrok Hospodina zástupů, vymýtím jméno Babylóna i pozůstatek po něm, nástupce i následníka, je výrok Hospodinův.
#14:23 Proměním jej v hnízdiště sýčků a ve slatiny, vymetu jej pometlem zkázy, je výrok Hospodina zástupů.“
#14:24 Hospodin zástupů přisáhl: „Jak jsem si předsevzal, tak se stane, a pro co jsem se rozhodl, to se uskuteční:
#14:25 Rozdrtím Ašúra ve své zemi, rozšlapu ho na svých horách a z nich sejmu jeho jho. Z ramene jim bude sňato jeho břímě.“
#14:26 To je rozhodnutí učiněné o celé zemi a to je paže napřažená proti všem pronárodům.
#14:27 Když Hospodin zástupů rozhodl, kdo to zruší? Jeho paže je napřažena, kdo ji odvrátí?
#14:28 V roce, kdy zemřel král Achaz, byl zjeven tento výnos:
#14:29 „Neraduj se, celá Pelišteo, že je polámána hůl, která tě bije, neboť z hadího kořene vzejde bazilišek a jeho plodem bude létající ohnivý had.
#14:30 Prvorození nuzných budou mít pastvu a ubožáci budou odpočívat v bezpečí. Tvůj kořen však umořím hladem a pozůstatek tvého lidu zajde.
#14:31 Kvílej, bráno! Křič, město! Celá Pelištea bude v zmatku. Neboť od severu se už valí dým a nikdo nezůstává stranou svých shromaždišť.“
#14:32 Co odpovědět poslům toho pronároda? Hospodin založil Sijón, k němu se přivinou utištění jeho lidu. 
#15:1 Výnos o Moábovi: Přes noc zajisté bude vyhuben Ar Moábský, zajde! Přes noc zajisté bude vyhuben Kír Moábský, zajde!
#15:2 Vystoupí v Díbónu do domu, na posvátná návrší, aby plakal. Nad Nebó, nad Médebou bude Moáb kvílet. Na každé hlavě lysina, každá brada ostříhána.
#15:3 V jeho ulicích se opasují žíněnými suknicemi, na jeho střechách a náměstích vše kvílí, rozplývá se v pláči.
#15:4 Úpí Chešbón i Eleále, jejich hlas až v Jahasu je slyšet. Proto křičí na poplach moábští ozbrojenci, jejich duše je poplašená.
#15:5 Moje srdce úpí nad Moábem. Jeho uprchlíci jsou až u Soáru, v Eglat-šelišiji. Po lúchítském svahu vystupují s pláčem. Na cestě do Chóronajimu propukají v úpění nad zkázou.
#15:6 Po nimrímských vodách zbude zpustošení, uschne tráva, bude po rostlinách, nezbude nic zeleného.
#15:7 Proto zbytek toho, na čem pracovali, co si uchovali, přenesou přes Topolový úval.
#15:8 Úpění obchází územím Moába, jeho kvílení zní až do Eglajimu, jeho kvílení zní až do Beer-élímu.
#15:9 Vždyť dímónské vody jsou plné krve. Proto ještě více dopustím na Dímón: lva na Moábce, kteří vyvázli, i na ty, kteří zůstali v zemi. 
#16:1 Pošlete beránka Vládci země ze Sély pouští na horu sijónské dcery.
#16:2 I budou jako vyplašení ptáci, vyhnaní z hnízda, moábské dcery při arnónských brodech.
#16:3 „Svolejte radu, učiňte rozhodnutí. V samé poledne ať je tvůj stín jak noc, ukryj zahnané, neprozraď vyplašené;
#16:4 Moábe, ať moji zahnaní jsou tvými hosty, před zhoubcem buď jejich skrýší. Utlačovatel, ten vezme za své, zhouba skončí. Ze země vymizejí ti, kdo ji podupali.
#16:5 Trůn bude upevněn milosrdenstvím a dosedne na něj v Davidově stanu ten, jenž bude soudit věrně, vyhledávat právo, rázně uplatňovat spravedlnost.“
#16:6 Slýchali jsme o pýše Moába přepyšného, o pyšné jeho povýšenosti a bezmezné zpupnosti; k ničemu nejsou jeho žvásty.
#16:7 Proto bude Moáb kvílet nad Moábem, všichni budou kvílet. Touhou po kírcharesetských hrozinkových koláčích budou vzdychat úplně zdeptáni.
#16:8 Vždyť zvadly viničné terasy Chešbónu, sibemská vinná réva. Páni pronárodů otloukli to ušlechtilé révoví, jež dosahovalo až k Jaezeru, vinulo se až k poušti, jeho výhonky se rozbujely a pronikly za moře.
#16:9 Proto pláči s plačícím Jaezerem, nad sibemskou vinnou révou. Svlažuji tě svými slzami, Chešbóne a Eleále, neboť při tvé letní sklizni a tvém vinobraní přestane vaše výskání.
#16:10 Přestaly v sadu radost a jásot, na vinicích se už neplesá a nehlaholí, nikdo nelisuje v lisu víno, tvému výskání jsem učinil přítrž.
#16:11 Proto všechno ve mně nad Moábem jak citara sténá, mé nitro lká nad Kír-cheresem.
#16:12 Až se Moáb ukáže před svými bohy a na posvátné návrší se dotrmácí, až vstoupí do své svatyně, aby se modlil, ničeho nedosáhne.
#16:13 Toto je slovo, které už tehdy Hospodin promluvil o Moábovi.
#16:14 Nyní Hospodin promluvil: „Za tři léta, jako jsou léta nádeníka, zlehčena bude sláva Moábova u celého jeho početného davu; zůstane mu malá, bezvýznamná hrstka.“ 
#17:1 Výnos o Damašku: „Hle, Damašek přestane být městem, stane se hromadou sutin.
#17:2 Opuštěná města Aróeru budou patřit stádům; tam budou odpočívat a nikdo je nevyděsí.
#17:3 Zanikne pevnost v Efrajimu i království v Damašku. S pozůstatkem Aramejců to padne jako se slávou Izraelců, je výrok Hospodina zástupů.“
#17:4 „V onen den vybledne Jákobova sláva a jeho tučné tělo zhubne.
#17:5 Bude tomu, jako když se při žni sklízí obilí, když se do náručí shromažďují klasy, jako když se klasy posbírají v dolině Refájců.
#17:6 Zbudou pak jen paběrky jako při srážení oliv, dvě tři olivky na samém vršku a čtyři pět na plodných větvích, je výrok Hospodina, Boha Izraele.“
#17:7 V onen den vzhlédne člověk k tomu, který ho učinil, a jeho oči budou patřit k Svatému Izraele.
#17:8 Už nebude vzhlížet k oltářům, dílu vlastních rukou, ani nebude patřit na to, co vyrobil svými prsty, na posvátné kůly a kadidlové oltáříky.
#17:9 V onen den budou jeho záštitná města jako opuštěné a houštím zarostlé vrcholky, jež pronárody opustily před syny Izraele; zpustnou.
#17:10 Vždyť jsi zapomněla na Boha, svou spásu, nepamatovala jsi na svoji záštitnou skálu. Proto vysazuješ sadbu rozkoší a pěstuješ cizokrajnou révu.
#17:11 V den, kdy je sázíš, oplocuješ je, za jitra pečuješ, aby tvá setba vyrašila. Ale žeň prchne v den nemoci a bolesti nevyléčitelné.
#17:12 Běda! Hluk četných čeledí lidských hlučí jak hlučící moře, hukot národů hučí podoben hukotu dravých vod.
#17:13 Národy hučí jak hukot mnohých vod, leč okřikne je a utekou daleko, hnány jako plevy větrem po horách a jako chmýří ve vichřici.
#17:14 Z večera ještě plno hrůzy, a než vzejde jitro, je po všem. Takový je úděl těch, kteří nás plení, los těch, kdo nás olupují. 
#18:1 Běda zemi okřídleného hmyzu při řekách Kúše;
#18:2 posílá vyslance po moři, po vodní hladině v plavidlech šáchorových. Jděte, poslové hbití, k urostlému pronárodu s lesklou pletí, k lidu daleko široko obávanému, k pronárodu nesrozumitelné řeči, který poražené pošlapává, jehož zemi protékají řeky!
#18:3 Všichni obyvatelé světa, vy, kdo přebýváte na zemi, až se vyzdvihne na horách korouhev, uvidíte, až se zatroubí na polnici, uslyšíte.
#18:4 Neboť Hospodin mi praví toto: „Klidně budu přihlížet ze svého stanoviště, budu jak sálavé horko v záři slunce, jako rosný oblak v horku žní.
#18:5 Přede žněmi, když odkvete réva a květenství se promění ve zrající hrozen, tu se jalové výhonky odřežou vinařským nožem a odnože se odstraňují, vylamují.
#18:6 Vesměs budou zanechány dravému ptactvu hor a zvířectvu země, takže dravé ptactvo na nich stráví léto a veškeré zvířectvo země na nich bude zimovat.“
#18:7 V onen čas bude přinášena Hospodinu zástupů pocta od urostlého lidu s lesklou pletí, od lidu široko daleko obávaného, od pronároda nesrozumitelné řeči, který poražené pošlapává, jehož zemi protékají řeky. Přinese ji na místo, kde přebývá jméno Hospodina zástupů, na horu Sijón. 
#19:1 Výnos o Egyptu: Hle, Hospodin jede na rychlém oblaku a vjíždí do Egypta. Egyptské modly se před ním kymácejí, Egypt svou odvahu nadobro ztratil.
#19:2 „Popudím Egypťany proti Egypťanům a bude bojovat bratr proti bratru a přítel proti příteli, město proti městu, království proti království.
#19:3 Egypt bude na duchu zlomen, jeho záměry zmatu, budou se dotazovat model a zaklínačů i duchů zemřelých a jasnovidců.
#19:4 Vydám Egypt v moc tvrdých pánů a mocný král bude nad nimi vládnout“, je výrok Pána, Hospodina zástupů.
#19:5 Vody se vypaří, než dojdou k moři, a řeka vyschne a vysuší se,
#19:6 říční ramena budou páchnout, voda v egyptských kanálech opadne a vyschne, třtina a rákos zvadnou.
#19:7 Lučiny při Nilu, při ústí Nilu, a všechna setba u Nilu uschne, bude odváta a nebude už.
#19:8 Tu se budou rybáři rmoutit a truchlit, všichni, kdo vrhají udici do Nilu; ochabnou, kdo na vodní hladině rozestírají sítě.
#19:9 Zklamáni budou pěstitelé lnu, česačky i tkalci bílého plátna.
#19:10 Podpěry země budou vyvráceny, bude smutno v duši všem, kdo pracují za mzdu.
#19:11 Jací pošetilci jsou sóanští velmožové! Moudří rádcové faraónovi, vaše rada je hloupá. Jak můžete říci faraónovi: „Jsem syn moudrých, syn dávných králů?“
#19:12 Kdepak jsou tví moudří? Ať ti to oznámí, vědí-li, jaký je úradek Hospodina zástupů o Egyptu.
#19:13 Pošetile si počínají sóanští velmožové, velmožové memfidští propadli klamu. Zavedli Egypt na scestí ti, kdo byli pilířem jeho kmenů,
#19:14 Hospodin namíchal do jeho nitra ducha závrati, takže zavedli Egypt na scestí ve všem jeho počínání, jako když opilec vrávorá ve vlastních zvratcích.
#19:15 Egyptu se nepodaří žádné dílo, ať je koná hlava nebo ocas, palmová ratolest nebo sítí.
#19:16 V onen den bude Egypt jako ženy: bude se třást strachem před mávnutím ruky Hospodina zástupů, až proti němu mávne.
#19:17 Země judská se stane postrachem Egypta; ať mu ji kdokoli připomene, Egypt dostane strach z úradku, který nad ním vynesl Hospodin zástupů.
#19:18 V onen den bude v egyptské zemi pět měst hovořících kenaanským jazykem a přísahajících při Hospodinu zástupů; jednomu se bude říkat „Město pobořené“.
#19:19 V onen den bude mít Hospodin oltář uprostřed egyptské země a posvátný sloup při jejich hranicích bude zasvěcen Hospodinu.
#19:20 Bude to znamením a svědectvím o Hospodinu zástupů v egyptské zemi. Když budou volat k Hospodinu kvůli utiskovatelům, pošle jim zachránce, aby je obhájil a vysvobodil je.
#19:21 Hospodin se dá poznat v Egyptě; Egypťané poznají v onen den Hospodina a budou ho uctívat obětním hodem a obětním darem, budou Hospodinu skládat a plnit sliby.
#19:22 Hospodin udeří na Egypt, udeří i vyléčí, a oni se obrátí k Hospodinu; on přijme jejich prosby a vyléčí je.
#19:23 V onen den povede silnice z Egypta do Asýrie, takže Asyřané budou chodit do Egypta a Egypťané do Asýrie a Egypťané budou uctívat Hospodina spolu s Asyřany.
#19:24 V onen den vytvoří Izrael s Egyptem a Asýrií trojici; bude požehnáním uprostřed země,
#19:25 protože Hospodin zástupů mu bude žehnat takto: „Požehnán buď lid můj egyptský a dílo mých rukou Asýrie a mé dědictví Izrael.“ 
#20:1 Toho roku, v němž tartán poslaný asyrským králem Sargónem přitáhl k Ašdódu, bojoval proti němu a dobyl jej,
#20:2 v onen čas promluvil Hospodin skrze Izajáše, syna Amósova: „Jdi a odpásej si z beder žíněnou suknici a zuj si opánky z nohou.“ On tak učinil a chodil nahý a bosý.
#20:3 I řekl Hospodin: „Jako chodí můj služebník Izajáš nahý a bosý po tři roky na znamení a předzvěst pro Egypt a pro Kúš,
#20:4 tak požene asyrský král zajatce egyptské a vysídlence kúšské, mladíky i starce, nahé a bosé, s odhaleným zadkem, Egyptu k hanbě.
#20:5 Budou plni děsu a hanby pro Kúš, k němuž vzhlíželi, a pro Egypt, jímž se chlubili.
#20:6 Obyvatelé toho pobřeží v onen den řeknou: ‚Hle, tak to dopadlo s tím, k němuž jsme vzhlíželi, k němuž jsme se utíkali o pomoc, aby nás vysvobodil od asyrského krále. Jak bychom mohli vyváznout my?‘“ 
#21:1 Výnos o přímořské stepi: Jako vichřice prohánějící se Negebem přijde ze stepi, z obávané země.
#21:2 Dostalo se mi tvrdého vidění: „Věrolomník jedná věrolomně a zhoubce hubí. Přitáhni, Élame, oblehni, Méde! Všem vzdechů učiním přítrž.“
#21:3 Proto jsou má bedra plná smrtelných úzkostí, křeče se mne zmocnily jako rodičky. Jsem zkrušen tím, co slyším, hrozím se toho, co vidím.
#21:4 Mé srdce vrávorá, náhlá hrůza mě jímá, kýžený soumrak se mi změnil v otřesnou chvíli.
#21:5 Chystá se stolování, prostírá se ubrus, jí se a pije. Vzhůru, velitelé, namažte své štíty!
#21:6 Neboť Panovník mi praví toto: „Jdi, postav hlídku; ať oznámí, co uvidí.
#21:7 Uvidí-li vozbu, koňská spřežení, jízdu na oslech, jízdu na velbloudech, ať sleduje pozorně, velmi pozorně.“
#21:8 I zvolal jako lev: „Na hlídce, Panovníku, stojím stále, po celé dny, a na své strážní stanoviště se stavím po všechny noci.“
#21:9 Hle, tu přijíždí vůz, muž, koňské spřežení! On hlásí: „Padl, padl Babylón a všechny tesané modly jeho božstev jsou roztříštěny o zem!“
#21:10 Můj podupaný lide, synu mého humna! Co jsem slyšel od Hospodina zástupů Boha Izraele, to vám oznamuji.
#21:11 Výnos o Dúmě: Volají na mě ze Seíru: „Strážce, kolik zbývá z noci? Strážce, kolik zbývá z noci?“
#21:12 Strážce praví: „Přichází jitro, ale ještě je noc. Chcete-li pátrat, pátrejte! Navraťte se, přijďte!“
#21:13 Výnos proti Arábii: Budete nocovat v arabské divočině, karavany Dedánců.
#21:14 Vstříc žíznivému! Neste vodu! Obyvatelé země Témy, vyjděte s chlebem naproti štvanci!
#21:15 Vždyť prchli před meči, před mečem taseným, před lukem napjatým, před tíhou boje.
#21:16 Panovník mi praví toto: „Do roka, jako jsou léta nádeníka, zanikne veškerá sláva Kédaru.
#21:17 A Kédarcům zůstane pranepatrný počet bohatýrských lučištníků.“ Tak promluvil Hospodin, Bůh Izraele. 
#22:1 Výnos o Údolí vidění: Co je s tebou, že jsi celé vystoupilo na střechy,
#22:2 město plné hluku, halasu, město rozjařené? Skolení tvoji nebyli skoleni mečem, nezemřeli v boji.
#22:3 Do jednoho prchli všichni tvoji náčelníci; byli jati bez výstřelu z luku. Všichni tvoji zastižení byli do jednoho jati v dáli na útěku.
#22:4 Proto jsem řekl: „Odvraťte ode mne oči, ať hořekuji a pláči; nesnažte se mě těšit nad zhoubou dcery mého lidu!“
#22:5 Neboť den poděšení a pošlapání a rozruchu, den Panovníka, Hospodina zástupů, nastal v Údolí vidění, den zhroucení zdí a křiku o pomoc k hoře.
#22:6 Élam zvedl toulec, má vozy s posádkou a koně, a Kír obnažil štít.
#22:7 Nejlepší z tvých dolin jsou plny vozů; jezdci zaujali postavení před branami.
#22:8 Odkryto bylo Judovo zaštítění. Onoho dne jsi vzhlížel k výzbroji Domu lesa.
#22:9 Viděli jste, jak mnoho bylo průlomů do města Davidova a shromažďovali jste vodu z Dolního rybníka.
#22:10 Pořídili jste soupis domů v Jeruzalémě a strhávali jste domy pro zpevnění hradeb.
#22:11 Mezi obojími hradbami jste zřídili nádrž pro vodu Starého rybníka. Nevzhlíželi jste však k tomu, který vše způsobil, a neohlíželi jste se na toho, který to odedávna připravoval.
#22:12 Onoho dne Panovník, Hospodin zástupů, vyzval k pláči a k nářku a k vyholení lysiny a k přepásání žíněnou suknicí,
#22:13 a hle, veselí a radost, porážení dobytka a zabíjení ovcí, jedení masa a pití vína. „Jezme a pijme, stejně zítra zemřeme!“
#22:14 Mému sluchu však Hospodin zástupů zjevil: „Tato vaše nepravost jistotně nedojde smíření, zemřete“, praví Panovník, Hospodin zástupů.
#22:15 Toto praví Panovník, Hospodin zástupů: Jdi, vydej se k tomu hodnostáři, promluv k Šebnovi, správci domu:
#22:16 „Co tu pohledáváš, koho tu máš, žes tu vytesal pro sebe hrob? Dal sis hrob vytesat vysoko, dal sis vyhloubit příbytek ve skále!
#22:17 Hle, Hospodin tě odvrhe, odvrhne tě, ty hrdino, zabalí tě,
#22:18 smotá, jako klubko tě svine, jako míč tě odhodí do široširé země. Tam umřeš a tam skončí tvé slavné vozy, pohano domu svého pána.
#22:19 Srazím tě z tvého postavení; ze svého úřadu budeš svržen.
#22:20 V onen den povolám svého služebníka Eljakíma, syna Chilkijášova.
#22:21 Obléknu ho do tvé suknice, připevním mu tvoji šerpu, tvou vladařskou moc vložím do jeho rukou a stane se otcem obyvatel Jeruzaléma i Judova domu.
#22:22 Na jeho rameno vložím klíč domu Davidova; když otevře, nikdo nezavře, a když zavře, nikdo neotevře.
#22:23 Zarazím ho jako hřeb na spolehlivém místě a stane se stolcem slávy otcovského domu.
#22:24 Ale zavěsí se na něj všechna váha otcovského domu, výhonky a odnože, všechny malé nádoby, jak misky, tak i všechny džbány.“
#22:25 „V onen den, je výrok Hospodina zástupů, povolí hřeb zaražený na spolehlivém místě, zlomí se, spadne a náklad, který je na něm, se zničí.“ Tak promluvil Hospodin. 
#23:1 Výnos o Týru: Kvilte, zámořské lodě! Týr je vypleněn, je bez domu, je bez přístavu; z Kitejské země to bylo rozhlášeno.
#23:2 Zmlkněte, obyvatelé pobřeží! Bylo v tobě plno sidónských obchodníků, mořeplavců.
#23:3 Přes velké vody dopravovali výnos zemí, setbu od Delty a sklizeň od Nilu; byls tržištěm pronárodů.
#23:4 Styď se, Sidóne! Vždyť prohlásilo moře, mořská pevnost řekla: „Nepracuji k porodu a nerodím ani nevychovávám jinochy, o panny nepečuji.“
#23:5 Jakmile ta zpráva dojde Egypťanům, budou se nad zprávou o Týru svíjet v křeči.
#23:6 Přeplavte se do Taršíše, obyvatelé pobřeží, kvilte!
#23:7 To že je váš rozjařený Týr? Jeho původ je v dávnověku; nohy ho nosí do daleka, tam se usazuje.
#23:8 Kdo takto rozhodl o Týru, který udílel koruny panovnické? Jeho obchodníci byli knížaty a jeho kramáři nejváženějšími v zemi.
#23:9 Hospodin zástupů tak rozhodl, aby zlomil pýchu vší nádhery, aby zlehčil všechny nejváženější v zemi.
#23:10 Rozlej se jako Nil po své zemi, taršíšská dcero; už tu není hráz.
#23:11 Hospodin vztáhl ruku na moře, otřásl královstvími, vydal příkaz proti Kenaanu o zničení jeho záštit.
#23:12 Vyhlásil: „Už nikdy nebudeš jásat, násilím potupená panno, dcero sidónská! Ke Kitejcům! Vstaň, přeplav se! Ani tam nenajdeš odpočinutí.“
#23:13 Pohleď na zemi Kaldejců; ten lid tu už není. Asýrie ji určila divé sběři. Postavili obléhací věže, paláce strhli, město obrátili v rozvaliny.
#23:14 Kvilte, zámořské lodě, vaše záštita je vypleněna!
#23:15 V onen den bude Týr zapomenut na sedmdesát let, jako by to byly dny jediného krále; do uplynutí sedmdesáti let se Týru povede jako v té písničce o nevěstce:
#23:16 „Vezmi citaru, obcházej město, nevěstko zapomenutá, hezky hrej, zpívej píseň za písní, ať se připomeneš.“
#23:17 Po uplynutí sedmdesáti let navštíví Hospodin Týr a ten se vrátí ke své nevěstčí mzdě; bude smilnit se všemi královstvími světa, co jich je na zemi.
#23:18 Ale jeho obchodní zisk a jeho nevěstčí mzda se stanou svatým majetkem Hospodinovým. Nebudou uloženy ani uschovány; jeho obchodní zisk bude patřit těm, kteří přebývají před tváří Hospodinovou, aby se dosyta najedli a dobře oblékli. 
#24:1 Hle, Hospodin vyplení zemi a zpustoší ji, rozvrátí její tvářnost a rozpráší její obyvatele.
#24:2 Bude na tom stejně lid i kněz, otrok i jeho pán, služka i její paní, kupující i prodavač, věřitel i dlužník, lichvář i zadlužený.
#24:3 Země bude zcela vypleněna a vyloupena; toto slovo promluvil Hospodin.
#24:4 Truchlit, vadnout bude země, chřadnout, vadnout bude svět, nejvznešenější z lidu země budou chřadnout.
#24:5 Zhanobena je země svými obyvateli, neboť přestoupili zákony, změnili nařízení, porušili věčnou smlouvu.
#24:6 Proto pozře zemi prokletí, kdo v ní přebývají, budou pykat; proto obyvatelé země zajdou v žáru a zůstane lidí maličko.
#24:7 Bude truchlit mošt, zvadne vinná réva, všichni, jejichž srdce se raduje, budou vzdychat.
#24:8 Ustane veselí bubnů, pomine hluk rozjařených, ustane veselí citary.
#24:9 Nebudou pít za zpěvu víno, opojný nápoj pijákům zhořkne.
#24:10 Zbořeno bude pusté město, vchod do každého domu bude zatarasen.
#24:11 V ulicích bude žalostný křik pro víno, nastane večer veškeré radosti, veselí se vystěhuje ze země.
#24:12 Ve městě zůstane hrozná spoušť, zničena, rozbita bude brána.
#24:13 I bude uprostřed země, mezi národy, tak jako když se srážejí olivy, jako paběrky, když končí vinobraní.
#24:14 Všude pozvednou svůj hlas a budou plesat, nad důstojností Hospodinovou budou jásat i při moři.
#24:15 Proto vzdejte Hospodinu čest i na východě, jménu Hospodina, Boha Izraele, na ostrovech mořských.
#24:16 Od kraje země jsme slyšeli prozpěvovat: „Sláva spravedlivému!“ Já jsem však řekl: „Je mi zle! Je mi zle! Běda mi!“ Věrolomní jednají věrolomně, věrolomní se dopouštějí věrolomnosti.
#24:17 Postrach, propast, past na tebe, obyvateli země!
#24:18 Kdo uteče před hlučícím postrachem, padne do propasti, a kdo vyleze z propasti, do pasti se lapí, otevřou se nebeské propusti shora a budou se třást základy země.
#24:19 Země se rozlomí, země se rozkymácí, země se rozpadne.
#24:20 Země se rozvrávorá, bude jak opilec, bude se zmítat jako budka. Dolehne na ni její nevěrnost, i padne a už nepovstane.
#24:21 V onen den Hospodin ztrestá nebeský zástup na výšině i pozemské krále na zemi.
#24:22 Budou sebráni, jako je sebrán vězeň do jámy, a budou zavřeni pod zámek; po mnohých pak dnech budou potrestáni.
#24:23 Zardí se bledá luna, zastydí se žárné slunko, až se Hospodin zástupů ujme kralování na hoře Sijónu a v Jeruzalémě před svými staršími v slávě. 
#25:1 Hospodine, ty jsi můj Bůh! Tebe budu vyvyšovat, vzdávat chválu tvému jménu, neboť činíš podivuhodné věci, tvé odvěké úradky jsou věrná pravda.
#25:2 Obrátil jsi město v hromadu sutin, opevněné město v rozvaliny. Palác cizáků z města zmizel, nebude vystavěn navěky.
#25:3 Proto tě bude ctít lid mocný, bude se tě bát město ukrutných pronárodů.
#25:4 Byl jsi záštitou slabého, v soužení záštitou ubožáka, útočištěm před přívalem i stínem v horku. Dech ukrutníků je jak příval na zeď,
#25:5 jako horko ve vyprahlém kraji. Hřmotící cizáky pokoříš jak horko stínem oblaku, prozpěvující ukrutníci budou poníženi.
#25:6 Hospodin zástupů připraví na této hoře všem národům hostinu tučnou, hostinu s vyzrálým vínem, jídla tučná s morkem, víno vyzrálé a přečištěné.
#25:7 Na této hoře odstraní závoj, který zahaluje všechny národy, přikrývku, která přikrývá všechny pronárody.
#25:8 Panovník Hospodin provždy odstraní smrt a setře slzu z každé tváře, sejme potupu svého lidu z celé země; tak promluvil Hospodin.
#25:9 V onen den se bude říkat: „Hle, to je náš Bůh. V něho jsme skládali naději a on nás spasil. Je to Hospodin, v něhož jsme skládali naději, budeme jásat a radovat se, že nás spasil.“
#25:10 Ruka Hospodinova spočine na této hoře, ale Moáb bude zašlapán, jako bývá šlapána sláma do hnoje,
#25:11 rozpřáhne tam potom ruce jako je rozpřahuje plavec při plavání; Bůh poníží jeho pýchu i obratnost jeho rukou.
#25:12 Tvé opevnění, tvé nedobytné hradby svalí, sníží a srazí k zemi, až do prachu. 
#26:1 V onen den se bude zpívat v judské zemi tato píseň: „Mocné je naše město! Jako hradby a val mu Bůh dal spásu.“
#26:2 Otevřete brány, ať vejde spravedlivý národ, který zachovává věrnost.
#26:3 Stvoření opírající se o tebe chráníš pokojem, pokojem, neboť v tebe doufá.
#26:4 Doufejte v Hospodina věčně, neboť Hospodin, jen Hospodin je skála věků.
#26:5 On sehnul obyvatele výšiny, město nedostupné, on je snižuje, až k zemi je snižuje, až do prachu je sráží.
#26:6 Pošlape je noha, nohy utištěného a kroky slabých.
#26:7 Stezka spravedlivého je přímá; spravedlivému sám urovnáváš dráhu Bože přímý.
#26:8 Také na stezce tvých soudů, Hospodine, skládáme naději v tebe, naše duše touží po tvém jménu, chce si připomínat tebe.
#26:9 Má duše v noci po tobě touží, můj duch ve mně za úsvitu tebe hledá. Když vykonáváš své soudy na zemi, obyvatelé světa se učí spravedlnosti.
#26:10 Dává-li se milost svévolníku, spravedlnosti se nenaučí; v zemi správných řádů bude jednat podle, na Hospodinovu důstojnost nebude hledět.
#26:11 Hospodine, ruka tvá je pozvednuta, a oni to přehlížejí. Ale zahanbeni spatří, jak ty horlíš pro svůj lid; tvé protivníky pozře oheň.
#26:12 Hospodine, jenom ty nám můžeš zjednat pokoj, tak jako jsi nám odplatil za všechny naše skutky.
#26:13 Hospodine, Bože náš, byli jsme pod mocí jiných, nikoli tvou, budeme však připomínat jedině tvé jméno!
#26:14 Mrtví neobživnou, stíny nepovstanou, neboť jsi je ztrestal, zahladils je, každou památku na ně jsi zničil.
#26:15 Rozmnožil jsi národ, Hospodine, rozmnožil jsi národ, oslavil ses, daleko jsi rozšířil všechny hranice země.
#26:16 Hospodine, po tobě se v soužení svém ohlíželi, vylévali tiché prosby, kdyžs je káral.
#26:17 Jako před porodem těhotná se svíjí a v bolestech křičí, takoví jsme byli před tvou tváří Hospodine.
#26:18 Svíjeli jsme se jako těhotní, a porodili jsme jen vítr. Zemi jsme vítězství nezískali, nepadli obyvatelé světa.
#26:19 Tvoji mrtví obživnou, má mrtvá těla vstanou! Probuďte se, plesejte, kdo přebýváte v prachu! Vždyť tvá rosa je rosou světel, porazíš i zemi stínů.
#26:20 Nuže, lide můj, již vejdi do svých komnat a zavři za sebou dveře. Skryj se na chviličku, dokud hrozný hněv se nepřežene.
#26:21 Hle, Hospodin vychází ze svého místa, aby ztrestal nepravost obyvatel země. I odhalí země krev prolitou na ní a nebude již přikrývat povražděné. 
#27:1 V onen den Hospodin ztrestá svým tvrdým, velkým a pevným mečem livjátana, hada útočného, livjátana, hada svinutého; zabije draka v moři.
#27:2 V onen den zpívejte o vinici, jež dává ohnivé víno.
#27:3 „Já Hospodin ji střežím, každou chvíli ji zavlažuji. Aby ji nic nepostihlo, nocí dnem ji budu střežit.
#27:4 Rozhořčen už na ni nejsem. Kdyby mi dávala bodláčí a křoví, zvedl bych proti ní boj a rázem bych ji spálil.
#27:5 Ať se chopí mé záštity a uzavře se mnou pokoj, ať se mnou uzavře pokoj.
#27:6 Přijdou dny, kdy Jákob zapustí kořeny, kdy bude pučet a rozkvete Izrael a naplní tvář světa svými plody.“
#27:7 Bil ho snad Bůh, jako bil ty, kdo bili jeho? Byl snad zavražděn jako jiní, kdo jím byli zavražděni?
#27:8 Vyhnal jsi ji, vyhostils ji, tak jsi s ní vedl spor. Odklidil ji svým ostrým větrem, když zavál od východu.
#27:9 Právě tím bude usmířena Jákobova nepravost. Až z něho sejme hřích, vydá toto ovoce: se všemi kameny oltáře naloží jako s kusy vápence, které se roztloukají. Nezůstanou stát posvátné kůly ani kadidlové oltáříky.
#27:10 Opevněné město zpustne, bude jak bydliště zavržené a poušt opuštěná; bude se tam pást býček, odpočívat tam a spásat jeho ratolesti.
#27:11 Až jeho snítky uschnou a polámou se, přijdou ženy a spálí je. Ten lid nic nechápe; proto ten, který jej učinil, se nad ním neslituje, jeho Tvůrce se nad ním nesmiluje.
#27:12 I stane se v onen den, že Hospodin vymlátí klasy od Řeky až k potoku Egyptskému; to vy, synové Izraelovi, budete sbíráni jeden po druhém.
#27:13 V onen den zaduje na velikou polnici a přijdou ti, kdo bloudí bez domova v zemi asyrské, i ti, kdo byli vyhnáni do egyptské země, a budou se klanět Hospodinu na svaté hoře v Jeruzalémě. 
#28:1 Běda pyšné koruně efrajimských opilců, nádherné ozdobě, květu vadnoucímu, městu ležícímu na vrchu nad žírným údolím; jsou zmoženi vínem.
#28:2 Hle, Panovníkův silný a udatný přijde jako příval krupobití, jako zhoubná bouře; jako příval vod, jak dravé proudy je srazí pěstí k zemi.
#28:3 Nohama bude pošlapána pyšná koruna efrajimských opilců,
#28:4 nádherná ozdoba, vadnoucí květ, město ležící na vrchu nad žírným údolím. Bude jako raný fík, než nastane léto: jen ho někdo uvidí, už jej má v hrsti a zhltne jej.
#28:5 V onen den se Hospodin zástupů stane ozdobnou korunou a nádherným věncem pozůstatku svého lidu
#28:6 a duchem práva tomu, kdo zasedá na soudu, i zmužilostí těm, kdo vracejí boj k bráně.
#28:7 Také tito blouzní z vína, po opojném nápoji se potácejí: kněz i prorok blouzní z opojného nápoje, jsou pomateni vínem, po opojném nápoji se potácejí, blouzní při viděních, při rozhodování kolísají.
#28:8 Všechny stoly jsou plné zvratků, samý výkal, místa není.
#28:9 Koho vyučit poznání? Komu vyložit zvěst? Odkojeným dětem, od prsů odstaveným?
#28:10 Stále jenom: žvást na žvást, žvást na žvást, tlach na tlach, tlach na tlach, trošku sem, trošku tam.
#28:11 Proto nesrozumitelnou řečí a cizím jazykem promluví k tomuto lidu ten,
#28:12 který mu říkal: „Zde je místo odpočinutí, nechte odpočinout znaveného, zde je místo míru!“ Nechtěli však slyšet.
#28:13 Slovo Hospodinovo jim bude: „Žvást na žvást, žvást na žvást, tlach na tlach, tlach na tlach, trošku sem, trošku tam.“ Jen ať jdou, však klesnou zpět a ztroskotají, padnou do léčky a uvíznou v ní.
#28:14 Proto slyšte slovo Hospodinovo, chvastouni, vládnoucí nad tímto lidem v Jeruzaléme!
#28:15 Říkáte: „Uzavřeli jsme smlouvu se smrtí a sjednali jsme úmluvu s podsvětím. Valí se bičující příval? Však se přežene, nás nezasáhne, vždyť jsme si svým útočištěm učinili lež a skryli jsme se v klamu.“
#28:16 Proto praví toto Panovník Hospodin: „Já to jsem, kdo za základ položil na Sijónu kámen, kámen osvědčený, úhelný a drahý, základ nejpevnější; kdo věří, nemusí spěchat.
#28:17 Za měřicí šňůru beru právo, za olovnici spravedlnost. Lživé útočiště smete krupobití a skrýši zaplaví vody.
#28:18 Vaše smlouva se smrtí bude zřušena a vaše úmluva s podsvětím neobstojí. Valí se bičující příval! Až se přežene, budete po něm jak zdupaná země.
#28:19 Kdykoli se přižene, sebere vás. Jitro co jitro se přežene, dnem i nocí. Hrůza ochromí ty, kdo pochopí tu zvěst.“
#28:20 Lůžko bude krátké, nebude možno narovnat se, a přikrývka úzká, že se do ní nezavineš.
#28:21 Jako na hoře Perasímu povstane Hospodin, chvět se bude země jak v dolině Gibeónské. Vykoná své dílo, dílo jemu nevlastní, udělá svou práci, práci jemu cizí.
#28:22 Nechovejte se už jako chvastouni, ať vám nemusí být přitažena pouta. Slyšel jsem od Panovníka, Hospodina zástupů, že je konec s celou zemí, že je rozhodnuto.
#28:23 Naslouchejte a slyšte můj hlas, pozorně sledujte a slyšte mou řeč.
#28:24 Cožpak oráč oře, kypří a vláčí svou roli bez ustání, aby mohl sít?
#28:25 Když urovná její povrch, rozsívá hned kopr, rozhazuje kmín, seje na určená místa pšenici, proso i ječmen a na okraj špaldu.
#28:26 K takovému řádu ho vycvičil a vyučil jeho Bůh.
#28:27 Kopr se nemlátí smykem a po kmínu se nejezdí koly vozů, nýbrž kopr se vyklepává prutem a kmín holí.
#28:28 Zrno na chléb se drtí, ale ne zcela; mlátí se, vjíždí se na ně koly vozů a spřežením, ale nerozdrtí se.
#28:29 Také toto pochází od Hospodina zástupů. Podivuhodný je ve svém úradku, veliký ve své pohotové pomoci. 
#29:1 Běda tobě, Aríeli, Aríeli, město, v němž se utábořil David! Jen pokračujte rok za rokem, ať probíhá koloběh svátků,
#29:2 ale dokročím na Aríela a nastane zármutek a smutek. Pak mi vskutku bude aríelem, oltářním ohništěm!
#29:3 Ze všech stran se proti tobě utábořím, sevřu tě obležením, navrším proti tobě náspy.
#29:4 Budeš sníženo, ze země budeš mluvit, tvá řeč přidušeně zazní z prachu. Tvůj hlas bude znít ze země jako hlas ducha zemřelých, z prachu bude řeč tvá sípat.
#29:5 Bude proti tobě dav cizáků jak zvířený prach, ukrutníků dav jak záplava plev. Stane se to náhle, znenadání.
#29:6 Od Hospodina zástupů bude město postiženo hromobitím a zemětřesením a mocným burácením, vichrem a smrští a plamenem sžírajícího ohně.
#29:7 A bude to jako sen, jak noční vidění: shluk všech pronárodů, které vytáhly do boje proti Aríeli; všechny vytáhly do boje proti němu a jeho pevnostem, aby na něj dokročily.
#29:8 Bude to však, jako když lačný má sen: zdá se mu, že jí, a když procitne, má prázdnou duši. Nebo jako když má sen žíznivý: zdá se mu, že pije, a když procitne, je ochablý a jeho duše prahne. Tak dopadne shluk všech pronárodů, které vytáhly do boje proti hoře Sijónu.
#29:9 Žasněte a ustrňte! Oslepněte, buďte slepí! Jsou opilí, ale ne vínem, potácejí se, ne však po opojném nápoji.
#29:10 To Hospodin vylil na vás ducha mrákoty, zavřel vaše oči, totiž proroky, a zahalil vaše hlavy, totiž vidoucí.
#29:11 Vidění toho všeho vám bude jako slova zapečetěné knihy. Dají ji tomu, kdo umí číst, se slovy: „Přečti to.“ On však odpoví: „Nemohu, je zapečetěná.“
#29:12 I bude kniha předána tomu, kdo neumí číst, se slovy: „Přečti to.“ A on odpoví: „Neumím číst.“
#29:13 Panovník praví: „Protože se tento lid přibližuje ke mně ústy a ctí mě svými rty, ale svým srdcem se ode mne vzdaluje a jejich bázeň přede mnou se stala jen naučeným lidským příkazem,
#29:14 proto i já budu dále podivuhodně jednat s tímto lidem, divně, předivně. Zanikne moudrost jeho moudrých a rozumnost jeho rozumných bude zakryta.“
#29:15 Běda těm, kdo své záměry skrývají hluboko před Hospodinem, v temnotách konají své činy a říkají: „Kdopak nás vidí a kdo o nás ví?“
#29:16 Je to vaše převrácenost, jestliže je hrnčíř oceňován stejně jako hlína. Což říká dílo tom, kdo je udělal: „On mě neudělal“? Což výtvor řekne o svém tvůrci: „On ničemu nerozumí“?
#29:17 Potrvá to jenom velmi krátce a Libanón se změní v sad a sad bude mít cenu lesa.
#29:18 I uslyší v onen den hluší slova knihy a oči slepých prohlédnou z temnoty a ze tmy.
#29:19 Pokorní se znovu budou radovat z Hospodina a nejubožejší z lidí budou jásat vstříc Svatému Izraele.
#29:20 Pryč zmizí ukrutník, po posměvači bude veta, vyhlazeni budou všichni, kdo jsou pohotovi k ničemnosti,
#29:21 kteří svým slovem svádějí člověka k hříchu, tomu, kdo je v bráně kárá, kladou léčky a spravedlivého strhují do nicoty.
#29:22 Proto praví toto Hospodin, který vykoupil Abrahama, o domu Jákobově: „Jákob už nepozná hanbu, ani mu nezblednou líce,
#29:23 uzří svoje děti, dílo mých rukou, ve svém středu. Budou dosvědčovat svatost mého jména, budou dosvědčovat svatost Svatého Jákobova, budou se třást před Bohem Izraele.
#29:24 Zbloudilí duchem poznají rozumnost, mému naučení se budou učit reptalové.“ 
#30:1 „Běda, synové umínění, je výrok Hospodinův, uskutečňujete záměry, ale ne moje, uzavíráte závazky, ale ne v mém duchu, vršíte hřích na hřích.
#30:2 Odcházíte a sestupujete do Egypta, ale mých úst jste se nedoptali, záštitu hledáte u faraóna, utíkáte se do stínu Egypta.
#30:3 Ale faraónova záštita vám bude k hanbě a utíkání do stínu Egypta k potupě.
#30:4 Vaši velmoži jsou v Sóanu a poslové dosáhli Chánesu;
#30:5 všichni se zklamou v lidu, který jim neprospěje, nebude jim ku pomoci ani ku prospěchu, ale k ostudě a hanbě.“
#30:6 Výnos o zvířatech v Negebu: Do země soužení a tísně, kde jsou lvice a lvi mezi nimi, zmije a ohnivý létající had, přivážejí Judejci na hřbetě oslů svůj majetek a na hrbech velbloudů své poklady lidu, který jim neprospěje.
#30:7 Egyptská pomoc je prázdný přelud. Proto jsem je nazval: „Obluda vyřízená.“
#30:8 Nyní jsi, napiš to před nimi na desku, zaznamenej do knihy a zůstane to svědectvím do posledního dne, navždy, navěky.
#30:9 Neboť je to lid vzpurný, synové prolhaní, synové, kteří nechtějí poslouchat Hospodinův zákon.
#30:10 Říkají vidoucím: „Nedívejte se“, a těm, kdo mají dar vidění: „Neukazujte nám, co je správné! Jen nám pochlebujte, jen nás balamuťte!
#30:11 Opusťte cestu, odbočte ze stezky, přestaňte se Svatým Izraele!“
#30:12 Proto praví Svatý Izraele toto: „Poněvadž pohrdáte tímto slovem a spoléháte na útisk a převrácenost a o ně se opíráte,
#30:13 proto vám tato nepravost bude jako rozestupující se trhlina v bortící se nedostupné hradbě, jež se zhroutí náhle, znenadání.
#30:14 Bude to, jako když někdo roztříští hliněný džbán napadrť, nelítostně, že se nenajde po rozbití ani střípek, aby se v něm dal přenést oheň z ohniště či nabrat voda z louže.“
#30:15 Neboť toto praví Panovník Hospodin, Svatý Izraele: „V obrácení a ztišení bude vaše spása, v klidu a důvěře vaše vítězství. Vy však nechcete,
#30:16 říkáte: ‚Nikoli! Utečeme na koních.‘ A vskutku budete utíkat. ‚Ujedeme na lehkonohých.‘ A vskutku vaši pronásledovatelé budou lehkonozí.
#30:17 Jeden tisíc vás uteče před pohrůžkou jediného, před pohrůžkou pěti utečete všichni, až zůstanete jako opuštěná žerď na vrcholu hory, jako korouhev na pahorku.“
#30:18 Hospodin vyčkává, chce se nad vámi smilovat, vyvýší se, slituje se nad vámi. Vždyť Hospodin je Bohem práva, blaze těm, kdo ho očekávají.
#30:19 Neboť lid bude sídlit na Sijónu, v Jeruzalémě. Nebudeš už nikdy plakat. Milostivě se nad tebou smiluje, až budeš úpěnlivě volat; uslyší a odpoví ti.
#30:20 Ač vám dával Panovník chléb soužení a vodu útlaku, on, tvůj učitel, už se nebude držet stranou. Na vlastní oči uzříš svého učitele
#30:21 a na vlastní uši uslyšíš za sebou slovo: „To je ta cesta, jděte po ní“, ať budete chtít doprava nebo doleva.
#30:22 Pak prohlásíš své stříbrem potažené vytesané modly i pozlacené lité modly za nečisté a jako nečisté je rozmetáš. Řekneš jim: „Táhni!“
#30:23 A Hospodin dá déšť tvé setbě, jíž oseješ roli; chléb z úrody role bude vydatný a výživný. V onen den se bude pást tvůj dobytek na širých lukách.
#30:24 Býci a osli, obdělávající roli, budou žrát šťavnatou a slanou píci, nakypřenou lopatou a vidlemi.
#30:25 Na každé vysoké hoře a na každém vyvýšeném pahorku budou potůčky tekoucích vod, v den velkého vraždění, kdy padnou věže.
#30:26 A bude světlo bledé luny jako světlo žárného slunka a světlo slunka bude sedmkrát větší, jako světlo sedmi dnů, v den, kdy Hospodin ováže zlomeniny svého lidu a rány jemu zasazené uzdraví.
#30:27 Hle, jméno Hospodinovo přichází zdáli. Jeho planoucí hněv se mocně zvedá. Rty mu překypují hrozným hněvem, jeho jazyk je jak sžírající oheň.
#30:28 Jeho dech je jako rozvodněný potok až k hrdlu sahající. Bude prosívat pronárody řešetem jejich šalebnosti, čelisti národů sevře uzdou svodu.
#30:29 Dáte se do zpěvu jako v noci, kdy se zasvěcuje svátek, srdce se naplní radostí jako tomu, kdo jde za zvuku píšťal, aby vstoupil na Hospodinovu horu, ke Skále Izraele.
#30:30 Hospodin se ohlásí svým velebným hlasem, dá pocítit svou dopadající paži v prudkém hněvu, v plameni sžírajícího ohně, blýskáním a průtrží mračen a kamenným krupobitím.
#30:31 Ašúr, jenž jiné bil holí, zděsí se Hospodinova hlasu.
#30:32 Každé švihnutí metlou, které mu Hospodin zasadí, dopadne za zvuku bubínků a citar; vybojuje proti němu bitvy pouhým mávnutím ruky.
#30:33 Neboť odedávna je už připraven Tófet, žároviště hluboké a široké, i pro krále. Na jeho hranici je mnoho ohně a dříví, jako proud síry je zapálí dech Hospodinův. 
#31:1 Běda těm, kdo sestupují pro pomoc do Egypta, na koně spoléhají, doufají ve vozy, že jich je mnoho, v jízdu, že je velmi zdatná; ale ke Svatému Izraele nevzhlížejí, nedotazují se Hospodina.
#31:2 Leč i on je moudrý. Přivodí zlé věci, neruší svá slova. Povstane proti domu zlovolníků, proti pomoci vyžádané od těch, kdo páchají ničemnosti.
#31:3 Egypt - to jsou jenom lidé, a ne Bůh, jejich koně tělo, a ne duch. Když Hospodin napřáhne svou ruku, tu klopýtne ten, kdo poskytuje pomoc, a padne ten, kdo ji přijímá: všichni spolu zhynou.
#31:4 Hospodin mi pravil toto: „Jako vrčí lev a lvíče nad úlovkem, i kdyby proti němu svolali všechny pastýře - jejich křik ho nepoděsí, před jejich halasem se nebude krčit -, tak sestoupí Hospodin zástupů k boji na horu Sijón, na její pahorek.
#31:5 Jako ptáci, kteří obletují hnízda, tak bude Hospodin zástupů štítem Jeruzalému, zaštítí a vysvobodí, ušetří a vyprostí.“
#31:6 Vraťte se k tomu, od něhož jste daleko odpadli, synové Izraele.
#31:7 Neboť v onen den každý z vás zavrhne své bůžky stříbrné i svoje bůžky zlaté, jež jste si vlastníma rukama hříšně vyrobili.
#31:8 „Ašúr padne, ne však mečem muže, pozře jej meč, ne však lidský. I kdyby utekl před mečem, jeho jinoši budou podrobeni nuceným pracím.
#31:9 I jeho skála odtáhne v děsu, před Hospodinovou korouhví se zděsí jeho velmožové“, je výrok Hospodina, který má oheň na Sijónu a pec v Jeruzalémě. 
#32:1 Hle, král bude kralovat spravedlivě a velmožové budou vládnout podle práva.
#32:2 Každý z nich bude jako skrýše před větrem a úkryt před průtrží mračen, jak tekoucí vody ve vyprahlém kraji, jako stín mohutné skály v žíznivé zemi.
#32:3 Oči těch, kdo vidí, se neodvrátí, uši těch, kdo slyší, budou pozorně naslouchat.
#32:4 Srdce nerozumných nabude poznání, jazyk koktavých bude hovořit hbitě a jasně.
#32:5 Bloud už nebude zván šlechetný, potměšilci nebude se říkat velkomyslný.
#32:6 Vždyť bloud mluví bludy a jeho srdce páchá ničemnosti, dopouští se rouhání a scestně mluví proti Hospodinu. Hladovou duši nechává lačnou a žíznícímu neposkytne nápoj.
#32:7 Zbraně potměšilcovy jsou zhoubné, jen k mrzkostem radí, chce zničit utištěné klamnými slovy, když se ubožák dovolává práva.
#32:8 Šlechetný však dává ušlechtilé rady a v ušlechtilém jednání setrvává.
#32:9 Ženy sebejisté, vzhůru, slyšte můj hlas! Bezstarostné dcery, popřejte mému výroku sluchu!
#32:10 Do roka a do dne se budete chvět, vy bezstarostné, neboť je konec s vinobraním, sklizeň se nedostaví.
#32:11 Děste se, vy sebejisté, chvějte se, vy bezstarostné, vysvlečte se, obnažte se, bedra si žínicí přepásejte.
#32:12 Bijí se v prsa, naříkají pro skvělá pole, pro úrodný vinný kmen,
#32:13 pro roli mého lidu, jež vydává trní a bodláčí, ano i pro domy plné veselí v rozjařeném městě.
#32:14 Neboť palác zchátrá, hluk města ustane. Návrší se strážnou věží budou navěky jeskyněmi, budou obveselením divokých oslů a pastvinou stád.
#32:15 Až bude na nás vylit z výše duch, poušť se stane sadem a sad bude mít cenu lesa.
#32:16 I na poušti bude přebývat právo a v sadu se usídlí spravedlnost.
#32:17 Spravedlnost vytvoří pokoj, spravedlnost zajistí klid a bezpečí navěky.
#32:18 Můj lid bude sídlit na nivách pokoje, v bezpečných příbytcích, v klidných místech odpočinku,
#32:19 i kdyby na les spadlo krupobití a město bylo srovnáno se zemí.
#32:20 Blaze vám, kteří budete osévat zemi všude zavlažovanou a necháte volně běhat býka a osla. 
#33:1 Běda tobě, který šíříš zhoubu a sám nejsi huben, který jednáš věrolomně, ač se vůči tobě nikdo věrolomně nezachoval. Až dokončíš zhoubu, propadneš sám zhoubě, až dovršíš věrolomnost, bude věrolomně naloženo s tebou.
#33:2 Hospodine, slituj se nad námi, v tebe skládáme naději! Buď paží svého lidu každé ráno i naší spásou v čase soužení.
#33:3 Národy se rozprchávají před mocným hřmotem, před tvou vyvýšeností se pronárody rozptylují.
#33:4 Kořist vám bude sebrána, jako by ji sebral hmyz, jako přepadávají kobylky, tak bude napadena.
#33:5 Vyvýšen je Hospodin, jenž přebývá na výšině, naplňuje Sijón spravedlností a právem.
#33:6 Bude věrnou jistotou tvé budoucnosti, klenotnicí spásy, moudrosti a poznání, jejímž pokladem je bázeň před Hospodinem.
#33:7 Hle, mužové Aríela úpějí na ulicích, poslové pokoje hořce pláčou.
#33:8 Silnice jsou liduprázdné, pocestní přestali chodit. Smlouva se porušuje, města jsou v nevážnosti, člověk nemá cenu.
#33:9 Truchlí, chřadne země, zahanben je Libanón a vadne. Šáron podobá se pustině, Bášan i Karmel shazují listí.
#33:10 „Nyní povstanu,“ praví Hospodin, „nyní se vyvýším, nyní se zvednu!“
#33:11 Obtěžkáte senem, porodíte slámu. Váš dech je oheň a pozře vás.
#33:12 Národy budou vypalovány jako vápno, jako posekané trní vzplanou v ohni.
#33:13 Slyšte, vzdálení, co činím, poznejte, blízcí, mou bohatýrskou sílu!
#33:14 Hříšníci na Sijónu dostali strach, rouhačů se zmocnilo úzkostné chvění: Kdo z nás může pobývat u sžírajícího ohně? Kdo z nás může pobývat u věčného žáru?
#33:15 Ten, kdo žije spravedlivě a mluví, co je správné, a zavrhuje zisk z vydírání, ten, kdo třepe rukama, že nevezme úplatek, kdo si zacpe uši, aby neslyšel výzvu k prolévání krve, kdo zavírá oči, aby nepřihlížel ke zlu,
#33:16 ten bude přebývat na výšinách; nepřístupné vrcholky skal mu budou nedobytným hradem, bude mu dán chléb, vody mu potečou neustále.
#33:17 Tvoje oči uzří krále v jeho kráse, spatří zemi rozprostírající se do dálky.
#33:18 Tvé srdce bude rozjímat o někdejší hrůze: Kde je ten, kdo předpisoval daně? Kde je výběrčí? Kde ten, kdo počítával věže?
#33:19 Nespatříš už nestoudný lid, lid temné řeči, která se poslouchat nedá, směšného jazyka, jemuž rozumět nelze.
#33:20 Pohleď na Sijón, město našich slavností! Tvé oči spatří Jeruzalém, nivu poklidnou, stan, který nebude stržen, jehož kolíky nikdy nebudou vytrženy a z jehož provazů se žádný nepřetrhne.
#33:21 Neboť tam je náš Vznešený, Hospodin. Je to místo, kde jsou řeky, říční ramena přeširoká; nevydá se tam veslice a vznosný koráb tam nepřepluje.
#33:22 Hospodin je náš soudce, Hospodin je náš zákonodárce, Hospodin je náš král, on nás spasí.
#33:23 Tvé provazy povolily, nebudou schopné udržet stěžeň ani napnout plachtu. Tehdy bude rozdělena hojná kořist, i kulhaví si naloupí lup.
#33:24 Nikdo z obyvatel neřekne: „Jsem nemocen.“ Lidu, který tam bydlí, bude odpuštěna nepravost. 
#34:1 Přistupte, pronárody, a slyšte, národy, pozorně poslouchejte! Ať to slyší země se vším, co je na ní, svět i vše, co na něm vzchází!
#34:2 Hospodin je rozlícen na všechny pronárody, rozhořčen na všechny jejich zástupy. Zničí je jako klaté, vydá je na porážku.
#34:3 Jejich skolení budou pohozeni, bude se táhnout puch z jejich mrtvol, hory se rozplynou v jejich krvi.
#34:4 Rozkladu propadne všechen nebeský zástup: nebesa se svinou jako kniha a všechen jejich zástup bude padat, jako opadává listí z vinné révy a padavče z fíkovníku.
#34:5 Můj meč na nebi je zpit hněvem. Hle, sestupuje vykonat soud nad Edómem, nad lidem, nad nímž jsem vyhlásil klatbu.
#34:6 Hospodinův meč je samá krev a od tuku je mastný, je od krve jehněčí a kozlí, od tuku z beraních ledvin. Neboť Hospodin má obětní hod v Bosře, v zemi edómské veliké obětní porážení.
#34:7 Klesnou zároveň jednorožci i býci a tuři, jejich země bude zpita krví, jejich prach nasákne tukem.
#34:8 Je to den Hospodinovy pomsty, rok odplaty v soudní při Sijónu.
#34:9 Potoky Edómu se změní v smůlu a jeho prach v síru, jeho země se stane hořící smolou.
#34:10 Nevyhasne v noci ani ve dne, kouř z ní bude stoupat věčně. Po všechna pokolení zůstane v troskách, nikdo už nikdy skrze ni neprojde.
#34:11 Obsadí ji sova a sýček, výr a krkavec v ní budou bydlet. Bude nad ní natažena měřicí šňůra pustoty a spuštěna olovnice prázdnoty.
#34:12 Její šlechtici tam nebudou povoláváni ke královské hodnosti, všichni její velmožové nebudou ničím.
#34:13 V jejích palácích poroste hloží, na jejích pevnostech bodláčí a trní. Budou tam nivy šakalů a tráva pro pštrosy.
#34:14 Bude se tam setkávat divá sběř s hyenami a běsové budou na sebe pokřikovat. Bude si tam hovět upír a najde si odpočinek.
#34:15 Uhnízdí se tam zmije šípová a naklade vejce, když se háďata vylíhnou, bude je chovat ve svém stínu. Také supi se tam shromáždí, sup vedle supa.
#34:16 Hledejte v knize Hospodinově a čtěte: „Jediná z těch příšer tam nebude chybět, ani jedna z nich nebude scházet, neboť ústa Boží to přikázala a jeho duch je shromáždí.“
#34:17 On sám tento los jim nechal padnout, jeho ruka jim měřicí šňůrou přidělila tuto zemi. Obsadí ji navěky, budou v ní bydlet po všechna pokolení. 
#35:1 Poušť i suchopár se rozveselí, rozjásá se pustina a rozkvete kvítím.
#35:2 Bujně rozkvete, radostně bude jásat a plesat. Bude jí dána sláva Libanónu, nádhera Karmelu a Šáronu. Ty uzří slávu Hospodinovu, nádheru našeho Boha.
#35:3 Dodejte síly ochablým rukám, pevnosti kolenům klesajícím.
#35:4 Řekněte nerozhodným srdcím: „Buďte rozhodní, nebojte se! Hle, váš Bůh přichází s pomstou, Bůh, který odplácí, vás přijde spasit.“
#35:5 Tehdy se rozevřou oči slepých a otevřou se uši hluchých.
#35:6 Tehdy kulhavý poskočí jako jelen a jazyk němého bude plesat. Na poušti vytrysknou vody, potoky na pustině.
#35:7 Ze sálající stepi se stane jezero a z žíznivé země vodní zřídla. Na nivách šakalů bude odpočívat dobytek, tráva tam poroste jako rákosí a sítí.
#35:8 Bude tam silnice a cesta a ta se bude nazývat cestou svatou. Nebude se po ní ubírat nečistý, bude jen pro lid Boží. Kdo půjde po této cestě, nezbloudí, i kdyby to byli pošetilci.
#35:9 Nebude tam lev, dravá zvěř na ni nevstoupí, vůbec se tam nevyskytne, nýbrž půjdou tudy vykoupení.
#35:10 Ti, za něž Hospodin zaplatil, se vrátí. Přijdou na Sijón s plesáním a věčná radost bude na jejich hlavách. Dojdou veselí a radosti, na útěk se dají starosti a nářek. 
#36:1 Ve čtrnáctém roce vlády krále Chizkijáše vytáhl Sancheríb, král asyrský, proti všem opevněným městům judským a zmocnil se jich.
#36:2 Asyrský král poslal nejvyššího číšníka z Lakíše do Jeruzaléma ke králi Chizkijášovi se silným vojskem. Ten se zastavil u strouhy Horního rybníka, který je u silnice k Valchářovu poli.
#36:3 Vyšel k němu Eljakím, syn Chilkijášův, který byl správcem domu, Šebna, písař, a Jóach, syn Asafův, kancléř.
#36:4 Nejvyšší číšník na ně zavolal: „Povězte Chizkijášovi: Toto praví velkokrál, král asyrský: Na co vlastně spoléháš?
#36:5 Říkáš: ‚Pouhé slovo přinese radu i zmužilost k boji.‘ Nuže, na koho spoléháš, že se proti mně bouříš?
#36:6 Hle, spoléháš na Egypt, na tu nalomenou třtinovou hůl. Kdokoli se o ni opře, tomu pronikne dlaní a propíchne ji. Takový je farao, král egyptský, vůči všem, kteří na něho spoléhají.
#36:7 Řekneš mi snad: ‚Spoléháme na Hospodina, svého Boha‘, ale je známo, že Chizkijáš odstranil jeho posvátná návrší a jeho oltáře a že Judovi a Jeruzalému poručil: ‚Pouze před tímto oltářem se budete klanět.‘
#36:8 Vsaď se tedy nyní s mým pánem, asyrským králem. Dám ti dva tisíce koní, dokážeš-li k nim sehnat jezdce!
#36:9 Jak bys mohl odrazit jediného místodržitele z nejmenších služebníků mého pána, i když se spoléháš na Egypt pro jeho vozbu a jízdu!
#36:10 Cožpak jsem vytáhl bez Hospodina proti této zemi, abych ji zničil? Hospodin mi nařídil: Vytáhni do této země a znič ji!“
#36:11 Eljakím, Šebna a Jóach odpověděli nejvyššímu číšníkovi: „Mluv raději ke svým služebníkům aramejsky, vždyť rozumíme. Nemluv na nás judsky, aby to neslyšel lid, který je na hradbách.“
#36:12 Ale nejvyšší číšník odvětil: „Zdalipak mě můj pán poslal, abych mluvil tato slova k tvému pánovi a k tobě? Zdali ne k mužům, kteří obsadili hradby a budou jíst svá lejna a pít svou moč spolu s vámi?“
#36:13 Nejvyšší číšník se postavil a volal judsky co nejhlasitěji: „Slyšte slova velkokrále, krále asyrského!
#36:14 Toto praví král: ‚Ať vás Chizkijáš nepodvádí‘, protože vás nedokáže vysvobodit!
#36:15 Ať vás Chizkijáš nevede k spoléhání na Hospodina slovy: ‚Hospodin nás určitě vysvobodí a toto město nebude vydáno do rukou asyrského krále!‘
#36:16 Neposlouchejte Chizkijáše. Toto praví král asyrský: ‚Sjednejte se mnou dohodu a vyjděte ke mně. Každý z vás bude jíst ze své vinné révy a ze svého fíkovníku a pít vodu ze své cisterny,
#36:17 dokud nepřijdu a nevezmu vás do země stejné, jako je země vaše, do země obilí a moštu, do země chleba a vinic.‘
#36:18 Jen ať vás Chizkijáš nepodněcuje slovy: ‚Hospodin nás vysvobodí.‘ Zdali někdo z bohů pronárodů vysvobodil svou zemi z rukou asyrského krále?
#36:19 Kde byli bohové Chamátu a Arpádu? Kde byli bohové Sefarvajimu? Což vysvobodili z mých rukou Samaří?
#36:20 Který ze všech bohů těchto zemí vysvobodil svou zemi z mých rukou? Že by Hospodin vysvobodil z mých rukou Jeruzalém?“
#36:21 Oni však mlčeli. Ani slůvkem mu neodpovídali. Králův příkaz totiž zněl: „Neodpovídejte mu.“
#36:22 Eljakím, syn Chilkijášův, správce domu, Šebna, píšař, a Jóach, syn Asafův, kancléř, přišli k Chizkijášovi s roztrženým rouchem a oznámili mu slova nejvyššího číšníka. 
#37:1 Když to král Chizkijáš uslyšel, roztrhl své roucho, zahalil se žíněnou suknicí a vešel do Hospodinova domu.
#37:2 Poslal Eljakíma, který byl správcem domu, písaře Šebnu a starší z kněží, zahalené žíněnými suknicemi, k proroku Izajášovi, synu Amósovu.
#37:3 Měli mu vyřídit: „Toto praví Chizkijáš: Tento den je den soužení, trestání a ponižování; plod je připraven vyjít z lůna, ale rodička nemá sílu k porodu.
#37:4 Kéž Hospodin, tvůj Bůh, slyší slova nejvyššího číšníka, kterého poslal jeho pán, král asyrský, aby se rouhal Bohu živému. Kéž jej Hospodin, tvůj Bůh, potrestá za ta slova, která slyšel. Pozdvihni hlas k modlitbě za pozůstatek lidu, který tu je.“
#37:5 Služebníci krále Chizkijáše přišli k Izajášovi.
#37:6 Izajáš jim řekl: „Vyřiďte svému pánu: Toto praví Hospodin: Neboj se těch slov, která jsi slyšel, jimiž mě hanobili sluhové asyrského krále.
#37:7 Hle, uvedu do něho ducha, že uslyší zprávu a vrátí se do své země. V jeho zemi jej nechám padnout mečem.“
#37:8 Když se nejvyšší číšník vracel, uslyšel, že asyrský král odtáhl od Lakíše. Zastihl ho, jak bojuje proti Libně.
#37:9 Asyrský král totiž uslyšel o Tirhákovi, králi kúšském: „Vytáhl proti tobě do boje!“ Když to uslyšel, poslal k Chizkijášovi posly ze vzkazem:
#37:10 „Toto vyřiďte Chizkijášovi, králi judskému: Ať tě nepodvede tvůj Bůh, na něhož spoléháš, že Jeruzalém nebude vydán do rukou asyrského krále.
#37:11 Hle, slyšel jsi o tom, co učinili králové asyrští všem zemím, že je zničili jako klaté. A ty bys byl vysvobozen?
#37:12 Zda bohové těch pronárodů, jimž přinesli zkázu moji otcové, mohli vysvobodit Gozan, Cháran, Resef a syny Edenu, kteří byli v Telasáru?
#37:13 Kde je král Chamátu a král Arpádu a král města Sefarvajímu, Heny a Ivy?“
#37:14 Chizkijáš vzal dopisy z ruky poslů, přečetl je a pak vstoupil do Hospodinova domu a rozložil je před Hospodinem.
#37:15 Chizkijáš se modlil k Hospodinu:
#37:16 „Hospodine zástupů, Bože Izraele, který sídlíš nad cheruby, ty sám jsi Bůh nade všemi královstvími země. Ty jsi učinil nebesa i zemi.
#37:17 Nakloň, Hospodine, své ucho a slyš, otevři, Hospodine, své oči a viz! Slyš všechna slova Sancheríba, který vyslal posly, aby haněli Boha živého.
#37:18 Opravdu, Hospodine, králové asyrští zničili všechny země, pronárody i jejich zemi.
#37:19 Jejich bohy vydali ohni, protože to nejsou bohové, nýbrž dílo lidských rukou, dřevo a kámen, proto je zničili.
#37:20 Ale teď, Hospodine, Bože náš, zachraň nás z jeho rukou, ať poznají všechna království země, že ty jsi Hospodin, ty sám.“
#37:21 I vzkázal Izajáš, syn Amósův, Chizkijášovi: „Toto praví Hospodin, Bůh Izraele: Modlil ses ke mně kvůli Sancheríbovi, králi asyrskému.
#37:22 Toto je slovo, které o něm promluvil Hospodin: Pohrdá tebou, vysmívá se ti panna, dcera sijónská! Potřásá nad tebou hlavou dcera jeruzalémská.
#37:23 Koho jsi haněl a hanobil? Proti komu jsi povýšil hlas a oči pyšně vzhůru zvedl? Proti Svatému Izraele.
#37:24 Svými služebníky jsi pohaněl Panovníka, když jsi řekl: ‚Se svou nespočetnou vozbou jsem vytáhl do horských výšin, na libanonské stráně. Pokácím tam statné cedry, skvělé cypřiše, vstoupím až do jeho nejvyšších končin, do křovin a hájů.
#37:25 Já jsem vykopal studny a napil se vody, svými chodidly jsem vysušil všechny průplavy Egypta.‘
#37:26 Což jsi neslyšel, že zdávna jsem to připravoval, už za dnů dávnověkých to chystal? Nyní to uskutečním: V hromady sutin se změní opevněná města.
#37:27 Jejich bezmocní obyvatelé jsou vyděšení a zostuzení, jsou jako bylina polní, zelenající se býlí, tráva na střechách, rez v obilí nepožatém.
#37:28 Vím o tobě, ať sedíš či vycházíš a vcházíš, jak proti mně běsníš.
#37:29 Protože proti mně tak běsníš a tvá drzost stoupá do mých uší, provleču ti chřípím kruh a do úst vložím uzdu. Odvedu tě cestou, po níž jsi přišel.“
#37:30 Toto ti bude znamením, Chizkijáši: V tomto roce budete jíst, co vyroste samo, i druhý rok, co samo vzejde, ale třetí rok sejte a sklízejte, vysazujte vinice a jezte jejich plody.
#37:31 Ti z Judova domu, kteří vyváznou a zůstanou, opět se zakoření a vydají ovoce.
#37:32 Z Jeruzaléma vyjde pozůstatek lidu a z hory Sijónu ti, kdo vyvázli. Horlivost Hospodina zástupů to učiní.
#37:33 Proto praví Hospodin o králi asyrském toto: „Nevejde do tohoto města. Ani šíp tam nevstřelí, se štíty proti němu nenastoupí, násep proti němu nenavrší.
#37:34 Cestou, kterou přišel, se zase vrátí, do tohoto města nevejde, je výrok Hospodinův.
#37:35 Budu štítem tomuto městu, zachráním je kvůli sobě a kvůli Davidovi, svému služebníku.“
#37:36 Tu vyšel Hospodinův anděl a pobil v asyrském táboře sto osmdesát pět tisíc. Za časného jitra, hle, všichni byli mrtví, všude mrtvá těla.
#37:37 Sancheríb, král asyrský, odtáhl pryč a vrátil se do Ninive a usadil se tam.
#37:38 Když se klaněl v chrámu svého boha Nisrocha, Adramelek a Sareser, jeho synové, ho zabili mečem a unikli do země Araratu. Po něm se stal králem jeho syn Esarchadón. 
#38:1 V oněch dnech Chizkijáš smrtelně onemocněl. Přišel k němu prorok Izajáš, syn Amósův, a řekl mu: „Toto praví Hospodin: Udělej pořízení o svém domě, protože zemřeš, nebudeš žít.“
#38:2 Chizkijáš se otočil tváří ke zdi a takto se k Hospodinu modlil:
#38:3 „Ach, Hospodine, rozpomeň se prosím, že jsem chodil před tebou opravdově a se srdcem nerozděleným a že jsem činil, co je dobré v tvých očích.“ A Chizkijáš se dal do velikého pláče.
#38:4 Tu se k Izajášovi stalo slovo Hospodinovo:
#38:5 „Jdi a vyřiď Chizkijášovi: Toto praví Hospodin, Bůh Davida, tvého otce: ‚Vyslyšel jsem tvou modlitbu, viděl jsem tvé slzy. Hle, přidám k tvým dnům patnáct let.
#38:6 Vytrhnu tebe i toto město ze spárů asyrského krále. Budu tomuto městu štítem.‘
#38:7 Toto ti bude znamením od Hospodina, že Hospodin splní to slovo, jež promluvil:
#38:8 Hle, o deset stupňů nazpět vrátím sluncem vržený stín, který sestoupil po stupních Achazových.“ A slunce se vrátilo o deset stupňů na stupních, po nichž sestoupilo.
#38:9 Zápis Chizkijáše, krále judského, o jeho nemoci, a jak se ze své nemoci navrátil k životu:
#38:10 Již jsem si říkal: „Uprostřed svých dnů se do bran podsvětí odeberu, je mi odečten zbytek mých let.“
#38:11 Říkal jsem si: „Nikdy už nespatřím Hospodina, Hospodina v zemi živých, člověka už nikdy nezahlédnu, přiřadím se k obyvatelům té říše zapomnění.
#38:12 Moje obydlí je strženo a stěhuje se ode mne jak pastýřský stan. Svinul jsem svůj život jako tkadlec dílo. On sám odřízl mě od osnovy. Do dnešního večera jsi se mnou hotov!
#38:13 Stavěl jsem si před oči až do jitra, že mi jako lev rozdrtí všechny kosti. Do dnešního večera jsi se mnou hotov.
#38:14 Jako rorejs a jeřáb, tak sípám, kvílím jako holubice, zemdlely mé oči upírající se na výšinu. Panovníku, jsem v tísni, zasaď se o mne!
#38:15 Proč bych dále mluvil? Vždyť to, co mi řekl, učiní bezpochyby. Doputuji veškerá svá léta s hořkostí své duše.
#38:16 Panovníku, kvůli tomuhle se žije? Žil můj duch pro tohle vše, co zažil? Uzdrav mě a zachovej mi život!
#38:17 Hle, ta hořkost přehořká mi byla ku pokoji. Ty sám vytrhl jsi z jámy zániku mou duši, za sebe jsi odhodil všechny mé hříchy.
#38:18 Vždyť podsvětí nevzdává ti chválu, smrt tě nedovede chválit, ti, kdo sestoupili do jámy, už nevyhlížejí tvou věrnost.
#38:19 Živý, jenom živý, vzdá ti chválu jako já dnes, otec seznámí s tvou věrností své syny.
#38:20 Hospodine, buď mou spásou! Budeme hrát na strunné nástroje po všechny dny svého žití v Hospodinově domě.“
#38:21 Izajáš poručil: „Ať přinesou suché fíky, potřou mu vřed a zůstane naživu.“
#38:22 Chizkijáš se otázal: „Co bude znamením, že vstoupím do Hospodinova domu?“ 
#39:1 V ten čas poslal Meródak-baladán, syn Baladánův, král babylónský, Chizkijášovi dopisy a dar; uslyšel totiž, že onemocněl a zotavil se.
#39:2 Chizkijáš měl z nich radost a ukázal poslům svou klenotnici, stříbro a zlato, různé balzámy, výborný olej i celou svou zbrojnici a všechno, co se nacházelo mezi jeho poklady. Nebylo nic, co by jim Chizkijáš ve svém domě a v celém svém vladařství nebyl ukázal.
#39:3 Prorok Izajáš přišel ke králi Chizkijášovi a zeptal se ho: „Co říkali ti muži a odkud k tobě přišli?“ Chizkijáš odpověděl: „Přišli ke mně z daleké země, z Babylóna.“
#39:4 Zeptal se: „Co viděli v tvém domě?“ Chizkijáš odvětil: „Viděli všechno, co je v mém domě. Nebylo nic, co bych jim nebyl ze svých pokladů ukázal.“
#39:5 I řekl Izajáš Chizkijášovi: „Slyš slovo Hospodina zástupů!
#39:6 Hle, přijdou dny a bude odneseno do Babylóna všechno, co je v tvém domě, poklady, které nahromadili tvoji otcové až do tohoto dne. Nic tu nezbude, praví Hospodin.
#39:7 I některé tvé syny, kteří z tebe vzejdou, které zplodíš, vezmou a stanou se kleštěnci v paláci krále babylónského.“
#39:8 Chizkijáš na to Izajášovi řekl: „Dobré je slovo Hospodinovo, které jsi mluvil.“ A dodal: „Ať je za mých dnů opravdový pokoj.“ 
#40:1 „Potěšte, potěšte můj lid,“ praví váš Bůh.
#40:2 Mluvte k srdci Jeruzaléma, provolejte k němu: Čas jeho služby se naplnil, odpykal si své provinění. Vždyť z Hospodinovy ruky přijal dvojnásobně za všechny své hříchy.
#40:3 Hlas volajícího: „Připravte na poušti cestu Hospodinu! Vyrovnejte na pustině silnici pro našeho Boha!
#40:4 Každé údolí ať je vyvýšeno, každá hora a pahorek sníženy. Pahorkatina ať v rovinu se změní a horské hřbety v pláně.
#40:5 I zjeví se Hospodinova sláva a všechno tvorstvo společně spatří, že promluvila Hospodinova ústa.“
#40:6 Hlas toho, jenž praví: „Volej!“ I otázal se: „Co mám volat?“ „Všechno tvorstvo je tráva a všechna jeho spolehlivost jak polní kvítí.
#40:7 Tráva usychá, květ vadne, zavane-li na něj vítr Hospodinův. Věru, lid je pouhá tráva.
#40:8 Tráva usychá, květ vadne, ale slovo Boha našeho je stálé navěky.“
#40:9 Vystup si na horu vysokou, Sijóne, který neseš radostnou zvěst, co nejvíc zesil svůj hlas, Jeruzaléme, který neseš radostnou zvěst, zesil jej, neboj se! Řekni judským městům: „Hle, váš Bůh!
#40:10 Panovník Hospodin přichází s mocí, jeho paže se ujme vlády. Hle, svoji mzdu má s sebou, u sebe svůj výdělek.
#40:11 Jak pastýř pase své stádo, beránky svou paží shromažďuje, v náručí je nosí, březí ovečky šetrně vede.“
#40:12 Kdopak svou hrstí odměřil vodstvo a pídí nebesa změřil? Kdo shrnul v odměrku všechen prach země a hory odvážil na vahadlech, pahorky na vážkách?
#40:13 Kdo změřil Hospodinova ducha a byl mu rádce a vedl ho k poznání?
#40:14 Kohopak o radu žádal, aby rozumnosti nabyl, aby ho poučil o stezce práva, naučil poznání a seznámil ho s cestami rozumnosti?
#40:15 Hle, pronárody jsou jako kapka ve vědru, jak prášek na vahách. Ale on pozvedá ostrovy jak smítko.
#40:16 Dříví Libanónu by na oheň nestačilo, jeho zvěř by nestačila pro zápalnou oběť.
#40:17 Všechny pronárody nejsou před ním ničím, jsou mu méně nežli nic, než nicota.
#40:18 Ke komu připodobníte Boha? Jakou podobu mu přisoudíte?
#40:19 Řemeslník odleje modlu a zlatotepec ji potáhne zlatem, stříbrné řetízky přidělá zlatník.
#40:20 Chudák, který na takovou oběť nemá, vybere dřevo, které netrouchniví, a vyhledá zručného řemeslníka, aby mu zhotovil modlu, jež by se neviklala.
#40:21 Což o tom nevíte? Což jste neslyšeli? Neoznámili vám to už na počátku? Což nechápete, kdo položil základy zemi?
#40:22 Ten, který sídlí nad obzorem země, jejíž obyvatelé jsou jako kobylky, ten, který nebesa jak závoj roztahuje a napíná je jako stan k obývání.
#40:23 Ten hodnostáře za nic nemá a jako s nicotou nakládá se soudci země.
#40:24 Sotva byli zasazeni, sotva byli zaseti, sotva jejich odnož kořeny do země zapustila, zaduje na ně a oni schnou a vichr je odnáší jako slámu.
#40:25 „Ke komu mě chcete připodobnit, aby mi byl roven?“ praví Svatý.
#40:26 „K výšině zvedněte zraky a hleďte: Kdo stvořil toto všechno?“ Ten, který v plném počtu vyvádí zástupy hvězd a všechny volá jménem; má obrovskou sílu a úžasnou moc, nechybí mu ani jedna.
#40:27 Proč říkáš, Jákobe, proč, Izraeli, mluvíš takto: „Má cesta je Hospodinu skryta, můj Bůh přehlíží mé právo“?
#40:28 Cožpak nevíš? Cožpak jsi neslyšel? Hospodin, Bůh věčný, stvořitel končin země, není zemdlený, není znavený, jeho rozumnost vystihnout nelze.
#40:29 On dává zemdlenému sílu a dostatek odvahy bezmocnému.
#40:30 Mladíci jsou zemdlení a unavení, jinoši se potácejí, klopýtají.
#40:31 Ale ti, kdo skládají naději v Hospodina, nabývají nové síly; vznášejí se jak orlové, běží bez únavy, jdou bez umdlení. 
#41:1 „Zmlkněte přede mnou, ostrovy, národy ať se oblečou v sílu, ať se přiblíží, ať potom promluví, přistupme společně k soudu.
#41:2 Kdo vzbudil od východu muže, jemuž v patách kráčí spravedlnost? Poddává mu pronárody, podrobuje krále. Jeho mečem rozmetá je v prach, jeho lukem jako rozházenou slámu.
#41:3 Žene je a sám klidně projde stezkou, na kterou jeho noha dosud nevstoupila.
#41:4 Kdo to uskutečnil a vykonal? Ten, jenž od počátku povolává všechna pokolení. Já Hospodin jsem první, já budu též u posledních věcí.“
#41:5 Spatří to ostrovy a uleknou se, končiny země se roztřesou, přiblíží se a přijdou.
#41:6 Navzájem si budou pomáhat a bratr bude povzbuzovat bratra: „Buď rozhodný!“
#41:7 Řemeslník pobízí zlatníka, kovotepec kováře. Praví: „Spojme to! Bude to dobré!“ Pak upevní modlu hřeby, aby se neviklala.
#41:8 „A ty, Izraeli, služebníku můj, Jákobe, tebe jsem vyvolil, potomku Abrahama, mého přítele,
#41:9 tebe jsem vychvátil z končin země, zavolal jsem tě z odlehlých míst, řekl jsem ti: ‚Ty jsi můj služebník, tebe jsem vyvolil, nezavrhl jsem tě.‘
#41:10 Neboj se, vždyť já jsem s tebou, nerozhlížej se úzkostlivě, já jsem tvůj Bůh. Dodám ti odvahu, pomocí ti budu, budu tě podpírat pravicí své spravedlnosti.
#41:11 Hle, budou se stydět a hanbit všichni, kdo proti tobě planou hněvem, budou jak nic a zahynou odpůrci tvoji.
#41:12 Budeš je hledat, ale nenajdeš ty, kdo na tebe dorážejí; budou jak nic, naprostá nicota, ti, kdo proti tobě válčí.
#41:13 Já jsem Hospodin, tvůj Bůh, držím tě za pravici, pravím ti: ‚Neboj se, já jsem tvá pomoc.‘
#41:14 Neboj se, červíčku Jákobův, hrstko Izraelova lidu. Já jsem tvá pomoc, je výrok Hospodinův, tvůj vykupitel je Svatý Izraele.
#41:15 Hle, učiním tě okovaným smykem na mlácení, novým, z obou stran ostrým. Budeš mlátit a drtit hory, s pahorky naložíš jako s plevami.
#41:16 Rozevěješ je a odnese je vítr, bouřlivý vichr je rozptýlí, ty však budeš jásat k chvále Hospodina, budeš se chlubit Svatým Izraele.“
#41:17 „Utištění ubožáci hledají vodu, ale žádná není; jazyk jim žízní prahne. Já Hospodin jim odpovím, já, Bůh Izraele, je neopustím.
#41:18 Na holých návrších otevřu vodní proudy, uprostřed plání prameny vod, poušť v jezero změním a zemi vyprahlou ve vodní zřídla.
#41:19 V poušti dám vyrůst cedrům, akáciím, myrtě a olivám, na pustině vysadím cypřiš, platan i zimostráz spolu,
#41:20 aby viděli a poznali, přesvědčili se a všichni pochopili, že toto učinila ruka Hospodinova a že to stvořil Svatý Izraele.“
#41:21 „Předložte svůj spor, praví Hospodin, předstupte se svými důkazy, praví Král Jákobův.
#41:22 Předstupte a oznamte nám, co se má přihodit. Oznamte, co bylo na počátku, vezmem si to k srdci. Chceme poznat budoucnost, ohlaste nám, co má nastat.
#41:23 Oznamte, co bude následovat, ať poznáme, že vy jste bohové. Jen předveďte něco dobrého či zlého, podíváme se a společně to uvidíme.
#41:24 Hle, vy jste pouhé nic a vaše skutky nestojí za nic. Zvolit vás je ohavnost.“
#41:25 „Vzbudil jsem muže od severu a přišel od východu slunce. Vzývá mé jméno. Přichází! Knížata jsou jako hlína, on jako hrnčíř, jenž šlape jíl.
#41:26 Kdo oznámil to od začátku, abychom to věděli, oznámil kdysi dávno, abychom mohli říci: ‚Je to správné‘? Nikdo nic neoznámil, nikdo nic neohlásil, nikdo vaši řeč neslyšel.
#41:27 Já první zvěstuji Sijónu: ‚Hle, už tu jsou!‘ a Jeruzalému dám toho, jenž nese radostnou zvěst.
#41:28 Dívám se a nikde nikdo, žádný rádce mezi nimi. Zeptal bych se jich, co odpovědí.
#41:29 Hle, všichni jsou ničemnost pouhá, k ničemu nejsou jejich činy; jejich lité modly jsou jen vítr a nicota.“ 
#42:1 „Zde je můj služebník, jehož podepírám, můj vyvolený, v němž jsem našel zalíbení. Vložil jsem na něho svého ducha, aby vyhlásil soud pronárodům.
#42:2 Nekřičí a hlas nepozvedá, nedává se slyšet na ulici.
#42:3 Nalomenou třtinu nedolomí, nezhasí knot doutnající. Soud vyhlásí podle pravdy.
#42:4 Neochabne, nezlomí se, dokud na zemi soud nevykoná. I ostrovy čekají na jeho zákon.“
#42:5 Toto praví Bůh Hospodin, který stvořil nebesa a roztáhl je, zemi překlenul i s tím, co na ní vzchází, jenž dává dech lidu na ní a ducha těm, kdo po ní chodí:
#42:6 „Já Hospodin jsem tě povolal ve spravedlnosti a uchopil tě za ruku; budu tě opatrovat, dám tě za smlouvu lidu a za světlo pronárodům,
#42:7 abys otvíral slepé oči, abys vyváděl vězně ze žaláře, z věznic ty, kdo sedí v temnotě.
#42:8 Já jsem Hospodin. To je mé jméno. Svou slávu nikomu nedám, svou chválu nepostoupím modlám.
#42:9 Hle, už nastalo, co bylo na počátku, teď oznamuji nové věci. Dříve než vyraší, vám je ohlašuji.“
#42:10 Zpívejte Hospodinu píseň novou! Ať zní jeho chvála ze všech končin země. Ti, kteří se vydávají na moře, i to, čeho je moře plno, ostrovy a ti, kdo na nich bydlí,
#42:11 ať pozvednou svůj hlas, i poušť a její města, dvorce, v nichž sídlí Kédar, ať plesají obyvatelé Sely a výskají z vrcholků hor.
#42:12 Ať vzdají čest Hospodinu, ať mu chvalořečí na ostrovech.
#42:13 Hospodin jak bohatýr se vyrazit už chystá, svou horlivost jak bojovník rozněcuje, pozvedá bojový pokřik, proti nepřátelům svým si bohatýrsky vede.
#42:14 „Byl jsem příliš dlouho zticha. Mám dál mlčet? Přemáhat se? Mám jen sténat, těžce jak rodička vzdychat a přitom po dechu lapat?
#42:15 Hory a pahorky postihnu suchem, nechám uschnout každou rostlinu na nich. V ostrovy proměním řeky, jezera vysuším.
#42:16 Slepé povedu cestou, již neznají, stezkami, o nichž nic nevědí, je budu vodit. Tmu před nimi změním v světlo, pahorkatiny v rovinu. Toto jsou věci, jež učiním, od toho neupustím.
#42:17 Ustoupí nazpět politi studem ti, kteří doufají v tesané modly, kdo říkají slitým modlám: ‚Jste naši bohové‘.“
#42:18 „Slyšte, hluší, prohlédněte, slepí, ať vidíte!
#42:19 Kdo byl slepý, ne-li můj služebník? Kdo byl hluchý jako můj posel, jehož jsem vyslal? Kdo byl slepý jako ten, za něhož jsem zaplatil, slepý jak služebník Hospodinův?
#42:20 Mnoho jsi viděl, ale nedbal jsi na to. S ušima otevřenýma neslyšel nic.“
#42:21 Pro vlastní spravedlnost se Hospodinu zalíbilo vyvýšit a zvelebit zákon.
#42:22 Lid je však oloupený a popleněný, jinoši všichni spoutáni, vsazení do vězení. Stali se loupežnou kořistí a není, kdo by vysvobodil, jsou vydáni v plen a není, kdo by řekl: „vrať to.“
#42:23 Kdo z vás tomu dopřeje sluchu? Kdo pozorně vyslechne, co bude následovat?
#42:24 Kdo vydal Jákoba v plen? A Izraele těm, kteří loupí? Což to nebyl Hospodin, proti němuž jsme zhřešili? Nechtěli jsme chodit po jeho cestách, jeho zákon jsme neposlouchali.
#42:25 Proto na něj vylil své hněvivé rozhořčení, krutou válku; plameny vyšlehly, a Jákob nic nepoznal, hořelo, a nebral si to k srdci. 
#43:1 Nyní toto praví Hospodin, tvůj stvořitel, Jákobe, tvůrce tvůj, Izraeli: „Neboj se, já jsem tě vykoupil, povolal jsem tě tvým jménem, jsi můj.
#43:2 Půjdeš-li přes vody, já budu s tebou, půjdeš-li přes řeky, nestrhne tě proud, půjdeš-li ohněm, nespálíš se, plamen tě nepopálí.
#43:3 Neboť já Hospodin jsem tvůj Bůh, Svatý Izraele, tvůj spasitel. Jako výkupné jsem dal za tebe Egypt, Kúš a Sebu dal jsem místo tebe.
#43:4 Protože jsi v očích mých tak drahý, vzácný, protože jsem si tě zamiloval, dám za tebe mnohé lidi a národy za tvůj život.
#43:5 Neboj se, já budu s tebou. Tvé potomstvo přivedu od východu, shromáždím tě od západu.
#43:6 Severu poručím: ‚Vydej!‘ a jihu: ‚Nezadržuj!‘ Přiveď mé syny zdaleka a mé dcery od končin země,
#43:7 každého, kdo se nazývá mým jménem a koho jsem stvořil ke své slávě, koho jsem vytvořil a učinil.“
#43:8 Vyveď ten lid slepý, ačkoli má oči, ty hluché, ač mají uši.
#43:9 Ať se vespolek shromáždí všechny pronárody, národy ať se spolu sejdou. Kdo z nich co oznámí? Kdo z nich nám ohlásí, co bylo na počátku? Ať postaví své svědky a ospravedlní se! Lidé o tom uslyší a řeknou: „Je to pravda.“
#43:10 „Mými svědky jste vy, je výrok Hospodinův, a můj služebník, jehož jsem vyvolil. Tak mě poznáte a uvěříte mi a pochopíte, že to jsem já. Přede mnou nebyl vytvořen Bůh a nebude ani po mně.
#43:11 Já, já jsem Hospodin, kromě mne žádný spasitel není.
#43:12 Já jsem oznamoval a zachraňoval, já jsem ohlašoval, a nikdo cizí mezi vámi. Jste moji svědkové, je výrok Hospodinův, a já jsem Bůh!
#43:13 Já jsem byl dřív než první den a není, kdo by z mých rukou vysvobodil. Kdo zvrátí, co já vykonám?“
#43:14 Toto praví Hospodin, váš vykupitel, Svatý Izraele: „Kvůli vám pošlu do Babylóna a srazím všechny ty útočné hady, Kaldejce, plesající na lodicích.
#43:15 Já Hospodin jsem váš Svatý, stvořitel Izraele, váš Král.“
#43:16 Toto praví Hospodin, který razí cestu mořem, dravými vodami stezku,
#43:17 jenž přivádí vozbu i koně, vojsko a válečnou moc i dokáže, že pospolu lehnou a nepovstanou, dohořeli, zhasli jako knot.
#43:18 „Nevzpomínejte na věci dřívější, o minulosti nepřemítejte.
#43:19 Hle, činím něco docela nového a už to raší. Nevíte o tom? Já povedu pouští cestu, pustou krajinou řeky.
#43:20 Čest mi vzdá zvěř pole, šakalové i pštrosi; obdařil jsem poušť vodou a pustou krajinu řekami, abych napojil svůj vyvolený lid.
#43:21 Lid, jejž jsem vytvořil pro sebe, ten bude vyprávět o mých chvályhodných činech.“
#43:22 „Nebyl jsem to já, koho jsi vzýval, Jákobe, kvůli mně ses neobtěžoval, Izraeli!
#43:23 Své jehňátko v oběť zápalnou jsi pro mne nepřivedl, nectil jsi mě svými obětními hody. Nenutil jsem tě sloužit mi obětními dary, ani kadidlem jsem tě neobtěžoval.
#43:24 Vonné koření jsi mi za stříbro nenakoupil, tukem svých obětí jsi mě neobčerstvil. Jenom svými hříchy chtěls mě nutit k službě, svými nepravostmi jsi mě obtěžoval.
#43:25 Já, já sám vymažu kvůli sobě tvoje nevěrnosti, na tvé hříchy nevzpomenu.
#43:26 Připomeň mi to, můžem se spolu soudit, sám si spočítej, pro co bys mohl být ospravedlněn.
#43:27 Tvůj otec byl první, který zhřešil, tvoji mluvčí mi byli nevěrní.
#43:28 Správce svatyně jsem zbavil posvěcení, Jákoba jsem vydal klatbě, Izraele zhanobení.“ 
#44:1 „A nyní slyš, Jákobe, můj služebníku, Izraeli, jehož jsem vyvolil.
#44:2 Toto praví Hospodin, který tě učinil, který tě vytvořil v životě matky a pomáhá ti: Neboj se, Jákobe, můj služebníku, Ješurúne, jehož jsem vyvolil.
#44:3 Já vyleji vody v místa zprahlá žízní, bystřiny na suchou zemi. Já vyleji svého ducha na tvé potomstvo a své požehnání na ty, kteří z tebe vzejdou.
#44:4 Porostou jak mezi trávou, budou jako topoly při tekoucích vodách.
#44:5 Onen řekne: ‚Já jsem Hospodinův‘ a jiný se nazve jménem Jákobovým, další si napíše na ruku: ‚Jsem Hospodinův‘ a dá si čestné jméno Izrael.“
#44:6 Toto praví Hospodin, král Izraele, jeho vykupitel, Hospodin zástupů: „Já jsem první i poslední, kromě mne žádného Boha není.
#44:7 Kdo je jako já? Jen ať se ozve! Ať to oznámí, ať mi to předloží! Od chvíle, kdy jsem navěky ustavil svůj lid, kdo mu oznamuje, co přijde a co má nastat?
#44:8 Nestrachujte se! Nebuďte zmateni. Což jsem ti vše neohlašoval a neoznamoval už předem? Vy jste moji svědkové. Což je Bůh kromě mne? Jiné skály není, já o žádné nevím.“
#44:9 Výrobci model jsou všichni k ničemu; modly, které mají za žádoucí, jim nijak neprospějí. Jejich svědkové, ti nic nevidí a nic nevědí, jsou jen pro ostudu.
#44:10 Kdo si boha vyrábí, jen modlu si odlévá; nebude z toho mít žádný prospěch.
#44:11 A všechny jeho společníky poleje stud. Řemeslníci jsou pouzí lidé. Ať se sem všichni shromáždí, ať se tu postaví; propadnou strachu a zastydí se spolu.
#44:12 Kovář kuje modlu, pracuje při žhoucím uhlí, kladivem jí dává tvar, zhotovuje ji silou své paže; přitom hladoví do vysílení, nemůže se ani napít, ač umdlévá.
#44:13 Tesař natahuje šňůru, rudkou načrtává modlu, opracovává ji dláty, rozměřuje kružidlem, až jí dá podobu muže, honosný vzhled člověka, a usadí ji v domě.
#44:14 Dá si porazit cedry nebo vezme dub, drnák či křemelák, vypěstoval si je mezi lesními stromy. Zasadil jasan a déšť mu dal vzrůst.
#44:15 Pro člověka je to na topení, něco z toho vezme a ohřeje se. Buď zatopí a upeče chleba, nebo zhotoví boha a klaní se mu, udělá z toho modlu a hrbí se před ní.
#44:16 Polovinu spálí pod masem, které bude jíst, upeče pečeni a nasytí se. Přitom se ohřeje a zvolá: „To jsem se zahřál, je mi teplo!“
#44:17 Z toho, co zůstalo, si udělá boha, modlu, před kterou se hrbí a jíž se klaní a k níž se modlí a žadoní: „Vysvoboď mě, jsi přece můj bůh!“
#44:18 Nevědí nic a nechápou, mají oči zalepené a nevidí, jejich srdce nevnímá.
#44:19 Ale on si to nebere k srdci, chybí mu poznání a rozum, aby si řekl: „Polovinu jsem spálil, na žhavém uhlí jsem napekl chleba a upekl maso a jedl jsem, ze zbytku jsem udělal tuhle ohavnost a hrbím se před špalkem.“
#44:20 Kdo se zabývá popelem, toho zavádí bláhovost srdce, ten svůj život nevysvobodí, ani se neptá: „Není to, co mám v pravici, pouhý klam?“
#44:21 „Pamatuj na to, Jákobe, pamatuj, Izraeli, že jsi můj služebník. Já jsem tě utvořil, jsi můj služebník, Izraeli, u mne nebudeš zapomenut.
#44:22 Zaženu tvou nevěru jak mračno a jako oblak tvé hříchy. Navrať se ke mně, já tě vykoupím.“
#44:23 Plesejte, nebesa, vždyť Hospodin to vykoná, hlaholte, nejhlubší útroby země, ať zvučně plesají hory, les a všechny jeho stromy, neboť Hospodin vykoupí Jákoba a proslaví se v Izraeli.
#44:24 Toto praví Hospodin, tvůj vykupitel, který tě vytvořil v životě matky: „Já jsem Hospodin, já konám všechno: sám nebesa roztahuji, překlenul jsem zemi. Kdo byl se mnou?“
#44:25 Hospodin ruší znamení žvanilů a z věštců činí pomatence, obrací mudrce nazpět a jejich poznání mate.
#44:26 Potvrzuje slovo svého služebníka, plní rozhodnutí ohlášené svými posly. Jeruzalému on praví: „Budeš osídlen!“ a judským městům: „Budete zbudována! Pozvednu jej z trosek.“
#44:27 Hlubině on praví: „Vyschni! Vysuším tvé vodní proudy.“
#44:28 O Kýrovi praví: „Hle, můj pastýř. Vyplní každé mé přání. Řekne Jeruzalému: ‚Budeš vybudován!‘ A chrámu: ‚Budeš založen!‘“ 
#45:1 Toto praví Hospodin o svém pomazaném, o Kýrovi: „Já jsem ho uchopil za pravici, pošlapu před ním pronárody, rozvážu opasky na bedrech králů, zotevírám před ním vrata, brány už nebudou zavírány.
#45:2 Já půjdu před tebou, vyrovnám nerovnosti, rozrazím bronzová vrata, železné závory zlomím.
#45:3 Tobě dám poklady v temnotě skryté i sklady nejtajnější a poznáš, že já jsem Hospodin, který tě volá jménem, Bůh Izraele.
#45:4 Kvůli svému služebníku Jákobovi, kvůli Izraeli, vyvolenci svému, jsem tě zavolal tvým jménem; dal jsem ti čestné jméno, ač jsi mě neznal.
#45:5 Já jsem Hospodin a jiného už není, mimo mne žádného Boha není. Přepásal jsem tě, ač jsi mě neznal,
#45:6 aby poznali od slunce východu až na západ, že kromě mne nic není. Já jsem Hospodin a jiného už není.
#45:7 Já vytvářím světlo a tvořím tmu, působím pokoj a tvořím zlo, já Hospodin konám všechny tyto věci.“
#45:8 Nebesa, vydejte krůpěje shůry, ať kane z oblaků spravedlnost; nechť se otevře země a urodí se spása a spravedlnost vyraší s ní. „Já Hospodin to stvořím.“
#45:9 Běda tomu, kdo se chce přít se svým tvůrcem, střep z hliněných střepů! Což smí hlína říci svému tvůrci: „Co to děláš?“ a tvůj výrobek: „On nemá ruce“?
#45:10 Běda tomu, kdo by řekl otci: „Cos to zplodil?“ Nebo ženě: „Cos to porodila?“
#45:11 Toto praví Hospodin, Svatý Izraele, jeho tvůrce: „Ptejte se mě na ty věci, které přijdou! Ohledně mých synů, díla mých rukou, budete mi něco přikazovat?
#45:12 Zemi jsem učinil já a člověka na ní jsem stvořil. Já jsem vlastníma rukama roztáhl nebesa a veškerému jejich zástupu jsem vydal příkazy.
#45:13 Já jsem jej vzbudil ke spravedlnosti a napřímím všechny jeho cesty. On zbuduje moje město, propustí mé přesídlence bez placení, bez úplatku,“ praví Hospodin zástupů.
#45:14 Toto praví Hospodin: „Co vytěžil Egypt a co vyzískala země Kúš, přejde k tobě a bude tvoje; i Sebajci, muži obrovití, půjdou ze tebou, přijdou v řetězech, budou se ti klanět a předkládat ti prosby. Jenom u tebe je Bůh a jiného už není, bohové nejsou nic.“
#45:15 Věru, ty jsi Bůh skrytý, Bůh Izraele, spasitel.
#45:16 Všechny je poleje stud a hanba, společně odtáhnou zahanbeni ti, kdo zhotovují modlářské výtvory.
#45:17 Izrael je spasen Hospodinem spásou věčnou. Stydět a hanbit se nebudete navěky, nikdy.
#45:18 Toto praví Hospodin, stvořitel nebe, onen Bůh, jenž vytvořil zemi, jenž ji učinil, ten, jenž ji upevnil na pilířích; nestvořil ji, aby byla pustá, vytvořil ji k obývání: „Já jsem Hospodin a jiného už není.
#45:19 Nemluvil jsem potají, v temném místě země, neřekl jsem potomstvu Jákobovu: Hledejte mě v pustotě. Já Hospodin vyhlašuji spravedlnost, prohlašuji právo.
#45:20 Shromážděte se a přijďte, přibližte se spolu, vy, kdo jste vyvázli z pronárodů. Neuvědomují si, že se nosí s modlou vytesanou ze dřeva, že se modlí k bohu, jenž nemůže spasit.
#45:21 Sdělte to dál, přibližte se! Jen ať se poradí spolu. Kdo to od pradávna ohlašoval? Kdo to oznamoval předem? Cožpak ne já, Hospodin? Kromě mne jiného Boha není! Bůh spravedlivý a spasitel není mimo mne.
#45:22 Obraťte se ke mně a dojdete spásy, veškeré dálavy země. Já jsem Bůh a jiného už není.
#45:23 Při sobě samém jsem přísahal, z mých úst vyšla spravedlnost, slovo, které se zpět nenavrátí. Přede mnou každý klesne na kolena a každý jazyk odpřisáhne:
#45:24 ‚Jenom v Hospodinu - řekne o mně - je spravedlnost i moc‘.“ Přijdou k němu a budou se stydět všichni ti, kdo proti němu pláli vzdorem.
#45:25 U Hospodina nalezne spravedlnost a jím se bude chlubit všechno potomstvo Izraele. 
#46:1 Bél klesl, Nébo se zhroutil, jejich modlářské stvůry byly naloženy na zvěř a na dobytek, ten náklad, který jste nosili, to břímě k zemdlení.
#46:2 Zvířata se zhroutila, klesla spolu, břímě nebyla s to zachránit; oni musí do zajetí.
#46:3 „Slyšte mě, dome Jákobův, všichni, kdo jste zůstali domu Izraelovu, vy, kteří jste chováni již od života matky, od matčina lůna na rukou nošeni:
#46:4 Já sám až do vašeho stáří, až do šedin vás budu nosit. Já jsem vás učinil a já vás ponesu, budu vás nosit a zachráním.“
#46:5 „Komu mě připodobníte, a s kým mě srovnáte, komu mě přirovnáte, abychom si byli podobni?
#46:6 Z měšce sypou zlato a na váze odvažují stříbro, najmou si zlatníka, aby jim udělal boha a pak se hrbí a klanějí.
#46:7 Nosí ho na rameni, přenášejí ho, postaví jej na podstavec, a on stojí, ze svého místa se nehne. Úpějí k němu, ale on se jim neozve, z jejich soužení je nezachrání.
#46:8 Připomeňte si to, vzmužte se, vezměte si to k srdci, nevěrníci.
#46:9 Pamatujte na to, co bylo na počátku od věků! Já jsem Bůh a jiného už není, jsem Bůh a nic není jako já.
#46:10 Od počátku oznamuji, co se v budoucnu stane, od pradávna, co se ještě nestalo. Pravím: Moje rozhodnutí platí a co se mi líbí, uskutečním.
#46:11 Od východu povolávám dravce, ze vzdálené země muže své volby. Jak jsem slíbil, tak to uskutečním, jak jsem si předsevzal, tak to splním.
#46:12 Naslouchejte mi, vy zarputilci, vzdálení od spravedlnosti:
#46:13 Přiblížil jsem svoji spravedlnost, daleko už není, nebude už prodlévat má spása. Obdařím Sijón spásou, oslavím se v Izraeli.“ 
#47:1 Sestup a seď v prachu, panno, dcero babylónská! Seď na zemi, připravena o trůn, kaldejská dcero! Už tě nebudou nazývat změkčilou a zhýčkanou.
#47:2 Chop se mlýnku a mel mouku! Odkryj závoj, zvedni vlečku, odkryj nohy, přebroď řeky!
#47:3 Ať se odkryje tvá nahota, ať se ukáže tvá hanba. Vykonám pomstu, nikdo mi nezabrání.
#47:4 Náš vykupitel je Svatý Izraele. Jeho jméno je Hospodin zástupů.
#47:5 Seď zticha, vstup do tmy, kaldejská dcero! Už nikdy tě nenazvou paní královských říší.
#47:6 Na svůj lid jsem se rozlítil, své dědictví znesvětit jsem nechal, vydal jsem je do tvé ruky. Neměla jsi s nimi slitování, svým jhem jsi i starce velmi obtížila.
#47:7 Řekla jsi: „Navěky budu paní!“ Nikdy sis nic k srdci nebrala, nepamatovalas na poslední věci.
#47:8 Proto slyš teď toto, požitkářko, jež trůníš v bezpečí a v srdci si namlouváš: „Nade mne už není! Já nebudu sedět jako vdova, nepoznám, co je to ztratit děti.“
#47:9 Dolehne na tebe obojí naráz, v jediném dni, bezdětnost i vdovství; dolehnou na tebe plnou měrou navzdory spoustě tvých kouzel a nesčíslným zaklínadlům.
#47:10 Ve své zlobě cítila ses bezpečná a řeklas: „Nikdo mě nevidí.“ Tvá moudrost a tvé vědění tě svedly na scestí. Říkala sis v srdci: „Jsem jenom já a nikdo víc už není.“
#47:11 Dolehne na tebe zlo a nebudeš je umět odčarovat, postihne tě neštěstí a zažehnat je nedokážeš. Dolehne na tebe náhlý zmar, ani si to nestačíš uvědomit.
#47:12 Postav se tu se svým zaklínámím, se spoustou svých kouzel, jimiž se od mládí zaměstnáváš! Snad budeš mít úspěch, snad naženeš strachu.
#47:13 Zmalátnělas přes svá velká rozhodnutí. Jen ať se postaví a zachrání tě ti, kdo pozorují nebesa, kdo zírají na hvězdy, kdo při novoluní uvádějí ve známost, co by tě mohlo potkat.
#47:14 Hle, jsou jako sláma, spálí je oheň. Ti nevysvobodí z moci plamene ani vlastní život. Nezbude ani žhavé uhlí pro ohřátí ani oheň, u něhož by bylo možno sedět.
#47:15 Tak dopadnou ti, jimiž se zaměstnáváš, s nimiž obchoduješ už od svého mládí. Každý z nich na svých cestách zbloudí, nespatří tě nikdo. 
#48:1 Slyšte to, dome Jákobův, vy, kdo jménem Izraelovým se nazýváte a z vod Judových jste vyšli, vy, kdo při Hospodinově jménu přísaháte, připomínáte si Boha Izraele, ne však pravdivě a spravedlivě.
#48:2 Podle města svatého se nazývají, hledají oporu v Bohu Izraele, jehož jméno je Hospodin zástupů.
#48:3 „Co se stalo na počátku, oznámil jsem předem, z mých úst to vyšlo, ohlásil jsem to; náhle jsem zasáhl a stalo se to.
#48:4 Přestože jsem věděl, jak jsi zatvrzelý, že tvá šíje je železná spona a tvé čelo z bronzu,
#48:5 oznamoval jsem ti všechno předem; dříve než co nastalo, jsem ti to ohlašoval, abys neříkal: ‚To udělala má modlářská stvůra, má tesaná a moje litá modla dala k tomu příkaz.‘
#48:6 Slyšels o tom. Teď se na to všechno dívej! A co vy? Což také něco oznámíte? Od nynějška ti chci ohlašovat nové věci, utajené, že o nich nic nevíš.
#48:7 Právě teď jsou stvořeny, nikoli předtím, přede dneškem jsi o nich nic neslyšel a říci nemůžeš: ‚Hle, já to znám.‘
#48:8 Nic jsi neslyšel, nic jsi nepoznal, tvé ucho předtím nic nezaslechlo. Vím, že jsi věrolomník věrolomný, říká se ti přece nevěrník už ze života matky;
#48:9 zdržoval jsem svůj hněv pro své jméno, pro svou chválu jsem se kvůli tobě krotil, nevyhladil jsem tě.
#48:10 Hle, přetavil jsem tě, ne však jako stříbro, vyzkoušel jsem tě v tavící peci utrpení.
#48:11 Kvůli sobě, kvůli sobě samému to dělám. Což smí být mé jméno znesvěceno? Svou slávu nikomu nedám.“
#48:12 „Poslouchej mě, Jákobe, Izraeli, můj povolaný. Já, já jsem ten první, já jsem i poslední.
#48:13 Ano, moje ruka založila zemi, má pravice nebesa rozvinula; zavolám-li na ně, stanou tady spolu.
#48:14 Vy všichni se shromažďte a slyšte: Kdo z nich oznámil tyto věci? Hospodin jej miluje, on splní jeho záměr proti Babylónu; jeho paže zasáhne Kaldejce.
#48:15 Já sám jsem to prohlásil, já jsem ho povolal, přivedl jsem ho a jeho cesta bude zdárná.
#48:16 Přistupte ke mně a slyšte toto: Od počátku nemluvím potají, od chvíle, kdy se to začalo dít, jsem při tom.“ A nyní mne posílá Panovník Hospodin a jeho duch.
#48:17 Toto praví Hospodin, tvůj vykupitel, Svatý Izraele: „Já jsem Hospodin, tvůj Bůh. Já tě vyučuji tomu, co je ku prospěchu, uvádím tě na cestu, po níž půjdeš.
#48:18 Jen kdybys dal pozor na má přikázání, byl by tvůj pokoj jako řeka a tvá spravedlnost jako mořské vlny,
#48:19 tvého potomstva by bylo jako písku a těch, kteří vzejdou z tvého lůna, jako jeho zrnek; nebylo by vyťato a zahlazeno přede mnou jeho jméno.“
#48:20 „Odejděte z Babylóna, prchněte od Kaldejců! S plesáním to oznamujte, všude rozhlašujte, rozneste to až do končin země, rcete: ‚Hospodin vykoupil Jákoba, svého služebníka.‘
#48:21 Nebudou žíznit, až je povede vyschlými kraji, vodu ze skály jim vyvede jak bystřinu, rozpoltí skálu a v hojnosti potekou vody.“
#48:22 „Nemají pokoj svévolníci,“ praví Hospodin. 
#49:1 Slyšte mě, ostrovy, daleké národy, dávejte pozor! Hospodin mě povolal z života mateřského; od nitra mé matky připomínal moje jméno.
#49:2 Učinil má ústa ostrým mečem, skryl mě ve stínu své ruky. Udělal ze mne výborný šíp, ukryl mě ve svém toulci.
#49:3 Řekl mi: „Ty jsi můj služebník, Izrael, v tobě se oslavím.“
#49:4 Já jsem však řekl: „Nadarmo jsem se namáhal, svou sílu jsem vydal pro nicotný přelud.“ A přece: U Hospodina je mé právo, můj výdělek u mého Boha.
#49:5 A nyní praví Hospodin, který mě vytvořil jako svého služebníka už v životě matky, abych k němu přivedl Jákoba nazpět, byť i nebyl shromážděn Izrael. Stal jsem se váženým v Hospodinových očích, můj Bůh je záštita moje.
#49:6 On dále řekl: „Nestačí, abys byl mým služebníkem, který má pozvednout Jákobovy kmeny a přivést zpátky ty z Izraele, kdo byli ušetřeni; dal jsem tě za světlo pronárodům, abys byl spása má do končin země.“
#49:7 Toto praví Hospodin, vykupitel Izraele, jeho Svatý, tomu, který je v opovržení, jehož má kdejaký pronárod v ohavnosti, služebníku vládců: „Spatří tě králové a povstanou, a velmožové se skloní, kvůli Hospodinu, který je věrný, kvůli Svatému Izraele, který tě vyvolil.“
#49:8 Toto praví Hospodin: „Odpovím ti v čase přízně, pomohu ti v den spásy, bude tě opatrovat, dám tě za smlouvu lidu, abys pozvedl zemi a zpustošená dědictví vrátil,
#49:9 abys řekl vězňům: ‚Vyjděte‘, těm, kdo jsou v temnotách: ‚Ukažte se!‘ Při cestách se budou pást a na všech holých návrších naleznou pastvu.
#49:10 Nebudou hladovět ani žíznit, nebude je ubíjet sálající step a sluneční žár, neboť je povede ten, jenž se nad nimi slitovává, a dovede je ke zřídlům vod.
#49:11 Na všech svých horách učiním cestu, mé silnice budou vyvýšeny.
#49:12 Hle, jedni přijdou zdaleka, jiní od severu a jiní od moře a jiní ze země Síňanů.“
#49:13 Plesejte, nebesa, a jásej, země, ať zvučně plesají hory, vždyť Hospodin potěšil svůj lid, slitoval se nad svými ujařmenými!
#49:14 Sijón říkával: „Hospodin mě opustil, Panovník na mě zapomenul.“
#49:15 Cožpak může zapomenout žena na své pacholátko, neslitovat se nad synem vlastního života? I kdyby některé zapomněly, já na tebe nezapomenu.
#49:16 Hle, vyryl jsem si tě do dlaní, tvé hradby mám před sebou stále.
#49:17 Tvoji synové už pospíchají. Ti, kdo tě bořili a ničili, od tebe odtáhnou.
#49:18 Rozhlédni se kolem a viz: Tito všichni se shromáždí a přijdou k tobě. Jakože živ jsem já, je výrok Hospodinův, jimi všemi se okrášlíš jako okrasou, ozdobíš se jimi jako nevěsta.
#49:19 Tvé trosky a tvá zpustošená města, tvá pobořená země ti teď budou příliš těsné pro množství obyvatel, až odtáhnou ti, kdo tě hubili.
#49:20 Opět ti budou říkat synové tvé bezdětnosti: ‚Toto místo je mi těsné! Dej mi prostor, ať mám kde sídlit.‘
#49:21 V srdci si řekneš: ‚Kdo mi zplodil tyto syny? Byla jsem bez dětí, bez manžela, přestěhovaná a zapuzená. Kdo je vychoval? Zůstala jsem zcela sama, kde se tito vzali?‘“
#49:22 Toto praví Panovník Hospodin: „Hle, pokynu rukou pronárodům, k národům vztáhnu svou korouhev; v náručí přinesou tvé syny, tvé dcery budou neseny na ramenou.
#49:23 Králové budou tvými pěstouny a jejich kněžny tvými chůvami. Klanět se ti budou tváří k zemi, budou lízat prach tvých nohou. I poznáš, že já jsem Hospodin, že se nezklamou ti, kdo skládají svou naději ve mne.“
#49:24 Lze vzít bohatýru to, co pobral? Může uniknout zajatý lid Spravedlivého?
#49:25 Ale Hospodin praví toto: „I bohatýru je možno vzít zajatce, co pobral ukrutník, může uniknout. Já s tvými odpůrci povedu spor, já spásu dám tvým synům.
#49:26 Tvé utiskovatele nakrmím jejich vlastním masem, jako moštem se opojí vlastní krví. I pozná všechno tvorstvo, že já Hospodin jsem tvůj spasitel a tvůj vykupitel, Přesilný Jákobův.“ 
#50:1 Toto praví Hospodin: „Kde je rozlukový lístek vaší matky, jímž jsem ji vyhostil? Anebo je tu někdo z mých věřitelů, jemuž jsem vás prodal? Hle, byli jste prodáni svou vlastní vinou, vaše matka byla vyhoštěna jen pro vaše nevěrnosti.
#50:2 Když jsem přicházel, proč tu nikdo nebyl? Volal jsem, a nikdo se neozýval. Což je má ruka tak krátká k vykoupení? Nemá k vysvobození dost síly? Hle, svou pohrůžkou vysušuji moře a řeky obrací v poušť; ryby v nich páchnou, protože není vody, a lekají žízní.
#50:3 Nebesa odívám v pochmurnou temnotu a přikrývám je žíněným rouchem.“
#50:4 Panovník Hospodin dal mi jazyk učedníků, abych uměl zemdleného podpírat slovem. On mě probouzí každého jitra, probouzí mi uši, abych slyšel jako učedníci.
#50:5 Panovník Hospodin mi otevřel uši a já nevzdoruji ani neuhýbám nazpět.
#50:6 Nastavuji záda těm, kteří mě bijí, a své líce těm, kdo rvou mé vousy, neukrývám svou tvář před potupami a popliváním.
#50:7 Panovník Hospodin je moje pomoc, proto nemohu být potupen, proto tvář svou nastavuji, jako kdyby byla z křemene, a vím, že nebudu zahanben.
#50:8 Blízko je ten, jenž mi zjedná spravedlnost. Kdo chce vést se mnou spory? Postavme se spolu! Kdo bude můj odpůrce na soudu? Ať ke mně přistoupí!
#50:9 Hle, Panovník Hospodin je moje pomoc. Kdo mě chce obvinit ze svévole? Hle, ti všichni zvetší jako šat, sežerou je moli.
#50:10 Kdo z vás se bojí Hospodina a poslouchá jeho služebníka? Kdo chodí v temnotách, kde není žádná záře, ten ať doufá v Hospodinovo jméno a opře se o svého Boha.
#50:11 Hle, vy všichni, kdo rozděláváte oheň, jiskrami se opásáte k boji. Jen choďte ve svitu svého ohně, v jiskrách, které jste zanítili! Má ruka vám způsobí, že ulehnete v trápení. 
#51:1 Slyšte mě, kdo usilujete o spravedlnost, kdo hledáte Hospodina. Pohleďte na skálu, z níž jste vytesáni, na hlubokou jámu, z níž jste vykopáni.
#51:2 Pohleďte na Abrahama, svého otce, a na Sáru, která vás v bolestech porodila. Jeho jediného jsem povolal, požehnal jsem mu a rozmnožil jsem ho.
#51:3 Hospodin potěší Sijón, potěší všechna jeho místa, jež jsou v troskách. Jeho poušť změní, bude jako Eden, jeho pustina jako zahrada Hospodinova. Bude v něm veselí a radost, vzdávání díků a prozpěvování.
#51:4 „Věnuj mi pozornost, můj lide, můj národe, naslouchej mi! Ode mne vyjde zákon a mé právo svitne všem lidem.
#51:5 Má spravedlnost je blízko, má spása vzejde, všechny lidi bude soudit má paže. Ve mne skládají naději ostrovy, čekají na mou paži.
#51:6 Pozvedněte své oči k nebi, pohleďte dolů na zem! Nebesa se rozplynou jako dým a země zvetší jako roucho, rovněž tak její obyvatelé pomřou. Ale má spása bude tu věčně, má spravedlnost neztroskotá.“
#51:7 „Slyšte mě, kdo znáte spravedlnost, lide, v jehož srdci je můj zákon. Nebojte se lidských pomluv, neděste se jejich hanobení!
#51:8 Moli je sežerou jako roucho, larvy je sežerou jako vlnu, ale má spravedlnost bude tu věčně a moje spása po všechna pokolení.“
#51:9 Probuď se, probuď, oděj se mocí, Hospodinova paže! Probuď se jako za dnů dávnověku, za dávných pokolení! Což právě ty jsi neskolila obludu, neproklála draka?
#51:10 Což právě ty jsi nevysušila moře, vody obrovské propastné tůně, a nezřídila v mořských hlubinách cestu, aby vykoupení mohli přejít?
#51:11 Ti, za něž Hospodin zaplatil, se vrátí, přijdou na Sijón s plesáním a věčná radost bude na jejich hlavě. Dojdou veselí a radosti, na útěk se dají starosti a nářek.
#51:12 „Já, já jsem váš utěšitel. Proč se tedy bojíš člověka, jenž umírá, lidského syna, který je jak tráva?
#51:13 Zapomínáš na Hospodina, který tě stvořil, který roztáhl nebesa a položil základy země. Proč máš neustále, po celé dny, strach ze vzteku utiskovatele, který ti chystal zkázu? Kde je ten utiskovatel i se svým vztekem?“
#51:14 Brzy se spoutanému otevře žalář; nezemře v jámě, ani chléb mu nebude scházet.
#51:15 „Já Hospodin, tvůj Bůh, vzdouvám moře, až jeho vlny hlučí. - Hospodin zástupů je jeho jméno. -
#51:16 Vložil jsem ti do úst svá slova, přikryl jsem tě stínem své ruky, jako stan jsem roztáhl nebesa a položil základy země, a řekl jsem Sijónu: ‚Ty jsi můj lid‘.“
#51:17 Probuď se, probuď, povstaň, jeruzalémská dcero! Pila jsi z Hospodinovy ruky pohár jeho rozhořčení. Ten kalich, pohár závrati, jsi vypila až do dna.
#51:18 Nikdo ji nevede, žádný ze synů, jež porodila, nikdo ji neuchopil za ruku, žádný se synů, jež vychovala.
#51:19 Toto obojí tě potkalo, a kdo s tebou cítil? Zhouba a zkáza, hlad a meč. Čímpak tě potěším?
#51:20 Tvoji synové leží vysíleni na nárožích všech ulic jak divoká ovce v pasti pod tíhou Hospodinova rozhořčení, pohrůžek tvého Boha.
#51:21 Proto slyš toto, ty utištěná, opojená, ne však vínem:
#51:22 Toto praví tvůj Panovník Hospodin, tvůj Bůh, jenž vede spor svého lidu: „Hle, beru ti z rukou ten pohár závrati, ten kalich, pohár mého rozhořčení; už nikdy z něho nebudeš pít.
#51:23 Vkládám jej do ruky těm, kdo tě sužovali, kteří ti poroučeli: ‚Skloň se, když přicházíme kolem.‘ A ty jsi musela nastavovat záda, byla jsi jako země, jak ulice těm, kdo po ní chodí.“ 
#52:1 Probuď se, probuď, oděj se silou, Sijóne, oděj se svými skvostnými rouchy, Jeruzaléme, město svaté, neboť už nikdy do tebe nevstoupí neobřezanec a nečistý.
#52:2 Setřes prach a povstaň, usídli se, Jeruzaléme, rozraz pouta na své šíji, zajatá dcero sijónská.
#52:3 Toto praví Hospodin: „Byli jste prodáni zadarmo, bez peněz budete vykoupeni.
#52:4 Toto praví Panovník Hospodin: Na začátku sestoupil můj lid do Egypta, aby tam pobýval jako host; na konci jej utiskovala Asýrie.
#52:5 Co teď mám udělat? - je výrok Hospodinův. Vždyť zadarmo byl vzat můj lid. Kdo mu vládnou, trápí jej, až kvílí, je výrok Hospodinův. Neustále, po celé dny, je znevažováno mé jméno.
#52:6 Můj lid však pozná mé jméno, pozná v onen den, že to jsem já, kdo prohlašuji: ‚Tu jsem‘.“
#52:7 Jak líbezné je, když po horách jdou nohy toho, jenž poselství nese a ohlašuje pokoj, jenž nese dobré poselství a ohlašuje spásu, jenž Sijónu hlásá: „Tvůj Bůh kraluje!“
#52:8 Slyš! Tvoji strážní pozvedají hlas, plesají společně, protože na vlastní oči vidí, že Hospodin se vrací na Sijón.
#52:9 Zvučně spolu plesejte, jeruzalémské trosky, vždyť Hospodin potěšil svůj lid, vykoupil Jeruzalém.
#52:10 Hospodin obnažil svou svatou paži před zraky všech pronárodů. I spatří všechny dálavy země spásu našeho Boha.
#52:11 Pryč, pryč odtud, vyjděte z Babylóna! Nedotýkejte se nečistého! Vyjděte z jeho středu! Očisťte se, vy, kdo nosíte Hospodinovo náčiní!
#52:12 Nemusíte však odcházet nakvap, nemusíte se dávat na útěk, protože Hospodin půjde před vámi, Bůh Izraele bude též uzavírat váš průvod.
#52:13 „Hle, můj služebník bude mít úspěch, zvedne se, povznese a vysoko se vyvýší.
#52:14 Jak mnozí ztrnuli úděsem nad tebou! Jeho vzezření bylo tak znetvořené, že nebyl podoben člověku, jeho vzhled takový, že nebyl podoben lidem.
#52:15 Avšak on pokropí mnohé pronárody krví, před ním si králové zakryjí ústa, protože spatří, co jim nebylo vyprávěno, porozumějí tomu, o čem neslyšeli.“ 
#53:1 Kdo uvěří naší zprávě? Nad kým se zjeví paže Hospodinova?
#53:2 Vyrostl před ním jako proutek, jak oddenek z vyprahlé země, neměl vzhled ani důstojnost. Viděli jsme ho, ale byl tak nevzhledný, že jsme po něm nedychtili.
#53:3 Byl v opovržení, kdekdo se ho zřekl, muž plný bolesti, zkoušený nemocemi, jako ten, před nímž si člověk zakryje tvář, tak opovržený, že jsme si ho nevážili.
#53:4 Byly to však naše nemoci, jež nesl, naše bolesti na sebe vzal, ale domnívali jsme se, že je raněn, ubit od Boha a pokořen.
#53:5 Jenže on byl proklán pro naši nevěrnost, zmučen pro naši nepravost. Trestání snášel pro náš pokoj, jeho jizvami jsme uzdraveni.
#53:6 Všichni jsme bloudili jako ovce, každý z nás se dal svou cestou, jej však Hospodin postihl pro nepravost nás všech.
#53:7 Byl trápen a pokořil se, ústa neotevřel; jako beránek vedený na porážku, jako ovce před střihači zůstal němý, ústa neotevřel.
#53:8 Byl zadržen a vzat na soud. Kdopak pomyslí na jeho pokolení? Vždyť byl vyťat ze země živých, raněn pro nevěrnost mého lidu.
#53:9 Byl mu dán hrob se svévolníky, s boháčem smrt našel, ačkoli se nedopustil násilí a v jeho ústech nebylo lsti.
#53:10 Ale Hospodinovou vůlí bylo zkrušit ho nemocí, aby položil svůj život v oběť za vinu. Spatří potomstvo, bude dlouho živ a zdárně vykoná vůli Hospodinovu.
#53:11 Zbaven svého trápení spatří světlo, nasytí se dny. „Tím, co zakusí, získá můj spravedlivý služebník spravedlnost mnohým; jejich nepravosti on na sebe vezme.
#53:12 Proto mu dávám podíl mezi mnohými a s četnými bude dělit kořist za to, že vydal sám sebe na smrt a byl počten mezi nevěrníky.“ On nesl hřích mnohých, Bůh jej postihl místo nevěrných. 
#54:1 Plesej, neplodná, která jsi nerodila, zvučně plesej a výskej, která ses v bolestech nesvíjela, protože synů osamělé bude více než synů provdané, praví Hospodin.
#54:2 Rozšiř místo ve svém stanu, ať napnou stanové houně tvých příbytků. Nerozpakuj se! Natáhni svá stanová lana a upevni kolíky.
#54:3 Rozmůžeš se napravo i nalevo, tvé potomstvo se zmocní pronárodů a osídlí zpustošená města.
#54:4 Neboj se, už se nemusíš stydět, nebuď zahanbená, už se nemusíš rdít studem. Zapomeneš na hanbu, kdy jsi byla neprovdána, nevzpomeneš na potupu svého vdovství.
#54:5 Tvým manželem je přece ten, jenž tě učinil, jeho jméno je Hospodin zástupů, tvým vykupitelem je Svatý Izraele; nazývá se Bohem celé země.
#54:6 Jako ženu opuštěnou a na duchu ztrápenou tě Hospodin povolal, ženu mladosti, jež byla zavržena, praví Bůh tvůj.
#54:7 „Na maličký okamžik jsem tě opustil, avšak shromáždím tě v převelikém slitování.
#54:8 V návalu rozlícení skryl jsem před tebou na okamžik svoji tvář, avšak ve věčném milosrdenství jsem se nad tebou slitoval, praví Hospodin, tvůj vykupitel.“
#54:9 „Je to pro mě jako za dnů Noeho, kdy jsem se přísahou zavázal, že už nikdy vody Noeho zemi nezatopí. Právě tak jsem se přísežně odřekl rozlícení i pohrůžek vůči tobě.
#54:10 I kdyby ustoupily hory a pohnuly se pahorky, moje milosrdenství od tebe neustoupí a smlouva mého pokoje se nepohne, praví Hospodin, tvůj slitovník.“
#54:11 „Ty utištěná, vichrem zmítaná, útěchy zbavená, hle, já ti do omítky vsadím drahokamy, za základ ti dám safíry.
#54:12 Cimbuří ti udělám rubínová, brány berylové a celé tvé obezdění z drahokamů.
#54:13 Všichni tví synové budou Hospodinovými učedníky; hojnost pokoje zjednám tvým synům.
#54:14 Podpírat tě bude spravedlnost, zbavena útisku nepocítíš bázně, děs se k tobě nepřiblíží.
#54:15 A kdyby se kdo proti tobě srotil, nebude to pocházet ode mne. Kdo se proti tobě srotí, v boji proti tobě padne.
#54:16 Hle, já jsem stvořil řemeslníka, jenž rozdmychuje uhlí v oheň a zhotovuje zbraně k užívání. A já jsem stvořil šiřitele zkázy, aby hubil.
#54:17 Žádná zbraň vyrobená proti tobě se nesetká se zdarem, každý jazyk, jenž proti tobě povstane na soudu, usvědčíš ze svévole. Toto je dědictví Hospodinových služebníků, jejich spravedlnost je ode mne, je výrok Hospodinův.“ 
#55:1 „Vzhůru! Všichni, kdo žízníte, pojďte k vodám, i ten, kdo peníze nemá. Pojďte, kupujte a jezte, pojďte a kupujte bez peněz a bez placení víno a mléko!
#55:2 Proč utrácíte peníze, ale ne za chléb? A svůj výdělek za to, co nenasytí? Poslechněte mě a jezte, co je dobré, ať se vaše duše kochá tukem!
#55:3 Nakloňte ucho a pojďte ke mně, slyšte a budete živi! Uzavřu s vámi smlouvu věčnou, obnovím milosrdenství věrně Davidovi prokázaná.“
#55:4 „Hle, dal jsem ho za svědka národům, národům za vévodu a zákonodárce.
#55:5 Hle, povoláš pronárod, který neznáš, pronárod, který tě nezná, přiběhne k tobě kvůli Hospodinu, tvému Bohu, za Svatým Izraele, který tě oslavil.“
#55:6 Dotazujte se Hospodina, dokud je možno ho najít, volejte ho, dokud je blízko.
#55:7 Svévolník ať opustí svou cestu, muž propadlý ničemnostem svoje úmysly; nechť se vrátí k Hospodinu, slituje se nad ním, k Bohu našemu, vždyť odpouštím mnoho.
#55:8 „Mé úmysly nejsou úmysly vaše a vaše cesty nejsou cesty moje, je výrok Hospodinův.
#55:9 Jako jsou nebesa vyšší než země, tak převyšují cesty mé cesty vaše a úmysly mé úmysly vaše.
#55:10 Spustí-li se lijavec nebo padá-li sníh z nebe, nevrací se zpátky, nýbrž zavlažuje zemi a činí ji plodnou a úrodnou, takže vydává símě tomu, kdo rozsívá, a chléb tomu, kdo jí.
#55:11 Tak tomu bude s mým slovem, které vychází z mých úst: Nevrátí se ke mně s prázdnou, nýbrž vykoná, co chci, vykoná zdárně, k čemu jsem je poslal.“
#55:12 „S radostí vyjdete a budete vedeni v pokoji. Hory a pahorky budou před vámi zvučně plesat a všechny stromy v poli budou tleskat.
#55:13 Místo trní vyroste cypřiš, místo plevele vzejde myrta. To bude k oslavě Hospodinova jména, za trvalé znamení, které nebude vymýceno.“ 
#56:1 Toto praví Hospodin: „Dodržujte právo, jednejte spravedlivě, protože má spása se už přiblížila, už se zjevuje má spravedlnost.
#56:2 Blaze člověku, který tak jedná, lidskému synu, který se toho drží a dbá na to, aby neznesvěcoval den odpočinku, který se má na pozoru, aby jeho ruka neučinila nic zlého.“
#56:3 Ať neříká nikdo z cizinců, kdo se připojil k Hospodinu: „Hospodin mě jistě odloučil od svého lidu.“ Ať neříká kleštěnec: „Hle, jsem strom suchý.“
#56:4 Neboť toto praví Hospodin: „Kleštěncům, kteří, dbají na mé dny odpočinku a volí to, co si přeji, kteří se pevně drží mé smlouvy,
#56:5 dám ve svém domě a na svých hradbách památník s jménem lepším než synů a dcer: Dám jim jméno věčné, jež nebude vymýceno.“
#56:6 Těm z cizinců, kteří se připojili k Hospodinu, aby mu sloužili a z lásky k jeho jménu se stali jeho služebníky, praví: „Všechny, kdo dbají na to, aby neznesvěcovali den odpočinku, kdo se pevně drží mé smlouvy,
#56:7 přivedu na svou svatou horu a ve svém domě modlitby je oblažím radostí, jejich oběti zápalné a obětní hody dojdou na mém oltáři zalíbení. Můj dům se bude nazývat domem modlitby pro všechny národy,
#56:8 je výrok Panovníka Hospodina, který shromažďuje zahnané z Izraele. Shromáždím k němu ještě další, k těm, kteří už k němu byli shromážděni.“
#56:9 Všechna polní zvěři, přijď se nažrat! Přijď, všechna zvěři lesů!
#56:10 Ti, kdo jsou na stráži, jsou slepí, nikdo z nich nic neví; všichni jsou to němí psi, nedovedou ani štěkat; jen mluví ze sna, když ulehnou, rádi podřimují.
#56:11 Ale jsou to psi hltaví, nenasytní. A to jsou pastýři! Bez zájmu a pochopení. Všichni jsou obráceni k svým cestám, každý za svým ziskem, odkudkoli.
#56:12 „Přijďte! Opatřím víno, opojným nápojem se opojíme. Zítřek bude jako dnešek, velký, přebohatý.“ 
#57:1 Spravedlivý hyne a není nikoho, kdo by si to bral k srdci. Zbožní muži bývají smeteni a nikoho nezajímá, že byl spravedlivý smeten zlem.
#57:2 Pokoje dosáhne ten, kdo jde správnou cestou. Odpočine na svém loži.
#57:3 „Předstupte sem, vy, synové té, která věští z mraků, potomstvo cizoložníka a nevěstky!
#57:4 Komu se tak škodolibě posmíváte? Komu že se pošklebujete? Na koho vyplazujete jazyk? Což nejste děti nevěrnosti, potomstvo zrady?
#57:5 Rozpalujete se u mohutných stromů, pod každým zeleným stromem, v roklích, pod skalními útesy zabíjíte děti.
#57:6 V rokli mezi hlazenými kameny je tvůj díl, ano, ty budou tvůj úděl. Také jim jsi přinášela úlitbu, obětovala dary. Tohle mě má těšit?
#57:7 Na hoře vysoko strmící zřídila sis lože, tam také stoupalas k obětním hodům.
#57:8 Za dveřmi a veřejemi sis dělala své pamětné znamení. Ode mne ses odvrátila, odhalovala ses, vystupovala jsi na své lože, rozšiřovala je, s oněmi ses spřáhla, milovala jsi jejich lože, zírala jsi na znamení údu.
#57:9 Mrhala jsi olejem pro Meleka, hýřila jsi svými mastmi, dodaleka vysílalas posly, snižovala ses až do podsvětí.
#57:10 Znavila ses množstvím svých cest, a neřeklas: ‚Nemá to význam.‘ Ve znamení údu jsi nalézala život, proto ses necítila vyčerpána.
#57:11 Koho ses polekala a koho se bojíš, že tolik lžeš? Na mne sis ani nevzpomněla, k srdci sis to nebrala. Jen proto, že jsem byl příliš dlouho zticha, nemáš přede mnou bázeň.
#57:12 Uvedu ve známost tvoji spravedlnost a tvé činy! Nic ti neprospějí.
#57:13 Až budeš křičet, ať si tě tvá sbírka bohů vysvobodí! Všechny je odnese vítr, odvane je vánek. Ale ten, kdo se utíká ke mně, dostane do dědictví zemi, do vlastnictví moji svatou horu.“
#57:14 Řekne: „Navršte, navršte násep, vyrovnejte cestu, odstraňte mému lidu překážky z cesty!“
#57:15 Toto praví Vznešený a Vyvýšený, jehož přebývání je věčné, jehož jméno je Svatý: „Přebývám ve vyvýšennosti a svatosti, ale i s tím, jenž je zdeptaný a poníženého ducha, abych oživil ducha ponížených, abych oživil srdce zdeptaných.
#57:16 Nechci vést spory navěky, nebudu trvale rozlícen, jinak by přede mnou každý duch zemdlel; vždyť vše, co dýchá, jsem učinil sám.
#57:17 Rozlítil jsem se, když se lid provinil chamtivostí, bil jsem jej, skryl jsem se v rozlícení, že se odvrátil a šel cestou svého srdce.
#57:18 Viděl jsem jeho cesty; vyléčím jej však a povedu, vrátím potěšení jemu a těm, kdo s ním truchlíš.
#57:19 Stvořím ovoce rtů. Pokoj, pokoj dalekým i blízkým, praví Hospodin. Vyléčím jej.
#57:20 Svévolní budou jak vzedmuté moře, které se nemůže uklidnit, jehož vody vyvrhují špínu a kal.
#57:21 Nemají pokoj svévolníci,“ praví můj Bůh. 
#58:1 „Volej z plna hrdla, bez zábran! Rozezvuč svůj hlas jako polnici! Mému lidu ohlas jeho nevěrnost, domu Jákobovu jeho hříchy.
#58:2 Den co den se mne dotazují, chtějí poznat moje cesty jako pronárod, jenž koná spravedlnost a řád svého Boha neopouští; na spravedlivé řády se mě doptávají, chtěli by mít Boha blízko.
#58:3 ‚Proč se postíme, a nevšímáš si toho? Pokořujeme se, a nebereš to na vědomí.‘ Právě v den, kdy se postíváte, hovíte svým zálibám a honíte všechny své dělníky.
#58:4 Postíte se jenom pro spory a hádky, abyste mohli svévolně udeřit pěstí. Nepostíte se tak, aby bylo slyšet váš hlas na výšině.
#58:5 Což to je půst, který si přeji? Den, kdy se člověk pokořuje, kdy hlavu sklání jako rákos, žínici obléká a popelem si podestýlá? Dá se toto nazvat postem, dnem, v němž má Hospodin zalíbení?
#58:6 Zdalipak půst, který já si přeji, není toto: Rozevřít okovy svévole, rozvázat jha, dát ujařmeným volnost, každé jho rozbít?
#58:7 Cožpak nemáš lámat svůj chléb hladovému, přijímat do domu utištěné, ty, kdo jsou bez přístřeší? Vidíš-li nahého, obléknout ho, nebýt netečný k vlastní krvi?
#58:8 Tehdy vyrazí jak jitřenka tvé světlo a rychle se zhojí tvá rána. Před tebou půjde tvá spravedlnost, za tebou se bude ubírat Hospodinova sláva.
#58:9 Tehdy zavoláš a Hospodin odpoví, vykřikneš o pomoc a on se ozve: ‚Tu jsem!‘ Odstraníš-li ze svého středu jho, hrozící prst a ničemná slova,
#58:10 budeš-li štědrý k hladovému a nasytíš-li ztrápeného, vzejde ti v temnotě světlo a tvůj soumrak bude jak poledne.“
#58:11 Hospodin tě povede neustále, bude tě sytit i v krajinách vyprahlých, zdatnost dodá tvým kostem; budeš jako zahrada zavlažovaná, jako vodní zřídlo, jemuž se vody neztrácejí.
#58:12 Co bylo od věků v troskách, vybudují ti, kdo z tebe vzejdou, opět postavíš, co založila minulá pokolení. Nazvou tě tím, jenž zazdívá trhliny a obnovuje stezky k sídlům.
#58:13 Jestliže v den odpočinku upustíš od svých pochůzek, od prosazování svých pochůzek, od prosazování svých zálib v můj svatý den, nazveš-li den odpočinku rozkošným, svatý den Hospodinův přeslavným, budeš-li jej slavit tak, že se vzdáš svých cest, že přestaneš hovět svým zálibám a nepovedeš plané řeči,
#58:14 tu nalezneš rozkoš v Hospodinu „a já ti dovolím jezdit po posvátných návrších země a z dědictví tvého otce Jákoba tě budu živit.“ Tak promluvila Hospodinova ústa. 
#59:1 Hle, Hospodinova ruka není krátká na spasení, jeho ucho není zalehlé, aby neslyšel.
#59:2 Jsou to právě vaše nepravosti, co vás odděluje od vašeho Boha, vaše hříchy zahalily jeho tvář před vámi, proto neslyší.
#59:3 Vaše dlaně jsou poskvrněny krví, vaše prsty nepravostí vaše rty mluví klam, váš jazyk přemílá podlosti.
#59:4 Nikdo nevolá po spravedlnosti, nikdo se nezastává pravdy. Doufají v nicotu a šalebně mluví, plodí trápení a rodí ničemnosti.
#59:5 Vyseděli baziliščí vejce, snují pavoučí vlákna. Kdo by jejich vejce pozřel, zemře, rozšlápne-li se, vyklouzne zmije.
#59:6 Jejich vlákna se nehodí na šat, svými výrobky se nepřikryjí. Jejich činy jsou jen ničemnosti, na dlaních jim lpí násilné skutky.
#59:7 Jejich nohy pádí za zlem, spěchají prolévat nevinnou krev. Zamýšlejí samé ničemnosti, na jejich silnicích číhá záhuba a zkáza.
#59:8 Cestu pokoje neznají, není práva v jejich stopách. Dělají si křivolaké cesty, kdo se jimi ubírá, nepozná pokoj.
#59:9 Proto se vzdálilo od nás právo, nedosahuje k nám spravedlnost. Čekáme na světlo, a je tma, na úsvit, a chodíme v šeru.
#59:10 Ohmatáváme stěnu jako slepí, hmatáme jako ti, kdo nemají zrak. V poledne klopýtáme, jako když se stmívá, mezi zdravými jsme jako mrtví.
#59:11 Hned všichni mručíme jako medvědi, hned zase lkáme jako holubice. Čekáme na právo, ale žádné tu není, na spásu, a ta se vzdálila od nás.
#59:12 Četné jsou naše nevěrnosti vůči tobě a naše hříchy svědčí proti nám, naše nevěrnosti se nás drží, známe svoje nepravosti:
#59:13 nevěrnost a zapírání Hospodina, odklon od našeho Boha, řeči o útisku, odpadnutí, v srdci nápady a úvahy o klamných věcech.
#59:14 Právo je úplně potlačeno, spravedlnost stojí někde v dáli, pravda klopýtá po náměstí, a co je správné, nemůže vstoupit.
#59:15 Pravda vyprchala, kořistí stane se ten, kdo se varuje zlého. Hospodin to však vidí a je zlé v jeho očích, že není žádného práva.
#59:16 Vidí, že není nikoho, žasne, že nikdo nezasáhne. Zachrání ho jeho paže, podepře ho jeho spravedlnost.
#59:17 Oděl se spravedlností jako pancířem a na hlavě má přilbu spásy, oděl se rouchem pomsty, jako pláštěm zahalil se rozhorlením.
#59:18 Podle vykonaných skutků odplatí svým protivníkům rozhořčením, svým nepřátelům za to, co páchali, ostrovům odplatí za spáchané skutky.
#59:19 I budou se bát Hospodinova jména na západě a jeho slávy na východu slunce, neboť se přivalí jako dravá řeka, kterou Hospodinův vítr žene vpřed.
#59:20 Vykupitel přijde k Sijónu, k těm, kdo se v Jákobovi odvrátili od nevěrnosti, je výrok Hospodinův.
#59:21 „Já pak s nimi uzavřu tuto smlouvu, praví Hospodin: Můj duch, který na tobě spočívá, a má slova, která jsem ti vložil do úst, nevzdálí se z úst tvých ani z úst tvého potomstva ani z úst potomstva tvého potomstva od nynějška až na věky,“ praví Hospodin. 
#60:1 „Povstaň, rozjasni se, protože ti vzešlo světlo, vzešla nad tebou Hospodinova sláva.
#60:2 Hle, temnota přikrývá zemi, soumrak národy, ale nad tebou vzejde Hospodin a ukáže se nad tebou jeho sláva.
#60:3 K tvému světlu přijdou pronárody a králové k jasu, jenž nad tebou vzejde.
#60:4 Rozhlédni se kolem a viz, tito všichni se shromáždí a přijdou k tobě; zdaleka přijdou tví synové a dcery tvé budou v náručí chovány.
#60:5 Až to spatříš, rozzáříš se, tvé ustrašené srdce se radostně rozbuší, neboť hučící moře tě zahrne svými dary, přijde k tobě bohatství pronárodů.
#60:6 Přikryje tě záplava velbloudů, mladých velbloudů z Midjánu a Éfy; přijdou všichni ze Sáby, ponesou zlato a kadidlo a budou radostně zvěstovat Hospodinovu chválu.
#60:7 K tobě se shromáždí všechny ovce z Kédaru, nebajótští berani ti budou k službám; budou přinášeni na můj oltář k mému zalíbení, oslavím dům své slávy.
#60:8 Kdože jsou ti, kteří přilétají jako oblak? Jako holubice ke svým děrám?
#60:9 Ke mně s nadějí vzhlížejí ostrovy a zámořské lodě už zdávna, aby tvé syny přivezly zdaleka a s nimi jejich stříbro a zlato pro jméno Hospodina, tvého Boha, pro Svatého, Boha Izraele, který tě oslavil.
#60:10 Cizinci vystavějí tvé hradby a jejich králové ti budou k službám. Bil jsem tě ve svém rozlícení, se zalíbením se však nad tebou slitovávám.
#60:11 Tvé brány budou neustále otevřené, nebudou zavírány ve dne ani v noci, aby k tobě mohly přijít pronárody se svým bohatstvím a přivést i své krále.
#60:12 Pronárod a království, jež by ti nesloužily, zhynou. Takové pronárody propadnou úplné zkáze.
#60:13 Přijde k tobě sláva Libanónu, cypřiš spolu s platanem a zimostrázem, aby oslavily místo mé svatyně; chci uctít podnož svých nohou.
#60:14 Zkroušeni přijdou k tobě synové těch, kdo tě ujařmovali, u tvých nohou se budou kořit všichni, kteří tě znevažovali; nazvou tě ‚Město Hospodinovo‘, ‚Sijón Svatého, Boha Izraele‘.“
#60:15 „Místo toho, abys bylo dále opuštěné, nenáviděné a bez poutníka, dám ti důstojnost navěky a veselí po všechna pokolení.
#60:16 Budeš sát mléko pronárodů, budeš sát z královských prsů. Poznáš, že já jsem Hospodin, tvůj spasitel, tvůj vykupitel, Přesilný Jákobův.
#60:17 Místo bronzu přinesu zlato a místo železa přinesu stříbro, místo dřeva bronz a místo kamení železo. Za dohlížitele ti dám pokoj a za poháněče spravedlnost.
#60:18 Nikdy už nebude slýcháno o násilí ve tvé zemi, o zhoubě a zkáze na tvém území. Své hradby budeš nazývat ‚Spása‘ a své brány ‚Chvála‘.
#60:19 Už nebudeš mít slunce za světlo dne, ani jas měsíce ti nebude svítit. Hospodin ti bude světlem věčným, tvůj Bůh tvou oslavou.
#60:20 Tvé slunce nikdy nezapadne, tvůj měsíc nebude ubývat, neboť Hospodin ti bude světlem věčným. Dny tvého smutku skončily.
#60:21 Tvůj lid, všichni budou spravedliví, navěky obdrží do vlastnictví zemi, oni, výhonek z mé sadby, dílo mých rukou k mé oslavě.
#60:22 Z nejmenšího jich bude tisíc, z nejnepatrnějšího mocný národ. Já jsem Hospodin, vykonám to spěšně, v pravý čas.“ 
#61:1 Duch Panovníka Hospodina je nade mnou. Hospodin mě pomazal k tomu, abych nesl radostnou zvěst pokorným, poslal mě obvázat rány zkroušených srdcem, vyhlásit zajatcům svobodu a vězňům propuštění,
#61:2 vyhlásit léto Hospodinovy přízně, den pomsty našeho Boha, potěšit všechny truchlící,
#61:3 pozvednout truchlící na Sijónu, dát jim místo popela na hlavu čelenku, olej veselí místo truchlení, závoj chvály místo ducha beznaděje. Nazvou je „Stromy spravedlnosti“ a „Sadba Hospodinova“ k jeho oslavě.
#61:4 Co bylo od věků v troskách, vybudují, postaví, co kdysi bylo zpustošeno, obnoví zničená města, zpustošená po celá pokolení.
#61:5 Stanou zde cizáci a budou vám pást ovce, synové ciziny budou vašimi rolníky a vinaři.
#61:6 Vy pak budete nazýváni „Hospodinovi kněží“, bude se vám říkat „Sluhové našeho Boha“. Budete užívat bohatství pronárodů a honosit se jejich slávou.
#61:7 Za svou dvojí ostudu a hanbu bude nad svým podílem lid plesat, dvojí dědictví obdrží v své zemi, budou mít věčně radost.
#61:8 „Neboť já Hospodin miluji právo, při zápalné oběti nenávidím vydírání. Jejich výdělek jim předám věrně a uzavřu s nimi smlouvu věčnou.
#61:9 Jejich potomstvo bude známé mezi pronárody a jejich potomci uprostřed národů. Všichni, kdo je spatří, rozpoznají na nich, že oni jsou to potomstvo, jemuž Hospodin žehná.“
#61:10 Velmi se veselím z Hospodina, má duše jásá k chvále mého Boha, neboť mě oděl rouchem spásy, zahalil mě pláštěm spravedlnosti jak ženicha, jenž si jako kněz čelenku bere, a jako nevěstu, která se krášlí svými šperky.
#61:11 Jako země dává vzrůst tomu, co klíčí, jako zahrada dává vzklíčit tomu, co bylo zaseto, tak Panovník Hospodin dá vzklíčit spravedlnosti a chvále přede všemi pronárody. 
#62:1 Kvůli Sijónu nebudu zticha, kvůli Jeruzalému si nedopřeji odpočinku, dokud jako záře nevzejde jeho spravedlnost, dokud jako pochodeň nevzplane jeho spása.
#62:2 Pronárody spatří tvoji spravedlnost a všichni králové tvou slávu. Nazvou tě novým jménem, jež určila Hospodinova ústa.
#62:3 Budeš nádhernou korunou v Hospodinově ruce a královským turbanem v dlaních svého Boha.
#62:4 Už nikdy o tobě neřeknou: „Opuštěná“. A nikdy o tvé zemi neřeknou: „Zpustošená“. Budou tě nazývat: „Oblíbená“ a tvou zemi: „Vdaná“, protože si tě Hospodin oblíbil a tvá země se vdala.
#62:5 Jako se mladík žení s pannou, tak se tví synové ožení s tebou. A jako se ženich veselí z nevěsty, tak se tvůj Bůh bude veselit z tebe.
#62:6 Na tvých hradbách, Jeruzaléme, jsem ustanovil strážce; po celý den a po celou noc ať nikdy nejsou zticha. Vy, kteří připomínáte Hospodina, nedopřávejte si klidu!
#62:7 Nedopřávejte mu klidu, dokud nepostaví Jeruzalém, dokud mu nevrátí v zemi chvalozpěv.
#62:8 Hospodin přisáhl svou pravicí, svou mocnou paží: „Už nikdy nedám tvé obilí za pokrm nepřátelům, už nikdy nebudou pít synové ciziny tvůj mošt, pro který jsi namáhavě pracovala.
#62:9 Co kdo sklidí, bude též jíst a bude chválit Hospodina, a kdo budou sbírat hrozny, budou z nich pít víno na mých svatých nádvořích.“
#62:10 Projděte, projděte branami! Připravte lidu cestu! Vyrovnejte, vyrovnejte silnici! Odstraňte kamení! Zvedněte korouhev nad národy!
#62:11 Hle, Hospodin dává slyšet až do dálav země výzvu: „Vyřiďte dceři sijónské: Hle, přichází tvá spása!“ Hle, svoji mzdu má s sebou, u sebe svůj výdělek.
#62:12 Nazvou je „Lid svatý“, „Hospodinovi vykoupení“. A tebe nazvou „Vyhledávaná“, „Město neopuštěné“. 
#63:1 Kdo to přichází z Edómu, kdo v rouchu rudém jde z Bosry? Kdo je ten v úboru velebném, hrdě kráčející ve své velké síle? „Já, který právem vyhlašuji, že na vítězství stačím.“
#63:2 Proč je tvůj úbor krvavě zbarven a tvé roucho jak toho, jenž v lisu šlape?
#63:3 „Sám jsem šlapal v lisovací kádi a nikdo z národů se mnou nebyl. Rozhněván jsem po nich šlapal, rozšlapal jsem je v svém rozhořčení, až šťáva z nich mi roucho postříkala, poskvrnil jsem si celý oděv.
#63:4 Den pomsty měl jsem v srdci, nastalo milostivé léto pro vykoupené.
#63:5 Rozhlédl jsem se, ale nebylo pomocníka, užasl jsem, opora žádná. K vítězství mi dopomohla vlastní paže, oporou mi bylo moje rozhořčení.
#63:6 Rozhněván jsem rozdupal národy, opojil jsem je svým rozhořčením; jejich mízu jsem nechal stékat k zemi.“
#63:7 Milosrdenství Hospodinovo budu připomínat, Hospodinovy chvályhodné činy, všechno, jak Hospodin nás odměňoval, velkou jeho dobrotu k izraelskému domu. On je odměňoval podle svého slitování, podle svého velikého milosrdenství.
#63:8 Prohlásil: „Vždyť oni jsou můj lid, synové, kteří nebudou klamat.“ Stal jsem se jim spasitelem.
#63:9 Každým jejich soužením byl sužován a anděl stojící před jeho tváří je zachraňoval; svou láskou a shovívavostí je vykupoval, bral je na svá ramena a nosil je po všechny dny dávné.
#63:10 Oni se však vzpírali a trápili jeho svatého ducha, proto se jim změnil v nepřítele, sám bojoval s nimi.
#63:11 Avšak jeho lid se rozpomínal na dny dávné, na dny Mojžíšovy. Kde je ten, jenž vyvedl je z moře s pastýři svých ovcí? Kde je ten, jenž do nitra mu vložil svatého ducha svého?
#63:12 Ten, který je po pravici Mojžíšově vedl svou proslavenou paží, který před nimi rozpoltil vodstvo, a tak si učinil jméno věčné?
#63:13 Propastnými tůněmi je vedl jako koně pouští, ani neklopýtli,
#63:14 jako dobytek sestupující na pláň. Hospodinův duch je vedl do místa odpočinutí. Tak jsi vedl svůj lid a proslavil jsi své jméno.
#63:15 Pohlédni z nebes a podívej se ze svého svatého, proslaveného obydlí! Kde je tvé horlení a bohatýrská síla? Tvé cituplné nitro a tvé slitování jsou mi uzavřeny?
#63:16 Jsi přece náš Otec! Abraham nás nezná, Izrael, ten o nás neví. Hospodine, tys náš Otec, náš vykupitel odedávna, to je tvé jméno.
#63:17 Proč jsi nás nechal zbloudit, Hospodine, ze svých cest? Zatvrdil jsi naše srdce, aby se tě nebálo. Navrať se kvůli svým služebníkům, kvůli kmenům svého dědictví!
#63:18 Dočasně si tvůj svatý lid přivlastnili, tvou svatyni pošlapali naši protivníci.
#63:19 My jsme tvoji odedávna. Jim jsi nepanoval, nazýváni nebyli tvým jménem. Kéž bys protrhl nebesa a sestoupil dolů, hory by se před tvou tváří potácely. 
#64:1 Jako když oheň spaluje suché roští a uvádí do varu vodu, tak dáš poznat svým protivníkům své jméno. Pronárody se budou před tebou chvět.
#64:2 Když jsi konal hrozné činy, jichž jsme se nenadáli, sestoupils, a hory se před tvou tváří potácely.
#64:3 Od věků se něco takového neslyšelo, k sluchu neproniklo, oko nespatřilo, že by jiný bůh, mimo tebe, učinil něco pro toho, kdo na něj čeká.
#64:4 Zasazuješ se o toho, kdo s radostí koná spravedlnost, o ty, kdo na tebe pamatují na tvých cestách. Hle, byl jsi rozlícen, že jsme hřešili, na tvých cestách odvěkých však budeme zachráněni.
#64:5 Jako nečistí jsme byli všichni, všechna naše spravedlnost jako poskvrněný šat. Uvadli jsme všichni jako listí, naše nepravosti nás unášely jako vítr.
#64:6 Nebylo nikoho, kdo by vzýval tvé jméno, kdo by procitl a pevně se tě chopil, neboť jsi před námi ukryl svou tvář a nechal nás zmítat se pod mocí naší nepravosti.
#64:7 Ale nyní, Hospodine, tys náš Otec! My jsme hlína, tys náš tvůrce, a my všichni jsme dílo tvých rukou.
#64:8 Nebuď už tak rozlícen, Hospodine, naši nepravost už nikdy nepřipomínej. Prosíme, hleď, všichni jsme tvůj lid.
#64:9 Svatá města tvá se stala pouští, pouští se stal Sijón, pustou krajinou Jeruzalém.
#64:10 Náš svatý a proslavený dům, kde tě chválívali otcové naši, stal se spáleništěm, v troskách je vše, co nám bylo vzácné.
#64:11 Cožpak můžeš vzhledem k tomu, Hospodine, přemáhat se, zůstat zticha a tolik nás ponižovat? 
#65:1 Dal jsem odpověď těm, kdo se neptali, dal jsem se nalézt těm, kdo mě nehledali. Řekl jsem: „Hle, tady jsem, jsem tady“ pronárodu, který nevzýval mé jméno.
#65:2 Po celé dny vztahoval jsem ruce k lidu svéhlavému, k těm, kdo chodí po nedobré cestě, za vlastními úmysly.
#65:3 Je to lid, jenž neustále do očí mě dráždí: Obětují v zahradách, na cihlách kadidlo pálí,
#65:4 vysedávají v hrobech, nocují v tajemných skrýších, jedí maso z vepřů a ve svých nádobách mají polévku ze závadných věcí.
#65:5 Říkají: „Jdi si po svém, nepřistupuj ke mně, jsem pro tebe svatý.“ Tito jsou v mých chřípích kouřem, ohněm, který po celé dny plane.
#65:6 Hle, je to přede mnou zapsáno: „Nebudu zticha, dokud neodplatím. Do klína jim odplatím
#65:7 za nepravosti vaše a nepravosti vašich otců zároveň, praví Hospodin“; pálili kadidlo na horách, tupili mě na pahorcích. „Odměřím jim do klína jejich výdělek z dřívějška.“
#65:8 Toto praví Hospodin: „Když se najde v hroznu šťáva, říkává se: ‚Nemař jej, je v něm požehnání.‘ Tak budu jednat kvůli svým služebníkům, aby se nic nezmařilo.
#65:9 Dám vzejít z Jákoba potomstvu a z Judy tomu, kdo dědičně obsadí mé hory. Obsadí je moji vyvolení, budou tam přebývat moji služebníci.
#65:10 Šáron se stane pastvinou ovcí, dolina Akór místem, kde bude dobytek odpočívat. To pro můj lid, který se mne bude dotazovat.
#65:11 Vás však, kteří opouštíte Hospodina, kdo zapomínáte na jeho svatou horu, kdo strojíte stůl pro bůžka štěstí a pro bůžka osudu naléváte kořeněné víno,
#65:12 vás jsem určil meči, vy všichni se musíte k popravě sklonit, neboť jsem volal, a vy jste neodpovídali, mluvil jsem, a vy jste neposlouchali, ale konali jste, co je zlé v mých očích, a zvolili jste si, co se mi nelíbí.“
#65:13 A tak toto praví Panovník Hospodin: „Hle, moji služebníci budou jíst, vy však budete hladovět. Hle, moji služebníci budou pít, vy však budete žíznit. Hle, moji služebníci se budou radovat, vy se však budete stydět.
#65:14 Hle, moji služebníci budou plesat s pohodou v srdci, vy však pro bolest srdce budete křičet a pro trýzeň ducha kvílet.
#65:15 Zanecháte své jméno mým vyvoleným k vyhlašování kletby: ‚Ať tě Panovník Hospodin usmrtí!‘ Své služebníky nazve pak Bůh jménem jiným.“
#65:16 Kdo si bude v té zemi dávat požehnání, bude si je dávat v Bohu pravém. Kdo bude v té zemi přísahat, bude přísahat při Bohu pravém. Minulá soužení budou zapomenuta, ukryta před mým zrakem.
#65:17 „Hle, já stvořím nová nebesa a novou zemi. Věci minulé nebudou připomínány, nevstoupí na mysl.
#65:18 Veselte se, jásejte stále a stále nad tím, co stvořím. Hle, já stvořím Jeruzalém k jásotu a jeho lid k veselí.
#65:19 I já budu nad Jeruzalémem jásat a veselit se ze svého lidu. Nikdy víc už nebude v něm slyšet pláč ani křik.
#65:20 Nikdy už tam nebude dítě, které zemře v několika dnech, ani stařec, který by se nedožil plnosti věku, protože bude mladíkem, kdo zemře ve stu letech. Ale hříšník, byť stoletý, bude zlořečen.
#65:21 Vystavějí domy a usadí se v nich, vysázejí vinice a budou jíst jejich plody.
#65:22 Nebudou stavět, aby se tam usadil jiný, nebudou sázet, aby z toho jedl jiný. Dny mého lidu budou jako dny stromu. Co svýma rukama vytvoří, to moji vyvolení sami spotřebují.
#65:23 Nebudou se namáhat nadarmo a nebudou rodit pro náhlý zánik, neboť jsou potomstvem těch, kdo byli Hospodinem požehnáni, oni i jejich potomci.
#65:24 Dříve než zavolají, já odpovím; budou ještě mluvit a já je už vyslyším.
#65:25 Vlk a beránek se budou pást spolu a lev jako dobytek bude žrát slámu, hadu však bude potravou prach. Nikdo už nebude páchat zlo a šířit zkázu na celé mé svaté hoře,“ praví Hospodin. 
#66:1 Toto praví Hospodin: „Mým trůnem jsou nebesa a podnoží mých nohou země. Kdepak je ten dům, který mi chcete vybudovat? Kdepak je místo mého odpočinutí?
#66:2 Všechny tyto věci učinila moje ruka. Tak vznikly všechny tyto věci, je výrok Hospodinův. Laskavě pohlédnu na toho, kdo je utištěný a na duchu ubitý, kdo se třese před mým slovem.“
#66:3 Člověk zabíjí v oběť býka i ubíjí člověka, obětuje jehňátko i láme vaz psu, přináší obětní dar i krev z vepřů dává, na připomínku pálí kadidlo i dobrořečí ničemné modle. Oni si vyvolují vlastní cesty, jejich duše si libuje v ohyzdných modlách.
#66:4 Já zase vyvolím jejich zvůli a uvedu na ně, čeho se lekají, protože jsem volal, a nikdo neodpovídal, mluvil jsem, a nikdo neposlouchal. Dopouštěli se toho, co je zlé v mých očích, a volili to, co se mi nelíbí.“
#66:5 Slyšte slovo Hospodinovo, vy, kdo se třesete při jeho slovu! Říkávají vaši bratři, kteří vás nenávidí, kteří vás vypovídají pro mé jméno: „Ať se Hospodin oslaví, ať vidíme vaši radost!“ Ale budou zahanbeni.
#66:6 Slyš! Hukot z města! Halas z chrámu! Hlas Hospodina, jenž odplácí svým nepřátelům za to, co spáchali.
#66:7 Dříve než ji přepadly porodní bolesti, porodila. Dříve než ji zachvátily porodní křeče, povila pacholíka.
#66:8 Kdo slyšel kdy něco takového? Kdo co takového spatřil? Což se zrodí země v jediném dni? Nebo pronárod snad bývá zplozen jedním rázem? Sotva se začala svíjet bolestí, už porodila sijónská dcera své syny.
#66:9 „Což já, který otvírám život, nemám umožnit narození? praví Hospodin. Když dávám plodnost, mám zavírat lůno? praví tvůj Bůh.“
#66:10 Radujte se s dcerou jeruzalémskou a jásejte nad ní všichni, kdo ji milujete! Veselte se s ní, veselte, všichni, kdo jste nad ní truchlívali.
#66:11 Budete sát do sytosti potěšení z jejích prsů, budete s rozkoší pít plnými doušky z prsů její slávy.
#66:12 Toto praví Hospodin: „Hle, já k ní přivedu pokoj jako řeku a jako rozvodněný potok slávu pro národů. Budete sát nošeni v náručí, hýčkáni na kolenou.
#66:13 Jako když někoho utěšuje matka, tak vás budu těšit. V Jeruzalémě dojdete potěšení.“
#66:14 Uvidíte to a vaše srdce se rozveselí, vaše kosti budou pučet jako mladá tráva. Bude zřejmé, že Hospodinova ruka je s jeho služebníky a že jeho hrozný hněv je proti jeho nepřátelům.
#66:15 Hle, Hospodin přichází v ohni a jeho vozy jsou jako vichřice, aby vylil hněv svůj v prchlivosti a své hrozby v plamenech ohně.
#66:16 Ohněm totiž a mečem povede Hospodin soud s veškerým tvorstvem. Mnoho bude těch, jež Hospodin skolí.
#66:17 „Ti, kdo se v zahradách posvěcují a očišťují po vzoru jednoho, který je uprostřed, ti, kdo jedí maso z vepřů a to, co je hodno opovržení, dokonce myši, společně zajdou“, je výrok Hospodinův.
#66:18 „Já zakročím proti jejich činům a úmyslům a shromáždím všechny pronárody a jazyky. I přijdou a spatří mou slávu.
#66:19 Vložím na ně znamení a ty z nich, kdo vyváznou, vyšlu k pronárodům, do Taršíše, do Púlu a Lúdu, k těm, co natahují lučiště, do Túbalu a do Jávanu, na daleké ostrovy, které zprávu o mně ještě neslyšely ani nespatřily moji slávu; budou hlásat moji slávu mezi pronárody.
#66:20 Přivedou také ze všech pronárodů všechny vaše bratry jako obětní dar Hospodinu na koních a na vozech a na nosítkách, na mezcích a dromedárech na mou svatou horu do Jeruzaléma, praví Hospodin, tak jako budou přinášet Izraelci obětní dar do Hospodinova domu v čisté nádobě.
#66:21 Také z nich si vezmu kněze, lévijce,“ praví Hospodin.
#66:22 „Jako nová nebesa a nová země, které učiním, budou stát přede mnou, je výrok Hospodinův, tak nepohnutelně bude stát vaše potomstvo a vaše jméno.
#66:23 O každém novoluní, v každý den odpočinku, přijde se sklonit veškeré tvorstvo přede mnou, praví Hospodin.
#66:24 Až vyjdou, spatří mrtvá těla mužů, kteří mi byli nevěrní. Jejich červ neumírá, jejich oheň neuhasne; budou strašlivou výstrahou všemu tvorstvu.“  

\book{Jeremiah}{Jer}
#1:1 Slova Jeremjáše, syna Chilkijášova, z kněží v Anatótu v zemi Benjamínově.
#1:2 K němu se stalo slovo Hospodinovo za dnů judského krále Jošijáše, syna Amónova, ve třináctém roce jeho kralování.
#1:3 Stávalo se též za dnů judského krále Jojakíma, syna Jošijášova, až do konce jedenáctého roku vlády judského krále Sidkijáše, syna Jóšijášova, až do přestěhování Jeruzaléma v pátém měsíci toho roku.
#1:4 Stalo se ke mně slovo Hospodinovo:
#1:5 „Dříve než jsem tě vytvořil v životě matky, znal jsem tě, dříve než jsi vyšel z lůna, posvětil jsem tě, dal jsem tě pronárodům za proroka.“
#1:6 Nato jsem odpověděl: „Ach, Panovníku Hospodine, nevím, jak bych mluvil. Jsem přece chlapec.“
#1:7 Ale Hospodin mi řekl: „Neříkej: Jsem chlapec. Všude, kam tě pošlu, půjdeš, a všechno, co ti přikážu, řekneš.
#1:8 Neboj se jich, já budu s tebou a vysvobodím tě, je výrok Hospodinův.“
#1:9 Hospodin vztáhl svou ruku a dotkl se mých úst. Pak mi Hospodin řekl: „Hle, vložil jsem ti do úst svá slova.
#1:10 Hleď, tímto dnem tě ustanovuji nad pronárody a nad královstvími, abys rozvracel a podvracel, abys ničil a bořil, stavěl a sázel.“
#1:11 Stalo se ke mně slovo Hospodinovo: „Co vidíš, Jeremjáši?“ Odpověděl jsem: „Vidím mandloňový prut.“
#1:12 Hospodin mi řekl: „Viděl jsi dobře. Bdím nad svým slovem, aby se uskutečnilo.“
#1:13 Slovo Hospodinovo stalo se ke mně podruhé: „Co vidíš?“ Odpověděl jsem: „Vidím překypující hrnec obrácený sem od severu.“
#1:14 Nato mi řekl Hospodin: „Od severu se přivalí zlo na všechny obyvatele země.
#1:15 Hle, povolám všechny čeledi severních království, je výrok Hospodinův. Přijdou a každá postaví svou soudnou stolici u vchodu do jeruzalémských bran, proti všem jeho hradbám kolkolem i proti všem judským městům.
#1:16 Vyhlásím jim své soudy nad všemi jejich zlými skutky, že mě opustili; jiným bohům pálili kadidlo a klaněli se dílu svých rukou.
#1:17 Ty však přepásej svá bedra, povstaň a vyřiď jim všechno, co jsem ti přikázal. Jen se jich neděs, jinak tě naplním děsem z nich.
#1:18 Hle, učinil jsem tě dnes opevněným městem, sloupem železným a bronzovou hradbou nad celou zemí proti králům judským, proti jejich velmožům, proti jejich kněžím i proti lidu země.
#1:19 Budou proti tobě bojovat, ale nic proti tobě nezmohou, neboť já budu s tebou a vysvobodím tě, je výrok Hospodinův.“ 
#2:1 Stalo se ke mně slovo Hospodinovo:
#2:2 „Jdi a provolávej v Jeruzalémě: Toto praví Hospodin: Připomínám ti zbožnost tvého mládí, tvou lásku před sňatkem, jak jsi za mnou chodila pouští, zemí neosívanou.
#2:3 Svatý byl Izrael Hospodinu, prvotina jeho úrody; všichni, kdo jej sžírají, se proviňují, dolehne na ně zlo, je výrok Hospodinův.“
#2:4 Slyšte slovo Hospodinovo, dome Jákobův i všechny čeledi domu Izraelova.
#2:5 Toto praví Hospodin: „Jaké bezpráví na mně našli vaši otcové, že se ode mne vzdálili? Chodili za přeludem a přeludem se stali.
#2:6 Neptali se: ‚Kde je Hospodin, který nás vyvedl z egyptské země, který nás vodil pouští, zemí pustou, plnou výmolů, zemí vyprahlou, zemí šeré smrti, zemí, jíž nikdo neprocházel, v níž člověk nesídlí.‘
#2:7 Vedl jsem vás do země sadů, abyste jedli její plody a to, co dobrého dává. Přišli jste a mou zemi jste poskrvnili, moje dědictví jste zohavili.
#2:8 Kněží se nezeptali: ‚Kde je Hospodin?‘ Nepoznali mě, kdo se obírají zákonem, pastýři mi byli nevěrní, proroci prorokovali skrze Baala, chodili za tím, co není k užitku.
#2:9 Proto již vedu s vámi spor, je výrok Hospodinův, povedu jej i se syny vašich synů.
#2:10 Projděte ostrovy Kitejců a hleďte, pošlete do Kédaru a dobře uvažujte, hleďte, zda se něco podobného kdy stalo.
#2:11 Zaměnil snad nějaký pronárod bohy, ač to žádní bohové nejsou? Můj lid zaměnil svou Slávu za to, co není k užitku.
#2:12 Děste se nad tím, nebesa, ustrňte nesmírnou hrůzou, je výrok Hospodinův.
#2:13 Dvojí zlo spáchal můj lid: Opustili mne, zdroj živých vod, a vytesali si cisterny, cisterny rozpukané, jež vodu neudrží.“
#2:14 „Což je Izrael otrokem? Je otrockého rodu? Proč se stal kořistí?
#2:15 Mladí lvi nad ním řvou, vydávají hlas. Z jeho země učinili spoušť, jeho města jsou vypálená, jsou bez obyvatel.
#2:16 I synové Memfidy a Tachpanchésu oholí ti kštici.
#2:17 Což sis to nezpůsobila sama? Opustilas Hospodina, svého Boha, v čase, kdy tě vodil svou cestou.
#2:18 Co máš teď z cesty egyptské, z toho, že piješ vody Nilu? Co máš z cesty asyrské, z toho, že se napájíš vodami Eufratu?
#2:19 Tvá vlastní zloba tě ztrestá, tvé odvrácení se ti vymstí. Uznej a pohleď, jak je to zlé a trpké, že opouštíš Hospodina, svého Boha, a ze mne strach nemáš, je výrok Panovníka, Hospodina zástupů.“
#2:20 „Už dávno jsi zlomila své jho, zpřetrhala svá pouta a řekla: ‚Nebudu sloužit modlám.‘ Ale na každém vysokém pahorku, pod každým zeleným stromem se, nevěstko, rozvaluješ.
#2:21 A já jsem tě zasadil, révu ušlechtilou, samou spolehlivou sadbu. Jak ses mi mohla proměnit v révu planou a cizí?
#2:22 Kdyby ses umyla potaší, vypotřebovala přemnoho louhu, zůstane přede mnou špína tvé nepravosti, je výrok Panovníka Hospodina.“
#2:23 „Jak můžeš říkat: ‚Neposkvrnila jsem se, za baaly jsem nechodila!‘ Podívej se na svou cestu v Údolí, přiznej, co jsi páchala, velbloudice, která se volně proháníš po svých cestách.
#2:24 Divoká oslice, navyklá poušti, toužebně větříš! Kdo ji udrží v čas říje? Každý, kdo ji hledá, nalezne ji v jejím měsíci bez námahy.
#2:25 Chraň se, ať nechodíš jednou bosa a hrdlo ať ti nevyschne žízní! Ale ty jsi řekla: ‚Zbytečné řeči. Nikoli, miluji cizáky a za nimi půjdu.‘
#2:26 Jako je zahanben dopadený zloděj, tak propadne hanbě dům izraelský i s králi a velmoži, s kněžími a proroky.
#2:27 Dřevu říkají: ‚Ty jsi můj otec‘ a kameni: ‚Ty jsi mě zplodil‘, a ke mně se obracejí zády, nikoli tváří. Ve zlý čas volají: ‚Povstaň a zachraň nás!‘
#2:28 Kde jsou tví bohové, které sis udělala? Ať vstanou a zachrání tě ve zlý čas, vždyť máš tolik bohů kolik měst, Judo.“
#2:29 „Proč se mnou vedete spor? Všichni jste mi byli nevěrní, je výrok Hospodinův.
#2:30 Nadarmo jsem bil vaše syny, nedali jste se napomenout. Váš meč požíral vaše proroky jako lev, který hubí.“
#2:31 Vy, toto pokolení, hleďte na slovo Hospodinovo: „Což jsem se stal Izraeli pouští nebo zemí temnot? Proč můj lid říká: ‚Chceme mít volnost, k tobě už nikdy nepůjdeme‘?
#2:32 Může panna zapomenout na své šperky, nevěsta na svůj svatební pás? Můj lid však na mě zapomíná po nespočetné dny.
#2:33 Jak prokážeš, že je tvá cesta dobrá, když vyhledáváš milkování? Proto také na svých cestách učíš zlým věcem.
#2:34 Dokonce na lemu tvého roucha je krev nevinných ubožáků, ač jsi je nepřistihla při vloupání. Ale při tom všem
#2:35 říkáš: ‚Jsem nevinná, jeho hněv se jistě ode mne odvrátil.‘ Hle, soudím tě za to, že říkáš: ‚Já jsem nezhřešila.‘
#2:36 Proč tak lehkovážně měníš svou cestu? I Egyptem se zklameš, jako tě zklamala Asýrie.
#2:37 Vyběhneš odtamtud s rukama nad hlavou, neboť Hospodin zavrhl to, več doufáš; s nimi neuspěješ.“ 
#3:1 Řekl: „Jestliže muž propustí svou ženu a ona od něho odejde a vdá se za jiného, smí se k ní znovu vrátit? Což by tím země nebyla nesmírně potřísněna? Smilnila jsi s mnoha druhy, avšak vrať se ke mně, je výrok Hospodinův.
#3:2 Pozdvihni své oči na holá návrší a pohleď: Kde ses neposkvrnila nevěrou? Kvůli nim jsi vysedávala při cestách jako Arab na poušti, potřísnila jsi zemi svým smilstvem, svými zlořády.
#3:3 Byla zadržena vláha, nepřišel podzimní déšť, ale ty máš čelo nevěstky, odvrhla jsi stud.
#3:4 Což ani teď ke mně nezvoláš: ‚Můj Otče, tys byl vůdce mého mládí‘?
#3:5 Ale říkáš: ‚Bude mě na věky hlídat, ustavičně stát na stráži?‘ Hle, mluvíš a pácháš zlé věci, jak jen můžeš.“
#3:6 Hospodin mi za dnů krále Jóšijáše řekl: „Viděl jsi, co provedla ta izraelská odpadlice? Chodila na kdejakou vysokou horu i pod kdejaký zelený strom a smilnila tam.
#3:7 Řekl jsem: Po tom všem, co provedla, se vrátí ke mně. Ale nevrátila se. Viděla ji její sestra, judská věrolomnice.
#3:8 Pro všechno cizoložství, jehož se ta izraelská odpadlice dopustila, jsem se rozhodl, že ji propustím, a dal jsem jí rozlukový list. Ale její sestra, judská věrolomnice, se nebála, šla a smilnila také.
#3:9 Svým lehkovážným smilstvem potřísnila zemi, cizoložila s kamenem i dřevem.
#3:10 Ani po tom všem se ke mně její sestra, judská věrolomnice, neobrátila celým srdcem, nýbrž licoměrně, je výrok Hospodinův.“
#3:11 Hospodin mi řekl: „Izraelská odpadlice je spravedlivější než ta věrolomnice judská.
#3:12 Jdi a provolej směrem k severu tato slova: Navrať se, izraelská odpadlice, je výrok Hospodinův, já se na vás neosopím, neboť jsem milosrdný, je výrok Hospodinův, nechovám hněv navěky.
#3:13 Poznej však svou nepravost, dopustila ses nevěry vůči Hospodinu, svému Bohu, zaměřila jsi své cesty k bohům cizím, pod kdejaký zelený strom, mne jsi neposlouchala, je výrok Hospodinův.“
#3:14 „Navraťte se, odpadlí synové, je výrok Hospodinův, neboť já jsem váš manžel. Přijmu vás, po jednom z města, po dvou z čeledi, a uvedu vás na Sijón.
#3:15 Dám vám pastýře podle svého srdce a ti vás budou pást obezřetně a prozíravě.
#3:16 A stane se, že se v oněch dnech na zemi rozmnožíte a rozplodíte, je výrok Hospodinův. Nebude se už říkat: ‚Schrána smlouvy Hospodinovy‘; ani jim na mysl nepřijde, nebudou ji připomínat, nebudou ji postrádat, nebude už zhotovena.
#3:17 V onen čas nazvou Jeruzalém Hospodinovým trůnem a shromáždí se v něm ke jménu Hospodina v Jeruzalémě všechny pronárody a nikdy už nebudou žít podle svého zarputilého a zlého srdce.
#3:18 V oněch dnech se dům judský s domem izraelským vydají společně na cestu ze země na severu do země, kterou jsem dal do dědictví jejich otcům.“
#3:19 Řekl jsem: „Jak tě mám připojit k synům a dát ti tu přežádoucí zemi, skvostné dědictví zástupů pronárodů? Domníval jsem se, že mě budeš nazývat Otcem a nebudeš se ode mne odvracet.
#3:20 Jako žena věrolomně opouští svého druha, tak věrolomně jste se zachovali vůči mně, domě izraelský, je výrok Hospodinův.“
#3:21 Na holých návrších je slyšet volání, pláč a prosby synů izraelských. Zvrátili svou cestu, zapomněli na Hospodina, svého Boha.
#3:22 „Vraťte se, synové odpadlí, já vaše odpadlictví vyléčím. ‚Zde jsme, přišli jsme k tobě, neboť ty jsi Hospodin, náš Bůh.
#3:23 Věru, klamavé jsou pahorky, halasící hory! Věru, jen v Hospodinu, našem Bohu, je spása Izraele.
#3:24 Ohava požírala už od našeho mládí, čeho otcové těžce nabyli: jejich ovce i dobytek, syny i dcery.
#3:25 Ležíme v hanbě, přikrytí pohanou, neboť jsme hřešili proti Hospodinu, svému Bohu, my i naši otcové, od svého mládí až do tohoto dne. Neposlouchali jsme Hospodina, svého Boha.‘“ 
#4:1 „Obrátíš-li se, Izraeli, je výrok Hospodinův, obrať se ke mně! Jestliže odstraníš své ohyzdné modly, abych je neměl na očích, a nebudeš se toulat,
#4:2 Budeš-li přísahat: ‚Jakože živ je Hospodin‘ a dbát pravdy, soudu a spravedlnosti, pak si budou jeho jménem žehnat pronárody a budou se jím chlubit.“
#4:3 Toto praví Hospodin mužům judským i jeruzalémským: „Zorejte si úhor, nesejte do trní!
#4:4 Obřežte se kvůli Hospodinu, obřežte svá neobřezaná srdce, mužové judští, obyvatelé Jeruzaléma, aby mé rozhořčení nevyšlehlo jako oheň a nehořelo a nikdo by je neuhasil, a to pro vaše zlé skutky.
#4:5 Oznamte v Judsku, rozhlaste v Jeruzalémě, mluvte, trubte na polnici v té zemi, volejte naplno a vyřiďte: ‚Shromážděte se! Vejděme do opevněných měst.‘
#4:6 Vztyčte korouhev na Sijónu! Bez prodlení prchněte do bezpečí! Od severu přivedu zlo a velikou zkázu.
#4:7 Z houštiny vystoupil lev. Ničitel pronárodů vyrazil vpřed, vyšel ze svého místa, aby z tvé země učinil spoušť. Tvá města budou vylidněna, budou bez obyvatel.
#4:8 Proto se opásejte žíněnou suknicí, naříkejte a bědujte, neboť se od nás neodvrátil Hospodinův planoucí hněv.“
#4:9 „A stane se v onen den, je výrok Hospodinův, král ztratí rozvahu, ztratí rozvahu i velmožové, kněží budou naplněni děsem a proroci budou trnout.“
#4:10 Říkám: „Ach, Panovníku Hospodine, podvedl jsi tento lid i Jeruzalém. Řekl jsi: ‚Budete mít pokoj‘ a zatím meč pronikl až k duši.“
#4:11 V onen čas bude řečeno tomuto lidu i Jeruzalému: „Žene se žhoucí vítr přes holá návrší v poušti na dceru mého lidu, nebude převívat, nebude pročišťovat.
#4:12 Přižene se ke mně vítr silnější než oni; teď také já nad nimi vyhlásím soud.
#4:13 Hle, vystupuje jako mračna, jako vichřice je jeho vozba, jeho koně jsou rychlejší než orlové. ‚Běda nám budeme vyhubeni!‘
#4:14 Smyj ze svého srdce zlo, Jeruzaléme, a budeš spasen. Jak dlouho budeš v svém nitru přechovávat ničemné myšlenky?
#4:15 Z Danu je slyšet zprávu, z Efrajimského pohoří zlověst.
#4:16 Hle, připomeňte pronárodům, rozhlaste po Jeruzalémě: ‚Přicházejí hlídky z daleké země, jejich hlas se rozléhá nad judskými městy.
#4:17 Budou kolem Jeruzaléma jako ti, kdo hlídají pole, protože se proti mně vzbouřil, je výrok Hospodinův.‘
#4:18 Způsobí ti to tvá cesta a tvé skutky. Je to plod tvé zloby, zhořkne ti a zasáhne tvé srdce!“
#4:19 Nitro, mé nitro! Jak se chvěji! Srdce se mi svírá, srdce mi buší, nemohu mlčet. Slyšíš, má duše, hlas polnice, válečný ryk?
#4:20 Zkáza za zkázou se hlásí, celá země je zpleněna. Náhle byly vypleněny mé stany, znenadání mé stanové houně.
#4:21 Jak dlouho budu vidět korouhev, slyšet hlas polnice?
#4:22 „Můj lid je pošetilý, nezná se ke mně, jsou to synové pomatení, nemají rozum. Jsou moudří, ale ke zlému, dobro konat nedovedou.“
#4:23 Viděl jsem zemi, a hle, je pustá a prázdná, nebesa jsou beze světla.
#4:24 Viděl jsem hory, a hle, třesou se, všechny pahorky se otřásají.
#4:25 Viděl jsem, a hle, nikde žádný člověk, i všechno nebeské ptactvo odletělo.
#4:26 Viděl jsem, a hle, sad je pouští, všechna města v něm jsou podvrácena Hospodinem, jeho planoucím hněvem.
#4:27 Toto praví Hospodin: „Celá země bude zpustošená, ale neskončím s ní docela.
#4:28 Země nad tím bude truchlit, zachmuří se nebe shůry, neboť jsem vyhlásil, co mám v úmyslu, a neželím toho a od toho neupustím.“
#4:29 Před hřmotem jezdců a lučištníků prchá celé město, zalezli do houští, vystoupili na skaliska. Celé město je opuštěné, není žádného, kdo by v něm bydlel.
#4:30 Ale ty, vypleněná, co to děláš? Odíváš se karmínem, zdobíš se zlatými ozdobami, líčíš si oči? Nadarmo se krášlíš, záletníci tebou pohrdají, budou ti ukládat o život.
#4:31 Slyším křik, jako když se rodička bolestí svíjí, sténání, jako když rodí poprvé, křik dcery sijónské, po dechu lapá, rozprostírá dlaně: „Běda mi, jsem tak bezmocná, vydána vrahů!“ 
#5:1 „Proběhněte ulicemi Jeruzaléma, prohledejte jeho náměstí, jen se dívejte a přesvědčte se, zda najdete někoho, kdo uplatňuje právo, kdo hledá pravdu, a já Jeruzalému odpustím.
#5:2 Říkají-li: ‚Jakože živ je Hospodin‘, jistě přísahají křivě.“
#5:3 Hospodine, tobě přece záleží na věrnosti. Biješ je, ale bolestí se nesvíjejí, ničíš je, ale tvá napomínání odmítají. Jejich tváře jsou tvrdší než skála, odmítají se vrátit.
#5:4 Řekl jsem: Jsou to jen nuzáci! Počínají si pošetile, protože neznají Hospodinovu cestu, řády svého Boha.
#5:5 Půjdu tedy k velkým, promluvím s nimi, ti přece znají Hospodinovu cestu, řády svého Boha. Avšak ti rovněž rozlomili jho, zpřetrhali pouta.
#5:6 Proto je zadáví lev z divočiny, zahubí je stepní vlk. U jejich měst bude číhat levhart a rozsápe každého, kdo z nich vyjde, neboť jejich nevěrností je tolik, tak četná jsou jejich odvrácení.
#5:7 „Jak bych ti to mohl odpustit? Tvoji synové mě opustili a přísahají při tom, co není Bůh. Sytím je, a oni cizoloží, v domě nevěstky si zasazují smuteční zářezy.
#5:8 Jsou to bujní hřebci, hned jak vstanou, každý plane vášní k ženě svého bližního.
#5:9 Což je za to nemám ztrestat? je výrok Hospodinův. Což takový pronárod nemá postihnout má pomsta?
#5:10 Vystupte na jeho zdi a ničte, ale ne docela, odstraňte jeho výhonky, protože nepatří Hospodinu.
#5:11 Věrolomně se vůči mně zachoval dům izraelský i dům judský, je výrok Hospodinův.
#5:12 Zapřeli Hospodina, říkají: ‚On to neudělá, nepotká nás nic zlého, nepocítíme meč ani hlad.
#5:13 Proroci přejdou jako vítr, nemají co mluvit; tak ať se jim stane.“
#5:14 Proto praví Hospodin, Bůh zástupů, toto: „Protože mluvíte takovými slovy, hle, já vkládám svá slova do tvých úst, ta budou ohněm a tento lid bude dřívím, jež pozře.“
#5:15 „Hle, přivedu na vás zdaleka pronárod, dome izraelský, je výrok Hospodinův, pronárod mohutný, pronárod odvěký, pronárod, jehož jazyk neznáš, nebudeš rozumět tomu, co mluví.
#5:16 Jeho toulec je otevřený hrob, všichni jsou bohatýři.
#5:17 Pozře tvou žeň, tvůj chléb, pozře tvé syny i tvé dcery, pozře tvé ovce i tvůj dobytek, pozře tvou révu i tvé fíky, mečem vyvrátí tvá opevněná města, na něž se spoléháš.“
#5:18 „Avšak ani v oněch dnech s vámi neskoncuji docela, je výrok Hospodinův.
#5:19 Až se zeptáte: ‚Proč nám to všechno Hospodin, náš Bůh, učinil‘, ty jim řekneš: ‚Tak jako jste vy opustili mne a sloužili ve své zemi cizím bohům, tak budete sloužit cizákům v zemi, která nebude vaše.‘“
#5:20 „Oznamte v domě Jákobově a rozhlašte v Judsku toto:
#5:21 Slyš to přece, lide pomatený, bez rozumu! Oči mají, a nevidí. Uši mají, a neslyší.
#5:22 Což se mě nebudete bát? je výrok Hospodinův, nebudete se přede mnou svíjet bolestí? Moři jsem jako hráz položil písek, je to řád věčný, který nepřestoupí. Bouří se, ale nic nezmůže, jeho vlny hučí, ale nepřestoupí jej.
#5:23 Avšak tento lid má srdce umíněné a vzpurné, odešli z umíněnosti.
#5:24 Ani je nenapadlo říci: ‚Bojme se Hospodina, svého Boha, který dává vydatný déšť, jarní i pozdní, v pravý čas, ke žni stanovené týdny nám zachovává.‘
#5:25 Vaše nepravosti to všechno narušily, vaše hříchy vás zbavily dobra.“
#5:26 „V mém lidu se objevují svévolníci, rozhlížejí se, krčí, jako ptáčníci líčí past a lapají lidi.
#5:27 Jako klec je plná ptáků, tak jsou jejich domy plné lsti. Proto se stali velkými, zbohatli.
#5:28 Tuční jsou, tlustí, zlé události netečně přecházejí, nikoho neobhájí, ani sirotka, a přesto je provází zdar. Za právo ubožáků se na soudu nepostaví.
#5:29 Což je za to nemám trestat? je výrok Hospodinův. Což takový pronárod nemá postihnout má pomsta?
#5:30 Úděsná, otřesná věc se děje v zemi.
#5:31 Proroci prorokují klamně, kněží vládnou na vlastní pěst a můj lid to má rád. Co však uděláte, až nastanou věci poslední?“ 
#6:1 „Prchejte do bezpečí, synové Benjamínovi, utečte z Jeruzaléma, v Tekóji zatrubte na polnici, vztyčte znamení nad Bét-keremem, protože ze severu vyhlíží zlo a veliká zkáza.
#6:2 Líbeznou a rozkošnou dceru sijónskou umlčím.
#6:3 Přitáhnou k ní pastýři se svými stády, postaví si kolem ní stany a každý spase, co bude moci.
#6:4 Posvěťte se k boji proti ní, vstaňte, vyrazme za poledne! Běda nám, neboť den se nachýlil a dlouží se večerní stíny.
#6:5 Vstaňte, vyrazme v noci a její paláce zničme!“
#6:6 Toto praví Hospodin zástupů: „Zporážejte stromy a navršte proti Jeruzalému násep. To město musí být celé potrestáno, uprostřed něho je útisk.
#6:7 Jako studna uchovává vodu čerstvou, tak ono uchovává své zlo. Je v něm slyšet o násilí a zhoubě, stále mám před sebou jeho chorobu a ránu.
#6:8 Jeruzaléme, nech se napomenout, jinak se od tebe odloučím, jinak tě obrátím v zpustošený kraj, v zemi neobyvatelnou.“
#6:9 Toto praví Hospodin zástupů: „Jako vinici ať paběrkují pozůstatek Izraele. Jako vinař vlož znovu svou ruku na úponky.“
#6:10 Ke komu mám mluvit, koho varovat, aby to slyšel? Hle, jejich uši jsou neobřezané, nemohou tomu věnovat pozornost. Hle, Hospodinovo slovo je jim potupou, nelíbí se jim.
#6:11 Jsem naplněn Hospodinovým rozhořčením, marně se snažím je ztlumit. „Vylej je na děti v ulicích i na skupiny jinochů; polapen bude zároveň muž i žena, stařec i ten, kdo dovršil své dny.
#6:12 Jejich domy, pole, spolu s nimi i ženy dostanou jiní, neboť napřáhnu svou ruku na obyvatele země, je výrok Hospodinův.
#6:13 Od nejmenšího až po největšího všichni propadli chamtivosti, od proroka až po kněze všichni klamou.
#6:14 Těžkou ránu mého lidu léčí lehkovážnými slovy: ‚Pokoj, pokoj!‘ Ale žádný pokoj není.
#6:15 Zastyděli se, že páchali ohavnosti? Ne, ti se nestydí, neznají zahanbení. Proto padnou s padajícími, klopýtnou v čase, kdy je budu trestat, praví Hospodin.“
#6:16 Toto praví Hospodin: „Stůjte na cestách a vyhlížejte, ptejte se na stezky věčnosti: Kde je ta dobrá cesta? Vydejte se po ní a vaše duše naleznou klid. Ale oni řekli: ‚Nepůjdeme.‘
#6:17 Ustanovil jsem nad vámi strážné: Věnujte pozornost hlasu polnice! Ale oni řekli: ‚Nebudeme dávat pozor.‘
#6:18 Proto slyšte, pronárody, poznej, pospolitosti, co se s nimi bude dít!
#6:19 Slyš, země: Hle, já uvedu na tento lid zlé věci, ovoce jejich úmyslů, neboť nevěnovali pozornost mým slovům a můj zákon si zprotivili.
#6:20 Nač je mi kadidlo, které přichází z Šeby, nejlepší vonné koření z daleké země? Vaše zápalné oběti nedojdou zalíbení, vaše obětní hody mi nejsou příjemné.“
#6:21 Proto praví Hospodin toto: „Hle, kladu před tento lid překážky a klopýtnou o ně otcové spolu se syny, sousedé i druhové zhynou.“
#6:22 Toto praví Hospodin: „Hle, ze severní země přichází lid, pronárod veliký povstává z nejodlehlejších koutů země.
#6:23 Chopí se luku a oštěpu, budou krutí a nebudou znát slitování, jejich hlas bude hučet jako moře, přijedou na koních seřazeni jako muži k boji proti tobě, dcero sijónská.“
#6:24 Jen jsme o něm zaslechli zprávu, ochably nám ruce, zachvátila nás úzkost a bolestné křeče jako rodičku.
#6:25 Nevycházej na pole, nechoď na cestu, nepřítel má meč. Kolkolem děs!
#6:26 Opásej se žínicí, dcero mého lidu, válej se v popelu, konej smuteční obřady jako nad jednorozeným, přehořce truchli, neboť náhle přitáhne na nás zhoubce.
#6:27 „Ustanovil jsem tě, abys přezkoušel můj lid, zkoumal jeho ryzost. Poznáš a přezkoušíš jejich cestu.
#6:28 Všichni jsou to umíněnci umínění, utrhají, kudy chodí, jsou z bronzu a ze železa, všichni jsou to zhoubci.
#6:29 Měch supí, z ohně vytéká olovo, tavič nadarmo přetavuje, zlí se neodloučí.
#6:30 Nazvou je stříbrem odvrženým, neboť je odvrhl Hospodin.“ 
#7:1 Slovo, které se stalo od Hospodina k Jeremjášovi:
#7:2 „Postav se do brány Hospodinova domu a volej tam toto slovo. Řekneš: Slyšte slovo Hospodinovo, všichni Judejci, kteří vcházíte těmito branami klanět se Hospodinu.“
#7:3 Toto praví Hospodin zástupů, Bůh Izraele: „Napravte své cesty a své skutky a nechám vás přebývat na tomto místě.
#7:4 Nespoléhejte na klamná slova: ‚Je to chrám Hospodinův, chrám Hospodinův, chrám Hospodinův.‘
#7:5 Jestliže napravíte své cesty a své skutky, budete-li mezi sebou zachovávat právo,
#7:6 nebudete-li utlačovat bezdomovce, sirotka a vdovu, nebudete-li na tomto místě prolévat nevinnou krev a nebudete-li chodit ke své škodě za jinými bohy,
#7:7 pak vás nechám přebývat na tomto místě, v zemi, kterou jsem dal vašim otcům na věky věků.
#7:8 Ale vy spoléháte na klamná slova, jež nejsou k užitku.
#7:9 Smí se krást, vraždit, cizoložit, křivě přísahat, pálit kadidlo baalům a chodit za jinými neznámými bohy?
#7:10 A přitom přicházíte a stavíte se přede mne v tomto domě, který je nazýván mým jménem, a říkáte: ‚Jsme vysvobozeni.‘ To proto, abyste mohli páchat všechny tyto ohavnosti?
#7:11 Což se stal ve vašich očích tento dům, který je nazýván mým jménem, úkrytem lupičů? Hle, také já vidím, je výrok Hospodinův.
#7:12 Jen jděte do Šíla, na místo, kde jsem dal zpočátku přebývat svému jménu, a podívejte se, jak jsem s ním pro zlobu svého izraelského lidu naložil.
#7:13 Nyní tedy, protože jste se dopustili všech těch činů, je výrok Hospodinův, mluvil jsem k vám, nepřetržitě jsem mluvil, ale neposlouchali jste, volával jsem vás, ale neodpovídali jste,
#7:14 proto naložím s domem, který je nazýván mým jménem a na který spoléháte, s místem, jež jsem dal vám a vašim otcům, stejně, jako jsem naložil se Šílem.
#7:15 Odmrštím vás od své tváře, jako jsem odmrštil všechny vaše bratry, všechno potomstvo Efrajimovo.“
#7:16 „Nemodli se za tento lid, nepozvedej svůj hlas k bědování a modlitbě za ně a nenaléhej na mě, neboť tě nevyslyším.
#7:17 Což sám nevidíš, čeho se v judských městech a na ulicích Jeruzaléma dopouštějí?
#7:18 Synové sbírají dříví, otcové zapalují oheň a ženy hnětou těsto, dělají obětní koláč pro královnu nebes a přinášejí úlitby jiným bohům, aby mě uráželi.
#7:19 Urážejí tím opravdu mne? je výrok Hospodinův. Což nedělají ostudu sami sobě?“
#7:20 Proto praví Panovník Hospodin toto: „Hle, můj hněv a mé rozhořčení se vyleje na toto místo, na lidi i dobytek, na stromoví na poli i na plody země a bude hořet a neuhasne.“
#7:21 Toto praví Hospodin zástupů, Bůh Izraele: „Přidejte své zápalné oběti ke svým obětním hodům a jezte maso!
#7:22 Nemluvil jsem k vašim otcům o zápalné oběti a obětním hodu v den, kdy jsem je vyvedl ze země egyptské; žádný takový příkaz jsem jim nedal.
#7:23 Přikázal jsem jim jedině toto: ‚Poslouchejte mě a budu vám Bohem a vy budete mým lidem, choďte po všech cestách, jak jsem vám přikázal, a povede se vám dobře.‘
#7:24 Ale neposlechli, své ucho nenaklonili, žili podle záměrů svého zarputilého a zlého srdce, obraceli se ke mně zády, a ne tváří,
#7:25 a to ode dne, kdy vaši otcové vyšli z egyptské země, až dodnes. Posílával jsem k nim nepřetržitě všechny své služebníky proroky, posílal jsem je denně,
#7:26 ale neuposlechli mě, své ucho nenaklonili, zůstali tvrdošíjní a jednali hůř než jejich otcové.
#7:27 Vyřidíš jim všechna tato slova, ale oni tě neposlechnou, budeš k nim volat, a neodpovědí ti.
#7:28 Řekneš jim: Toto je pronárod, který neposlouchá Hospodina, svého Boha, a nedá se napomenout. Ztratila se věrnost, je odťata od jejich úst.“
#7:29 Odstřihni si vlasy svého zasvěcení a odhoď je, zapěj na holých návrších žalozpěv, protože Hospodin zavrhl a odmrštil pokolení, na něž se hněvá.
#7:30 Judští synové se dopouštěli toho, co je zlé v mých očích, je výrok Hospodinův. V domě, který je nazýván mým jménem, postavili ohyzdné modly, a tak jej poskrvnili.
#7:31 V Tófetu, který je v Údolí syna Hinómova, postavili posvátná návrší a spalovali své syny a své dcery ohněm. To jsem jim nepřikázal, ani mi to nepřišlo na mysl.
#7:32 Proto hle, přicházejí dny, je výrok Hospodinův, kdy se už nebude říkat: ‚Tófet‘ ani ‚Údolí syna Hinómova‘, nýbrž ‚Údolí vraždění‘. V Tófetu se bude pohřbívat, a až nebude místo,
#7:33 stanou se mrtvoly tohoto lidu pokrmem nebeskému ptactvu a zemskému zvířectvu, jež nikdo nevyplaší.
#7:34 Způsobím, že v judských městech a na ulicích Jeruzaléma přestane hlas veselí, hlas radosti, hlas ženicha i hlas nevěsty, neboť země bude obrácena v trosky.“ 
#8:1 „V onen čas, je výrok Hospodinův, vynesou z hrobů kosti judských králů, kosti judských velmožů, kosti kněží, kosti proroků i kosti obyvatel Jeruzaléma
#8:2 a předhodí je slunci, měsíci a celému nebeskému zástupu, které milovali a kterým sloužili, za nimiž chodili, jichž se dotazovali a jimž se klaněli. Nebudou sebrány, nebudou pohřbeny, budou jako hnůj na povrchu země.
#8:3 A všichni ostatní, kteří zůstanou z této zlé čeledi na všech ostatních místech, kam jsem je zahnal, raději zvolí smrt než život, je výrok Hospodina zástupů.“
#8:4 Řekneš jim: Toto praví Hospodin: „Což ti, kteří padli, už nemohou povstat? Kdo se odvrátil, nemůže se vrátit?
#8:5 Proč se tento odpadlý jeruzalémský lid odvrátil natrvalo? Pevně se drží lsti a zdráhá se navrátit?
#8:6 Pozorně jsem naslouchal a slyšel. Není v pořádku, co mluví, nikdo nelituje zla, jež spáchal, neřekne: ‚Čeho jsem se to dopustil!‘ Všichni se znovu dávají v běh jako kůň, který pádí do bitvy.
#8:7 I čáp na nebi zná svůj čas, hrdlička, vlaštovka a jeřáb dodržují čas svého příletu, ale můj lid nezná Hospodinovy řády.
#8:8 Jak můžete říkat: ‚Jsme moudří, my máme Hospodinův zákon‘? Ano, jenže jej falšuje falešné rydlo písařů.
#8:9 Stud polil moudré, zachvátil je děs, jsou polapeni. Hle, odvrhli Hospodinovo slovo, k čemu jim bude moudrost?
#8:10 Proto dám jejich ženy jiným, jejich pole podmanitelům; neboť od nejmenšího až po největšího všichni propadli chamtivosti. Od proroka až po kněze všichni klamou.
#8:11 Těžkou ránu dcery mého lidu léčí lehkovážnými slovy: ‚Pokoj, pokoj!‘ Ale žádný pokoj není.
#8:12 Zastyděli se, že páchali ohavnosti? Ne, ti se nestydí, neznají zahanbení. Proto padnou s padajícími, klopýtnou v čase, kdy je budu trestat, praví Hospodin.“
#8:13 „Sklidím je nadobro, je výrok Hospodinův, na vinné révě nezbudou žádné hrozny, na fíkovníku žádné fíky, jen zvadlé listí. Pustím na ně ty, kdo se přes ně přeženou.
#8:14 ‚Nač tu sedíme? Seberte se, vstupme do opevněných měst a tam zmlkneme, neboť nás umlčel Hospodin, náš Bůh, napojil nás otrávenou vodou, protože jsme hřešili proti Hospodinu.
#8:15 S nadějí jsme vyhlíželi pokoj, ale nic dobrého nepřichází, čas uzdravení, a hle, předěšení!‘
#8:16 Od Danu je už slyšet frkání jeho koní, hlučným ržáním jeho hřebců se chvěje celá země. Přitáhnou a vyjedí zemi se vším, co je na ní, město i jeho obyvatele.
#8:17 Hle, posílám na vás hady, bazilišky, na něž neplatí žádné zaříkání, a uštknou vás, je výrok Hospodinův.“
#8:18 Což mohu okřát v těch strastech? Srdce mám zemdlené.
#8:19 Hlasitě volá o pomoc dcera mého lidu ze země předaleké. Což není Hospodin na Sijónu? Jeho Král tam není? „Proč mě uráželi svými modlami, cizáckými přeludy?“
#8:20 Minula žeň, skončilo léto, a spása nikde.
#8:21 Těžkou ranou dcery svého lidu jsem těžce raněn, tíží mě chmury, zmocnil se mě úděs.
#8:22 Což není v Gileádu balzám, což tam není lékař? Proč není zhojena rána dcery mého lidu?
#8:23 Kdo poskytne mé hlavě vodu a očím zdroj slzí, abych oplakával dnem i nocí skolené syny dcery svého lidu? 
#9:1 Kéž bych měl na poušti místo, kde mohou pocestní přenocovat. Mám opustit svůj lid a odejít od nich? Neboť jsou to samí cizoložníci, shromáždění věrolomných.
#9:2 „Napínají svůj jazyk jako luk, klamem, a ne pravdou se zmocňují země. Neboť jdou od zla ke zlu a ke mně se neznají, je výrok Hospodinův.
#9:3 Střezte se svého druha, na žádného bratra nespoléhejte, neboť každý bratr je samý úskok a každý druh utrhá, kudy chodí.
#9:4 Kdekdo chce svého druha přelstít a neřekne pravdu. Učí svůj jazyk mluvit jen klam, vyčerpávají se nepravostmi.
#9:5 Bydlíš uprostřed lsti; ve své lsti se mě zdráhají poznat, je výrok Hospodinův.“
#9:6 Proto praví Hospodin zástupů toto: „Hle, bude je tříbit a budu zkoumat, jak bych měl naložit s dcerou svého lidu.
#9:7 Šíp, který zabíjí, je jejich jazyk, jejich ústa mluví lstivě. S bližním mluví o pokoji, ale ve svém nitru strojí úklady.
#9:8 Což je za to nemám trestat? je výrok Hospodinův. Což takový pronárod nemá postihnout má pomsta?
#9:9 Běduji a pláči nad horami, nad stepními pastvinami pěji žalozpěv, že byly vypáleny; už jimi nikdo neprochází, ani bečení stád není slyšet. Od nebeského ptactva až po dobytek se všechno vyplašilo a odtáhlo.
#9:10 Z Jeruzaléma učiním hromadu kamení, domov šakalů, a z judských měst zpustošený kraj bez obyvatel.“
#9:11 Kdo je tak moudrý, aby to pochopil a rozhlašoval, co k němu promluvila Hospodinova ústa? Proč přišla nazmar země, je vypálenou stepí, jíž nikdo neprochází?
#9:12 Hospodin praví: „Poněvadž opustili můj zákon, který jsem jim předložil, neposlouchali mě a jím se neřídili,
#9:13 ale v zarputilosti svého srdce chodili za baaly, jak je tomu naučili jejich otcové,“
#9:14 proto praví Hospodin zástupů, Bůh Izraele, toto: „Hle, nakrmím tento lid pelyňkem a napojím je otrávenou vodou,
#9:15 rozptýlím je mezi pronárody, které neznali oni ani jejich otcové, a pošlu za nimi meč, dokud s nimi neskoncuji.“
#9:16 Toto praví Hospodin zástupů: „Pochopte to a zavolejte plačky, ať přijdou! Pošlete pro moudré ženy, ať přijdou!
#9:17 Ať rychle přiběhnou a bědují nad námi, ať z našich očí stékají slzy a z našich řas proudí voda.“
#9:18 Ze Sijónu je slyšet hlasité bědování: „Jak jsme popleněni, velice zahanbeni! Museli jsme opustit zemi, naše příbytky strhli.“
#9:19 „Slyšte, ženy, slovo Hospodinovo, nechť vaše ucho přijme slovo z jeho úst. Své dcery učte bědování, jedna druhou žalozpěvu.
#9:20 Našimi okny vstoupila smrt, vnikla do našich paláců. Vyhubila děti v ulicích, jinochy na náměstí.“
#9:21 Vyhlas: „Toto je výrok Hospodinův: Mrtvá lidská těla leží jako hnůj na povrchu pole, jak snopky za žencem, jež nikdo nesbírá.“
#9:22 Toto praví Hospodin: „Ať se moudrý nechlubí svou moudrostí, ať se bohatýr nechlubí svou bohatýrskou silou, ať se boháč nechlubí svým bohatstvím.
#9:23 Chce-li se něčím chlubit, ať se chlubí, že je prozíravý a zná mne; neboť já Hospodin prokazuji milosrdenství a vykonávám na zemi soud a spravedlnost; to jsem si oblíbil, je výrok Hospodinův.“
#9:24 „Hle, přicházejí dny, je výrok Hospodinův, kdy potrestám všechny, kteří se obřezávají, a přesto jsou neobřezaní:
#9:25 Egypt, Judu, Edóma, Amónovce, Moába a všechny, kdo si vyholují skráně a sídlí na stepi, neboť všechny tyto pronárody jsou neobřezané. I celý dům izraelský má neobřezané srdce!“ 
#10:1 Dome izraelský, slyšte slovo, které proti vám promluvil Hospodin.
#10:2 Toto praví Hospodin: „Neučte se cestě pronárodů a neděste se nebeských znamení, i když se jich děsí pronárody.
#10:3 To, čím se řídí národy, je pouhý přelud, dřevo poražené v lese, výrobek přitesaný řemeslnou rukou.
#10:4 Krášlí jej stříbrem a zlatem, hřebíky a kladivy upevňují, aby to nebylo vratké.
#10:5 Jsou jako strašák v okurkovém poli. Nemluví, musí se nosit, sami neudělají ani krok. Nebojte se jich, nemohou udělat nic zlého ani dobrého.“
#10:6 Nikdo není jako ty, Hospodine, jsi veliký a veliké je tvé jméno pro tvou bohatýrskou sílu.
#10:7 Kdo by se tě nebál, Králi pronárodů? Úcta patří jenom tobě! Ze všech mudrců pronárodů a ze všech království není nikdo jako ty.
#10:8 Jsou tupí a hloupí do jednoho, dají se vodit přeludy, pouhým dřevem,
#10:9 tepaným stříbrem dovezeným z Taršíše a zlatem z Úfazu, výrobkem řemeslníka a rukou zlatníkových; je oděno purpurem a nachem, ale vše je výrobek dovedných řemeslníků.
#10:10 Avšak Hospodin je Bůh pravý. On je Bůh živý a Král věčný. Před jeho rozlícením se chvěje země, pronárody neobstojí před jeho hrozným hněvem.
#10:11 Řekněte jim toto: „Bohové, kteří neudělali nebe ani zemi, zmizí ze země i zpod nebes.“
#10:12 On svou silou učinil zemi, svou moudrostí upevnil svět, svým rozumem napjal nebesa.
#10:13 Když vydá hlas, shlukují se na nebi vody, přivádí mlhu od končin země, déšť provází blesky, ze svých zásobnic vyvádí vítr.
#10:14 Každý člověk je tupec, neví-li, že se každý zlatník pro své modly dočká hanby. Vždyť jeho lité modly jsou klam, ducha v nich není.
#10:15 Jsou přelud, podvodný výtvor. V čase, kdy je budu trestat, zhynou.
#10:16 Díl Jákobův není jako oni, vždyť všechno vytvořil on. Izrael je dědičným kmenem toho, jehož jméno je Hospodin zástupů.
#10:17 Seber ze země svůj uzlík, ty která sídlíš v obležení.
#10:18 Toto praví Hospodin: „Hle, tentokrát začnu metat z praku obyvatele země a budu je soužit, aby to pocítili.“
#10:19 Běda mi pro mou těžkou ránu, jsem proboden, raněn. Řekl jsem: Je to nemoc, kterou musím snášet.
#10:20 Můj stan je zpustošen a všechna má stanová lana zpřetrhána. Moji synové ode mne odešli, nejsou tu. Není tu nikdo, kdo by ještě postavil můj stan a napjal mé stanové houně.
#10:21 Pastýři jsou tupí, nedotazují se Hospodina. Nejsou prozíraví, všechna jejich stáda jsou rozprášena.
#10:22 Slyš, zpráva! Hle, přichází: „Velké dunění ze severní země učiní z judských měst zpustošený kraj, domov šakalů.“
#10:23 Je mi známo, Hospodine, že cesta nezávisí na člověku ani jistota kroku na tom, kdo jde.
#10:24 Potrestej mě, Hospodine, podle práva, ale ne ve hněvu, abys mě nezničil.
#10:25 Vylej svoje rozhořčení na pronárody, jež neznají se k tobě, na čeledi, které nevzývají tvoje jméno, vždyť pozřely Jákoba, pozřely ho zcela, zpustošily jeho nivy. 
#11:1 Slovo, které se stalo k Jeremjášovi od Hospodina:
#11:2 „Slyšte slova této smlouvy. Vyhlásíš je mužům judským a obyvatelům Jeruzaléma.
#11:3 Řekneš jim: Toto praví Hospodin, Bůh Izraele: Proklet buď každý, kdo neposlouchá slova této smlouvy,
#11:4 jak jsem přikázal vašim otcům v den, kdy jsem je vyvedl z egyptské země, z tavicí pece železné: Poslouchejte mě a plňte má slova, všechno, co vám přikazuji, a budete mým lidem a já vám budu Bohem.
#11:5 Tak splním přísahu, jíž jsem se zapřísáhl vašim otcům, že jim dám zemi oplývající mlékem a medem, jak je tomu dnes.“ Odpověděl jsem a řekl: „Zajisté, Hospodine.“
#11:6 Dále mi Hospodin řekl: „Vyhlas v judských městech a na ulicích Jeruzaléma všechna tato slova: Slyšte slova této smlouvy a plňte je.
#11:7 Důrazně jsem varoval vaše otce ode dne, kdy jsem je vyvedl z egyptské země, až dodnes. Nepřetržitě jsem je varoval slovy: Poslouchejte mě!
#11:8 Ale neposlouchali, ucho nenaklonili; každý žil podle svého zarputilého a zlého srdce. Proto na ně uvedu všechna slova této smlouvy, kterou jsem přikázal plnit, ale oni ji neplnili.“
#11:9 Hospodin mi řekl: „Byla shledána zrada judských mužů a obyvatelů Jeruzaléma:
#11:10 Vrátili se k dřívějším nepravostem svých otců, kteří se vzpírali poslouchat má slova. Chodili za jinými bohy a sloužili jim. Dům izraelský i dům judský zrušily mou smlouvu, kterou jsem uzavřel s jejich otci.“
#11:11 Proto praví Hospodin toto: „Hle, já na ně uvedu zlé věci, jimž nebudou moci ujít. Budou ke mně úpět, ale já je nevyslyším.
#11:12 Ať si judská města a obyvatelé Jeruzaléma jdou úpět k bohům, jimž pálí kadidlo; nebudou s to je spasit, až na ně dolehne zlý čas.
#11:13 Vždyť máš tolik bohů, kolik měst, Judo. A kolik je jeruzalémských ulic, tolik oltářů jste zřídili Ohavě, oltářů pro pálení kadidla Baalovi.
#11:14 Nemodli se za tento lid. Nepozvedej svůj hlas k bědování a modlitbě za ně. V čas, kdy budou ke mně pro to zlo volat, nebudu slyšet.“
#11:15 „Co pohledává Jákob, můj milý, v mém domě? Pletichy tam kují mnozí. Může obětní maso přenést zlo, které tě stihne? Budeš potom jásat?
#11:16 Zelenající se olivou, krásnou a s nádherným ovocem, tě nazval Hospodin. S velikým hukotem zanítí pod ní oheň a ten stráví její ratolesti.“
#11:17 Hospodin zástupů, který tě zasadil, promluvil proti tobě zlé věci pro zlo, jehož se dopustil dům izraelský a dům judský: aby mě uráželi, pálili kadidlo Baalovi.
#11:18 Hospodin mi dal vědět o jejich skutcích; znám je. Kdysi jsi mi je ukázal.
#11:19 Byl jsem důvěřivý jako beránek vedený na porážku. Nevěděl jsem, co proti mně zamýšlejí. „Porazme strom i s ovocem, vytněme jej ze země živých, ať už není připomínáno jeho jméno.“
#11:20 Hospodine zástupů, soudce spravedlivý, ty zkoumáš ledví i srdce, kéž spatřím tvou pomstu nad nimi. Vždyť tobě jsem předložil svůj spor.
#11:21 Proto praví Hospodin toto o anatótských mužích, kteří ti ukládají o život. Říkají: „Neprorokuj ve jménu Hospodina a nezemřeš naší rukou.“
#11:22 Proto praví Hospodin zástupů toto: „Hle, já je ztrestám. Jinoši zemřou mečem, synové a dcery jim zemřou hladem.
#11:23 Nikdo po nich nezůstane, neboť uvedu zlé věci na anatótské muže v roce, kdy je budu trestat.“ 
#12:1 Jsi spravedlivý, Hospodine, i když s tebou vedu spor. Chci s tebou mluvit o tvých soudech: Proč je úspěšná cesta svévolníků? Všichni, kdo se dopouštějí věrolomnosti, žijí si v klidu.
#12:2 Zasadil jsi je a oni zakořenili, vyrůstají a nesou plody. Jsi blízko, mají tě v ústech, ale daleko jsi jejich ledví.
#12:3 Ty mě, Hospodine, znáš a vidíš mě, zkoušíš mé srdce, je s tebou. Vytřiď je jak ovce na porážku, zasvěť je ke dni pobíjení.
#12:4 Jak dlouho ještě bude země truchlit a bylina schnout na všech polích? Pro zlobu těch, kdo na ní bydlí, jsou vyhlazována zvířata i ptactvo. Ale říkají: „Do naší budoucnosti Bůh nevidí.“
#12:5 Běžel jsi s pěšími a unavili tě; jak bys chtěl závodit s koni? V pokojné zemi žiješ v bezpečí, co si počneš v houštinách Jordánu?
#12:6 Vždyť i tvoji bratři a dům tvého otce se vůči tobě zachovali věrolomně. Také oni z plna hrdla za tebou volali. Nevěř jim, i když s tebou pěkně mluví.“
#12:7 „Svůj dům jsem opustil, odvrhl jsem své dědictví; co mi bylo nejmilejší, vydal jsem do rukou nepřátel.
#12:8 Mé dědictví se mi stalo lvem z divočiny, vydává hlas proti mně. Proto je nemám rád.
#12:9 Je mé dědictví pro mne kropenatý dravec? Je obklopeno dravci? Jděte, sežeňte všechnu zvěř polí, přiveďte ji, ať se nažere.
#12:10 Mnozí pastýři zničili mou vinici, pošlapali svůj podíl, můj žádoucí podíl obrátili v poušť a zpustošený kraj.
#12:11 Učinili z ní zpustošený kraj; leží přede mnou truchlivá a zpustošená. Zpustošená je celá země a není nikdo, kdo by si to bral k srdci.“
#12:12 Po všech lysých návrších pouště přicházejí zhoubci. Hospodinův meč bude požírat od jednoho konce země ke druhému a žádný tvor nebude mít pokoj.
#12:13 Zaseli pšenici, sklidí trní; vyčerpávali se bez užitku. Styďte se za to, co jste vytěžili pro Hospodinův planoucí hněv.
#12:14 Toto praví Hospodin o všech mých zlých sousedech, kteří sahají na dědictví, jež dal Izraeli, svému lidu: „Hle, já je vyvrátím z jejich půdy, i dům Judův vyvrátím z jejich středu.
#12:15 Až je vyvrátím, potom se nad nimi opět slituji a přivedu každého zpět do jeho dědictví v jeho zemi.
#12:16 A jestliže by se pilně učili cestám mého lidu a přísahali při mém jménu: ‚Jakože živ je Hospodin‘, jak učili můj lid přísahat při Baalovi, pak jim bude zbudován dům mezi mým lidem.
#12:17 Jestliže však nebudou poslouchat, zcela vyvrátím onen pronárod a nechám jej zahynout, je výrok Hospodinův.“ 
#13:1 Toto mi řekl Hospodin: „Jdi si koupit lněný pás, dej si jej kolem beder a nesmáčej jej ve vodě.“
#13:2 Koupil jsem si tedy pás, jak ke mně Hospodin promluvil, a dal jsem si jej kolem beder.
#13:3 Slovo Hospodinovo stalo se ke mně podruhé:
#13:4 „Vezmi pás, který sis koupil a jejž máš kolem beder, vstaň a jdi k Eufratu a tam jej ukryj ve skalní trhlině.“
#13:5 I šel jsem a ukryl jsem jej u Eufratu, jak mi Hospodin přikázal.
#13:6 Po uplynutí mnoha dnů mi Hospodin opět řekl: „Vstaň a jdi k Eufratu a vezmi odtud pás, který jsem ti přikázal tam ukrýt.“
#13:7 Šel jsem tedy k Eufratu a vyhrabal jsem pás z místa, kam jsem jej ukryl. A hle, pás byl zničen, k ničemu se nehodil.
#13:8 Tu se ke mně stalo slovo Hospodinovo:
#13:9 „Toto praví Hospodin: Tak zničím pýchu Judy a nesmírnou pýchu Jeruzaléma.
#13:10 Tento zlý lid, který se vzpírá slyšet má slova a žije si podle svého zarputilého srdce, chodí za jinými bohy, slouží jim a klaní se jim, bude jako tento pás, který se k ničemu nehodí.
#13:11 Jako pás přilne k bedrům muže, tak jsem k sobě přimkl celý dům izraelský a dům judský, je výrok Hospodinův, aby se mi stali lidem, jménem, chválou a oslavou. Ale neuposlechli.“
#13:12 Proto jim vyřídíš toto slovo: „Toto praví Hospodin, Bůh Izraele: Každý měch se plní vínem. Namítnou ti: ‚Cožpak nevíme, že se každý měch plní vínem?‘
#13:13 Nato jim řekneš: Toto praví Hospodin: Hle, já opojím všechny obyvatele této země, krále, kteří sedí na Davidovoě trůnu, kněze a proroky i všechny obyvatele Jeruzaléma,
#13:14 a roztříštím je jednoho o druhého, otce i syny, je výrok Hospodinův; budu bez soucitu, bez lítosti a bez slitování, až je budu ničit.“
#13:15 Slyšte a naslouchejte, nebuďte domýšliví, tak promluvil Hospodin.
#13:16 Vzdejte slávu Hospodinu, svému Bohu, dříve než způsobí tmu, dříve než nohama narazíte o zšeřelé hory. S nadějí se upínáte k světlu, ale on učinil z něho šero smrti, obestřel je černým mračnem.
#13:17 Jestliže o tom nechcete slyšet, vskrytu bude plakat má duše kvůli vaší povýšenosti, slzy mi proudem potečou z očí, až bude Hospodinovo stádce vedeno do zajetí.
#13:18 Poruč králi a královně: „Seďte v ponížení, vždyť spadla z vašich hlav koruna, která vás zdobila.
#13:19 Města na jihu byla uzavřena a není nikoho, kdo by je otevřel. Celý Juda byl přestěhován, přestěhován do jednoho.“
#13:20 Pozvedni oči a hleď na ty, kdo přicházejí od severu. „Kde je tobě svěřené stádo, ovce, jež byly tvou ozdobou?
#13:21 Co řekneš, až nad tebou ustanoví tvými pohlaváry a vůdci ty, které jsi učila? Což tě pak nezachvátí porodní bolesti jako rodičku?
#13:22 Řekneš-li si v srdci: ‚Proč mě to potkalo?‘ Pro množství tvých nepravostí byla ti vyhrnuta sukně a bylas vydána násilí.
#13:23 Může snad Kúšijec změnit svou kůži? Či levhart svou skvrnitost? Jak vy byste mohli jednat dobře, když jste se naučili páchat zlo? -
#13:24 Rozptýlím je jako stébla slámy, jež poletují ve větru z pouště.“
#13:25 „To je tvůj los, úděl, jejž jsem ti vyměřil, je výrok Hospodinův. Poněvadž jsi na mě zapomněla, spoléhala ses na klam.
#13:26 To já jsem ti zvedl sukni přes tvář a byla vidět tvá hanba,
#13:27 tvé cizoložství a tvá vášnivost, mrzký skutek tvého smilstva. Na pahorcích v poli jsem viděl tvé ohyzdné modly. Běda tobě, Jeruzaléme. Nejsi čistý! Jak dlouho ještě?“ 
#14:1 O slovu Hospodinově, které se stalo k Jeremjášovi ohledně sucha.
#14:2 Juda truchlí, jeho brány zchátraly a chmurně klesly k zemi. Z Jeruzaléma stoupá žalostný křik.
#14:3 Jejich vznešení posílají svou čeleď pro vodu. Ta chodí k nádržím a vodu nenachází, vrací se s prázdnými nádobami. Stydí se a hanbí a zakrývá si hlavu.
#14:4 Jsou zděšeni nad rolemi, neboť v zemi nenastává doba dešťů. Oráči se stydí, zakrývají si hlavu.
#14:5 I laň v poli opouští, co porodila, neboť není zeleň.
#14:6 Na lysých návrších postávají divocí osli, větří jako šakalové, vypovídá jim zrak, neboť nejsou byliny.
#14:7 „Jestliže mluví proti nám naše nepravosti, jednej, Hospodine, pro své jméno. Mnohokráte jsme se odvrátili, hřešili jsme proti tobě.
#14:8 Naděje Izraele, jeho spasiteli v čase soužení! Proč jsi v zemi jako host, jak pocestný, který se zdrží jen přes noc?
#14:9 Proč jsi jak zdrcený člověk, jak bohatýr, který není s to spasit? Ty jsi Hospodine, uprostřed nás a nazývají nás tvým jménem. Nezříkej se nás!“
#14:10 Toto praví Hospodin o tomto lidu: „Milují toulky a své nohy nešetří„; proto v nich nemá Hospodin zalíbení. Bude nyní připomínat jejich nepravost a trestat jejich hřích.
#14:11 Hospodin mi poručil: „Nemodli se za dobro tohoto lidu.
#14:12 Ani když se budou postit, nevyslyším jejich bědování, ani když připraví oběť zápalnou a přídavnou, nenajdu v nich zalíbení. Zcela s nimi skoncuji mečem, hladem a morem.“
#14:13 Nato jsem řekl: „Ach, Panovníku Hospodine, hle, proroci jim říkají: Nespatříte meč a hlad mít nebudete, nýbrž dám vám na tomto místě pravý pokoj.“
#14:14 Hospodin mi řekl: „Ti proroci prorokují mým jménem klam. Neposlal jsem je a nepřikázal jsem jim to a nemluvil jsem k nim. Prorokují vám klamné vidění, nicotnou věštbu a lest vlastního srdce.“
#14:15 Toto praví Hospodin proti těm prorokům, kteří prorokují jeho jménem: „Já jsem je neposlal. To oni říkají: ‚V této zemi nebude meč a hlad.‘ Mečem a hladem vymřou ti proroci do jednoho.
#14:16 A lid, jemuž prorokovali, bude ležet po ulicích Jeruzaléma sražen hladem a mečem a nikdo je nepohřbí, ani jejich ženy ani jejich syny a dcery. Tak vyleji na ně jejich zlobu.“
#14:17 Řekni jim toto slovo: Z mých očí tekou slzy dnem i nocí bez přestání, neboť těžká rána rozdrtila panenskou dceru mého lidu velikou, bolestnou ranou.
#14:18 Vyjdu-li na pole, hle, skolení mečem, přijdu-li do města, hle, zesláblí hlady; jak prorok, tak kněz procházejí zemí a nevědí si rady.“
#14:19 „Což jsi opravdu zavrhl Judu? Tvá duše si zprotivila Sijón? Proč nás biješ, což pro nás není uzdravení? S nadějí jsme vyhlíželi pokoj, ale nic dobrého nepřichází, čas uzdravení, a hle předěšení.
#14:20 Hospodine, uznáváme svou svévoli, i nepravost svých otců, hřešili jsme proti tobě.
#14:21 Neodmítej nás však pro své jméno, nenech zhanobit trůn své slávy. Rozpomeň se a neruš svou smlouvu s námi.
#14:22 Což jsou mezi přeludy pronárodů ti, kdo by mohli dát déšť? Což nebesa vydávají deště? Nejsi to ty sám, Hospodine, náš Bože? Skládáme naději v tebe, neboť ty konáš všechny tyto věci!“ 
#15:1 Hospodin mi řekl: „I kdyby se přede mne postavil Mojžíš a Samuel, nepřiklonil bych se k tomuto lidu. Pošli jej ode mne pryč, ať odejde!
#15:2 Až se tě budou ptát: ‚Kam máme odejít?‘, řekneš jim: Toto praví Hospodin: Kdo je určen smrti, propadne smrti, kdo meči, meči, kdo hladu, hladu, kdo zajetí, zajetí.“
#15:3 Ztrestám je na čtverý způsob, je výrok Hospodinův: mečem, by zabíjel, psy, aby vláčeli, nebeským ptactvem a zemským zvířectvem, aby žralo a ničilo.
#15:4 Učiním je obrazem hrůzy pro všechna království země kvůli Menašeovi, synu judského krále Chizkijáše, kvůli tomu, co páchal v Jeruzalémě.“
#15:5 „Kdo s tebou bude mít soucit, Jeruzaléme? Kdo ti projeví účast? Kdo k tobě zajde, aby se zeptal, jak se ti daří?
#15:6 Ty sám sis mě přestal všímat, je výrok Hospodinův, odešel jsi ode mne. Napřáhnu na tebe ruku a zničím tě; stálé ohledy mě unavují.
#15:7 Převěju je lopatou na obilí v branách země, o děti připravím, zahubím svůj lid: od svých cest se neodvrátili.
#15:8 Zůstane mi v něm víc vdov než písku v mořích. Za poledne na ně přivedu zhoubce, na matku i na mládence. Náhle přepadnu město hněvem a hrůzou.
#15:9 Zchřadne sedminásobná rodička, vypustí duši. Ještě za dne jí zajde slunce. Bude se stydět, rdít studem. Ty kteří zůstanou, vydám meči před tváří jejich nepřátel, je výrok Hospodinův.“
#15:10 Běda mi, matko má, že jsi mě zrodila, muže sporu a sváru pro celou zemi. Nic jsem si nevypůjčil, nikdo mi nepůjčil, všichni mi klnou.
#15:11 Hospodin praví: „Nepřichystal jsem ti dobro? Nesrazil jsem kvůli tobě nepřítele v čase zlém, v čase soužení?
#15:12 Dá se přerazit železo, železo ze severu, či bronz?
#15:13 Tvé zboží a poklady vydám bez náhrady v plen za všechny tvé hříchy, páchané po celém tvém území.
#15:14 Dám tě odvést tvými nepřáteli do země, kterou neznáš. Rozhoří se proti vám oheň roznícený mým hněvem.“
#15:15 Tys o tom věděl, Hospodine, pamatuj na mě a ujmi se mne. Vykonej za mne pomstu na těch, kdo mě pronásledují. Pro svou shovívavost mě odtud ještě neber. Věz, že pro tebe snáším potupu.
#15:16 Jakmile se objevila tvá slova, pozřel jsem je. Tvá slova mi byla veselím a radostí srdce. Nazývají mě tvým jménem, Hospodine, Bože zástupů.
#15:17 Nesedám v kruhu těch, kdo se vysmívají, a nejásám. Kvůli tvé ruce sedím osamocen, neboť jsi mě naplnil svým hrozným hněvem.
#15:18 Proč je má bolest trvalá a má rána nevyléčitelná a nechce se hojit? Stal ses mi tím, kdo jako by lhal, vodou nestálou.
#15:19 Proto praví Hospodin toto: „Jestliže se obrátíš, vrátím tě k sobě, staneš přede mnou. Budeš-li pronášet vzácná slova, nic bezcenného, budeš mými ústy. Oni se musejí obrátit k tobě, ty se k nim neobracej.
#15:20 Učinil jsem tě vůči tomuto lidu nedostupnou bronzovou hradbou. Budou proti tobě bojovat, ale nepřemohu tě, neboť já jsem s tebou, abych tě spasil a vysvobodil, je výrok Hospodinův.
#15:21 Vysvobodím tě z rukou zlovolníků a vykoupím tě ze spárů katanů.“ 
#16:1 Stalo se ke mně slovo Hospodinovo:
#16:2 „Neber si ženu. Nebudeš mít na tomto místě syny ani dcery.
#16:3 Toto praví Hospodin o synech a dcerách zrozených na tomto místě, o jejich matkách, které je porodily, a o jejich otcích, kteří je zplodili v této zemi:
#16:4 Zemřou na smrtelné nemoci. Nebude se nad nimi naříkat a nebudou pohřbeni. Budou hnojem na povrchu role. Skoncuje s nimi meč a hlad. Jejich mrtvoly budou za pokrm nebeskému ptactvu a zemskému zvířectvu.“
#16:5 Toto praví Hospodin: „Nevcházej do domu smuteční hostiny, nechoď tam, kde se naříká, a neprojevuj jim soustrast, neboť já jsem tomuto lidu vzal svůj pokoj, je výrok Hospodinův, spolu s milosrdenstvím a slitováním.
#16:6 Velcí i malí v této zemi zemřou a nebudou pohřbeni. Nebude se nad nimi naříkat a nikdo si pro ně nebude dělat smuteční zářezy ani lysinu.
#16:7 Při truchlení nebudou lámat chléb, aby potěšili toho, kdo truchlí nad mrtvým, nedají napít z kalicha útěchy těm, kdo truchlí nad otcem či matkou.
#16:8 Nevejdeš ani do domu, v němž se hoduje, abys tam s nimi poseděl, pojedl a popil.“
#16:9 Toto praví Hospodin zástupů, Bůh Izraele: „Hle, způsobím, že na tomto místě, před vaším zrakem a za vašich dnů přestane veselí a hlas radosti, hlas ženicha a hlas nevěsty.“
#16:10 „Až oznámíš tomuto lidu všechna tato slova, zeptají se tě: ‚Proč Hospodin promluvil proti nám všechno toto velké zlo? Jaká je naše nepravost? Jaký je náš hřích? V čem jsme hřešili proti Hospodinu, našemu Bohu?‘
#16:11 Na to jim odpovíš: V tom, že mě vaši otcové opustili, je výrok Hospodinův, a chodili za jinými bohy, sloužili jim a klaněli se jim; mne opustili a můj zákon nedodržovali.
#16:12 Vy však jste činili horší věci než vaši otcové; hle, každý z vás si žije podle svého zarputilého a zlého srdce a mne neposlouchá.
#16:13 Z této země vás uvrhnu do země, kterou jste nepoznali vy ani vaši otcové, a tam budete dnem i nocí sloužit jiným bohům. Už pro vás nemám žádné smilování.“
#16:14 „Hle, přicházejí dny, je výrok Hospodinův, kdy se už nebude říkat: ‚Jakože živ je Hospodin, který vyvedl syny Izraele z egyptské země‘,
#16:15 nýbrž: ‚Jakože živ je Hospodin, který vyvedl syny Izraele ze severní země a ze všech zemí, kam je rozehnal‘. Přivedu je zpět do země, kterou jsem dal jejich otcům.“
#16:16 „Hle, já posílám k mnohým rybářům, je výrok Hospodinův, a ti je vyloví jako ryby, pak také pošlu k mnohým lovcům a ti je budou lovit na každé hoře a na každém pahorku a ve skalních trhlinách.
#16:17 Mám na zřeteli všechny jejich cesty; neskryjí se přede mnou, před mým zrakem svou nepravost neschovají.
#16:18 Napřed jim totiž za jejich nepravost a za jejich hřích dvojnásobně odplatím. Znesvětili mou zemi mršinami svých ohyzdných model a naplnili mé dědictví svými ohavnostmi.“
#16:19 Hospodine, má sílo a má záštito, mé útočiště v den soužení, k tobě přijdou pronárody ze všech dálav země a řeknou: „Jistě naši otcové v dědictví obdrželi klam, přelud, který k užitku jim nebyl.“
#16:20 „Může si člověk udělat bohy? Žádní bohové to nejsou.
#16:21 Proto hle, já jim dám poznat, tentokrát jim dám poznat svou moc a bohatýrskou sílu, poznají, že moje jméno je Hospodin.“ 
#17:1 Judův hřích je zapsán železným rydlem, je vyryt diamantovým hrotem na tabulku jejich srdcí, na rohy vašich oltářů.
#17:2 I jejich synové si připomínají jejich oltáře a posvátné kůly u zelených stromů na vysokých pahorcích.
#17:3 „Horo má v poli, vydám v plen tvé zboží, všechny tvé poklady, tvá posvátná návrší pro hřích páchaný po celém tvém území.
#17:4 Budeš muset uvolnit své dědictví, které jsem ti dal. Dám tě tvým nepřátelům do otroctví v zemi, kterou neznáš. Zanítili jste oheň mého hněvu, navěky bude hořet.“
#17:5 Toto praví Hospodin: „Proklet buď muž, který doufá v člověka, opírá se o pouhé tělo a srdcem se odvrací od Hospodina.
#17:6 Bude jako jalovec v pustině, který neokusí přicházející dobro. Bude přebývat ve vyprahlém kraji, v poušti, v zemi solných plání, kde nelze bydlet.
#17:7 Požehnán buď muž, který doufá v Hospodina, který důvěřuje Hospodinu.
#17:8 Bude jako strom zasazený u vody; své kořeny zapustil u vodního toku, nezakusí přicházející žár. Jeho listí je zelené, v roce sucha se ničeho neobává, nepřestává nést plody.“
#17:9 „Nejúskočnější ze všeho je srdce a nevyléčitelné. Kdopak je zná?
#17:10 Já Hospodin zpytuji srdce a zkoumám ledví, já každému splatím podle jeho cesty, podle ovoce jeho skutků.“
#17:11 Koroptev sedí na vejcích, ale mladé nevyvede. Tak ten, kdo nabude bohatství nespravedlivě, v půlce života je opustí a skončí jako bloud.
#17:12 Trůne slávy, od počátku vyvýšený, místo naší svatyně,
#17:13 naděje Izraele, Hospodine, všichni, kdo tě opouštějí, propadnou hanbě. Ti, kdo se ode mne odvracejí, jsou zapsáni v zemi stínů, protože opustili zdroj živých vod, Hospodina.
#17:14 Uzdrav mě, Hospodine, a budu zdráv, spas mě a budu spasen, neboť ty jsi můj chvalozpěv.
#17:15 Hle, oni mi říkají: „Kde je Hospodinovo slovo? Ať se přece splní!“
#17:16 Nedral jsem se o to být tvým pastýřem, netoužil jsem po dnu zkázy. Ty víš, co vyšlo z mých rtů; je to přímo před tebou.
#17:17 Nebuď mi tím, kdo děsí. Tys mé útočiště ve zlý den.
#17:18 Ať propadnou hanbě ti, kdo mě pronásledují, a já ať se hanbit nemusím. Ať oni jsou zděšeni, a já ať zděšen nejsem. Uveď na ně zlý den a dvojnásob těžkou ranou je rozdrť!
#17:19 Toto mi praví Hospodin: „Jdi a postav se v bráně synů mého lidu, jíž vcházejí a vycházejí judští králové, a ve všech jeruzalémských branách.
#17:20 Řekni jim: Judští králové, celé Judsko a všichni obyvatelé Jeruzaléma, kteří vcházíte těmito branami, slyšte slovo Hospodinovo:
#17:21 Toto praví Hospodin: Bedlivě dbejte, abyste nenosili břemena v den odpočinku a nepřinášeli je do jeruzalémských bran.
#17:22 Nevynášejte břemena ze svých domů v den odpočinku a nevykonávejte žádnou práci, ale ať je vám den odpočinku svatý, jak jsem přikázal vašim otcům.
#17:23 Ale neposlechli a ucho nenaklonili, nýbrž zůstali tvrdošíjní, neslyšeli a nepřijali napomenutí.
#17:24 Budete-li mě opravdu poslouchat, je výrok Hospodinův, a v den odpočinku nebudete přinášet břemena do bran tohoto města, ale bude vám den odpočinku svatý a nebudete v něm konat žádnou práci,
#17:25 budou vcházet branami tohoto města králové a velmožové, kteří dosednou na Davidův trůn, budou jezdit na voze a na koních, oni i jejich velmožové, muži judští a obyvatelé Jeruzaléma, a toto město bude trvat navěky.
#17:26 Z judských měst a z okolí Jeruzaléma, ze země Benjamínovy, z Přímořské nížiny, z pohoří, z Negebu přijdou a přinesou do Hospodinova domu oběť zápalnou, obětní hod i oběť přídavnou a kadidlo, přinesou i oběť děkovnou.
#17:27 Nebudete-li mě poslouchat a den odpočinku vám nebude svatý, ale budete-li v den odpočinku nosit břemena a vcházet s nimi do jeruzalémských bran, zanítím v jeho branách oheň a ten pozře jeruzalémské paláce a neuhasne.“ 
#18:1 Slovo, které se stalo od Hospodina k Jeremjášovi:
#18:2 „Vstaň a sestup do hrnčířova domu a tam ti ohlásím svá slova.“
#18:3 Sestoupil jsem tedy do hrnčířova domu, právě když pracoval na hrnčířském kruhu.
#18:4 Ale nádoba, kterou vlastní rukou z hlíny zhotovoval, se nepovedla. Začal znovu a udělal z ní nádobu jinou; dělal, co se mu zdálo být správné.
#18:5 I stalo se ke mně slovo Hospodinovo:
#18:6 „Cožpak nemohu naložit s vámi jako ten hrnčíř, dome izraelský? je výrok Hospodinův. Hle, jste v mých rukou jako hlína v rukou hrnčířových, dome izraelský.
#18:7 Jednou promluvím proti pronárodu a proti království, že je vyvrátím, podvrátím a zničím.
#18:8 Avšak odvrátí-li se onen pronárod od zla, jež páchal a proti němuž jsem mluvil, budu litovat toho, že jsem zamýšlel způsobit mu něco zlého.
#18:9 Jindy promluvím o pronárodu a o království, že je vybuduji a zasadím.
#18:10 Budou-li se však dopouštět toho, co je zlé v mých očích, a nebudou mě poslouchat, budu litovat toho, že jsem slíbil prokázat jim dobré věci.
#18:11 A nyní vyřiď mužům judským a obyvatelům Jeruzaléma: Toto praví Hospodin: Hle, já připravuji proti vám zlo, to zamýšlím proti vám. Navraťte se už každý ze své zlé cesty, napravte své cesty a své skutky.“
#18:12 Oni však řekli: „Zbytečné řeči. Půjdeme za svými úmysly, každý z nás bude jednat podle svého zarputilého a zlého srdce.“
#18:13 Proto praví Hospodin toto: „Jen se dotažte mezi pronárody! Kdo kdy slyšel něco takového? Hrůzostrašné věci se dopustila panna izraelská.
#18:14 Zmizí sněhová pole ze skal Libanónu? Vyschnou snad tyto cizí vody, chladné a bystré?
#18:15 Na mne však můj lid zapomněl, pálí kadidlo falešným bohům. Přivedu je k pádu na jejich cestách, na stezkách věčnosti, půjdou po neschůdných, neupravených cestách.
#18:16 Jejich země bude vzbuzovat úděs a věčný posměch, každý kolemjdoucí bude nad ní v úděsu potřásat hlavou.
#18:17 Jako východním větrem, tak je rozptýlím před nepřítelem. Obrátím se k nim zády, a ne tváří, v den jejich běd.“
#18:18 Ale oni řekli: „Pojďme a něco si na Jeremjáše vymyslíme. Knězi přece nechybí zákon, mudrci úradek, proroku slovo. Pojďme, utlučeme ho jazykem, nedáme na žádné jeho slovo.“
#18:19 Pozornost mi věnuj, Hospodine, a slyš hlas těch, kdo se mnou vedou spor.
#18:20 Což má být odměňováno dobro zlem, že mi kopu jámu? Pamatuj, jak jsem před tebou stával a mluvil v jejich prospěch a odvracel od nich tvé rozhořčení.
#18:21 Proto vydej jejich syny hladu, vydej je na pospas meči, jejich ženy ať jsou bez dětí a ovdovělé a muže ať skolí smrt, jejich jinoši ať jsou pobiti v boji mečem.
#18:22 Z jejich domů ať se rozléhá úpění, přiveď na ně znenadání hordu, neboť vykopali jámu, aby mě polapili, nastražili osidla mým nohám.
#18:23 Ty, Hospodine, víš o každém jejich záměru, že mi chtějí způsobit smrt. Nezprošťuj je viny, jejich hřích ať u tebe smazán není, ať jsou před tebou přivedeni k pádu, tak s nimi nalož v čas svého hněvu. 
#19:1 Toto praví Hospodin: „Jdi a kup u hrnčíře hliněnou lahvici a s některými staršími lidu a staršími z kněžstva
#19:2 vyjdi do Údolí syna Hinómova, které je před Střepnou branou, a volej tam slova, která k tobě budu mluvit.
#19:3 Řekni jim: Králové judští a obyvatelé Jeruzaléma, slyšte slovo Hospodinovo. Toto praví Hospodin zástupů, Bůh Izraele: Hle, já uvedu na toto místo zlo. Tomu, kdo o něm uslyší, bude znít v uších.
#19:4 Opustili mě, cizím mi učinili toto místo a pálili na něm kadidlo jiným bohům, které neznali oni ani jejich otcové ani judští králové, a toto místo naplnili krví nevinných.
#19:5 Vybudovali posvátná návrší Baalova, aby Baalovi jako zápalné oběti pálili v ohni své syny. K tomu jsem jim přece nedal příkaz, ani jsem o tom nemluvil, ani mi to na mysl nepřišlo.
#19:6 Proto hle, přicházejí dny, je výrok Hospodinův, kdy toto místo se už nebude nazývat ‚Tófet‘ či ‚Údolí syna Hinómova‘, nýbrž ‚Údolí vraždění‘.
#19:7 Na tomto místě rozbiji na střepy záměr Judy a Jeruzaléma, způsobím to, že padnou mečem před svými nepřáteli, do rukou těch, kdo jim ukládají o život; jejich mrtvoly vydám za pokrm nebeskému ptactvu a zemskému zvířectvu.
#19:8 Způsobím, že toto město bude vzbuzovat úděs a posměch; každý kolemjdoucí nad všemi jeho ranami posměšně zasykne úžasem.
#19:9 Způsobím, že budou jíst maso svých synů a maso svých dcer, jeden bude požírat maso druhého v obležení a tísni, až je budou tísnit jejich nepřátelé a ti, kdo jim ukládají o život.
#19:10 Nato rozbiješ tu lahvici před muži, kteří půjdou s tebou,
#19:11 a řekneš jim: Toto praví Hospodin zástupů: Právě tak, jak se rozbíjí hliněná nádoba a nelze ji už opravit, rozbiji tento lid a toto město; v Tófetu se bude pohřbívat, protože jinde místo k pohřbívání nebude.
#19:12 Právě tak naložím s tímto místem, je výrok Hospodinův, a s jeho obyvateli. Učiním tomuto městu jako Tófetu.
#19:13 Domy v Jeruzalémě a domy judských králů budou nečisté jako místo Tófet; to se týká všech domů, na jejichž střechách se pálilo kadidlo veškerému nebeskému zástupu a konaly úlitby jiným bohům.“
#19:14 Když přišel Jeremjáš z Tófetu, kam jej Hospodin poslal prorokovat, postavil se na nádvoří Hospodinova domu a řekl všemu lidu:
#19:15 „Toto praví Hospodin zástupů, Bůh Izraele: Hle, já uvedu na toto město i na všechna města kolem všechno zlo, které jsem nad ním vyhlásil, neboť jsou tvrdošíjní a neposlouchají má slova.“ 
#20:1 Pašchúr, syn kněze Iméra, jenž byl vrchním dohlížitelem v Hospodinově domě, slyšel Jeremjáše prorokovat tato slova.
#20:2 Proto dal Pašchúr proroka Jeremjáše zbít a vsadit do klády, která byla v horní Benjamínské bráně, u Hospodinova domu.
#20:3 Nazítří, když Pašchúr vyvedl Jeremjáše z klády, řekl mu Jeremjáš: „Hospodin tě nepojmenoval Pašchúr, nýbrž Magór mi-sábíb (to je Kolkolem děs).“
#20:4 Toto praví Hospodin: „Děsu vydám tebe i všechny tvé přátele. Před tvými zraky padnou mečem svých nepřátel. Celé Judsko vydám do rukou babylónskému králi. On je přestěhuje do Babylóna a pobije mečem.
#20:5 Vydám též celou klenotnici tohoto města s veškerým jeho jměním a s veškerými jeho drahocennostmi, rovněž všechny poklady judských králů vydám do rukou jejich nepřátel; uloupí je, vezmou a odnesou do Babylóna.
#20:6 A ty, Pašchúre, i všichni obyvatelé tvého domu, půjdete do zajetí; přijdeš do Babylóna a tam zemřeš a tam budeš pohřben ty i všichni tvoji přátelé, jimž jsi klamně prorokoval.“
#20:7 Přemlouvals mě, Hospodine, a dal jsem se přemluvit. Zdolal jsi mě a přemohl. Po celé dny jsem jen pro smích, každý se mi vysmívá.
#20:8 Sotvaže promluvím, upím, přivolávám násilí a zhoubu, neboť Hospodinovo slovo mi přináší jen potupu a pošklebky po celé dny.
#20:9 Řekl jsem: „Nebudu je připomínat, už nebudu v jeho jménu mluvit“, avšak je v mém srdci jak hořící oheň, je uzavřeno v mých kostech, jsem vyčerpán tím, co musím snášet, dál už nemohu.
#20:10 Z mnoha stran pomluvy slyším. Kolkolem děs! „Udejte ho!“ „Udáme ho!“ Kdekdo z těch, s nimiž jsem pokojně žil, čeká, až se zhroutím. „Snad se dá nachytat, pak ho přemůžeme, zajmeme ho a pomstíme se mu.“
#20:11 Ale Hospodin je se mnou jako přesilný bohatýr, proto moji pronásledovatelé upadnou a dál nebudou moci, velice se budou stydět, že nebudou mít úspěch, jejich věčná hanba nebude zapomenuta.
#20:12 Hospodine zástupů, který zkoumáš spravedlivého, ty vidíš do ledví i do srdce, kéž spatřím tvou pomstu nad nimi. Vždyť tobě jsem předložil svůj spor.
#20:13 Zpívejte Hospodinu, chvalte Hospodina, protože vysvobodil ubožáka z rukou zlovolníků.
#20:14 Proklet buď den, v němž jsem se narodil. Den, v kterém mě má matka porodila, ať není požehnán.
#20:15 Proklet buď muž, jenž zvěstoval mému otci: „Narodilo se ti dítě, chlapec“, a udělal mu velkou radost.
#20:16 Ať je ten muž jako města, která Hospodin nelítostně vyvrátil; ať zrána slyší úpěnlivé volání a za poledne válečný povyk.
#20:17 Že mě neusmrtil hned v matčině lůně, aby mně má matka byla hrobem, aby její lůno bylo věčně těhotné!
#20:18 Proč jsem vyšel z matčina lůna? Abych viděl trápení a strasti, aby v hanbě skončily mé dny? 
#21:1 Slovo, které se stalo od Hospodina k Jeremjášovi, když k němu král Sidkijáš poslal Pašchúra, syna Malkijášova, a Sefanjáše, syna kněze Maasejáše:
#21:2 „Dotaž se prosím ohledně nás Hospodina, neboť Nebúkadnesar, král babylónský, začal proti nám válčit. Snad s námi Hospodin naloží podle své divuplné moci a on od nás odtáhne.“
#21:3 Jeremjáš jim odpověděl: „Toto vyřídíte Sidkijášovi:
#21:4 Toto praví Hospodin, Bůh Izraele: Hle, odejmu válečné zbraně, které máte ve svých rukou a jimiž bojujete vně hradeb proti babylónskému králi a Kaldejcům, kteří vás obléhají, a shromáždím je doprostřed tohoto města.
#21:5 A sám budu proti vám bojovat napřaženou rukou a pevnou paží, s hněvem, rozhořčením a velkým rozlícením.
#21:6 Obyvatele tohoto města, lidi i dobytek, raním těžkým morem a pomřou.
#21:7 Potom, je výrok Hospodinův, vydám Sidkijáše, judského krále, i jeho služebníky a lid, ty, kdo zůstanou v tomto městě po moru, meči a hladu, do rukou Nebúkadnesara, krále babylónského, do ruku jejich nepřátel, do rukou těch, kdo jim ukládají o život. Ten je pobije ostřím meče, bez lítosti, bez soucitu, bez slitování.“
#21:8 A tomuto lidu řekneš: „Toto praví Hospodin: Hle, předkládám vám cestu života a cestu smrti.
#21:9 Kdo bude sídlit v tomto městě, zemře mečem, hladem a morem. Kdo však vyjde a skloní se před Kaldejci, kteří vás obléhají, zůstane naživu, jako kořist získá svůj život.
#21:10 Neboť jsem obrátil svou tvář proti tomuto městu ke zlému, a ne k dobrému, je výrok Hospodinův. Bude vydáno do rukou babylónskému králi a ten je spálí ohněn.“
#21:11 Domu judského krále řekni: „Slyšte slovo Hospodinovo.
#21:12 Dome Davidův, toto praví Hospodin: Za jitra zjednávejte právo a vysvobozujte oloupeného z rukou utiskovatele, nebo mé rozhořčení vzplane jako oheň a bude hořet, aniž je kdo uhasí, a to pro vaše zlé skutky.
#21:13 Chystám se na tebe, která trůníš v dolině, skálo na rovině, je výrok Hospodinův, na ty, kdo říkají: ‚Kdo by k nám vnikl? Kdo by se dostal do našich obydlí?‘
#21:14 Ztrestám vás podle ovoce vašich skutků, je výrok Hospodinův. Ve vašem lese zanítím oheň a ten pozře vše kolem.“ 
#22:1 Toto praví Hospodin: „Sestup do domu krále judského a vyhlas tam toto slovo.
#22:2 Řekni: Slyš slovo Hospodinovo, králi judský, který sedíš na trůně Davidově, ty i tvoji služebníci i tvůj lid, ti, kdo přicházejí do těchto bran.
#22:3 Toto praví Hospodin: Uplatňujte právo a spravedlnost, vysvobozujte oloupeného z rukou utiskovatele, netýrejte bezdomovce, sirotka a vdovu, nedopouštějte se násilí, neprolévejte na tomto místě nevinnou krev.
#22:4 Budete-li vskutku podle tohoto slova jednat, budou branami tohoto domu vcházet králové, kteří budou sedět po Davidovi na jeho trůnu, budou jezdit na voze i na koních, král i jeho služebníci i jeho lid.
#22:5 Neuposlechnete-li těchto slov, přísahám při sobě, je výrok Hospodinův, že tento dům bude obrácen v trosky.“
#22:6 Toto praví Hospodin o domě krále judského: „Byl jsi mi Gileádem a vrcholkem Libanónu. Učiním tě však pouští, městy, jež nebudou obývána.
#22:7 Posvětím proti tobě šiřitele zkázy, každého s jeho zbrojí, ti porazí tvé nejlepší cedry a vhodí do ohně.“
#22:8 Mnohé pronárody budou přecházet kolem tohoto města a jeden druhého se bude ptát: „Proč Hospodin takhle naložil s tímto velkým městem?“
#22:9 Dostanou odpověď: „Protože opustili smlouvu Hospodina, svého Boha, klaněli se jiným bohům a sloužili jim.“
#22:10 Neoplakávejte mrtvého, neprojevujte soustrast, plačte usedavě nad tím, kdo odchází, poněvadž se už nevrátí a neuvidí svou rodnou zem.
#22:11 Toto praví Hospodin o Šalúmovi, synu Jóšijášovu, králi judském, který kraloval po svém otci Jóšijášovi, a z tohoto místa odešel: „Již se sem nevrátí.
#22:12 Na místě, kam ho přestěhovali, zemře, tuto zemi už neuvidí.“
#22:13 „Běda, kdo staví svůj dům nespravedlností a své pokojíky na střeše bezprávím, svého bližního nutí pracovat zadarmo a jeho výdělek mu nedává.
#22:14 Kdo říká: ‚Vystavím si rozměrný dům s prostornými pokojíky na střeše, prorazím si v něm okna, obložím jej cedrem a natřu rudkou.‘
#22:15 Zda proto budeš králem, když se budeš naparovat v cedroví? Zdali se tvůj otec nenajedl a nenapil? Když uplatňoval právo a spravedlnost, tehdy se mu dobře vedlo.
#22:16 Když se zastával utištěného a ubožáka, tehdy bývalo dobře. Zda to neznamená, že mě znal? je výrok Hospodinův.
#22:17 Ale tvé oči a tvé srdce nezajímá nic než vlastní zisk, prolévání nevinné krve, páchání útisku a křivd.“
#22:18 Proto praví Hospodin o Jójakímovi, synu Jóšijášovu, králi judském, toto: „Nebudou nad ním naříkat: ‚Běda, můj bratře, běda, sestro!‘ Nebudou nad ním naříkat: ‚Běda, pane, běda, veličenstvo!‘
#22:19 Bude pohřben, jako se pohřbívá osel, bude vyvlečen a odhozen ven za jeruzalémské brány.“
#22:20 „Vystup na Libanón, dcero, a křič, ať zazní tvůj hlas v Bášanu. Křič z Abarímu, protože všichni tvoji milenci jsou roztříštěni.
#22:21 Mluvil jsem k tobě, dokud jsi měla klid. Řekla jsi: ‚Nebudu poslouchat!‘ Byla jsi zvyklá už od svého mládí neposlouchat můj hlas.
#22:22 Všechny tvé pastýře spase vítr, tvoji milenci půjdou do zajetí. Tehdy se budeš stydět a hanbit za všechno zlo, jež jsi spáchala.
#22:23 Ty, která ses usídlila na Libanónu, uhnízdila se na cedrech, jak budeš vzdychat, až na tebe přijdou bolesti, až se budeš svíjet jako rodička!“
#22:24 „Jakože jsem živ, je výrok Hospodinův, i kdyby byl Konjáš, syn Jójakímův, král judský, pečetním prstenem na mé pravé ruce, strhnu tě odtamtud.
#22:25 Vydám tě do rukou těch, kdo ti ukládají o život, do rukou těch, jichž se tak lekáš, do rukou Nebúkadnesara, krále babylónského, a do rukou Kaldejců.
#22:26 Uvrhnu tebe i tvou matku, která tě porodila, do jiné země, kde jste se nenarodili, a tam zemřete.
#22:27 Ale do země, ke které se budou upínat svou duší v touze po návratu, tam se nenavrátí.“
#22:28 Což je tento muž, Konjáš, stvůrou hodnou opovržení a rozbití, je nádobou neoblíbenou? Proč byli on i jeho potomstvo odvrženi a odhozeni do země, kterou neznali?
#22:29 Země, země, země! Slyš slovo Hospodinovo.
#22:30 Toto praví Hospodin: „Zapište tohoto muže jako bezdětného, muže, který nebude mít ve svých dnech zdar. Nikomu z jeho potomků se nepodaří dosednout na Davidův trůn a být opět v Judsku vladařem!“ 
#23:1 „Běda pastýřům, kteří hubí a rozptylují ovce mé pastvy,“ je výrok Hospodinův.
#23:2 Proto Hospodin, Bůh Izraele, praví proti těm pastýřům, pastýřům svého lidu, toto: „Mé ovce rozptylujete, rozháníte je a nedohlížíte na ně. Hle, já vás za vaše zlé skutky ztrestám, je výrok Hospodinův.
#23:3 Sám shromáždím pozůstatek svých ovcí ze všech zemí, kam jsem je rozehnal, a přivedu je zpět na jejich nivy a budou plodné a rozmnoží se.
#23:4 Ustanovím nad nimi pastýře a ti je budou pást, nebudou se již bát ani děsit a žádná nebude pohřešována, je výrok Hospodinův.“
#23:5 „Hle, přicházejí dny, je výrok Hospodinův, kdy Davidovi vzbudím výhonek spravedlivý. Kralovat bude jako král a bude prozíravý a bude v zemi uplatňovat právo a spravedlnost.
#23:6 V jeho dnech dojde Judsko spásy a Izrael bude přebývat v bezpečí. A nazvou ho tímto jménem: ‚Hospodin - naše spravedlnost‘.“
#23:7 „Hle, přicházejí dny, je výrok Hospodinův, kdy se už nebude říkat: ‚Jakože živ je Hospodin, který vyvedl syny Izraele z egyptské země‘,
#23:8 nýbrž: ‚Jakože živ je Hospodin, který vyvedl a přivedl potomstvo domu Izraelova ze severní země a ze všech zemí, kam je rozehnal‘. Usadí se ve své zemi.“
#23:9 O prorocích: „Mé srdce je zlomeno v mém nitru, všechny mé kosti se chvějí, jsem jako opilý člověk, jako muž zmožený vínem, kvůli Hospodinu, kvůli jeho svatým slovům.
#23:10 Poněvadž země je plná cizoložníků, truchlí pod kletbou a pastviny na stepi vyschly; oni však běhají za zlem a zmužile si vedou v tom, co není správné.
#23:11 Vždyť jak prorok, tak kněz se rouhají, i ve svém domě nalézám jejich zlé činy, je výrok Hospodinův.
#23:12 Proto bude jejich cesta kluzká, budou vyhnáni do temnoty a v ní padnou; přivedu na ně zlo, rok jejich trestu, je výrok Hospodinův.“
#23:13 Na samařských prorocích jsem viděl tuto nepatřičnost: Prorokovali ve jménu Baalově a sváděli Izraele, můj lid.
#23:14 Také u jeruzalémských proroků jsem viděl hroznou věc: cizoložství a neustálé klamání. Posilují ruce zlovolníků, aby se nikdo neodvrátil od svých zlých činů. Jsou pro mne všichni jako Sodoma a obyvatelé města jako Gomora.
#23:15 Proto Hospodin zástupů praví proti těm prorokům toto: „Hle, nakrmím je pelyňkem a napojím je otrávenou vodou, protože od jeruzalémských proroků vyšlo rouhačství nad celou zemí.“
#23:16 Toto praví Hospodin zástupů: „Neposlouchejte slova proroků, kteří vám prorokují, obluzují vás přeludy, ohlašují vám vidění svých srdcí, a ne to, co vyšlo z úst Hospodinových.“
#23:17 Stále říkají těm, kteří mě znevažují: „Hospodin promluvil: Budete mít pokoj“, a každému, kdo chodí se zarputilým srdcem, slibují: „Nedolehne na vás nic zlého.“
#23:18 Kdo však byl v důvěrném obecenství s Hospodinem? Kdo viděl a slyšel jeho slovo? Kdo věnoval pozornost jeho slovu a naslouchal?
#23:19 Hle, vichr Hospodinův! Vzplálo rozhořčení. Vichr víří, stáčí se na hlavu svévolníků.
#23:20 Hospodinův hněv se neodvrátí, dokud nevykoná a nesplní záměry jeho srdce. V posledních dnech to určitě pochopíte.
#23:21 „Já jsem ty proroky neposlal, a přesto běží, nemluvil jsem k nim, a přesto prorokují.
#23:22 Kdyby byli v důvěrném obecenství se mnou a hlásali má slova mému lidu, odvrátili by je od jejich zlé cesty a od jejich zlých skutků.
#23:23 Jsem Bůh, jenom když jsem blízko? je výrok Hospodinův; jsem-li daleko, Bůh už nejsem?
#23:24 Může se někdo skrýt ve skrýších a já ho neuvidím? je výrok Hospodinův. Nenaplňuji snad nebe i zemi? je výrok Hospodinův.“
#23:25 „Slyšel jsem, co říkají proroci, kteří v mém jménu prorokují klam. Říkají: ‚Měl jsem sen, měl jsem sen.‘
#23:26 Jak dlouho ještě? Je něco v srdci proroků, kteří prorokují klam, a proroků, kteří prorokují lest svých srdcí?
#23:27 Myslí si, že mé jméno vymýtí z paměti mého lidu svými sny, které si navzájem vypravují? Tak jako zapomněli na mé jméno jejich otcové kvůli Baalovi?
#23:28 Prorok, který má sen, ať vypravuje sen, ale kdo má mé slovo, ať mluví mé slovo pravdivě. „Co je slámě do obilí?“, je výrok Hospodinův.
#23:29 „Není mé slovo jako oheň, je výrok Hospodinův, jako kladivo tříštící skálu?“
#23:30 „Proto hle, já jsem proti těm prorokům, je výrok Hospodinův, kteří kradou jeden druhému má slova.
#23:31 Hle, já jsem proti těm prorokům, je výrok Hospodinův, kteří používají svého jazyka a tvrdí: ‚Výrok Hospodinův‘.
#23:32 Hle, já jsem proti těm, kdo prorokují klamné sny, je výrok Hospodinův. Vypravují je a svádějí můj lid svými klamy a svou chvástavostí. Já jsem je ani neposlal ani nepověřil, tomuto lidu nejsou k užitku, je výrok Hospodinův.“
#23:33 „Zeptá-li se tě tento lid, prorok nebo kněz: ‚Jaký je výnos Hospodinův?‘, odvětíš jim: ‚Jaký výnos? Odvrhnu vás, je výrok Hospodinův.‘
#23:34 Proroka i kněze a lid, kteří užívají rčení: ‚Výnos Hospodinův‘, potrestám, každého i jeho dům.
#23:35 Takto se budete ptát jeden druhého, každý svého bratra: ‚Co odpověděl Hospodin? Co Hospodin promluvil?‘
#23:36 Nebudete si už připomínat výnos Hospodinův, neboť výnosem je každému jeho vlastní slovo. Vy překrucujete slova Boha živého, Hospodina zástupů, našeho Boha.
#23:37 Takto se budeš ptát proroka: ‚Co ti odpověděl Hospodin?‘ a ‚Co Hospodin promluvil?‘
#23:38 Budete-li říkat: ‚Výnos Hospodinův‘, tak tedy toto praví Hospodin: Protože užíváte rčení: ‚Výnos Hospodinův‘, vzkazuji vám: Neříkejte: ‚Výnos Hospodinův‘!
#23:39 Hle, propůjčím vás nepříteli a odvrhnu od své tváře vás i město, které jsem dal vám i vašim otcům.
#23:40 Uvalím na vás věčnou hanbu a věčnou potupu, která nebude zapomenuta.“ 
#24:1 Hospodin mi ukázal, hle, dva koše fíků postavené před Hospodinovým chrámem poté, co Nebúkadnesar, král babylónský, přestěhoval Jekonjáše, syna Jójakímova, krále judského i judské velmože, řemeslníky a kovotepce z Jeruzaléma a přivedl je do Babylóna.
#24:2 V jednom koši byly fíky velmi dobré, jako jsou fíky rané, v druhém koši byli fíky velmi špatné, tak špatné, že se nedaly jíst.
#24:3 Hospodin se mne zeptal: „Co vidíš, Jeremjáši?“ Odvětil jsem: „Fíky. Ty dobré fíky jsou velmi dobré, ty špatné jsou velmi špatné, tak špatné, že se nedají jíst.“
#24:4 Tu se ke mně stalo slovo Hospodinovo:
#24:5 „Toto praví Hospodin, Bůh Izraele: Jako na tyto dobré fíky se dívám na judské přesídlence, které jsem pro jejich dobro poslal z tohoto místa do země Kaldejců.
#24:6 Obracím k nim svůj zrak v dobrém, přivedu je zpět do této země. Znovu je vybuduji a už je nezbořím, zasadím je a nevykořením.
#24:7 Dám jim srdce, aby mě poznali, že já jsem Hospodin. Budou mým lidem a já jim budu Bohem, neboť se ke mně navrátí celým srdcem.
#24:8 Ale jako se špatnými fíky, které jsou tak špatné, že se nedají jíst, praví Hospodin, právě tak naložím se Sidkijášem, králem judským, a jeho velmoži, i pozůstatkem jeruzalémského lidu, s těmi, kdo zůstali v této zemi, i s těmi, kdo bydlí v zemi egyptské.
#24:9 Učiním je obrazem hrůzy a zla pro všechna království země, potupou a pořekadlem, předmětem výsměchu a zlořečením na všech místech, kam je vyženu.
#24:10 Pošlu na ně meč, hlad a mor, dokud nebudou do posledního vyhubeni ze země, kterou jsem dal jim a jejich otcům.“ 
#25:1 Slovo, které se stalo k Jeremjášovi o veškerém lidu judském čtvrtého roku vlády Jójakíma, syna Jóšijášova, krále judského, což byl první rok Nebúkadnesara, krále babylónského.
#25:2 To slovo prorok Jeremjáš mluvil o veškerém lidu judském a o všech obyvatelích Jeruzaléma:
#25:3 „Slovo Hospodinovo se ke mně stávalo od třináctého roku vlády Jóšijáše, syna Amónova, krále judského, až dodnes, tedy dvacet tři léta. Já jsem k vám mluvil, nepřetržitě jsem mluvil, ale vy jste neposlouchali.
#25:4 Hospodin k vám posílal všechny své služebníky proroky, nepřetržitě je posílal, ale neposlouchali jste, ucho jste k slyšení nenaklonili.
#25:5 Říkávali: ‚Ať se každý odvrátí od své zlé cesty a od svých zlých skutků a budete přebývat v zemi, kterou Hospodin dal vám a vašim otcům od věků na věky.
#25:6 Nechoďte za jinými bohy, neslužte jim a neklaňte se jim. Neurážejte mě dílem svých rukou a já s vámi nenaložím zle.‘
#25:7 Ale neposlouchali jste mě, je výrok Hospodinův, dílem svých rukou jste mě uráželi, sobě k zlému.“
#25:8 Proto Hospodin zástupů praví toto: „Že jste neposlouchali má slova,
#25:9 hle, já pošlu pro všechny čeledi severu, je výrok Hospodinův, i pro Nebúkadnesara, krále babylónského, svého služebníka, a přivedu je na tuto zemi i na všechny její obyvatele i na všechny tyto okolní pronárody a vyhubím je jako klaté. Způsobím, že budou vzbuzovat úděs a posměch a budou troskami navěky.
#25:10 Způsobím jim, že se ztratí hlas veselí a hlas radosti, hlas ženicha a hlas nevěsty, hlas mlýnku i světlo lampy.
#25:11 Tak se stane celá tato země troskami a bude budit úděs. Tyto pronárody budou sloužit králi babylónskému po sedmdesát let.
#25:12 Až se naplní sedmdesát let, budu stíhat na králi babylónském a na onom pronárodu, totiž na zemi Kaldejců, jejich vinu, je výrok Hospodinův, a způsobím, že bude navěky zpustošena.
#25:13 Uvedu na tuto zemi všechna svá slova, která jsem proti ní mluvil, všechno, co je zapsáno v této knize, co prorokoval Jeremjáš proti všem pronárodům.
#25:14 I oni budou sloužit četným pronárodům a velkým králům. Odplatím jim podle jejich skutků a podle díla jejich rukou.“
#25:15 Toto mi řekl Hospodin, Bůh Izraele: „Vezmi z mé ruky tuto číši vína rozhořčení a napoj jím všechny pronárody, k nimž tě posílám,
#25:16 ať pijí a vrávorají a třeští před mečem, který mezi ně pošlu!“
#25:17 Vzal jsem číši z Hospodinovy ruky a napojil jsem všechny pronárody, k nimž mě Hospodin poslal:
#25:18 Jeruzalém i judská města, jeho krále, jeho velmože, abych je obrátil v trosky, aby vzbuzovali úděs, posměch a zlořečení, jak je tomu dodnes.
#25:19 Napojil jsem i faraóna, krále egyptského, jeho služebníky, jeho velmože a všechen jeho lid
#25:20 i všechen přimíšený lid, všechny krále země Úsu, všechny krále země pelištejské, Aškalón, Gázu, Ekrón i pozůstatek lidu ašdódského,
#25:21 Edóma, Moába, Amónovce
#25:22 i všechny krále Týru, všechny krále Sidónu i krále zámořských ostrovů,
#25:23 také Dedána, Tému, Búza a všechny, kdo si vyholují skráně,
#25:24 všechny krále Arábie i všechny krále přimíšeného lidu, kteří pobývají v poušti,
#25:25 všechny krále Zimrí, všechny krále Élamu i všechny krále Médie,
#25:26 všechny krále severu, blízké i vzdálené, jednoho po druhé, i všechna království země, co jich je na zemi. Po nich bude pít král Šéšak.
#25:27 Řekneš jim: „Toto praví Hospodin zástupů, Bůh Izraele: Pijte, opijte se, zvracejte a padněte. Už nepovstanete před mečem, který mezi vás pošlu.“
#25:28 Budou-li se zdráhat vzít číši z tvé ruky a pít z ní, řekneš jim: „Toto praví Hospodin zástupů: Musíte pít.
#25:29 Hle, v městě, které se nazývá mým jménem, již začínám ono zlo, a vy byste zůstali bez trestu? Nezůstanete bez trestu, neboť povolám meč na všechny obyvatele země, je výrok Hospodina zástupů.“
#25:30 Ty jim budeš prorokovat všechna tato slova. Řekneš jim: „Hospodin vydá řev z výšiny, ze svého svatého obydlí vydá svůj hlas. Hlasitě řve nad svou nivou - zní to jak výskání těch, kdo lisují víno-, řve na všechny obyvatele země.
#25:31 Hukot vřavy dolehne až do končin země, neboť Hospodin vede spor s pronárody, soudí se sám s veškerým tvorstvem, svévolníky vydá meči, je výrok Hospodinův.“
#25:32 Toto praví Hospodin zástupů: „Hle, vyjde zlo od pronároda k pronárodu, přižene se veliký vichr z nejodlehlejších koutů země.“
#25:33 Ti, které Hospodin v onen den skolí, zůstanou ležet od jednoho konce země ke druhému. Nebude se nad nimi naříkat, nebudou sebráni, nebudou pohřbeni, budou jak hnůj na povrchu země.
#25:34 Kvilte, pastýři, a úpěte, válejte se v prachu, vznešení vůdcové stáda! Vaše dny se naplnily, půjdete na porážku, rozpráším vás, padnete, jako když spadne vzácná nádoba.
#25:35 Pro pastýře nebude žádného útočiště, nebude úniku pro vznešené vůdce stáda.
#25:36 Slyš! Úpění pastýřů, kvílení vznešených vůdců stáda, neboť Hospodin pustoší jejich pastvu.
#25:37 Pokojné nivy zajdou Hospodinovým planoucím hněvem.
#25:38 Opouští svůj úkryt jako lvíče. Jejich země bude zpustošena, až se jeho hudící hněv rozpálí, až jeho hněv vzplane. 
#26:1 Na začátku kralování Jójakíma, syna Jóšijášova, krále judského, stalo se od Hospodina toto slovo:
#26:2 „Toto praví Hospodin: Postav se na nádvoří Hospodinova domu a mluv proti všem judským městům, která se přicházejí klanět do Hospodinova domu, všechna slova, která jsem ti přikázal, abys jim mluvil; neubereš ani slovíčko.
#26:3 Snad uslyší a odvrátí se každý od své zlé cesty a já budu litovat toho, že jsem s nimi chtěl zle naložit za jich zlé skutky.
#26:4 Řekneš jim: Toto praví Hospodin: Jestliže mě neuposlechnete a nebudete se řídit mým zákonem, který jsem vám vydal,
#26:5 a nebudete-li poslouchat slova mých služebníků proroků, které vám nepřetržitě posílám - a vy jste neposlechli! -
#26:6 naložím s tímto domem jako se Šílem a toto město vydám v zlořečení u všech pronárodů země.“
#26:7 Kněží, proroci i všechen lid slyšeli Jeremjáše mluvit v Hospodinově domě všechna tato slova.
#26:8 Když Jeremjáš domluvil, co mu Hospodin přikázal mluvit ke všemu lidu, kněží, proroci i všechen lid ho chytili a křičeli: „Zemřeš!
#26:9 Proč jsi v Hospodinově jménu prorokoval, že tento dům bude jako Šílo a že toto město bude v troskách, bez obyvatele?“ Všechen lid se proti Jeremjášovi v Hospodinově domě srotil.
#26:10 Když judští velmožové slyšeli tato slova, vystoupili z domu královského do Hospodinova domu a posadili se při vchodu do Hospodinovy Nové brány.
#26:11 Kněží a proroci řekli velmožům a všemu lidu: „Tento muž zasluhuje smrt za to, co o tomto městě prorokoval, jak jste na vlastní uši slyšeli.“
#26:12 Jeremjáš všem velmožům a všemu lidu odpověděl: „Hospodin mě poslal, abych prorokoval o tomto domě a o tomto městě všechna slova, která jste slyšeli.
#26:13 Nyní napravte své cesty a své skutky, poslouchejte Hospodina, svého Boha, a Hospodin bude litovat toho, že proti vám mluvil zlé věci.
#26:14 Pokud jde o mne, jsem ve vašich rukou, naložte se mnou, jak pokládáte za dobré a správné.
#26:15 Ale vězte, když mě usmrtíte, uvedete na sebe, na toto město i na jeho obyvatele nevinnou krev. Vždyť mě k vám Hospodin opravdu posílá, abych vám přednesl všechna tato slova.“
#26:16 Velmožové i všechen lid nato kněžím a prorokům odvětili: „Tento muž nezasluhuje smrt. Vždyť k nám mluvil ve jménu Hospodina, našeho Boha.“
#26:17 Někteří muži ze starších země povstali a řekli celému shromáždění lidu:
#26:18 „Za dnů Chizkijáše, krále judského, prorokoval Micheáš Mórešetský a řekl všemu judskému lidu: ‚Toto praví Hospodin zástupů: Sijón bude zorán jako pole, z Jeruzaléma budou sutiny, z hory Hospodinova domu návrší zarostlá křovím.‘
#26:19 Dal ho snad Chizkijáš, král judský, a celý Juda usmrtit? Zdalipak se nebál Hospodina a neprosil Hospodina o shovívavost? Hospodin pak litoval, že proti nim mluvil ty zlé věci. Dopustili bychom se velikého zla sami proti sobě.“
#26:20 V Hospodinově jménu prorokoval také muž Úrijáš, syn Šemajášův z Kirat-jearímu. Prorokoval proti tomuto městu a proti této zemi stejnými slovy jako Jeremjáš.
#26:21 Král Jójakím a všichni jeho bohatýři i všichni velmožové slyšeli jeho slova. Král ho chtěl usmrtit. Když o tom Úrijáš uslyšel, bál se a uprchl a přišel až do Egypta.
#26:22 Ale král Jójakím poslal do Egypta muže s Elnátanem, synem Akbórovým, poslal je do Egypta.
#26:23 Odvedli Úrijáše z Egypta a přivedli ho ke králi Jójakímovi. Ten ho zabil mečem a jeho mrtvolu pohodil u hrobů prostého lidu.
#26:24 Avšak s Jeremjášem byla ruka Achíkama, syna Šáfanova. Ten jej nevydal do rukou lidu, který jej chtěl usmrtit. 
#27:1 Na začátku kralování Jójakíma, syna Jóšijášova, krále judského, stalo se k Jeremjášovi od Hospodina toto slovo.
#27:2 Toto mi pravil Hospodin: „Udělej si pouta a jařmo a dej si je na šíji.
#27:3 Pak je pošli králi Edómu, králi Moábu, králi Amónovců, králi Týru a králi Sidónu po poslech přicházejících do Jeruzaléma k Sidkijášovi, králi judskému.
#27:4 Dáš jim pro jejich panovníky tento příkaz: Toto praví Hospodin zástupů, Bůh Izraele: Toto řeknete svým panovníkům:
#27:5 Já jsem učinil zemi, člověka i zvířata, která jsou na zemi, svou velikou silou a svou vztaženou paží a dávám ji tomu, kdo je toho v mých očích hoden.
#27:6 Nyní jsem všechny tyto země dal do rukou Nebúkadnesarovi, králi babylónskému, svému služebníku; dal jsem mu i polní zvěř, aby mu sloužila.
#27:7 Budou mu sloužit všechny pronárody, i jeho synu a vnuku. Pak nadejde čas jeho zemi i jemu a podrobí si ho v službu mnohé pronárody a velicí králové.
#27:8 Ale pronárod či království, které by nesloužilo jemu, Nebúkadnesarovi, králi babylónskému, a které by svou šíji nesklonilo pod jho babylónského krále, ten pronárod postihnu mečem a hladem a morem, je výrok Hospodinův, až je úplně vydám do jeho rukou.
#27:9 Neposlouchejte své proroky ani své věštce ani své sny ani své mrakopravce ani své čaroděje, kteří vám říkají: ‚Králi babylónskému sloužit nebudete!‘
#27:10 Prorokují vám klam, aby vás odvedli daleko od vaší země. A já vás vyženu a vy zahynete.
#27:11 Ale pronárod, který vloží svou šíji pod jho babylónského krále a bude mu sloužit, ponechám v jeho zemi, je výrok Hospodinův, a bude ji obdělávat a v ní sídlit.“
#27:12 K Sidkijášovi, králi judskému, jsem mluvil přesně tato slova: „Vložte svou šíji pod jho babylónského krále a služte jemu i jeho lidu a budete živi.
#27:13 Proč bys měl umírat ty i tvůj lid mečem, hladem a morem, jak mluvil Hospodin o pronárodu, který by nesloužil babylónskému králi?
#27:14 Neposlouchejte slova proroků, kteří vám říkají: ‚Králi babylónskému sloužit nebudete.‘ Prorokují vám klam.
#27:15 Já jsem je neposlal, je výrok Hospodinův, oni prorokují v mém jménu klam, abych vás vyhnal. Zahynete vy i proroci, kteří vám prorokují.“
#27:16 Ke kněžím a ke všemu tomuto lidu jsem mluvil: „Toto praví Hospodin: Neposlouchejte slova svých proroků, kteří vám prorokují: ‚Hle, předměty z Hospodinova domu budou velice brzy z Babylóna vráceny.‘ Ti vám prorokují klam.
#27:17 Neposlouchejte je! Služte babylónskému králi a budete živi. Proč by toto město mělo být obráceno v trosky?
#27:18 Jestliže jsou proroci a mají Hospodinovo slovo, ať naléhavě žádají Hospodina zástupů, aby se do Babylóna nedostaly i předměty, které v domě Hospodinově a v domě krále judského a v Jeruzalémě zbyly.“
#27:19 Toto praví Hospodin zástupů o sloupech, o bronzovém moři a o podstavcích a o zbytku předmětů, zbylých v tomto městě,
#27:20 které nevzal Nebúkadnesar, král babylónský, když přestěhoval z Jeruzaléma do Babylóna Jekonjáše, syna Jójakímova, krále judského, a všechny šlechtice Judska a Jeruzaléma.
#27:21 Toto praví Hospodin zástupů, Bůh Izraele, o předmětech zbylých v domě Hospodinově a v domě krále judského a v Jeruzalémě:
#27:22 „Budou doneseny do Babylóna a budou tam až do dne, kdy je budu pohřešovat, je výrok Hospodinův; pak je dám přinést a vrátím je opět na toto místo.“ 
#28:1 Týž rok, na začátku kralování Sidkijáše, krále judského, pátého měsíce čtvrtého roku, mi řekl Chananjáš, syn Azúrův, prorok z Gibeónu, v Hospodinově domě za přítomnosti kněží a všeho lidu:
#28:2 „Toto praví Hospodin zástupů, Bůh Izraele: Jho babylónského krále jsem zlomil.
#28:3 Do dvou let vrátím na toto místo všechny předměty Hospodinova domu, které vzal z tohoto místa Nebúkadnesar, král babylónský a odnesl je do Babylóna.
#28:4 A Jekonjáše, syna Jójakímova, krále judského, i všechny judské přesídlence, kteří přišli do Babylóna, přivedu zpět na toto místo, je výrok Hospodinův, neboť jho babylónského krále zlomím.“
#28:5 Prorok Jeremjáš odpověděl proroku Chananjášovi za přítomnosti kněží a všeho lidu, stojícího v Hospodinově domě.
#28:6 Prorok Jeremjáš řekl: „Amen. Kéž tak učiní Hospodin! Kéž Hospodin splní tvá slova, která jsi prorokoval, že vrátí předměty Hospodinova domu a přivede zpět z Babylóna na toto místo všechny přesídlence.
#28:7 Ale poslechni toto slovo, které ohlašuji tobě i všemu lidu.
#28:8 Proroci, kteří byli odedávna přede mnou a před tebou, prorokovali o četných zemí a proti velkým královstvím o válce a zlu a moru.
#28:9 Prorok, který prorokoval o pokoji, byl uznán za proroka, kterého opravdu poslal Hospodin, až když došlo na slovo toho proroka.“
#28:10 I sňal prorok Chananjáš z šíje proroka Jeremjáše jařmo a zlomil je.
#28:11 Chananjáš řekl za přítomnosti všeho lidu: „Toto praví Hospodin: Tak zlomím do dvou let jho Nebúkadnesara, krále babylónského, které je na šíji všech národů.“ Nato prorok Jeremjáš odešel svou cestou.
#28:12 Potom, když prorok Chananjáš zlomil jařmo na šíji proroka Jeremjáše, stalo se k Jeremjášovi slovo Hospodinovo:
#28:13 „Jdi a řekni Chananjášovi: Toto praví Hospodin: Zlomil jsi jařma dřevěná a udělal jsi místo nich jařma železná.
#28:14 Toto praví Hospodin zástupů, Bůh Izraele: Na šíji všech těchto pronárodů vložím jho železné, aby sloužily Nebúkadnesarovi, králi babylónskému, a budou mu sloužit. Dal jsem mu i polní zvěř.“
#28:15 Dále řekl prorok Jeremjáš proroku Chananjášovi: „Slyš, Chananjáši, Hospodin tě neposlal. Ty jsi svedl tento lid ke klamnému doufání.
#28:16 Proto Hospodin praví toto: Hle, zapudím tě z povrchu země. Do roka zemřeš, protože jsi scestně mluvil o Hospodinu.“
#28:17 A prorok Chananjáš zemřel týž rok, sedmého měsíce. 
#29:1 Toto jsou slova dopisu, který poslal prorok Jeremjáš z Jeruzaléma přesídlencům, zbytku starších, kněžím, prorokům a všemu lidu, které přestěhoval Nebúkadnesar z Jeruzaléma do Babylóna,
#29:2 když musel odejít z Jeruzaléma král Jekonjáš, královna, dvořané, velmožové judští a jeruzalémští i tesaři a kováři.
#29:3 Poslal jej po Eleasovi, synu Šáfanovu, a po Gemerjášovi, synu Chilkijášovu, které vyslal Sidkijáš, král judský, k Nebúkadnesarovi, králi babylónskému, do Babylóna.
#29:4 „Toto praví Hospodin zástupů, Bůh Izraele, všem přesídlencům, které jsem přestěhoval z Jeruzaléma do Babylóna:
#29:5 Stavějte domy a bydlete v nich, vysazujte zahrady a jezte jejich plody.
#29:6 Berte si ženy, ploďte syny a dcery. Berte ženy pro své syny, provdávejte své dcery za muže, ať rodí syny a dcery, rozmnožujte se tam, ať vás neubývá.
#29:7 Usilujte o pokoj toho města, do něhož jsem vás přestěhoval, modlete se za ně k Hospodinu, neboť v jeho pokoji i vy budete mít pokoj.
#29:8 Toto praví Hospodin zástupů, Bůh Izraele: Nedejte se podvádět svými proroky, kteří jsou uprostřed vás, ani svými věštci. Neposlouchejte své sny, které míváte.
#29:9 Prorokují vám klam v mém jménu, já jsem je neposlal, je výrok Hospodinův.
#29:10 Toto praví Hospodin: Až se vyplní sedmdesát let Babylóna, navštívím vás a splním na vás své dobré slovo, že vás přivedu zpět na toto místo.
#29:11 Neboť to, co s vámi zamýšlím, znám jen já sám, je výrok Hospodinův, jsou to myšlenky o pokoji, nikoli o zlu: chci vám dát naději do budoucnosti.
#29:12 Budete mě volat a chodit ke mně, modlit se ke mně a já vás vyslyším.
#29:13 Budete mě hledat a naleznete mě, když se mne budete dotazovat celým svým srdcem.
#29:14 Dám se vám nalézt, je výrok Hospodinův, a změním váš úděl, shromáždím vás ze všech pronárodů a ze všech míst, kam jsem vás vyhnal, je výrok Hospodinův, a přivedu vás zpět na místo, odkud jsem vás přestěhoval.“
#29:15 „Říkali jste: ‚Hospodin nám v Babylóně vzbudil proroky.‘
#29:16 Toto praví Hospodin o králi, který sedí na Davidově trůnu, a o všem lidu, který sídlí v tomto městě, o vašich bratřích, kteří s vámi při přesídlení neodešli.
#29:17 Toto praví Hospodin zástupů: Hle, pošlu na ně meč, hlad a mor, naložím s nimi jako s trpkými fíky, které se nedají jíst, jak jsou odporné.
#29:18 Budu je pronásledovat mečem, hladem a morem. Učiním je obrazem hrůzy pro všechna království země, předmětem kletby, úděsu, posměchu a potupy ve všech pronárodech, kam jsem je vyhnal,
#29:19 protože neposlouchali má slova, je výrok Hospodinův, když jsem k nim posílal své služebníky proroky. Posílal jsem je nepřetržitě, ale ani vy jste neposlouchali, je výrok Hospodinův.
#29:20 Vy všichni přesídlenci, které jsem poslal z Jeruzaléma do Babylóna, slyšte Hospodinovo slovo.
#29:21 Toto praví Hospodin zástupů, Bůh Izraele, o Achabovi, synu Kólajášovu, a o Sidkijášovi, synu Maasejášovu, kteří vám prorokují v mém jménu klam: Hle, vydám je do rukou Nebúkadnesara, krále babylónského; on je dá před vašima očima zabít.
#29:22 Proto se bude mezi všemi judskými přesídlenci v Babylóně zlořečit slovy: ‚Ať s tebou Hospodin naloží jako se Sidkijášem a Achabem, které král babylónský pražil na ohni‘,
#29:23 protože se dopouštěli v Izraeli hanebnosti, cizoložili s ženami svých bližních, a co mluvili v mém jménu, byl klam, což jsem jim nepřikázal. Já to vím a jsem toho svědkem, je výrok Hospodinův.“
#29:24 Šemajášovi Nechelamskému řekni:
#29:25 „Toto praví Hospodin zástupů, Bůh Izraele: Ty jsi posílal svým jménem všemu lidu, který je v Jeruzalémě, a knězi Sefanjášovi, synu Maasejášovu, a všem kněžím dopisy:
#29:26 ‚Hospodin tě ustanovil knězem místo kněze Jójady, abys byl dohližitelem v domě Hospodinově nad každým potřeštěným, nad tím, kdo vystupuje jako prorok, abys jej mohl dát do klády či do obojku.
#29:27 Proč jsi tedy neokřikl Jeremjáše Anatótského, který u vás vystupuje jako prorok?
#29:28 Vždyť i k nám do Babylóna poslal vzkaz: Bude to dlouho trvat; stavějte domy, bydlete v nich, vysazujte zahrady a jezte jejich plody‘.“
#29:29 Kněz Sefanjáš přečetl tento dopis proroku Jeremjášovi.
#29:30 I stalo se slovo Hospodinovo k Jeremjášovi:
#29:31 „Pošli vzkaz všem přesídlencům: Toto praví Hospodin o Šemajášovi Nechaelamském: Že vám Šemajáš prorokoval, ačkoli jsem ho neposlal, a svedl vás ke klamnému doufání,
#29:32 proto praví Hospodin toto: Hle, ztrestám Šemajáše Nechelamského i jeho potomstvo. Nebude mít nikoho, kdo by bydlel uprostřed tohoto lidu, a neuvidí dobré věci, které svému lidu učiním, je výrok Hospodinův, protože scestně mluvil proti Hospodinu.“ 
#30:1 Slovo, které se stalo k Jeremjáši od Hospodina:
#30:2 „Toto praví Hospodin, Bůh Izraele: Napiš si do knihy všechna slova, která jsem k tobě mluvil.
#30:3 Neboť hle, přicházejí dny, je výrok Hospodinův, že změním úděl svého lidu, Izraele i Judy, praví Hospodin, a přivedu je zpět do země, kterou jsem dal jejich otcům. A obsadí ji.“
#30:4 Toto jsou slova, která promluvil Hospodin o Izraeli a Judovi.
#30:5 Toto praví Hospodin: „Slyšeli jsme děsuplný hlas. Strach a nikde pokoj.
#30:6 Jen se ptejte, podívejte se, zda rodí muž. Proč jsem viděl každého muže s rukama na bedrách jako rodičku, a všechny tváře byly sinalé.
#30:7 Běda, bude to veliký den, bez obdoby, čas soužení pro Jákoba, ale bude z něho zachráněn.“
#30:8 „V onen den, je výrok Hospodina zástupů, zlomím jeho jho na tvé šíji a zpřetrhám tvá pouta. Nebudou už sloužit cizáků,
#30:9 budou sloužit Hospodinu, svému Bohu, a Davidovi, svému králi, kterého jim vzbudím.“
#30:10 „Ty, Jákobe, můj služebníku, neboj se, je výrok Hospodinův, neděs se, Izraeli. Hle, já tě zachráním, i když jsi daleko, i tvé potomky, ze země zajetí. Jákob se vrátí a bude žít v míru, bezstarostně, a nikdo jej nevyděsí.
#30:11 Já budu s tebou, je výrok Hospodinův, a zachráním tě. Učiním konec všem pronárodům, mezi než jsem tě rozprášil, ale s tebou neskončím docela. Potrestám tě ovšem podle práva, bez trestu tě neponechám.“
#30:12 Toto praví Hospodin: „Nevyléčitelné je tvé poranění, bolestná je tvá rána.
#30:13 Nikdo nevede tvou při, tvá otevřená rána se nehojí, nezaceluje.
#30:14 Všichni tvoji milenci na tebe zapomněli, nevyhledávají tě. Ranou nepřítele jsem tě ranil, nelítostným pokáráním pro množství tvých nepravostí, pro tvé nesčíslné hříchy.
#30:15 Proč křičíš pro svou těžkou ránu, že je tvá bolest nevyléčitelná? Způsobil jsem ti to pro množství tvých nepravostí, pro tvé nesčíslné hříchy.
#30:16 Avšak všichni, kdo tě požírají, budou pozřeni, všichni tvoji protivníci do jednoho půjdou do zajetí. Kdo tě plení, budou popleněni, všechny, kdo tě loupí, vydám k oloupení.
#30:17 Zahojím a vyléčím tvé rány, je výrok Hospodinův. Nazývali tě zapuzenou: říkali o tobě: ‚To je ten Sijón, nikdo jej nevyhledává.‘“
#30:18 Toto praví Hospodin: „Hle, změním úděl Jákobových stanů, slituji se nad jeho příbytky. Město bude vystavěno na svých sutinách a palác bude obýván podle svého řádu.
#30:19 Zazní jejich děkovná píseň a hlasitý smích. Rozmnožím je, nebude jich ubývat, oslavím je, nebude jich málo.
#30:20 Jeho synů bude jako kdysi, jeho pospolitost přede mnou obstojí. Všechny jeho utlačovatele ztrestám.
#30:21 Jeho vznešený vůdce bude pocházet z něho, jeho vládce vzejde z jeho středu. Dovolím mu přiblížit se a on bude ke mně přistupovat. Kdo by se sám mohl odvážit ke mně přistupovat? je výrok Hospodinův.
#30:22 Budete mým lidem a já vám budu Bohem.“
#30:23 Hle, vichr Hospodinův! Vzplálo rozhořčení. Vichr je tu hostem, stáčí se na hlavu svévolníků.
#30:24 Hospodinův planoucí hněv se neodvrátí, dokud nevykoná a nesplní záměry jeho srdce. V posledních dnech to pochopíte. 
#31:1 V onen čas, je výrok Hospodinův, budu Bohem všem čeledím Izraele a oni budou mým lidem.
#31:2 Toto praví Hospodin: „Milost na poušti nalezl lid, který vyvázl před mečem. Jdu, abych přinesl mír Izraeli.“
#31:3 Hospodin se mi ukázal zdaleka: „Miloval jsem tě odvěkou láskou, proto jsem ti tak trpělivě prokazoval milosrdenství.
#31:4 Znovu tě zbuduji a budeš zbudována, panno izraelská. Znovu se ozdobíš bubínky a vyjdeš k tanci s těmi, kdo se smějí.
#31:5 Znovu budeš vysazovat vinice na samařských horách. Budou vysazovat, a ti, kteří budou sázet, budou i sklízet.
#31:6 Vždyť už tu je den, kdy zavolají hlídači na Efrajimském pohoří: ‚Vzhůru, vydejme se na Sijón k Hospodinu, svému Bohu.‘“
#31:7 Toto praví Hospodin: „Radostně plesejte vstříc Jákobovi, jásejte nad tím, který je hlavou pronárodů, vzdávejte chválu, rozhlašujte: ‚Hospodin spasil tvůj lid, pozůstatek Izraele.‘
#31:8 Hle, přivedu je ze země severní, shromáždím je z nejodlehlejších koutů země. Bude mezi nimi slepý a kulhavý, těhotná, i ta, jež právě porodila. Vrátí se sem veliké shromáždění.
#31:9 Přijdou s pláčem a s prosbami o smilování, já je povedu. Dovedu je k potokům, jež mají vodu, cestou přímou, na níž neklopýtnou. Budu Izraeli otcem, Efrajim bude můj prvorozený.“
#31:10 Slyšte, pronárody, Hospodinovo slovo, na vzdálených ostrovech oznamte toto: „Ten, který rozmetal Izraele, shromáždí jej, bude jej střežit jako pastýř své stádo.“
#31:11 Hospodin zaplatí za Jákoba, vykoupí ho z rukou silnějšího.
#31:12 Přijdou a budou plesat na výšině sijónské, budou proudit k Hospodinově dobrotě za obilím, moštem a čerstvým olejem, za mladým bravem a skotem. Jejich duše bude jako zavlažovaná zahrada. Už nikdy nebudou tesknit.
#31:13 Tehdy se bude v tanci radovat panna i jinoši a starci. „Jejich truchlení změním ve veselí, místo strasti jim dám útěchu a radost.
#31:14 Duši kněží zavlažím tukem a můj lid se bude sytit mými dobrými dary, je výrok Hospodinův.“
#31:15 Toto praví Hospodin: „V Rámě je slyšet hlasité bědování, přehořký pláč. Ráchel oplakává své syny, odmítá útěchu, protože její synové už nejsou.“
#31:16 Toto praví Hospodin: „Přestaň hlasitě plakat a ronit slzy, vždyť tu je mzda za to, co jsi vykonala, je výrok Hospodinův, však oni se vrátí z nepřátelské země.
#31:17 Je naděje pro tvé potomstvo, je výrok Hospodinův. Synové se vrátí na své území.
#31:18 Zřetelně jsem slyšel Efrajima, jak si stýská: ‚Potrestal jsi mě a byl jsem ztrestán, býval jsem jak nezkrocený býček. Obrať mě, chci se vrátit, vždyť ty, Hospodine, jsi můj Bůh.
#31:19 Po svém návratu chci činit pokání, po svém poučení budu se bít v prsa, stydět se a hanbit, že nesu potupu svého mládí.‘ -
#31:20 Což je mi Efrajim syn tak drahý, dítě mého potěšení? Kdykoli však o něm mluvím, znovu a znovu si ho připomínám. Proto je mé nitro nad ním zneklidněno. Slituji, slituji se nad ním, je výrok Hospodinův.“
#31:21 Postav si milníky, vztyč ukazatele, zamysli se nad cestou upravenou, nad cestou, po níž jsi chodila. Vrať se, izraelská panno, vrať se do těchto svých měst!
#31:22 Jak dlouho budeš pobíhat sem a tam, dcero odpadlice? Hospodin stvoří na zemi novou věc: žena se bude ucházet o muže.
#31:23 Toto praví Hospodin zástupů, Bůh Izraele: „Opět budou mluvit toto slovo v zemi judské a v jejích městech, až změním jejich úděl: ‚Požehnej ti Hospodin, nivo spravedlnosti, horo svatá.‘
#31:24 Bude na ní bydlet Juda a zároveň všechna jeho města, oráči i ti, kteří táhnou se stádem.
#31:25 Občerstvím duši znavenou a každou duši tesknící ukojím.“
#31:26 Potom jsem se probudil a prohlédl jsem. Můj spánek mi byl příjemný.
#31:27 „Hle, přicházejí dny, je výrok Hospodinův, kdy oseji dům izraelský i dům judský lidmi i zvířaty.
#31:28 A jako jsem nad nimi bděl, abych rozvracel a podvracel, bořil a ničil a škodil, tak budu nad nimi bdít, abych budoval a sázel, je výrok Hospodinův.
#31:29 V oněch dnech už nebudou říkat: ‚Otcové jedli nezralé hrozny a synům trnou zuby‘,
#31:30 nýbrž každý zemře pro vlastní nepravost. Každému, kdo jí nezralé hrozny, budou trnout zuby.“
#31:31 „Hle, přicházejí dny, je výrok Hospodinův, kdy uzavřu s domem izraelským i s domem judským novu smlouvu.
#31:32 Ne takovou smlouvu, jakou jsem uzavřel s jejich otci v den, kdy jsem je uchopil za ruku, abych je vyvedl z egyptské země. Oni mou smlouvu porušili, ale já jsem zůstal jejich manželem, je výrok Hospodinův.
#31:33 Toto je smlouva, kterou uzavřu s domem izraelským po oněch dnech, je výrok Hospodinův: Svůj zákon jim dám do nitra, vepíši jim jej do srdce. Budu jim Bohem a oni budou mým lidem.
#31:34 Už nebude učit každý svého bližního a každý svého bratra: ‚Poznávejte Hospodina!‘ Všichni mě budou znát, od nejmenšího do největšího z nich, je výrok Hospodinův. Odpustím jim jejich nepravost a jejich hřích už nebudu připomínat.“
#31:35 Toto praví Hospodin, který dává slunce za světlo ve dne, měsíc a hvězdy za světlo v noci podle svých ustanovení, který vzdouvá moře, takže jeho vlny hučí, jehož jméno je Hospodin zástupů:
#31:36 „Jestliže přestanou tato ustanovení přede mnou platit, je výrok Hospodinův, také potomstvo Izraele nebude přede mnou už ani pronárodem po všechny dny.“
#31:37 Toto praví Hospodin: „Jestliže budou změřena nebesa nahoře a prozkoumány základy země dole, i já zavrhnu všechno potomstvo Izraele, kvůli všemu, čeho se dopustili, je výrok Hospodinův.“
#31:38 „Hle, přicházejí dny, je výrok Hospodinův, kdy toto město bude opět vystavěno pro Hospodina, od věže Chananeelu až k bráně Nárožní.
#31:39 Odtud vyjde měřící šňůra přímo k pahorku Garébu a stočí se do Goje.
#31:40 Celá dolina s mrtvými těly a popelem z obětí i všechna lada až k úvalu Kidrónskému, až k rohu brány Koňské na východě, to vše bude svaté pro Hospodina. Už nikdy to nebude vyvráceno ani zbořeno.“ 
#32:1 Slovo, které se stalo od Hospodina k Jeremjášovi v desátém roce vlády Sidkijáše, krále judského; ten rok byl rok osmnáctý vlády Nebúkadnesarovy.
#32:2 Tehdy vojsko babylónského krále obléhalo Jeruzalém a prorok Jeremjáš byl uvězněn na nádvoří stráží při domě judského krále.
#32:3 Uvěznil ho Sidkijáš, judský král. Řekl: „Proč prorokuješ: ‚Toto praví Hospodin: Hle, vydám toto město do rukou babylónského krále a ten je dobude.
#32:4 Ani Sidkijáš, král judský, nevyvázne z rukou Kaldejců. Určitě bude vydán do rukou babylónského krále, bude s ním mluvit ústy k ústům a pohlédne mu do očí.
#32:5 Ten zavede Sidkijáše do Babylóna a bude tam, dokud jej nenavštívím, je výrok Hospodinův. I když budete proti Kaldejcům bojovat, nic se vám nezdaří.“
#32:6 Jeremjáš řekl: „Stalo se ke mně slovo Hospodinovo.
#32:7 Hle, přijde k tobě Chanameel, syn tvého strýce Šalúma, s nabídkou: ‚Kup si mé pole v Anatótu, protože máš výkupní právo.‘
#32:8 Když ke mně podle Hospodinova slova přišel Chanameel, syn mého strýce, na nádvoří stráží, řekl mi: ‚Kup prosím mé pole, které je v Anatótu v zemi Benjamínově, neboť máš právo získat je do vlastnictví; máš totiž výkupní právo, kup si je.‘ I poznal jsem, že to je slovo Hospodinovo.
#32:9 Koupil jsem tedy od Chanameela, syna svého strýce, pole, které je v Anatótu, a odvážil jsem mu stříbro, sedmnáct šekelů stříbra.
#32:10 Napsal jsem listinu, zapečetil, povolal jsem svědky a odvážil na vahách stříbro.
#32:11 Pak jsem vzal kupní listinu, jak zapečetěnou podle nařízených směrnic, tak otevřenou,
#32:12 a dal jsem tu kupní listinu Bárukovi, synu Nerijáše, syna Masejášova, v přítomnosti Chanameela, syna svého strýce, a v přítomnosti svědků, kteří podepsali kupní listinu, v přítomnosti všech Judejců, kteří se usadili na nádvoří stráží.
#32:13 Bárukovi jsem v jejich přítomnosti přikázal:
#32:14 Toto praví Hospodin zástupů, Bůh Izraele: Vezmi tyto listiny, tu kupní listinu zapečetěnou i listinu otevřenou, a vlož je do hliněné nádoby, aby se uchovaly po mnoho dní.
#32:15 Toto praví Hospodin zástupů, Bůh Izraele: Opět se v této zemi budou kupovat domy, pole i vinice!“
#32:16 Potom, když jsem dal kupní listinu Bárukovi, synu Nerijášovu, modlil jsem se k Hospodinu:
#32:17 „Ach, Panovníku Hospodine, hle, ty jsi učinil nebesa i zemi svou velikou mocí a svou vztaženou paží. Tobě není nic nemožného.
#32:18 Prokazuješ milosrdenství tisícům a splácíš nepravost otců do klína jejich synů, kteří přijdou po nich. Ty jsi Bůh veliký, bohatýr, jehož jméno je Hospodin zástupů,
#32:19 veliký v úradku a mocný v skutcích, tvé oči jsou otevřené na všechny cesty synů lidských, abys každému odplatil podle jeho cest a podle ovoce jeho skutků.
#32:20 Ty jsi činil znamení a zázraky v zemi egyptské a až do tohoto dne, jak v Izraeli, tak mezi všemi lidmi. Tak sis učinil jméno, jak je tomu v tento den.
#32:21 Vyvedl jsi Izraele, svůj lid, ze země egyptské se znameními a zázraky, pevnou rukou a vztaženou paží a za velké bázně.
#32:22 Dal jsi jim tuto zemi, o níž ses zapřisáhl jejich otců, že jim ji dáš, zemi, oplývající mlékem a medem.
#32:23 Když však vešli a dostali ji, neposlouchali tě a tvým zákonem se neřídili. Nic z toho, co jsi jim přikázal konat, nečinili. Proto jsi na ně přivolal všechno toto zlo.
#32:24 Hle, náspy! Přitáhli k tomuto městu, aby je dobyli. Město bude vydáno do rukou Kaldejců, kteří proti němu bojují; padne mečem, hladem a morem. Co jsi mluvil, to se stane, ty sám na to budeš patřit.
#32:25 A ty, Panovníku Hospodine, mi teď říkáš: ‚Kup si stříbro pole a povolej svědky!‘ Vždyť město bude vydáno do rukou Kaldejců.“
#32:26 I stalo se slovo Hospodinovo k Jeremjášovi:
#32:27 „Hle, já jsem Hospodin, Bůh veškerého tvorstva. Je pro mne něco nemožného?
#32:28 Proto praví Hospodin toto: Hle, vydám toto město do rukou Kaldejců a do rukou Nebúkadnesara, krále babylónského, a ten je dobude.
#32:29 Přitáhnou Kaldejci a budou proti tomuto městu bojovat, založí v tomto městě oheň a spálí je i všechny domy, na jejichž střechách pálili kadidlo Baalovi a přinášeli úlitby jiným bohům, aby mě uráželi.
#32:30 Vždyť Izraelci i Judejci se už od svého mládí dopouštěli toho, co je zlé v mých očích, Izraelci mě uráželi tím, co si vlastníma rukama udělali, je výrok Hospodinův.
#32:31 Toto město mi bylo jen pro hněv a rozhořčení ode dne, kdy je vystavěli, až do toho dne, kdy je vymýtím od své tváře,
#32:32 pro všechno zlo, které Izraelci i Judejci páchali, aby mě uráželi, oni, jejich králové, velmožové, kněží i proroci, judští mužové i obyvatelé Jeruzaléma.
#32:33 Obrátili se ke mně zády a ne tváří; ačkoli jsem je poučoval nepřetržitě a horlivě, neposlouchali a nepřijímali napomenutí.
#32:34 Ohyzdné modly umístili v domě, který se nazývá mým jménem, a tak jej potřísnili.
#32:35 Baalovi vybudovali posvátná návrší, která jsou v Údolí syna Hinómova, a prováděli své syny a dcery ohněm Molekovi. To jsem jim nepřikázal, ba ani mi na mysl nepřišlo, že se budou dopouštět takových ohavností, a tím svedou Judu k hříchu.“
#32:36 Proto nyní toto praví Hospodin, Bůh Izraele, o tomto městě, o němž říkáte, že bude vydáno do rukou krále babylónského mečem, hladem a morem:
#32:37 „Hle, já je shromáždím ze všech zemí, do nichž jsem je ve svém hněvu, rozhořčení a velikém rozlícení rozehnal. Přivedu je zpět na toto místo a usadím je v bezpečí.
#32:38 Budou mým lidem a já jim budu Bohem.
#32:39 Dám jim jedno srdce a jednu cestu, aby se mě báli po všechny dny, aby dobře bylo jim i jejich synům po nich.
#32:40 Uzavřu s nimi smlouvu věčnou, že už jim nepřestanu prokazovat dobro. Do jejich srdcí dám bázeň přede mnou, aby ode mne neodstupovali.
#32:41 Budu se z nich veselit, z celého srdce a z celé duše prokazovat dobro; v té zemi je opravdu zasadím.“
#32:42 Toto praví Hospodin: „Jako jsem uvedl na tento lid všechno to veliké zlo, tak na ně uvedu všechno to dobro, které jsem jim slíbil.
#32:43 A budou kupovat pole v této zemi, o níž říkáte: ‚Je zpustošená, není v ní člověka ani dobytka, je vydáno do rukou Kaldejců.‘
#32:44 Budou kupovat pole za stříbro, sepisovat listiny, zapečeťovat je, přivádět svědky v zemi Benjamínově, v okolí Jeruzaléma, v městech judských, v městech vrchoviny, v městech Přímořské nížiny i v městech Negebu, protože jejich úděl změním, je výrok Hospodinův.“ 
#33:1 I stalo se slovo Hospodinovo k Jeremjášovi podruhé, když byl ještě ve vazbě v nádvoří stráží:
#33:2 „Toto praví Hospodin, který to učiní, Hospodin, který vytvoří, co zůstane nepohnutelné, Hospodin je jeho jméno.
#33:3 Volej ke mně a odpovím ti. Chci ti oznámit veliké a nedostupné věci, které neznáš.“
#33:4 Toto praví Hospodin, Bůh Izraele, o domech tohoto města a o domech judských králů, které byly strženy kvůli náspům a kvůli meči,
#33:5 když se Judejci chystali do boje s Kaldejci: „Budou naplněny mrtvými těli lidí, které zabiji ve svém hněvu a rozhořčení, protože jsem skryl svou tvář před tímto městem kvůli všemu zlu, které páchali.
#33:6 Hle, jeho ránu zahojím a vyléčím, uzdravím je a zjevím jim hojnost pokoje a pravdy.
#33:7 Změním úděl Judy i úděl Izraele a vybuduji je jako na začátku.
#33:8 Očistím je ode všech jejich nepravostí, jimiž proti mně hřešili, a odpustím jim všechny nepravosti, jimiž proti mně hřešili a jimiž se mi vzepřeli.
#33:9 A já budu mít jméno k veselí, chvále a oslavě mezi všemi pronárody na zemi, až uslyší o všem dobru, které pro svůj lid učiním. Budou se chvět strachem a rozčilením pro všechno to dobro a pro všechen pokoj, který svému lidu zjednám.
#33:10 Toto praví Hospodin: Říkáte: ‚Toto místo leží v troskách. V judských městech není člověka ani dobytka. Jeruzalémské ulice jsou zpustošené, bez člověka, bez obyvatele, bez dobytka.‘ Opět zde však bude slyšet
#33:11 hlas veselí, hlas radosti, hlas ženicha a hlas nevěsty, hlas těch, kteří vybízejí: ‚Chválu vzdejte Hospodinu zástupů, protože Hospodin je dobrý, jeho milosrdenství je věčné.‘ Budou tu i ti, kteří přinášejí do Hospodinova domu děkovnou oběť. Změním totiž úděl této země jako na počátku, praví Hospodin.
#33:12 Toto praví Hospodin zástupů: Na tomto místě, které je v troskách, bez člověka a bez dobytka, i ve všech jeho městech budou opět nivy pastýřů, kde budou s ovcemi odpočívat.
#33:13 Ve městech vrchoviny i ve městech Přímořské nížiny i ve městech Negebu, v zemi Benjamínově i v okolí Jeruzaléma a v městech judských budou opět procházet ovce pod rukama toho, který je počítá, praví Hospodin.“
#33:14 „Hle, přicházejí dny, je výrok Hospodinův, kdy splním to dobré slovo, které jsem vyhlásil o domě Izraelově a o domě Judově.
#33:15 V oněch dnech a v onen čas způsobím, aby Davidovi vyrašil výhonek spravedlivý, a ten bude v zemi uplatňovat právo a spravedlnost.
#33:16 V oněch dnech dojde Judsko spásy, Jeruzalém bude přebývat v bezpečí a takto jej nazvou: ‚Hospodin - naše spravedlnost.‘
#33:17 Toto praví Hospodin: Nebude vyhlazen Davidovi následník na trůnu Izraelova domu.
#33:18 Ani lévijským kněžím nebude vyhlazen přede mnou ten, kdo by přinášel oběti zápalné, v dým obracel oběť přídavnou a po všechny dny připravoval obětní hod.“
#33:19 I stalo se slovo Hospodinovo k Jeremjášovi:
#33:20 „Toto praví Hospodin: Podaří-li se vám zrušit moji smlouvu se dnem a moji smlouvu s nocí, takže nebude dne ani noci v určený čas,
#33:21 pak bude zrušena i má smlouva s Davidem, mým služebníkem, a nebude mít syna, který by kraloval na jeho trůnu, i s lévijskými kněžími, mými sluhy.
#33:22 Jak je nesčíslný nebeský zástup a nezměřitelný mořský písek, tak rozmnožím potomstvo Davida, svého služebníka, i lévijců, kteří mi slouží.“
#33:23 I stalo se slovo Hospodinovo k Jeremjášovi:
#33:24 „Nevidíš, co tento lid mluví? Obě čeledi, které Hospodin vyvolil, prý zavrhl. Můj lid znevažují, jako by pro ně nebyl už ani pronárodem.
#33:25 Toto praví Hospodin: Jakože jsem ustanovil svou smlouvu se dnem a nocí jako směrnici pro nebe i zemi,
#33:26 právě tak nezavrhnu potomstvo Jákoba a Davida, svého služebníka; z jeho potomstva budu brát vládce pro potomstvo Abrahamovo, Izákovo a Jákobovo, až změním jejich úděl a slituji se nad nimi.“ 
#34:1 Slovo, které se stalo od Hospodina, k Jeremjášovi, když Nebúkadnesar, král babylónský, a všechno jeho vojsko a všechna království země ovládaná jeho rukou i všechny národy bojovali proti Jeruzalému a proti všem jeho městům:
#34:2 „Toto praví Hospodin, Bůh Izraele: Jdi a řekni Sidkijášovi, králi judskému: Toto praví Hospodin: Hle, vydám toto město do rukou babylónského krále a ten je vypálí.
#34:3 Ani ty mu z rukou neunikneš. Určitě budeš chycen, budeš mu vydán do rukou. Pohlédneš babylónskému králi do očí a on s tebou bude mluvit ústy k ústům. Přijdeš do Babylóna.
#34:4 Ale slyš slovo Hospodinovo, Sidkijáši, králi judský, toto o tobě praví Hospodin: Nezemřeš mečem,
#34:5 zemřeš pokojně. Jako spalovali vonné látky k poctě tvých otců, králů dřívějších, kteří byli před tebou, tak je budou spalovat i tobě. Budou nad tebou naříkat: ‚Běda, pane!‘ Já jsem promluvil to slovo, je výrok Hospodinův.“
#34:6 Prorok Jeremjáš mluvil v Jeruzalémě k Sidkijášovi, králi judskému, všechna tato slova.
#34:7 Vojsko babylónského krále bojovalo proti Jeruzalému a proti všem zbylým judským městům, proti Lakíši a Azece; z opevněných judských měst zůstala jen tato města.
#34:8 Slovo, které se stalo od Hospodina k Jeremjášovi, když král Sidkijáš uzavřel s veškerým lidem v Jeruzalémě smlouvu o vyhlášení svobody:
#34:9 „Každý ať propustí svého hebrejského otroka a každý svou hebrejskou otrokyni na svobodu, aby nikdo z Judejců neotročil svému bratru.“
#34:10 Všichni velmožové i všechen lid, kteří přistoupili k smlouvě, uposlechli a každý propustil svého otroka i svou otrokyni na svobodu, takže jim už nesloužili. Uposlechli a propustili je.
#34:11 Ale potom zas obrátili a donutili k návratu otroky a otrokyně, které propustili na svobodu. Podmanili si je opět za otroky a otrokyně.
#34:12 I stalo se od Hospodina k Jeremjášovi slovo Hospodinovo.
#34:13 „Toto praví Hospodin, Bůh Izraele: Já jsem uzavřel smlouvu s vašimi otci v den, kdy jsem je vyvedl z egyptské země, z domu otroctví:
#34:14 ‚Koncem sedmého roku propustíte každý svého hebrejského bratra, který se ti prodal. Bude ti sloužit šest let a pak ho propustíš od sebe na svobodu.‘ Ale vaši otcové mě neposlechli a nenaklonili ucho.
#34:15 Vy jste dnes obrátili a učinili jste, co je správné v mých očích, když jste každý vyhlásil svému bližnímu volnost. Smlouvu jste uzavřeli přede mnou v domě, který se nazývá mým jménem.
#34:16 Ale pak jste opět obrátili a mé jméno jste znesvětili. Donutili jste k návratu každý svého otroka a otrokyni, které jste propustili na svobodu. Podmanili jste si je opět, aby byli vašimi otroky a otrokyněmi.
#34:17 Proto Hospodin praví toto: Protože jste mě neuposlechli a nevyhlásili každý svobodu svému bratru a bližnímu, hle, vyhlašuji proti vám, je výrok Hospodinův, volnost meči, moru a hladu. Učiním vás obrazem hrůzy pro všechna království země.
#34:18 Vydám muže, kteří přestupují mou smlouvu, kteří neplní slova smlouvy, kterou přede mnou uzavřeli. Rozťali býčka na dvě poloviny a prošli mezi jeho díly.
#34:19 Velmožové judští i velmožové jeruzalémští, dvořané i kněží a všechen lid země prošli mezi díly býčka.
#34:20 Vydám je do rukou jejich nepřátel, do rukou těch, kteří jim ukládají o život. Jejich mrtvoly budou za pokrm nebeskému ptactvu a zemskému zvířectvu.
#34:21 I Sidkijáše, krále judského, a jeho velmože vydám do rukou jejich nepřátel, do rukou těch, kteří jim ukládají o život, do rukou vojska babylónského krále, které od vás odtáhlo.
#34:22 Hle, dám příkaz, je výrok Hospodinův, a přivedu je zpět na toto město. Budou proti němu bojovat, dobudou je a vypálí je. Z judských měst učiním zpustošený kraj, budou bez obyvatele.“ 
#35:1 Slovo, které se stalo k Jeremjášovi od Hospodina za dnů Jójakíma, syna Jóšijášova, krále judského.
#35:2 „Jdi do domu Rekábejců, promluv k nim a doveď je do jedné ze síní Hospodinova domu a dej jim pít víno.“
#35:3 Vzal jsem tedy Jaazanjáše, syna Jeremjáše, syna Chabasinjášova, a jeho bratry a všechny jeho syny, celý dům Rekábejců.
#35:4 Uvedl jsem je do Hospodinova domu, do síně synů Chánana, syna Jigdaljáše, muže Božího; ta byla vedle síně velmožů, která byla nad síní Maasejáše, syna Šalúmova, strážce prahu.
#35:5 Postavil jsem před syny Rekábejců kalichy plné vína a číše a řekl jsem jim: „Napijte se vína!“
#35:6 Ale oni odmítli: „Nebudeme pít víno, neboť náš otec Jónadab, syn Rekábův, nám přikázal: ‚Nebudete nikdy pít víno ani vy ani vaši synové!
#35:7 Domy nestavějte, semeno nerozsívejte a vinice nevysazujte, nic vám nebude patřit. Budete po všechny své dny přebývat ve stanech a tak budete žít po mnohé dny na této zemi, v níž jste jen hosté.‘
#35:8 Uposlechli jsme svého otce Jónadaba, syna Rekábova, ve všem, co nám přikázal, takže nepijeme víno po všechny své dny ani my ani naše ženy ani naši synové ani naše dcery,
#35:9 nestavíme domy, abychom v nich bydleli, nemáme vinice ani pole a nesejeme.
#35:10 Přebýváme ve stanech. Uposlechli jsme. Konáme přesně to, co nám přikázal náš otec Jónadab.
#35:11 Když vtrhl Nebúkadnesar, král babylónský, do země, řekli jsme si: ‚Pojďte, vejděme před vojskem Kaldejců a před vojskem Aramejců do Jeruzaléma.‘ Proto přebýváme v Jeruzalémě.“
#35:12 I stalo se slovo Hospodinovo k Jeremjášovi:
#35:13 „Toto praví Hospodin zástupů, Bůh Izraele: Jdi a řekni mužům judským a obyvatelům Jeruzaléma: Což nepřijmete napomenutí a neuposlechnete mých slov? je výrok Hospodinův.
#35:14 Rekábejci plní slova Jónadaba, syna Rekábova, který zakázal svým synům pít víno. Nepijí je až dodnes, protože poslouchají příkaz svého otce. A já jsem k vám mluvil, nepřetržitě jsem mluvil, ale vy jste mě neposlouchali.
#35:15 Posílal jsem k vám všechny své služebníky proroky, nepřetržitě jsem je posílal se slovy: ‚Obraťte se každý od své zlé cesty, napravte své jednání, nechoďte za jinými bohy, neslužte jim. Budete bydlet v zemi, kterou jsem dal vám i vašim otcům.‘ Ale nenaklonili jste své ucho a neuposlechli jste mě.
#35:16 Synové Jónadaba, syna Rekábova, plní příkaz svého otce, který jim dal; mne však tento lid neposlouchá.
#35:17 Proto Hospodin, Bůh zástupů, Bůh Izraele, praví toto: Hle, uvedu na Judu i na všechny obyvatele Jeruzaléma všechno zlo, jež jsem proti nim vyhlásil, protože jsem k nim mluvíval, a neposlouchali, volával jsem na ně, a neodpovídali.“
#35:18 A domu Rekábejců Jeremjáš řekl: „Toto praví Hospodin zástupů, Bůh Izraele: Protože jste byli poslušni příkazu svého otce Jónadaba, zachovávali jste všechny jeho příkazy a konali jste přesně to, co vám přikázal,
#35:19 proto Hospodin zástupů, Bůh Izraele, praví toto: Z rodu Jónadaba, syna Rekábova, nebude vyhlazen ten, kdo by stál v mých službách po všechny dny.“ 
#36:1 Ve čtvrtém roce vlády Jójakíma, syna Jóšijášova, krále judského, stalo se od Hospodina k Jeremjášovi toto slovo:
#36:2 „Vezmi si svitek knihy a napiš do něho všechna slova, která jsem k tobě mluvil o Izraeli a Judovi i o všech pronárodech ode dne, kdy jsem k tobě promluvil, ode dnů Jóšijášových až dodnes.
#36:3 Snad až dům judský uslyší o všem zlu, které jim zamýšlím učinit, odvrátí se každý od své zlé cesty a já jim jejich nepravost a jejich hřích odpustím.“
#36:4 Jeremjáš tedy zavolal Báruka, syna Nerijášova. Báruk zapsal do svitku knihy všechna slova z Jeremjášových úst, která k němu Hospodin mluvil.
#36:5 Pak Jeremjáš Bárukovi přikázal: „Jsem ve vazbě a nesmím chodit do Hospodinova domu.
#36:6 Ale ty jdi a ze svitku, do něhož jsi zapsal Hospodinova slova z mých úst, předčítej lidu v domě Hospodinově v den postu, předčítej je i všem Judejcům, kteří přijdou ze svých měst.
#36:7 Snad padnou na kolena a budou prosit před Hospodinem o smilování a každý se odvrátí od své zlé cesty. Vždyť veliký je hněv a rozhořčení, o němž Hospodin tomuto lidu mluvil!“
#36:8 Báruk, syn Nerijášův, učinil všechno, co mu přikázal prorok Jeremjáš, a četl v Hospodinově domě z knihy Hospodinova slova.
#36:9 V pátem roce vlády Jójakíma, syna Jóšijášova, krále judského, v devátém měsíci, vyhlásili půst před Hospodinem pro všechen lid v Jeruzalémě i pro všechen lid, přicházející do Jeruzaléma z judských měst.
#36:10 Báruk předčítal z knihy Jeremjášova slova všemu lidu v Hospodinově domě v síni písaře Gemarjáše, syna Šáfanova, na horním nádvoří u vchodu do Nové brány Hospodinova domu.
#36:11 Míkajáš, syn Gemarjáše, syna Šáfanova, slyšel z té knihy všechna Hospodinova slova.
#36:12 Sešel dolů do královského domu do síně písařovy, kde zasedali všichni velmožové: písař Elíšama, Delajáš, syn Šemajášův, Elnátan, syn Akbórův, Gemarjáš, syn Šáfanův, a Sidkijáš, syn Chananjášův. Byli tam všichni velmožové.
#36:13 Mikajáš jim oznámil všechna slova, která slyšel, když Báruk lidu předčítal z knihy.
#36:14 Všichni ti velmožové poslali k Bárukovi Jehúdího, syna Netanjáše, syna Šelemjáše, syna Kúšího, se vzkazem: „Vezmi s sebou svitek, z něhož jsi lidu předčítal, a pojď.“ Báruk, syn Nerijášův, vzal s sebou svitek a přišel k nim.
#36:15 Vyzvali jej: „Posaď se a předčítej nám.“ Báruk jim tedy předčítal.
#36:16 Když slyšeli všechna ta slova, chvěli se jeden jako druhý strachem a řekli Bárukovi: „Všechna tato slova musíme oznámit králi.“
#36:17 Požádali Báruka: „Sděl nám, jak jsi všechna tato slova napsal. Z jeho úst?“
#36:18 Báruk jim odpověděl: „Vlastními ústy ke mně všechna toto slova pronášel a já jsem je zapisoval černidlem do knihy.“
#36:19 Velmožové Bárukovi řekli: „Jdi a skryj se i s Jeremjášem, ať nikdo neví, kde jste!“
#36:20 Pak vešli ke králi do dvorany. Svitek uschovali v síni písaře Elíšamy a oznámili králi všechna ta slova.
#36:21 Král poslal Jehúdího pro svitek. Ten jej vzal ze síně písaře Elíšamy a předčítal z něho králi i všem velmožům, kteří stáli kolem krále.
#36:22 Král bydlel v zimním domě, byl totiž devátý měsíc, a na ohništi před ním hořelo.
#36:23 Jak Jehúdí přečetl tři čtyři sloupce, král je ze svitku odřízl nožíkem a házel je do ohně na ohništi, až celý svitek na ohništi shořel.
#36:24 Král ani jeho služebníci se nezachvěli strachem a neroztrhli svá roucha, když slyšeli všechna ta slova.
#36:25 Ačkoli Elnátan, Delajáš a Gemarjáš na krále naléhali, aby svitek nepálil, neuposlechl je.
#36:26 Král přikázal Jerachmeelovi, synu královskému, Serajášovi, synu Azríelovu, a Šelemjášovi, synu Abdeelovu, aby písaře Báruka i proroka Jeremjáše jali; Hospodin je však ukryl.
#36:27 I stalo se k Jeremjášovi slovo Hospodinovo poté, když král spálil svitek se slovy, která zapsal Báruk z Jeremjášových úst:
#36:28 „Znovu si vezmi jiný svitek a napiš na něj všechna předešlá slova, která byla na svitku předešlém, který Jójakím, král judský, spálil.
#36:29 A Jójakímovi, králi judskému, řekni: Toto praví Hospodin: Ty jsi ten svitek spálil a řekl jsi: ‚Proč jsi na něm napsal, že určitě přitáhne babylónský král a přinese této zemi zkázu, takže v ní nebudou lidé ani dobytek?‘
#36:30 Proto praví Hospodin o Jójakímovi, králi judském, toto: Nebude mít nikoho, kdo by seděl na Davidově trůnu. Jeho mrtvola bude pohozena na denním vedru a na nočním chladu.
#36:31 Jej i jeho potomky a jeho služebníky za jejich nepravosti ztrestám. Uvedu na ně a na obyvatele Jeruzaléma i na muže judské všechno to zlo, které jsem jim ohlásil, a oni neposlechli.“
#36:32 Jeremjáš vzal tedy jiný svitek, dal je písaři Bárukovi, synu Nerijášovu, a ten do něho zapsal z Jeremjášových úst všechna slova té knihy, kterou Jójakím, král judský, v ohni spálil. Bylo k nim přidáno ještě mnoho podobných slov. 
#37:1 Král Sidkijáš, syn Jóšijášův, se stal králem místo Konjáše, syna Jójakímova. Ustanovil ho králem v zemi judské Nebúkadnesar, král babylónský.
#37:2 Ale ani on ani jeho služebníci ani lid země neuposlechli slov Hospodinových, která mluvil skrze proroka Jeremjáše.
#37:3 Král Sidkijáš poslal Jehúkala, syna Šelemjášova, a kněze Sefanjáše, syna Maasejášova, k proroku Jeremjášovi se vzkazem: „Modli se prosím za nás k Hospodinu, našemu Bohu.“
#37:4 Jeremjáš vcházel a vycházel uprostřed lidu, dosud jej nedali do žaláře.
#37:5 Faraónovo vojsko vytáhlo z Egypta. Když Kaldejci, kteří obléhali Jeruzalém, uslyšeli o nich zprávu, odtáhli od Jeruzaléma.
#37:6 I stalo se k proroku Jeremjášovi slovo Hospodinovo:
#37:7 „Toto praví Hospodin, Bůh Izraele: Toto řekněte králi judskému, který vás ke mně poslal, abyste se mě dotazovali. Hle, vojsko faraónovo, které vám vytáhlo na pomoc, se navrátí do své země, do Egypta,
#37:8 a vrátí se Kaldejci, budou bojovat proti tomuto městu, dobudou je a vypálí.
#37:9 Toto praví Hospodin: Nepodvádějte sami sebe slovy: ‚Kaldejci od nás určitě odejdou.‘ Ti neodejdou!
#37:10 I kdybyste pobili celé kaldejské vojsko, které proti vám bojuje, takže by zůstali jen muži skolení, každý z nich by ve svém stanu povstal a toto město by vypálili.“
#37:11 Když kaldejské vojsko odtáhlo od Jeruzaléma před vojskem faraónovým,
#37:12 vyšel Jeremjáš z Jeruzaléma a šel do země Benjamínovy, aby se tam ujal svého podílu uprostřed lidu.
#37:13 Už byl v Benjamínské bráně, kde byl dohlížitel jménem Jirijáš, syn Šelemjáše, syna Chananjášova; ten proroka Jeremjáše chytil. Řekl: „Chceš přeběhnout ke Kaldejcům!“
#37:14 Jeremjáš odpověděl: „Mýlíš se, nepřecházím ke Kaldejcům.“ Ale Jirijáš Jeremjáše neposlouchal, chytil ho a dovedl k velmožům.
#37:15 Velmožové se na Jeremjáše rozlítili, zbili ho a dali ho do vězení v domě písaře Jónatana, který proměnili v žalář.
#37:16 Jeremjáš se dostal do sklepení, totiž do kobek. Pobyl si tam mnoho dní.
#37:17 Pak pro něj král Sidkijáš poslal, přijal ho tajně ve svém domě a zeptal se ho: „Je nějaké slovo od Hospodina?“ Jeremjáš odpověděl: „Je“, a pokračoval: „Budeš vydán do rukou babylónského krále.“
#37:18 Potom se Jeremjáš krále Sidkijáše tázal: „Čím jsem proti tobě a tvým služebníkům a proti tomuto lidu zhřešil, že jste mě dali do žaláře?
#37:19 Kde jsou vaši proroci, kteří vám prorokovali: ‚Král babylónský nepřitáhne proti vám ani proti této zemi‘?
#37:20 Teď poslyš prosím, králi, můj pane, předkládám ti prosbu o smilování: Neposílej mě zpět do domu písaře Jónatana, abych tam nezemřel.“
#37:21 Král Sidkijáš přikázal dát Jeremjáše pod dohled na nádvoří stráží. Každý den mu dávali z ulice pekařů bochníček chleba, dokud nebyl všechen chléb ve městě sněden. Tak Jeremjáš pobýval v nádvoří stráží. 
#38:1 Šefatjáš, syn Matánův, Gedaljáš, syn Pašchúrův, Júkal, syn Šelemjášův, a Pašchúr, syn Malkijášův, slyšeli slova, která mluvil Jeremjáš ke všemu lidu:
#38:2 „Toto praví Hospodin: Kdo zůstane v tomto městě, zemře mečem, hladem nebo morem, ale kdo vyjde ke Kaldejcům, zůstane naživu a jako kořist získá svůj život, zůstane živ.
#38:3 Toto praví Hospodin: Toto město bude určitě vydáno do rukou vojska babylónského krále a ten je dobude.“
#38:4 Proto řekli velmožové králi: „Nechť je ten muž usmrcen, vždyť oslabuje ruce bojovníků, kteří v tomto městě zbývají, i ruce všeho lidu, když jim mluví podobná slova. Ten muž neusiluje o pokoj pro tento lid, nýbrž o zlo.“
#38:5 Král Sidkijáš odpověděl: „Hle, je ve vašich rukou, neboť král proti vám nic nezmůže.“
#38:6 Chopili se tedy Jeremjáše a vhodili ho do cisterny Malkijáše, syna královského, která byla na nádvoří stráží. Spustili tam Jeremjáše po provazech. V cisterně nebyla voda, jen bláto, takže se Jeremjáš topil v blátě.
#38:7 Kúšijec Ebedmelek, dvořan, který byl v králově domě, uslyšel, že Jeremjáše dali do cisterny. Když král seděl v Benjamínské bráně,
#38:8 vyšel Ebedmelek z královského domu a promluvil ke králi:
#38:9 „Králi, můj pane, zle si počínali tito muži, že to všechno proroku Jeremjášovi provedli. Vhodili ho do cisterny, aby tam dole umřel hlady; ve městě už není žádný chléb.“
#38:10 Král Kúšijci Ebedmelekovi přikázal: „Vezmi s sebou odtud třicet mužů a vytáhni proroka Jeremjáše z cisterny, než tam zemře.“
#38:11 Ebedmelek vzal tedy s sebou muže, vešel do královského domu dolů do skladu, vzal odtud roztrhané hadry a rozedrané šaty a spustil je po provazech Jeremjášovi do cisterny.
#38:12 Kúšijec Ebedmelek řekl Jeremjášovi: „Podlož si ty roztrhané a rozedrané hadry do podpaží pod provazy.“ Jeremjáš to tak udělal.
#38:13 Pak táhli Jeremjáše na provazech a vytáhli ho z cisterny. Jeremjáš opět pobýval na nádvoří stráží.
#38:14 Král Sidkijáš si dal přivést proroka Jeremjáše k třetímu vchodu do Hospodinova domu. Král Jeremjášovi pravil: „Ptám se tě na Hospodinovo slovo. Nic přede mnou nezatajuj!“
#38:15 Jeremjáš Sidkijášovi odpověděl: „Když ti je oznámím, cožpak mě nevydáš na smrt? A když ti poradím, neposlechneš mě.“
#38:16 Král Sidkijáš Jeremjášovi tajně přisáhl: „Jakože živ je Hospodin, který nám dal tento život, nevydám tě na smrt ani do rukou těch mužů, kteří ti ukládají o život.“
#38:17 I řekl Jeremjáš Sidkijášovi: „Toto praví Hospodin, Bůh zástupů, Bůh Izraele: Jestliže vyjdeš k velmožům babylónského krále, zůstaneš sám naživu a toto město nebude vypáleno; zůstaneš naživu ty i tvůj dům.
#38:18 Jestliže však k velmožům babylónského krále nevyjdeš, bude toto město vydáno do rukou Kaldejců a ti je vypálí. Ani ty jim z rukou neunikneš.“
#38:19 Král Sidkijáš Jeremjášovi odvětil: „Obávám se Judejců, kteří přeběhli ke Kaldejcům, že mě Kaldejci vydají do jejich rukou a ti naloží se mnou podle své zvůle.“
#38:20 Jeremjáš odpověděl: „Nevydají. Uposlechni prosím Hospodina, jak ti ohlašuji, a povede se ti dobře a zůstaneš naživu.
#38:21 Jestliže se budeš zdráhat vyjít, nastane to, co mi Hospodin ukázal:
#38:22 Hle, všechny ženy které zůstaly v domě judského krále, budou vyvedeny k velmožům babylónského krále a řeknou: ‚Podněcovali tě a přemohli tě mužové, kteří ti slibovali pokoj. Teď, když tvé nohy tonou v bahně, stáhli se zpět.‘
#38:23 Všechny tvé ženy a tvé syny vyvedou ke Kaldejcům. Ani ty jim z rukou neunikneš. Budeš chycen rukou babylónského krále a toto město bude vypáleno.“
#38:24 Sidkijáš pravil Jeremjášovi: „Ať se nikdo o těchto slovech nedoví, a nezemřeš.
#38:25 Jestliže velmožové uslyší, že jsem s tebou mluvil, a přijdou k tobě a řeknou ti: ‚Pověz nám, co jsi s králem mluvil, nic před námi nezatajuj a nevydáme tě na smrt. A co král mluvil s tebou?‘,
#38:26 řekneš jim: ‚Předložil jsem králi prosbu o smilování, aby mě nedával opět do domu Jónatanova, abych tam nezemřel.‘“
#38:27 I přišli k Jeremjášovi všichni velmožové a vyslýchali ho. On jim pověděl všechno tak, jak král přikázal. Mlčky od něho odešli a nikdo o té věci neuslyšel.
#38:28 Jeremjáš pobýval na nádvoří stráží až do dne, kdy byl Jeruzalém dobyt. I stalo se při dobývání Jeruzaléma, 
#39:1 devátého roku vlády Sidkijáše, krále judského, desátého měsíce, že přitáhl Nebúkadnesar, král babylónský, s celým svým vojskem k Jeruzalému a oblehl jej.
#39:2 Jedenáctého roku vlády Sidkijášovy, čtvrtého měsíce, devátého dne, byly prolomeny hradby města.
#39:3 Všichni velmožové babylónského krále vstoupili do Prostřední brány a usadili se tam: Nergal-sareser, Samgar-nebú, Sarsekím, přední z dvořanů, Nergal-sareser, přední z mágů, a všichni ostatní velitelé babylónského krále.
#39:4 Jakmile je Sidkijáš, král judský, a všichni bojovníci spatřili, uprchli. V noci vyšli z města směrem ke královské zahradě branou mezi dvěma hradbami. Dali se směrem k pustině.
#39:5 Kaldejské vojsko je pronásledovalo a dostihlo Sidkijáše na Jerišských pustinách. Chopili se ho a přivedli jej k Nebúkadnesarovi, králi babylónskému, do Ribly v zemi chamátské, kde nad ním vynesl rozsudek.
#39:6 Babylónský král dal popravit v Rible Sidkijášovy syny před jeho očima; i všechny judské šlechtice dal babylónský král popravit.
#39:7 Sidkijáše oslepil a spoutal ho bronzovými řetězy a odvedl ho do Babylóna.
#39:8 Královský dům i domy lidu Kaldejci vypálili, jeruzalémské hradby zbořili.
#39:9 Zbytek lidu, který zůstal v městě, a přeběhlíky, kteří k němu přeběhli, ten zbytek lidu, který zůstal, přestěhoval Nebúzaradán, velitel tělesné stráže, do Babylóna.
#39:10 Ty nejchudší z lidu, kteří nic neměli, zanechal Nebúzaradán, velitel tělesné stráže, v zemi judské, a dal jim onoho dne vinice a ornou půdu.
#39:11 Nebúkadnesar, král babylónský, vydal prostřednictvím velitele tělesné stráže Nebúzaradána příkaz o Jeremjášovi:
#39:12 „Vezmi jej do své péče a nedělej mu nic zlého. Učiň mu, oč tě požádá.“
#39:13 I poslali tedy Nebúzaradán, velitel tělesné stráže, Nebúšazbán, přední z dvořanů, Nergal-sareser, přední z mágů, a všichni hodnostáři babylónského krále
#39:14 pro Jeremjáše a vzali jej z nádvoří stráží a předali ho Gedaljášovi, synu Achíkama, syna Šáfanova, aby ho přivedl do domu. Tak bydlil uprostřed lidu.
#39:15 K Jeremjášovi, když byl ještě ve vazbě na nádvoří stráží, se stalo slovo Hospodinovo:
#39:16 „Jdi a řekni Kúšijci Ebedmelekovi: Toto praví Hospodin zástupů, Bůh Izraele: Hle, způsobím, že dojde na má slova o tomto městě; bude to ke zlému a ne k dobrému. Stane se tak onoho dne v tvé přítomnosti.
#39:17 Ale tebe v onen den vysvobodím, je výrok Hospodinův. Nebudeš vydán do rukou těch mužů, kterých se lekáš.
#39:18 Určitě tě zachráním, nepadneš mečem a jako kořist získáš svůj život, protože jsi ve mne doufal, je výrok Hospodinův.“ 
#40:1 Slovo, které se stalo od Hospodina k Jeremjášovi poté, co ho Nebúzaradán, velitel tělesné stráže, propustil z Rámy; řetězy spoutaného jej vzal ze středu všech přesídlenců z Jeruzaléma a Judska, těch, kteří měli být přestěhováni do Babylóna.
#40:2 Velitel tělesné stráže vzal tedy Jeremjáše a řekl mu: „Hospodin, tvůj Bůh, mluvil proti tomuto místu o tomto zlu.
#40:3 A Hospodin je uvedl a učinil, jak mluvil, protože jste proti Hospodinu hřešili a neposlouchali jste ho. Proto se vám toto stalo.
#40:4 Nyní, hle, rozvážu ti dnes řetězy na rukou. Pokládáš-li za dobré jít se mnou do Babylóna, pojď, budu o tebe pečovat. Pokládáš-li za zlé jít se mnou do Babylóna, nemusíš chodit. Hleď, celá země je před tebou. Kam pokládáš za dobré a správné jít, tam jdi.
#40:5 Už se to nezvrátí. Obrať se ke Gedaljášovi, synu Achíkama, syna Šáfanova, kterého babylónský král ustanovil správcem judských měst. Přebývej s ním uprostřed lidu nebo jdi, kam pokládáš za správné jít.“ A velitel tělesné stráže mu dal jídlo na cestu a dar a propustil ho.
#40:6 Jeremjáš přišel ke Gedaljášovi, synu Achíkamovu, do Mispy a přebýval s ním uprostřed lidu, který byl ponechán v zemi.
#40:7 Když uslyšeli všichni velitelé vojsk, kteří byli se svým mužstvem v poli, že babylónský král ustanovil správcem země Gedaljáše, syna Achíkamova, a že mu svěřil muže, ženy a malé děti a z chudiny země ty, kteří nebyli přestěhováni do Babylóna,
#40:8 šli ke Gedaljášovi do Mispy. Byli to: Jišmael, syn Netanjášův, Jóchanan a Jónatan, synové Káreachovi, Serajáš, syn Tanchumetův, synové Efaje Netófského a Jezanjáš, syn Maakaťanův, každý se svým mužstvem.
#40:9 Gedaljáš, syn Achíkama, syna Šáfanova, je i jejich mužstvo přísežně ujistil: „Nebojte se sloužit Kaldejcům. Zůstaňte v zemi, služte babylónskému králi a povede se vám dobře.
#40:10 A já, hle, bydlím v Mispě a zastupuji vás před Kaldejci, kteří k nám přicházejí. Sklízejte víno, letní ovoce i olej, ukládejte do svých nádob a bydlete ve svých městech, která držíte.“
#40:11 I všichni Judejci, kteří byli v Moábsku a mezi Amónovci a v Edómsku a kteří byli ve všech zemích, uslyšeli, že babylónský král zanechal Judovi pozůstatek lidu a že nad nimi ustanovil Gedaljáše, syna Achíkama, syna Šáfanova.
#40:12 Vrátili se tedy všichni Judejci ze všech míst, kam byli zahnáni, přišli do země judské do Mispy ke Gedaljášovi a sklidili velice mnoho vína a letního ovoce.
#40:13 Jóchanan, syn Káreachův, a všichni velitelé vojsk, kteří byli v poli, přišli ke Gedaljášovi do Mispy
#40:14 a řekli mu: „Zdalipak víš, že Baalis, král Amónovců, poslal Jišmaela, syna Netanjášova, aby tě zabil?“ Ale Gedaljáš, syn Achíkamův, jim nevěřil.
#40:15 Jóchanan, syn Káreachův, tajně navrhl Gedaljášovi v Mispě: „Když dovolíš, půjdu a zabiji Jišmaela, syna Netanjášova. Nikdo se to nedoví. Proč by měl zabít on tebe? Všichni Judejci, kteří se k tobě shromáždili, byli by rozptýleni a pozůstatek judského lidu by zahynul.“
#40:16 Ale Gedaljáš, syn Achíkamův, Jóchananovi, synu Káreachovu, řekl: „Nedělej to, mluvíš o Jišmaelovi křivě.“ 
#41:1 V sedmém měsíci přišel Jišmael, syn Netanjáše, syna Elíšamova, z královského potomstva, i královští hodnostáři s deseti muži ke Gedaljášovi, synu Achíkamovu, do Mispy. Tam v Mispě pojedli společně chléb.
#41:2 Jišmael, syn Netanjášův, a deset mužů, kteří byli s ním, pak povstali a Gedaljáše, syna Achíkama, syna Šáfanova, zabili mečem. Tak usmrtil toho, kterého babylónský král ustanovil správcem země.
#41:3 Jišmael pobil i všechny Judejce, kteří byli s Gedaljášem v Mispě, i kaldejské bojovníky, kteří tam byli.
#41:4 Druhý den po usmrcení Gedaljáše, když o tom ještě nikdo nevěděl,
#41:5 přišli muži ze Šekemu, Šíla a ze Samaří, bylo jich osmdesát, s ostříhanými vousy, s roztrženými rouchy a se smutečními zářezy. Měli s sebou obětní dar a kadidlo; chtěli to přinést do Hospodinova domu.
#41:6 Jišmael, syn Netanjášův, jim vyšel z Mispy vstříc, šel a plakal. Když se s nimi setkal, řekl jim: „Pojďte ke Gedaljášovi, synu Achíkamovu.“
#41:7 Když přišli do středu města, tu je Jišmael, syn Netanjášův, s muži, kteří byli s ním, zabil a hodil do cisterny.
#41:8 Ale našlo se mezi nimi deset mužů, kteří Jišmaelovi řekli: „Neusmrcuj nás; máme v poli skryté zásoby: pšenici a ječmene, olej i med.“ I nechal je a neusmrtil je s jejich bratřími.
#41:9 Cisterna, do níž Jišmael vhodil všechna mrtvá těla mužů, které zabil s Gedaljášem, byla ta, kterou udělal král Ása, když bojoval proti Baešovi, králi izraelskému. Tu naplnil Jišmael, syn Netanjášův, skolenými.
#41:10 Jišmael zajal celý pozůstatek lidu, který byl v Mispě, královské dcery i všechen lid, který zůstal v Mispě, nad nímž ustanovil Nebúzaradán, velitel tělesné stráže, Gedaljáše, syna Achíkamova. Jišmael, syn Netanjášův, je zajal a chystal se přejít k Amónovcům.
#41:11 Když uslyšel Jóchanan, syn Káreachův, a všichni velitelé vojsk, kteří byli s ním, o všem tom zlu, které spáchal Jišmael, syn Netanjášův,
#41:12 vzali všechno své mužstvo a šli bojovat proti Jišmaelovi, synu Netanjášovu. Našli ho u Veliké vody v Gibeónu.
#41:13 Když všechen lid, který byl s Jišmaelem, spatřil Jóchanana, syna Káreachova, a všechny velitele vojsk, kteří byli s ním, zaradovali se.
#41:14 A všechen lid z Mispy, který Jišmael zajal, se obrátil a šel k Jóchananovi, synu Káreachovu.
#41:15 Ale Jišmael, syn Netanjášův, s osmi muži před Jóchananem unikl a odešel k Amónovcům.
#41:16 Pak Jóchanan, syn Káreachův, se všemi veliteli vojsk, kteří byli s ním, vzal celý pozůstatek lidu z Mispy, který přivedl zpět od Jišmaela, syna Netanjášova, poté co zabil Gedaljáše, syna Achíkamova: muže, bojovníky, ženy, děti i dvořany, a odvedl je z Gibeónu.
#41:17 Šli a pobyli v Kinhámově útulku pro pocestné, který byl poblíž Betléma. Chtěli odejít do Egypta
#41:18 před Kaldejci, jichž se báli, protože Jišmael, syn Netanjášův, zabil Gedaljáše, syna Achíkamova, kterého babylónský král ustanovil správcem země. 
#42:1 Pak přistoupili všichni velitelé vojsk s Jóchanenem, synem Káreachovým, Jizanjášem, synem Hóšajášovým, a vším lidem od nejmenšího do největšího
#42:2 a řekli proroku Jeremjášovi: „Dovol, abychom ti předložili svou prosbu. Modli se za nás k Hospodinu, svému Bohu, za celý tento pozůstatek lidu. Vždyť nás zůstalo jen maličko z mnoha, jak na vlastní oči vidíš.
#42:3 Ať nám Hospodin, tvůj Bůh, oznámí, kam máme jít a co máme dělat.“
#42:4 Prorok Jeremjáš jim odpověděl: „Vyslyším vás, hle, budu se modlit k Hospodinu, vašemu Bohu, podle vašich slov. Každé slovo, které vám Hospodin odpoví, vám oznámím. Nic vám nezatajím.“
#42:5 Oni nato Jeremjášovi řekli: „Nechť je nám Hospodin spolehlivým a věrným svědkem, jestliže se nezachováme přesně podle slova, s nímž tě k nám Hospodin, tvůj Bůh pošle.
#42:6 Ať to bude dobré nebo zlé, Hospodina, svého Boha, k němuž tě posíláme, uposlechneme, aby nám bylo dobře; budeme poslouchat Hospodina, svého Boha.“
#42:7 Po uplynutí deseti dnů stalo se k Jeremjášovi slovo Hospodinovo.
#42:8 Zavolal tedy Jóchanana, syna Káreachova, a všechny velitele vojsk, kteří s ním byli, a všechen lid od nejmenšího do největšího
#42:9 a řekl jim: „Toto praví Hospodin, Bůh Izraele, k němuž jste mě poslali, abych mu předložil vaše prosby:
#42:10 Jestliže se opět usídlíte v této zemi, budu vás budovat a nikoli bořit, zasadím vás a nevyvrátím, protože lituji těch zlých věcí, které jsem vám učinil.
#42:11 Nebojte se babylónského krále, jehož se bojíte; nebojte se ho, je výrok Hospodinův, neboť jsem s vámi, spasím vás a vysvobodím z jeho rukou.
#42:12 Prokážu vám slitování: On se nad vámi slituje a přivede vás zpět do vaší země.
#42:13 Jestliže řeknete: ‚Nebudeme bydlet v této zemi‘ a nebudete chtít uposlechnout Hospodina, svého Boha,
#42:14 řeknete-li: ‚Nikoli, ale půjdeme do země egyptské, kde neuvidíme válku a neuslyšíme zvuk polnice a nebudeme lačnět po chlebu; tam budeme bydlet‘,
#42:15 pak slyšte nyní Hospodinovo slovo, pozůstatku judského lidu. Toto praví Hospodin zástupů, Bůh Izraele: Jestliže máte v úmyslu vstoupit do Egypta, abyste se tam usadili jako hosté,
#42:16 meč, jehož se bojíte, vás tam v zemi egyptské dostihne, a hlad, jehož se obáváte, se vás bude v Egyptě držet; zemřete tam.
#42:17 Všichni muži, kteří mají v úmyslu vstoupit do Egypta, aby tam pobývali jako hosté, zemřou mečem, hladem či morem a žádný z nich nepřežije a neunikne tomu zlu, které na ně uvedu.
#42:18 Toto praví Hospodin zástupů, Bůh Izraele: Jako se vylil můj hněv a mé rozhořčení na obyvatele Jeruzaléma, tak se vylije mé rozhořčení na vás, půjdete-li do Egypta. Stanete se kletbou, úděsem, zlořečením a potupou. Toto místo už neuzříte.
#42:19 Hospodin k vám promluvil, pozůstatku judského lidu. Nechoďte do Egypta. Teď to víte, vždyť vás dnes varuji.
#42:20 Sami se svádíte. Poslali jste mě přece k Hospodinu, svému Bohu, a řekli jste: ‚Modli se za nás, k Hospodinu, našemu Bohu. Všechno, co řekne Hospodin, náš Bůh, nám oznam a my to učiníme.‘
#42:21 Dnes jsem vám to oznámil, ale nechcete uposlechnout Hospodina, svého Boha, v ničem, s čím mě k vám poslal.
#42:22 Nyní tedy vězte, že zemřete mečem, hladem či morem na místě, kam chcete jít a pobývat tam jako hosté.“ 
#43:1 I domluvil Jeremjáš ke všemu lidu všechna slova Hospodina, jejich Boha. Hospodin, jejich Bůh, ho totiž k nim poslal se všemi těmito slovy.
#43:2 Tu Azarjáš, syn Hóšajášův, a Jóchanan, syn Káreachův, a všichni ostatní zpupní muži Jeremjášovi řekli: „Co mluvíš, je klam. Hospodin, náš Bůh, tě neposlal se vzkazem: ‚Nevstupujte do Egypta, abyste tam pobývali jako hosté.‘
#43:3 To Bárúk, syn Nerijášův, tě proti nám podněcuje, aby nás vydal do rukou Kaldejců, aby nás usmrtili nebo přestěhovali do Babylóna.“
#43:4 A Jóchanan, syn Káreachův, a nikdo z velitelů vojsk a nikdo z lidu Hospodina neuposlechli, že mají zůstat v zemi judské.
#43:5 Jóchanan, syn Káreachův, a všichni velitelé vojsk vzali celý pozůstatek judského lidu, ty, kdo se vrátili ze všech pronárodů, kam byli zanháni, aby pobývali v zemi judské jako hosté:
#43:6 muže, ženy i děti, královské dcery i všechny, které Nebúzaradán, velitel tělesné stráže, zanechal s Gedeljášem, synem Achíkama, syna Šáfanova, i proroka Jeremjáše a Báruka, syna Nerijášova.
#43:7 Vstoupili do země egyptské, Hospodina neuposlechli. Přišli až do Tachpanchésu.
#43:8 V Tachpanchésu stalo se k Jeremjášovi slovo Hospodinovo:
#43:9 „Vezmi do rukou velké kameny a ukryj je za přítomnosti judských mužů do jílu v cihelně u vchodu do faraónova domu v Tachpanchésu.
#43:10 Řekneš jim: Toto praví Hospodin zástupů, Bůh Izraele: Hle, já pošlu pro Nebúkadnesara, krále babylónského, svého služebníka, a postavím jeho trůn na těchto kamenech, které jsem ukryl, a roztáhne na nich svůj velkolepý stan.
#43:11 Přitáhne a bude bít egyptskou zemi. Kdo je určen smrti, propadne smrti, kdo zajetí, zajetí, kdo meči, meči.
#43:12 V domech egyptských bohů zanítím oheň. Vypálí je a zajme. Zahalí se egyptskou zemí, jako se pastýř halí do svého šatu. Pak odtud vyjde v pokoji.
#43:13 Roztříští posvátné sloupy v Bét-šemeši, který je v egyptské zemi, a domy egyptských bohů vypálí.“ 
#44:1 Slovo, které se stalo k Jeremjášovi pro všechny Judejce, kteří bydleli v egyptské zemi, kteří bydleli v Migdólu, Tachpanchésu, Memfidě a v zemi Patrósu:
#44:2 „Toto praví Hospodin zástupů, Bůh Izraele: Viděli jste všechno zlo, které jsem uvedl na Jeruzalém a na všechna judská města. Hle, jsou to dnes jen trosky, nikdo v nich nebydlí,
#44:3 a to pro jejich zlé skutky, které páchali. Uráželi mě, když chodili pálit kadidlo a sloužit jiným bohů, které neznali oni ani vy ani vaši otcové.
#44:4 Posílal jsem k vám všechny své služebníky proroky, nepřetržitě jsem je posílal se vzkazem: ‚Nedopouštějte se té ohavnosti, kterou nenávidím!‘
#44:5 Ale neposlechli a nenaklonili své ucho, aby se odvrátili od svých zlých skutků a nepálili kadidlo jiným bohům.
#44:6 Proto se vylilo mé rozhořčení a můj hněv a hořel v městech judských a na ulicích Jeruzaléma, takže se staly troskami a zpustošeným krajem, jak je tomu dodnes.“
#44:7 Nyní toto praví Hospodin, Bůh zástupů, Bůh Izraele: „Proč sami sobě působíte tak veliké zlo? Vytínáte ze společenství Judy muže i ženu, nemluvně i kojence, takže vám nezbude ani pozůstatek lidu.
#44:8 Urážíte mě výtvory svých rukou, když pálíte kadidlo jiným bohům v zemi egyptské, kam jste vstoupili, abyste tam pobývali jako hosté. Vytínáte sami sebe a stanete se zlořečením a potupou mezi všemi pronárody země.
#44:9 Což jste zapomněli na zlé skutky svých otců a na zlé skutky judských králů a na zlé skutky jejich žen i na zlé skutky své a na zlé skutky svých žen, jichž jste se dopouštěli v zemi judské a v ulicích Jeruzaléma?
#44:10 Dodnes nejsou zkrušeni, nebojí se, neřídí se mým zákonem ani mými nařízeními, která jsem předložil vám a vašim otcům.
#44:11 Proto Hospodin zástupů, Bůh Izraele, praví toto: Hle, zaměřím se na vás, ale ke zlému, a celého Judu vytnu.
#44:12 Vezmu pozůstatek judského lidu, ty, kteří pojali úmysl vstoupit do egyptské země, aby tam pobývali jako hosté: Všichni v egyptské zemi zajdou, padnou mečem či hladem, zajdou malí i velcí, zemřou mečem a hladem a stanou se kletbou, úděsem, zlořečením a potupou.
#44:13 Ty, kdo se usadili v egyptské zemi, ztrestám mečem, hladem či morem, jako jsem ztrestal Jeruzalém.
#44:14 Nevyvázne a nepřežije nikdo z pozůstatku judského lidu, z těch, kteří přišli do egyptské zemi, aby tam pobývali jako hosté. Nevrátí se do země judské, ačkoli celou duší prahnou po tom, aby se mohli vrátit a bydlit tam. Nevrátí se, jen někteří vyváznou.“
#44:15 Nato odpověděli Jeremjášovi všichni muži, kteří věděli, že jejich ženy pálívaly kadidlo jiným bohům, i všechny ženy, které tu stály - bylo jich veliké shromáždění -, všechen lid, který pobýval v Patrósu v egyptské zemi:
#44:16 „V tom, co jsi nám mluvil ve jménu Hospodinově, tě neposlechneme.
#44:17 Určitě však splníme všechno, co vyšlo z našich úst, že budeme pálit kadidlo královně nebes a že jí budeme přinášet úlitby tak, jak jsme to dělávali my i naši otcové, naši králové i velmožové v městech judských a na ulicích Jeruzaléma. Tehdy jsme měli chleba dosyta, bylo nám dobře, nic zlého jsme nezakoušeli.
#44:18 Od té doby, kdy jsme přestali pálit kadidlo královně nebes a přinášet jí úlitby, máme nedostatek ve všem a hyneme mečem a hladem.“
#44:19 Ženy dodaly: „Jestliže pálíme kadidlo královně nebes a přinášíme jí úlitby, cožpak jí bez vědomí svých mužů pečeme obětní koláče, jimiž ji zpodobňujeme, a přinášíme jí úlitby?“
#44:20 Jeremjáš řekl všemu lidu, mužům, ženám i všemu lidu, kteří s ním vedli rozhovor:
#44:21 „Cožpak si Hospodin nepřipomněl ten obětní dým z kadidla, jež jste pálili v městech judských a na ulicích Jeruzaléma vy i vaši otcové, vaši králové a vaši velmožové i lid země? Nepřišlo mu to na mysl?
#44:22 Hospodin už nemohl snést vaše zlé skutky a vaše ohavnosti, jichž jste se dopouštěli. Proto se vaše země stala troskami, úděsem a zlořečením, bez obyvatele, jak je tomu dnes.
#44:23 Protože jste pálili kadidlo a hřešili proti Hospodinu a neposlouchali Hospodina, že jste se neřídili jeho zákonem, jeho nařízeními ani jeho svědectvím, potkalo vás toto zlo, jak je tomu dnes.“
#44:24 A Jeremjáš řekl všemu lidu a všem ženám: „Slyšte slovo Hospodinovo, celý Judo v egyptské zemi.
#44:25 Toto praví Hospodin zástupů, Bůh Izraele: Vy i vaše ženy, co jste svými ústy vyslovili, to jste svýma rukama naplnili. Řekli jste: ‚Splníme sliby, které jsme dali. Budeme pálit kadidlo královně nebes a budeme jí přinášet úlitby.‘ Tak si tedy dodržujte své sliby, vyplňujte své sliby!
#44:26 Ale slyšte slovo Hospodinovo, všechen Judo, který pobýváš v egyptské zemi: Hle, zapřísáhl jsem se při svém velikém jménu, praví Hospodin, že mé jméno už nebude vzýváno v celém Egyptě ústy žádného judského muže, který by říkal: ‚Jakože živ je Panovník Hospodin!‘
#44:27 Hle, budu nad nimi bdít ke zlému, a ne k dobrému. Všichni judští muži, kteří jsou v egyptské zemi, do jednoho zajdou mečem a hladem.
#44:28 Jen hrstka lidí unikne meči a vrátí se z egyptské země do země judské. A celý pozůstatek judského lidu, který přišel do egyptské země, aby tam pobýval jako host, pozná, čí slovo obstojí, zda mé či jejich.
#44:29 A toto budete mít jako znamení, je výrok Hospodinův: Na tomto místě vás ztrestám, a tak poznáte, že má slova proti vám obstojí, vám ke zlému.
#44:30 Toto praví Hospodin: Hle, vydám faraóna Chofru, krále egyptského, do rukou jeho nepřátel, do rukou těch, kteří mu ukládají o život, jako jsem vydal Sidkijáše, krále judského, do rukou Nebúkadnesara, krále babylónského, jeho nepřítele, který mu ukládal o život.“ 
#45:1 Slovo, které promluvil prorok Jeremjáš k Bárukovi, synu Nerijášovu, když psal tato slova z Jeremjášových úst do knihy ve čtvrtém roce vlády Jójakíma, syna Jóšijášova, krále judského.
#45:2 „Toto praví Hospodin, Bůh Izraele, o tobě, Báruku.
#45:3 Řekl jsi: ‚Běda mi, neboť Hospodin přidal k mé bolesti ještě starost. Do zemdlení vzdychám, a odpočinutí nenalézám.‘
#45:4 Toto mu povíš: Toto praví Hospodin: Hle, bořím, co jsem vybudoval, vyvracím, co jsem zasadil, celou tuto zemi!
#45:5 A ty bys chtěl pro sebe usilovat o veliké věci? Neusiluj. Neboť hle, já uvedu zlo na všechno tvorstvo v ní, je výrok Hospodinův, ale tobě dám jako kořist život na všech místech, kamkoli půjdeš.“ 
#46:1 Slovo Hospodinovo, které se stalo k proroku Jeremjášovi proti pronárodům.
#46:2 O Egyptu, proti vojsku faraóna Néka, krále egyptského, který byl u řeky Eufratu u Karkemíše a kterého porazil Nebúkadnesar, král babylónský, ve čtvrtém roce vlády Jójakíma, syna Jóšijášova, krále judského.
#46:3 „Připravte štít a pavézu, chystejte se k boji!
#46:4 Zapřahejte koně, nasedejte, jezdci! Nastupte v přilbách, vyleštěte oštěpy, oblékněte krunýře!
#46:5 Cože to vidím? Jsou naplněni děsem, ustupují zpět, jejich bohatýři jsou rozdrceni, dali se na zběsilý útěk, ani se neohlédnou. Kolkolem děs, je výrok Hospodinův.
#46:6 Ať neuteče ani hbitý, ať ani bohatýr neunikne! Na severu u řeky Eufratu zakopnou a padnou.
#46:7 Kdo to vystupuje jako Nil, jako řeky, jejichž vody se vzdouvají?
#46:8 Egypt vystupuje jak Nil, jako řeky, jejichž vody se vzdouvají. Řekl: ‚Vystoupím, přikryji zemi. Zničím město i jeho obyvatele!‘
#46:9 Vzhůru tedy, koně! Divoce se žeňte, vozy! Vyjdou bohatýři, Kúšijci a Pútejci, štítonoši, Luďané, lučištníci.
#46:10 Onen den bude dnem pomsty pro Panovníka, Hospodina zástupů, aby vykonal pomstu na svých protivnících. Meč bude požírat, nasytí se a vydatně se napije jejich krve. Bude to oběť pro Panovníka, Hospodina zástupů, v severní zemi, u řeky Eufratu.
#46:11 Vystup do Gileádu, přines balzám, panno, dcero egyptská! Nadarmo jsi shromáždila léky, rána se ti nezacelí!
#46:12 Pronárody uslyšely o tvém pohanění a tvého žalostného křiku je plná země. Vždyť bohatýr zakopne o bohatýra, oba dva společně padnou.“
#46:13 Slovo, které promluvil Hospodin k proroku Jeremjášovi, že Nebúkadnesar, král babylónský, přitáhne a bude bít egyptskou zemi.
#46:14 „Oznamte v Egyptě, ohlaste v Migdólu, ohlaste to v Memfidě i v Tachpanchésu! Povězte: Postav se, připrav se, meč už pozřel všechno kolem tebe.
#46:15 Jak je možné, že byl smeten tvůj silný? Neobstál. Hospodin jej srazil.
#46:16 Mnohokrát klopýtl, až padl. Jeden druhému říkají: ‚Vstaňme, vraťme se ke svému lidu, do země, kde jsme se zrodili, před hubícím mečem.‘
#46:17 Tam zavolají: ‚Faraóne, králi egyptský, slyš, hukot vřavy, lhůta prošla.‘
#46:18 Jakože živ jsem já, je výrok Krále, jehož jméno je Hospodin zástupů: jako je Tábor mezi horami a Karmel při moři, tak určitě to přijde.
#46:19 Připrav si věci k přesídlení, trůnící dcero egyptská. Memfis bude zpustošena, vylidní se, bude bez obyvatele.
#46:20 Egypt je překrásná jalovice, ale střeček ze severu již letí, letí.
#46:21 I jeho žoldnéři uprostřed něho jsou jako vykrmení býčci. Obrátí se, utečou společně, neobstojí. I na ně přijde den jejich běd, čas, kdy budou ztrestáni.
#46:22 Egypt bude syčet jako had, až se přiblíží vojsko, přijdou na něj jako drvoštěpové se sekerami.
#46:23 Pokácejí jeho les, je výrok Hospodinův, i když je neproniknutelný. Budou četnější než kobylky, bude jich bez počtu.
#46:24 Egyptská dcera bude zahanbena, vydána do rukou lidu ze severu.“
#46:25 Hospodin zástupů, Bůh Izraele, praví: „Hle, já ztrestám Amóna z Théb i faraóna, Egypt, jeho bohy, jeho krále, faraóna i ty, kdo v něho doufají.
#46:26 A vydám je do rukou těch, kdo jim ukládají o život, do ruku Nebúkadnesara, krále babylónského, a do rukou jeho služebníků. A potom bude bydlet jako za dnů dřívějších, je výrok Hospodinův.“
#46:27 „Ty, Jákobe, můj služebníku, neboj se, neděs se, Izraeli, hle, já tě zachráním, i když jsi daleko, i tvé potomky ze země zajetí. Jákob se vrátí a bude žít v klidu, bezstarostně, a nikdo jej nevyděsí.
#46:28 Ty, Jákobe, můj služebníku, neboj se, je výrok Hospodinův. Já budu s tebou. Učiním konec všem pronárodům, mezi něž jsem tě vyhnal, ale s tebou neskončím docela, i když tě potrestám podle práva; bez trestu tě neponechám.“ 
#47:1 Slovo Hospodinovo, které se stalo k proroku Jeremjášovi o Pelištejcích, než farao dobyl Gázu.
#47:2 Toto praví Hospodin: „Hle, od severu vystupují vody a stanou se rozvodněným tokem. Zaplaví zemi se vším, co je na ní, město i ty, kdo v něm bydlí. Lidé budou úpět, kvílet bude každý, kdo obývá zemi,
#47:3 před dusotem kopyt jeho hřebců, před duněním jeho vozby, před hlukem jeho kol. Otcové se na syny ani neohlédnou, ochabnou jim ruce.
#47:4 Přijde den zhouby na všechny Pelištejce. V Týru a Sidónu bude vyhlazen každý, kdo přežil a mohl by pomoci. Hospodin zahubí Pelištejce, pozůstatek lidu z ostrova Kaftóru.
#47:5 Gáza si vyholí lysinu, Aškalón zajde. Pozůstatku lidu z jejich doliny, dlouho si ještě budeš zasazovat smuteční zářezy?“
#47:6 Běda, Hospodinův meči! Jak dlouho si ještě nedopřeješ klidu? Ukliď se do pochvy, mírni se, ustaň!
#47:7 Jak bys mohl mít klid? Hospodin mu dal příkaz proti Aškalónu a proti mořskému pobřeží. Tam jej vykázal. 
#48:1 O Moábovi: Toto praví Hospodin zástupů, Bůh Izraele: „Běda, zahubeno bude Nebó. Zahanben, dobyt bude Kirjatajim. Zahanben bude nedobytný hrad a naplněn děsem.
#48:2 Už je po chloubě Moábově v Chešbónu, zamýšlejí proti němu zlo: ‚Vzhůru, vyhlaďme jej, ať není ani pronárodem!‘ Také ty, Madméne, zajdeš, meč půjde za tebou.
#48:3 Slyš křik z Chóronajimu. ‚Zhouba a veliká zkáza!‘
#48:4 Roztříštěn je Moáb! Je slyšet úpění jeho nejmenších.
#48:5 Do svahu lúchítského půjde v usedavém pláči. Ze stráně chóronajimské uslyší protivníci úpění nad zkázou.
#48:6 ‚Utecte, zachraňte se, budete jako jalovec na poušti!‘
#48:7 Protože spoléháš na své činy a na své poklady, i ty budeš dobyt! A Kemóš půjde s přesídlenci, spolu s ním jeho kněží a velmožové.
#48:8 Na každé město přitáhne zhoubce, žádné město se nezachrání. Dolina zahyne, rovina bude zahlazena, jak praví Hospodin.
#48:9 Dejte Moábovi květ, neboť musí odejít, jeho města budou zpustošena, nikdo v nich nebude bydlet.
#48:10 Proklet buď, kdo koná Hospodinovo dílo nespolehlivě! Proklet buď, kdo odpírá jeho meči krev!“
#48:11 „Moáb byl bezstarostný od svého mládí, klidně si hověl na svém kalu. Nebyl přelíván z nádoby do nádoby, nebyl přestěhován. Proto si podržel svou příchuť, jeho vůně se nezměnila.“
#48:12 „Proto hle, přicházejí dny, je výrok Hospodinův, a pošlu na něj ty, kdo stáčejí víno, a ti jej stočí, jeho nádoby vyprázdní a jejich měchy roztrhají.
#48:13 A Moáb bude zahanben kvůli Kemóšovi tak, jak byl zahanben dům izraelský kvůli Bét-elu, v nějž doufal.“
#48:14 „Jak můžete říkat: ‚Jsme bohatýři, válečníci schopní boje?‘
#48:15 Zahuben bude Moáb, na jeho města vytáhne nepřítel, jeho nejlepší jinoši klesnou jako na porážce, je výrok Krále, jehož jméno je Hospodin zástupů.
#48:16 Blíží se bědy, jež dolehnou na Moába. Zlo se na něj rychle řítí.
#48:17 Projevte mu účast všichni, kdo jste kolem něho, všichni, kdo znáte jeho jméno. Povězte: ‚Jak byla zlomena silná hůl, překrásný prut!‘
#48:18 Ustup ze slávy, seď a žízni, trůnící dcero díbónská. Vytáhl na tebe zhoubce Moábu, tvé pevnosti zničil.
#48:19 Stůj na cestě a vyhlížej, ty, jež bydlíš v Aróeru. Ptej se toho, kdo utekl, a té, která unikla. Zeptej se: ‚Co se stalo?‘
#48:20 Moáb byl zahanben, ztroskotal! Kvilte a úpějte! Oznamte v údolí Arnónu, že Moáb je zahuben.“
#48:21 Soud přišel na rovinatou zemi, na Cholón, na Jahasu, proti Méfaatu,
#48:22 proti Díbónu, proti Nebó, proti Bét-diblatajimu,
#48:23 proti Kirjatajimu, proti Bét-gamúlu, proti Bét-meónu,
#48:24 proti Kerijótu a proti Bosře i proti všem městům moábské země, vzdáleným i blízkým.
#48:25 „Moábův roh byl odseknut, jeho paže roztříštěna, je výrok Hospodinův.
#48:26 Opojte ho, protože se vyvyšoval nad Hospodina! Moáb se udáví svým zvratkem, také on bude na posměch.
#48:27 Což tobě nebyl Izrael na posměch? Byl snad dopaden mezi zloději, že potřásáš hlavou, kdykoli o něm mluvíš?“
#48:28 „Opusťte města, usaďte se na skalisku, obyvatelé Moábu! Buďte jako holubice, která hnízdí nad samou propastí.
#48:29 Slýchali jsme o pýše Moába přepyšného, o jeho domýšlivosti, o pyšné jeho povýšenosti, o jeho nadutém srdci.
#48:30 Já znám, je výrok Hospodinův, jeho zpupnost, k ničemu nejsou jeho žvásty, k ničemu není, co dělají.
#48:31 Proto kvílím nad Moábem, kvůli celému Moábu úpím, nad muži Kír-cheresu se bude lkát.
#48:32 Pláču nad tebou, sibemská vinná révo, víc než jsem plakal nad Jaezerem! Tvé úponky pronikly za moře, až k moři, Jaezeru, dosahovaly. Na tvou letní sklizeň i na tvé vinobraní vpadl zhoubce.
#48:33 Přestaly radost a jásot v sadu i v celé moábské zemi. Zastavil jsem víno, z lisu nepoteče, nebude se už lisovat s výskáním. Výskání nebude!“
#48:34 „Protože křik Chešbónu dolehl až do Eleále, až do Jahasu, vydávají svůj hlas od Sóaru až do Chóronajimu a Eglat-šelišije; i povodí Nimrímu bude zpustošeným krajem.
#48:35 Učiním přítrž Moábovi, je výrok Hospodinův, tomu, jenž vystupuje na posvátné návrší a pálí kadidlo svým bohům.
#48:36 Proto mi bude kvůli Moábovi hlučet srdce jako píšťaly, mé srdce bude hlučet jako píšťaly pro muže Kír-cheresu, protože všechno, co zanechal, zahyne.
#48:37 Na každé hlavě bude lysina, každá brada bude přistřižena, na všech rukou budou smuteční zářezy a na bedrech žíněné suknice.
#48:38 Na všech moábských střechách a na všech jeho prostranstvích se budou konat smuteční obřady, protože jsem roztříštil Moába jako nádobu, která se nelíbí, je výrok Hospodinův.
#48:39 Jak ztroskotal! Kvilte! Jak se Moáb zahanbeně obrátil zády! Moáb bude předmětem posměchu a zděšení všem, kteří jsou kolem něho.“
#48:40 Neboť toto praví Hospodin: „Hle, zhoubce se vrhne dolů jako orel a roztáhne křídla nad Moábem,
#48:41 dobyta budou města i hrady zdolány, srdce moábských bohatýrů v onen den bude jako srdce ženy sevřené bolestí.
#48:42 Moáb bude vyhlazen, už nebude lidem, protože se vyvyšoval nad Hospodina.
#48:43 Postrach, propast, past na tebe, obyvateli Moábu, je výrok Hospodinův.
#48:44 Kdo uteče před postrachem, padne do propasti, a kdo vyleze z propasti, do pasti se lapí, protože na něj, na Moába uvedu rok potrestání, je výrok Hospodinův.
#48:45 Ve stínu Chešbónu stanou vysílení utečenci, avšak z Chešbónu vyšlehne oheň a z prostředku Síchonu plamen a pozře skráně Moába, temeno hřmotícího lidu.
#48:46 Moábe, běda tobě, Kemóšův lid zhyne. Tvoji synové budou odvedeni do zajetí, do zajetí půjdou i tvé dcery.
#48:47 Ale v posledních dnech úděl Moábův změním, je výrok Hospodinův.“ - Až potud soud nad Moábem. 
#49:1 O Amónovcích. Toto praví Hospodin: „Což Izrael nemá syny? Což nemá dědice? Proč tedy Milkóm obsadil Gáda a jeho lid se usadil v jeho městech?
#49:2 Proto hle, přicházejí dny, je výrok Hospodinův, v nichž dám zaznít válečnému křiku proti Rabě Amónovců. Bude z ní zpustošený pahorek, její vesnice budou vypáleny a Izrael obsadí, co patřilo jejím dědicům, praví Hospodin.
#49:3 Kvil, Chešbóne, vždyť je Aj vyhuben, křičte, dcery Raby, přepásejte se žíněnou suknicí, naříkejte! Pobíhejte po zdech, vždyť Milkóm půjde s přesídlenci, spolu se svými kněžími a velmoži.
#49:4 Proč se chlubíš dolinami? Že z tvé doliny prýští voda, dcero odpadlice? Spoléháš na své poklady a říkáš: ‚Kdo by na mne přišel?‘
#49:5 Hle, přivedu na tebe postrach, je výrok Panovníka, Hospodina zástupů, ze všech stran kolem tebe. Budete rozehnáni každý zvlášť. A nebude nikoho, kdo by shromáždil vyplašené.
#49:6 Posléze však úděl Amónovců změním, je výrok Hospodinův.“
#49:7 O Edómu. Toto praví Hospodin zástupů: „Což už není v Témanu žádná moudrost? Ztratili rozumní rozvahu? Jejich moudrost je zmařena?
#49:8 Utečte, obraťte se, usídlete se v dolinách, obyvatelé Dedánu. Přivedu na něj bědy Ezauovy v čas, když jej ztrestám.
#49:9 Jestliže na tebe přijdou sběrači hroznů, nezanechají paběrky, přijdou-li zloději v noci, zničí, co se dá.
#49:10 Neboť já jsem odhalil Ezaua, odkryl jsem jeho skrýše, nebude se moci skrývat. Jeho potomstvo bude vyhubeno, jeho bratři i jeho sousedé; nebude nikoho mít.
#49:11 Zanech své sirotky, já je zachovám při životě, ať ve mne doufají tvé vdovy.“
#49:12 Toto praví Hospodin: „Hle, ti, jimž nebylo určeno pít kalich, ti jej musejí pít, a ty bys zůstal bez trestu? Nezůstaneš bez trestu, určitě budeš pít.
#49:13 Vždyť jsem přísahal sám při sobě, je výrok Hospodinův, že Bosra bude budit úděs, stane se potupou, troskami, zlořečením a všechna její města budou navěky v troskách.“
#49:14 Uslyšel jsem zprávu od Hospodina, že mezi pronárody byl poslán vyslanec. „Shromážděte se, přijďte proti němu, vzhůru do boje!
#49:15 Hle, mezi pronárody tě činím maličkým, opovrženým mezi lidmi.
#49:16 Zavedla tě hrůza, kterou šíříš, opovážlivost tvého srdce. Bydlíš v skalních rozsedlinách, zmocnil ses vysokého pahorku. I když sis založil hnízdo vysoko jak orel, strhnu tě odtud, je výrok Hospodinův.“
#49:17 Edóm bude vzbuzovat úděs. Každý kolemjdoucí posměšně zasykne úžasem nad všemi jeho ranami.
#49:18 „Jako když Bůh podvrátil Sodomu a Gomoru a jejich sousední města, praví Hospodin, už tam nebude nikdo sídlit a člověk tam nebude pobývat ani jako host.
#49:19 Hle, vystupuje jako lev z jordánské houštiny na stále zelené nivy, ale v okamžiku ho z nich zaženu a ustanovím nad nimi toho, jenž je vyvolený. Vždyť kdo je jako já? Kdo mě může předvolat k přelíčení? A kdo je ten pastýř, který přede mnou obstojí?“
#49:20 Proto poslyšte Hospodinovo rozhodnutí, jak rozhodl o Edómu, i co zamýšlí proti obyvatelům Témanu: „Nejchatrnější ze stáda je vyvlečou, a jejich niva se nad nimi zhrozí.
#49:21 Hlukem jejich pádu se zachvěje země, jejich úpění bude slyšet až u moře Rákosového.
#49:22 Hle, zhoubce vzlétne a vrhne se dolů jako orel, roztáhne křídla nad Bosrou. A srdce edómských bohatýrů bude v onen den jako srdce ženy sevřené bolestí.“
#49:23 O Damašku. „Stydí se Chamát i Arpád, neboť slyší zlou zprávu; zmítají se hrůzou jako moře, nemohou se uklidnit.
#49:24 Damašek klesá, obrací se na útěk, zmocnilo se ho zděšení, soužení a útrapy ho zachvátily jako rodičku.
#49:25 Jakže? Nebyl snad opuštěn Jeruzalém, město mé chlouby, osada, která mě obveselovala?
#49:26 Proto jeho nejlepší padnou na jeho prostranstvích a všichni jeho bojovníci v onen den zajdou, je výrok Hospodina zástupů.
#49:27 Na hradbách Damašku zanítím oheň a ten pozře paláce Ben-hadadovy.“
#49:28 O Kédaru a královstí Chasóru, která porazil Nebúkadnesar, král babylónský. Toto praví Hospodin: „Povstaňte, táhněte na Kédar, vyhubte syny východu!
#49:29 Vezmou jim stany i brav, jejich stanové houně i všechno nářadí, odvlekou jim i jejich velbloudy a budou na ně volat: ‚Kolkolem děs!‘
#49:30 Utečte, rozprchněte se, usídlete se v dolinách, obyvatelé Chasóru, je výrok Hospodinův, neboť Nebúkadnesar, král babylónský, se proti vám rozhodl a má s vámi své úmysly.
#49:31 Vstaňte, dejte se na pochod proti pronárodu poklidnému, který přebývá v bezpečí, je výrok Hospodinův, nemají vrata ani závory, bydlí osaměle.
#49:32 Jejich velbloudi jim budou uloupeni, množství jejich stád bude ukořistěno. Rozptýlím je do všech větrů, ty s ostříhanými skráněmi, a přivedu na ně bědy ze všech protilehlých stran, je výrok Hospodinův.
#49:33 Chasór se stane domovem šakalů, pustinou navěky. Už tam nebude pobývat ani jako host.“
#49:34 Slovo Hospodinovo, které se stalo k proroku Jeremjášovi o Élamu na začátku kralování Sidkijáše, krále judského:
#49:35 Toto praví Hospodin zástupů: „Hle, zlomím luk Élamu, prvotinu jejich bohatýrské síly.
#49:36 Přivedu na Élam čtyři větry ze čtyř končin nebes a rozptýlím Élamce do všech těch větrů, takže nebude pronároda, kam by élamští vyhnanci nepřišli.
#49:37 Předěsím Élam před jejich nepřáteli, před těmi, kteří jim ukládají o život. Uvedu na ně zlo, svůj planoucí hněv, je výrok Hospodinův. Pošlu za nimi meč a skoncuji s nimi.
#49:38 Pak postavím v Élamu svůj trůn a zničím odtud krále i velmože, je výrok Hospodinův.
#49:39 Ale v posledních dnech úděl Élamu změním, je výrok Hospodinův.“ 
#50:1 Slovo, které promluvil Hospodin o Babylónu, o zemi kaldejské, skrze proroka Jeremjáše:
#50:2 „Oznamte to mezi pronárody a rozhlaste, zvedněte korouhev, rozhlašujte, nic neskrývejte, řekněte: ‚Babylón je dobyt, Bél je zostuzen, Marduk se děsí.‘ Jeho modlářské stvůry jsou zostuzeny, děsí se jeho hnusné modly!“
#50:3 Přitáhne na něj pronárod ze severu, ten jeho zemi zpustoší, nikdo v ní nebude bydlet, ani člověk ani dobytek, všichni odejdou jako psanci.
#50:4 „V oněch dnech a v onen čas, je výrok Hospodinův, přijdou Izraelci spolu s Judejci, budou přicházet s pláčem a hledat Hospodina, svého Boha.
#50:5 Budou se ptát na cestu na Sijón, tam se obrátí a řeknou: ‚Pojďte!‘ A přidruží se k Hospodinu smlouvou věčnou, která nebude zapomenuta.
#50:6 Můj lid je bloudícím stádem. Jeho pastýři jej nechali bloudit, obraceli jej k horám; chodil od hory k pahorku, na místo, kde odpočíval, zapomněl.
#50:7 Kdokoli jej našli, požírali jej. Jejich protivníci říkali: ‚Nebudeme vinni, protože hřešili proti Hospodinu, proti nivě spravedlnosti a naději svých otců, Hospodinu.‘“
#50:8 „Prchněte ze středu Babylóna, vyjděte z kaldejské země. Buďte jako kozlové před stádem.
#50:9 Neboť hle, já vzbudím a přivedu proti Babylónu shromáždění velikých pronárodů ze severní země; seřadí se proti němu, odtud bude dobyt. Jejich šípy budou jako šípy bohatýra, který přivádí na sirobu, nevrátí se s prázdnou.
#50:10 Kaldejsko se stane kořistí, všichni, kteří z něho budou kořistit, se nasytí, je výrok Hospodinův.
#50:11 Tak vy se tedy radujete a jásáte, kdo pleníte mé dědictví! Bujně poskakujete jako jalovice při mlácení a řehtáte jako hřebci!
#50:12 Ale vaše matka se bude velice stydět, rdít se bude ta, která vás porodila; hle, bude jako poslední z pronárodů, poušt, suchopár a pustina.
#50:13 Kvůli Hospodinovu rozezlení nebude obývána, celá bude zpustošena. Každý, kdo přejde kolem Babylóna, posměšně zasykne úžasem nad všemi jeho ranami.
#50:14 Seřaďte se kolem proti Babylónu, všichni, kdo napínáte luk, střílejte na něj, nešetřete šípy, vždyť zhřešil proti Hospodinu!
#50:15 Spusťte proti němu ze všech stran válečný pokřik! Už se vzdává, jeho pilíře padly, byly strženy jeho hradby. Toto je Hospodinova pomsta. Vykonejte nad ním pomstu! Učiňte mu to, co činil on.
#50:16 Vytněte z Babylóna rozsévače i toho, kdo se chápe srpu v čas žně! Před hubícím mečem se každý obrátí ke svému lidu, každý uteče do své země.
#50:17 Izrael je zaplašená ovce, zahnali jej lvi. Nejdřív jej požíral král asyrský a teď nakonec jeho kosti hryže Nebúkadnesar, král babylónský.“
#50:18 Proto Hospodin zástupů, Bůh Izraele, praví toto: „Hle, potrestám babylónského krále a jeho zemi, jako jsem potrestal krále asyrského.
#50:19 A přivedu Izraele zpět na jeho nivu a bude se pást na Karmelu i v Bášanu a sytit se v Efrajimském pohoří a v Gileádu.
#50:20 V oněch dnech a v onen čas, je výrok Hospodinův, budou hledat Izraelovu nepravost, ale žádná nebude, a Judův hřích, ale nebude nalezen, neboť odpustím těm, které zanechám.“
#50:21 „Proti zemi meratajimské! Táhni proti ní i na obyvatele Pekódu! Znič, zahlaď jako klaté jejich potomstvo, je výrok Hospodinův, vykonej všechno, co jsem ti přikázal!
#50:22 Hlas boje zní zemí, veliká zkáza.
#50:23 Jak je rozsekáno a roztříštěno kladivo, které doléhalo na celou zemi, jak zpustošen je Babylón mezi pronárody!
#50:24 Políčil jsem na tebe a byl jsi lapen, Babylóne, a nevzal jsi to na vědomí. Byl jsi nalezen i chycen, protože ses chtěl potýkat s Hospodinem.
#50:25 Hospodin otevřel svou pokladnici, vynesl nástroje svého hrozného hněvu. Vždyť to je dílo Panovníka, Hospodina zástupů, v zemi Kaldejců.
#50:26 Přitáhněte na něj od končin země, otevřete jeho obilnice, navršte vše na hromady, zničte to jako klaté, ať mu nezůstane ani pozůstatek lidu.
#50:27 Zničte všechny jeho býčky, ať sejdou na porážku! Běda jim! Přišel jejich den, čas, kdy je ztrestám.“
#50:28 Slyš! Utečenci a ti, kdo vyvázli ze země babylónské, přicházejí zvěstovat na Sijón pomstu Hospodina, našeho Boha, pomstu za jeho chrám.
#50:29 „Vyzvěte střelce proti Babylónu, všechny, kdo napínají luk. Ať se utáboří kolem něho, ať není vyváznutí.Odplaťte mu podle jeho skutků! Podle toho, co dělal, udělejte jemu! Vždyť se opovážlivě choval vůči Hospodinu, vůči Svatému Izraele.“
#50:30 „Jeho nejlepší padnou na jeho prostranstvích a všichni jeho bojovníci v onen den zajdou, je výrok Hospodinův.“
#50:31 „Chystám se na tebe, opovážlivče! je výrok Panovníka, Hospodina zástupů. Už přišel tvůj den, čas, kdy tě ztrestám.
#50:32 Klopýtne opovážlivec, padne a nebude nikoho, kdo by ho pozvedl. V jeho městech zanítím oheň a ten pozře celé jeho okolí.“
#50:33 Toto praví Hospodin zástupů: „Izraelci jsou utiskováni a Judejci s nimi. Všichni, kdo je odváděli do zajetí, je pevně uchopili a odmítli je propustit.
#50:34 Jejich Vykupitel je však silný, jeho jméno je Hospodin zástupů. Pevně povede jejich spor, přinese zemi pokoj a obyvatele Babylóna pokoje zbaví.“
#50:35 „Meč proti Kaldejcům! je výrok Hospodinův. I na obyvatele Babylóna a na jeho velmože i jeho mudrce.
#50:36 Meč na tlachaly, ukáží se jako pošetilci, meč na jeho bohatýry, vyděsí se.
#50:37 Meč na jeho koně i na jeho vozbu a na všechen přimíšený lid, který je uprostřed něho; budou jako ženy. Meč na jeho poklady, budou uloupeny.
#50:38 Sucho na jeho vody, vyschnou. Vždyť je to země model, kvůli příšerám třeští.
#50:39 Proto tam bude sídlit divá sběř a hyeny, usídlí se v ní pštrosi. Už nikdy nebude obývána a nebude obydlena od pokolení do pokolení.
#50:40 Jako když Bůh podvrátil Sodomu a Gomoru a jejich sousední města, je výrok Hospodinův, už tam nebude nikdo sídlit a člověk tam nebude pobývat ani jako host.
#50:41 Hle, přijde lid ze severu, veliký pronárod, a četní králové se vypraví z nejodlehlejších koutů země.
#50:42 Chopí se luku a oštěpu, budou krutí a nebudou znát slitování. Jejich hlas bude hučet jako moře, přijedou na koních, seřazeni jako muži k boji proti tobě, dcero babylónská.
#50:43 Babylónský král uslyší o nich pověst, jeho ruce ochabnou a zmocní se ho úzkost jako bolest rodičky.
#50:44 Hle, vystupuje jako lev z jordánské houštiny na stále zelené nivy, ale v okamžiku je z nich zaženu a ustanovím nad nimi toho, jenž je vyvolený. Vždyť kdo je jako já? Kdo mě může předvolat k přelíčení? A kdo je ten pastýř, který přede mnou obstojí?
#50:45 Proto poslyšte Hospodinovo rozhodnutí, jak rozhodl o Babylóně, i co zamýšlí proti zemi Kaldejců: Nejchatrnější ze stáda je vyvlečou a jejich niva se nad nimi zhrozí.
#50:46 Voláním: ‚Byl dobyt Babylón!‘ se bude chvět země a mezi pronárody bude slyšet úpění.“ 
#51:1 Toto praví Hospodin: „Hle vzbudím proti Babylónu zkázonosný vítr, proti obyvatelům onoho ‚srdce mých protivníků‘.
#51:2 Pošlu na Babylón cizáky a ti jej rozptýlí, vylidní jeho zemi, budou proti němu ze všech stran v jeho zlý den.
#51:3 Tomu, kdo mocně napíná luk a vypíná se ve svém pancíři, pravím: S jeho jinochy nemějte soucit, všechen jeho zástup zasvěťte záhubě.
#51:4 V kaldejské zemi budou padat skolení a probodení na ulicích Babylóna.
#51:5 Izrael a Juda nebudou však opuštěni jako vdova od svého Boha, od Hospodina zástupů, přestože se jejich země tolik provinila proti Svatému Izraele.
#51:6 Utečte z Babylóna, zachraňte se každý, ať pro jeho nepravost nezajdete, neboť toto je čas Hospodinovy pomsty, on mu odplatí za to, co spáchal.
#51:7 Zlatým kalichem byl Babylón v Hospodinově ruce, opájel všechny země; pronárody pily jeho víno, proto pronárody třeštily.
#51:8 Avšak Babylón náhle padne a roztříští se. Kvilte pro něho, na jeho bolest vezměte balzám, snad bude vyléčen.
#51:9 Léčili jsme Babylón, ale k vyléčení není. Opusťte jej, pojďme každý do své země! Soud nad ním se dotýká nebes, až do oblak sahá.“
#51:10 Hospodin nechal vzejít naší spravedlnost. Pojďte, budeme na Sijónu vyprávět o díle Hospodina, svého Boha.
#51:11 Naostřete šípy, chystejte štíty! Hospodin vzbudí ducha médských králů, protože hodlá uvrhnout Babylón do zkázy. A Hospodinova pomsta bude pomstou za jeho chrám.
#51:12 Vztyčte na babylónských hradbách korouhev, zesilte stráže, postavte strážné, připravte zálohy! K čemu se Hospodin odhodlal, to vykoná, to, co promluvil proti obyvatelům Babylóna.
#51:13 Dcero babylónská, jež přebýváš u hojných vod a oplýváš poklady, přišel tvůj konec, míra tvé zištnosti je dovršena.
#51:14 Hospodin zástupů přísahá při sobě samém: „Naplním tě lidmi jako žravými kobylkami a budou nad tebou zpívat a výskat.“
#51:15 On svou silou učinil zemi, svou moudrostí upevnil svět, svým rozumem napjal nebesa.
#51:16 Když vydá hlas, shlukují se na nebi vody, přivádí mlhu od končin země, déšť provází blesky, ze svých zásobnic vyvádí vítr.
#51:17 Každý člověk je tupec, neví-li, že se každý zlatník pro své modly dočká hanby. Vždyť jeho lité modly jsou klam, ducha v nich není.
#51:18 Jsou přelud, podvodný výtvor. V čase, kdy budu trestat, zhynou.
#51:19 Díl Jákobův není jako oni, vždyť všechno vytvořil on. Izrael je dědičným kmenem toho, jehož jméno je Hospodin zástupů.
#51:20 „Tys byl mé kladivo, má válečná výzbroj, tebou jsem rozrážel pronárody, tebou jsem vrhal do zkázy království.
#51:21 Tebou jsem rozrazil koně i jeho jezdce, tebou jsem rozrazil vůz i jeho osádku.
#51:22 Tebou jsem rozrazil muže i ženu, tebou jsem rozrazil starce i mladíka, tebou jsem rozrazil jinocha i pannu.
#51:23 Tebou jsem rozrazil pastýře i jeho stádo, tebou jsem rozrazil oráče i jeho spřežení, tebou jsem rozrazil místodržitele i zemské správce.“
#51:24 Odplatím před vašimi zraky Babylónu a všem obyvatelům Kaldejska za všechno to zlo, jež spáchali na Sijónu, je výrok Hospodinův.
#51:25 „Chystám se na tebe, horo zkázy, je výrok Hospodinův, horo zkázy pro celou zemi! Napřáhnu na tebe svou ruku, svalím tě ze skal, změním tě v horu - spáleniště.
#51:26 Nevezmou z tebe úhelný kámen ani kámen do základů, budeš věčně zpustošeným místem, je výrok Hospodinův.
#51:27 Vztyčte v zemi korouhev, zatrubte mezi pronárody na polnici, posvěťte proti němu k boji pronárody, svolejte proti němu království Ararat, Miní a Aškenaz. Ustanovte proti němu úředníky, vyveďte koňstvo jako roje kobylek.
#51:28 Posvěťte proti němu k boji pronárody, krále médské, místodržitele a všechny zemské správce i všechny země jimž vládne.
#51:29 Země se bude třást a svíjet bolestí, neboť Hospodin povstal proti Babylónu a zamýšlí proměnit babylónskou zemi ve zpustošený kraj bez obyvatele.
#51:30 Babylónští bohatýři ustanou v boji a usadí se na nepřístupných vrcholcích skal, jejich bohatýrská síla vyschne, budou jako ženy. Jejich příbytky budou vypáleny, závory Babylóna budou rozraženy.
#51:31 Běžec se bude potkávat s běžcem, posel se bude potkávat s poslem, aby babylónskému králi podali zprávu, že okraj jeho města je dobyt,
#51:32 že byly zabrány brody, že i to rákosí spálili ohněm a bojovníci že jsou naplněni hrůzou.“
#51:33 Toto praví Hospodin zástupů, Bůh Izraele: „Dcera babylónská je jak humno v čase, kdy se dusá. Ještě maličko, a nastane jí čas žně.
#51:34 ‚Stravuje mě, ve zmatek mě uvedl Nebúkadnesar, král babylónský. Odložil mě, nádobu vyprázdněnou, chtěl mě spolknout jako drak, naplnil si břicho mými rozkošemi, teď mne odhodil.
#51:35 Násilí spáchané na mně a mém těle padni na Babylón,‘ praví ta, jež sídlí na Sijónu. ‚Moje krev na kaldejské obyvatele,‘ praví Jeruzalém.“
#51:36 Proto praví Hospodin toto: „Ujmu se tvé pře, vykonám za tebe pomstu, jeho moře vysuším, zasáhnu suchem jeho prameny.
#51:37 Babylón bude hromadou kamení, domovem šakalů, bude k úděsu a posměchu, bez obyvatele.
#51:38 Společně budou řvát jak mladí lvi, skučet jako lví mláďata.
#51:39 Až se rozpálí, uspořádám jim hostinu, opojím je tak, že se rozjaří, ale pak usnou spánkem věčným a neprocitnou, je výrok Hospodinův.
#51:40 Povedu je na na porážku jako jehňata, jako berany s kozly.“
#51:41 „Jak je lapen Šéšak, zajata chlouba celé země! Jak je zpustošen Babylón mezi pronárody!
#51:42 Nad Babylón vystoupí moře, bude přikryt jeho hučícím vlnobitím.
#51:43 Jeho města budou zpustošena, budou zemí bezvodou a pustou, zemí, v níž se nikdo neusadí, jíž člověk neprojde.
#51:44 Ztrestám Béla v Babylóně, vyrvu mu z úst, co zhltl, nebudou již k němu proudit pronárody, hradby Babylóna padnou.
#51:45 Vyjděte z něho, můj lide, zachraňte se každý před Hospodinovým planoucím hněvem.“
#51:46 Nezoufejte, nebojte se zprávy, která se v zemi proslýchá; v tomto roce přijde jedna zpráva a v dalším roce další zpráva: Násilí vládne v zemi a vládce bude proti vládci.
#51:47 A tak, hle, přicházejí dny, kdy ztrestám modly Babylóna. Celá jeho země se bude stydět a všichni v něm padnou skoleni.
#51:48 Nad Babylónem budou plesat nebesa i země se vším, co je v nich, neboť od severu přijdou na něj zhoubci, je výrok Hospodinův.
#51:49 Také na Babylón dojde. Jako padali skolení v Izraeli, tak i v Babylóně budou padat skolení po celé zemi.
#51:50 Vy, kteří jste vyvázli před mečem, jděte, nestůjte, až budete daleko, vzpomeňte na Hospodina a mějte na srdci Jeruzalém.
#51:51 Stydíme se, vždyť jsme museli vyslechnout tupení. Hanba pokryla naše tváře, protože do svatyně Hospodinova domu vstoupili cizáci.
#51:52 „Právě proto, je výrok Hospodinův, přicházejí dny, kdy ztrestám jeho modly a po celé jeho zemi budou naříkat skolení.
#51:53 I kdyby Babylón až k nebi vystoupil a vyhnal mocné opevnění až k výšinám, přijdou na něj ode mne zhoubci, je výrok Hospodinův.“
#51:54 Úpěnlivé volání z Babylóna! Zkáza veliká v kaldejské zemi!
#51:55 Hospodin hubí Babylón, vyhladí z něho velký hřmot; i kdyby jeho vlnobití hlučelo jako mnohé vody, jejich hřmot pomine.
#51:56 Neboť přišel na něj, na Babylón, zhoubce, lapeni budou jeho bohatýři, jejich luky se polámou, neboť Bohem odplaty je Hospodin, vrchovatě jim odplatí.
#51:57 „Opojím jeho velmože, mudrce, místodržitele, zemské správce a bohatýry a usnou věčným spánkem a neprocitnou,“ je výrok Krále, jehož jméno je Hospodin zástupů.
#51:58 Toto praví Hospodin zástupů: „Babylónské široké hradby budou strženy do základů, jeho vysoké brány budou vypáleny ohněm. Lidé se namáhají pro nic za nic, národy se lopotí a pozře to oheň.“
#51:59 Slovo, jež přikázal prorok Jeremjáš Serajášovi, synu Nerijáše, syna Machsejášova, který se vypravil do Babylóna se Sidkijášem, králem judským, ve čtvrtém roce jeho kralování. Serajáš byl velitelem ležení při odpočinku.
#51:60 Všechno zlo, jež mělo postihnout Babylón, zapsal Jeremjáš do jedné knihy; všechna ta slova jsou napsána proti Babylónu.
#51:61 Jeremjáš řekl Serajášovi: „Až přijdeš do Babylóna, až jej spatříš, budeš všechna tato slova předčítat.
#51:62 Řekneš: ‚Hospodine, ty jsi prohlásil o tomto místě, že je vyhladíš, takže v něm nebude obyvatele, vyhladíš lidi i dobytek a stane se místem věčně zpustošeným.‘
#51:63 Až skončíš předčítání z této knihy, přivaž k ní kámen a hoď ji doprostřed Eufratu.
#51:64 Řekni: ‚Tak bude potopen Babylón a nepovstane ze zla, které já na něj uvedu.‘ Všichni zemdlejí.“ Zde končí slova Jeremjášova. 
#52:1 Sidkijášovi bylo jedenadvacet let, když začal kralovat, a kraloval v Jeruzalémě jedenáct let. Jeho matka se jmenovala Chamútal; byla to dcera Jirmejášova z Libny.
#52:2 Dopouštěl se toho, co je zlé v Hospodinových očích, zcela tak, jak se toho dopouštěl Jójakím.
#52:3 Bylo to proto, že Hospodin svým hněvem stíhal Jeruzalém i Judu, až je od své tváře i zavrhl. Sidkijáš se vzbouřil proti babylónskému králi.
#52:4 V devátém roce Sidkijášova kralování, v desátém měsíci, desátého dne toho měsíce, přitáhl Nebúkadnesar, král babylónský, s celým svým vojskem proti Jeruzalému, položili se proti němu a zbudovali proti němu dokola obléhací val.
#52:5 Město bylo obleženo do jedenáctého roku vlády krále Sidkijáše.
#52:6 Ve čtvrtém měsíci, devátého dne toho měsíce, když už tvrdě doléhal na město hlad a lid země neměl co jíst,
#52:7 byly hradby města prolomeny. Všichni bojovníci uprchli. V noci vyšli z města branou mezi hradbami u královské zahrady. Kolem celého města byli Kaldejci. Uprchlíci se dali cestou k pustině.
#52:8 Kaldejské vojsko krále pronásledovalo a dostihlo Sidkijáše na Jerišských pustinách. Celé jeho vojsko se od něho rozprchlo.
#52:9 Krále chytili a přivedli jej k babylónskému králi do Ribly v zemi chamátské, kde nad ním Nebúkadnesar vynesl rozsudek.
#52:10 Babylónský král dal popravit Sidkijášovy syny před jeho očima; i všechny judské velitele dal v Rible popravit.
#52:11 Sidkijáše babylónský král oslepil a spoutal ho bronzovými řetězy a odvedl ho do Babylóna. Vsadil ho do žaláře, kde byl až do dne své smrti.
#52:12 Desátého dne pátého měsíce v devatenáctém roce vlády Nebúkadnesara, krále babylónského, přitáhl do Jeruzaléma Nebúzaradán, velitel tělesné stráže, který stával před babylónským králem.
#52:13 Zapálil Hospodinův dům i dům královský a všechny domy v Jeruzalémě; všechny význačné domy vypálil.
#52:14 Celé kaldejské vojsko, které bylo s velitelem tělesné stráže, zbořilo celé hradby kolem Jeruzaléma.
#52:15 Část chudiny a zbytek lidu, který zůstal v městě, a přeběhlíky, kteří přeběhli k babylónskému králi, ten zbývající houf, Nebúzaradán, velitel tělesné stráže přestěhoval.
#52:16 Část chudiny země ponechal Nebúzaradán, velitel tělesné stráže, jako vinaře a rolníky.
#52:17 Bronzové sloupy, které byly u Hospodinova domu, podstavce i bronzové moře, které bylo v Hospodinově domě, Kaldejci rozbili a všechen bronz z nich odvezli do Babylóna.
#52:18 Pobrali i hrnce, lopatky, kleště na knoty, kropenky a pánvičky a všechno bronzové náčiní potřebné k bohoslužbě.
#52:19 Také misky, kadidelnice, kropenky, hrnce, svícny, pánvičky i obětní misky, jak zlaté, tak stříbrné, velitel tělesné stráže pobral.
#52:20 Bronzu ze všech těch předmětů, ze dvou sloupů, jednoho moře, dvanácti bronzových býků a podstavců, které dal zhotovit král Šalomoun pro Hospodinův dům, bylo tolik, že se nedal ani zvážit.
#52:21 Co se sloupů týká, jeden sloup byl osmnáct loket vysoký a bylo lze jej obepnout šňůrou dvanáct loket dlouhou; silný byl na čtyři prsty a byl dutý.
#52:22 Na něm byla bronzová hlavice. Jedna hlavice byla pět loket vysoká. Kolem hlavice bylo mřížování a granátová jablka. To vše bylo z bronzu. Právě tak vypadal druhý sloup včetně granátových jablek.
#52:23 Granátových jablek bylo po stranách devadesát šest, celkem bylo na mřížování dokola sto granátových jablek.
#52:24 Velitel tělesné stráže vzal Serajáše, hlavního kněze, i Sefanjáše, druhého kněze, a tři strážce prahu.
#52:25 Z města vzal jednoho dvořana, který dohlížel na bojovníky, a sedm mužů z těch, kdo směli patřit na královu tvář, kteří byli v městě, i písaře, velitele vojska, jenž povolával lid země do služby, a šedesát mužů z lidu země, kteří byli v městě.
#52:26 Velitel tělesné stráže Nebúzaradám je vzal a dovedl k babylónskému králi do Ribly.
#52:27 Babylónský král je dal v Rible v zemi chamátské popravit. Tak byl Juda přestěhován ze své země.
#52:28 Počet lidu, který dal přestěhovat Nebúkadnesar v sedmém roce své vlády, byl tento: tři tisíce dvacet tři Judejci.
#52:29 V osmnáctém roce vlády Nebúkadnesarovy byly z Jeruzaléma přestěhovány osm set třicet dvě duše.
#52:30 Ve dvacátém třetím roce Nebúkadnesarovy vlády přestěhoval Nebúzaradán, velitel tělesné stráže, sedm set čtyřicet pět Judejců. Celkem to bylo čtyři tisíce šest set duší.
#52:31 V třicátém sedmém roce po přestěhování Jójakína, krále judského, dvacátého pátého dne dvanáctého měsíce, udělil babylónský král Evíl-merodak, v tom roce, kdy začal kralovat, milost Jójakínovi, králi judskému, a propustil ho z vězení.
#52:32 Mluvil s ním vlídně a jeho křeslo dal postavit výše než křesla králů, kteří byli u něho v Babylóně.
#52:33 Změnil také jeho vězeňský šat a on pak po všechny dny svého života jídal každodenně před ním chléb.
#52:34 Vše, co potřeboval, bylo mu každodenně babylónským králem poskytováno, den co den, až do dne jeho smrti, po všechny dny jeho života.  

\book{Lamentations}{Lam}
#1:1 Jak samotno sedí město, které mělo tolik lidu! Je jako vdova, ač bylo mocné mezi pronárody. Bylo kněžnou mezi krajinami, a je podrobeno nuceným pracím.
#1:2 Za noci pláče a pláče, po líci slzy jí kanou, ze všech jejích milovníků není nikdo, kdo by ji potěšil. Všichni její druhové se k ní zachovali věrolomně, stali se jejími nepřáteli.
#1:3 Juda odešel do vyhnanství ponížen a nesmírně zotročen. Usadil se mezi pronárody, odpočinutí nenachází. Všichni jeho pronásledovatelé ho dostihli v jeho úzkostech.
#1:4 Cesty na Sijón truchlí, nikdo nepřichází ke slavnosti. Všechny jeho brány jsou zpustošené, jeho kněží vzdychají, jeho panny jsou zarmoucené, hořko je sijónské dceři.
#1:5 Její protivníci nabyli vrchu, její nepřátelé si žijí v klidu, neboť Hospodin ji zarmoutil pro množství jejích nevěrností. Její pacholátka odešla před tváří protivníka do zajetí.
#1:6 Odešla od sijónské dcery veškerá její důstojnost. Její velmožové jsou jako jeleni, když nenacházejí pastvu: táhnou vysíleni před tváří pronásledovatele.
#1:7 Jeruzalém se rozpomíná ve dnech svého ponížení a zmateného toulání na všechno, co míval za žádoucí ode dnů dávnověkých. Když jeho lid padl do rukou protivníka, nebyl nikdo, kdo by mu pomohl. Protivníci to viděli a posmívali se jeho zániku.
#1:8 Těžce zhřešila jeruzalémská dcera, proto se stala nečistou. Všichni, kdo si jí vážili, ji mají za bezectnou, neboť vidí její nahotu; a ona vzdychá a odvrací se.
#1:9 Má na lemu roucha nečistotu, nepamatovala na svůj konec. Předivně sešla a není nikdo, kdo by ji potěšil. „Pohleď, Hospodine, na mé ponížení, jak se nepřítel vypíná.“
#1:10 Protivník vztáhl ruku na všechno, co měla za žádoucí. Ba, vidí pronárody vcházet do své svatyně, ač jsi o nich přikázal: „Nevejdou do tvého shromáždění.“
#1:11 Veškeren její lid vzdychá a žebrá o chléb. Za pokrm dávají všechno, co měli za žádoucí, jen aby se udrželi při životě. „Pohleď, Hospodine, popatř, jak jsem zbavena cti.“
#1:12 „Je vám to lhostejné, vy všichni, kteří jdete kolem? Popatřte a hleďte, je-li jaká bolest jako bolest moje, ta, která mi byla způsobena, jíž mě zarmoutil Hospodin v den svého planoucího hněvu.
#1:13 Z výšiny seslal do mých kostí oheň a pošlapal je. Před nohy mi rozprostřel síť, obrátil mě nazpět. Učinil mě místem zpustošeným, nečistým po všechny dny.
#1:14 Neustále mě tlačí moje nevěrnosti, jež jeho ruka spletla ve jho, dostaly se mi na šíji; podlomil mou sílu, Panovník mě vydal do rukou těch, před nimiž neobstojím.
#1:15 Panovník pohrdl všemi udatnými v mém středu, svolal proti mně slavnostní shromáždění, aby rozdrtil mé junáky. Panovník pošlapal v lisu panenskou dceru judskou.
#1:16 Propukám nad tím v pláč, slzy se mi proudem řinou z očí, vzdálil se ten, kdo by mě potěšil, kdo by mě udržel při životě. Moji synové úděsem trnou, neboť nepřítel ukázal svou moc.“
#1:17 Sijón rozpíná své ruce, není nikdo, kdo by jej potěšil. Hospodin vydal Jákobovi příkaz jeho protivníkům vůkol něho. Jeruzalém je mezi nimi jako nečistý.
#1:18 „Hospodin je spravedlivý, já jsem vzdorovala jeho ústům. Slyšte tedy, všichni národové, hleďte na mou bolest. Mé panny a moji junáci odešli do zajetí.
#1:19 Volala jsem své milence, oni mě však oklamali. Moji kněží a moji starší hynuli v městě, když si vyžebrávali pokrm, aby se udrželi při životě.
#1:20 Pohleď, Hospodine, jak se soužím, v mém nitru to bouří, srdce se mi svírá v hrudi, neboť jsem zarputile vzdorovala. Venku sirobu přinášel meč a v domě řádila smrt.
#1:21 Slyšeli mě, jak vzdychám, nebyl nikdo, kdo by mě potěšil, všichni moji nepřátelé uslyšeli o mém neštěstí a veselili se, že jsi to způsobil ty. Přiveď den, který jsi vyhlásil, a budou na tom jako já.
#1:22 Všechno zlo, jež způsobili, ať vstoupí před tvou tvář a nalož s nimi, jako jsi naložil se mnou za všechny mé nevěrnosti. Hluboce vzdychám a mé srdce je choré.“ 
#2:1 Jak zahalil oblakem ve svém hněvu Panovník sijónskou dceru! Srazil z nebe na zem okrasu Izraele. Nebyl pamětliv podnože svých nohou v den svého hněvu.
#2:2 Panovník pohltil bez soucitu všechny Jákobovy nivy, rozbořil ve své prchlivosti pevnosti dcery judské, srovnal se zemí, znesvětil království a jeho velmože.
#2:3 V planoucím hněvu srazil každý roh Izraele, svou pravici stáhl nazpět před tváří nepřítele, vzplanul v Jákobovi jako plápolající oheň, sžírající všechno vůkol.
#2:4 Napjal svůj luk jako nepřítel, pozvedl pravici jako protivník a hubil všechny, kdo byli žádoucí oku; ve stanu sijónské dcery vylil svoje rozhořčení jako oheň.
#2:5 Panovník byl jako nepřítel, pohltil Izraele; pohltil všechny její paláce, její pevnosti zničil, rozmnožil v dceři judské smutek a zármutek.
#2:6 Násilně strhl své zahradě oplocení, své slavnostní shromáždění zničil. Hospodin přivedl na Sijónu v zapomnění slavnost i den odpočinku, zavrhl ve svém hrozném hněvu krále i kněze.
#2:7 Panovník zanevřel na svůj oltář, svou svatyni zrušil, hradby jejích paláců vydal do rukou nepřítele. V domě Hospodinově bylo hluku jako v den slavnostního shromáždění.
#2:8 Hospodin se rozhodl zničit hradby sijónké dcery. Natáhl měřící šňůru, už neodvrátí svou ruku od ničení. Tak zarmoutil val i hradbu, rázem je po nich veta.
#2:9 Do země se ponořily její brány, její závory zničil a roztříštil, její král a velmožové jsou mezi pronárody a zákona není; ani její proroci nemívají vidění od Hospodina.
#2:10 Usedli na zem a ztichli starcové sijónské dcery, házeli si na hlavu prach, oděli se do žíněných suknic. Jeruzalémské panny svěsily hlavu k zemi.
#2:11 Zrak mi pro slzy slábne, v mém nitru to bouří, játra mi vyhřezla na zem pro těžkou ránu dcery mého lidu, neboť zkomírá pachole a kojenec na prostranstvích města.
#2:12 Svých matek se ptají: „Kde je obilí a víno?“ když jako smrtelně raněný zkomírají na prostranstvích města a na klíně svých matek vypouštějí duši.
#2:13 Jaké svědectví o tobě vydám, čemu tě připodobním, jeruzalémská dcero? K čemu tě přirovnám, čím tě potěším, panno, dcero sijónská? Tvá těžká rána je veliká jak moře, kdo tě uzdraví?
#2:14 Tvoji proroci ti zvěstovali šalebná a bláhová vidění, neodhalovali tvoje nepravosti, aby změnili tvůj úděl. Vidění, která ti zvěstovali, byla šalba a svody.
#2:15 Tleskají nad tebou rukama všichni, kdo jdou kolem, syknou a potřesou hlavou nad jeruzalémskou dcerou. „Toto je město, o němž se říkalo: Místo dokonalé krásy k potěše celé země?“
#2:16 Rozevírají na tebe ústa všichni tvoji nepřátelé, syknou, vycení zuby a řeknou: „Zhltli jsme ji. Ano, toto je den, na nějž jsme čekali, našli jsme, viděli jsme.“
#2:17 Hospodin provedl svůj záměr, splnil, co řekl, to, co přikázal již za dnů dávnověkých. Bořil bez soucitu, dopustil, aby se nepřítel nad tebou radoval, vyvýšil roh tvých protivníků.
#2:18 Jejich srdce úpí k Panovníku; hradbo sijónské dcery, prolévej slzy potokem ve dne i v noci, neochabuj v pláči, nedopřej svým očím klidu.
#2:19 Povstaň a běduj za noci na počátku hlídek, vylévej své srdce jako vodu před tváří Panovníka. Zvedej k němu své dlaně za život svých pacholátek, zkomírajících hladem na každém nároží.
#2:20 Pohleď, Hospodine, popatř, s kým jsi kdy tak naložil! Což mají ženy jíst svůj plod, pacholátka, jež hýčkají? Což v Panovníkově svatyni má být vražděn kněz a prorok?
#2:21 Na zemi po ulicích leží mladík i stařec, mé panny a moji junáci padli mečem. Zahubil jsi je v den svého hněvu, pobíjel jsi bez soucitu.
#2:22 Voláš jako ke dni slavnostního shromáždění z okolí ty, jichž se děsím, takže v den Hospodinova hněvu nebude nikdo, kdo by vyvázl a přežil. Těm, jež jsem hýčkala a vychovala, připravil můj nepřítel konec. 
#3:1 Já jsem muž, jenž zakusil ponížení pod holí jeho prchlivosti.
#3:2 Hnal mě a odvedl do temnoty beze světla.
#3:3 Ano, obrací znovu a znovu svou ruku proti mně každého dne.
#3:4 Vetchým učinil mé tělo i kůži, roztříštil mé kosti.
#3:5 Obstavěl a obklíčil mě jedem a útrapami.
#3:6 Do temnot mě vsadil jako od věků mrtvé.
#3:7 Postavil kolem mě zeď, že nemohu vyjít, obtížil mě bronzovým řetězem.
#3:8 Jakkoli úpím a o pomoc volám, umlčuje mou modlitbu.
#3:9 Zazdil mé cesty kvádry, mé stezky rozvrátil.
#3:10 Stal se pro mne medvědem číhajícím, lvem ukrytým v skrýších.
#3:11 Zmátl mé cesty, rozdrásal mě, způsobil, že úděsem trnu.
#3:12 Napjal svůj luk a učinil mě terčem svých šípů.
#3:13 Do mých ledví vystřílel obsah svého toulce.
#3:14 Jsem pro smích všemu svému lidu, předmětem jejich popěvků každodenně.
#3:15 Nasytil mě hořkostmi, napojil pelyňkem.
#3:16 Nechává mě o štěrk si drtit zuby, přitiskl mě do popela.
#3:17 Vytrhl mne z pokojného života, zapomněl jsem na všechno dobré.
#3:18 Proto jsem řekl: „Je veta po mně i po mém čekání na Hospodina.“
#3:19 Rozpomeň se na mé ponížení a zmatené toulání, na pelyněk, na jed.
#3:20 Má duše se rozpomíná, rozpomíná a hroutí se ve mně.
#3:21 Beru si však k srdci a s důvěrou očekávám
#3:22 Hospodinovo milosrdenství, jež nepomíjí, jeho slitování, jež nekončí.
#3:23 Obnovuje se každého rána, tvá věrnost je neskonalá.
#3:24 „Můj podíl je Hospodin,“ praví má duše, proto na něj čekám.
#3:25 Dobrý je Hospodin k těm, kdo v něho naději složí, k duši, jež se na jeho vůli dotazuje.
#3:26 Je dobré, když člověk potichu čeká na spásu od Hospodina.
#3:27 Dobré je muži, jestliže nosil jho už ve svém mládí.
#3:28 Ať usedne osamocen a ztichne, když je na něho vložil.
#3:29 Ať položí do prachu svá ústa, snad ještě naděje zbývá.
#3:30 Ať nastaví líce tomu, kdo ho bije, a potupou se sytí.
#3:31 Neboť Panovník navěky nezavrhne.
#3:32 Zarmoutí-li, slituje se pro své velké milosrdenství.
#3:33 Z rozmaru totiž lidské syny nepokoří ani nezarmoutí.
#3:34 Když jsou nohama deptáni všichni vězňové v zemi,
#3:35 když se převrací právo muže před tváří Nejvyššího,
#3:36 když se křivdí člověku v jeho při, což to Panovník nevidí?
#3:37 Kdo řekne a stane se, když Panovník nepřikázal?
#3:38 Nevychází z úst Nejvyššího zlé i dobré?
#3:39 Na co si může člověk naříkat, pokud žije? Ať si muž naříká na své hříchy.
#3:40 Zkoumejme a zpytujme své cesty, vraťme se zpět k Hospodinu.
#3:41 Pozvedněme s dlaněmi i srdce k Bohu na nebesích.
#3:42 My jsme byli nevěrní a vzpurní a ty jsi neodpustil.
#3:43 Zastřel ses hněvem a stíhal jsi nás, hubil jsi bez soucitu.
#3:44 Obestřel ses oblakem, aby modlitba k tobě nepronikla.
#3:45 Učinil jsi nás odporným smetím uprostřed národů.
#3:46 Rozevírají na nás ústa všichni naši nepřátelé.
#3:47 Naším údělem se staly postrach a propast, zmar a těžká rána.
#3:48 Slzy se mi proudem řinou z očí nad těžkou ranou dcery mého lidu.
#3:49 Oči mi slzí bez ustání, bez ochabnutí,
#3:50 dokud nepohlédne, dokud se nepodívá Hospodin z nebe.
#3:51 Mé oko vyčerpává mou duši pláčem kvůli dcerám mého města.
#3:52 Lovili mě jako lovci ptáče, bez důvodu, moji nepřátelé.
#3:53 Umlčeli můj život v jámě a zaházeli mě kamením.
#3:54 Až nad hlavu mě zatopily vody; řekl jsem si: „Jsem ztracen.“
#3:55 Vzýval jsem tvé jméno, Hospodine, z nejhlubší jámy.
#3:56 Slyšel jsi můj hlas, nezakrývej si ucho, dopřej mi úlevu, když o pomoc volám.
#3:57 Byl jsi mi blízko v den, kdy jsem tě vzýval, řekl jsi: „Neboj se!“
#3:58 Panovníku, ujal ses mých sporů, vykoupil jsi můj život.
#3:59 Viděl jsi, Hospodine, jak mi křivdí, dopomoz mi k právu.
#3:60 Viděl jsi všechnu jejich pomstychtivost, všechny jejich záměry vůči mně.
#3:61 Slyšel jsi, Hospodine, jak mě tupí, všechny jejich záměry proti mně,
#3:62 řeči mých odpůrců i o čem uvažují proti mně každého dne.
#3:63 Popatř: ať sedí nebo stojí, posměšně si o mně popěvují.
#3:64 Podle zásluhy jim odplať, Hospodine, podle skutků jejich rukou.
#3:65 Ponech jim zavilé srdce, to bude tvá kletba na ně.
#3:66 V hněvu je pronásleduj, dokud je nevyhladíš zpod nebes, Hospodine. 
#4:1 Jak zčernalo zlato, změnil se jasný zlatý třpyt! Svaté kameny leží rozmetány po nárožích všech ulic.
#4:2 Vzácní synové Sijónu, cenění nad ryzí zlato, jak jsou pokládáni za hliněné džbány, vyrobené rukou hrnčíře!
#4:3 I šakalí matky podají prs, kojí svá mláďata, dcera mého lidu je však krutá jako pštrosové na poušti.
#4:4 Kojenci se žízní lepí jazyk k patru, pacholátka prosí o chléb, a nikdo jim nenaláme.
#4:5 Ti, kdo jídali lahůdky, budí úděs na ulicích. Ti, kdo byli chováni v purpuru, válejí se v hnoji.
#4:6 Větší byla nepravost dcery mého lidu nežli hřích Sodomy, která byla podvrácena v okamžení, aniž ji zasáhla ruka.
#4:7 Její zasvěcenci byli čistší než sníh, bělejší než mléko, brunátnější nad korály, jejich žíly byly jako safír.
#4:8 Avšak nyní jejich postava potemněla víc než saze, na ulicích je nepoznají, na kostech se jim svraštila kůže, vyschla, až byla jak dřevo.
#4:9 Lépe jsou na tom skolení mečem nežli skolení hladem, ti, kteří vykrváceli probodeni, než ti, kdo zemřeli pro neúrodu pole.
#4:10 Ženy, které bývají tak milosrdné, vařili vlastníma rukama své děti a pojídaly je při těžké ráně dcery mého lidu.
#4:11 Hospodin dovršil své rozhořčení, vylil svůj planoucí hněv, zanítil na Sijónu oheň, aby jej pozřel do základů.
#4:12 Nevěřili králové země ani všichni obyvatelé světa, že vejde protivník a nepřítel do jeruzalémských bran.
#4:13 To pro hříchy jeho proroků, pro nepravosti kněží, kteří prolévali uprostřed něho krev spravedlivých.
#4:14 Potáceli se po ulicích slepí, potřísněni krví, nikdo se nesměl dotknout jejich oděvu.
#4:15 „Odstupte! Nečistý!“ volali na ně, „Odstupte! Odstupte! Nedotýkejte se!“ Když se odpotáceli, říkalo se mezi pronárody: „Už tu nikdy nebudou hosty.“
#4:16 Hospodin sám je rozdělil, už nikdy na ně nepohlédne. Nebrali ohled na kněze, nad starci se neslitovali.
#4:17 Naše oči se vyčerpávaly vyhlížením pomoci, byl to však přelud; na své hlídce jsme vyhlíželi pronárod neschopný zachránit.
#4:18 Sledovali nás na každém kroku, abychom nechodili po svých prostranstvích, náš konec se přiblížil, naše dny se naplnili, nastal náš konec.
#4:19 Naši pronásledovatelé byli rychlejší než nebeští orli, zahnali nás do hor, číhali na nás v poušti.
#4:20 Dech našeho chřípí, Hospodinův pomazaný, byl lapen do jejich jámy. A o něm jsme říkávali: „V jeho stínu budeme žít mezi pronárody.“
#4:21 Jenom se vesel a raduj, dcero edómská, bydlící v zemi Úsu! I na tebe přijde kalich, opojíš se a budeš obnažena.
#4:22 Tvá nepravost však, dcero sijónská, skončí, Bůh tě už nikdy nenechá odvést do vyhnanství, ale tebe, dcero edómská, ztrestá za tvou nepravost a odhalí tvé hříchy. 
#5:1 Rozpomeň se, Hospodine, co se nám stalo, popatř a pohleď na naši potupu!
#5:2 Naše dědictví připadlo cizákům, naše domy cizincům.
#5:3 Stali jsme se sirotky, nemáme otce, naše matky jsou jako vdovy.
#5:4 Vlastní vodu pijeme za stříbro, své dřevo musíme platit.
#5:5 Jsme pronásledováni, visí nám na šíji, vynakládáme námahu a nedopřejí nám klidu.
#5:6 Egyptu podáváme ruku, též Asýrii, abychom se nasytili chlebem.
#5:7 Naši otcové zhřešili, už nejsou, a my neseme jejich nepravosti.
#5:8 Otroci se stali našimi vládci a nikdo nás z jejich rukou nevytrhne.
#5:9 S nasazením života přinášíme chléb, ohrožováni mečem z pouště.
#5:10 Kůže nám žhne jako pec od úporného hladu.
#5:11 Na Sijónu ponižují ženy, v judských městech panny.
#5:12 Jejich rukou jsou věšeni velmožové, tváře starců nejsou ctěny.
#5:13 Junáci nosí mlýnek, mladíci klopýtají pod dřívím.
#5:14 Starci přestali zasedat v bráně a junáci zpívat písničky.
#5:15 Přestalo veselí našeho srdce, v truchlení se proměnil náš tanec.
#5:16 Spadla koruna z naší hlavy, běda nám, neboť jsme hřešili.
#5:17 Nad tím zemdlelo naše srdce, nad oním se nám zatmělo v očích,
#5:18 nad horou Sijónem, že je tak zpustošená; honí se po ní šakalové.
#5:19 Ty, Hospodine, budeš trůnit věčně, tvůj trůn přetrvá všechna pokolení.
#5:20 Proč na nás pořád zapomínáš, opustil jsi nás až do nejdelších časů?
#5:21 Obrať nás, Hospodine, k sobě a my se navrátíme, obnov naše dny jak za dnů dávnověkých.
#5:22 Nebo jsi nás úplně zavrhl? Rozlítil ses na nás převelice.  

\book{Ezekiel}{Ezek}
#1:1 Třicátého roku ve čtvrtém měsíci, pátého dne toho měsíce, když jsem byl mezi přesídlenci u průplavu Kebaru, otevřela se nebesa a měl jsem různá vidění od Boha.
#1:2 Pátého dne toho měsíce, byl to pátý rok přestěhování krále Jójakína,
#1:3 událo se slovo Hospodinovo ke knězi Ezechielovi, synu Buzího, v kaldejské zemi u průplavu Kebaru. Tam na něm spočinula Hospodinova ruka.
#1:4 Viděl jsem, jak se přihnal bouřlivý vítr od severu, veliké mračno a šlehající oheň; okolo něho byla zář a uprostřed ohně jakýsi třpyt oslnivého vzácného kovu.
#1:5 Uprostřed něho bylo cosi podobného čtyřem bytostem, které se vzhledem podobaly člověku.
#1:6 Každá z nich měla čtyři tváře a každá čtyři křídla.
#1:7 Nohy měly rovné, ale chodidla byla jako chodidla býčka, jiskřila jako vyleštěný bronz.
#1:8 Ruce měly lidské, a to pod křídly na čtyřech stranách; měly po čtyřech tvářích a křídlech.
#1:9 Svými křídly se přimykaly jedna k druhé. Při chůzi se neotáčely, každá se pohybovala přímo vpřed.
#1:10 Jejich tváře se podobaly tváři lidské, zprava měly všechny čtyři tváře lví a zleva měly všechny čtyři tváře býčí a všechny čtyři měly také tváře orlí.
#1:11 Takové byly jejich tváře; jejich křídla byla rozpjata vzhůru. Každá se přimykala dvěma křídly k druhé a dvěma si přikrývaly těla.
#1:12 Každá se pohybovala přímo vpřed. Chodily podle toho, kam je vedl duch; při chůzi se neotáčely.
#1:13 Svým vzhledem se ty bytosti podobaly hořícímu řeřavému uhlí. Vypadaly jako pochodně a oheň procházel mezi bytostmi a zářil, totiž z toho ohně šlehal blesk.
#1:14 A ty bytosti pobíhaly sem a tam, takže vypadaly jako blýskavice.
#1:15 Když jsem na ty bytosti hleděl, hle, na zemi u těch bytostí, před každou z těch čtyř, bylo po jednom kole.
#1:16 Vzhled a vybavení kol bylo toto: třpytila se jako chrysolit a všechna čtyři se sobě podobala; jejich vzhled a vybavení se jevilo tak, jako by bylo kolo uvnitř kola.
#1:17 Když jela, mohla jet na všechny čtyři strany a při jízdě se nezatáčela.
#1:18 Jejich loukotě byly mohutné a šla z nich bázeň; ta čtyři kola měla loukotě kolem dokola plné očí.
#1:19 Když se bytosti pohybovaly, pohybovala se s nimi i kola, a když se bytosti vznášely nad zemí, vznášela se i kola.
#1:20 Kam je vedl duch, tam šly, tam vedl duch i kola; vznášela se spolu s nimi, neboť duch bytostí byl v kolech.
#1:21 Když ony šly, jela, když se ony zastavily, stála, a když se ony vznášely nad zemí, vznášela se kola spolu s nimi, neboť duch bytostí byl v kolech.
#1:22 Nad hlavami bytostí bylo cosi podobného klenbě jako třpyt oslňujícího křišťálu rozpjatého nahoře nad jejich hlavami.
#1:23 Pod tou klenbou pak byla jejich křídla vztažena jedno k druhému; každá bytost měla dvě křídla, jimiž se přikrývala, a každá měla dvě křídla, jimiž přikrývala své tělo.
#1:24 I slyšel jsem zvuk jejich křídel jako zvuk mnohých vod, jako hlas Všemocného, když se pohybovaly; bylo to jako hřmění, jako zvuk válečného tábora; když stály, svěsily křídla.
#1:25 Zvuk se šířil svrchu nad klenbou, kterou měly nad hlavou; když stály, svěsily křídla.
#1:26 A nahoře nad klenbou, kterou měly nad hlavou, bylo cosi, co vypadalo jako člověk.
#1:27 I viděl jsem, jako by se třpytil oslnivý vzácný kov; vypadalo to jako oheň uvnitř i okolo; směrem od toho, co vypadalo jako bedra, nahoru a směrem od toho dolů jsem viděl, co vypadalo jako oheň šířící záři dokola.
#1:28 Vypadalo to jako duha, která bývá na mračnu za deštivého dne, tak vypadala ta záře dokola; byl to vzhled a podoba Hospodinovy slávy. Když jsem to spatřil, padl jsem na tvář a slyšel jsem hlas mluvícího. 
#2:1 Řekl mi: „Lidský synu, postav se na nohy; budu s tebou mluvit.“
#2:2 Jakmile ke mně promluvil, vstoupil do mě duch a postavil mě na nohy. I slyšel jsem ho k sobě mluvit.
#2:3 Řekl mi: „Lidský synu, posílám tě k izraelským synům, k těm bouřícím se pronárodům, které se vzbouřily proti mně. Oni i jejich otcové mi byli nevěrni až do tohoto dne.
#2:4 I synové jsou zatvrzelí a mají tvrdé srdce. Posílám tě k nim a řekneš jim: ‚Toto praví Panovník Hospodin‘,
#2:5 ať poslechnou nebo ne, jsou dům vzpurný. Poznají však, že byl uprostřed nich prorok.
#2:6 Ty, lidský synu, se jich neboj, neboj se ani jejich slov, když jsou vůči tobě zarputilí a jako trní, jako bys bydlel mezi štíry. Neboj se jejich slov a neděs se jich, jsou dům vzpurný.
#2:7 Promluvíš k nim má slova, ať poslechnou nebo ne; jsou to vzpurníci.
#2:8 Ty, lidský synu, slyš, co já k tobě mluvím. Nebuď vzpurný jako ten vzpurný dům. Rozevři ústa a sněz, co ti dávám.“
#2:9 Tu jsem viděl, že je ke mně vztažena ruka, a hle, v ní knižní svitek.
#2:10 Rozvinul jej přede mnou a byl popsán na vnitřní i vnější straně; byly na něm napsány žalozpěvy, lkání a bědování. 
#3:1 Řekl mi: „Lidský synu, sněz, co máš před sebou, sněz tento svitek a jdi, mluv k izraelskému domu.“
#3:2 Otevřel jsem tedy ústa a on mi dal ten svitek sníst.
#3:3 Řekl mi: „Lidský synu, nakrm své břicho a naplň své útroby tímto svitkem, který ti dávám.“ Snědl jsem jej a byl mi v ústech sladký jako med.
#3:4 Řekl mi: „Lidský synu, nyní jdi k izraelskému domu a mluv k nim mými slovy.
#3:5 Nejsi přece posílán k lidu temné mluvy a těžkého jazyka, nýbrž k domu izraelskému,
#3:6 ne k mnohým národům temné mluvy a těžkého jazyka, jejichž řeči bys nerozuměl; kdybych tě poslal k nim, poslechli by tě.
#3:7 Izraelský dům tě však nebude chtít poslechnout, poněvadž nejsou ochotni poslechnout mne; celý izraelský dům má tvrdé čelo a zatvrzelé srdce.
#3:8 Hle, dávám ti tvář právě tak tvrdou, jako je jejich, a čelo právě tak tvrdé, jako je jejich.
#3:9 Dávám ti čelo jako křemen, tvrdší než oblázek. Neboj se jich a neděs se jich, jsou dům vzpurný.“
#3:10 Dále mi řekl: „Lidský synu, srdcem přijmi a ušima slyš všechna má slova, jež k tobě mluvím.
#3:11 Nuže, jdi k přesídlencům, k synům svého lidu. Budeš k nim mluvit a řekneš jim: ‚Toto praví Panovník Hospodin‘, ať poslechnou nebo ne.“
#3:12 Tu mě duch zvedl a uslyšel jsem za sebou mocné dunění: „Požehnána buď Hospodinova sláva vycházející ze svého místa!“
#3:13 a zvuk křídel těch bytostí, která se vzájemně těsně dotýkala, i zvuk těch kol, vznášejících se spolu s nimi, i mocné dunění.
#3:14 Duch mě zvedl a odnesl mě a já jsem šel v duchu roztrpčen a rozhořčen, ale Hospodinova ruka na mně pevně spočívala.
#3:15 Tak jsem přišel do Tel Abíbu k přesídlencům usazeným u průplavu Kebaru, k těm totiž, kteří tam byli usazeni, a seděl jsem tam po sedm dní a vzbuzoval mezi nimi úděs.
#3:16 Když uplynulo sedm dní, stalo se ke mně slovo Hospodinovo:
#3:17 „Lidský synu, ustanovuji tě strážcem izraelského domu. Uslyšíš-li z mých úst slovo, vyřídíš jim mé varování.
#3:18 Řeknu-li o svévolníkovi: ‚Zemřeš‘, a ty bys nepromluvil a svévolníka nevaroval před jeho svévolnou cestou, abys ho přivedl k životu, ten svévolník zemře za svou nepravost, ale za jeho krev budu volat k odpovědnosti tebe.
#3:19 Jestliže budeš svévolníka varovat, ale on se od své svévole a své svévolné cesty neodvrátí, zemře pro svou nepravost, ale ty jsi svou duši vysvobodil.
#3:20 Když se odvrátí spravedlivý od své spravedlnosti a bude se dopouštět bezpráví, položím mu do cesty nástrahu a zemře. Jestliže jsi ho v jeho hříchu nevaroval, zemře a nebude pamatováno na jeho spravedlnost, kterou konal, ale za jeho krev budu volat k odpovědnosti tebe.
#3:21 Jestliže bys však spravedlivého varoval, aby nehřešil, a on přestane hřešit, bude žít, protože se dal varovat, a ty jsi svou duši vysvobodil.“
#3:22 I spočinula tam na mně Hospodinova ruka a řekl mi: „Vstaň a jdi na pláň, budu tam k tobě mluvit.“
#3:23 Vstal jsem tedy a odešel jsem na pláň. A hle, stála tam Hospodinova sláva jako tehdy, když jsem ji viděl u průplavu Kebaru. I padl jsem na tvář.
#3:24 Vstoupil však do mne duch a postavil mě na nohy a on ke mně mluvil. Řekl mi: „Jdi a zavři se uvnitř svého domu.
#3:25 Lidský synu, hle, budeš svázán provazy a spoután jimi, takže nebudeš moci vycházet mezi ně.
#3:26 Způsobím, že ti jazyk přilne k patru a budeš němý. Nebudeš je už kárat; jsou dům vzpurný.
#3:27 Ale až k tobě budu mluvit, otevřu ti ústa a řekneš jim: ‚Toto praví Panovník Hospodin.‘ Kdo chce poslouchat, ať poslouchá, a kdo ne, ať neposlouchá; jsou dům vzpurný.“ 
#4:1 „Ty, lidský synu, slyš. Vezmi si cihlu, polož ji před sebe a vyryj na ni město Jeruzalém.
#4:2 Polož proti němu obležení: Zbuduj proti němu obléhací valy, navrš proti němu násep, rozmísti proti němu vojsko a rozestav proti němu ze všech stran válečné berany.
#4:3 Pak si vezmi železnou pánev a polož ji jako železnou stěnu mezi sebe a město a zaměř na ně svou tvář. Tak se octne v obležení, tím je sevřeš. To bude znamením pro izraelský dům.
#4:4 Polož se pak na levý bok; vložíš na něj nepravost izraelského domu a tolik dnů, kolik budeš na něm ležet, poneseš jejich nepravost.
#4:5 Za léta jejich nepravosti ti ukládám počet dnů, totiž tři sta devadesát dní, v nichž poneseš nepravost izraelského domu.
#4:6 Až skončí tyto dny, lehneš si podruhé, a to na pravý bok, a poneseš nepravost domu judského po čtyřicet dní; ukládám ti za každý rok jeden den.
#4:7 Pak zaměříš na obležený Jeruzalém svou tvář a svou obnaženou paži a budeš proti němu prorokovat.
#4:8 A já tě svážu provazy, takže nebudeš s to obrátit se z boku na bok, dokud neskončíš dny svého obléhání.
#4:9 Vezmi si pšenici, ječmen, boby, čočku, proso a špaldu a nasyp to vše do jedné nádoby; budeš si z toho připravovat pokrm po tolik dní, co budeš ležet na boku, budeš to jíst po tři sta devadesát dní.
#4:10 Pokrm, který budeš jíst, bude vážit dvacet šekelů na den; budeš jej jíst vždy v jistý čas.
#4:11 I vodu budeš pít odměřenou, šestinu hínu; budeš pít vždy v jistý čas.
#4:12 A pokud budeš chtít jíst ječný podpopelný chléb, upečeš si jej před jejich očima na lidských výkalech.“
#4:13 Hospodin řekl: „Právě tak budou Izraelci jíst svůj pokrm nečistý mezi pronárody, kam je zaženu.“
#4:14 Namítl jsem: „Ach, Panovníku Hospodine, já jsem se nikdy neposkvrnil; od mládí až do nynějška jsem nejedl zdechlé či rozsápané zvíře ani jsem nedal do úst závadné maso.“
#4:15 Řekl mi: „Pohleď, dovoluji ti použít kravinců místo lidských výkalů; na nich si připravuj pokrm.“
#4:16 Dále mi řekl: „Lidský synu, zlámu v Jeruzalémě hůl chleba. Budou jíst chléb odvážený a s obavami a vodu budou pít odměřenou a v skleslosti,
#4:17 budou mít nedostatek chleba i vody; jeden jako druhý budou trnout děsem a zahynou pro svou nepravost.“ 
#5:1 „Ty, lidský synu, slyš. Vezmi si ostrý meč; použiješ ho jako holičské břitvy. Oholíš si hlavu i bradu, pak si připravíš váhu a rozvážíš to na díly.
#5:2 Třetinu spálíš na ohništi uprostřed města, že se dovršují dny obležení; potom vezmeš další třetinu a rozsekáš ji okolo něho mečem, a třetinu rozvěješ do větru, neboť s taseným mečem jim budu v zádech.
#5:3 Hrstku toho však vezmeš a zavážeš do cípu svého pláště,
#5:4 ale i z toho ještě něco odebereš, hodíš do ohně a spálíš; odtud vyjde oheň na celý izraelský dům.“
#5:5 Toto praví Panovník Hospodin: „To je Jeruzalém. Postavil jsem jej doprostřed pronárodů a okolo něho jsou země.
#5:6 Vzepřel se však proti mým řádům a nařízením svévolněji než pronárody a země, jež jsou okolo něho; mé řády znevážili a mými nařízeními se neřídili.“
#5:7 Proto praví Panovník Hospodin toto: „Poněvadž hlučíte více než pronárody, které jsou okolo vás, ale mými nařízeními se neřídíte a mé řády nezachováváte, ba ani nejednáte podle zvyklostí pronárodů, jež jsou okolo vás,
#5:8 proto praví Panovník Hospodin toto: Jsem proti tobě, Jeruzaléme, a vykonám uprostřed tebe soudy před očima pronárodů.
#5:9 Pro všechny tvé ohavnosti vykonám v tobě, co jsem ještě nevykonal a už nikdy nic takového nevykonám.
#5:10 Otcové budou jíst uprostřed tebe syny a synové budou jíst své otce. Tak vykonám v tobě soudy a celý pozůstatek tvého lidu rozvěji do všech větrů.
#5:11 Jakože jsem živ, je výrok Panovníka Hospodina, poněvadž jsi poskrvnil mou svatyni všelijakými ohyzdnými a ohavnými modlami, ani já ti nic neslevím a nebude mi tě líto, ani já nebudu znát soucit.
#5:12 Třetina v tobě zemře morem a zajde uprostřed tebe hladem, třetina padne okolo tebe mečem a třetinu rozvěji do všech větrů a s tasemým mečem jim budu v zádech.
#5:13 Tím se dovrší můj hněv, upokojím své rozhořčení proti nim, a tak se potěším. I poznají, že já Hospodin jsem promluvil ve své žárlivosti, že mé rozhořčení proti nim se dovršilo.
#5:14 Obrátím tě v trosky a budeš potupou mezi pronárody, které jsou okolo tebe, před očima každého, kdo půjde kolem.
#5:15 Budeš pronárodům, jež jsou okolo tebe, předmětem tupení a hanobení, trestu a úděsu, až budu nad tebou konat soudy v hněvu, v rozhořčení a rozhořčeným trestáním. Já Hospodin jsem promluvil.
#5:16 Vystřelím na vás zhoubné šípy hladu, které přinesou zkázu, vystřelím je k vaší zkáze, uvedu na vás hlad a zlámu vám hůl chleba.
#5:17 Pošlu na vás hlad a dravou zvěř, a budeš bez dětí, projde tebou mor a krev, uvede na tebe meč. Já Hospodin jsem promluvil.“ 
#6:1 I stalo se ke mně slovo Hospodinovo:
#6:2 „Lidský synu, postav se proti izraelským horám a prorokuj proti nim.
#6:3 Řekni: Hory izraelské, slyšte slovo Panovníka Hospodina! Toto praví Panovník Hospodin horám a pahorkům, řečištím a údolím: Hle, já na vás přivedu meč a zničím vaše posvátná návrší.
#6:4 Vaše oltáře budou zpustošeny, vaše kadidlové oltáříky rozbity. Způsobím, že vaši skolení padnou před vaše hnusné modly.
#6:5 Mrtvá těla izraelských synů pohodím před jejich hnusné modly a vaše kosti rozmetám okolo vašich oltářů.
#6:6 Kdekoli sídlíte, budou města obrácena v trosky a posvátná návrší budou zpustošena, takže vaše oltáře budou ležet v troskách a sutinách, vaše hnusné modly budou roztříštěny a odstraněny, vaše kadidlové oltáříky skáceny a vaše výtvory zahlazeny.
#6:7 Skolení budou padat uprostřed vás, i poznáte, že já jsem Hospodin.
#6:8 Avšak ponechám vám mezi pronárody zbytek, ty, kdo vyváznou před mečem, až budete rozmetáni do různých zemí.
#6:9 Tu se ti z vás, kdo vyváznou, na mě rozpomenou mezi pronárody, kam budou odvlečeni do zajetí, až dopustím, aby bylo rozbito jejich smilné srdce, jež se ode mne odvrátilo, a oči smilně se dívající po jejich hnusných modlách. Tím se zhnusí sami sobě pro zlo, které všemi svými ohavnostmi napáchali.
#6:10 I poznají, že já jsem Hospodin; ne nadarmo jsem prohlásil, že jim způsobím tyto zlé věci.“
#6:11 Toto praví Panovník Hospodin: „Udeř do dlaně, dupni nohou a řekni: ‚Běda izraelskému domu pro všechny zlé ohavnosti; padnou mečem, hladem a morem.
#6:12 Vzdálený zemře morem a blízký padne mečem, kdo zůstane a bude ušetřen, zemře hladem; tím se dovrší mé rozhořčení proti nim.
#6:13 I poznáte, že já jsem Hospodin, až budou skolení ležet mezi jejich hnusnými modlami okolo jejich oltářů na každém vysokém pahorku, na všech vrcholech hor i pod každým zeleným stromem a pod každým košatým posvátným stromem na místě, kde připravovali libovonné oběti všem svým hnusným modlám.
#6:14 Napřáhnu na ně svou ruku a obrátím zemi, kdekoli oni sídlí, ve zpustošený kraj, úděsnější než je Diblatská poušť. I poznají, že já jsem Hospodin.‘“ 
#7:1 I stalo se ke mně slovo Hospodinovo:
#7:2 „Ty, lidský synu, slyš. Toto praví Panovník Hospodin o izraelské zemi: Je konec, přišel konec na všechny čtyři strany země!
#7:3 Tvůj konec už je tady. Dám průchod svému hněvu proti tobě, budu tě soudit podle tvých cest a obrátím na tebe všechny tvé ohavnosti.
#7:4 Nebude mi tě líto a nebudu znát soucit, obrátím tvé cesty proti tobě a tvé ohavnosti vyvstanou uprostřed tebe. I poznáte, že já jsem Hospodin.“
#7:5 Toto praví Panovník Hospodin: „Hle, přichází zlo za zlem.
#7:6 Přišel konec, konec přišel, bude s tebou skoncováno, hle, konec přišel!
#7:7 Naplňuje se, obyvateli země, tvůj úděl, přichází čas, blízký je den zmatku; na horách ustane výskání.
#7:8 Už brzy na tebe vyleji své rozhořčení, tím se dovrší můj hněv proti tobě, budu tě soudit podle tvých cest a obrátím na tebe všechny tvé ohavnosti.
#7:9 Nebude mi tě líto a nebudu znát soucit, obrátím tvé cesty proti tobě a vyvstanou tvé ohavnosti uprostřed tebe. I poznáte, že jsem to já Hospodin, kdo bije.
#7:10 Hle, ten den, hle, přišel! Úděl se naplnil. Rozkvetla hůl, vypučela zpupnost.
#7:11 Násilí se rozrostlo v hůl svévole. Nic nebude z nich, nic nebude z jejich hlučení s shlukování, nebude nad nimi bědováno.
#7:12 Přišel čas, nastal ten den. Kdo kupuje, ať se neraduje, kdo prodává, ať netruchlí, poněvadž přichází rozhorlení na všechen hlučící dav země.
#7:13 Kdo prodává, nevrátí se k tomu, co prodal, i kdyby zůstal naživu. Vidění týkající se všeho jejího hlučícího davu se nezvrátí, nikdo, kdo žije ve své nepravosti, se nevzchopí.
#7:14 Zatroubí se na trubku a všechno bude připraveno, ale do boje nevyjde nikdo, poněvadž přichází mé rozhorlení na všechen její hlučící dav.
#7:15 Venku řádí meč a doma mor a hlad; kdo je na poli, zemře mečem, a kdo v městě, toho pozře hlad a mor.
#7:16 A vyváznou-li někteří z nich a budou na horách, jako holubi v roklích, budou všichni sténat, každý pro vlastní nepravost.
#7:17 Všechny ruce ochabnou a všechna kolena se rozplynou jak voda.
#7:18 Lidé se přepásají žíněnou suknicí, přikryje je zděšení, na všech tvářích bude stud, na všech hlavách lysina.
#7:19 Vyházejí své stříbro na ulice a zlato jim bude nečistotou. V den Hospodinovy prchlivosti je jejich stříbro a zlato nevysvobodí. Nenasytí se a své útroby si nenaplní, poněvadž jejich nepravost je přivede k pádu.
#7:20 Nádherné okrasy Boží užili ve své pýše k zhotovení obrazů svých ohavných, ohyzdných model; proto jsem jim ji změnil v nečistotu.
#7:21 Vydám ji v lup do rukou cizáků a za kořist svévolníků v zemi, a ti ji znesvětí.
#7:22 Odvrátím od nich svou tvář, cizáci znesvětí můj poklad, přijdou na něj lupiči a znesvětí jej.
#7:23 Udělej řetěz, neboť země je plná krve, jež volá po soudu, a město je plné násilí.
#7:24 Proto přivedu nejhorší z pronárodů a ti obsadí jejich domy. Učiním přítrž pýše mocných a znesvěceni budou, kdo je posvěcují.
#7:25 Přichází tíseň, budou hledat pokoj, ale žádný nebude.
#7:26 Valí se neštěstí za neštěstím a přichází zpráva za zprávou. Budou vyhledávat vidění od proroka, ale kněz bude bez zákona a starci bez soudnosti.
#7:27 Král bude truchlit, kníže se obleče v úděs a ruce lidu země ochrnou hrůzou. Naložím s nimi podle jejich cesty a za jejich zvyklosti je budu soudit. I poznají, že já jsem Hospodin.“ 
#8:1 Šestého roku v šestém měsíci, pátého dne toho měsíce, jsem seděl ve svém domě a přede mnou seděli judští starší. Tam na mne dolehla ruka Panovníka Hospodina.
#8:2 Měl jsem vidění. Hle, cosi vzhledem podobného ohni. Od beder dolů to vypadalo jako oheň a od beder nahoru to vypadalo jako záře, jako třpyt oslnivého vzácného kovu.
#8:3 Vztáhl cosi, co se podobalo ruce, uchopil mě za kštici na hlavě a duch mě vynesl mezi zemi a nebe; v Božích viděních mě přivedl do Jeruzaléma ke vchodu vnitřní brány, která je obrácena na sever, v níž je modla žárlivosti, která vyvolává žárlivost Hospodinovu.
#8:4 A hle, byla tam sláva Boha Izraele. Vypadala jako tehdy, když jsem ji viděl na pláni.
#8:5 I řekl mi: „Lidský synu, pozvedni své oči k severu.“ Pozvedl jsem oči k severu, a hle, na sever od brány byl oltář; ta modla žárlivosti byla v průchodu.
#8:6 Řekl mi: „Lidský synu, vidíš, co páchají? Jak veliké ohavnosti zde dům izraelský páchá, abych se vzdálil od své svatyně? A uvidíš ještě větší ohavnosti.“
#8:7 Uvedl mě ke vchodu do nádvoří, kde jsem spatřil jakýsi otvor ve zdi.
#8:8 Řekl mi: „Lidský synu, prokopej se tou zdí!“ Prokopal jsem se tedy zdí, a hle, jakýsi vchod.
#8:9 Řekl mi: „Vejdi a podívej se na ty zlé ohavnosti, které zde páchají.“
#8:10 Vešel jsem a uviděl všelijaká zpodobení plazů a ohyzdných zvířat a všelijaké hnusné modly izraelského domu, vyryté po zdech kolem dokola.
#8:11 A před nimi stojí sedmdesát mužů ze starších domu izraelského, uprostřed nich Jaazanjáš, syn Šáfanův. Každý z nich má v ruce svou kadidelnici a vzhůru stoupá oblak vonného kouře z kadidla.
#8:12 Řekl mi: „Viděl jsi, lidský synu, co páchají starší izraelského domu potmě, každý v pokojíku s obrazy svých model? Říkají: ‚Hospodin nás nevidí, Hospodin zemi opustil‘.“
#8:13 Řekl mi: „A uvidíš ještě větší ohavnosti, které páchají.“
#8:14 A přivedl mě do brány Hospodinova domu, která vede k severu, a hle, sedí tam ženy a oplakávají Tamúza.
#8:15 Řekl mi: „Viděl jsi, lidský synu? A uvidíš ještě větší ohavnosti než tyto.“
#8:16 Uvedl mě do vnitřního nádvoří Hospodinova domu, a hle, při vchodu do Hospodinova chrámu, mezi předsíní a oltářem, bylo asi dvacet pět mužů, zády ke chrámu Hospodinovu a tváří k východu; klaněli se východním směrem slunci.
#8:17 Řekl mi: „Viděl jsi, lidský synu? Domu Judovu nestačí tyto ohavnosti, které zde páchají; navíc naplnili celou zemi násilím. Stále znovu mě urážejí, a hle, jak si k nosu pozvedají ratolest.
#8:18 Také já budu jednat v rozhořčení, nebude mi jich líto a nebudu znát soucit. Pak budou ke mně volat mocným hlasem, ale já je nevyslyším.“ 
#9:1 I zavolal na mne mocným hlasem: „Blíží se ti, kteří budou trestat město; každý se svou zkázonosnou zbraní v ruce.“
#9:2 A hle, šest mužů přichází cestou od Horní brány, obrácené k severu, každý se svou ničivou zbraní v ruce. Jeden z nich je oděn lněným šatem a má na bedrech písařský kalamář. Když přišli, postavili se u bronzového oltáře.
#9:3 Tu se přenesla sláva Boha Izraele z cheruba, na němž spočívala, k prahu domu. Hospodin zavolal muže oděného lněným šatem, který měl na bedrech písařský kalamář,
#9:4 a poručil mu: „Projdi středem města, středem Jeruzaléma, a označ znamením na čele muže, kteří vzdychají a sténají nad všemi ohavnostmi, které se v něm páchají.“
#9:5 A slyšel jsem, jak ostatním poručil: „Procházejte městem za ním a bijte bez lítosti a bez soucitu.
#9:6 Starce, mladíka, pannu, děti i ženy zabíjejte, šiřte zkázu. Nepřistupujte však k nikomu z těch, na nichž je znamení. Začněte od mé svatyně!“ I začali od starších, kteří byli před domem.
#9:7 Nařídil jim: „Poskvrňte dům, naplňte nádvoří skolenými; jděte!“ I vyšli do města a pobíjeli.
#9:8 Když je pobíjeli a já zůstal sám, padl jsem na tvář a úpěl: „Běda, Panovníku Hospodine, chceš uvést zkázu na celý pozůstatek Izraele, že vyléváš na Jeruzalém své rozhořčení?“
#9:9 Řekl mi: „Nepravost domu izraelského i judského je nesmírně veliká. Země je plná prolité krve a město je plné křivd. Vždyť řekli: ‚Hospodin zemi opustil, Hospodin nic nevidí‘.
#9:10 Proto mi jich nebude líto a nebudu znát soucit. Jejich cestu obrátím na jejich hlavu.“
#9:11 A hle, muž oděný lněným šatem, s kalamářem na bedrech, podal zprávu: „Provedl jsem, co jsi mi přikázal.“ 
#10:1 Tu jsem uviděl, hle, na klenbě nad hlavou cherubů bylo cosi jako safírový kámen; ukázalo se nad nimi cosi, co se vzhledem podobalo trůnu.
#10:2 I řekl Hospodin muži oděnému lněným šatem: „Vstup do soukolí pod cherubem, naber si plné hrsti žhavého uhlí z místa mezi cheruby a rozhoď po městě!“ I vstoupil tam před mýma očima.
#10:3 Cherubové stáli po pravé straně domu, když ten muž vstupoval, a vnitřní nádvoří naplňoval oblak.
#10:4 Hospodinova sláva se vznesla od cheruba k prahu domu, takže dům byl naplněn oblakem a nádvoří bylo plné jasu Hospodinovy slávy.
#10:5 A zvuk křídel cherubů bylo slyšet až do vnějšího nádvoří jako hlas Boha Všemocného, když mluví.
#10:6 I přikázal muži oděnému lněným šatem: „Naber oheň ze soukolí, z místa mezi cheruby.“ On vstoupil a stanul u kola.
#10:7 Tu jeden z cherubů vztáhl ruku z místa mezi cheruby k ohni, který byl mezi cheruby, nabral z něho a dal do hrsti muži oděnému lněným šatem. Ten to vzal a vyšel.
#10:8 U cherubů bylo totiž vidět pod křídly cosi podobného lidské ruce.
#10:9 Vedle cherubů jsem viděl čtyři kola; jedno kolo u jednoho cheruba, další kolo u dalšího cheruba. Kola vypadala jako třpyt drahokamu chrysolitu.
#10:10 Svým vzhledem se všechna čtyři sobě podobala, jako by kolo bylo uprostřed kola.
#10:11 Při jízdě se na všechny čtyři strany pohybovala bez zatáčení; jela vždy k místu, kam se obrátila hlava, bez zatáčení.
#10:12 Celé tělo, záda, ruce i křídla cherubů, stejně jako kola, byla dokola plná očí, u všech čtyř kol.
#10:13 Slyšel jsem, že ta kola nazval soukolím.
#10:14 Každá bytost měla čtyři tváře. První byla tvář cheruba, druhá lidská, třetí lví, čtvrtá orlí.
#10:15 A cherubové se vznášeli - to jsou ty bytosti, které jsem viděl u průplavu Kebaru.
#10:16 Když se cherubové pohybovali, pohybovala se s nimi i kola, a když cherubové zvedali křídla a vznesli se nad zemi, kola se nevzdálila, zůstala při nich.
#10:17 Když se zastavili, zastavila se také. Když se vznesli, vznesla se s nimi, neboť v nich byl duch těch bytostí.
#10:18 I vyšla Hospodinova sláva od prahu domu a stanula nad cheruby.
#10:19 Cherubové zvedli křídla a vznesli se před mýma očima ze země. Spolu s nimi se hnula i kola. Sláva stanula u vchodu východní brány Hospodinova domu; sláva Boha Izraele byla nahoře nad nimi.
#10:20 To jsou ty bytosti, které jsem viděl pod trůnem Boha Izraele u průplavu Kebaru. Poznal jsem, že to jsou cherubové.
#10:21 Každý měl čtyři tváře a čtyři křídla a pod křídly ruce podobné lidským.
#10:22 Jejich tváře se svým vzhledem a znaky podobaly těm, které jsem viděl při průplavu Kebaru. Každý se ubíral přímo před sebe. 
#11:1 Tu mě duch zvedl a přenesl k východní bráně Hospodinova domu, obrácené k východu. A hle, u vchodu do brány dvacet pět mužů. Uprostřed nich jsem uviděl Jaazanjáše, syna Azúrova, a Pelatjáše, syna Benajášova, velmože lidu.
#11:2 Hospodin mi řekl: „Lidský synu, to jsou ti muži, kteří vymýšlejí ničemnosti a dávají v tomto městě zlou radu.
#11:3 Říkají: ‚Nestavějte v blízké době domy. Město je hrnec a my jsme maso‘.
#11:4 Proto prorokuj proti nim, prorokuj, lidský synu!“
#11:5 Duch Hospodinův se na mne snesl a Hospodin mi poručil: „Řekni: Toto praví Hospodin: Takto jste mluvili, dome izraelský, a já vím, co vstoupilo do vašeho ducha.
#11:6 Rozmnožili jste počet skolených v tomto městě, naplnili jste skolenými jeho ulice.“
#11:7 Proto praví Panovník Hospodin toto: „Skolení, které jste v něm pohodili, jsou maso a město je hrnec; vás však z jeho středu vyvedu.
#11:8 Bojíte se meče a já na vás meč uvedu, je výrok Panovníka Hospodina.
#11:9 Vyvedu vás z města, ale vydám vás do rukou cizáků. Tak nad vámi vykonám soudy.
#11:10 Padnete mečem na izraelském pomezí. Budu vás soudit, i poznáte, že já jsem Hospodin.
#11:11 Město se vám nestane hrncem a vy v něm nebudete masem. Budu vás soudit na izraelském pomezí.
#11:12 I poznáte, že já jsem Hospodin, jehož nařízeními jste se neřídili a podle jeho řádů jste nejednali, nýbrž jednali jste podle zvyklostí pronárodů, které jsou kolem vás.“
#11:13 I stalo se, když jsem prorokoval, že Pelatjáš, syn Benajášův, zemřel. Padl jsem na tvář a mocným hlasem jsem úpěl: „Běda, Panovníku Hospodine, chceš skoncovat s pozůstatkem Izraele?“
#11:14 I stalo se ke mně slovo Hospodinovo:
#11:15 „Lidský synu, tvým bratřím, mužům z tvého příbuzenstva a celému domu izraelskému říkají obyvatelé Jeruzaléma: ‚Zůstaňte si daleko od Hospodina! Tato země byla dána do vlastnictví nám.‘
#11:16 Proto řekni: Toto praví Panovník Hospodin: Ano, vzdálil jsem je mezi pronárody a rozptýlil po zemích; avšak v zemích, do nichž přišli, jsem se jim stal na ten krátký čas svatyní.
#11:17 Proto jim řekni: Toto praví Panovník Hospodin: Shromáždním vás z pronárodů, posbírám vás ze zemí, do nichž jste byli rozptýleni, a dám vám zemi izraelskou.
#11:18 Přijdou tam a odstraní z ní všechny její ohyzdné a ohavné modly.
#11:19 A dám jim jedno srdce a vložím do jejich nitra nového ducha, odstraním z jejich těla srdce kamenné a dám jim srdce z masa,
#11:20 aby se řídili mými nařízeními, zachovávali moje řády a jednali podle nich. I budou mým lidem a já jim budu Bohem.
#11:21 Ale cestu těch, jejichž srdce se drží ohyzdných a ohavných model, uvalím na jejich hlavu, je výrok Panovníka Hospodina.“
#11:22 Tu zvedli cherubové křídla a kola se vznesla spolu s nimi a sláva Boha Izraele byla nahoře nad nimi.
#11:23 Pak se Hospodinova sláva, která byla nad středem města, vznesla a stanula nad horou na východ od města.
#11:24 A duch mě zvedl a přenesl do Kaldejska k přesídlencům. Stalo se tak ve vidění mocí ducha Božího; a vidění, jež jsem měl, se ode mne vzneslo.
#11:25 I sdělil jsem přesídlencům všechno, co mi Hospodin ukázal. 
#12:1 I stalo se ke mně slovo Hospodinovo:
#12:2 „Lidský synu, bydlíš uprostřed vzpurného domu. Mají oči k vidění, ale nevidí, mají uši k slyšení, ale neslyší. Jsou dům vzpurný.“
#12:3 „Ty, lidský synu, slyš. Připrav si věci k přesídlení a přestěhuj se za denního světla před jejich očima. Přestěhuješ se před jejich očima ze svého místa na jiné místo. Snad nahlédnou, že jsou dům vzpurný.
#12:4 Vyneseš své věci jako věci k přesídlení za denního světla před jejich očima. Sám pak vyjdeš večer před jejich očima jako ti, kdo jsou přesídlováni.
#12:5 Před jejich očima se prokopej zdí a vyjdi.
#12:6 Vezmeš své věci na záda, za soumraku je přeneseš, jim před očima. Zastřeš si tvář, abys neviděl zemi. Učinil jsem tě předzvěstí pro izraelský dům.“
#12:7 Vykonal jsem, co mi bylo přikázáno. Za denního světla jsem vynesl své věci jako věci k přesídlení. Večer jsem se prokopal zdí a za soumraku jsem před jejich očima odnesl ty věci na zádech.
#12:8 Za jitra se ke mně stalo slovo Hospodinovo:
#12:9 „Lidský synu, což se tě nikdo z izraelského domu, domu vzpurného, nezeptal, co to děláš?
#12:10 Řekni jim: Toto praví Panovník Hospodin: Tento výnos se týká knížete v Jeruzalémě a celého domu izraelského, který je v městě.
#12:11 A pokračuj: Já jsem předzvěstí toho, co vás čeká; co jsem dělal já, to udělají jim. Budou přesídleni, půjdou do zajetí.
#12:12 I kníže, který je ve vašem středu, vezme za soumraku své věci na záda a vyjde. Prokopou zeď, aby ho vyvedli. Zastře si tvář, aby nespatřil svýma očima zemi.
#12:13 Rozprostřel jsem na něho síť, bude lapen do mé lovecké sítě. Zavedu ho do Babylóna v kaldejské zemi, ale ani ji neuvidí a zemře tam.
#12:14 Všechny, kdo jsou kolem něho jemu ku pomoci, i všechny jeho voje rozpráším do všech větrů a s taseným mečem jim budu v zádech.
#12:15 I poznají, že já jsem Hospodin, až je rozptýlím mezi pronárody a rozpráším po zemích.
#12:16 Ale určitý počet lidí z nich zachovám před mečem, hladem a morem, aby vyprávěli o všech svých ohavnostech mezi pronárody, mezi něž půjdou. I poznají, že já jsem Hospodin.“
#12:17 I stalo se ke mně slovo Hospodinovo:
#12:18 „Lidský synu, budeš jíst svůj chléb s třesením a pít vodu s chvěním a obavami.
#12:19 A řekneš lidu země: Toto praví Panovník Hospodin o obyvatelích Jeruzaléma v zemi izraelské: S obavami budou jíst svůj chléb a v úděsu budou pít vodu, neboť jejich země bude zpustošena, zbavena své hojnosti pro násilnosti všech svých obyvatel.
#12:20 Obydlená města se obrátí v trosky, země se stane zpustošeným krajem. I poznáte, že já jsem Hospodin.“
#12:21 I stalo se ke mně slovo Hospodinovo:
#12:22 „Lidský synu, copak to máte za pořekadlo o izraelské zemi: ‚Dny uběhnou a každé vidění selže‘?
#12:23 Proto jim řekni: Toto praví Panovník Hospodin: Učiním konec tomuto pořekadlu, už je v Izraeli nebudou říkat. A promluv k nim: Blízko jsou dny, kdy se splní všechna vidění.
#12:24 Už nebude uprostřed izraelského domu žádné šalebné vidění a lichotivá věštba.
#12:25 Když já Hospodin promluvím, mluvím slovo, které se splní bez odkladu. Za vašich dnů, vzpurný dome, mluvím slovo a také je splním, je výrok Panovníka Hospodina.“
#12:26 I stalo se ke mně slovo Hospodinovo:
#12:27 „Lidský synu, hle, izraelský dům říká: ‚Vidění, které on vidí, se splní za mnoho dnů; on prorokuje pro vzdálené časy.‘
#12:28 Proto jim řekni: Toto praví Panovník Hospodin: Už nebude odloženo žádné mé slovo. Promluvím slovo a ono se splní, je výrok Panovníka Hospodina.“ 
#13:1 I stalo se ke mně slovo Hospodinovo:
#13:2 „Lidský synu, prorokuj proti izraelským prorokům, kteří prorokují sami od sebe. Řekni těm, kteří prorokují podle vlastního srdce: Slyšte slovo Hospodinovo!
#13:3 Toto praví Panovník Hospodin: Běda bláznivým proroků, kteří následují svého vlastního ducha, ale pranic neviděli.
#13:4 Jako lišky ve zříceninách jsou tvoji proroci, Izraeli.
#13:5 Nenastoupili jste do trhlin a nezazdívali jste zeď izraelského domu, aby obstál v boji v den Hospodinův.
#13:6 Jejich vidění je šalba a jejich věštba lež. Říkají: ‚Výrok Hospodinův‘, ale Hospodin je neposlal, a přece čekají na potvrzení slova.
#13:7 Copak jste neviděli šalebná vidění a nezvěstovali lživou věštbu? A přece říkáte: ‚Výrok Hospodinův‘, přestože jsem já nepromluvil.“
#13:8 Proto praví Panovník Hospodin toto: „Protože jsou vaše slova šalba a vaše vidění lež, chystám se na vás, je výrok Panovníka Hospodina.
#13:9 Moje ruka je pozdvižena proti prorokům, kteří mají šalebná vidění a věští lež. Nebudou náležet do kruhu rádců mého lidu, nebudou vepsáni do soupisu izraelského domu a nevstoupí na izraelskou půdu. I poznáte, že já jsem Panovník Hospodin.
#13:10 Znovu a znovu svádějí můj lid slovy: ‚Pokoj‘, ač žádný pokoj není. Lid staví chatrnou stěnu, a hle, oni ji nahazují omítkou.
#13:11 Řekni těm, kteří nahazují stěnu omítkou: Padne! Spustí se lijavec, který spláchne všechno. Nechám padat kroupy jako kameny, rozpoutá se bouřlivá vichřice,
#13:12 a hle, zeď padne. Což vám nebylo řečeno: ‚Kde je to, čím jste ji nahodili?‘“
#13:13 Proto praví Panovník Hospodin toto: „Rozhořčen rozpoutám bouřlivou vichřici, rozhněván a rozhořčen spustím lijavec, který všechno spláchne, a kroupy jako kameny; a skoncuji s ní.
#13:14 Zbořím zeď, kterou nahodili omítkou, srazím ji na zem a odkryje se její základ. Padne a vy skončíte uprostřed sutin. I poznáte, že já jsem Hospodin.
#13:15 Tak dovrším své rozhořčení proti zdi i proti těm, kdo ji nahazují omítkou. Řeknu vám: Už není zeď, už nejsou ti, kdo ji nahazovali,
#13:16 totiž proroci izraelští, kteří prorokovali o Jeruzalému, a ti, kteří mívali o něm vidění pokoje, ač žádný pokoj nebyl, je výrok Panovníka Hospodina.
#13:17 „Ty, lidský synu, slyš. Postav se proti dcerám svého lidu, které prorokují podle vlastního srdce, a prorokuj proti nim.
#13:18 Řekni: Toto praví Panovník Hospodin: Běda těm, které šijí kouzelné váčky na každé zápěstí a dělají kukly na hlavu každé velikosti, aby lovily duše. Lovíte duše mého lidu a svou duši chcete zachovat naživu?
#13:19 Znesvěcujete mě před mým lidem pro hrst ječmene a kus chleba; svým lhaním lidu, který poslouchá lži, usmrcujete duše, které neměly zemřít, a při životě chcete zachovat duše, které nemají žít.“
#13:20 Proto praví Panovník Hospodin toto: „Chystám se na vaše kouzelné váčky, do nichž lovíte duše jako ptáky. Strhám je z vašich rukou a duše, které lovíte jako ptáky, pustím.
#13:21 A vaše kukly vám strhnu a svůj lid vám vytrhnu z rukou. Už nebudou úlovkem ve vašich rukou. I poznáte, že já jsem Hospodin.
#13:22 Poněvadž mučíte srdce spravedlivého podvodem, zatímco já jsem ho nechtěl zarmucovat, a posilujete ruce svévolníka, aby se neodvrátil od své zlé cesty a nebyl zachován naživu,
#13:23 proto už nebudete mívat šalebná vidění a věštit věštby, ale vytrhnu svůj lid z vašich rukou. I poznáte, že já jsem Hospodin.“ 
#14:1 Přišli ke mně někteří ze starších izraelských a posadili se přede mnou.
#14:2 I stalo se ke mně slovo Hospodinovo:
#14:3 „Lidský synu, tito mužové nosí v srdci své hnusné modly a kladou před sebe svou nepravost, aby o ni klopýtli. Těm mám dávat odpověď na dotazy?
#14:4 Proto k nim promluv. Řekni jim: Toto praví Panovník Hospodin: Každému z domu izraelského, kdo nosí v srdci své hnusné modly a klade před sebe svou nepravost, aby o ni klopýtl, a jde za prorokem, já Hospodin dám odpověď, jakou si pro své hnusné modly zaslouží.
#14:5 A tak bude izraelský dům polapen svým vlastním srdcem, protože všichni se mi svými hnusnými modlami odcizili.
#14:6 Proto řekni izraelskému domu: Toto praví Panovník Hospodin: Obraťte se a odvraťte se od svých hnusných model, odvraťte svou tvář od všech svých ohavností!
#14:7 Já Hospodin dám odpověď každému z izraelského domu i z bezdomovců, kdo pobývá v Izraeli a odcizil se mi, každému, kdo nosí v srdci své hnusné modly a klade před sebe svou nepravost, aby o ni klopýtl, a jde za prorokem, aby se ho na mne dotazoval.
#14:8 Proti takovému muži obrátím svou tvář a učiním ho výstražným znamením a pořekadlem, vytnu ho ze svého lidu. I poznáte, že já jsem Hospodin.
#14:9 Dá-li se prorok svést a promluví slovo, svedl jsem toho proroka já Hospodin. Napřáhnu na něho ruku a vyhladím ho ze svého izraelského lidu.
#14:10 I ponese svou nepravost; nepravost dotazujícího se bude stejná jako nepravost prorokova.
#14:11 Učiním to, aby už izraelský dům nebloudil, nevzdaloval se mi a neposkvrňoval se všemožnými svými nevěrnostmi. I budou mým lidem a já jim budu Bohem, je výrok Panovníka Hospodina.“
#14:12 I stalo se ke mně slovo Hospodinovo:
#14:13 „Lidský synu, jestliže by země proti mně zhřešila a zronevěřila se a já bych na ni napřáhl svou ruku, zlámal jí hůl chleba a seslal na ni hlad, takže bych z ní vymýtil lidi i dobytek,
#14:14 a byli by v ní tito tři muži: Noe, Daniel a Jób, ti vysvobodí svou spravedlností jen sami sebe, je výrok Panovníka Hospodina.
#14:15 Kdybych přivedl na zemi dravou zvěř, aby ji vylidnila, takže by byla zpustošena a nikdo by jí kvůli zvěři neprocházel,
#14:16 ti tři muži v ní, jakože jsem živ, je výrok Panovníka Hospodina, nevysvobodí ani své syny a dcery; vysvobodí pouze sebe, země však bude zpustošena.
#14:17 Nebo kdybych na tu zemi uvedl meč a rozhodl bych, aby meč prošel zemí, a vymýtil bych z ní lidi i dobytek,
#14:18 ti tři muži v ní, jakože jsem živ, je výrok Panovníka Hospodina, nevysvobodí ani své syny a dcery, vysvobodí pouze sebe.
#14:19 Nebo kdybych na tu zemi poslal mor a vylil na ni své rozhořčení krveprolitím, aby z ní byli vymýceni lidé i dobytek,
#14:20 Noe, Daniel a Jób v ní, jakože jsem živ, je výrok Panovníka Hospodina, nevysvobodí ani syna ani dceru. Svou spravedlností vysvobodí jen sami sebe.“
#14:21 Toto praví Panovník Hospodin: „I když pošlu na Jeruzalém tyto své čtyři přísné soudy: meč, hlad, dravou zvěř a mor, abych z něho vymýtil lidi i dobytek,
#14:22 hle, zbude v něm hrstka synů a dcer, kteří vyváznou; budou vyvedeni a vyjdou k vám. Uvidíte jejich cestu a jejich skutky a najdete útěchu po zlu, které jsem uvedl na Jeruzalém, po tom všem, co jsem na něj uvedl.
#14:23 Oni vás potěší, až uvidíte jejich cestu a jejich skutky, a poznáte, že jsem nedělal zbytečně nic z toho, co jsem v něm vykonal, je výrok Panovníka Hospodina.“ 
#15:1 I stalo se ke mně slovo Hospodinovo:
#15:2 „Lidský synu, čím předčí dřevo vinné révy ostatní druhy dřeva, čím je její odnož mezi lesními stromy?
#15:3 Bere se z něho dřevo ke zhotovení nějakého výrobku? Nebo z něho vezmou dřevo na kolík, na který se zavěšuje různé náčiní?
#15:4 Hle, přikládá se na oheň a ten je pozře; když oba jeho konce pozře oheň a vzplane i jeho prostředek, hodí se k nějakému výrobku?
#15:5 Ani dokud bylo celé, nedal se z něho zhotovit žádný výrobek, což teprv, když konce pozřel oheň a celé vzplane, dá se z něho ještě zhotovit nějaký výrobek?“
#15:6 Proto praví Panovník Hospodin toto: „Jako jsem vydal ohni k pozření dřevo vinné révy mezi lesními stromy, tak jsem vydal obyvatele Jeruzaléma.
#15:7 Obrátil jsem proti nim svou tvář; dostali se z ohně, ale oheň je pozře. I poznáte, že já jsem Hospodin, až se proti nim postavím.
#15:8 Učiním ze země zpustošený kraj za to, že se mi zpronevěřili, je výrok Panovníka Hospodina.“ 
#16:1 I stalo se ke mně slovo Hospodinovo:
#16:2 „Lidský synu, seznam Jeruzalém s jeho ohavnostmi.
#16:3 Řekni: Toto praví Panovník Hospodin dceři jeruzalémské: Původem a rodištěm jsi z kenaanské země, tvůj otec byl Emorejec a tvá matka Chetejka.
#16:4 A tvoje narození? V den tvého narození nebyla odříznuta tvá pupeční šňůra, nebyla jsi umyta vodou, abys byla čistá, nebyla jsi potřena solí ani zavinuta do plének.
#16:5 Žádné oko nad tebou nepojala lítost, aby ti aspoň něco z toho udělali ze soucitu s tebou. V den svého narození jsi byla pohozena v poli, protože jsi vzbuzovala hnus.
#16:6 Tu jsem šel kolem tebe a uviděl jsem tě, jak se třepeš ve vlastní krvi, a řekl jsem ti, když jsi ležela ve vlastní krvi: Žij! Řekl jsem ti, když jsi ležela ve své krvi: Žij!
#16:7 Dal jsem ti růst jako tomu, co raší na poli, vyrostla jsi a jsi velká. Rozvinula ses do plného půvabu, ňadra dostala tvar, vyrostlo ti ochlupení, ale byla jsi docela nahá.
#16:8 Šel jsem kolem tebe, uviděl jsem tě, a hle, byl právě tvůj čas, čas milování. I rozprostřel jsem na tebe svůj plášť, přikryl jsem tvou nahotu. Pak jsem ti přísahal a vešel s tebou v smlouvu, je výrok Panovníka Hospodina, a stala ses mou.
#16:9 Umyl jsem tě vodou, smyl jsem z tebe tvoji krev a potřel jsem tě olejem.
#16:10 Pak jsem tě oblékl do pestrého šatu, obul ti opánky z tachaší kůže, ovinul ti pás z jemného plátna a zahalil tě hedvábím.
#16:11 Ozdobil jsem tě ozdobami, na ruce jsem ti dal náramky a na hrdlo náhrdelník.
#16:12 Navlékl jsem ti do chřípí kroužek a do uší náušnice, na hlavu jsem ti vložil překrásný věnec.
#16:13 Byla jsi ozdobena zlatem a stříbrem, tvé oděvy byly z jemného plátna, hedvábí a pestrých látek, jedla jsi pokrmy z bílé mouky, medu a oleje. Stala ses nesmírně krásnou a dosáhla jsi královských poct.
#16:14 Tvé jméno proniklo mezi pronárody pro tvou krásu; byla dokonalá pro důstojnost, kterou jsem na tebe vložil, je výrok Panovníka Hospodina.
#16:15 Ty jsi však spoléhala na svou krásu a zhanobila jsi své jméno smilstvím. Zahrnovala jsi svým smilstvím každého, kdo šel kolem, ať to byl kdo byl.
#16:16 A vzala jsi část svých rouch a křiklavě sis jimi vyzdobila posvátná návrší. Na nich jsi smilnila, jak tomu nikdy nebylo a nebude.
#16:17 Vzala jsi překrásné předměty z mého zlata a stříbra, které jsem ti dal, udělala sis obrazy mužského pohlaví a smilnila jsi s nimi.
#16:18 Vzala jsi svá pestrá roucha a přikrývala jsi je jimi a kladla jsi před ně můj olej a mé kadidlo.
#16:19 Můj chléb, který jsem ti dal, bílou mouku, olej a med, jimiž jsem tě živil, jsi kladla před ně jako libou vůni. Tak to bylo, je výrok Panovníka Hospodina.
#16:20 Vzala jsi také své syny a dcery, které jsi mi porodila, a jim jsi je obětovala za pokrm, jako by bylo tvého smilstva málo.
#16:21 Zabíjela jsi mé syny v oběť a jim jsi je vydávala tím, že jsi je prováděla ohněm.
#16:22 Při všech svých ohavnostech a smilstvech sis nevzpomněla na dny svého mládí, jak jsi byla docela nahá a jak ses třepala ve své krvi.
#16:23 Po všech tvých zločinech se stalo, běda, běda tobě, je výrok Panovníka Hospodina,
#16:24 že sis vybudovala modlářskou svatyňku a udělala sis vyvýšené místo na kdejakém prostranství.
#16:25 Na každém rozcestí sis vybudovala své vyvýšené místo, a tak jsi zohavila svou krásu. Své nohy jsi roztahovala pro každého, kdo šel kolem, a tak jsi množila své smilstvo.
#16:26 Smilnila jsi s Egypťany, se svými sousedy statného těla; tak jsi množila své smilstvo a urážela mne.
#16:27 Hle, napřáhl jsem na tebe ruku, snížil tvůj příděl a vydal tě rozvášnění těch, kdo tě nenávidí, dcer pelištejských, které se stydí za tvou mrzkou cestu.
#16:28 Smilnila jsi také s Asyřany a neměla jsi nikdy dost. Smilnila jsi s nimi, a přece jsi neměla dost.
#16:29 Rozmnožila jsi své smilstvo po zemi kenaanské i kaldejské a ani tak jsi neměla dost.
#16:30 Jak zchátralo tvé srdce, je výrok Panovníka Hospodina, tím, že to všechno děláš, řemeslo nevázané nevěstky,
#16:31 když sis vybudovala modlářskou svatyňku na kdejakém rozcestí a udělala sis vyvýšené místo na kdejakém prostranství. Ale na rozdíl od nevěstky jsi nedbala o mzdu.
#16:32 Jako cizoložná manželka brala sis cizí místo svého muže.
#16:33 Všem nevěstkám se dává dar, tys však dávala dary všem svým milencům a odměňovala jsi je za to, že k tobě vcházeli z celého okolí a smilnili s tebou.
#16:34 Tím se lišíš ve svém smilstvu od jiných žen, ani to po tobě žádná nevěstka neudělá, že dáváš mzdu a nedává se mzda tobě; to je ten rozdíl.
#16:35 Proto slyš, nevěstko, slovo Hospodinovo!
#16:36 Toto praví Panovník Hospodin: Protože byla rozhazována tvá měď a při tvém smilstvu byla odkrývána tvá nahota před tvými milenci, před všemi tvými hnusnými, ohavnými modlami, a pro krev tvých synů, které jsi jim dala,
#16:37 proto hle, já shromáždím všechny tvé milence, s nimiž ti bylo tak příjemně, všechny, které miluješ, i všechny, které nenávidíš. Shromáždím je u tebe z celého okolí a odkryji před nimi tvou nahotu, takže tě uvidí v celé tvé nahotě.
#16:38 Budu tě soudit podle právních ustanovení o cizoložnicích a vražedkyních a vydám tě krveprolití v rozhořčení a žárlivosti.
#16:39 Vydám tě do jejich rukou; tvoji modlářskou svatyňku zboří a tvá vyvýšená místa strhnou, vysvléknou tě z tvých šatů, seberou tvé překrásné šperky a zanechají tě zcela nahou.
#16:40 Pak na tebe přivedou shromáždění, budou tě kamenovat kameny a sekat svými meči.
#16:41 Tvé domy spálí ohněm a vykonají nad tebou soudy před zraky mnoha žen. Tak učiním přítrž tvému smilství a nebudeš už dávat mzdu.
#16:42 Upokojím své rozhořčení proti tobě a moje žárlivost se od tebe vzdálí, uklidním se, už se nebudu urážet.
#16:43 Protože jsi nepamatovala na dny svého mládí, ale popouzela jsi mě tím vším, teď já uvalím tvou cestu na tvou hlavu, je výrok Panovníka Hospodina, nebudeš páchat své mrzkosti u všelijakých svých ohavných model.
#16:44 Hle, každý, kdo užívá přísloví, užije o tobě tohoto: ‚Jaká matka, taková dcera‘.
#16:45 Jsi dcerou své matky, která si zprotivila vlastního muže i syny, jsi sestrou svých sester, které si zprotivily své muže i své syny. Vaše matka byla Chetejka a váš otec Emorejec.
#16:46 Tvou větší sestrou je Samaří se svými dcerami; ta sedí po tvé levici. Tvou menší sestrou je Sodoma se svými dcerami; ta sedí po tvé pravici.
#16:47 Ale ty jsi nechodila jen jejich cestami, ani jsi nepáchala jen jejich ohavnosti; to ti bylo málo. Na všech svých cestách jsi byla horší než ony.
#16:48 Jakože jsem živ, je výrok Panovníka Hospodina, tvá sestra Sodoma se svými dcerami nenapáchala tolik zla jako ty se svými dcerami.
#16:49 Hle, toto byla nepravost tvé sestry Sodomy; pýcha, sytost chleba a sebejistý klid, který měla i se svými dcerami. Ale ruku utištěného ubožáka neposilovala.
#16:50 Povyšovaly se, páchaly přede mnou ohavnost, a tak jsem je odstranil, jak jsem uznal za vhodné.
#16:51 A Samaří, to nehřešilo ani z polovice jako ty. Tys však rozhojnila své ohavnosti víc než ony, takže jsi ospravedlnila své sestry všemi ohavnostmi, které jsi páchala.
#16:52 Nes tedy svou hanbu, za níž jsi odsuzovala své sestry. Pro tvé hříchy, že jsi jednala ohavněji než ony, jeví se spravedlivějšími než ty. Styď se a nes svou hanbu, vždyť jsi ospravedlnila své sestry.
#16:53 Změním jejich úděl, úděl Sodomy a jejich dcer, úděl Samaří a jeho dcer, a mezi nimi změním i tvůj úděl,
#16:54 ale poneseš svou hanbu a budeš zahanbena za všechno, co jsi páchala; tím budou potěšeny.
#16:55 Tvá sestra Sodoma i se svými dcerami se navrátí k tomu, čím bývaly kdysi, též Samaří se svými dcerami se navrátí k tomu, čím bývaly kdysi, i ty se svými dcerami se navrátíš k tomu, čím jste bývaly kdysi.
#16:56 Což nebylo o tvé sestře Sodomě slyšet z tvých úst v době tvé pýchy,
#16:57 dokud nebyla odhalena tvá špatnost? Teď nastal čas, kdy dcery Aramu a celého jeho okolí tě tupí a okolní dcery pelištejské tebou pohrdají.
#16:58 Svou mrzkost a své ohavnosti poneseš sama, je výrok Hospodinův.“
#16:59 Toto praví Panovník Hospodin: „Budu zacházet s tebou, jako jsi ty zacházela se mnou, když jsi pohrdla přísahou a porušila smlouvu.
#16:60 Avšak rozpomenu se na svou smlouvu s tebou za dnů tvého mládí a ustavím s tebou smlouvu věčnou.
#16:61 Vzpomeneš si na své cesty a budeš zahanbena, až dostaneš své sestry, ty větší i menší než ty. Dám ti je za dcery, ale nebudou účastny tvé smlouvy.
#16:62 Já ustavím svou smlouvu s tebou. I poznáš, že já jsem Hospodin,
#16:63 a budeš si to připomínat a stydět se a už neotevřeš ústa pro svou hanbu, až tě smířím se sebou přes všechno, co jsi páchala, je výrok Panovníka Hospodina.“ 
#17:1 I stalo se ke mně slovo Hospodinovo:
#17:2 „Lidský synu, dej izraelskému domu hádanku a předlož mu podobenství.
#17:3 Řekni: Toto praví Panovník Hospodin: Veliký orel s velkými křídly, dlouhými perutěmi, celý opeřený, pestře zbarvený, přilétl na Libanón a vzal vrcholek cedru.
#17:4 Utrhl vršek jeho koruny a přenesl jej do země kramářů, položil jej v městě obchodníků.
#17:5 Vzal sazenici té země a vsadil ji do orné půdy, vzal ji u hojných vod a zasadil jako vrbu.
#17:6 I vyrašila a stala se z ní bujná vinná réva nízkého vzrůstu; její větvoví se k němu obracelo a její kořeny byly pod ním; sazenice se stala vinnou révou, ta se rozvětvila a vyhnala ratolesti.
#17:7 Byl však ještě jeden veliký orel s velkými křídly a bohatě opeřený. A hle, vinná réva upnula své kořeny k němu, i své větvoví k němu vztáhla, aby ji svlažoval mimo záhony, kde byla vysazena.
#17:8 Byla zasazena v dobrém poli u hojných vod, aby se rozvětvila a nesla plody, a stala se nádhernou révou.
#17:9 Pověz: Toto praví Panovník Hospodin: Vydaří se? Nevytrhne ji orel i s kořeny, a neotrhá jí plody, takže zaschne a všechno listí, jež na ní vyrašilo, uschne? Nebude třeba veliké síly a mnohého lidu, aby byla vyrvána z kořene.
#17:10 Hle, byla zasazena; ale vydaří se? Neuschne úplně, až ji zasáhne východní vítr? Na záhonech, kde vyrašila, uschne.“
#17:11 I stalo se ke mně slovo Hospodinovo:
#17:12 „Řekni vzpurnému domu: Což nevíte, co tyto věci znamenají? Řekni: Hle, do Jeruzaléma přitáhl babylónský král a zajal jeho krále i velmože a odvedl je k sobě do Babylóna.
#17:13 Vzal z královského potomstva jednoho a uzavřel s ním smlouvu, zavázal ho přísahou a zajal mocné té země,
#17:14 aby království bylo poníženo, aby se nepozvedlo, aby dbalo na jeho smlouvu, mělo-li obstát.
#17:15 Ale on se proti němu vzbouřil. Poslal své posly do Egypta, aby mu dali koně a mnoho lidu. Zdaří se to? Unikne, kdo páchá takové věci? Unikne, kdo porušuje smlouvu?
#17:16 Jakože jsem živ, je výrok Panovníka Hospodina, zemře v Babylóně, v sídle toho krále, který ho ustanovil králem, jehož přísahou pohrdl a jehož smlouvu porušil.
#17:17 Ani s velikým vojskem a s mnohými sbory nebude mu farao nic platný v bitvě, až navrší násep a zbuduje obléhací valy, aby vyhladil mnoho duší.
#17:18 Pohrdl přísahou, porušil smlouvu. Hle, dal na to ruku a dělá toto všechno. Neunikne.“
#17:19 Proto praví Panovník Hospodin toto: „Jakože jsem živ, přísahu mou, kterou pohrdl, a smlouvu mou, kterou porušil, uvalím na jeho hlavu.
#17:20 Rozprostřu na něho svou síť a bude polapen do mé lovecké sítě. Zavedu ho do Babylóna a tam se s ním budu soudit za to, že se mi zpronevěřil.
#17:21 Všichni jeho uprchlíci ve všech jeho vojích padnou mečem; kdo zůstanou, budou rozehnáni na všechny strany. I poznáte, že já Hospodin jsem promluvil.“
#17:22 Toto praví Panovník Hospodin: „Já vezmu ratolest z vrcholku vysokého cedru a zasadím ji, z vršku jeho koruny utrhnu snítku a zasadím na vysoko čnící hoře.
#17:23 Zasadím ji na vyvýšené hoře izraelské a vyžene větve, ponese plody a stane se nádherným cedrem. Pod ním bude bydlit všechno ptactvo, všechno, co má křídla, bude bydlit ve stínu jeho větvoví.
#17:24 Všechny stromy pole poznají, že já Hospodin jsem ponížil strom vysoký a povýšil strom nízký; nechal jsem uschnout strom zelený a dal vypučet stromu suchému. Já Hospodin jsem promluvil a také to učiním.“ 
#18:1 I stalo se ke mně slovo Hospodinovo:
#18:2 „Co si myslíte, když říkáte o izraelské zemi toto přísloví: ‚Otcové jedli nezralé hrozny a synům trnou zuby‘?
#18:3 Jakože jsem živ, je výrok Panovníka Hospodina, toto přísloví se už nebude mezi vámi v Izraeli říkat.
#18:4 Hle, mně patří všechny duše; jak duše otcova, tak duše synova jsou mé. Zemře ta duše, která hřeší.
#18:5 Je-li někdo spravedlivý a jedná podle práva a spravedlnosti,
#18:6 nehoduje na horách a nepozvedá oči k hnusným modlám izraelského domu, neposkvrňuje ženu svého bližního a nepřibližuje se k ženě v čase její nečistoty,
#18:7 nikoho neutiskuje, dlužníkovi vrací jeho zástavu, nikoho neodírá, hladovému dává svůj chléb a nahého přikrývá rouchem,
#18:8 nepůjčuje lichvářsky a nebere úrok, odvrací se od bezpráví, vykonává pravdivý soud mezi mužem a mužem,
#18:9 řídí se mými nařízeními, zachovává mé řády a jedná věrně, takový spravedlivý jistě bude žít, je výrok Panovníka Hospodina.
#18:10 Pokud však zplodí syna rozvratníka, který bude prolévat krev a dopouštět se proti bratru čehokoli z těchto věcí
#18:11 a nebude činit nic z oněch věcí dobrých, ale bude hodovat na horách, poskvrňovat ženu svého bližního,
#18:12 utiskovat utištěného ubožáka, druhého odírat, nevracet zástavu, pozvedat oči k hnusným modlám, dopouštět se ohavností,
#18:13 půjčovat lichvářsky a brát úrok, bude žít? Nebude žít; dopouštěl se všech těchto ohavností, jistě zemře, jeho krev bude na něm.
#18:14 Zplodí-li však syna, který uvidí všechny hříchy svého otce, jichž se dopouští, a ulekne se, nebude se jich dopouštět,
#18:15 nebude jíst na horách a pozvedat oči k hnusným modlám izraelského domu, nebude poskvrňovat ženu svého bližního,
#18:16 nebude nikoho utiskovat, neponechá si zástavu a nikoho nebude odírat, bude dávat hladovému svůj chléb a nahého přikrývat rouchem,
#18:17 neodtáhne svou ruku od utištěného, nevezme lichvu ani úrok, nýbrž bude jednat podle mých řádů a řídit se mými nařízeními, ten pro nepravost svého otce nezemře, jistě bude žít.
#18:18 Jeho otec, protože se dopouštěl útisku, odíral bratra a nekonal dobro mezi svým lidem, zemře pro svou nepravost.
#18:19 Vy však říkáte: ‚Proč nepyká syn za otcovu nepravost?‘ Bude-li syn jednat podle práva a spravedlnosti, dbát na všechna má nařízení a plnit je, jistě bude žít.
#18:20 Duše, která hřeší, ta umře; syn nebude pykat za nepravost otcovu a otec nebude pykat za nepravost synovu; spravedlnost zůstane na spravedlivém a zvůle zůstane na svévolníkovi.
#18:21 Kdyby se svévolník odvrátil ode všech svých hříchů, jichž se dopouštěl, a dbal by na všechna má nařízení a jednal podle práva a spravedlnosti, jistě bude žít, nezemře.
#18:22 Žádná jeho nevěrnost, jíž se dopustil, mu nebude připomínána, bude žít pro svou spravedlnost, podle níž jednal.
#18:23 Což si libuji v smrti svévolníka? je výrok Panovníka Hospodina. Zdalipak nechci, aby se odvrátil od svých cest a byl živ?
#18:24 Když se spravedlivý odvrátí od své spravedlnosti a bude se dopouštět bezpráví podle všech ohavností, jichž se dopouští svévolník, měl by žít? Žádné jeho spravedlivé činy, které konal, nebudou připomínány, zemře za to, že se zpronevěřil, za svůj hřích, kterým se prohřešil.
#18:25 Vy však říkáte: ‚Cesta Panovníkova není správná.‘ Nuže, slyšte, dome izraelský: Má cesta že není správná? Nejsou to vaše cesty, jež nejsou správné?
#18:26 Když se spravedlivý odvrátí od své spravedlnosti a dopouští se bezpráví, umře za to; zemře pro své bezpráví, jehož se dopouštěl.
#18:27 Když se však svévolník odvrátí od své zvůle, jíž se dopouštěl, a jedná podle práva a spravedlnosti, zachová svou duši při životě.
#18:28 Prohlédl totiž a odvrátil se ode všech svých nevěrností, jichž se dopouštěl; jistě bude žít, nezemře.
#18:29 Avšak dům izraelský říká: ‚Cesta Panovníkova není správná.‘ Mé cesty že nejsou správné, dome izraelský? Nejsou to vaše cesty, jež nejsou správné?
#18:30 Proto budu soudit každého z vás, dome izraelský, podle jeho cest, je výrok Panovníka Hospodina. Obraťte se a odvraťte se ode všech svých nevěrností a vaše nepravost vám nebude k pádu.
#18:31 Odhoďte od sebe všechny nevěrnosti, jichž jste se dopouštěli, a obnovte své srdce a svého ducha. Proč byste měli zemřít, dome izraelský?
#18:32 Vždyť já si nelibuji ve smrti toho, kdo umírá, je výrok Panovníka Hospodina. Obraťte se tedy a budete žít.“ 
#19:1 Ty pak začni žalozpěv nad izraelskými předáky.
#19:2 Zapěj: Kdo byla tvá matka? Lvice, jež odpočívala mezi lvy, mezi mladými lvy chovala svá lvíčata.
#19:3 Když odchovala jedno ze svých lvíčat, stal se z něho mladý lev, naučil se trhat kořist a požírat lidi.
#19:4 Doslechly se o něm pronárody, byl polapen v jejich pasti; odvlekli ho za háky v chřípí do egyptské země.
#19:5 Když lvice viděla, že čeká marně, pozbyla naděje. I vzala jedno ze svých lvíčat a učinila z něho mladého lva.
#19:6 Ten se procházel mezi lvy, stal se mladým lvem, naučil se trhat kořist a požírat lidi.
#19:7 Bořil jejich paláce a v trosky obracel jejich města, jeho řevem zpustla země se vším, co je na ní.
#19:8 Vypravily se na něho pronárody z okolních krajin, rozprostřely na něho svou síť, byl polapen v jejich pasti.
#19:9 Dali jej do klece, za háky jej dovlekli ke králi babylónskému, dovlekli ho do pevností, aby již nebylo slyšet jeho řev na horách izraelských.
#19:10 Tvá matka jako vinná réva, podobně jako ty, byla zasazena u vody; byla plodná a bujně vzrostlá pro hojnost vod.
#19:11 Měla mohutné sněti, vhodné na žezla vladařů, svým vzrůstem se vypínala nad košaté stromy, bylo ji vidět pro její výšku i množství větvoví.
#19:12 Avšak byla vyrvána v rozhořčení, pohozena na zem a východní vítr vysušil její plody. Její mohutné sněti byly ulámány, uschly a pozřel je oheň.
#19:13 Nyní je přesazena do stepi, do země vyprahlé a žíznivé.
#19:14 Ze snětí vyšlehl oheň, pozřel její větve i plody; nezbyla na ní žádná mohutná sněť, odnož vhodná na vladařské žezlo. To je žalozpěv a jako žalozpěv se bude zpívat. 
#20:1 Sedmého roku, pátého měsíce a desátého dne toho měsíce, přišli někteří z izraelských starších dotazovat se Hospodina a posadili se přede mnou.
#20:2 I stalo se ke mně slovo Hospodinovo:
#20:3 „Lidský synu, mluv k izraelským starším a řekni jim: ‚Toto praví Panovník Hospodin: Přicházíte se mě dotazovat? Jakože jsem živ, nebudu vám dávat odpovědi na dotazy, je výrok Panovníka Hospodina.‘
#20:4 Chceš je soudit, chceš se ujmout soudu, lidský synu? Seznam je s ohavnostmi jejich otců,
#20:5 řekni jim: Toto praví Panovník Hospodin: V den, kdy jsem si vyvolil Izraele, zvedl jsem ruku k přísaze potomstvu Jákobova domu a dal jsem se jim poznat v egyptské zemi, zvedl jsem ruku a přisáhl jim: Já jsem Hospodin, váš Bůh.
#20:6 V onen den jsem zvedl ruku a přísahal jim, že je vyvedu z egyptské země do země, kterou jsem pro ně vyhlédl a která oplývá mlékem a medem a je skvostem všech zemí.
#20:7 Řekl jsem jim: Odvrhněte každý ohyzdné modly, k nimž vzhlížíte, a neposkvrňujte se egyptskými hnusnými modlami, já jsem Hospodin, váš Bůh.
#20:8 Ale oni se mi vzepřeli a nechtěli mě ani slyšet. Žádný neodhodil ohyzdné modly, k nimž vzhlížel, a neopustil egyptské hnusné modly. I řekl jsem, že na ně vyleji své rozhořčení, a tak dovrším svůj hněv proti nim uprostřed egyptské země.
#20:9 Učinil jsem to pro své jméno, aby nebylo znesvěcováno v očích pronárodů, uprostřed nichž byli, před jejich zraky jsem se jim dal poznat, když jsem je vyvedl z egyptské země.
#20:10 Vyvedl jsem je tedy z egyptské země a uvedl na poušť.
#20:11 Vydal jsem jim svá nařízení a seznámil je se svými řády, skrze něž má člověk život, když je plní.
#20:12 Dal jsem jim také své dny odpočinku, aby byly znamením mezi mnou a jimi, aby věděli, že já Hospodin je posvěcuji.
#20:13 Ale dům izraelský se mi na poušti vzepřel, mými nařízeními se neřídili a znevážili mé řády, skrze něž má člověk život, když je plní, a hrubě znesvěcovali mé dny odpočinku. I řekl jsem, že na ně vyleji na poušti své rozhořčení, a tak s nimi skoncuji.
#20:14 Učinil jsem to pro své jméno, aby nebylo znesvěcováno před očima pronárodů, před jejichž zraky jsem je vyvedl.
#20:15 Já jsem též zvedl ruku a přisáhl jim na poušti, že je neuvedu do země, kterou jsem jim dal, do země oplývající mlékem a medem, jež je skvostem všech zemí,
#20:16 a to proto, že znevážili mé řády, neřídili se mými nařízeními a znesvěcovali mé dny odpočinku, neboť jejich srdce chodilo za jejich hnusnými modlami.
#20:17 Ale bylo mi líto je zničit a neskoncoval jsem s nimi na poušti.
#20:18 Jejich synům jsem na poušti řekl: ‚Neřiďte se nařízeními svých otců, nedbejte na jejich řády a neposkvrňujte se jejich hnusnými modlami,
#20:19 já jsem Hospodin, váš Bůh. Řiďte se mými nařízeními, dodržujte mé řády a zachovávejte je.
#20:20 Svěťte mé dny odpočinku a budou znamením mezi mnou a vámi, abyste věděli, že já jsem Hospodin, váš Bůh.‘
#20:21 Ale i synové se mi vzepřeli, neřídili se mými nařízeními, nedodržovali mé řády, skrze něž má člověk život, když je plní, a znesvěcovali mé dny odpočinku. I řekl jsem, že na ně vyleji své rozhořčení, a tak na poušti dovrším svůj hněv proti nim.
#20:22 Ale zadržel jsem svou ruku. Učinil jsem to pro své jméno, aby nebylo znesvěceno před očima pronárodů, před jejichž zraky jsem je vyvedl.
#20:23 Já jsem též zvedl ruku a přisáhl jim na poušti, že je rozptýlím mezi pronárody a rozpráším po zemích,
#20:24 protože se nedrželi mých řádů a znevažovali má nařízení, znesvěcovali mé dny odpočinku a vzhlíželi k hnusným modlám svých otců.
#20:25 A tak jsem jim dal nedobrá nařízení a řády, skrze něž nebudou mít život;
#20:26 poskvrnil jsem je jejich vlastními dary, když prováděli ohněm vše, co otvírá lůno, abych je naplnil úděsem a oni poznali, že já jsme Hospodin.
#20:27 Proto mluv, lidský synu, k izraelskému domu a řekni jim: Toto praví Panovník Hospodin: Vaši otcové mě dál a dále hanobili tím, že se mi zronevěřovali.
#20:28 Já jsem je uvedl do země, o níž jsem zvednutím ruky přisáhl, že ji dám jim, ale oni, jak uviděli nějaký vyvýšený pahorek a nějaký listnatý strom, hned tam obětovali své oběti a přinášeli tam své popouzející dary, kladli tam své libovonné oběti a vylévali tam své úlitby.
#20:29 Řekl jsem jim: Na jakéto posvátné návrší tam chodíte? Nazývají je ‚Posvátné návrší‘ až dodnes.“
#20:30 Proto řekni izraelskému domu: Toto praví Panovník Hospodin: Což se musíte poskvrňovat cestou svých otců a smilnit s jejich ohyzdnými modlami?
#20:31 Až dodnes se poskvrňujete všemi svými hnusnými modlami, přinášíte své dary a provádíte své syny ohněm, a já bych vám měl dávat odpovědi na dotazy, dome izraelský? Jakože jsem živ, je výrok Panovníka Hospodina, nebudu vám odpovídat na dotazy!
#20:32 Co vám vstoupilo na mysl, určitě se nestane. Říkejte si: ‚Budeme jako pronárody, jako čeledi jiných zemí, budeme sloužit dřevu a kameni.‘
#20:33 Jakože jsem živ, je výrok Panovníka Hospodina, budu nad vámi kralovat pevnou rukou a vztaženou paží v rozhořčení, které na vás vyleji.
#20:34 A vyvedu vás z národů a shromáždím vás ze zemí, do nichž jste byli rozptýleni pevnou rukou a vztaženou paží v rozhořčení, které jsem na vás vylil.
#20:35 Uvedu vás do pouště národů a budu se tam s vámi soudit tváří v tvář.
#20:36 Jako jsem se soudil s vašimi otci v poušti egyptské země, tak se budu soudit s vámi, je výrok Panovníka Hospodina.
#20:37 A provedu vás pod holí a uvedu vás do závazku smlouvy.
#20:38 Vytřídím z vás ty, kdo se vzbouřili a byli mi nevěrní. Vyvedu je ze země jejich dočasného pobytu, ale na izraelskou půdu nevstoupí. I poznáte, že já jsem Hospodin.
#20:39 Vy tedy, dome izraelský, slyšte. Toto praví Panovník Hospodin: Jděte a služte si každý svým hnusným modlám i nadále, nechcete-li poslouchat mne, ale neznesvěcujte už mé svaté jméno svými dary a svými hnusnými modlami.
#20:40 Na mé svaté hoře, na vysoké hoře izraelské, je výrok Panovníka Hospodina, tam mi bude sloužit celý izraelský dům, až se usadí celý v zemi, tam v nich najdu zalíbení a tam budu vyhledávat vaše oběti pozdvihování a prvotiny vašich darů se všemi vašimi svatými dary.
#20:41 Najdu ve vás zalíbení pro libou vůni vašich obětí, až vás vyvedu z národů a shromáždím vás z těch zemí, do nichž jste byli rozptýleni, a budu ve vás posvěcen před očima pronárodů.
#20:42 I poznáte, že já jsem Hospodin, až vás přivedu na izraelskou půdu, do země, o níž jsem zvednutím ruky přisáhl, že ji dám vašim otcům.
#20:43 Tam si připomenete své cesty a všechny své skutky, jimiž jste se poskvrňovali, a zhnusíte se sami sobě pro všechny zlé činy, jichž jste se dopouštěli.
#20:44 I poznáte, že já jsme Hospodin, až pro své jméno s vámi naložím ne podle vašich zlých cest ani podle vašich zvrhlých skutků, dome izraelský, je výrok Panovníka Hospodina.“ 
#21:1 I stalo se ke mně slovo Hospodinovo:
#21:2 „Lidský synu, postav se směrem k jihu a vynes věštbu proti polední straně a prorokuj proti lesu na poli Negebu.
#21:3 Řekni Negebskému lesu: Slyš Hospodinovo slovo! Toto praví Panovník Hospodin: Hle, já v tobě zaněcuji oheň a ten v tobě pozře každý strom zelený i každý strom suchý. Planoucí plamen neuhasne a budou jím popáleny všechny tváře od Negebu až na sever.
#21:4 I uzří všechno tvorstvo, že jsem je zažehl já Hospodin; neuhasne.“
#21:5 Tu jsem řekl: „Ach, Panovníku Hospodine, říkají o mně: ‚Což nám nevypráví samá podobenství?‘“
#21:6 I stalo se ke mně slovo Hospodinovo:
#21:7 „Lidský synu, postav se proti Jeruzalému a vynes věštbu proti svatyním a prorokuj proti izraelské zemi.
#21:8 Řekni izraelské zemi: Toto praví Hospodin: Chystám se na tebe a tasím svůj meč z pochvy; vyhladím z tebe spravedlivého i svévolníka.
#21:9 Proto, abych vyhladil z tebe spravedlivého i svévolníka, tasím svůj meč z pochvy na všechno tvorstvo od Negebu až na sever.
#21:10 I pozná všechno tvorstvo, že já Hospodin jsem tasil svůj meč z pochvy; již se tam nevrátí.“
#21:11 „Ty, lidský synu, slyš. Vzdychej jako bys měl polámaná bedra, hořce vzdychej před jejich zraky.
#21:12 Až se tě zeptají: ‚Nad čím vzdycháš?‘, odpovíš: ‚Nad zprávou, která přichází‘. Každé srdce ztratí odvahu a všechny ruce ochabnou, každý duch pohasne a všechna kolena se rozplynou jak voda. Hle, už to přichází a nastává, je výrok Panovníka Hospodina.“
#21:13 I stalo se ke mně slovo Hospodinovo:
#21:14 „Lidský synu, prorokuj a řekni: Toto praví Panovník: Řekni: Meč, ano meč je naostřen a vyleštěn.
#21:15 Je naostřen, by porážel jako na jatkách, vyleštěn, aby se blýskal. ‚Což nemáme důvod se veselit? Vždyť řekl: Hůl mého syna zneváží každý strom!‘
#21:16 Ale on dal vyleštit meč, aby se ho chopil. Meč je naostřen a vyleštěn, aby jej dal do ruky zabíjejícího.
#21:17 Úpěj a kvílej, lidský synu, neboť je na můj lid, na všechny izraelské předáky propadlé meči spolu s mým lidem; proto se bij v bedra.
#21:18 Je to zkouška: Co, když žádná znevažující hůl nebude? je výrok Panovníka Hospodina.“
#21:19 „Ty, lidský synu, slyš. Prorokuj a tleskni rukama: Meč dopadne dvakrát i třikrát, bude to meč skolených, meč velikého klání, kroužící kolem nich.
#21:20 Aby zakolísalo srdce a rozmnožily se pády, na všechny jejich brány jsem seslal meč, aby zabíjel. Byl udělán, aby se blýskal, je nabroušen, aby porážel.
#21:21 Rozpřáhni se napravo, zasáhni doleva, kamkoli se zaměří tvé ostří.
#21:22 Také já tlesknu rukama, až upokojím své rozhořčení. Já Hospodin jsem promluvil.“
#21:23 I stalo se ke mně slovo Hospodinovo:
#21:24 „Ty, lidský synu, slyš. Naznač si dvě cesty, jimiž přijde meč babylónského krále. Obě budou vycházet z jedné země. A na rozcestí k město postav ukazatele, postav jej.
#21:25 Naznač cestu, kterou přijde meč k Rabě Amónovců a do Judska, do opevněného Jeruzaléma.
#21:26 Babylónský král se zastaví na křižovatce, na rozcestí obou cest, aby si vyžádal věštbu. Bude potřásat šípy, bude se doptávat domácích bůžků a nahlížet do jater.
#21:27 V pravici bude mít věštbu ohledně Jeruzaléma, že se mají postavit berany, že se má dát povel k vraždění, že se má zvednout válečný pokřik, že se mají postavit berany proti branám a navršit násep, že se má zbudovat obléhací val.
#21:28 Těm, kdo se zavázali přísahami, se to bude zdát šalebnou věštbou. On však připomene nepravost, pro niž budou chyceni.“
#21:29 Proto praví Panovník Hospodin toto: „Protože připomínáte svou nepravost a odhalily se vaše nevěrnosti a vaše hříchy se staly zřejmými ve všech vašich skutcích, protože jste se sami připomněli, budete tou ruku chyceni.
#21:30 Ty pak, bídný svévolníku, izraelský kníže, jehož den nadchází v čase, kdy nepravost spěje ke konci,
#21:31 toto praví Panovník Hospodin: Sejmi turban a sundej korunu, nastává změna. Co je nízké, ať je vyvýšeno, a vysoké ať je sníženo.
#21:32 Sutiny, sutiny, sutiny ze všeho nadělám, jaké nikdy nebyly, dokud nepřijde ten, jemuž přísluší soud; já jsem ho jím pověřil.“
#21:33 „Ty, lidský synu, slyš. Prorokuj a řekni: Toto praví Panovník Hospodin o Amónovcích a o jejich utrhání. Řekni: Meč, ano meč je tasen, aby porážel, je vyleštěn, aby požíral, ať se blýská.
#21:34 Mají pro tebe šalebné vidění, lživě ti věští, aby tě připojili k svévolníkům s proťatými hrdly. Jejich den nadchází v čase, kdy nepravost spěje ke konci.
#21:35 Vrať meč do pochvy. Budu tě soudit v místě, kde jsi byla stvořena, dcero amónská, v zemi tvého původu.
#21:36 Vyleji na tebe svůj hrozný hněv, budu na tebe soptit oheň své prchlivosti, vydám tě do rukou surovců, kteří chystají zkázu.
#21:37 Staneš se pokrmem ohně, tvá krev bude prolita v zemi, už nebudeš připomínána, neboť já Hospodin jsem promluvil.“ 
#22:1 I stalo se ke mně slovo Hospodinovo:
#22:2 „Ty, lidský synu, slyš. Chceš vynést rozsudek? Rozsudek nad městem, v němž je prolévána krev? Seznam je se všemi jeho ohavnostmi.
#22:3 Řekni: Toto praví Panovník Hospodin: Je to město prolévající krev ve svém středu, přichází jeho čas; zhotovuje si hnusné modly a poskrvňuje se jimi.
#22:4 Svou krví, kterou jsi prolilo, ses provinilo a svými hnusnými modlami, které jsi zhotovilo, ses poskvrnilo. Zavinilo jsi, že se přiblížily tvé dny a docházejí tvá léta. Proto tě vydávám v potupu pronárodům a za posměch všem zemím.
#22:5 Blízké i daleké se ti budem posmívat pro nečisté jméno a velké zmatky.
#22:6 Hle, izraelští předáci v tobě jsou pohotoví k prolévání krve, každý vlastní paží.
#22:7 Otce i matku v tobě zlehčují, bezdomovce uprostřed tebe utlačují, sirotka a vdovu v tobě utiskují.
#22:8 Pohrdáš mými svatými dary, znesvěcuješ mé dny odpočinku.
#22:9 Jsou v tobě utrhači pohotoví k prolévání krve a ti, kdo hodují na horách, páchají uprostřed tebe mrzkost.
#22:10 V tobě syn odkrývá otcovu nahotu, v tobě ponižují ženu v období její nečistoty.
#22:11 V tobě se jeden dopouští ohavností se ženou druhého, tchán mrzce poskrňuje svou snachu, bratr ponižuje svou sestru, dceru svého otce.
#22:12 V tobě berou úplatek, jsou pohotoví k prolévání krve, bereš lichvářský úrok, chamtivě vydíráš svého bližního a na mne zapomínáš, je výrok Panovníka Hospodina.
#22:13 Hle, udeřím pěstí do toho, co jsi uchvátilo, a to kvůli prolité krvi, která je uprostřed tebe.
#22:14 Obstojí tvá odvaha, budeš jednat rozhodně ve dnech, kdy proti tobě zakročím? Já Hospodin jsem promluvil a učiním to.
#22:15 Rozptýlím tě mezi pronárody, rozpráším tě po zemích a dokonale tě zbavím tvé nečistoty.
#22:16 Samo se znesvětíš před očima pronárodů. I poznáš, že já jsem Hospodin.“
#22:17 I stalo se ke mně slovo Hospodinovo:
#22:18 „Lidský synu, dům izraelský se mi stal struskou, všichni jsou pouhá měď a cín, železo a olovo uvnitř tavicí pece, stali se struskou po tavbě stříbra.
#22:19 Proto praví Panovník Hospodin toto: Protože jste se stali všichni struskou, hle, já vás shromáždím doprostřed Jeruzaléma.
#22:20 Jako se shromažďuje stříbro a měď, železo, olovo a cín dovnitř tavící pece a kolem ní se rozdmýchá oheň k tavbě, tak vás shromáždím ve svém hněvu a rozhořčení, vložím do pece a roztavím.
#22:21 Shrnu vás dohromady, rozdmýchám kolem vás oheň své prchlivosti a vy se tam budete tavit.
#22:22 Jako se taví stříbro uvnitř tavicí pece, tak budete roztaveni uprostřed města. I poznáte, že já Hospodin jsem na vás vylil své rozhořčení.“
#22:23 I stalo se ke mně slovo Hospodinovo:
#22:24 „Lidský synu, řekni zemi: Ty jsi země, která není čistá, která nebude v den hrozného hněvu zavlažena.
#22:25 Spiknutí jejích proroků je uprostřed ní; jsou jako řvoucí lev, který trhá kořist, požírají duše, berou klenoty a skvosty, rozmnožují počet jejích vdov uprostřed ní.
#22:26 Její kněží znásilňují můj zákon a znesvěcují moje svaté věci. Nerozlišují svaté od nesvatého, neseznamují s rozdílem mezi nečistým a čistým, přehlížejí mé dny odpočinku, jsem mezi nimi znesvěcován.
#22:27 Její velmožové jsou uprostřed ní jako vlci, kteří trhají kořist, prolévají krev a hubí duše, neboť chamtivě shánějí zisk.
#22:28 Její proroci jim všechno nahazují omítkou, mají šalebná vidění a věští jim lživé věci. Říkají: ‚Toto praví Panovník Hospodin‘, ačkoli Hospodin nepromluvil.
#22:29 Lid země tvrdě utlačuje kdekoho, odírá ho, poškozuje utištěného ubožáka a protiprávně utlačuje bezdomovce.
#22:30 Hledal jsem mezi nimi muže, který by zazdíval zeď a postavil se v trhlině před mou tvář za tuto zemi, abych ji neuvrhl do zkázy, ale nenašel jsem.
#22:31 Vylil jsem tedy na ně svůj hrozný hněv, v ohni své prchlivosti jsem s nimi skoncoval, jejich cestu jsem jim uvalil na hlavu, je výrok Panovníka Hospodina.“ 
#23:1 I stalo se ke mně slovo Hospodinovo:
#23:2 „Lidský synu, dvě ženy byla dcerami jedné matky.
#23:3 Smilnily v Egyptě, smilnily od svého mládí. Tam byla mačkána jejich ňadra a ohmatávány jejich panenské bradavky.
#23:4 Starší se jmenovala Ohola a její sestra Oholíba. Byly mé a rodily syny a dcery. Jménem Ohola je míněno Samaří, Oholíba je Jeruzalém.
#23:5 Ohola smilnila; místo aby zůstala má, vášnivě se oddávala svým milencům, blízkým Asyřanům,
#23:6 místodržitelům a zemským správcům, oblečeným do fialového purpuru; vesměs to skvělým jinochům, jezdcům, jedzdícím na ořích.
#23:7 Nabízela se jim ke smilnění, veškerému výkvětu asyrských synů; poskvrňovala se se všemi, jimž se vášnivě oddávala, se všemi jejich hnusnými modlami.
#23:8 Nezanechala ani svého smilnění s Egypťany, neboť v jejím mládí s ní léhali, ohmatávali její panenské bradavky a vylévali na ni své smilství.
#23:9 Proto jsem ji vydal do rukou jejích milenců, do rukou asyrských synů, kterým se vášnivě oddávala.
#23:10 Ti odkryli její nahotu, vzali jí syny a dcery a zavraždili ji mečem. I stala se pověstnou mezi ženami, když nad ní vykonali soud.
#23:11 Její sestra Oholíba to viděla, jednala však ve své vášnivosti ještě hůř než ona a její smilnění bylo horší než smilstvo její sestry.
#23:12 Vášnivě se oddávala asyrským synům, místodržitelům, zemským správců, blízkým, dokonale vystrojeným, jezdcům jezdícím na ořích, vesměs to skvělým jinochům.
#23:13 Viděl jsem, že se poskvrnila, že obě jdou touž cestou.
#23:14 A šla ve svém smilnění ještě dál, když uviděla muže namalované na zdi, obrazy Kaldejců nakreslené rudkou,
#23:15 přepásané pásem kolem beder, s převislými turbany na hlavách; svým vzhledem byli všichni jako osádky válečných vozů, podobou synové Babylóna, jejichž rodnou zemí je Kaldejsko.
#23:16 Jim se vášnivě oddávala, když je spatřily její oči, a posílala k nim do Kaldejska posly.
#23:17 Synové Babylóna k ní vcházeli na milostné lůžko a poskvrňovali je svým smilněním. A když se s nimi poskvrnila, odvrátila se od nich.
#23:18 Odhalila své smilnění, odhalila svou nahotu, a já jsem se od ní odvrátil, tak jako jsem se odvrátil od její sestry.
#23:19 Stále víc propadala svému smilnění při vzpomínce na dny svého mládí, kdy smilnila v egyptské zemi,
#23:20 a vášnivě se oddávala svým záletníkům, jejichž úd je jako úd oslů a jejich výron jako výron hřebců.
#23:21 Tak ses ohlížela po mrzkosti svého mládí, kdy v Egyptě ohmatávali bradavky tvých mladých ňader.
#23:22 Proto, Oholíbo, toto praví Panovník Hospodin: Hle, já vzbudím proti tobě tvé milence, od kterých ses odvrátila, a přivedu je proti tobě z okolních zemí:
#23:23 syny Babylóna a všechny Kaldejce, Pekóda, Šóu a Kóu, všechny syny Asýrie s nimi, skvělé jinochy, všechny místodržitele, představené, osádky válečných vozů a slovutné, všechny ty, kteří jezdí na ořích.
#23:24 A tak proti tobě přitáhne válečná vozba a káry se společenstvím národů; pavézy, štíty a přilby se nepřátelsky položí kolem tebe. A pověřím je soudem, aby tě soudili podle svých práv.
#23:25 Postihnu tě svou žárlivostí a oni s tebou v rozhořčení naloží takto: uříznou ti nos a uši, a co po tobě zbude, padne mečem. Vezmou ti syny a dcery, a co po tobě zbude, bude pozřeno ohněm.
#23:26 Svléknou ti roucho a vezmou ti okrasné předměty.
#23:27 Tak od tebe odstraním tvou mrzkost a tvé smilstvo z egyptské země; už k nim nepozvedneš oči a na Egypt už nevzpomeneš.
#23:28 Toto praví Panovník Hospodin: Hle, vydávám tě do rukou těch, které nenávidíš, do rukou těch, od kterých ses odvrátila.
#23:29 Naloží s tebou nenávistně a vezmou ti všechno, cos vytěžila, a nechají tě nahou a obnaženou. A bude odkryto v celé nahotě tvé smilstvo, tvá mrzkost a tvé smilnění.
#23:30 Toto tě postihne, protože jsi smilnila po vzoru pronárodů, za to, že ses poskvrňovala jejich hnusnými modlami.
#23:31 Šla jsi po cestě své sestry, a tak ti dám do ruky její kalich.“
#23:32 Toto praví Panovník Hospodin: „Budeš pít kalich své sestry, hluboký a široký, bude příčinou smíchu a posměchu, pojme mnoho.
#23:33 Zcela se opojíš strastí, z kalicha úděsu a zpustošení, z kalicha své sestry Samaří.
#23:34 Vypiješ jej do dna, rozhryžeš jej na střepy, rozdrásáš si ňadra, neboť já jsem promluvil, je výrok Panovníka Hospodina.“
#23:35 Proto praví Panovník Hospodin toto: „Poněvadž jsi na mne zapomněla a odhodila mě za záda, sama nes následky své mrzkosti a svého smilnění.“
#23:36 Hospodin mi řekl: „Lidský synu, chceš vynést rozsudek nad Oholou a Oholíbou? Oznam jim tedy jejich ohavnosti!
#23:37 Cizoložily a na jejich rukou je krev. Cizoložily se svými hnusnými modlami a přiváděly jim dokonce za pokrm své syny, které porodily mně.
#23:38 A navíc mi činily toto: Téhož dne poskvrňovali mou svatyni a znesvěcovaly mé dny odpočinku. -
#23:39 Téhož dne, kdy zabíjeli své syny svým hnusným modlám, vcházeli do mé svatyně, a tak ji znesvěcovaly. Toho se dopouštěli v mém domě. -
#23:40 Ony si dokonce posílaly pro muže přicházející z dálky. Ti, když k nim byl poslán posel, přicházeli. Pro ně ses omývala, oči si líčila a ozdobou zdobila.
#23:41 Sedala sis na nádherné lože, před nímž byl prostřený stůl, a kladla jsi na něj moje kadidlo a můj olej.
#23:42 Bezstarostný dav množství lidí halasil v městě vstříc zpitým mužům přiváděným z pouště; ti dávali ženám na ruce náramky a na hlavy okrasnou korunu.
#23:43 Řekl jsem o té, která cizoložstvím sešla: Ještě teď s ní páchají smilstva a ona s nimi.
#23:44 Vcházejí k ní, jako se vchází k nevěstce. Tak vcházejí k Ohole a k Oholíbě, mrzkým ženám.
#23:45 Avšak budou je soudit spravedliví mužové podle práva o cizoložnicích a o těch, kdo prolévají krev. Jsou to cizoložnice a na jejich rukou je krev.“
#23:46 Toto praví Panovník Hospodin: „Ať vystoupí proti nim shromáždění, a vydám je hrůze a oloupení.
#23:47 Shromáždění je uhází kamením a rozseká je svými meči. Jejich syny a dcery povraždí a jejich domy spálí ohněm.
#23:48 Tak odstraním ze země mrzkost a všechny ženy přijmou výstrahu a nebudou se dopouštět mrzkosti jako vy.
#23:49 Za vaši mrzkost vás budou stíhat a ponesete hříchy svého hnusného modlářství. I poznáte, že já jsem Panovník Hospodin.“ 
#24:1 I stalo se ke mně slovo Hospodinovo v devátém roce, desátého měsíce, desátého dne toho měsíce:
#24:2 „Lidský synu, napiš si jméno dne, právě tohoto dne. Právě v tento den napadl babylónský král Jeruzalém.
#24:3 Předlož vzpurnému domu podobenství a řekni jim: Toto praví Panovník Hospodin: Přistav hrnec, přistav, a také do něho nalej vodu.
#24:4 Naházej do něho díly masa, samé dobré díly, kýtu a plece, naplň jej vybranými kostmi.
#24:5 Vezmi vybraný kus bravu, kosti narovnej vespod a uveď to do největšího varu; ať se v něm vaří i ty kosti.“
#24:6 Proto praví Panovník Hospodin toto: „Běda městu, v němž teče krev, hrnci, v němž zůstává připálenina, z něhož připálenina nesejde. Vytahuj z něho díl po dílu, o město se nebude losovat,
#24:7 neboť krev, kterou prolilo, zůstává uprostřed něho, nechalo ji téci na holou skálu, nevylilo ji na zem, aby ji přikryl prach.
#24:8 Ať se zvedne rozhořčení, ať vzplane pomsta! Nechám tu krev na holé skále a nebude přikryta.“
#24:9 Proto praví Panovník Hospodin toto: „Běda městu, v němž teče krev, teď já udělám velkou hranici.
#24:10 Nalož hodně dříví, rozdmýchej oheň, dej maso vařit a úplně je rozvař, až se kosti připálí.
#24:11 Prázdný hrnec postav na žhavé uhlí, aby se rozpálil a jeho měď se rozžhavila; ať se v něm roztaví jeho nečistota a jeho připálenina vezme za své.
#24:12 Marná námaha! Množství jeho připáleniny z něho nesejde. Do ohně s jeho připáleninou!
#24:13 Ve tvé nečistotě je mrzkost; pročišťoval jsem tě, ale neočistilo ses od své nečistoty. Nebudeš už čisté, dokud na tobě neuspokojím své rozhořčení.
#24:14 Já Hospodin jsem promluvil; však to přijde, učiním to, nepolevím, nepocítím lítost a nebudu toho želet. Budu tě soudit podle tvé cesty a tvých skutků, je výrok Panovníka Hospodina.“
#24:15 I stalo se ke mně slovo Hospodinovo:
#24:16 „Lidský synu, hle, já ti náhlou ranou vezmu žádost tvých očí, ale ty nenaříkej, neplač, neuroň ani slzu.
#24:17 Sténej potichu, nekonej smuteční obřady za mrtvé, oviň si turban, na nohy si obuj opánky, nezahaluj si vous a nejez smuteční chléb, který ti lidé přinesou.“
#24:18 Ráno jsem o tom mluvil k lidu a večer mi zemřela žena. A já jsem ráno učinil, jak mi bylo přikázáno.
#24:19 I řekl mi lid: „Neoznámíš nám, co pro nás znamená to, co děláš?“
#24:20 Odvětil jsem jim: „Stalo se ke mně slovo Hospodinovo:
#24:21 Řekni izraelskému domu: Toto praví Panovník Hospodin: Hle, já znesvětím svou svatyni, pýchu vaší moci, žádost vašich očí a libost vaší duše. A vaši synové a vaše dcery, které zanecháte, padnou mečem.
#24:22 Pak se zachováte tak, jak jsem se zachoval já. Nezahalíte si vous a nebude jíst smuteční chléb, který vám lidé přinesou.
#24:23 Na hlavách budete mít turbany a na nohou opánky, nebudete naříkat ani plakat; zahynete pro své nepravosti a jen budete vzdychat jeden přes druhého.
#24:24 Ezechiel je vám předzvěstí. Zachováte se ve všem jako on. Až k tomu dojde, poznáte, že já jsem Panovník Hospodin.“
#24:25 „Ty, lidský synu, slyš. Zdali jim v ten den nevezmu jejich záštitu, jejich veselí a ozdobu, žádost jejich očí a to, k čemu tíhne jejich duše, jejich syny a dcery?
#24:26 V onen den k tobě přijde jeden z těch, kdo vyváznou, a podá zprávu.
#24:27 V onen den se ti otevřou ústa v přítomnosti toho, kdo vyvázl, a ty promluvíš a nebudeš už němý. Jsi jim předzvěstí. I poznají, že já jsem Hospodin.“ 
#25:1 I stalo se ke mně slovo Hospodinovo:
#25:2 „Lidský synu, postav se proti Amónovcům a prorokuj proti nim.
#25:3 Řekni Amónovcům: Slyšte slovo Panovníka Hospodina. Toto praví Panovník Hospodin: O mé svatyni, když byla znesvěcena, a o izraelské zemi, když byla pustošena, a o Judovu domu, když byli přesídlováni, jsi říkal: ‚Dobře jim tak!‘
#25:4 Proto hle, já tě vydám do vlastnictví synům východu a ti si v tobě postaví hradiště a založí si příbytky. Ti budou jíst tvé ovoce, ti budou pít tvé mléko.
#25:5 Učiním Rabu pastvinou velbloudů a zemi Amónovců místem, kde budou odpočívat ovce. I poznáte, že já jsem Hospodin.“
#25:6 Toto praví Panovník Hospodin: „Zatleskal jsi rukou a zadupal nohou a zcela bezostyšně ses radoval nad zemí izraelskou.
#25:7 Proto hle, já na tebe napřáhnu ruku, dám tě za kořist pronárodům, vymýtím tě z národů, vyhubím tě ze zemí a vyhladím tě. I poznáš, že já jsem Hospodin.“
#25:8 „Toto praví Panovník Hospodin: Moáb a Seír říkali: ‚Hle, judský dům je jako všechny pronárody.‘
#25:9 Proto hle, já otevřu horský hřeben Moábův, takže bude bez měst, bez svých měst až po samé hranice, bez chlouby země - Bét-ješimótu, Baal-meónu a Kirjátajimu.
#25:10 Vydám jej spolu s Amónovci synům východu; dám jim je do vlastnictví, aby se mezi pronárody nevzpomínalo na Amónovce.
#25:11 Také na Moábu vykonám soudy. I poznají, že já jsem Hospodin.“
#25:12 „Toto praví Panovník Hospodin: Edómci se tvrdě mstili domu Judovu a tím, že se na něm mstili, velice se provinili.
#25:13 Proto praví Panovník Hospodin toto: Napřáhnu na Edóm ruku a vymýtím z něho lidi i dobytek. Obrátím jej v trosky; od Témanu po Dedán budou padat mečem.
#25:14 Pomstu nad Edómem vykonám skrze Izraele, svůj lid; oni s Edómem naloží podle mého hněvu a rozhořčení. I poznají moji pomstu, je výrok Panovníka Hospodina.“
#25:15 „Toto praví Panovník Hospodin: Pelištejci se mstili, krutě a bezostyšně se mstili a z odvěkého nepřátelství šířili zkázu.
#25:16 Proto praví Panovník Hospodin toto: Hle, já napřáhnu na Pelištejce ruku a vyplením Keretejce. Vyhubím i pozůstatek lidu mořského pobřeží.
#25:17 A vykonám na nich velkou pomstu, v rozhořčení je potrestám. I poznají, že já jsem Hospodin, až na nich vykonám svou pomstu.“ 
#26:1 V jedenáctém roce, prvního dně měsíce, stalo se ke mně slovo Hospodinovo:
#26:2 „Lidský synu, protože Týr o Jeruzalému říká: ‚Ha, je rozbit, on, vrata národů, otevřel se mi vstříc, naplním se tím, co je v troskách‘.
#26:3 Proto praví Panovník Hospodin toto: Jsem proti tobě, Týre, přivedu na tebe mnohé pronárody, jako když moře přivádí svá vlnobití.
#26:4 Zničí hradby Týru, zboří jeho věže; i prach z něho smetu, učiním jej holou skálou.
#26:5 Staneš se místem v moři, kde se suší sítě. Já jsem promluvil, je výrok Panovníka Hospodina, lupem pronárodů se stane.
#26:6 Jeho dcery, které jsou na poli, budou povražděny mečem. I poznají, že já jsem Hospodin.“
#26:7 Toto praví Panovník Hospodin: „Hle, od severu přivádím na Týr Nebúkadnesara, krále babylónského, krále králů, s koni, vozy a jezdci, a shromáždění početného lidu.
#26:8 Tvé dcery na poli povraždím mečem. Zřídí proti tobě obléhací valy, navrší proti tobě násep, postaví proti tobě pavézy.
#26:9 Beranem bude bušit do tvých hradeb, tvé věže rozmetá svou zbrojí.
#26:10 Pokryje tě prach, zvířený spoustou jeho koní, tvé hradby se budou otřásat rachotem jeho jízdy, kol a vozů, až vstoupí do tvých bran, jako se vstupuje do poraženého města.
#26:11 Kopyta jeho koní zdusají všechny tvé ulice, tvůj lid povraždí mečem, posvátné sloupy tvé moci klesnou na zem.
#26:12 Ukořistí tvé jmění, uloupí tvé zboží, zboří tvé hradby a rozmetají tvé skvělé domy; tvé kamení, dříví i prach svrhnou do vod.
#26:13 Učiním přítrž hluku tvých písní, už se nebude rozléhat zvuk tvých citar.
#26:14 Učiním tě holou skálou, budeš místem, kde se suší sítě, nikdy už nebudeš vystavěn, neboť já Hospodin jsem promluvil, je výrok Panovníka Hospodina.“
#26:15 Toto praví Panovník Hospodin o Týru: „Což se nebudou ostrovy otřásat rachotem tvého pádu a sténáním skolených při hrozném vraždění uprostřed tebe?
#26:16 Ze svých trůnů sestoupí všechna knížata moře, sejmou své pláště, vysvlečou svá pestrá roucha, jejich šatem se stane třesení, posadí se na zem a budou se neustále třást úděsem nad tebou.
#26:17 A začnou nad tebou pět žalozpěv: ‚Jak jsi zmizelo z moří, ty město obydlené, vychvalované! Silné bylo na moři i se svými obyvateli, z nichž padal děs na všechny sídlící kolem.
#26:18 Nyní, v den tvého pádu, se třesou ostrovy. Mořské ostrovy pojala hrůza nad tvým zánikem‘.“
#26:19 Toto praví Panovník Hospodin: „Učiním z tebe město ležící v troskách, stejné jako města dávno neobydlená, přivedu na tebe propastnou tůň, přikryje tě obrovské vodstvo.
#26:20 Srazím tě s těmi, kdo sestupují do jámy, k lidu předvěkému, usadím tě v nejhlubších útrobách země jako odvěké trosky s těmi, kdo sestupují do jámy, a nebudeš už obýván. Tak se proslavím v zemi živých.
#26:21 Dám tě za odstrašující příklad, zanikneš. Budou tě hledat, ale nikdy, navěky, tě nenajdou, je výrok Panovníka Hospodina.“ 
#27:1 I stalo se ke mně slovo Hospodinovo:
#27:2 „Ty, lidský synu, slyš. Začni nad Týrem žalozpěv.
#27:3 Zapěj o Týru: Ty, který trůníš nad přístupy k moři a obchoduješ s národy na mnoha ostrovech, toto praví Panovník Hospodin: Týre, ty jsi říkal: ‚Já jsem dokonale krásný.‘
#27:4 Až k srdci moře sahá tvé pomezí, tvoji stavitelé vtiskli tvé kráse dokonalost.
#27:5 Z cypřišů senírských nadělali všechna obložení tvých korábů, libanónských cedrů použili ke zhotovení tvých stěžňů.
#27:6 Tvá vesla udělali z bášanských dubů a tvé deštění vyrobili ze slonoviny a ze zimostrázů z ostrovů kitejských.
#27:7 Tvým plachtovím bylo pestré, jemné egyptské plátno, tvou korouhví fialový purpur, přikrýval tě nach z elíšských ostrovů.
#27:8 Tvými veslaři byli obyvatelé Sidónu a Arvadu, své nejzkušenější jsi zaměstnával jako velitele lodí, Týre.
#27:9 Starší Gebalu a jeho zkušené jsi zaměstnával spravováním toho, co chátralo, kotvily v tobě všechny mořské lodě se svými lodníky a směňovaly tvé zboží.
#27:10 Bojovníky tvého vojska byli Peršané, Lúďané a Pútejci, ve tvých zbrojnicích zavěšovali štíty a přilby, dodávali ti lesku.
#27:11 S tvým vojskem byli na tvých hradbách kolem dokola synové Arvadu a na tvých věžích byli Gamáďané, zavěšovali kolkolem na tvé hradby své štíty, vtiskli tvé kráse dokonalost.
#27:12 Taršíš s tebou kupčil s množstvím různého zboží. Dodával na tvé trhy stříbro, železo, cín i olovo.
#27:13 Obchodovali s tebou Jávan, Túbal a Mešek; směňovali s tebou otroky a bronzové nádoby.
#27:14 Koně tažné i jezdecké a mezky na tvé trhy dodávali z Bét-togarmy.
#27:15 Obchodovali s tebou synové Dedánu, mnohé ostrovy byly tvými překupníky; poctou ti přinášeli sloní kly a ebenové dřevo.
#27:16 Kupčil s tebou Aram s mnohými tvými výrobky a dodával na tvé trhy malachit, nach, pestré látky, bělostné plátno, korály a rubín.
#27:17 Obchodovaly s tebou Judsko a izraelská země; směňovaly s tebou mínitskou pšenici, proso, med, olej a mastix.
#27:18 I Damašek s tebou kupčil s mnohými tvými výrobky, s množstvím různého zboží, s chelbónským vínem a sacharskou vlnou.
#27:19 Dan a Jávan dodávali na tvé trhy přízi; směňovali s tebou zpracované železo, skořici a jiné koření.
#27:20 Dedán s tebou obchodoval s pokrývkami na sedla pro jízdu.
#27:21 Tvými překupníky byli Arabové a všichni kédarští předáci; obchodovali s tebou s jehňaty, berany a kozly.
#27:22 Obchodovali s tebou obchodníci Šéby a Raemy; dodávali na tvé trhy všechny nejvzácnější balzámy, rozmanité drahokamy a zlato.
#27:23 Obchodovali s tebou Cháran, Kané a Eden, obchodníci Šéby, Asýrie a Kilmadu.
#27:24 Ti s tebou obchodovali; na tvých tržištích byla skvostná roucha, pláště fialově purpurové i pestré, barevné tkaniny, lana pletená i točená.
#27:25 Zámořské lodě a karavany s tvým směnným zbožím tě velice zaplnily a proslavily až v srdci moře.
#27:26 Tvoji veslaři tě zavezli na širé vody, ale v srdci moře tě roztříští východní vítr.
#27:27 Tvůj majetek a trhy, zboží, tvoji lodníci a velitelé lodí, ti, kdo opravovali, co zchátralo, obchodníci směňující tvé zboží, všichni tvoji bojovníci, celé tvé shromáždění uprostřed tebe, všichni padnou do srdce moře v den tvého pádu.
#27:28 Pro úpěnlivé volání tvých velitelů lodí se budou chvět pastviny.
#27:29 Všichni veslaři vystoupí ze svých lodí, lodníci a všichni velitelé mořských lodí stanou na zemi.
#27:30 Budou se nad tebou rozléhat jejich hlasy, budou hořce úpět, budou si na hlavu házet prach, válet se v popelu.
#27:31 Vyholí si lysinu, opásají se žíněným rouchem, budou nad tebou plakat v hořkosti duše, nastane přehořký nářek.
#27:32 Při svém bědování začnou nad tebou žalozpěv, žalostně zapějí: ‚Kdo byl umlčen jako Týr uprostřed moře?‘
#27:33 Dokud jsi vyplouval na moře za trhem, sytil jsi mnohé národy, z množství tvého směnného zboží bohatli králové země.
#27:34 Až je moře roztříští v hlubokých vodách, padne tvůj směnný obchod i celé tvé shromáždění uprostřed tebe.
#27:35 Úděs nad tebou zachvátí všechny obyvatelé ostrovů, jejich králové se budou třást hrůzou, obličej se jim zkřiví.
#27:36 Kupčíci mezi národy nad tebou syknou, budeš odstrašujícím příkladem, zanikneš navěky.“ 
#28:1 I stalo se ke mně slovo Hospodinovo:
#28:2 „Lidský synu, řekni týrskému vévodovi: Toto praví Panovník Hospodin: Ve svém domýšlivém srdci si říkáš: ‚Jsem Bůh, sedím na božském trůnu v srdci moří.‘ Jsi člověk, a ne Bůh, i když své srdce vydáváš za srdce božské!
#28:3 Hle, jsi moudřejší než Daniel, nic tajného se před tebou neutají.
#28:4 Svou moudrostí a rozumností ses domohl blahobytu, do svých pokladnic jsi získal zlato a stříbro.
#28:5 Množstvím své moudrosti a svými obchody jsi rozhojnil svůj blahobyt a pro ten tvůj blahobyt se tvé srdce stalo domýšlivým.“
#28:6 Proto praví Panovník Hospodin toto: „Protože své srdce vydáváš za srdce božské,
#28:7 hle, přivedu na tebe cizí, kruté pronárody; na krásu tvé moudrosti vytasí své meče, znesvětí tvou skvělost.
#28:8 Spustí tě do jámy, zemřeš smrtí skolených v srdci moří.
#28:9 Budeš ještě prohlašovat: ‚Jsem Bůh‘, před tím, kdo tě zabije? Jsi člověk, a ne Bůh, jsi v rukou těch, kdo tě skolí.
#28:10 Zemřeš smrtí neobřezanců, rukou cizáků. Já jsem promluvil, je výrok Panovníka Hospodina.“
#28:11 I stalo se ke mně slovo Hospodinovo:
#28:12 „Lidský synu, začni žalozpěv nad týrským králem. Zapěj o něm: Toto praví Panovník Hospodin: Byl jsi věrným obrazem pravzoru, plný moudrosti a dokonale krásný.
#28:13 Byl jsi v Edenu, zahradě Boží, ozdoben všemi drahokamy: rubínem, topasem, jaspisem, chrysolitem, karneolem, onyxem, safírem, malachitem a smaragdem. Tvé bubínky a flétny byly zhotoveny ze zlata; byly připraveny v den, kdy jsi byl stvořen.
#28:14 Byl jsi zářivý cherub ochránce, k tomu jsem tě určil, pobýval jsi na svaté hoře Boží, procházel ses uprostřed ohnivých kamenů,
#28:15 na svých cestách jsi byl bezúhonný ode dne svého stvoření, dokud se v tobě nenašla podlost.
#28:16 Pro množství tvých obchodů se tvé nitro naplnilo násilím a ty jsi zhřešil. I skolím tě, srazím z hory Boží, cherube ochránce, vyhladím tě zprostředka ohnivých kamenů.
#28:17 Pro tvou krásu se stalo tvé srdce domýšlivým, pro svou skvělost jsi zkazil svoji moudrost, svrhnu tě k zemi, dám tě za podívanou králům.
#28:18 Množstvím svých nepravostí, nepoctivostí svých obchodů jsi znesvětil své svatyně. Způsobím, že z tvého středu vyšlehne oheň a stráví tě, pohodím tě na zem jako popel před očima všech, kdo tě spatří.
#28:19 Všude mezi lidmi, kde tě znají, strnou nad tebou v úděsu, staneš se odstrašujícím příkladem, zanikneš navěky.“
#28:20 I stalo se ke mně slovo Hospodinovo:
#28:21 „Lidský synu, postav se proti Sidónu a prorokuj proti němu.
#28:22 Řekni: Toto praví Panovník Hospodin: Jsem proti tobě, Sidóne, a budu oslaven v tvém středu. I poznají, že já jsem Hospodin; až v něm vykonám soudy, budu v něm posvěcen.
#28:23 Sešlu na něj mor, krev do jeho ulic, budou uprostřed něho padat skolení, meč na něj dolehne ze všech stran. I poznají, že já jsem Hospodin.“
#28:24 „Dům izraelský už nebude mít zhoubný trn, trní sužující ze všech stran kolkolem ty, kdo jím pohrdají. I poznají, že já jsem Panovník Hospodin.“
#28:25 Toto praví Panovník Hospodin: „Shromáždím dům izraelský z národů, mezi kterými jsou rozptýleni, a budu mezi nimi posvěcen před očima pronárodů. Budou bydlet ve své zemi, kterou jsem dal Jákobovi, svému služebníku.
#28:26 Budou v ní bydlet bezpečně, postaví si domy, vysadí vinice, bezpečně budou bydlet, až vykonám soudy nade všemi sousedy kolem, kteří jimi pohrdají. I poznají, že já jsem Hospodin, jejich Bůh.“ 
#29:1 V desátém roce, dvanáctého dne desátého měsíce, stalo se ke mně slovo Hospodinovo:
#29:2 „Lidský synu, postav se proti faraónovi, králi egyptskému, a prorokuj proti němu i proti celému Egyptu.
#29:3 Mluv: Toto praví Panovník Hospodin: Jsem proti tobě, faraóne, králi egyptský, ohromný draku, který odpočíváš uprostřed svých toků a říkáš: ‚Ta řeka je má, já jsem si ji udělal.‘
#29:4 Vetknu do tvých čelistí háky a způsobím, že ryby tvých toků přilnou k tvým šupinám, vytáhnu tě z tvých toků i se všemi rybami tvých toků, jež přilnuly k tvým šupinám.
#29:5 Pohodím tě na poušti, tebe i všechny ryby tvých toků, dopadneš na povrch pole, nebudeš sebrán a nebudeš shromážděn, dám tě za pokrm zemské zvěři a nebeskému ptactvu.
#29:6 I poznají všichni obyvatelé Egypta, že já jsem Hospodin. To proto, že byli domu izraelskému třtinovou oporou.
#29:7 Když se tě zachytili rukama, zlomil ses a rozerval jsi jim celé rámě, když se o tebe opřeli, roztříštil ses, a ochromils jim celá bedra.“
#29:8 Proto praví Panovník Hospodin toto: „Hle, uvedu na tebe meč a vymýtím z tebe lidi i dobytek.
#29:9 Egyptská země bude zpustošena a obrácena v trosky. I poznají, že já jsem Hospodin. To proto, že farao řekl: ‚Ta řeka je má, já jsem si ji udělal‘.
#29:10 Chystám se na tebe i na tvé toky. Obrátím egyptskou zemi v naprosté trosky, v zpustošený kraj od Migdólu po Sevénu a až k pomezí kúšskému.
#29:11 Neprojde jí lidská noha, neprojde jí ani noha zvířete, po čtyřicet let nebude obývána.
#29:12 Obrátím egyptskou zemi v zpustošený kraj, stane se jednou ze zpustošených zemí a její města budou patřit mezi města v troskách, budou zpustošena po čtyřicet let. Egypťany rozptýlím mezi pronárody a rozpráším je po zemích.“
#29:13 Toto praví Panovník Hospodin: „Po uplynutí čtyřiceti let shromáždím Egypťany z národů, kam byli rozptýleni.
#29:14 Změním úděl Egypťanů a přivedu je zpět do země Patrósu, do země jejich původu, ale budou tam královstvím poníženým.
#29:15 Budou poníženější než jiná království a nebudou se už vypínat nad pronárody, ztenčím jejich řady, aby nad pronárody nepanovali.
#29:16 Už v ně nebude dům izraelský doufat a připomínat si nepravost tím, že by se za nimi obracel. I poznají, že já jsem Panovník Hospodin.“
#29:17 Ve dvacátém sedmém roce, prvního dne prvního měsíce, stalo se ke mně slovo Hospodinovo:
#29:18 „Lidský synu, Nebúkadnesar, král babylónský, uložil svému vojsku těžkou otrockou službu proti Týru. Každá hlava je odřená a každé rameno rozedřené. Avšak mzdu za tu otrockou službu, kterou proti Týru konal, nemá ani on ani jeho vojsko.
#29:19 Proto praví Panovník Hospodin toto: Hle, dávám Nebúkadnesarovi, králi babylónskému, egyptskou zemi, aby odnesl její majetek, aby ukořistil, co se ukořistit dá, aby uloupil, co se uloupit dá; to bude mzdou pro jeho vojsko.
#29:20 Jako výdělek za jeho službu mu dám egyptskou zemi, protože to učinil pro mne, je výrok Panovníka Hospodina.
#29:21 V onen den způsobím, že vyraší roh domu izraelského a tobě dám otevřeně promluvit mezi nimi. I poznají, že já jsem Hospodin.“ 
#30:1 I stalo se ke mně slovo Hospodinovo:
#30:2 „Lidský synu, prorokuj a řekni: Toto praví Panovník Hospodin: Kvílejte, ach, ten den!
#30:3 Neboť blízko je den, blízko je den Hospodinův; den oblaku, čas pronárodů.
#30:4 Na Egypt dolehne meč, smrtelná úzkost zachvátí Kúš, až budou padat skolení v Egyptě. Vezmou mu jeho majetek, bude zbořen do základů.“
#30:5 Kúš, Pút, Lúd a všechen přimíšený lid i Kúb i synové země smlouvy s nimi padnou mečem.
#30:6 Toto praví Hospodin: „Padnou ti, kteří jsou Egyptu oporou, klesne jeho pyšná moc, padnou v něm mečem od Migdólu po Sevénu, je výrok Panovníka Hospodina.
#30:7 Bude zpustošen, stane se jednou ze zpustošených zemí a jeho města budou patřit mezi města v troskách.
#30:8 I poznají, že já jsem Hospodin, až založím v Egyptě požár, až budou všichni jeho pomocníci rozdrceni.
#30:9 V onen den vyplují ode mne na lodích poslové, aby vyděsili sebejistý Kúš. V den Egypta je zachvátí smrtelná úzkost; hle, už přichází.“
#30:10 Toto praví Panovník Hospodin: „Učiním přítrž egyptskému hlučícímu davu skrze Nebúkadnesara, krále babylónského.
#30:11 On a jeho lid s ním, kruté pronárody, budou přivedeni, aby přinesli zemi zkázu. Vytasí na Egypt své meče a naplní zemi skolenými.
#30:12 Proměním nilská ramena v suchou zemi, vydám tu zemi do rukou zlovolníkům, rukou cizáků zpustoším zemi se vším, co je na ní; já Hospodin jsem promluvil.“
#30:13 Toto praví Panovník Hospodin: „Zničím hnusné modly, odklidím z Memfidy bůžky, z egyptské země nevzejde už žádný kníže. Sešlu na egyptskou zemi bázeň.
#30:14 Zpustoším Patrós, založím požár v Sóanu, vykonám soudy v Thébách,
#30:15 vyleji své rozhořčení na Sín, záštitu Egypta, vyhladím thébský hlučící dav.
#30:16 Založím v Egyptě požár, Sín se bude svíjet v křeči, Théby budou rozpolceny, Memfidu sevřou protivníci za jasného dne.
#30:17 Jinoši Ávenu a Pí-besetu padnou mečem a dívky půjdou do zajetí.
#30:18 V Tachpanchésu potemní den, až tam bude zlomeno jařmo Egypta a bude učiněna přítrž jeho pyšné moci. Přikryje ho mračno a jeho dcery půjdou do zajetí.
#30:19 Vykonám na Egyptu soudy. I poznají, že já jsem Hospodin.“
#30:20 V jedenáctém roce, sedmého dne prvního měsíce, stalo se ke mně slovo Hospodinovo:
#30:21 „Lidský synu, zlomil jsem paži faraóna, krále egyptského, a nebudou přiloženy léky a nebude připevněna dlaha, nebude ošetřena, aby se jí nevrátila síla a nechopila se meče.“
#30:22 Proto praví Panovník Hospodin toto: „Chystám se na faraóna, krále egyptského; přerazím mu obě paže, silnou i zlomenou, a nechám vypadnout meč z jeho ruky.
#30:23 Rozptýlím Egypťany mezi pronárody a rozpráším je po zemích.
#30:24 Posílím paže krále babylónského a do ruky mu vložím svůj meč, ale paže faraóna zlomím; i bude před ním sténat jako sténá probodený.
#30:25 Posílím paže krále babylónského, ale paže faraónovy poklesnou. I poznají, že já jsem Hospodin, až vložím svůj meč do ruky babylónského krále a on jej napřáhne na egyptskou zemi.
#30:26 Rozptýlím Egypťany mezi pronárody a rozpráším je po zemích. I poznají, že já jsem Hospodin.“ 
#31:1 Jedenáctého roku v třetím měsíci, prvního dne toho měsíce, stalo se ke mně slovo Hospodinovo:
#31:2 „Lidský synu, řekni faraónovi, králi egyptskému, i jeho hlučícímu davu: Komu se podobáš ve své velikosti?
#31:3 Hle, Ašúr byl jako cedr na Libanónu, krásně rozvětvený, vrhající hluboký stín, vysokého vzrůstu, jeho vrcholek byl mezi košatými korunami.
#31:4 Vody mu daly velikost, propastná tůň ho vyhnala do výše, její proudy protékaly kolem místa, kde byl vysazen; své potůčky vysílala ke všem stromům pole.
#31:5 Proto se svým vzrůstem vynášel nad všechny stromy pole, jeho větve se rozrostly, ratolesti se prodloužily díky hojným vodám, když vyrůstal.
#31:6 V jeho větvích hnízdilo kdejaké nebeské ptactvo, pod jeho ratolestmi se rodila kdejaká polní zvěř, v jeho stínu sídlily kdejaké početné pronárody.
#31:7 Byl krásný svou velikostí a délkou svého větvoví, neboť zakořenil u hojných vod.
#31:8 Nepřesahovaly ho cedry v zahradě Boží, cypřiše se nevyrovnaly jeho větvím, platany nebyly jako jeho ratolesti. Žádný strom v zahradě Boží se mu krásou nevyrovnal.
#31:9 Krásným jsem jej učinil množstvím jeho větvoví, záviděly mu všechny stromy Edenu, které byly v zahradě Boží.“
#31:10 Proto praví Panovník Hospodin toto: „Protože se vynáší svým vzrůstem, vyhnal totiž svůj vrcholek mezi košaté koruny a pro jeho výšku se pozdvihlo jeho srdce,
#31:11 vydám jej do rukou samovládce nad pronárody a ten s ním naloží podle své zvůle; zapudím ho.
#31:12 Cizáci, nejkrutější z pronárodů, ho vytnou a odvrhnou. Jeho větvoví padne na hory a do všech údolí. Jeho ratolesti se roztříští a padnou do všech řečišť země. Všechny národy země vystoupí z jeho stínu a odvrhnou ho.
#31:13 Až padne, usadí se na něm kdejaké nebeské ptactvo, mezi jeho ratolestmi bude kdejaká polní zvěř.
#31:14 Aby se žádný strom u vod nevynášel svým vzrůstem a nevyháněl svůj vrcholek mezi košaté koruny, aby se žádný mohutný strom nestavěl svou výškou nad ostatní stromy napájené vodami, budou všichni vydáni smrti, do nejhlubších útrob země, mezi lidské syny, k těm, kdo sestupují do jámy.“
#31:15 Toto praví Panovník Hospodin: „V den, kdy sestoupí do podsvětí, zavřu nad ním propastnou tůň a způsobím, aby truchlila, zadržím její proudy, hojné vody budou zastaveny. Zármutkem nad ním sklíčím Libanón a všechny stromy pole pro něho povadnou.
#31:16 Hřmotem jeho pádu otřesu pronárody, až jej srazím do podsvětí s těmi, kdo sestupují do jámy. V nejhlubších útrobách země se potěší všechny stromy Edenu, vše výborné a líbezné, co bylo na Libanónu, co bývalo napájeno vodou.
#31:17 Také ti, kdo byli jeho paží, ti, kdo přebývali v jeho stínu mezi pronárody, sestoupí spolu s ním do podsvětí ke skoleným mečem.
#31:18 Kterému ze stromů Edenu ses podobal slávou i velikostí? Se stromy Edenu budeš sražen do nejhlubších útrob země, mezi neobřezance, budeš ležet mezi skolenými mečem. Tak dopadne farao i celý jeho hlučící dav, je výrok Panovníka Hospodina.“ 
#32:1 Dvanáctého roku ve dvanáctém měsíci, prvního dne toho měsíce, stalo se ke mně slovo Hospodinovo:
#32:2 „Lidský synu, začni žalozpěv nad faraónem, králem egyptským. Zapěj o něm: K mladému lvu mezi pronárody jsi byl přirovnán, byl jsi jako drak v mořích, prodíral ses svými proudy, nohama jsi kalil vodu a čeřil její proudy.“
#32:3 Toto praví Panovník Hospodin: „Rozestřu na tebe svou síť společenstvím četných národů a vytáhnou tě mým nevodem.
#32:4 Vyvrhnu tě na zem, pohodím tě na povrch pole, usadím na tobě kdejaké nebeské ptactvo, nasytím tebou zvěř celé země.
#32:5 Rozházím tvé tělo po horách, hromadami tvého masa naplním údolí.
#32:6 Proudem tvé krve napojím zemi až k horám, řečiště se naplní krví z tebe.
#32:7 Až budeš zhasínat, zakryji nebesa, zatemním jejich hvězdy, slunce zakryji oblakem a měsíc nevydá světlo.
#32:8 Zatemním nad tebou na nebi všechna jasná světla, na tvou zemi sešlu tmu, je výrok Panovníka Hospodina.
#32:9 Srdce četných národů naplním hořem, až uvedu tvou zkázu ve známost mezi pronárody, v zemích, které neznáš.
#32:10 Úděsem nad tebou naplním četné národy, jejich králové strnou nad tebou hrůzou, až proti nim zamávám svým mečem, všichni se budou neustále třást o svůj život v den tvého pádu.“
#32:11 Toto praví Panovník Hospodin: „Dolehne na tebe meč babylónského krále;
#32:12 meči bohatýrů srazím tvůj hlučící dav, meči všech krutých pronárodů. Ti svrhnou v záhubu pýchu Egypta; vyhlazen bude všechen jeho hlučící dav.
#32:13 Vyhubím všechna jeho zvířata u hojných vod, nezkalí je už lidská noha, nezkalí je ani kopyta zvířat.
#32:14 Tehdy čirými učiním jejich vody a jejich proudy potečou jako olej, je výrok Panovníka Hospodina.
#32:15 Egyptskou zemi učiním zpustošeným krajem, zpustošena bude země se vším, co je na ní, až v ní pobiji všechny obyvatele. I poznají, že já jsem Hospodin.“
#32:16 „To je žalozpěv a jako žalozpěv se bude zpívat, jako žalozpěv jej budou zpívat dcery pronárodů, žalozpěv nad Egyptem a vším jeho hlučícím davem, je výrok Panovníka Hospodina.“
#32:17 Dvanáctého roku, patnáctého dne téhož měsíce, stalo se ke mně slovo Hospodinovo:
#32:18 „Lidský synu, běduj nad hlučícím davem Egypta, sraz jej dolů, jej i dcery pronárodů i jejich vznešené do nejhlubších útrob země s těmi, kdo sestupují do jámy.
#32:19 Jsi snad milejší než kdokoli jiný? Sestup, budeš uložen k neobřezancům.“
#32:20 Padnou mezi skolené mečem; meč je už předán. Odtáhněte jej i všechen jeho hlučící dav.
#32:21 Zprostředka podsvětí nejsilnější bohatýři budou mluvit k němu a jeho pomahačům: ‚Sestoupili, ulehli neobřezanci skolení mečem.‘
#32:22 Tam je Asýrie a celé její shromáždění; kolem krále jsou hroby všech skolených, padlých mečem.
#32:23 Vykázali jí hroby v nejhlubší jámě; kolem jejího hrobu je i její shromáždění, všichni skolení, padlí mečem, ti, kteří šířili děs v zemi živých.
#32:24 Tam je Élam, kolem jeho hrobu všechen jeho hlučící dav, všichni skolení, padlí mečem, neobřezanci, kteří sestoupili do nejhlubších útrob země; šířili svůj děs v zemi živých, nesou hanbu s těmi, kdo sestupují do jámy.
#32:25 Dali mu lože mezi skolenými, se vším jeho hlučícím davem. Kolem krále jsou hroby všech jeho neobřezanců, skolených mečem; šířili svůj děs v zemi živých, nesou hanbu s těmi, kdo sestupují do jámy. Vykázali mu hrob mezi skolenými.
#32:26 Tam je Mešek-Túbal a všechen jejich hlučící dav; kolem krále jsou hroby všech neobřezanců, skolených mečem; šířili svůj děs v zemi živých.
#32:27 Nebudou ležet s bohatýry, padli jako neobřezanci. Oni sestoupili do podsvětí se svou válečnou zbrojí a pod hlavy jim byly dány jejich meče. Jejich nepravosti však lpí na jejich kostech, neboť děs oněch bohatýrů byl v zemi živých. -
#32:28 I ty, faraóne, budeš zlomen, svržen mezi neobřezance, ulehneš ke skoleným mečem.
#32:29 Tam je Edóm, jeho králové a všechna jeho knížata, i se svou bohatýrskou silou; jsou uloženi ke skoleným mečem, ulehli k neobřezancům, k těm, kdo sestupují do jámy.
#32:30 Tam jsou všichni vůdcové severu a všichni Sidóňané, sestoupili ke skoleným. Pro děs, který šel z jejich bohatýrské síly, jsou zahanbeni; ulehli, neobřezanci, ke skoleným mečem a nesou hanbu s těmi, kdo sestupují do jámy.
#32:31 Farao je spatří a potěší se vším tím hlučícím davem. Jsou skoleni mečem, farao i celé jeho vojsko, je výrok Panovníka Hospodina;
#32:32 šířil jsem jím svůj děs v zemi živých. Uložen je farao i všechen jeho hlučící dav mezi neobřezance ke skoleným mečem, je výrok Panovníka Hospodina.“ 
#33:1 I stalo se ke mně slovo Hospodinovo:
#33:2 „Lidský synu, mluv k synům svého lidu a řekni jim: Uvedu-li na některou zemi meč, lid země si vybere ze sebe jednoho muže a ustanoví jej strážcem.
#33:3 Když spatří, že na zemi přichází meč, zaduje na polnici a bude varovat lid.
#33:4 Uslyší-li kdo hlas polnice, ale nedá se varovat a meč přijde a zachvátí jej, jeho krev padne na jeho hlavu.
#33:5 Slyšel hlas polnice, ale nedal se varovat, proto jeho krev padne na něj; kdyby se byl dal varovat, byl by se zachránil.
#33:6 Jestliže však strážce uvidí meč přicházet, ale nezaduje na polnici a lid nebude varován a meč přijde a zachvátí někoho z nich, ten bude pro svou nepravost zachvácen, ale strážného budu volat za jeho krev k odpovědnosti.“
#33:7 „Ty, lidský synu, slyš. Ustanovuji tě strážcem izraelského domu. Uslyšíš-li z mých úst slovo, vyřídíš jim mé varování.
#33:8 Řeknu-li o svévolníkovi: ‚Svévolníku, zemřeš,‘ a ty bys nepromluvil a nevaroval ho před jeho cestou, ten svévolník zemře za svou nepravost, ale za jeho krev budu volat k odpovědnosti tebe.
#33:9 Jestliže budeš svévolníka varovat před jeho cestou, aby se od ní odvrátil, ale on se od své cesty neodvrátí, zemře pro svou nepravost, ale ty jsi svou duši vysvobodil.“
#33:10 „Ty, lidský synu, slyš. Řekni izraelskému domu: Toto říkáváte: ‚Lpí na nás naše nevěrnosti a naše hříchy a my pro ně zahyneme. Jak budeme moci žít?‘
#33:11 Řekni jim: Jakože jsem živ, je výrok Panovníka Hospodina, nechci, aby svévolník zemřel, ale aby se odvrátil od své cesty a byl živ. Odvraťte se, odvraťte se od svých zlých cest! Proč byste měli zemřít, dome izraelský?“
#33:12 „Ty, lidský synu, slyš. Řekni synům svého lidu: Spravedlnost nevysvobodí spravedlivého v den jeho nevěrnosti, svévole nezpůsobí pád svévolníka v den, kdy se odvrátí od své svévole. Spravedlivý nebude moci kvůli ní zůstat naživu v den, kdy zhřeší.
#33:13 Řeknu-li o spravedlivém, že bude žít, ale on se začne spoléhat na svou spravedlnost a dopouštět se bezpráví, nebude se připomínat žádný z jeho spravedlivých činů, ale zemře pro bezpráví, kterého se dopustil.
#33:14 Řeknu-li o svévolníkovi: ‚Zemřeš!‘, ale on se od svého hříchu odvrátí a bude jednat pole práva a spravedlnosti,
#33:15 vrátí zástavu, nahradí, oč koho odral, bude se řídit nařízeními vedoucími k životu a nebude se dopouštět bezpráví, bude určitě žít a nezemře.
#33:16 Žádný z jeho hříchů, kterých se dopustil, se mu nebude připomínat, začal jednat podle práva a spravedlnosti, jistě bude žít.“
#33:17 „Synové tvého lidu říkají: ‚Panovníkova cesta není správná.‘ Nejsou však správné jejich cesty.
#33:18 Odvrátí-li se spravedlivý od své spravedlnosti a dopustí se bezpráví, zemře pro ně.
#33:19 Odvrátí-li se svévolník od svévole a bude jednat podle práva a spravedlnosti, díky tomu bude žít.
#33:20 Vy však říkáte: ‚Panovníkova cesta není správná.‘ Každého z vás, dome izraelský, budu soudit podle jeho cest!“
#33:21 Dvanáctého roku našeho přesídlení, v desátém měsíci, pátého dne toho měsíce, přišel ke mně jeden z těch, kdo vyvázli z Jeruzaléma, se slovy: „Město je vybito!“
#33:22 Večer předtím, než přišel ten, jenž vyvázl, spočinula na mně ruka Hospodinova. Hospodin mi otevřel ústa předtím, než ke mně ráno přišel ten, jenž vyvázl. Má ústa se otevřela a už jsem nebyl němý.
#33:23 I stalo se ke mně slovo Hospodinovo:
#33:24 „Lidský synu, obyvatelé oněch trosek v izraelské zemi říkají: ‚Abrahám byl samojediný a obdržel zemi. Nás je mnoho. Tato země byla dána do vlastnictví nám.‘
#33:25 Proto jim řekni: Toto praví Panovník Hospodin: Protože jíte maso s krví, zvedáte oči ke svým hnusným modlám a proléváte krev, byste měli obdržet tuto zemi?
#33:26 Opíráte se o svůj meč, dopouštíte se ohavnosti, poskvrňujete každý ženu svého bližního, a měli byste obdržet zemi?“
#33:27 „Toto jim řekni: Toto praví Panovník Hospodin: Jakože jsem živ, ti, kteří jsou na troskách, padnou mečem. Ty, kteří jsou na poli, vydán za potravu zvěři. A ti, kteří jsou ve skalních skrýších a slujích, pomřou morem.
#33:28 Obrátím zemi ve zpustošený a úděs budící kraj, ustane její pyšná moc. Izraelské hory budou zpustošeny, nikdo jimi nebude procházet.
#33:29 I poznají, že já jsem Hospodin, až obrátím zemi ve zpustošený a úděs budící kraj pro všechny jejich ohavnosti, jichž se dopouštěli.“
#33:30 „Ty, lidský synu, slyš. Synové tvého lidu si o tobě povídají u zdí a u vchodů do domů a mluví mezi sebou, jeden s druhým: ‚Pojďte a poslechněte, jaké slovo vychází od Hospodina!‘
#33:31 Přicházejí k tobě, jako když se schází lid k rokování, a sedají si před tebou jako můj lid. Poslouchají tvá slova, ale podle nich nejednají. V ústech mají horoucí slova o tom, co udělají, ale jejich srdce tíhne za jejich mrzkým ziskem.
#33:32 Tys pro ně jako ten, kdo horoucně a krásně zpívá a pěkně hraje. Poslouchají tvá slova, ale vůbec podle nich nejednají.
#33:33 Až to přijde, a už to nadchází, poznají, že byl mezi nimi prorok.“ 
#34:1 I stalo se ke mně slovo Hospodinovo:
#34:2 „Lidský synu, prorokuj proti pastýřům Izraele. Prorokuj a řekni těm pastýřům: Toto praví Panovník Hospodin: Běda pastýřům Izraele, kteří pasou sami sebe. Což pastýři nemají pást ovce?
#34:3 Pojídáte tuk, oblékáte se vlnou, porážíte vykrmené, ale ovce nepasete.
#34:4 Neduživé jste neposílili, nemocné jste neléčili, polámanou jste neovázali, zaběhlou jste nepřivedli nazpět, po ztracené jste nepátrali, panovali jste nad nimi násilně a surově.
#34:5 Jsou rozptýlené, jsou bez pastýře; staly se potravou veškeré polní zvěři a zůstávají rozptýleny.
#34:6 Mé ovce bloudí všude po horách, po kdejakém vysokém pahorku, jsou rozptýleny po celé zemi a není, kdo by je hledal, kdo by po nich pátral.
#34:7 Slyšte tedy, pastýři, slovo Hospodinovo:
#34:8 Jakože jsem živ, je výrok Panovníka Hospodina, mé ovce jsou loupeny a stávají se potravou veškeré polní zvěři, protože nemají pastýře a moji pastýři mé ovce nehledají; pasou sami sebe, ale mé ovce nepasou.
#34:9 Proto, pastýři, slyšte slovo Hospodinovo!
#34:10 Toto praví Panovník Hospodin: Hle, chystám se na ty pastýře, budu je volat k odpovědnosti za své ovce. Nedovolím jim už pást ovce, aby místo nich pásli sami sebe. Vysvobodím své ovce z jejich chřtánu, nebudou jim potravou.“
#34:11 Toto praví Panovník Hospodin: „Hle, já sám vyhledám své ovce a budu o ně pečovat.
#34:12 Tak jako pastýř pečuje o své stádo, když je uprostřed svěřených ovcí, tak budu pečovat o své ovce a vysvobodím je ze všech míst, kam byly rozptýleny v den oblaku a mrákoty.
#34:13 Vyvedu je z národů, shromáždím je ze zemí a přivedu je do jejich země. Budu je pást na izraelských horách, při potocích a na všech sídlištích v zemi.
#34:14 Budu je pást na dobré pastvě; jejich pastviny budou na výšinách izraelských hor. Budou odpočívat na dobrých pastvinách, budou se pást na tučné pastvě na horách izraelských.
#34:15 Sám budu pást své ovce a dám jim odpočívat, je výrok Panovníka Hospodina.
#34:16 Ztracenou vypátrám, zaběhlou přivedu zpět, polámanou ovážu a nemocnou posílím, kdežto tučnou a silnou zahladím. Budu je pást a soudit.“
#34:17 „Pokud jde o vás, mé ovce, toto praví Panovník Hospodin: Hle, já vykonám soud mezi ovcí a ovcí, mezi berany a kozly.
#34:18 Což je vám málo vypásat nejlepší pastvu? Proč zbytek pastvy zašlapáváte nohama? Pijete nejčistší vodu; proč tu ostatní nohama kalíte?
#34:19 A moje ovce se mají pást na tom, co jste nohama zašlapali, a pít to, co jste nohama zkalili?
#34:20 Proto o nich praví Panovník Hospodin toto: Hle, já vykonám soud mezí ovcí vykrmenou a ovcí hubenou,
#34:21 protože odstrkujete bokem a plecemi všechna neduživá zvířata a trkáte je svými rohy, takže jste je rozptýlili mimo stádo.
#34:22 Já zachráním své ovce a nikdo je už nebude loupit. Já vykonám soud mezi ovcí a ovcí.
#34:23 Ustanovím nad nimi jednoho pastýře, který je bude pást, Davida, svého služebníka. Ten je bude pást a ten bude jejich pastýřem.
#34:24 Já Hospodin jim budu Bohem a David, můj služebník, bude uprostřed nich knížetem. Já Hospodin jsem promluvil.
#34:25 Uzavřu s nimi smlouvu pokoje a odstraním ze země dravou zvěř, takže i na poušti budou moci bezpečně sídlit a spát v divočině.
#34:26 Obdařím je i okolí svého pahorku požehnáním a v pravý čas sešlu vydatný déšť; budou to deště požehnání.
#34:27 Polní stromoví vydá své ovoce, země vydá svou úrodu a budou na své roli v bezpečí. I poznají, že já jsem Hospodin, až rozlámu břevno jejich jha a vysvobodím je z rukou těch, kteří je zotročují.
#34:28 Už se nestanou loupeží pronárodů a zemská zvěř je nebude požírat, ale budou sídlit v bezpečí a nikdo je nevyděsí.
#34:29 Dám jim, aby byli sadbou k slávě mého jména. Nebudou už v zemi hynout hladem, už nebudou snášet hanbu od pronárodů.
#34:30 I poznají, že já Hospodin, jejich Bůh, jsem s nimi a oni, dům izraelský, jsou mým lidem, je výrok Panovníka Hospodina.
#34:31 Budete mými ovcemi, ovcemi, které já pasu, vy lidé, a já vám budu Bohem, je výrok Panovníka Hospodina.“ 
#35:1 I stalo se ke mně slovo Hospodinovo:
#35:2 „Lidský synu, postav se proti Seírskému pohoří a prorokuj proti němu.
#35:3 Řekni: Toto praví Panovník Hospodin: Chystám se na tebe, Seírské pohoří. Napřáhnu na tebe ruku a obrátím tě ve zpustošený a úděs budící kraj.
#35:4 Z tvých měst nadělám trosky, staneš se zpustošeným krajem. I poznáš, že já jsem Hospodin.
#35:5 Chováš odvěké nepřátelství, vydalo jsi syny izraelské napospas meči v čase jejich běd, v čase, kdy svou nepravostí přivodili konec.
#35:6 Proto, jakože jsem živ, je výrok Panovníka Hospodina, připravím ti krveprolití a krev tě bude pronásledovat; jelikož jsi nemělo v nenávisti krev, krev tě bude pronásledovat.
#35:7 Seírské pohoří obrátím v úděsně zpustošený kraj a vymýtím odtud každého, kdo tudy bude procházet tam či zpět.
#35:8 Jeho hory naplním skolenými; mečem skolení padnou na tvých pahorcích, v tvých údolích a u všech tvých potoků.
#35:9 Obrátím tě ve věčně zpustošený kraj, tvá města už nebudou obývána. I poznáte, že já jsem Hospodin.
#35:10 Protože jsi říkalo: ‚Oba tyto pronárody a obě země budou moje a obsadíme je, i když je tam Hospodin‘,
#35:11 proto, jakože jsem živ, je výrok Panovníka Hospodina, budu jednat podle tvého hněvu a tvé závisti, jak jsi jednalo ve své nenávisti vůči nim, a dám se mezi nimi poznat, až tě budu soudit.
#35:12 I poznáš, že já Hospodin jsem slyšel všechna tvá znevažování, která jsi vyřklo proti izraelským horám. Říkalo jsi: ‚Jsou zpustošeny, jsou nám vydány za pokrm!‘
#35:13 Vyvyšovali jste se nade mne svými ústy a zavalili jste mě svými slovy. Já jsem to slyšel.“
#35:14 Toto praví Panovník Hospodin: „Co bylo radostí celé země, bude zpustošeno; tak s tebou naložím.
#35:15 Jako ses radovalo nad tím, že dědictví domu izraelského bylo zpustošeno, tak učiním tobě. Budeš zpustošeno, Seírské pohoří, i celý Edóm se vším všudy. I poznají, že já jsem Hospodin.“ 
#36:1 „Ty, lidský synu, slyš. Prorokuj o izraelských horách. Řekni: Izraelské hory, slyšte Hospodinovo slovo!
#36:2 Toto praví Panovník Hospodin: Nepřítel o vás říká: ‚Dobře vám tak! I věčná návrší se stanou naším vlastnictvím.‘
#36:3 Proto prorokuj a řekni: Toto praví Panovník Hospodin: Právě za to, že vás pustošili a ze všech stran po vás šlapali, takže jste se staly vlastnictvím ostatních pronárodů, za to, že jste se dostaly do řečí a pomluv lidu,
#36:4 slyšte tedy, izraelské hory, slovo Panovníka Hospodina. Toto praví Panovník Hospodin horám a pahorkům, potokům a údolím, zpustošeným troskám a opuštěným městům, která se stala loupeží ostatních pronárodů kolem dokola a jsou jim pro smích.“
#36:5 Proto praví Panovník Hospodin toto: „Vskutku, v ohni svého rozhorlení jsem mluvil proti ostatním pronárodům, totiž proti celému Edómu, proti těm, kteří si s radostí, z celého srdce a bezostyšně zabrali do vlastnictví moji zemi a uloupili její pastviny.
#36:6 Proto prorokuj o izraelské zemi a řekni horám a pahorkům potokům a údolím: Toto praví Panovník Hospodin: Hle, mluvím ve svém rozhorlení a rozhořčení nad tím, že jste musely snášet hanbu od pronárodů.
#36:7 Proto praví Panovník Hospodin toto: Zvedám svou ruku k přísaze: Vskutku, pronárody okolo vás ponesou svou hanbu.
#36:8 A vy, izraelské hory, vyženete své větvoví a ponesete své plody mému izraelskému lidu; jeho příchod je blízko.
#36:9 Vždyť já jsem s vámi, obrátím se k vám a budete obdělávány a osívány.
#36:10 A rozmnožím na vás lidi, celý izraelský dům se vším všudy, města budou osídlena, a co je v troskách, bude vystavěno.
#36:11 Ano, rozmnožím na vás lidi i dobytek, rozmnoží a rozplodí se, osídlím vás jako za vašich dávných časů. Budu ještě štědřejší než ve vašich počátcích. I poznáte, že já jsem Hospodin.
#36:12 Uvedu na vás lidi, Izraele, svůj lid, a ten vás obsadí, stanete se jejich dědictvím a nebudete je už připravovat o děti.“
#36:13 Toto praví Panovník Hospodin: „Protože o vás říkají: ‚Ty, země, požíráš lidi a svůj vlastní národ připravuješ o děti‘,
#36:14 nuže, už nebudeš požírat lidi ani nepřivedeš svůj národ k pádu, je výrok Panovníka Hospodina.
#36:15 Nedovolím, aby bylo slyšet, že tě pronárody haní, nebudeš už snášet potupu od kdejakého lidu, nepřivedeš už svůj národ k pádu, je výrok Panovníka Hospodina.“
#36:16 I stalo se ke mně slovo Hospodinovo:
#36:17 „Lidský synu, když izraelský dům sídlil na své půdě, poskvrnili ji svou cestou a svými skutky. Jejich cesta přede mnou byla jako ženská nečistota.
#36:18 Vylil jsem na ně své rozhořčení za krev, kterou na zemi prolévali, a za to, že ji poskvrnili svými hnusnými modlami.
#36:19 Rozptýlil jsem je mezi pronárody, jsou roztroušeni po zemích, soudil jsem je podle jejich cest a skutků.
#36:20 Ale když přišli mezi pronárody, znesvěcovali mé svaté jméno, kamkoli přišli. Říkalo se o nich: ‚Je to lid Hospodinův, ale z jeho země museli odejít.‘
#36:21 I jala mě lítost pro mé svaté jméno, které oni, izraelský dům, znesvětili mezi pronárody, kamkoli přišli.
#36:22 Řekni proto izraelskému domu: Toto praví Panovník Hospodin: Nečiním to kvůli vám, izraelský dome, nýbrž kvůli svému svatému jménu, které jste znesvěcovali mezi pronárody, kamkoli jste přišli.
#36:23 Opět posvětím své veliké jméno, znesvěcené mezi pronárody, jméno, které jste vy uprostřed nich znesvětili. I poznají pronárody, že já jsem Hospodin, je výrok Panovníka Hospodina, až na vás ukáži před jejich očima svou svatost.
#36:24 Vezmu vás z pronárodů, shromáždím vás ze všech zemí a přivedu vás do vaší země.
#36:25 Pokropím vás čistou vodou a budete očištěni; očistím vás ode všech vašich nečistot a ode všech vašich hnusných model.
#36:26 A dám vám nové srdce a do nitra vám vložím nového ducha. Odstraním z vašeho těla srdce kamenné a dám vám srdce z masa.
#36:27 Vložím vám do nitra svého ducha; učiním, že se budete řídit mými nařízeními, zachovávat moje řády a jednat podle nich.
#36:28 Pak budete sídlit v zemi, kterou jsem dal vašim otcům, budete mým lidem a já vám budu Bohem.
#36:29 Zachráním vás ze všech vašich nečistot, přivolám obilí a rozhojním je a nedopustím na vás hlad.
#36:30 Rozhojním ovoce na stromech i polní úrodu, takže už neponesete mezi pronárody potupu kvůli hladu.
#36:31 Až si vzpomenete na své zlé cesty a na své nedobré skutky, budete se sami sobě ošklivit pro své nepravosti a ohavnosti.
#36:32 Nečiním to kvůli vám, je výrok Panovníka Hospodina, to vám buď známo; styď a hanbi se za své cesty, dome izraelský!“
#36:33 Toto praví Panovník Hospodin: „V den, kdy vás očistím ode všech vašich nepravostí, osídlím města, a co je v troskách, bude vystavěno.
#36:34 Zpustošená země bude obdělána, nebude už ležet zpustošená před zraky všech, kdo jdou kolem.
#36:35 Potom řeknou: ‚Tato zpustošená země je jako zahrada v Edenu: města ležící v troskách, zpustlá a zbořená, jsou nyní opevněna a osídlena.‘
#36:36 I poznají pronárody, které zůstanou vůkol, že já Hospodin jsem vystavěl, co bylo zbořeno, a osázel zpustošený kraj; já Hospodin jsem to vyhlásil i vykonal.
#36:37 Toto praví Panovník Hospodin: Opět budu izraelskému domu odpovídat na dotazy a prokážu jim toto: Rozmnožím lidi jako ovce.
#36:38 Bude jich jak ovcí posvěcených k obětem, jak ovcí v Jeruzalémě při slavnostních shromážděních; města obrácená v trosky naplní se lidmi jako ovcemi. I poznají, že já jsem Hospodin.“ 
#37:1 Spočinula na mně ruka Hospodinova. Hospodin mě svým duchem vyvedl a postavil doprostřed pláně, na níž bylo plno kostí,
#37:2 a provedl mě kolem nich. A hle, na té pláni bylo velice mnoho kostí a byly velice suché.
#37:3 I otázal se mne: „Lidský synu, mohou tyto kosti ožít?“ Odpověděl jsem: „Panovníku Hospodine, ty to víš.“
#37:4 Tu mi řekl: „Prorokuj nad těmi kostmi a řekni jim: „Slyšte, suché kosti, Hospodinovo slovo!
#37:5 Toto praví Panovník Hospodin těmto kostem: Hle, já do vás uvedu ducha a oživnete.
#37:6 Dám na vás šlachy, pokryji vás svalstvem, potáhnu vás kůží a vložím do vás ducha a oživnete. I poznáte, že já jsem Hospodin.“
#37:7 Prorokoval jsem tedy, jak mi bylo přikázáno. A zatímco jsem prorokoval, ozval se hluk, nastalo dunění a kosti se přibližovaly jedna ke druhé.
#37:8 Viděl jsem, jak je najednou pokryly šlachy a svaly a navrch se potáhly kůží, avšak duch v nich ještě nebyl.
#37:9 Tu mi řekl: „Prorokuj o duchu, lidský synu, prorokuj a řekni mu: Toto praví Panovník Hospodin: Přijď, duchu, od čtyř větrů a zaduj na tyto povražděné, ať ožijí!“
#37:10 Když jsem prorokoval, jak mi přikázal, vešel do nich duch a oni ožili. Postavili se na nohy a bylo to převelmi veliké vojsko.
#37:11 Potom mi řekl: „Lidský synu, tyto kosti, to je všechen dům izraelský. Hle, říkají: ‚Naše kosti uschly, zanikla naše naděje, jsme ztraceni.‘
#37:12 Proto prorokuj a řekni jim: Toto praví Panovník Hospodin: Hle, já otevřu vaše hroby a vyvedu vás z vašich hrobů, můj lide, a přivedu vás do izraelské země.
#37:13 I poznáte, že já jsem Hospodin, až otevřu vaše hroby a vyvedu vás z hrobů, můj lide.
#37:14 Vložím do vás svého ducha a oživnete. Dám vám odpočinutí ve vaší zemi. I poznáte, že já Hospodin jsem to vyhlásil i vykonal, je výrok Hospodinův.“
#37:15 I stalo se ke mně slovo Hospodinovo:
#37:16 „Ty, lidský synu, slyš. Vezmi si jedno dřevo a napiš na ně: ‚Judovi a synům izraelským, jeho spojencům.‘ Pak vezmi druhé dřevo a napiš na ně: ‚Josefovi‘, to je dřevo Efrajimovo a celého domu izraelského, jeho spojenců.
#37:17 Přilož jedno k druhému a budeš je mít za jedno dřevo; budou v tvé ruce jedním celkem.
#37:18 Až se tě synové tvého lidu zeptají: ‚Nepovíš nám, co to pro nás znamená?‘,
#37:19 promluv k nim: toto praví Panovník Hospodin: ‚Hle, já vezmu dřevo Josefovo, které je v ruce Efrajimově a izraelských kmenů, jeho spojenců, přidám k němu dřevo Judovo a učiním je dřevem jedním a budou v mé ruce jedno.‘
#37:20 Až budou ta dřeva, na která jsi psal, v tvé ruce před jejich očima,
#37:21 promluvíš k nim: Toto praví Panovník Hospodin: Hle, já vezmu syny Izraele zprostřed pronárodů, kamkoli odešli, shromáždím je ze všech stran a přivedu je do jejich země.
#37:22 Učiním z nich jediný národ v zemi, na izraelských horách, a jediný král bude králem všech. Nebudou to už dva národy a nebudou už rozdělení na dvě království.
#37:23 Nebudou se už poskvrňovat svými hnusnými a ohyzdnými modlami ani žádnými svými nevěrnostmi. Zachráním je ze všech míst, kde sídlili a ve kterých hřešili, očistím je a budou mým lidem a já jim budu Bohem.
#37:24 David, můj služebník, bude nad nimi králem a jediným pastýřem všech, budou se řídit mými řády, budou zachovávat má nařízení a jednat podle nich.
#37:25 Budou sídlit v zemi, kterou jsem dal Jákobovi, svému služebníku, v níž sídlili vaši otcové. Budou v ní sídlit oni i jejich synové a synové jejich synů navěky a David, můj služebník, bude jejich knížetem navěky.
#37:26 Uzavřu s nimi smlouvu pokoje; bude to věčná smlouva s nimi, dám jim ji a rozmnožím je a v jejich středu zřídím svou svatyni navěky.
#37:27 Můj příbytek bude nad nimi a já jim budu Bohem a oni budou mým lidem.
#37:28 I poznají pronárody, že já Hospodin jsem ten, kdo posvěcuje Izraele, až má svatyně bude navěky v jejich středu.“ 
#38:1 I stalo se ke mně slovo Hospodinovo:
#38:2 „Lidský synu, postav se proti Gógovi v zemi Magógu, proti velkoknížeti Mešeku a Túbalu, a prorokuj proti němu.
#38:3 Řekni: Toto praví Panovník Hospodin: Chystám se na tebe, Gógu, velkokníže Mešeku a Túbalu!
#38:4 Odvedu tě, dám ti do čelistí háky a odvleču tebe i celé tvé vojsko, koně i jezdce, všechny dokonale vystrojené, velké shromáždění s pavézami a štíty, všechny, kdo vládnou mečem,
#38:5 Peršany, s nimi Kúšany a Pútejce, všechny se štíty a přilbami,
#38:6 Gómera a všechny jeho voje, Bét-togarmu na nejzazším severu a všechny jeho voje, mnoho národů s tebou.
#38:7 Připrav se, ano, buď připraven ty a všechny tvé sbory, které se kolem tebe shromáždily; budeš jejich strážcem.
#38:8 Před mnohými dny jsi byl určen k tomu, abys na sklonku let vtáhl do země, která se zatím zotaví po meči. Její lid, shromážděný z mnohých národů na hory izraelské, které ustavičně zůstávaly pusté, bude vyveden z národů a všichni budou bezpečně bydlet.
#38:9 Tu přitáhneš a přiženeš se jako bouře, jako oblak přikryješ zemi, ty i všechny tvé voje a mnoho národů s tebou.“
#38:10 Toto praví Panovník Hospodin: „V onen den ti vstoupí na srdce různé věci a pojmeš zlý úmysl.
#38:11 Řekneš si: ‚Potáhnu do země neopevněných osad, půjdu na ty, kdo sídlí v klidu a bezpečí, na všechny obyvatele, kteří nemají hradby ani závory ani vrata.
#38:12 Poberu kořist a uloupím lup.‘ Vztáhneš svou ruku na osídlená místa, která byla kdysi v troskách, a na lid shromážděný z pronárodů, který si hledí jen stád a zboží a sídlí na Pupku země.
#38:13 Šeba a Dedán, kupci z Taršíše a všechna jeho lvíčata ti řeknou: ‚Přicházíš, abys pobral kořist? Sebral jsi své sbory, abys uloupil lup? Abys odnesl stříbro a zlato, pobral stáda a zboží, abys ukořistil velikou kořist?‘
#38:14 Proto prorokuj, lidský synu, a řekni Gógovi: Toto praví Panovník Hospodin: Což se nedozvíš v onen den, že Izrael, můj lid, sídlí v bezpečí?
#38:15 Vyjdeš ze svého místa, z nejzazšího severu, ty i mnoho národů s tebou, všichni jezdci na koních, velké shromáždění, velké vojsko.
#38:16 Potáhneš na Izraele, můj lid a pokryješ zemi jako mrak. Na sklonku dnů tě přivedu na svou zemi, aby mě pronárody poznaly, až ukážu svou svatost před jejich očima na tobě, Gógu!“
#38:17 Toto praví Panovník Hospodin: „Což to nejsi ty, o kterém jsem mluvil v dávných dobách skrze izraelské proroky, své služebníky, kteří prorokovali v oněch dnech, po léta, že tě na ně přivedu?
#38:18 I stane se v onen den, v den, kdy přijde Góg na izraelskou zemi, je výrok Panovníka Hospodina, že se zvedne mé rozhořčení v hněvu
#38:19 a promluvím ve svém rozhorlení, v ohni své prchlivosti: Jistojistě přijde v onen den na izraelskou zemi veliké zemětřesení
#38:20 a budou se přede mnou třást mořské ryby i nebeské ptactvo, polní zvěř a všichni plazi, kteří se plazí po zemi, i každý člověk, který je na tváři země. Hory se zbortí, skalní srázy se zřítí a každá hradba se skácí na zem.
#38:21 Zavolám proti němu meč na všechny své hory, je výrok Panovníka Hospodina, meč jednoho se obrátí proti druhému.
#38:22 Vykonám nad ním soud morem a krví; spustím příval deště a kroupy jako kameny, oheň a síru na něho a na všechny jeho voje a na množství národů, které jsou s ním.
#38:23 Ukážu svou velikost a svatost a dám se poznat před očima mnohých pronárodů. I poznají, že já jsem Hospodin.“ 
#39:1 Ty, lidský synu, slyš. Prorokuj proti Gógovi a řekni: Toto praví Panovník Hospodin: Chystám se na tebe, Gógu, velkokníže Mešeku a Túbalu.
#39:2 Obrátím tě, povedu a odvleču tě z nejzazšího severu a přivedu tě na izraelské hory.
#39:3 Ale z levé ruky ti vyrazím luk a z pravé ruky ti vytrhnu šípy.
#39:4 Na izraelských horách padneš ty, všechny tvé voje i národy, které jsou s tebou. Vydám tě za pokrm dravému ptactvu, kdekterému okřídlenci i polní zvěři.
#39:5 Padneš na poli. Já jsem promluvil, je výrok Panovníka Hospodina.
#39:6 Pošlu také oheň na zemi Magóg a na sebejisté obyvatele ostrovů. I poznají, že já jsem Hospodin.
#39:7 Uprostřed svého izraelského lidu dám poznat své svaté jméno a už nedopustím, aby mé svaté jméno bylo znesvěcováno. I poznají pronárody, že já jsem Hospodin, Svatý v Izraeli.
#39:8 Hle, co mělo přijít, nastalo, je výrok Panovníka Hospodina, to je ten den, o němž jsem mluvil.
#39:9 Obyvatelé izraelských měst vyjdou a zapálí zbroj, štít i pavézu, luk i šípy, oštěp a kopí, a spálí je. Budou z nich rozdělávat oheň po sedm let.
#39:10 Nebudou nosit dříví z pole ani je rubat v lesích, protože budou rozdělávat oheň ze zbroje. Vezmou kořist těm, kdo ji vzali jim, a oloupí ty, kdo olupovali je, je výrok Panovníka Hospodina.“
#39:11 „Onoho dne se stane, že určím Gógovi místo pro hrob v Izraeli tam v Údolí nájezdníků na východ od moře. To uzavře nájezdníkům cestu. Tam pohřbí Góga i všechen jeho hlučící dav a nazvou je Údolí Gógova hlučícího davu.
#39:12 Po sedm měsíců je bude dům izraelský pohřbívat, aby očistil zemi.
#39:13 Bude je pohřbívat všechen lid země a získají si tím jméno. V ten den budu oslaven, je výrok Panovníka Hospodina.
#39:14 Mužové k tomu oddělení budou neustále procházet zemí a pohřbívat nájezdníky zbylé na tváři země, aby ji očistili; po sedm měsíců ji budou prohledávat.
#39:15 Budou procházet zemí křížem krážem. Kdo uvidí lidskou kost, postaví vedle ní znamení, než ji hrobaři pohřbí v Údolí Gógova hlučícího davu.
#39:16 Také jedno město bude mít jméno Hamóna (to je Hlučení). Tak očistí zemi.“
#39:17 „Ty, lidský synu, slyš. Toto praví Panovník Hospodin: Řekni ptactvu, každému okřídlenci a veškeré polní zvěři: Shromážděte se a přijďte, seberte se ze všech stran k mému obětnímu hodu, který vám připravuji, k velikému obětnímu hodu na izraelských horách. Budete jíst maso a pít krev.
#39:18 Budete jíst maso bohatýrů a pít krev knížat země. Ti všichni budou místo beranů, jehňat, kozlů, býčků a vykrmeného bášanského dobytka.
#39:19 Dosyta se najíte tuku a až do opojení se napijete krve z mého obětního hodu, který jsem vám připravil.
#39:20 Nasytíte se u mého stolu koni a jezdci, bohatýry a všemi bojovníky, je výrok Panovníka Hospodina.
#39:21 Dám poznat svou slávu mezi pronárody. Všechny pronárody spatří můj soud, který vykonám, a mou ruku, kterou na ně vložím.
#39:22 I pozná dům izraelský, že já Hospodin jsem jejich Bůh od onoho dne i potom.
#39:23 I poznají pronárody, že dům izraelský byl přesídlen pro svou nepravost. Protože se mi zpronevěřili, skryl jsem před nimi svou tvář a vydal je do rukou jejich nepřátel, takže všichni padli mečem.
#39:24 Jednal jsem s nimi podle jejich nečistoty a nevěrnosti. Skryl jsem před nimi svou tvář.“
#39:25 Proto praví Panovník Hospodin toto: „Nyní změním Jákobův úděl a smiluji se nad celým domem izraelským. Budu horlit pro své svaté jméno.
#39:26 Odpykali si svou hanbu a všechnu svou zpronevěru, které se vůči mně dopustili; budou bezpečně bydlet ve své zemi a nikdo je nevyděsí.
#39:27 Ukážu na nich svou svatost před očima mnohých pronárodů tím, že je přivedu zpět z národů a shromáždím ze zemí jejich nepřátel.
#39:28 I poznají, že já jsem Hospodin, jejich Bůh, který je přesídlil mezi pronárody. Já je shromáždím do jejich země. A neponechám tam už nikoho z nich.
#39:29 Svou tvář už před nimi nebudu skrývat, neboť jsem vylil na dům izraelský svého ducha, je výrok Panovníka Hospodina.“ 
#40:1 Desátého dne toho měsíce na začátku roku, dvacátého pátého roku našeho přesídlení a čtrnáctého roku po dobytí města, právě toho dne spočinula na mně ruka Hospodinova a uvedl mě tam.
#40:2 Uvedl mě ve vidění Božím do izraelské země a postavil mě na velmi vysokou horu, na které bylo z jižní strany zbudováno cosi jako město.
#40:3 Uvedl mě tam, a hle, byl tam muž, který vypadal, jako by byl z bronzu. Stál v bráně, v ruce měl lněnou šňůru a prut k měření.
#40:4 „Lidský synu“, oslovil mě ten muž, „pozorně se dívej a napjatě poslouchej a vezmi si k srdci všechno, co ti ukážu; kvůli tomu, co ti bude ukázáno, byl jsi sem přiveden. Oznam domu izraelskému všechno, co uvidíš.“
#40:5 A hle, zvenčí kolem dokola domu Božího byla hradba. Délka prutu k měření, který měl muž v ruce, byla šest loket; každý z nich byl o dlaň delší. Změřil šířku stavby; byla jeden prut a výška také jeden prut.
#40:6 Pak vstoupil do brány, která směřovala na východ. Vystoupil po schodišti a změřil práh brány; byl jeden prut široký. Práh byl široký jeden prut.
#40:7 Postranní místnosti v bráně byly jeden prut dlouhé a jeden prut široké. Prostor mezi místnostmi měřil pět loket. Práh dvorany v bráně směrem k domu měřil jeden prut.
#40:8 Změřil i dvoranu brány směrem k domu; měřila jeden prut.
#40:9 Pak změřil dvoranu brány s pilíři; měřila osm loket. Její pilíře měřily dva lokte. Dvoranu brány měřil směrem k domu.
#40:10 Brána směřující na východ měla po obou stranách tři postranní místnosti. Rozměry těch tří místností byly stejné a rozměry pilířů po obou stranách byly také stejné.
#40:11 Dále změřil vchod do brány; byl deset loket široký. Brána byla dlouhá třináct loket.
#40:12 Ohraničující zídka před postranními místnostmi z jedné i z druhé strany byla vysoká jeden loket. Každá místnost měřila šestkrát šest loket.
#40:13 Pak změřil bránu od jednoho kraje střechy nad místností k druhému kraji střechy; byla široká dvacet pět loket, od vchodu ke vchodu.
#40:14 Byly tam pilíře, šedesát loket. U pilíře začínalo nádvoří a prostíralo se kolem dokola brány.
#40:15 A od průčelí vstupu do brány až k průčelí dvorany brány z vnitřní strany bylo padesát loket.
#40:16 Kolem dokola brány byla okna zužující se dovnitř místností mezi pilíři. Tak tomu bylo i u dvoran; kolem dokola byla okna dovnitř a na pilířích byly palmy.
#40:17 Uvedl mě pak na vnější nádvoří. Tam byly kolem dokola vydlážděného nádvoří komory; bylo jich na tom dlážděném nádvoří třicet.
#40:18 Dlažba byla položena až ke stěně všech bran po celé jejich délce; byla to nižší dlažba.
#40:19 Potom změřil nádvoří; od průčelí dolní brány až k okraji vnitřního nádvoří bylo zvenčí široké sto loket na východ i na sever.
#40:20 Změřil také délku i šířku brány směřující na sever; vedla do vnějšího nádvoří.
#40:21 Na každé straně měla tři postranní místnosti, pilíře a dvoranu. Měřila jako brána první; byla dlouhá padesát loket a dvacet pět loket široká.
#40:22 Její okna, dvorana i palmy byly stejných rozměrů jako u brány směřující na východ. Vystupovalo se k ní po sedmi schodech. Dvorana byla z vnitřní strany.
#40:23 Brána do vnitřního nádvoří byla proti vnější bráně na severu i na východě. Od jedné brány k druhé naměřil sto loket.
#40:24 Pak mě zavedl k jihu. Tam byla brána směřující na jih. Změřil také její pilíře a dvoranu; měřily jako předchozí.
#40:25 Kolem dokola brány i dvorany byla okna jako okna bran předchozích. Brána byla padesát loket dlouhá a dvacet pět loket široká.
#40:26 Vedlo k ní sedm schodů. Dvorana byla z vnitřní strany. Také na jejích pilířích byly palmy, na každé straně jedna.
#40:27 Podobně u jižní brány vedoucí do vnitřního nádvoří. Od vnější brány k vnitřní bráně na jižní straně naměřil také sto loket.
#40:28 Pak mě zavedl do jižní brány k vnitřnímu nádvoří. Změřil jižní bránu; měřila stejně jako předchozí.
#40:29 Také její postranní místnosti, pilíře i dvorana měřily stejně jako předchozí. Kolem dokola brány i dvorany byla okna. Brána byla padesát loket dlouhá a dvacet pět loket široká.
#40:30 Kolem dokola byly dvorany dvacet pět loket dlouhé a pět loket široké.
#40:31 Dvorana brány směřovala k vnějšímu nádvoří. Na jejích pilířích byla palmy a vedlo k ní osm schodů.
#40:32 Pak mě zavedl k východní straně vnitřního nádvoří a změřil bránu; měřila stejně jako předchozí.
#40:33 I její místnosti, pilíře i dvorana měřily stejně jako předchozí. Kolem dokola brány i dvorany byla okna. Brána byla padesát loket dlouhá a dvacet pět loket široká.
#40:34 Její dvorana směřovala k vnějšímu nádvoří. Na pilířích byly palmy, na každé straně jedna. Vedlo k ní osm schodů.
#40:35 Pak mě zavedl k severní bráně. Naměřil stejně jako u předchozích, když změřil ji,
#40:36 její místnosti, pilíře i dvoranu i okna dolem dokola. Byla padesát loket dlouhá a dvacet pět loket široká.
#40:37 Její dvorana směřovala k vnějšímu nádvoří. Na pilířích byly po obou stranách palmy. Vedlo k ní osm schodů.
#40:38 Při pilířích každé brány byla komora s vchodem; tam se oplachovala zápalná oběť.
#40:39 A v dvoraně brány byly na obou stranách dva stoly; na nich se porážela zvířata k oběti zápalné, k oběti za hřích a k oběti za vinu.
#40:40 Zvenku při boční stěně severní brány, tam, kde se vystupuje ke vchodu, byly dva stoly a při protější boční stěně u dvorany té brány byly také dva stoly.
#40:41 Z obou stran u boční stěny stály čtyři stoly, dohromady osm stolů, na kterých se porážejí zvířata.
#40:42 Čtyři stoly k oběti zápalné byly z tesaných kamenů; každý byl jeden a půl lokte dlouhý, jeden a půl lokte široký a jeden loket vysoký. Na ně se kladlo nářadí k porážení zvířat pro oběť zápalnou a obětní hod.
#40:43 Uvnitř kolem dokola byly upevněny dvojité závěsné háky, na jednu dlaň. Na stoly se kladlo také maso obětního daru.
#40:44 Na vnější straně vnitřní brány byly komory zpěváků na vnitřním nádvoří; jedna při straně brány severní s průčelím k jihu, druhá při straně brány jižní s průčelím na sever.
#40:45 I řekl mi: „Ta komora s průčelím k jihu je určena kněžím, kteří drží stráž u domu,
#40:46 a komora s průčelím na sever je určena kněžím, kteří drží stráž u oltáře. Jsou to Sádokovci, jediní z Léviovců, kteří se smějí přiblížit k Hospodinu, aby mu přisluhovali.“
#40:47 Potom změřil nádvoří; bylo sto loket dlouhé a sto loket široké, tvořilo čtverec. Před domem stál oltář.
#40:48 Přivedl mě k chrámové předsíni a změřil pilíře předsíně z jedné i z druhé strany; měřily pět loket. Brána byla široká tři lokte z jedné i z druhé strany.
#40:49 Předsíň byla dvacet loket dlouhá a jedenáct loket široká a vystupovalo se k ní po schodech. Vedle pilířů stály sloupy, z každé strany jeden. 
#41:1 Přivedl mě k chrámu a změřil pilíře; byly široké šest loket z obou stran jako šířka stanu.
#41:2 Vchod byl široký deset loket a stěny po obou stranách vchodu po pěti loktech. Dále změřil délku chrámu, činila čtyřicet loket a šířka dvacet loket.
#41:3 Vstoupil dovnitř a změřil pilíř u vchodu; měřil dva lokte a průchod šest loket, šířka stěny vchodu sedm loket.
#41:4 V přední části chrámu naměřil dvacet loket délky a dvacet loket šířky. A řekl mi: „To je velesvatyně.“
#41:5 Změřil stěny domu, měřily šest loket. Ochoz domu byl široký čtyři lokte.
#41:6 Ochozy byly tři nad sebou po třiceti komůrkách. Pro ochozy byly po celém obvodu domu ve zdi výstupky pro nosníky, takže nosníky nezasahovaly do stěny domu.
#41:7 Stěna se rozšiřovala k jednotlivým ochozům stupeň za stupněm kolem dokola. Protože obklopovala stupňovitě dům po celém jeho obvodu, rozšiřoval se stupňovitě i dům. Ze spodního ochozu se vystupovalo k hornímu i prostřednímu.
#41:8 Po celém obvodu domu jsem viděl vyvýšenou plošinu. Základy ochozů měřily celý prut, to je šest dlouhých loket.
#41:9 Stěna z venkovní strany ochozu byla pět loket široká. Volný prostor mezi ochozy u domu
#41:10 a mezi komorami byl široký dvacet loket po celém obvodu domu.
#41:11 Z ochozu byl vchod do volného prostoru, jeden severním směrem a druhý jižním. Volný prostor zabíral po celém obvodu místo široké pět loket.
#41:12 Budova, která stála u západního okraje odděleného prostoru, byla sedmdesát loket široká. Stěna budovy byla po celém obvodu široká pět loket, délka budovy byla devadesát loket.
#41:13 Změřil dům; byl sto loket dlouhý. Oddělený prostor, budova a její stěny byly dlouhé také sto loket.
#41:14 Dům i s odděleným prostorem na východ byl sto loket široký.
#41:15 Změřil budovu i s pavlačemi na obou stranách při zadním okraji odděleného prostoru; byla dlouhá sto loket. Vnitřek chrámu i předsíně vedoucí na nádvoří,
#41:16 prahy i zúžená okna a pavlače byly ze tří stran naproti prahu obloženy dřevem kolem dokola. Obloženy byly stěny od země až k oknům, i okna byla pokryta.
#41:17 Obložení sahalo až nad vchod uvnitř i zevně domu. Na každé stěně souměrně po celém vnějším obvodu
#41:18 byli vyřezáni cherubové a palmy. Mezi dvěma cheruby byla vždy palma. Každý cherub měl dvě tváře:
#41:19 Tvář lidská směřovala k palmě na jedné straně a tvář lví k palmě na druhé straně. Tak to bylo kolem celého domu.
#41:20 Cherubové i palmy byli vyřezáni i na stěně chrámu od země až nad vchod.
#41:21 Chrám měl čtverhranné věřeje. Před svatyní bylo vidět
#41:22 dřevěný oltář tři lokte vysoký a dva lokte dlouhý, hranatý. Stěny po celé jeho délce byly dřevěné. Promluvil ke mně: „Toto je stůl, který je před Hospodinem.“
#41:23 Dvoje dveře vedly do chrámu i do svatyně.
#41:24 Dveře měly dvě křídla a obě křídla byla otočná. První i další dveře měly po dvou křídlech.
#41:25 Na dveřích chrámu byla řezba cherubů a palem stejných jako na stěnách. Zvenčí před předsíní bylo dřevěné přístřeší.
#41:26 Zúžená okna a palmy byly na stěnách předsíně na obou stranách i na ochozech domu a na přístřešcích. 
#42:1 Pak mě vedl severní cestou směrem k vnějšímu nádvoří a přivedl mě ke komorám, které byly na sever od odděleného prostoru při budově.
#42:2 Komory byly v délce sto loket s vchodem od severu a v šířce padesáti loket.
#42:3 V délce dvaceti loket při vnitřním nádvoří, proti dlažbě vnějšího nádvoří, byly pavlače ve třech řadách nad sebou.
#42:4 Přístup ke komorám byl široký deset loket. Dovnitř vedl chodník jeden loket široký. Vchody ke komorám byly na severu.
#42:5 Vrchní komory při budově byly užší než spodní a prostřední, protože pavlače zabíraly část místa.
#42:6 Byly postaveny ve třech poschodích bez podpěrných sloupů, jaké byly na nádvořích. Proto se stavba stupňovitě zužovala zezdola od země, od spodních komor přes prostřední.
#42:7 Venkovní zídka dlouhá padesát loket, souběžná s komorami, směřovala k vnějšímu nádvoří. Byla při přední straně komor.
#42:8 Komory obrácené k vnějšímu nádvoří byly dlouhé padesát loket, zatímco obrácené k chrámu byly dlouhé sto loket.
#42:9 Do spodní části komor vedl vchod z východní strany, kterým se do nich vcházelo z vnějšího nádvoří.
#42:10 Při zdi nádvoří, při odděleném prostoru směrem k východu, a při budově byly další komory.
#42:11 Před nimi byl chodník. Komory vypadaly stejně jako na severní straně; byly stejně dlouhé i široké, i se všemi jejich východy a vnitřním zařízením.
#42:12 Jako vchody ke komorám na jižní straně byl stejný hlavní vstup na začátku chodníku, chodníku vedoucího podél ochranné zdi od východu. Tudy se přicházelo.
#42:13 I řekl mi: „Komory na severní i jižní straně, které jsou při odděleném prostoru, jsou komory svaté; tam smějí jíst velesvaté dary jen kněží, kteří přistupují k Hospodinu. Tam také ukládají velesvaté dary k oběti přídavné, k oběti za hřích a k oběti za vinu, neboť je to svaté místo.
#42:14 Kněží, kteří tam vstoupí, než vyjdou ze svatého prostoru do vnějšího nádvoří, odloží tam svá roucha, v nichž konali službu, protože jsou svatá. Oblečou se do jiných rouch a teprve pak mohou vstoupit do prostoru určeného pro lid.“
#42:15 Když dokončil měření uvnitř domu, vedl mě cestou k bráně s průčelím k východu a změřil celou stavbu kolem dokola.
#42:16 Prutem k měření změřil východní stranu; měřila celkem pět set prutů, měřeno prutem k měření.
#42:17 Pak změřil prutem k měření severní stranu; měřila celkem pět set prutů.
#42:18 Také jižní stranu změřil prutem k měření; měřila pět set prutů.
#42:19 Potom se obrátil k západní straně; naměřil prutem k měření pět set prutů.
#42:20 Celou stavbu změřil ze čtyř stran. Obklopovala ji hradba; ta byla dlouhá pět set a široká také pět set prutů. Měla oddělovat svaté od nesvatého. 
#43:1 Pak mě zavedl k bráně směřující na východ,
#43:2 a hle, od východu přicházela sláva Boha Izraele. Zvuk jeho příchodu byl jako zvuk mnohých vod a země zářila jeho slávou.
#43:3 Vidění, které jsem nyní uviděl, se podobalo vidění, které jsem viděl, když přišel zničit město, a také tomu vidění, které jsem viděl u průplavu Kebaru. I padl jsem na tvář.
#43:4 Hospodinova sláva vstoupila do domu branou s průčelím na východ.
#43:5 Duch mě zvedl a uvedl mě do vnitřního nádvoří. A hle, Hospodinova sláva naplnila dům.
#43:6 Uslyšel jsem, jak ke mně někdo mluví z domu, zatímco muž stál vedle mne.
#43:7 Řekl mi: „Lidský synu, viděl jsi místo, kde je můj trůn, místo, kde je podnož mých nohou. Tam chci přebývat uprostřed synů Izraele. Ani dům izraelský ani jeho králové už neposkvrní mé svaté jméno svým smilstvem a mrtvými těly svých králů ani svými posvátnými návršími.
#43:8 Položili svůj práh vedle mého prahu a své veřeje vedle mých veřejí, takže mezi mnou a jimi zůstala pouze stěna. Mé svaté jméno poskvrnili svými ohavnostmi, jichž se dopouštěli, takže jsem je ve svém hněvu pozřel.
#43:9 Nyní však vzdálí daleko ode mne své smilstvo i mrtvá těla svých králů a já budu navěky bydlet uprostřed nich.“
#43:10 „Ty, lidský synu, slyš. Pověz izraelskému domu o tomto domě, aby se hanbili za své nepravosti. Ať vezmou míru podle tohoto vzoru.
#43:11 Budou-li se opravdu hanbit za všechno, čeho se dopouštěli, pak je seznam s uspořádáním domu a s jeho zařízením, s předpisy, jak vycházet a vcházet, s celým jeho uspořádáním i se všemi nařízeními, s celým jeho uspořádáním i se všemi jeho zákony. To vše napiš před jejich očima, aby dbali na celé jeho uspořádání a všechna jeho nařízení a plnili je.
#43:12 Toto je zákon o domě: Celé jeho území kolem dokola na vrcholu hory je velesvaté. Hle, takový je zákon o domě.“
#43:13 Toto jsou míry oltáře v loktech o dlaň delších. Podstavec je jeden loket vysoký a šířka jeho okraje měří také loket. Okraj je ohraničen obrubou jednu dlaň širokou. A toto je výška oltáře:
#43:14 Na podstavci položeném na zemi je spodní stupeň vysoký dva lokte. Šířka okraje je jeden loket. Na tomto nízkém stupni je stupeň vysoký čtyři lokte; šířka okraje je jeden loket.
#43:15 Ohniště na oltáři je vysoké čtyři lokte. Od ohniště vzhůru jsou čtyři rohy.
#43:16 Ohniště je dlouhé dvanáct loket a široké také dvanáct loket; čtyři jeho strany tvoří čtverec.
#43:17 Stupeň je na všech čtyřech stranách dlouhý čtrnáct loket a široký také čtrnáct loket. Ohraničení kolem něho je v šíři půl lokte. Podstavec ze všech stran je na loket vysoký; vystupuje se k němu od východu.
#43:18 I řekl mi: „Lidský synu, toto praví Panovník Hospodin: Toto jsou nařízení o oltáři pro den, kdy bude zhotoven. Bude na něm zapálena oběť zápalná a bude kropen krví;
#43:19 lévijským kněžím, kteří jsou potomky Sádokovými a přistupují ke mně, aby mi přisluhovali, dáš mladého býčka k oběti za hřích, je výrok Panovníka Hospodina.
#43:20 Pak vezmeš trochu jeho krve; tou potřeš čtyři rohy oltáře, čtyři úhly stupně a jeho ohraničení kolem. Tak jej očistíš od hříchu a vykonáš za něj smírčí obřady.
#43:21 Potom vezmeš býčka k oběti za hřích a spálíš ho na ustanoveném místě domu mimo svatyni.
#43:22 Druhého dne přivedeš kozla bez vady k oběti za hřích. I očistí oltář od hříchu tak, jako jej očistili obětí býčka.
#43:23 Když ukončíš očišťování od hříchu, přivedeš mladého býčka bez vady a z bravu berana bez vady.
#43:24 Přivedeš je před Hospodina. Kněží je posypou solí a spálí je jako zápalnou oběť pro Hospodina.
#43:25 Po sedm dní denně budeš obětovat kozla jako oběť za hřích; dále budou obětovat mladého býčka a berana z bravu, vše bez vady.
#43:26 Po sedm dní budou vykonávat za oltář smírčí obřady; očistí a vysvětí jej.
#43:27 Až skončí tyto dny, osmého dne a dále budou kněží přinášet na oltáři vaše oběti zápalné i vaše oběti pokojné a já ve vás naleznu zalíbení, je výrok Panovníka Hospodina.“ 
#44:1 Pak mě odvedl cestou k vnější bráně do svatyně, obrácené k východu. Byla zavřená.
#44:2 Hospodin mi řekl: „Tato brána zůstane zavřená; nebude otvírána a nikdo jí nebude vstupovat, neboť skrze ni vstoupil Hospodin, Bůh Izraele. Proto zůstane uzavřena.
#44:3 Jen kníže, protože je knížetem, bude v ní sedat, aby jedl před Hospodinem chléb. Bude vcházet dvoranou brány a touž cestou vyjde.“
#44:4 Pak mě uvedl severní branou před dům; díval jsem se, a hle, dům naplnila Hospodinova sláva. I padl jsem na tvář.
#44:5 Hospodin mi řekl: „Lidský synu, vezmi si k srdci, pozorně hleď a napjatě poslouchej všechno, co ti řeknu o všech nařízeních týkajících se Hospodinova domu, o všech zákonných ustanoveních. Vezmi si vše k srdci, jak se má do domu vcházet a ze svatyně vycházet.“
#44:6 „A řekni vzpurnému domu izraelskému: Toto praví Panovník Hospodin: Dost už vašich ohavností, dome izraelský!
#44:7 Přiváděli jste do mé svatyně cizince neobřezaného srdce a neobřezaného těla, aby byli v mé svatyni, a tak znesvěcovali můj dům. Jako pokrm jste mi přinášeli tuk a krev, ale všemi svými ohavnostmi jste porušovali mou smlouvu.
#44:8 Nedrželi jste stráž nad mými svatými věcmi, stavěli jste na stráž při mé svatyni koho se vám zachtělo.
#44:9 Toto praví Panovník Hospodin: Žádný cizinec neobřezaného srdce a neobřezaného těla nevstoupí do mé svatyně ze všech cizinců, kteří žijí uprostřed synů Izraele.
#44:10 Pokud jde o lévijce, kteří se ode mne vzdálili, když Izrael bloudil, kteří zbloudili pryč ode mne za svými hnusnými modlami, ti ponesou svoji nepravost.
#44:11 Budou v mé svatyni jen přisluhovat dozorem při branách domu, budou přisluhovat v domě, budou pro lid porážet zvířata k zápalné oběti a k obětnímu hodu, budou stát před lidem, aby mu přisluhovali.
#44:12 Protože přisluhovali před jeho hnusnými modlami a jejich nepravost byla izraelskému domu k pádu, proto jsem pozvedl proti nim ruku k přísaze, je výrok Panovníka Hospodina. Ponesou svoji nepravost.
#44:13 Nepřistoupí ke mně, aby mi konali kněžkou službu; nepřistoupí k mým svatým darům, natož k velesvatým a ponesou hanbu za ohavnosti, jichž se dopouštěli.
#44:14 Ustanovil jsem je, aby drželi stráž při domě a konali veškerou službu při všem, co se v něm koná.
#44:15 Avšak lévijští kněží Sádokovci, kteří drželi stráž při mé svatyni, když Izraelci zbloudili ode mne pryč, ti se budou ke mně přibližovat, aby mi přisluhovali, a budou přede mnou stávat, aby mi přinášeli tuk a krev, je výrok Panovníka Hospodina.
#44:16 Ti vstoupí do mé svatyně a ti se přiblíží k mému stolu, budou mi přisluhovat a držet stráž.
#44:17 Kdykoli vstoupí do bran vnitřního nádvoří, oblečou se do lněného roucha. Nevezmou na sebe nic vlněného, pokud budou přisluhovat v branách vnitřního nádvoří a v domě.
#44:18 Na hlavě budou mít lněné turbany, kolem beder lněné spodky, ale neopášou se tak, aby se potili.
#44:19 Když budou vycházet do vnějšího nádvoří k lidu, svléknou si roucho, v němž přisluhovali, uloží je do svatých komor a obléknou si jiné roucho, aby svým rouchy neposvěcovali lid.“
#44:20 „Kněží si nevyholí hlavu ani nenechají vlát vlasy, ale řádně si vlasy přistřihnou.
#44:21 Žádný kněz nebude pít víno, má-li vstoupit do vnitřního nádvoří.
#44:22 Neožení se s vdovou nebo se zapuzenou ženou, ale s pannou z potomstva izraelského domu. Vdovu po knězi si však mohou vzít.
#44:23 A budou učit můj lid rozdílu mezi svatým a nesvatým, budou je seznamovat s rozdílem mezi nečistým a čistým.
#44:24 Dojde-li ke sporu, oni povstanou, aby soudili, a rozsoudí jej podle mých řádů. Budou dbát na mé zákony i na má nařízení o všech mých slavnostních shromážděních a světit mé dny odpočinku.
#44:25 Nikdo z nich nepřistoupí k mrtvému člověku, aby se neposkvrnil, leč by to byl otec nebo matka, syn nebo dcera, bratr nebo neprovdaná sestra; při těch se nemusí vyhýbat poskvrnění.
#44:26 A po svém očištění si připočte ještě sedm dní.
#44:27 V den, kdy opět vstoupí do svatyně, do vnitřního nádvoří, aby přisluhoval ve svatyni, přinese svou oběť za hřích, je výrok Panovníka Hospodina.
#44:28 A toto budou mít za dědictví: jejich dědictvím budu já. Nedáte jim v Izraeli žádné trvalé vlastnictví, jejich trvalým vlastnictvím budu já.
#44:29 Budou jíst obětní dary, oběť za hřích i za vinu, bude jim patřit všechno postižené klatbou v Izraeli.
#44:30 Prvotiny všech raných plodů a všech obětí pozdvihování, ze všech vašich obětí pozdvihování, budou náležet kněžím, i prvotiny obilní tluče dáte knězi, aby spočinulo na vašem domě požehnání.
#44:31 Kněží nebudou jíst nic, co uhynulo nebo bylo rozsápáno, ať je to pták nebo dobytče.“ 
#45:1 „Až si přidělíte losem do dědictví zemi, odevzdáte jako oběť pozdvihování Hospodinu svatý díl země pětadvacet tisíc loket dlouhý a deset tisíc loket široký; ten bude po celém svém území svatý v celém obvodu.
#45:2 A z toho připadne pro svatyni čtverec pět set loket na pět set loket a k tomu padesát loket pastvin okolo.
#45:3 Z celé výměry tedy odměříš díl pětadvacet tisíc loket dlouhý a deset tisíc loket široký. V něm bude svaté místo, velesvatyně.
#45:4 Bude to svatý díl země. Bude patřit kněžím přisluhujícím ve svatyni, totiž těm, kteří přistupují k Hospodinově službě. Tam budou mít místo pro domy a svaté místo pro svatyni.
#45:5 Díl pětadvacet tisíc loket dlouhý a deset tisíc loket široký budou mít lévijci přisluhující při domě. Připadne jim jako trvalé vlastnictví dvacet komor.
#45:6 Jako trvalé vlastnictví města určíte díl pět tisíc loket široký a pětadvacet tisíc loket dlouhý, souběžně s dílem odděleným jako svatá oběť pozdvihování. To bude patřit celému domu izraelskému.
#45:7 Knížeti bude patřit díl po obou stranách oběti pozdvihování a trvalého vlastnictví města, a to podél svaté oběti pozdvihování a podél trvalého vlastnictví města, na západní straně část západní a na východní straně část východní. Délka těch souběžných dílů bude stejná jak při západním, tak při východním pomezí.
#45:8 To územní v Izraeli připadne knížeti do trvalého vlastnictví. A má knížata už nebudou můj lid utiskovat, ale dají ostatní zemi domu izraelskému podle jeho kmenů.“
#45:9 Toto praví Panovník Hospodin: „Příliš mnoho si osobujete, knížata izraelská. Zanechte násilí a útisku, jednejte podle práva a spravedlnosti, přestaňte můj lid vyhánět z jeho vlastnictví, je výrok Panovníka Hospodina.
#45:10 Budete mít spravedlivé váhy, spravedlivou éfu, spravdlivý bat.
#45:11 Éfa i bat ať mají stejnou míru, takže bat bude mít obsah desetiny chómeru, rovněž tak éfa. Obsah éfy se bude vyměřovat podle chómeru.
#45:12 A šekel bude mít dvacet zrn. Dvacet šekelů a pětadvacet šekelů a patnáct šekelů budete mít na jednu hřivnu.“
#45:13 „Toto je oběť pozdvihování, kterou budete odevzdávat: šestinu éfy z chómeru pšenice a šestinu éfy z chómeru ječmene.
#45:14 Stanovená míra pro olej je bat: desetina batu z kóru, jenž má deset batů, či z chómeru, neboť deset batů je jeden chómer.
#45:15 Dále ze zavlažované oblasti izraelské přinesete jako obětní dar a oběť zápalnou a oběti pokojné jeden kus bravu ze dvou set kusů. Bude to smírčí oběť za vás, je výrok Panovníka Hospodina.
#45:16 Všechen lid země bude odvádět tuto oběť pozdvihování knížeti v Izraeli.
#45:17 Kníže je povinen přinášet oběti zápalné, oběť přídavnou i úlitbu o svátcích, o novoluních i o dnech odpočinku, o všech slavnostních shromážděních domu izraelského. On připraví oběť za hřích i oběť přídavnou, oběť zápalnou i oběti pokojné jako smírčí oběť za dům izraelský.“
#45:18 Toto praví Panovník Hospodin: „První den prvního měsíce vezmeš mladého býčka bez vady a očistíš svatyni od hříchu.
#45:19 Kněz vezme trochu krve z oběti za hřích a potře veřeje domu i čtyři úhly stupně na oltáři i veřeje brány vnitřního nádvoří.
#45:20 Právě tak učiníš sedmého dne toho měsíce za každého, kdo zhřešil neúmyslně a z nevědomosti; budete konat smírčí obřady za dům.
#45:21 Čtrnáctého dne prvního měsíce budete mít hod beránka a po sedm dní slavnosti se budou jíst nekvašené chleby.
#45:22 Toho dne připraví kníže býčka jako oběť za hřích svůj i všeho lidu země.
#45:23 V sedmi dnech té slavnosti připraví k zápalné oběti Hospodinu sedm býčků a sedm beranů bez vady, každý den po sedm dní, a kozla jako oběť za hřích každý den.
#45:24 A jako přídavnou oběť připraví éfu mouky na býčka a éfu na berana a na každou éfu hín oleje.
#45:25 Patnáctého dne sedmého měsíce při slavnosti připraví totéž jako v oněch sedmi dnech, jak oběť za hřích, tak oběť zápalnou i přídavnou a olej.“ 
#46:1 Toto praví Panovník Hospodin: „Brána vnitřního nádvoří, obrácená k východu, zůstane zavřená po šest pracovních dní, ale v den odpočinku a v den novoluní bude otevřená.
#46:2 Kníže vstoupí zvenčí dvoranou brány a postaví se u veřejí brány. Kněží připraví jeho zápalnou oběť a jeho oběti pokojné a on se bude klanět na prahu brány. Pak z ní vyjde a brána se nezavře až do večera.
#46:3 Také lid země se bude u vchodu do té brány klanět před Hospodinem ve dnech odpočinku a o novoluních.
#46:4 Zápalnou obětí, kterou kníže přinese Hospodinu v den odpočinku, bude šest beránků bez vady a beran bez vady.
#46:5 A přídavnou obětí bude éfa mouky na berana; a k beránkům, co může dát jako přídavnou oběť; a hín oleje na éfu.
#46:6 V den novoluní připraví mladého býčka bez vady a šest beránků a berana; i ti budou bez vady.
#46:7 Připraví také jako přídavnou oběť éfu mouky na býčka, éfu na berana a k beránkům podle svých možností; a hín oleje na éfu.
#46:8 Když bude kníže vstupovat, projde dvoranou brány a stejnou cestou vyjde.
#46:9 Ale když bude vstupovat lid země k slavnostním shromážděním před Hospodina, ten, kdo bude vstupovat severní branou, aby se klaněl, vyjde branou jižní, a kdo bude vstupovat branou jižní, bude vycházet branou severní. Nevrátí se branou, kterou vstoupil, ale vyjde branou protější.
#46:10 Kníže vejde uprostřed vstupujících a s vycházejícími vyjde.
#46:11 O svátcích a slavnostních shromážděních bude přídavnou obětí éfa mouky na býčka, éfa na berana a k beránkům, co může dát; a hín oleje na éfu.
#46:12 Připraví-li kníže Hospodinu dobrovolnou oběť zápalnou nebo dobrovolné oběti pokojné, ať je mu otevřena brána obrácená k východu. Jako svou oběť zápalnou a své oběti pokojné připraví totéž, co připravil v den odpočinku. Potom vyjde a po jeho odchodu bude brána uzavřena.
#46:13 Denně připravíš jako zápalnou oběť Hospodinu ročního beránka bez vady; připravíš jej každého jitra.
#46:14 A k němu jako přídavnou oběť připravíš každého jitra šestinu éfy mouky a třetinu hínu oleje k pokropení bílé mouky. Je to každodenní přídavná oběť Hospodinu podle provždy platných nařízení.
#46:15 Každého jitra připraví beránka a oběť přídavnou i olej. To je každodenní oběť zápalná.“
#46:16 Toto praví Panovník Hospodin: „Jestliže dá kníže dar některému ze svých synů, bude to jeho synům náležet jako dědičný podíl. Bude to tedy jejich trvalé vlastnictví.
#46:17 Když však dá dar z dědičného podílu některému ze svých služebníků, zůstane mu to do léta osvobození a pak se vrátí knížeti. Jeho dědičný podíl náleží výhradně jeho synům.
#46:18 Kníže nevezme nic z dědičného podílu lidu, nebude je vytlačovat z jejich trvalého vlastnictví. Svým synům dá dědičný podíl z vlastního trvalého vlastnictví, aby nikdo z mého lidu nemusel ze svého trvalého vlastnictví pryč.“
#46:19 I vedl mě vchodem v postranní části brány do svatých komor pro kněze, které jsou obráceny k severu, a hle, bylo tam místo v odlehlém koutě na západ.
#46:20 Řekl mi: „Toto je místo, kde kněží vaří maso oběti za vinu a oběti za hřích, kde pečou chleby z oběti přídavné; nesmějí to vynášet do vnějšího nádvoří, aby neposvěcovali lid.“
#46:21 Pak mě vyvedl do vnějšího nádvoří a provedl čtyřmi jeho rohy, A hle, v každém rohu nádvoří byl dvůr.
#46:22 Ve čtyřech rozích nádvoří byly nezastřešené dvory čtyřicet loket dlouhé a třicet loket široké. Všechny čtyři měly týž rozměr a byly vestavěny do rohů.
#46:23 A všechny čtyři měly kolem dokola kamennou ohradu a dole kolem ohrazení byla udělána topeniště.
#46:24 I řekl mi: „Toto jsou kuchyně, kde služebníci domu budou vařit oběti lidu.“ 
#47:1 Přivedl mě znovu ke vchodu do domu. A hle, zespodu prahu domu vytékala k východu voda; průčelí domu je totiž obráceno k východu a voda tekla dolů zpod pravého boku domu na jižní straně oltáře.
#47:2 A vyvedl mě branou severní a vedl mě venku okolo k vnější bráně směrem na východ. A hle, voda se řinula z pravého boku domu.
#47:3 Když odcházel ten muž na východ, měl v ruce měřící šňůru. Odměřil tisíc loket a provedl mě vodou; vody bylo po kotníky.
#47:4 Znovu odměřil tisíc a provedl mě vodou; vody bylo po kolena. Odměřil další tisíc a provedl mě; vody bylo po bedra.
#47:5 Odměřil ještě tisíc a potok nebylo možno přebrodit; voda vystoupila a muselo se v ní plavat; byl to potok, který se už nedal přebrodit.
#47:6 Řekl mi: „Vidíš, lidský synu?“ Vedl mě dál a znovu mě přivedl k břehu toho potoka.
#47:7 Když jsem se tam vrátil, hle, na břehu potoka bylo po obou stranách velmi mnoho stromoví.
#47:8 I řekl mi: „Tato voda odtéká do východního obvodu, teče dolů do Araby a vlévá se do moře. Tam, kde vtéká do moře, stávají se vody zdravými.
#47:9 Každý hemžící se živočich bude žít všude, kam se vlévá potok. A ryb bude velmi mnoho. Neboť kam se vlévají ony vody, tam se vody moře stávají zdravými a zůstane naživu všechno tam, kam se ten potok vlévá.
#47:10 I postaví se u něho rybáři; od Én-gedí až k Én-eglajimu se budou rozprostírat sítě. Ryb rozličného druhu bude velmi mnoho jako ryb ve Velkém moři.
#47:11 Avšak jeho bažiny a mělčiny nebudou ozdravěny, bude se tam získávat sůl.
#47:12 Při potoku na obou březích vzejde rozličné ovocné stromoví, jemuž nebude vadnout listí, jež nepřestane plodit, ale každý měsíc přinese rané plody, neboť vody, které je zavlažují, vytékají ze svatyně. Jeho ovoce bude sloužit za pokrm a jeho listí jako lék.“
#47:13 Toto praví Panovník Hospodin: „Toto je hranice, v níž si vezmete do dědictví zemi pro dvanáct izraelských kmenů, Josefovi provazce dva.
#47:14 Jeden jako druhý obdržíte do dědictví zemi, o níž jsem zvedl ruku k přísaze, že ji dám vašim otcům. Tato země vám připadne za dědictví.
#47:15 A toto je pomezí té země: Na severu od Velkého moře směrem k Chetlónu při cestě do Sedadu
#47:16 Chamát, Beróta, Sibrajim, který je pomezím Damašku a pomezím Chamátu, a Chasér-tíkón, který je při pomezí Chavránu.
#47:17 Pomezí povede od moře k Chasar-énónu na pomezí Damašku daleko na severu a k pomezí Chamátu. To bude severní strana.
#47:18 Na východní straně vyměříte pomezí mezi Chavránem a Damaškem a mezi Gileádem a izraelskou zemí podél Jordánu k Východnímu moři. To bude východní strana.
#47:19 Na jižní straně půjde pomezí jižně od Támaru až k Vodám sváru v Kádeši, k Nachale u Velkého moře. To bude jižní strana směrem k Negebu.
#47:20 Na západní straně bude tvořit pomezí Velké moře od jižního pomezí až naproti cestě do Chamátu. To bude západní strana.
#47:21 Tuto zemi si rozdělíte mezi izraelské kmeny.
#47:22 Přidělíte losováním dědičný podíl sobě i bezdomovcům, kteří mezi vámi pobývají jako hosté a kteří mezi vámi zplodili syny; budou u vás jako rodilí Izraelci. Spolu s vámi si určí losováním dědičný podíl mezi izraelskými kmeny.
#47:23 V tom kmeni, u něhož bezdomovec pobývá jako host, mu dáte jeho dědičný podíl, je výrok Panovníka Hospodina.“ 
#48:1 Toto jsou jména kmenů: Nejdále na severu směrem k Chetlónu až k cestě do Chamátu a Chasar-énanu - hranice Damašku je na sever stranou Chamátu - od východní strany až k moři připadne jeden podíl Danovi.
#48:2 Při pomezí Danově od strany východní po západní připadne jeden podíl Ašerovi.
#48:3 Při pomezí Ašerově od východní strany po západní připadne jeden podíl Neftalímu.
#48:4 Při pomezí Neftalíově od východní strany po západní připadne jeden podíl Manasesovi.
#48:5 Při pomezí Manasesově od východní strany po západní připadne jeden podíl Efrajimovi.
#48:6 Při pomezí Efrajimově od východní strany po západní připadne jeden podíl Rúbenovi.
#48:7 Při pomezí Rúbenově od východní strany po západní připadne jeden podíl Judovi.
#48:8 Při pomezí Judově od východní strany po západní bude oběť pozdvihování, kterou přinesete, dvacet pět tisíc loket široká a dlouhá jako jeden z podílů od strany východní po západní; a uprostřed ní bude svatyně.
#48:9 Oběť pozdvihování, kterou přinesete Hospodinu, bude v první části dvacet pět tisíc loket dlouhá a deset tisíc loket široká.
#48:10 Ti, kterým bude ta svatá oběť pozdvihování patřit, budou kněží. Bude na severu dvacet pět tisíc loket dlouhá, na západě deset tisíc loket široká, na východě deset tisíc loket široká a na jihu dvacet pět tisíc loket dlouhá; uprostřed ní bude Hospodinova svatyně.
#48:11 Bude patřit kněžím, posvěceným ze synů Sádokových, kteří střeží to, co jsem jim svěřil. Oni nezbloudili jako lévijci, když bloudili synové izraelští.
#48:12 Jim bude patřit tato obětina z oběti pozdvihování jako nejposvátnější část země při pomezí lévijců.
#48:13 Souběžně s pomezím kněží budou mít podíl lévijci, a to dvacet pět tisíc loket dlouhý a deset tisíc loket široký. Délka každého bude dvacet pět tisíc loket a šířka deset tisíc loket.
#48:14 Nebudou z podílu odprodávat ani vyměňovat či na jiného převádět prvotiny země; jsou svaté Hospodinu.
#48:15 Zbylých pět tisích loket našíř podél těch dvaceti pěti tisíc loket bude díl nesvatý patřící městu, pro sídliště a pastviny; město bude uprostřed něho.
#48:16 A toto jsou jeho rozměry: na severní straně čtyři tisíce pět set loket, na jižní straně čtyři tisíce pět set loket, při východní straně čtyři tisíce pět set loket a na západní straně čtyři tisíce pět set loket.
#48:17 A město bude mít pastviny o rozměrech: na severu dvě stě padesát loket, na jihu dvě stě padesát loket, na východě dvě stě padesát loket, na západě dvě stě padesát loket.
#48:18 Zbytek na délku souběžně se svatou obětí pozdvihování bude měřit deset tisíc loket na východě a deset tisíc loket na západě; bude souběžný se svatou obětí pozdvihování a výtěžky z něho budou sloužit za pokrm pro služebníky města.
#48:19 Služebníci města budou ze všech izraelských kmenů ti, kdo pracují na tom dílu.
#48:20 Celou oběť pozdvihování, čtverec dvacet pět tisíc loket na dvacet pět tisíc loket, oddělíte jako svatou oběť pozdvihování do trvalého vlastnictví města.
#48:21 Zbytek po obou stranách svaté oběti pozdvihování a trvalého vlastnictví města bude patřit knížeti; bude to dvacet pět tisíc loket podél oběti pozdvihování při pomezí východním a na západě dvacet pět tisíc loket podél pomezí západního, souběžně s kmenovými podíly; to bude patřit knížeti. Svatá oběť pozdvihování a svatyně Hospodinova domu budou uprostřed.
#48:22 I po obou stranách trvalého vlastnictví lévijců a trvalého vlastnictví města budou části patřící knížeti, uprostřed mezi pomezím Judovým a Benjamínovým. To bude patřit knížeti.
#48:23 Pokud jde o ostatní kmeny: Od východní strany po západní připadne jeden podíl Benjamínovi.
#48:24 Při pomezí Benjamínově od strany východní po západní připadne jeden podíl Šimeónovi.
#48:25 Při pomezí Šimeónově od strany východní po západní připadne jeden podíl Isacharovi.
#48:26 Při pomezí Isacharově od strany východní po západní připadne jeden podíl Zabulónovi.
#48:27 Při pomezí Zabulónově od strany východní po západní připadne jeden podíl Gádovi.
#48:28 Při pomezí Gádově na straně negebské k jihu půjde pomezí od Támaru k Vodám sváru v Kádeši, k Nachale u Velkého moře.
#48:29 To je země, kterou dostanete přidělenou do dědictví pro izraelské kmeny, a toto jsou jejich příděly, je výrok Panovníka Hospodina.
#48:30 A toto jsou východy z města: Ze severní strany bude město měřit čtyři tisíce pět set loket;
#48:31 tam budou městské brány, pojmenované podle izraelských kmenů. Na severu tři brány: jedna brána Rúbenova, jedna brána Judova a jedna brána Léviova.
#48:32 Na východní straně bude měřit čtyři tisíce pět set loket; tam budou tři brány: jedna brána Josefova, jedna brána Benjamínova a jedna brána Danova.
#48:33 Strana jižní bude měřit čtyři tisíce pět set loket; tam budou také tři brány: jedna brána Šimeónova, jedna brána Isacharova a jedna brána Zabulónova.
#48:34 Strana západní bude měřit čtyři tisíce pět set loket; tam budou rovněž tři brány: jedna brána Gádova, jedna brána Ašérova a jedna brána Neftalíova.
#48:35 Obvod bude měřit osmnáct tisíc loket. Jméno města od onoho dne bude: „Zde je Hospodin“.  

\book{Daniel}{Dan}
#1:1 Ve třetím roce kralování Jójakíma, krále judského, přitáhl Nebúkadnesar, babylónský král, k Jeruzalému a oblehl jej.
#1:2 Panovník Hospodin mu vydal do rukou judského krále Jójakíma a část nádob z Božího domu. Nebúkadnesar je dopravil do země Šineáru, do domu svého božstva; nádoby dal dopravit do klenotnice svého boha.
#1:3 Pak rozkázal král Ašpenazovi, vrchnímu nad dvořany, aby přivedl z Izraelců, a to z královského potomstva a ze šlechty,
#1:4 jinochy bez jakékoli vady, pěkného vzhledu, zběhlé ve veškeré moudrosti, kteří si osvojili poznání, rozumějí všemu vědění a jsou schopni stávat v královském paláci a naučit se kaldejskému písemnictví a jazyku.
#1:5 Král pro ně určil každodenní příděl z královských lahůdek a z vína, které pil při svých hodech, a dal je vychovávat po tři roky. Po jejich uplynutí měli stávat před králem.
#1:6 Z Judejců byli mezi nimi Daniel, Chananjáš, Míšael a Azarjáš.
#1:7 Velitel dvořanů jim změnil jména: Danielovi dal jméno Beltšasar, Chananjášovi Šadrak, Míšaelovi Méšak a Azarjášovi Abed-nego.
#1:8 Ale Daniel si předsevzal, že se neposkvrní královskými lahůdkami a vínem, které pil král při svých hodech. Požádal velitele dvořanů, aby se nemusel poskvrňovat.
#1:9 A Bůh dal Danielovi dojít u velitele dvořanů milosrdenství a slitování.
#1:10 Velitel dvořanů však Danielovi řekl: „Bojím se krále, svého pána, který vám určil pokrm a nápoj. Když uvidí, že jste v tváři přepadlejší než jinoši z vašich řad, připravíte mě u krále o hlavu.“
#1:11 Daniel tedy navrhl opatrovníkovi, kterého určil velitel dvořanů nad Danielem, Chananjášem, Míšaelem a Azarjášem:
#1:12 „Zkus to se svými služebníky po deset dní. Ať nám dávají k jídlu zeleninu a k pití vodu.
#1:13 Potom porovnáš vzhled náš a vzhled jinochů, kteří jedli královské lahůdky, a učiň se svými služebníky podle toho, co uvidíš.“
#1:14 Opatrovník je v té věci vyslyšel a zkusil to s nimi po deset dní.
#1:15 Po uplynutí deseti dnů se ukázalo, že jejich vzhled je lepší; byli statnější než ostatní jinoši, kteří jedli královské lahůdky.
#1:16 Opatrovník tedy odnášel jejich lahůdky a víno, které měli pít, a dával jim zeleninu.
#1:17 A Bůh dal těm čtyřem jinochům vědění a zběhlost ve veškerém písemnictví a moudrosti. Danielovi dal nadto porozumět všem viděním a snům.
#1:18 Po uplynutí doby, kdy podle králova nařízení měli být předvedeni, přivedl je velitel dvořanů před Nebúkadnesara.
#1:19 Král s nimi rozmlouval a žádný mezi nimi nebyl shledán takový jako Daniel, Chananjáš, Míšael a Azarjáš. Proto stávali před králem.
#1:20 Pokud šlo o porozumění moudrosti, které od nich král vyžadoval, shledal, že desetkrát předčí všechny věštce a zaklínače, kteří byli v celém jeho království.
#1:21 A Daniel tam zůstal až do prvního roku vlády krále Kýra. 
#2:1 Ve druhém roce svého kralování měl Nebúkadnesar sen. Rozrušil se a nemohl spát.
#2:2 Král tedy rozkázal zavolat věštce, zaklínače, čaroděje a hvězdopravce, aby mu pověděli, co se mu zdálo. I přišli a postavili se před králem.
#2:3 Král jim řekl: „Měl jsem sen a jsem rozrušen; chci ten sen znát.“
#2:4 Hvězdopravci promluvili ke králi aramejsky: „Králi, navěky buď živ! Pověz svým služebníkům ten sen a my ti sdělíme výklad.“
#2:5 Král hvězdopravcům odpověděl: „Mé slovo je příkaz: Jestliže mi neoznámíte sen a jeho výklad, budete rozsekáni na kusy a z vašich domů se stane hnojiště.
#2:6 Jestliže mi však sen a jeho výklad sdělíte, dostane se vám ode mne darů, odměny a velké pocty. Sdělte mi tedy sen a jeho výklad!“
#2:7 Odpověděli znovu: „Ať král poví svým služebníkům sen a my sdělíme výklad.“
#2:8 Král odpověděl: „Já vím jistě, že chcete získat čas, protože vidíte, že mé slovo je příkazem:
#2:9 Jestliže mi ten sen neoznámíte, čeká vás jediný rozsudek. Domluvili jste se, že mi budete říkat lživé a zlé věci, dokud se nezmění čas. A proto mi řekněte ten sen a poznám, že jste schopni sdělit mi i výklad.“
#2:10 Hvězdopravci králi odpověděli: „Není na zemi člověka, který by dovedl sdělit to, co král rozkázal. Nadto žádný velký a mocný král nežádal od žádného věštce, zaklínače nebo hvězdopravce takovou věc.
#2:11 Věc, kterou král žádá, je těžká a není nikoho jiného, kdo by ji králi sdělil, mimo bohy, kteří nepřebývají mezi smrtelníky.“
#2:12 Král se proto rozhněval, velice se rozzlobil a rozkázal všechny babylónské mudrce zahubit.
#2:13 Když byl vydán rozkaz a mudrci byli zabíjeni, hledali též Daniela a jeho druhy, aby byli zabiti.
#2:14 Tehdy se Daniel rozvážně a uvážlivě obrátil na Arjóka, velitele královy tělesné stráže, který vyšel zabíjet babylónské mudrce.
#2:15 Otázal se Arjóka, králova zmocněnce: „Proč je králův rozkaz tak přísný?“ Arjók Danielovi tu věc oznámil.
#2:16 Daniel vešel ke králi a prosil ho, aby mu dal určitou dobu, že králi ten výklad sdělí.
#2:17 Pak Daniel odešel do svého domu a oznámil tu věc svým druhům, Chananjášovi, Míšaelovi a Azarjášovi.
#2:18 Vyzval je, aby prosili Boha nebes o slitování ve věci toho tajemství, aby Daniel a jeho druhové nebyli zahubeni se zbytkem babylónských mudrců.
#2:19 I bylo to tajemství Danielovi zjeveno v nočním vidění. Daniel dobrořečil Bohu nebes.
#2:20 Promlouval takto: „Požehnáno buď jméno Boží od věků až na věky. Jeho moudrost i bohatýrská síla.
#2:21 On mění časy i doby, krále sesazuje, krále ustanovuje, dává moudrost moudrým, poznání těm, kdo mají rozum.
#2:22 Odhaluje hlubiny a skryté věci, poznává to, co je ve tmě, a světlo s ním bydlí.
#2:23 Tobě, Bože otců mých, chci vzdávat čest a chválu, neboť jsi mi dal moudrost a bohatýrskou sílu. Oznámils mi nyní, oč jsme tě prosili, oznámil jsi nám královu záležitost.“
#2:24 Daniel tedy vešel k Arjókovi, kterého král ustanovil, aby zahubil babylónské mudrce. Přišel a řekl mu toto: „Babylónské mudrce nehub! Uveď mě před krále, sdělím králi výklad.“
#2:25 Arjók neprodleně uvedl Daniela před krále a řekl mu: „Našel jsem muže z judských přesídlenců, který králi oznámí výklad.“
#2:26 Král oslovil Daniela, který měl jméno Beltšasar: „Jsi schopen oznámit mi sen, který jsem měl, a jeho výklad?“
#2:27 Daniel králi odpověděl: „Tajemství, na které se král ptá, nemohou králi sdělit ani mudrci ani zaklínači ani věštci ani planetáři.
#2:28 Ale je Bůh v nebesích, který odhaluje tajemství. On dal králi Nebúkadnesarovi poznat, co se stane v posledních dnech. Toto je sen, totiž vidění, která ti prošla hlavou na tvém lůžku:
#2:29 Tobě, králi vyvstaly na lůžku starosti o to, co se v budoucnu stane, a ten, který odhaluje tajemství, ti oznámil, co se stane.
#2:30 Já pak to nemám z moudrosti, které bych měl více než ostatní živé bytosti, ale to tajemství mi bylo odhaleno, aby výklad byl králi oznámen a abys poznal myšlení svého srdce.
#2:31 Ty jsi, králi, viděl jakousi velikou sochu. Byla to obrovská socha a její lesk byl mimořádný. Stála proti tobě a měla strašný vzhled.
#2:32 Hlava té sochy byla z ryzího zlata, její hruď a paže ze stříbra, břicho a boky z mědi,
#2:33 stehna ze železa, nohy dílem ze železa a dílem z hlíny.
#2:34 Viděl jsi, jak se bez zásahu rukou utrhl kámen a udeřil do železných a hliněných nohou sochy a rozdrtil je,
#2:35 a rázem bylo rozdrceno železo, hlína, měď, stříbro i zlato, a byly jako plevy na mlatě v letní době. Odnesl je vítr a nezbylo po nich ani stopy. A ten kámen, který do sochy udeřil, se stal obrovskou skálou a zaplnil celou zemi.
#2:36 Toto je sen. Též jeho výklad řekneme králi:
#2:37 Ty, králi, jsi král králů. Bůh nebes ti dal království, moc, sílu a slávu.
#2:38 A všechna místa, kde bydlí lidé, polní zvěř a nebeské ptactvo, dal ti do rukou a dal ti moc nad tím vším. Ty jsi ta zlatá hlava.
#2:39 Po tobě povstane další království, nižší než tvé, a pak další, třetí království, měděné, které bude mít moc nad celou zemí.
#2:40 Čtvrté království bude tvrdé jako železo, neboť železo drtí a drolí vše, a to království jako železo, které tříští všechno, bude drtit a tříštit.
#2:41 Že jsi viděl nohy a prsty dílem z hrnčířské hlíny a dílem ze železa, znamená, že království bude rozdělené a bude v něm něco z pevnosti železa, neboť jsi viděl železo smíšené s jílovitou hlínou.
#2:42 Prsty nohou dílem ze železa a dílem z hlíny znamenají, že království bude zčásti tvrdé a dílem křehké.
#2:43 Že jsi viděl železo smíšené s jílovitou hlínou, znamená, že se bude lidské pokolení mísit, avšak nepřilnou k sobě navzájem, jako se nesmísí železo s hlínou.
#2:44 Ve dnech těch králů dá Bůh nebes povstat království, které nebude zničeno navěky, a to království nebude předáno jinému lidu. Rozdrtí a učiní konec všem těm královstvím, avšak samo zůstane navěky,
#2:45 neboť jsi viděl, že se utrhl ze skály kámen bez zásahu rukou a rozdrtil železo, měď, hlínu, stříbro i zlato. Veliký Bůh dal králi poznat, co se v budoucnu stane. Sen je pravdivý a výklad spolehlivý.“
#2:46 Tu král Nebúkadnesar padl na tvář, poklonil se před Danielem a rozkázal, aby mu byla obětována oběť přídavná s vonnými dary.
#2:47 Král Daniela oslovil: „Vpravdě, váš Bůh je Bohem bohů a Pán králů, který odhaluje tajemství, neboť jsi mi dokázal toto tajemství odhalit.“
#2:48 Král pak Daniela povýšil, dal mu mnoho velikých darů i moc nad celou babylónskou krajinou a učinil ho nejvyšším správcem všech babylónských mudrců.
#2:49 Ale Daniel prosil krále, aby správou babylónské krajiny pověřil Šadraka, Méšaka a Abed-nega. Daniel sám zůstal na královském dvoře. 
#3:1 Král Nebúkadnesar dal zhotovit zlatou sochu, jejíž výška byla šedesát loket a šířka šest loket. Postavil ji na pláni Dúra v babylónské krajině.
#3:2 Král Nebúkadnesar poslal pro satrapy, zemské správce a místodržitele, poradce, správce pokladu, soudce, vysoké úředníky a všechny zmocněnce nad krajinami, aby přišli k posvěcení sochy, kterou král Nebúkadnesar postavil.
#3:3 Tehdy se shromáždili satrapové, zemští správcové a místodržitelé, poradci, správcové pokladu, soudcové, vysocí úředníci a všichni zmocněnci nad krajinami k posvěcení sochy, kterou král Nebúkadnesar postavil. Stáli proti soše, kterou postavil Nebúkadnesar.
#3:4 Hlasatel mocně volal: „Poroučí se vám, lidé různých národností a jazyků:
#3:5 Jakmile uslyšíte hlas rohu, flétny, citary, harfy, loutny, dud a rozmanitých strunných nástrojů, padnete a pokloníte se před zlatou sochou, kterou postavil král Nebúkadnesar.
#3:6 Kdo nepadne a nepokloní se, bude v tu hodinu vhozen do rozpálené ohnivé pece.“
#3:7 Proto v určenou dobu, jakmile všichni lidé uslyšeli hlas rohu, flétny, citary, harfy, loutny a rozmanitých strunných nástrojů, všichni lidé různých národností a jazyků padli a klaněli se před zlatou sochou, kterou král Nebúkadnesar postavil.
#3:8 V té době přišli muži hvězdopravci a udali Judejce.
#3:9 Hlásili králi Nebúkadnesarovi: „Králi, navěky buď živ!
#3:10 Ty jsi, králi, vydal rozkaz, aby každý člověk, až uslyší hlas rohu, flétny, citary, harfy, loutny a dud a rozmanitých strunných nástrojů, padl a poklonil se před zlatou sochou.
#3:11 Kdo nepadne a nepokloní se, má být vhozen do rozpálené ohnivé pece.
#3:12 Jsou zde muži Judejci, které jsi pověřil správou babylónské krajiny, Šadrak, Méšak a Abed-nego. Tito muži nedbají, králi, na tvůj rozkaz, tvé bohy neuctívají a před zlatou sochou, kterou jsi postavil, se neklanějí.“
#3:13 Tehdy Nebúkadnesar, rozlícen a rozhořčen, rozkázal přivést Šadraka, Méšaka a Abed-nega. Tito muži byli hned přivedeni před krále.
#3:14 Nebúkadnesar se jich otázal: „Je to tak, Šadraku, Méšaku a Abed-nego, že mé bohy neuctíváte a před zlatou sochou, kterou jsem postavil, jste se nepoklonili?
#3:15 Nuže, jste ochotni v čase, kdy uslyšíte hlas rohu, flétny, citary, harfy, loutny a dud a rozmanitých strunných nástrojů, padnout a poklonit se před sochou, kterou jsem udělal? Jestliže se nepokloníte, v tu hodinu budete vhozeni do rozpálené ohnivé pece. A kdo je ten Bůh, který by vás vysvobodil z mých rukou!“
#3:16 Šadrak, Méšak a Abed-nego odpověděli králi: „Nebúkadnesare, nám není třeba dávat ti odpověď.
#3:17 Jestliže náš Bůh, kterého my uctíváme, nás bude chtít vysvobodit z rozpálené ohnivé pece i z tvých rukou, králi, vysvobodí nás.
#3:18 Ale i kdyby ne, věz, králi, že tvé bohy uctívat nebudeme a před zlatou sochou, kterou jsi postavil, se nepokloníme.“
#3:19 Tu se Nebúkadnesar velice rozlítil a výraz jeho tváře se vůči Šadrakovi, Méšakovi a Abed-negovi změnil. Rozkázal vytopit pec sedmkrát víc, než se obvykle vytápěla.
#3:20 Mužům, statečným bohatýrům, kteří byli v jeho vojsku, rozkázal Šadraka, Méšaka a Abed-nega svázat a vhodit je do rozpálené ohnivé pece.
#3:21 Ti muži byli hned svázáni ve svých pláštích a suknicích i s čepicemi a celým oblečením a vhozeni do rozpálené ohnivé pece.
#3:22 Protože královo slovo bylo přísné a pec byla nadmíru vytopena, ony muže, kteří Šadraka, Méšaka a Abed-nega vynesli, usmrtil plamen ohně.
#3:23 A ti tři muži, Šadrak, Méšak a Abed-nego, padli svázaní do rozpálené ohnivé pece.
#3:24 Tu král Nebúkadnesar užasl a chvatně vstal. Otázal se královské rady: „Což jsme nevhodili do ohně tři svázané muže?“ Odpověděli králi: „Jistěže, králi.“
#3:25 Král zvolal: „Hle, vidím čtyři muže, jsou rozvázaní a procházejí se uprostřed ohně bez jakékoli úhony. Ten čtvrtý se svým vzhledem podobá božímu synu.“
#3:26 I přistoupil Nebúkadnesar k otvoru rozpálené ohnivé pece a zvolal: „Šadraku, Méšaku a Abed-nego, služebníci Boha nejvyššího, vyjděte a pojďte sem!“ Šadrak, Méšak a Abed-nego vyšli z ohně.
#3:27 Satrapové, zemští správci a místodržitelé a královská rada se shromáždili, aby viděli ty muže, nad jejichž těly neměl oheň moc; ani vlas jejich hlavy nebyl sežehnut, jejich pláště nedoznaly změny, ani nebyly cítit ohněm.
#3:28 Nebúkadnesar zvolal: „Požehnán buď Bůh Šadrakův, Méšakův a Abed-negův, který poslal svého anděla a vysvobodil své služebníky, kteří na něj spoléhali. Přestoupili královo slovo a vydali svá těla, aby nemuseli vzdát poctu a klanět se nějakému jinému bohu než bohu svému.
#3:29 Vydávám rozkaz: Kdokoli z lidí kterékoli národnosti a jazyka by řekl něco proti Bohu Šadrakovu, Méšakovu a Abed-negovu, ať je rozsekán na kusy a jeho dům ať je učiněn hnojištěm, neboť není jiného Boha, který by mohl vyprostit jako tento Bůh.“
#3:30 A král zařídil, aby se Šadrakovi, Méšakovi a Abed-negovi v babylónské krajině dobře dařilo.
#3:31 Král Nebúkadnesar všem lidem různých národností a jazyků, kteří bydlí na celé zemi: „Rozhojněn buď váš pokoj!
#3:32 Zalíbilo se mi sdělit vám, jaká znamení a jaké divy učinil na mně Bůh nejvyšší.
#3:33 Jak veliká jsou jeho znamení, jak mocné jsou jeho divy! Jeho království je království věčné, jeho vladařská moc po všechna pokolení. 
#4:1 Já, Nebúkadnesar, jsem spokojeně pobýval ve svém domě, pln svěžesti ve svém paláci.
#4:2 Viděl jsem sen a ten mě vystrašil. Představy ve snu na lůžku, vidění, která i prošla hlavou, mě naplnily hrůzou.
#4:3 Vydal jsem rozkaz, aby ke mně byli uvedeni všichni babylónští mudrci, aby mi sen vyložili.
#4:4 Přišli tedy věštci, zaklínači, hvězdopravci a planetáři. Vyprávěl jsem jim sen, ale jeho výklad mi nemohli oznámit.
#4:5 Konečně ke mně přišel Daniel, který má jméno Beltšasar podle jména mého boha; v něm je duch svatých bohů. Vyprávěl jsem mu sen:
#4:6 Beltšasare, nejvyšší z věštců, vím, že v tobě je duch svatých bohů a že žádné tajemství ti nedělá potíže. Pověz mi výklad vidění snu, který jsem viděl.
#4:7 Ve viděních, která mi prošla hlavou na mém lůžku jsem viděl: Hle, strom stál uprostřed země, jeho výška byla obrovská.
#4:8 Strom rostl a sílil, až jeho výška sahala k nebi. Bylo jej vidět od samého konce země.
#4:9 Měl nádherné listí a mnoho plodů, byla na něm potrava pro všechny. Polní zvěř pod ním nalézala stín, v jeho větvích bydleli nebeští ptáci a sytilo se z něho všechno tvorstvo.
#4:10 Ve viděních, která mi prošla hlavou na mém lůžku, jsem viděl: Hle, posel, a to svatý, sestupoval z nebe.
#4:11 Mocně volal a nařizoval toto: ‚Skácejte strom! Osekejte mu větve! Otrhejte mu listí! Rozházejte jeho plody! Ať uteče zvěř, která byla pod ním, i ptáci z jeho větví!
#4:12 Avšak pařez s kořeny ponechte v zemi, sevřený obručí z železa a bronzu, ve svěží zeleni pole; ať je skrápěn nebeskou rosou a se zvěří ať se dělí o rostliny země.
#4:13 Jeho srdce ať je jiné, než je srdce lidské, ať je mu dáno srdce zvířecí, dokud nad ním neuplyne sedm let.
#4:14 V rozhodnutí nebeských poslů je rozsudek, výpovědí svatých je věc uzavřená. Z toho živí poznají, že Nejvyšší má moc nad lidským královstvím a komu chce, je dává; může nad ním ustanovit i nejnižšího z lidí.‘
#4:15 Tento sen jsem viděl já, král Nebúkadnesar, a ty, Beltšasare, mi řekni jeho výklad. Žádný z mudrců mého království mi nemohl výklad oznámit. Ty však jsi toho schopen, neboť v tobě je duch svatých bohů.“
#4:16 Tu Daniel, který měl jméno Beltšasar, zůstal skoro hodinu strnulý a jeho myšlenky ho plnily hrůzou. Král mu řekl: „Beltšasare, snu ani výkladu se nehroz.“ Beltšasar odpověděl: „Můj pane, kéž by sen platil tvým nepřátelům a jeho výklad tvým protivníkům.
#4:17 Strom, který jsi viděl, který rostl a sílil, až jeho výška sahala k nebi a bylo ho vidět po celé zemi,
#4:18 který měl nádherné listí a mnoho plodů, na němž byla potrava pro všechny, pod nímž bydlela polní zvěř a v jehož větvích přebývali nebeští ptáci,
#4:19 jsi ty, králi, který jsi rostl a sílil. Tvá velikost rostla, až dosáhla k nebi, tvá vladařská moc až na konec země.
#4:20 Král viděl potom posla, a to svatého, jak sestupoval z nebe a nařizoval: ‚Skácejte strom a zničte jej, avšak pařez s kořeny ponechte v zemi sevřený obručí z železa a bronzu ve svěží zeleni pole, ať je skrápěn nebeskou rosou a ať má podíl s polní zvěří, dokud nad ním neuplyne sedm let.‘
#4:21 Toto je výklad, králi: Je to rozhodnutí Nejvyššího, které dopadlo na krále, mého pána.
#4:22 Vyženou tě pryč od lidí a budeš bydlet s polní zvěří. Za pokrm ti dají rostliny jako dobytku a nechají tě skrápět nebeskou rosou. Tak nad tebou uplyne sedm let, dokud nepoznáš, že Nejvyšší má moc nad lidským královstvím a že je dává, komu chce.
#4:23 A že bylo řečeno, aby byl pařez toho stromu i s kořeny ponechán, tvé království se ti opět dostane, až poznáš, že Nebesa mají moc.
#4:24 Kéž se ti, králi, zalíbí má rada: Překonej své hříchy spravedlností a svá provinění milostí k strádajícím; snad ti bude prodloužen klid.“
#4:25 To všechno dopadlo na krále Nebúkadnesara.
#4:26 Uplynulo dvanáct měsíců. Král se procházel po královském paláci v Babylóně
#4:27 a řekl: „Zdali není veliký tento Babylón, který jsem svou mocí a silou vybudoval jako královský dům ke slávě své důstojnosti?“
#4:28 Ještě to slovo bylo v ústech krále, když se snesl hlas z nebe: „Tobě je to řečeno, králi Nebúkadnesare: Tvé království od tebe odešlo.
#4:29 Vyženou tě pryč od lidí a budeš bydlit s polní zvěří. Dají ti za pokrm rostliny jako dobytku. Tak nad tebou uplyne sedm let, dokud nepoznáš, že Nejvyšší má moc nad lidským královstvím a že je dává, komu chce.“
#4:30 V tu hodinu se to slovo na Nebúkadnesarovi splnilo. Byl vyhnán pryč od lidí, pojídal rostliny jako dobytek, jeho tělo bylo skrápěno nebeskou rosou, až mu narostly vlasy jako peří orlům a nehty jako drápy ptákům.
#4:31 „Když uplynuly ty dny, pozdvihl jsem já Nebúkadnesar své oči k nebi a rozum se mi vrátil. Dobrořečil jsem Nejvyššímu a chválil jsem a velebil Věčně živého, neboť jeho vladařská moc je věčná, jeho království po všechna pokolení.
#4:32 Všichni obyvatelé země jsou považováni za nic. Podle své vůle nakládá s nebeským vojskem i s obyvateli země. Není, kdo by mohl zabraňovat jeho ruce a ptát se ho: ‚Co to děláš?‘
#4:33 Tou dobou se mi vrátil rozum a ke slávě mého království mi opět byla vrácena má důstojnost a lesk. Moje královská rada a hodnostáři mě vyhledali, opět jsem byl dosazen do svého království a byla mi přidána mimořádná velikost.
#4:34 Nyní tedy já Nebúkadnesar chválím, vyvyšuji a velebím Krále nebes. Všechno jeho dílo je pravda, jeho cesty právo. Ty, kteří si vedou pyšně, má moc ponížit.“ 
#5:1 Král Belšasar vystrojil velikou hostinu svým tisíci hodnostářům a před těmito tisíci pil víno.
#5:2 Při popíjení vína poručil Belšasar přinést zlaté a stříbrné nádoby, které odnesl Nebúkadnesar, jeho otec, z jeruzalémského chrámu, aby z nich pili král i jeho hodnostáři, jeho ženy i ženiny.
#5:3 Hned tedy přinesli zlaté nádoby odnesené z chrámu, to je z Božího domu v Jeruzalémě, a pili z nich král i jeho hodnostáři, jeho ženy i ženiny.
#5:4 Pili víno a chválili bohy zlaté a stříbrné, bronzové, železné, dřevěné a kamenné.
#5:5 V tu hodinu se ukázaly prsty lidské ruky a něco psaly na omítku zdi královského paláce naproti svícnu. Král viděl zápěstí ruky, která psala.
#5:6 Tu se barva králova obličeje změnila a myšlenky ho naplnily hrůzou, poklesl v kyčlích a kolena mu tloukla o sebe.
#5:7 Král mocně zvolal, aby přivedli zaklínače, hvězdopravce a planetáře. Babylónským mudrcům král řekl: „Kdokoli přečte toto písmo a sdělí mi výklad, bude oblečen do purpuru, na krk mu bude dán zlatý řetěz a bude mít v království moc jako třetí po mně.“
#5:8 Všichni královští mudrci tedy vstoupili, ale nebyli schopni písmo přečíst a oznámit králi výklad.
#5:9 Král Belšasar byl pln hrůzy a barva jeho obličeje se změnila. I hodnostáři byli zmateni.
#5:10 Po slovech krále a hodnostářů vešla do domu, kde hodovali, královna a řekla: „Králi, navěky buď živ! Nechť tě tvé myšlenky neplní hrůzou a barva tvého obličeje ať se nemění.
#5:11 Je v tvém království muž, v němž je duch svatých bohů. Za dnů tvého otce bylo shledáno, že je osvícený a zběhlý v moudrosti, která je jako moudrost bohů. Král Nebúkadnesar, tvůj otec, ho ustanovil nejvyšším z věštců, zaklínačů, hvězdopravců a planetářů, ano, tvůj otec, králi,
#5:12 neboť bylo shledáno, že Daniel, jemuž král dal jméno Beltšasar, má mimořádného ducha a poznání a že je zběhlý ve vykládání snů, řešení záhad a vysvětlování věcí nesnadných. Nechť je Daniel nyní zavolán a sdělí výklad.“
#5:13 Daniel byl hned přiveden ke králi. Král se Daniela otázal: „Ty jsi Daniel z judských přesídlenců, kterého přivedl král, můj otec, z Judska?
#5:14 Slyšel jsem o tobě, že je v tobě duch bohů a že bylo shledáno, že jsi osvícený a zběhlý v mimořádné moudrosti.
#5:15 Byli ke mně přivedeni mudrci, zaklínači, aby mi přečetli toto písmo a oznámili mi jeho výklad, ale nejsou schopni výklad té věci sdělit.
#5:16 O tobě jsem slyšel, že dokážeš podat výklad a vysvětlit nesnadné. Nyní tedy, dokážeš-li to písmo přečíst a výklad mi oznámit, budeš oblečen do purpuru, na krk ti bude dán zlatý řetěz a budeš mít v království moc jako třetí.“
#5:17 Daniel na to králi odpověděl: „Své dary si ponech a své odměny dej jinému. To písmo však králi přečtu a výklad mu oznámím.
#5:18 Slyš, králi! Bůh nejvyšší dal Nebúkadnesarovi, tvému otci, království a velikost, slávu a důstojnost.
#5:19 Pro velikost, kterou mu dal, se před ním třásli všichni lidé různých národností a jazyků a obávali se ho. Koho chtěl, zabil, koho chtěl, nechal žít, koho chtěl, povýšil, koho chtěl, ponížil.
#5:20 Když se jeho srdce povýšilo a jeho duch se stal náramně zpupný, byl svržen ze svého královského stolce a jeho sláva mu byla odňata.
#5:21 Byl vyhnán pryč od lidí, jeho srdce se stalo podobné zvířecímu, bydlil s divokými osly, za pokrm mu dávali rostliny jako dobytku a jeho tělo bylo skrápěno nebeskou rosou, dokud nepoznal, že Bůh nejvyšší má moc nad lidským královstvím a že nad ním ustanovuje, koho chce.
#5:22 Ani ty, jeho synu Belšasare, jsi neponížil své srdce, ačkoli jsi o tom všem věděl,
#5:23 ale povýšil ses nad Pána nebes. Přinesli před tebe nádoby z jeho domu a pil jsi z nich víno ty i tvoji hodnostáři, tvé ženy i ženiny, a chválil jsi bohy stříbrné a zlaté, bronzové, železné, dřevěné a kamenné, kteří nic nevidí, neslyší ani nevědí. Boha, v jehož rukou je tvůj dech a všechny tvé cesty, jsi však nevelebil.
#5:24 Proto bylo od něho posláno zápěstí ruky a napsáno toto písmo.
#5:25 Toto pak je písmo, které bylo napsáno: ‚Mené, mené, tekel ú-parsín‘.
#5:26 Toto je výklad těch slov: Mené - Bůh sečetl tvé kralování a ukončil je.
#5:27 Tekel - byl jsi zvážen na vahách a shledán lehký.
#5:28 Peres - tvé království bylo rozlomeno a dáno Médům a Peršanům.“
#5:29 Belšasar hned poručil, aby Daniela oblékli do purpuru, na krk mu dali zlatý řetěz a rozhlásili o něm, že má v království moc jako třetí.
#5:30 Ještě té noci byl kaldejský král Belšasar zabit. 
#6:1 Darjaveš médský se ujal království ve věku šedesáti dvou let.
#6:2 Darjavešovi se zalíbilo ustanovit nad královstvím sto dvacet satrapů, aby byli po celém království.
#6:3 Nad nimi byli tři říšští vládcové, z nichž jedním byl Daniel. Těm podávali satrapové hlášení, aby se tím král nemusel obtěžovat.
#6:4 Daniel pak vynikal nad říšské vládce i satrapy, neboť v něm byl mimořádný duch. Král ho zamýšlel ustanovit nad celým královstvím.
#6:5 Tu se říšští vládcové a satrapové snažili nalézt proti Danielovi záminku ohledně jeho správy království, ale žádnou záminku ani zlé jednání nalézt nemohli, neboť byl věrný. Žádnou nedbalost ani zlé jednání na něm neshledali.
#6:6 Proto si ti muži řekli: „Nenajdeme proti Danielovi žádnou záminku, ledaže bychom našli proti němu něco, co se týče zákona jeho Boha.“
#6:7 Říšští vládcové a satrapové se shlukli ke králi a naléhali na něj: „Králi Darjaveši, navěky buď živ!
#6:8 Všichni královští vládci, zemští správcové a satrapové, královská rada a místodržitelé se uradili, abys královským výnosem potvrdil zákaz: Každý, kdo by se v údobí třiceti dnů obracel v modlitbě na kteréhokoli boha nebo člověka kromě na tebe, králi, ať je vhozen do lví jámy.
#6:9 Nyní, králi, vydej zákaz a podepiš přípis, který by podle nezrušitelného zákona Médů a Peršanů nesměl být změněn.“
#6:10 Král Darjaveš tedy podepsal přípis a zákaz.
#6:11 Když se Daniel dověděl, že byl podepsán přípis, vešel do svého domu, kde měl v horní pokoji otevřená okna směrem k Jeruzalému. Třikrát za den klekal na kolena, modlil se a vzdával čest svému Bohu, jako to činíval dříve.
#6:12 Tu se ti muži shlukli a přistihli Daniela, jak se modlí a prosí svého Boha o milost.
#6:13 Hned šli ke králi a dovolávali se královského zákazu: „Zdali jsi nepodepsal zákaz, že každý člověk, který by se v údobí třiceti dnů modlil ke kterémukoli bohu nebo člověku kromě k tobě, králi, bude vhozen do lví jámy?“ Král odpověděl: „To slovo platí podle nezrušitelného zákona Médů a Peršanů.“
#6:14 Na to králi řekli: „Daniel z judských přesídlenců, králi, na tebe a zákaz, který si podepsal nedbá. Třikrát za den se modlí svou modlitbu.“
#6:15 Když král slyšel takovou řeč, byl velmi znechucen. Usilovně přemýšlel, jak by Daniela vysvobodil. Namáhal se až do západu slunce, jak by ho vyprostil.
#6:16 Ti muži se však shlukli ke králi a naléhali na něj: „Věz, králi, podle zákona Médů a Peršanů žádný zákaz ani výnos, který král vydá, nesmí být změněn.“
#6:17 Král tedy poručil, aby přivedli Daniela a vhodili ho do jámy, v níž byli lvi. Danielovi řekl: „Kéž tě tvůj Bůh, kterého stále uctíváš, vysvobodí.“
#6:18 Donesli jeden kámen a položili ho na otvor jámy. Král jej zapečetil pečetním prstenem svým a pečetními prsteny svých hodnostářů, aby se v Danielově záležitosti nedalo nic změnit.
#6:19 Pak se král odebral do svého paláce a ulehl, aniž co pojedl. Nedopřál si žádné obveselení a spánek se mu vyhýbal.
#6:20 Jak se začalo rozednívat, hned za úsvitu, král vstal a chvatně odešel k jámě, kde byli lvi.
#6:21 Když přišel k jámě, zarmouceným hlasem zavolal na Daniela. Řekl Danielovi: „Danieli, služebníku Boha živého, dokázal tě Bůh, kterého stále uctíváš, zachránit před lvy?“
#6:22 Tu Daniel promluvil ke králi: „Králi, navěky buď živ!
#6:23 Můj Bůh poslal a svého anděla a zavřel ústa lvům, takže mi neublížili. Vždyť jsem byl před ním shledán čistý a ani proti tobě, králi, jsem se ničeho zlého nedopustil.“
#6:24 Král tím byl velice potěšen a poručil, aby Daniela vytáhli z jámy. Daniel byl tedy z jámy vytažen a nebyla na něm shledána žádná úhona, protože věřil ve svého Boha.
#6:25 Král pak poručil, aby přivedli ty muže, kteří Daniela udali, a hodili je do lví jámy i s jejich syny a ženami. Ještě nedopadli na dno jámy, už se jich zmocnili lvi a rozdrtili jim všechny kosti.
#6:26 Tehdy král Darjaveš napsal všem lidem různých národností a jazyků, bydlícím po celé zemi: „Rozhojněn buď váš pokoj!
#6:27 Vydávám rozkaz, aby se v celé mé královské říši všichni třásli před Danielovým Bohem a obávali se ho, neboť on je Bůh živý a zůstává navěky, jeho království nebude zničeno a jeho vladařská moc bude až do konce.
#6:28 Vysvobozuje a vytrhuje, činí znamení a divy na nebi i na zemi. On vysvobodil Daniela ze lvích spárů.“
#6:29 Danielovi se pak dobře dařilo v království Darjavešově i v království Kýra perského. 
#7:1 V prvním roce vlády Belšasara, krále babylónského, viděl Daniel sen, vidění mu prošla hlavou na jeho lůžku. Hned tedy v hlavních rysech ten sen popsal.
#7:2 Daniel řekl: „Viděl jsem v nočním vidění, hle, čtyři nebeské větry rozbouřily Velké moře.
#7:3 A z moře vystoupila čtyři veliká zvířata, odlišná jedno od druhého.
#7:4 První bylo jako lev a mělo orlí křídla. Viděl jsem, že mu byla křídla oškubána, bylo pozvednuto od země a postaveno na nohy jako člověk a dáno lidské srdce.
#7:5 Hle, další zvíře, druhé, se podobalo medvědu. Bylo postaveno tváří k jedné straně. Mělo v tlamě mezi zuby tři žebra a bylo mu řečeno: ‚Vstaň a hojně se nažer masa!‘
#7:6 Potom jsem viděl, hle, další zvíře bylo jako levhart a mělo na hřbetě čtyři ptačí křídla. Bylo to zvíře čtyřhlavé a byla mu dána vladařská moc.
#7:7 Potom jsem v nočním vidění viděl, hle, čtvrté zvíře, strašné, příšerné a mimořádně mocné. Mělo veliké železné zuby, žralo a drtilo a zbytek rozšlapávalo svýma nohama. Bylo odlišné ode všech předešlých zvířat a mělo deset rohů.
#7:8 Prohlížel jsem rohy, a hle, vyrostl mezi nimi další malý roh a tři z dřívějších rohů byly před ním vyvráceny. Hle, na tom rohu byly oči jako oči lidské a ústa, která mluvila troufale.
#7:9 Viděl jsem, že byly postaveny stolce a že usedl Věkovitý. Jeho oblek byl bílý jako sníh, vlasy jeho hlavy jako čistá vlna, jeho stolec - plameny ohně, jeho kola - hořící oheň.
#7:10 Řeka ohnivá proudila a vycházela od něho, tisíce tisíců sloužily jemu a desetitíce desetitisíců stály před ním. Zasedl soud a byly otevřeny knihy.
#7:11 Tu jsem viděl, že pro ta troufalá slova, která roh mluvil, viděl jsem, že to zvíře bylo zabito, jeho tělo zničeno a dáno k spálení ohněm.
#7:12 Zbylým zvířatům odňali jejich vladařskou moc a byl jim ponechán život do určité doby a času.
#7:13 Viděl jsem v nočním vidění, hle, s nebeskými oblaky přicházel jakoby Syn člověka; došel až k Věkovitému, přivedli ho k němu.
#7:14 A byla mu dána vladařská moc, sláva a království, aby ho uctívali všichni lidé různých národností a jazyků. Jeho vladařská moc je věčná, která nepomine, a jeho království nebude zničeno.“
#7:15 Můj duch, můj, Danielův, byl uvnitř své schránky zmatený a vidění, která mi prošla hlavou, mě naplnila hrůzou.
#7:16 Přistoupil jsem k jednomu z těch, kteří tam stáli, a prosil jsem ho o hodnověrný výklad toho všeho. Řekl mi to a oznámil mi výklad té věci:
#7:17 „Ta čtyři veliká zvířata, to čtyři králové povstanou v zemi.
#7:18 Ale království se ujmou svatí Nejvyššího a budou mít království v držení až na věky, totiž až na věky věků.“
#7:19 Chtěl jsem mít jistotu o tom čtvrtém zvířeti, které bylo odlišné ode všech ostatních a bylo mimořádně strašné: mělo železné zuby, bronzové drápy, žralo a drtilo a zbytek rozšlapávalo svýma nohama,
#7:20 i o deseti rozích, které mělo na hlavě, a o dalším, který vyrostl a před nímž tři spadly, totiž o tom rohu, který měl oči a ústa mluvící troufale a jevil se větší než ostatní.
#7:21 Viděl jsem, že ten roh vedl válku proti svatým a přemáhal je,
#7:22 až přišel Věkovitý a soud byl předán svatým Nejvyššího; nadešla doba a království dostali do držení svatí.
#7:23 Řekl toto: „Čtvrté zvíře - na zemi bude čtvrté království, to se bude ode všech království lišit; pozře celou zemi, podupe ji a rozdrtí.
#7:24 A deset rohů - z toho království povstane deset králů. Po nich povstane jiný, ten se bude od předchozích lišit a sesadí tři krále.
#7:25 Bude mluvit proti Nejvyššímu a bude hubit svaté Nejvyššího. Bude se snažit změnit doby a zákon. Svatí budou vydáni do jeho rukou až do času a časů a poloviny času,
#7:26 avšak zasedne soud a vladařskou moc mu odejmou, a bude úplně vyhlazen a zahuben.
#7:27 Království, vladařská moc a velikost všech království pod celým nebem budou dány lidu svatých Nejvyššího. Jeho království bude království věčné a všechny vladařské moci ho budu uctívat a poslouchat.“
#7:28 Zde končí to slovo. Já, Daniel, jsem se velice zhrozil těch myšlenek, barva mé tváře se změnila, ale to slovo jsem uchoval ve svém srdci. 
#8:1 V třetím roce kralování krále Belšasara ukázalo se mně, Danielovi, vidění, po onom, které se mi ukázalo na počátku.
#8:2 Viděl jsem ve vidění - byl jsem ve vidění na hradě Šúšanu v élamské krajině - viděl jsem tedy ve vidění, že jsem u řeky Úlaje:
#8:3 Pozvedl jsem oči a spatřil jsem, hle, jeden beran stál před řekou a měl dva rohy. Ty rohy byly veliké, jeden však byl větší než druhý; větší vyrostl jako poslední.
#8:4 Viděl jsem berana trkat směrem k moři, na sever a na jih. Žádné zvíře před ním neobstálo a nikdo nic nevyprostil z jeho moci. Dělal, co se mu zlíbilo, a vzmohl se.
#8:5 Pozoroval jsem, a hle, ze západu přicházel kozel na celou zemi, ale země se nedotýkal. Ten kozel měl mezi očima nápadný roh.
#8:6 Přišel až k dvourohému beranovi, kterého jsem viděl stát nad řekou. Přiběhl k němu silně rozzuřen.
#8:7 Viděl jsem, že dostihl berana, rozlíceně do něho vrazil a zlomil mu oba rohy a beran neměl sílu mu odolat. Kozel ho povalil na zem a rozšlapal a nebyl nikdo, kdo by berana vyprostil z jeho moci.
#8:8 Kozel se velice vzmohl. Když byl na vrcholu moci, zlomil se ten velký roh a místo něho vyrostly čtyři nápadné rohy do čtyř nebeských větrů.
#8:9 Z jednoho z nich vyrazil jeden maličký roh, který se velmi vzmáhal na jih a na východ i k nádherné zemi.
#8:10 Vzmohl se tak, že sahal až k nebeskému zástupu, srazil na zem část toho zástupu, totiž hvězd, a rozšlapal je.
#8:11 Vypjal se až k veliteli toho zástupu, zrušil každodenní oběť a rozvrátil příbytek jeho svatyně.
#8:12 Zástup byl sveden ke vzpouře proti každodenní oběti. Pravdu srazil na zem a dařilo se mu, co činil.
#8:13 Slyšel jsem, jak jeden svatý mluví. Jiný svatý se toho mluvícího otázal: „Jak dlouho bude platit vidění o každodenní oběti a o vzpouře, která pustoší a dovoluje šlapat po svatyni i zástupu?“
#8:14 Řekl mi: „Až po dvou tisících a třech stech večerech a jitrech dojde svatyně spravedlnosti.“
#8:15 Když jsem já Daniel uviděl to vidění a snažil se mu porozumět, hle, postavil se naproti mně kdosi podobný muži
#8:16 a uslyšel jsem nad Úlajem lidský hlas, který takto volal: „Gabrieli, vysvětli mu to vidění.“
#8:17 Přišel tedy tam, kde jsem stál; zatímco přicházel, byl jsem ohromen a padl jsem tváří k zemi. Řekl mi: „Pochop, lidský synu, že to vidění se týká doby konce.“
#8:18 Když se mnou mluvil, ležel jsem v mrákotách tváří na zemi. Dotkl se mě a postavil mě na mé místo.
#8:19 Řekl : „Hle, sdělím ti, co se stane v posledním hrozném hněvu, neboť se to týká konce času.
#8:20 Dvourohý beran, kterého jsi viděl, jsou králové médští a perští.
#8:21 Chlupatý kozel je král řecký a veliký roh, který měl mezi očima, je první král.
#8:22 To, že se roh zlomil a místo něho vyvstaly čtyři, znamená, že vyvstanou čtyři království z toho pronároda, ale nebudou mít jeho sílu.
#8:23 Ke konci jejich kralování, až se naplní míra vzpurných, povstane král nestoudný, který bude rozumět hádankám.
#8:24 Bude zdatný svou silou, a nejen svou silou, bude přinášet neobyčejnou zkázu a jeho konání bude provázet zdar. Uvrhne do zkázy zdatné a lid svatých.
#8:25 Obezřetně a se zdarem bude jeho ruka lstivě jednat; ve svém srdci se bude vypínat, nerušeně uvrhne do zkázy mnohé. Postaví se proti Veliteli velitelů, avšak bude zlomen bez zásahu ruky.
#8:26 Vidění o večerech a jitrech, jak ti bylo pověděno, je pravdivé. Ty pak podrž to vidění v tajnosti, neboť se uskuteční za mnoho dnů.“
#8:27 Já Daniel jsem zůstal bez sebe a byl jsem těžce nemocen po řadu dní. Pak jsem vstal a konal své povolání u krále. Žasl jsem nad tím viděním, ale nikdo to nechápal. 
#9:1 V prvním roce vlády Darjaveše, syna Achašvérošova, který byl médského původu a byl králem nad královstvím kaldejským,
#9:2 v tom prvním roce jeho kralování jsem já Daniel porozuměl z knih počtu roků, o nichž se stalo slovo Hospodinovo proroku Jeremjášovi; vyplní se, že Jeruzalém bude po sedmdesát let v troskách.
#9:3 Obrátil jsem se k Panovníku Bohu, abych ho vyhledal modlitbou a prosbami o smilování v postu, žíněném rouchu a popelu.
#9:4 Modlil jsem se k Hospodinu, svému Bohu, a vyznával se mu slovy: „Ach, Panovníku, Bože veliký a hrozný, který dbáš na smlouvu a milosrdenství vůči těm, kteří tě milují a dodržují tvá přikázání!
#9:5 Zhřešili jsme a provinili se, jednali jsme svévolně, bouřili se a uchýlili od tvých přikázání a soudů.
#9:6 A neposlouchali jsme tvé služebníky proroky, kteří mluvili ve tvém jménu našim králům, našim velmožům, našim otcům a všemu lidu země.
#9:7 Na tvé straně, Panovníku, je spravedlnost, na nás je zjevná hanba až do tohoto dne; je na každém judském muži, na obyvatelích Jeruzaléma, na celém Izraeli, na blízkých i dalekých, ve všech zemích, do nichž jsi je rozehnal pro jejich zpronevěru, které se vůči tobě dopustili.
#9:8 Hospodine, na nás je zjevná hanba, na našich králích, na našich velmožích a na našich otcích, neboť jsme proti tobě zhřešili.
#9:9 Na Panovníku, našem Bohu, závisí slitování a odpuštění, neboť jsme se bouřili proti němu
#9:10 a neposlouchali jsme Hospodina, našeho Boha, a neřídili se jeho zákony, které nám vydával skrze své služebníky proroky.
#9:11 Celý Izrael přestoupil tvůj zákon a odchýlil se a neposlouchal tebe. Na nás je vylita kletba a zlořečení, jak je o tom psáno v zákoně Mojžíše, Božího služebníka, protože jsme proti tobě hřešili.
#9:12 Dodržel jsi své slovo, které jsi promluvil proti nám a proti našim soudcům, kteří nás soudili, že uvedeš na nás zlo tak veliké, že takové nebylo učiněno pod celým nebem; tak bylo učiněno v Jeruzalémě.
#9:13 Jak je psáno v zákoně Mojžíšově, přišlo na nás všechno to zlo. Neprosili jsme Hospodina, svého Boha, o shovívavost a neodvrátili se od svých nepravostí a nejednali prozíravě podle jeho pravdy.
#9:14 Proto Hospodin bděl nad tím zlem a uvedl je na nás, neboť Hospodin, náš Bůh, je spravedlivý ve všech svých činech, které koná, ale my jsme ho neposlouchali.
#9:15 Nyní, Panovníku, náš Bože, který jsi svůj lid vyvedl z egyptské země pevnou rukou, a tak sis učinil jméno, jaké máš až dodnes, zhřešili jsme, byli jsme svévolní.
#9:16 Panovníku, nechť se prosím podle tvé veškeré spravedlnosti odvrátí tvůj hněv a tvé rozhořčení od tvého města Jeruzaléma, tvé svaté hory, neboť pro naše hříchy a viny našich otců je Jeruzalém a tvůj lid tupen všemi, kteří jsou kolem nás.
#9:17 Nyní tedy, Bože náš, slyš modlitbu svého služebníka a jeho prosby o smilování a rozjasni tvář nad svou zpustošenou svatyní, kvůli sobě, Panovníku.
#9:18 Nakloň, můj Bože, své ucho a slyš, otevři své oči a viz, jak jsme zpustošeni my i město, které se nazývá tvým jménem. Vždyť ne pro své spravedlivé činy ti předkládáme své prosby o smilování, ale pro tvé velké slitování.
#9:19 Panovníku, slyš! Panovníku, odpusť! Panovníku, pozoruj a jednej! Neprodlévej! Kvůli sobě, můj Bože! Vždyť tvé město i tvůj lid se nazývá tvým jménem.“
#9:20 Ještě jsem rozmlouval a modlil se, vyznával hřích svůj i hřích Izraele, svého lidu, a předkládal Hospodinu, svému Bohu své prosby o smilování za svatou horu Boží,
#9:21 tedy ještě jsem rozmlouval v modlitbě, když Gabriel, ten muž, kterého jsem prve viděl ve vidění, spěšně přilétl a dotkl se mě v době večerního obětního daru.
#9:22 Poučil mě, když se mnou mluvil. Řekl: „Danieli, nyní jsem vyšel, abych ti posloužil k poučení.
#9:23 Prve při tvých prosbách o smilování vyšlo slovo a já jsem přišel, abych ti je oznámil, neboť jsi vzácný Bohu. Pochop to slovo a rozuměj vidění.
#9:24 Sedmdesát týdnů let je stanoveno tvému lidu a tvému svatému městu, než bude skoncováno s nevěrností, než budou zapečetěny hříchy, než dojde k zproštění viny, k uvedení věčné spravedlnosti, k zapečetění vidění a proroctví, k pomazání svatyně svatých.
#9:25 Věz a pochop! Od vyjití slova o navrácení a vybudování Jeruzaléma až k pomazanému vévodovi uplyne sedm týdnů. Za šedesát dva týdny bude opět vybudováno prostranství a příkop. Ale budou to svízelné doby.
#9:26 Po uplynutí šedesáti dvou týdnů bude pomazaný zahlazen a nebude již. Město a svatyni uvrhne do zkázy lid vévody, který přijde. Sám skončí v povodni, ale až do konce bude válka. Je rozhodnuto o pustošení.
#9:27 Vnutí svou smlouvu mnohým v jednom týdnu a v polovině toho týdne zastaví obětní hod i oběť přídavnou. Hle, pustošitel na křídlech ohyzdné modly, než se naplní čas a na pustošitele bude vylito rozhodnutí.“ 
#10:1 V třetím roce vlády Kýra, krále perského, bylo Danielovi, který byl pojmenován Beltšasar, zjeveno slovo. A to slovo je pravdivé; týká se velké strasti. Pochopil to slovo. Pochopení mu bylo dáno ve viděních.
#10:2 V těch dnech jsem já, Daniel, truchlil po celé tři týdny.
#10:3 Chutný chléb jsem nejedl, maso a víno jsem nevzal do úst, ani jsem se nepotíral mastí až do uplynutí celých tří týdnů.
#10:4 Dvacátého čtvrtého dne prvního měsíce jsem byl na břehu veliké řeky zvané Chidekel.
#10:5 Pozvedl jsem oči a spatřil jsem, hle, jakýsi muž oblečený ve lněném oděvu. Na bedrech měl pás z třpytivého zlata z Úfazu.
#10:6 Tělo měl jako chrysolit, tvář jako blesk, oči jako hořící pochodně, paže a nohy jako lesknoucí se bronz a zvuk jeho slov byl jako hrozný hluk.
#10:7 Já Daniel jsem to vidění viděl sám. Muži, kteří byli se mnou, žádné vidění neviděli, ale padlo na ně veliké zděšení a uprchli do úkrytu.
#10:8 Zůstal jsem sám a viděl jsem to veliké vidění, ale nezůstala ve mně síla. Velebnost mé tváře se změnila a byla zcela porušena; nezachoval jsem si sílu.
#10:9 Slyšel jsem zvuk jeho slov a jak jsem zvuk jeho slov uslyšel, přišly na mě mrákoty a padl jsem na tvář, totiž tváří na zem.
#10:10 A hle, dotkla se mě ruka a zatřásla mnou, takže jsem se pozvedl na kolena a dlaně rukou.
#10:11 Muž mi řekl: „Danieli, muži vzácný, pochop slova, která k tobě budu mluvit. Stůj na svém místě. Jsem poslán k tobě.“ Když se mnou mluvil to slovo, stál jsem a chvěl jsem se.
#10:12 Řekl mi: „Neboj se, Danieli, neboť od prvního dne, kdy ses rozhodl porozumět a pokořit se před svým Bohem, byla tvá slova vyslyšena a já jsem proto přišel.
#10:13 Avšak ochránce perského království stál proti mně po jednadvacet dní. Dokud mi nepřišel na pomoc Míkael, jeden z předních ochránců, zůstal jsem tam u perských králů.
#10:14 Přišel jsem, abych tě poučil o tom, co potká tvůj lid v posledních dnech, neboť vidění se týká těchto dnů.“
#10:15 Když ke mně promluvil tato slova, sklonil jsem tvář k zemi a oněměl jsem.
#10:16 A hle, kdosi podobný lidským synům se dotkl mých rtů. Otevřel jsem ústa a promluvil jsem k tomu, který stál proti mně: „Můj pane, pro to vidění mě sevřely křeče a nezachoval jsem si sílu.
#10:17 Jak by mohl služebník tohoto mého pána mluvit s tímto mým pánem? Od té doby není ve mně síla a nezůstal ve mně dech.“
#10:18 Opět se mě dotkl kdosi, kdo vypadal jako člověk, a dodal mi sílu.
#10:19 Řekl: „Neboj se, muži vzácný, pokoj tobě! Vzchop se, vzchop se!“ Když se mnou mluvil, nabyl jsem síly. Řekl jsem: „Ať můj pán mluví, neboť jsem již posílen.“
#10:20 Řekl: „Víš, proč jsem k tobě přišel? Nyní se opět vrátím, abych bojoval s ochráncem Peršanů. Odcházím, a hle, přichází ochránce Řeků.
#10:21 Zajisté, oznámím ti, co je zapsáno ve spisu pravdy. Není nikoho, kdo by mi dodával sílu v těch věcech, kromě vašeho ochránce Míkaela.“ 
#11:1 „V prvním roce vlády Darjaveše médského jsem stál při něm, abych mu dodával sílu a byl mu záštitou.
#11:2 Nyní ti tedy oznámím pravdu: Hle, v Persii povstanou ještě tři králové. Čtvrtý pak nabude většího bohatství než ostatní. Jakmile získá sílu ze svého bohatství, vyburcuje všechno proti řeckému království.
#11:3 Povstane však bohatýrský král, bude vládnout nad obrovskou říší a dělat, co se mu zlíbí.
#11:4 Ale až bude pevně stát, bude jeho království rozlomeno, rozděleno podle čtyř nebeských větrů, avšak ne jeho potomkům; nebude to už říše, jaké vládl on, neboť jeho království bude rozvráceno a dostane se jiným, nikoli jím.
#11:5 Tu se vzmůže král Jihu, ale jeden z jeho velitelů se vzmůže víc než on a bude vládnout. Bude vládnout nad nesmírnou říší.
#11:6 Po uplynutí několika let se spojí: dcera krále Jihu přijde ke králi Severu, aby ujednala smír; avšak paže si neuchová sílu, jeho paže neobstojí. Dcera bude v té době vydána záhubě i s těmi, kteří ji přivedli, i s tím, který ji zplodil, a s tím, kdo jí byl oporou.
#11:7 Avšak na jeho místo postoupí výhonek z jejích kořenů, přitáhne proti vojsku a vejde do pevnosti krále Severu a bude tam prosazovat svou sílu.
#11:8 I jejich bohy s jejich litými sochami a vzácnými nádobami, stříbro a zlato odveze do zajetí do Egypta; pak několik let nechá krále Severu na pokoji.
#11:9 Ten sice vtrhne do království krále Jihu, ale vrátí se do své země.
#11:10 Jeho synové však budou válčit a shromáždí nesmírné množství vojska. Jeden z nich přitáhne, přižene se jako povodeň a bude válčit dál až k jeho pevnosti.
#11:11 Král Jihu se rozhořčí, vytáhne a bude sním, s králem Severu, bojovat; ten sice postaví nesmírné množství vojska, ale to množství bude vydáno do rukou onoho.
#11:12 To množství bude odvedeno. Jeho srdce zpychne. Pobije desetitisíce, avšak moc si neudrží.
#11:13 Král Severu se vrátí a postaví ještě nesmírnější množství, než bylo první, a po určité době, po několika letech, přitáhne s velikým vojskem a obrovským vybavením.
#11:14 V té době se mnozí postaví proti králi Jihu a synové rozvratníků tvého lidu se pozdvihnou, aby se potvrdilo vidění, avšak klesnou.
#11:15 Král Severu opět přitáhne, nasype náspy a zmocní se opevněného města. Paže Jihu neobstojí, ani lid jeho vybraných sborů nebude mít sílu, aby obstál.
#11:16 Ten, který přitáhne proti němu, bude dělat, co se mu zlíbí, a nikdo před ním neobstojí. Zastaví se i v nádherné zemi a jeho ruka přinese zkázu.
#11:17 Pojme úmysl zmocnit se celého království, ujedná s ním smír a dá mu jednu z dcer, aby království strhla do zkázy. Ale ona nebude stát při něm.
#11:18 Pak obrátí tvář k ostrovům a mnohých se zmocní, avšak jeden konsul učiní přítrž jeho tupení; a nejen to: jeho tupení mu oplatí.
#11:19 Obrátí tedy tvář k pevnostem své země, ale klesne, padne a nenajdou ho.
#11:20 Na jeho místo povstane někdo, kdo dá projít výběrčímu vznešeným královstvím, ale po několika letech bude zlomen, a to ani hněvem ani bojem.
#11:21 Na jeho místo povstane Opovrženíhodný. Královská důstojnost mu nebude udělena, nýbrž nerušeně přitáhne a úskočně se zmocní království.
#11:22 Jako povodní budou před ním odplaveny a zlomeny paže protivníků, právě tak i vůdce smlouvy.
#11:23 S těmi, kteří se s ním spojí, bude jednat lstivě. Opovážlivě vytáhne s hrstkou pronároda.
#11:24 Nerušeně přitáhne do žirných krajin a učiní, co nečinili jeho otcové ani otcové jeho otců. Rozdělí mezi své lidi loupež, kořist a majetek a opevněným místům bude strojit úklady, až do času.
#11:25 Vyburcuje svou sílu i srdce proti králi Jihu, potáhne s velikým vojskem. Král Jihu s vojskem velice velkým a zdatným se s ním utká v boji, ale neobstojí, protože mu budou nastrojeny úklady.
#11:26 Ti totiž, kteří jídají jeho lahůdky, ho zlomí, jeho vojsko bude odplaveno, mnoho jich bude skoleno a padne.
#11:27 Srdce obou těch králů budou plná zloby a u jednoho stolu budou mluvit lež, ale nezdaří se to, neboť konec je ještě odložen do jistého času.
#11:28 Navrátí se tedy do své země s velikým jměním, ale jeho srdce bude proti svaté smlouvě. Podle toho bude jednat; pak se vrátí do své země.
#11:29 Po jistém čase opět potáhne proti Jihu, ale podruhé to nebude tak jako poprvé.
#11:30 Přitáhne na něj loďstvo Kitejců a bude zkrušen. Opět bude soptit a jednat proti svaté smlouvě. Obrátí se a přikloní k těm, kdo opustili svatou smlouvu.
#11:31 Jeho paže se napřáhnou a znesvětí svatyni i pevnost, vymýtí kadodenní oběť a dají tam ohyzdnou modlu pustošitele.
#11:32 Ty, kteří jednají svévolně vůči smlouvě, přivede úslisnostmi k rouhání. Avšak lid, ti, kteří se znají ke svému Bohu, zůstanou pevní a budou podle toho jednat.
#11:33 Prozíraví v lidu budou poučovat mnohé, ale budou po nějaký čas klesat pod mečem a plamenem, zajetím a loupeží.
#11:34 Když budou klesat, naleznou trochu pomoci, ale mnozí se k nim připojí úskočně.
#11:35 Někteří z prozíravých budou klesat, budou zkoušeni, tříbeni a běleni pro dobu konce, totiž do jistého času.
#11:36 Král bude dělat, co se mu zlíbí, bude se vypínat a činit větším nad každého boha a bude divně mluvit i proti Bohu bohů, co mu nepřísluší, a bude ho provázet zdar, dokud se nedovrší hrozný hněv, neboť rozhodnutí bude vykonáno.
#11:37 Nepřikloní se ani k bohům svých otců ani k Oblíbenci žen, nepřikloní se k žádnému bohu, neboť se bude činit větším nade všechny.
#11:38 Jen boha pevností bude ctít na jeho místě, zlatem, stříbrem, drahokamem a drahocennostmi bude ctít boha, kterého jeho otcové neznali.
#11:39 Cizího boha uvede do opevněných pevností; kdo ho uzná, toho zahrne slávou. Takovým svěří vládu nad mnohými a jako odměnu jim přidělí půdu.
#11:40 V době konce se s ním srazí král Jihu, ale král Severu proti němu zaútočí s vozbou a jízdou a obrovským loďstvem. Přitáhne proti zemím, zaplaví je a potáhne dál.
#11:41 Přitáhne i do nádherné země a mnozí klesnou. Z jeho rukou uniknou jen tito: Edóm, Moáb a přední z Amónovců.
#11:42 Vztáhne svou ruku na četné země; ani egyptská země nevyvázne.
#11:43 Získá vládu nad skrytými poklady zlata a stříbra a všemi egyptskými drahocennostmi. V jeho průvodu budou i Lúbijci a Kúšijci.
#11:44 Ale vyděsí ho zprávy z východu a ze severu. Vytáhne s velikým rozhořčením, aby mnohé zahladil a vyhubil jako klaté.
#11:45 Postaví své přepychové stany od moří k hoře svaté nádhery. Pak přijde jeho konec a nikdo mu nepomůže.“ 
#12:1 „V oné době povstane Míkael, velký ochránce, a bude stát při synech tvého lidu. Bude to doba soužení, jaké nebylo od vzniku národa až do této doby. V oné době bude vyproštěn tvůj lid, každý, kdo je zapsán v Knize.
#12:2 Mnozí z těch, kteří spí v prachu země, procitnou; jedni k životu věčnému, druzí k pohaně a věčné hrůze.
#12:3 Prozíraví budou zářit jako záře oblohy, a ti, kteří mnohým dopomáhají k spravedlnosti, jako hvězdy, navěky a navždy.
#12:4 A ty, Danieli, udržuj ta slova v tajnosti a zapečeť tuto knihu až do doby konce. Mnozí budou zmateně pobíhat, ale poznání se rozmnoží.“
#12:5 Já Daniel jsem viděl toto: Hle, povstali dva další muži, jeden na tomto břehu veletoku, druhý na onom břehu veletoku,
#12:6 a ten se otázal muže oblečeného ve lněném oděvu, který byl nad vodami veletoku: „Kdy nastane konec těch podivuhodných věcí?“
#12:7 I slyšel jsem muže oblečeného ve lněném oděvu, který byl nad vodami veletoku. Zvedl svou pravici i levici k nebi a přísahal při Živém navěky, že k času a časům a k polovině, až se dovrší roztříštění moci svatého lidu, dovrší se i všechno toto.
#12:8 Slyšel jsem, ale neporozuměl jsem. Řekl jsem: „Můj pane, jaké bude zakončení toho všeho?“
#12:9 Řekl: „Jdi, Danieli, tajuplná a zapečetěná budou ta slova až do doby konce.
#12:10 Mnozí se vytříbí, zbělí a budou vyzkoušeni. Svévolníci budou jednat svévolně; žádný svévolník se nepoučí, ale prozíraví se poučí.
#12:11 Od doby, kdy bude odstraněna každodenní oběť a vztyčena ohyzdná modla pustošitele, uplyne tisíc dvě stě devadesát dní.
#12:12 Blaze tomu, kdo se v důvěře dočká tisíce tří set třiceti pěti dnů.
#12:13 Ty vytrvej do konce. Pak odpočineš, ale na konci dnů povstaneš ke svému údělu.  

\book{Hosea}{Hos}
#1:1 Slovo Hospodinovo, které se stalo k Ozeášovi, synu Beérovu, za judských králů Uzijáše, Jótama, Achaza a Chizkijáše a za izraelského krále Jarobeáma, syna Jóašova.
#1:2 Začíná Hospodinova řeč skrze Ozeáše. Hospodin Ozeášovi řekl: „Jdi, vezmi si nevěstku a ze smilstva měj děti. Země jen smilní a smilní, odvrací se od Hospodina.“
#1:3 On tedy šel a vzal si Gomeru, dceru diblajimskou. Ta otěhotněla a porodila mu syna.
#1:4 I řekl mu Hospodin: „Pojmenuj ho Jizreel, neboť já zakrátko potrestám Jehúův dům za krev prolitou v Jizreelu; zruším království izraelského domu.
#1:5 Stane se v onen den, že přelomím Izraelovo lučiště v dolině Jizreelu.“
#1:6 Opět otěhotněla a porodila dceru. Řekl mu: „Pojmenuj ji Lórucháma (to je Neomilostněná), protože nadále nebudu izraelskému domu milostiv; dost jsem jim promíjel.
#1:7 Domu judskému však budu milostiv. Zachráním je skrze sebe, Hospodina, jejich Boha; nezachráním je lukem, mečem a válkou, koni a jezdci.“
#1:8 Když odstavila Lóruchámu, otěhotněla a porodila syna.
#1:9 I Řekl: „Pojmenuj ho Lóami (to je Nejste-lid-můj), neboť vy nejste můj lid a já nejsem váš Bůh.“ 
#2:1 Izraelských synů bude však zase jak mořského písku: nikdo je nezměří, nikdo je nesečte. Na místě, kde se jim říká: „Nejste lid můj“, bude se jim říkat: „Synové živého Boha.“
#2:2 Synové judští a synové izraelští budou pospolu shromážděni, ustanoví si jedinou hlavu a vyjdou z této země. Veliký bude den Jizreelu.
#2:3 Řekněte svým bratřím: „Lide můj“, svým sestrám: „Omilostněná“.
#2:4 Veďte spor proti své matce, veďte spor, vždyť ona není mou ženou a já nejsem jejím mužem. Ať odstraní ze své tváře znamení smilství, zprostřed svých ňader znak cizoložství!
#2:5 Jinak ji svléknu do naha, vystavím ji, jak byla v den svého zrození. Učiním z ní poušť, obrátím ji v suchopár, umořím ji žízní.
#2:6 Ani nad jejími syny se neslituji, jsou to synové smilstva.
#2:7 Smilstva se dopouští jejich matka, hanebnosti jejich rodička; říká: „Půjdu za svými milenci, kteří mi dávají vodu a chléb, vlnu a len, olej a nápoje.“
#2:8 Proto zahradím tvou cestu hložím. Postavím před ni zeď, aby nenašla své stezky.
#2:9 Bude se honit za svými milenci, ale nedostihne je, bude je hledat, ale nenajde je. Pak si řekne: „Půjdu a vrátím se ke svému prvnímu muži, tehdy mi bývalo lépe než teď.“
#2:10 Nechce pochopit, že já jsem jí dával obilí, mošt i čerstvý olej, že jsem ji zahrnoval stříbrem a zlatem - a oni to dávali Baalovi.
#2:11 Proto vezmu své obilí zpět v jeho době, i mošt v příhodný čas; strhnu z ní svou vlnu i svůj len, jež měly přikrývat její nahotu.
#2:12 Tehdy odkryji její hanbu před zraky jejích milenců. Žádný ji nevytrhne z mé ruky.
#2:13 Všemu jejímu veselí učiním přítrž, jejím svátkům, novoluním i dnům odpočinku, všem jejím slavnostem.
#2:14 Zpustoším její révu i fíkovník, o nichž říká: „To je má odměna, kterou mi dali moji milenci.“ Proměním je v divočinu, bude je požírat polní zvěř.
#2:15 Ztrestám ji za dny baalů, že jim pálí kadidlo a zdobí se kroužkem a náhrdelníkem, že chodí za svými milenci a na mne zapomněla, je výrok Hospodinův.
#2:16 Proto ji přemluvím, uvedu ji na poušť, budu jí promlouvat k srdci.
#2:17 Zas jí dám její vinice, dolinu Akór jako bránu k naději. Tam mi opět odpoví jako za dnů mládí, jako v den, kdy vystoupila z egyptské země.
#2:18 V onen den, je výrok Hospodinův, budeš ke mně volat: „Můj muži“, a nenazveš mě už: „Můj Baale“.
#2:19 Odstraním z jejích úst jména baalů; jejich jména nebude už nikdo připomínat.
#2:20 V onen den pro ně uzavřu smlouvu s polní zvěří a s nebeským ptactvem i se zeměplazy. Vymýtím ze země luk, meč i válku a dám jim uléhat v bezpečí.
#2:21 Zasnoubím si tě navěky, zasnoubím si tě spravedlností a právem, milosrdenstvím a slitováním,
#2:22 zasnoubím si tě věrností a poznáš Hospodina.
#2:23 V onen den odpoví, je výrok Hospodinův, odpovím nebesům a ona odpovědí zemi,
#2:24 země pak odpoví obilím moštu a oleji a ony odpovědí Jizreelu (to je Bůh rozsívá).
#2:25 Vseji jej pro sebe do země, Neomilostněné budu milostiv, těm, kdo Nejsou-lid-můj, řeknu: „Tys můj lid“, a on řekne: „Můj Bože!“ 
#3:1 Hospodin mi řekl: „Jdi opět a miluj ženu, milenku jiného, cizoložnici. Právě tak miluje Hospodin izraelské syny, i když se obracejí za jinými bohy a milují koláče z hroznů.“
#3:2 Opatřil jsem si ji tedy za patnáct šekelů stříbra a za půldruhého chómeru ječmene.
#3:3 Řekl jsem jí: „Po mnoho dní zůstaneš se mnou, nebudeš smilnit, nebudeš patřit jinému; také já budu jen s tebou.“
#3:4 Po mnoho dní zůstanou izraelští synové bez krále, bez velmože, bez obětních hodů, bez posvátného sloupu, bez efódu a domácích bůžků.
#3:5 Potom se však izraelští synové obrátí a budou hledat Hospodina, svého Boha, i svého krále Davida. Se strachem přiběhnou k Hospodinu a jeho dobrotě v posledních dnech. 
#4:1 Slyšte slovo Hospodinovo, synové izraelští! Hospodin vede při s obyvateli země, protože není věrnost ani milosrdenství ani poznání Boha v zemi.
#4:2 Kletby a přetvářka, vraždy a krádeže a cizoložství se rozmohly, krveprolití stíhá prolitou krev.
#4:3 Proto země truchlí, chřadnou všichni její obyvatelé, polní zvěř a nebeské ptactvo; hynou i mořské ryby.
#4:4 Nikdo se nepři, nechtěj druhého kárat! Tvůj lid, kněže, jako by chtěl vést při,
#4:5 ale upadneš za dne, a za noci upadne s tebou prorok; zahladím i tvou matku.
#4:6 Můj lid zajde, protože odmítá poznání. Ty jsi zavrhl poznání a já zavrhnu tebe; nebudeš mým knězem. Zapoměls na zákon svého Boha; i já zapomenu na tvé syny.
#4:7 Čím je jich víc, tím více proti mně hřeší; já jejich váženost zlehčím.
#4:8 Krmí se obětmi mého lidu za hřích, přiklánějí se k jeho nepravostem.
#4:9 Proto dojde jak na lid, tak na kněze. Ztrestám ho za jeho cesty, za jeho skutky mu odplatím.
#4:10 Budou jíst, a nenasytí se, budou smilnit, a nerozmohou se, protože opustili Hospodina
#4:11 a hleděli si smilstva, vínem a moštem omamují srdce.
#4:12 Můj lid se doptává svého dřeva, jeho hůlka mu předpovídá; tak jej zavádí duch smilstva. Smilstvím se odvrací od svého Boha.
#4:13 Obětují na vrcholcích hor, na pahorcích pálí kadidlo, pod dubem a topolem a kdejakým posvátným stromem, protože jejich stín je příjemný. Proto vaše dcery smilní, vaše snachy cizoloží.
#4:14 Ale nepotrestám za smilstvo jen vaše dcery, za cizoložství vaše snachy. Vždyť sami kněží se spolčují s nevěstkami, s kněžkami obětují! Je to lid nerozumný, padne.
#4:15 Jestli smilníš ty, Izraeli, ať se neproviňuje Juda. Nevcházejte do Gilgálu, neputujte do Bét-ávenu, nepřísahejte: „Jakože živ je Hospodin!“
#4:16 Izrael je umíněný jako umíněná kráva. Má je teď Hospodin pást volně jako beránka?
#4:17 Spolčencem modlářských stvůr se stal Efrajim, nech ho být.
#4:18 Propadl modlářským pitkám, smilní a smilní, jsou posedlí hanebnou láskou, ta je jim štítem.
#4:19 Bude jim úzko, vítr je zachvátí svými perutěmi, za svoje oběti se budou stydět. 
#5:1 Slyšte to, kněží, napni pozornost, dome izraelský, naslouchej, dome královský, neboť vám byl svěřen soud. Stali jste se však osidlem pro Mispu, rozprostřenou sítí na Táboru,
#5:2 jámou, kterou vyhloubili odpadlíci. Ale já je všechny ztrestám.
#5:3 Znám Efrajima, Izrael se přede mnou neukryje. Kdykoli, Efrajime, svádíš ke smilstvu, bývá Izrael poskvrněn.
#5:4 Jejich skutky jim nedovolí vrátit se k jejich Bohu; je v nich duch smilstva, neznají Hospodina.
#5:5 Ten, jenž je Pýchou Izraele, obrátí se proti němu. Izrael i Efrajim upadnou pro svoji nepravost, upadne s nimi i Juda.
#5:6 Se svým bravem a skotem pak půjdou hledat Hospodina, ale nenajdou. Unikl jim.
#5:7 Zachovali se vůči Hospodinu věrolomně, zplodili cizí syny. Nyní je i s jejich podíly pohltí novoluní.
#5:8 Zatrubte v Gibeji na polnici, na pozoun v Rámě, křičte na poplach v Bét-ávenu! Táhnou proti tobě, Benjamíne!
#5:9 Efrajim bude vzbuzovat úděs v den, kdy bude ztrestán; izraelské kmeny seznamuji s pravdou.
#5:10 Judští velmožové jsou jako ti, kdo přenášejí mezníky. Vyleji na ně svou prchlivost jako vodu.
#5:11 Efrajim bude poroben a zkrušen soudem, protože se rád honí za žvástem.
#5:12 Budu Efrajimovi jako hnisavá rána a jako kostižer Judovu domu.
#5:13 Když spatřil Efrajim svoji nemoc a Juda svou otevřenou ránu, šel Efrajim k Ašúrovi, poslal k velkokráli. Ten vás však vyléčit nemůže, vaši otevřenou ránu nevyhojí!
#5:14 Neboť já budu na Efrajima jako mladý lev, na Judův dům jako lvíče. Já, já rozsápu a odejdu, uchvátím a nikdo nevysvobodí.
#5:15 Odejdu, vrátím se ke svému místu, dokud nevyznají svou vinu a nezačnou mě hledat. Ve svém soužení mě budou hledat za úsvitu. 
#6:1 „Pojďte, vraťme se k Hospodinu, on nás rozsápal a také zhojí, zranil nás a také obváže.
#6:2 Po dvou dnech nám vrátí život, třetího dne nám dá povstat a my před ním budeme žít.
#6:3 Poznávejme Hospodina, usilujme ho poznat. Jako jitřenka, tak jistě on vyjde. Přijde k nám jako přívaly dešťů a jako jarní déšť, jenž svlažuje zemi.“
#6:4 Co mám s tebou dělat, Efrajime? Co mám s tebou dělat, Judo? Vaše zbožnost je jak jitřní obláček, jako rosa, která hned po ránu mizí.
#6:5 Proto jsem je otesával skrze proroky, ubíjel jsem je výroky svých úst; z mých soudů nad tebou ti vzejde světlo.
#6:6 Chci milosrdenství, ne oběť, poznání Boha je nad zápaly.
#6:7 Oni však po způsobu lidí přestoupili smlouvu, ve všem se vůči mně zachovali věrolomně.
#6:8 Gileád je městem těch, kdo páchají ničemnosti, městem krvavých stop.
#6:9 Jako hordy na někoho číhající spolčují se kněží, vraždí na cestě do Šekemu, prosazují své mrzké plány.
#6:10 V izraelském domě vidím strašnou věc: Izrael se tam poskvrňuje Efrajimovým smilstvem.
#6:11 A tobě, Judo, nastane žeň, až změním úděl svého lidu. 
#7:1 Když chci léčit Izraele, vychází najevo nepravost Efrajimova a zlořády Samaří, jak proradně jednají. Zloděj se vloupává, horda přepadává na ulici.
#7:2 Ani jim nepřijde na mysl, že na všechny jejich zlořády pamatuji. Jejich skutky je obklopují, mám je před očima.
#7:3 Svými zlořády dělají radost králi, svými přetvářkami velmožům.
#7:4 Jsou to samí cizoložníci, jsou jako sálající pec bez pekaře, který přestal bdít, když dal zadělané těsto kynout.
#7:5 V den našeho krále jsou velmožové rozpálení vínem; spřáhl se s posměvači.
#7:6 Záludně se k němu blíží, jejich srdce je jako pec; po celou noc spí jejich pekař, za jitra shoří plápolajícím plamenem.
#7:7 Každý z nich sálá jako pec, požírají své soudce; všichni jejich králové padají, nikdo z nich nevolá ke mně.
#7:8 Efrajim se směšuje s kdejakým lidem, Efrajim je jako neobrácený podpopelný chléb.
#7:9 Cizáci stravují jeho sílu, on to nepoznává; ačkoli prokvétá šedinami, on to nepoznává.
#7:10 Ten, jenž je Pýchou Izraele, obrátí se proti němu. K Hospodinu, svému Bohu se nevracejí, hledat ho je ani nenapadne.
#7:11 Efrajim je jako nerozumná holubice, snadno se dá zlákat. Volají k Egyptu, chodí do Asýrie.
#7:12 Až zase půjdou, rozestřu na ně svou síť, schytám je jako nebeské ptactvo, ztrestám je, jak o tom slýchali ve svém shromáždění.
#7:13 Běda jim, protože přede mnou ulétli jak ptáče. Zhouba na ně, že mi byli nevěrní. Já jsem je vykoupil, oni však o mně mluví lži.
#7:14 Ve svém srdci neúpějí ke mně, když kvílejí na svých ložích. Scházejí se při obilí a moštu a mně se vyhýbají.
#7:15 Ať jsem je trestal nebo posiloval jejich paži, oni o mně smýšleli zle.
#7:16 Obracejí se, ne však k Nejvyššímu, jsou jako záludný luk. Mečem padají jejich velmožové pro jazyk, jenž soptí hněvem. Proto sklidí jen úšklebek v egyptské zemi. 
#8:1 Polnici k ústům! Jak orel na dům Hospodinův! Přestoupili moji smlouvu, proti mému zákonu se vzepřeli.
#8:2 Budou ke mně úpět: „Bože můj, my synové Izraele se k tobě známe.“
#8:3 Izrael zanevřel na to, co je dobré, bude jej honit nepřítel.
#8:4 Krále si ustanovili, ale beze mne, dosadili velmože, ale já se k nim neznám. Ze svého stříbra a zlata si udělali modlářské stvůry ke své vlastní zkáze.
#8:5 Tvůj býček na tebe, Samaří, zanevřel. Můj hněv proti nim plane! Jak dlouho ještě? Bez trestu nezůstanou.
#8:6 Neboť ten býček je z Izraele, udělal jej řemeslník, není to Bůh. Ze samařského býčka zbudou jen třísky.
#8:7 Zaseli vítr, sklidí bouři. Nevyroste jim ani stéblo, co vzklíčí, nevydá žádnou mouku; kdyby snad i něco vydalo, pohltí to cizáci.
#8:8 Izrael bude pohlcen. Hle, budou mezi pronárody jako bezcenná nádoba.
#8:9 Vystupovali k Ašúrovi, k osamělému divokému oslu; Efrajim kupčí láskou.
#8:10 Ale i kdyby kupčili se všemi pronárody, přece je seženu dohromady a brzo začnou pociťovat břemeno krále velmožů.
#8:11 Efrajim postavil mnoho oltářů, aby hřešil; má tedy oltáře a hřeší.
#8:12 Kdybych mu napsal ze svého zákona sebevíc, bude to pokládáno za něco cizího.
#8:13 Když mi obětují dary, obětují maso, aby se najedli. Hospodin v nich nemá zalíbení. Však si připomene jejich nepravost a ztrestá je za jejich hříchy. Půjdou zpátky do Egypta.
#8:14 Izrael zapomněl na svého Učinitele a nastavěl chrámků, Juda postavil mnoho opevněných měst. Ale já pošlu na jeho města oheň a ten pozře jejich paláce. 
#9:1 Neraduj se, Izraeli, rozpustilým jásotem jako národy! Smilstvem se odvracíš od svého Boha, miluješ nevěstčí mzdu na každém obilním humně.
#9:2 Humno ani lis jim obživu nedá, mošt v zemi selže.
#9:3 Nebudou sídlit v Hospodinově zemi. Efrajim se vrátí do Egypta, v Asýrii budou jíst poskvrněné věci.
#9:4 Už nepřinesou Hospodinu úlitbu vína, nebudou mu příjemné jejich obětní hody. Stanou se jim chlebem zármutku. Všichni, kdo ho jedí, se poskvrní. Kdo se živí jejich chlebem, nesmí vstoupit do Hospodinova domu.
#9:5 Co chcete připravit pro slavnostní den, pro sváteční den Hospodinův?
#9:6 Neboť i když vyjdou ze zhouby, sebere je Egypt, Memfis je pohřbí. Jejich převzácné stříbro podědí bodláčí, v jejich stanech se rozroste trní.
#9:7 Přišly dny navštívení, přišly dny odvety, Izrael to pozná! Je prorok pošetilý? Muž ducha je potřeštěnec? To pro tvou velkou nepravost, pro velkou nevraživost.
#9:8 Ten, kdo spolu s mým Bohem hlídá Efrajima, je prorok. Na všech jeho cestách je čihařské osidlo, v domě jeho Boha nevraživost.
#9:9 Zabředli hluboko, propadli zkáze jako za dnů Gibeje. Na jejich nepravost Bůh pamatuje, ztrestá je za jejich hříchy.
#9:10 Jak hrozny na poušti našel jsem Izraele, jak první plody fíkovníku, ty nejranější, spatřil jsem vaše otce. Oni však odešli za Baal-peórem, zasvětili se té ohavě a stali se ohyzdami jako jejich miláček.
#9:11 Efrajimova sláva odlétne jako ptáče. Nebudou plodit ani rodit, ani neotěhotní.
#9:12 I kdyby své syny odchovali, připravím je o ně, žádný nezůstane. Běda i jim, běda, až od nich odstoupím.
#9:13 Efrajim, jak jsem viděl, tíhne k Týru ležícímu nad nivami; teď však Efrajim bude muset vyvést před toho vraha své syny.
#9:14 Dej jim, Hospodine! Co jim dáš? Neplodné lůno a vyschlé prsy jim dej!
#9:15 V Gilgálu je zdroj všech jejich zlořádů, proto jsem je tam začal nenávidět. Pro jejich zlé skutky je vypudím ze svého domu. Lásku jim už znovu neprokážu. Všichni jejich velmoži jsou umíněnci.
#9:16 Efrajim bude pobit, jejich kořen uschne, žádné ovoce neponesou. I kdyby něco zplodili, usmrtím to nejvzácnější, plody jejich lůna. -
#9:17 Můj Bůh je zavrhne; protože ho neposlouchali, stanou se psanci mezi pronárody. 
#10:1 Izrael je rozbujelá réva, plody jen pro sebe nasazuje. Čím více má plodů, tím více oltářů staví, čím je jeho země lepší, tím lépe zdobí posvátné sloupy.
#10:2 Jejich srdce je plné úlisnosti, budou však za vinu pykat. On jejich oltáře strhne, zničí jejich posvátné sloupy.
#10:3 Teď říkají: „Nemáme krále. Nebojíme se Hospodina a král - co by nám mohl udělat?“
#10:4 Zaklínají se falešnými slovy, když uzavírají smlouvu; jejich soud bují jako jedovaté býlí na zoraném poli.
#10:5 Obyvatelé Samaří se budou třást strachem o betávenské jalovice; jeho lid bude truchlit nad býčkem. Ať si jeho žreci jásají nad jeho slávou, ta se od něho odstěhuje.
#10:6 I jeho socha bude odnesena do Asýrie jako obětní dar velkokráli. Efrajim sklidí ostudu, Izrael bude zahanben pro své rozhodnutí.
#10:7 Zajde Samaří i jeho král, bude jako pěna na vodách.
#10:8 Zpleněna budou posvátná návrší Ávenu, hřích Izraele. Na jejich oltářích vyroste trní a hloží. Tu řeknou horám: „Přikryjte nás!“ a pahorkům: „Padněte na nás!“
#10:9 Ode dnů Gibeje jsi hřešil, Izraeli. Tam obstáli, nezasáhla je gibejská válka proti pachatelům bezpráví.
#10:10 Budu je trestat, jak sám budu chtít, národy se proti nim shromáždí, aby je spoutaly pro jejich dvojí nepravost.
#10:11 Efrajim je jalovice jhu přivyklá, bývá ráda při výmlatu. Přistoupil jsem k její pěkné šíji: Efrajima zapřáhnu, Juda bude orat, Jákob bude vláčet.
#10:12 Rozsívejte si pro spravedlnost, sklízejte pro milosrdenství, zorejte svůj úhor. Je čas dotázat se Hospodina. Až přijde, svlaží vás spravedlností.
#10:13 Orali jste svévoli, sklízeli bezpráví, jedli jste ovoce přetvářky. Doufal jsi v svou cestu, v množství svých bohatýrů.
#10:14 V tvém lidu nastane rozbroj, všechny tvé pevnosti budou vypleněny, jako když Šalman vyplenil Bét-arbél, v den boje byla rozdrcena matka nad syny.
#10:15 Právě tak naloží s vámi Bét-el pro vaše strašné zlořády. Za úsvitu bude izraelský král nadobro umlčen. 
#11:1 Když byl Izrael mládenečkem, zamiloval jsem si ho, zavolal jsem svého syna z Egypta.
#11:2 Proroci je volali, oni se však od nich odvraceli, obětovali baalům, pálili kadidlo tesaným modlám.
#11:3 Ačkoli jsem sám naučil Efrajima chodit, on na své rámě bral modly. Nepoznali, že já jsem je uzdravoval. -
#11:4 Provázky lidskými jsem je táhl, provazy milování, byl jsem jako ti, kdo jim nadlehčují jho, když jsem se k němu nakláněl a krmil jej.
#11:5 Nevrátí se do egyptské země, ale jeho králem bude Ašúr, neboť odmítli vrátit se ke mně.
#11:6 V jeho městech bude řádit meč, skoncuje s jeho mluvky, pro jejich záměry je pozře.
#11:7 Můj lid se zdráhá vrátit se ke mně; když ho volají k Nejvyššímu, nikdo se nepozvedne.
#11:8 Což bych se tě, Efrajime, mohl vzdát, mohl bych tě, Izraeli, jen tak vydat? Cožpak bych tě mohl vydat jako Admu, naložit s tebou jako se Sebójimem? Mé vlastní srdce se proti mně vzepřelo, jsem pohnut hlubokou lítostí.
#11:9 Nedám průchod svému planoucímu hněvu, nezničím Efrajima, protože jsem Bůh, a ne člověk, jsem Svatý uprostřed tebe; nepřijdu s hněvivostí.
#11:10 Budou následovat Hospodina, až zařve jako lev. Až vydá řev, přiběhnou s třesením synové od moře,
#11:11 přilétnou s třesením z Egypta jako ptáče, z asyrské země jako holubice. A já je usadím v jejich domech, je výrok Hospodinův. 
#12:1 Efrajim mě obklíčil přetvářkou, dům izraelský záludností; též Juda je před Bohem vrtkavý stále, i když je věrný svatým obyčejům.
#12:2 Efrajim se živí větrem, za východním větrem se honí po celé dny, množí lež a zhoubu. S Asýrií uzavírají smlouvu, do Egypta donášejí olej.
#12:3 Též s Judou povede Hospodin spor, Jákoba ztrestá za jeho cesty, odplatí mu podle jeho skutků.
#12:4 Už v mateřském životě úskočně jednal se svým bratrem, ale v mužném věku jako kníže zápasil s Bohem.
#12:5 Jako kníže se utkal s andělem a obstál. Plakal a prosil o smilování. V Bét-elu nalézá toho, který tam s námi rozmlouvá.
#12:6 Hospodin je Bůh zástupů, jménem Hospodin je připomínán.
#12:7 Navrať se tedy ke svému Bohu, dbej na milosrdenství a právo, s nadějí vytrvale čekej na svého Boha.
#12:8 Efrajim je kramář, v ruce má falešná závaží, rád klame.
#12:9 A říká: „Ano, zbohatl jsem, získal jsem si jmění. Na ničem, čeho jsem nabyl, nelze shledat hříšnou nepravost.“
#12:10 Já však jsem Hospodin, tvůj Bůh, už z egyptské země. Zase tě usadím ve stanech jako za dnů, kdy jsem se s tebou setkával.
#12:11 Mluvím k prorokům a dávám mnohá vidění: skrze proroky předkládám podobenství.
#12:12 Gileád páchá ničemnosti, propadne zmaru; v Gilgálu obětují býky, jejich oltáře se stanou hromadami sutě mezi brázdami pole.
#12:13 Jákob uprchl na pole Aramské, Izrael pro ženu sloužil, pro ženu střežil stáda.
#12:14 Skrze proroka vyvedl Hospodin Izraele z Egypta, skrze proroka nad ním držel stráž.
#12:15 Efrajim vzbudil urážkami Boží rozhořčení; krev, kterou prolil, uvrhne jeho Pán na něho a jeho tupení se mu vrátí. 
#13:1 Kdykoli Efrajim promluvil, nastalo zděšení, byl vznešený v Izraeli, ale provinil se Baalem a propadl smrti.
#13:2 A přesto hřeší dál, odlévají si modly, modlářské stvůry ze stříbra podle svých nápadů. Všechno je to jen výrobek řemeslníků. O těch modlách říkají: „Ti, kdo obětují, ať líbají býčky.“
#13:3 Proto budou jak jitřní obláček, jako rosa, která hned po ránu mizí, jako plevy, jež vichr odvane z humna, jako kouř z dymníku.
#13:4 Ale já jsem Hospodin, tvůj Bůh, už z egyptské země. Nepoznal jsi Boha kromě mne; mimo mne jiného zachránce není.
#13:5 Já jsem tě poznal na poušti, ve vyprahlé zemi.
#13:6 Měli pastvy do sytosti. Nasytili se a jejich srdce zpyšnělo, zapomněli na mě.
#13:7 Budu na ně jako mladý lev, budu číhat u cesty jak levhart,
#13:8 střetnu se s nimi jako medvědice zbavená mláďat. Hruď až k srdci jim roztrhám, sežeru je tam jako lvice; polní zvěř je rozsápe.
#13:9 Je to tvá zkáza, Izraeli, že jsi proti mně, proti své pomoci.
#13:10 Kdepak je tvůj král? Ať tě zachrání ve všech tvých městech. Kde jsou tvoji soudcové, o něž jsi žádal: „Dej mi krále a velmože“?
#13:11 Dávám ti krále rozhněván a beru ti jej v prchlivosti.
#13:12 Efrajimovy nepravosti jsou svázány, jeho hříchy uschovány.
#13:13 Dolehnou na něho bolesti rodičky. Je to syn nemoudrý; nechtěl opustit včas lůno rodičky.
#13:14 Mám je vyplatit ze spárů podsvětí, vykoupit ze smrti? Kde je, smrti, tvá morová rána? Podsvětí, kde je tvůj smrtící žár? Lítost však bude mým očím skryta.
#13:15 I když Efrajim mezi bratry ponese ovoce, přižene se východní vítr, vítr Hospodinův; už se zdvihá z pouště. Jeho zdroj zklame, jeho pramen vyschne, vypleněn bude poklad veškerých vzácností. 
#14:1 Samaří bude pykat, že se stavělo na odpor svému Bohu. Padnou mečem, jejich pacholátka budou rozdrcena, těhotné ženy rozpolceny.
#14:2 Navrať se, Izraeli, k Hospodinu, svému Bohu, neboť jsi upadl pro svoji nepravost.
#14:3 Vezměte s sebou tato slova, obraťte se k Hospodinu, řekněte mu: „Promiň nám všechnu nepravost, přijmi nás laskavě, když chceme oběti býčků nahradit svými rty.
#14:4 Ašúr nás nezachrání, na oře nevsedneme a výrobku svých rukou nebudeme už říkat: ‚Náš Bože!‘ U tebe přece nalézá sirotek slitování.“
#14:5 Jejich odvrácení uzdravím, rád si je zamiluji, neboť můj hněv se od nich odvrátil.
#14:6 Budu Izreli rosou, rozkvete jako lilie, zapustí kořeny jako libanónský cedr.
#14:7 Jeho ratolesti se rozloží, ve své velebnosti bude podoben olivě a jeho vůně bude jako vůně Libanónu.
#14:8 Usednou opět v jeho stínu, budou pěstovat obilí, rozkvetou jako réva připomínající libanónské víno. -
#14:9 Co je mi do nějakých modlářských stvůr, Efrajime? Řekl jsem přece, že na něho shlédnu. Jsem jako zelený cypřiš; ode mne budeš mít ovoce, které poneseš.
#14:10 Kdo je moudrý, aby těmto věcem porozuměl, rozumný, aby je pochopil? Hospodinovy cesty jsou přímé, spravedliví budou po nich chodit, nevěrní však na nich upadnou.  

\book{Joel}{Joel}
#1:1 Slovo Hospodinovo, které se dostalo k Jóelovi, synu Petúelovu.
#1:2 Slyšte to, starší, pozorně naslouchejte, všichni obyvatelé země: Stalo se něco takového za vašich dnů anebo za dnů vašich otců?
#1:3 Vypravujte o tom svým synům a vaši synové svým synům a jejich synové dalšímu pokolení:
#1:4 Co zbylo po housenkách, sežraly kobylky, co zbylo po kobylkách, sežrali brouci, co zbylo po broucích, sežrala jiná havěť.
#1:5 Vystřízlivějte, opilci, a plačte! Kvílejte, všichni pijani vína, že je vám mladé víno odtrženo od úst.
#1:6 Vždyť na mou zemi vytáhl pronárod mocný a nesčíslný; má zuby lví, tesáky jako lvice.
#1:7 Zpustošil mou vinnou révu, polámal můj fíkovník, kůru z něho sloupal a pohodil, způsobil, že vinné výhonky zežloutly.
#1:8 Kvílej, jako panna oděná žíněnou suknicí kvílí pro ženicha svého mládí.
#1:9 Obětní dary a úlitby jsou odtrženy od Hospodinova domu; truchlí kněží, sluhové Hospodinovi.
#1:10 Pole je popleněno, truchlí role; je popleněno obilí, vyschl mošt, došel čerstvý olej.
#1:11 Oráči se hanbí, vinaři kvílejí pro pšenici a ječmen; sklizeň na poli přišla nazmar.
#1:12 Vinná réva uschla, zvadl fíkovník, granátový strom i datlovník a jabloň, všechno polní stromoví je suché. Lidským synům vyschl zdroj veselí.
#1:13 Kněží, opásejte se k naříkání, kvílejte, sluhové oltáře, vejděte, nocujte v žíněných suknicích, sluhové mého Boha, neboť dům vašeho Boha je zbaven obětních darů a úliteb.
#1:14 Uložte půst, svolejte slavnostní shromáždění, shromážděte starší, všechny obyvatele země, do domu Hospodina, svého Boha, a úpěte k Hospodinu.
#1:15 Běda, ten den! Blízko je den Hospodinův! Přivalí se jako zhouba od Všemohoucího.
#1:16 Což nám není přímo před očima odtržen pokrm, od domu našeho Boha radost a jásot?
#1:17 Zrno zaschlo pod hroudami, sklady jsou zpustošené, sýpky rozbořené, obilí uschlo.
#1:18 Jak těžce oddychuje dobytek! Stáda skotu se plaší, nemají pastvu; stáda bravu se plouží.
#1:19 K tobě, Hospodine, volám. Stepní pastviny pozřel oheň, všechno polní stromoví sežehl plamen.
#1:20 Dobytek na poli po tobě prahne, v potocích vyschla voda, stepní pastviny pozřel oheň. 
#2:1 Trubte na polnici na Sijónu, křičte na poplach na mé svaté hoře, ať se třesou všichni obyvatelé země, neboť přichází den Hospodinův. Je blízko
#2:2 den tmy a temnot, den oblaku a mrákoty. Jako úsvit na horách se rozprostírá lid četný a mocný, jakého nebylo od věků, aniž kdy bude po něm až do let posledního pokolení.
#2:3 Před ním je oheň sžírající, za ním sežehující plamen; před ním je země jak zahrada Eden, za ním poušť, zpustošený kraj. A vyváznout před ním nelze.
#2:4 Vzhledem připomíná koně, je jako jízda v divém cvalu.
#2:5 Jako hřmící válečná vozba poskakuje po vrcholcích hor, praská jak ohnivý plamen, když sžírá strniště. Je jako mocný lid seřazený k boji.
#2:6 Národy se před ním svíjejí v křeči, tváře všech blednou.
#2:7 Běží jako bohatýři, jako bojovníci ztéká hradby. Každý jde svou cestou, neodbočí od své dráhy.
#2:8 Jeden druhému nepřekáží, každý jde určeným směrem. Když narazí na oštěpy, neprořidnou,
#2:9 vnikají do města, běhají po hradbách, na domy vystupují, lezou okny jak zloděj.
#2:10 Před ním se roztřese země, zachvějí nebesa, slunce a měsíc se zachmuří a hvězdy ztratí svou zář.
#2:11 Sám Hospodin dá povel svému vojsku. Jeho šiky jsou nesčíslné. Mocný je ten, kdo vykoná jeho rozkaz. Hospodinův den je veliký a přehrozný! Kdo mu odolá?
#2:12 Nyní tedy, je výrok Hospodinův, navraťte se ke mně celým srdcem, v postu, pláči a nářku.
#2:13 Roztrhněte svá srdce, ne oděv, navraťte se k Hospodinu, svému Bohu, neboť je milostivý a plný slitování, shovívavý a nejvýš milosrdný. Jímá ho lítost nad každým zlem.
#2:14 Kdo ví, nepojme-li ho opět lítost a nezanechá-li za sebou požehnání, a zase budou obětní dary a úlitby pro Hospodina, vašeho Boha.
#2:15 Trubte na polnici na Sijónu, uložte půst, svolejte slavnostní shromáždění!
#2:16 Shromážděte lid, posvěťte sbor, sezvěte starce, shromážděte pacholátka i kojence od prsů, ať vyjde ženich ze svého pokojíku a nevěsta ze své komůrky.
#2:17 Ať mezi chrámovou předsíní a oltářem pláčou kněží, sluhové Hospodinovi, ať prosí: „Ušetři, Hospodine, svůj lid, nevydávej své dědictví potupě, ať nad nimi nevládnou pronárody. Proč se má mezi národy říkat: ‚Kde je jejich Bůh?‘“
#2:18 Horlivě se ujme Hospodin své země a se svým lidem bude mít soucit.
#2:19 Hospodin se ozve a řekne svému lidu: Hle, posílám vám obilí, mošt i čerstvý olej, abyste se nasytili, a už vás nevydám potupě mezi pronárody.
#2:20 Vzdálím od vás Seveřana, zaženu ho do vyprahlé, zpustošené země, jeho předvoj do moře východního, jeho zadní voj do moře západního. A bude z něho vystupovat zápach, hnilobný puch, neboť chtěl vykonat veliké věci.
#2:21 Neboj se, země, jásej a raduj se, neboť veliké věci vykoná Hospodin.
#2:22 Neboj se, dobytku na poli, stepní pastviny se zazelenají, strom zase ponese ovoce, fíkovník a réva vydají úrodu.
#2:23 Jásejte, synové Sijónu, radujte se z Hospodina, svého Boha, neboť vám dá učitele spravedlnosti a jako na začátku vám sešle hojnost dešťů podzimních i jarních.
#2:24 Humna budou plná obilí, lisy budou přetékat moštem a čerstvým olejem.
#2:25 Tak vám nahradím, co po léta požíraly kobylky a brouci, různá havěť a housenky, mé veliké vojsko, které jsem na vás posílal.
#2:26 Budete jíst dosyta a budete chválit jméno Hospodina, svého Boha, který s vámi tak podivuhodně jednal. A můj lid nebude navěky zahanben.
#2:27 Poznáte, že jsem uprostřed Izraele. Já jsem Hospodin, váš Bůh, a jiného Boha není. A můj lid nebude navěky zahanben. 
#3:1 I stane se potom: Vyleji svého ducha na každé tělo. Vaši synové i vaše dcery budou prorokovat, vaši starci budou mít sny, vaši jinoši budou mít prorocká vidění.
#3:2 Rovněž na otroky a otrokyně vyleji v oněch dnech svého ducha.
#3:3 Způsobím, že budou na nebi i na zemi divné úkazy: krev a oheň a sloupy dýmu.
#3:4 Slunce se zastře tmou a měsíc krví, dříve než přijde den Hospodinův, veliký a hrozný.
#3:5 Avšak každý, kdo vzývá Hospodinovo jméno, se zachrání. Na hoře Sijónu a v Jeruzalémě budou ti, kdo vyvázli, jak řekl Hospodin, spolu s těmi, kdo přežili, jež Hospodin povolá. 
#4:1 Hle, v oněch dnech a v onom čase, kdy změním úděl Judy a Jeruzaléma,
#4:2 shromáždím všechny pronárody, svedu je do Doliny Jóšafatu (to je Hospodin bude soudit) a budu je tam soudit kvůli svému lidu, kvůli svému dědictví, kvůli Izraeli, kterého rozehnali mezi pronárody. Mou zemi rozdělili
#4:3 a o můj lid losovali. Dávali chlapce za nevěstku a děvče prodávali za víno - a pili.
#4:4 Co jste vy proti mně, Týre a Sidóne i všechny pelištejské kraje? Chystáte proti mně odvetu? Přikročíte-li k odvetě, rychle a bez meškání vám to vrátím na vaši hlavu.
#4:5 Pobrali jste mé stříbro a zlato, mé nejkrásnější skvosty jste zanesli do svých chrámů.
#4:6 Syny judské a jeruzalémské jste prodávali synům Jávanců, aby je zavlekli daleko od jejich území.
#4:7 Hle, já je probudím k návratu z místa, kam jste je prodali, a co jste spáchali, vám vrátím na vaši hlavu:
#4:8 Vydám vaše syny a vaše dcery napospas synům judským a oni je prodají Šebajcům, dalekému pronárodu. Tak promluvil Hospodin.
#4:9 Provolejte mezi pronárody toto: Vyhlaste svatý boj, probuďte bohatýry! Ať nastoupí, ať přitáhnou všichni bojovníci!
#4:10 Překujte své radlice v meče, vinařské nože v oštěpy! Slaboch ať zvolá: „Jsem bohatýr!“
#4:11 Pospěšte na pomoc, všechny okolní pronárody, a shromážděte se! - Přiveď sem, Hospodine, své bohatýry! -
#4:12 Ať se pronárody probudí a přitáhnou do Doliny Jóšafatu, zasednu tam a budu soudit všechny pronárody vůkol.
#4:13 Chopte se srpu, již dozrála žeň, pojďte, šlapejte, lis je už plný, kádě přetékají. Jejich zlovůle je velká.
#4:14 Davy a davy jsou v Dolině rozhodnutí a den Hospodinův v Dolině rozhodnutí je blízko.
#4:15 Slunce a měsíc se zachmuří a hvězdy ztratí svou zář.
#4:16 Hospodin vydá řev ze Sijónu, vydá hlas z Jeruzaléma, zachvějí se nebesa i země. Hospodin je útočiště svého lidu a záštita synů Izraele.
#4:17 I poznáte, že jsem Hospodin, váš Bůh, že přebývám na Sijónu, na své svaté hoře. Jeruzalém bude opět svatý a nebudou jím už procházet cizáci.
#4:18 A stane se v onen den, že z hor bude kanout mladé víno, z pahorků poteče mléko, všemi judskými potoky bude proudit voda, z Hospodinova domu vytryskne pramen a napojí Úval akácií.
#4:19 Z Egypta bude zpustošený kraj, z Edómu zpustošená step za násilí na synech judských, za to, že v jejich vlastní zemi prolévali nevinnou krev.
#4:20 Ale Judsko bude osídleno navěky, Jeruzalém od pokolení do pokolení.
#4:21 Jejich krev prohlásím za nevinnou, nikoho nenechám bez trestu. Hospodin přebývá na Sijónu.  

\book{Amos}{Amos}
#1:1 Slova Ámose, který byl z tekójských drobopravců. Měl vidění o Izraeli za dnů judského krále Uzijáše a za dnů izraelského krále Jarobeáma, syna Jóašova, dva roky před zemětřesením.
#1:2 Řekl: Hospodin vydá řev ze Sijónu, vydá hlas z Jeruzaléma; budou truchlit pastviny pastýřů, vrchol Karmelu zprahne.
#1:3 Toto praví Hospodin: Pro trojí zločin Damašku, ba pro čtverý, toto neodvolám: Protože mlátili Gileáda železnými smyky,
#1:4 sešlu na Chazaelův dům oheň a ten pozře paláce Ben-hadadovy.
#1:5 Přerazím závoru Damašku, vyhladím z Pláně kouzel toho, jenž tam sídlí, z Domu rozkoše toho, jenž drží žezlo; aramejský lid bude přesídlen do Kíru, praví Hospodin.
#1:6 Toto praví Hospodin: Pro trojí zločin Gázy, ba pro čtverý, toto neodvolám: Protože úplně přestěhovali obyvatelstvo a přesídlence vydali v plen Edómu,
#1:7 sešlu na hradby Gázy oheň a ten pozře její paláce.
#1:8 Vyhladím z Ašdódu toho, jenž tam sídlí, z Aškalónu toho, jenž drží žezlo, proti Ekrónu obrátím svou ruku; vyhyne pozůstatek Pelištejců, praví Panovník Hospodin.
#1:9 Toto praví Hospodin: Pro trojí zločin Týru, ba pro čtverý, toto neodvolám: Protože úplně vydali v plen Edómu všechny přesídlence a nepamatovali na bratrskou smlouvu,
#1:10 sešlu na hradby Týru oheň a ten pozře jeho paláce.
#1:11 Toto praví Hospodin: Pro trojí zločin Edómu, ba pro čtverý, toto neodvolám: Protože pronásledoval svého bratra mečem, odvrhl slitování, živil v sobě neustále hněv a setrvával ve své prchlivosti,
#1:12 sešlu na Téman oheň a ten pozře vládce Bosry.
#1:13 Toto praví Hospodin: Pro trojí zločin Amónovců, ba pro čtverý, toto neodvolám: Protože roztínali těhotné ženy gileádské, aby rozšířili své území,
#1:14 zanítím na hradbách Raby oheň a ten pozře její paláce za válečného ryku v den boje, za vichřice v den bouře.
#1:15 Jejich král bude přestěhován, spolu s ním i jeho velmožové, praví Hospodin. 
#2:1 Toto praví Hospodin: Pro trojí zločin Moábu, ba pro čtverý, toto neodvolám: Protože na vápno spálil kosti edómského krále,
#2:2 sešlu na Moáb oheň a ten pozře paláce Kerijótu. Moáb zemře ve hřmotu vřavy za válečného ryku při zvuku polnice.
#2:3 Vyplením soudce z jeho středu, všechny jeho velmože pobiji s ním, praví Hospodin.
#2:4 Toto praví Hospodin: Pro trojí zločin Judy, ba pro čtverý, toto neodvolám: Protože zavrhli Hospodinův zákon a nedbali na jeho nařízení - zavedli je jejich lživé modly, za nimiž chodili jejich otcové -,
#2:5 sešlu na Judu oheň a ten pozře paláce Jeruzaléma.
#2:6 Toto praví Hospodin: Pro trojí zločin Izraele, ba pro čtverý, toto neodvolám, protože za stříbro prodávají spravedlivého a ubožáka pro pár opánků.
#2:7 Baží dostat hlavy nuzáků do prachu země, pokorné zavádějí na scestí, syn i otec chodí za nevěstkou, a tak znesvěcují moje svaté jméno.
#2:8 Rozvalují se na zabavených oděvech při každém oltáři. Vydřené pokuty propíjejí ve víně v domě svého boha.
#2:9 Já jsem zničil před nimi Emorejce vysoké jako cedry a pevné jako duby. Zničil jsem je od koruny až ke kořenům.
#2:10 Já jsem vás vyvedl z egyptské země, vodil jsem vás čtyřicet let po poušti a pak jste obsadili emorejskou zemi.
#2:11 Z vašich synů jsem povolával proroky, z vašich jinochů nazíry. Není tomu tak, synové Izraele? je výrok Hospodinův.
#2:12 Avšak vy jste dávali nazírům pít víno a prorokům jste přikazovali: „Neprorokujte!“
#2:13 Hle, já vás budu tlačit k zemi jako povoz plný snopů.
#2:14 Hbitý neuteče, silný nebude moci užít své síly, bohatýr se nezachrání.
#2:15 Lučišník neobstojí, rychlonohý neunikne, jezdec na koni se nezachrání.
#2:16 Z bohatýrů i ten nejsrdnatější v onen den uteče nahý, je výrok Hospodinův. 
#3:1 Slyšte toto slovo, jež Hospodin promluvil proti vám, synové izraelští, proti celé čeledi, kterou jsem vyvedl z egyptské země.
#3:2 Pouze k vám jsem se znal ze všech čeledí země, a proto vás ztrestám za všechny vaše nepravosti.
#3:3 Půjdou spolu dva, jestliže se nedohodli?
#3:4 Řve v divočině lev, nemá-li úlovek? Ozve se lvíče ze svého doupěte, kdyby nic nelapilo?
#3:5 Chytí se pták do osidla na zemi, není-li nastraženo? Vymrští se osidlo ze země, když nicnepolapilo?
#3:6 Když se v městě troubí na polnici, zda se lid netřese? Stane-li se v městě něco zlého, zda nejedná Hospodin?
#3:7 Ovšem, Panovník Hospodin nečiní nic, aniž by zjevil své tajemství prorokům, svým služebníkům.
#3:8 Lev řve, kdo by se nebál? Panovník Hospodin mluví, kdo by neprorokoval?
#3:9 Vyhlaste na palácích v Ašdódu a na palácích v egyptské zemi řekněte: Shromážděte se na Samařské hory! Pohleďte, co je tam zmatků, kolik útisku!
#3:10 Správně jednat neumějí, je výrok Hospodinův. Uskladňují ve svých palácích násilí a zhoubu jako poklady.
#3:11 Proto praví Panovník Hospodin toto: Tuto zemi obklíčí protivník, tvá opevnění strhne, tvoje paláce budou vyloupeny.
#3:12 Toto praví Hospodin: Jako pastýř vyrve lvu z tlamy dva hnáty či boltec, tak budou vyrváni izraelští synové, kteří si hoví v Samaří v pohodlí lehátka, na polštářích pohovky.
#3:13 Slyšte to a dejte výstrahu Jákobovu domu, je výrok Panovníka Hospodina, Boha zástupů:
#3:14 V den, kdy budu trestat Izraele za jeho nevěrnosti, ztrestám i oltáře bételské; rohy oltáře budou odseknuty, padnou k zemi.
#3:15 Domem zimním udeřím o dům letní, zaniknou domy ze slonoviny, smeteno bude mnoho domů, je výrok Hospodinův. 
#4:1 Slyšte toto slovo, bášanské krávy, které jste na Samařské hoře, které utiskujete nuzné a křivdíte ubožákům. Říkáte jejich Panovníkovi: „Dávej, ať hodujeme!“
#4:2 Panovník Hospodin se zapřísáhl při své svatosti: Hle, přijdou na vás dny, kdy vás vytáhnou na hácích a vaše potomky na rybářských udicích.
#4:3 Vyjdete trhlinami, každá tou, před níž právě bude, a budete vrženy až k Harmónu, je výrok Hospodinův.
#4:4 Choďte si do Bét-elu oddávat se nevěrnosti, jen se dopouštějte ještě více nevěrnosti v Gilgálu, přinášejte své oběti z jitra, své desátky třetího dne.
#4:5 Spalujte v děkovnou oběť kvašené chleby, hlučně ohlašujte oběti dobrovolné, vždyť to tak máte rádi, synové izraelští, je výrok Panovníka Hospodina.
#4:6 Způsobil jsem, že jste neměli co do úst v žádném svém městě, že jste měli nedostatek chleba na všech svých místech. A přece jste se ke mně nenavrátili, je výrok Hospodinův.
#4:7 Já jsem vám také odepřel hojnost dešťů tři měsíce přede žněmi, jedno město jsem svlažoval a druhé jsem nesvlažil, jeden díl země byl zavlažován, zatímco díl, na který nepršelo, zprahl.
#4:8 Dvě tři města vrávorala k jinému městu, aby se napila vody, žízeň však neuhasila. A přece jste se ke mně nenavrátili, je výrok Hospodinův.
#4:9 Bil jsem vás obilnou rzí a snětí; mnoho vašich zahrad a vinic, vaše fíkovníky a vaše olivy sežraly housenky. A přece jste se ke mně nenavrátili, je výrok Hospodinův.
#4:10 Poslal jsem na vás mor jako na Egypt, vaše jinochy jsem pobil mečem, i vaši koně byli odvlečeni, dal jsem vám čichat puch vašich táborů. A přece jste se ke mně nenavrátili, je výrok Hospodinův.
#4:11 Podvrátil jsem vás, jako jsem podvrátil, já Bůh, Sodomu a Gomoru; byli jste jako oharek vyrvaný z plamenů ohniště. A přece jste se ke mně nenavrátili, je výrok Hospodinův.
#4:12 Proto s tebou, Izraeli, takto naložím. A protože s tebou naložím takto, připrav se, Izraeli, na setkání se svým Bohem!
#4:13 Neboť hle, on je tvůrce hor a stvořitel větru, oznamuje člověku, co má na mysli, působí úsvit i soumrak, šlape po posvátných návrších země. Jeho jméno je Hospodin, Bůh zástupů. 
#5:1 Slyšte toto slovo, které pronáším, žalozpěv nad vámi, dome izraelský:
#5:2 „Padla izraelská panna, už nevstane, leží bez povšimnutí na vlastní zemi, nezvedne ji nikdo.“
#5:3 Neboť toto praví Panovník Hospodin: Městu, které vytáhne s tisícem, zůstane sto, a tomu, jež vytáhdne se stem, zůstane pro dům izraelský deset.
#5:4 Toto praví Hospodin domu izraelskému: Dotazujte se mne a budete žít!
#5:5 Nedotazujte se Bét-elu, nevcházejte do Gilgálu, neputujte do Beer-šeby, neboť Gilgál bude přesídlen, Bét-el se ukáže jako ničemný klam.
#5:6 Dotazujte se Hospodina a budete žít! Kéž nezachvátí dům Josefův jako oheň. Pozřel by jej a nikdo by neuhasil Bét-el,
#5:7 ty, kdo převracejí právo v pelyněk a spravedlnost srážejí k zemi.
#5:8 Ten, který učinil Plejády a Orióna, obrací šero smrti v jitro, ale i den zatmívá nocí, povolává mořské vody a vylévá je na tvář země. Jeho jméno je Hospodin.
#5:9 On mocného uvrhne do záhuby a zhouba pronikne do pevnosti.
#5:10 Oni však toho, kdo trestá v bráně, nenávidí, pravdomluvný se jim hnusí.
#5:11 Protože hanebně vydíráte nuzáka a vymáháte na něm obilnou daň, mohli jste si vybudovat domy z kvádrů, bydlet v nich však nebudete; vysadili jste si skvělé vinice, avšak víno z nich pít nebudete.
#5:12 Já znám vaše četné nevěrnosti, vaše nehorázné hříchy. Nevražíte na spravedlivého, berete úplatek, ubožáky v bráně odstrkujete.
#5:13 Proto v oné době prozíravý zmlkne, bude to zlý čas.
#5:14 Hledejte dobro a ne zlo a budete žít, a tak Hospodin, Bůh zástupů, bude s vámi, jak říkáte.
#5:15 Mějte v nenávisti zlo a milujte dobro, uplatňujte v bráně právo! Snad se Hospodin, Bůh zástupů, smiluje nad pozůstatkem lidu Josefova.
#5:16 Proto praví Hospodin, Bůh zástupů, Panovník, toto: Na všech náměstích bude nářek, na všech ulicích budou křičet: „Běda, běda!“ Zavolají k truchlení oráče, ty, kdo umějí bědovat, k naříkání.
#5:17 Na všech vinicích bude nářek, neboť projdu tvým středem, praví Hospodin.
#5:18 Běda těm, kdo touží po dni Hospodinově! K čemu vám bude den Hospodinův? Bude tmou, a ne světlem!
#5:19 Jako když se dá někdo na útěk před lvem a narazí na něho medvěd; nebo vejde do domu, opře se rukou o stěnu a uštkne ho had.
#5:20 Což není den Hospodinův tmou, a ne světlem, temnotou bez jasu?
#5:21 Nenávidím vaše svátky, zavrhl jsem je, ani vystát nemohu vaše slavnostní shromáždění.
#5:22 Když mi přinášíte zápalné oběti a své oběti přídavné, nemám v nich zalíbení, na pokojnou oběť z vašeho vykrmeného dobytka ani nepohlédnu.
#5:23 Pryč ode mne s halasem tvých písní, tvé brnkání na harfy už nechci slyšet.
#5:24 Ať se valí právo jako vody, spravedlnost jak proudící potok.
#5:25 Připravovali jste mi obětní hody a přídavné oběti po čtyřicet let na poušti, dome izraelský?
#5:26 Ponesete Sikúta, svého krále, Kijúna, své obrazy, hvězdu svého boha, to, co jste si udělali.
#5:27 Přesídlím vás až za Damašek, praví ten, jehož jméno je Hospodin, Bůh zástupů. 
#6:1 Běda bezstarostným na Sijónu i těm, kdo doufají v Samařskou horu, běda vůdcům nejpřednějšího pronároda, k nimž přichází dům izraelský!
#6:2 Projděte Kalné a podívejte se. Odtud jděte do velkého Chamátu a sestupte do pelištejského Gatu. Jste lepší než tato království? Je teď jejich území větší než vaše?
#6:3 Zažehnáváte zlý den, ale nastolujete násilí.
#6:4 Běda těm, kdo lehají na ložích ze slonoviny, povalují se na pohovkách, jídají jehňata ze stáda a telata z chléva,
#6:5 blábolí za zvuku harfy, vymýšlejí si hudební nástroje jako David,
#6:6 pijí víno z obětních misek, nejlepšími oleji se maží, ale nad Josefovou těžkou ranou se netrápí.
#6:7 Proto nyní budou přestěhováni v čele přesídlenců, ustane hodokvas povalečů.
#6:8 Přísahal Panovník Hospodin při sobě samém, je výrok Hospodina, Boha zástupů: Hnusí se mi Jákobova pýcha, jeho paláce mám v nenávisti. V plen vydám město se vším, co je v něm.
#6:9 Zbude-li pak deset mužů v jednom domě, i ti zemřou.
#6:10 Někoho vezme jeho strýc, spalovač mrtvol, aby vynesl kosti z domu, a zeptá se toho, kdo se skryl v odlehlém koutě domu: „Je tu ještě někdo s tebou?“ On odpoví: „Už nikdo.“ A dodá: „Tiše! Jen nepřipomínat jméno Hospodinovo.“
#6:11 Neboť hle, Hospodin vydá příkaz a velký dům rozbije na padrť a malý dům na třísky.
#6:12 Mohou běhat koně po skalisku? Může být zoráno dobytkem? A přece jste učinili z práva jed, z ovoce spravedlnosti pelyněk.
#6:13 Radujete se, a nemáte z čeho, říkáte: „Což jsme vlastní silou nezískali pro sebe moc?“
#6:14 Já však proti vám postavím pronárod, dome izraelský, je výrok Hospodina, Boha zástupů, a ten vás bude utlačovat od cesty do Chamátu až po Úval pustiny. 
#7:1 Toto mi ukázal Panovník Hospodin: Hle, připravuje kobylky, když začíná růst otava. Po královské senoseči, sotvaže otava vzešla,
#7:2 stalo se, že docela sežraly byliny země. I řekl jsem: „Panovníku Hospodine, odpusť prosím! Jak Jákob obstojí? Je tak nepatrný!“
#7:3 Hospodin byl nad tím jat lítostí: „Nestane se to,“ řekl Hospodin.
#7:4 Toto mi ukázal Panovník Hospodin: Hle, Panovník Hospodin volá, že povede spor ohněm. Ten pozřel obrovskou propastnou tůň a zžírá už i Jákobův podíl.
#7:5 I řekl jsem: „Panovníku Hospodine, ustaň prosím! Jak Jákob obstojí? Je tak nepatrný!“
#7:6 Hospodin byl nad tím jat lítostí: „Nestane se ani toto,“ řekl Panovník Hospodin.
#7:7 Toto mi ukázal: Hle, Panovník stojí na hradbách s olovnicí, s olovnicí v ruce. I řekl mi Hospodin: „Ámosi, co vidíš?“
#7:8 Odpověděl jsem: „Olovnici.“ A Panovník řekl: „Hle, spustím olovnici doprostřed Izraele, svého lidu. Už mu nebudu dál promíjet.
#7:9 Posvátná návrší Izákova budou zpustošena, Izraelovy svatyně budou obráceny v trosky, proti domu Jarobeámovu se postavím s mečem.“
#7:10 Bételský kněz Amasjáš poslal izraelskému králi Jarobeámovi zprávu: „Spikl se proti tobě Ámos přímo v izraelském domě. Není možné, aby země snášela všechna jeho slova.
#7:11 Ámos totiž říká: Jarobeám zemře mečem a Izrael bude zcela jistě přesídlen ze své země.“
#7:12 Pak řekl Amasjáš Ámosovi: „Seber se, vidoucí, a prchej do judské země! Tam chleba jez a tam si prorokuj!
#7:13 A nikdy už neprorokuj v Bét-elu, neboť je to svatyně králova, královský dům.“
#7:14 Ámos Amasjášovi odpověděl: „Nebyl jsem prorok ani prorocký žák, zabýval jsem se dobytkem a sykomorami.
#7:15 Hospodin mě vzal od ovcí, Hospodin mi rozkázal: ‚Jdi a prorokuj Izraeli, mému lidu!‘
#7:16 Slyš tedy slovo Hospodinovo. Ty říkáš: ‚Neprorokuj proti Izraeli, nevynášej věštbu proti Izákovu domu!‘
#7:17 Proto Hospodin praví toto: ‚Tvoje žena bude v městě provozovat smilství, tvoji synové a dcery padnou mečem, tvá půda bude rozměřena provazcem a ty zemřeš v zemi nečisté.‘ Izrael bude zcela jistě přesídlen ze své země.“ 
#8:1 Toto mi ukázal Panovník Hospdin: Hle, koš zralého ovoce.
#8:2 A zeptal se: „Ámosi, co vidíš?“ Odpověděl jsem: „Koš zralého ovoce.“ Hospodin mi řekl: „Izrael, můj lid, je zralý pro konec. Už mu nebudu dál promíjet.“
#8:3 V onen den se obrátí chrámové zpěvy v kvílení, je výrok Panovníka Hospodina, mnoho mrtvých těl bude pohozeno na všech místech. Tiše!
#8:4 Slyšte to vy, kdo bažíte po ubožákovi a kdo chcete odstranit pokorné v zemi.
#8:5 Říkáte: „Kdy už pomine novoluní, abychom zas prodávali obilí, a den odpočinku, abychom otevřeli sýpku, na míře ubírali, na ceně přidávali a podváděli falešnou váhou,
#8:6 abychom si koupili nuzáky za stříbro, ubožáka pro pár opánků, a abychom prodali obilní zadinu.“
#8:7 Přísahal Hospodin při sobě, při Pýše Jákobově: Nikdy nezapomenu na žádný jejich skutek.
#8:8 Nebude se nad tím třást země a truchlit každý její obyvatel? Celá se vzedme jako Nil a zase ustoupí a opadne jako egyptská řeka.
#8:9 V onen den způsobím, je výrok Panovníka Hospodina, že slunce zapadne v poledne, tmou zahalím zemi za jasného dne.
#8:10 Vaše slavnosti proměním ve smutek, všechny vaše písně v žalozpěvy. Na všechna bedra vložím žíněnou suknici, na každé hlavě bude lysina. Učiním, že bude v zemi smutek jako nad jednorozeným a její poslední chvíle jako den hořkosti.
#8:11 Hle, přicházejí dny, je výrok Panovníka Hospodina, kdy pošlu na zemi hlad, ne hlad po chlebu ani žízeň po vodě, nýbrž po slyšení slov Hospodinových.
#8:12 Budou vrávorat od moře k moři a ze severu na východ; budou pobíhat a hledat slovo Hospodinovo, ale nenajdou.
#8:13 V onen den budou omdlévat žízní krásné panny i jinoši.
#8:14 Ti, kteří přísahají při samařském provinění a říkají: „Jakože živ je tvůj bůh, Dane, a jakože živa je cesta Beer-šeby“, padnou a již nepovstanou. 
#9:1 Spatřil jsem Panovníka, jak stojí nad oltářem. Řekl: „Udeř do hlavice sloupu, až se zachvějí prahy! Sraz jim to všechno na hlavu, ty poslední vybiji mečem. Nikdo z nich neuteče, neunikne, nevyvázne.
#9:2 I kdyby se prokopali do podsvětí, má ruka je odtud vezme. Kdyby vystoupili na nebesa, strhnu je odtud.
#9:3 Ukryjí-li se na vrcholu Karmelu, vypátrám je a vezmu je odtud. Skryjí-li se před mým zrakem na mořském dně, poručím hadu a vyštípe je odtud.
#9:4 Půjdou-li před svými nepřáteli do zajetí, poručím meči, aby je tam pobil, neboť můj zrak spočinul na nich ke zlému a ne k dobrému.“
#9:5 Panovník Hospodin zástupů sotva se dotkne země, už se zmítá a všichni její obyvatelé truchlí. Celá se vzedme jako Nil a zase opadne jako egyptská řeka.
#9:6 On buduje stupně svého domu na nebesích, nad zemí zakládá svou klenbu. On povolává mořské vody a vylévá je na tvář země. Jeho jméno je Hospodin.
#9:7 Nejste snad pro mne jako Kúšijci, vy, synové Izraele? je výrok Hospodinův. Nevyvedl jsem Izraele z egyptské země a Pelištejce z Kaftóru a Arama z Kíru?
#9:8 Hle, oči Panovníka Hospodina jsou upřeny na toto hříšné království: Vyhladím je z povrchu země. Dům Jákobův však zcela nevyhladím, je výrok Hospodinův.
#9:9 Neboť hle, přikázal jsem a zatřesu izraelským domem mezi všemi pronárody, jako když se třese řešetem, a žádný kamínek nepropadne na zem.
#9:10 Zemřou mečem všichni hříšníci mého lidu, kteří říkají: „Nás nic nepostihne, nás nepotká nic zlého.“
#9:11 V onen den postavím padající Davidův stánek a jeho trhliny zazdím, opravím, co na něm pobořeno, a zbuduji jej jako za dnů dávných.
#9:12 Podrobí si pozůstatek lidu edómského a všech pronárodů; i budou nazývány mým jménem, je výrok Hospodina, který toto činí.
#9:13 Hle, přicházejí dny, je výrok Hospodinův, kdy půjde oráč hned za žencem a ten, kdo šlape hrozny, hned za rozsévačem. Z hor budce kanout mladé víno a všechny pahorky budou oplývat vláhou.
#9:14 Úděl Izraele, svého lidu, změním, oni znovu vybudují zpustošená města a osídlí je, vysadí vinice a budou z nich pít víno, založí zahrady a budou z nich jíst ovoce.
#9:15 Zasadím je do jejich půdy a již nikdy nebudou vykořeněni ze své země, kterou jsem jim dal, praví Hospodin, tvůj Bůh.  

\book{Obadiah}{Obad}
#1:1 Vidění Abdiášovo. Toto praví Panovník Hospodin o Edómu. Slyšeli jsme zprávu od Hospodina, že mezi pronárody byl poslán vyslanec: „Vzhůru, povstaňme k boji proti němu!“
#1:2 Hle, mezi pronárody tě činím maličkým, budeš ve velkém opovržení.
#1:3 Opovážlivost tvého srdce tě zavedla. Bydlíš v skalních rozsedlinách, sídlíš vysoko, v srdci si říkáš: „Kdo by mě strhl k zemi?“
#1:4 I když sis založil hnízdo vysoko jak orel a položil je mezi hvězdy, strhnu tě odtud, je výrok Hospodinův.
#1:5 Až tě přepadnou zloději, škůdcové noční, jak zajdeš! Což nenakradou tolik, aby měli dost? Až na tebe přijdou sběrači hroznů, nenechají ani paběrky.
#1:6 Tak bude Ezau propátrán, jeho úkryty proslíděny.
#1:7 Poženou tě až na hranice, podvedou tě všichni, s nimiž máš smlouvu, zmocní se tě ti, s nimiž v pokoji žiješ, ti, kteří jedí tvůj chléb, zákeřně ti zasadí ránu. On však rozum nemá.
#1:8 Zdali v onen den, je výrok Hospodinův, nevyhubím z Edómu mudrce a rozumnost z Ezauovy hory?
#1:9 Témane, tvoji bohatýři se zděsí, neboť na Ezauově hoře budou všichni vyhlazeni, protože vraždili.
#1:10 Pro násilí na tvém bratru Jákobovi přikryje tě hanba, budeš vyhlazen navěky.
#1:11 V ten den ses postavil proti němu, v den, kdy cizáci jímali jeho vojsko. Když vcházeli do jeho bran cizinci a losovali o Jeruzalém, i tys byl jako jeden z nich.
#1:12 Jen se nepas pohledem na den svého bratra v den jeho neštěstí, neraduj se nad judskými syny v den jejich záhuby, nepošklebuj se jim v den soužení!
#1:13 Nevcházej do brány mého lidu v den jeho běd; právě ty se nepas pohledem na zlo, jež ho stihlo v den jeho běd; ať tvé ruce nesahají po jeho majetku v den jeho běd!
#1:14 A nestůj nad průrvou, abys pobíjel ty, kdo vyvázli, nevydávej ty, kdo přežili, v den soužení!
#1:15 Blízko je den Hospodinův proti všem národům. Co jsi učinil ty, to bude učiněno tobě, vrátí se ti to na hlavu, jak zasluhuješ.
#1:16 Jak jste se napili na mé svaté hoře vy, tak budou ustavičně pít všechny pronárody; budou pít, až se budou zalykat, a zmizí, jako by nikdy nebývaly.
#1:17 Avšak na hoře Sijónu budou ti, kteří vyvázli, a bude svatá, a Jákobův dům obsadí ty, kdo jej obsadili.
#1:18 I bude dům Jákobův ohněm a dům Josefův plamenem, zatímco dům Ezauův bude strništěm; oba se proti němu rozpálí a pozřou jej, nikdo z Ezauova domu nepřežije. Tak promluvil Hospodin.
#1:19 Lid Negebu obsadí Ezauovu horu a lid Přímořské nížiny Pelištejce, obsadí též pole Efrajimovo a pole Samařské a Benjamín obsadí Gileád.
#1:20 Přesídlenci, to vojsko synů Izraele, obsadí, co je kenaanské, až do Sarepty a přesídlenci z Jeruzaléma, kteří jsou v Sefaradu, obsadí negebská města.
#1:21 Vítězové pak vystoupí na horu Sijón, aby soudili Ezauovu horu. A nastane království Hospodinovo.  

\book{Jonah}{Jonah}
#1:1 Stalo se slovo Hospodinovo k Jonášovi, synu Amítajovu:
#1:2 „Vstaň, jdi do Ninive, toho velikého města, a volej proti němu, neboť zlo, které páchají, vystoupilo před mou tvář.“
#1:3 Ale Jonáš vstal, aby uprchl do Taršíše, pryč od Hospodina. Sestoupil do Jafy a vyhledal loď, která plula do Taršíše. Zaplatil za cestu a vstoupil na loď, aby se s nimi plavil do Taršíše, pryč od Hospodina.
#1:4 I uvrhl Hospodin na moře veliký vítr a na moři se rozpoutala veliká bouře. Lodi hrozilo ztroskotání.
#1:5 Lodníci se báli a úpěli každý ke svému bohu a vrhali do moře předměty, které měli na lodi, aby jí odlehčili. Ale Jonáš sestoupil do podpalubí, ulehl a tvrdě usnul.
#1:6 Přišel k němu velitel lodi a řekl mu: „Co je s tebou, ospalče! Vstaň a volej k svému bohu! Snad si nás tvůj bůh povšimne a nezahyneme.“
#1:7 Zatím se lodníci mezi sebou smluvili: „Pojďte, budeme losovat a poznáme, kvůli komu nás postihlo toto neštěstí.“ Losovali tedy a los padl na Jonáše.
#1:8 Řekli mu: „Pověz nám, kvůli komu nás postihlo toto neštěstí. Čím se zabýváš? Odkud přicházíš? Z které země, z kterého lidu?“
#1:9 Odpověděl jim: „Jsem Hebrej a bojím se Hospodina, Boha nebes, který učinil moře i pevninu.“
#1:10 Tu padla na ty muže veliká bázeň a řekli mu: „Cos to udělal?“ Dozvěděli se totiž, že prchá od Hospodina, sám jim to pověděl.
#1:11 Zeptali se ho: „Co teď s tebou máme udělat, aby nás moře nechalo napokoji?“ Neboť moře se stále více bouřilo.
#1:12 Odpověděl jim: „Vezměte mě a uvrhněte do moře, a moře vás nechá napokoji. Vím, že vás tahle veliká bouře přepadla kvůli mně.“
#1:13 Ti muži však veslovali, aby se vrátili na pevninu, ale marně. Moře se proti nim bouřilo stále víc.
#1:14 Volali tedy k Hospodinu: „Prosíme, Hospodine, ať nezahyneme pro život tohoto muže, nestíhej nás za nevinnou krev. Ty jsi Hospodin, jak si přeješ, tak činíš.“
#1:15 I vzali Jonáše a uvrhli ho do moře. A moře přestalo běsnit.
#1:16 Na ty muže padla veliká bázeň před Hospodinem. Přinesli Hospodinu oběť a zavázali se sliby. 
#2:1 Hospodin však nastrojil velikou rybu, aby Jonáše pohltila. Jonáš byl v útrobách ryby tři dny a tři noci.
#2:2 I modlil se v útrobách ryby k Hospodinu, svému Bohu. Řekl:
#2:3 „V soužení jsem volal k Hospodinu, on mi odpověděl. Z lůna podsvětí jsem volal o pomoc a vyslyšels mě.
#2:4 Vhodil jsi mě do hlubin, do srdce moře, obklíčil mě proud, všechny tvé příboje, tvá vlnobití se přese mne převalily.
#2:5 A já jsem si řekl: Jsem zapuzen, nechceš mě už vidět. Tak rád bych však zase hleděl na tvůj svatý chrám!
#2:6 Zachvátily mě vody, propastná tůň mě obklíčila, chaluhy mi ovinuly hlavu.
#2:7 Sestoupil jsem ke kořenům horstev, závory země se za mnou zavřely navěky. Tys však vyvedl můj život z jámy, Hospodine, můj Bože!
#2:8 Když jsem byl v duši tak skleslý, Hospodina jsem si připomínal; má modlitba vešla k tobě ve tvůj svatý chrám.
#2:9 Ti, kdo se šalebných přeludů drží, o milosrdenství se připravují.
#2:10 Já ti však s díkůvzdáním přinesu oběť, co jsem slíbil, splním. U Hospodina je spása!“
#2:11 I rozkázal Hospodin rybě, a vyvrhla Jonáše na pevninu. 
#3:1 I stalo se slovo Hospodinovo k Jonášovi podruhé:
#3:2 „Vstaň, jdi do Ninive, toho velikého města, a provolávej v něm, co ti uložím.“
#3:3 Jonáš tedy vstal a šel do Ninive, jak mu Hospodin uložil. Ninive bylo veliké město před Bohem, muselo se jím procházet tři dny.
#3:4 Jonáš vešel do města, procházel jím jeden den a volal: „Ještě čtyřicet dní, a Ninive bude vyvráceno.“
#3:5 I uvěřili ninivští muži Bohu, vyhlásili půst a oblékli si žíněné suknice od největšího až po nejmenšího.
#3:6 Když to slovo proniklo k ninivskému králi, vstal ze svého trůnu, odložil svůj plášť, zahalil se do žíněné suknice a sedl si do popela.
#3:7 Potom dal v Ninive rozhlásit: „Podle vůle krále a jeho mocných rádců! Lidé ani zvířata, skot ani brav ať nic neokusí, ať se nepasou a nepijí vodu.
#3:8 Ať se zahalí do žíněné suknice, lidé i zvířata, a naléhavě ať volají k Bohu. Každý ať se odvrátí od své zlé cesty a od násilí, které mu lpí na rukou.
#3:9 Kdo ví, možná že se Bůh v lítosti obrátí a odvrátí od svého planoucího hněvu a nezahyneme.“
#3:10 I viděl Bůh, jak si počínají, že se odvracejí od své zlé cesty, a litoval, že jim chtěl učinit zlo, které ohlásil. - A neučinil tak. 
#4:1 Jonáš se velice rozezlil a planul hněvem.
#4:2 Modlil se k Hospodinu a řekl: „Ach, Hospodine, což jsem to neříkal, když jsem byl ještě ve své zemi? Proto jsem dal přednost útěku do Taršíše! Věděl jsem, že jsi Bůh milostivý a plný slitování, shovívavý a nesmírně milosrdný, že tě jímá lítost nad každým zlem.
#4:3 Nyní, Hospodine, vezmi si prosím můj život. Lépe abych umřel, než abych žil.“
#4:4 Hospodin se však otázal: „Je dobře, že tak planeš?“
#4:5 Jonáš totiž vyšel z města, usadil se na východ od něho a udělal si tam přístřešek. Seděl v jeho stínu, aby viděl, co se bude ve městě dít.
#4:6 Hospodin Bůh nastrojil skočec, který vyrostl nad Jonášem, aby mu stínil hlavu a zbavil ho zloby. Jonáš měl ze skočce velikou radost.
#4:7 Příštího dne za svítání nastrojil však Bůh červa, který skočec nahlodal, takže uschl.
#4:8 Když pak vzešlo slunce, nastrojil Bůh žhavý východní vítr a slunce bodalo Jonáše do hlavy, až úplně zemdlel a přál si umřít. Řekl: „Lépe abych umřel, než abych žil.“
#4:9 Bůh se však Jonáše otázal: „Je dobře, že pro ten skočec tak planeš?“ Odpověděl: „Je to dobře. Planu hněvem až k smrti.“
#4:10 Hospodin řekl: „Tobě je líto skočce, s kterým jsi neměl žádnou práci, jemuž jsi nedal vzrůst; přes noc vyrostl, přes noc zašel.
#4:11 A mně by nemělo být líto Ninive, toho velikého města, v němž je víc než sto dvacet tisíc lidí, kteří nedovedou rozeznat pravici od levice, a v němž je i tolik dobytka?“  

\book{Micah}{Mic}
#1:1 Slovo Hospodinovo, které se stalo k Micheáši Mórešetskému za dnů judských králů Jótama, Achaza a Chizkijáše; bylo to vidění o Samaří a o Jeruzalému.
#1:2 Slyšte to, všichni lidé, napni pozornost, země, se vším, co je na tobě! Panovník Hospodin buď proti vám svědkem, Panovník ze svého svatého chrámu!
#1:3 Neboť hle, Hospodin vychází ze svého místa, sestupuje a šlape po posvátných návrších země.
#1:4 Hory se pod ním rozplývají, doliny pukají, jsou jako vosk v ohni, jako vody řítící se ze stráně.
#1:5 To všechno pro nevěrnost Jákobovu, pro hříchy izraelského domu. Co je Jákobova nevěrnost, zdali ne Samaří? A co jsou posvátná návrší Judy, zdali ne Jeruzalém?
#1:6 Proto udělám ze Samaří pole sutin a vysadím na něm vinici, kameny z něho svalím do údolí, odkryji jeho základy.
#1:7 Rozbity budou veškeré jeho tesané modly, veškeré jeho nevěstčí odměny budou spáleny ohněm, zpustoším všechny jeho modlářské stvůry. Protože je shromáždilo ze smilné nevěstčí odměny, opět se stanou smilnou odměnou nevěstek.
#1:8 Naříkat nad tím musím a kvílet, chodit bosý a nahý, dám se do nářku jak šakalové, do truchlení jako pštrosi,
#1:9 protože jsou nevyléčitelné jeho rány. Došlo i na Judu; zasažena je brána mého lidu, sám Jeruzalém.
#1:10 Neoznamujte to v Gatu, ať se nerozléhá váš pláč. V Bét-leafře (to je v Domě prachu) válejte se v prachu.
#1:11 Obyvatelé Šafíru (to je Nádhery), stěhujte se jinam, nazí a v hanbě. Neujdou obyvatelé Saanánu. Nářek v Bét-eselu vás připraví o oporu.
#1:12 Ačkoli obyvatelé Marótu čekali dobro, sestoupilo od Hospodina zlo k bráně Jeruzaléma.
#1:13 Zapřahejte do válečného vozu, obyvatelé Lakíše; tady je počátek hříchu sijónské dcery. Vždyť v tobě byly shledány nevěrnosti Izraele!
#1:14 Proto se rozejdete s Mórešet-gatem; domy Akzíbu (to je Lži) obelžou izraelské krále.
#1:15 I na vás přivedu dobyvatele, obyvatelé Maréši, až do Adulámu bude muset vejít sláva Izraele.
#1:16 Vyholte si lysinu a ostříhejte se kvůli svým rozkošným synům; udělejte si širokou lysinu jako sup, že byli od vás přesídleni. 
#2:1 Běda těm, kdo strojí ničemnosti, páchají zlo na svých ložích, dopouštějí se ho za jitřního světla; na to mají dost sil.
#2:2 Žádostivě dychtí po polích - a uchvacují, po domech - a zabírají. Utiskují muže i jeho dům, člověka i jeho dědictví.
#2:3 Proto praví Hospodin toto: Hle, já na tu čeleď strojím zlo, z něhož nevyvléknete své šíje. Přestanete chodit povýšeně, bude to zlý čas.
#2:4 V onen den o vás užijí pořekadel, strhne se žalostné bědování: „Je po všem!“ „Přišla na nás záhuba“, řeknete, „podíl mého lidu je pryč, kterak jsem o něj přišel! Naše pole připadlo odpadlíku.“
#2:5 Ano, nebudeš mít nikoho, kdo by ti losem určil podíl v Hospodinově shromáždění.
#2:6 Říkají: „Nevěštěte!“ Budou věštit! Ale nebudou věštit těmto; ti od potupných věcí neodstoupí.
#2:7 Vy, kteří se nazýváte domem Jákobovým, což nepokoušíte Hospodinovu trpělivost? Tohle že jsou jeho skutky? Zdali nejsou má slova k dobrému pro toho, kdo chodí přímo?
#2:8 Dříve se můj lid stavěl proti nepříteli. Teď však strháváte z plášťů ozdoby těm, kdo po návratu z boje užívají bezpečí.
#2:9 Ženy mého lidu zapuzujete z jejich rozkošného domova, jejich děti chcete navěky připravit o důstojnost, kterou jsem jim dal.
#2:10 Nuže, jděte si, tady odpočinutí nedojdete. Pro svou nečistotu budete zahubeni strašnou záhubou.
#2:11 Kdyby někdo posedlý duchem podvodu nalhával: „Budu ti věštit při víně a opojném nápoji“, to by byl věštec pro tento lid!
#2:12 Jistotně tě celého posbírám, Jákobe, jistotně shromáždím pozůstatek Izraele, svedu jej dohromady jak ovce do ohrady, jako stádo doprostřed pastviště, zahemží se to zas lidmi.
#2:13 Už vytáhl ten, kdo jim bude razit cestu. Prorazili a procházejí branou a z ní vycházejí. Jejich Král prošel před nimi, Hospodin v jejich čele! 
#3:1 Pravím: Slyšte, představitelé Jákobovi, vůdcové izraelského domu! Což není vaším úkolem znát právo?
#3:2 Dobro nenávidíte a milujete zlo, stahujete z lidu kůži a z jeho kostí maso.
#3:3 Požírají maso mého lidu, sdírají z nich kůži a lámou jim kosti, jako by je kouskovali do hrnce, jak do kotle maso.
#3:4 Až budou úpět k Hospodinu, neodpoví jim; skryje před nimi tvář v onen čas, jak zasluhují za svoje zlé skutky.
#3:5 Toto praví Hospodin proti prorokům, kteří svádějí můj lid: Když mají svými zuby co kousat, mluví o pokoji; když pak jim někdo do úst nic nedá, vyhlašují proti němu svatý boj.
#3:6 Proto vám nastane noc bez vidění, tma bez věštby. Nad těmi proroky zajde slunce, den se nad nimi zachmuří.
#3:7 Tu se budou jasnovidci stydět, rdít se budou věštci, všichni si zakryjí ústa, neboť Bůh neodpoví.
#3:8 Já však jsem naplněn mocí, duchem Hospodinovým, právem a bohatýrskou silou, abych mohl Jákobovi povědět o jeho nevěrnosti, Izraeli o jeho hříchu.
#3:9 Slyšte to, představitelé domu Jákobova, vůdcové izraelského domu, vy, kteří si hnusíte právo a překrucujete všechno, co je přímé!
#3:10 Sijón budujete krveprolitím, Jeruzalém bezprávím.
#3:11 Jeho představitelé soudí za úplatek, jeho kněží učí za odměnu, jeho proroci věští za stříbro. Přitom spoléhají na Hospodina a říkají: „Což není Hospodin uprostřed nás? Na nás nepřijde nic zlého.“
#3:12 Proto bude Sijón kvůli vám zorán jako pole, z Jeruzaléma budou sutiny, z hory Hospodinova domu návrší zarostlá křovím. 
#4:1 I stane se v posledních dnech, že se hora Hospodinova domu bude tyčit nad vrcholy hor, bude povznesena nad pahorky a budou k ní proudit národy.
#4:2 Mnohé pronárody půjdou a budou se pobízet: „Pojďte, vystupme na Hospodinovu horu, do domu Boha Jákobova. Bude nás učit svým cestám a my budeme chodit po jeho stezkách.“ Ze Sijónu vyjde zákon, slovo Hospodinovo z Jeruzaléma.
#4:3 On bude soudit mnohé národy, ztrestá mocné pronárody, i ty nejvzdálenější. I překují své meče na radlice, svá kopí na vinařské nože. Pronárod nepozdvihne meč proti pronárodu, nebudou se již cvičit v boji.
#4:4 Každý bude bydlit pod svou vinnou révou, pod svým fíkovníkem, a nikdo ho nevyděsí. Tak promluvila ústa Hospodina zástupů.
#4:5 Každý jiný lid si chodí ve jménu svých bohů, ale my budeme chodit ve jménu Hospodina, našeho Boha, navěky a navždy.
#4:6 V onen den, je výrok Hospodinův, se ujmu chromé, shromáždím zapuzenou, tu, s níž jsem zle naložil.
#4:7 Chromou budu mít za pozůstatek lidu, vypuzenou za mocný národ. A Hospodin bude nad nimi kralovat na hoře Sijónu od nynějška až navěky.
#4:8 Ty, věži stáda, návrší dcery sijónské, tobě připadne, k tobě se navrátí dřívější vladařství, království dcery jeruzalémské.
#4:9 Proč nyní tolik hořekuješ? Což nemáš u sebe krále? Zahynul tvůj rádce, že tě zachvátila bolest jako rodičku?
#4:10 Svíjej a válej se v křečích jako rodička, sijónská dcero, neboť teď musíš vyjít z města a bydlet na poli a přijdeš až do Babylóna; tam budeš vysvobozena, tam tě Hospodin vykoupí z rukou tvých nepřátel.
#4:11 Nyní se proti tobě sbírají mnohé pronárody. Říkají: „Sijón je zhanoben, pokochejme se pohledem na něj.“
#4:12 Oni však neznají smýšlení Hospodinovo, nerozumějí jeho úradku, že je shromažďuje jako snopy na humno.
#4:13 Dej se do mlácení, sijónská dcero, neboť učiním tvůj roh ze železa, kopyta z bronzu. Rozdrtíš mnoho národů. Jejich zisk zasvětím Hospodinu jako klatý a jejich majetek Pánu celé země.
#4:14 Nyní si však zasazuj smuteční zářezy, dcero porobená; sevře nás obležení. Izraelova soudce budou bít do tváře holí. 
#5:1 A ty, Betléme efratský, ačkoli jsi nejmenší mezi judskými rody, z tebe mi vzejde ten, jenž bude vládcem v Izraeli, jehož původ je odpradávna, ode dnů věčných.
#5:2 I když je Hospodin vydá v plen do chvíle, než rodička porodí, zbytek jeho bratří se vrátí zpět k synům Izraele.
#5:3 I postaví se a bude je pást v Hospodinově moci, ve vyvýšeném jménu Hospodina, svého Boha, a budou bydlet bezpečně; jeho velikost bude nyní sahat až do dálav země.
#5:4 A on sám bude pokoj. Až vtrhne Ašúr do naší země, až bude šlapat po našich palácích, postavíme proti němu sedm pastýřů, osm vojevůdců lidu.
#5:5 Ti budou spravovat asyrskou zemi mečem v přístupech do země Nimrodovy. On nás vysvobodí před Ašúrem, až vtrhne do naší země, až bude šlapat po našem pomezí.
#5:6 I bude pozůstatek Jákoba uprostřed mnoha národů jako rosa od Hospodina, jako vláha pro bylinu, která neskládá naději v člověka, nečeká na syny lidské.
#5:7 Pozůstatek Jákoba bude mezi pronárody, uprostřed mnoha národů, jako lev mezi zvířaty divočiny, jak lvíče ve stádech ovcí: přežene-li se, rozšlape je, rozsápe a nikdo mu nic nevyrve.
#5:8 Budeš vyvýšen nad své protivníky, všichni tvoji nepřátelé budou vyhlazeni.
#5:9 I stane se v onen den, je výrok Hospodinův, že vyhladím tvé koně z tvého středu a zničím tvé vozy.
#5:10 Vyhladím města tvé země a všechny tvé pevnosti zbořím.
#5:11 Vyhladím z tvých rukou tvé čáry, nebudeš mít mrakopravce.
#5:12 Vyhladím tvé tesané modly a posvátné sloupy z tvého středu; dílu svých rukou se již nebudeš klanět.
#5:13 Vyvrátím tvé posvátné kůly z tvého středu a zahladím tvá města.
#5:14 V hněvu a rozhořčení vykonám pomstu na pronárodech, které mě neposlouchají. 
#6:1 Slyšte nyní, co praví Hospodin: Povstaň, veď soud s horami, nechť slyší pahorky tvůj hlas!
#6:2 Slyšte Hospodinův spor, hory i nepohnutelné základy země! Hospodin vede spor se svým lidem, činí výtky Izraeli:
#6:3 Lide můj, co jsem ti udělal? Jaké potíže jsem ti působil? Odpověz mi.
#6:4 Vždyť jsem tě vyvedl z egyptské země, vykoupil jsem tě z domu otroctví, poslal jsem před tebou Mojžíše, Árona a Mirjam.
#6:5 Lide můj, jenom si vzpomeň na záměr Baláka, krále moábského, a co mu odpověděl Bileám, Beórův syn, jak jsi pak táhl od Šitímu do Gilgálu, abys pochopil Hospodinovy spravedlivé činy.
#6:6 „Jak předstoupím před Hospodina? S čím se mám sklonit před Bohem na výšině? Mohu před něj předstoupit s oběťmi zápalnými, s ročními býčky?
#6:7 Cožpak má Hospodin zalíbení v tisících beranů, v deseti tisících potoků oleje? Což smím dát za svou nevěrnost svého prvorozence, v oběť za svůj hřích plod svého lůna?“
#6:8 Člověče, bylo ti oznámeno, co je dobré a co od tebe Hospodin žádá: jen to, abys zachovával právo, miloval milosrdenství a pokorně chodil se svým Bohem.
#6:9 Slyš, Hospodin volá na město. - Pohotovou pomocí je vzhlížet k tvému jménu. - Slyšte o metle! Kdo ji přivolal?
#6:10 Což stále budou ve svévolníkově domě poklady nabyté svévolí a míra ošizená, popouzející k hněvu?
#6:11 Mám snad za poctivé prohlásit nespravedlivé váhy a váček s falešným závažím?
#6:12 Jeho boháči jsou plni násilí, jeho obyvatelé mluví zrádně, v jejich ústech je lstivý jazyk.
#6:13 Proto na tebe sešlu nemoci a rány, pro tvoje hříchy tě zpustoším.
#6:14 Budeš jíst, ale nenasytíš se, budeš jen hladovět; nevyvázneš s tím, co sis dalo stranou, a s čím vyvázneš, vydám meči.
#6:15 Budeš sít, a žní se nedočkáš; budeš lisovat olivy, a olejem se nepomažeš; vylisuješ mošt, a vína se nenapiješ.
#6:16 Ustanovení Omrího se drží toto město, všech skutků Achabova domu; řídíte se jejich radami. Proto učiním, že budeš vzbuzovat úděs a tvoji obyvatelé posměch. Ponesete potupu mého lidu. 
#7:1 Běda mně! Jsem jako paběrky letní sklizně, jako paběrky po vinobraní. Není tu hrozen k snědku ani raný fík, po němž toužím.
#7:2 Zbožný vymizel ze země, přímého mezi lidmi není. Všichni strojí vražedné úklady, jeden druhého do sítě loví.
#7:3 Oběma rukama páchají zlo pro svůj prospěch, velmož i soudce úplatky vymáhají, mocný mluví, jak se mu zachce; a všechno zpřevracejí.
#7:4 Nejlepší z nich je jak potměchuť, nejpoctivější jako plot z trní. Přichází den, který vyhlíželi tvoji proroci, den tvého navštívení; již nastává mezi nimi rozruch.
#7:5 Nevěřte bližnímu, nespoléhejte na přítele; před tou, jež uléhá po tvém boku, se střez otevřít ústa.
#7:6 Syn tupí otce, dcera povstává proti matce, snacha proti tchyni; každý má nepřátele ve vlastním domě.
#7:7 Ale já budu vyhlížet k Hospodinu, čekat na Boha, který mě spasí. Můj Bůh mě vyslyší.
#7:8 „Neraduj se nade mnou, má nepřítelkyně! Padla-li jsem, povstanu, sedím-li ve tmě, mým světlem je Hospodin.
#7:9 Chci nést Hospodinův hrozný hněv, neboť jsem proti němu hřešila, dokud on neurovná můj spor a nezjedná mi právo. Vyvede mě na světlo a uzřím jeho spravedlnost.
#7:10 Má nepřítelkyně to uvidí a pokryje ji hanba. To je ta, která mi říká: „Kde je Hospodin, tvůj Bůh?“ Na vlastní oči ji spatřím, až bude pošlapána jak bláto na ulicích.“
#7:11 To bude den budování tvých zdí. Onoho dne se rozšíří tvé hranice.
#7:12 Bude to den, kdy přijdou k tobě z Asýrie a egyptských měst, od Egypta až k řece Eufratu, od moře k moři, od hory k hoře.
#7:13 Ale země bude zpustošena kvůli svým obyvatelům, pro ovoce jejich skutků.
#7:14 Pas berlou svou svůj lid, ovce dědictví svého, ty, který sám jediný přebýváš v lesní divočině na Karmelu. Ať se pasou v Bášanu a Gileádu jako za dnů odvěkých,
#7:15 jako za dnů, kdy jsi vycházel z egyptské země. Ukáži lidu podivuhodné věci.
#7:16 Spatří to pronárody a zastydí se za všechno své siláctví. Přiloží ruku na ústa, uši jim ohluchnou.
#7:17 Prach budou lízat jako had; jako zeměplazi polezou ve zmatku ze svých hradišť. Se strachem se budou obracet k Hospodinu, našemu Bohu. Tebe se budou bát.
#7:18 Kdo je Bůh jako ty, který snímá nepravost, promíjí nevěrnost pozůstatku svého dědictví! Nesetrvává ve svém hněvu, neboť si oblíbil milosrdenství.
#7:19 Opět se nad námi slituje, rozšlape naše nepravosti. Do mořských hlubin vhodíš všechny jejich hříchy,
#7:20 prokážeš věrnost Jákobovi, milosrdenství Abrahamovi, jak jsi za dnů pradávných přísahal našim otcům.  

\book{Nahum}{Nah}
#1:1 Výnos o Ninive. Kniha vidění Nahuma Elkóšského.
#1:2 Hospodin je Bůh žárlivý a mstitel. Mstitel je Hospodin, vládce rozhořčený. Pomstou stíhá Hospodin své protivníky, hněvem pronásleduje nepřátele.
#1:3 Hospodin je shovívavý a velkorysý, ale viníka bez trestu neponechá. Jeho cesta vede vichřicí a bouří, mračna jsou prach zvířený jeho nohama.
#1:4 Oboří se na moře a vysuší je, všechny řeky nechá vyschnout. Uvadá Bášan a Karmel, vadne i výhonek Libanónu.
#1:5 Hory se před ním třesou, pahorky se zmítají, země před jeho tváří se vzdouvá, svět a všichni, kteří na něm bydlí.
#1:6 Kdo odolá jeho hroznému hněvu? Kdo obstojí před jeho planoucím hněvem? Jeho rozhořčení sálá jak oheň, skály se před ním drolí.
#1:7 Hospodin je dobrý, je záštitou v den soužení, zná se k těm, kteří se k němu utíkají.
#1:8 Valící se povodní učiní konec tomuto místu, tma bude pronásledovat jeho nepřátele.
#1:9 Co zamýšlíte proti Hospodinu? On učiní konec. Soužení už podruhé nenastane.
#1:10 Zapletou se v hloží svými pitkami zpiti. Jako suché strniště budou zcela pozřeni ohněm.
#1:11 Z tebe vyšel ten, který zamýšlí proti Hospodinu zlé věci, ničemný rádce.
#1:12 Toto praví Hospodin: Ať si žijí sebepokojněji, ať je jich sebevíc, budou skoseni; ničemník zajde. Pokořoval jsem tě, Sijóne, už tě nebudu pokořovat.
#1:13 Jeho jařmo nyní přerazím a spadne z tebe, zpřetrhám tvá pouta.
#1:14 A proti tobě, Ašúre, vydává Hospodin příkaz: Nebude již nikoho, kdo by dále rozséval tvé jméno. Z domu tvých bohů vyhladím modly tesané i lité. Připravil jsem ti hrob. Jsi zlořečený. 
#2:1 Hle, po horách přichází zvěstovatel radosti, jenž ohlašuje pokoj. Slav, Judo, své slavnosti, plň své sliby. Už nikdy na tebe nepřitáhne ničemník, je zcela zahlazen.
#2:2 Ninive, už na tebe táhne ten, jenž tě rozmetá. Jen si hlídej pevnost. Číhej u cesty. Posilni svá bedra. Seber veškerou svou sílu.
#2:3 Hospodin vrátí Jákobovi důstojnost, důstojnost, jaká patří Izraeli. Potřeli jej vetřelci, zničili jeho odnože.
#2:4 Rudý je štít bohatýrů, v šarlat oděni jsou válečníci. Jiskří jako oheň ocel vozů, když se připravují k bitvě. Cypřišová ratiště se kymácejí.
#2:5 Vozy se šíleně řítí ulicemi, ženou se přes prostranství podobny pochodním, míhají se jako blesky.
#2:6 S připomínkou svých vznešených božstev klopýtavým během ženou se k ninivským hradbám. Je zřízen ochranný kryt.
#2:7 Brány při řece jsou otevřeny; chrám se hroutí.
#2:8 Je rozhodnuto. Obnažena a odvlékána je Ištar, její děvy kvílí jako holubice a bijí se v prsa.
#2:9 Ninive bývalo odedávna jako rybník plný vod. A nyní utíkají. Zastavte se, stůjte! Nikdo se však ani neohlédne.
#2:10 Rabujte stříbro, rabujte zlato! Jaké nekonečné množství zásob! Jak oslnivý lesk všemožných vzácností!
#2:11 Vše pusté! Zpustlé! Zpustošené! Ztratili odvahu, kolena se třesou. Smrtelná úzkost zachvátila veškerá bedra, tváře všech zbledly.
#2:12 Kam se podělo to doupě lvů, místo, kde lvíčata krmívali? Když lev vycházel za kořistí, lvíče tam zůstávalo a nikdo je nevyplašil.
#2:13 Lev přinášel hojnost úlovku mláďatům, lvicím to, co zadávil, brlohy si naplňoval kořistí, svá doupata tím, co ulovil.
#2:14 Chystám se však na tebe, je výrok Hospodina zástupů. Ninivskou vozbu spálím na prach, tvá lvíčata pozře meč. Vymýtím ze země tvoje kořistnictví, nebude už slyšet hlas tvých poslů! 
#3:1 Běda městu, jež se brodí v krvi! Je samá přetvářka, je plné loupeží, kořistění nebere konce.
#3:2 Slyš! Bičů svist, dunění kol! Dusot koní, hřmot vozby!
#3:3 Útok jezdců, zášlehy mečů, blýskání kopí! Přemnoho skolených, hromady mrtvých těl, bezpočet mrtvol, až přes ně klopýtají!
#3:4 To všechno pro mnohá smilstva Nevěstky, přesvůdné mistryně kouzel. Svým smilstvem kupčí s pronárody, svými kouzly kupčí s čeleděmi.
#3:5 Chystám se však na tebe, je výrok Hospodina zástupů. Až přes tvář ti vyhrnu sukni a ukáži tvou nahotu pronárodům, královstvím tvou hanbu.
#3:6 Nakydám na tebe neřád, potupím tě, postavím na pranýř.
#3:7 Kdokoli tě spatří, prchne od tebe. Řekne: „Ninive propadlo záhubě, kdo se nad ním ustrne?“ Kde vyhledám ty, kdo by tě potěšili?
#3:8 Jsi snad lepší, než byly Théby Amónovy, trůnící mezi proudy, obklopeny vodami? Moře jim bylo valem, jejich hradby se zdvihaly z moře.
#3:9 Kúš tak mocný, Egypt nekonečný, Pút i Lúbim byly tvými pomocníky.
#3:10 I Théby byly přesídleny, odešly do zajetí. Jejich nemluvňata byla drcena na rozích všech ulic; o jejich slavné se losovalo, všichni jejich velmožové byli spoutáni řetězy.
#3:11 I ty se opojíš a budeš omámeno. I ty budeš hledat záštitu před nepřítelem.
#3:12 Všechny tvé pevnosti však budou jako fíkovníky s ranými plody: až se jimi zatřese, spadnou do úst tomu, kdo je chce jíst.
#3:13 Tvůj lid uprostřed tebe jsou baby. Brány tvé země se dokořán otevřou tvým nepřátelům, tvé závory pozře oheň.
#3:14 Navaž si vodu, než budeš obleženo! Zpevni svá opevnění! Šlapej jíl, hněť hlínu, chop se formy na cihly!
#3:15 Oheň tě tam pozře, vytne meč, pozře tě jako brouci. Byť vás bylo spousta jako brouků, byť vás bylo spousta jako kobylek,
#3:16 byť tvých překupníků bylo víc než nebeských hvězd, brouci se vykuklí a odlétnou.
#3:17 Tvoji strážci jsou jak kobylky, tvoji úředníci jak hejna sarančat: v zimním čase zalezou do škvír zídek, ale jak vysvitne slunce, ulétnou a nevíš, kam se poděly.
#3:18 Již dřímou tvoji pastýři, asyrský králi, ulehli tvoji vznešení. Tvůj lid je rozehnán po horách, nikdo jej neshromáždí.
#3:19 Nikdo neošetří tvé těžké poranění, bolestná zůstane tvá rána. Všichni, kdo o tobě uslyší, zatleskají nad tebou v dlaně. Vždyť na koho nedoléhala bez přestání tvá zloba?  

\book{Habakkuk}{Hab}
#1:1 Výnos, který přijal ve vidění prorok Abakuk.
#1:2 Jak dlouho již volám o pomoc, Hospodine, a ty neslyšíš. Úpím k tobě pro násilí, a ty nezachraňuješ.
#1:3 Proč mi dáváš vidět ničemnosti a mlčky na trápení hledíš? Doléhají na mne zhouba a násilí, rozrostli se spory a sváry.
#1:4 Proto je tak ochromen zákon a nikdy se neprosadí právo. Spravedlivého obkličuje svévolník, proto je právo tak překrouceno.
#1:5 Pohleďte na pronárody, popatřte! Ustrnete údivem nad tím, co vykonám za vašich dnů. Nebudete věřit, až se o tom bude vypravovat.
#1:6 Já totiž povolám Kaldejce, pronárod krutý a prchlivý, jenž projde široširou zemí, aby se zmocnil příbytků, které mu nepatří.
#1:7 Je příšerný a hrozný, vyhlašuje vlastní svrchované právo.
#1:8 Jeho koně jsou hbitější než levharti, dravější než vlci za večera; jeho jezdci uhánějí tryskem, jeho jízda se zdaleka žene, letí jako orlice, jež spěchá za kořistí.
#1:9 Každý z nich se žene za násilím, dychtivě míří vpřed. Sebral zajatců jak písku.
#1:10 Z králů si tropí žerty, hodnostáři jsou mu k smíchu, každé pevnosti se směje, nahrne prach a dobude ji.
#1:11 Potom přelétne jak vítr a táhne dál obtížen vinou, za boha má vlastní sílu.
#1:12 Cožpak nejsi od pradávna, Hospodine, svatý Bože můj? My nezemřeme. Hospodine, pověřil jsi ho soudem, Skálo, uložil jsi mu, aby trestal.
#1:13 Tvé oči jsou čisté, nemohou se dívat na zlo a hledět na trápení. Proč tedy trpíš věrolomné, mlčíš, když svévolník pohlcuje spravedlivějšího?
#1:14 Což jsi učinil lidstvo jako mořské ryby, jako havěť, nad níž nevládne nikdo?
#1:15 On udicí vytahuje všechny a lapá je do své sítě, do svého nevodu je chytá. Proto se raduje a jásá.
#1:16 Proto obětuje své síti a svému nevodu pálí kadidlo; neboť skrze ně se mu dostává výborného podílu a nejtučnějšího pokrmu.
#1:17 Což bude svou síť vyprazdňovat bez přestání, bez soucitu vraždit pronárody? 
#2:1 Postavím se na své strážné stanoviště, budu stát na hlásce a vyhlížet, abych seznal, co ke mně promluví a jakou odpověď dostanu na svoji stížnost.
#2:2 Hospodin mi odpověděl, řekl: „Zapiš to vidění, zaznamenej je na tabulky, aby si je čtenář mohl snadno přečíst.
#2:3 Vidění už ukazuje k určitému času, míří neomylně k cíli; prodlévá-li vyčkej, neboť přijde zcela jistě, zadržet se nedá.“
#2:4 Pozor na opovážlivce; není v něm duše přímá. Spravedlivý bude žít pro svou věrnost.
#2:5 Jako víno oklame, tak neobstojí troufalý muž. Rozevírá chřtán jako podsvětí, zůstane jako smrt nenasytný, i kdyby pro sebe zabral všechny pronárody a všechny národy shromáždil k sobě.
#2:6 Což ti všichni neužijí proti němu pořekadel, posměšných popěvků a narážek na něj? Bude se říkat: Běda tomu, kdo hromadí, co mu nepatří. Jak dlouho? I tomu, kdo zástavou zatěžuje.
#2:7 Což tvoji dlužníci náhle nepovstanou, nevschopí se ti, kteří se třesou strachem? Budeš jim vydán v plen.
#2:8 Za to, že jsi plenil mnohé pronárody, budou plenit všechny ostatní národy tebe za prolitou lidskou krev a za násilí páchané na zemi, na městu i všech jeho obyvatelích.
#2:9 Běda tomu, kdo chamtivě shání mrzký zisk pro svůj dům, aby si založil hnízdo na výšině, aby se vyprosil ze spárů zla.
#2:10 Rozhodl ses k hanbě svého domu učinit konec mnohým národům; hřešíš sám proti sobě.
#2:11 I kámen ze zdi bude křičet, trám z kovu mu odpovídat.
#2:12 Běda tomu, kdo staví město na prolité krvi a zabezpečuje tvrz bezprávím.
#2:13 Hle, což to není od Hospodina zástupů, když „lidé se namáhají, a pozře to oheň, národy se lopotí pro nic za nic“?
#2:14 Země bude naplněna poznáním Hospodinovy slávy, jako vody pokrývají moře.
#2:15 Běda tomu, kdo napájí svého bližního, tobě, který přiměšuješ jed svého hněvu a opíjíš ho a na jeho nahotu se díváš.
#2:16 Dosyta se najíš pohany, ne slávy. I ty budeš pít a ukáže se tvá neobřezanost. I na tebe dojde číše z pravice Hospodinovy a zlořečení na tvou slávu.
#2:17 Násilí na Libanónu se obrátí proti tobě, pobíjení zvířat ti nažene děsu za prolitou lidskou krev a za násilí páchané na zemi, na městu i všech jeho obyvatelích.
#2:18 Co prospěje tesaná modla, již vytesal její tvůrce, modla litá, učitel lži? Ať si v ni doufá její tvůrce, zhotovuje pouze němé bůžky.
#2:19 Běda tomu, kdo říká dřevu: „Procitni“, kdo říká němému kameni: „Vzbuď se.“ Něco takového má být učitelem? I když je to potaženo zlatem a stříbrem, nemá to žádného ducha.-
#2:20 Hospodin je ve svém svatém chrámu. Ztiš se před ním, celá země! 
#3:1 Modlitba proroka Abakuka; na způsob tklivé písně.
#3:2 Hospodine, slyšel jsem tvou zprávu; bojím se o tvoje dílo, Hospodine, v tento čas je zachovej, v tento čas je uveď v známost. V nepokoji pamatuj na slitování!
#3:3 Z Témanu přichází Bůh, z hory Páranu Svatý. Nebesa přikrývá velebnost jeho, země je plná chvalozpěvů.
#3:4 Září jako světlo, po straně má rohy, v nichž se skryla jeho síla.
#3:5 Před ním se žene mor, nákaza valí se za ním.
#3:6 Stanul, a měří zemi, pohlédl, a zatřásl pronárody; pukají odvěká horstva, pahorky pravěké se hroutí; cesta věčnosti patří jemu.
#3:7 Vidím, jak hrouží se v nicotu kúšanské stany, jak se chvějí stanové houně midjanské země.
#3:8 Vzpanul snad Hospodin proti řekám? Zda proti řekám plane tvůj hněv nebo tvá prchlivost proti moři? Jedeš na svých ořích, tvoje vozy přivážejí spásu.
#3:9 Luk máš připravený ke splnění přísah, daných dávným pokolením. Řekami rozpolcuješ zemi.
#3:10 Spatřily tě hory a chvějí se v křeči. Přehnala se průtrž mračen, propastná tůň do křiku se dala, vysoko vzpíná své ruce.
#3:11 Do svého obydlí se stáhlo slunce i měsíc, utekly před světlem tvých šípů, před září tvého blištivého kopí.
#3:12 Rozlícen po zemi kráčíš, po pronárodech ve hněvu dupeš.
#3:13 Vyšel jsi spasit svůj lid, spasit pomazaného svého. Rozdrtil jsi hlavu svévolníkova domu, obnažils ho od základů k hrdlu.
#3:14 Jeho vlastními holemi rozrazil jsi hlavu jeho knížat, ženoucích se jako smršť, aby mě rozprášili; jásali divoce, jako by už tajně pozřeli utištěného.
#3:15 Svými oři jsi pošlapal moře, vzdouvající se nesmírná vodstva.
#3:16 Uslyšel jsem o tom a celý se třesu, chci se ozvat a rty se mi chvějí; jakoby kostižer zachvátil mé kosti, podlamují se pode mnou nohy. Budu však klidně očekávat den soužení, až přitáhnou a přepadnou lid.
#3:17 I kdyby fíkovník nevypučel, réva nedala výnos, selhala plodnost olivy, pole nevydala pokrm, z ohrady zmizel brav, ve chlévech dobytek nebyl,
#3:18 já budu jásotem oslavovat Hospodina, jásat ke chvále Boha, který je má spása.
#3:19 Panovník Hospodin je moje síla. Učinil mé nohy hbité jako nohy laně, po posvátných návrších mi dává šlapat. Pro předního zpěváka za doprovodu strunných nástrojů.  

\book{Zephaniah}{Zeph}
#1:1 Slovo Hospodinovo, které se stalo k Sofonjášovi, synu Kúšiho, syna Gedaljáše, syna Amarjáše, syna Chizkijášova, za dnů judského krále Jóšijáše, syna Amónova:
#1:2 Sklidím z povrchu země úplně všechno, je výrok Hospodinův.
#1:3 Sklidím lidi i dobytek, sklidím nebeské ptactvo i mořské ryby, pohoršení spolu se svévolníky. Vyhladím člověka z povrchu země, je výrok Hospodinův.
#1:4 Napřáhnu svou ruku na Judu, na všechny obyvatele Jeruzaléma, vyhladím z tohoto místa, co zůstalo z Baala, i jméno žreců a kněží,
#1:5 rovněž ty, kdo se na střechách klanějí zástupu nebeskému, ty, kdo se klanějí hned Hospodinu a jím se zapřísahají, a hned zase přísahají při svém Melekovi,
#1:6 ty, kdo odpadli od Hospodina, kdo Hospodina nehledají a nedotazují se ho.
#1:7 Ztiš se před Panovníkem Hospodinem, vždyť den Hospodinův je blízko! Hospodin připravil oběť, jako svaté oddělil ty, které pozval.
#1:8 Stane se v den Hospodinova obětního hodu: Ztrestán velmože i královské syny, všechny ty, kdo si oblékají cizokrajný šat;
#1:9 onoho dne ztrestám každého, kdo přeskakuje práh, ty, kdo naplňují dům svého Pána násilím a lstí.
#1:10 V onen den, je výrok Hospodinův, se ozve úpění od Rybné brány a kvílení z druhé strany, veliký třeskot od pahorků.
#1:11 Kvílejte, obyvatelé kotliny, všechen lid kramářů zajde. Budou vyhlazeni všichni, kdo odvažují stříbro.
#1:12 V té době prohledám Jeruzalém se svítilnami, ztrestám muže, kteří jsou jako zkyslé víno nad svým kalem, kteří si v srdci říkají: „Hospodin neudělá nic dobrého ani zlého.“
#1:13 V plen bude vydán jejich blahobyt a jejich domovy ve zpustošení; vystaví domy, a nebudou v nich bydlet, vysázejí vinice, a vína z nich neokusí.
#1:14 Veliký den Hospodinův je blízko, je blízký a převelice rychlý. Slyš, Hospodinův den je tady! Zoufale volá bohatýr do boje.
#1:15 Onen den bude dnem prchlivosti, dnem soužení a tísně, dnem ničení a zkázy, dnem tmy a temnot, dnem oblaku a mrákoty,
#1:16 dnem polnice a válečného ryku nad opevněnými městy, nad vyvýšenými cimbuřími.
#1:17 Sešlu na lidi soužení a budou tápat jako slepci, neboť zhřešili proti Hospodinu; jejich krev bude odklizena jako prach, jejich vnitřnosti jako mrva.
#1:18 Jejich stříbro ani zlato je nedokáže v den Hospodinovy prchlivosti vysvobodit; ohněm jeho rozhorlení bude pozřena celá země. Ano, učiní náhlý konec všem obyvatelům země! 
#2:1 Vzchop se, vzpamatuj, pronárode beze studu,
#2:2 dříve než vstoupí v platnost výnos, vždyť jako plevy přelétne den, dříve než vás stihne planoucí hněv Hospodinův, dříve než vás stihne den Hospodinova hněvu.
#2:3 Hledejte Hospodina, všichni pokorní země, kdo jednáte podle jeho práva. Hledejte spravedlnost, hledejte pokoru, snad se skryjete v den Hospodinova hněvu.
#2:4 Gáza bude opuštěna, Akšalón se stane zpustošeným místem, Ašdód bude vyhnán za poledne, vyrván bude i s kořeny Ekrón.
#2:5 Běda vám, obyvatelé Přímoří, pronárode Keréťanů! Slovo Hospodinovo je proti vám. Kenaane, země pelištejská, vyhubím tě do posledního obyvatele.
#2:6 Přímoří se stane krajem pastvin s chýšemi pastýřů, s ovčími ohradami,
#2:7 připadne pozůstatku Judova domu; budou na nich pást a večer budou odpočívat v aškalónských domech, neboť je navštíví Hospodin, jejich Bůh, a změní jejich úděl.
#2:8 Slyším utrhání Moábovo, hanobení od Amónovců. Utrhají mému lidu, proti jeho pomezí se vypínají.
#2:9 Avšak, jakože jsem živ, je výrok Hospodina zástupů, Boha Izraele, Moáb bude jako Sodoma, Amónovci jako Gomora, stanou se panstvím kopřiv, solnou proláklinou, krajem navěky zpustošeným. Oloupí je pozůstatek mého lidu, zbytek mého národa je zdědí.
#2:10 To vše se jim stane za jejich pýchu, neboť utrhali lidu Hospodina zástupů a vypínali se nad něj.
#2:11 Hrozný bude Hospodin, až udeří na ně: stihne úbitěmi všechny bohy země. Každý se mu bude klanět ze svého místa, i všechny ostrovy pronárodů.
#2:12 Také vy, Kúšijci, budete skoleni mým mečem!
#2:13 Napřáhne svou ruku proti severu, zahubí Ašúra, z Ninive učiní zpustošené místo, stepní suchopár;
#2:14 v jeho středu budou odpočívat stáda, všechna zvěř toho pronároda. Na hlavicích sloupů tam bude nocovat pelikán a sýček. Jaký to zpěv zazní z oken! Na prahu trosky, cedrové deštění vytrháno.
#2:15 Tohle je město jásotu, které tak bezpečně trůnilo a v srdci si namlouvalo: „Není nade mne.“ Jak bylo zpustošeno! Jen zvěř tam odpočívá. Každý, kdo přejde kolem, posměšně sykne a mávne rukou. 
#3:1 Běda městu, které hubí, vzpurnému a potřísněnému!
#3:2 Poslouchat nechce, napomenutí nepřijímá, nedoufá v Hospodina, nepřibližuje se k svému Bohu.
#3:3 Jeho velmožové jsou uprostřed něho jako řvoucí lvi, jeho soudcové jak vlci za večera, do jitra nenechají nic neohryzaného;
#3:4 jeho proroci jsou chvastouni, muži věrolomní; jeho kněží znesvěcují, co je svaté, znásilňují zákon.
#3:5 Spravedlivý Hospodin je uprostřed něho, bezpráví neučiní, každé jitro vynáší na světlo svůj soud, denně, bez ustání, ale ten, kdo se dopouští bezpráví, nezná studu.
#3:6 Vyhladil jsem národy, jejich cimbuří zpustla, jejich ulice jsem zničil, nikdo jimi neprochází; jejich města jsou vylidněna, nikdo v nich nesídlí.
#3:7 Řekl jsem: „Měj přede mnou bázeň a přijmi mé napomenutí!“ Jeho domov by nebyl vyhlazen žádnou z těch věcí, jimiž je ztrestám. Ale oni hned za časného jitra kazí všechno, čím se zabývají.
#3:8 A tak mě očekávejte, je výrok Hospodinův, v den, kdy povstanu ke kořisti. To je můj rozsudek: Seberu pronárody, shromáždím království a vyleji na ně svůj hrozný hněv, všechen svůj planoucí hněv. Ohněm mého rozhorlení bude pozřena celá země.
#3:9 Tehdy očistím rty každého lidu a všichni budou vzývat Hospodinovo jméno a sloužit mu společnou paží.
#3:10 Z druhé strany kúšských řek moji vyznavači, má rozptýlená dcera, přinesou mi obětní můj dar.
#3:11 V onen den se už nemusíš stydět za žádné své skutky, jimiž ses proti mně vzepřelo. Tehdy z tvého středu odstraním tvé pyšné rozjařence a nepřipustím, aby ses opět začalo povyšovat na mé svaté hoře.
#3:12 Uprostřed tebe zanechám utištěný a nuzný lid, který se uteče k Hospodinovu jménu.
#3:13 Pozůstatek Izraele se již nebude dopouštět bezpráví a mluvit lživě, v jejich ústech se nenajde jazyk lstivý. Budou se pást a odpočívat a nikdo je nevyplaší.
#3:14 Zaplesej, sijónská dcero, zahlahol Izraeli! Raduj se a jásej z celého srdce, dcero jeruzalémská!
#3:15 Rozsudek nad tebou Hospodin zrušil, zbavil tě nepřítele. Král Izraele, Hospodin, je uprostřed tebe, neboj se už zlého!
#3:16 V onen den bude Jeruzalému řečeno: „Neboj se, Sijóne, nechť tvé ruce neochabnou!
#3:17 Hospodin, tvůj Bůh, je uprostřed tebe, bohatýr, který zachraňuje, raduje se z tebe a veselí, láskou umlká a opět nad tebou jásá a plesá.“
#3:18 Posbírám ty, kdo jsou zarmouceni, odloučeni od slavnosti, neboť jsou z tebe; břemeno potupy na nich leží.
#3:19 Hle, já si to v onen čas vyřídím se všemi, kdo tě pokořují, zachráním chromou, shromáždím zapuzenou, dám jim chválu a jméno všude v zemi jejich hanby.
#3:20 V onen čas vás přivedu, v ten čas vás shromáždím, dám vám jméno a chválu mezi všemi národy země, až změním váš úděl před vašimi zraky, praví Hospodin.  

\book{Haggai}{Hag}
#1:1 V druhém roce vlády krále Dareia, prvního dne šestého měsíce, stalo se slovo Hospodinovo skrze proroka Agea k judskému místodržiteli Zerubábelovi, synu Šealtíelovu a k veleknězi Jóšuovi, synu Jósadakovu.
#1:2 „Toto praví Hospodin zástupů: Tento lid říká: ‚Ještě nepřišel čas, čas k budování Hospodinova domu.‘“
#1:3 I stalo se slovo Hospodinovo skrze proroka Agea:
#1:4 „Je snad čas k tomu, abyste si bydleli v domech vykládaných dřevem, zatímco tento dům je v troskách?
#1:5 Nyní toto praví Hospodin zástupů: Vezměte si k srdci své cesty!
#1:6 Sejete mnoho, a sklízí se málo. Jen jezte, nenasytíte se; jen pijte, žízeň neuhasíte; jen se oblékejte, nezahřejete se. Kdo se dává najmout za mzdu, ukládá ji do děravého váčku.“
#1:7 „Toto praví Hospodin zástupů: Vezměte si k srdci své cesty.
#1:8 Vystupte na horu, přivezte dříví a budujte dům! V něm budu mít zalíbení, v něm se oslavím, praví Hospodin.
#1:9 Pachtíte se za mnoha věcmi a máte z toho málo. Co přinesete domů, já rozvěji. Proč se to děje? je výrok Hospodina zástupů. Protože můj dům je v troskách, zatímco vy se staráte každý jen o svůj dům.
#1:10 Proto nebe nad vámi zadrželo rosu a země zadržela svoji úrodu.
#1:11 Přivolal jsem sucho na zemi i na hory, na obilí i na vinný mošt a na čerstvý olej, na všechno, co přináší role, také na lidi i na dobytek, na všechno, co se rukama vytěží.“
#1:12 Tehdy Zerubábel, syn Šealtíelův, a velekněz Jóšua, syn Jósadakův, i celý pozůstatek lidu uposlechli hlasu Hospodina, svého Boha, a slov proroka Agea, protože ho poslal Hospodin, jejich Bůh, a lid se začal bát Hospodina.
#1:13 Hospodinův posel Ageus řekl z Hospodinova pověření lidu: „Já jsem s vámi, je výrok Hospodinův.“
#1:14 A Hospodin probudil ducha judského místodržitele Zerubábela, syna Šeltíelova, a ducha velekněze Jóšuy, syna Jósadakova, i ducha celého pozůstatku lidu. I přišli a dali se do díla na domě Hospodina zástupů, svého Boha,
#1:15 dvacátého čtvrtého dne šestého měsíce druhého roku vlády krále Dareia. 
#2:1 Dvacátého prvního dne sedmého měsíce stalo se slovo Hospodinovo skrze proroka Agea:
#2:2 „Řekni judskému místodržiteli Zerubábelovi, synu Šealtíelovu, a veleknězi Jóšuovi, synu Jósadakovu, a pozůstatku lidu:
#2:3 Kdo zůstal mezi vámi z těch, kteří viděli tento dům v jeho prvotní slávě? A jaký jej vidíte nyní? Není ve vašich očích jen pouhé nic?
#2:4 Buď rozhodný, Zerubábeli, je výrok Hospodinův, buď rozhodný, veleknězi Jóšuo, synu Jósadakův, buď rozhodný, všechen lide země, je výrok Hospodinův. Dejte se do díla, neboť jsem s vámi, je výrok Hospodina zástupů,
#2:5 podle slova, kterým jsem se vám zavázal, když jste vyšli z Egypta. Můj duch stojí uprostřed vás. Nebojte se!
#2:6 „Toto praví Hospodin zástupů: Ještě jednou, a bude to brzy, otřesu nebem i zemí, mořem i souší.
#2:7 Otřesu všemi pronárody, a přijdou s tím nejvzácnějším, co mají; a naplním tento dům slávou, praví Hospodin zástupů.
#2:8 Mé je stříbro, mé je zlato, je výrok Hospodina zástupů.
#2:9 Sláva tohoto nového domu bude větší nežli prvního, praví Hospodin zástupů. A na tomto místě budu udílet pokoj, je výrok Hospodina zástupů.“
#2:10 Dvacátého čtvrtého dne devátého měsíce druhého roku vlády Dareirovy stalo se slovo Hospodinovo k proroku Ageovi.
#2:11 Toto praví Hospodin zástupů: „Požádej kněze o poučení.
#2:12 ‚Kdyby někdo nesl svaté maso v cípu svého šatu a dotkl by se tím cípem chleba, kaše, vína, oleje nebo jakéhokoli jídla, bude to tím posvěceno?‘“ Kněží odpověděli: „Nikoli.“
#2:13 I řekl Ageus: „Kdyby se někdo poskvrněný mrtvolou dotkl něčeho z těchto věcí, poskvrní se to tím?“ Kněží odpověděli: „Poskvrní.“
#2:14 Ageus na to pokračoval: „Tak je tomu přede mnou s tímto lidem, s tímto pronárodem, je výrok Hospodinův, i se vším dílem jejich rukou: cokoli sem přinášejí je poskvrněné.
#2:15 Ale vezměte si k srdci, co bude od tohoto dne dál. Dříve než byl v Hospodinově domě kladen kámen ke kameni,
#2:16 co se dálo? Když někdo přišel k hromadě obilí pro dvacet měr, bylo na ní jen deset. Když někdo přišel k lisu, aby nabral z kádě padesát měr, bylo v ní jen dvacet.
#2:17 Bil jsem rzí, obilnou směsí a krupobitím vás i všechno dílo vašich rukou, a přece se nikdo z vás ke mně neobrátil, je výrok Hospodinův.
#2:18 Vezměte si k srdci, co bude od tohoto dne dál, od dvacátého čtvrtého dne devátého měsíce, ode dne, kdy byl znovu založen chrám Hospodinův! Vezměte si to k srdci.
#2:19 Ačkoli ještě není zrno na sýpce a vinná réva, fíkovník, granátovník a oliva nepřinesly dosud ovoce, od tohoto dne budu žehnat.“
#2:20 Dvacátého čtvrtého dne toho měsíce stalo se slovo Hospodinovo k Ageovi ještě jednou:
#2:21 „Řekni judskému místodržiteli Zerubábelovi: Otřesu nebem i zemí.
#2:22 Podvrátím trůny všech království a zlomím sílu království pronárodů, převrátím vůz i s těmi, kteří na něm jezdí, klesnou koně i jezdci, každý padne mečem svého bratra.
#2:23 V onen den, je výrok Hospodina zástupů, vezmu tebe, svého služebníka, Zerubábeli, synu Šealtíelův, je výrok Hospodinův, a učiním tě pečetním prstenem, neboť jsem tě vyvolil, je výrok Hospodina zástupů.“  

\book{Zechariah}{Zech}
#1:1 Osmého měsíce druhého roku vlády Dareiovy stalo se slovo Hospodinovo k proroku Zacharjášovi, synu Berekjáše, syna Idova:
#1:2 „Hospodin se hrozně rozlítil na vaše otce.
#1:3 Řekni jim tedy: Toto praví Hospodin zástupů: Obraťte se ke mně, je výrok Hospodina zástupů, a já se obrátím k vám, praví Hospodin zástupů.
#1:4 Nebuďte jako vaši otcové, k nimž volávali dřívější proroci: ‚Toto praví Hospodin zástupů: Obraťte se přece od svých zlých cest a od svých zlých skutků!‘ Ale neposlechli, pozornost tomu nevěnovali, je výrok Hospodinův.
#1:5 Vaši otcové - kde jsou? A proroci - mohli žít věčně?
#1:6 Avšak moje slova a nařízení, která jsem přikázal svým služebníkům prorokům, nezasáhla snad vaše otce? I obrátili se a přiznali: ‚Jak si Hospodin zástupů předsevzal, že s námi naloží podle našich cest a podle našich skutků, tak s námi naložil.‘“
#1:7 Dvacátého čtvrtého dne jedenáctého měsíce, to je měsíce šebátu, druhého roku vlády Dareiovy, stalo se slovo Hospodinovo k proroku Zacharjášovi, synu Berekjáše, syna Idova.
#1:8 V noci jsem viděl, hle, muž sedí na ryzáku, stojí mezi myrtovím v hlubině a za ním koně ryzí, plaví a bílí.
#1:9 Zeptal jsem se ho: „Co to znamená, můj pane?“ Boží posel, který se mnou mluvil, mi odvětil: „Já ti vysvětlím, co to znamená.“
#1:10 A muž stojící mezi myrtovím pokračoval: „To jsou ti, které vyslal Hospodin, aby procházeli zemí.“
#1:11 Oni pak oslovili Hospodinova posla stojícího mezi myrtovím: „Procházeli jsme zemí a celá země si poklidně bydlí.“
#1:12 Nato Hospodinův posel zvolal: „Hospodine zástupů, jak dlouho se ještě nechceš slitovat nad Jeruzalémem a nad judskými městy, jimž dáváš pociťovat svůj hrozný hněv již po sedmdesát let?“
#1:13 Hospodin poslovi, který se mnou mluvil, odpověděl slovy dobrými, slovy útěšnými.
#1:14 Posel, který se mnou mluvil, mi nato řekl: „Provolej, že toto praví Hospodin zástupů: ‚Horlím pro Jeruzalém a Sijón velkou horlivostí,
#1:15 jsem hrozně rozlícen na sebejisté pronárody, které hned přiložily ruku k zlému, jakmile jsem se na svůj lid jen maličko rozlítil.‘
#1:16 Proto praví Hospodin toto: ‚Obrátím se k Jeruzalému s hojným slitováním; bude v něm zbudován můj dům, je výrok Hospodina zástupů, a nad Jeruzalémem bude nataženo měřící pásmo.‘
#1:17 Dále provolej, že toto praví Hospodin zástupů: ‚Má města budou znovu oplývat dobrem.‘ Hospodin znovu potěší Sijón a znovu si vyvolí Jeruzalém.“ 
#2:1 Pak jsem se rozhlédl a vidím, hle, čtyři rohy.
#2:2 Zeptal jsem se posla, který se mnou mluvil: „Co to znamená?“ Odvětil mi: „To jsou ty rohy, které rozmetaly Judu, to je Izraele, a Jeruzalém.“
#2:3 I ukázal mi Hospodin čtyři kováře.
#2:4 Otázal jsem se: „Co se to chystají udělat?“ Odvětil mi: „To jsou ty rohy, které rozmetaly Judu, takže žádný nebyl schopen pozvednout hlavu. Tu přišli tito kováři, aby jimi zalomcovali a srazili rohy pronárodům, které pozdvihly roh proti judské zemi a chtěly ji rozmetat.“
#2:5 Opět jsem se rozhlédl a vidím, hle, muž s měřícím provazcem v ruce.
#2:6 Zeptal jsem se: „Kam jdeš?“ Odpověděl mi: „Vyměřit Jeruzalém, abych viděl, jak bude široký a dlouhý.“
#2:7 A hle, vychází posel, který se mnou mluvil. Jiný posel mu vyšel vstříc
#2:8 a oslovil ho: „Běž a promluv s tím mládencem. Řekni: Jeruzalém bude městem nehrazeným, tolik v něm bude lidí i dobytka.
#2:9 Já sám, je výrok Hospodinův, budu ohnivou hradbou kolem něho a slávou uprostřed něho.
#2:10 Běda, běda! Utečte ze severní země, je výrok Hospodinův. Rozprostřel jsem vás do čtyř nebeských větrů, je výrok Hospodinův.
#2:11 Běda! Zachraň se, Sijóne, ty poklidně si bydlící u babylónské dcery!
#2:12 Toto praví Hospodin zástupů, který mě poslal pro svoji slávu k pronárodům, které vás plenily: Kdo se vás dotkne, dotkne se zřítelnice mého oka.
#2:13 Hle, mávnu proti nim rukou a budou je plenit ti, kdo jim otročili. ‚I poznáte, že mě poslal Hospodin zástupů.‘
#2:14 Plesej a raduj se, sijónská dcero, neboť už přicházím a budu bydlet uprostřed tebe, je výrok Hospodinův.
#2:15 V onen den se mnoho pronárodů přidruží k Hospodinu. Stanou se mým lidem a já budu bydlet uprostřed tebe. ‚I poznáš, že mě k tobě poslal Hospodin zástupů.‘
#2:16 Pak se Hospodin ujme Judy jako svého dědičného podílu ve svaté zemi a znovu vyvolí Jeruzalém.
#2:17 Ztiš se před Hospodinem, veškeré tvorstvo, neboť se vypravil ze svého svatého příbytku.“ 
#3:1 Potom mi ukázal velekněze Jóšuu, jak stojí před Hospodinovým poslem, a po pravici mu stál satan, aby proti němu vnesl žalobu.
#3:2 Hospodin však satanovi řekl: „Hospodin ti dává důtku, satane, důtku ti dává Hospodin, který si vyvolil Jeruzalém. Což to není oharek vyrvaný z ohně?“
#3:3 Jóšua totiž, jak stál před poslem, byl oblečen do špinavého šatu.
#3:4 A Hospodin se obrátil k těm, kteří tu před ním stáli: „Svlékněte z něho ten špinavý šat.“ Jemu pak řekl: „Pohleď, sňal jsem z tebe tvou nepravost a dal jsem tě obléci do slavnostního roucha.
#3:5 Řekl jsem: Vstavte mu na hlavu čistý turban.“ Tu mu vstavili na hlavu čistý turban a oblékli mu šat. Hospodinův posel stál při tom.
#3:6 A Hospodinův posel Jóšuovi dosvědčil:
#3:7 „Toto praví Hospodin zástupů: Budeš-li chodit po mých cestách a budeš-li střežit, co jsem ti svěřil, budeš obhajovat můj dům a střežit má nádvoří a já ti dám právo přicházet mezi ty, kteří zde stojí. -
#3:8 Nuže slyš, veleknězi Jošuo, ty i tvoji druhové, kteří sedí před tebou: Tito muži jsou předzvěstí toho, že přivedu svého služebníka, zvaného Výhonek.
#3:9 Hle, tu je kámen, který kladu před Jóšuu: jeden kámen, na něm sedm očí. Já sám na něm vyryji znak, je výrok Hospodina zástupů, a odklidím nepravost této země v jediném dni.
#3:10 Onoho dne pak, je výrok Hospodina zástupů, pozvete jeden druhého pod vinnou révu a pod fíkovník.“ 
#4:1 Potom mě posel, který se mnou mluvil, znovu vzbudil jako toho, kdo musí být probuzen ze spánku,
#4:2 a zeptal se mě: „Co vidíš?“ Odvětil jsem: „Hle, vidím svícen, celý ze zlata, a na něm nahoře mísa se sedmi kahany. Všechny kahany na něm mají nahoře po sedmi hubičkách.
#4:3 A nad ním dvě olivy, jedna z pravé strany a druhá z levé strany mísy.“
#4:4 Nato jsem se zeptal posla, který se mnou mluvil: „Co to znamená, můj pane?“
#4:5 Posel, který se mnou mluvil, mi odpověděl. Řekl mi: „Ty nevíš, co to znamená?“ Řekl jsem: „Nevím, můj pane.“
#4:6 Nato mi řekl: „Toto je slovo Hospodinovo k Zerubábelovi: Ne mocí ani silou, nýbrž mým duchem, praví Hospodin zástupů.
#4:7 Čím jsi ty, veliká horo, před Zerubábelem? Rovinou. On vynese poslední kámen za hlučného provolávání: ‚Jeruzalém došel milosti!‘“
#4:8 I stalo se ke mně slovo Hospodinovo:
#4:9 „Ruce Zerubábelovy tento dům založily, jeho ruce jej také dokončí. ‚I poznáš, že mě k vám poslal Hospodin zástupů.‘
#4:10 A kdo pohrdal dnem malých začátků, radostně bude hledět na olovnici v ruce Zerubábelově.“ „Těch sedm, to jsou oči Hospodinovy, prohledávající celou zemi.“
#4:11 Nato jsem mu řekl: „Co znamenají ty dvě olivy po pravé i po levé straně svícnu?“
#4:12 A dále jsem se ho otázal: „Co znamenají ty dva trsy oliv, které jsou nad dvěma zlatými žlábky, jimiž vytéká zlato?“
#4:13 Tu mi řekl: „Ty nevíš, co to znamená?“ Řekl jsem: „Nevím, můj pane.“
#4:14 Odvětil: „To jsou ti dva synové nového oleje, kteří stojí před Pánem celé země.“ 
#5:1 Znovu jsem se rozhlédl a vidím, hle, letí svitek.
#5:2 Posel se mě otázal: „Co vidíš?“ Odvětil jsem: „Vidím letět svitek dvacet loket dlouhý a deset loket široký.“
#5:3 Řekl mi: „To je kletba, která vychází na celou zemi. Bude jí odstraněn každý zloděj, každý křivopřísežník.
#5:4 Vychází z mého rozhodnutí, je výrok Hospodina zástupů. Vejde do zlodějova domu i do domu toho, který v mém jménu křivě přísahá, a uhostí se uvnitř jeho domu a zničí jeho dřevo i kámen.“
#5:5 Potom vyšel posel, který se mnou mluvil, a vyzval mě: „Jen se rozhlédni a podívej se. Co se tu objevuje?“
#5:6 Otázal jsem se: „Co je to?“ Odvětil mi: „To se objevuje éfa“, a dodal: „Tak to bude s nimi vypadal v celé zemi.“
#5:7 A hle, pozvedlo se olověné víko, a tam uvnitř éfy sedí jakási ženština.
#5:8 Řekl mi: „To je Svévole“, mrštil jí do éfy a přiklopil otvor olověným poklopem.
#5:9 Vtom jsem se rozhlédl a vidím, hle, vycházejí dvě ženy. Vítr se jim opíral do křídel; měly totiž křídla jako čáp. Zvedly éfu mezi zemi a nebe.
#5:10 Otázal jsem se posla, který se mnou mluvil: „Kam chtějí tu éfu dopravit?“
#5:11 Odvětil mi: „Chtějí té ženštině vystavět v šineárské zemi dům, a až bude připraven, bude tam uložena na svůj podstavec.“ 
#6:1 Znovu jsem se rozhlédl a vidím, hle, mezi dvěma horami vyjíždějí čtyři vozy. Ty hory byly měděné.
#6:2 V prvém voze byly zapřaženi ryzáci, v druhém vraníci,
#6:3 v třetím bělouši a ve čtvrtém grošáci, statní koně.
#6:4 Obrátil jsem se na posla, který se mnou mluvil, a otázal jsem se ho: „Co to znamená, můj pane?“
#6:5 Posel odpověděl a řekl mi: „To vycházejí čtyři nebeští duchové, kteří stojí před Pánem celé země.“
#6:6 Vraníci vyjíždějí do severní země. Po nich vyrazili bělouši, zatímco grošáci vyjeli do země jižní.
#6:7 Ti statní koně tedy vyrazili ve snaze projet křížem krážem zemi, neboť posel řekl: „Projeďte zemi křížem krážem.“ I projížděli zemí.
#6:8 Pak na mne zvolal a promluvil ke mně: „Pohleď, ti, kteří vyjeli do severní země, uklidnili mého ducha, co se severní země týče.“
#6:9 I stalo se ke mně slovo Hospodinovo
#6:10 o převzetí darů od přesídlenců, od Cheldaje, Tóbijáše a Jedajáše: „Onoho dne přijdeš, vejdeš do domu Jóšijáše, syna Sefanjášova, kam oni přišli z Babylóna,
#6:11 a převezneš stříbro i zlato. Dáš zhotovit tiáru, vstavíš ji na hlavu velekněze Jóšuy, syna Jósadakova,
#6:12 a řekneš mu: ‚Toto praví Hospodin zástupů: Hle, přijde muž jménem Výhonek, vyraší odspodu; ten zbuduje Hospodinův chrám.
#6:13 Ano, on zbuduje chrám Hospodinův a bude obdařen velebností. Bude sedět na svém trůnu a vládnout a bude na svém trůnu knězem; mezi obojím bude pokojná shoda.‘
#6:14 Tiára pak bude v Hospodinově chrámě pamětním znamením Chelemovi, Tóbijášovi, Jedajášovi, a Chénovi, synu Sefanjášovu.
#6:15 I vzdálení budou přicházet a pomáhat při budování Hospodinova chrámu. ‚I poznáte, že mě k vám poslal Hospodin zástupů‘, ovšem, budete-li opravdu poslouchat Hospodina, svého Boha.“ 
#7:1 Čtvrtého roku vlády krále Dareia, čtvrtého dne devátého měsíce, totiž měsíce kislevu, stalo se slovo Hospodinovo k Zacharjášovi.
#7:2 Tehdy poslal Bét-el Sar-esera s Regem-melekem a jeho muži, aby si naklonil Hospodina
#7:3 a aby se při domě Hospodina zástupů otázali kněží a proroků: „Mám se nadále pátého měsíce oddávat pláči a odříkání, jak jsem to po léta činil?“
#7:4 I stalo se ke mně slovo Hospodina zástupů:
#7:5 „Řekni všemu lidu země i kněžím: Když jste se postili a naříkali pátého a sedmého měsíce, teď už po sedmdesát let, postili jste se opravdu kvůli mně?
#7:6 A když jíte a pijete, nejíte a nepijete si pro sebe?
#7:7 Neprovolával Hospodin právě toto skrze dřívější proroky, když si ještě Jeruzalém a města kolem něho bydlily v klidu a obydlen byl Negeb i Přímořská nížina?“
#7:8 I stalo se slovo Hospodinovo k Zacharjášovi:
#7:9 „Toto praví Hospodin zástupů: Vynášejte pravdivé rozsudky a ať každý prokazuje svému bratru milosrdenství a slitování.
#7:10 Neutiskujte vdovu a sirotka, bezdomovce a trpícího, nikdo ať ve svém srdci nezamýšlí proti bratru nic zlého.
#7:11 Ale oni tomu odmítli věnovat pozornost, svéhlavě se obrátili zády a zacpali si uši, aby neslyšeli.
#7:12 Srdce měli jako z křemene a neslyšeli zákon ani slova, která posílal Hospodin zástupů svým duchem skrze dřívější proroky. Proto je postihlo velké rozlícení Hospodina zástupů.
#7:13 Kdykoli volal, oni neslyšeli. Ať si volají, já je neslyším, praví Hospodin zástupů.
#7:14 Odvanu je vichrem mezi nejrůznější pronárody, které neznají, a země bude za nimi zpustošena, takže jí nikdo nebude procházet. Tak pustošili přežádoucí zemi.“ 
#8:1 I stalo se slovo Hospodina zástupů.
#8:2 „Toto praví Hospodin zástupů: Horlím pro Sijón velkou horlivostí, horlím pro něj velmi rozhořčeně.
#8:3 Toto praví Hospodin: Vrátím se na Sijón a budu bydlet uprostřed Jeruzaléma. Jeruzalém bude nazýván Město věrné a hora Hospodina zástupů Hora svatá.
#8:4 Toto praví Hospodin zástupů: Opět budou sedat na náměstích Jeruzaléma starci a stařeny, každý s holí v ruce pro vysoké stáří.
#8:5 Náměstí toho města budou plná chlapců a děvčat; ti si budou hrát na jeho náměstích.
#8:6 Toto praví Hospodin zástupů: Jestliže to bude v očích pozůstatku tohoto lidu v oněch dnech nemožné, bude to nemožné i v mých očích? je výrok Hospodina zástupů.
#8:7 Toto praví Hospodin zástupů: Hle, zachráním svůj lid ze země východní i ze země, kde slunce zapadá.
#8:8 Přivedu je a budou bydlet v Jeruzalémě. Budou mým lidem a já jim budu Bohem věrným a spravedlivým.
#8:9 Toto praví Hospodin zástupů: Jednejte rozhodně, vy, kdo v těchto dnech slyšíte z úst proroků tato slova o dni, kdy byl položen základ domu Hospodina zástupů, aby byl zbudován chrám.
#8:10 Před těmito dny nebylo totiž co dát za lidskou práci, natož za práci dobytka. Kdo vycházel a vcházel, neměl pokoj od protivníka, dopustil jsem, že stáli všichni proti všem.
#8:11 Ale nyní nebudu jednat s pozůstatkem tohoto lidu jako za dřívějších dnů, je výrok Hospodina zástupů.
#8:12 Vzejde setba pokoje: vinná réva vydá své plody, země svou úrodu a nebesa svou rosu. To všechno dám jako dědictví pozůstatku tohoto lidu.
#8:13 Jako jste byli zlořečením mezi pronárody, dome judský a dome izraelský, tak budete požehnáním, až vás zachráním. Nebojte se! Jednejte rozhodně.
#8:14 Toto praví Hospodin zástupů: Jako jsem pojal úmysl naložit s vámi zle, když mě vaši otcové rozlítili, praví Hospodin zástupů, a nelitoval jsem toho,
#8:15 teď naopak mám v úmyslu obdařit v těchto dnech Jeruzalém a judský dům dobrem. Nebojte se!
#8:16 Konejte toto: Mluvte každý svému bližnímu pravdu. Vynášejte ve svých branách pravdu a pokojný soud.
#8:17 Nezamýšlejte nikdo ve svém srdci proti bližnímu nic zlého. „Nelibujte si v křivé přísaze. To všechno nenávidím, je výrok Hospodinův.“
#8:18 I stalo se ke mně slovo Hospodina zástupů.
#8:19 „Toto praví Hospodin zástupů: Půst čtvrtého, půst pátého, půst sedmého a půst desátého měsíce se judskému domu obrátí v radostné veselí a utěšené slavnosti. Jen milujte pravdu a pokoj!
#8:20 Toto praví Hospodin zástupů: Znovu budou přicházet národy a obyvatelé mnoha měst.
#8:21 Obyvatelé jednoho města půjdou do druhého a řeknou: ‚Pojďme si naklonit Hospodina, pojďme vyhledat Hospodina zástupů. Já půjdu také.‘
#8:22 A budou přicházet četné národy i mocné pronárody, aby hledaly v Jeruzalémě Hospodina zástupů a naklonily si ho.
#8:23 Toto praví Hospodin zástupů: V oněch dnech se chopí deset mužů z pronárodů všech jazyků pevně cípu roucha jednoho Judejce a řeknou: ‚Půjdeme s vámi. Slyšeli jsme, že s vámi je Bůh.‘“ 
#9:1 Výnos. Slovo Hospodinovo proniklo do země Chadráku, Damašek je místem, na němž spočine. K Hospodinu budou vzhlížet všichni lidé, všechny kmeny Izraele.
#9:2 Proniklo i do Chamátu, do jeho pomezí, do Týru i Sidónu, tak přemoudrého.
#9:3 Týr si vybudoval opevnění, nahromadil stříbra jako prachu, zlata ryzího jak bláta na ulicích.
#9:4 Avšak Panovník si jej podrobí, jeho val srazí do moře; bude pozřen ohněm.
#9:5 Spatří to Aškalón a bude se bát, Gáza se bude svíjet v hrozné křeči, Ekrón se zklame v tom, več doufal. Z Gázy zmizí král, Aškalón nebude už osídlen,
#9:6 v Ašdódu se usadí míšenec. Tak vyhladím pýchu Pelištejců.
#9:7 Krvavé sousto mu vyrvu z úst, ohyzdné oběti z jeho zubů. Tak bude i on zachován našemu Bohu, stane se v Judsku pohlavárem a Ekrón bude jako Jebúsejec.
#9:8 „Položím se táborem u svého domu proti vojsku, proti každému, kdo by sem chtěl vtrhnout; zotročovatel se už nebude přes ně valit, nespustím je nyní z očí.“ -
#9:9 Rozjásej se, sijónská dcero, dcero jeruzalémská, propukni v hlahol! Hle, přichází k tobě tvůj král, spravedlivý a zachráněný, pokořený, jede na oslu, na oslátku, osličím mláděti.
#9:10 „Vymýtím vozy z Efrajima a z Jeruzaléma koně; válečný luk bude vymýcen.“ Vyhlásí pronárodům pokoj; jeho vláda bude od moře k moři od Řeky až do dálav země.
#9:11 „Pro krev smlouvy s tebou propustím tvé vězně z cisterny, v níž není vody.
#9:12 Navraťte se do pevnosti, vězňové, jimž naděje vzchází. Dnes to oznamuji znovu: Všechno ti nahradím dvojnásobně.
#9:13 Judu si napnu jako lučiště, na tětivu vložím Efrajima. Zburcují tvé syny, Sijóne, proti synům tvým, Jávane. Učiním tě mečem bohatýra.“
#9:14 Ukáže se nad nimi Hospodin, jako blesk vyletí jeho střela; Panovník Hospodin zaduje na polnici, vydá se na pochod ve vichřicích z jihu.
#9:15 Hospodin zástupů jim bude štítem a oni budou jíst a kameny do praku pošlapou, budou pít a halasit jako při víně a budou plní jak obětní miska, skropení jak rohy na oltáři.
#9:16 Tak je Hospodin, jejich Bůh, v onen den zachrání, stádečko svého lidu. Jako drahokamy v čelence budou zářit nad jeho zemí.
#9:17 Jaká je jeho dobrota, jaká je jeho krása! Působí, že mládenci prospívají jako obilí a panny jako mošt. 
#10:1 Vyproste si od Hospodina déšť v čas jarních dešťů. Hospodin tvoří bouřková mračna, hojnými dešti je naplňuje, bylinu na poli dopřeje každému.
#10:2 Bůžkové mluvili ničemnosti, co viděli věštci, byl klam; rozhlašují šalebné sny, přeludem utěšují. Proto táhl lid jako stádo, ujařmený, bez pastýře. -
#10:3 Můj hněv plane proti pastýřům, zakročím proti kozlům. Hospodin zástupů navštívil své stádo, judský dům; vystrojí je nádherně jako svého válečného oře.
#10:4 Od něho je cimbuří, od něho stanový kolík, od něho válečné lučiště, od něho pochází i každý zotročovatel.
#10:5 Budou jako bohatýři, rozšlapou všechno do bláta ulic v té bitvě. Budou bojovat a Hospodin bude s nimi, jezdci na koních budou zahanbeni.
#10:6 Obdařím judský dům bohatýrskou silou a dům Josefův zachráním. Přivedu je zpět a usadím je do bezpečí, protože jsem se nad nimi slitoval. Budou zas, jako bych na ně nebyl zanevřel, já jsem Hospodin, jejich Bůh, já jim odpovím.
#10:7 Efrajimci budou jako bohatýři, jejich srdce se bude radovat jako po víně; až to spatří jejich synové, též se zaradují, jejich srdce bude jásat kvůli Hospodinu.
#10:8 Hvízdnu na ně a shromáždím je, protože jsem je vykoupil, bude jich zase mnoho, jako jich bylo kdysi.
#10:9 Rozseji je mezi národy, budou se na mě rozpomínat v dalekých krajích, zůstanou naživu i se svými syny a opět se navrátí.
#10:10 Přivedu je zpět z egyptské země, shromáždím je z Asýrie a uvedu je do země gileádské a liabanónské, ani jim nebude stačit místo.
#10:11 Projde s nimi mořem soužení, bude však bít do mořských vln, až vyschnou všechny hlubiny Veletoku. Svržena bude pýcha Asýrie, Egyptu bude odňato žezlo.
#10:12 Obdařím je bohatýrskou silou od Hospodina, budou chodit v jeho jménu, je výrok Hospodinův. 
#11:1 Otevři svá vrata, Libanóne, ať tvé cedry pozře oheň.
#11:2 Kvílej, cypřiši, že padl cedr, že vznešené stromy propadly záhubě. Kvílejte, bášanské duby, že padl neprostupný hvozd.
#11:3 Slyš, jak pastýři kvílejí, že jejich vznešené nivy propadly záhubě. Slyš, jak lvíčata řvou, že pýcha Jordánu propadla záhubě.
#11:4 Toto praví Hospodin, můj Bůh: „Pas ovce určené k zabití.
#11:5 Ti, kdo je kupují, zabíjejí je a za vinu nepykají; kdo je prodávají, říkají: ‚Požehnán buď Hospodin, to jsem se obohatil‘; a žádný z jejich pastýřů nemá s nimi soucit.
#11:6 Ani já už nebudu mít soucit s obyvateli země, je výrok Hospodinův. Hle, dopustím, aby člověk padl do rukou člověka, do rukou svého krále. Na kusy rozbijí zemi a já ji z jejich rukou nevytrhnu.“
#11:7 Pásl jsem tedy ovce určené k zabití, totiž ty nejutištěnější ze stáda. Vzal jsem si dvě hole; jednu jsem nazval „Vlídnost“ a druhou jsem nazval „Pouto“ a pásl jsem ovce.
#11:8 V jediném měsíci jsem vyhnal tři pastýře, neboť jsem s nimi ztratil trpělivost a také oni o mě nestáli.
#11:9 Řekl jsem: „Nebudu vás pást. Co má umřít, ať umře; co má být zahnáno, ať je zahnáno; a které zůstanou, ať se požerou navzájem.“
#11:10 Vzal jsem svou hůl Vlídnost a zlomil jsem ji na znamení, že ruším svou smlouvu, kterou jsem uzavřel se všemi národy.
#11:11 I byla onoho dne zrušena a ty nejutištěnější ze stáda, které se mě držely, poznaly, že je to slovo Hospodinovo.
#11:12 Řekl jsem tedy: „Pokládáte-li to za dobré, vyplaťte mi mzdu; ne-li, nechte být.“ Tu mi odvážili jako mzdu třicet šekelů stříbra.
#11:13 Nato mi Hospodin řekl: „Hoď to tavičovi, tu nádhernou cenu, jíž mě ocenili.“ I vzal jsem těch třicet šekelů stříbra a hodil jsem je v Hospodinově domě tavičovi.
#11:14 Pak jsem zlomil druhou hůl, Pouto, na znamení, že ruším bratrství mezi Judou a Izraelem.
#11:15 Ale Hospodin mi řekl: „Znovu si vezmi výstroj pošetilého pastýře,
#11:16 neboť dám povstat v zemi pastýři, jenž zahnané nevyhledá, nebude pátrat po ztraceném, polámanou nebude léčit, o vyčerpanou se nepostará. Bude se krmit masem těch nejtučnějších, strhne jim paznehty.
#11:17 Běda pastýři ničemnému, který opouští stádo! Meč proti jeho paži a proti jeho pravému oku! Paže ať mu nadobro uschne a pravé oko úplně vyhasne.“ 
#12:1 Výnos. Slovo Hospodinovo o Izraeli, výrok Hospodina, který rozprostřel nebesa a položil základy země i vytvořil ducha člověkova v jeho nitru.
#12:2 „Hle, postavil jsem Jeruzalém před všechny okolní národy jako číši, po níž se zapotácejí. Při obléhání Jeruzaléma dojde i na Judu.
#12:3 V onen den učiním Jeruzalém vzpěračským balvanem pro všechny národy. Všichni, kteří jej budou chtít vzepřít, zraní se až do krve. Všechny pronárody země se proti němu srotí.
#12:4 V onen den,je výrok Hospodinův, udeřím každý a kůň se splaší a jezdec na něm zešílí. Na dům judský však obrátím svůj zrak, kdežto všechny koně národů raním slepotou.
#12:5 Tu si každý judský pohlavár v srdci řekne: ‚Obyvatelé Jeruzaléma jsou mi posilou pro Hospodina zástupů, svého Boha.‘
#12:6 V onen den učiním judské pohlaváry pánví s ohněm ve dříví a hořící pochodní v otepi slámy. Pozřou všechny okolní národy napravo i nalevo. Jeruzalém však dále zůstane na svém místě, v Jeruzalémě.“
#12:7 Hospodin nejprve zachrání judské stany, aby čest domu Davidova, čest toho, jenž sídlí v Jeruzalémě, nebyla větší než Judova.
#12:8 V onen den bude Hospodin štítem tomu, kdo sídlí v Jeruzalémě; klopýtající mezi nimi bude v onen den jako David a dům Davidův bude jako Bůh, bude před nimi jako anděl Hospodinův.
#12:9 „V onen den vyhledám všechny pronárody, které přitáhly na Jeruzalém, a zahladím je.
#12:10 Ale na dům Davidův, na toho, jenž sídlí v Jeruzalémě, vyleji ducha milosti a proseb o smilování. Budou vzhlížet ke mně, kterého probodli. Budou nad ním naříkat, jako se naříká nad smrtí jednorozeného, budou nad ním hořce lkát, jako se hořce lká nad prvorozeným.
#12:11 V onen den se bude v Jeruzalémě rozléhat nářek jako nářek pro Hadad-rimóna na pláni u Megida.
#12:12 Naříkat bude země, každá čeleď zvlášť: zvlášť čeleď domu Davidova a jejich ženy zvlášť, zvlášť čeleď domu Nátanova a jejich ženy zvlášť,
#12:13 zvlášť čeleď domu Léviova a jejich ženy zvlášť, zvlášť čeleď šimeíská a jejich ženy zvlášť,
#12:14 všechny ostatní čeledi, každá čeleď zvlášť a jejich ženy zvlášť.“ 
#13:1 V onen den vytryskne pro dům Davidův i pro obyvatele Jeruzaléma pramen k obmytí hříchu a nečistoty.
#13:2 „V onen den, je výrok Hospodina zástupů, vyhladím ze země jména modlářských stvůr, takže už nikdy nebudou připomínána. Vypudím ze země rovněž proroky a ducha nečistoty.“
#13:3 Bude-li pak ještě někdo porokovat, řeknou mu vlastní rodiče, otec i matka: „Nezůstaneš naživu, protože jsi ve jménu Hospodinově klamal.“ A jeho vlastní rodiče, otec i matka, ho probodnou, že prorokoval.
#13:4 V onen den se stane, že proroci budou zahanbeni, každý pro své prorocké vidění. Už neobléknou chlupatý plášť a nebudou obelhávat.
#13:5 Budou říkat: „Nejsem prorok, jsem zemědělský nevolník, koupený hned v mládí.“
#13:6 Řekne-li mu kdo: „Co to máš na prsou za rány?“, odpoví: „Tak jsem byl zbit v domě těch, kdo mě milovali.“
#13:7 „Probuď se, meči, proti mému pastýři, proti reku, mému společníku, je výrok Hospodina zástupů. Bij pastýře a ovce se rozprchnou! K maličkým však obrátím svou ruku.
#13:8 Z celé země, je výrok Hospodinův, budou dvě třetiny vyhlazeny, zajdou, jen třetina v ní zbude.
#13:9 Tu třetinu však provedu ohněm. Přetavím je, jako se taví stříbro, přezkouším je, jako se zkouší zlato. Ti budou vzývat mé jméno a já jim odpovím. Řeknu: ‚Toto je můj lid.‘ A oni řeknou: ‚Hospodin je můj Bůh.‘“ 
#14:1 Hle, přichází den Hospodinův. Budou se mezi vámi o tebe dělit jako o kořist.
#14:2 Shromáždím k Jeruzalému do boje všechny pronárody. Město bude dobyto, domy vypleněny, ženy zhanobeny. Polovina města bude vysídlena, zbytek lidu však nebude z města vyhlazen.
#14:3 Hospodin vytáhne a bude bojovat proti oněm pronárodům, jako bojoval kdysi v den bitvy.
#14:4 V onen den stanou jeho nohy na hoře Olivové, ležící na východ od Jeruzaléma. Olivová hora se rozpoltí vpůli od východu na západ velmi širokým údolím. Polovina hory uhne k severu a polovina k jihu.
#14:5 Vy pak budete utíkat údolím mezi těmito mými horami, neboť to horské údolí bude sahat až k Asalu. Budete utíkat, jako jste utíkali před zemětřesením za dnů judského krále Uzijáše. Pak přijde Hospodin, můj Bůh, a s tebou všichni svatí, Bože.
#14:6 V onen den nebude světlo, vše vzácné ztuhne mrazem.
#14:7 A Nastane den jediný, známý jen Hospodinu, kdy nebude ani den noc; i za večerního času bude světlo.
#14:8 V onen den poplynou z Jeruzaléma živé vody, polovina k moři východnímu, polovina k západnímu, v létě jako v zimě.
#14:9 Hospodin bude Králem nad celou zemí; v onen den bude Hospodin jediný a jeho jméno jediné.
#14:10 Celá země se promění jakoby v rovinu, od Geby až k Rimónu jižně od Jeruzaléma; ten bude vysoko a v klidu na svém místě, od brány Benjamínské až k místu první brány, totiž k bráně Nárožní, od věže Chananeelu až ke Královským lisům.
#14:11 Budou tu sídlit a nebude už klatby. Jeruzalém bude bydlet v bezpečí.
#14:12 A toto bude porážka, kterou Hospodin připraví všem národům, jež vytáhly do boje proti Jeruzalému: tělo jim shnije ještě vstoje, oči jim shnijí v důlcích a jazyk jim shnije v ústech.
#14:13 V onen den je Hospodin uvede do zmatku tak velikého, že jeden bude chytat druhého za ruku, ale ta ruka se pozvedne proti ruce přítelově.
#14:14 Také Juda bude bojovat proti Jeruzalému. Nashromáždí se bohatství všech okolních pronárodů, zlato, stříbro a nesmírné množství oděvů.
#14:15 Stejná porážka postihne koně, mezky, velbloudy i osly a všechen dobytek, který bude v jejich leženích.
#14:16 Všichni pak, kdo zbyli ze všech pronárodů, které vytáhly proti Jeruzalému, budou každoročně putovat, aby se klaněli Králi, Hospodinu zástupů, a slavili slavnost stánků.
#14:17 Jestliže některá z čeledí země nebude putovat do Jeruzaléma a klanět se Králi, Hospodinu zástupů, nedostaví se jim déšť.
#14:18 Jestliže nebude putovat a nepřijde egyptská čeleď, ani jí se nedostaví déšť. Utrpí porážku, kterou připraví Hospodin pronárodům, jež nebudou putovat, aby slavily slavnost stánků.
#14:19 Prohřeší se Egypt a prohřeší se každý jiný pronárod, který nebude putovat, aby slavil slavnost stánků.
#14:20 V onen den bude na zvoncích koní nápis: „Svatý Hospodinu“, a hrnce v domě Hospodinově budou jako obětní misky před oltářem.
#14:21 I každý hrnec v Jeruzalémě a v Judsku bude zasvěcen Hospodinu zástupů. Všichni obětníci budu přicházet a brát si z nich a v nich vařit. A nebude už kramáře v domě Hospodina zástupů v onon den.  

\book{Malachi}{Mal}
#1:1 Výnos. Slovo Hospodinovo o Izraeli skrze Malachiáše.
#1:2 „Zamiloval jsem si vás, praví Hospodin. Vy však se ptáte: ‚Kde je důkaz, že náš miluješ?‘ Což nebyl Ezau Jákobův bratr? je výrok Hospodinův. Jákoba jsem si zamiloval,
#1:3 Ezaua však nenávidím. Proto jsem obrátil v poušť jeho hory, jeho dědictví v step pro šakaly.“
#1:4 Říká-li Edóm: „Jsme zničeni, avšak na troskách budeme znovu stavět“, praví Hospodin zástupů toto: „Jen ať si stavějí, já budu bořit.“ I nazvou je územím zvůle, lidem, na který Hospodin navěky zanevřel hněvu.
#1:5 Na vlastní oči to spatříte a řeknete sami: „Hospodin je veliký i za pomezím Izraele.“
#1:6 Syn ctí svého otce a služebník svého pána. „Jsem-li Otec, kde je úcta ke mně? Jsem-li Pán, kde je bázeň přede mnou, praví Hospodin zástupů vám, kněží, kteří zlehčujete mé jméno. Ptáte se: ‚Čím zlehčujeme tvé jméno?‘
#1:7 Přinášíte na můj oltář poskvrněný chléb a ptáte se: ‚Čím jsme tě poskvrnili?‘ Tím, že říkáte, že Hospodinův stůl není třeba brát vážně.
#1:8 Když přivádíte k oběti slepé zvíře, to není nic zlého? Když přivádíte kulhavé a nemocné, to není nic zlého? Jen to dones svému místodržiteli, získáš-li tak jeho přízeň a přijme-li tě, praví Hospodin zástupů.
#1:9 Tak tedy proste Boha o shovívavost, aby se nad námi smiloval. Vlastní rukou jste to činívali. Cožpak mohu někoho z vás přijmout? praví Hospodin zástupů.
#1:10 Kéž by se mezi vámi našel někdo, kdo by zavřel dveře, abyste marně nezapalovali na mém oltáři oheň! Nemám ve vás zalíbení, praví Hospodin zástupů, a dary z vaší ruky jsem si neoblíbil.
#1:11 Od východu slunce až na západ bude mé jméno veliké mezi pronárody. Na každém místě budou přinášet mému jménu kadidlo a čistý obětní dar. Zajisté bude mé jméno veliké mezi pronárody, praví Hospodin zástupů.
#1:12 Ale vy je znesvěcujete, když říkáte, že Panovníkův stůl je možno poskvrňovat, a to, co je na něm, obětovaný pokrm, není třeba brát vážně.
#1:13 Říkáte: ‚Ach, nač se trápit?‘ a vše odbýváte, praví Hospodin zástupů. Přinášíte, co bylo vzato násilím, kulhavé i nemocné přinášíte jako obětní dar. Což si mohu oblíbit, co je z vaší ruky? praví Hospodin.
#1:14 Buď proklet chytrák, jenž má ve svém stádu samce, a složí-li slib, obětuje Panovníku zvíře vykleštěné. Neboť já jsem velkký Král, praví Hospodin zástupů, mé jméno budí mezi pronárody bázeň.“ 
#2:1 Nuže, kněží, je o vás rozhodnuto takto:
#2:2 „Jestliže neposlechnete a nevezmete si k srdci, že máte oslavovat mé jméno, praví Hospodin zástupů, stihnu vás kletbou a vaše žehnání prokleji. A již jsem je proklel, neboť si to k srdci neberete.
#2:3 Hle, obořím se na vaše potomstvo, vmetu vám do tváře výměty, výměty z vašich svátečních obětí. Vynesou vás i s nimi.
#2:4 Věděli jste to, vždyť jsem vám poslal toto rozhodnutí: Chci mít smlouvu s Lévim, praví Hospodin zástupů.
#2:5 Má smlouva s ním byla život a pokoj. Dal jsem mu toto: bázeň, aby žil v mé bázni a děsil se mého jména.
#2:6 Zákon pravdy byl v jeho ústech, bezpráví se na jeho rtech neobjevilo; chodil se mnou pokojně a bezelstně a mnohé odvrátil od nepravosti.
#2:7 Vždyť rty kněze střeží poznání; z jeho úst se vyhledává zákon, neboť on je poslem Hospodina zástupů.
#2:8 Vy jste však z této cesty sešli, mnohé jste přivedli k tomu, že klopýtli o zákon, porušili jste lévijskou smlouvu, praví Hospodin zástupů.
#2:9 Proto i já jsem způsobil, že jste v nevážnosti a ponížení u všeho lidu, za to, že nestřežíte mé cesty a při výkladu zákona straníte osobám.“
#2:10 Což nemáme my všichni jednoho Otce? Což nás nestvořil jediný Bůh? Proč jsme vůči sobě věrolomní a znesvěcujeme tak smlouvu svých otců?
#2:11 Věrolomně jedná Juda, ohavnost se páchá v Izraeli i v Jeruzalémě. Juda znesvěcuje svatyni, kterou Hospodin miluje, oženil se s dcerou cizího božstva.
#2:12 Každého, kdo se toho dopouští, vyhladí Hospodin z Jákobových stanů, bdícího i odpovídajícího, i toho, kdo přináší obětní dar Hospodinu zástupů.
#2:13 A ještě se dopouštíte další věci: Hospodinův oltář smáčíte slzami, pláčete a sténáte, protože již nehledí na obětní dar a nemá zalíbení v tom, co přinášíte.
#2:14 Ptáte se proč? Proto, že Hospodin je svědkem mezi tebou a ženou tvého mládí, vůči níž ses zachoval věrolomně, ačkoli je to tvá družka a žena podle smlouvy.
#2:15 Což on neučinil člověka jednoho a nedal mu částku ducha? Oč má ten jeden usilovat? O Boží potomstvo. Střezte svého ducha, nikdo ať se nezachová věrolomně k ženě svého mládí.
#2:16 „Každý ať nenávidí rozvod, praví Hospodin, Bůh Izraele, ať na svém oděvu přikryje násilí, praví Hospodin zástupů.“ Střezte tedy svého ducha, nejednejte věrolomně!
#2:17 Unavujete Hospodina svými slovy. Ptáte se: „Čím ho unavujeme?“ Tím, že říkáte: „Každý, kdo se dopouští zlého, líbí se Hospodinu, takovým on přeje.“ Anebo: „Kde je Bůh zastávající právo?“ 
#3:1 „Hle, posílám svého posla, aby připravil přede mnou cestu. I vstoupí nenadále do svého chrámu Pán, kterého hledáte, posel smlouvy, po němž toužíte. Opravdu přijde, praví Hospodin zástupů.
#3:2 Kdo však snese den jeho příchodu? Kdo obstojí, až se on ukáže? Bude jako oheň taviče, jako louh těch, kdo bělí plátno.
#3:3 Tavič usedne a pročistí stříbro, pročistí syny Léviho a přetaví je jako zlato a stříbro. I budou patřit Hospodinu a spravedlivě přinášet obětní dary.
#3:4 Obětní dary Judy a Jeruzaléma budou pak Hospodinu vítány jako za dávných dnů, jako v dřívějších letech.
#3:5 Předtím vás však přijdu soudit, rychle usvědčím cizoložníky a čaroděje, křivopřísežníky, utiskovatele námezdníků, vdov a sirotků, ty, kdo odpírají právo bezdomovci a mě se nebojí, praví Hospodin zástupů.“
#3:6 „Já Hospodin jsem se nezměnil, ani vy jste nepřestali být syny Jákobovými.
#3:7 Už za dnů svých otců jste se odchýlili od mých nařízení a nedbali jste na ně. Navraťte se ke mně a já se navrátím k vám, praví Hospodin zástupů. Ptáte se: ‚Jak se máme vrátit?‘
#3:8 Smí člověk okrádat Boha? Vy mě okrádáte. Ptáte se: ‚Jak tě okrádáme?‘ Na desátcích a na obětech pozdvihování.
#3:9 Jste stiženi kletbou proto, že mě okrádáte, celý ten pronárod!
#3:10 Přinášejte do mých skladů úplné desátky. Až bude ta potrava v mém domě, pak to se mnou zkuste, praví Hospodin zástupů: Neotevřu vám snad nebeské průduchy a nevyleji na vás požehnání? A bude po nedostatku.
#3:11 Kvůli vám se obořím na škůdce, aby vám nekazil plodiny země, abyste na poli neměli neplodnou vinnou révu, praví Hospodin zástupů.
#3:12 Všechny národy vám budou blahořečit a stanete se vytouženou zemí, praví Hospodin zástupů.“
#3:13 „Příliš smělá jsou vaše slova proti mně, praví Hospodin. Ptáte se: ‚Co mluvíme proti tobě?‘
#3:14 Říkáte: ‚Sloužit Bohu není k ničemu. Co z toho, že jsme před ním drželi stráž a že jsme chodili před Hospodinem zástupů zachmuřeně?
#3:15 Proto za šťastné pokládáme opovážlivce. Mají úspěch, ač se dopouštějí svévolností, pokoušejí Boha, a přece uniknou.‘“
#3:16 Tehdy ti, kteří se bojí Hospodina, o tom rozmlouvali; Hospodin to pozoroval a slyšel. A byla před ním sepsána pamětní kniha se jmény těch, kteří se bojí Hospodina a mají na mysli jeho jméno.
#3:17 „Ti budou, praví Hospodin zástupů, v den, který připravuji, mým zvláštním vlastnictvím, budu k nim shovívavý, jako bývá shovívavý otec k synu, jenž mu slouží.“
#3:18 Potom uvidíte rozdíl mezi spravedlivým a svévolníkem, mezi tím, kdo Bohu slouží, a tím, kdo mu sloužit nechce.
#3:19 „Hle, přichází ten den hořící jako pec; a všichni opovážlivci i všichni, kdo páchají svévolnosti, se stanou strništěm. A ten přicházející den je sežehne, praví Hospodin zástupů; nezůstane po nich kořen ani větev.
#3:20 Ale vám, kdo se bojíte mého jména, vzejde slunce spravedlnosti se zdravím na paprscích. Rozběhnete se a budete poskakovat jako vykrmení býčci,
#3:21 rozšlapete svévolníky, že budou jako popel pod chodidly vašich nohou v ten den, který připravuji, praví Hospodin zástupů.
#3:22 Pamatujte na zákon mého služebníka Mojžíše, jemuž jsem vydal na Chorébu pro celého Izraele nařízení a práva.
#3:23 Hle, posílám k vám proroka Elijáše, dříve než přijde den Hospodinův veliký a hrozný.
#3:24 On obrátí srdce otců k synům a srdce synů k otcům, abych při svém příchodu nestihl zemi klatbou.“   

\book{Matthew}{Matt}
#1:1 Listina rodu Ježíše Krista, syna Davidova, syna Abrahamova.
#1:2 Abraham měl syna Izáka, Izák Jákoba, Jákob Judu a jeho bratry,
#1:3 Juda Farese a Záru z Támary, Fares měl syna Chesróma, Chesróm Arama.
#1:4 Aram měl syna Amínadaba, Amínadab Naasona, Naason Salmóna,
#1:5 Salmón měl syna Boaze z Rachaby, Boaz Obéda z Rút, Obéd Isaje
#1:6 a Isaj Davida krále. David měl syna Šalomouna z ženy Uriášovy,
#1:7 Šalomoun Roboáma, Roboám Abiu, Abia Asafa,
#1:8 Asaf Jóšafata, Jóšafat Jórama, Jóram Uziáše.
#1:9 Uziáš měl syna Jótama, Jótam Achaza, Achaz Ezechiáše,
#1:10 Ezechiáš Manase, Manase měl syna Amose Amos Joziáše,
#1:11 Joziáš Jechoniáše a jeho bratry za babylónského zajetí.
#1:12 Po babylónském zajetí Jechoniáš měl syna Salatiela, Salatiel Zorobabela,
#1:13 Zorobabel Abiuda, Abiud Eljakima, Eljakim Azóra
#1:14 Azór Sádoka, Sádok Achima. Achim měl syna Eliuda,
#1:15 Eliud Eleazara, Eleazar Mattana, Mattan Jákoba,
#1:16 Jákob pak měl syna Josefa, muže Marie, z níž se narodil Ježíš řečený Kristus.
#1:17 Všech pokolení od Abrahama do Davida bylo tedy čtrnáct, od Davida po babylónské zajetí čtrnáct a od babylónského zajetí až po Krista čtrnáct.
#1:18 Narození Ježíšovo událo se takto: Jeho matka Maria byla zasnoubena Josefovi, ale dříve, než se sešli, shledalo se, že počala z Ducha svatého.
#1:19 Její muž Josef byl spravedlivý a nechtěl ji vystavit hanbě; proto se rozhodl propustit ji potají.
#1:20 Ale když pojal ten úmysl, hle, anděl Páně se mu zjevil ve snu a řekl: „Josefe, syny Davidův, neboj se přijmout Marii, svou manželku; neboť co v ní bylo počato, je z Ducha svatého.
#1:21 Porodí syna a dáš mu jméno Ježíš; neboť on vysvobodí svůj lid z jeho hříchů.“
#1:22 To všechno se stalo, aby se splnilo, co řekl Hospodin ústy proroka:
#1:23 „Hle, panna počne a porodí syna a dají mu jméno Immanuel“, to jest přeloženo „Bůh s námi“.
#1:24 Když se Josef probudil ze spánku, učinil, jak mu přikázal anděl Hospodinův, a přijal svou manželku k sobě.
#1:25 Ale nežili spolu, dokud neporodila syna, a dal mu jméno Ježíš. 
#2:1 Když se narodil Ježíš v Judském Betlémě za dnů krále Heroda, hle, mudrci od východu se objevili v Jeruzalémě a ptali se:
#2:2 „Kde je ten právě narozený král Židů? Viděli jsme na východě jeho hvězdu a přišli jsme se mu poklonit.“
#2:3 Když to uslyšel Herodes, znepokojil se a s ním celý Jeruzalém;
#2:4 svolal proto všechny velekněze a zákoníky lidu a vyptával se jich, kde se má Mesiáš narodit.
#2:5 Oni mu odpověděli: „V judském Betlémě; neboť tak je psáno u proroka:
#2:6 „A ty Betléme v zemi judské, zdaleka nejsi nejmenší mezi knížaty judskými, neboť z tebe vyjde vévoda, který bude pastýřem mého lidu, Izraele.“
#2:7 Tedy Herodes tajně povolal mudrce a podrobně se jich vyptal na čas, kdy se hvězda ukázala. Potom je poslal do Betléma a řekl:
#2:8 „Jděte a pátrejte důkladně po tom dítěti; a jakmile je naleznete, oznamte mi, abych se mu i já šel poklonit.“
#2:9 Oni krále vyslechli a dali se na cestu. A hle,hvězda, kterou viděli na východě, šla před nimi, až se zastavila nad místem, kde bylo to dítě.
#2:10 Když spatřili hvězdu, zaradovali se velikou radostí.
#2:11 Vešli do domu a uviděli dítě s Marií, jeho matkou; padli na zem, klaněli se mu a obětovali mu přinesené dary - zlato, kadidlo a myrhu.
#2:12 Potom, na pokyn ve snu, aby se nevraceli k Herodovi, jinudy odcestovali do své země.
#2:13 Když odešli, hle, anděl Hospodinův se ukázal Josefovi ve snu a řekl: „Vstaň, vezmi dítě i jeho matku, uprchni do Egypta a buď tam, dokud ti neřeknu; neboť Herodes bude hledat dítě, aby je zahubil.“
#2:14 On tedy vstal, vzal v noci dítě i jeho matku, odešel do Egypta
#2:15 a byl tam až do smrti Herodovy. Tak se splnilo, co řekl Pán ústy proroka: ‚Z Egypta jsem povolal svého syna.‘
#2:16 Když Herodes poznal, že ho mudrci oklamali, rozlítil se a dal povraždit všechny chlapce v Betlémě a v celém okolí ve stáří do dvou let, podle času, který vyzvěděl od mudrců.
#2:17 Tehdy se splnilo, co je řečeno ústy proroka Jeremiáše:
#2:18 ‚Hlas v Ráma je slyšet, pláč a veliký nářek; Ráchel oplakává své děti a nedá se utešit, protože jich není.‘
#2:19 Ale když Herodes umřel, hle, anděl Hospodinův se ukázal Josefovi v Egyptě
#2:20 a řekl: „Vstaň, vezmi dítě i jeho matku a jdi do země izraelské; neboť již zemřeli ti, kteří ukládali dítěti o život.“
#2:21 On tedy vstal, vzal dítě i jeho matku a vrátil se do izraelské země.
#2:22 Když však uslyšel, že Archelaos kraluje v Judsku po svém otci Herodovi, bál se tam jít; ale na pokyn ve snu se obrátil do končin galilejských
#2:23 a usadil se v městě zvaném Nazaret - aby se splnilo, co je řečeno ústy proroků, že bude nazván Nazaretský. 
#3:1 Za těch dnů vystoupil Jan Křtitel a kázal v Judské poušti:
#3:2 „Čiňte pokání, neboť se přiblížilo království nebeské.“
#3:3 To je ten, o němž je řečeno ústy proroka Izaiáše: ‚Hlas volajícího na poušti: Připravte cestu Páně, vyrovnejte mu stezky!‘
#3:4 Jan měl na sobě šat z velbloudí srsti, kožený pás kolem boků a potravou mu byly kobylky a med divokých včel.
#3:5 Tehdy vycházel k němu celý Jeruzalém i Judsko a celé okolí Jordánu,
#3:6 vyznávali své hříchy a dávali se od něho v řece Jordánu křtít.
#3:7 Ale když spatřil, že mnoho farizeů a saduceů přichází ke křtu, řekl jim: „Plemeno zmijí, kdo vám ukázal, že můžete utéci před nadcházejícím hněvem?
#3:8 Neste tedy ovoce, které ukazuje, že činíte pokání.
#3:9 Nemyslete si, že můžete říkat: ‚Náš otec je Abraham!‘ Pravím vám, že Bůh může Abrahamovi stvořit děti z tohoto kamení.
#3:10 Sekera je už na kořeni stromů; a každý strom, který nenese dobré ovoce, bude vyťat a hozen do ohně.
#3:11 Já vás křtím vodou k pokání; ale ten, který přichází za mnou, je silnější než já - nejsem hoden ani toho, abych mu zouval obuv; on vás bude křtít Duchem svatým a ohněm.
#3:12 Lopata je v jeho ruce; a pročistí svůj mlat, svou pšenici shromáždí do sýpky, ale plevy spálí neuhasitelným ohněm.“
#3:13 Tu přišel Ježíš z Galileje k Jordánu za Janem, aby se dal od něho pokřtít.
#3:14 Ale on mu bránil a říkal: „Já bych měl být pokřtěn od tebe, a ty jdeš ke mně?“
#3:15 Ježíš mu odpověděl: „Připusť to nyní; neboť tak je třeba, abychom naplnili všechno, co Bůh žádá.“ Tu mu již Jan nebránil.
#3:16 Když byl Ježíš pokřtěn, hned vystoupil z vody, a hle, otevřela se nebesa a spatřil Ducha Božího, jak sestupuje jako holubice a přichází na něho.
#3:17 A z nebe promluvil hlas: „Toto je můj milovaný Syn, jehož jsem si vyvolil.“ 
#4:1 Tehdy byl Ježíš Duchem vyveden na poušť, aby byl pokoušen od ďábla.
#4:2 Postil se čtyřicet dní a čtyřicet nocí, až nakonec vyhladověl.
#4:3 Tu přistoupil pokušitel a řekl mu: „Jsi-li Syn Boží, řekni, ať z těchto kamenů jsou chleby.“
#4:4 On však odpověděl: „Je psáno: ‚Ne jenom chlebem bude člověk živ, ale každým slovem, které vychází z Božích úst.‘“
#4:5 Tu ho vezme ďábel na do svatého města, postaví ho na vrcholek chrámu
#4:6 a řekne mu: „Jsi-li Syn Boží, vrhni se dolů; vždyť je psáno: ‚Svým andělům dá příkaz a na ruce tě vezmou, abys nenarazil nohou na kámen‘!“
#4:7 Ježíš mu pravil: „Je také psáno: ‚Nebudeš pokoušet Hospodina, Boha svého.‘“
#4:8 Pak ho ďábel vezme na velmi vysokou horu, ukáže mu všechna království světa i jejich slávu
#4:9 a řekne mu: „Toto všechno ti dám, padneš-li přede mnou a budeš se mi klanět.“
#4:10 Tu mu Ježíš odpoví: „Jdi z cesty, satane; neboť je psáno: ‚Hospodinu, Bohu svému, se budeš klanět a jeho jediného uctívat.‘
#4:11 V té chvíli ho ďábel opustil, a hle, andělé přistoupili a obsluhovali ho.
#4:12 Když Ježíš uslyšel, že Jan je uvězněn, odebral se do Galileje.
#4:13 Opustil Nazaret a usadil se v Kafarnaum při moři, v území Zabulón a Neftalím,
#4:14 aby se splnilo, co je řečeno ústy proroka Izaiáše:
#4:15 ‚Země Zabulón a Neftalím, směrem k moři za Jordánem, Galilea pohanů -
#4:16 lid bydlící v temnotách uvidí veliké světlo; světlo vzejde těm, kdo seděli v krajině stínu smrti.‘
#4:17 Od té chvíle začal Ježíš kázat: „Čiňte pokání, neboť se přiblížilo království nebeské.“
#4:18 Když procházel podél Galilejského moře, uviděl dva bratry, Šimona zvaného Petr a jeho bratra Ondřeje, jak vrhají síť do moře; byli totiž rybáři.
#4:19 Řekl jim: „Pojďte za mnou a učiním z vás rybáře lidí.“
#4:20 Oni hned nechali sítě a šli za ním.
#4:21 O něco dále uviděl jiné dva bratry, Jakuba Zebedeova a jeho bratra Jana, jak na lodi se svým otcem Zebedeem spravují sítě; a povolal je.
#4:22 Ihned opustili loď i svého otce a šli za ním.
#4:23 Ježíš chodil po celé galileji, učil v jejich synagógách, kázal evangelium království Božího a uzdravoval každou nemoc a každou chorobu v lidu.
#4:24 Pověst o něm se roznesla po celé Sýrii; přinášeli k němu všechny nemocné, postižené rozličnými neduhy a trápením, posedlé, náměsíčné, ochrnuté, a uzdravoval je.
#4:25 A velké zástupy z Galileje, Desetiměstí, z Jeruzaléma, Judska i ze Zajordání ho následovaly. 
#5:1 Když spatřil zástupy, vystoupil na horu; a když se posadil, přistoupili k němu jeho učedníci.
#5:2 Tu otevřel ústa a učil je:
#5:3 „Blaze chudým v duchu, neboť jejich je království nebeské.
#5:4 Blaze těm, kdo pláčou, neboť oni budou potěšeni.
#5:5 Blaze tichým, neboť oni dostanou zemi za dědictví.
#5:6 Blaze těm, kdo hladovějí a žízní po spravedlnosti, neboť oni budou nasyceni.
#5:7 Blaze milosrdným, neboť oni dojdou milosrdenství.
#5:8 Blaze těm, kdo mají čisté srdce, neboť oni uzří Boha.
#5:9 Blaze těm, kdo působí pokoj, neboť oni budou nazváni syny Božími.
#5:10 Blaze těm, kdo jsou pronásledováni pro spravedlnost, neboť jejich je království nebeské.
#5:11 Blaze vám, když vás budou tupit a pronásledovat a lživě mluvit proti vám všecko zlé kvůli mně.
#5:12 Radujte se a jásejte, protože máte hojnou odměnu v nebesích; stejně pronásledovali i proroky, kteří byli před vámi.
#5:13 Vy jste sůl země; jestliže však sůl pozbude chuti, čím bude osolena? K ničemu již není, než aby se vyhodila ven a lidé po ní šlapali.
#5:14 Vy jste světlo světa. Nemůže zůstat skryto město ležící na hoře.
#5:15 A když rozsvítí lampu, nestaví ji pod nádobu, ale na svícen; a svítí všem v domě.
#5:16 Tak ať svítí světlo vaše před lidmi, aby viděli vaše dobré skutky a vzdali slávu vašemu Otci v nebesích.
#5:17 Nedomnívejte se, že jsem přišel zrušit Zákon nebo Proroky; nepřišel jsem zrušit, nýbrž naplnit.
#5:18 Amen, pravím vám,: DOkud nepomine nebe a země, nepomine ani jediné písmenko ani jediná čárka ze Zákona, dokud se všechno nestane.
#5:19 Kdo by tedy zrušil jediné z těchto nejmenších přikázání a tak učil lidi, bude v království nebeském vyhlášen za nejmenšího; kdo by však zachovával a učil, bude v království nebeském vyhlášen velkým.
#5:20 Neboť pravím vám: Nebude-li vaše spravedlnost o mnoho přesahovat spravedlnost zákoníků a farizeů, jistě nevejdete do království nebeského.
#5:21 Slyšeli jste, že bylo řečeno otcům: Nezabiješ! Kdo by zabil, bude vydán soudu.
#5:22 Já však pravím, že již ten, kdo se hněvá na svého bratra, bude vydán soudu; kdo snižuje svého bratra, bude vydán radě; kdo svého bratra zatracuje, propadne ohnivému peklu.
#5:23 Přinášíš-li tedy svůj dar na oltář a tam se rozpomeneš, že tvůj bratr má něco proti tobě,
#5:24 nech svůj dar před oltářem a jdi se nejprve smířit se svým bratrem; potom teprve přijď a přines svůj dar.
#5:25 Dohodni se svým protivníkem včas, dokud jsi na cestě k soudu, aby tě neodevzdal soudci a soudce žalářníkovi a byl bys uvržen do vězení.
#5:26 Amen, pravím ti, že odtud nevyjdeš, dokud nezaplatíš do posledního haléře.
#5:27 Slyšeli jste, že bylo řečeno: ‚Nezcizoložíš.‘
#5:28 Já však vám pravím, že každý, kdo hledí na ženu chtivě, již s ní zcizoložil ve svém srdci.
#5:29 Jestliže tě svádí tvé pravé oko, vyrvi je a odhoď pryč, neboť je pro tebe lépe, aby zahynul jeden ze tvých údů, než aby celé tvé tělo bylo uvrženo do pekla.
#5:30 A jestliže tě svádí tvá pravá ruka, utni ji a odhoď pryč, neboť je pro tebe lépe aby zahynul jeden ze tvých údů, než aby celé tvé tělo bylo uvrženo do pekla.
#5:31 Také bylo řečeno: ‚Kdo propustí svou manželku manželku, ať jí dá rozlukový lístek!‘
#5:32 Já však vám pravím, že každý, kdo propouští svou manželku mimo případ smilstva, uvádí ji do cizoložství; a kdo by se s propuštěnou ženou oženil, cizoloží.
#5:33 Dále jste slyšeli, že řečeno bylo otcům: ‚Nebudeš přísahat křivě, ale splníš Hospodinu přísahy své.‘
#5:34 Já však vám praví, abyste nepřísahali vůbec; ani při nebi, protože nebe je trůn Boží;
#5:35 ani při zemi, protože země je podnož jeho nohou; ani při Jeruzalému, protože je to město velikého krále;
#5:36 ani při své hlavě nepřísahej, protože nemůžeš způsobit, aby ti jediný vlas zbělel nebo zčernal.
#5:37 Vaše slovo buď ‚ano, ano - ne, ne‘; co je nad to, je ze zlého.
#5:38 Slyšeli jste, že bylo řečeno: ‚Oko za oko, zub za zub‘.
#5:39 Já však vám pravím, abyste se zlým nejednali jako on s vámi; ale kdo tě uhodí do pravé tváře, nastav mu i druhou;
#5:40 a tomu, kdo by se chtěl s tebou soudit o košili, nech i svůj plášť.
#5:41 Kdo tě donutí k službě na jednu míli, jdi s ním dvě.
#5:42 Kdo tě prosí, tomu dej, a kdo si chce od tebe vypůjčit, od toho se neodvracej.
#5:43 Slyšeli jste, že bylo řečeno: ‚Milovati budeš bližního svého a nenávidět nepřítele svého.‘
#5:44 Já však pravím: Milujte své nepřátele a modlete se za ty, kdo vás pronásledují,
#5:45 abyste byli syny nebeského Otce; protože on dává svému slunci svítit na zlé i dobré a déšť posílá na spravedlivé i nespravedlivé.
#5:46 Budete-li milovat ty, kdo milují vás, jaká vás čeká odměna? Což i celníci nečiní totéž?
#5:47 A jestliže zdravíte jenom své bratry, co činíte zvláštního? Což i pohané nečiní totéž?
#5:48 Buďte tedy dokonalí, jako je dokonalý váš nebeský Otec. 
#6:1 Varujte se konat skutky spravedlnosti před lidmi, jim na odiv; jinak nemáte odměnu u svého Otce v nebesích.
#6:2 Když prokazuješ dobrodiní, nechtěj budit pozornost, jako činí pokrytci v synagógách a na ulicích, aby došli slávy u lidí; amen pravím vám, už mají svou odměnu.
#6:3 Když ty prokazuješ dobrodiní, ať neví tvá levice, co činí pravice,
#6:4 aby tvé dobrodiní zůstalo skryto, a tvůj Otec, který vidí, co je skryto, ti odplatí.
#6:5 A když se modlíte, nebuďte jako pokrytci: ti se s oblibou modlí v synagógách a na nárožích, aby byli lidem na očích; amen pravím vám, už mají svou odměnu.
#6:6 Když se modlíš, vejdi do svého pokojíku, zavři za sebou dveře a modli se k svému Otci, který zůstává skryt; a tvůj Otec, který vidí, co je skryto, ti odplatí.
#6:7 Při modlitbě pak nemluvte naprázdno jako pohané; oni si myslí, že budou vyslyšeni pro množství svých slov.
#6:8 Nebuďte jako oni; vždyť váš Otec ví, co potřebujete, dříve než ho prosíte.
#6:9 Vy se modlete takto: Otče náš, jenž jsi na nebesích, buď posvěceno tvé jméno.
#6:10 Přijď tvé království. Staň se tvá vůle jako v nebi, tak i na zemi.
#6:11 Náš denní chléb dej nám dnes.
#6:12 A odpusť nám naše viny, jako i my jsme odpustili těm, kdo se provinili proti nám.
#6:13 A nevydej nás v pokušení, ale vysvoboď nás od zlého.
#6:14 Neboť jestliže odpustíte lidem jejich přestoupení, i vám odpustí váš nebeský Otec;
#6:15 jestliže však neodpustíte lidem, ani váš Otec vám neodpustí vaše přestoupení.
#6:16 A když se postíte, netvařte se utrápeně jako pokrytci; ti zanedbávají svůj vzhled, aby lidem ukazovali, že se postí; amen, pravím vám, už mají svou odměnu.
#6:17 Když se postíš, potři svou hlavu olejem a tvář svou umyj,
#6:18 abys neukazoval lidem, že se postíš, ale svému Otci, který zůstává skryt; a tvůj Otec, který vidí, co je skryto, ti odplatí.
#6:19 Neukládejte si poklady na zemi, kde je ničí mol a rez a kde je zloději vykopávají a kradou.
#6:20 Ukládejte si poklady v nebi, kde je neničí mol ani rez a kde je zloději nevykopávají a nekradou.
#6:21 Neboť kde je tvůj poklad, tam bude i tvé srdce.
#6:22 Světlem těla je oko. Je-li tedy tvé oko čisté, celé tvé tělo bude mít světlo.
#6:23 Je-li však tvé oko špatné, celé tvé tělo bude ve tmě. Jestliže i světlo v tobě je temné, jak velká bude potom tma?
#6:24 Nikdo nemůže sloužit dvěma pánům. Neboť jednoho bude nenávidět a druhého milovat, k jednomu se přidá a druhým potom pohrdne. Nemůžete sloužit Bohu i majetku.
#6:25 Proto vám pravím: Nemějte starost o svůj život, co budete jíst, ani o tělo, co budete mít na sebe. Což není život víc než pokrm a tělo víc než oděv?
#6:26 Pohleďte na nebeské ptactvo: neseje, nežne, nesklízí do stodol, a přece je váš nebeský Otec živí. Což vy nejste o mnoho cennější?
#6:27 Kdo z vás může o jedinou píď prodloužit svůj život, bude-li se znepokojovat?
#6:28 A o oděv proč si děláte starosti? Podívejte se na polní lilie, jak rostou: nepracují, nepředou -
#6:29 a pravím vám, že ani Šalomoun v celé své nádheře nebyl tak oděn, jako jedna z nich.
#6:30 Jestliže tedy Bůh tak obléká polní trávu, která tu dnes je a zítra bude hozena do pece, neobleče tím spíše vás, malověrní?
#6:31 Nemějte starost a neříkejte: co budeme jíst? Co budeme pít? Co si budeme oblékat?
#6:32 Po tom všem se shánějí pohané. Váš nebeský Otec přece ví, že to všechno potřebujete.
#6:33 Hledejte především jeho království a spravedlnost, a všechno ostatní vám bude přidáno.
#6:34 Nedělejte si tedy starosti o zítřek; zítřek bude mít své starosti. Každý den má dost na svém trápení. 
#7:1 Nesuďte, abyste nebyli souzeni.
#7:2 Neboť jakým soudem soudíte, takovým budete souzeni, a jakou měrou měříte, takovou Bůh naměří vám.
#7:3 Jak to, že vidíš třísku v oku svého bratra, ale trám ve vlastním oku nepozoruješ?
#7:4 Anebo jak to, že říkáš svému bratru: ‚Dovol, ať ti vyjmu třísku z oka‘ - a hle, trám ve tvém vlastním oku!
#7:5 Pokrytče, nejprve vyjmi ze svého oka trám, a pak teprve prohlédneš, abys mohl vyjmout třísku z oka svého bratra.
#7:6 Nedávejte psům, co je svaté. Neházejte perly před svině, nebo je nohama zašlapou, otočí se a roztrhají vás.
#7:7 Proste, a bude vám dáno; hledejte a naleznete; tlučte a bude vám otevřeno.
#7:8 Neboť každý, kdo prosí, dostává, a kdo hledá, nalézá, a kdo tluče, tomu bude otevřeno.
#7:9 Což by někdo z vás dal svému synu kámen, když ho prosí o chléb?
#7:10 Nebo by mu dal hada, když ho prosí o rybu?
#7:11 Jestliže tedy vy, ač jste zlí, umíte svým dětem dávat dobré dary, tím spíše váš Otec v nebesích dá dobré těm, kdo ho prosí!
#7:12 Jak byste chtěli, aby lidé jednali s vámi, tak vy jednejte s nimi; v tom je celý Zákon, i Proroci.
#7:13 Vejděte těsnou branou; prostorná je brána a široká cesta, která vede do záhuby; a mnoho je těch, kdo tudy vcházejí.
#7:14 Těsná je brána a úzká cesta, která vede k životu, a málokdo ji nalézá.
#7:15 Střezte se lživých proroků, kteří k vám přicházejí v rouchu ovčím, ale uvnitř jsou draví vlci.
#7:16 Po jejich ovoci je poznáte. Což sklízejí z trní hrozny nebo z bodláčí fíky?
#7:17 Tak každý dobrý strom dává dobré ovoce, ale špatný strom dává špatné ovoce.
#7:18 Dobrý strom nemůže nést špatné ovoce a špatný strom nemůže nést dobré ovoce.
#7:19 Každý strom, který nedává dobré ovoce, bude vyťat a hozen do ohně.
#7:20 A tak je poznáte po jejich ovoci.
#7:21 Ne každý, kdo mi říká ‚Pane, Pane‘, vejde do království nebeského; ale ten, kdo činí vůli mého Otce v nebesích.
#7:22 Mnozí mi řeknou v onen den: ‚Pane, Pane, což jsme ve tvém jménu neprorokovali a ve tvém jménu nevymítali zlé duchy a ve tvém jménu neučinili mnoho mocných činů?‘
#7:23 A tehdy já prohlásím: ‚Nikdy jsem vás neznal; jděte ode mne, kdo se dopouštíte nepravosti.‘
#7:24 A tak každý, kdo slyší tato má slova a plní je, bude podoben rozvážnému muži, který postavil svůj dům na skále.
#7:25 Tu spadl příval, přihnaly se vody, zvedla se vichřice, a vrhly se na ten dům; ale nespadl, neboť měl základy na skále.
#7:26 Ale každý, kdo slyší tato má slova a neplní je, bude podoben muži bláznivému, který postavil svůj dům na písku.
#7:27 A spadl příval, přihnaly se vody, zvedla se vichřice, a obořily se na ten dům; a padl, a jeho pád byl veliký.“
#7:28 Když Ježíš dokončil tato slova, zástupy žasly nad jeho učením;
#7:29 neboť je učil jako ten, kdo má moc, a ne jako jejich zákoníci. 
#8:1 Když sestoupil z hory, šly za ním velké zástupy.
#8:2 Tu k němu přistoupil malomocný, padl před ním na zem a řekl: „Pane, chceš-li, můžeš mě očistit.“
#8:3 On vztáhl ruku, dotkl se ho a řekl: „Chci, buď čist.“ A hned byl očištěn od svého malomocenství.
#8:4 Tu mu Ježíš pravil: „Ne abys o tom někomu říkal! Ale jdi, ukaž se knězi a obětuj dar, který Mojžíš přikázal - jim na svědectví.“
#8:5 Tu když přišel do Kafarnaum, přistoupil k němu jeden setník a prosil ho:
#8:6 „Pane, můj sluha leží doma ochrnutý a hrozně trpí.“
#8:7 Řekl mu: „Já přijdu a uzdravím ho.“
#8:8 Setník však odpověděl: „Pane, nejsem hoden, abys vstoupil pod mou střechu; ale řekni jen slovo, a můj sluha bude uzdraven.
#8:9 Vždyť i já podléhám rozkazům a vojákům rozkazuji; řeknu-li některému ‚jdi‘, pak jde; jinému ‚pojď sem‘, pak přijde; a svému otroku ‚udělej to‘, pak to udělá.“
#8:10 Když to Ježíš uslyšel, podivil se, a řekl těm, kdo ho následovali: „Amen, amen, pravím vám, tak velikou víru jsem v Izraeli nenalezl u nikoho.
#8:11 Pravím vám, že mnozí od východu i západu přijdou a budou stolovat s Abrahamem, Izákem a Jákobem v království nebeském;
#8:12 ale synové království budou vyvrženi ven do tmy; tam bude pláč a skřípění zubů.“
#8:13 Potom řekl Ježíš setníkovi: „Jdi, a jak jsi uvěřil, tak se ti staň.“ A v tu chvíli se sluha uzdravil.
#8:14 Když Ježíš vstoupil do domu Petrova, spatřil, že jeho tchyně leží v horečce.
#8:15 Dotkl se její ruky a horečka ji opustila; i vstala a obsluhovala ho.
#8:16 Když nastal večer, přinesli k němu mnoho posedlých; i vyhnal duchy svým slovem a všechny nemocné uzdravil,
#8:17 aby se naplnilo, co je řečeno ústy proroka Izaiáše: ‚On slabosti naše na sebe vzal a nemoci nesl.‘
#8:18 Když Ježíš viděl kolem sebe zástup, rozkázal odjet na druhý břeh.
#8:19 Jeden zákoník přišel a řekl mu: „Mistře, budu tě následovat, kamkoli půjdeš.“
#8:20 Ale Ježíš mu odpověděl: „Lišky mají doupata a ptáci hnízda, ale Syn člověka nemá, kde by hlavu složil.“
#8:21 Jiný z učedníků mu řekl: „Pane, dovol mi napřed odejít a pochovat svého otce.“
#8:22 Ale Ježíš mu řekl: „Následuj mě a nech mrtvé, ať pochovávají své mrtvé.“
#8:23 Vstoupil na loď a učedníci ho následovali.
#8:24 V tom se strhla na moři veliká bouře, takže loď už mizela ve vlnách; ale on spal.
#8:25 I přistoupili a probudili ho se slovy: „Pane, zachraň nás, nebo zahyneme!“
#8:26 Řekl jim: „Proč jste tak ustrašeni, malověrní?“ Vstal, pohrozil větrům i moři; a nastalo veliké ticho.
#8:27 Lidé užasli a říkali: „Kdo to jen je, že ho poslouchají větry i moře?.“
#8:28 Když přijel na druhý břeh moře do gadarenské krajiny, vyšli proti němu dva posedlí, kteří vystoupili z hrobů; byli velmi nebezpeční, takže se nikdo neodvážil tudy chodit.
#8:29 A dali se do křiku: „Co je ti po nás, Synu Boží? Přišel jsi nás trápit dříve, než nastal čas?“
#8:30 Opodál se páslo veliké stádo vepřů.
#8:31 A zlí duchové ho prosili: „Když už nás vyháníš, pošli nás do toho stáda vepřů!“
#8:32 On jim řekl: „Jděte!“ Tu vyšli a vešli do vepřů; a hle, celé stádo se hnalo střemhlav po srázu do moře a zahynulo ve vodách.
#8:33 Pasáci utekli, přišli do města a vyprávěli všechno, i o těch posedlých.
#8:34 A celé město vyšlo naproti Ježíšovi, a když ho spatřili, prosili ho, aby se vzdálil z jejich končin. 
#9:1 Ježíš vstoupil na loď a přeplavil se na druhou stranu a přišel do svého města.
#9:2 A hle, přinesli k němu ochrnutého, ležícího na lůžku. Když Ježíš viděl jejich víru, řekl ochrnutému: „Buď dobré mysli, synu, odpouštějí se ti hříchy.“
#9:3 Ale někteří ze zákoníků si řekli: „Ten člověk se rouhá!“
#9:4 Ježíš však poznal jejich myšlenky a řekl: „Proč o tom smýšlíte tak zle?
#9:5 Je snadnější říci ‚odpouštějí se ti hříchy‘, nebo říci ‚vstaň a choď‘?
#9:6 Abyste však věděli, že Syn člověka má moc na zemi odpouštět hříchy“ - tu řekne ochrnutému: „Vstaň, vezmi své lože a jdi domů!“
#9:7 On vstal a odešel domů.
#9:8 Když to viděly zástupy, zmocnila se jich bázeň a chválili Boha, že dal takovou moc lidem.
#9:9 Když šel Ježíš odtud dál, viděl v celnici sedět člověka jménem Matouš a řekl mu: „Pojď za mnou!“ On vstal a šel za ním.
#9:10 Když potom seděl u stolu v domě, hle, mnoho celníků a jiných hříšníků stolovalo s Ježíšem a jeho učedníky.
#9:11 Farizeové to uviděli a řekli jeho učedníkům: „Jak to, že váš Mistr jí s celníky a hříšníky?“
#9:12 On to uslyšel a řekl: „Lékaře nepotřebují zdraví, ale nemocní.
#9:13 Jděte a učte se, co to je: ‚Milosrdenství chci, a ne oběť‘. Nepřišel jsem pozvat spravedlivé, ale hříšníky.“
#9:14 Tehdy k němu přišli Janovi učedníci a ptali se ho: „Jak to, že my a farizeové se postíme, ale tvoji učedníci se nepostí?“
#9:15 Ježíš jim řekl: „Mohou hosté na svatbě truchlit, dokud je ženich s nimi? Přijdou však dny, kdy od nich bude ženich vzat; potom se budou postit.
#9:16 Nikdo nezalátá starý šat záplatou z neseprané látky; nebo se ten přišitý kus ze šatu vytrhne a díra bude ještě větší.
#9:17 A mladé víno se nedává do starých měchů, jinak se měchy roztrhnou, víno vyteče a měchy přijdou nazmar. Nové víno se dává do nových měchů, a tak se uchová obojí.“
#9:18 Zatímco k nim takto mluvil, přišel jeden z představených, klaněl se před ním a řekl: „Má dcera právě skonala; ale pojď, vlož na ni svou ruku, a bude žít!“
#9:19 Ježíš vstal a šel s ním i se svými učedníky.
#9:20 A hle, žena trpící už dvanáct let krvácením přišla zezadu a dotkla se jeho šatu.
#9:21 Říkala si totiž: „Dotknu-li se aspoň jeho šatu, budu zachráněna!“
#9:22 Ježíš se obrátil a spatřiv ji, řekl: „Buď dobré mysli, dcero, tvá víra tě zachránila.“ A od té hodiny byla ta žena zdráva.
#9:23 Když Ježíš vstoupil do domu toho představeného a uviděl hudebníky a hlučící zástup,
#9:24 řekl: „Jděte odtud! Ta dívka neumřela, ale spí.“ Oni se mu posmívali.
#9:25 A když byl zástup vyhnán, vešel Ježíš dovnitř, vzal dívku za ruku a ona vstala.
#9:26 Pověst o tom se rozšířila po celé té krajině.
#9:27 Když šel Ježíš odtamtud dál, šli za ním dva slepci a křičeli: „Smiluj se nad námi, Synu Davidův!“
#9:28 A když vešel do domu, přistoupili ti slepci k němu. Ježíš jim řekl: „Věříte, že to mohu učinit?“ Odpověděli mu: „Ano, Pane.“
#9:29 Tu se dotkl jejich očí a řekl: „Podle vaší víry se vám staň.“
#9:30 A otevřely se jim oči. Ježíš jim pohrozil: „Ne, aby se to někdo dověděl!“
#9:31 Oni však šli a rozhlásili ho po celé té krajině.
#9:32 Když odcházeli, přivedli k němu němého člověka, posedlého zlým duchem.
#9:33 A zlý duch byl vyhnán a němý mluvil. Zástupy v údivu říkaly: „Něco takového nebylo v Izraeli nikdy vídáno.“
#9:34 Ale farizeové říkali: „Ve jménu knížete démonů vyhání démony.“
#9:35 Ježíš obcházel všechna města i vesnice, učil v jejich synagógách, kázal evangelium království a uzdravoval každou nemoc a každou chorobu.
#9:36 Když viděl zástupy, bylo mu jich líto, protože byli vysílení a skleslí jako ovce bez pastýře.
#9:37 Tehdy řekl svým učedníkům: „Žeň je velká, dělníků málo.“
#9:38 Proste proto Pána žně ať vyšle dělníky na svou žeň!“ 
#10:1 Zavolal svých dvanáct učedníků a dal jim moc nad nečistými duchy, aby je vymítali a uzdravovali každou nemoc a každou chorobu.
#10:2 Jména těch dvanácti jsou: první Šimon zvaný Petr, jeho bratr Ondřej, Jakub Zebedeův, jeho bratr Jan,
#10:3 Filip, Bartoloměj, Tomáš, celník Matouš, Jakub Alfeův, Tadeáš,
#10:4 Šimon Kananejský a Iškariotský Jidáš, který ho pak zradil.
#10:5 Těchto dvanáct Ježíš vyslal a přikázal jim: „Na cestu k pohanům nevstupujte, do samařské obce nechoďte;
#10:6 jděte raději ke ztraceným ovcím z lidu izraelského.
#10:7 Jděte a kažte, že se přiblížilo království nebeské.
#10:8 Nemocné uzdravujte, mrtvé probouzejte k životu, malomocné očišťujte, démony vymítejte; zadarmo jste dostali, zadarmo dejte.
#10:9 Neberte od nikoho zlato, stříbro ani měďáky do opasku;
#10:10 neberte si na cestu mošnu ani dvoje šaty ani obuv ani hůl, neboť ‚hoden je dělník své mzdy‘.
#10:11 Když přijdete do některého města nebo vesnice, vyptejte se, kdo z nich je toho hoden; u něho zůstaňte, dokud nebudete odcházet.
#10:12 Když vstoupíte do domu, řekněte: ‚Pokoj vám.‘
#10:13 A budou-li toho hodni, ať na ně přijde váš pokoj. Nebudou-li toho hodni, ať se váš pokoj vrátí k vám.
#10:14 A když vás někdo nepřijme a nebude chtít slyšet vaše slova, vyjděte ven z toho domu nebo města a setřeste prach svých nohou.
#10:15 Amen, pravím vám, lehčeji bude zemi sodomské a gomorské v den soudu, než tomu městu.
#10:16 Hle, já vás posílám jako ovce mezi vlky; buďte tedy obezřetní jako hadi a bezelstní jako holubice.
#10:17 Mějte se na pozoru před lidmi; neboť vás budou vydávat soudům, ve svých synagógách vás budou bičovat,
#10:18 budou vás vodit před vládce a krále kvůli mně, abyste vydali svědectví jim i národům.
#10:19 A když vás obžalují, nedělejte si starosti, jak a co budete mluvit; neboť v tu hodinu vám bude dáno, co máte mluvit.
#10:20 Nejste to vy, kdo mluvíte, ale mluví ve vás Duch vašeho Otce.
#10:21 Vydá na smrt bratr bratra a otec dítě, povstanou děti proti rodičům a připraví je o život.
#10:22 Budou vás všichni nenávidět pro mé jméno; ale kdo vytrvá až do konce, bude spasen.
#10:23 Když vás budou pronásledovat v jednom městě, prchněte do jiného; amen, pravím vám, že nebudete hotovi se všemi izraelskými městy, než přijde Syn člověka.
#10:24 Žák není nad učitele ani sluha nad svého pána.
#10:25 Stačí, aby žák byl jako jeho učitel a sluha jako jeho pán. Když hospodáře nazvali Belzebulem, čím spíše jeho čeleď!
#10:26 Nebojte se jich tedy; neboť není nic zahaleného, co nebude jednou odhaleno, a nic skrytého, co nebude poznáno.
#10:27 Co vám říkám ve tmě, povězte na světle; a co slyšíte v soukromí, hlásejte se střech.
#10:28 A nebojte se těch, kdo zabíjejí tělo, ale duši zabít nemohou; bojte se toho, který může duši i tělo zahubit v pekle.
#10:29 Neprodávají se dva vrabci za haléř? A ani jeden z nich nepadne na zem bez dopuštění vašeho Otce.
#10:30 U vás pak jsou spočteny i všechny vlasy na hlavě.
#10:31 Nebojte se tedy; máte větší cenu než mnoho vrabců.
#10:32 Každý, kdo se ke mně přizná před lidmi, k tomu se i já přiznám před svým Otcem v nebi;
#10:33 kdo mně však zapře před lidmi, toho i já zapřu před svým Otcem v nebi.
#10:34 Nemyslete si, že jsem přišel na zem uvést pokoj; nepřišel jsem uvést pokoj, ale meč.
#10:35 Neboť jsem přišel postavit syna proti otci, dceru proti matce, snachu proti tchyni;
#10:36 a ‚nepřítelem člověka bude jeho vlastní rodina‘.
#10:37 Kdo miluje otce a matku více nežli mne, není mne hoden.
#10:38 Kdo nenese svůj kříž a nenásleduje mne, není mne hoden.
#10:39 Kdo nalezne svůj život, ztratí jej; kdo ztratí svůj život pro mne, nalezne jej.
#10:40 Kdo vás přijímá, přijímá mne; a kdo přijímá mne, přijímá toho, který mne poslal.
#10:41 Kdo přijme proroka, protože je to prorok, obdrží odměnu proroka; kdo přijme spravedlivého, protože je to spravedlivý, obdrží odměnu spravedlivého.
#10:42 A kdo by napojil třebas jen číší studené vody jednoho z těchto nepatrných, protože je to učedník, amen, pravím vám, nepřijde o svou odměnu.“ 
#11:1 Když Ježíš dokončil tyto příkazy svým dvanácti učedníkům, šel odtud učit a kázat v tamějších městech.
#11:2 Jan uslyšel ve vězení o činech Kristových; poslal k němu vzkaz po svých služebnících:
#11:3 „Jsi ten, který má přijít, nebo máme čekat jiného?“
#11:4 Ježíš jim odpověděl: „Jděte, zvěstujte Janovi, co vidíte a slyšíte:
#11:5 Slepí vidí, chromí chodí, malomocní jsou očišťováni, hluší slyší, mrtví vstávají, chudým se zvěstuje evangelium.
#11:6 A blaze tomu, kdo se nade mnou neuráží.“
#11:7 Když Janovi učedníci odcházeli, začal Ježíš mluvit k zástupům o Janovi: „Na co jste se vyšli na poušť podívat? Na rákos, kterým kývá vítr?
#11:8 Nebo co jste vyšli shlédnout? Člověka oblečeného do drahých šatů? Ti, kdo nosí drahé šaty, jsou v domech královských.
#11:9 Nebo proč jste vyšli? Vidět proroka? Ano, pravím vám, a víc než proroka.
#11:10 To je ten, o němž je psáno: ‚Hle, já posílám posla před svou tváří, aby ti připravil cestu.‘
#11:11 Amen, pravím vám, mezi těmi, kdo se narodili z ženy, nevystoupil nikdo větší, než Jan Křtitel; avšak i ten nejmenší v království nebeském je větší nežli on.
#11:12 Ode dnů Jana Křtitele až podnes království nebeské trpí násilí a násilníci po něm sahají.
#11:13 Neboť všichni proroci i Zákon prorokovali až po Jana.
#11:14 A chcete-li to přijmout, on je Eliáš, který má přijít.
#11:15 Kdo má uši, slyš.
#11:16 Čemu připodobním toto pokolení? Je jako děti, které sedí na tržišti a pokřikují na své druhy:
#11:17 ‚Hráli jsme vám, a vy jste netancovali; naříkali jsme, a vy jste nelomili rukama.‘
#11:18 Přišel Jan, nejedl, nepil - a říkají: ‚Je posedlý.‘
#11:19 Přišel Syn člověka, jí a pije - a říkají: ‚Hle, milovník hodů a pitek, přítel celníků a hříšníků!‘ Ale moudrost je ospravedlněna svými skutky.“
#11:20 Tehdy počal kárat města, ve kterých se stalo nejvíc jeho mocných skutků, že nečinila pokání:
#11:21 „Běda ti, Chorazin, běda ti, Betsaido! Kdyby se byly v Týru a Sidónu dály takové mocné skutky jako u vás, dávno by byli oblékli žíněný šat, sypali se popelem a činili pokání.
#11:22 Ale pravím vám: Týru a Sidónu bude lehčeji v den soudu, nežli vám.
#11:23 A ty, Kafarnaum, budeš snad vyvýšeno až do nebe? Až do propasti klesneš! Neboť kdyby se byly v Sodomě odehrály takové mocné skutky, jako u vás, stála by podnes.
#11:24 Ale pravím vám: zemi Sodomské bude lehčeji v den soudu, nežli tobě.“
#11:25 V ten čas řekl Ježíš: „Velebím, tě Otče, Pane nebes i země, že jsi ty věci skryl před moudrými a rozumnými a zjevil jsi je maličkým.
#11:26 Ano, Otče, tak se ti zalíbilo.
#11:27 Všechno je mi dáno od Otce; a nikdo nezná Syna než Otec, ani Otce nezná nikdo než Syn - a ten, komu by to Syn chtěl zjevit.
#11:28 Pojďte ke mně všichni, kdo se namáháte a jste obtíženi břemeny, a já vám dám odpočinout.
#11:29 Vezměte na sebe mé jho a učte se ode mne, neboť jsem tichý a pokorného srdce: a naleznete odpočinutí svým duším.
#11:30 Vždyť mé jho netlačí a břemeno netíží.“ 
#12:1 V ten čas šel Ježíš v sobotu obilím. Jeho učedníci dostali hlad a začali mnout zrní z klasů a jíst.
#12:2 Když to viděli farizeové, řekli mu: „Hle, tvoji učedníci dělají, co se nesmí dělat v sobotu!“
#12:3 On jim však řekl: „Nečetli jste, co udělal David, když měl hlad, on a ti kdo byli s ním?
#12:4 Jak vešel do domu Božího a jedli posvátné chleby, ačkoli to nebylo dovoleno jemu ani těm, kdo ho doprovázeli, nýbrž kněžím?
#12:5 A nečetli jste v Zákoně, že kněží službou v chrámu porušují sobotu, a přesto jsou bez viny?
#12:6 Pravím vám, že zde je víc než chrám.
#12:7 Kdybyste věděli, co znamená, ‚milosrdenství chci, a ne oběť‘, neodsuzovali byste nevinné.
#12:8 Vždyť Syn člověka je pánem nad sobotou.“
#12:9 Odtud šel dál a přišel do jejich synagógy.
#12:10 A byl tam člověk s odumřelou rukou. Otázali se Ježíše: „Je dovoleno v sobotu uzdravovat?“ Chtěli ho totiž obžalovat.
#12:11 On jim řekl: „Kdyby někdo z vás měl jedinou ovečku, a ona by mu v sobotu spadla do jámy, neuchopil by ji a nevytáhl?
#12:12 A oč je člověk cennější než ovce! Proto je dovoleno v sobotu činit dobře.“
#12:13 Potom řekl tomu člověku: „Zvedni tu ruku!“ Zvedl jim, a byla zase zdravá jako ta druhá.
#12:14 Farizeové vyšli a smluvili se proti němu, že ho zahubí.
#12:15 Ježíš to poznal a odešel odtamtud. Mnozí šli za ním a on všechny nemocné uzdravil;
#12:16 ale přikázal jim, aby ho nikomu neprozrazovali -
#12:17 aby se splnilo, co je řečeno ústy proroka Izaiáše:
#12:18 ‚Hle, služebník můj, kterého jsem vyvolil, milovaný můj, kterého si oblíbila duše má. Vložím na něho svého Ducha. A vyhlásí soud národům.
#12:19 Nebude se přít ani rozkřikovat, na ulicích nikdo neuslyší jeho hlas.
#12:20 Nalomenou třtinu nedolomí a doutnající knot neuhasí, až dovede právo k vítězství.
#12:21 A v jeho jménu bude naděje národů.‘
#12:22 Tehdy k němu přivedli posedlého, který byl slepý a němý; a uzdravil ho, takže ten němý mluvil i viděl.
#12:23 Zástupy žasly a říkaly: „Není to Syn Davidův?“
#12:24 Když to slyšeli farizeové, řekli: „On nevyhání démony jinak, než ve jménu Belzebula, knížete démonů.“
#12:25 Protože znal jejich smýšlení, řekl jim: „Každé království vnitřně rozdělené pustne a žádná obec ani dům vnitřně rozdělený nemůže obstát.
#12:26 A vyhání-li satan satana, pak je v sobě rozdvojen; jak tedy bude moci obstát jeho království?
#12:27 Jestliže já vyháním démony ve jménu Belzebula, ve jménu koho je vyhánějí vaši žáci? Proto oni budou vašimi soudci.
#12:28 Jestliže však vyháním démony Duchem Božím, pak už vás zastihlo Boží království.
#12:29 Což může někdo vejít do domu silného muže a uloupit jeho věci, jestliže dříve toho siláka nespoutá? Pak teprve vyloupí jeho dům.
#12:30 Kdo není se mnou, je proti mně; a kdo se mnou neshromažďuje, rozptyluje.
#12:31 Proto pravím vám, že každý hřích i rouhání bude lidem odpuštěno, ale rouhání proti Duchu svatému nebude odpuštěno.
#12:32 I tomu, kdo by řekl slovo proti Synu člověka, bude odpuštěno; ale kdo by řekl slovo proti Duchu svatému, tomu nebude odpuštěno v tomto věku ani v budoucím.
#12:33 Zasaďte dobrý strom, i jeho ovoce bude dobré. Zasaďte špatný strom, i jeho ovoce bude špatné. Strom se pozná po ovoci.
#12:34 Plemeno zmijí: Jak může být vaše řeč dobrá, když jste zlí? Čím srdce přetéká, to ústa mluví.
#12:35 Dobrý člověk z dobrého pokladu srdce vynáší dobré; zlý člověk ze zlého pokladu vynáší zlé.
#12:36 Pravím vám, že z každého planého slova, jež lidé promluví, budou skládat účty v den soudu.
#12:37 Neboť podle svých slov budeš ospravedlněn a podle svých slov odsouzen.“
#12:38 Tehdy mu na to řekli někteří ze zákoníků a farizeů: „Mistře, chceme od tebe vidět znamení.“
#12:39 On jim však odpověděl: „Pokolení zlé a zpronevěřilé si hledá znamení; ale znamení mu nebude dáno, leč znamení proroka Jonáše.
#12:40 Jako byl Jonáš v břiše mořské obludy tři dny a tři noci, tak bude Syn člověka tři dny a tři noci v srdci země.
#12:41 Mužové z Ninive povstanou na soudu s tímto pokolením a usvědčí je, neboť oni se obrátili po Jonášově kázání - a hle, zde je víc než Jonáš.
#12:42 Královna jihu povstane na soudu s tímto pokolením a usvědčí je, protože ona přišla z nejzazších končin země, aby slyšela moudrost Šalomounovu - a hle, zde je víc než Šalomoun.
#12:43 Když nečistý duch vyjde z člověka, bloudí po pustých místech a hledá odpočinutí, ale nenalézá.
#12:44 Tu řekne: ‚Vrátím se do svého domu, odkud jsem vyšel.‘ Přijde a nalezne jej prázdný, vyčištěný a uklizený.
#12:45 Tu jde a přivede s sebou sedm jiných duchů, horších, než je sám, a vejdou a bydlí tam; a konce toho člověka jsou horší, než začátky. Tak bude i s tímto zlým pokolením.“
#12:46 Ještě když mluvil k zástupům, hle, jeho matka a bratři stáli venku a chtěli s ním mluvit.
#12:47 Někdo mu řekl: „Hle, tvá matka a tvoji bratři stojí venku a chtějí s tebou mluvit.“
#12:48 On však odpověděl tomu, kdo mu to řekl: „Kdo je má matka a kdo jsou moji bratři?“
#12:49 Ukázal na své učedníky a řekl: „Hle, moje matka a moji bratři.
#12:50 Neboť kdo činí vůli mého Otce v nebesích, to je můj bratr, má sestra i matka.“ 
#13:1 Toho dne vyšel Ježíš z domu a posadil se u moře.
#13:2 Shromáždil se k němu tak veliký zástup, že musel vstoupit na loď; posadil se v ní a celý zástup stál na břehu.
#13:3 I mluvil k nim mnoho v podobenstvích: „Vyšel rozsévač rozsívat.
#13:4 Když rozsíval, padla některá zrna podél cesty, a přilétli ptáci a sezobali je.
#13:5 Jiná padla na skalnatou půdu, kde neměla dost země, a hned vzešla, protože nebyla hluboko v zemi.
#13:6 Ale když vyšlo slunce, spálilo je; a protože neměla kořen, uschla.
#13:7 Jiná zas padla mezi trní; trní vzrostlo a udusilo je.
#13:8 A jiná zrna padla do dobré země a dala užitek, některé sto zrn, jiné šedesát a jiné třicet.
#13:9 Kdo má uši, slyš!“
#13:10 Učedníci k němu přistoupili a řekli: „Proč k nim mluvíš v podobenstvích?“
#13:11 On jim odpověděl: „Protože vám je dáno znáti tajemství království nebeského, jim však není dáno.
#13:12 Kdo má, tomu bude dáno a bude mít ještě víc; ale kdo nemá, tomu bude odňato i to, co má.
#13:13 Proto k nim mluvím v podobenstvích, že hledíce nevidí a slyšíce neslyší a nechápou.
#13:14 A plní se na nich proroctví Izaiášovo: ‚Budete stále poslouchat, a nepochopíte, ustavičně budete hledět a neuvidíte.
#13:15 Neboť obrostlo tukem srdce tohoto lidu, ušima nedoslýchají a oči zavřeli, takže nevidí očima a ušima neslyší, srdcem nepochopí a neobrátí se - a já je neuzdravím.‘
#13:16 Blažené vaše oči, že vidí, i vaše uši, že slyší.
#13:17 Amen, pravím vám, že mnozí proroci a spravedliví toužili vidět, na co vy hledíte, ale neviděli, a slyšet, co vy slyšíte, a neslyšeli.
#13:18 Vy tedy slyšte výklad podobenství o rozsévači.
#13:19 Pokaždé, když někdo slyší slovo o království a nechápe, přichází ten zlý a vyrve, co bylo zaseto do jeho srdce; to je ten, u koho se zaselo podél cesty.
#13:20 U koho bylo zaseto na skalnatou půdu, to je ten, kdo slyší slovo a hned je s radostí přijímá;
#13:21 ale nezakořenilo v něm a je nestálý: když přijde tíseň nebo pronásledování pro to slovo, hned odpadá.
#13:22 U koho bylo zaseto do trní, to je ten, kdo slyší slovo, ale časné starosti a vábivost majetku slovo udusí, a zůstane bez úrody.
#13:23 U koho bylo zaseto do dobré země, to je ten, kdo slovo slyší i chápe a přináší úrodu, jeden stonásobnou, druhý šedesátinásobnou, třetí třicetinásobnou.“
#13:24 Předložil jim jiné podobenství: „S královstvím nebeským je to tak, jako když jeden člověk zasel dobré semeno na svém poli.
#13:25 Když však lidé spali, přišel nepřítel, nasel plevel do pšenice a odešel.
#13:26 Když vyrostlo stéblo a nasadilo klas, tu ukázal se i plevel.
#13:27 Přišli sluhové toho hospodáře a řekli mu: ‚Pane, cožpak jsi nezasel na svém poli dobré semeno? Kde se vzal ten plevel?‘
#13:28 On jim odpověděl: ‚To udělal nepřítel.‘ Sluhové mu řeknou: ‚Máme jít a plevel vytrhat?‘
#13:29 On však odpoví: ‚Ne, protože při trhání plevele byste vyrvali z kořenů i pšenici.
#13:30 Nechte, ať spolu roste obojí až do žně; a v čas žně řeknu žencům: Seberte nejprve plevel a svažte jej do otýpek k spálení, ale pšenici shromážděte do mé stodoly.‘“
#13:31 Ještě jiné podobenství jim předložil: „Království nebeské je jako hořčičné zrno, které člověk zaseje na svém poli;
#13:32 je sice menší, než všecka semena, ale když vyroste, je větší, než ostatní byliny a je z něho strom, takže přilétají ptáci a hnízdí v jeho větvích.“
#13:33 Pověděl jim i toto podobenství: „Království nebeské je jako kvas, který žena vmísí do tří měřic mouky, až se všecko prokvasí.“
#13:34 Toto vše mluvil Ježíš k zástupům v podobenstvích; bez podobenství k nim vůbec nemluvil,
#13:35 aby se splnilo, co bylo řečeno ústy proroka: ‚Otevřu v podobenstvích ústa svá, vyslovím, co je skryto od založení světa.‘
#13:36 Potom opustil zástupy a vešel do domu. Učedníci za ním přišli a řekli mu: „Vylož nám to podobenství o plevelu na poli!“
#13:37 On jim odpověděl: „Rozsévač, který rozsívá dobré semeno, je Syn člověka
#13:38 a pole je tento svět. Dobré semeno, to jsou synové království,plevel jsou synové toho zlého;
#13:39 nepřítel, který jej nasel, je ďábel. Žeň je skonání věku a ženci jsou andělé.
#13:40 Tak jako se tedy sbírá plevel a pálí ohněm, tak bude i při skonání věku.
#13:41 Syn člověka pošle své anděly, ti vyberou z jeho království každé pohoršení a každého, kdo se dopouští nepravosti,
#13:42 a hodí je do ohnivé pece; tam bude pláč a skřípění zubů.
#13:43 Tehdy spravedliví zazáří jako slunce v království svého Otce. Kdo má uši, slyš!“
#13:44 „Království nebeské je jako poklad ukrytý v poli, který někdo najde a skryje; z radosti nad tím jde, prodá všecko, co má, a koupí to pole.
#13:45 Anebo je království nebeské jako když obchodník, který kupuje krásné perly,
#13:46 objeví jednu drahocennou perlu; jde, prodá všecko, co má, a koupí ji.
#13:47 Anebo je království nebeské jako síť, která se spustí do moře a zahrne všecko možné;
#13:48 když je plná, vytáhnou ji na břeh, sednou, a co je dobré, vybírají do nádob, co je špatné, vyhazují ven.
#13:49 Tak bude i při skonání věku: vyjdou andělé, oddělí zlé od spravedlivých
#13:50 a hodí je do ohnivé pece; tam bude pláč a skřípění zubů.“
#13:51 „Pochopili jste to všecko?“ Odpověděli: „Ano.“
#13:52 A on jim řekl: „Proto každý zákoník, který se stal učedníkem království nebeského, je jako hospodář, který vynáší ze svého pokladu nové i staré.“
#13:53 Když Ježíš dokončil tato podobenství, odebral se odtud.
#13:54 Přišel do svého domova a učil v jejich synagóze, takže v úžasu říkali: „Odkud se u toho člověka vzala taková moudrost a mocné činy?
#13:55 Což to není syn tesaře? Což se jeho matka nejmenuje Maria a jeho bratři Jakub, Josef, Šimon a Juda?
#13:56 A nejsou všechny jeho sestry u nás? Odkud to tedy ten člověk všecko má?“
#13:57 A byl jim kamenem úrazu. Ale Ježíš jim řekl: „Prorok není beze cti, leda ve své vlasti a ve svém domě.“
#13:58 A neučinil tam mnoho mocných činů pro jejich nevěru. 
#14:1 V ten čas se tetrarcha Herodes doslechl o Ježíšovi
#14:2 a řekl svým sluhům: „To je Jan Křtitel, který vstal z mrtvých; proto v něm působí mocné síly.“
#14:3 Tento Herodes totiž zatkl Jana a dal ho v poutech vsadit do žaláře kvůli Herodiadě, manželce svého bratra Filipa,
#14:4 neboť Jan mu říkal: „Není dovoleno, abys ji měl za ženu.“
#14:5 Byl by ho rád zbavil života, ale bál se lidu, protože měli Jana za proroka.
#14:6 Na Herodovy narozeniny však dcera té Herodiady tančila uprostřed hostů. Zalíbila se Herodovi,
#14:7 a on jí přísahou slíbil dát, o cokoli požádá.
#14:8 A ona, navedena svou matkou, řekla: „Dej mi sem přinést na míse hlavu Jana Křtitele.“
#14:9 Král se zarmoutil, ale pro přísahu před spolustolovníky poručil, aby jí vyhověli,
#14:10 a dal Jana v žaláři stít.
#14:11 Tak přinesli jeho hlavu na míse, dali ji dívce a ona ji donesla své matce.
#14:12 Janovi učedníci potom přišli, odnesli jeho tělo a pohřbili je; pak šli a oznámili to Ježíšovi.
#14:13 Když to Ježíš uslyšel, odplul lodí na pusté místo, aby byl sám; ale zástupy o tom uslyšely a pěšky šly z měst za ním.
#14:14 Když vystoupil, uviděl velký zástup a bylo mu jich líto. I uzdravoval jejich nemocné.
#14:15 Když nastal večer, přistoupili k němu učedníci a řekli: „Toto místo je pusté a je už pozdní hodina. Propusť zástupy, ať jdou do vesnic kopit si jídlo.“
#14:16 Ale Ježíš jim řekl: „Nemusejí odcházet, dejte vy jim jíst!“
#14:17 Oni odpověděli: „Máme tu jen pět chlebů a dvě ryby.“
#14:18 On však řekl: „Přineste je sem!“
#14:19 Poručil, aby se zástupy rozsadily po trávě. Potom vzal těch pět chlebů a dvě ryby, vzhlédl k nebi, vzdal díky, lámal chleby a dával učedníkům a učedníci zástupům.
#14:20 I jedli všichni a nasytili se; a sebrali nalámaných chlebů, které zbyly, dvanáct plných košů.
#14:21 A jedlo tam na pět tisíc mužů kromě žen a dětí.
#14:22 Hned nato přiměl Ježíš učedníky, aby vstoupili na loď a jeli před ním na druhý břeh, než propustí zástupy.
#14:23 Když je propustil, vystoupil na horu, aby se o samotě modlil. Když nastal večer, byl tam sám.
#14:24 Loď byla daleko od země a vlny ji zmáhaly, protože vítr vál proti ní.
#14:25 K ránu šel k nim, kráčeje po moři.
#14:26 Když ho učedníci viděli kráčet po moři, vyděsili se, že je to přízrak, a křičeli strachem.
#14:27 Ježíš na ně hned promluvil a řekl jim: „Vzchopte se, já jsem to, nebojte se!“
#14:28 Petr mu odpověděl: „Pane, jsi-li to ty, poruč mi, ať přijdu k tobě po vodách!“
#14:29 A on řekl: „Pojď!“ Petr vystoupil z lodi, vykročil na vodu a šel k Ježíšovi.
#14:30 Ale když viděl, jaký je vítr, přepadl ho strach, začal tonout a vykřikl: „Pane, zachraň mne!“
#14:31 Ježíš hned vztáhl ruku, uchopil ho a řekl mu: „Ty malověrný, proč jsi pochyboval?“
#14:32 Když vstoupil na loď, vítr se utišil.
#14:33 Ti, kdo byli na lodi, klaněli se mu a říkali: „Jistě jsi Boží Syn.“
#14:34 Když se dostali na druhý břeh, přistáli u Genezaretu.
#14:35 Lidé z toho místa ho poznali a vzkázali do celého okolí. Přinášeli mu všechny nemocné
#14:36 a prosili ho, aby se směli aspoň dotknout třásní jeho roucha. A kdo se dotkli, byli uzdraveni. 
#15:1 Tehdy přišli k Ježíšovi z Jeruzaléma farizeové a zákoníci a řekli:
#15:2 „Proč tvoji učedníci porušují tradici otců? Vždyť si před jídlem neomývají ruce!“
#15:3 On jim odpověděl: „A proč vy přestupujete přikázání Boží kvůli své tradici?
#15:4 Vždyť Bůh řekl: ‚Cti otce i matku‘ a ‚kdo zlořečí otci nebo matce, ať je potrestán smrtí.‘
#15:5 Vy však učíte: Kdo řekne otci nebo matce: ‚To, čím bych ti měl pomoci, je obětní dar‘,
#15:6 ten již to nemusí dát svému otci nebo matce. A tak jste svou tradicí zrušili slovo Boží.
#15:7 Pokrytci, dobře prorokoval o vás Izaiáš, když řekl:
#15:8 ‚Lid tento ctí mě rty, ale srdce jejich je daleko ode mne;
#15:9 marně mě uctívají, neboť učí naukám, jež jsou jen příkazy lidskými.‘“
#15:10 Svolal zástupy a řekl: „Slyšte a rozumějte:
#15:11 Ne co vchází do úst, znesvěcuje člověka, ale co z úst vychází, to člověka znesvěcuje.“
#15:12 Tu přišli učedníci a řekli mu: „Víš, že se farizeové urazili, když slyšeli to slovo?“
#15:13 Ale on jim odpověděl: „Každá rostlina, kterou nezasadil můj nebeský Otec, bude vykořeněna.
#15:14 Nechte je: slepí vedou slepé. A když vede slepý slepého, oba spadnou do jámy.“
#15:15 Petr mu na to řekl: „Vysvětli nám to podobenství!“
#15:16 On odpověděl: „I vy jste ještě nechápaví?
#15:17 Nerozumíte, že to, co vchází do úst, přijde do břicha a jde do hnoje?
#15:18 Však to, co z úst vychází, jde ze srdce, a to člověka znesvěcuje.
#15:19 Neboť ze srdce vycházejí špatné myšlenky, vraždy, cizoložství, smilství, loupeže, křivá svědectví, urážky.
#15:20 To jsou věci, které člověka znesvěcují; ale jíst neomytýma rukama člověka neznesvěcuje.“
#15:21 Ježíš odtud odešel až do okolí Týru a Sidónu.
#15:22 A hle, jedna kananejská žena z těch končin vyšla a volala: „Smiluj se nade mnou, Synu Davidův! Má dcera je zle posedlá.“
#15:23 Ale on jí neodpověděl ani slovo. Jeho učedníci přistoupili a žádali ho: „Zbav se jí, vždyť za námi křičí!“
#15:24 On odpověděl: „Jsem poslán ke ztraceným ovcím z lidu izraelského.“
#15:25 Ale ona přistoupila, klaněla se mu a řekla: „Pane, pomoz mi!“
#15:26 On jí odpověděl: „Nesluší se vzít chleba dětem a hodit jej psům.“
#15:27 Ona řekla: „Ovšem, Pane, jenže i psi se živí z drobtů, které spadnou ze stolu jejich pánů.“
#15:28 Tu jí Ježíš řekl: „Ženo, tvá víra je veliká; staň se ti tak, jak chceš.“ A od té hodiny byla její dcera zdráva.
#15:29 Odtud Ježíš přešel zase ke Galilejskému moři; vystoupil na horu a posadil se tam.
#15:30 Tu se k němu sešly celé zástupy a měly s sebou chromé, mrzáky, slepé, hluchoněmé a mnohé jiné. Kladli je k jeho nohám a on je uzdravil,
#15:31 takže se zástupy divily, když viděly, že němí mluví, mrzáci jsou zdraví, chromí chodí a slepí vidí; i velebili Boha izraelského.
#15:32 Ježíš si zavolal své učedníky a řekl: „Je mi líto zástupu, neboť již tři dny jsou se mnou a nemají co jíst. Poslat je domů hladové nechci, aby nezemdleli na cestě.“
#15:33 Učedníci mu namítli: „Kde vezmeme na poušti tolik chleba, abychom nasytili takový zástup?“
#15:34 Ježíš jim řekl: „Kolik máte chlebů?“ Odpověděli: „Sedm a několik rybiček.“
#15:35 I nařídil zástupu usednout na zem;
#15:36 potom vzal těch sedm chlebů i ryby, vzdal díky, lámal a dával učedníkům a učedníci zástupům.
#15:37 I jedli všichni a nasytili se; a zbylých nalámaných chlebů sebrali ještě sedm plných košů.
#15:38 Těch, kteří jedli, bylo čtyři tisíce mužů kromě žen a dětí.
#15:39 potom propustil zástupy, vstoupil na loď a připlul na území Magadan. 
#16:1 Přišli saduceové a farizeové a pokoušeli ho; žádali na něm, aby jim ukázal znamení z nebe.
#16:2 On však jim odpověděl: „Večer říkáte: ‚Bude krásně, je pěkný západ.‘
#16:3 A ráno: ‚Dnes bude nečas, slunce vychází do mraků.‘ Vzhled oblohy umíte posoudit a znamení časů nemůžete?
#16:4 Pokolení zlé a zpronevěřilé hledá znamení; ale nebude mu dáno znamení, leč znamení Jonášovo.“ Opustil je a odešel.
#16:5 Když byli učedníci na druhém břehu, shledali, že si zapomněli vzít chleby.
#16:6 Ježíš jim řekl: „Hleďte se mít na pozoru před kvasem saduceů a farizeů!“
#16:7 Oni však u sebe uvažovali: „Nevzali jsme chleba.“
#16:8 Když to Ježíš zpozoroval, řekl: „Proč mluvíte o tom, malověrní, že nemáte chleba?
#16:9 Ještě mi nerozumíte ani se nepamatujete na těch pět chlebů pro pět tisíc, a kolik košů jste sebrali?
#16:10 Ani těch sedm chlebů pro čtyři tisíce, a kolik košů jste sebrali?
#16:11 Což nerozumíte, že jsem k vám nemluvil o chlebech? Mějte se na pozoru před kvasem saduceů a farizeů!“
#16:12 Tehdy pochopili, že jim neřekl, aby se měli na pozoru před kvasem, nýbrž před učením farizeů a saduceů.
#16:13 Když Ježíš přišel do končin Cesareje Filipovy, ptal se svých učedníků: „Za koho lidé pokládají Syna člověka?“
#16:14 Oni řekli: „Jedni za Jana Křtitele, druzí za Eliáše, jiní za Jeremiáše nebo za jednoho z proroků.“
#16:15 Řekl jim: „A za koho mne pokládáte vy?“
#16:16 Šimon Petr odpověděl: „Ty jsi Mesiáš, Syn Boha živého.“
#16:17 Ježíš mu odpověděl: „Blaze tobě, Šimone Jonášův, protože ti to nezjevilo tělo a krev, ale můj Otec v nebesích.
#16:18 A já ti pravím, že ty jsi Petr; a na té skále zbuduji svou církev a brány pekel ji nepřemohou.
#16:19 Dám ti klíče království nebeského, a co odmítneš na zemi, bude odmítnuto v nebi, a co přijmeš na zemi, bude přijato v nebi.“
#16:20 Tehdy nařídil učedníkům, aby nikomu neříkali, že je Mesiáš.
#16:21 Od té doby začal Ježíš ukazovat svým učedníkům, že musí jít do Jeruzaléma a mnoho trpět od starších, velekněží a zákoníků, být zabit a třetího dne vzkříšen.
#16:22 Petr si ho vzal stranou a začal ho kárat: „Buď toho uchráněn, Pane, to se ti nemůže stát!“
#16:23 Ale on se obrátil a řekl Petrovi: „Jdi mi z cesty, satane! Jsi mi kamenem úrazu, protože tvé smýšlení není z Boha, ale z člověka!“
#16:24 Tehdy řekl Ježíš svým učedníkům: „Kdo chce jít za mnou, zapři sám sebe, vezmi svůj kříž a následuj mne.
#16:25 Neboť kdo by chtěl zachránit svůj život, ten o něj přijde; kdo však ztratí svůj život pro mne, nalezne jej.
#16:26 Jaký prospěch bude mít člověk, získá-li celý svět, ale svůj život ztratí? A zač získá člověk svůj život zpět?
#16:27 Syn člověka přijde v slávě svého Otce se svými svatými anděly, a tehdy odplatí každému podle jeho jednání.
#16:28 Amen, pravím vám, že někteří z těch, kteří tu stojí, neokusí smrti, dokud nespatří Syna člověka přicházejícího se svým královstvím.“ 
#17:1 Po šesti dnech vzal s sebou Ježíš Petra a Jakuba a jeho bratra Jana a vyvedl je na vysokou horu, kde byli sami.
#17:2 A byl proměněn před jejich očima; jeho tvář zářila jako slunce a jeho šat byl oslnivě bílý.
#17:3 A hle, zjevil se jim Mojžíš a Eliáš, jak s ním rozmlouvají.
#17:4 Nato promluvil Petr a řekl Ježíšovi: „Pane, je dobré, že jsme zde; chceš-li, udělám tu tři stany, jeden tobě, jeden Mojžíšovi a jeden Eliášovi.“
#17:5 Ještě nedomluvil, a hle, světlý oblak je zastínil a z oblaku promluvil hlas: „Toto jest můj milovaný Syn, kterého jsem si vyvolil; toho poslouchejte.“
#17:6 Když to učedníci uslyšeli, padli tváří k zemi a velmi se báli.
#17:7 Ale Ježíš k nim přistoupil, dotkl se jich a řekl: „Vstaňte a nebojte se.“
#17:8 Oni pozvedli oči a neviděli už nikoho jiného, než Ježíše samotného.
#17:9 Když sestupovali s hory, přikázal jim Ježíš: „Nikomu o tom vidění neříkejte, dokud Syn člověka nebude vzkříšen z mrtvých.“
#17:10 Učedníci se ho ptali: „Jak to, že říkají zákoníci, že napřed musí přijít Eliáš?“
#17:11 On jim odpověděl: „Ano, Eliáš přijde a obnoví všecko.
#17:12 Avšak pravím vám, že Eliáš již přišel, ale nepoznali ho a udělali s ním, co se jim zlíbilo; tak i syn člověka bude od nich trpět.“
#17:13 Tehdy učedníci pochopili, že mluvil o Janu Křtiteli.
#17:14 Když přišli k zástupu, přistoupil k němu jeden člověk a na kolenou prosil:
#17:15 „Pane, smiluj se nad mým synem, neboť je náměsíčný a je na tom zle: často padá do ohně a často do vody.
#17:16 A přivedl jsem ho k tvým učedníkům a nemohli ho uzdravit.“
#17:17 Ježíš odpověděl: „Pokolení nevěřící a zvrácené, jak dlouho ještě budu s vámi? Jak dlouho vás ještě mám snášet? Přiveďte mi ho sem!“
#17:18 Ježíš mu pohrozil, a zlý duch z něho vyšel; od té chvíle byl chlapec zdráv.
#17:19 Když byli učedníci s Ježíšem sami, přistoupili k němu a řekli: „Proč jsme ho nemohli vyhnat my?“
#17:20 On jim řekl: „Pro vaši malověrnost! Amen, pravím vám, budete-li mít víru jako zrnko hořčice, řeknete této hoře: ‚Přejdi odtud tam‘, a přejde; a nic vám nebude nemožné.“
#17:21 Takový duch nevyjde jinak než modlitbou a postem.
#17:22 Když byli spolu v Galileji, řekl jim Ježíš: „Syn člověka bude vydán do rukou lidí;
#17:23 zabijí ho, a třetí den bude vzkříšen.“ Velice se zarmoutili.
#17:24 Když přišli do Kafarnaum, přistoupili k Petrovi výběrčí chrámové daně a řekli: „Váš Mistr neplatí chrámovou daň?“
#17:25 On řekl: „Platí!“ Když přišel domů, ještě než promluvil, řekl mu Ježíš: „Co myslíš, Šimone, od koho vybírají pozemští králové poplatky a daně? Od svých synů nebo od cizích lidí?“
#17:26 Když odpověděl: „Od cizích“, pravil mu Ježíš: „Synové jsou tedy svobodni.
#17:27 Ale abychom je nepohoršili, jdi k moři, hoď udici; vytáhni rybu, která se první chytí, otevři jí ústa a najdeš peníz; ten vezmi a dej jim za mne i za sebe.“ 
#18:1 V tu hodinu přišli učedníci k Ježíšovi s otázkou: „Kdo je vlastně největší v království nebeském?“
#18:2 Ježíš zavolal dítě, postavil je doprostřed
#18:3 a řekl: „Amen, pravím vám, jestliže se neobrátíte a nebudete jako děti, nevejdete do království nebeského.
#18:4 Kdo se pokoří a bude jako toto dítě, ten je největší v království nebeském.
#18:5 A kdo přijme jediné takové dítě ve jménu mém, přijímá mne.
#18:6 Kdo by svedl k hříchu jednoho z těchto nepatrných, kteří ve mne věří, pro toho by bylo lépe, aby mu pověsili na krk mlýnský kámen a potopili ho do mořské hlubiny.
#18:7 Běda světu, že svádí k hříchu! Svody sice nutně přicházejí, ale běda tomu, skrze koho přijdou.
#18:8 Jestliže tě tvá ruka nebo noha svádí k hříchu, utni ji a odhoď pryč; lépe je pro tebe, vejdeš-li do života zmrzačený nebo chromý, než abys byl s oběma rukama či nohama uvržen do věčného ohně.
#18:9 Jestliže tě tvé oko svádí k hříchu, vyrvi je a odhoď pryč; lépe je pro tebe, vejdeš-li do života jednooký, než abys byl s oběma očima uvržen do ohnivého pekla.
#18:10 Mějte se na pozoru, abyste nepohrdali ani jedním z těchto maličkých. Pravím vám, že jejich andělé v nebi jsou neustále v blízkosti mého nebeského Otce.
#18:11 Vždyť Syn člověka přišel spasit, co zahynulo.
#18:12 Co myslíte? Má-li někdo sto ovcí a jedna z nich mu zabloudí, nenechá těch devadesát devět na horách a nejde hledat tu, která zabloudila?
#18:13 A podaří-li se mu ji nalézt, amen, pravím vám, bude se z ní radovat víc než z těch devadesáti devíti, které nezabloudily.
#18:14 Právě tak je vůle vašeho nebeského Otce, aby nezahynul ani jediný z těchto maličkých.
#18:15 Když tvůj bratr zhřeší, jdi a pokárej ho mezi čtyřma očima; dá-li si říci, získal jsi svého bratra.
#18:16 Nedá-li si říci, přiber k sobě ještě jednoho nebo dva, aby ‚ústy dvou nebo tří svědků byla potvrzena každá výpověď‘.
#18:17 Jestliže ani potom neuposlechne, oznam to církvi; jestliže však neuposlechne ani církev, ať je ti jako pohan nebo celník.
#18:18 Amen, pravím vám, cokoli odmítnete na zemi, bude odmítnuto v nebi, a cokoli přijmete na zemi, bude přijato v nebi.
#18:19 Opět vám pravím, shodnou-li se dva z vás na zemi v prosbě o jakoukoli věc, můj nebeský Otec jim to učiní.
#18:20 Neboť kde jsou dva nebo tři shromážděni ve jménu mém, tam jsem já uprostřed nich.“
#18:21 Tehdy přistoupil Petr k Ježíšovi a řekl mu: „Pane, kolikrát mám odpustit svému bratru, když proti mně zhřeší? Snad až sedmkrát?“
#18:22 Ježíš mu na to odpověděl: „Pravím ti, ne sedmkrát, ale až sedmdesát sedmkrát.“
#18:23 „S královstvím nebeským je to tak, jako když se jeden král rozhodl vyžádat účty od svých služebníků.
#18:24 Když začal účtovat, přivedli mu jednoho, který byl dlužen mnoho tisíc hřiven.
#18:25 Protože mu je nemohl vrátit, rozkázal ho pán prodat i s ženou a dětmi a se vším, co měl, a nahradit ztrátu.
#18:26 Tu mu ten služebník padl k nohám a na kolenou prosil: ‚Měj se mnou strpení a všechno ti vrátím!‘
#18:27 Pán se ustrnul na oním služebníkem, propustil ho a dluh mu odpustil.
#18:28 Sotva však ten služebník vyšel, potkal jednoho ze svých spoluslužebníků, který mu byl dlužen sto denárů; chytil ho za krk a křičel: ‚Zaplať mi, co jsi dlužen!‘
#18:29 Jeho spoluslužebník mu padl k nohám a prosil ho: ‚Měj se mnou strpení, a zaplatím ti to!‘
#18:30 On však nechtěl, ale šel a dal ho do vězení, dokud nezaplatí dluh.
#18:31 Když jeho spoluslužebníci viděli, co se přihodilo, velice se zarmoutili; šli a oznámili svému pánu všecko, co se stalo.
#18:32 Tu ho pán zavolal a řekl mu: ‚Služebníku zlý, celý tvůj dluh jsem ti odpustil, když jsi mě prosil;
#18:33 neměl ses také ty smilovat nad svým spoluslužebníkem, jako jsem se já smiloval nad tebou?‘
#18:34 A rozhněval se jeho pán a dal ho do vězení, dokud nezaplatí celý dluh. -
#18:35 Tak bude jednat s vámi můj nebeský Otec, jestliže ze srdce neodpustíte každý svému bratru.“ 
#19:1 Když Ježíš dokončil tato slova, odebral se z Galileje do krajin judských za Jordán.
#19:2 Velké zástupy šly za ním, a on je tam uzdravil.
#19:3 Tu k němu přišli farizeové a pokoušeli ho: „Je dovoleno propustit manželku z jakékoli příčiny?“
#19:4 Odpověděl jim: „Nečetli jste, že Stvořitel od počátku ‚muže a ženu učinil je‘?
#19:5 A řekl: ‚Proto opustí muž otce i matku a připojí se ke své manželce, a budou ti dva jedno tělo‘;
#19:6 takže již nejsou dva, ale jeden. A proto co Bůh spojil, člověk nerozlučuj!“
#19:7 Namítnou mu: „Proč tedy Mojžíš ustanovil, že muž smí propustit svou manželku tím, že jí dá rozlukový lístek?“
#19:8 Odpoví jim: „Pro tvrdost vašeho srdce vám Mojžíš dovolil propustit manželku. Od počátku to však nebylo.
#19:9 Pravím vám, kdo propustí svou manželku z jiného důvodu než pro smilstvo a vezme si jinou, cizoloží.“
#19:10 Učedníci mu řekli: „Jestliže je to s mužem a ženou takové, pak je lépe se neženit.“
#19:11 On jim odpověděl: „Ne všichni pochopí to slovo; jen ti, kterým je dáno.
#19:12 Někteří nežijí v manželství, protože jsou k tomu od narození nezpůsobilí; jiní nežijí v manželství, protože je nezpůsobilými učinili lidé; a někteří nežijí v manželství, protože se ho zřekli pro království nebeské. Kdo to může pochopit, pochop.“
#19:13 Tehdy k němu přinášeli děti, aby na ně vložil ruce a pomodlil se; ale učedníci jim to zakazovali.
#19:14 Ježíš však řekl: „Nechte děti a nebraňte jim jít ke mně; neboť takovým patří království nebeské.“
#19:15 Požehnal jim a šel dál.
#19:16 A hle, kdosi k němu přišel a zeptal se ho: „Mistře, co dobrého mám udělat, abych získal věčný život?“
#19:17 On mu řekl: „Proč se mě ptáš na dobré? Jediný je dobrý! A chceš-li vejít do života, zachovávej přikázání!“
#19:18 Otázal se ho: „Která?“ Ježíš odpověděl: „Nebudeš zabíjet, cizoložit, krást, křivě svědčit,
#19:19 cti otce a matku, miluj svého bližního jako sám sebe.“
#19:20 Mladík mu řekl: „To jsem všechno dodržoval! Co mi ještě schází?“
#19:21 Ježíš mu odpověděl: „Chceš-li být dokonalý, jdi, prodej, co ti patří, rozdej chudým, a budeš mít poklad v nebi; pak přijď a následuj mne.“
#19:22 Když mladík uslyšel to slovo, smuten odešel, neboť měl mnoho majetku.
#19:23 Ježíš řekl svým učedníkům: „Amen, pravím vám, že bohatý těžko vejde do království nebeského.
#19:24 Znovu vám říkám, snáze projde velbloud uchem jehly než bohatý do Božího království.“
#19:25 Když to učedníci slyšeli, velice se zhrozili a řekli: „Kdo potom může být spasen?“
#19:26 Ježíš na ně pohleděl a řekl: „U lidí je to nemožné, ale u Boha je možné všecko.“
#19:27 Na to mu řekl Petr: „Hle, my jsme opustili všecko a šli jsme za tebou! Co tedy budeme mít?“
#19:28 Ježíš jim řekl: „Amen, pravím vám, až se Syn člověka při obnovení všeho posadí na trůn své slávy, tehdy i vy, kteří jste mě následovali, usednete na dvanáct trůnů a budete soudit dvanáct pokolení Izraele.
#19:29 A každý, kdo opustil domy nebo bratry nebo sestry nebo otce nebo matku nebo děti nebo pole pro mé jméno, stokrát víc dostane a bude mít podíl na věčném životě.
#19:30 Mnozí první budou poslední a poslední první.“ 
#20:1 „Neboť s královstvím nebeským je to tak, jako když jeden hospodář hned ráno vyšel najmout dělníky na svou vinici.
#20:2 Smluvil s dělníky denár na den a poslal je na vinici.
#20:3 Když znovu vyšel o deváté hodině, viděl, jak jiní stojí nečinně na trhu,
#20:4 a řekl jim: ‚Jděte i vy na mou vinici, a já vám dám, co bude spravedlivé.‘
#20:5 Oni šli. Vyšel opět kolem poledne i kolem třetí hodiny odpoledne a učinil právě tak.
#20:6 Když vyšel kolem páté hodiny odpoledne, našel tam další, jak tam stojí, a řekl jim: ‚Co tu stojíte celý den nečinně?‘
#20:7 Odpověděli mu: ‚Nikdo nás nenajal.‘ On jim řekne: ‚Jděte i vy na mou vinici.‘
#20:8 Když byl večer, řekl pán vinice svému správci: ‚Zavolej dělníky a vyplať jim mzdu, a to od posledních k prvním!‘
#20:9 Tak přišli ti, kteří pracovali od pěti odpoledne, a každý dostal denár.
#20:10 Když přišli ti první, měli za to, že dostanou víc; ale i oni dostali po denáru.
#20:11 Vzali ho a reptali proti hospodáři:
#20:12 ‚Tihle poslední dělali jedinou hodinu, a tys jim dal stejně jako nám, kteří jsme nesli tíhu dne a vedro!‘
#20:13 On však odpověděl jednomu z nich: ‚Příteli, nekřivdím ti! Nesmluvil jsi se mnou denár za den?
#20:14 Vezmi si, co ti patří a jdi! Já chci tomu poslednímu dát jako tobě;
#20:15 nemohu si se svým majetkem udělat, co chci? Nebo snad tvé oko závidí, že jsem dobrý?‘
#20:16 Tak budou poslední první a první poslední.“
#20:17 Když Ježíš šel do Jeruzaléma, vzal si stranou dvanáct učedníků a cestou jim řekl:
#20:18 „Hle, jdeme do Jeruzaléma a Syn člověka bude vydán velekněžím a zákoníkům; odsoudí ho na smrt
#20:19 a vydají pohanům, aby se mu posmívali, zbičovali ho a ukřižovali; a třetího dne bude vzkříšen.“
#20:20 Tehdy k němu přistoupila matka synů Zebedeových se svými syny, klaněla se mu a chtěla ho o něco požádat.
#20:21 On jí řekl: „Co chceš?“ Řekla: „Ustanov, aby tito dva synové měli místo jeden po tvé pravici a jeden po tvé levici ve tvém království.“
#20:22 Ježíš však odpověděl: „Nevíte, oč žádáte. Můžete pít kalich, který já mám pít?“ Řekli mu: „Můžeme.“
#20:23 Praví jim: „Můj kalich budete pít, ale udělovat místa po mé pravici či levici není má věc; ta místa patří těm, jimž je připravil můj Otec.“
#20:24 Když to uslyšelo ostatních deset, rozmrzeli se na oba bratry.
#20:25 Ale Ježíš si je zavolal a řekl: „Víte, že vládcové panují na národy a velicí je utlačují.
#20:26 Ne tak bude mezi vámi: kdo se mezi vámi chce stát velkým, buď vaším služebníkem;
#20:27 a kdo chce být mezi vámi první, buď vaším otrokem.
#20:28 Tak, jako Syn člověka nepřišel, aby si dal sloužit, ale aby sloužil a dal svůj život jako výkupné za mnohé.“
#20:29 Když vycházeli z Jericha, následoval ho velký zástup.
#20:30 A hle, dva slepí seděli u cesty; když uslyšeli, že jde kolem Ježíš, začali křičet: „Smiluj se nad námi, Pane, Synu Davidův!“
#20:31 Zástup je napomínal, aby mlčeli, ale oni křičeli ještě víc: „Smiluj se nad námi, Pane, Synu Davidův!“
#20:32 Ježíš se zastavil, zavolal je a řekl jim: „Co chcete, abych pro vás učinil?“
#20:33 Odpověděli mu: „Pane, ať se otevřou naše oči!“
#20:34 Ježíš pohnut soucitem, dotkl se jejich očí, a hned prohlédli; a šli za ním. 
#21:1 Když se přiblížili k Jeruzalému a přišli do Betfage na Olivové hoře, poslal Ježíš dva učedníky
#21:2 a řekl jim: „Jděte do vesnice, která je před vámi, a hned naleznete přivázanou oslici a u ní oslátko. Odvažte je a přiveďte ke mně.
#21:3 A kdyby vám někdo něco říkal, odpovězte: ‚Pán je potřebuje.‘ A ten člověk je hned pošle.“
#21:4 To se stalo, aby se splnilo, co je řečeno ústy proroka:
#21:5 ‚Povězte dceři siónské: Hle, král tvůj přichází k tobě, tichý a sedící na oslici, na oslátku té, která je podrobena jhu.‘
#21:6 Učedníci šli a učinili, co jim Ježíš přikázal.
#21:7 Přivedli oslici i oslátko, položili na ně pláště a on se na ně posadil.
#21:8 A mohutný zástup prostíral na cestu své pláště, jiní odsekával ratolesti stromů a stlali je na cestu.
#21:9 Zástupy, které šly před ním i za ním, volaly: „Hosanna Synu Davidovu! Požehnaný, který přichází ve jménu Hospodinově! Hosanna na výsostech!“
#21:10 Když vjel Ježíš do Jeruzaléma, po celém městě nastal rozruch; ptali se: „Kdo to je?“
#21:11 Zástupy odpovídaly: „To je ten prorok Ježíš z Nazareta v Galileji.“
#21:12 Ježíš vešel do chrámu a vyhnal prodavače a kupující v nádvoří, zpřevracel stoly směnárníků i stánky prodavačů holubů;
#21:13 řekl jim: „Je psáno: ‚Můj dům bude zván domem modlitby‘, ale vy z něho děláte doupě lupičů.“
#21:14 I přistoupili k němu v chrámě slepí a chromí a on je uzdravil.
#21:15 Když velekněží a zákoníci viděli jeho udivující činy i děti volající v chrámě „Hosanna Synu Davidovu“, rozhněvali se
#21:16 a řekli mu: „Slyšíš, co to říkají?“ Ježíš jim odpověděl: „Ovšem! Nikdy jste nečetli: ‚Z úst nemluvňátek a kojenců připravil sis chválu‘?“
#21:17 Opustil je a vyšel ven z města do Betanie; tam přenocoval.
#21:18 Když se ráno vracel do města, dostal hlad.
#21:19 Spatřil u cesty fíkovník a šel k němu; ale nic na něm nenalezl, jen listí. I řekl mu: „Ať se na tobě na věky neurodí ovoce!“ A ten fíkovník najednou uschl.
#21:20 Když to učedníci viděli, podivli se: „Jak najednou ten fíkovník uschl!“
#21:21 Ježíš jim odpověděl: „Amen, pravím vám, budete-li mít víru a nebudete pochybovat, učiníte nejen to, co se stalo s fíkovníkem; ale i kdybyste této hoře řekli: ‚Zdvihni se a vrhni se do moře‘ - stane se to.
#21:22 A věříte-li, dostanete všecko, oč budete v modlitbě prosit.“
#21:23 Když Ježíš přišel do chrámu a učil, přistoupili k němu velekněží a starší lidu a řekli: „Jakou mocí to činíš? A kdo ti tu moc dal?“
#21:24 Ježíš jim odpověděl: „Já vám také položím otázku; jestliže ji zodpovíte, i já vám povím, jakou mocí to činím.
#21:25 Odkud měl Jan pověření křtít? Z nebe, či od lidí?“ Oni se mezi sebou dohadovali: „Řekneme-li, ‚z nebe‘, namítne nám: ‚Proč jste mu tedy neuvěřili?‘
#21:26 Řekneme-li však ‚z lidí‘, máme obavy ze zástupu; vždyť všichni mají Jana za proroka.“
#21:27 Odpověděli tedy Ježíšovi: „Nevíme.“ Tu jim řekl i on: „Ani já vám nepovím, jakou mocí to činím.“
#21:28 „Co myslíte? Jeden člověk měl dva syny. Přišel a řekl prvnímu: ‚Synu, jdi dnes pracovat na vinici!‘
#21:29 On odpověděl: ‚Nechce se mi.‘ Ale potom toho litoval a šel.
#21:30 Otec přišel k druhému a řekl mu totéž. Ten odpověděl: ‚Ano, pane.‘ Ale nešel.
#21:31 Kdo z těch dvou splnil vůli svého otce?“ Odpověděli: „Ten první!“ Ježíš jim řekl: „Amen, pravím vám, že celníci a nevěstky předcházejí vás do Božího království.
#21:32 Přišel k vám Jan po cestě spravedlnosti, a vy jste mu neuvěřili. Ale celníci a nevěstky mu uvěřili. Vy jste to viděli, ale ani potom jste toho nelitovali a neuvěřili mu.
#21:33 Poslyšte jiné podobenství: Jeden hospodář vysadil vinici, obehnal ji zdí, vykopal v ní lis a vystavěl strážní věž; potom vinici pronajal vinařům a odcestoval.
#21:34 Když se přiblížil čas vinobraní, poslal své služebníky k vinařům, aby převzali jeho díl úrody.
#21:35 Ale vinaři jeho služebníky chytili, jednoho zbili, druhého zabili, dalšího ukamenovali.
#21:36 Znovu poslal další služebníky, a to více než před tím, ale naložili s nimi právě tak.
#21:37 Nakonec k nim poslal svého syna; řekl si: ‚Na mého syna budou mít přece ohled!‘
#21:38 Když však vinaři shlédli syna, řekli si mezi sebou: ‚To je dědic. Pojďme, zabijme ho, a dědictví připadne nám!‘
#21:39 Chytili ho, vyvlekli ven z vinice a zabili.
#21:40 Když nyní přijde pán vinice, co udělá těm vinařům?“
#21:41 Řekli mu: „Zlé bez milosti zahubí a vinici pronajme jiným vinařům, kteří mu budou odvádět výnos v určený čas.“
#21:42 Ježíš jim řekl: „Což jste nikdy nečetli v Písmech: ‚Kámen, který stavitelé zavrhli, stal se kamenem úhelným; Hospodin to učinil a je to podivuhodné v našich očích‘?
#21:43 Proto vám pravím, že vám Boží království bude odňato a bude dáno národu, který ponese jeho ovoce.
#21:44 Kdo padne na ten kámen, roztříští se, a na koho on padne, toho rozdrtí.“
#21:45 Když slyšeli velekněží a farizeové tato podobenství, poznali, že mluví o nich.
#21:46 Hleděli se ho zmocnit, ale báli se zástupů, protože ty ho měly za proroka. 
#22:1 A Ježíš k nim znovu mluvil v podobenstvích:
#22:2 „S královstvím Božím je to tak, jako když jeden král vystrojil svatbu svému synu.
#22:3 Poslal služebníky, aby přivedli pozvané na svatbu, ale oni nechtěli jít.
#22:4 Poslal znovu jiné služebníky se slovy: ‚Řekněte pozvaným: Hle, hostinu jsem uchystal, býčci a krmný dobytek je poražen, všechno je připraveno; pojďte na svatbu!‘
#22:5 Ale oni nedbali a odešli, jeden na své pole, druhý za svým obchodem.
#22:6 Ostatní chytili jeho služebníky, potupně je ztýrali a zabili je.
#22:7 Tu se král rozhněval, poslal svá vojska, vrahy zahubil a jejich město vypálil.
#22:8 Potom řekl svým služebníkům: ‚Svatba je připravena, ale pozvaní nebyli jí hodni;
#22:9 jděte tedy na rozcestí a koho najdete, pozvěte na svatbu.‘
#22:10 Služebníci vyšli na cesty a shromáždili všechny, které nalezli, zlé i dobré; a svatební síň se naplnila stolovníky.
#22:11 Když král vstoupil mezi stolovníky, spatřil tam člověka, který nebyl oblečen na svatbu.
#22:12 Řekl mu: ‚Příteli, jak ses sem dostal, když nejsi oblečen na svatbu?‘ On se nezmohl ani na slovo.
#22:13 Tu řekl král sloužícím: ‚Svažte mu ruce i nohy a uvrhněte ho ven do temnot; tam bude pláč a skřípění zubů.‘
#22:14 Neboť mnozí jsou pozváni, ale málokdo bude vybrán.“
#22:15 Tehdy farizeové šli a radili se, jak by Ježíšovi nějakým slovem nastražili léčku.
#22:16 Poslali za ním své učedníky s herodiány, aby řekli: „Mistře, víme, že jsi pravdivý a učíš cestě Boží podle pravdy; na nikoho se neohlížíš a nebereš ohled na postavení člověka.
#22:17 Pověz nám tedy, co myslíš: Je dovoleno dávat daň císaři, nebo ne?“
#22:18 Ale Ježíš poznal jejich zlý úmysl a řekl: „Co mě pokoušíte, pokrytci?
#22:19 „Ukažte mi peníz daně!“ Podali mu denár.
#22:20 On jim řekl: „Čí je tento obraz a nápis?“
#22:21 Odpověděli: „Císařův.“ Tu jim řekl: „Odevzdejte tedy, co je císařovo, císaři, co je Boží, dejte Bohu.“
#22:22 Když to slyšeli, podivili se, nechali ho a odešli.
#22:23 V ten den přišli za ním saduceové, kteří říkají, že není vzkříšení, a předložili mu dotaz:
#22:24 „Mistře, Mojžíš řekl: ‚Zemře-li někdo bez dětí, ať se s jeho manželkou podle řádu švagrovství ožení jeho bratr a zplodí svému bratru potomka.‘
#22:25 U nás bylo sedm bratří. První po svatbě zemřel, a protože neměl potomka, zanechal svou ženu svému bratru.
#22:26 Stejně i druhý, třetí a nakonec všech sedm.
#22:27 Naposledy ze všech zemřela ta žena.
#22:28 Až bude vzkříšení, komu z těch sedmi bude patřit? Vždyť ji měli všichni!“
#22:29 Ježíš jim odpověděl: „Mýlíte se, neznáte Písma ani moc Boží.
#22:30 Po vzkříšení se lidé nežení ani nevdávají, ale jsou jako nebeští andělé.
#22:31 A pokud jde o vzkříšení mrtvých, nečetli jste, co vám Bůh pravil:
#22:32 ‚Já jsem Bůh Abrahamův, Bůh Izákův a Bůh Jákobův?‘ On přece není Bohem mrtvých, nýbrž živých.“
#22:33 Když to slyšely zástupy, žasly nad jeho učením.
#22:34 Když se farizeové doslechli, že umlčel saduceje, smluvili se
#22:35 a jeden jejich zákoník se ho otázal, aby ho pokoušel:
#22:36 „Mistře, které přikázání v zákoně je největší?“
#22:37 On mu řekl: „‚Miluj Hospodina, Boha svého, celým svým srdcem, celou svou duší a celou svou myslí.‘
#22:38 To je největší a první přikázání.
#22:39 Druhé je mu podobné: ‚Miluj svého bližního jako sám sebe.‘
#22:40 Na těch dvou přikázáních spočívá celý Zákon i Proroci.“
#22:41 Když se farizeové sešli, zeptal se jich Ježíš:
#22:42 „Co si myslíte o Mesiášovi? Čí je syn?“ Odpověděli mu: „Davidův.“
#22:43 Řekl jim: „Jak to tedy, že ho David v Duchu svatém nazývá Pánem, když praví:
#22:44 ‚Řekl Hospodin mému Pánu: Usedni po mé pravici, dokud ti nepoložím nepřátele pod nohy.‘
#22:45 Jestliže tedy David nazývá Mesiáše Pánem, jak může být jeho synem?“
#22:46 A nikdo nebyl s to odpovědět mu ani slovo; od toho dne se ho již nikdo neodvážil tázat. 
#23:1 Tehdy Ježíš mluvil k zástupům i k svým učedníkům:
#23:2 „Na stolici Mojžíšově zasedli zákoníci a farizeové.
#23:3 Proto čiňte a zachovávejte všechno, co vám řeknou; ale podle jejich skutků nejednejte: neboť oni mluví a nečiní.
#23:4 Svazují těžká břemena a nakládají je lidem na ramena, ale sami se jich nechtějí dotknout ani prstem.
#23:5 Všechny své skutky konají tak, aby je lidé viděli: rozšiřují si modlitební řemínky a prodlužují třásně,
#23:6 mají rádi přední místa na hostinách a přední sedadla v synagógách,
#23:7 líbí se jim, když je lidé na ulici zdraví a říkají jim ‚Mistře‘.
#23:8 Vy však si nedávejte říkat ‚Mistře‘: jediný je váš Mistr, vy všichni jste bratří.
#23:9 A nikomu na zemi nedávejte jméno ‚Otec‘: jediný je váš Otec, ten nebeský.
#23:10 Ani si nedávejte říkat ‚Učiteli‘: váš učitel je jeden, Kristus.
#23:11 Kdo je z vás největší, bude váš služebník.
#23:12 Kdo se povyšuje, bude ponížen, a kdo se ponižuje, bude povýšen.
#23:13 Běda vám, zákoníci a farizeové, pokrytci! Zavíráte lidem království nebeské, sami nevcházíte a zabraňujete těm, kdo chtějí vejít.
#23:14 Běda vám, zákoníci a farizeové, pokrytci! Vyjídáte domy vdov pod záminkou dlouhých modliteb; proto vás postihne tím přísnější soud.
#23:15 Běda vám, zákoníci a farizeové, pokrytci! Obcházíte moře i zemi, abyste získali jednoho novověrce; a když ho získáte, učiníte z něho syna pekla, dvakrát horšího, než jste sami.
#23:16 Běda vám, slepí vůdcové! Říkáte: ‚Kdo by přísahal u chrámu, nic to neznamená; ale kdo by přísahal při chrámovém zlatu, je vázán.‘
#23:17 Blázni a slepci, co je větší: zlato, nebo chrám, který to zlato posvěcuje?
#23:18 Nebo: ‚Kdo by přísahal při oltáři, nic to neznamená; ale kdo by přísahal při oběti na něm, je vázán.‘
#23:19 Slepci, co je větší: oběť nebo oltář, který tu oběť posvěcuje?
#23:20 Kdo tedy přísahá při oltáři, přísahá při něm a při všem, co na něm leží.
#23:21 A kdo přísahá při chrámu, přísahá při něm a při tom, kdo v něm přebývá.
#23:22 Kdo přísahá při nebi, přísahá při trůnu Božím a při tom, který na něm sedí.
#23:23 Běda vám, zákoníci a farizeové, pokrytci! Odevzdáváte desátky z máty, kopru a kmínu, a nedbáte na to, co je v Zákoně důležitější: právo, milosrdenství a věrnost. Toto bylo třeba činit a to ostatní nezanedbávat.
#23:24 Slepí vůdcové, cedíte komára, ale velblouda spolknete!
#23:25 Běda vám, zákoníci a farizeové, pokrytci! Očišťujete číše a talíře zvenčí, ale uvnitř jsou plné hrabivosti a chtivosti.
#23:26 Slepý farizeji, vyčisť především vnitřek číše, a bude čistý i vnějšek.
#23:27 Běda vám, zákoníci a farizeové, pokrytci! Podobáte se obíleným hrobům, které zvenčí vypadají pěkně, ale uvnitř jsou plné lidských kostí a všelijaké nečistoty.
#23:28 Tak i vy se navenek zdáte lidem spravedliví, ale uvnitř jste samé pokrytectví a nepravost.
#23:29 Běda vám, zákoníci a farizeové, pokrytci! Stavíte náhrobky prorokům, zdobíte pomníky spravedlivých
#23:30 a říkáte: ‚Kdybychom my byli na místě svých otců, neměli bychom podíl na smrti proroků.‘
#23:31 Tak svědčíte sami proti sobě, že jste synové těch, kteří zabíjeli proroky.
#23:32 Dovršte tedy činy svých otců!
#23:33 Hadi, plemeno zmijí, jak uniknete pekelnému trestu?
#23:34 Hle, proto vám posílám proroky a učitele moudrosti i zákoníky; a vy je budete zabíjet a křižovat, budete je bičovat ve svých synagógách a pronásledovat z místa na místo,
#23:35 aby na vás padla všechna spravedlivá krev prolitá na zemi, od krve spravedlivého Ábela až po krev Zachariáše syna Bachariášova, kterého jste zabili mezi chrámem a oltářem.
#23:36 Amen, pravím vám, to vše padne na toto pokolení.
#23:37 Jeruzaléme, Jeruzaléme, který zabíjíš proroky a kamenuješ ty, kdo byli k tobě posláni; kolikrát jsem chtěl shromáždit tvé děti, tak jako kvočna shromažďuje kuřátka pod svá křídla, a nechtěli jste!
#23:38 Hle, váš dům se vám ponechává pustý.
#23:39 Neboť vám pravím, že mě neuzříte od nynějška až do chvíle, kdy řeknete: ‚Požehnaný, který přichází ve jménu Hospodinově.‘“ 
#24:1 Když Ježíš vyšel z chrámu a odcházel odtud, přistoupili k němu učedníci a ukazovali mu chrámové stavby.
#24:2 On však jim řekl: „Vidíte toto všechno? Amen, pravím vám, že tu nezůstane kámen na kameni, všecko bude rozmetáno.“
#24:3 Když seděl na Olivové hoře a byli sami, přistoupili k němu učedníci a řekli: „Pověz nám, kdy to nastane a jaké bude znamení tvého příchodu a skonání věku!“
#24:4 Ježíš jim odpověděl: „Mějte se na pozoru, aby vás někdo nesvedl.
#24:5 Neboť mnozí přijdou v mém jménu a budou říkat ‚já jsem Mesiáš‘ a svedou mnohé.
#24:6 Budete slyšet válečný ryk a zvěsti o válkách; hleďte, abyste se nelekali. Musí to být, ale to ještě není konec.
#24:7 Povstane národ proti národu a království proti království, bude hlad a zemětřesení na mnoha místech.
#24:8 Ale to vše bude teprve začátek bolestí.
#24:9 Tehdy vás budou vydávat v soužení i na smrt a všechny národy vás budou nenávidět pro mé jméno.
#24:10 A tehdy mnozí odpadnou a navzájem se budou zrazovat a jedni druhé nenávidět;
#24:11 povstanou lživí proroci a mnohé svedou
#24:12 a protože se rozmůže nepravost, vychladne láska mnohých.
#24:13 Ale kdo vytrvá až do konce, bude spasen.
#24:14 A toto evangelium o království bude kázáno po celém světě na svědectví všem národům, a teprve potom přijde konec.
#24:15 Když pak uvidíte ‚znesvěcující ohavnost‘, o níž je řeč u proroka Daniele, jak stojí na místě svatém - kdo čteš, rozuměj -
#24:16 tehdy ti, kdo jsou v Judsku, ať uprchnou do hor;
#24:17 kdo je na střeše, ať nesestupuje, aby si vzal něco z domu;
#24:18 a kdo je na poli, ať se nevrací, aby si vzal plášť.
#24:19 Běda těhotným a kojícím v oněch dnech!
#24:20 Modlete se, abyste se nemuseli dát na útěk v zimě nebo v sobotu.
#24:21 Neboť tehdy nastane hrozné soužení, jaké nebylo od počátku světa až do nynějška a nikdy již nebude.
#24:22 A kdyby nebyly ty dny zkráceny, nebyl by spasen žádný člověk; ale kvůli vyvoleným budou tyto dny zkráceny.
#24:23 Tehdy, řekne-li vám někdo: ‚Hle, tu je Mesiáš nebo tam‘, nevěřte!
#24:24 Neboť vyvstanou lžimesiášové a lžiproroci a budou předvádět veliká znamení a zázraky, že by svedli i vyvolené, kdyby to bylo možné.
#24:25 Hle, řekl jsem vám to už předem.
#24:26 Když vám řeknou: ‚Hle, je na poušti‘, nevycházejte! ‚Hle, v tajných úkrytech‘, nevěřte tomu!
#24:27 Neboť jako blesk ozáří oblohu od východu až na západ, takový bude příchod Syna člověka.
#24:28 Kde je mrtvola, slétnou se i supi.
#24:29 Hned po soužení těch dnů se zatmí slunce, měsíc ztratí svou zář, hvězdy budou padat z nebe a mocnosti nebeské se zachvějí.
#24:30 Tehdy se ukáže znamení Syna člověka na nebi; a tu budou lomit rukama všechny čeledi země a uzří Syna člověka přicházet na oblacích nebeských s velkou mocí a slávou.
#24:31 On vyšle své anděly s mohutným zvukem polnice a ti shromáždí jeho vyvolené do čtyř úhlů světa, od jedněch konců nebe ke druhým.
#24:32 Od fíkovníku si vezměte poučení: Když už jeho větev raší a vyráží listí, víte, že je léto blízko.
#24:33 Tak i vy, až toto všecko uvidíte, vězte, že ten čas je blízko, přede dveřmi.
#24:34 Amen, pravím vám, že nepomine toto pokolení, než se to všechno stane.
#24:35 Nebe a země pominou, ale má slova nepominou.
#24:36 O onom dni a hodině však neví nikdo, ani andělé v nebi, ani Syn; jenom Otec sám.
#24:37 Až přijde Syn člověka, bude to jako za dnů Noé:
#24:38 Jako tehdy před potopou hodovali a pili, ženili se a vdávaly až do dne, kdy Noé vešel do korábu,
#24:39 a nic nepoznali, až přišla potopa a zachvátila všecky - takový bude i příchod Syna člověka.
#24:40 Tehdy budou dva na poli, jeden bude přijat a druhý zanechán.
#24:41 Dvě budou mlít obilí, jedna bude přijata a druhá zanechána.
#24:42 Bděte tedy, protože nevíte, v který den váš Pán přijde.
#24:43 Uvažte přece: Kdyby hospodář věděl, v kterou noční hodinu přijde zloděj, bděl by a zabránil by mu vloupat se do domu.
#24:44 Proto i vy buďte připraveni, neboť Syn člověka přijde v hodinu, kdy se nenadějete.
#24:45 Když pán ustanovuje nad svou čeledí služebníka, aby jim včas podával pokrm, který služebník je věrný a rozumný?
#24:46 Blaze tomu služebníku, kterého pán při svém příchodu nalezne, že tak činí.
#24:47 Amen, pravím vám, že ho ustanoví nade vším, co mu patří.
#24:48 Když si však špatný služebník řekne: ‚Můj pán nejde‘,
#24:49 a začne bít své spoluslužebníky, hodovat a pít s opilci,
#24:50 tu pán toho služebníka přijde v den, kdy to nečeká, a v hodinu, kterou netuší,
#24:51 vyžene ho a vykáže mu úděl mezi pokrytci; tam bude pláč a skřípění zubů. 
#25:1 Tehdy bude království nebeské, jako když deset družiček vzalo lampy a vyšlo naproti ženichovi.
#25:2 Pět z nich bylo pošetilých a pět rozumných.
#25:3 Pošetilé vzaly lampy, ale nevzaly si s sebou olej.
#25:4 Rozumné si vzaly s lampami i olej v nádobkách.
#25:5 Když ženich nepřicházel, na všechny přišla ospalost a usnuly.
#25:6 Uprostřed noci se rozlehl křik: ‚Ženich je tu, jděte mu naproti!‘
#25:7 Všechny družičky procitly a dávaly do pořádku své lampy.
#25:8 Tu řekly pošetilé rozumným: ‚Dejte nám trochu oleje, naše lampy dohasínají!‘
#25:9 Ale rozumné odpověděly: ‚Nemůžeme, nedostávalo by se nám ani vám. Jděte raději ke kupcům a kupte si!‘
#25:10 Ale zatímco šly kupovat, přišel ženich, a které byly připraveny, vešly s ním na svatbu; a dveře byly zavřeny.
#25:11 Potom přišly i ty ostatní družičky a prosily: ‚Pane, pane, otevři nám!‘
#25:12 Ale on odpověděl: ‚Amen, pravím vám, neznám vás.‘
#25:13 Bděte tedy, protože neznáte den ani hodinu.
#25:14 Bude tomu, jako když člověk, který se chystal na cestu, zavolal své služebníky a svěřil jim svůj majetek;
#25:15 jednomu dal pět hřiven, druhému dvě a třetímu jednu, každému podle jeho schopností, a odcestoval.
#25:16 Ten, který přijal pět hřiven, ihned se s nimi dal do podnikání a vyzískal jiných pět.
#25:17 Tak i ten, který měl dvě, vyzískal jiné dvě.
#25:18 Ten, který přijal jednu, šel, vykopal jámu a ukryl peníze svého pána.
#25:19 Po dlouhé době se pán těch služebníků vrátil a začal účtovat.
#25:20 Přistoupil tedy ten, který přijal pět hřiven, přinesl jiných pět a řekl: ‚Pane, svěřil jsi mi pět hřiven; hle, jiných pět jsem jimi získal.‘
#25:21 Jeho pán mu odpověděl: ‚Správně, služebníku dobrý a věrný, nad málem jsi byl věrný, ustanovím tě nad mnohým; vejdi a raduj se u svého pána.‘
#25:22 Přistoupil ten se dvěma hřivnami a řekl: ‚Pane, svěřil jsi mi dvě hřivny; hle, jiné dvě jsem získal.‘
#25:23 Jeho pán mu odpověděl: ‚Správně, služebníku dobrý a věrný, nad málem jsi byl věrný, ustanovím tě nad mnohým; vejdi a raduj se u svého pána.‘
#25:24 Přistoupil i ten, který přijal jednu hřivnu, a řekl: ‚Pane, poznal jsem tě, že jsi tvrdý člověk a sklízíš, kde jsi nesel, a sbíráš, kde jsi nerozsypal.
#25:25 Bál jsem se, a proto jsem šel a ukryl tvou hřivnu v zemi. Hle, zde máš, co ti patří.‘
#25:26 Jeho pán mu odpověděl: ‚Služebníku špatný a líný, věděl jsi, že žnu, kde jsem nezasel, a sbírám, kde jsem nerozsypal.
#25:27 Měl jsi tedy dát mé peníze peněžníkům, abych přišel a to, co mi patří, si vybral s úrokem.
#25:28 Vezměte mu tu hřivnu a dejte tomu, který má deset hřiven!
#25:29 Neboť každému, kdo má, bude dáno a přidáno; kdo nemá, tomu bude odňato i to, co má.
#25:30 A toho neužitečného služebníka uvrhněte ven do temnot; tam bude pláč a skřípění zubů.‘
#25:31 Až přijde Syn člověka ve své slávě a všichni andělé s ním, posadí se na trůnu své slávy;
#25:32 a budou před něho shromážděny všechny národy. I oddělí jeden od druhých, jako pastýř odděluje ovce od kozlů,
#25:33 ovce postaví po pravici a kozly po levici.
#25:34 Tehdy řekne král těm po pravici: ‚Pojďte, požehnaní mého Otce, ujměte se království, které je vám připraveno od založení světa.
#25:35 Neboť jsem hladověl, a dali jste mi jíst, žíznil jsem, a dali jste mi pít, byl jsem na cestách, a ujali jste se mne,
#25:36 byl jsem nahý, a oblékli jste mě, byl jsem nemocen, a navštívili jste mě, byl jsem ve vězení, a přišli jste za mnou.‘
#25:37 Tu mu ti spravedliví odpoví: ‚Pane, kdy jsme tě viděli hladového, a nasytili jsme tě, nebo žíznivého, a dali jsme ti pít?
#25:38 Kdy jsme tě viděli jako pocestného, a ujali jsme se tě, nebo nahého, a oblékli jsme tě?
#25:39 Kdy jsme tě viděli nemocného nebo ve vězení, a přišli jsme za tebou?‘
#25:40 Král jim odpoví a řekne jim: ‚Amen, pravím vám, cokoliv jste učinili jednomu z těchto mých nepatrných bratří, mně jste učinili.‘
#25:41 Potom řekne těm na levici: ‚Jděte ode mne, prokletí, do věčného ohně, připraveného ďáblu a jeho andělům!
#25:42 Hladověl jsem, a nedali jste mi jíst, žíznil jsem, a nedali jste mi pít,
#25:43 byl jsem na cestách, a neujali jste se mne, byl jsem nahý, a neoblékli jste mě, byl jsem nemocen a ve vězení, a nenavštívili jste mě.‘
#25:44 Tehdy odpoví i oni: ‚Pane, kdy jsme tě viděli hladového, žíznivého, pocestného, nahého, nemocného nebo ve vězení, a neposloužili jsme ti?‘
#25:45 On jim odpoví: ‚Amen, pravím vám, cokoliv jste neučinili jednomu z těchto nepatrných, ani mně jste neučinili.‘
#25:46 A půjdou do věčných muk, ale spravedliví do věčného života.“ 
#26:1 Když Ježíš dokončil všechna tato slova, řekl svým učedníkům:
#26:2 „Víte, že za dva dny jsou velikonoce, a Syn člověka bude vydán, aby byl ukřižován.“
#26:3 Tehdy se sešli velekněží a starší lidu ve dvoře velekněze, který se jmenoval Kaifáš,
#26:4 a uradili se, že se Ježíš zmocní lstí a že ho zabijí.
#26:5 Říkali: „Jen ne při svátečním shromáždění, aby se lid nebouřil.“
#26:6 Když byl Ježíš v Betanii v domě Šimona Malomocného,
#26:7 přišla za ním žena, která měla alabastrovou nádobku drahocenného oleje, a vylila ji na jeho hlavu, jak seděl u stolu.
#26:8 Když to viděli učedníci, hněvali se: „Nač taková ztráta?
#26:9 Mohlo se to prodat za mnoho peněz a ty dát chudým!“
#26:10 Ježíš to zpozoroval a řekl jim: „Proč trápíte tu ženu? Vykonala dobrý skutek.
#26:11 Vždyť chudé máte stále kolem sebe, mne však nemáte stále.
#26:12 Když vylila ten olej na mé tělo, učinila to k mému pohřbu.
#26:13 Amen, pravím vám, všude, po celém světě, kde bude kázáno toto evangelium, bude se mluvit na její památku o tom, co ona učinila.“
#26:14 Tehdy šel jeden z dvanácti, jménem Jidáš Iškariotský, k velekněžím
#26:15 a řekl: „Co mi dáte? Já vám ho zradím.“ Oni mu určili třicet stříbrných.
#26:16 Od té chvíle hledal vhodnou příležitost, aby ho zradil.
#26:17 Prvního dne o svátcích nekvašených chlebů přišli učedníci za Ježíšem a řekli: „Kde chceš, abychom ti připravili velikonoční večeři?“
#26:18 On je poslal do města k jistému člověku, aby mu řekli: „Mistr vzkazuje: Můj čas je blízko, u tebe budu jíst se svými učedníky velikonočního beránka.“
#26:19 Učedníci učinili, jak jim Ježíš nařídil, a připravili velikonočního beránka.
#26:20 Navečer usedl s Dvanácti ke stolu,
#26:21 a když jedli, řekl jim: „Amen, pravím vám, že jeden z vás mne zradí.“
#26:22 Velice je to zarmoutilo a začali se ho jeden po druhém ptát: „Snad to nejsem já, Pane?“
#26:23 On odpověděl: „Kdo se mnou omočil ruku v míse, ten mě zradí.
#26:24 Syn člověka sice odchází, jak je o něm psáno; ale běda tomu, který Syna člověka zrazuje. Pro toho by bylo lépe, kdyby se byl vůbec nenarodil!“
#26:25 Na to řekl Jidáš, který ho zrazoval: „Jsem to snad já, Mistře?“ Řekl mu: „Ty sám jsi to řekl.“
#26:26 Když jedli, vzal Ježíš chléb, požehnal, lámal a dával učedníkům se slovy: „Vezměte, jezte, toto jest mé tělo.“
#26:27 Pak vzal i kalich, vzdal díky a podal jim ho se slovy: „Pijte z něho všichni.
#26:28 Neboť toto jest má krev, která zpečeťuje smlouvu a prolévá se za mnohé na odpuštění hříchů.
#26:29 Pravím vám, že již nebudu pít z tohoto plodu vinné révy až do toho dne, kdy budu s vámi pít kalich nový v království svého Otce.“
#26:30 Potom zazpívali chvalozpěv a vyšli na Olivovou horu.
#26:31 Tu jim Ježíš řekl: „Vy všichni ode mne této noci odpadnete, neboť jest psáno: ‚Budu bít pastýře a rozprchnou se ovce stáda.‘
#26:32 Po svém vzkříšení však vás předejdu do Galileje.“
#26:33 Na to mu řekl Petr: „Kdyby všichni od tebe odpadli, já nikdy ne!“
#26:34 Ježíš mu odpověděl: „Amen, pravím ti, že ještě této noci, dřív než kohout zakokrhá, třikrát mě zapřeš.“
#26:35 Petr prohlásil: „I kdybych měl s tebou umřít, nezapřu tě.“ Podobně mluvili všichni učedníci.
#26:36 Tu s nimi Ježíš přišel na místo zvané Getsemane a řekl učedníkům: „Počkejte zatím zde, já půjdu dál, abych se modlil.“
#26:37 Vzal s sebou Petra a oba syny Zebedeovy; tu na něho padl zármutek a úzkost.
#26:38 Tehdy jim řekl: „Má duše je smutná až k smrti. Zůstaňte zde a bděte se mnou!“
#26:39 Poodešel od nich, padl tváří k zemi a modlil se: „Otče můj, je-li možné, ať mne mine tento kalich; avšak ne jak já chci, ale jak ty chceš.“
#26:40 Potom přišel k učedníkům a zastihl je v spánku. Řekl Petrovi: „To jste nemohli jedinou hodinu bdít se mnou?
#26:41 Bděte a modlete se, abyste neupadli do pokušení. Váš duch je odhodlán, ale tělo slabé.“
#26:42 Odešel podruhé a modlil se: „Otče můj, není-li možné, aby mne tento kalich minul, a musím-li jej pít, staň se tvá vůle.“
#26:43 A když se vrátil, zastihl je opět spící; nemohli oči udržet.
#26:44 Nechal je, zase odešel a potřetí se modlil stejnými slovy.
#26:45 Potom přišel k učedníkům a řekl jim: „Ještě spíte a odpočíváte? Hle, přiblížila se hodina, a Syn člověka je vydáván do rukou hříšníků.
#26:46 Vstaňte, pojďme! Hle, přiblížil se ten, který mě zrazuje!“
#26:47 Ještě ani nedomluvil a přišel Jidáš, jeden z dvanácti. Velekněží a starší s ním poslali zástup, ozbrojený meči a holemi.
#26:48 Jeho zrádce s nimi domluvil znamení: „Koho políbím, ten to je; toho zatkněte.“
#26:49 A hned přistoupil k Ježíšovi a řekl: „Buď zdráv, Mistře“, a políbil ho.
#26:50 Ježíš mu odpověděl: „Příteli, konej svůj úkol!“ Tu přistoupili k Ježíšovi, vztáhli na něho ruce a zmocnili se ho.
#26:51 Jeden z těch, kdo byli s ním, sáhl po meči, napadl veleknězova sluhu a uťal mu ucho.
#26:52 Ježíš mu řekl: „Vrať svůj meč na jeho místo; všichni, kdo se chápou meče, mečem zajdou.
#26:53 Či myslíš, že bych nemohl poprosit svého Otce, a poslal by mi ihned víc než dvanáct legií andělů?
#26:54 Ale jak by se potom splnila Písma, že to tak musí být?“
#26:55 V té hodině řekl Ježíš zástupům: „Vyšli jste na mne jako na povstalce s meči a holemi, abyste mě zajali. Denně jsem sedával v chrámě a učil, a nezmocnili jste se mne.
#26:56 Toto všechno se však stalo, aby se splnilo, co psali proroci.“ A tu ho všichni učedníci opustili a utekli.
#26:57 Ti, kteří Ježíše zatkli, odvedli ho k veleknězi Kaifášovi, kde se shromáždili zákoníci a starší.
#26:58 Petr šel za ním zpovzdálí až do veleknězova dvora; vstoupil dovnitř a posadil se mezi sluhy, aby viděl konec.
#26:59 Velekněží a celá rada hledali křivé svědectví proti Ježíšovi, aby ho mohli odsoudit k smrti.
#26:60 Ale nenalezli, ačkoli předstupovalo mnoho křivých svědků. Konečně přišli dva
#26:61 a vypovídali: „On řekl: Mohu zbořit chrám a ve třech dnech jej vystavět.“
#26:62 Tu velekněz vstal a řekl mu: „Nic neodpovídáš na to, co tihle proti tobě svědčí?“
#26:63 Ale Ježíš mlčel. A velekněz mu řekl: „Zapřísahám tě při Bohu živém, abys nám řekl, jsi-li Mesiáš, Syn Boží!“
#26:64 Ježíš odpověděl: „Ty sám jsi to řekl. Ale pravím vám, od nynějška uzříte Syna člověka sedět po pravici Všemohoucího a přicházet s oblaky nebeskými.“
#26:65 Tu velekněz roztrhl svá roucha a řekl: „Rouhal se! Nač ještě potřebujeme svědky? Hle, teď jste slyšeli rouhání!
#26:66 Co o tom soudíte?“ Jejich výrok zněl: „Je hoden smrti.“
#26:67 Pak mu plivali do obličeje, bili ho po hlavě, někteří ho tloukli do tváře
#26:68 a říkali: „Hádej, Mesiáši, kdo tě udeřil!“
#26:69 Petr seděl venku v nádvoří. Tu k němu přistoupila jedna služka a řekla: „I ty jsi byl s tím Galilejským Ježíšem!“
#26:70 A on přede všemi zapřel: „Nevím, co mluvíš.“
#26:71 Vyšel k bráně, ale uviděla ho jiná a řekla těm, kdo tam byli: „Tenhle byl s tím Nazaretským Ježíšem.“
#26:72 On znovu zapřel s přísahou: „Neznám toho člověka.“
#26:73 Ale zakrátko přistoupili ti, kdo tam stáli, a řekli Petrovi: „Jistě i ty jsi jeden z nich, vždyť i tvé nářečí tě prozrazuje!“
#26:74 Tu se začal zaklínat a zapřísahat: „Neznám toho člověka.“ V tom zakokrhal kohout;
#26:75 tu se Petr rozpomněl na slova, která mu Ježíš řekl: ‚Dříve než kohout zakokrhá, třikrát mě zapřeš.‘ Vyšel ven a hořce se rozplakal. 
#27:1 Když bylo ráno, uradili se všichni velekněží a starší lidu proti Ježíšovi, že ho připraví o život.
#27:2 Spoutali ho, odvedli a vydali vladaři Pilátovi.
#27:3 Když Jidáš, který ho zradil, viděl, že Ježíše odsoudili, pocítil výčitky, vrátil třicet stříbrných velekněžím a starším
#27:4 a řekl: „Zhřešil jsem, zradil jsem nevinnou krev!“ Ale oni mu odpověděli: „Co je nám po tom? To je tvoje věc!“
#27:5 A on odhodil peníze v chrámě a utekl; šel a oběsil se.
#27:6 Velekněží sebrali peníze a řekli: „Není dovoleno dát je do chrámové pokladny, je to odměna za krev.“
#27:7 Uradili se tedy a koupili za ně pole hrnčířovo k pohřbívání cizinců.
#27:8 Proto se to pole jmenuje ‚Pole krve‘ až dodnes.
#27:9 Tak se splnilo, co je řečeno ústy proroka Jeremiáše: ‚Vzali třicet stříbrných, cenu člověka, na kterou ho ocenili synové Izraele;
#27:10 a dali ji za pole hrnčířovo, jak jim přikázal Hospodin.‘
#27:11 A Ježíš byl postaven před vladaře. Vladař mu položil otázku: „Ty jsi král Židů?“ Ježíš odpověděl: „Ty sám to říkáš.“
#27:12 Na žaloby velekněží a starších nic neodpovídal.
#27:13 Tu mu řekl Pilát: „Neslyšíš, co všechno proti tobě svědčí?“
#27:14 On mu však neodpověděl ani na jedinou věc, takže se vladař velice divil.
#27:15 O svátcích měl vladař ve zvyku propouštět zástupu jednoho vězně, kterého si přáli.
#27:16 Tehdy tam měli pověstného vězně jménem Barabáš.
#27:17 Když se zástupy shromáždily, řekl jim Pilát: „Koho vám mám propustit, Barabáše, nebo Ježíše zvaného Mesiáš?“
#27:18 Věděl totiž, že mu ho vydali ze zášti.
#27:19 Když seděl na soudné stolici, poslala k němu jeho žena se vzkazem: „Nezačínej si nic s tím spravedlivým! Dnes mě kvůli němu pronásledovaly zlé sny.“
#27:20 Velekněží a starší však přiměli zástup, aby si vyžádali Barabáše, a Ježíše zahubili.
#27:21 Vladař jim řekl: „Koho vám z těch dvou mám propustit?“ Oni volali: „Barabáše!“
#27:22 Pilát jim řekl: „Co tedy mám učinit s Ježíšem zvaným Mesiáš?“ Všichni volali: „Ukřižovat!“
#27:23 Namítl jim: „Čeho se vlastně dopustil?“ Ale oni ještě více křičeli: „Ukřižovat!“
#27:24 Když pak Pilát viděl, že nic nepořídí, ale že pozdvižení je čím dál větší, omyl si ruce před očima zástupu a pravil: „Já nejsem vinen krví tohoto člověka; je to vaše věc.“
#27:25 A všechen lid mu odpověděl: „Krev jeho na nás a naše děti!“
#27:26 Tu jim propustil Barabáše, Ježíše dal zbičovat a vydal ho, aby byl ukřižován.
#27:27 Vladařovi vojáci dovedli Ježíše do místodržitelství a svolali na něj celou setninu.
#27:28 Svlékli ho a oblékli mu nachový plášť,
#27:29 upletli korunu z trní a posadili mu ji na hlavu, do pravé ruky mu dali hůl, klekali před ním a posmívali se mu: „Buď zdráv, židovský králi!“
#27:30 Plivali na něj, brali tu hůl a bili ho po hlavě.
#27:31 Když se mu dost naposmívali, svlékli mu plášť a oblékli ho zase do jeho šatů. A odvedli ho k ukřižování.
#27:32 Cestou potkali jednoho člověka z Kyrény, jménem Šimona; toho přinutili, aby nesl jeho kříž.
#27:33 Když přišli na místo zvané Golgota, to znamená ‚Lebka‘,
#27:34 dali mu napít vína smíchaného se žlučí; ale když je okusil, nechtěl pít.
#27:35 Ukřižovali ho a losem si rozdělili jeho šaty;
#27:36 pak tam stáli a střežili ho.
#27:37 Nad hlavu mu dali nápis o provinění: „To je Ježíš, král Židů.“
#27:38 S ním byli ukřižováni dva povstalci, jeden po pravici a druhý po levici.
#27:39 Kolemjdoucí ho uráželi; potřásali hlavou
#27:40 a říkali: „Když chceš zbořit chrám a ve třech dnech jej znovu postavit, zachraň sám sebe; jsi-li Syn Boží, sestup z kříže!“
#27:41 Podobně se mu posmívali i velekněží spolu se zákoníky a staršími. Říkali:
#27:42 :Jiné zachránil, sám sebe zachránit nemůže. Je král izraelský - ať nyní sestoupí s kříže a uvěříme v něho!
#27:43 Spolehl se na Boha, ať ho vysvobodí, stojí-li o něj. Vždyť řekl: ‚Jsem Boží Syn!‘“
#27:44 Stejně ho tupili i povstalci spolu s ním ukřižovaní.
#27:45 V poledne nastala tma po celé zemi až do tří hodin.
#27:46 Kolem třetí hodiny zvolal Ježíš mocným hlasem: „Eli, Eli, lama sabachtani?“, to jest: ‚Bože můj, Bože můj, proč jsi mne opustil?‘
#27:47 Když to uslyšeli, říkali někteří z těch, kdo tu stáli: „On volá Eliáše.“
#27:48 Jeden z nich hned odběhl, vzal houbu, naplnil ji octem, nabodl na tyč a dával mu pít.
#27:49 Ostatní však říkali: „Nech ho, ať uvidíme, jestli přijde Eliáš a zachrání ho!“
#27:50 Ale Ježíš znovu vykřikl mocným hlasem a skonal.
#27:51 A hle, chrámová opona se roztrhla v půli odshora až dolů, země se zatřásla, skály pukaly,
#27:52 hroby se otevřely a mnohá těla zesnulých svatých byla vzkříšena;
#27:53 vyšli z hrobů a po jeho vzkříšení vstoupili do svatého města a mnohým se zjevili.
#27:54 Setník a ti, kdo s ním střežili Ježíše, když viděli zemětřesení a všechno, co se dálo, velmi se zděsili a řekli: „On byl opravdu Boží Syn!“
#27:55 Zpovzdálí přihlíželo mnoho žen, které provázely Ježíše z Galileje, aby se o něj staraly;
#27:56 mezi nimi Marie z Magdaly, Marie, matka Jakubova i Josefova, a matka synů Zebedeových.
#27:57 Když nastal večer, přišel zámožný člověk z Arimatie, jménem Josef, který také patřil k Ježíšovým učedníkům.
#27:58 Ten přišel k Pilátovi a požádal o Ježíšovo tělo. Pilát přikázal, aby mu je dali.
#27:59 Josef tělo přijal, zavinul je do čistého plátna
#27:60 a položil je do svého nového hrobu, který měl vytesaný ve skále; ke vchodu hrobu přivalil veliký kámen a odešel.
#27:61 Byla tam Marie z Magdaly a jiná Marie, které seděly naproti hrobu.
#27:62 Nazítří, po pátku, shromáždili se velekněží a farizeové u Piláta
#27:63 a řekli: „Pane, vzpomněli jsme si, že ten podvodník řekl ještě za svého života: ‚Po třech dnech budu vzkříšen.‘
#27:64 Dej proto rozkaz, ať je po tři dny hlídán jeho hrob,aby nepřišli jeho učedníci, neukradli ho a neřekli lidu, že byl vzkříšen z mrtvých; to by pak byl poslední podvod horší než první.“
#27:65 Pilát jim odpověděl: „Zde máte stráž, dejte hrob hlídat, jak uznáte za dobré.“
#27:66 Oni odešli, zapečetili kámen a postavili k hrobu stráž. 
#28:1 Když uplynula sobota a začínal první den týdne, přišly Marie z Magdaly a jiná Marie, aby se podívaly k hrobu.
#28:2 A hle, nastalo velké zemětřesení, neboť anděl Páně sestoupil s nebe, odvalil kámen a usedl na něm.
#28:3 Jeho vzezření bylo jako blesk a jeho roucho bílé jako sníh.
#28:4 Strážci byli strachem bez sebe a strnuli jako mrtví.
#28:5 Anděl řekl ženám: „Vy se nebojte. Vím, že hledáte Ježíše, který byl ukřižován.
#28:6 Není zde; byl vzkříšen, jak řekl. Pojďte se podívat na místo, kde ležel.
#28:7 Jděte rychle povědět jeho učedníkům, že byl vzkříšen z mrtvých; jde před nimi do Galileje, tam ho spatří. Hle, řekl jsem vám to.“
#28:8 Tu rychle opustily hrob a se strachem i s velikou radostí běžely to oznámit jeho učedníkům.
#28:9 A hle, Ježíš je potkal a řekl: „Buďte pozdraveny.“ Ženy přistoupily, objímaly jeho nohy a klaněly se mu.
#28:10 Tu jim Ježíš řekl: „Nebojte se. Jděte a oznamte mým bratřím, aby šli do Galileje; tam mě uvidí.“
#28:11 Když se ženy vzdálily, někteří ze stráže přišli do města a oznámili velekněžím, co se všechno stalo.
#28:12 Ti se sešli se staršími, poradili se a dali vojákům značné peníze
#28:13 s pokynem: „Řekněte, že jeho učedníci přišli v noci a ukradli ho, když jste spali.
#28:14 A doslechne-li se to vladař, my to urovnáme a postaráme se, abyste neměli těžkosti.“
#28:15 Vojáci vzali peníze a udělali to tak, jak se jim řeklo. A ten výklad je rozšířen mezi židy až podnes.
#28:16 Jedenáct apoštolů se pak odebralo do Galileje, na horu, kterou jim Ježíš určil.
#28:17 Spatřili ho a klaněli se mu; ale někteří pochybovali.
#28:18 Ježíš přistoupil a řekl jim: „Je mi dána veškerá moc na nebi i na zemi.
#28:19 Jděte ke všem národům a získávejte mi učedníky, křtěte ve jméno Otce i Syna i Ducha svatého
#28:20 a učte je, aby zachovávali všecko, co jsem vám přikázal. A hle, já jsem s vámi po všecky dny až do skonání tohoto věku.“  

\book{Mark}{Mark}
#1:1 Počátek evangelia Ježíše Krista, Syna Božího.
#1:2 Je psáno u proroka Izaiáše: ‚Hle, já posílám posla před tvou tváří, by ti připravil cestu.
#1:3 Hlas volajícího na poušti: Připravte cestu Páně, vyrovnejte mu stezky!‘
#1:4 To se stalo, když Jan Křtitel vystoupil na poušti a kázal: „Čiňte pokání a dejte se pokřtít na odpuštění hříchů.“
#1:5 Celá judská krajina i všichni z Jeruzaléma vycházeli k němu, vyznávali své hříchy a dávali se od něho křtít v řece Jordánu.
#1:6 Jan byl oděn velbloudí srstí, měl kožený pás kolem boků a jedl kobylky a med divokých včel.
#1:7 A kázal: „Za mnou přichází někdo silnější, než jsem já; nejsem hoden, abych se sklonil a rozvázal řemínek jeho obuvi.
#1:8 Já jsem vás křtil vodou, on vás bude křtít Duchem svatým.
#1:9 V těch dnech přišel Ježíš z Nazareta v Galileji a byl v Jordánu od Jana pokřtěn.
#1:10 V tom, jak vystupoval z vody, uviděl nebesa rozevřená a Ducha, který jako holubice sestupuje na něj.
#1:11 A z nebe se ozval hlas: „Ty jsi můj milovaný Syn, tebe jsem si vyvolil.“
#1:12 A hned ho Duch vyvedl na poušť.
#1:13 Byl na poušti čtyřicet dní a satan ho pokoušel; byl mezi dravou zvěří a andělé ho obsluhovali.
#1:14 Když byl Jan uvězněn, přišel Ježíš do Galileje a kázal Boží evangelium.
#1:15 Naplnil se čas a přiblížilo se království Boží. Čiňte pokání a věřte evangeliu.“
#1:16 Když šel podél Galilejského moře, uviděl Šimona a jeho bratra Ondřeje, jak vrhají síť do moře; byli totiž rybáři.
#1:17 Ježíš jim řekl: „Pojďte za mnou a učiním z vás rybáře lidí.“
#1:18 Ihned opustili sítě a šli za ním.
#1:19 O něco dále uviděl Jakuba Zebedeova a jeho bratra Jana; ti byli na lodi a spravovali sítě.
#1:20 Hned je povolal. A zanechali na lodi svého otce Zebedea s pomocníky a šli za ním.
#1:21 Když přišli do Kafarnaum, hned v sobotu šel do synagógy a učil.
#1:22 I žasli nad jeho učením, neboť učil jako ten, kdo má moc, a ne jako zákoníci.
#1:23 V jejich synagóze byl právě člověk, posedlý nečistým duchem. Ten vykřikl:
#1:24 „Co je ti do nás, Ježíši Nazaretský? Přišel jsi nás zahubit? Vím, kdo jsi. Jsi svatý Boží.“
#1:25 Ale Ježíš mu pohrozil: „Umlkni a vyjdi z něho!“
#1:26 Nečistý duch jím zalomcoval a s velikým křikem z něho vyšel.
#1:27 Všichni užasli a jeden druhého se ptali: „Co to je? Nové učení plné moci - i nečistým duchům přikáže, a poslechnou ho.“
#1:28 A pověst o něm se rychle rozšířila po celé galilejské krajině.
#1:29 Když vyšel ze synagógy, vstoupil s Jakubem a Janem do domu Šimonova a Ondřejova.
#1:30 Šimonova tchyně ležela v horečce. Hned mu o ní pověděli.
#1:31 Přistoupil k ní, vzal ji za ruku a pozvedl ji. Horečka ji opustila a ona je obsluhovala.
#1:32 Když nastal večer a slunce zapadlo, přinášeli k němu všechny nemocné a posedlé.
#1:33 Celé město se shromáždilo u dveří.
#1:34 I uzdravil mnoho nemocných rozličnými neduhy a mnoho zlých duchů vyhnal. A nedovoloval zlým duchům mluvit, protože věděli, kdo je.
#1:35 Časně ráno, ještě za tmy, vstal a vyšel z domu; odešel na pusté místo a tam se modlil.
#1:36 Šimon a jeho druhové se pustili za ním.
#1:37 Když ho nalezli, řekli: „Všichni tě hledají.“
#1:38 Řekne jim: „Pojďte jinam do okolních městeček, abych i tam kázal, nebo proto jsem vyšel.“
#1:39 A tak šel, kázal v jejich synagógách po celé Galileji a vyháněl zlé duchy.
#1:40 Přijde k němu malomocný a na kolenou ho prosí: „Chceš-li, můžeš mě očistit.“
#1:41 Ježíš se slitoval, vztáhl ruku, dotkl se ho a řekl: „Chci, buď čist.“
#1:42 A hned se jeho malomocenství ztratilo a byl očištěn.
#1:43 Ježíš mu pohrozil, poslal jej ihned pryč
#1:44 a nařídil mu: „Ne, abys někomu něco říkal! Ale jdi, ukaž se knězi a obětuj za své očištění, co Mojžíš přikázal - jim na svědectví.“
#1:45 On však odešel a mnoho o tom vyprávěl a rozhlašoval, takže Ježíš už nemohl veřejně vejít do města, ale zůstal venku na opuštěných místech. A chodili k němu odevšad. 
#2:1 Když se po několika dnech vrátil do Kafarnaum, proslechlo se, že je doma.
#2:2 Sešlo se tolik lidí, že už ani přede dveřmi nebylo k hnutí. A mluvil k nim.
#2:3 Tu k němu přišli s ochrnutým; čtyři ho nesli.
#2:4 Protože se pro zástup nemohli k němu dostat, odkryli střechu tam, kde byl Ježíš, prorazili otvor a spustili dolů nosítka, na kterých ochrnutý ležel.
#2:5 Když Ježíš viděl jejich víru, řekl ochrnutému: „Synu, odpouštějí se ti hříchy.“
#2:6 Seděli tam někteří ze zákoníků a v duchu uvažovali:
#2:7 „Co to ten člověk říká? Rouhá se! Kdo jiný může odpouštět hříchy než Bůh?“
#2:8 Ježíš hned svým duchem poznal, o čem přemýšlejí, a řekl jim: „Jak to, že tak uvažujete?
#2:9 Je snadnější říci ochrnutému: ‚Odpouštějí se ti hříchy,‘ anebo říci: ‚Vstaň, vezmi své lože a choď?‘
#2:10 Abyste však věděli, že Syn člověka má moc na zemi odpouštět hříchy“ - řekne ochrnutému:
#2:11 „Pravím ti, vstaň, vezmi své lože a jdi domů!“
#2:12 On vstal, vzal hned své lože a vyšel před očima všech, takže všichni žasli a chválili Boha: „Něco takového jsme ještě nikdy neviděli.“
#2:13 Vyšel opět k moři. Všechen lid k němu přicházel a on je učil.
#2:14 A když šel dál, viděl Leviho, syna Alfeova, jak sedí v celnici, a řekl mu: „Pojď za mnou.“ On vstal a šel za ním.
#2:15 Když byl u stolu v jeho domě, stolovalo s Ježíšem a jeho učedníky mnoho celníků a jiných hříšníků; bylo jich totiž mnoho mezi těmi, kteří ho následovali.
#2:16 Když zákoníci z farizejské strany viděli, že jí s hříšníky a celníky, říkali jeho učedníkům: „Jak to, že jí s celníky a hříšníky?“
#2:17 Ježíš to uslyšel a řekl jim: „Lékaře nepotřebují zdraví, ale nemocní. Nepřišel jsem pozvat spravedlivé, ale hříšníky.“
#2:18 Učedníci Janovi a farizeové se postili. Přišli k němu a ptali se: „Jak to, že se učedníci Janovi a učedníci farizeů postí, ale tvoji učedníci se nepostí?“
#2:19 Ježíš jim řekl: „Mohou se hosté na svatbě postit, když je ženich s nimi? Pokud mají ženicha mezi sebou, nemohou se postit.
#2:20 Přijdou však dny, kdy od nich ženich bude vzat; potom, v ten den, se budou postit.
#2:21 Nikdo nepřišívá záplatu z neseprané látky na starý šat, jinak se ten přišitý kus vytrhne, nové od starého, a díra bude ještě větší.
#2:22 A nikdo nedává mladé víno do starých měchů, jinak víno roztrhne měchy a měchy i víno přijdou nazmar. Nové víno do nových měchů!“
#2:23 Jednou v sobotu procházel obilím a jeho učedníci začali cestou mnout zrní z klasů.
#2:24 Farizeové mu řekli: „Jak to, že dělají v sobotu, co se nesmí!“
#2:25 Odpověděl jim: „Nikdy jste nečetli, co udělal David, když měl hlad a neměl co jíst, on i ti, kdo byli s ním?
#2:26 Jak za velekněze Abiatara vešel do domu Božího a jedl posvátné chleby, které nesmí jíst nikdo kromě kněží, a dal i těm, kdo ho provázeli?“
#2:27 A řekl jim: „Sobota je učiněna pro člověka, a ne člověk pro sobotu.
#2:28 Proto je Syn člověka pánem i nad sobotou.“ 
#3:1 Vešel opět do synagógy; a byl tam člověk s odumřelou rukou.
#3:2 Číhali na něj, uzdraví-li ho v sobotu, aby jej obžalovali.
#3:3 On řekl tomu člověku s odumřelou rukou: „Vstaň a pojď doprostřed!“
#3:4 Pak se jich zeptal: „Je dovoleno v sobotu jednat dobře, či zle, život zachránit, či utratit?“ Ale oni mlčeli.
#3:5 Tu se po nich rozhlédl s hněvem, zarmoucen tvrdostí jejich srdce, a řekl tomu člověku: „Zvedni ruku!“ Zvedl ji, a jeho ruka byla zase zdravá.
#3:6 Když farizeové vyšli ven, hned se proti němu s herodiány umlouvali, že ho zahubí.
#3:7 Ježíš se se svými učedníky uchýlil k moři. Šlo za ním množství lidí z Galileje; ale i z Judska,
#3:8 Jeruzaléma, Idumeje, ze Zajordání a z okolí Týru a Sidónu přišlo k němu veliké množství, když slyšeli, co všechno činí.
#3:9 Požádal učedníky, ať pro něho připraví loď, aby se zástup na něho netlačil.
#3:10 Mnohé totiž uzdravil, proto se ti, kdo trpěli chorobami, tlačili k němu, aby se ho dotkli.
#3:11 A nečistí duchové, jakmile ho viděli, padali před ním na zem a křičeli: „Ty jsi Syn Boží!“
#3:12 On však jim přísně nakazoval, aby ho neprozrazovali.
#3:13 Vystoupil na horu a zavolal k sobě ty, které si vyvolil; i přišli k němu.
#3:14 Ustanovil jich dvanáct, aby byli s ním, aby je posílal kázat
#3:15 a aby měli moc vymítat zlé duchy.
#3:16 Ustanovil těchto dvanáct: Petra - toto jméno dal Šimonovi -
#3:17 Jakuba Zebedeova a jeho bratra Jana, jimž dal jméno Boanerges, což znamená ‚synové hromu‘,
#3:18 Ondřeje, Filipa, Bartoloměje, Matouše, Tomáše, jakuba Alfeova, Tadeáše, Šimona Kananejského
#3:19 a Iškariotského Jidáše, který ho pak zradil.
#3:20 Vešel do domu a opět shromáždil zástup, takže nemohli ani chleba pojíst.
#3:21 Když to uslyšeli jeho příbuzní, přišli, aby se ho zmocnili; říkali totiž, že se pomátl.
#3:22 Zákoníci, kteří přišli z Jeruzaléma, říkali: „Je posedlý Belzebulem. Ve jménu knížete démonů vyhání démony.“
#3:23 Zavolal je k sobě a mluvil k nim v podobenstvích: „Jak může satan vyhánět satana?
#3:24 Je-li království vnitřně rozděleno, nemůže obstát.
#3:25 Je-li dům vnitřně rozdělen, nebude moci obstát.
#3:26 A povstane-li satan sám proti sobě a je rozdvojen, nemůže obstát a je s ním konec.
#3:27 Nikdo nemůže vejít do domu silného muže a uloupit jeho věci, jestliže toho siláka dříve nespoutá. Pak teprve vyloupí jeho dům.
#3:28 Amen, pravím vám, že všecko bude lidem odpuštěno, hříchy i všechna možná rouhání.
#3:29 Kdo by se však rouhal proti Duchu svatému, nemá odpuštění na věky, ale je vinen věčným hříchem.“
#3:30 Toto pravil, protože řekli: „Má nečistého ducha.“
#3:31 Tu přišla jeho matka a jeho bratři. Stáli venku a vzkázali mu, aby k nim přišel.
#3:32 Kolem něho seděl zástup; řekli mu: „Hle, tvoje matka a tvoji bratři jsou venku a hledají tě.“
#3:33 Odpověděl jim: „Kdo je má matka a moji bratři?“
#3:34 Rozhlédl se po těch, kteří seděli v kruhu kolem něho, a řekl: „Hle, moje matka a moji bratři!
#3:35 Kdo činí vůli Boží, to je můj bratr, má sestra i matka.“ 
#4:1 Opět začal učit u moře. Shromáždil se k němu tak veliký zástup, že musel vstoupit na loď na moři; posadil se v ní a celý zástup byl na břehu.
#4:2 Učil je mnohému v podobenstvích. Ve svém učení jim řekl:
#4:3 „Slyšte! Vyšel rozsévač rozsívat.
#4:4 Když rozsíval, padlo některé zrno podél cesty, a přiletěli ptáci a sezobali je.
#4:5 Jiné padlo na skalnatou půdu, kde nemělo dost země, a hned vzešlo, protože nebylo hluboko v zemi.
#4:6 Ale když vyšlo slunce, spálilo je; a protože nemělo kořen, uschlo.
#4:7 Jiné zase padlo do trní; trní vzrostlo, udusilo je, a zrno nevydalo úrodu.
#4:8 A jiná zrna padla do dobré země a vzcházela, rostla, dávala úrodu a přinášela užitek i třicetinásobný i šedesátinásobný i stonásobný.“
#4:9 A řekl: „Kdo má uši k slyšení, slyš!“
#4:10 A když byl o samotě, vyptávali se ho ti, kdo byli s ním, co znamenají podobenství.
#4:11 I řekl jim: „Vám je dáno znát tajemství Božího království; ale těm, kdo jsou vně, děje se všechno v podobenstvích,
#4:12 aby ‚hleděli a hleděli, ale neviděli, poslouchali a poslouchali, ale nechápali, aby se snad neobrátili a nebylo jim odpuštěno‘.“
#4:13 Řekl jim: „Nerozumíte tomuto podobenství? Jak porozumíte všem ostatním?
#4:14 Rozsévač rozsívá slovo.
#4:15 Toto jsou ti podél cesty, kde se rozsívá slovo: Když je uslyší, hned přichází satan a odnímá slovo do nich zaseté.
#4:16 A podobně ti, u nichž je zaseto na skalnatou půdu: Ti slyší slovo a hned je s radostí přijímají.
#4:17 Nemají však v sobě kořen a jsou nestálí; když pak přijde tíseň nebo pronásledování pro to slovo, hned odpadají.
#4:18 U jiných je zaseto do trní: Ti slyší slovo,
#4:19 ale časné starosti, vábivost majetku a chtivost ostatních věcí vnikají do nitra a dusí slovo, takže zůstane bez úrody.
#4:20 Toto pak jsou ti, u nichž je zaseto do dobré země: Ti slyší slovo, přijímají je a nesou úrodu třicetinásobnou i šedesátinásobnou i stonásobnou.“
#4:21 A řekl jim: „Přichází snad světlo, aby bylo dáno pod nádobu nebo pod postel, a ne na svícen?
#4:22 Nic není skrytého, aby to jednou nebylo zjeveno, a nic nebylo utajeno, leč aby vyšlo najevo.
#4:23 Kdo má uši k slyšení, slyš!“
#4:24 Řekl jim také: „Dávejte pozor na to, co slyšíte! Jakou mírou měříte, takovou vám bude naměřeno, a ještě přidáno.
#4:25 Neboť kdo má, tomu bude dáno, a kdo nemá, tomu bude odňato i to co má.“
#4:26 Dále řekl: „S královstvím Božím je to tak, jako když člověk vhodí semeno do země;
#4:27 ať spí či bdí, v noci i ve dne, semeno vzchází a roste, on ani neví jak.
#4:28 Země sama od sebe plodí nejprve stéblo, potom klas a nakonec zralé obilí v klasu.
#4:29 A když úroda dozraje, hned hospodář pošle srp, protože nastala žeň.“
#4:30 Také řekl: „K čemu přirovnáme Boží království nebo jakým podobenstvím je znázorníme?
#4:31 Je jako hořčičné zrno: Když je zaseto do země, je menší, než všecka semena na zemi;
#4:32 ale když je zaseto, vzejde, přerůstá všechny byliny a vyhání tak velké větve, že ptáci mohou hnízdit v jejich stínu.“
#4:33 V mnoha takových podobenstvích mluvil jim slovo tak, jak mohli slyšet.
#4:34 Bez podobenství k nim nemluvil, ale v soukromí svým učedníkům vykládal.
#4:35 Téhož dne večer jim řekl: „Přeplavme se na druhou stranu!“
#4:36 I opustili zástup a odvezli ho lodí, na které byl. A jiné lodi ho doprovázely.
#4:37 Tu se strhla velká bouře s vichřicí a vlny se valily na loď, že už byla skoro plná.
#4:38 On však na zádi lodi na podušce spal. I probudí ho a řeknou mu: „Mistře, tobě je jedno, že zahyneme?“
#4:39 Tu vstal, pohrozil větru a řekl moři: „Zmlkni a utiš se!“ I ustal vítr a bylo veliké ticho.
#4:40 A řekl jim: „Proč jste tak ustrašení? Což nemáte víru?“
#4:41 Zděsili se velikou bázní a říkali jeden druhému: „Kdo to jen je, že ho poslouchá i vítr i moře?“ 
#5:1 Přijeli na protější břeh moře do krajiny gerasenské.
#5:2 Sotva vystoupil z lodi, vyšel proti němu z hrobů člověk nečistého ducha.
#5:3 Ten bydlel v hrobech a nikdo ho nedokázal spoutat už ani řetězy.
#5:4 Často totiž byl už spoután okovy a řetězy, ale řetězy se sebe strhal a okovy rozlámal. Nikdo neměl sílu ho zkrotit.
#5:5 A stále v noci i ve dne křičel mezi hroby a na horách a bil do sebe kamením.
#5:6 Když spatřil z dálky Ježíše, přiběhl, padl před ním na zem
#5:7 a hrozně křičel: „Co je ti po mně, Ježíši, synu Boha nejvyššího? Při Bohu tě zapřísahám, netrap mě!“
#5:8 Ježíš mu totiž řekl: „Duchu nečistý, vyjdi z toho člověka!“
#5:9 A zeptal se ho: „Jaké je tvé jméno?“ Odpověděl: „Mé jméno je ‚legie‘, poněvadž je nás mnoho.“
#5:10 A velmi ho prosil, aby je neposílal pryč z té krajiny.
#5:11 Páslo se tam na svahu hory velké stádo vepřů.
#5:12 I prosili ho ti zlí duchové: „Pošli nás, ať vejdeme do těch vepřů!“
#5:13 On jim to dovolil. Tu nečistí duchové vyšli z posedlého a vešli do vepřů. Stádo pak - bylo jich na dva tisíce - hnalo se střemhlav po srázu do moře a v moři se utopilo.
#5:14 Pasáci utekli a donesli o tom zprávu do města i vesnic. Lidé se šli podívat, co se stalo.
#5:15 Přišli k Ježíšovi a spatřili toho posedlého, který míval množství zlých duchů, jak sedí oblečen a chová se rozumně; a zděsili se.
#5:16 Ti, kteří to viděli, vyprávěli o posedlém a také o vepřích, co se s nimi stalo.
#5:17 Tu počali Ježíše prosit, aby odešel z jejich končin.
#5:18 Když vstupoval na loď, prosil ho ten člověk dříve posedlý, aby směl s ním.
#5:19 Ale nedovolil mu to a řekl: „Jdi domů k svým a pověz jim, jak veliké věci ti učinil Pán, když se nad tebou smiloval.“
#5:20 Tehdy odešel a začal zvěstovat v Dekapoli, jak veliké věci mu učinil Ježíš; a všichni se divili.
#5:21 Když se Ježíš přeplavil v lodi opět na druhou stranu, shromáždil se k němu veliký zástup, když byl ještě na břehu moře.
#5:22 Tu přišel k němu jeden představený synagógy, jménem Jairos, a sotva Ježíše spatřil, padl mu k nohám
#5:23 a úpěnlivě ho prosil: „Má dcerka umírá. Pojď, vlož na ni ruce, aby byla zachráněna a žila!“
#5:24 Ježíš odešel s ním. Velký zástup šel za ním a tlačil se na něj.
#5:25 A byla tam žena, která měla dvanáct let krvácení.
#5:26 Podstoupila mnohé léčení u mnoha lékařů a vynaložila všecko, co měla, ale nic jí nepomohlo, naopak, šlo to s ní stále k horšímu.
#5:27 Když zaslechla o Ježíšovi, přišla ze zadu v zástupu a dotkla se jeho šatu.
#5:28 Říkala si totiž: „Dotknu-li se aspoň jeho šatu, budu vysvobozena!“
#5:29 A rázem přestalo jí krvácení a ucítila v těle, že je vyléčena ze svého trápení.
#5:30 Ježíš hned poznal, že z něho vyšla síla, otočil se v zástupu a řekl: „Kdo se to dotkl mého šatu?“
#5:31 Jeho učedníci mu řekli: „Vidíš, jak se na tebe zástup tlačí, a ptáš se: ‚Kdo se mne to dotkl‘?“
#5:32 I rozhlížel se, aby našel tu, která to učinila.
#5:33 Ta žena věděla, co se s ní stalo, a tak s bázní a chvěním přišla, padla mu k nohám a pověděla celou pravdu.
#5:34 A on jí řekl: „Dcero, tvá víra tě zachránila. Odejdi v pokoji, uzdravena ze svého trápení!“
#5:35 Když ještě mluvil, přišli lidé z domu představeného synagógy a řekli: „Tvá dcera zemřela; proč ještě obtěžuješ mistra?“
#5:36 Ale Ježíš nedbal na ta slova a řekl představenému synagógy: „Neboj se, jen věř!“
#5:37 A nedovolil nikomu, aby šel s ním, kromě Petra, Jakuba a jeho bratra Jana.
#5:38 Když přišli do domu představeného synagógy, spatřili velký rozruch, pláč a kvílení.
#5:39 Vešel dovnitř a řekl jim: „Proč ten rozruch a pláč? Dítě neumřelo, ale spí.“
#5:40 Oni se mu posmívali. On však všecky vyhnal, vzal s sebou otce dítěte, matku a ty, kdo byli s ním, a vystoupil tam, kde dítě leželo.
#5:41 Vzal ji za ruku a řekl: „Talitha kum,“ což znamená: „Děvče, pravím ti, vstaň!“
#5:42 Tu děvče hned vstalo a chodilo; bylo jí dvanáct let. A zmocnil se jich úžas a zděšení.
#5:43 Přísně jim nařídil, že se to nikdo nesmí dovědět, a řekl, aby jí dali něco k jídlu. 
#6:1 Vyšel odtamtud a šel do svého domova; učedníci šli s ním.
#6:2 Když přišla sobota, počal učit v synagóg. Mnoho lidí ho poslouchalo a v úžasu říkali: „Odkud to ten člověk má? Jaká je to moudrost, jež mu byla dána? A jak mocné činy se dějí jeho rukama!
#6:3 Což to není ten tesař, syn Mariin a bratr Jakubův, Josefův, Judův a Šimonův? A nejsou jeho sestry tady u nás?“ A byl jim kamenem úrazu.
#6:4 Tu jim Ježíš řekl: „Prorok není beze cti, leda ve své vlasti, u svých příbuzných a ve svém domě.“
#6:5 A nemohl tam učinit žádný mocný čin, jen na několik málo nemocných vložil ruce a uzdravil je.
#6:6 A podivil se jejich nevěře. Obcházel pak okolní vesnice a učil.
#6:7 Zavolal svých dvanáct, počal je posílat dva a dva a dával jim moc nad nečistými duchy.
#6:8 Přikázal jim, aby si nic nebrali na cestu, jen hůl: ani chleba, ani mošnu, ani peníze do opasku;
#6:9 aby šli jen v sandálech a nebrali si dvoje šaty.
#6:10 A řekl jim: „Když přijdete někam do domu, tam zůstávejte, dokud z těch míst neodejdete.
#6:11 A které místo vás nepřijme a kde vás nebudou chtít slyšet, vyjděte odtamtud a setřeste prach svých nohou na svědectví proti nim.“
#6:12 I vyšli a volali k pokání;
#6:13 vymítali mnoho zlých duchů, potírali olejem mnoho nemocných a uzdravovali je.
#6:14 Uslyšel o tom král Herodes, neboť jméno Ježíšovo se stalo známým; říkalo se: „Jan Křtitel vstal z mrtvých, a proto v něm působí mocné síly.“
#6:15 Jiní pak říkali: „Je to Eliáš!“ A zase jiní: „Je to prorok - jeden z proroků.“
#6:16 Když to Herodes uslyšel, řekl: „To vstal Jan, kterého jsem dal stít.“
#6:17 Tento Herodes totiž dal Jana zatknout a vsadit v poutech do žaláře kvůli Herodiadě, manželce svého bratra Filipa, protože si ji vzal za ženu.
#6:18 Jan totiž říkal Herodovi: „Není dovoleno, abys měl manželku svého bratra!“
#6:19 Herodias byla plná zloby proti Janovi, ráda by ho zbavila života, ale nemohla.
#6:20 Herodes se totiž Jana bál, neboť věděl, že je to muž spravedlivý a svatý, a chránil ho; když ho slyšel, byl celý nejistý, a přece mu rád naslouchal.
#6:21 Vhodná chvíle nastala, když Herodes o svých narozeninách uspořádal hostinu pro své dvořany, důstojníky a významné lidi z Galileje.
#6:22 Tu vstoupila dcera té Herodiady, tančila a zalíbila se králi Herodovi i těm, kdo s ním hodovali. Král řekl dívce: „Požádej mě oč chceš, a já ti to dám.“
#6:23 Zavázal se jí přísahou: „O cokoli požádáš, dám tobě, až do polovice mého království.“
#6:24 Ona vyšla a zeptala se matky: „Oč mám požádat?“ Ta odpověděla: „O hlavu Jana Křtitele.“
#6:25 Spěchala ihned dovnitř ke králi a přednesla mu svou žádost: „Chci, abys mi ihned dal na míse hlavu Jana Křtitele.“
#6:26 Král se velmi zarmoutil, ale pro přísahu před spolustolovníky nechtěl ji odmítnout.
#6:27 I poslal hned kata s příkazem přinést Janovu hlavu. Ten odešel, sťal ho v žaláři
#6:28 a přinesl jeho hlavu na míse; dal ji dívce a dívka ji dala své matce.
#6:29 Když to uslyšeli Janovi učedníci, přišli, vzali jeho tělo a uložili je do hrobu.
#6:30 Apoštolové se shromáždili k Ježíšovi a oznámili mu všecko, co činili a učili.
#6:31 Řekl jim: „Pojďte sami stranou na pusté místo a trochu si odpočiňte!“ Stále totiž přicházelo a odcházelo mnoho lidí, a neměli ani čas se najíst.
#6:32 Odjeli tedy lodí na pusté místo, aby byli sami.
#6:33 Mnozí spatřili, jak odjíždějí a poznali je; pěšky se tam ze všech měst sběhli a byli tam před nimi.
#6:34 Když vystoupil, uviděl velký zástup a bylo mu jich líto, protože byli jako ovce bez pastýře. I začal je učit mnohým věcem.
#6:35 Když už čas pokročil, přistoupili k němu jeho učedníci a řekli: „Toto místo je pusté a čas už pokročil.
#6:36 Propusť je, ať si jdou do okolních dvorů a vesnic koupit něco k jídlu.“
#6:37 Odpověděl jim: „Dejte vy jim jíst!“ Řekli mu: „Máme jít nakoupit za dvě stě denárů chleba a dát jim jíst?“
#6:38 Zeptal se jich: „Kolik chlebů máte? Jděte se podívat!“ Když to zjistili, řekli: „Pět, a dvě ryby.“
#6:39 Přikázal jim, aby všecky rozsadili po skupinách na zelený trávník.
#6:40 I rozložili se oddíl za oddílem stokrát po padesáti.
#6:41 Potom vzal těch pět chlebů a dvě ryby, vzhlédl k nebi, dobrořečil, lámal chleby a dával učedníkům, aby jim je předkládali. I ty dvě ryby rozdělil všem.
#6:42 A jedli všichni a nasytili se.
#6:43 A ještě sebrali dvanáct plných košů nalámaných chlebů i ryb.
#6:44 Těch pak, kteří jedli chleby, bylo pět tisíc mužů.
#6:45 Hned na to přiměl své učedníky, aby vstoupili na loď a jeli napřed na druhý břeh k Betsaidě, než on propustí zástup.
#6:46 Rozloučil se s nimi a šel na horu, aby se modlil.
#6:47 A když nastal večer, byla loď daleko na moři a on jediný na zemi.
#6:48 Spatřil je zmožené veslováním, neboť vítr vál proti nim; tu k nim před svítáním kráčel po moři a chtěl jít dál kolem nich.
#6:49 Když ho uviděli kráčet po moři, vykřikli v domnění, že je to přízrak;
#6:50 všichni ho totiž viděli a vyděsili se. On však na ně hned promluvil: „Vzchopte se, já jsem to, nebojte se!“
#6:51 Vstoupil k nim na loď a vítr se utišil. I byli celí ohromení úžasem.
#6:52 Nepochopili totiž, jak to bylo s chleby, neboť jejich mysl byla zatvrzelá.
#6:53 Když se dostali na druhý břeh, přistáli u Genezaretu.
#6:54 Jakmile vystoupili z lodi, hned ho lidé poznali,
#6:55 zběhali celou tu krajinu a začali nosit na nosítkách nemocné na každé místo, kde slyšeli, že jest.
#6:56 A kamkoli vcházel do vesnic, měst i dvorců, kladli nemocné na tržiště a prosili ho, aby se směli dotknout byť jen třásně jeho roucha. A kdo se ho dotkli, byli uzdraveni. 
#7:1 Shromáždili se k němu farizeové a někteří ze zákoníků, kteří přišli z Jeruzaléma.
#7:2 Uviděli některé z jeho učedníků, jak jedí znesvěcujícíma, to jest neomytýma rukama.
#7:3 (Farizeové totiž a všichni židé se drží tradice otců a nejedí, dokud si k zápěstí neomyjí ruce.
#7:4 A po návratu z trhu nejedí, dokud se neočistí. A je ještě mnoho jiných tradic, kterých se drží: ponořování pohárů, džbánů a měděných mis.)
#7:5 Farizeové a zákoníci se ho zeptali: „Proč se tvoji učedníci neřídí podle tradice otců a jedí znesvěcujícíma rukama?“
#7:6 Řekl jim: „Dobře prorokoval Izaiáš o vás pokrytcích, jak je psáno: ‚Tento lid ctí mě rty, ale srdce jejich je daleko ode mne;
#7:7 marná je zbožnost, kterou mne ctí, učíce naukám, jež jsou jen příkazy lidskými.‘
#7:8 Opustili jste přikázání Boží a držíte se lidské tradice.“
#7:9 A ještě řekl: „Jak dovedně rušíte Boží přikázání, abyste zachovali svou tradici!
#7:10 Vždyť Mojžíš řekl: ‚Cti svého otce i svou matku‘ a ‚kdo zlořečí otci nebo matce, ať je potrestán smrtí‘.
#7:11 Vy však učíte: Řekne-li někdo otci nebo matce: ‚To, čím jsem ti zavázán pomoci, je korbán (to jest dar Bohu)‘,
#7:12 již podle vás nemusí pro otce nebo matku nic udělat;
#7:13 tak rušíte Boží slovo svou tradicí, kterou pěstujete. A takových podobných věcí činíte mnoho.“
#7:14 Když znovu svolal zástup, řekl jim: „Slyšte mě všichni a rozumějte:
#7:15 Nic, co zvenčí vchází do člověka, nemůže ho znesvětit; ale co z člověka vychází, to jej znesvěcuje.
#7:16 Kdo má uši k slyšení, slyš!“
#7:17 Když opustil zástup a vešel do domu, ptali se ho jeho učedníci na to podobenství.
#7:18 Řekl jim: „Tak i vy jste nechápaví? Nerozumíte, že nic, co zvenčí vchází do člověka, nemůže ho znesvětit,
#7:19 poněvadž mu nevchází do srdce, ale do břicha a jde do hnoje?“ Tak prohlásil všechny pokrmy za čisté.
#7:20 A řekl: „Co vychází z člověka, to ho znesvěcuje.
#7:21 Z nitra totiž, z lidského srdce, vycházejí zlé myšlenky, smilství, loupeže, vraždy,
#7:22 cizoložství, chamtivost, zlovolnost, lest, bezuzdnost, závistivý pohled, urážky, nadutost, opovážlivost.
#7:23 Všecko toto zlé vychází z nitra a znesvěcuje člověka.“
#7:24 Vstal a šel odtud do končin týrských. Vešel do jednoho domu a nechtěl, aby o tom někdo věděl. Nemohlo se to však utajit;
#7:25 hned o něm uslyšela jedna žena, jejíž dcerka měla nečistého ducha. Přišla a padla mu k nohám;
#7:26 ta žena byla pohanka, rodem Syrofeničanka. Prosila ho, aby vyhnal zlého ducha z její dcery.
#7:27 On jí řekl: „Nech napřed nasytit děti. Neboť se nesluší vzít dětem chléb a hodit jej psům.“
#7:28 Odpověděla mu: „Ovšem, pane, jenže i psi se pod stolem živí z drobtů po dětech.“
#7:29 Pravil jí: „Žes to řekla, jdi, zlý duch vyšel z tvé dcery.“
#7:30 Když se vrátila domů, nalezla dítě ležící na lůžku a zlý duch byl pryč.
#7:31 Ježíš se vrátil na území Týru a šel přes Sidón k jezeru Galilejskému územím Dekapole.
#7:32 Tu k němu přivedou člověka hluchého a špatně mluvícího a prosí ho, aby na něj vložil ruku.
#7:33 Vzal ho stranou od zástupu, vložil prsty do jeho uší, dotkl se slinou jeho jazyka,
#7:34 vzhlédl k nebi, povzdechl a řekl: „Effatha“, což znamená ‚otevři se!‘
#7:35 I otevřel se mu sluch, uvolnilo se pouto jeho jazyka a mluvil správně.
#7:36 Ježíš jim nařídil, aby to nikomu neříkali. Čím víc jim to však nařizoval, tím více to rozhlašovali.
#7:37 Nadmíru se divili a říkali: „Dobře všecko učinil. I hluchým dává sluch a němým řeč.“ 
#8:1 Když s ním v těch dnech opět byl velký zástup a neměli co jíst, zavolal si učedníky a řekl jim:
#8:2 „Je mi líto zástupu, neboť již tři dny jsou se mnou a nemají co jíst.
#8:3 Když je pošlu domů hladové, zemdlí na cestě; vždyť někteří z nich jsou zdaleka.“
#8:4 Jeho učedníci mu odpověděli: „Odkud by kdo mohl tady na poušti vzít chléb, aby všecky nasytil?“
#8:5 Zeptal se jich: „Kolik chlebů máte?“ Řekli: „Sedm.“
#8:6 Nařídil tedy zástupu usednout na zem; vzal těch sedm chlebů, vzdal díky, lámal a dával svým učedníkům, aby je předkládali; oni je předložili zástupu.
#8:7 Měli i několik rybiček; vzdal za ně díky a nařídil, aby je také předkládali.
#8:8 I jedli, nasytili se a sebrali zbylých nalámaných chlebů sedm košů.
#8:9 Těch lidí bylo asi čtyři tisíce. Pak je propustil.
#8:10 Hned na to vstoupil se svými učedníky na loď a plul do končin dalmanutských.
#8:11 I přišli farizeové a začali se s ním přít; žádali na něm znamení z nebe a tak ho pokoušeli.
#8:12 V duchu si povzdechl a řekl: „Proč toto pokolení žádá znamení? Amen, pravím vám, tomuhle pokolení nebude dáno žádné znamení.
#8:13 Odešel od nich, vstoupil opět na loď a odjel na druhý břeh.
#8:14 Zapomněli si vzít s sebou chleby; na lodi měli jen jeden chléb.
#8:15 Domlouval jim: „Hleďte se varovat kvasu farizeů a kvasu Herodova!“
#8:16 I začali mezi sebou rozmlouvat, že nemají chleba.
#8:17 Když to Ježíš zpozoroval, řekl jim: „Proč mluvíte o tom, že nemáte chleba? Ještě nerozumíte a nechápete? Je vaše mysl zatvrzelá?
#8:18 Oči máte, a nevidíte, uši máte, a neslyšíte! Nepamatujete se,
#8:19 když jsem lámal těch pět chlebů pěti tisícům, kolik plných košů nalámaných chlebů jste sebrali?“ Řekli mu: „Dvanáct.“
#8:20 „A když sedm chlebů čtyřem tisícům, kolik plných košů nalámaných chlebů jste sebrali?“ Odpověděli mu: „Sedm.“
#8:21 Řekl jim: „Ještě nechápete?“
#8:22 Přišli do Betsaidy. Přivedli k němu slepce a prosili jej, aby se ho dotkl.
#8:23 I vzal toho slepého za ruku a vyvedl ho z vesnice; potřel mu slinou oči, vložil na něho ruce a ptal se ho: „Vidíš něco?“
#8:24 On pozvedl oči a řekl: „Vidím lidi, vypadají jako stromy a chodí.“
#8:25 Potom mu znovu položil ruce na oči; slepý prohlédl, byl uzdraven a viděl všecko zcela zřetelně.
#8:26 Ježíš ho poslal domů a přikázal mu: „Ale do vesnice nechoď!“
#8:27 Ježíš se svými učedníky vyšel do vesnic u Cesareje Filipovy. Cestou se učedníků ptal: „Za koho mě lidé pokládají?“
#8:28 Řekli mu: „Za Jana Křtitele, jiní za Eliáše a někteří za jednoho z proroků.“
#8:29 Zeptal se jich: „A za koho mne pokládáte vy?“ Petr mu odpověděl: „Ty jsi Mesiáš.“
#8:30 I přikázal jim, aby nikomu o něm neříkali.
#8:31 A začal je učit, že Syn člověka musí mnoho trpět, být zavržen od starších, velekněží a zákoníků, být zabit a po třech dnech vstát.
#8:32 A mluvil o tom otevřeně. Petr si ho vzal stranou a začal ho kárat.
#8:33 On se však obrátil, podíval se na učedníky a pokáral Petra: „Jdi mi z cesty, satane!; tvé smýšlení není z Boha, ale z člověka!“
#8:34 Zavolal k sobě zástup s učedníky a řekl jim: „Kdo chce jít se mnou, zapři sám sebe, vezmi svůj kříž a následuj mne.
#8:35 Neboť kdo by chtěl zachránit svůj život, ten o něj přijde; kdo však přijde o život pro mne a pro evangelium, zachrání jej.
#8:36 Co prospěje člověku, získá-li celý svět, ale ztratí svůj život?
#8:37 Zač by mohl člověk získat zpět svůj život?
#8:38 Kdo se stydí za mne a za má slova v tomto zpronevěřilém a hříšném pokolení, za toho se bude stydět i Syn člověka, až přijde v slávě svého Otce se svatými anděly.“ 
#9:1 Řekl jim také: „Amen, pravím vám, že někteří z těch, kteří tu stojí, neokusí smrti, dokud nespatří Boží království, přicházející v moci.“
#9:2 Po šesti dnech vzal s sebou Ježíš jen Petra, Jakuba a Jana a vyvedl je na vysokou horu, kde byli sami. A byl proměněn před jejich očima.
#9:3 Jeho šat byl zářivě bílý, jak by jej žádný bělič na zemi nedovedl vybílit.
#9:4 Zjevil se jim Eliáš a Mojžíš a rozmlouvali s Ježíšem.
#9:5 Petr promluvil a řekl Ježíšovi: „Mistře, je dobré, že jsme zde; udělejme tři stany, jeden tobě, jeden Mojžíšovi a jeden Eliášovi.“
#9:6 Nevěděl, co by řekl, tak byli zděšeni.
#9:7 Tu přišel oblak a zastínil je a z oblaku se ozval hlas: „Toto jest můj milovaný Syn, toho poslouchejte.“
#9:8 Když se pak rychle rozhlédli, neviděli u sebe již nikoho jiného, než Ježíše samotného.
#9:9 Když sestupovali s hory, přikázal jim, aby nikomu nevypravovali, co viděli, dokud Syn člověka nevstane z mrtvých.
#9:10 To slovo je zaujalo a společně rozebírali, co to znamená vstát z mrtvých.
#9:11 A vyptávali se: „Proč říkají zákoníci, že napřed musí přijít Eliáš?“
#9:12 Řekl jim: „Eliáš má přijít napřed a obnovit všecko. Jak to však, že je psáno o Synu člověka, že má mnoho vytrpět a být v opovržení?
#9:13 Ale pravím vám, že Eliáš již přišel a učinili mu, co se jim zlíbilo, jak je to o něm psáno.“
#9:14 Když přišli k ostatním učedníkům, spatřili kolem nich veliký zástup a zákoníky, kteří se s nimi přeli.
#9:15 A celý zástup, jakmile ho uviděl, užasl; přibíhali k němu a zdravili ho.
#9:16 Ježíš se jich otázal: „Oč se s nimi přete?“
#9:17 Jeden člověk ze zástupu odpověděl: „Mistře, přivedl jsem k tobě svého syna, který má zlého ducha a nemůže mluvit.
#9:18 Kdekoli se ho zmocní, povalí ho a on má pěnu u úst, skřípe zuby a strne. Požádal jsem tvé učedníky, aby ducha vyhnali, ale nedokázali to.“
#9:19 Odpověděl jim: „Pokolení nevěřící, jak dlouho ještě budu s vámi? Jak dlouho vás mám ještě snášet? Přivěďte ho ke mně!“
#9:20 I přivedli ho k němu. Když ten duch Ježíše spatřil,, hned chlapce zkroutil křečí; padl na zem, svíjel se a měl pěnu u úst.
#9:21 Ježíš se zeptal jeho otce: „Od kdy to má?“ Odpověděl: „Od dětství.“
#9:22 A často jej zlý duch srazil, dokonce do ohně i do vody, aby ho zahubil. Ale můžeš-li, slituj se nad námi a pomoz nám.“
#9:23 Ježíš mu řekl: „Můžeš-li! Všechno je možné tomu, kdo věří.“
#9:24 Chlapcův otec rychle vykřikl: „Věřím, pomoz mé nedověře!“
#9:25 Když Ježíš viděl, že se sbíhá zástup, pohrozil nečistému duchu: „Duchu němý a hluchý, já ti nařizuji, vyjdi z něho a nikdy už do něho nevcházej!“
#9:26 Duch vykřikl, silně jím zalomcoval a vyšel; chlapec zůstal jako mrtvý, takže mnozí říkali, že umřel.
#9:27 Ale Ježíš ho vzal za ruku, pozvedl ho a on vstal.
#9:28 Když vešel do domu a jeho učedníci s ním byli sami, ptali se ho: „Proč jsme ho nemohli vyhnat my?“
#9:29 Řekl jim: „Takový duch nemůže vyjít jinak, než modlitbou a postem.“
#9:30 Když odtamtud vyšli, procházeli Galileou; Ježíš však nechtěl, aby se o tom vědělo,
#9:31 neboť učil své učedníky a říkal jim: „Syn člověka je vydáván do rukou lidí a zabijí ho; a až bude zabit, po třech dnech vstane.
#9:32 Oni však tomu slovu nerozuměli a báli se ho zeptat.
#9:33 Přišli do Kafarnaum. Když byl doma, ptal se jich: „O čem jste cestou uvažovali?“
#9:34 Ale oni mlčeli, neboť se cestou mezi sebou dohadovali, kdo je největší.
#9:35 Ježíš usedl, zavolal svých Dvanáct a řekl jim: „Kdo chce být první, buď ze všech poslední a služebník všech.“
#9:36 Pak vzal dítě, postavil je doprostřed nich a řekl jim:
#9:37 „Kdo přijme jedno z takových dětí v mém jménu, přijímá mne; a kdo mne přijme, nepřijímá mne, ale toho, který mě poslal.“
#9:38 Jan mu řekl: „Mistře, viděli jsme kohosi, kdo v tvém jménu vyhání démony, ale s námi nechodil; i bránili jsme mu, protože s námi nechodil.“
#9:39 Ježíš však řekl: „Nebraňte mu! Žádný, kdo učiní mocný čin v mém jménu, nemůže mi hned na to zlořečit.
#9:40 Kdo není proti nám, je pro nás.
#9:41 Kdokoli vám podá číši vody, protože jste Kristovi, amen, pravím vám, nepřijde o svou odměnu.“
#9:42 „Kdo by svedl k hříchu jednoho z těchto nepatrných, kteří ve mne věří, lépe by mu bylo, kdyby mu dali na krk mlýnský kámen a hodili ho do moře.
#9:43 Svádí-li tě k hříchu tvá ruka, utni ji; lépe je pro tebe, vejdeš-li do života zmrzačen, než abys šel s oběma rukama do pekla, do ohně neuhasitelného.
#9:44 ---
#9:45 A svádí-li tě k hříchu noha, utni ji; je lépe pro tebe, vejdeš-li do života chromý, než abys byl s oběma nohama uvržen do pekla.
#9:46 ---
#9:47 A jestliže tě svádí oko, vyloupni je; je lépe pro tebe, vejdeš-li do Božího království jednooký, než abys byl s oběma očima uvržen do pekla,
#9:48 ‚kde jejich červ neumírá a oheň nehasne.‘
#9:49 Každý bude solen ohněm.
#9:50 Sůl je dobrá, ztratí-li však svou slanost, čím ji osolíte? Mějte sůl v sobě a žijte mezi sebou v pokoji.“ 
#10:1 I vstal a šel odtamtud do judských krajin a za Jordán. Opět se k němu shromáždily zástupy, a on je zase učil, jak bylo jeho zvykem.
#10:2 Tu přišli farizeové a zkoušeli ho: ptali se ho, je-li muži dovoleno propustit manželku.
#10:3 Odpověděl jim: „Co vám ustanovil Mojžíš?“
#10:4 Řekli: „Mojžíš dovolil napsat rozlukový lístek a propustit.“
#10:5 Ježíš jim řekl: „Pro tvrdost vašeho srdce vám napsal toto ustanovení.
#10:6 Od počátku stvoření ‚Bůh učinil člověka jako muže a ženu;
#10:7 proto opustí muž svého otce i matku a připojí se ke své manželce,
#10:8 a budou ti dva jedno tělo‘; takže již nejsou dva, ale jeden.
#10:9 A proto, co Bůh spojil, člověk nerozlučuj!“
#10:10 V domě se ho učedníci znovu na tu věc ptali.
#10:11 I řekl jim: „Kdo propustí svou manželku a vezme si jinou, dopouští se vůči ní cizoložství;
#10:12 a jestliže manželka propustí svého muže a vezme si jiného, dopouští se cizoložství.“
#10:13 Tu mu přinášeli děti, aby se jich dotkl, ale učedníci jim to zakazovali.
#10:14 Když to Ježíš uviděl, rozhněval se a řekl: „Nechte děti přicházet ke mně, nebraňte jim, neboť takovým patří království Boží.
#10:15 Amen, pravím vám, kdo nepřijme Boží království jako dítě, jistě do něho nevejde.“
#10:16 Objímal je, vzkládal na ně ruce a žehnal jim.
#10:17 Když se vydával na cestu, přiběhl k němu nějaký člověk, a poklekl před ním a ptal se ho: „Mistře dobrý, co mám dělat, abych měl podíl na věčném životě?“
#10:18 Ježíš mu řekl: „Proč mi říkáš dobrý? Nikdo není dobrý, jedině Bůh.
#10:19 Přikázání znáš: Nezabiješ, nezcizoložíš, nebudeš krást, nevydáš křivé svědectví, nebudeš podvádět, cti svého otce i svou matku!“
#10:20 On mu na to řekl: „Mistře, to všechno jsem dodržoval od svého mládí.“
#10:21 Ježíš na něj s láskou pohleděl a řekl: „Jedno ti schází. Jdi, prodej všecko, co máš, rozdej chudým a budeš mít poklad v nebi; pak přijď a následuj mne!“
#10:22 On po těch slovech svěsil hlavu a smuten odešel, neboť měl mnoho majetku.
#10:23 Ježíš se rozhlédl po svých učednících a řekl jim: „Jak těžko vejdou do Božího království ti, kdo mají bohatství!“
#10:24 Učedníky ta slova zarazila. Ježíš jim ještě jednou řekl: „Dítky, jak těžké je vejít do království Božího!
#10:25 Snáze projde velbloud uchem jehly, než aby bohatý vešel do Božího království.“
#10:26 Ještě více se zhrozili a říkali si: „Kdo tedy může být spasen?“
#10:27 Ježíš na ně pohleděl a řekl: „U lidí je to nemožné, ale ne u Boha; vždyť u Boha je možné všecko.“
#10:28 Tu se Petr ozval: „Hle, my jsme opustili všecko a šli jsme za tebou.“
#10:29 Ježíš jim řekl: „Amen, pravím vám, není nikoho, kdo opustil dům nebo bratry nebo sestry nebo matku nebo otce nebo děti nebo pole pro mne a pro evangelium,
#10:30 aby nyní, v tomto čase, nedostal spolu s pronásledováním stokrát více domů, bratří, sester, matek, dětí i polí a v přicházejícím věku život věčný.
#10:31 Mnozí první budou poslední a poslední první.“
#10:32 Byli na cestě do Jeruzaléma a Ježíš šel před nimi; byli zaraženi a ti, kteří šli za nimi, se báli. Vzal k sobě opět svých Dvanáct a začal mluvit o tom, co ho má potkat:
#10:33 „Hle, jdeme do Jeruzaléma a Syn člověka bude vydán velekněžím a zákoníkům; odsoudí ho na smrt a vydají pohanům,
#10:34 budou se mu posmívat, poplivají ho, zbičují a zabijí; a po třech dnech i vstane.“
#10:35 Přistoupili k němu Jakub a Jan, synové Zebedeovi, a řekli mu: „Mistře, chtěli bychom, abys nám učinil, oč tě požádáme.“
#10:36 Řekl jim: „Co chcete, abych vám učinil?“
#10:37 Odpověděli mu: „Dej nám, abychom měli místo jeden po tvé pravici a druhý po levici v tvé slávě.“
#10:38 Ale Ježíš jim řekl: „Nevíte, oč žádáte. Můžete pít kalich, který já piji, nebo být pokřtěni křtem, kterým já jsem křtěn?“
#10:39 Odpověděli: „Můžeme.“ Ježíš jim řekl: „Kalich, který já piji, budete pít a křtem, kterým já jsem křtěn, budete pokřtěni.
#10:40 Ale udělovat místa po mé pravici či levici není má věc; ta místa patří těm, jimž jsou připravena.“
#10:41 Když to uslyšelo ostatních deset, začali se hněvat na Jakuba a Jana.
#10:42 Ježíš je zavolal k sobě a řekl jim: „Víte, že ti, kdo platí u národů za první, nad nimi panují, a kdo jsou u nich velcí, utlačují je.
#10:43 Ne tak bude mezi vámi; ale kdo se mezi vámi chce stát velkým, buď vaším služebníkem;
#10:44 a kdo chce být mezi vámi první, buď otrokem všech.
#10:45 Vždyť ani Syn člověka nepřišel, aby si dal sloužit, ale aby sloužil a dal svůj život jako výkupné za mnohé.“
#10:46 Přišli do Jericha. A když vycházel s učedníky a s velkým zástupem z Jericha, seděl u cesty syn Timaiův, Bartimaios, slepý žebrák.
#10:47 Když uslyšel, že je to Ježíš Nazaretský, dal se do křiku: „Ježíši, Synu Davidův, smiluj se nade mnou!“
#10:48 Mnozí ho napomínali, aby mlčel. On však tím více křičel: „Synu Davidův, smiluj se nade mnou!“
#10:49 Ježíš se zastavil a řekl: „Zavolejte ho!“ I zavolali toho slepého a řekli mu: „Vzchop se, vstaň, volá tě!“
#10:50 Odhodil svůj plášť, vyskočil a přišel k Ježíšovi.
#10:51 Ježíš mu řekl: „Co chceš abych pro tebe učinil?“ Slepý odpověděl: „Pane, ať vidím!“
#10:52 Ježíš mu řekl: „Jdi, tvá víra tě zachránila.“ Hned prohlédl a šel tou cestou za ním. 
#11:1 Když se blížili k Jeruzalému, k Betfage a Betanii u Olivové hory, poslal dva ze svých učedníků
#11:2 a řekl jim: „Jděte do vesnice, která je před vámi, a hned, jak do ní vejdete, naleznete přivázané oslátko, na němž dosud nikdo z lidí neseděl. Odvažte je a přiveďte!
#11:3 A řekne-li vám někdo: ‚Co to děláte?‘, odpovězte: ‚Pán je potřebuje a hned je sem zase vrátí.‘“
#11:4 Vyšli a na rozcestí nalezli oslátko přivázané venku u dveří. Když je odvazovali,
#11:5 někteří z těch, kteří tam stáli, jim řekli: „Co to děláte, že odvazujete oslátko?“
#11:6 Odpověděli jim tak, jak Ježíš přikázal, a oni je nechali.
#11:7 Oslátko přivedli Ježíšovi, přehodili přes ně své pláště a on se na ně posadil.
#11:8 Mnozí rozprostřeli na cestu své pláště a jiní zelené ratolesti z polí.
#11:9 A ti, kdo šli před ním i za ním, volali: „Hosanna!
#11:10 Požehnaný, který přichází ve jménu Hospodinově, požehnáno buď přicházející království našeho otce Davida. Hosanna na výsostech!“
#11:11 Ježíš vjel do Jeruzaléma a vešel do chrámu. Po všem se rozhlédl, a poněvadž již bylo pozdě večer, odešel s dvanácti do Betanie.
#11:12 Když vyšli druhého dne z Betanie, dostal hlad.
#11:13 Spatřil z dálky fíkovník, který měl listí, a šel se podívat, zda na něm něco nalezne. Když k němu přišel, nenalezl nic, než listí, neboť nebyl čas fíků.
#11:14 I řekl mu: „Ať z tebe již na věky nikdo nejí ovoce!“ Učedníci to slyšeli.
#11:15 Přišli do Jeruzaléma. Když vešel do chrámu, začal vyhánět prodavače a kupující v nádvoří, zpřevracel stoly směnárníků a stánky prodavačů holubů;
#11:16 nedovoloval ani to, aby někdo s čímkoliv procházel nádvořím.
#11:17 A učil je: „Což není psáno: ‚Můj dům bude zván domem modlitby pro všechny národy‘? Vy však jste z něho udělali doupě lupičů.“
#11:18 Velekněží a zákoníci to slyšeli a hledali, jak by ho zahubili; báli se ho totiž, protože všechen lid byl nadšen jeho učením.
#11:19 A když nastal večer, vyšel Ježíš s učedníky z města.
#11:20 Ráno, když šli kolem, uviděl ten fíkovník uschlý od kořenů.
#11:21 Petr se rozpomenul a řekl: „Mistře, pohleď, fíkovník, který jsi proklel, uschl.“
#11:22 Ježíš jim odpověděl: „Mějte víru v Boha!
#11:23 Amen, pravím vám, že kdo řekne této hoře: ‚Zdvihni se a vrhni do moře‘ - a nebude pochybovat, ale bude věřit, že se stane, co říká, bude to mít.
#11:24 Proto vám pravím: Věřte, že všecko, oč v modlitbě poprosíte, je vám dáno a budete to mít.
#11:25 A kdykoli povstáváte k modlitbě, odpouštějte, co proti druhým máte, aby i váš Otec, který je v nebesích, vám odpustil vaše přestoupení.“
#11:26 +Jestliže však vy neodpustíte, ani váš Otec, který je v nebesích, vám neodpustí vaše přestoupení.
#11:27 Znovu přišli do Jeruzaléma. Když procházel chrámem, přišli k němu velekněží, zákoníci a starší
#11:28 a řekli mu: „Jakou mocí to činíš? A kdo ti dal moc, abys to činil?“
#11:29 Ježíš jim řekl: „I já vám položím jednu otázku; odpovězte mně, a já vám povím, jakou mocí to činím!
#11:30 Odkud měl Jan pověření křtít? Z nebe či od lidí? Odpovězte mi!“
#11:31 I dohadovali se mezi sebou: „Řekneme-li ‚z nebe‘, namítne nám: ‚Proč jste mu tedy neuvěřili‘?
#11:32 Řekneme-li však ‚od lidí‘?“ - to se zase báli zástupu; neboť všichni měli za to, že Jan byl opravdu prorok.
#11:33 Odpověděli tedy Ježíšovi: „Nevíme.“ A Ježíš jim řekl: „Ani já vám nepovím, jakou mocí to činím.“ 
#12:1 Začal k nim mluvit v podobenstvích: „Jeden člověk vysadil vinici, obehnal ji zdí, vykopal nádrž k lisu a vystavěl strážní věž; pak ji pronajal vinařům a odcestoval.
#12:2 V stanovený čas poslal k vinařům služebníka, aby od nich vybral podíl z výnosu vinice.
#12:3 Ale oni ho chytili, zbili a poslali zpět s prázdnou.
#12:4 Opět k nim poslal jiného služebníka. Toho zpolíčkovali a zneuctili.
#12:5 Poslal dalšího, toho zabili; a mnoho jiných - jedny zbili, jiné zabili.
#12:6 Měl ještě jednoho: svého milovaného syna. Toho k nim poslal nakonec a říkal si: ‚Na mého syna budou mít přece ohled.‘
#12:7 Ale ti vinaři si mezi sebou řekli: ‚To je dědic. Pojďme a zabijme ho a dědictví bude naše!‘
#12:8 A chytili ho, zabili a vyhodili ven z vinice.
#12:9 Co udělá pán vinice? Přijde, zahubí vinaře a vinici dá jiným.
#12:10 Nečetli jste v Písmu slovo: ‚Kámen, který stavitelé zavrhli, stal se kamenem úhelným;
#12:11 Hospodin to učinil a je to podivuhodné v našich očích‘?“
#12:12 Nepřátelé se chtěli Ježíše zmocnit, neboť poznali, že to podobenství řekl proti nim, ale báli se lidu. I nechali ho a odešli.
#12:13 Poslali k němu některé z farizeů a herodiánů, aby ho chytili za slovo.
#12:14 Ti šli a řekli: „Mistře, víme, že jsi pravdivý a že se na nikoho neohlížíš; ty přece nebereš ohled na postavení člověka, nýbrž učíš cestě Boží podle pravdy. Je dovoleno dávat daň císaři, nebo ne? Máme dávat, nebo nemáme?“
#12:15 On však prohlédl jejich úskok a řekl jim: „Co mě pokoušíte? Ukažte mi denár!“
#12:16 Když mu jej podali, zeptal se jich: „Čí je tento obraz a nápis?“ Odpověděli: „Císařův.“
#12:17 Ježíš jim řekl: „Co je císařovo, odevzdejte císaři, a co je Boží, Bohu.“ Velice se nad tím podivili.
#12:18 A přišli k němu saduceové, kteří říkají, že není vzkříšení, a ptali se ho:
#12:19 „Mistře, Mojžíš nám ustanovil: ‚Zemře-li něčí bratr a zanechá manželku bezdětnou, ať se s ní ožení jeho bratr a zplodí svému bratru potomka.‘
#12:20 Bylo sedm bratří. Oženil se první a zemřel bez potomka.
#12:21 Jeho manželku si vzal druhý, ale zemřel a také nezanechal potomka. A stejně třetí.
#12:22 A nikdo z těch sedmi nezanechal potomka. Naposledy ze všech zemřela i ta žena.
#12:23 Komu z nich bude patřit, až po vzkříšení vstanou? Všech sedm ji přece mělo za manželku.“
#12:24 Ježíš jim řekl: „Mýlíte se, neznáte Písmo ani moc Boží!
#12:25 Když lidé vstanou z mrtvých, nežení se ani nevdávají, ale jsou jako nebeští andělé.
#12:26 A pokud jde o mrtvé, že vstanou, nečetli jste v knize Mojžíšově, ve vyprávění o hořícím keři, jak Bůh Mojžíšovi řekl: ‚Já jsem Bůh Abrahamův, Bůh Izákův a Bůh Jákobův‘?
#12:27 On přece není Bohem mrtvých, nýbrž Bohem živých. Velmi se mýlíte!“
#12:28 Přistoupil k němu jeden ze zákoníků, který slyšel jejich rozhovor a shledal, že jim dobře odpověděl. Zeptal se ho: „Které přikázání je první ze všech?“
#12:29 Ježíš odpověděl: „První je toto: ‚Slyš, Izraeli, Hospodin, Bůh náš, jest jediný pán;
#12:30 miluj Hospodina, Boha svého, z celého svého srdce, z celé své mysli a z celé své síly!‘
#12:31 Druhé pak je toto: ‚Miluj bližního svého jako sám sebe!‘ Většího přikázání nad tato dvě není.“
#12:32 I řekl mu ten zákoník: „Správně, Mistře, podle pravdy jsi řekl, že jest jediný Bůh a že není jiného kromě něho;
#12:33 a milovati jej z celého srdce, z celého rozumu i z celé síly a milovat bližního jako sám sebe je víc, než přinášet Bohu oběti a dary.“
#12:34 Když Ježíš viděl, že moudře odpověděl, řekl mu: „Nejsi daleko od Božího království.“ Potom se ho již nikdo otázat neodvážil.
#12:35 Když Ježíš učil v chrámu, řekl jim ještě: „Jak mohou zákoníci říkat, že Mesiáš je syn Davidův?
#12:36 Sám David řekl v Duchu svatém: ‚Řekl Hospodin mému Pánu: usedni po mé pravici, dokud ti nepoložím nepřátele pod nohy.‘
#12:37 Sám David nazývá Mesiáše Pánem, jak tedy může být jeho synem?“ Početný zástup mu rád naslouchal.
#12:38 Když učil, řekl: „Varujte se zákoníků, kteří se rádi procházejí v dlouhých řízách, stojí o pozdravy na ulicích,
#12:39 o přední sedadla v synagógách a přední místa na hostinách.
#12:40 Vyjídají domy vdov a dlouho se naoko modlí. Ty postihne tím přísnější soud.“
#12:41 Sedl si naproti chrámové pokladnici a díval se, jak do ní lidé vhazují peníze. A mnozí bohatí dávali mnoho.
#12:42 Přišla také jedna chudá vdova a vhodila dvě drobné mince, dohromady čtyrák.
#12:43 Zavolal své učedníky a řekl jim: „Amen, pravím vám, tato chudá vdova dala víc, než všichni ostatní, kteří dávali do pokladnice.
#12:44 Všichni totiž dávali ze svého nadbytku, ona však ze svého nedostatku: dala, co měla, všechno, z čeho měla být živa.“ 
#13:1 A když vycházel z chrámu, řekl mu jeden z jeho učedníků: „Pohleď, Mistře, jaké to kameny a jaké stavby!“
#13:2 Ježíš mu řekl: „Obdivuješ ty velké stavby? Nezůstane z nich kámen na kameni, všechno bude rozmetáno.“
#13:3 Když seděl na Olivové hoře naproti chrámu a byli sami, zeptali se ho Petr, Jakub, Jan a Ondřej:
#13:4 „Pověz nám, kdy to nastane a jaké bude znamení, až se začne všechno schylovat ke konci!“
#13:5 Ježíš jim odpověděl: „Mějte se na pozoru, aby vás někdo nesvedl.
#13:6 Mnozí přijdou v mém jménu a budou říkat: ‚Já jsem to‘ a svedou mnohé.
#13:7 Až uslyšíte válečný ryk a zvěsti o válkách, nelekejte se! Musí to být, ale ještě nebude konec.
#13:8 Povstane národ proti národu a království proti království, v mnohých krajinách budou zemětřesení, bude hlad. To bude teprve začátek bolestí.
#13:9 Vy sami se mějte na pozoru! Budou vás vydávat soudům, budete biti v synagógách, budete stát před vládci i králi kvůli mně, abyste před nimi vydali svědectví.
#13:10 Ale dříve musí být evangelium kázáno všem národům.
#13:11 Až vás povedou před soud, nemějte předem starost, co budete mluvit; ale co vám bude v té hodině dáno, to mluvte. Nejste to vy, kdo mluví, ale Duch svatý.
#13:12 Vydá na smrt bratr bratra a otec dítě, povstanou děti proti rodičům a připraví je o život.
#13:13 Budou vás všichni nenávidět pro mé jméno; ale kdo vytrvá až do konce, bude spasen.
#13:14 Když pak uvidíte ‚znesvěcující ohavnost‘ stát tam, kde být nemá - kdo čte, rozuměj - tehdy ti, kdo jsou v Judsku, ať uprchnou do hor;
#13:15 kdo je na střeše, ať nesestupuje a nevchází do domu, aby si odtud něco vzal;
#13:16 a kdo je na poli, ať se nevrací domů, aby si vzal plášť.
#13:17 Běda těhotným a kojícím v oněch dnech!
#13:18 Modlete se, aby to nebylo v zimě.
#13:19 S těmi dny přijde takové soužení, jaké nebylo od počátku světa, který stvořil Bůh, až do dneška nikdy nebude.
#13:20 A kdyby Pán nezkrátil ty dny, nebyl by spasen žádný člověk. Ale kvůli svým vyvoleným zkrátí ty dny.
#13:21 A tehdy, řekne-li vám někdo: ‚Hle, tu je Mesiáš, hle tam‘, nevěřte!
#13:22 Vyvstanou lžimesiášové a lžiproroci a budou předvádět znamení a zázraky, aby svedli vyvolené, kdyby to bylo možné.
#13:23 Vy však se mějte na pozoru! Všecko jsem vám řekl předem.
#13:24 Ale v těch dnech po onom soužení zatmí se slunce, a měsíc ztratí svou záři,
#13:25 hvězdy budou padat z nebe a mocnosti, které jsou v nebesích, se zachvějí.
#13:26 A tehdy uzří Syna člověka přicházet v oblacích s velikou mocí a slávou.
#13:27 Tehdy vyšle anděly a shromáždí své vyvolené od čtyř úhlů světa, od nejzazších konců země po nejzazší konec nebe.
#13:28 Od fíkovníku si vezměte poučení: Když už jeho větev raší a vyráží listí, víte, že léto je blízko.
#13:29 Tak i vy, až uvidíte, že se toto děje, vězte, že ten čas je blízko, přede dveřmi.
#13:30 Amen, pravím vám, že nepomine toto pokolení, než se to všecko stane.
#13:31 Nebe a země pominou, ale má slova nepominou.
#13:32 O onom dni či hodině neví nikdo, ani andělé v nebi, ani SYn, jenom Otec.
#13:33 Mějte se na pozoru, neboť nevíte, kdy ten čas přijde.
#13:34 Jako člověk, který je na cestách: než opustil svůj dům, dal každému služebníku odpovědnost za jeho práci a vrátnému nařídil, aby bděl.
#13:35 Bděte tedy, neboť nevíte, kdy pán domu přijde, zda večer, či o půlnoci, nebo za kuropění, nebo ráno;
#13:36 aby vás nenalezl spící, až znenadání přijde.
#13:37 Co vám říkám, říkám všem: Bděte!“ 
#14:1 Bylo dva dny před velikonocemi, svátkem nekvašených chlebů. Velekněží a zákoníci přemýšleli, jak by se Ježíše lstí zmocnili a zabili ho.
#14:2 Říkali: „Jen ne při svátečním shromáždění, aby se lid nevzbouřil.“
#14:3 Když byl v Betanii v domě Šimona Malomocného a seděl u stolu, přišla žena, která měla alabastrovou nádobku pravého vzácného oleje z nardu. Rozbila ji a olej vylila na jeho hlavu.
#14:4 Někteří se hněvali: „Nač ta ztráta oleje?
#14:5 Mohl se prodat za víc než tři sta denárů a ty se mohly dát chudým.“ A osopili se na ni.
#14:6 Ježíš však řekl: „Nechte ji! Proč ji trápíte? Vykonala na mě dobrý skutek.
#14:7 Vždyť chudé máte stále kolem sebe, a kdykoli chcete, můžete jim činit dobře; mne však nemáte stále.
#14:8 Ona učinila, co měla; už napřed pomazala mé tělo k pohřbu.
#14:9 Amen, pravím vám, všude po celém světě, kde bude kázáno evangelium, bude se mluvit na její památku také o tom, co ona učinila.“
#14:10 Jidáš Iškariotský, jeden z Dvanácti, odešel k velekněžím, aby jim ho zradil.
#14:11 Ti se zaradovali, když to slyšeli, a slíbili mu peníze. I hledal vhodnou příležitost, jak by jim ho vydal.
#14:12 Prvního dne nekvašených chlebů, když se zabíjel velikonoční beránek, řekli Ježíšovi jeho učedníci: „Kde chceš, abychom ti připravili velikonoční večeři?“
#14:13 Poslal dva ze svých učedníků a řekl jim: „Jděte do města a potká vás člověk, který nese džbán vody. Jděte za ním,
#14:14 a kam vejde, řekněte hospodáři: ‚Mistr vzkazuje: Kde je pro mne světnice, v níž bych jedl se svými učedníky velikonočního beránka?‘
#14:15 A on vám ukáže velkou horní místnost, zařízenou a připravenou; tam pro nás připravte večeři!“
#14:16 Učedníci šli, a když přišli do města, nalezli všecko, jak jim to řekl; a připravili velikonočního beránka.
#14:17 Večer přišel Ježíš s Dvanácti.
#14:18 A když byli u stolu a jedli, řekl: „Amen, pravím vám, že jeden z vás mě zradí, ten, který se mnou jí.“
#14:19 Zarmoutilo je to a začali se ho jeden po druhém ptát: „Snad ne já?“
#14:20 Řekl jim: „Jeden z Dvanácti, který se mnou namáčí chléb v téže míse.
#14:21 Syn člověka odchází, jak je o něm psáno, ale běda tomu, který Syna člověka zrazuje. Pro toho by bylo lépe, kdyby se byl vůbec nenarodil.“
#14:22 Když jedli, vzal chléb, požehnal. lámal a dával jim se slovy: „Vezměte, toto jest mé tělo.“
#14:23 Pak vzal kalich, vzdal díky, podal jim ho a pili z něho všichni.
#14:24 A řekl jim: „Toto jest má krev, která zpečeťuje smlouvu a prolévá se za mnohé.
#14:25 Amen, pravím vám, že nebudu již píti z plodu vinné révy až do toho dne, kdy budu píti nový kalich v Božím království.“
#14:26 Když zazpívali chvalozpěv, šli na Olivovou horu.
#14:27 Ježíš jim řekl: „Všichni ode mne odpadnete, neboť je psáno: ‚Budou bít pastýře a ovce se rozprchnou.‘
#14:28 Ale po svém vzkříšení vás předejdu do Galileje.“
#14:29 Tu mu Petr řekl: „I kdyby všichni odpadli, já ne.“
#14:30 Ježíš mu odpověděl: „Amen, pravím tobě, že dnes, této noci, dřív než kohout dvakrát zakokrhá, právě ty mě třikrát zapřeš.“
#14:31 On však ještě horlivěji prohlašoval: „I kdybych měl spolu s tebou umřít, nezapřu tě.“ Tak mluvili všichni.
#14:32 Přišli na místo zvané Getsemane. Ježíš řekl svým učedníkům: „Počkejte tu, než se pomodlím.“
#14:33 Pak vzal s sebou Petra, Jakuba a Jana. Přepadla ho hrůza a úzkost.
#14:34 I řekl jim: „Má duše je smutná až k smrti. Zůstaňte zde a bděte!“
#14:35 Poodešel od nich, padl na zem a modlil se, aby ho, je-li možné, minula tato hodina.
#14:36 Řekl: „Abba, Otče, tobě je všecko možné; odejmi ode mne tento kalich, ale ne, co já chci, nýbrž co ty chceš.“
#14:37 Přišel k učedníkům a zastihl je v spánku. Řekl Petrovi: „Šimone, ty spíš? Nedokázal jsi jedinou hodinu bdít?
#14:38 Bděte a modlete se, abyste neupadli do pokušení. Váš duch je odhodlán, ale tělo slabé.“
#14:39 Znovu odešel a modlil se stejnými slovy.
#14:40 A když se vrátil, zastihl je spící; oči se jim zavíraly a nevěděli, co by mu odpověděli.
#14:41 Přišel potřetí a řekl jim: „Ještě spíte a odpočíváte? Už dost! Přišla hodina, hle, Syn člověka je vydáván do rukou hříšníků.
#14:42 Vstaňte, pojďme! Hle, přibližuje se ten, který mě zrazuje.“
#14:43 Ještě ani nedomluvil a přišel Jidáš, jeden z Dvanácti. Velekněží, zákoníci a starší s ním poslali zástup, ozbrojený meči a holemi.
#14:44 Jeho zrádce s nimi domluvil znamení. Řekl jim: „Kterého políbím, ten to je; toho zatkněte a pod dozorem odveďte.“
#14:45 Když Jidáš přišel, hned k Ježíšovi přistoupil a řekl: „Mistře!“ a políbil ho.
#14:46 Oni pak na něho vztáhli ruce a zmocnili se ho.
#14:47 Tu jeden z těch, kdo stáli kolem, tasil meč, zasáhl veleknězova sluhu a uťal mu ucho.
#14:48 Ale Ježíš jim řekl: „Vyšli jste na mne jako na povstalce s meči a holemi, abyste mě zajali.
#14:49 Denně jsem vás učíval v chrámě a nezmocnili jste se mne. Ale je třeba, aby se naplnila Písma!“
#14:50 Všichni ho opustili a utekli.
#14:51 Šel za ním nějaký mladík, který měl na sobě jen kus plátna přes nahé tělo; toho chytili.
#14:52 On jim však nechal plátno v rukou a utekl.
#14:53 Ježíše odvedli k veleknězi, kde se shromáždili nejvyšší kněží, starší a zákoníci.
#14:54 Petr šel zpovzdálí za Ježíšem až dovnitř do veleknězova dvora; seděl tam spolu se sluhy a ohříval se u ohně.
#14:55 Velekněží a celá rada hledali proti Ježíšovi svědectví, aby ho mohli odsoudit k smrti, ale nenalézali.
#14:56 Mnozí sice proti němu křivě svědčili, ale svědectví se neshodovala.
#14:57 Někteří pak vystoupili a křivě proti němu svědčili:
#14:58 „Slyšeli jsme ho říkat: Já zbořím tento chrám udělaný rukama a ve třech dnech vystavím jiný, ne rukama udělaný.“
#14:59 Ale ani jejich svědectví se neshodovala.
#14:60 Tu velekněz vstal, postavil se uprostřed a otázal se Ježíše: „Nic neodpovídáš na to, co tihle proti tobě svědčí?“
#14:61 On však mlčel a nic neodpověděl. Opět se ho velekněz zeptal: „Jsi Mesiáš, Syn Požehnaného?“
#14:62 Ježíš řekl: „Já jsem. A uzříte Syna člověka sedět po pravici Všemohoucího a přicházet s oblaky nebeskými.“
#14:63 Tu velekněz roztrhl svá roucha a řekl: „Nač potřebujeme ještě svědky?
#14:64 Slyšeli jste rouhání. Co o tom soudíte?“ Oni pak všichni rozhodli, že je hoden smrti.
#14:65 Někteří na něj počali plivat, zakrývali mu obličej, bili ho po hlavě a říkali mu: „Prorokuj!“ A sluhové ho tloukli do tváře.
#14:66 Když byl Petr dole v nádvoří, přišla jedna z veleknězových služek,
#14:67 a když uviděla Petra, jak se ohřívá, pohlédla na něj a řekla: „I ty jsi byl s tím Nazaretským Ježíšem!“
#14:68 On však zapřel: „O ničem nevím a vůbec nechápu, co říkáš.“ A vyšel ven do předního dvora. V tom zakokrhal kohout.
#14:69 Ale služka ho uviděla a zase začala říkat těm, kteří stáli poblíž: „Ten člověk je z nich!“
#14:70 On však opět zapíral. A zakrátko zase ti, kdo stáli kolem, řekli Petrovi: „Jistě jsi z nich, vždyť jsi z Galileje!“
#14:71 On se však začal zaklínat a zapřísahat: „Neznám toho člověka, o němž mluvíte.“
#14:72 V tom kohout zakokrhal podruhé. Tu se Petr rozpomněl na slova, která mu Ježíš řekl: „Dřív než kohout dvakrát zakokrhá, třikrát mě zapřeš.“ A dal se do pláče. 
#15:1 A hned zrána se poradili velekněží, starší a zákoníci, celá rada; spoutali Ježíše, odvedli jej a vydali Pilátovi.
#15:2 Pilát se ho otázal: „Ty jsi král Židů?“ On mu odpověděl: „Ty sám to říkáš.“
#15:3 Velekněží na něj mnoho žalovali.
#15:4 Tu se ho Pilát znovu otázal: „Nic neodpovídáš? Pohleď, co všecko na tebe žalují!“
#15:5 Ježíš však už nic neodpověděl, takže se Pilát divil.
#15:6 O svátcích jim propouštíval jednoho vězně, o kterého žádali.
#15:7 Ve vězení byl mezi vzbouřenci, kteří se při vzpouře dopustili vraždy, muž jménem Barabáš.
#15:8 Zástup přišel Piláta požádat o to, v čem jim obvykle vyhověl.
#15:9 Pilát jim na to řekl: „Chcete, abych vám propustil židovského krále?“
#15:10 Věděl totiž, že mu ho velekněží vydali ze zášti.
#15:11 Velekněží však podnítili zástup, ať jim raději propustí Barabáše.
#15:12 Pilát se jich znovu zeptal: „Co tedy mám učinit s tím, kterému říkáte židovský král?“
#15:13 Tu se znovu dali do křiku: „Ukřižuj ho!“
#15:14 Pilát jim řekl: „A čeho se vlastně dopustil?“ Oni však ještě víc křičeli: „Ukřižuj ho!“
#15:15 Tu Pilát, aby vyhověl zástupu, propustil jim Barabáše; Ježíše dal zbičovat a vydal ho, aby byl ukřižován.
#15:16 Vojáci ho odvedli do místodržitelského dvora a svolali celou setninu.
#15:17 Navlékli mu purpurový plášť, upletli trnovou korunu, vsadili mu ji na hlavu
#15:18 a začali ho zdravit: „Buď zdráv, židovský králi!“
#15:19 Bili ho po hlavě holí, plivali na něj, klekali na kolena a padali před ním na zem.
#15:20 Když se mu dost naposmívali, svlékli mu purpurový plášť a oblékli ho zase do jeho šatů. Pak ho vyvedli ven, aby ho ukřižovali.
#15:21 Cestou přinutili nějakého Šimona z Kyrény, otce Alexandrova a Rufova, který šel z venkova, aby mu nesl kříž.
#15:22 A přivedli ho na místo zvané Golgota, což v překladu znamená ‚Lebka‘.
#15:23 Dávali mu víno okořeněné myrhou, on je však nepřijal.
#15:24 Ukřižovali ho a rozdělili si jeho šaty; losovali o ně, co si kdo vezme.
#15:25 Bylo devět hodin, když ho ukřižovali.
#15:26 Jeho provinění oznamoval nápis: „Král Židů.“
#15:27 S ním ukřižovali dva povstalce, jednoho po pravici a druhého po levici.
#15:28 ---
#15:29 Kolemjdoucí ho uráželi: potřásali hlavou a říkali: „Ty, který chceš zbořit chrám a ve třech dnech jej postavit,
#15:30 zachraň sám sebe a sestup s kříže!“
#15:31 Podobně se mu mezi sebou posmívali velekněží spolu se zákoníky. Říkali: „Jiné zachránil, sám sebe zachránit nemůže.
#15:32 Ať nyní sestoupí s kříže, ten Mesiáš, král izraelský, abychom to viděli a uvěřili!“ Tupili ho i ti, kteří byli ukřižováni spolu s ním.
#15:33 Když bylo poledne, nastala tma po celé zemi až do tří hodin.
#15:34 O třetí hodině zvolal Ježíš mocným hlasem: „Eloi, Eloi, lema sabachtani?“, což přeloženo znamená: ‚Bože můj, Bože můj, proč jsi mne opustil?‘
#15:35 Když to uslyšeli, říkali někteří z těch, kdo stáli okolo: „Hle, volá Eliáše.“
#15:36 Kdosi pak odběhl, namočil houbu v octě, nabodl ji na prut a dával mu pít se slovy: „Počkejte, uvidíme, přijde-li ho Eliáš sejmout.“
#15:37 Ale Ježíš vydal mocný hlas a skonal.
#15:38 Tu se chrámová opona roztrhla vpůli odshora až dolů.
#15:39 A když uviděl setník, který stál před ním, že takto skonal, řekl: „Ten člověk byl opravdu Syn Boží.“
#15:40 Zpovzdálí se dívaly také ženy, mezi nimi Marie z Magdaly, Marie, matka Jakuba mladšího a Josefa, a Salome,
#15:41 které ho provázely a staraly se o něj, když byl v Galileji, a mnohé jiné, které se spolu s ním vydaly do Jeruzaléma.
#15:42 A když už nastal večer - byl totiž den přípravy, předvečer soboty -
#15:43 přišel Josef z Arimatie, vážený člen rady, který také očekával království Boží, dodal si odvahy, vešel k Pilátovi a požádal o Ježíšovo tělo.
#15:44 Pilát se podivil, že Ježíš už zemřel; zavolal setníka a zeptal se ho, je-li už dlouho mrtev.
#15:45 A když mu to setník potvrdil, daroval mrtvé tělo Josefovi.
#15:46 Ten koupil plátno, sňal Ježíše z kříže, zavinul ho do plátna a položil do hrobu, který byl vytesán ve skále, a ke vchodu přivalil kámen.
#15:47 Marie z Magdaly a Marie, matka Josefova, se dívaly, kam byl uložen. 
#16:1 Když uplynula sobota, Marie z Magdaly, Marie, matka Jakubova, a Salome nakoupily vonné masti, aby ho šly pomazat.
#16:2 Brzy ráno prvního dne po sobotě, sotva vyšlo slunce, šly k hrobu.
#16:3 Říkaly si mezi sebou: „Kdo nám odvalí kámen od vchodu do hrobu?“
#16:4 Ale když vzhlédly, viděly, že kámen je odvalen; a byl velmi veliký.
#16:5 Vstoupily do hrobu a uviděly mládence, který seděl po pravé straně a měl na sobě bílé roucho; i zděsily se.
#16:6 Řekl jim: „Neděste se! Hledáte Ježíše, toho Nazaretského, který byl ukřižován. Byl vzkříšen, není zde. Hle, místo, kam ho položily.
#16:7 Ale jděte, řekněte učedníkům, zvláště Petrovi: ‚Jde před vámi do Galileje; tam ho spatříte, jak vám řekl.‘“
#16:8 Ženy šly a utíkaly od hrobu, protože na ně padla hrůza a úžas. A nikomu nic neřekly, neboť se bály.
#16:9 Když Ježíš ráno prvního dne po sobotě vstal, zjevil se nejprve Marii z Magdaly, z níž kdysi vyhnal sedm démonů.
#16:10 Ona to šla oznámit těm, kteří bývali s ním a nyní truchlili a plakali.
#16:11 Ti, když uslyšeli, že žije a že se jí ukázal, nevěřili.
#16:12 Potom se zjevil v jiné podobě dvěma z nich cestou, když šli na venkov.
#16:13 Ti to šli oznámit ostatním; ale ani těm nevěřili.
#16:14 Konečně se zjevil samým jedenácti, když byli u stolu; káral jejich nevěru a tvrdost srdce, poněvadž nevěřili těm, kteří ho viděli vzkříšeného.
#16:15 A řekl jim: „Jděte do celého světa a kažte evangelium všemu stvoření.
#16:16 Kdo uvěří a přijme křest, bude spasen; kdo však neuvěří, bude odsouzen.
#16:17 Ty, kdo uvěří, budou provázet tato znamení: Ve jménu mém budou vyhánět démony a mluvit novými jazyky;
#16:18 budou brát hady do ruky, a vypijí-li něco smrtícího, nic se jim nestane; na choré budou vzkládat ruce a uzdraví je.“
#16:19 Když jim to Pán řekl, byl vzat vzhůru do nebe a usedl po pravici Boží.
#16:20 Oni pak vyšli, všude kázali; a Pán s nimi působil a jejich slovo potvrzoval znameními.  

\book{Luke}{Luke}
#1:1 I když se již mnozí pokusili sepsat vypravování o událostech, které se mezi námi naplnily,
#1:2 jak nám je předali ti, kteří byli od počátku očitými svědky a služebníky slova,
#1:3 rozhodl jsem se také já, když jsem vše znovu důkladně prošel, že ti to v pravém sledu vypíši, vznešený Theofile,
#1:4 abys poznal hodnověrnost toho, v čem jsi byl vyučován.
#1:5 Za dnů judského krále Heroda žil kněz, jménem Zachariáš, z oddílu Abiova; měl manželku z dcer Áronových a ta se jmenovala Alžběta.
#1:6 Oba byli spravedliví před Bohem a žili bezúhonně podle všech Hospodinových příkazů a ustanovení.
#1:7 Neměli však děti, neboť Alžběta byla neplodná a oba byli již pokročilého věku.
#1:8 Když jednou přišla řada na Zachariášův oddíl a on konal před Bohem kněžskou službu,
#1:9 připadlo na něj losem podle kněžského řádu, aby vešel do svatyně Hospodinovy a obětoval kadidlo.
#1:10 Venku se v hodinu oběti modlilo veliké množství lidu.
#1:11 Tu se ukázal anděl Páně stojící po pravé straně oltáře, kde se obětovalo kadidlo.
#1:12 Když ho Zachariáš uviděl, zděsil se a padla na něho bázeň.
#1:13 Anděl mu řekl: „Neboj se, Zachariáši, neboť tvá prosba byla vyslyšena; tvá manželka Alžběta ti porodí syna a dáš mu jméno Jan.
#1:14 Budeš mít radost a veselí a mnozí se budou radovat z jeho narození.
#1:15 Bude veliký před Pánem, víno a opojný nápoj nebude pít, už od mateřského klína bude naplněn Duchem svatým.
#1:16 A mnohé ze synů izraelských obrátí k Pánu, jejich Bohu;
#1:17 sám půjde před ním v duchu a moci Eliášově, aby obrátil srdce otců k synům a vzpurné k moudrosti spravedlivých a připravil Pánu lid pohotový.“
#1:18 Zachariáš řekl andělovi: „Podle čeho to poznám? Vždyť já jsem stařec a moje žena pokročilého věku.“
#1:19 Anděl mu odpověděl: „Já jsem Gabriel, který stojí před Bohem; byl jsem poslán, abych k tobě promluvil a oznámil ti tuto radostnou zvěst.
#1:20 Hle, oněmíš a nepromluvíš až do dne, kdy se to stane, poněvadž jsi neuvěřil mým slovům, která se svým časem naplní.“
#1:21 Lid čekal na Zachariáše a divil se, že tak dlouho prodlévá v chrámě
#1:22 Když vyšel, nemohl k nim promluvit, a tak poznali, že měl v chrámě vidění; dával jim jen znamení a zůstal němý.
#1:23 Jakmile skončily dny jeho služby, odešel domů.
#1:24 Po těch dnech jeho manželka Alžběta počala, ale tajila se po pět měsíců a říkala si:
#1:25 „Toto mi učinil Pán; sklonil se ke mně v těchto dnech, aby mne zbavil mého pohanění mezi lidmi.“
#1:26 Když byla Alžběta v šestém měsíci, byl anděl Gabriel poslán od Boha do Galilejského města, které se jmenuje Nazaret,
#1:27 k panně zasnoubené muži jménem Josef, z rodu Davidova; jméno té panny bylo Maria.
#1:28 Přistoupil k ní a řekl: „Buď zdráva, milostí zahrnutá, Pán s tebou.“
#1:29 Ona se nad těmi slovy velmi zarazila a uvažovala, co ten pozdrav znamená.
#1:30 Anděl jí řekl: „Neboj se, Maria, vždyť jsi nalezla milost u Boha.
#1:31 Hle, počneš a porodíš syna a dáš mu jméno Ježíš.
#1:32 Ten bude veliký a bude nazván synem Nejvyššího a Pán Bůh mu dá trůn jeho otce Davida.
#1:33 Na věky bude kralovat na rodem Jákobovým a jeho království nebude konce.“
#1:34 Maria řekla andělovi: „Jak se to může stát, vždyť nežiji s mužem?“
#1:35 Anděl jí odpověděl: „Sestoupí na tebe Duch svatý a moc Nejvyššího tě zastíní; proto i tvé dítě bude svaté a bude nazváno Syn Boží.
#1:36 Hle, i tvá příbuzná Alžběta počala ve svém stáří syna a již je v šestém měsíci, ač se o ní říkalo, že je neplodná.
#1:37 Neboť ‚u Boha není nic nemožného‘.“
#1:38 Maria řekla: „Hle, jsem služebnice Páně; staň se mi podle tvého slova.“ Anděl pak od ní odešel.
#1:39 V těch dnech se Maria vydala na cestu a spěchala do hor do města Judova.
#1:40 Vešla do domu Zachariášova a pozdravila Alžbětu.
#1:41 Když Alžběta uslyšela Mariin pozdrav, pohnulo se dítě v jejím těle; byla naplněna Duchem svatým
#1:42 a zvolala velikým hlasem: „Požehnaná jsi nade všechny ženy a požehnaný plod tvého těla.
#1:43 Jak to, že ke mně přichází matka mého Pána?
#1:44 Hle, jakmile se zvuk tvého hlasu dotkl mých uší, pohnulo se radostí dítě v mém těle.
#1:45 A blahoslavená, která uvěřila, že se splní to, co jí bylo řečeno od Pána.“
#1:46 Maria řekla: „Duše má velebí Pána
#1:47 a můj duch jásá v Bohu, mém spasiteli,
#1:48 že se sklonil ke své služebnici v jejím ponížení. Hle, od této chvíle budou mne blahoslavit všechna pokolení,
#1:49 že se mnou učinil veliké věci ten, který je mocný. Svaté jest jeho jméno
#1:50 a milosrdenství jeho od pokolení do pokolení k těm, kdo se ho bojí.
#1:51 Prokázal sílu svým ramenem, rozptýlil ty, kdo v srdci smýšlejí pyšně;
#1:52 vladaře svrhl s trůnu a ponížené povýšil,
#1:53 hladové nasytil dobrými věcmi a bohaté poslal pryč s prázdnou.
#1:54 Ujal se svého služebníka Izraele, pamětliv svého milosrdenství,
#1:55 jež slíbil našim otcům, Abrahamovi a jeho potomkům na věky.“
#1:56 Maria zůstala s Alžbětou asi tři měsíce a pak se vrátila domů.
#1:57 Alžbětě se naplnil čas a přišla její hodina: narodil se jí syn.
#1:58 A když uslyšeli její sousedé a příbuzní, že jí Pán prokázal tak veliké milosrdenství, radovali se spolu s ní.
#1:59 Osmého dne se sešli k obřízce dítěte a chtěli mu dát jméno po otci Zachariáš.
#1:60 Jeho matka na to řekla: „Nikoli, bude se jmenovat Jan.“
#1:61 Řekli jí: „Nikdo z tvého příbuzenstva se tak nejmenuje!“
#1:62 Obrátili se na otce, jaké mu chce dát jméno.
#1:63 On požádal o tabulku a napsal na ni: Jeho jméno je Jan. A všichni se tomu divili.
#1:64 Ihned se uvolnila jeho ústa i jazyk a on mluvil a chválil Boha.
#1:65 Tu padla bázeň na všechny sousedy a všude po judských horách se mluvilo o těch událostech.
#1:66 Všichni, kteří to uslyšeli, uchovávali to v mysli a říkali: „Čím toto dítě bude?“ A ruka Hospodinova byla s ním.
#1:67 Jeho otec Zachariáš byl naplněn Duchem svatým a takto prorocky promluvil:
#1:68 „Pochválen buď Hospodin, Bůh Izraele, protože navštívil a vykoupil svůj lid
#1:69 a vzbudil nám mocného spasitele z rodu Davida, svého služebníka,
#1:70 jak mluvil ústy svatých proroků od pradávna;
#1:71 zachránil nás od našich nepřátel a z rukou těch, kteří nás nenávidí,
#1:72 slitoval se nad našimi otci a rozpomenul se na svou svatou smlouvu,
#1:73 na přísahu, kterou učinil našemu otci Abrahamovi, že nám dá,
#1:74 abychom vysvobozeni z rukou nepřátel a prosti strachu
#1:75 jej zbožně a spravedlivě ctili po všechny dny svého života.
#1:76 A ty, synu, budeš nazván prorokem Nejvyššího, neboť půjdeš před Pánem, abys mu připravil cestu
#1:77 a dal jeho lidu poznat spásu v odpuštění hříchů,
#1:78 pro slitování a milosrdenství našeho Boha, jímž nás navštíví Vycházející z výsosti,
#1:79 aby se zjevil těm, kdo jsou ve tmě a stínu smrti, a uvedl naše kroky na cestu pokoje.“
#1:80 Chlapec rostl a sílil na duchu; a žil na poušti až do dne, kdy vystoupil před Izrael. 
#2:1 Stalo se v oněch dnech, že vyšlo nařízení od císaře Augusta, aby byl po celém světě proveden soupis lidu.
#2:2 Tento první majetkový soupis se konal, když Sýrii spravoval Quirinius.
#2:3 Všichni se šli dát zapsat, každý do svého města.
#2:4 Také Josef se vydal z Galileje, města Nazareta, do Judska, do města Davidova, které se nazývá Betlém, poněvadž byl z domu a rodu Davidova,
#2:5 aby se dal zapsat s Marií, která mu byla zasnoubena a čekala dítě.
#2:6 Když tam byli, naplnily se dny a přišla její hodina.
#2:7 I porodila svého prvorozeného syna, zavinula jej do plenek a položila do jeslí, protože se pro ně nenašlo místo pod střechou.
#2:8 A v té krajině byli pastýři pod širým nebem a v noci se střídali v hlídkách u svého stáda.
#2:9 Náhle při nich stál anděl Páně a sláva Páně se rozzářila kolem nich. Zmocnila se jich veliká bázeň.
#2:10 Anděl jim řekl: „Nebojte se, hle, zvěstuji vám velikou radost, která bude pro všechen lid.
#2:11 Dnes se vám narodil Spasitel, Kristus Pán, v městě Davidově.
#2:12 Toto vám bude znamením: Naleznete děťátko v plenkách, položené do jeslí.“
#2:13 A hned tu bylo s andělem množství nebeských zástupů a takto chválili Boha:
#2:14 „Sláva na výsosti Bohu a na zemi pokoj mezi lidmi; Bůh v nich má zalíbení.“
#2:15 Jakmile andělé od nich odešli do nebe, řekli si pastýři: „Pojďme až do Betléma a podívejme se na to, co se tam stalo, jak nám Pán oznámil.“
#2:16 Spěchali tam a nalezli Marii a Josefa i to děťátko položené do jeslí.
#2:17 Když je spatřili, pověděli, co jim bylo řečeno o tom dítěti.
#2:18 Všichni, kdo to slyšeli, užasli nad tím, co pastýři vyprávěli.
#2:19 Ale Marie to všechno v mysli zachovávala a rozvažovala o tom.
#2:20 Pastýři se pak navrátili oslavujíce a chválíce Boha za všechno, co slyšeli a viděli, jak jim to bylo řečeno.
#2:21 Když uplynulo osm dní a nastal čas k obřízce, dali mu jméno Ježíš, které dostal od anděla dříve, než jej matka počala.
#2:22 Když uplynuly dny jejich očišťování podle zákona Mojžíšova, přinesli Ježíše do Jeruzaléma, aby s ním předstoupili před Hospodina -
#2:23 jak je psáno v zákoně Páně: ‚vše, co je mužského rodu a otvírá život matky, bude zasvěceno Hospodinu‘ -
#2:24 a aby podle ustanovení zákona obětovali dvě hrdličky nebo dvě holoubata.
#2:25 V Jeruzalémě žil muž jménem Simeon; byl to člověk spravedlivý a zbožný, očekával potěšení Izraele a Duch svatý byl s ním.
#2:26 Jemu bylo Duchem svatým předpověděno, že neuzří smrti, dokud nespatří Hospodinova Mesiáše.
#2:27 A tehdy veden Duchem přišel do chrámu. Když pak rodiče přinášeli Ježíše, aby splnili, co o dítěti předepisoval Zákon,
#2:28 vzal ho Simeon do náručí a takto chválil Boha:
#2:29 „Nyní propouštíš v pokoji svého služebníka, Pane, podle svého slova,
#2:30 neboť mé oči viděly tvé spasení,
#2:31 které jsi připravil přede všemi národy -
#2:32 světlo, jež bude zjevením pohanům, slávu pro tvůj lid Izrael.“
#2:33 Ježíšův otec a matka byli plni údivu nad slovy, která o něm slyšeli.
#2:34 A Simeon jim požehnal a řekl jeho matce Marii: „Hle, on jest dán k pádu i k povstání mnohých v Izraeli a jako znamení, kterému se budou vzpírat
#2:35 - i tvou vlastní duši pronikne meč - aby vyšlo najevo myšlení mnohých srdcí.“
#2:36 Žila tu i prorokyně Anna, dcera Fanuelova, z pokolení Ašerova. Byla již pokročilého věku; když se jako dívka provdala, žila se svým mužem sedm let
#2:37 a pak byla vdovou až do svého osmdesátého čtvrtého roku. Nevycházela z chrámu, ale dnem i nocí sloužila Bohu posty i modlitbami.
#2:38 A v tu chvíli k nim přistoupila, chválila Boha a mluvila o tom dítěti všem, kteří očekávali vykoupení Jeruzaléma.
#2:39 Když Josef a Maria vše řádně vykonali podle zákona Páně, vrátili se Galileje do svého města Nazareta.
#2:40 Dítě rostlo v síle a moudrosti a milost Boží byla s ním.
#2:41 Každý rok chodívali jeho rodiče o velikonočních svátcích do Jeruzaléma.
#2:42 Také když my bylo dvanáct let, šli tam, jak bylo o svátcích obyčejem.
#2:43 A když v těch dnech všechno vykonali a vraceli se domů, zůstal chlapec Ježíš v Jeruzalémě, aniž to jeho rodiče věděli.
#2:44 Protože se domnívali, že je někde s ostatními poutníky, ušli den cesty a pak jej hledali mezi svými příbuznými a známými.
#2:45 Když ho nenalezli, vrátili se a hledali ho v Jeruzalémě.
#2:46 Po třech dnech jej nalezli v chrámě, jak sedí mezi učiteli, naslouchá a dává jim otázky.
#2:47 Všichni, kteří ho slyšeli, divili se rozumnosti jeho odpovědí.
#2:48 Když ho rodiče spatřili, užasli a jeho matka mu řekla: „Synu, co jsi nám udělal? Hle, tvůj otec a já jsme tě s úzkostí hledali.“
#2:49 On jim řekl: „Jak to, že jste mě hledali? Což jste nevěděli, že musím být tam, kde jde o věc mého Otce?“
#2:50 Ale oni jeho slovu neporozuměli.
#2:51 Pak se s nimi vrátil do Nazareta a poslouchal je. Jeho matka uchovávala to vše ve svém srdci.
#2:52 A Ježíš prospíval na duchu i na těle a byl milý Bohu i lidem. 
#3:1 V patnáctém roce vlády císaře Tiberia, když Pontius Pilát spravoval Judsko a v galileji vládl Herodes, jeho bratr Filip na území Itureje a Trachonitidy a Lyzanias v Abiléně,
#3:2 za nejvyššího velekněze Annáše a Kaifáše, stalo se slovo Boží k Janovi, synu Zachariášovu, na poušti.
#3:3 I začal Jan procházet celé okolí Jordánu a kázal: „Čiňte pokání a dejte se pokřtít na odpouštění hříchů“,
#3:4 jak je psáno v knize slov proroka Izaiáše: ‚Hlas volajícího na poušti: Připravte cestu Páně, vyrovnejte mu stezky!
#3:5 Každá propast bude zasypána, hory i pahorky budou srovnány; co je křivé, bude přímé, hrbolaté cesty budou rovné;
#3:6 a každý tvor uzří spasení Boží.‘
#3:7 Zástupům, které vycházely, aby se od něho daly pokřtít, Jan říkal: „Plemeno zmijí, kdo vám ukázal, že můžete utéci před nastávajícím hněvem?
#3:8 Neste tedy ovoce, které ukazuje, že činíte pokání, a nezačínejte si říkat: ‚Náš otec jest Abraham!‘ Pravím vám, že Bůh může Abrahamovi stvořit děti z tohoto kamení.
#3:9 Sekera už je na kořeni stromů; a každý strom, který nenese dobré ovoce, bude vyťat a hozen do ohně.“
#3:10 Zástupy se Jana ptaly: „Co jen máme dělat?“
#3:11 On jim odpověděl: „Kdo má dvoje oblečení, dej tomu, kdo nemá žádné, a kdo má co k jídlu, udělej také tak.“
#3:12 Přišli i celníci, aby se dali pokřtít, a ptali se: „Mistře, co máme dělat?“
#3:13 On jim řekl: „Nevymáhejte víc, než máte nařízeno.“
#3:14 Tázali se ho i vojáci: „A co máme dělat my?“ Řekl jim: „Nikomu nečiňte násilí, nikoho nevydírejte, spokojte se se svým žoldem.“
#3:15 Lidé byli plni očekávání a všichni ve svých myslích uvažovali o Janovi, není-li on snad Mesiáš.
#3:16 Na to Jan všem řekl: „Já vás křtím vodou. Přichází však někdo silnější než jsem já; nejsem ani hoden, abych rozvázal řemínek jeho obuvi; on vás bude křtít Duchem svatým a ohněm.
#3:17 Lopata je v jeho ruce, aby pročistil svůj mlat a pšenici shromáždil do své sýpky; ale plevy spálí ohněm neuhasitelným.“
#3:18 A ještě mnohým jiným způsobem napomínal lid a kázal radostnou zvěst.
#3:19 Ale když káral vládce Heroda kvůli Herodiadě, manželce jeho bratra, a za všechno zlé, co činil,
#3:20 Herodes všechno dovršil ještě tím, že dal Jana zavřít do vězení.
#3:21 Když se všechen lid dával křtít a když byl pokřtěn i Ježíš a modlil se, otevřelo se nebe
#3:22 a Duch svatý sestoupil na něj v tělesné podobě jako holubice a z nebe se ozval hlas: „Ty jsi můj milovaný Syn, tebe jsem si vyvolil.“
#3:23 Když Ježíš začínal své dílo, bylo mu asi třicet let. Jak se mělo za to, byl syn Josefa, jehož předkové byli: Heli,
#3:24 Mathat, Levi, Melchi, Janai, Josef,
#3:25 Matathias, Amos, Nahum, Esli, Nagai,
#3:26 Mahat, Matathias, Semei, Josech, Joda,
#3:27 Johanan, Resa, Zorobabel, Salathiel, Neri,
#3:28 Melchi, Addi, Kosan, Elamadam, Er,
#3:29 Jesus, Eliezer, Jorim, Mathat, Levi,
#3:30 Simeon, Juda, Josef, Jonam, Eliakim,
#3:31 Melea, Menna, Mattath, Natham, David,
#3:32 Isaj, Obéd, Boaz, Sala, Naason,
#3:33 Amínadab, Admin, Arni, Chesróm, Fares, Juda,
#3:34 Jákob, Izák, Abraham, Tare, Nachor,
#3:35 Seruch, Ragau, Falek, Heber, Sala,
#3:36 Kainan, Arfaxad, Sem, Noé, Lámech,
#3:37 Matusalem, Henoch, Jared, Maleleel, Kainan,
#3:38 Enóš, Šét a Adam, který byl od Boha. 
#4:1 Plný Ducha svatého vrátil se Ježíš od Jordánu; Duch ho vodil po poušti
#4:2 čtyřicet dní a ďábel ho pokoušel. V těch dnech nic nejedl, a když se skončily, vyhladověl.
#4:3 Ďábel mu řekl: „Jsi-li Syn Boží, řekni tomuto kamení, ať je z něho chléb.“
#4:4 Ježíš mu řekl: „Je psáno: Člověk nebude živ jenom chlebem ‚ale každým slovem Božím.“
#4:5 Pak ho ďábel vyvedl vzhůru, v jediném okamžiku mu ukázal všechna království země
#4:6 a řekl: „Tobě dám všechnu moc i slávu těch království, poněvadž mně je dána, a komu chci, tomu ji dám:
#4:7 Budeš-li se mi klanět, bude to všechno tvé.“
#4:8 Ježíš mu odpověděl: „Je psáno: Budeš se klanět Hospodinu, Bohu svému, a jeho jediného uctívat.“
#4:9 Pak ho ďábel přivedl do Jeruzaléma, postavil ho na vrcholek chrámu a řekl mu: „Jsi-li Syn Boží, vrhni se odtud dolů;
#4:10 vždyť je psáno ‚andělům svým dá o tobě příkaz, aby tě ochránili‘
#4:11 a ‚na ruce tě vezmou, abys nenarazil nohou svou na kámen‘.“
#4:12 Ježíš mu odpověděl: „Je psáno: nebudeš pokoušet Hospodina, Boha svého.“
#4:13 Když ďábel skončil všechna svá pokušení, odešel od něho až do dané chvíle.
#4:14 Ježíš se vrátil v moci Ducha do Galileje a pověst o něm se rozšířila po celém okolí.
#4:15 Učil v jejich synagógách a všichni ho velmi chválili.
#4:16 Přišel do Nazareta, kde vyrostl. Podle svého obyčeje vešel v sobotní den do synagógy a povstal, aby četl z Písma.
#4:17 Podali mu knihu proroka Izaiáše; otevřel ji a nalezl místo, kde je psáno:
#4:18 ‚Duch Hospodinův jest nade mnou; proto mne pomazal, abych přinesl chudým radostnou zvěst; poslal mne, abych vyhlásil zajatcům propuštění a slepým vrácení zraku, abych propustil zdeptané na svobodu,
#4:19 abych vyhlásil léto milosti Hospodinovy.‘
#4:20 Pak zavřel knihu, dal ji sluhovi a posadil se; a oči všech v synagóze byly na něj upřeny.
#4:21 Promluvil k nim: „Dnes se splnilo toto Písmo, které jste právě slyšeli.“
#4:22 Všichni mu přisvědčovali a divili se slovům milosti, vycházejícím z jeho úst. A říkali: „Což to není syn Josefův?“
#4:23 On jim odpověděl: „Jistě mi řeknete toto přísloví: Lékaři, uzdrav sám sebe! O čem jsme slyšeli, že se dálo v Kafarnaum, učiň i zde,kde jsi doma.“
#4:24 Řekl: „Amen, pravím vám, žádný prorok není vítán ve své vlasti.
#4:25 Po pravdě vám říkám: Mnoho vdov bylo v Izraeli za dnů Eliášových, kdy se zavřelo nebe na tři a půl roku a na celou zemi přišel veliký hlad.
#4:26 A k žádné z nich nebyl Eliáš poslán, nýbrž jen k oné vdově do Sarepty v zemi sidonské.
#4:27 A mnoho malomocných bylo v Izraeli za proroka Elizea, a žádný z nich nebyl očištěn, jen syrský Náman.“
#4:28 Když to slyšeli, byli všichni v synagóze naplněni hněvem.
#4:29 Vstali, vyhnali ho z města a vedli až na sráz hory, na níž bylo jejich město vystavěno, aby ho svrhli dolů.
#4:30 On však prošel jejich středem a bral se dál.
#4:31 Odešel do galilejského města Kafarnaum a učil je v sobotu.
#4:32 Žasli nad jeho učením, poněvadž jeho slovo mělo moc.
#4:33 V synagóze byl člověk, který byl posedlý nečistým duchem; ten vzkřikl velikým hlasem:
#4:34 „Co je ti do nás, Ježíši Nazaretský? Přišel jsi nás zahubit? Vím kdo jsi. Jsi svatý Boží.“
#4:35 Ale Ježíš mu pohrozil: „Umlkni a vyjdi z něho!“ Zlý duch jím smýkl doprostřed a vyšel z něho, aniž mu uškodil.
#4:36 Na všechny padl úžas a říkali si navzájem: „Jaké je to slovo, že v moci a síle přikazuje nečistým duchům a oni vyjdou!“
#4:37 A pověst o něm se rozhlásila po všech místech okolní krajiny.
#4:38 Povstal a odešel ze synagógy do Šimonova domu. Šimonova tchyně byla soužena silnou horečkou; i prosili ho za ni.
#4:39 Postavil se nad ní, pohrozil horečce a ta ji opustila. Ihned vstala a obsluhovala je.
#4:40 Když slunce zapadlo, všichni k němu přiváděli své nemocné, kteří trpěli rozličnými neduhy; on vzkládal ruce na každého z nich a uzdravoval je.
#4:41 Z mnohých vycházeli i démoni a křičeli: „Ty jsi Boží Syn!“ Hrozil jim a nedovoloval jim mluvit, protože věděli, že je Mesiáš.
#4:42 Když nastal den, vyšel z domu a šel na pusté místo; zástupy ho hledaly. Přišly až k němu a zdržovaly ho, aby od nich neodcházel.
#4:43 Řekl jim: „Také ostatním městům musím zvěstovat Boží království, vždyť k tomu jsem byl poslán.“
#4:44 A kázal v judských synagógách. 
#5:1 Jednou se na něj lidé tlačili, aby slyšeli Boží slovo, a on stál u břehu jezera Genezaretského;
#5:2 tu uviděl, že u břehu jsou dvě lodi. Rybáři z nich vystoupili a vypírali sítě.
#5:3 Vstoupil do jedné z lodí, která patřila Šimonovi, a požádal ho, aby odrazil kousek od břehu. Posadil se a z lodi učil zástupy.
#5:4 Když přestal mluvit, řekl Šimonovi: „Zajeď na hlubinu a spusťte sítě k lovu!“
#5:5 Šimon mu odpověděl: „Mistře, namáhali jsme se celou noc a nic jsme nechytili. Ale na tvé slovo spustím sítě.“
#5:6 Když to učinili, zahrnuli veliké množství ryb, až se jim sítě trhaly.
#5:7 Dali znamení svým společníkům na druhé lodi, aby jim přišli na pomoc. Oni přijeli a naplnili rybami obě lodi, že se až potápěly.
#5:8 Když to Šimon Petr uviděl, padl Ježíšovi k nohám a řekl: „Odejdi ode mne, Pane, vždyť já jsem člověk hříšný.“
#5:9 Neboť jeho i všechny, kteří s ním byli, pojal úžas nad tím lovem ryb;
#5:10 stejně i Jakuba a Jana, syny Zebedeovy, kteří byli Šimonovými druhy. Ježíš řekl Šimonovi: „Neboj se, od této chvíle budeš lovit lidi.“
#5:11 Přirazili s loďmi k zemi, všechno tam nechali a šli za ním.
#5:12 V jednom městě, kam Ježíš přišel, byl muž plný malomocenství. Jakmile Ježíše spatřil, padl tváří k zemi a prosil ho: „Pane, chceš-li, můžeš mě očistit.“
#5:13 On vztáhl ruku, dotkl se ho a řekl: „Chci, buď čist.“ A hned se jeho malomocenství ztratilo.
#5:14 Přikázal mu, aby nikomu nic neříkal: „Ale jdi,“ pravil, „ukaž se knězi a obětuj za své očištění, co Mojžíš přikázal - jim na svědectví.“
#5:15 Zvěst o něm se šířila stále víc a scházely se k němu celé zástupy, aby ho slyšely a byly uzdravovány ze svých nemocí.
#5:16 On však odcházíval na pustá místa a tam se modlil.
#5:17 Jednoho dne učil a kolem seděli farizeové a učitelé Zákona, kteří se sešli ze všech galilejských a judských vesnic i z Jeruzaléma. Moc Páně byla s ním, aby uzdravoval.
#5:18 A hle, muži nesli na nosítkách člověka, který byl ochrnutý, a snažili se ho vnést dovnitř a položit před něj.
#5:19 Když viděli, že ho nepronesou zástupem, vystoupili na střechu, udělali otvor v dlaždicích a spustili ochrnutého i s lůžkem přímo před Ježíše.
#5:20 Když viděl jejich víru, řekl tomu člověku: „Tvé hříchy jsou ti odpuštěny.“
#5:21 Zákoníci a farizeové začali uvažovat: „Kdo je ten člověk, že mluví tak rouhavě? Kdo může odpustit hříchy, než sám Bůh?“
#5:22 Ježíš však poznal jejich myšlenky a odpověděl jim: „Jak to, že tak uvažujete?
#5:23 Je snadnější říci ‚Jsou ti odpuštěny tvé hříchy‘, nebo říci ‚Vstaň a choď?‘
#5:24 Abyste však věděli, že Syn člověka má moc na zemi odpouštět hříchy,“ řekl ochrnutému: „Pravím ti, vstaň, vezmi své lůžko a jdi domů.“
#5:25 A ihned před nimi vstal, vzal to, na čem ležel, šel domů a chválil Boha.
#5:26 Všechny zachvátil úžas a chválili Boha. Byli naplněni bázní a říkali: „Co jsme dnes viděli, je nad naše chápání.“
#5:27 Pak vyšel a spatřil celníka jménem Levi, jak sedí v celnici, a řekl mu: Pojď za mnou.“
#5:28 Levi nechal všechno, vstal a šel za ním.
#5:29 A ve svém domě mu připravil velikou hostinu. Bylo tam množství celníků a jiných, kteří s ním stolovali.
#5:30 Ale farizeové a jejich zákoníci reptali a řekli jeho učedníkům: „Jak to, že jíte s celníky a hříšníky?“
#5:31 Ježíš jim odpověděl: „Lékaře nepotřebují zdraví, ale nemocní.
#5:32 Nepřišel jsem volat k pokání spravedlivé, ale hříšníky.“
#5:33 Oni mu řekli: „Učedníci Janovi se často postí a konají modlitby, stejně tak i učedníci farizeů; tvoji však jedí a pijí.“
#5:34 Ježíš jim řekl: „Můžete chtít, aby se hosté na svatbě postili, když je ženich s nimi?
#5:35 Přijdou však dny, kdy od nich bude ženich vzat; potom, v těch dnech, se budou postit.“
#5:36 Řekl jim i podobenství: „Nikdo neutrhne kus látky z nového šatu a nezalátá jím starý šat; jinak bude mít díru v novém a ke starému se se záplata z nového nehodí.
#5:37 A nikdo nedává mladé víno do starých měchů; jinak mladé víno roztrhne měchy a vyteče a měchy přijdou nazmar.
#5:38 Nové víno se musí dát do nových měchů.
#5:39 Kdo se napije starého, nechce nové; řekne: ‚Staré je lepší.‘“ 
#6:1 Jednou v sobotu procházeli obilím a jeho učedníci trhali klasy, mnuli z nich rukama zrní a jedli.
#6:2 Někteří z farizeů řekli: „Jak to, že děláte, co se v sobotu nesmí?“
#6:3 Ježíš jim odpověděl: „Nečetli jste, co udělal David, když měl hlad on a ti, kdo byli s ním?
#6:4 Jak vešel do domu Božího a vzal posvátné chleby, jedl je a dal i těm, kteří ho provázeli? A to byly chleby, které nesmí jíst nikdo, jen kněží.“
#6:5 A řekl jim: „Syn člověka je pánem nad sobotou.“
#6:6 V jinou sobotu vešel do synagógy a učil. Byl tam člověk, jehož pravá ruka byla odumřelá.
#6:7 Zákoníci a farizeové si na Ježíše dávali pozor, uzdravuje-li v sobotu, aby měli proč ho obžalovat.
#6:8 On však znal jejich myšlenky. Řekl tomu muži s odumřelou rukou: „Vstaň a postav se doprostřed.“ On se zvedl a postavil se tam.
#6:9 Ježíš jim řekl: „Ptám se vás: Je dovoleno v sobotu činit dobře, či zle, život zachránit, či zahubit?“
#6:10 Rozhlédl se po nich a řekl tomu člověku: „Zvedni tu ruku!“ On to učinil a jeho ruka byla zase zdravá.
#6:11 Tu se jich zmocnila zlost a radili se spolu, co by měli s Ježíšem udělat.
#6:12 V těch dnech vyšel na horu k modlitbě; a celou noc se tam modlil k Bohu.
#6:13 Když nastal den, zavolal k sobě své učedníky a vyvolil z nich dvanáct, které také nazval apoštoly:
#6:14 Šimona, kterému dal jméno Petr, jeho bratra Ondřeje, Jakuba, Jana, Filipa, Bartoloměje,
#6:15 Matouše, Tomáše, Jakuba Alfeova, Šimona zvaného Zélóta,
#6:16 Judu Jakubova a Jidáše Iškariotského, který se pak stal zrádcem.
#6:17 Sešel s nimi dolů a na rovině zůstal stát; a s ním veliký zástup lidu z celého Judska i z Jeruzaléma, z pobřeží týrského i sidonského;
#6:18 ti všichni přišli, aby ho slyšeli a byli uzdraveni ze svých nemocí. Uzdravovali se i ti, kteří byli sužováni nečistými duchy.
#6:19 A každý ze zástupu se ho snažil dotknout, poněvadž z něho vycházela moc a uzdravovala všechny.
#6:20 Ježíš pohlédl na učedníky a řekl: „Blaze vám, chudí, neboť vaše je království Boží.
#6:21 Blaze vám, kdo nyní hladovíte, neboť budete nasyceni. Blaze vám, kdo nyní pláčete, neboť se budete smát.
#6:22 Blaze vám, když vás lidé budou nenávidět a když vás vyloučí, potupí a vymaží vaše jméno jako proklaté pro Syna člověka.
#6:23 Veselte se v ten den a jásejte radostí; hle, máte hojnou odměnu v nebi. Vždyť právě tak jednali jejich otcové s proroky.
#6:24 ALe běda vám, bohatým, vždyť vám se už potěšení dostalo.
#6:25 Běda vám, kdo jste nyní nasyceni, neboť budete hladovět. Běda, kdo se nyní smějete, neboť budete plakat a naříkat.
#6:26 Běda, když vás budou všichni lidé chválit; vždyť stejně se chovali jejich otcové k falešným prorokům.
#6:27 Ale vám, kteří mě slyšíte, pravím: Milujte své nepřátele. Dobře čiňte těm, kteří vás nenávidí.
#6:28 Žehnejte těm, kteří vás proklínají, modlete se za ty, kteří vám ubližují.
#6:29 Tomu, kdo tě udeří do tváře, nastav i druhou, a bude-li ti brát plášť, nech mu i košili!
#6:30 Každému, kdo tě prosí, dávej, a co ti někdo vezme, nepožaduj zpět.
#6:31 Jak chcete, aby lidé jednali s vámi, jednejte i vy s nimi.
#6:32 Jestliže milujete jen ty, kdo vás milují, můžete za to očekávat Boží uznání? Vždyť i hříšníci milují ty, kdo je milují.
#6:33 Činíte-li dobře těm, kdo vám dobře činí, můžete za to očekávat Boží uznání? Vždyť totéž činí i hříšníci.
#6:34 Půjčujete-li těm, u nichž je naděje, že vám to vrátí, můžete za to očekávat Boží uznání? Vždyť i hříšníci půjčují hříšníkům, aby to zase dostali nazpátek.
#6:35 Ale milujte své nepřátele; čiňte dobře, půjčujte a nic nečekejte zpět. A vaše odměna bude hojná: budete syny Nejvyššího, neboť on je dobrý k nevděčným i zlým.
#6:36 Buďte milosrdní, jako je milosrdný váš Otec.
#6:37 Nesuďte a nebudete souzeni; nezavrhujte, a nebudete zavrženi; odpouštějte a bude vám odpuštěno.
#6:38 Dávejte a bude vám dáno; dobrá míra, natlačená, natřesená, vrchovatá vám bude dána do klína. Neboť jakou měrou měříte, takovou Bůh naměří vám.“
#6:39 Řekl jim také podobenství: „Může vést slepý slepého? Nepadnou oba do jámy?
#6:40 Žák není nad učitele. Je-li zcela vyučen, bude jako jeho učitel.
#6:41 Jak to, že vidíš třísku v oku svého bratra, ale trám ve vlastním oku nepozoruješ?
#6:42 Jak můžeš říci svému bratru: ‚Bratře, dovol, ať ti vyjmu třísku, kterou máš v oku‘, a sám ve svém oku trám nevidíš? Pokrytče, nejprve vyjmi trám ze svého oka, a pak teprve prohlédneš, abys mohl vyjmout třísku z oka svého bratra.
#6:43 Dobrý strom nedává špatné ovoce a špatný strom nedává dobré ovoce.
#6:44 Každý strom se pozná po svém ovoci. Vždyť z trní nesklízejí fíky a z hloží hrozny.
#6:45 Dobrý člověk z dobrého pokladu svého srdce vydává dobré a zlý ze zlého vydává zlé. Jeho ústa mluví, čím srdce přetéká.
#6:46 Proč mne oslovujete ‚Pane, Pane‘, a nečiníte, co říkám?
#6:47 Víte, komu se podobá ten, kdo slyší tato má slova a plní je?
#6:48 Je jako člověk, který stavěl dům: kopal, hloubil, až položil základy na skálu. Když přišla povodeň, přivalil se proud na ten dům, ale nemohl jím pohnout, protože byl dobře postaven.
#6:49 Kdo však uslyšel má slova a nejednal podle nich, je jako člověk, který vystavěl dům na zemi bez základů. Když se na něj proud přivalil, hned se zřítil; a zkáza toho domu byla veliká.“ 
#7:1 Když to všechno svým posluchačům pověděl, odešel do Kafarnaum.
#7:2 Tam měl jeden setník otroka, na němž mu velmi záleželo; ten byl na smrt nemocen.
#7:3 Když setník uslyšel o Ježíšovi, poslal k němu židovské starší a žádal ho, aby přišel a zachránil život jeho otroka.
#7:4 Ti přišli k Ježíšovi a snažně ho prosili: „Je hoden, abys mu to udělal;
#7:5 neboť miluje náš národ, i synagógu nám vystavěl.“
#7:6 Ježíš šel s nimi. A když už byl nedaleko jeho domu, poslal setník své přátele se vzkazem: „Pane, neobtěžuj se; vždyť nejsem hoden, abys vstoupil pod mou střechu.
#7:7 Proto jsem se ani neodvážil k tobě přijít. Ale dej rozkaz, a můj sluha bude zdráv.
#7:8 Vždyť i já podléhám rozkazům a vojákům rozkazuji; řeknu-li některému ‚jdi‘, pak jde; jinému ‚pojď sem‘, pak přijde; a svému otroku ‚udělej to‘, pak to udělá.“
#7:9 Když to Ježíš uslyšel, podivil se, obrátil se k zástupu, který ho následoval, a řekl: „Pravím vám, že tak velikou víru jsem nenalezl ani v Izraeli.“
#7:10 Když se poslové navrátili do setníkova domu, nalezli toho otroka zdravého.
#7:11 Hned nato odešel do města, které se nazývalo Naim. S ním šli jeho učedníci veliký zástup lidí.
#7:12 Když se blížili k městské bráně, hle, vynášeli mrtvého; byl to jediný syn své matky a ta byla vdova. Velký zástup z města ji doprovázel.
#7:13 Když ji Pán uviděl, bylo mu jí líto a řekl jí: „Neplač!“
#7:14 Přistoupil k márám a dotkl se jich; ti kteří je nesli, se zastavili. Řekl: „Chlapče, pravím ti, vstaň!“
#7:15 Mrtvý se posadil a začal mluvit; Ježíš ho vrátil jeho matce.
#7:16 Všech se zmocnila bázeň, oslavovali Boha a říkali: „Veliký prorok povstal mezi námi“ a „Bůh navštívil svůj lid“.
#7:17 A tato zvěst se rozšířila o něm po celém Judsku a po celém okolí.
#7:18 O tom všem se dověděl Jan od svých učedníků. Zavolal si dva z nich
#7:19 a poslal je k Ježíšovi s otázkou: „Jsi ten, který má přijít, nebo máme čekat jiného?“
#7:20 Ti muži k němu přišli a řekli: „Poslal nás Jan Křtitel a ptá se: Jsi ten, který má přijít, nebo máme čekat jiného?“
#7:21 A v tu hodinu uzdravil Ježíš mnoho lidí z nemocí, utrpení a z moci zlých duchů a mnohým slepým daroval zrak.
#7:22 Odpověděl jim: „Jděte, zvěstujte Janovi, co jste viděli a slyšeli: Slepí vidí, chromí chodí, malomocní jsou očišťováni, hluší slyší, mrtví vstávají, chudým se zvěstuje evangelium.
#7:23 A blaze tomu, kdo se nade mnou neuráží.“
#7:24 Když pak Janovi poslové odešli, začal Ježíš o něm mluvit zástupům: „Na co jste vyšli se na poušť podívat? Na rákos, kterým kývá vítr?
#7:25 Nebo co jste vyšli shlédnout? Člověka oblečeného do drahých šatů? Ti, kdo mají skvělý šat a žijí v přepychu, jsou v královských palácích.
#7:26 Nebo co jste vyšli shlédnout? Proroka? Ano, pravím vám, a víc než proroka.
#7:27 To je ten, o němž je psáno: ‚Hle, já posílám posla před tvou tváří, aby ti připravil cestu.‘
#7:28 Pravím vám, mezi těmi, kdo se narodili z ženy, nikdo není větší než Jan; avšak i ten nejmenší v království Božím jest větší, nežli on.
#7:29 A všechen lid, který ho slyšel, i celníci dali Bohu za pravdu tím, že přijali Janův křest.
#7:30 Ale farizeové a zákoníci zavrhli úmysl, který Bůh s nimi měl, když se nedali pokřtít od Jana.
#7:31 Čemu tedy připodobním lidi tohoto pokolení? Komu se podobají?
#7:32 Jsou jako děti, které sedí na tržišti a pokřikují na sebe: ‚Hráli jsme vám, a vy jste netancovali; naříkali jsem, a vy jste neplakali,‘
#7:33 Přišel Jan Křtitel, nejedl chléb a nepil víno - říkáte: ‚Je posedlý.‘
#7:34 Přišel Syn člověka, jí a pije - říkáte: ‚Hle, milovník hodů a pitek, přítel celníků a hříšníků!‘
#7:35 Ale moudrost je ospravedlněna u všech svých dětí.“
#7:36 Jeden z farizeů pozval Ježíše k jídlu. Vešel tedy do domu toho farizea a posadil se ke stolu.
#7:37 V tom městě byla žena hříšnice. Jakmile se dověděla, že Ježíš je u stolu v domě farizeově, přišla s alabastrovou nádobkou vzácného oleje,
#7:38 s pláčem přistoupila zezadu k jeho nohám, začala mu je smáčet slzami a otírat svými vlasy, líbala je a mazala vzácným olejem.
#7:39 Když to spatřil farizeus, který ho pozval, řekl si v duchu: „Kdyby to byl prorok, musel by poznat, co to je za ženu, která se ho dotýká, že je to hříšnice.“
#7:40 Ježíš mu na to řekl: „Šimone, chci ti něco povědět.“ On řekl: „Pověz, Mistře!“ -
#7:41 „Jeden věřitel měl dva dlužníky. První byl dlužen pět set denárů, druhý padesát.
#7:42 Když neměli čím splatit dluh, odpustil oběma. Který z nich ho bude mít raději?“
#7:43 Šimon mu odpověděl: „Mám za to, že ten, kterému odpustil víc.“ Řekl mu: „Správně jsi usoudil!“
#7:44 Pak se obrátil k ženě a řekl Šimonovi: „Pohleď na tu ženu! Vešel jsem do tvého domu, ale vodu na nohy jsi mi nepodal, ona však skropila mé nohy slzami a otřela je svými vlasy.
#7:45 Nepolíbil jsi mne, ale ona od té chvíle, co jsem vešel, nepřestala líbat mé nohy.
#7:46 Nepomazal jsi mou hlavu olejem, ona však vzácným olejem pomazala mé nohy.
#7:47 Proto ti pravím: Její mnohé hříchy jsou jí odpuštěny, protože projevila velikou lásku. Komu se málo odpouští, málo miluje.“
#7:48 Řekl jí: „Jsou ti odpuštěny hříchy.“
#7:49 Ti, kteří s ním byli u stolu, si začali říkat: „Kdo to jen je, že dokonce odpouští hříchy?“
#7:50 A řekl ženě: „Tvá víra tě zachránila, jdi v pokoji!“ 
#8:1 Potom Ježíš porocházel městy a vesnicemi a přinášel radostnou zvěst o Božím království; bylo s ním dvanáct učedníků
#8:2 a některé ženy uzdravené od zlých duchů a nemocí: Marie, zvaná Magdalská, z níž vyhnal sedm démonů,
#8:3 Jana, žena Herodova správce Chuzy, Zuzana a mnohé jiné, které se o ně ze svých prostředků staraly.
#8:4 Lidé se k němu scházeli ve velkých zástupech a přicházeli z mnoha měst. Mluvil k nim v podobenství:
#8:5 „Vyšel rozsévač rozsívat semeno. Když rozsíval, padlo některé zrno podél cesty, bylo pošlapáno a ptáci je sezobali.
#8:6 Jiné padlo na skálu, vzešlo a uschlo, protože nemělo vláhu.
#8:7 Jiné padlo doprostřed trní; trní rostlo a udusilo je.
#8:8 A jiné padlo do země dobré, vzrostlo a přineslo stonásobný užitek.“ To řekl a zvolal: „Kdo má uši k slyšení, slyš!“
#8:9 Jeho učedníci se ho ptali, co to podobenství má znamenat.
#8:10 On řekl: „Vám je dáno znát tajemství Božího království, ostatním však jen v podobenstvích, aby hledíce neviděli a slyšíce nechápali.
#8:11 Toto podobenství znamená: Semenem je Boží slovo.
#8:12 Podél cesty - to jsou ti, kteří uslyší, ale pak přichází ďábel a bere slovo z jejich srdcí, aby neuvěřili a nebyli zachráněni.
#8:13 Na skále - to jsou ti, kteří s radostí přijímají slovo, když je uslyšeli; protože v nich však nezakořenilo, věří jen nějaký čas a v čas pokušení odpadají.
#8:14 Semeno padlé do trní jsou ti, kteří uslyší, ale potom je starosti, majetek a rozkoše života dusí, takže nepřinesou úrodu.
#8:15 Semeno v dobré zemi jsou ti, kteří uslyší slovo, zachovávají je v dobrém a upřímném srdci a s vytrvalostí přinášejí úrodu.
#8:16 Nikdo přece nerozsvítí světlo a nepřikryje nádobu ani je nedá pod postel, ale dá je na svícen, aby ti, kdo vcházejí, viděli světlo.
#8:17 Nic není skrytého, co jednou nebude zjeveno, a nic utajeného, co by se nepoznalo a nevyšlo najevo.
#8:18 Dávejte tedy pozor, jak slyšíte: Neboť kdo má, tomu bude dáno, a kdo nemá, tomu bude odňato i to, co myslí, že má.“
#8:19 Přišla za ním jeho matka a bratři, ale nemohli se k němu dostat pro zástup.
#8:20 Lidé mu oznámili: „Tvoje matka a bratři stojí venku a chtějí se s tebou setkat.“
#8:21 On jim odpověděl: „Má matka a moji bratři jsou ti, kdo slyší slovo Boží a podle toho jednají.“
#8:22 Jednoho dne vstoupil on i jeho učedníci na loď; řekl jim: „Přeplavme se na druhý břeh jezera.“ Když odrazili od břehu
#8:23 a plavili se, usnul. Tu se snesla na jezero bouře s vichřicí, takže nabírali vodu a byli ve velkém nebezpečí.
#8:24 Přistoupili a probudili ho se slovy: „Mistře, Mistře, zahyneme!“ On vstal, pohrozil větru a valícím se vlnám; i ustaly a bylo ticho.
#8:25 Řekl jim: „Kde je vaše víra?“ Oni se zděsili a užasli. Říkali mezi sebou: „Kdo to jen je, že rozkazuje i větru a vodám a poslouchají ho?“
#8:26 Přeplavili se do krajiny gerasenské, která leží proti Galileji.
#8:27 Když vystoupil na břeh, vyšel proti němu jakýsi muž z toho města, který byl posedlý démony a už dlouhou dobu nenosil oděv a nebydlel v domě, nýbrž v hrobech.
#8:28 Když spatřil Ježíše, vykřikl, padl před ním na zem a hlasitě zvolal: „Co je ti po mně, Ježíši, Synu Boha nejvyššího? Žádám tě, abys mne netrápil.“
#8:29 Ježíš totiž nečistému duchu přikazoval, aby z toho člověka vyšel. Neboť ho velice často zachvacoval; tehdy ho poutali řetězy a okovy a hlídali, ale on pouta vždy přerval a byl démonem hnán do pustých míst.
#8:30 Ježíš se ho zeptal: „Jaké je tvé jméno?“ Odpověděl: „Legie“, protože do něho vešlo mnoho zlých duchů.
#8:31 A prosili Ježíše, jen aby jim nepřikazoval odejít do pekelné propasti.
#8:32 Bylo tam veliké stádo vepřů, které se páslo na svahu hory. Démoni ho prosili, aby jim dovolil do nich vejít; on jim to dovolil.
#8:33 Tu vyšli z toho člověka, vešli do vepřů, a stádo se hnalo po příkré srázu do jezera a utopilo se.
#8:34 Když pasáci viděli, co se stalo, utekli a donesli o tom zprávu do města i do vesnic.
#8:35 Lidé se šli podívat, co se stalo; přišli k Ježíšovi a nalezli toho člověka, z něhož vyšli démoni, jak sedí u Ježíšových nohou oblečen a chová se rozumně. A zděsili se.
#8:36 Ti, kteří viděli, jak byl ten posedlý vysvobozen, jim o tom pověděli.
#8:37 A všichni obyvatelé gerasenské krajiny prosili Ježíše, aby od nich odešel, poněvadž se jich zmocnila veliká bázeň. Vstoupil tedy na loď, aby se vrátil.
#8:38 Ale ten muž, z něhož vyšli démoni, ho prosil, aby směl být s ním; on ho však poslal zpět a řekl mu:
#8:39 „Vrať se domů a vypravuj, jak veliké věci ti učinil Bůh.“ I odešel a zvěstoval po celém městě, jak veliké věci mu učinil Ježíš.
#8:40 Když se Ježíš vracel, zástup ho přivítal, protože na něj už všichni čekali.
#8:41 Tu k němu přišel muž, který se jmenoval Jairos; byl to představený synagógy. Padl Ježíšovi k nohám a úpěnlivě ho prosil, aby přišel do jeho domu,
#8:42 protože měl jedinou dceru, asi dvanáctiletou, a ta umírala. Když tam Ježíš šel, zástupy ho velmi tísnily.
#8:43 A byla tam žena, která měla dvanáct let krvácení a nikdo ji nemohl uzdravit.
#8:44 Přišla zezadu a dotkla se třásní jeho šatu, a hned jí krvácení přestalo.
#8:45 Ježíš řekl: „Kdo se mne to dotkl?“ Když všichni zapírali, řekl Petr a ti, kdo byli s ním: „Mistře, kolem tebe jsou zástupy a tlačí se na tebe!“
#8:46 Ale Ježíš řekl: „Někdo se mne dotkl. Já jsem poznal, že ze mne vyšla síla.“
#8:47 Když žena viděla, že se to neutají, přišla chvějíc se, padla mu k nohám a přede všemi lidmi vypověděla, proč se ho dotkla a jak hned byla uzdravena.
#8:48 On jí řekl: „Dcero, tvá víra tě zachránila, jdi v pokoji.“
#8:49 Když ještě mluvil, přišel kdosi z domu představeného synagógy a řekl: „Tvá dcera je mrtva, už Mistra neobtěžuj.“
#8:50 Ježíš to uslyšel a řekl: „Neboj se, jen věř a bude zachráněna.“
#8:51 Když přišel k němu do domu, nedovolil, aby s ním někdo šel dovnitř, jen Petr, Jan a Jakub s otcem té dívky i s její matkou.
#8:52 Všichni nad mrtvou plakali a naříkali. On řekl: „Neplačte. Nezemřela, ale spí.“
#8:53 Posmívali se mu, protože věděli, že zemřela.
#8:54 On však ji vzal za ruku a zvolal: „Dcerko, vstaň!“
#8:55 Tu se jí život vrátil a hned vstala. Nařídil, aby jí dali něco k jídlu.
#8:56 Jejích rodičů se zmocnil úžas. On jim však přikázal, aby nikomu neříkali, co se stalo. 
#9:1 Ježíš svolal svých Dvanáct a dal jim sílu a moc vyhánět všechny démony a léčit nemoci.
#9:2 Poslal je zvěstovat Boží království a uzdravovat.
#9:3 A řekl jim: „Nic si neberte na cestu, ani hůl ani mošnu ani chléb ani peníze ani dvoje šaty.
#9:4 Když přijdete do některého domu, tam zůstávejte a odtud vycházejte.
#9:5 A když vás někde nepřijmou, odejděte z onoho města a setřeste prach se svých nohou na svědectví proti nim.
#9:6 Vydali se na cestu, chodili od vesnice k vesnici, přinášeli všude radostnou zvěst a uzdravovali.
#9:7 Tetrarcha Herodes slyšel o všem, co se dálo, ale nevěděl, co si má myslit, poněvadž někteří říkali, že Jan vstal z mrtvých,
#9:8 jiní, že se zjevil Eliáš, a jiní zase, že vstal jeden z dávných proroků.
#9:9 Herodes řekl: „Jana jsem přece dal stít. Kdo to tedy je, že o něm slyším takové věci?“ A přál si ho vidět.
#9:10 Když se apoštolové navrátili, vypravovali Ježíšovi, co všechno činili. Vzal je s sebou a odešli sami do města Betsaidy.
#9:11 Když to zástupy zpozorovaly, šly za ním; on je přijal, mluvil jim o Božím království a uzdravoval ty, kteří to potřebovali.
#9:12 Schylovalo se k večeru. Tu k němu přistoupilo dvanáct učedníků a řekli: „Propusť zástup, ať jdou do okolních vesnic a dvorů opatřit si nocleh a něco k jídlu, protože jsme zde na pustém místě.“
#9:13 On jim řekl: „Dejte jim jíst vy!“ Řekli mu: „Nemáme víc než pět chlebů a dvě ryby; nebo snad máme jít a nakoupit pokrm pro všechen tento lid?“
#9:14 Bylo tam asi pět tisíc mužů. Řekl svým učedníkům: „Usaďte je ve skupinách asi po padesáti.“
#9:15 Učinili to a rozsadili je všechny.
#9:16 Potom vzal těch pět chlebů a dvě ryby, vzhlédl k nebi, vzdal díky, lámal a dával učedníkům, aby je předložili zástupu.
#9:17 A najedli a nasytili se všichni. A sebralo se zbylých nalámaných chlebů dvanáct košů.
#9:18 Když se o samotě modlil a byli s ním jeho učedníci, otázal se jich: „Za koho mne zástupy pokládají?“
#9:19 Oni mu odpověděli: „Za Jana Křtitele, jiní za Eliáše a někteří myslí, že vstal jeden z dávných proroků.“
#9:20 Řekl jim: „A za koho mne pokládáte vy?“ Petr mu odpověděl: „Za Božího Mesiáše.“
#9:21 On jim však důrazně přikázal, aby to nikomu neříkali, a pravil jim:
#9:22 „Syn člověka musí mnoho trpět, být zavržen od starších, velekněží i zákoníků, být zabit a třetího dne vzkříšen.“
#9:23 Všem pak řekl: „Kdo chce jít za mnou, zapři sám sebe, nes každého dne svůj kříž a následuj mne.
#9:24 Neboť kdo by chtěl zachránit svůj život, ten o něj přijde; kdo však přijde o život pro mne, zachrání jej.
#9:25 Jaký prospěch má člověk, který získá celý svět, ale sám sebe ztratí nebo zmaří?
#9:26 Kdo se stydí za mne a za má slova, za toho se bude stydět Syn člověka, až přijde v slávě své i Otcově a svatých andělů.
#9:27 Říkám vám po pravdě: Někteří z těch, kdo tu stojí, neokusí smrti, dokud nespatří království Boží.“
#9:28 Za týden po této rozmluvě vzal Ježíš s sebou Petra, Jana a Jakuba a vystoupil na horu, aby se modlil.
#9:29 A když se modlil, nabyla jeho tvář nového vzhledu a jeho roucho bělostně zářilo.
#9:30 A hle, rozmlouvali s ním dva muži - byli to Mojžíš a Eliáš;
#9:31 zjevili se v slávě a mluvili o cestě, kterou měl dokonat v Jeruzalémě.
#9:32 Petra a jeho druhy obestřel těžký spánek. Když se probrali, spatřili jeho slávu i ty dva muže, kteří byli s ním.
#9:33 V tom se ti muži začali od něho vzdalovat; Petr mu řekl: „Mistře, je dobré, že jsme zde; udělejme tři stany, jeden tobě, jeden Mojžíšovi a jeden Eliášovi.“ Nevěděl, co mluví.
#9:34 Než to dopověděl, přišel oblak a zastínil je. Když se ocitli v oblaku, zmocnila se jich bázeň.
#9:35 A z oblaku se ozval hlas: „Toto jest můj vyvolený Syn, toho poslouchejte.“
#9:36 Když se hlas ozval, byl už Ježíš sám. Oni umlkli a nikomu tehdy neřekli nic o tom, co viděli.
#9:37 Když příštího dne sestoupili s hory, vyšel mu vstříc veliký zástup.
#9:38 A hle, jakýsi muž ze zástupu volal: „Mistře, prosím tě, ujmi se mého syna, vždyť je to mé jediné dítě;
#9:39 hle, zachvacuje ho duch, takže znenadání vykřikuje, a lomcuje jím, až má pěnu kolem úst; jen stěží od něho odchází a tak ho moří.
#9:40 Prosil jsem už tvé učedníky, aby ho vyhnali, ale nemohli.“
#9:41 Ježíš odpověděl: „Pokolení nevěřící a zvrácené, jak dlouho ještě mám být s vámi a snášet vás? Přiveď sem svého syna!“
#9:42 Ještě než k němu přišel, démon ho povalil a zkroutil v křeči. Ježíš pohrozil nečistému duchu, uzdravil chlapce a vrátil jej otci.
#9:43 Všichni užasli nad velikou Boží mocí. Když se všichni divili, co všechno učinil, řekl svým učedníkům:
#9:44 „Slyšte a dobře si pamatujte tato slova: Syn člověka bude vydán do rukou lidí.“
#9:45 Oni však tomu slovu nerozuměli a jeho smysl jim zůstával skryt; proto nechápali, ale báli se ho na to slovo zeptat.
#9:46 Přišlo jim na mysl, kdo z nich je asi největší.
#9:47 Když Ježíš poznal, čím se obírají, vzal dítě, postavil je vedle sebe
#9:48 a řekl jim: „Kdo přijme takové dítě ve jménu mém, přijímá mne; a kdo přijme mne, přijme toho, který mne poslal. Kdo je nejmenší mezi všemi vámi, ten je veliký.“
#9:49 Jan mu na to řekl: „Mistře, viděli jsme kohosi, jak v tvém jménu vyhání démony, a bránili jsme mu, protože tě nenásleduje jako my.“
#9:50 Ježíš mu řekl: „Nebraňte mu! Kdo není proti vám, je pro vás.“
#9:51 Když se naplňovaly dny, kdy měl být vzat vzhůru, upjal svou mysl k cestě do Jeruzaléma
#9:52 a poslal před sebou posly. Vydali se na cestu a přišli do jedné samařské vesnice, aby pro něho vše připravili.
#9:53 Ale tam je nepřijali, protože jeho tvář byla obrácena k Jeruzalému.
#9:54 Když to uviděli učedníci Jakub a Jan, řekli: „Pane, máme přivolat oheň s nebe, aby je zahubil, jako to učinil Eliáš?“
#9:55 Obrátil se a pokáral je: „Nevíte, jakého jste ducha.
#9:56 Syn člověka nepřišel lidi zahubit, ale zachránit.“ A šli do jiné vesnice.
#9:57 Když se ubírali cestou, řekl mu kdosi: „Budu tě následovat, kamkoli půjdeš.“
#9:58 Ale Ježíš mu odpověděl: „Lišky mají doupata a ptáci hnízda, ale Syn člověka nemá, kam by hlavu složil.“
#9:59 Jinému řekl: „Následuj mne!“ On odpověděl: „Dovol mi Pane, abych šel napřed pochovat svého otce.“
#9:60 Řekl mu: „Nech mrtvé, ať pochovávají své mrtvé. Ale ty jdi a všude zvěstuj království Boží.“
#9:61 A jiný mu řekl: „Budu tě následovat, Pane. Ale napřed mi dovol, abych se rozloučil se svou rodinou.“
#9:62 Ježíš mu řekl: „Kdo položí ruku na pluh a ohlíží se zpět, není způsobilý pro království Boží.“ 
#10:1 Potom určil Pán ještě sedmdesát jiných a poslal je před sebou po dvou do každého města i místa, kam měl sám jít.
#10:2 Řekl jim: „Žeň je mnohá a dělníků málo. Proste proto Pána žně, ať vyšle dělníky na svou žeň.
#10:3 Jděte! Hle, posílám vás jako ovce mezi vlky.
#10:4 Neberte si měšec ani mošnu ani obuv. S nikým se na cestě nepozdravujte.
#10:5 Když vejdete do některého domu, řekněte nejprve: ‚Pokoj tomuto domu!‘
#10:6 A přijmou-li pozdrav pokoje, váš pokoj na nich spočine; ne-li, vrátí se opět k vám.
#10:7 V tom domě zůstávejte, jezte a pijte, co vám dají, neboť ‚hoden je dělník své mzdy‘! Nepřecházejte z domu do domu.
#10:8 A když přijdete do některého města a tam vás přijmou, jezte, co vám předloží;
#10:9 uzdravujte nemocné a vyřiďte jim: ‚Přiblížilo se k vám království Boží.‘
#10:10 Když však přijdete do některého města a nepřijmou vás, vyjděte do jeho ulic a řekněte:
#10:11 ‚Vytřásáme na vás i ten prach z vašeho města, který ulpěl na našich nohou! Ale to vězte: Přiblížilo se království Boží‘.
#10:12 Pravím vám, že Sodomě bude v onen den lehčeji, než tomu městu.
#10:13 Běda ti, Chorazin, běda ti, Betsaido! Kdyby se byly v Týru a Sidónu udály takové mocné skutky jako u vás, dávno by byli seděli v žíněném šatě, sypali se popelem a činili pokání.
#10:14 Ale Týru a Sidónu bude na soudu lehčeji než vám.
#10:15 A ty, Kafarnaum, budeš vyvýšeno až do nebe? Až do propasti klesneš!
#10:16 Kdo slyší vás, slyší mne, a kdo odmítá vás, odmítá mne; kdo odmítá mne, odmítá toho, který mě poslal.“
#10:17 Těch sedmdesát se vrátilo s radostí a říkali: „Pane, i démoni se nám podrobují ve tvém jménu.“
#10:18 Řekl jim: „Viděl jsem, jak satan padá s nebe jako blesk.
#10:19 Hle, dal jsem vám moc šlapat po hadech a štírech a po veškeré síle nepřítele, takže vám v ničem neuškodí.
#10:20 Ale neradujte se z toho, že se vám podrobují duchové; radujte se, že vaše jména jsou zapsána v nebesích.“
#10:21 V té hodině zajásal v Duchu svatém a řekl: „Velebím tě, Otče, Pane nebes i země, že jsi tyto věci skryl před moudrými a rozumnými, a zjevil jsi je maličkým. Ano, Otče, tak se ti zlíbilo.
#10:22 Všechno je mi dáno od mého Otce; a nikdo neví, kdo je Syn, než Otec, ani kdo je Otec, než Syn a ten, komu by to Syn chtěl zjevit.“
#10:23 Když byli sami, obrátil se na své učedníky a řekl jim: „Blahoslavené oči, které vidí, co vy vidíte.
#10:24 Říkám vám, že mnozí proroci a králové chtěli vidět, na co vy hledíte, ale neviděli; a slyšte, co vy slyšíte, ale neslyšeli.“
#10:25 Tu vystoupil jeden zákoník a zkoušel ho: „Mistře, co mám dělat, abych měl podíl na věčném životě?“
#10:26 Ježíš mu odpověděl: „Co je psáno v Zákoně? Jak to tam čteš?“
#10:27 On mu řekl: „‚Miluj Hospodina, Boha svého, z celého svého srdce, celou svou duší, celou svou silou a celou svou myslí‘ a ‚miluj svého bližního jako sám sebe‘.“
#10:28 Ježíš mu řekl: „Správně jsi odpověděl. To čiň a budeš živ.“
#10:29 Zákoník se však chtěl ospravedlnit a proto Ježíšovi řekl: „A kdo je můj bližní?“
#10:30 Ježíš mu odpověděl: „Jeden člověk šel z Jeruzaléma do Jericha a padl do rukou lupičů; ti jej obrali, zbili a nechali tam ležet polomrtvého.
#10:31 Náhodou šel tou cestou kněz, ale když ho uviděl, vyhnul se mu.
#10:32 A stejně se mu vyhnul i levita, když přišel k tomu místu a uviděl ho.
#10:33 Ale když jeden Samařan na své cestě přišel k tomu místu a uviděl ho, byl hnut soucitem;
#10:34 přistoupil k němu, ošetřil jeho rány olejem a vínem, obvázal mu je, posadil jej na svého mezka, zavezl do hostince a tam se o něj staral.
#10:35 Druhého dne dal hostinskému dva denáry a řekl: ‚Postarej se o něj, a bude-li tě to stát víc, já ti to zaplatím, až se budu vracet.‘
#10:36 Kdo z těch tří, myslíš, byl bližním tomu, který upadl mezi lupiče?“
#10:37 Zákoník odpověděl: „Ten, který prokázal milosrdenství.“ Ježíš mu řekl: „Jdi a jednej také tak.“
#10:38 Když šel Ježíš s učedníky dál, vešel do jedné vesnice. Tam jej přijala do svého domu žena jménem Marta,
#10:39 která měla sestru Marii; ta si sedla k nohám Ježíšovým a poslouchala jeho slovo.
#10:40 Ale Marta měla plno práce, aby ho obsloužila. Přišla k němu a řekla: „Pane, nezáleží ti na tom, že mne má sestra nechala sloužit samotnou? Řekni jí přece, ať mi pomůže!“
#10:41 Pán jí odpověděl: „Marto, Marto, děláš si starosti a trápíš se pro mnoho věcí.
#10:42 Je jednoho je třeba. Marie volila dobře. Vybrala si to, oč nepřijde.“ 
#11:1 Jednou se Ježíš na nějakém skrytém místě modlil; když přestal, řekl mu jeden z jeho učedníků: „Pane, nauč nás modlit se, jako tomu učil své učedníky i Jan.“
#11:2 Odpověděl jim: „Když se modlíte, říkejte: Otče náš, jenž jsi v nebesích, buď posvěceno tvé jméno. Přijď tvé království. Staň se vůle tvá jako v nebi, tak i na zemi.
#11:3 Náš denní chléb nám dávej každého dne.
#11:4 A odpusť nám naše hříchy, neboť i my odpouštíme každému, kdo se proviňuje proti nám. A nevydej nás do pokušení, ale vysvoboď nás od zlého.
#11:5 Řekl jim: „Někdo z vás bude mít přítele, půjde k němu o půlnoci a řekne mu: ‚Příteli, půjč mi tři chleby,
#11:6 protože právě teď ke mně přišel přítel, který je na cestách, a já mu nemám co dát.‘
#11:7 On mu zevnitř odpoví: ‚Neobtěžuj mne! Dveře jsou již zavřeny a děti jsou se mnou na lůžku. Nebudu přece vstávat, abych ti to dal.‘
#11:8 Pravím vám, i když nevstane a nevyhoví mu, ač je jeho přítel, vstane a vyhoví mu pro jeho neodbytnost a dám mu vše, co potřebuje.
#11:9 A tak vám pravím: Proste, a bude vám dáno; hledejte, a naleznete; tlučte, a bude vám otevřeno.
#11:10 Neboť každý, kdo prosí, dostává, a kdo hledá, nalézá, a kdo tluče, tomu bude otevřeno.
#11:11 Což je mezi vámi otec, který by dal svému synu hada, když ho prosí o rybu?
#11:12 Nebo by mu dal štíra, když ho prosí o vejce?
#11:13 Jestliže tedy vy, ač jste zlí, umíte svým dětem dávat dobré dary, čím spíše váš Otec z nebe dá Ducha svatého těm, kdo ho o to prosí!“
#11:14 Jednou vyháněl Ježíš zlého ducha z němého člověka. Když ten duch vyšel, němý promluvil. Zástupy se divily.
#11:15 Někteří z nich však řekli: „Démony vyhání ve jménu Belzebula, knížete démonů.“
#11:16 Jiní ho chtěli podrobit zkoušce; žádali od něho znamení z nebe.
#11:17 Protože znal jejich myšlenky, řekl jim: „Každé království vnitřně rozdělené pustne a dům za domem padá.
#11:18 Je-li i satan v sobě rozdvojen, jak bude moci obstát jeho království? Říkáte přece, že vyháním démony ve jménu Belzebula.
#11:19 Jestliže já vyháním démony ve jménu Belzebula, ve jménu koho je vyhánějí vaši žáci? Proto budou oni vašimi soudci.
#11:20 Jestliže však vyháním démony prstem Božím, pak už vás zastihlo Boží království.
#11:21 Střeží-li silný muž v plné zbroji svůj palác, jeho majetek je v bezpečí.
#11:22 Napadne-li ho však někdo silnější a přemůže ho, vezme mu všechnu jeho zbroj, na kterou spoléhal, a kořist rozdělí.
#11:23 Kdo není se mnou, je proti mně; kdo se mnou neshromažďuje, rozptyluje.
#11:24 Když nečistý duch vyjde z člověka, bloudí po pustých místech a hledá odpočinutí, ale když je nenalezne, řekne: ‚Vrátím se do svého domu, odkud jsem vyšel.‘
#11:25 Přijde a nalezne jej vyčištěný a uklizený.
#11:26 Tu jde a přivede sedm jiných duchů, horších než je sám, vejdou a bydlí tam; a konce toho člověka jsou horší než začátky.“
#11:27 Když toto mluvil, zvolala jedna žena ze zástupu: „Blaze té, která tě zrodila a odkojila!“
#11:28 Ale on řekl: „Blaze těm, kteří slyší slovo Boží a zachovávají je.“
#11:29 Když se u něho shromažďovaly zástupy, začal mluvit: „Toto pokolení je zlé; žádá si znamení, ale znamení mu nebude dáno, leč znamení Jonášovo.
#11:30 Jako byl Jonáš znamením pro Ninivské, tak bude i Syn člověka tomuto pokolení.
#11:31 Královna jihu povstane na soudu s muži tohoto pokolení a usvědčí je, protože ona přišla z nejzazších končin země, aby slyšela moudrost Šalomounovu - a hle, zde je více než Šalomoun.
#11:32 Mužové ninivští povstanou na soudu s tímto pokolením, neboť oni se obrátili po Jonášovu kázání - a hle, zde je více než Jonáš.
#11:33 Nikdo nerozsvítí světlo, aby je postavil do kouta nebo pod nádobu, ale dá je na svícen, aby ti, kdo vcházejí, viděli.
#11:34 Světlem tvého těla je oko. Je-li tvé oko čisté, i celé tvé tělo má světlo. Je-li však tvé oko špatné, i tvé tělo je ve tmě.
#11:35 Hleď tedy, ať světlo v tobě není tmou.
#11:36 Má-li celé tvé tělo světlo a žádná jeho část není ve tmě, bude celé tak jasné, jako když tě osvítí světlo svou září.“
#11:37 Když domluvil, pozval ho k jídlu jeden farizeus. Ježíš k němu vešel a posadil se ke stolu.
#11:38 Farizeus se podivil, když viděl, že se před jídlem nejprve neomyl.
#11:39 Ale Pán mu řekl: „Vy farizeové očišťujete číše a mísy zvenčí, ale vaše nitro je plné hrabivosti a špatnosti.
#11:40 Pošetilci! Což ten, který stvořil zevnějšek, nestvořil také to, co je uvnitř?
#11:41 Rozdejte chudým, co je v mísách, a hle, všechno vám bude čisté.
#11:42 Ale běda vám farizeům! Odevzdáváte desátky z máty, routy a ze všech zahradních rostlin, ale nedbáte na spravedlnost a lásku, kterou žádá Bůh. Toto bylo třeba činit a to ostatní neopomíjet.
#11:43 Běda vám farizeům! S oblibou sedáte na předních místech v synagógách a líbí se vám, když vás lidé na ulici zdraví.
#11:44 Běda vám! Jste jako zapomenuté hroby, po nichž lidé nahoře chodí a nevědí o nich.“
#11:45 Na to mu jeden ze zákoníků odpověděl: „Mistře, když toto říkáš, urážíš také nás!“
#11:46 On mu řekl: „I vám zákoníkům běda! Zatěžujete lidi břemeny, která nemohou unést, a sami se těch břemen nedotknete ani jediným prstem.
#11:47 Běda vám! Stavíte pomníky prorokům, které zabili vaši otcové.
#11:48 Tak dosvědčujete a potvrzujete činy svých otců: oni proroky zabíjeli, vy jim budujete pomníky.
#11:49 Proto také Moudrost Boží promluvila: Pošlu k nim proroky a apoštoly a oni je budou zabíjet a pronásledovat,
#11:50 aby tomuto pokolení byla připočtena vina za krev všech proroků prolitou od založení světa,
#11:51 od krve Ábelovy až po krev Zachariáše, který zahynul mezi oltářem a svatyní. Ano, pravím vám, tomuto pokolení bude přičtena vina.
#11:52 Běda vám zákoníkům! Vzali jste klíč poznání, sami jste nevešli, a těm, kteří chtěli vejít, jste v tom zabránili.“
#11:53 Když odtud vyšel, začali na něj zákoníci a farizeové zle dotírat a na mnohé se vyptávat,
#11:54 činíce mu tak nástrahy, aby jej mohli chytit za slovo. 
#12:1 Mezitím se shromáždily nespočetné zástupy, že se lidé div neušlapali. Ježíš začal mluvit nejprve ke svým učedníkům: „Mějte se na pozoru před kvasem farizeů, to jest před pokrytectvím.
#12:2 Nic není zahaleného, co nebude jednou odhaleno, a nic skrytého, co nebude jednou poznáno.
#12:3 Proto vše, co jste řekli ve tmě, bude slyšet na světle, co jste šeptem mluvili v tajných úkrytech, bude se hlásat ze střech.
#12:4 Říkám to vám, svým přátelům: Nebojte se těch, kdo zabíjejí tělo, ale víc už vám udělat nemohou.
#12:5 Ukážu vám, koho se máte bát. Bojte se toho, který má moc vás zabít a ještě uvrhnout do pekla. Ano, pravím vám, toho se bojte!
#12:6 Což neprodávají pět vrabců za dva haléře? A přece ani jeden z nich není zapomenut před Bohem.
#12:7 Ano i vlasy na vaší hlavě jsou spočteny. Nebojte se, máte větší cenu než mnoho vrabců.
#12:8 Pravím vám: Každý, kdo se ke mně přizná před lidmi, k tomu se i Syn člověka přizná před Božími anděly.
#12:9 Kdo mě však před lidmi zapře, bude zapřen před Božími anděly.
#12:10 Každému, kdo řekne slovo proti Synu člověka, bude odpuštěno. Avšak tomu, kdo se rouhá proti Duchu svatému, odpuštěno nebude.
#12:11 Když vás povedou do synagóg a před úřady a soudy, nedělejte si starosti, jak a čím se budete hájit a co řeknete.
#12:12 Vždyť Duch svatý vás v té hodině naučí, co je třeba říci.“
#12:13 Někdo ze zástupu ho požádal: „Mistře, domluv mému bratru, ať se rozdělí se mnou o dědictví.“
#12:14 Ježíš mu odpověděl: „Člověče, kdo mne ustanovil nad vámi soudcem nebo rozhodčím?“
#12:15 A řekl jim: „Mějte se na pozoru před každou chamtivostí, neboť i když člověk má nadbytek, není jeho život zajištěn tím, co má.“
#12:16 Pak jim pověděl toto podobenství: „Jednomu bohatému člověku se na polích hojně urodilo.
#12:17 Uvažoval o tom, a říkal si: ‚Co budu dělat, když nemám kam složit svou úrodu?‘
#12:18 Pak si řekl: ‚Tohle udělám: Zbořím stodoly, postavím větší a tam shromáždím všechno své obilí i ostatní zásoby
#12:19 a řeknu si: Teď máš velké zásoby na mnoho let; klidně si žij, jez, pij, buď veselé mysli.‘
#12:20 Ale Bůh mu řekl: ‚Blázne! Ještě této noci si vyžádají tvoji duši, a čí bude to, co jsi nashromáždil?‘
#12:21 Tak je to s tím, kdo si hromadí poklady a není bohatý před Bohem.“
#12:22 Svým učedníkům řekl: „Proto vám pravím: Nemějte starost o život, co budete jíst, ani o tělo, co budete mít na sebe.
#12:23 Život je vždycky víc než pokrm a tělo než oděv.
#12:24 Všimněte si havranů: nesejí, nežnou, nemají komory ani stodoly, a přece je Bůh živí. Oč větší cenu máte vy než ptáci!
#12:25 Kdo z vás může jen o píď prodloužit svůj život, bude-li se znepokojovat?
#12:26 Nedokážete-li tedy ani to nejmenší, proč si děláte starosti o to ostatní?
#12:27 Všimněte si lilií, jak rostou: nepředou, netkají - a pravím vám, že ani Šalomoun v celé své nádheře nebyl tak oděn jako jedna z nich.
#12:28 Jestliže tedy Bůh tak obléká trávu, která dnes je na poli a zítra bude hozena do pece, čím spíše obleče vás, malověrní!
#12:29 A neshánějte se, co budete jíst a co pít, a netrapte se tím.
#12:30 Potom všem se shánějí lidé tohoto světa. Váš Otec přece ví, že to potřebujete.
#12:31 Vy však hledejte jeho království a to ostatní vám bude přidáno.
#12:32 Nebijte se, malé stádce, neboť vašemu Otci se zalíbilo dát vám království.
#12:33 Prodejte, co máte, a rozdejte to. Opatřte si měšce, které se nerozpadnou, nevyčerpatelný poklad v nebi, kam se zloděj nedostane a mol neničí.
#12:34 Neboť kde je váš poklad, tam bude i vaše srdce.
#12:35 Buďte připraveni a vaše lampy ať hoří.
#12:36 Buďte jako lidé, kteří čekají na svého pána, až se vrátí ze svatby, aby mu hned otevřeli, až přijde a zatluče na dveře.
#12:37 Blaze těm služebníkům, které pán, až přijde, zastihne bdící. Amen, pravím vám, že se opáše, posadí je ke stolu a sám je bude obsluhovat.
#12:38 Přijde-li po půlnoci, či dokonce při rozednění a zastihne je vzhůru, blaze jim.
#12:39 Uvažte přece: kdyby hospodář věděl, v kterou hodinu přijde zloděj, nedovolil by mu vloupat se do domu.
#12:40 I vy buďte připraveni, neboť Syn člověka přijde v hodinu, kdy se toho nenadějete.“
#12:41 Petr mu řekl: „Pane, říkáš toto podobenství jenom nám, nebo všem?“
#12:42 On odpověděl: „Když Pán ustanovuje nad svým služebnictvem správce, aby jim včas rozdílel pokrm, který správce je věrný a rozumný?
#12:43 Blaze tomu služebníku, kterého pán při svém příchodu nalezne, že tak činí.
#12:44 Vpravdě pravím vám, že ho ustanoví nade vším, co mu patří.
#12:45 Když si pak onen služebník řekne: ‚Můj pán dlouho nejde‘ a začne bít sluhy i služky, jíst a pít i opíjet se,
#12:46 tu pán toho služebníka přijde v den, kdy to nečeká, a v hodinu, kterou netuší, vyžene ho a vykáže mu úděl mezi nevěrnými.
#12:47 Ten služebník, který zná vůli svého pána, a přece není hotov podle vůle jednat, bude velmi bit.
#12:48 Ten, kdo ji nezná a udělá něco, zač si zaslouží bití, bude bit méně. Komu bylo mnoho dáno, od toho se mnoho očekává, a komu mnoho svěřili, od toho budou žádat tím více.
#12:49 Oheň jsem přišel uvrhnout na zemi, a jak si přeji, aby se už vzňal!
#12:50 Křtem mám být pokřtěn, a jak je mi úzko, dokud se nedokoná!
#12:51 Myslíte, že jsem přišel dát zemi pokoj? Ne, pravím vám, ale rozdělení!
#12:52 Neboť od této chvíle bude rozděleno v jednom domě pět lidí: tři proti dvěma a dva proti třem,
#12:53 budou rozděleni otec proti synu a syn proti otci, matka proti dceři a dcera proti matce, tchyně proti snaše a snacha proti tchyni.“
#12:54 Také zástupům řekl: „Když pozorujete, že na západě vystupuje mrak, hned říkáte: ‚Přijde déšť‘ - bývá tak;
#12:55 a vane-li jižní vítr, říkáte: ‚Bude vedro‘ - a bývá.
#12:56 Pokrytci, umíte posoudit to, co vidíte na zemi i na obloze; jak to, že nedovedete rozpoznat tento čas?
#12:57 Proč nejste s to sami od sebe posoudit, co je správné?
#12:58 Když jdeš se svým protivníkem k soudu, učiň vše, aby ses s ním ještě cestou vyrovnal; jinak tě povleče k soudci, soudce tě odevzdá dozorci a dozorce tě uvrhne do vězení.
#12:59 Pravím ti, že odtud nevyjdeš, dokud nezaplatíš do posledního haléře.“ 
#13:1 Právě tehdy k němu přišli někteří se zprávou o Galilejcích, jejichž krev smísil Pilát s krví jejich obětí.
#13:2 On jim na to řekl: „Myslíte, že tito Galilejci byli větší hříšníci než ti ostatní, že to museli vytrpět?“
#13:3 Ne, pravím vám, ale nebudete-li činit pokání, všichni podobně zahynete.
#13:4 Nebo si myslíte, že těch osmnáct, na které spadla věž v Siloe a zabila je, byli větší viníci než ostatní obyvatelé Jeruzaléma?
#13:5 Ne, pravím vám, ale nebudete-li činit pokání, všichni právě tak zahynete.“
#13:6 Potom jim pověděl toto podobenství: „Jeden člověk měl na své vinici fíkovník; přišel si pro jeho ovoce, ale nic na něm nenalezl.
#13:7 Řekl vinaři: ‚Hle, už po tři léta přicházím pro ovoce z tohoto fíkovníku a nic nenalézám. Vytni jej! Proč má kazit i tu zem?‘
#13:8 On mu odpověděl: ‚Pane, ponech ho ještě tento rok, až jej okopám a pohnojím.
#13:9 Snad příště ponese ovoce; jestliže ne, dáš jej porazit‘.“
#13:10 V sobotu učil v jedné synagóze.
#13:11 Byla tam žena, která byla stižena nemocí už osmnáct let; byla úplně sehnutá a nemohla se vůbec napřímit.
#13:12 Když ji Ježíš spatřil, zavolal ji a řekl: „Ženo, jsi zproštěna své nemoci“ a vložil na ni ruce;
#13:13 ona se ihned vzpřímila a velebila Boha.
#13:14 Avšak představený synagógy, pobouřen tím, že Ježíš uzdravuje v sobotu, řekl zástupu: „Je šest dní, kdy se má pracovat; v těch tedy přicházejte, abyste byli uzdravováni, a ne v den sobotní.“
#13:15 Na to Pán odpověděl: „Pokrytci! Neodvazuje každý z vás v sobotu vola nebo osla od žlabu a nevede ho napájet?
#13:16 A tato žena, dcera Abrahamova, kterou držel satan spoutanou po osmnáct let, neměla být vysvobozena z těchto pout v den sobotní?“
#13:17 Těmito slovy byli všichni jeho protivníci zahanbeni, ale celý zástup se radoval nad podivuhodnými činy, které Ježíš konal.
#13:18 Řekl: „Čemu se podobá Boží království a k čemu je přirovnám?
#13:19 Je jako hořčičné zrno, které člověk zasel do své zahrady; vyrostlo, je z něho strom a ptáci se uhnízdili v jeho větvích.“
#13:20 A dále řekl: „K čemu přirovnám Boží království?
#13:21 Je jako kvas, který žena vmísí do tří měřic mouky, až všechno prokvasí.“
#13:22 Ježíš procházel městy i vesnicemi, učil a přitom stále směřoval k Jeruzalému.
#13:23 Kdosi mu řekl: „Pane, je opravdu málo těch, kteří budou spaseni?“ On jim odpověděl:
#13:24 „Snažte se vejít úzkými dveřmi, neboť mnozí, pravím vám, se budou snažit vejít, ale nebudou schopni.
#13:25 Jakmile už jednou hospodář vstane a zavře dveře a vy zůstanete venku, začnete tlouct na dveře a volat: ‚Pane, otevři nám‘, tu on vám odpoví: ‚Neznám vás, odkud jste!‘
#13:26 Pak budete říkat: ‚Jedli jsme s tebou i pili a na našich ulicích jsi učil!‘
#13:27 On však odpoví: ‚Neznám vás, odkud jste. Odstupte ode mne všichni, kdo se dopouštíte bezpráví.‘
#13:28 Tam bude pláč a skřípění zubů, až spatříte Abrahama, Izáka a Jákoba i všechny proroky v Božím království, a vy budete vyvrženi ven.
#13:29 A přijdou od východu i západu, od severu i jihu, a budou stolovat v Božím království.
#13:30 Hle, jsou poslední, kteří budou první, a jsou první, kteří budou poslední.“
#13:31 V tu chvíli přišli někteří farizeové a řekli mu: „Rychle odtud odejdi, protože Herodes tě chce zabít.“
#13:32 On jim řekl: „Jděte a vyřiďte té lišce: Hle, já vyháním démony a uzdravuji dnes i zítra, a třetího dne dojdu do svého cíle.
#13:33 Avšak dnes, zítra i pozítří musím jít svou cestou, neboť není možné, aby prorok zahynul mimo Jeruzalém.“
#13:34 „Jeruzaléme, Jeruzaléme, který zabíjíš proroky a kamenuješ ty, kdo byli k tobě posláni, kolikrát jsem chtěl shromáždit tvé děti, tak jako kvočna shromažďuje kuřátka pod svá křídla, ale nechtěli jste!
#13:35 Hle, ve svém domě zůstanete sami. Pravím vám, že mě neuzříte, dokud nepřijde chvíle, kdy řeknete: Požehnaný, který přichází ve jménu Hospodinově.“ 
#14:1 Jednou v sobotu vešel Ježíš do domu jednoho z předních farizeů, aby jedl u jeho stolu; a oni si na něj dávali pozor.
#14:2 Tu se před ním objevil nějaký člověk stižený vodnatelností.
#14:3 Ježíš se obrátil na zákoníky a farizeje a otázal se jich: „Je dovoleno v sobotu uzdravovat, nebo ne?“
#14:4 Oni však mlčeli. I dotkl se ho a uzdravil jej a propustil.
#14:5 Jim pak řekl: „Spadne-li někomu z vás syn nebo vůl do nádrže, nevytáhnete ho hned i v den sobotní?“
#14:6 Na to mu nedovedli dát odpověď.
#14:7 Když pozoroval, jak si hosté vybírají přední místa, pověděl jim toto podobenství:
#14:8 „Pozve-li tě někdo na svatbu, nesedej si dopředu; vždyť mezi pozvanými může být někdo váženější, než jsi ty, a ten, kdo vás oba pozval, přijde a řekne ti:
#14:9 ‚Uvolni mu své místo!‘ a ty pak musíš s hanbou dozadu.
#14:10 Ale jsi-li pozván, jdi a posaď se na poslední místo; potom přijde ten, který tě pozval a řekne ti: ‚Příteli, pojď dopředu!‘ Pak budeš mít čest přede všemi hosty.
#14:11 Neboť každý, kdo se povyšuje, bývá ponížen, a kdo se ponižuje, bude povýšen.“
#14:12 Tomu, kdo jej pozval, Ježíš řekl: „Dáváš-li oběd nebo večeři, nezvi své přátele ani své bratry ani příbuzné ani bohaté sousedy, poněvadž oni by tě také pozvali a tak by se ti dostalo odplaty.
#14:13 Ale dáváš-li hostinu, pozvi chudé, zmrzačené, chromé a slepé.
#14:14 Blaze tobě, neboť nemají, čím ti odplatit; ale bude ti odplaceno při vzkříšení spravedlivých.“
#14:15 Když to uslyšel jeden z hostí, řekl mu: „Blaze tomu, kdo bude jíst chléb v království Božím.“
#14:16 Ježíš mu řekl: „Jeden člověk chystal velikou večeři a pozval mnoho lidí.
#14:17 Když měla hostina začít, poslal svého služebníka, aby řekl pozvaným: ‚Pojďte, vše už je připraveno.‘
#14:18 A začali se jeden jako druhý vymlouvat. První mu řekl: ‚Koupil jsem pole a musím se jít na ně podívat. Prosím tě, přijmi mou omluvu.‘
#14:19 Druhý řekl: ‚Koupil jsem pět párů volů a jdu je vyzkoušet. Prosím tě, přijmi mou omluvu!‘
#14:20 Další řekl: ‚Oženil jsem se, a proto nemohu přijít.‘
#14:21 Služebník se vrátil a oznámil to svému pánu. Tu se pán domu rozhněval a řekl svému služebníku: ‚Vyjdi rychle na náměstí a do ulic města a přiveď sem chudé, zmrzačené, slepé a chromé.‘
#14:22 A služebník řekl: ‚Pane, stalo se, jak jsi rozkázal a ještě je místo.‘
#14:23 Pán řekl služebníkovi: ‚Vyjdi za lidmi na cesty a k ohradám a přinuť je, ať přijdou, aby se můj dům naplnil.
#14:24 Neboť vám pravím: Nikdo z těch mužů, kteří byli pozváni, neokusí mé večeře.‘“
#14:25 Šly s ním veliké zástupy; obrátil se k nim a řekl:
#14:26 „Kdo přichází ke mně a nedovede se zříci svého otce a matky, své ženy a dětí, svých bratrů a sester, ano i sám sebe, nemůže být mým učedníkem.
#14:27 Kdo nenese svůj kříž a nejde za mnou, nemůže být mým učedníkem.
#14:28 Chce-li někdo z vás stavět věž, což si napřed nesedne a nespočítá náklad, má-li dost na dokončení stavby?
#14:29 Jinak - až položí základ a nebude moci dokončit - vysmějí se mu všichni, kteří to uvidí.
#14:30 ‚To je ten člověk‘, řeknou, ‚který začal stavět, ale nemohl dokončit.‘
#14:31 Nebo má-li nějaký král táhnout do boje, aby se střetl s jiným králem, což nezasedne nejprve k poradě, zda se může s deseti tisíci postavit tomu, kdo táhne s dvaceti tisíci?
#14:32 Nemůže-li, vyšle poselstvo, dokud je protivník ještě daleko, a žádá o podmínky míru.
#14:33 Tak ani žádný z vás, kdo se nerozloučí se vším, co má, nemůže být mým učedníkem.
#14:34 Dobrá je sůl. Jestliže však i sůl pozbude chuti, co jí dodá slanosti?
#14:35 Nehodí se na pole ani na hnojiště: vyhodí se ven. Kdo má uši k slyšení, slyš.“ 
#15:1 Do jeho blízkosti přicházeli samí celníci a hříšníci, aby ho slyšeli.
#15:2 Farizeové a zákoníci mezi sebou reptali: „On přijímá hříšníky a jí s nimi!“
#15:3 Pověděl jim toto podobenství:
#15:4 „Má-li někdo z vás sto ovcí a ztratí jednu z nich, což nenechá těch devadesát devět na pustém místě a nejde za tou, která se ztratila, dokud ji nenalezne?
#15:5 Když ji nalezne, vezme si ji s radostí na ramena,
#15:6 a když přijde domů, svolá své přátele a sousedy a řekne jim: ‚Radujte se se mnou, protože jsem nalezl ovci, která se mi ztratila.‘
#15:7 Pravím vám, že právě tak bude v nebi větší radost nad jedním hříšníkem, který činí pokání, než nad devadesáti devíti spravedlivými, kteří pokání nepotřebují.
#15:8 Nebo má-li nějaká žena deset stříbrných mincí a ztratí jednu z nich, což nerozsvítí lampu, nevymete dům a nehledá pečlivě, dokud ji nenajde?
#15:9 A když ji nalezne, svolá své přítelkyně a sousedky a řekne: ‚Radujte se se mnou, poněvadž jsem nalezla peníz, který jsem ztratila.‘
#15:10 Pravím vám, právě tak je radost před anděly Božími nad jedním hříšníkem, který činí pokání.“
#15:11 Řekl také: „Jeden člověk měl dva syny.
#15:12 Ten mladší řekl otci: ‚Otče, dej mi díl majetku, který na mne připadá.‘ On jim rozdělil své jmění.
#15:13 Po nemnoha dnech mladší syn všechno zpeněžil, odešel do daleké země a tam rozmařilým životem svůj majetek rozházel.
#15:14 A když už všechno utratil, nastal v té zemi veliký hlad a on začal mít nouzi.
#15:15 Šel a uchytil se u jednoho občana té země; ten ho poslal na pole pást vepře.
#15:16 A byl by si chtěl naplnit žaludek slupkami, které žrali vepři, ale ani ty nedostával.
#15:17 Tu šel do sebe a řekl: ‚Jak mnoho nádeníků u mého otce má chleba nazbyt, a já tu hynu hladem!
#15:18 Vstanu, půjdu ke svému otci a řeknu mu: Otče, zhřešil jsem proti nebi i vůči tobě.
#15:19 Nejsem už hoden nazývat se tvým synem; přijmi mne jako jednoho ze svých nádeníků.‘
#15:20 I vstal a šel ke svému otci. Když byl ještě daleko, otec ho spatřil a hnut lítostí běžel k němu, objal ho a políbil.
#15:21 Syn mu řekl: ‚Otče, zhřešil jsem proti nebi i vůči tobě. Nejsem už hoden nazývat se tvým synem.‘
#15:22 Ale otec rozkázal svým služebníkům: ‚Přineste ihned nejlepší oděv a oblečte ho; dejte mu na ruku prsten a obuv na nohy.
#15:23 Přiveďte vykrmené tele, zabijte je, hodujme a buďme veselí,
#15:24 protože tento můj syn byl mrtev, a zase žije, ztratil se, a je nalezen.‘ A začali se veselit.
#15:25 Starší syn byl právě na poli. Když se vracel a byl už blízko domu, uslyšel hudbu a tanec.
#15:26 Zavolal si jednoho ze služebníků a ptal se ho, co to má znamenat.
#15:27 On mu odpověděl: ‚Vrátil se tvůj bratr, a tvůj otec dal zabít vykrmené tele, že ho zase má doma živého a zdravého.‘
#15:28 I rozhněval se a nechtěl jít dovnitř. Otec vyšel a domlouval mu.
#15:29 Ale on odpověděl: ‚Tolik let už ti sloužím a nikdy jsem neporušil žádný tvůj příkaz; a mě jsi nikdy nedal ani kůzle, abych se poveselil se svými přáteli.
#15:30 Ale když přišel tenhle tvůj syn, který s děvkami prohýřil tvé jmění, dal jsi pro něho zabít vykrmené tele.‘
#15:31 On mu řekl: ‚Synu, ty jsi stále se mnou a všecko, co mám, je tvé.
#15:32 Ale máme proč se veselit a radovat, poněvadž tento tvůj bratr byl mrtev, a zase žije, ztratil se, a je nalezen.‘“ 
#16:1 Svým učedníkům řekl: „Byl jeden bohatý člověk a ten měl správce, kterého obvinili, že špatně hospodaří s jeho majetkem.
#16:2 Zavolal ho a řekl mu: ‚Čeho ses dopustil? Slož účty ze svého správcovství, protože dále nemůžeš být správcem.‘
#16:3 Správce si řekl: ‚Co budu dělat, když mne pán zbavuje správcovství? Na práci nejsem, žebrat se stydím.
#16:4 Vím, co udělám, aby mne někde přijali do domu, až budu zbaven správcovství!‘
#16:5 Zavolal si dlužníky svého pána jednoho po druhém a řekl prvnímu: ‚Kolik jsi dlužen mému pánovi?‘ On řekl: ‚Sto věder oleje.‘
#16:6 Řekl mu: ‚Tu je tvůj úpis; rychle sedni a napiš nový na padesát.‘
#16:7 Pak řekl druhému: ‚A kolik jsi dlužen ty?‘ Odpověděl: ‚Sto měr obilí.‘ Řekl mu: ‚Tu je tvůj úpis; napiš osmdesát.‘
#16:8 Pán pochválil toho nepoctivého správce, že jednal prozíravě. Vždyť synové tohoto světa jsou vůči sobě navzájem prozíravější, než synové světla.
#16:9 Já vám pravím: I nespravedlivým mamonem si můžete získat přátele; až majetek pomine, budete přijati do věčných příbytků.
#16:10 Kdo je věrný v nejmenší věci, je věrný také ve velké; kdo je v nejmenší věci nepoctivý, je nepoctivý i ve velké.
#16:11 Jestliže jste nespravovali věrně ani nespravedlivý majetek, kdo vám svěří to pravé bohatství?
#16:12 Jestliže jste nebyli věrni v tom, co vám nepatří, kdo vám dá, co vám právem patří?
#16:13 Žádný sluha nemůže sloužit dvěma pánům. Neboť jednoho bude nenávidět, a druhého milovat, k jednomu se přidá, druhým pohrdne. Nemůžete sloužit Bohu i majetku.“
#16:14 Toto slyšeli farizeové, kteří měli rádi peníze, a posmívali se mu.
#16:15 Řekl jim: „Vy před lidmi vystupujete jako spravedliví, ale Bůh zná vaše srdce: neboť co lidé cení vysoko, je před Bohem ohavnost.
#16:16 Zákon a proroci až do Jana; od té chvíle se zvěstuje království Boží a každý si do něho vynucuje vstup.
#16:17 Spíše pomine nebe a země, než aby padla jediná čárka Zákona.
#16:18 Každý, kdo propouští svou manželku a vezme si jinou, cizoloží; kdo se ožení s tou, kterou muž pustil, cizoloží.
#16:19 Byl jeden bohatý člověk, nádherně a vybraně se strojil a den co den skvěle hodoval.
#16:20 U vrat jeho domu lehával nějaký chudák, jménem Lazar, plný vředů,
#16:21 a toužil nasytit se aspoň tím, co spadlo se stolu toho boháče; dokonce přibíhali psi a olizovali jeho vředy.
#16:22 I umřel ten chudák a andělé ho přenesli k Abrahamovi; zemřel i ten boháč a byl pohřben.
#16:23 A když v pekle pozdvihl v mukách oči, uviděl v dáli Abrahama a u něho Lazara.
#16:24 Tu zvolal: ‚Otče Abrahame, smiluj se nade mnou a pošli Lazara, ať omočí aspoň špičku prstu ve vodě a svlaží mé rty, neboť se trápím v tomto plameni.‘
#16:25 Abraham řekl: ‚Synu, vzpomeň si, že se ti dostalo všeho dobrého už za tvého života, a Lazarovi naopak všeho zlého. Nyní on se raduje a ty trpíš.
#16:26 A nad to vše je mezi námi a vámi veliká propast, takže nikdo - i kdyby chtěl, nemůže odtud k vám ani překročit od vás k nám.‘
#16:27 Řekl: ‚Prosím tě tedy, otče, pošli jej do mého rodného domu,
#16:28 neboť mám pět bratrů, ať je varuje, aby také oni nepřišli do tohoto místa muk.‘
#16:29 Ale Abraham mu odpověděl: ‚Mají Mojžíše a Proroky, ať je poslouchají!‘
#16:30 On řekl: ‚Ne tak, otče Abrahame, ale přijde-li k nim někdo z mrtvých, budou činit pokání.‘
#16:31 Řekl mu: ‚Neposlouchají-li Mojžíše a Proroky, nedají se přesvědčit, ani kdyby někdo vstal z mrtvých.‘“ 
#17:1 Ježíš řekl svým učedníkům: „Není možné, aby nepřišla pokušení; běda však tomu, skrze koho přicházejí.
#17:2 Bylo by pro něho lépe, kdyby mu dali na krk mlýnský kámen a uvrhli ho do moře, než aby svedl k hříchu jednoho z těchto nepatrných.
#17:3 Mějte se na pozoru! Když tvůj bratr zhřeší, pokárej ho, a bude-li toho litovat, odpusť mu.
#17:4 A jestliže proti tobě zhřeší sedmkrát za den, a sedmkrát k tobě přijde s prosbou: ‚Je mi to líto‘, odpustíš mu!“
#17:5 Apoštolové řekli Pánu: „Dej nám více víry!“
#17:6 Pán jim řekl: „Kdybyste měli víru jako zrnko hořčice, řekli byste této moruši: ‚Vyrvi se i s kořeny a přesaď se do moře‘, a ona by vás poslechla.“
#17:7 „Řekne snad někdo svému služebníku, který se vrátil z pole, kde oral nebo pásl: ‚Pojď si hned sednout ke stolu‘?
#17:8 Neřekne mu spíše: ‚Připrav mi něco k jídlu a přistroj se k obsluze, dokud se nenajím a nenapiji; pak budeš jíst a pít ty!‘?
#17:9 Děkuje snad svému služebníku, že udělal, co mu bylo přikázáno?
#17:10 Tak i vy, když učiníte všechno, co vám bylo přikázáno, řekněte: ‚Jsme jenom služebníci, učinili jsme to, co jsme byli povinni učinit.‘“
#17:11 Na cestě do Jeruzaléma procházel Samařskem a Galileou.
#17:12 Když přicházel k jedné vesnici, šlo mu vstříc deset malomocných; zůstali stát opodál
#17:13 a hlasitě volali: „Ježíš, Mistře, smiluj se nad námi!“
#17:14 Když je uviděl, řekl jim: „Jděte a ukažte se kněžím!“ A když tam šli, byli očištěni.
#17:15 Jeden z nich, jakmile zpozoroval, že je uzdraven, hned se vrátil a velikým hlasem velebil Boha;
#17:16 padl tváří k Ježíšovým nohám a děkoval mu. A byl to Samařan.
#17:17 Nato Ježíš řekl: „Nebylo jich očištěno deset? Kde je těch devět?
#17:18 Nikdo z nich se nenašel, kdo by se vrátil a vzdal Bohu chválu, než tento cizinec?“
#17:19 Řekl mu: „Vstaň a jdi, tvá víra tě zachránila.“
#17:20 Když se ho farizeové otázali, kdy přijde Boží království, odpověděl jim: „Království Boží nepřichází tak, abyste to mohli pozorovat;
#17:21 ani se nedá říci: ‚Hle, je tu‘ nebo ‚je tam‘! Vždyť království Boží je mezi vámi!“
#17:22 Svým učedníkům řekl: „Přijdou dny, kdy si budete toužebně přát, abyste spatřili aspoň jediný ze dnů Syna člověka, ale nespatříte.
#17:23 Řeknou vám: ‚Hle, tam je, hle, tu‘; zůstaňte doma a nechoďte za nimi.
#17:24 Jako když se zableskne a rázem osvětlí všecko pod nebem z jednoho konce nebe na druhý, tak bude Syn člověka ve svém dni.
#17:25 Ale nejprve musí mnoho trpět a být zavržen od tohoto pokolení.
#17:26 Jako bylo za dnů Noé, tak bude i za dnů Syna člověka:
#17:27 Jedli, pili, ženili se a vdávaly až do dne, kdy Noé vešel do korábu a přišla potopa a zahubila všechny.
#17:28 Stejně tak bylo za dnů Lotových: Jedli, pili, kupovali, prodávali, sázeli a stavěli;
#17:29 v ten den, kdy Lot vyšel ze Sodomy, spustil se oheň a síra z nebe a zahubil všechny.
#17:30 Právě tak bude v den, kdy se zjeví Syn člověka.
#17:31 Kdo bude v onen den na střeše, ale věci bude mít v domě, ať nesestupuje, aby si je vzal; a stejně tak kdo bude na poli, ať se nevrací zpět.
#17:32 Vzpomeňte si na Lotovu ženu!
#17:33 Kdo by usiloval svůj život zachovat, ztratí jej, a kdo jej ztratí, zachová jej.
#17:34 Pravím vám: Té noci budou dva na jednom loži, jeden bude přijat a druhý zanechán.
#17:35 Dvě budou mlít spolu obilí, jedna bude přijata, druhá zanechána.“
#17:36 Dva budou na poli, jeden bude přijat a druhý zanechán.
#17:37 Když to slyšeli, otázali se ho: „Kde to bude, Pane?“ Řekl jim: „Kde bude tělo, tam se slétnou i supi.“ 
#18:1 Vypravoval jim podobenství, aby ukázal, jak je třeba stále se modlit a neochabovat:
#18:2 „V jednom městě byl soudce, který se Boha nebál a z lidí si nic nedělal.
#18:3 V tom městě byla i vdova, která k němu ustavičně chodila a žádala: ‚Zastaň se mne proti mému odpůrci.‘
#18:4 Ale on se k tomu dlouho neměl. Potom si však řekl: ‚I když se Boha nebojím a z lidí si nic nedělám,
#18:5 dopomohu jí k právu, poněvadž mi nedává pokoj. Jinak mi sem stále bude chodit a nakonec mě umoří.‘“
#18:6 A Pán řekl: „Všimněte si, co praví ten nespravedlivý soudce!
#18:7 Což teprve Bůh! Nezjedná on právo svým vyvoleným, kteří k němu dnem i nocí volají, i když jim s pomocí prodlévá?
#18:8 Ujišťuji vás, že se jich brzo zastane. Ale nalezne Syn člověka víru na zemi, až přijde?“
#18:9 O těch, kteří si na sobě zakládali, že jsou spravedliví, a ostatními pohrdali, řekl toto podobenství:
#18:10 „Dva muži vstoupili do chrámu, aby se modlili; jeden byl farizeus, druhý celník.
#18:11 Farizeus se postavil a takto se sám u sebe modlil: ‚Bože, děkuji ti, že nejsem jako ostatní lidé, vyděrači, nepoctivci, cizoložníci, nebo i jako tento celník.
#18:12 Postím se dvakrát za týden a dávám desátky ze všeho, co získám.‘
#18:13 Avšak celník stál docela vzadu a neodvážil se ani oči k nebi pozdvihnout; bil se do prsou a říkal: ‚Bože, slituj se nade mnou hříšným.‘
#18:14 Pravím vám, že ten celník se vrátil ospravedlněn do svého domu, a ne farizeus. Neboť každý, kdo se povyšuje, bude ponížen, a kdo se ponižuje, bude povýšen.“
#18:15 Přinášeli mu i nemluvňátka, aby se jich dotýkal. Když to učedníci viděli, zakazovali jim to.
#18:16 Ježíš si je zavolal k sobě a řekl: „Nechte děti přicházet ke mně a nebraňte jim, neboť takovým patří království Boží.
#18:17 Amen, pravím vám, kdo nepřijme království Boží jako dítě, jistě do něho nevejde.“
#18:18 Jeden z předních mužů se ho otázal: „Mistře dobrý, co mám dělat, abych měl podíl na věčném životě?“
#18:19 Ježíš mu řekl: „Proč mi říkáš ‚dobrý‘? Nikdo není dobrý, jedině Bůh.
#18:20 Přikázání znáš: Nezcizoložíš, nezabiješ, nebudeš krást, nevydáš křivé svědectví, cti otce svého i matku.“
#18:21 On řekl: „To všechno jsem dodržoval od svého mládí.“
#18:22 Když to Ježíš uslyšel, řekl mu: „Jedno ti ještě schází. Prodej všechno co máš, rozděl chudým a budeš mít poklad v nebi; pak přijď a následuj mne!“
#18:23 On se velice zarmoutil, když to slyšel, neboť byl velmi bohatý.
#18:24 Když Ježíš viděl, jak se zarmoutil, řekl: „Jak těžko vejdou do Božího království ti, kdo mají bohatství!
#18:25 Snáze projde velbloud uchem jehly, než aby bohatý člověk vešel do Božího království.“
#18:26 Ti, kdo to slyšeli, řekli: „Kdo tedy může být spasen?“
#18:27 Odpověděl: „Nemožné u lidí je u Boha možné.“
#18:28 Petr řekl: „Hle, my jsme opustili, co bylo naše, a šli jsme za tebou.“
#18:29 On jim řekl: „Amen, pravím vám, není nikoho, kdo opustil dům nebo ženu nebo bratry nebo rodiče nebo děti pro Boží království,
#18:30 aby v tomto čase nedostal mnohokrát víc a v přicházejícím věku život věčný.“
#18:31 Vzal k sobě svých Dvanáct a řekl jim: „Hle, jdeme do Jeruzaléma a na Synu člověka se naplní všechno, co je psáno u proroků.
#18:32 Neboť bude vydán pohanům a budou se mu posmívat a tupit ho a plivat na něj,
#18:33 zbičují ho a zabijí; a třetího dne vstane.“
#18:34 Oni však ničemu z toho nerozuměli, smysl těch slov jim zůstal skryt a nepochopili, co říkal.
#18:35 Když se Ježíš přiblížil k Jerichu, seděl u cesty jeden slepec a žebral.
#18:36 Když uslyšel, že kolem prochází zástup lidí, ptal se, co se to děje.
#18:37 Řekli mu, že tudy jde Ježíš Nazaretský.
#18:38 Tu zvolal: „Ježíši, Synu Davidův, smiluj se nade mnou!“
#18:39 Ti, kteří šli vpředu, ho napomínali, aby mlčel. On však tím více křičel: „Synu Davidův, smiluj se nade mnou!“
#18:40 Ježíš se zastavil a přikázal, aby ho k němu přivedli.
#18:41 Když se přiblížil, Ježíš se ho otázal: „Co chceš, abych učinil?“ On odpověděl: „Pane, ať vidím.“
#18:42 Ježíš mu řekl: „Prohlédni! Tvá víra tě uzdravila.“
#18:43 Ihned prohlédl, šel za ním a oslavoval Boha. A všechen lid, který to viděl, vzdal Bohu chválu. 
#19:1 Ježíš vešel do Jericha a procházel jím.
#19:2 Tam byl muž jménem Zacheus, vrchní celník a veliký boháč;
#19:3 toužil uvidět Ježíše, aby poznal, kdo to je, ale poněvadž byl malé postavy, nemohl ho pro zástup spatřit.
#19:4 Běžel proto napřed a vylezl na moruši, aby ho uviděl, neboť tudy měl jít.
#19:5 Když Ježíš přišel k tomu místu, pohlédl vzhůru a řekl: „Zachee, pojď rychle dolů, neboť dnes musím zůstat v tvém domě.“
#19:6 On rychle slezl a s radostí jej přijal.
#19:7 Všichni, kdo to uviděli, reptali: „On je hostem u hříšného člověka!“
#19:8 Zacheus se zastavil a řekl Pánu: „Polovinu svého jmění, Pane, dávám chudým, a jestliže jsem někoho ošidil, nahradím mu to čtyřnásobně.“
#19:9 Ježíš mu řekl: „Dnes přišlo spasení do tohoto domu; vždyť je to také syn Abrahamův.
#19:10 Neboť Syn člověka přišel, aby hledal a spasil, co zahynulo.“
#19:11 Těm, kteří to slyšeli, pověděl ještě podobenství, protože byl blízko Jeruzaléma a oni se domnívali, že království Boží se má zjevit ihned.
#19:12 Proto řekl: „Jeden muž vznešeného rodu měl odejít do daleké země, aby si odtud přinesl královskou hodnost.
#19:13 Zavolal si deset svých služebníků a dal jim deset hřiven a řekl jim: ‚Hospodařte s nimi, dokud nepřijdu.‘
#19:14 Ale občané ho nenáviděli a poslali vzápětí poselstvo, aby vyřídilo: ‚Nechceme tohoto člověka za krále!‘
#19:15 Když se však jako král vrátil, dal si předvolat služebníky, kterým svěřil peníze, aby se přesvědčil, jak s nimi kdo hospodařil.
#19:16 Přišel první a řekl: ‚Pane, tvoje hřivna vynesla deset hřiven.‘
#19:17 Řekl mu: ‚Správně, služebníku dobrý, poněvadž jsi byl věrný v docela malé věci, budeš vládnout nad deseti městy.‘
#19:18 Přišel druhý a řekl: ‚Pane, tvoje hřivna vynesla pět hřiven.‘
#19:19 Řekl mu: ‚Ty vládni nad pěti městy.‘
#19:20 Přišel další a řekl: ‚Pane, tu je tvoje hřivna; měl jsem ji schovanou v šátku,
#19:21 neboť jsem se tě bál. Jsi přísný člověk: bereš, co jsi nedal, a sklízíš, co jsi nezasel.‘
#19:22 Řekne mu: ‚Jsi špatný služebník. Soudím tě podle tvých vlastních slov: věděl jsi, že jsem přísný a beru, co jsem nedal, a sklízím, co jsem nezasel.
#19:23 Proč jsi aspoň mé peníze neuložil, a já bych si je teď vybral i s úrokem.‘
#19:24 Své družině pak řekl: ‚Vezměte mu tu hřivnu a dejte ji tomu, kdo má deset hřiven!‘
#19:25 Řekli mu: ‚Pane, už má deset.‘
#19:26 Pravím vám: ‚Každému, kdo má, bude dáno; kdo nemá, tomu bude odňato i to, co má.
#19:27 Ale mé nepřátele, kteří nechtěli, abych byl králem, přiveďte sem a přede mnou je pobijte.‘“
#19:28 Po těchto slovech pokračoval Ježíš v cestě do Jeruzaléma.
#19:29 Když se přiblížil k Betfage a k Betanii u hory, která se nazývá Olivová, poslal dva ze svých učedníků
#19:30 a řekl jim: „Jděte naproti do vesnice, a jak do ní vejdete, naleznete přivázané oslátko, na němž dosud nikdo z lidí neseděl. Odvažte je a přiveďte!
#19:31 Zeptá-li se vás někdo, proč je odvazujete, odpovězte mu: ‚Pán je potřebuje.‘“
#19:32 Šli, kam je poslal, a nalezli vše, jak jim řekl.
#19:33 Když oslátko odvazovali, řekli jim jeho majitelé: „Proč to oslátko odvazujete?“
#19:34 Oni odpověděli: „Pán je potřebuje.“
#19:35 Přivedli oslátko k Ježíšovi, hodili přes ně své pláště a Ježíše na ně posadili.
#19:36 A jak jel, prostírali mu své pláště na cestu.
#19:37 Když už se blížil ke svahu Olivové hory, počal celý zástup učedníků radostně a hlasitě chválit Boha za všechny mocné činy, které viděli.
#19:38 Volali: „Požehnaný král, který přichází ve jménu Hospodinově. Na nebi pokoj a sláva na výsostech!“
#19:39 Tu mu řekli někteří farizeové ze zástupu: „Mistře, napomeň své učedníky!“
#19:40 Odpověděl: „Pravím vám, budou-li oni mlčet, bude volat kamení.“
#19:41 Když už byli blízko a uzřel město, dal se nad ním do pláče
#19:42 a řekl: „Kdybys poznalo v tento den i ty, co vede k pokoji! Avšak je to skryto tvým očím.
#19:43 Přijdou na tebe dny, kdy tvoji nepřátelé postaví kolem tebe val, obklíčí a sevřou tě ze všech stran.
#19:44 Srovnají tě se zemí a s tebou i tvé děti; nenechají v tobě kámen na kameni, poněvadž jsi nepoznalo čas, kdy se Bůh k tobě sklonil.“
#19:45 Když vešel do chrámu, začal vyhánět ty, kdo tam prodávali,
#19:46 a řekl jim: „Je psáno: ‚Můj dům bude domem modlitby‘, ale vy jste z něho udělali doupě lupičů.“
#19:47 Každý den učil v chrámě; velekněží však a zákoníci i přední mužové z lidu usilovali o to, aby jej zahubili,
#19:48 ale nevěděli, jak by to měli udělat, poněvadž všechen lid mu visel na rtech. 
#20:1 Jednoho dne, když učil lid v chrámě a zvěstoval evangelium, přistoupili k němu velekněží a zákoníci se staršími
#20:2 a řekli: „Pověz nám, jakou mocí to činíš a kdo je ten, který ti tu moc dal.“
#20:3 Odpověděl jim: „I já vám položím otázku. Řekněte mi,
#20:4 odkud měl Jan pověření křtít. Z nebe či od lidí?“
#20:5 Oni o tom mezi sebou uvažovali: „Řekneme-li ‚z nebe‘, namítne nám: ‚Proč jste mu neuvěřili?‘
#20:6 Řekneme-li ‚od lidí‘, všechen lid nás bude kamenovat, protože jsou přesvědčeni, že Jan byl prorok.“
#20:7 A tak odpověděli, že nevědí, odkud.
#20:8 Ježíš jim řekl: „Ani já vám nepovím, jakou mocí to činím.“
#20:9 Pak začal vypravovat lidu toto podobenství: „Jeden člověk vysadil vinici, pronajal ji vinařům a odcestoval.
#20:10 V stanovený čas poslal k vinařům služebníka, aby mu odevzdali podíl z výnosu vinice. Ale vinaři ho zbili a poslali zpět s prázdnou.
#20:11 Poslal k nim ještě jiného služebníka; oni i toho zbili zneuctili a poslali zpět s prázdnou.
#20:12 Poslal ještě třetího; i toho zbili do krve a vyhnali.
#20:13 Tu řekl pán vinice: ‚Co mám dělat? Pošlu svého milovaného syna, na něho snad budou mít ohled.‘
#20:14 Když ho však vinaři spatřili, domlouvali se mezi sebou: ‚To je dědic. Zabijme ho, a dědictví bude naše.‘
#20:15 A vyvlekli ho ven z vinice a zabili. Co tedy s nimi udělá pán vinice?
#20:16 Přijde, zahubí ty vinaře a vinici dá jiným.“ Když to uslyšeli, řekli: „To přece ne!“
#20:17 On na ně pohleděl a řekl: „Co tedy znamená slovo Písma: ‚Kámen, který stavitelé zavrhli, stal se kamenem úhelným?‘
#20:18 Každý, kdo padne na ten kámen, roztříští se, a na koho padne, toho rozdrtí.“
#20:19 Zákoníci a velekněží ho chtěli v tu hodinu dostat do rukou, ale báli se lidu; poznali totiž, že to podobenství řekl proti nim.
#20:20 Nespustili ho však z očí. Poslali své lidi, kteří měli předstírat, že to myslí upřímně, aby jej přistihli při výroku, pro nějž by ho mohli vydat vladařově moci a soudu.
#20:21 Otázali se ho: „Mistře, víme, že správně mluvíš a učíš a nestraníš nikomu, nýbrž učíš cestě Boží podle pravdy.
#20:22 Je nám dovoleno dávat daň císaři, nebo ne?“
#20:23 Ježíš však prohlédl jejich záludnost a řekl jim:
#20:24 „Ukažte mi denár! Čí má obraz a nápis?“ Odpověděli: „Císařův.“
#20:25 Řekl jim: „Odevzdejte tedy to, co je císařovo, císaři, a co je Boží, Bohu.“
#20:26 A tak se jim nepodařilo, aby ho před lidmi přistihli v řeči; podivili se jeho odpovědi a umlkli.
#20:27 Přišli k němu někteří ze saduceů - ti popírají vzkříšení - a otázali se ho:
#20:28 „Mistře, Mojžíš nám ustanovil: ‚Zemře-li nečí bratr ženatý, ale bezdětný, ať se s jeho manželkou ožení jeho bratr a zplodí svému bratru potomka.‘
#20:29 Bylo tedy sedm bratří. Oženil se první a zemřel bezdětný.
#20:30 Jeho manželku si vzal druhý,
#20:31 pak třetí a stejně všech sedm; nezanechali děti a zemřeli.
#20:32 Nakonec zemřela i ta žena.
#20:33 Komu z nich bude tato žena patřit při vzkříšení? Všech sedm ji přece mělo za manželku.“
#20:34 Ježíš jim řekl: „Lidé přítomného věku se žení a vdávají.
#20:35 Avšak ti, kteří byli hodni dosáhnout budoucího věku a vzkříšení z mrtvých, nežení se ani nevdávají.
#20:36 Vždyť už nemohou zemřít, neboť jsou rovni andělům a jsou syny Božími, poněvadž jsou účastni vzkříšení.
#20:37 A že mrtví vstanou, naznačil i Mojžíš ve vyprávění o hořícím keři. když nazývá Hospodina ‚Bohem Abrahamovým, Bohem Izákovým a Bohem Jákobovým‘.
#20:38 On přece není Bohem mrtvých, nýbrž živých, neboť před ním jsou všichni živi.“
#20:39 Někteří ze zákoníků na to řekli: „Mistře, dobře jsi odpověděl.“
#20:40 A už se neodvážili položit mu jinou otázku.
#20:41 Řekl jim: „Jak mohou nazývat Mesiáše synem Davidovým?
#20:42 Vždyť sám David praví v Knize žalmů: ‚Řekl Hospodin mému Pánu: Usedni po mé pravici,
#20:43 dokud nepoložím tvé nepřátele za podnož tvých nohou.‘
#20:44 David tedy nazývá Mesiáše Pánem; jak potom může být jeho synem?“
#20:45 Když všechen lid poslouchal, řekl učedníkům:
#20:46 „Dejte si pozor na zákoníky, kteří se rádi procházejí v dlouhých řízách, mají v oblibě pozdravy na ulicích, přední sedadla v synagógách a přední místa na hostinách.
#20:47 Vyjídají domy vdov a dlouho se na oko modlí. Ty postihne tím přísnější soud.“ 
#21:1 Ježíš pozoroval, jak bohatí vhazují své dary do chrámové pokladnice.
#21:2 Uviděl i jednu nuznou vdovu, jak tam hodila dvě drobné mince,
#21:3 a řekl: „Vpravdě vám pravím, že tato chudá vdova dala víc než všichni ostatní.
#21:4 Neboť ti všichni dali dary ze svého nadbytku, ona však ze svého nedostatku; dala všechno, z čeho měla být živa.“
#21:5 Když někteří mluvili o chrámu, jak je vyzdoben krásnými kameny a pamětními dary, řekl:
#21:6 „Přijdou dny, kdy z toho, co vidíte, nezůstane kámen na kameni, všechno bude rozmetáno.“
#21:7 Otázali se ho: „Mistře, kdy to nastane? A jaké bude znamení, až se to začne dít?“
#21:8 Odpověděl: „Mějte se na pozoru, abyste se nedali svést. Neboť mnozí přijdou v mém jménu a budou říkat: ‚Já jsem to‘ a ‚nastal čas‘. Nechoďte za nimi.
#21:9 Až uslyšíte o válkách a povstáních, neděste se: neboť to musí nejprve být, ale konec nenastane hned.“
#21:10 Tehdy jim řekl: „POvstane národ proti národu a království proti království,
#21:11 budou veliká zemětřesení a v mnohých krajinách hlad a mor, hrůzy a veliká znamení z nebes.
#21:12 Ale před tím vším na vás vztáhnou ruce a budou vás pronásledovat; budou vás vydávat synagógám na soud a do vězení a vodit před krále a vládce pro mé jméno.
#21:13 To vám bude příležitostí k svědectví.
#21:14 Vezměte si k srdci, abyste se předem nepřipravovali, jak se budete hájit.
#21:15 Neboť já vám dám řeč i moudrost, kterou nedokáže přemoci ani vyvrátit žádný váš protivník.
#21:16 Zradí vás i rodiče, bratři, příbuzní a přátelé a někteří z vás budou zabiti.
#21:17 A všichni vás budou nenávidět pro mé jméno.
#21:18 Ale ani vlas z vaší hlavy se neztratí.
#21:19 Když vytrváte, získáte své životy.
#21:20 Když uvidíte, že Jeruzalém obkličují vojska, tu poznáte, že se přiblížila jeho zkáza.
#21:21 Tehdy ti, kdo jsou v Judsku, ať uprchnou do hor, kteří jsou v Jeruzalémě, ať z něho odejdou, a kteří jsou po venkově, ať do něho nevcházejí,
#21:22 poněvadž jsou to dny odplaty, v nichž se má naplnit vše, co je psáno.
#21:23 Běda těhotným a kojícím v oněch dnech! Neboť bude veliké soužení na zemi a hněv proti tomuto lidu.
#21:24 Padnou ostřím meče, budou jako zajatci odvedeni mezi všecky národy, po Jeruzalému budou šlapat pohané, dokud se jejich čas neskončí.
#21:25 Budou znamení na slunci, měsíci a hvězdách a na zemi úzkost národů, bezradných, kam se podít před řevem valícího se moře.
#21:26 Lidé budou zmírat strachem a očekáváním toho, co přichází na celý svět. Neboť mocnosti pekelné se zachvějí.
#21:27 A tehdy uzří Syna člověka přicházet v oblaku s mocí a velikou slávou.
#21:28 Když se toto začne dít, napřimte se a zvedněte hlavy, neboť vaše vykoupení je blízko.“
#21:29 Vypravoval jim podobenství: „Podívejte se na fíkovník nebo jiný strom:
#21:30 Když se už zelenají, sami víte, že léto je blízko.
#21:31 Tak i vy, až uvidíte, že se toto děje, vězte, že je blízko království Boží.
#21:32 Amen, pravím vám, že nepomine toto pokolení,než se toto všechno stane.
#21:33 Nebe a země pominou, ale má slova nikdy nepominou.
#21:34 Mějte se na pozoru, aby vaše srdce nebyla zatížena nestřídmostí, opilstvím a starostmi o živobytí a aby vás onen den nepřekvapil jako past.
#21:35 Neboť přijde na všechny, kteří přebývají na zemi.
#21:36 Buďte bdělí a proste v každý čas, abyste měli sílu uniknout všemu tomu, co se bude dít, a mohli stanout před Synem člověka.“
#21:37 Ve dne učil v chrámě, ale na noc odcházel na horu, která se nazývala Olivová.
#21:38 A všechen lid k němu přicházel už časně zrána do chrámu, aby ho poslouchal. 
#22:1 Blížil se svátek nekvašených chlebů, velikonoce.
#22:2 Velekněží a zákoníci přemýšleli, jak by ho zahubili; báli se však lidu.
#22:3 Tu vstoupil satan do Jidáše, nazývaného Iškariotský, který byl z počtu Dvanácti.
#22:4 Odešel, aby se domluvil s velekněžími a veliteli stráže, že jim ho zradí.
#22:5 Oni se zaradovali a dohodli se, že mu dají peníze.
#22:6 Jidáš s tím souhlasil a hledal vhodnou příležitost, aby jim ho vydal, až při tom nebude zástup.
#22:7 Nastal den nekvašených chlebů, kdy měl být zabit velikonoční beránek.
#22:8 Ježíš poslal Petra a Jakuba a řekl jim: „Jděte a připravte nám beránka, abychom slavili velikonoční večeři.“
#22:9 Oni mu řekli: „Kde chceš, abychom ji připravili?“
#22:10 Řekl jim: „Když vejdete do města, potkáte člověka, který nese džbán vody. Jděte za ním do domu, do něhož vejde, a řekněte hospodáři:
#22:11 ‚Mistr ti vzkazuje: Kde je světnice, v níž bych jedl se svými učedníky velikonočního beránka?‘
#22:12 A on vám ukáže upravenou velkou horní místnost; tam připravte večeři.“
#22:13 Odešli a nalezli všechno, jak jim řekl, a připravili velikonočního beránka.
#22:14 Když nastala hodina, usedl ke stolu a apoštolové s ním.
#22:15 Řekl jim: „Velice jsem toužil jísti s vámi tohoto beránka, dříve, než budu trpět.
#22:16 Neboť vám pravím, že ho již nebudu jíst, dokud vše nedojde naplnění v království Božím.“
#22:17 Vzal kalich, vzdal díky a řekl: „Vezměte a podávejte mezi sebou.
#22:18 Neboť vám pravím, že od této chvíle nebudu píti z plodu vinné révy, dokud nepřijde království Boží.“
#22:19 Pak vzal chléb, vzdal díky, lámal a dával jim se slovy: „Toto je mé tělo, které se za vás vydává. To čiňte na mou památku.“
#22:20 A právě tak, když bylo po večeři, vzal kalich a řekl: „Tento kalich je nová smlouva zpečetěná mou krví, která se za vás prolévá.“
#22:21 „Avšak hle, můj zrádce je se mnou u stolu.
#22:22 Syn člověka jde, jak je určeno, běda však tomu člověku, který ho zrazuje.“
#22:23 A oni se začali mezi sebou dohadovat, který z nich je ten, kdo to učiní.
#22:24 Vznikl mezi nimi spor, kdo z nich je asi největší.
#22:25 Řekl jim: „Králové panují nad národy, a ti, kdo jsou u moci, dávají si říkat dobrodinci.
#22:26 Avšak vy ne tak: Kdo je mezi vámi největší, buď jako poslední, a kdo je v čele, buď jako ten, který slouží.
#22:27 Neboť kdo je větší: ten, kdo sedí za stolem, či ten, kdo obsluhuje? Zdali ne ten, kdo sedí za stolem? Ale já jsem mezi vámi jako ten, který slouží.
#22:28 A vy jste ti, kdo se mnou v mých zkouškách vytrvali.
#22:29 Já vám uděluji království, jako je můj Otec udělil mně,
#22:30 abyste v mém království jedli a pili u mého stolu; usednete na trůnech a budete soudit dvanáct pokolení Izraele.“
#22:31 „Šimone, Šimone, hle, satan si vyžádal, aby vás směl tříbit jako pšenici.
#22:32 Já jsem však za tebe prosil, aby tvá víra neselhala; a ty, až se obrátíš, buď posilou svým bratřím.“
#22:33 Řekl mu: „Pane, s tebou jsem hotov jít i do vězení a na smrt.“
#22:34 Ježíš mu řekl: „Pravím ti, Petře, ještě se ani kohout neozve, a ty už třikrát zapřeš, že mne znáš.“
#22:35 Řekl jim: „Když jsem vás vyslal bez měšce, mošny a obuvi, měli jste v něčem nedostatek?“ Oni mu odpověděli: „Neměli.“
#22:36 Řekl jim: „Nyní však, kdo má měšec, vezmi jej a stejně tak i mošnu; kdo nemá, prodej plášť a kup si meč.
#22:37 Pravím vám, že se na mně musí naplnit to, co je psáno: ‚Byl započten mezi zločince.‘ Neboť to, co se na mne vztahuje, dochází svého cíle.“
#22:38 Oni mu řekli: „Pane, tu jsou dva meče.“ Na to jim řekl: „To stačí.“
#22:39 Potom se jako obvykle odebral na Olivovou horu; učedníci ho následovali.
#22:40 Když došel na místo, řekl jim: „Modlete se, abyste neupadli do pokušení.“
#22:41 Pak se od nich vzdálil, co by kamenem dohodil, klekl a modlil se:
#22:42 „Otče, chceš-li, odejmi ode mne tento kalich, ale ne má nýbrž tvá vůle se staň.“
#22:43 Tu se mu zjevil anděl z nebe a dodával mu síly.
#22:44 Ježíš v úzkostech zápasil a modlil se ještě usilovněji; jeho pot kanul na zem jako krůpěje krve.
#22:45 Pak vstal od modlitby, přišel k učedníkům a shledal, že zármutkem usnuli.
#22:46 Řekl jim: „Jak to, že spíte? Vstaňte a modlete se, abyste neupadli do pokušení.“
#22:47 Ještě ani nedomluvil, a hle, zástup, a vpředu ten, který se jmenoval Jidáš, jeden z Dvanácti; přistoupil k Ježíšovi, aby ho políbil.
#22:48 Ježíš mu řekl: „Jidáši, políbením zrazuješ Syna člověka?“
#22:49 Když ti, kteří byli s Ježíšem, viděli co nastává, řekli: „Pane, máme se bít mečem?“
#22:50 A jeden z nich napadl sluhu veleknězova a uťal mu pravé ucho.
#22:51 Ježíš však řekl: „Přestaňte s tím.“ Dotkl se jeho ucha a uzdravil ho.
#22:52 Pak řekl Ježíš těm, kteří pro něho přišli, kněžím, velitelům stráže a starším: „Jako na povstalce jste na mne vyšli s meči a holemi.
#22:53 Denně jsem byl mezi vámi v chrámě, a nevztáhli jste na mne ruce. Ale toto je vaše hodina, vláda tmy.“
#22:54 Pak ho zatkli a odvedli do veleknězova domu. Petr šel zpovzdálí za nimi.
#22:55 Když zapálili uprostřed nádvoří oheň a sesedli se okolo, přisedl mezi ně i Petr.
#22:56 A jak seděl tváří k ohni, všimla si ho jedna služka, pozorně se na něj podívala a řekla: „Tenhle byl také s ním!“
#22:57 Ale on zapřel: „Vůbec ho neznám.“
#22:58 Zakrátko ho spatřil někdo jiný a řekl: „Ty jsi také jeden z nich.“ Petr odpověděl: „Nejsem!“
#22:59 Když uplynula asi hodina, tvrdil zase někdo: „I tenhle byl určitě s ním, vždyť je z Galileje!“
#22:60 Petr řekl: „Vůbec nevím, o čem mluvíš!“ A ihned, ještě než domluvil, zakokrhal kohout.
#22:61 Tu se Pán obrátil a pohleděl na Petra; a Petr se rozpomenul na slovo, které mu Pán řekl: „Dřív než dnes kohout zakokrhá, zapřeš mne třikrát.“
#22:62 Vyšel ven a hořce se rozplakal.
#22:63 Muži, kteří Ježíš hlídali, posmívali se mu a bili ho;
#22:64 zavázali mu oči a ptali se ho: „Hádej, proroku, kdo tě uhodil.“
#22:65 A ještě mnoha jinými slovy ho uráželi.
#22:66 Jakmile nastal den, shromáždili se starší z lidu, velekněží a zákoníci, odvedli ho před svou radu a řekli mu:
#22:67 „Jsi-li Mesiáš, pověz nám to.“ Odpověděl jim: „I když vám to řeknu, neuvěříte.
#22:68 Položím-li otázku já vám, neodpovíte.
#22:69 Ale od této chvíle bude Syn člověka sedět po pravici všemohoucího Boha.“
#22:70 Tu mu řekli všichni: „Jsi tedy Syn Boží?“ On jim odpověděl: „Vy sami říkáte, že já jsem.“
#22:71 Oni řekli: „Nač ještě potřebujeme svědectví? Vždyť jsme to slyšeli z jeho úst.“ 
#23:1 Tu povstalo celé shromáždění a odvedli ho k Pilátovi.
#23:2 Vznesli proti němu žalobu: „Podle našeho zjištění rozvrací tento člověk náš národ, brání odvádět císaři daně a prohlašuje se za Mesiáše krále.“
#23:3 Pilát mu položil otázku: „Ty jsi král Židů?“ On mu odpověděl: „Ty sám to říkáš.“
#23:4 Pilát řekl velekněžím a zástupům: „Já na tomto člověku žádnou vinu neshledávám.“
#23:5 Ale oni na něj naléhali: „Svým učením pobuřuje lid po celém Judsku; začal v Galileji a přišel až sem.“
#23:6 Jakmile to Pilát uslyšel, otázal se, zda je ten člověk z Galileje.
#23:7 Když se dověděl, že podléhá Herodově pravomoci, poslal ho k němu, protože Herodes byl právě v těch dnech také v Jeruzalémě.
#23:8 Když Herodes Ježíše spatřil, velmi se zaradoval; už dávno si ho totiž přál vidět, poněvadž o něm mnoho slyšel, a doufal, že uvidí, jak dělá nějaký zázrak.
#23:9 Kladl mu mnoho otázek, ale on mu na nic neodpovídal.
#23:10 Byli při tom i velekněží a zákoníci a neústupně na něj žalovali.
#23:11 Tu se od něho Herodes se svými vojáky pohrdavě odvrátil, vysmál se mu, dal ho obléci ve slavnostní šat a poslal ho zase k Pilátovi.
#23:12 Toho dne se Herodes a Pilát stali přáteli; před tím totiž bylo mezi nimi nepřátelství.
#23:13 Pilát svolal velekněze, členy rady i lid
#23:14 a řekl jim: „Přivedli jste přede mne tohoto člověka, že pobuřuje lid; já jsem ho, jak vidíte, před vámi vyslechl a neshledal jsem na něm nic, z čeho jej obviňujete.
#23:15 Ani Herodes ne; vždyť nám ho poslal zpět. Je zřejmé, že nespáchal nic, proč by zasluhoval smrt.
#23:16 Dám ho na místě potrestat a pak ho propustím.“
#23:17 Musel jim totiž o svátcích propustit vždy jednoho člověka.
#23:18 Ale oni všichni najednou křičeli: „Pryč s ním! Propusť nám Barabáše!“
#23:19 To byl člověk, kterého uvrhli do vězení pro jakousi vzpouru ve městě a vraždu.
#23:20 Tu k nim Pilát znovu promluvil, neboť chtěl Ježíše propustit.
#23:21 Avšak oni křičeli: „Na kříž, na kříž s ním!“
#23:22 Promluvil k nim potřetí: „Čeho se vlastně dopustil? Neshledal jsem na něm nic, proč by měl zemřít. Dám ho zbičovat a pak ho propustím.“
#23:23 Ale oni na něm s velkým křikem vymáhali, aby ho dal ukřižovat; a jejich křik se stále stupňoval.
#23:24 A tak se Pilát rozhodl jim vyhovět.
#23:25 Propustil toho, který byl vsazen do vězení pro vzpouru a vraždu a o kterého žádali; Ježíše vydal, aby se s ním stalo, co chtěli.
#23:26 Když jej odváděli, zastavili nějakého Šimona z Kyrény, který šel z pole, a vložili na něho kříž, aby jej nesl za Ježíšem.
#23:27 Za ním šel veliký zástup lidu; ženy nad ním naříkaly a oplakávaly ho.
#23:28 Ježíš se k nim obrátil a řekl: „Dcery jeruzalémské, nade mnou neplačte! Plačte nad sebou a svými dětmi;
#23:29 hle, přicházejí dny, kdy budou říkat: ‚Blaze neplodným, blaze těm, které nikdy nerodily a nekojily!‘
#23:30 Tehdy ‚řeknou horám: Padněte na nás, a pahrbkům: Přiryjte nás!‘
#23:31 Neboť děje-li se toto se zeleným stromem, co se stane se suchým?“
#23:32 Spolu s ním byli vedeni na smrt ještě dva zločinci.
#23:33 Když přišli na místo, které se nazývá Lebka, ukřižovali jej i ty zločince, jednoho po jeho pravici a druhého po levici.
#23:34 Ježíš řekl: „Otče, odpusť jim, vždyť nevědí, co činí.“ O jeho šaty se rozdělili losem.
#23:35 Lid stál a díval se. Členové rady se mu vysmívali a říkali: „Jiné zachránil, ať zachrání sám sebe, je-li Mesiáš, ten vyvolený Boží.“
#23:36 Posmívali se mu i vojáci; chodili k němu a podávali mu ocet
#23:37 a říkali: „Když jsi židovský král, zachraň sám sebe.“
#23:38 Nad ním byl nápis písmem řeckým, latinským a hebrejským: „Toto je král Židů.“
#23:39 Jeden z těch zločinců, kteří viseli na kříži, se mu rouhal: „To jsi Mesiáš? Zachraň sebe i nás!“
#23:40 Tu ho ten druhý okřikl: „Ty se ani Boha nebojíš? Vždyť jsi sám odsouzen ke stejnému trestu.
#23:41 A my jsme odsouzeni spravedlivě, dostáváme zaslouženou odplatu, ale on nic zlého neudělal.“
#23:42 A řekl: „Ježíši, pamatuj na mne, až přijdeš do svého království.“
#23:43 Ježíš mu odpověděl: „Amen, pravím ti, dnes budeš se mnou v ráji.“
#23:44 Bylo už kolem poledne; tu nastala tma po celé zemi až do tří hodin, protože se zatmělo slunce.
#23:45 Chrámová opona se roztrhla v půli.
#23:46 A Ježíš zvolal mocným hlasem: „Otče, do tvých rukou odevzdávám svého ducha.“ Po těch slovech skonal.
#23:47 Když setník viděl, co se stalo, velebil Boha a řekl: „Tento člověk byl vskutku spravedlivý.“
#23:48 A ti, kdo se v celých zástupech sešli na tu podívanou, když viděli, co se stalo, odcházeli bijíce se do prsou.
#23:49 Všichni jeho přátelé stáli opodál, i ženy, které Ježíše doprovázely z Galileje a všechno to viděly.
#23:50 Členem židovské rady byl muž jménem Josef, člověk dobrý a spravedlivý,
#23:51 který nesouhlasil s jejich rozhodnutím a činem. Pocházel z židovského města Arimatie a patřil k těm, kdo očekávali království Boží.
#23:52 Ten přišel k Pilátovi a požádal ho o Ježíšovo tělo;
#23:53 sňal je z kříže, zavinul do plátna a položil do hrobu, vytesaného ve skále, kde ještě nikdo nebyl pochován.
#23:54 Byl pátek a začínala sobota.
#23:55 Ženy, které šly s Ježíšem z Galileje, šly za ním; viděly hrob i to, jak bylo tělo pochováno.
#23:56 Potom se vrátily, aby připravily vonné masti a oleje. Ale v sobotu zachovaly podle přikázání sváteční klid. 
#24:1 Prvního dne po sobotě, za časného jitra, přišly k hrobu s vonnými mastmi, které připravily.
#24:2 Nalezly však kámen od hrobu odvalený.
#24:3 Vešly dovnitř, ale tělo Pána Ježíše nenašly.
#24:4 A jak nad tím byly bezradné, stanuli u nich dva muži v zářícím rouchu.
#24:5 Zachvátil je strach a sklonily se tváří k zemi. Ale oni jim řekli: „Proč hledáte živého mezi mrtvými?
#24:6 Není zde, byl vzkříšen. Vzpomeňte si, jak vám řekl, když byl ještě v Galileji,
#24:7 že Syn člověka musí být vydán do rukou hříšných lidí, být ukřižován a třetího dne vstát.“
#24:8 Tu se rozpomenuly na jeho slova,
#24:9 vrátily se od hrobu a oznámily to všecko jedenácti učedníkům i všem ostatním.
#24:10 Byla to Marie z Magdaly, Jana a Marie Jakubova a s nimi ještě jiné, které pověděly apoštolům.
#24:11 Těm však ta slova připadala jako blouznění a nevěřili jim.
#24:12 Petr se rozběhl ke hrobu, nahlédl dovnitř a uviděl tam ležet jen plátna. Vrátil se v údivu nad tím, co se stalo.
#24:13 Téhož dne se dva z nich ubírali do vsi jménem Emaus, která je od Jeruzaléma vzdálena asi tři hodiny cesty,
#24:14 a rozmlouvali spolu o tom všem, co se událo.
#24:15 A jak to v řeči probírali, připojil se k nim sám Ježíš a šel s nimi.
#24:16 Ale něco jako by bránilo jejich očím, aby ho poznali.
#24:17 Řekl jim: „O čem to spolu rozmlouváte?“ Oni zůstali stát plni zármutku.
#24:18 Jeden z nich, jménem Kleofáš, mu odpověděl: „Ty jsi asi jediný z Jeruzaléma, kdo neví, co se tam v těchto dnech stalo!“
#24:19 On se jich zeptal: „A co to bylo?“ Oni mu odpověděli: „Jak Ježíš Nazaretského, který byl prorok mocný slovem i skutkem před Bohem i přede vším lidem,
#24:20 naši velekněží a členové rady vydali, aby byl odsouzen na smrt, a ukřižovali ho.
#24:21 A my jsme doufali, že on je ten, který má vykoupit Izrael. Ale už je to dnes třetí den, co se to stalo.
#24:22 Ovšem některé z našich žen nás ohromily: Byly totiž zrána u hrobu
#24:23 a nenalezly jeho tělo; přišly a vyprávěly, že měly i vidění andělů, kteří říkali, že je živ.
#24:24 Někteří z nás pak odešli ke hrobu a shledali, že je to tak, jak ženy vypravovaly, jeho však neviděli.“
#24:25 A on jim řekl: Jak jste nechápaví! To je vám tak těžké uvěřit všemu, co mluvili proroci!
#24:26 Což neměl Mesiáš to vše vytrpět a vejít do slávy?“
#24:27 Potom začal od Mojžíše a všech proroků a vykládal jim to, co se na něho vztahovalo ve všech částech Písma.
#24:28 Když už byli blízko vesnice, do které šli, on jako by chtěl jít dál.
#24:29 Oni ho však začali přemlouvat: „Zůstaň s námi, vždyť už je k večeru a den se schyluje.“ Vešel tedy a zůstal s nimi.
#24:30 Když byl s nimi u stolu, vzal chléb, vzdal díky, lámal a rozdával jim.
#24:31 Tu se jim otevřely oči a poznali ho; ale on zmizel jejich zrakům.
#24:32 Řekli si spolu: „Což nám srdce nehořelo, když s námi na cestě mluvil a otvíral nám Písma?“
#24:33 A v tu hodinu vstali a vrátili se do Jeruzaléma; nalezli jedenáct učedníků a jejich druhy pohromadě.
#24:34 Ti jim řekli: „Pán byl opravdu vzkříšen a zjevil se Šimonovi.“
#24:35 Oni pak vypravovali, co se jim stalo na cestě a jak se jim dal poznat, když lámal chléb.
#24:36 Když o tom mluvili, stál tu on sám uprostřed nich a řekl jim: „Pokoj vám.“.
#24:37 Zděsili se a byli plni strachu, poněvadž se domnívali, že vidí ducha.
#24:38 Řekl jim: „Proč jste tak zmateni a proč vám takové věci přicházejí na mysl?
#24:39 Podívejte se na mé ruce a nohy: vždyť jsem to já. Dotkněte se mne a přesvědčte se: duch přece nemá maso a kosti, jako to vidíte na mně.“
#24:40 To řekl a ukázal jim ruce a nohy.
#24:41 Když tomu pro samou radost nemohli uvěřit a jen se divili, řekl jim: „Máte tu něco k jídlu?“
#24:42 Podali mu kus pečené ryby.
#24:43 Vzal si a pojedl před nimi.
#24:44 Řekl jim: „To jsem měl na mysli, když jsem byl ještě s vámi a říkal vám, že se musí naplnit všechno, co je o mně psáno v zákoně Mojžíšově, v Prorocích a Žalmech.“
#24:45 Tehdy jim otevřel mysl, aby rozuměli Písmu.
#24:46 Řekl jim: „Tak je psáno: Kristus bude trpět a třetího dne vstane z mrtvých;
#24:47 v jeho jménu se bude zvěstovat pokání na odpuštění hříchů všem národům, počínajíc Jeruzalémem.
#24:48 Vy jste toho svědky.
#24:49 Hle, sesílám na vás, co slíbil můj Otec; zůstaňte ve městě, dokud nebudete vyzbrojeni mocí z výsosti.“
#24:50 Potom je vyvedl až k Betanii, zvedl ruce a požehnal jim;
#24:51 a když jim žehnal, vzdálil se od nich a byl nesen do nebe.
#24:52 Oni před ním padli na kolena; potom se s velikou radostí vrátili do Jeruzaléma,
#24:53 byli stále v chrámě a chválili Boha.  

\book{John}{John}
#1:1 Na počátku bylo Slovo, to Slovo bylo u Boha, to Slovo byl Bůh.
#1:2 To bylo na počátku u Boha.
#1:3 Všechno povstalo skrze ně a bez něho nepovstalo nic, co jest.
#1:4 V něm byl život a život byl světlo lidí.
#1:5 To světlo ve tmě svítí a tma je nepohltila.
#1:6 Od Boha byl poslán člověk, jménem Jan.
#1:7 Ten přišel proto, aby vydal svědectví o tom světle, aby všichni uvěřili skrze něho.
#1:8 Jan sám nebyl tím světlem, ale přišel, aby o tom světle vydal svědectví.
#1:9 Bylo tu pravé světlo, které osvěcuje každého člověka; to přicházelo do světa.
#1:10 Na světě byl, svět skrze něj povstal, ale svět ho nepoznal.
#1:11 Přišel do svého vlastního, ale jeho vlastní ho nepřijali.
#1:12 Těm pak, kteří ho přijali a věří v jeho jméno, dal moc stát se Božími dětmi.
#1:13 Ti se nenarodili, jen jako se rodí lidé, jako děti pozemských otců, nýbrž se narodili z Boha.
#1:14 A Slovo se stalo tělem a přebývalo mezi námi. Spatřili jsme jeho slávu, slávu, jakou má od Otce jednorozený Syn, plný milosti a pravdy.
#1:15 Jan o něm vydal svědectví a volal: „To je ten, o němž jsem řekl: Přichází za mnou, ale je větší, protože tu byl dříve než já.“
#1:16 Z jeho plnosti jsme byli obdarováni my všichni milostí za milostí.
#1:17 Neboť Zákon byl dán skrze Mojžíše, milost a pravda se stala skrze Ježíše Krista.
#1:18 Boha nikdy nikdo neviděl; jednorozený Syn, který je v náruči Otcově, nám o něm řekl.
#1:19 Toto je svědectví Janovo, když k němu Židé z Jeruzaléma poslali kněze a levity, aby se ho otázali: „Kdo jsi?“
#1:20 Nic nepopřel a otevřeně vyznal: „Já nejsem Mesiáš.“
#1:21 Znovu se ho zeptali: „Jak to tedy je? Jsi Eliáš?“ Řekl: „Nejsem.“ „Jsi ten Prorok?“ Odpověděl: „Ne.“
#1:22 Řekli mu tedy: „Kdo jsi? Ať můžeme přinést odpověď těm, kdo nás poslali. Za koho se sám pokládáš?“
#1:23 Řekl: „Jsem hlas volajícího na poušti: Urovnejte cestu Páně - jak řekl prorok Izaiáš.“
#1:24 Ti vyslaní byli z řad farizeů.
#1:25 Otázali se ho: „Proč tedy křtíš, když nejsi ani Mesiáš ani Eliáš ani ten Prorok?“
#1:26 Jan jim odpověděl: „Já křtím vodou. Uprostřed vás stojí, koho vy neznáte -
#1:27 ten, který přichází za mnou; jemu nejsem hoden ani rozvázat řemínek u jeho obuvi.“
#1:28 To se stalo v Betanii, na druhém břehu Jordánu, kde Jan křtil.
#1:29 Druhého dne spatřil Jan Ježíše, jak jde k němu, a řekl: „Hle, beránek Boží, který snímá hřích světa.
#1:30 To je ten, o němž jsem řekl: Za mnou přichází někdo větší, neboť byl dříve, než já.
#1:31 Já jsem nevěděl, kdo to je, ale přišel jsem křtít vodou proto, aby ho poznal Izrael.“
#1:32 Jan vydal svědectví: „Spatřil jsem, jak Duch sestoupil jako holubice z nebe a zůstal na něm.
#1:33 A já jsem stále nevěděl, kdo to je, ale ten, který mě poslal křtít vodou, mi řekl: ‚Na koho spatříš sestupovat Ducha a zůstávat na něm, to je ten, který křtí Duchem svatým.‘
#1:34 Já jsem to viděl a dosvědčuji, že toto je Syn Boží.“
#1:35 Druhého dne tam byl opět Jan s dvěma ze svých učedníků.
#1:36 Spatřil Ježíše, jak jde okolo, a řekl: „Hle, beránek Boží.“
#1:37 Ti dva učedníci slyšeli, co řekl, a šli za Ježíšem.
#1:38 Když se Ježíš obrátil a uviděl, že jdou za ním, otázal se jich: „Co chcete?“ Řekli mu: „Rabi (což přeloženo znamená: Mistře), kde bydlíš?“
#1:39 Odpověděl jim: „Pojďte a uvidíte!“ Šli tedy, viděli, kde bydlí, a zůstali ten den u něho. Bylo kolem čtyř hodin odpoledne.
#1:40 Jeden z těchto dvou, kteří slyšeli, co Jan řekl, a Ježíše následovali, byl Ondřej, bratr Šimona Petra.
#1:41 Vyhledal nejprve svého bratra Šimona a řekl mu: „Nalezl jsem Mesiáše (což je v překladu: Kristus).“
#1:42 Přivedl ho k Ježíšovi. Ježíš na něj pohleděl a řekl: „Ty jsi Šimon, syn Janův; budeš se jmenovat Kéfas (což se překládá: Petr).“
#1:43 Druhého dne se Ježíš rozhodl vydat na cestu do Galileje. Vyhledal Filipa a řekl mu: „Následuj mě!“
#1:44 Filip byl z Betsaidy, města Ondřejova a Petrova;
#1:45 Filip zase vyhledal Natanaela a řekl mu: „Nalezli jsme toho, o němž psal Mojžíš v Zákoně i proroci, Ježíše, syna Josefova z Nazareta.“
#1:46 Natanael mu namítl: „Z Nazareta? Co odtamtud může vzejít dobrého?“ Filip mu odpoví: „Pojď a přesvědč se!“
#1:47 Ježíš spatřil Natanaela, jak k němu přichází a řekl o něm: „Hle, pravý Izraelita, v němž není lsti.“
#1:48 Řekl mu Natanael: „Odkud mě znáš?“ Ježíš mu odpověděl: „Dříve, než tě Filip zavolal, viděl jsem tě pod fíkem.“
#1:49 „Mistře“, řekl mu Natanael, „ty jsi Syn Boží, ty jsi král Izraele.“
#1:50 Ježíš mu odpověděl: „Ty věříš proto, že jsem ti řekl: ‚Viděl jsem tě pod fíkem?‘ Uvidíš věci daleko větší.“
#1:51 A dodal: „Amen, amen, pravím vám, uzříte nebesa otevřená a anděly Boží vystupovat a sestupovat na Syna člověka.“ 
#2:1 Třetího dne byla svatba v Káně Galilejské. Byla tam Ježíšova matka;
#2:2 na svatbu byl pozván také Ježíš a jeho učedníci.
#2:3 Když se nedostávalo vína, řekla Ježíšovi jeho matka: „Už nemají víno.“
#2:4 Ježíš jí řekl: „Co to ode mne žádáš! Ještě nepřišla má hodina.“
#2:5 Matka řekla služebníkům: „Udělejte, cokoli vám nařídí.“
#2:6 Bylo tam šest kamenných nádob, určených k židovskému očišťování, každá na dvě až tři vědra.
#2:7 Ježíš řekl služebníkům: „Naplňte ty nádoby vodou!“ I naplnili je až po okraj.
#2:8 Pak jim přikázal: „Teď z nich naberte a doneste správci hostiny!“ Učinili tak.
#2:9 Jakmile správce hostiny ochutnal vodu proměněnou ve víno - nevěděl, odkud je, ale služebníci, kteří vodu nabírali, to věděli - zavolal si ženicha
#2:10 a řekl mu: „Každý člověk podává nejprve dobré víno, a teprve když už se hosté napijí, víno horší. Ty jsi však uchoval dobré víno až pro tuto chvíli.“
#2:11 Tak učinil Ježíš v Káně Galilejské počátek svých znamení a zjevil svou slávu. A jeho učedníci v něho uvěřili.
#2:12 Potom odešel Ježíš, jeho matka, bratři i učedníci do Kafarnaum a zůstali tam několik dní.
#2:13 Byly blízko židovské velikonoce a Ježíš se vydal na cestu do Jeruzaléma.
#2:14 V chrámu našel prodavače dobytka, ovcí a holubů i penězoměnce, jak sedí za stoly.
#2:15 Udělal si z provazů bič a všechny z chrámu vyhnal, i s ovcemi a dobytkem, směnárníkům rozházel mince, stoly zpřevracel
#2:16 a prodavačům holubů poručil: „Pryč s tím odtud! Nedělejte z domu mého Otce tržiště!“
#2:17 Jeho učedníci si vzpomněli, že je psáno: ‚Horlivost pro tvůj dům mne stráví.‘
#2:18 Židé mu řekli: „Jakým znamením nám prokážeš, že to smíš činit?“
#2:19 Ježíš jim odpověděl: „Zbořte tento chrám a ve třech dnech jej postavím.“
#2:20 Tu řekli Židé: „Čtyřicet šest let byl tento chrám budován, a ty jej chceš postavit ve třech dnech?“
#2:21 On však mluvil o chrámu svého těla.
#2:22 Když byl pak vzkříšen z mrtvých, rozpomenuli se jeho učedníci, že to říkal, a uvěřili Písmu i slovu, které Ježíš pověděl.
#2:23 Když byl v Jeruzalémě o velikonočních svátcích, mnozí uvěřili v jeho jméno, protože viděli znamení, která činil.
#2:24 Ježíš jim však nesvěřoval, kdo je, poněvadž všechny lidi znal;
#2:25 nepotřeboval, aby mu někdo o něm říkal svůj soud. Sám dobře věděl, co je v člověku. 
#3:1 Mezi farizeji byl člověk jménem Nikodém, člen židovské rady.
#3:2 Ten přišel k Ježíšovi v noci a řekl mu: „Mistře, víme, že jsi učitel, který přišel od Boha. Neboť nikdo nemůže činit ta znamení, která činíš ty, není-li Bůh s ním.“
#3:3 Ježíš mu odpověděl: „Amen, amen, pravím tobě, nenarodí-li se kdo znovu, nemůže spatřit království Boží.“
#3:4 Nikodém mu řekl: „Jak se může člověk narodit, když už je starý? Nemůže přece vstoupit do těla své matky a podruhé se narodit.“
#3:5 Ježíš odpověděl: „Amen, amen, pravím tobě, nenarodí-li se kdo z vody a z Ducha, nemůže vejít do království Božího.
#3:6 Co se narodilo z těla, je tělo, co se narodilo z Ducha, je duch.
#3:7 Nediv se, že jsem ti řekl: Musíte se narodit znovu.
#3:8 Vítr vane kam chce, jeho zvuk slyšíš, ale nevíš, odkud přichází a kam směřuje. Tak je to s každým, kdo se narodil z Ducha.“
#3:9 Nikodém se ho otázal: „Jak se to může stát?“
#3:10 Ježíš mu řekl: „Ty jsi učitel Izraele, a tohle nevíš?
#3:11 Amen, amen, pravím tobě, že mluvíme o tom, co známe, a svědčíme o tom, co jsme viděli, ale vy naše svědectví nepřijímáte.
#3:12 Jestliže nevěříte, když jsem vám mluvil o pozemských věcech, jak uvěříte, budu-li mluvit o nebeských?
#3:13 Nikdo nevstoupil na nebesa, leč ten, který sestoupil z nebes, Syn člověka.
#3:14 Jako Mojžíš vyvýšil hada na poušti, tak musí být vyvýšen Syn člověka,
#3:15 aby každý, kdo v něho věří, měl život věčný.
#3:16 Neboť Bůh tak miloval svět, že dal svého jediného Syna, aby žádný, kdo v něho věří, nezahynul, ale měl život věčný.
#3:17 Vždyť Bůh neposlal svého Syna na svět, aby soudil, ale aby skrze něj byl svět spasen.
#3:18 Kdo v něho věří, není souzen. Kdo nevěří, je již odsouzen, neboť neuvěřil ve jméno jednorozeného Syna Božího.
#3:19 Soud pak je v tom, že světlo přišlo na svět, ale lidé si zamilovali více tmu než světlo, protože jejich skutky byly zlé.
#3:20 Neboť každý, kdo dělá něco špatného, nenávidí světlo a nepřichází k světlu, aby jeho skutky nevyšly najevo.
#3:21 Kdo však činí pravdu, přichází ke světlu, aby se ukázalo, že jeho skutky jsou vykonány v Bohu.“
#3:22 Potom Ježíš odešel se svými učedníky do judské země; tam s nimi pobýval a křtil.
#3:23 Také Jan křtil v Ainon, blízko Salim, protože tam byl dostatek vody; lidé přicházeli a dávali se křtít.
#3:24 To bylo ještě před Janovým uvězněním.
#3:25 Mezi učedníky Janovými a Židy došlo ke sporu o očišťování.
#3:26 Přišli k Janovi a řekli mu: „Mistře, ten který byl s tebou na druhém břehu Jordánu, o němž jsi vydal dobré svědectví, nyní sám křtí a všichni chodí k němu.“
#3:27 Jan odpověděl: „Člověk si nic nemůže přisvojit, není-li mu to dáno z nebe.
#3:28 Vy sami jste svědkové, že jsem řekl: Já nejsem Mesiáš, ale jsem vyslán jako jeho předchůdce.
#3:29 Ženich je ten, kdo má nevěstu. Ženichův přítel, který u něho stojí a čeká na jeho rozkaz, upřímně se raduje, když uslyší ženichův hlas. A tak je má radost dovršena.
#3:30 On musí růst, já však se menšit.
#3:31 Kdo přichází shůry, je nade všecky. Kdo pochází ze země, náleží zemi a mluví o pozemských věcech. Kdo přichází z nebe je nade všecky,
#3:32 svědčí o tom, co viděl a slyšel, ale jeho svědectví nikdo nepřijímá.
#3:33 Kdo však jeho svědectví přijal, potvrdil tím, že Bůh je pravdivý.
#3:34 Ten, koho poslal Bůh, mluví slova Boží, neboť Bůh udílí svého Ducha v plnosti.
#3:35 Otec miluje Syna a všecku moc dal do jeho rukou.
#3:36 Kdo věří v Syna, má život věčný. Kdo Syna odmítá, neuzří život, ale hněv Boží na něm zůstává.“ 
#4:1 Když se Pán dověděl, že farizeové uslyšeli, jak on získává a křtí více učedníků než Jan -
#4:2 ač Ježíš sám nekřtil, nýbrž jeho učedníci -
#4:3 opustil Judsko a odešel opět do Galileje.
#4:4 Musel však projít Samařskem.
#4:5 Na té cestě přišel k samařskému městu jménem Sychar, v blízkosti pole, jež dal Jákob svému syny Josefovi;
#4:6 tam byla Jákobova studna. Ježíš, unaven cestou, usedl u té studny. Bylo kolem poledne.
#4:7 Tu přichází samařská žena, aby načerpala vody. Ježíš jí řekne: „Dej mi pít!“ -
#4:8 Jeho učedníci odešli před tím do města, aby nakoupili něco k jídlu. -
#4:9 Samařská žena mu odpoví: „Jak ty jako Žid, můžeš chtít ode mne, Samařanky, abych ti dala napít?“ Židé se totiž se Samařany nestýkají.
#4:10 Ježíš jí odpověděl: „Kdybys znala, co dává Bůh, a věděla, kdo ti říká, abys mu dala pít, požádala bys ty jeho, a on by ti dal vodu živou.“
#4:11 Žena mu řekla: „Pane, ani vědro nemáš a studna je hluboká, kde tedy vezmeš tu živou vodu?
#4:12 Jsi snad větší než náš praotec Jákob, který nám tuto studnu dal? Sám z ní pil, stejně jako jeho synové i jeho stáda.“
#4:13 Ježíš jí odpověděl: „Každý, kdo pije tuto vodu, bude mít opět žízeň
#4:14 Kdo by se však napil vody, kterou mu dám já, nebude žíznit navěky. Voda, kterou mu dám, stane se v něm pramenem, vyvěrajícím k životu věčnému.“
#4:15 Ta žena mu řekla: „Pane, dej mi té vody, abych už nežíznila a nemusela už sem chodit pro vodu.“
#4:16 Ježíš jí řekl: „Jdi, zavolej svého muže a přijď sem!“
#4:17 Žena mu řekla: „Nemám muže.“ Nato jí řekl Ježíš: „Správně jsi odpověděla, že nemáš muže.
#4:18 Vždyť jsi měla pět mužů, a ten, kterého máš nyní, není tvůj muž. To jsi řekla pravdu.“
#4:19 Žena mu řekla: „Pane, vidím, že jsi prorok.
#4:20 Naši předkové uctívali Boha na této hoře, ale vy říkáte, že místo, na němž má být Bůh uctíván, je v Jeruzalémě!“
#4:21 Ježíš jí odpoví: „Věř mi, ženo, že přichází hodina, kdy nebudete ctít Otce ani na této hoře ani v Jeruzalémě.
#4:22 Vy uctíváte, co neznáte; my uctíváme, co známe, neboť spása je ze Židů.
#4:23 Ale přichází hodina, ano již je tu, kdy ti, kteří Boha opravdově ctí, budou ho uctívat v Duchu a v pravdě. A Otec si přeje, aby ho lidé takto ctili.
#4:24 Bůh je Duch a ti, kdo ho uctívají, mají tak činit v Duchu a v pravdě.“
#4:25 Žena mu řekla: „Vím, že přichází Mesiáš, zvaný Kristus. Ten až přijde, oznámí nám všecko.“
#4:26 Ježíš jí řekl: „Já jsem to - ten, který k tobě mluví.“
#4:27 Vtom přišli jeho učedníci a divili se, že rozmlouvá s ženou. Nikdo však neřekl ‚nač se ptáš‘ nebo ‚proč s ní mluvíš?‘
#4:28 Žena tam nechala svůj džbán a odešla do města a řekla lidem:
#4:29 „Pojďte se podívat na člověka, který mi řekl všecko co jsem dělala. Není to snad Mesiáš?“
#4:30 Vyšli tedy z města a šli k němu.
#4:31 Mezi tím ho prosili jeho učedníci: „Mistře, pojez něco!“
#4:32 On jim řekl: „Já mám k nasycení pokrm, který vy neznáte.“
#4:33 Učedníci si mezi sebou říkali: „Přinesl mu snad někdo něco k jídlu?“
#4:34 Ježíš jim řekl: „Můj pokrm jest, abych činil vůli toho, který mě poslal, a dokonal jeho dílo.
#4:35 Neříkáte snad: Ještě čtyři měsíce a budou žně? Hle, pravím vám, pozvedněte zraky a pohleďte na pole, že již zbělela ke žni.
#4:36 Již přijímá odměnu ten, kdo žne, a shromažďuje úrodu k věčnému životu, aby se společně radovali rozsévač i žnec.
#4:37 Přitom je pravdivé rčení, že jeden rozsévá a druhý žne.
#4:38 Já jsem vás poslal, abyste žali tam, kde jste nepracovali. Jiní pracovali a vy v jejich práci pokračujete.“
#4:39 Mnoho Samařanů z onoho města v něho uvěřilo pro slovo té ženy, která svědčila: „Všecko mi řekl, co jsem dělala.“
#4:40 Když k němu ti Samařané přišli, prosili ho, aby u nich zůstal. I zůstal tam dva dny.
#4:41 A ještě mnohem víc jich uvěřilo pro jeho slovo.
#4:42 Oné ženě pak říkali: „Teď už věříme ne proto, cos nám o něm řekla; sami jsme ho slyšeli a víme, že toto je opravdu Spasitel světa.“
#4:43 Po dvou dnech odešel Ježíš odtamtud do Galileje.
#4:44 Sám totiž dosvědčil, že prorok nemá vážnosti ve své vlasti.
#4:45 Když přišel do Galileje, Galilejští jej přijali, protože viděli všecko, co učinil v Jeruzalémě o svátcích, které tam také slavili.
#4:46 Přišel tedy opět do Kány Galilejské, kde před tím proměnil vodu ve víno. V Kafarnaum byl jeden královský služebník, jehož syn byl nemocen.
#4:47 Když uslyšel, že Ježíš přišel z Judska do Galileje, vydal se k němu a prosil ho, aby přišel a uzdravil jeho syna, který už byl blízek smrti.
#4:48 Ježíš mu odpověděl: „Neuvidíte-li zázraky a znamení, neuvěříte.“
#4:49 Královský služebník mu řekl: „Pane, pojď, než mé dítě umře!“
#4:50 Ježíš mu odpověděl: „Vrať se domů, tvůj syn je živ!“ Ten člověk uvěřil slovu, které mu Ježíš řekl, a šel.
#4:51 Ještě když byl na cestě, šli mu naproti jeho sluhové a oznámili mu: „Tvůj syn žije.“
#4:52 Zeptal se jich, v kterou hodinu se mu začalo dařit lépe. Odpověděli mu: „Včera hodinu po poledni mu přestala horečka.“
#4:53 Tu otec poznal, že to bylo právě v tu chvíli, kdy mu Ježíš řekl: „Tvůj syn je živ.“ A uvěřil on i všichni v jeho domě. -
#4:54 To druhé znamení učinil Ježíš opět v Galileji, kam přišel z Judska. 
#5:1 Potom byly židovské svátky a Ježíš se vydal do Jeruzaléma.
#5:2 V Jeruzalémě je u Ovčí brány rybník, hebrejsky zvaný Bethesda, a u něho pět sloupořadí.
#5:3 V nich lehávalo množství nemocných, slepých, chromých a ochrnutých čekajících na pohyb vody.
#5:4 Neboť anděl Páně čas od času sestupoval do rybníka a vířil vodu; kdo první po tom zvíření vstoupil do vody, býval uzdraven, ať trpěl kteroukoli nemocí.
#5:5 Byl tam i jeden člověk, nemocný již třicet osm let.
#5:6 Když Ježíš spatřil, jak tam leží, a poznal, že je už dlouho nemocen, řekl mu: „Chceš být zdráv?“
#5:7 Nemocný mu odpověděl: „Pane, nemám nikoho, kdo by mě odnesl do rybníka, jakmile se voda rozvíří. Než se tam sám dostanu, jiný mě předejde.“
#5:8 Ježíš mu řekl: „Vstaň, vezmi lože své a choď!“
#5:9 A hned byl ten člověk uzdraven; vzal své lože a chodil. Toho dne však byla sobota.
#5:10 Židé řekli tomu uzdravenému: „Je sobota, a proto nesmíš nosit lože.“
#5:11 Odpověděl jim: „Ten, který mě uzdravil, mi řekl: Vezmi své lože a choď!“
#5:12 Zeptali se ho: „Kdo byl ten člověk, který ti řekl: Vezmi je a choď?“
#5:13 Ale uzdravený nevěděl, kdo to je, neboť Ježíš se mu ztratil v zástupu, který tam byl.
#5:14 Později vyhledal Ježíš toho člověka v chrámě a řekl mu: „Hle, jsi zdráv. Už nehřeš, aby tě nepotkalo něco horšího!“
#5:15 Ten člověk šel a oznámil Židům, že je to Ježíš, kdo ho uzdravil.
#5:16 A proto Židé začali Ježíš pronásledovat, že takové věci dělal v sobotu.
#5:17 On však jim odpověděl: „Můj Otec pracuje bez přestání, proto i já pracuji.“
#5:18 To bylo příčinou, že Židé ještě více usilovali Ježíše zabít, protože nejen znesvěcoval sobotu, ale dokonce nazýval Boha svým vlastním Otcem, a tak se mu stavěl na roveň.
#5:19 Ježíš jim řekl: „Amen, amen, pravím vám: Syn nemůže sám od sebe činit nic než to, co vidí činit Otce. Co činí Otec, stejně činí i jeho Syn.
#5:20 Vždyť Otec miluje Syna a ukazuje mu všecko, co sám činí; a ukáže mu ještě větší skutky, takže užasnete.
#5:21 Jako Otec mrtvé křísí a probouzí k životu, tak i Syn probouzí k životu, které chce.
#5:22 Otec nikoho nesoudí, ale všechen soud dal do rukou Synovi,
#5:23 aby všichni ctili Syna, jako ctí Otce. Kdo nemá v úctě Syna, nemá v úctě ani Otce, který ho poslal.
#5:24 Amen, amen, pravím vám, kdo slyší mé slovo a věří tomu, který mě poslal, má život věčný a neopodléhá soudu, ale přešel již ze smrti do života.
#5:25 Amen, amen, pravím vám, přichází hodina, ano, už je tu, kdy mrtví uslyší hlas Božího Syna, a kteří uslyší, budou žít.
#5:26 Neboť jako Otec má život sám v sobě, tak dal i Synovi, aby měl život sám v sobě.
#5:27 A dal mu moc konat soud, poněvadž je Syn člověka.
#5:28 Nedivte se tomu, neboť přichází hodina, kdy všichni v hrobech uslyší jeho hlas
#5:29 a vyjdou; ti, kdo činili dobré, vstanou k životu, a ti, kdo činili zlé, vstanou k odsouzení.
#5:30 Sám od sebe nemohu dělat nic; jak mi Bůh přikazuje, tak soudím, a můj soud je spravedlivý, neboť nehledám vůli svou, ale vůli toho, který mě poslal.“
#5:31 „Svědčím-li sám o sobě, mé svědectví není pravé.
#5:32 Je však jiný, který o mě svědčí, a já vím, že svědectví, které o mě vydává, je pravé.
#5:33 Poslali jste k Janovi a on vydal svědectví pravdě.
#5:34 Já nepotřebuji svědectví od člověka - ale říkám to proto, abyste vy byli spaseni.
#5:35 Jan byl svíce hořící a zářící, a vy jste se chtěli na chvíli radovat v jeho světle.
#5:36 Svědectví, které mám já, je větší než Janovo: skutky, jež mi Otec svěřil, abych je vykonal. Tyto skutky, které činím, svědčí o tom, že mě Otec poslal.
#5:37 A sám Otec, který mě poslal, vydal o mě svědectví. Vy jste však nikdy neslyšeli jeho hlas ani jste nespatřili jeho tvář
#5:38 a jeho slovo je ve vás nezůstává, poněvadž nevěříte tomu, koho on poslal.
#5:39 Zkoumáte Písma a myslíte si, že v nich máte věčný život; a Písma svědčí o mně.
#5:40 Ale vy nechcete přijít ke mně, abyste měli život.
#5:41 Nečekám slávu od lidí.
#5:42 Ale o vás jsem se přesvědčil, že v sobě nemáte lásku k Bohu.
#5:43 Přišel jsem ve jménu svého Otce, ale nepřijímáte mne. Kdyby přišel někdo ve svém vlastním jménu, toho přijmete.
#5:44 Jak byste mohli uvěřit, když oslavujete sebe navzájem, ale slávu od samého Boha nehledáte!
#5:45 Nedomnívejte se, že já budu na vás u Otce žalovat; vaším žalobcem je Mojžíš, v něhož jste složili svou naději.
#5:46 Kdybyste opravdu věřili Mojžíšovi, věřili byste i mně, neboť on psal o mně.
#5:47 Nevěříte-li tomu, co on napsal, jak uvěříte mým slovům?“ 
#6:1 Potom odešel Ježíš na druhý břeh Tiberiadského jezera v Galileji.
#6:2 Šel za ním velký zástup, poněvadž viděli znamení, která činil na nemocných.
#6:3 Ježíš vystoupil na horu a tam se posadil se svými učedníky.
#6:4 Byly blízko židovské svátky velikonoční.
#6:5 Když se Ježíš rozhlédl a viděl, že k němu přichází četný zástup, řekl Filipovi: „Kde nakoupíme chleba, aby se všichni najedli?“
#6:6 To však řekl, aby ho zkoušel; sám totiž věděl, co chce učinit.
#6:7 Filip mu odpověděl: „Ani za dvě stě denárů chleba nepostačí, aby se na každého aspoň něco dostalo.“
#6:8 Řekne mu jeden z učedníků, Ondřej, bratr Šimona Petra:
#6:9 „Je tu jeden chlapec, který má pět ječných chlebů a dvě ryby; ale co je to pro tolik lidí!“
#6:10 Ježíš řekl: „Ať se všichni posadí!“ Na tom místě bylo mnoho trávy. Posadili se tedy, mužů bylo asi pět tisíc.
#6:11 Pak vzal Ježíš chleby, vzdal díky a rozdílel sedícím; stejně i ryby, kolik kdo chtěl.
#6:12 Když se nasytili, řekl svým učedníkům: „Seberte zbylé nalámané chleby, aby nic nepřišlo nazmar!“
#6:13 Sebrali je tedy a naplnili dvanáct košů nalámanými díly, které z těch pěti ječných chlebů po jídle zbyly.
#6:14 Když lidé viděli znamení, které Ježíš učinil, říkali: „Opravdu je to ten Prorok, který má přijít na svět!“
#6:15 Když Ježíš poznal, že chtějí přijít a zmocnit se ho, aby ho provolali králem, odešel opět na horu, zcela sám.
#6:16 Když nastal večer, sestoupili jeho učedníci k moři,
#6:17 vstoupili na loď a jeli na druhý břeh do Kafarnaum. Už se setmělo a Ježíš s nimi stále ještě nebyl.
#6:18 Moře se vzdouvalo mocným náporem větru.
#6:19 Když veslovali asi pětadvacet nebo třicet stadií, spatřili Ježíše, jak kráčí po moři a blíží se k lodi; zmocnil se jich strach.
#6:20 On jim však řekl: „Já to jsem, nebojte se!“
#6:21 Chtěli jej vzít na loď, a hned se loď ocitla u břehu, k němuž jeli.
#6:22 Zástup zůstal na protějším břehu moře. Druhého dne si uvědomili, že tam byl jen jeden člun a že Ježíš na něj nevstoupil společně se svými učedníky, ale že učedníci odjeli sami.
#6:23 Jiné čluny z Tiberiady však přistály nedaleko místa, kde jedli chléb, nad nímž Pán vzdal díky.
#6:24 Když tedy zástup shledal, že Ježíš ani jeho učedníci tam nejsou, vstoupili na ty čluny a jeli ho do Kafarnaum hledat.
#6:25 Když jej na druhém břehu moře nalezli, řekli mu: „Mistře, kdy ses sem dostal?“
#6:26 Ježíš jim odpověděl: „Amen, amen, pravím vám, hledáte mě ne proto, že jste viděli znamení, ale proto, že jste jedli chléb a nasytili jste se.
#6:27 Neusilujte o pomíjející pokrm, ale o pokrm zůstávající pro život věčný; ten vám dá Syn člověka, jemuž jeho Otec, Bůh, vtiskl svou pečeť.“
#6:28 Řekli mu: „Jak máme jednat, abychom konali skutky Boží?“
#6:29 Ježíš jim odpověděl: „Toto je skutek, který žádá Bůh: abyste věřili v toho, koho on poslal.“
#6:30 Řekli mu: „Jaké znamení učiníš, abychom je viděli a uvěřili ti? Co dokážeš?
#6:31 Naši otcové jedli na poušti manu, jak je psáno: ‚Dal jim jíst chléb z nebe‘.“
#6:32 Ježíš jim řekl: „Amen, amen, pravím vám, chléb z nebe vám nedal Mojžíš; pravý chléb z nebe vám dává můj Otec.
#6:33 Neboť Boží chléb je ten, který sestupuje z nebe a dává život světu.“
#6:34 Řekli mu: „Pane, dávej nám ten chléb stále!“
#6:35 Ježíš jim řekl: „Já jsem chléb života; kdo přichází ke mně, nikdy nebude hladovět, a kdo věří ve mně, nikdy nebude žíznit.
#6:36 Ale řekl jsem vám: Viděli jste mě, a přece nevěříte.
#6:37 Všichni, které mi Otec dává, přijdou ke mně; a kdo ke mně přijde, toho nevyženu ven,
#6:38 neboť jsem sestoupil z nebe, ne abych činil vůli svou, ale abych činil vůli toho, který mě poslal;
#6:39 a jeho vůle jest, abych neztratil nikoho z těch, které mi dal, ale vzkřísil je v poslední den.
#6:40 Neboť to je vůle mého Otce, aby každý, kdo vidí Syna a věří v něho, měl život věčný; a já jej vzkřísím v poslední den.“
#6:41 Židé proti němu reptali, že řekl: ‚Já jsem chléb, který sestoupil z nebe‘.
#6:42 A říkali: „Což tohle není Ježíš, syn Josefův? Vždyť známe jeho otce i matku! Jak tedy může říkat: ‚Sestoupil jsem z nebe‘!“
#6:43 Ježíš jim odpověděl: „Nereptejte mezi sebou!
#6:44 Nikdo nemůže přijít ke mně, jestliže ho nepřitáhne Otec, který mě poslal; a já ho vzkřísím v poslední den.
#6:45 Je psáno v prorocích: ‚Všichni budou vyučeni od Boha‘. Každý, kdo slyšel Otce a vyučil se u něho, přichází ke mně.
#6:46 Ne že by někdo Otce viděl; jen ten, kdo je u Boha, viděl Otce.
#6:47 Amen, amen, pravím vám, kdo věří, má život věčný.
#6:48 Já jsem chléb života.
#6:49 Vaši Otcové jedli na poušti manu, a zemřeli.
#6:50 Toto je chléb, který sestupuje z nebe: kdo z něho jí, nezemře.
#6:51 Já jsem ten chléb živý, který sestoupil z nebe; kdo jí z toho chleba, živ bude na věky. A chléb, který já dám, je mé tělo, dané za život světa.“
#6:52 Židé se mezi sebou přeli: „Jak nám ten člověk může dát k jídlu své tělo?“
#6:53 Ježíš jim řekl: „Amen, amen, pravím vám, nebudete-li jíst tělo Syna člověka a pít jeho krev, nebudete mít v sobě život.
#6:54 Kdo jí mé tělo a pije mou krev, má život věčný a já ho vzkřísím v poslední den.
#6:55 Neboť mé tělo je pravý pokrm a má krev pravý nápoj.
#6:56 Kdo jí mé tělo a pije mou krev, zůstává ve mně a já v něm.
#6:57 Jako mne poslal živý Otec a já mám život z Otce, tak i ten, kdo mne jí, bude mít život ze mne.
#6:58 Toto je ten chléb, který sestoupil z nebe - ne jako jedli vaši otcové, a zemřeli. Kdo jí tento chléb, živ bude navěky.“
#6:59 To řekl, když učil v synagóze v Kafarnaum.
#6:60 Když to jeho učedníci slyšeli, mnozí z nich řekli: „To je hrozná řeč! Kdo to může poslouchat?“
#6:61 Ježíš poznal, že učedníci na to reptají, a řekl jim: „Nad tím se urážíte?
#6:62 Což až uvidíte Syna člověka vystupovat tam, kde byl dříve?
#6:63 Co dává život, je Duch, tělo samo nic neznamená. Slova, která jsem k vám mluvil, jsou Duch a jsou život.
#6:64 Ale někteří z vás nevěří.“ Ježíš totiž od počátku věděl, kteří nevěří a kdo je ten, který ho zradí. -
#6:65 A řekl: „Proto jsem vám pravil, že nikdo ke mně nemůže přijít, není-li mu to dáno od Otce.“
#6:66 Od té chvíle ho mnoho jeho učedníků opustilo a už s ním nechodili.
#6:67 Ježíš řekl Dvanácti: „I vy chcete odejít?“
#6:68 Šimon Petr mu odpověděl: „Pane, ke komu bychom šli? Ty máš slova věčného života.
#6:69 A my jsme uvěřili a poznali, že ty jsi ten Svatý Boží.“
#6:70 Ježíš jim odpověděl: „Nevyvolil jsem si vás dvanáct? A přece jeden z vás je ďábel.“
#6:71 Mínil Jidáše, syna Šimona Iškariotského. Ten ho totiž měl zradit, jeden z Dvanácti. 
#7:1 Potom chodil Ježíš po Galileji; nechtěl chodit po Judsku, protože mu Židé ukládali o život.
#7:2 Byly blízko židovské svátky stánků,
#7:3 a jeho bratří mu řekli: „Jdi odtud do Judska, aby tvoji učedníci viděli skutky, které činíš.
#7:4 Nikdo přece nezůstává se svými skutky v ústraní, chce-li být známý na veřejnosti. Činíš-li takové věci, ukaž se světu!“
#7:5 Ani jeho bratři v něj totiž nevěřili.
#7:6 Ježíš jim řekl: „Můj čas ještě nenastal, ale pro vás je stále vhodný čas.
#7:7 Vás nemůže svět nenávidět, mne však nenávidí, protože ho usvědčuji z jeho zlých skutků.
#7:8 Vy na svátky jděte, já na tyto svátky ještě nejdu, protože můj čas se dosud nenaplnil.“
#7:9 To ji řekl a zůstal v Galileji.
#7:10 Když jeho bratři odešli na svátky, tu šel také on - ale nepozorovaně, aby se o tom nevědělo.
#7:11 Židé ho o svátcích hledali a říkali: „Kde je?“
#7:12 Bylo o něm mezi lidmi mnoho dohadů. Jedni říkali: „Je dobrý.“ Jiní říkali: „Není, vždyť svádí lid.“
#7:13 Nikdo ovšem ze strachu před Židy o něm nemluvil veřejně.
#7:14 Když bylo uprostřed svátků, vstoupil Ježíš do chrámu a učil.
#7:15 Židé se divili a říkali: „Jak to přijde, že se vyzná v Písmech, když ho tomu nikdo neučil?“
#7:16 Ježíš jim odpověděl: „Mé učení není mé, ale toho, kdo mě poslal.
#7:17 Kdo chce činit jeho vůli, pozná, zda je mé učení z Boha, nebo mluvím-li sám za sebe.
#7:18 Kdo mluví sám za sebe, hledá svou vlastní slávu; kdo však hledá slávu toho, který ho poslal, ten je pravdivý a není v něm nepravosti.
#7:19 Nedal vám Mojžíš zákon? A nikdo z vás zákon neplní. Proč mne chcete zabít?“
#7:20 Zástup odpověděl: „Jsi posedlý? Kdo tě chce zabít?“
#7:21 Ježíš jim odpověděl: „Jediný skutek jsem učinil a všichni se nad tím pozastavujete.
#7:22 Mojžíš vám dal obřízku - ne že by pocházela teprve od Mojžíše, je od praotců - a vy obřezáváte člověka i v sobotu.
#7:23 Přijímá-li člověk v sobotu obřízku, aby byl dodržen Mojžíšův zákon, proč se na mne zlobíte, že jsem v sobotu uzdravil celého člověka?
#7:24 Nesuďte podle zdání, ale suďte spravedlivým soudem!“
#7:25 Tu řekli někteří Jeruzalémští: „Není to ten, kterého chtějí zabít?
#7:26 A hleďte, mluví veřejně, a nic mu neříkají. Snad nedošli přední muži opravdu k poznání, že je to Mesiáš?
#7:27 Ale o tomto člověku víme, odkud je. Až přijde Mesiáš, nikdo nebude vědět, odkud je.“
#7:28 Tu Ježíš, když učil v chrámě, hlasitě zvolal: „Znáte mě a víte také, odkud jsem. A přece jsem nepřišel sám od sebe, ale poslal mě ten, který je pravdivý; toho vy neznáte.
#7:29 Já ho však znám, neboť jsem od něho a on mě poslal.“
#7:30 Tehdy se ho chtěli zmocnit; ale nikdo na něj nevztáhl ruku, neboť ještě nepřišla jeho hodina.
#7:31 Ale mnozí ze zástupu v něj uvěřili a říkali: „Až přijde Mesiáš, bude snad činit více znamení než on?“
#7:32 Farizeové se doslechli, že se to v zástupu o něm říká, a tak společně s velekněžími poslali chrámovou stráž, aby ho zatkli.
#7:33 Ježíš řekl: „Ještě krátký čas budu s vámi a pak odejdu k tomu, který mě poslal.
#7:34 Budete mě hledat, ale nenajdete. A kde jsem já, tam vy přijít nemůžete.“
#7:35 Židé si říkali mezi sebou: „Kam hodlá jít, že ho nenajdeme? Chce snad jít mezi pohany a tam učit?
#7:36 Co znamenají slova, jež řekl: ‚Budete mě hledat a nenajdete, a tam, kde jsem já, vy nemůžete přijít‘?“
#7:37 V poslední, velký den svátků Ježíš vystoupil a zvolal: „Jestliže kdo žízní, ať přijde ke mně a pije!
#7:38 Kdo věří ve mne, ‚proud živé vody poplyne z jeho nitra‘, jak praví Písmo.“
#7:39 To řekl o Duchu, jejž měli přijmout ti, kteří v něj uvěřili. Dosud totiž Duch svatý nebyl dán, neboť Ježíš ještě nebyl oslaven.
#7:40 Když někteří ze zástupu slyšeli ta slova, říkali: „To je skutečně ten Prorok!“
#7:41 Druzí prohlašovali: „Je to Mesiáš!“ Jiní namítali: „Což přijde Mesiáš z Galileje?
#7:42 Neříká Písmo, že Mesiáš vzejde z potomstva Davidova a z Betléma, odkud byl David?“
#7:43 A tak došlo v zástupu kvůli němu k roztržce.
#7:44 Někteří ho chtěli zadržet; ale nikdo na něj nevztáhl ruku.
#7:45 Chrámová stráž se vrátila k velekněžím a farizeům a ti se jich ptali: „Proč jste ho nepřivedli?“
#7:46 Oni odpověděli: „Nikdo nikdy takto nemluvil!“
#7:47 Farizeové jim řekli: „I vy jste se dali svést?
#7:48 Uvěřil v něj někdo z předních mužů či farizeů?
#7:49 Jen tahle chátra, která nezná zákon - kletba na ně!“
#7:50 Jeden z nich, Nikodém, který za Ježíšem již předtím přišel, jim řekl:
#7:51 „Odsoudí náš zákon někoho, aniž ho napřed vyslechne a zjistí, čeho se dopustil?“
#7:52 Řekli mu: „Nejsi ty také z Galileje? Hledej v Písmu a uvidíš, že z Galileje prorok nikdy nepovstane!“
#7:53 Všichni se vrátili do svých domovů 
#8:1 Ježíš však odešel na Olivovou horu.
#8:2 Na úsvitě přišel opět do chrámu a všechen lid se k němu shromažďoval. On se posadil a učil je.
#8:3 Tu k němu zákoníci a farizeové přivedou ženu, přistiženou při cizoložství; postavili ji doprostřed
#8:4 a řeknou mu: „Mistře, tato žena byla přistižena při činu jako cizoložnice.
#8:5 V zákoně nám Mojžíš přikázal takové kamenovat. Co říkáš ty?“
#8:6 Tou otázkou ho zkoušeli, aby ho mohli obžalovat. Ježíš se sklonil a psal prstem po zemi.
#8:7 Když však na něj nepřestávali naléhat, zvedl se a řekl: „Kdo z vás je bez hříchu, první hoď na ni kamenem!“
#8:8 A opět se sklonil a psal po zemi.
#8:9 Když to uslyšeli, zahanbeni ve svém svědomí vytráceli se jeden po druhém, starší nejprve, až tam zůstal sám s tou ženou, která stála před ním.
#8:10 Ježíš se zvedl a řekl jí: „Ženo, kde jsou ti, kdo na tebe žalovali? Nikdo tě neodsoudil?“
#8:11 Ona řekla: „Nikdo, Pane.“ Ježíš jí řekl: „Ani já tě neodsuzuji. Jdi a už nehřeš!“
#8:12 Ježíš k nim opět promluvil a řekl: „Já jsem světlo světa; kdo mě následuje, nebude chodit ve tmě, ale bude mít světlo života.“
#8:13 Farizeové mu řekli: „Ty vydáváš svědectví sám o sobě, proto tvé svědectví není pravé.“
#8:14 Ježíš jim odpověděl: „I když vydávám svědectví sám o sobě, moje svědectví je pravé, neboť vím, odkud jsem přišel a kam jdu. Vy nevíte, odkud přicházím a kam jdu.
#8:15 Vy soudíte podle zdání. Já nesoudím nikoho.
#8:16 Jestliže já přece soudím, je můj soud pravdivý, neboť nesoudím sám, ale se mnou i ten, který mě poslal.
#8:17 I ve vašem zákoně je přece psáno, že svědectví dvou osob je pravé.
#8:18 Jsem to já, kdo svědčí sám o sobě; a svědčí o mně také Otec, který mě poslal.“
#8:19 Zeptali se ho: „Kde je tvůj otec?“ Ježíš odpověděl: „Neznáte ani mě ani mého Otce. Kdybyste znali mne, znali byste i mého Otce.“
#8:20 Ta slova řekl v síni pokladnic, když učil v chrámě. Ale nezatkli ho, protože dosud nepřišla jeho hodina.
#8:21 Opět jim řekl: „Já odcházím; budete mě hledat, ale umřete ve svém hříchu. Kam já jdu, tam vy přijít nemůžete.“
#8:22 Židé řekli: „Chce si snad vzít život, že říká: Kam já jdu, tam vy nemůžete přijít?“
#8:23 I řekl jim: „Vy jste zdola, ale já jsem shůry. Vy jste z tohoto světa.
#8:24 Proto jsem vám řekl, že zemřete ve svých hříších. Jestliže neuvěříte, že já to jsem, zemřete ve svých hříších.“
#8:25 Řekli jemu: „Kdo jsi ty?“ Ježíš jim odpověděl: „Co vám od začátku říkám.
#8:26 Mám o vás mnoho co říci, a to slova soudu; ten, který mě poslal, je pravdivý, a já oznamuji světu to, co jsem slyšel od něho.“
#8:27 Nepoznali, že k nim mluví o Otci.
#8:28 Ježíš jim řekl: „Teprve, až vyvýšíte Syna člověka, poznáte, že já jsem to a že sám od sebe nečiním nic, ale mluvím tak, jak mě naučil Otec.
#8:29 Ten, který mě poslal, je se mnou; nenechal mě samotného, neboť stále dělám, co se líbí jemu.“
#8:30 Když takto mluvil, mnozí v něho uvěřili.
#8:31 Ježíš řekl Židům, kteří mu uvěřili: „Zůstanete-li v mém slovu, jste opravdu mými učedníky.
#8:32 Poznáte pravdu a pravda vás učiní svobodnými.“
#8:33 Odpověděli mu: „Jsme potomci Abrahamovi a nikdy jsme nikomu neotročili. Jak můžeš říkat: stanete se svobodnými?“
#8:34 Ježíš jim odpověděl: „Amen, amen, pravím vám, že každý, kdo hřeší, je otrokem hříchu.
#8:35 Otrok nezůstává v domě navždy; navždy zůstává syn.
#8:36 Když vás Syn osvobodí, budete skutečně svobodni.
#8:37 Vím, že jste potomci Abrahamovi; ale chcete mě zabít, neboť pro mé slovo není u vás místa.
#8:38 Já mluvím o tom, co jsem viděl u Otce; a vy děláte, co jste slyšeli od vašeho otce.“
#8:39 Odpověděli mu: „Náš otec je Abraham.“ Ježíš jim řekl: „Kdybyste byli děti Abrahamovy, jednali byste jako on.
#8:40 Já jsem mluvil pravdu, kterou jsem slyšel od Boha, a vy mě chcete zabít. Tak Abraham nejednal.
#8:41 Vy konáte skutky svého otce.“ Řekli mu: „Nenarodili jsme se ze smilstva, máme jednoho Otce, Boha.“
#8:42 Ježíš jim řekl: „Kdyby Bůh byl váš Otec, milovali byste mě, neboť jsem od Boha vyšel a od něho přicházím. Nepřišel jsem sám od sebe, ale on mě poslal.
#8:43 Proč mou řeč nechápete? Proto, že nemůžete snést mé slovo.
#8:44 Váš otec je ďábel a vy chcete dělat, co on žádá. On byl vrah od počátku a nestál v pravdě, poněvadž v něm pravda není. Když mluví, nemůže jinak než lhát, protože je lhář a otec lži.
#8:45 Já mluvím pravdu, a proto mi nevěříte.
#8:46 Kdo z vás mě usvědčí z hříchu? Mluvím-li pravdu, proč mi nevěříte?
#8:47 Kdo je z Boha, slyší Boží řeč. Vy proto neslyšíte, že z Boha nejste.“
#8:48 Židé mu odpověděli: „Neřekli jsme správně, že jsi Samařan a jsi posedlý zlým duchem?“
#8:49 Ježíš odpověděl: „Nejsem posedlý, ale vzdávám čest svému Otci, vy však mi čest upíráte.
#8:50 Já sám nehledám svou slávu. Jest, kdo ji pro mne hledá, a ten soudí.
#8:51 Amen, amen, pravím vám, kdo zachovává mé slovo, nezemře navěky.“
#8:52 Židé mu řekli: „Teď jsme poznali, že jsi posedlý! Umřel Abraham, stejně i proroci, a ty pravíš: Kdo zachovává mé slovo, neokusí smrti navěky.“
#8:53 Jsi snad větší, než náš otec Abraham, který umřel? Také proroci umřeli. Co ze sebe děláš?“
#8:54 Ježíš odpověděl: „Kdybych oslavoval sám sebe, má sláva by nic nebyla. Mne oslavuje můj Otec, o němž vy říkáte, že je to váš Bůh.
#8:55 Vy jste ho nepoznali, ale já ho znám. Kdybych řekl, že ho neznám, budu lhář jako vy. Ale znám ho a jeho slovo zachovávám.
#8:56 Váš otec Abraham zajásal, že spatří můj den; spatřil jej a zaradoval se.“
#8:57 Židé mu řekli: „Ještě ti není padesát a viděl jsi Abrahama?“
#8:58 Ježíš jim odpověděl: „Amen, amen, pravím vám, dříve než se Abraham narodil, já jsem.“
#8:59 Tu se chopili kamenů a chtěli je po něm házet. Ježíš se však ukryl v zástupu a vyšel z chrámu. 
#9:1 Cestou uviděl člověka, který byl od narození slepý.
#9:2 Jeho učedníci se ho zeptali: „Mistře, kdo se prohřešil, že se ten člověk narodil slepý? On sám, nebo jeho rodiče?“
#9:3 Ježíš odpověděl: „Nezhřešil ani on ani jeho rodiče; je slepý, aby se na něm zjevily skutky Boží.
#9:4 Musíme konat skutky toho, který mě poslal, dokud je den. Přichází noc, kdy nikdo nebude moci pracovat.
#9:5 Pokud jsem na světě, jsem světlo světa.“
#9:6 Když to řekl, plivl na zem, udělal ze sliny bláto, potřel slepému tím blátem oči
#9:7 a řekl mu: „Jdi, umyj se v rybníce Siloe.“ (To jméno znamená ‚Poslaný‘.) On tedy šel, umyl se, a když se vrátil, viděl.
#9:8 Sousedé a ti, kteří jej dříve vídali žebrat, se ptali: „Není to ten, kdo tu sedával a žebral?“
#9:9 Jedni říkali: „Je to on.“ Jiní pak: „Není, ale je mu podoben.“ On sám řekl: „Jsem to já.“
#9:10 I řekli mu: „Jak to, že se ti otevřely oči?“
#9:11 Odpověděl: „Člověk jménem Ježíš udělal bláto, potřel mi oči a řekl mi: Jdi k Siloe a umyj se! Šel jsem tedy, umyl jsem se a vidím.“
#9:12 Řekli mu: „Kde je ten člověk?“ Odpověděl: „To nevím.“
#9:13 Přivedou toho, který byl dříve slepý, k farizeům;
#9:14 toho dne, kdy Ježíš udělal bláto a otevřel mu oči, byla totiž sobota.
#9:15 Proto se ho farizeové znovu dotazovali, jak nabyl zraku. A on jim řekl: „Položil mi bláto na oči, umyl jsem se a vidím.“
#9:16 Někteří z farizeů říkali: „Ten člověk není od Boha, protože nezachovává sobotu.“ Jiní naopak říkali: „Jak by mohl hříšný člověk činit taková znamení?“ A došlo mezi nimi k roztržce.
#9:17 Řekli tedy znovu tomu slepému: „Za koho ty jej pokládáš, když ti otevřel oči?“ On odpověděl: „Je to prorok.“
#9:18 Židé nevěřili, že byl slepý a že prohlédl, dokud si nezavolali jeho rodiče
#9:19 a nezeptali se jich: „Je to váš syn, o němž říkáte, že se narodil slepý? Jak to že nyní vidí?“
#9:20 Rodiče odpověděl: „Víme, že je to náš syn a že se narodil slepý.
#9:21 Jak to, že nyní vidí, to nevíme, a kdo mu otevřel oči, také ne. Jeho se zeptejte, je dospělý, ať mluví sám za sebe!“
#9:22 To řekli jeho rodiče, protože se báli Židů, neboť Židé se usnesli, aby ten, kdo kdo Ježíš vyzná jako Mesiáše, byl vyloučen ze synagógy.
#9:23 Proto řekli jeho rodiče: Je dospělý, zeptejte se ho!
#9:24 Zavolali tedy ještě jednou toho člověka, který byl dříve slepý, a řekli mu: „Vyznej před Bohem pravdu! My víme, že ten člověk je hříšník.“
#9:25 Odpověděl: „Je-li hříšník, nevím; jedno však vím, že jsem byl slepý a nyní vidím.“
#9:26 Řekli mu: „Co s tebou učinil? Jak ti otevřel oči?“
#9:27 Odpověděl jim: „Již jsem vám to řekl, ale vy jste to nevzali na vědomí. Proč to chcete slyšet znovu? Chcete se snad i vy stát jeho učedníky?“
#9:28 Osopili se na něho: „Ty jsi jeho učedník, ale my jsme učedníci Mojžíšovi.
#9:29 My víme, že k Mojžíšovi mluvil Bůh, o tomhle však nevíme, odkud je.“
#9:30 Ten člověk jim odpověděl: „To je právě divné: Vy nevíte odkud je - a otevřel mi oči!
#9:31 Víme, že hříšníky Bůh neslyší; slyší však toho, kdo ctí a činí jeho vůli.
#9:32 Co je svět světem, nebylo slýcháno, že by někdo otevřel oči slepého od narození.
#9:33 Kdyby tento člověk nebyl od Boha, nemohl by nic takového učinit.“
#9:34 Odpověděli mu: „Celý ses narodil v hříchu, a nás chceš poučovat?“ A vyhnali ho.
#9:35 Ježíš se dověděl, že ho vyhnali; vyhledal ho a řekl mu: „Věříš v Syna člověka?“
#9:36 Odpověděl: „A kdo je to, pane, abych v něho uvěřil?“
#9:37 Ježíš mu řekl: „Vidíš ho; je to ten, kdo s tebou mluví.“
#9:38 On na to řekl: „Věřím, Pane,“ a padl před ním na kolena.
#9:39 Ježíš řekl: „Přišel jsem na tento svět k soudu: aby ti, kdo nevidí, viděli, a ti, kdo vidí, byli slepí.“
#9:40 Farizeové, kteří tam byli, to slyšeli a řekli mu: „Jsme snad i my slepí?“
#9:41 Ježíš jim odpověděl: „Kdybyste byli slepí, hřích byste neměli. Vy však říkáte: Vidíme. A tak zůstáváte v hříchu.“ 
#10:1 „Amen, amen, pravím vám: Kdo nevchází do ovčince dveřmi, ale přelézá ohradu, je zloděj a lupič.
#10:2 Kdo však vchází dveřmi, je pastýř ovcí.
#10:3 Vrátný mu otvírá a ovce slyší jeho hlas. Volá své ovce jménem a vyvádí je.
#10:4 Když je má všecky venku, kráčí před nimi a ovce jdou za ním, protože znají jeho hlas.
#10:5 Za cizím však nepůjdou, ale utečou od něho, protože hlas cizích neznají.“
#10:6 Toto přirovnání jim Ježíš řekl; oni však nepochopili, co tím chtěl říci.
#10:7 Řekl jim tedy Ježíš znovu: „Amen, amen, pravím vám, já jsem dveře pro ovce.
#10:8 Všichni, kdo přišli přede mnou, jsou zloději a lupiči. Ale ovce je neposlouchaly.
#10:9 Já jsem dveře. Kdo vejde skrze mne, bude zachráněn, bude vcházet i vycházet a nalezne pastvu.
#10:10 Zloděj přichází, jen aby kradl, zabíjel a ničil. Já jsem přišel, aby měly život a měly ho v hojnosti.
#10:11 Já jsem dobrý pastýř. Dobrý pastýř položí svůj život za ovce.
#10:12 Ten, kdo není pastýř, kdo pracuje jen za mzdu a ovce nejsou jeho vlastní, opouští je a utíká, když vidí, že se blíží vlk. A vlk ovce trhá a rozhání.
#10:13 Tomu, kdo je najat za mzdu, na nich nezáleží.
#10:14 Já jsem dobrý pastýř; znám své ovce a ony znají mne,
#10:15 tak jako mě zná Otec a já znám Otce. A svůj život dávám za ovce.
#10:16 Mám i jiné ovce, které nejsou z tohoto ovčince. I ty musím přivést. Uslyší můj hlas a bude jedno stádo a jeden pastýř.
#10:17 Proto mě Otec miluje, že dávám svůj život, abych jej opět přijal.
#10:18 Nikdo mi ho nebere, ale já jej dávám sám od sebe. Mám moc svůj život dát a mám moc jej opět přijmout. Takový příkaz jsem přijal od svého Otce.“
#10:19 Pro tato slova došlo mezi Židy opět k roztržce.
#10:20 Mnozí z nich říkali: „Je posedlý zlým duchem a blázní. Proč ho posloucháte?“
#10:21 Jiní říkali: „Tak nemluví posedlý. Což může zlý duch otevřít oči slepých?“
#10:22 Byly právě svátky posvěcení Jeruzalémského chrámu; bylo to v zimě.
#10:23 Ježíš se procházel v chrámě, v sloupoví Šalomounově.
#10:24 Židé ho obklopili a řekli mu: „Jak dlouho nás chceš držet v nejistotě? Jsi-li Mesiáš, řekni nám to otevřeně!“
#10:25 Ježíš jim odpověděl: „Řekl jsem vám to, a nevěříte. Skutky, které činím ve jménu Otce, ty o mně vydávají svědectví.
#10:26 Ale vy nevěříte, protože nejste z mých ovcí.
#10:27 Moje ovce slyší můj hlas, já je znám, jdou za mnou
#10:28 a já jim dávám věčný život: nezahynou navěky a nikdo je z mé ruky nevyrve.
#10:29 Můj Otec, který mi je dal, je větší nade všecky, a nikdo je nemůže vyrvat z Otcovy ruky.
#10:30 Já a Otec jsme jedno.“
#10:31 Židé se opět chopili kamenů, aby ho ukamenovali.
#10:32 Ježíš jim řekl: „Ukázal jsem vám mnoho dobrých skutků od Otce. Pro který z nich mne chcete kamenovat?“
#10:33 Židé mu odpověděli: „Nechceme tě kamenovat pro dobrý skutek, ale pro rouhání: jsi člověk a tvrdíš, že jsi Bůh.“
#10:34 Ježíš jim řekl: „Ve vašem zákoně je přece psáno: ‚Řekl jsem: jste bohové.‘
#10:35 Jestliže Bůh ty, jichž se týká toto slovo, nazval bohy - a Písmo musí platit -
#10:36 jak můžete obviňovat mne, kterého Otec posvětil a poslal do světa, že se rouhám, protože jsem řekl: Jsem Boží Syn?
#10:37 Nečiním-li skutky svého Otce, nevěřte mi!
#10:38 Jestliže je však činím a nevěříte mně, věřte těm skutkům, abyste jednou provždy pochopili, že Otec je ve mně a já v Otci.“
#10:39 Opět se ho chtěli zmocnit, on jim však unikl.
#10:40 Odešel znovu na druhý břeh Jordánu, na místo, kde dříve křtil Jan, a tam se zdržoval.
#10:41 Mnozí k němu přicházeli a říkali: „Jan sice neučinil žádné znamení, ale vše, co o něm řekl, je pravda.“
#10:42 A mnoho lidí tam v Ježíše uvěřilo. 
#11:1 Byl nemocen jeden člověk, Lazar z Betanie, z vesnice, kde bydlela Marie a její sestra Marta.
#11:2 To byla ta Marie, která pomazala Pána vzácným olejem a nohy mu otřela svými vlasy; a její bratr Lazar byl nemocen.
#11:3 Sestry mu vzkázaly: „Pane, ten, kterého máš rád, je nemocen.“
#11:4 Když to Ježíš uslyšel, řekl: „Ta nemoc není k smrti, ale k slávě Boží, aby Syn Boží byl skrze ni oslaven.“
#11:5 Ježíš Martu, její sestru i Lazara miloval.
#11:6 Když uslyšel, že je Lazar nemocen, zůstal ještě dva dny na tom místě, kde byl.
#11:7 Teprve potom řekl svým učedníkům: „Pojďme opět do Judska!“
#11:8 Učedníci mu řekl: „Mistře, není to dávno, co tě chtěli Židé ukamenovat, a zase tam chceš jít?“
#11:9 Ježíš odpověděl: „Což nemá den dvanáct hodin? Kdo chodí ve dne, neklopýtne, neboť vidí světlo tohoto světa.
#11:10 Kdo však chodí v noci, klopýtá, poněvadž v něm není světla.“
#11:11 To pověděl a dodal: „Náš přítel Lazar usnul. Ale jdu ho probudit.“
#11:12 Učedníci mu řekli: „Pane, spí-li, uzdraví se.“
#11:13 Ježíš mluvil o jeho smrti, ale oni mysleli, že mluví o pouhém spánku.
#11:14 Tehdy jim Ježíš řekl: „Lazar umřel.
#11:15 A jsem rád, že jsem tam nebyl, kvůli vám, abyste uvěřili. Pojďme k němu!“
#11:16 Tomáš, jinak Didymos, řekl ostatním učedníkům: „Pojďme i my, ať zemřeme spolu s ním!“
#11:17 Když Ježíš přišel, shledal, že Lazar je již čtyři dny v hrobě.
#11:18 Betanie byla blízko Jeruzaléma, necelou hodinu cesty,
#11:19 a mnozí z Židů přišli k Marii a Martě, aby je potěšili v jejich zármutku nad jejich bratrem.
#11:20 Když Marta uslyšela, že Ježíš přichází, šla mu naproti. Marie zůstala doma.
#11:21 Marta řekla Ježíšovi: „Pane, kdybys byl zde, nebyl by můj bratr umřel.
#11:22 Ale i tak vím, že začkoli požádáš Boha, Bůh ti dá.“
#11:23 Ježíš jí řekl: „Tvůj bratr vstane.“
#11:24 Řekla mu Marta: „Vím, že vstane při vzkříšení v poslední den.“
#11:25 Ježíš jí řekl: „Já jsem vzkříšení i život. Kdo věří ve mne, i kdyby umřel, bude žít.
#11:26 A každý, kdo žije a věří ve mne, neumře navěky. Věříš tomu?“
#11:27 Řekla mu: „Ano, Pane. Já jsem uvěřila, že ty jsi Mesiáš, Syn Boží, který má přijít na svět.“
#11:28 S těmi slovy odešla, zavolala svou sestru Marii stranou a řekla jí: „Je tu Mistr a volá tě.“
#11:29 Jak to Marie uslyšela, rychle vstala a šla k němu.
#11:30 Ježíš totiž ještě nedošel do vesnice a byl na tom místě, kde se s ním Marta setkala.
#11:31 Když viděli Židé, kteří byli s Marií v domě a těšili ji, že rychle vstala a vyšla, šli za ní; domnívali se, že jde k hrobu, aby se tam vyplakala.
#11:32 Jakmile Marie přišla tam, kde byl Ježíš, a spatřila ho, padla mu k nohám a řekla: „Pane, kdybys byl zde, nebyl by můj bratr umřel.“
#11:33 Když Ježíš viděl, jak pláče a jak pláčou i Židé, kteří přišli s ní, v Duchu se rozhorlil a vzrušen
#11:34 řekl: „Kam jste ho položili?“ Řekli mu: „Pane, pojď se podívat!“
#11:35 Ježíšovi vstoupily do očí slzy.
#11:36 Židé říkali: „Hle, jak jej miloval!“
#11:37 Někteří z nich však řekli: „Když otevřel oči slepému, nemohl způsobit, aby tento člověk neumřel?“
#11:38 Ježíš, znovu rozhorlen, přichází k hrobu. Byla to jeskyně a na ní ležel kámen.
#11:39 Ježíš řekl: „Zvedněte ten kámen!“ sestra zemřelého Marta mu řekla: „Pane, už je v rozkladu, vždyť je to čtvrtý den.“
#11:40 Ježíš jí odpověděl: „Neřekl jsem ti, že uvidíš slávu Boží, budeš-li věřit?“
#11:41 Zvedli tedy kámen. Ježíš pohlédl vzhůru a řekl: „Otče, děkuji ti, žes mě vyslyšel.
#11:42 Věděl jsem sice, že mě vždycky slyšíš, ale řekl jsem to kvůli zástupu, který stojí kolem, aby uvěřili, že ty jsi mě poslal.“
#11:43 Když to řekl, zvolal mocným hlasem: „Lazare, pojď ven!“
#11:44 Zemřelý vyšel, měl plátnem svázány ruce i nohy a tvář měl zahalenu šátkem. Ježíš jim řekl: „Rozvažte ho a nechte odejít!“
#11:45 Mnozí z Židů, kteří přišli k Marii a viděli, co Ježíš učinil, uvěřili v něho.
#11:46 Ale někteří z nich šli k farizeům a oznámili jim, co učinil.
#11:47 Velekněží a farizeové svolali radu a řekli: „Co si počneme? Ten člověk činí mnohá znamení
#11:48 Když proti němu nezakročíme, všichni v něj uvěří, a přijdou Římané a zničí nám toto svaté místo i národ.“
#11:49 Jeden z nich, Kaifáš, velekněz toho roku, jim řekl: „Vy ničemu nerozumíte;
#11:50 nechápete, že je pro vás lépe, aby jeden člověk zemřel za lid, než aby zahynul celý národ.“
#11:51 To však neřekl sám ze sebe, ale jako velekněz toho roku vyřkl proroctví, že Ježíš má zemřít za národ,
#11:52 a nejenom za národ, ale také proto, aby rozptýlené děti Boží shromáždil v jedno.
#11:53 Od toho dne byli tedy smluveni, že ho zabijí.
#11:54 Proto už Ježíš nechodil veřejně mezi Židy, ale odešel odtud do kraje blízko pouště, do města jménem Efraim, a tam zůstal s učedníky.
#11:55 Bylo krátce před židovskými velikonocemi a mnozí z té krajiny přišli před svátky do Jeruzaléma, aby se očistili.
#11:56 Hledali Ježíše, a jak stáli v chrámě, říkali si mezi sebou: „Co myslíte? Přijde na svátky?“
#11:57 Velekněží a farizeové vydali totiž nařízení, že každý, kdo by věděl, kde se zdržuje, má to oznámit, aby ho mohli zatknout. 
#12:1 Šest dní před velikonocemi přišel Ježíš do Betanie, kde bydlel Lazar, kterého vzkřísil z mrtvých.
#12:2 Připravili mu tam večeři; Marta při ní obsluhovala a Lazar byl jeden z těch, kteří byli s Ježíšem u stolu.
#12:3 Tu vzala Marie libru drahého oleje z pravého nardu, pomazala Ježíšovi nohy a otřela je svými vlasy. Dům se naplnil vůní té masti.
#12:4 Jidáš Iškariotský, jeden z jeho učedníků, který jej měl zradit, řekl:
#12:5 „Proč nebyl ten olej prodán za tři sta denárů a peníze dány chudým?“
#12:6 To řekl ne proto, že by mu záleželo na chudých, ale že byl zloděj: měl na starosti pokladnici a bral z toho, co se do ní dávalo.
#12:7 Ježíš řekl: „Nech ji, uchovala to ke dni mého pohřbu!
#12:8 Chudé máte vždycky s sebou, ale mne nemáte vždycky.“
#12:9 Velký zástup Židů se dověděl, že tam Ježíš je; a přišli nejen kvůli němu, ale také aby viděli Lazara, kterého vzkřísil z mrtvých.
#12:10 Proto se velekněží uradili, že zabijí i Lazara;
#12:11 neboť mnozí Židé kvůli němu odcházeli a věřili v Ježíše.
#12:12 Druhého dne se dovědělo mnoho poutníků, kteří přišli na svátky, že Ježíš přichází do Jeruzaléma.
#12:13 Vzali palmové ratolesti, šli ho uvítat a volali: „Hosanna, požehnaný, jenž přichází ve jménu Hospodinově, král izraelský.“
#12:14 Ježíš nalezl oslátko a vsedl na ně, jak je psáno:
#12:15 ‚Neboj se, dcero Siónská, hle, král tvůj přichází, sedě na oslátku.‘
#12:16 Jeho učedníci tomu v té chvíli nerozuměli, ale když byl Ježíš oslaven, tu se rozpomenuli, že to o něm bylo psáno a že se tak stalo.
#12:17 Zástup, který byl s ním, když vyvolal Lazara z hrobu z vzkřísil ho z mrtvých, vydával o tom svědectví.
#12:18 Proto ho také přišlo uvítat množství lidu, neboť slyšeli, že učinil toto znamení.
#12:19 Farizeové si řekli: „Vidíte, že nic nezmůžete! Celý svět se dal za ním.“
#12:20 Někteří z poutníků, kteří se přišli o svátcích klanět Bohu, byli Řekové.
#12:21 Ti přistoupili k Filipovi, který byl z Betsaidy v Galileji, a prosili ho: „Pane, rádi bychom viděli Ježíše.“
#12:22 Filip šel a řekl to Ondřejovi, Ondřej a Filip to šli říci Ježíšovi.
#12:23 Ježíš jim odpověděl: „Přišla hodina, aby byl oslaven Syn člověka.
#12:24 Amen, amen, pravím vám, jestliže pšeničné zrno nepadne do země a nezemře, zůstane samo. Zemře-li však, vydá mnohý užitek.
#12:25 Kdo miluje svůj život, ztratí jej; kdo nenávidí svůj život v tomto světě, uchrání jej pro život věčný.
#12:26 Kdo mně chce sloužit, ať mě následuje, a kde jsem já, tam bude i můj služebník. Kdo mně slouží, dojde cti od Otce.“
#12:27 „Nyní je má duše sevřena úzkostí. Mám snad říci: Otče, zachraň mě od této hodiny? Vždyť pro tuto hodinu jsem přišel.
#12:28 Otče, oslav své jméno!“ Z nebe zazněl hlas: „Oslavil jsem a ještě oslavím.“
#12:29 Zástup, který tam stál a slyšel to, říkal, že zahřmělo. Jiní tvrdili: „Anděl k němu promluvil.“
#12:30 Ježíš na to řekl: „Tento hlas se neozval kvůli mně, ale kvůli vám.
#12:31 Nyní je soud nad tímto světem, nyní bude vládce tohoto světa vyvržen ven.
#12:32 A já, až budu vyvýšen ze země, přitáhnu všecky k sobě.“
#12:33 To řekl, aby naznačil, jakou smrtí má zemřít.
#12:34 Zástup mu odpověděl: „My jsme slyšeli ze zákona, že Mesiáš má zůstat navěky; jak ty můžeš říkat, že Syn člověka musí být vyvýšen? Kdo je ten Syn člověka?“
#12:35 Ježíš jim řekl: „Ještě jen malou chvíli je světlo mezi vámi. Dokud máte světlo, neustávejte v cestě, aby vás nepřekvapila tma; kdo chodí ve tmě, neví, kam jde.
#12:36 Dokud máte světlo, věřte ve světlo, abyste se stali syny světla.“ Tak promluvil Ježíš; potom odešel a skryl se před nimi.
#12:37 Ač před nimi učinil taková znamení, nevěřili v něho;
#12:38 a tak se naplnilo slovo, které řekl prorok Izaiáš: ‚Hospodine, kdo uvěřil našemu kázání? A komu byla zjevena moc Hospodinova?‘
#12:39 Proto nemohli věřit, neboť Izaiáš také řekl:
#12:40 ‚Oslepil jim oči a zatvrdil jim srdce, takže nevidí očima a srdce nepochopí, neobrátí se a já je neuzdravím.‘
#12:41 Tak řekl Izaiáš, neboť viděl jeho slávu a mluvil o něm.
#12:42 Přesto v něho uvěřili mnozí z předních mužů, ale kvůli farizeům se k němu nepřiznávali, aby nebyli vyloučeni ze synagógy.
#12:43 Zamilovali si lidskou slávu víc než slávu Boží.
#12:44 Ježíš hlasitě zvolal: „Kdo věří ve mne, ne ve mne věří, ale v toho, který mě poslal.
#12:45 A kdo vidí mne, vidí toho, který mě poslal.
#12:46 Já jsem přišel na svět jako světlo, aby nikdo, kdo ve mne věří, nezůstal ve tmě.
#12:47 Kdo slyší má slova a nezachovává je, toho já nesoudím. Nepřišel jsem, abych soudil svět, ale abych svět spasil.
#12:48 Kdo mě odmítá a nepřijímá moje slova, má, kdo by jej soudil: Slovo, které jsem mluvil, to jej bude soudit v poslední den.
#12:49 Neboť jsem nemluvil sám ze sebe, ale Otec, který mě poslal, přikázal mi, jak mám mluvit a co říci.
#12:50 A vím, že jeho přikázání je věčný život. Co tedy mluví, mluvím tak, jak mi pověděl Otec.“ 
#13:1 Bylo před velikonočními svátky. Ježíš věděl, že přišla jeho hodina, aby z tohoto světa šel k Otci; miloval své, kteří jsou ve světě, a prokázal svou lásku k nim až do konce.
#13:2 Když byli u večeře a ďábel již vložil do srdce Jidáše Iškariotského, syna Šimonova, aby ho zradil,
#13:3 Ježíš vstal od stolu a vědom si toho, že mu Otec dal všecko do rukou a že od Boha vyšel a k Bohu odchází,
#13:4 odložil svrchní šat, vzal lněné plátno a přepásal se;
#13:5 pak nalil vodu do umyvadla a začal učedníkům umývat nohy a utírat je plátnem, jímž byl přepásán.
#13:6 Přišel k Šimonu Petrovi a ten mu řekl: „Pane, ty mi chceš mýt nohy?“
#13:7 Ježíš odpověděl: „Co já činím, nyní nechápeš, potom však to pochopíš.“
#13:8 Petr mu řekl: „Nikdy mi nebudeš mýt nohy!“ Ježíš odpověděl: „Jestliže tě neumyji, nebudeš mít se mnou podíl.“
#13:9 Řekl mu Šimon Petr: „Pane, pak tedy nejenom nohy, ale i ruce a hlavu!“
#13:10 Ježíš mu řekl: „Kdo je vykoupán, nepotřebuje než nohy umýt, neboť je celý čistý. I vy jste čistí, ale ne všichni.“
#13:11 Věděl, kdo ho zradí, a proto řekl: Ne všichni jste čistí.
#13:12 Když jim umyl nohy, oblékl si svůj šat, opět se posadil a řekl jim: „Chápete, co jsem vám učinil?
#13:13 Nazýváte mě Mistrem a Pánem, a máte pravdu: Skutečně jsem.
#13:14 Jestliže tedy já, Pán a Mistr, jsem vám umyl nohy, i vy máte jeden druhému nohy umývat.
#13:15 Dal jsem vám příklad, abyste i vy jednali, jako jsem jednal já.
#13:16 Amen, amen, pravím vám, sluha není větší než jeho pán a posel není větší než ten, kdo ho poslal.
#13:17 Když to víte, blaze vám, jestliže to také činíte.
#13:18 Nemluvím o vás všech. Já vím, které jsem vyvolil. Ale má se naplnit slovo Písma: ‚Ten, který se mnou jí chléb, zvedl proti mně patu.‘
#13:19 Říkám vám to již nyní předem, abyste potom, až se to stane, uvěřili, že já jsem to.
#13:20 Amen, amen, pravím vám, kdo přijímá toho, koho pošlu, mne přijímá. A kdo přijímá mne, přijímá toho, který mě poslal.
#13:21 Když to Ježíš řekl, v hlubokém zármutku dosvědčil: „Amen, amen, pravím vám, jeden z vás mě zradí.“
#13:22 Učedníci se dívali jeden na druhého v nejistotě, o kom to mluví.
#13:23 Jeden z učedníků, kterého Ježíš miloval, byl u stolu po jeho boku.
#13:24 Na toho se Šimon obrátil a řekl: „Zeptej se, o kom to mluví!“
#13:25 Ten učedník se naklonil těsně k Ježíšovi a zeptal se: „Pane, kdo to je?“
#13:26 Ježíš odpověděl: „Je to ten pro koho omočím tuto skývu chleba a podám mu ji.“ Omočil tedy skývu, vzal ji a dal Jidášovi Iškariotskému, synu Šimonovu.
#13:27 Tehdy, po té skývě vstoupil do něho satan. Ježíš mu řekl: „Co chceš učinit, učiň hned!“
#13:28 Nikdo u stolu nepochopil, proč mu to řekl.
#13:29 Protože měl Jidáš u sebe pokladnici, domnívali se někteří, že ho poslal nakoupit, co potřebují na svátky, nebo dát něco chudým.
#13:30 Jidáš přijal skývu, a hned vyšel ven. byla noc.
#13:31 Když Jidáš vyšel ven, Ježíš řekl: „Nyní byl oslaven Syn člověka a Bůh byl oslaven v něm;
#13:32 Bůh jej také oslaví v sobě a oslaví jej hned.
#13:33 Dítky, ještě jen krátký čas jsem s vámi. Budete mě hledat, a jako jsem řekl Židům, tak nyní říkám i vám: Kam já odcházím, tam vy přijít nemůžete.
#13:34 Nové přikázání vám dávám, abyste se navzájem milovali; jako já jsem miloval vás, i vy se milujte navzájem.
#13:35 Podle toho všichni poznají, že jste moji učedníci, budete-li mít lásku jedni k druhým.“
#13:36 Šimon Petr mu řekl: „Pane, kam odcházíš?“ Ježíš odpověděl: „Kam já jdu, tam mne nyní následovat nemůžeš; budeš mne však následovat později.“
#13:37 Šimon Petr mu řekl: „Pane, proč tě nemohu nyní následovat?“ Svůj život za tebe položím.“
#13:38 Ježíš odpověděl: „Svůj život za mne položíš? Amen, amen, pravím tobě: Než kohout zakokrhá, třikrát mě zapřeš.“ 
#14:1 „Vaše srdce ať se nechvěje úzkostí! Věříte v Boha, věřte i ve mne.
#14:2 V domě mého Otce je mnoho příbytků; kdyby tomu tak nebylo, řekl bych vám to. Jdu, abych vám připravil místo.
#14:3 A odejdu-li, abych vám připravil místo, opět přijdu a vezmu vás k sobě, abyste i vy byli, kde jsem já.
#14:4 A cestu, kam jdu, znáte.“
#14:5 Řekne mu Tomáš: „Pane, nevíme, kam jdeš. Jak bychom mohli znát cestu?“
#14:6 Ježíš mu odpověděl: „Já jsem ta cesta, pravda i život. Nikdo nepřichází k Otci než skrze mne.
#14:7 Kdybyste znali mne, znali byste i mého Otce. Nyní ho již znáte, neboť jste ho viděli.“
#14:8 Filip mu řekl: „Pane, ukaž nám Otce, a víc nepotřebujeme!“
#14:9 Ježíš mu odpověděl: „Tak dlouho jsem s vámi, Filipe, a ty mě neznáš? Kdo vidí mne, vidí Otce. Jak tedy můžeš říkat: Ukaž nám Otce?
#14:10 Nevěříš, že já jsem v Otci a Otec je ve mně? Slova, která vám mluvím, nemluvím sám od sebe; Otec, který ve mně přebývá, činí své skutky.
#14:11 Věřte mi, že já jsem v Otci a Otec ve mně; ne-li, věřte aspoň pro ty skutky!
#14:12 Amen, amen, pravím vám: Kdo věří ve mne, i on bude činit skutky, které já činím, a ještě větší, neboť já jdu k Otci.
#14:13 A začkoli budete prosit ve jménu mém, učiním to, aby byl Otec oslaven v Synu.
#14:14 Budete-li mne o něco prosit ve jménu mém, já to učiním.
#14:15 Milujete-li mne, budete zachovávat má přikázání;
#14:16 a já požádám Otce a on vám dá jiného Přímluvce, aby byl s vámi navěky -
#14:17 Ducha pravdy, kterého svět nemůže přijmout, poněvadž ho nevidí ani nezná. Vy jej znáte, neboť s vámi zůstává a ve vás bude.
#14:18 Nezanechám vás osiřelé, přijdu k vám.
#14:19 Ještě malou chvíli a svět mne už neuzří, vy však mě uzříte, poněvadž já jsem živ a také vy budete živi.
#14:20 V onen den poznáte, že já jsem ve svém Otci, vy ve mně a já ve vás.
#14:21 Kdo přijal má přikázání a zachovává je, ten mě miluje. A toho, kdo mě miluje, bude milovat můj Otec; i já ho budu milovat a dám mu to poznat.“
#14:22 Řekl mu Juda, ne ten Iškariotský: „Pane, jak to, že se chceš dát poznat nám, ale ne světu?“
#14:23 Ježíš mu odpověděl: „Kdo mě miluje, bude zachovávat mé slovo, a můj Otec ho bude milovat; přijdeme k němu a učiníme si u něho příbytek.
#14:24 Kdo mě nemiluje, nezachovává má slova. A slovo, které slyšíte, není moje, ale mého Otce, který mě poslal.
#14:25 Toto vám pravím, dokud jsem s vámi.
#14:26 Ale Přímluvce, Duch svatý, kterého pošle Otec ve jménu mém, ten vás naučí všemu a připomene vám všecko, co jsem vám řekl.
#14:27 Pokoj vám zanechávám, svůj pokoj vám dávám; ne jako dává svět, já vám dávám. Ať se srdce vaše nechvěje a neděsí!
#14:28 Slyšeli jste, že jsem vám řekl: Odcházím - a přijdu k vám. Jestliže mě milujete, měli byste se radovat, že jdu k Otci; neboť Otec je větší než já.
#14:29 Řekl jsem vám to nyní předem, abyste potom, až se to stane, uvěřili.
#14:30 Již s vámi nebudu mnoho mluvit, neboť přichází vládce tohoto světa. Proti mně nic nezmůže.
#14:31 Ale svět má poznat, že miluji Otce a jednám, jak mi přikázal. - Vstaňte, pojďme odtud!“ 
#15:1 „Já jsem pravý vinný kmen a můj Otec je vinař.
#15:2 Každou mou ratolest, která nenese ovoce, odřezává, a každou, která nese ovoce, čistí, aby nesla hojnější ovoce.
#15:3 Vy jste již čisti pro slovo, které jsem k vám mluvil.
#15:4 Zůstaňte ve mně, a já ve vás. Jako ratolest nemůže nést ovoce sama od sebe, nezůstane-li při kmeni, tak ani vy, nezůstanete-li při mně.
#15:5 Já jsem vinný kmen, vy jste ratolesti. Kdo zůstává ve mně a já v něm, ten nese hojné ovoce; neboť beze mne nemůžete činit nic.
#15:6 Kdo nezůstane ve mně, bude vyvržen ven jako ratolest a uschne; pak ji seberou, hodí do ohně a spálí.
#15:7 Zůstanete-li ve mně a zůstanou-li má slova ve vás, proste, oč chcete, a stane se vám.
#15:8 Tím bude oslaven můj Otec, když ponesete hojné ovoce a budete mými učedníky.
#15:9 Jako si Otec zamiloval mne, tak jsem si já zamiloval vás. Zůstaňte v mé lásce.
#15:10 Zachováte-li má přikázání, zůstanete v mé lásce, jako já zachovávám přikázání svého Otce a zůstávám v jeho lásce.
#15:11 To jsem vám pověděl, aby moje radost byla ve vás a vaše radost aby byla plná.“
#15:12 „To je mé přikázání, abyste se milovali navzájem, jako jsem já miloval vás.
#15:13 Nikdo nemá větší lásku než ten, kdo položí život za své přátele.
#15:14 Vy jste moji přátelé, činíte-li, co vám přikazuji.
#15:15 Už vás nenazývám služebníky, protože služebník neví, co činí jeho pán. Nazval jsem vás přáteli, neboť jsem vám dal poznat všechno, co jsem slyšel od svého Otce.
#15:16 Ne vy jste vyvolili mne, ale já jsem vyvolil vás a ustanovil jsem vás, abyste šli a nesli ovoce a vaše ovoce aby zůstalo; a Otec vám dá, oč byste ho prosili v mém jménu.
#15:17 To vám přikazuji, abyste jeden druhého milovali.
#15:18 Nenávidí-li vás svět, vězte, že mě nenáviděl dříve než vás.
#15:19 Kdybyste náleželi světu, svět by miloval to, co je jeho. Protože však nejste ze světa, ale já jsem vás ze světa vyvolil, proto vás svět nenávidí.
#15:20 Vzpomeňte si na slovo, které jsem vám řekl: Sluha není nad svého pána. Jestliže pronásledovali mne, i vás budou pronásledovat - jestliže mé slovo zachovali, i vaše zachovají.
#15:21 Ale to vše vám učiní pro mé jméno, poněvadž neznají toho,který mě poslal.
#15:22 Kdybych byl nepřišel a nemluvil k nim, byli by bez hříchu. Nyní však nemají výmluvu pro svůj hřích.
#15:23 Kdo nenávidí mne, nenávidí i mého Otce.
#15:24 Kdybych byl mezi nimi nečinil skutky, jaké nikdo jiný nedokázal, byli by bez hříchu. Ale oni je viděli, a přece mají v nenávisti i mne i mého Otce.
#15:25 To proto, aby se naplnilo slovo napsané v jejich zákoně: ‚Nenáviděli mě bez příčiny‘.
#15:26 Až přijde Přímluvce, kterého vám pošlu od Otce, Duch pravdy, jenž od Otce vychází, ten o mně vydá svědectví.
#15:27 Také vy vydávejte svědectví, neboť jste se mnou od začátku. 
#16:1 To jsem vám pověděl, abyste se nedali svést.
#16:2 Budou vás vylučovat za synagóg; ano, přichází hodina, že ten, kdo vás zabije, bude se domnívat, že tím uctívá Boha.
#16:3 To s vámi budou činit, protože nepoznali Otce ani mne.
#16:4 Ale to jsem vám pověděl, abyste si vzpomněli na má slova, až přijde ta hodina.“ „Neřekl jsem vám to na začátku, poněvadž jsem byl s vámi.
#16:5 Nyní však odcházím k tomu, který mě poslal, a nikdo z vás se mě nezeptá: Kam jdeš?
#16:6 Ale že jsem s vámi tak mluvil, zármutek naplnil vaše srdce.
#16:7 Říkám vám však pravdu: Prospěje vám, abych odešel. Když neodejdu, Přímluvce k vám nepřijde. Odejdu-li, pošlu ho k vám.
#16:8 On přijde a ukáže světu, v čem je hřích, spravedlnost a soud:
#16:9 Hřích v tom, že ve mne nevěří;
#16:10 spravedlnost v tom, že odcházím k Otci a již mne nespatříte;
#16:11 soud v tom, že vládce tohoto světa je již odsouzen.
#16:12 Ještě mnoho jiného bych vám měl povědět, ale nyní byste to neunesli.
#16:13 Jakmile však přijde on, Duch pravdy, uvede vás do veškeré pravdy, neboť nebude mluvit sám ze sebe, ale bude mluvit, co uslyší. A oznámí vám, co má přijít.
#16:14 On mě oslaví, neboť vám bude zvěstovat, co přijme ode mne.
#16:15 Všecko, co má Otec, jest mé. Proto jsem řekl, že vám bude zvěstovat, co přijme ode mne.
#16:16 Zanedlouho mě již nespatříte a zanedlouho mě opět uzříte.“
#16:17 Někteří z jeho učedníků si mezi sebou řekli: „Co znamenají slova ‚zanedlouho mě nespatříte a zanedlouho mě opět uzříte‘ a ‚odcházím k Otci‘?“
#16:18 Říkali: „Co znamená ono: zanedlouho? Nevíme, o čem mluví.“
#16:19 Ježíš poznal, že se ho chtějí otázat, a řekl jim: „Dohadujete se mezi sebou o tom, že jsem řekl: Zanedlouho mě nespatříte a zanedlouho mě opět uzříte?
#16:20 Amen, amen, pravím vám, vy budete plakat a naříkat, ale svět se bude radovat; vy se budete rmoutit, ale váš zármutek se promění v radost.
#16:21 Žena, když rodí, má zármutek, neboť přišla její hodina; ale když porodí dítě, nevzpomíná už na soužení pro radost, že na svět přišel člověk.
#16:22 I vy máte nyní zármutek. Uvidím vás však opět a vaše srdce se zaraduje a vaši radost vám nikdo nevezme.
#16:23 V onen den se mě nebudete již na nic ptát. Amen, amen, pravím vám, budete-li o něco prosit Otce ve jménu mém, dá vám to.
#16:24 Až dosud jste o nic neprosili v mém jménu. Proste a dostanete, aby vaše radost byla plná.
#16:25 To vše jsem vám říkal v obrazech. Přichází hodina, kdy k vám už nebudu mluvit o Otci v obrazech, ale budu jej zvěstovat přímo.
#16:26 V onen den budete prosit v mém jménu a neříkám vám, že já budu prosit Otce za vás;
#16:27 vždyť Otec sám vás miluje, protože vy milujete mne a uvěřili jste, že jsem vyšel od Boha.
#16:28 Vyšel jsem od Otce a přišel jsem na svět. Teď svět opouštím a navracím se k Otci.“
#16:29 Jeho učedníci mu řekli: „Nyní mluvíš přímo a bez obrazů.
#16:30 Nyní víme, že víš všecko a že nepotřebuješ, aby ti někdo kladl otázky. Proto věříme, že jsi vyšel od Boha.“
#16:31 Ježíš jim odpověděl: „Teď věříte?
#16:32 Hle, přichází hodina, a již je zde, kdy se rozprchnete každý do svého domova a mne necháte samotného. Ale nejsem sám, neboť Otec je se mnou.
#16:33 To jsem vám pověděl, abyste nalezli ve mně pokoj. Ve světě máte soužení. Ale vzchopte se, já jsem přemohl svět. 
#17:1 Po těch slovech Ježíš pozvedl oči k nebi a řekl: „Otče, přišla má hodina. Oslav svého Syna, aby Syn oslavil tebe,
#17:2 stejně, jako jsi učinil, když jsi mu dal moc nad všemi lidmi, aby vše, co jsi mu svěřil, dal jim: život věčný.
#17:3 A život věčný je v tom, když poznají tebe, jediného pravého Boha, a toho, kterého jsi poslal, Ježíš Krista.
#17:4 Já jsem tě oslavil na zemi, když jsem dokonal dílo, které jsi mi svěřil.
#17:5 A nyní ty, Otče, oslav mne svou slávou, kterou jsem měl u tebe, dříve, než byl svět.
#17:6 Zjevil jsem tvé jméno lidem, které jsi mi ze světa dal. Byli tvoji a mně jsi je dal; a tvoje slovo zachovali.
#17:7 Nyní poznali, že všecko, co jsi mi dal, je od tebe;
#17:8 neboť slova, která jsi mi svěřil, dal jsem jim a oni je přijali. V pravdě poznali, že jsem od tebe vyšel, a uvěřili, že ty jsi mě poslal.
#17:9 Za ně prosím. Ne za svět prosím, ale za ty, které jsi mi dal, neboť jsou tvoji;
#17:10 a všecko mé je tvé, a co je tvé, je moje. V nich jsem oslaven.
#17:11 Již nejsem ve světě, ale oni jsou ve světě, a já jdu k tobě. Otče svatý, zachovej je ve svém jménu, které jsi mi dal; nechť jsou jedno jako my.
#17:12 Dokud jsem byl s nimi, zachovával jsem je v tvém jménu, které jsi i dal; ochránil jsem je, takže žádný z nich nezahynul, kromě toho, který byl zavržen, aby se naplnilo Písmo.
#17:13 Nyní jdu k tobě, ale toto mluvím ještě na světě, aby v sobě měli plnost mé radosti.
#17:14 Dal jsem jim tvé slovo, ale svět k nim pojal nenávist, poněvadž nejsou ze světa, jako ani já nejsem ze světa.
#17:15 Neprosím, abys je vzal ze světa, ale abys je zachoval od zlého.
#17:16 Nejsou ze světa, jako ani já nejsem ze světa.
#17:17 Posvěť je pravdou; tvoje slovo je pravda.
#17:18 Jako ty jsi mne poslal do světa, tak i já jsem je poslal do světa.
#17:19 Sám sebe za ně posvěcuji, aby i oni byli v pravdě posvěceni.
#17:20 Neprosím však jen za ně, ale i za ty, kteří skrze jejich slovo ve mne uvěří;
#17:21 aby všichni byli jedno jako ty, Otče, ve mně a já v tobě, aby i oni byli v nás, aby tak svět uvěřil, že jsi mě poslal.
#17:22 Slávu, kterou jsi mi dal, dal jsem jim, aby byli jedno, jako my jsme jedno -
#17:23 já v nich a ty ve mně; aby byli uvedeni v dokonalost jednoty a svět aby poznal, že ty jsi mě poslal a zamiloval sis je tak jako mne.
#17:24 Otče, chci, aby také ti, které jsi mi dal, byli se mnou tam, kde jsem já; ať hledí na mou slávu, kterou jsi mi dal, neboť jsi mě miloval již před založením světa.
#17:25 Spravedlivý Otče, svět tě nepoznal, ale já jsem tě poznal, a také oni poznali, že jsi mě poslal.
#17:26 Dal jsem jim poznat tvé jméno a ještě dám poznat, aby v nich byla láska, kterou máš ke mně, a já abych byl v nich.“ 
#18:1 Po těch slovech odešel Ježíš s učedníky za potok Cedron, kde byla zahrada; a do ní vstoupil on i jeho učedníci.
#18:2 Také Jidáš, který ho zradil, znal to místo, neboť Ježíš tam se svými učedníky často býval.
#18:3 Jidáš vzal s sebou oddíl vojáků, k tomu stráž od velekněží a farizeů, a přišli tam s pochodněmi, lucernami a zbraněmi.
#18:4 Ježíš, který věděl všecko, co ho má potkat, vyšel a řekl jim: „Koho hledáte?“
#18:5 Odpověděli mu: „Ježíše Nazaretského.“ Řekl jim: „To jsem já.“ Stál s nimi i Jidáš, který ho zradil.
#18:6 Jakmile jim řekl ‚to jsem já‘, couvli a padli na zem.
#18:7 Opět se jich otázal: „Koho hledáte?“ A oni řekli: „Ježíše Nazaretského.“
#18:8 Ježíš odpověděl: „Řekl jsem vám, že jsem to já. Hledáte-li mne, nechte ostatní odejít.“
#18:9 Tak se mělo naplnit slovo: ‚Z těch, které jsi mi dal, neztratil jsem ani jednoho.‘
#18:10 Šimon Petr vytasil meč, který měl u sebe, zasáhl jednoho veleknězova sluhu a uťal mu ucho. Ten sluha se jmenoval Malchos.
#18:11 Ježíš řekl Petrovi: „Schovej ten meč do pochvy! Což nemám pít kalich, který mi dal Otec?“
#18:12 Vojáci s důstojníkem a židovská stráž se zmocnili Ježíše a spoutali ho.
#18:13 Přivedli ho nejprve k Annášovi; byl totiž tchánem Kaifáše, který byl toho roku veleknězem.
#18:14 Byl to ten Kaifáš, který poradil židům, že je lépe, aby jeden člověk zemřel za lid.
#18:15 Za Ježíšem šel Šimon Petr a jiný učedník. Ten učedník byl znám veleknězi a vešel do nádvoří veleknězova domu.
#18:16 Petr zůstal venku před vraty. Ten druhý učedník, který byl znám veleknězi, vyšel, promluvil s vrátnou a zavedl Petra dovnitř.
#18:17 Tu řekla služka vrátná Petrovi: „Nepatříš i ty k učedníkům toho člověka?“ On řekl: „Nepatřím.“
#18:18 Poněvadž bylo chladno, udělali si sluhové a strážci oheň a stáli kolem něho, aby se ohřáli. I Petr stál s nimi a ohříval se.
#18:19 Velekněz se dotazoval Ježíše na jeho učedníky a na jeho učení.
#18:20 Ježíš mu řekl: „Já jsem mluvil k světu veřejně. Vždycky jsem učil v synagóze a v chrámě, kde se shromažďují všichni židé, a nic jsem neříkal tajně.
#18:21 Proč se mě ptáš? Zeptej se těch, kteří slyšeli, co jsem říkal. Ti přece vědí, co jsem řekl.“
#18:22 Po těch slovech jeden ze strážců, který stál poblíž, udeřil Ježíše do obličeje a řekl: „Tak se odpovídá veleknězi?“
#18:23 Ježíš mu odpověděl: „Řekl-li jsem něco špatného, prokaž, že je to špatné. Jestliže to bylo správné, proč mě biješ?“
#18:24 Annáš jej tedy poslal spoutaného veleknězi Kaifášovi.
#18:25 Šimon Petr stál ještě u ohně a zahříval se. Řekli mu: „Nejsi i ty z jeho učedníků?“ On to zapřel: „Nejsem.“
#18:26 Jeden z veleknězových sluhů, příbuzný toho, jemuž Petr uťal ucho, řekl: „Cožpak jsem tě s ním neviděl v zahradě?“
#18:27 Petr opět zapřel a vtom zakokrhal kohout.
#18:28 Od Kaifáše vedli Ježíše do místodržitelského paláce. Bylo časně zrána. Židé sami do paláce nevešli, aby se neposkvrnili a mohli jíst velikonočního beránka.
#18:29 Pilát k nim vyšel a řekl: „Jakou obžalobu vznášíte proti tomu člověku?“
#18:30 Odpověděli: „Kdyby nebyl zločinec, nebyli bychom ti ho vydali.“
#18:31 Pilát jim řekl: „Vezměte si ho vy a suďte podle svého zákona!“ Židé mu odpověděli: „Není nám dovoleno nikoho popravit.“
#18:32 To aby se naplnilo slovo Ježíšovo, kterým naznačil, jakou smrtí má zemřít.
#18:33 Pilát vešel opět do svého paláce, zavolal Ježíše a řekl mu: „Ty jsi král židovský?“
#18:34 Ježíš odpověděl: „Říkáš to sám od sebe, nebo ti to o mně řekli jiní?“
#18:35 Pilát odpověděl: „Jsem snad žid? Tvůj národ a velekněží mi tě vydali. Čím ses provinil?“
#18:36 Ježíš řekl: „Moje království není z tohoto světa. Kdyby mé království bylo z tohoto světa, moji služebníci by bojovali, abych nebyl vydán Židům; mé království však není odtud.“
#18:37 Pilát mu řekl: „Jsi tedy přece král?“ Ježíš odpověděl: „Ty sám říkáš, že jsem král. Já jsem se proto narodil a proto jsem přišel na svět, abych vydal svědectví pravdě. Každý, kdo je z pravdy, slyší můj hlas.“
#18:38 Pilát mu řekl: „Co je pravda?“ Po těch slovech vyšel opět k Židům a řekl jim: „Já na něm žádnou vinu nenalézám.
#18:39 Je zvykem, že vám o velikonocích propouštím na svobodu jednoho vězně. Chcete-li, propustím vám tohoto židovského krále.“
#18:40 Na to se dali do křiku: „Toho ne, ale Barabáše!“ Ten Barabáš byl vzbouřenec. 
#19:1 Tehdy dal Pilát Ježíše zbičovat.
#19:2 Vojáci upletli z trní korunu, vložili mu ji na hlavu a přes ramena mu přehodili purpurový plášť.
#19:3 Pak před něho předstupovali a říkali: „Buď pozdraven, králi židovský!“ Přitom ho bili do obličeje.
#19:4 Pilát znovu vyšel a řekl Židům: „Hleďte, vedu vám ho ven, abyste věděli, že na něm nenalézám žádnou vinu.“
#19:5 Ježíš vyšel ven s trnovou korunou na hlavě a v purpurovém plášti. Pilát jim řekl: „Hle, člověk!“
#19:6 Když ho velekněží a jejich služebníci uviděli, dali se do křiku: „Ukřižovat, ukřižovat!“ Pilát jim řekl: „Vy sami si ho křižujete! Já na něm vinu nenalézám.“
#19:7 Židé mu odpověděli: „My máme zákon, a podle toho zákona musí zemřít, protože se vydával za Syna Božího.“
#19:8 Když to Pilát uslyšel, ještě více se ulekl.
#19:9 Vešel znovu do paláce a řekl Ježíšovi: „Odkud jsi?“ Ale Ježíš mu nedal žádnou odpověď.
#19:10 Pilát řekl: „Nemluvíš se mnou? Nevíš, že mám moc tě propustit, a mám moc tě ukřižovat?“
#19:11 Ježíš odpověděl: „Neměl bys nade mnou žádnou moc, kdyby ti nebyla dána shůry. Proto ten, kdo mě tobě vydal, má větší vinu.“
#19:12 Od té chvíle ho Pilát usiloval propustit. Ale Židé křičeli: „Jestliže ho propustíš, jsi nepřítel císařův. Každý, kdo se vydává za krále, je proti císaři.“
#19:13 Když to Pilát uslyšel, dal vyvést Ježíše ven a zasedl k soudu na místě zvaném ‚Na dláždění‘, hebrejsky Gabbatha.
#19:14 Byl den přípravy před svátky velikonočními, kolem poledne. Pilát Židům řekl: „Hle, váš král!“
#19:15 Oni se dali do křiku: „Pryč s ním, pryč s ním, ukřižuj ho!“ Pilát jim řekl: „Vašeho krále mám dát ukřižovat?“ Velekněží odpověděli: „Nemáme krále, jen císaře.“
#19:16 Tu jim ho vydal, aby byl ukřižován, a oni se Ježíše chopili.
#19:17 Nesl svůj kříž a vyšel z města na místo zvané Lebka, hebrejsky Golgota.
#19:18 Tam ho ukřižovali a s ním jiné dva, z každé strany jednoho a Ježíše uprostřed.
#19:19 Pilát dal napsat nápis a připevnit jej na kříž. Stálo tam: Ježíš Nazaretský, král židovský.
#19:20 Ten nápis četlo mnoho Židů, neboť místo, kde byl ukřižován, bylo blízko města; byl napsán hebrejsky, latinsky a řecky.
#19:21 Židovští velekněží řekli Pilátovi: „Neměls psát ‚židovský král‘, nýbrž ‚vydával se za židovského krále‘.“
#19:22 Pilát odpověděl: „Co jsem napsal, napsal jsem.“
#19:23 Když vojáci Ježíše ukřižovali, vzali jeho šaty a rozdělili je na čtyři díly, každému vojákovi díl; zbýval ještě spodní šat. Ten byl beze švů, od shora vcelku utkaný.
#19:24 Řekli mezi sebou: „Netrhejme jej, ale losujme, čí bude.!“ To proto, aby se naplnilo Písmo: ‚Rozdělili si mé šaty a o oděv můj metali los.‘ To tedy vojáci provedli.
#19:25 U Ježíšova kříže stály jeho matka a sestra jeho matky, Marie Kleofášova a Marie Magdalská.
#19:26 Když Ježíš spatřil matku a vedle ní učedníka, kterého miloval, řekl matce: „Ženo, hle, tvůj syn!“
#19:27 Potom řekl tomu učedníkovi: „Hle, tvá matka!“ V tu hodinu ji onen učedník přijal k sobě.
#19:28 Ježíš věděl, že vše je již dokonáno; a proto, aby se až do konce naplnilo Písmo, řekl: „Žízním.“
#19:29 Stála tam nádoba plná octa; namočili tedy houbu do octa a na yzopu mu ji podali k ústům.
#19:30 Když Ježíš okusil octa, řekl: „Dokonáno jest.“ A nakloniv hlavu, skonal.
#19:31 Poněvadž byl den přípravy a těla nesměla zůstat přes sobotu na kříži - na tu sobotu totiž připadal veliký svátek - požádali Židé Piláta, aby odsouzeným byly zlámány kosti a aby byli sňati s kříže.
#19:32 Přišli tedy vojáci a zlámali kosti prvnímu i druhému, kteří byli ukřižováni s ním.
#19:33 Když přišli k Ježíšovi a viděli, že je již mrtev, kosti mu nelámali,
#19:34 ale jeden z vojáků mu probodl kopím bok; a ihned vyšla krev a voda.
#19:35 A ten, který to viděl, vydal o tom svědectví, a jeho svědectví je pravdivé; on ví, že mluví pravdu, abyste i vy uvěřili.
#19:36 Neboť se to stalo, aby se naplnilo Písmo: ‚Ani kost mu nebude zlomena‘.
#19:37 A na jiném místě Písmo praví: ‚Uvidí, koho probodli.‘
#19:38 Potom požádal Piláta Josef z Arimatie - byl to Ježíšův učedník, ale tajný, protože se bál Židů - aby směl Ježíšovo tělo sejmout z kříže. Když Pilát k tomu dal souhlas, Josef šel a tělo sňal.
#19:39 Přišel také Nikodém, který kdysi navštívil Ježíše v noci, a přinesl sto liber směsi myrhy a aloe.
#19:40 Vzali Ježíšovo tělo a zabalili je s vonnými látkami do lněných pláten, jak je to u Židů při pohřbu zvykem.
#19:41 V těch místech, kde byl Ježíš ukřižován, byla zahrada a v ní nový hrob, v němž dosud nikdo nebyl pochován.
#19:42 Tam položili Ježíše, protože byl den přípravy a hrob byl blízko. 
#20:1 První den po sobotě, když ještě byla tma, šla Marie Magdalská k hrobu a spatřila, že kámen je od hrobu odvalen.
#20:2 Běžela k Šimonu Petrovi a k tomu učedníkovi, kterého Ježíš miloval, a řekla jim: „Vzali Pána z hrobu, a nevíme, kam ho položili.“
#20:3 Petr a ten druhý učedník vstali a šli k hrobu.
#20:4 Oba dva běželi, ale ten druhý učedník předběhl Petra a byl u hrobu první.
#20:5 Sehnul se a viděl tam ležet lněná plátna, ale dovnitř nevešel.
#20:6 Po něm přišel Šimon Petr a vešel do hrobu. Uviděl tam ležet lněná plátna,
#20:7 ale šátek, jímž ovázali Ježíšovu hlavu, neležel mezi plátny, nýbrž byl svinut na jiném místě.
#20:8 Potom vešel dovnitř i ten druhý učedník, který přišel k hrobu dřív; spatřil vše a uvěřil.
#20:9 Dosud totiž nevěděli, že podle Písma musí vstát z mrtvých.
#20:10 Oba učedníci se pak vrátili domů.
#20:11 Ale Marie stála venku před hrobem a plakala. Přitom se naklonila do hrobu
#20:12 a spatřila dva anděly v bílém rouchu, sedící na místě, kde před tím leželo Ježíšovo tělo, jednoho u hlavy a druhého u nohou.
#20:13 Otázali se Marie: „Proč pláčeš?“ Odpověděla jim: „Odnesli mého Pána a nevím, kam ho položili.“
#20:14 Po těch slovech se obrátila a spatřila za sebou Ježíše; ale nepoznala, že je to on.
#20:15 Ježíš jí řekl: „Proč pláčeš? Koho hledáš?“ V domnění, že je to zahradník, mu odpověděla: „Jestliže tys jej, pane, odnesl, řekni mi, kam jsi ho položil, a já pro něj půjdu.“
#20:16 Ježíš jí řekl: „Marie!“ Obrátila se a zvolala hebrejsky: „Rabbuni“, to znamená ‚Mistře‘.
#20:17 Ježíš jí řekl: „Nedotýkej se mne, dosud jsem nevystoupil k Otci. Ale jdi k mým bratřím a pověz jim, že vystupuji k Otci svému i Otci vašemu a k Bohu svému i Bohu vašemu.“
#20:18 Marie Magdalská šla k učedníkům a oznámila jim: „Viděla jsem Pána a toto mi řekl.“
#20:19 Téhož dne večer - prvního dne po sobotě - když byli učedníci ze strachu před Židy shromážděni za zavřenými dveřmi, přišel Ježíš a postavil se uprostřed nich a řekl: „Pokoj vám.“
#20:20 Když to řekl, ukázal jim ruce a bok. Učedníci se zaradovali, když spatřili Pána.
#20:21 Ježíš jim znovu řekl: „Pokoj vám. Jako mne poslal Otec, tak já posílám vás.“
#20:22 Po těch slovech na ně dechl a řekl jim: „Přijměte Ducha svatého.
#20:23 Komu odpustíte hříchy, tomu jsou odpuštěny, a komu je neodpustíte, tomu odpuštěny nejsou.“
#20:24 Tomáš, jinak Didymos, jeden z dvanácti učedníků, nebyl s nimi, když Ježíš přišel.
#20:25 Ostatní mu řekli: „Viděli jsme Pána.“ Odpověděl jim: „Dokud neuvidím na jeho rukou stopy po hřebech a dokud nevložím do nich svůj prst a svou ruku do rány v jeho boku, neuvěřím.“
#20:26 Osmého dne potom byli učedníci opět uvnitř a Tomáš s nimi. Ač byly dveře zavřeny, Ježíš přišel, postavil se a řekl: „Pokoj vám.“
#20:27 Potom řekl Tomášovi: „Polož svůj prst sem, pohleď na mé ruce a vlož svou ruku do rány v mém boku. Nepochybuj a věř!“
#20:28 Tomáš mu odpověděl: „Můj Pán a můj Bůh.“
#20:29 Ježíš mu řekl: „Že jsi mě viděl, věříš. Blahoslavení, kteří neviděli, a uvěřili.“
#20:30 Ještě mnoho jiných znamení učinil Ježíš před očima učedníků, a ta nejsou zapsána v této knize.
#20:31 Tato však zapsána jsou, abyste věřili, že Ježíš je Kristus, Syn Boží, a abyste věříce měli život v jeho jménu. 
#21:1 Potom se Ježíš opět zjevil učedníků u jezera Tiberiadského.
#21:2 Stalo se to takto: Byli spolu Šimon Petr, Tomáš, jinak Didymos, Natanael z Kány Galilejské, synové Zebedeovi a ještě dva z jeho učedníků.
#21:3 Šimon Petr jim řekl: „Jdu lovit ryby.“ Odpověděli mu: „I my půjdeme s tebou.“ Šli a vstoupili na loď. Té noci však nic neulovili.
#21:4 Když začalo svítat, stál Ježíš na břehu, ale učedníci nevěděli, že je to on.
#21:5 Ježíš jim řekl: „Nemáte něco k jídlu?“ Odpověděli: „Nemáme.“
#21:6 Řekl jim: „Hoďte síť na pravou stranu lodi, tam ryby najdete.“ Hodili síť a nemohli ji ani utáhnout pro množství ryb.
#21:7 Onen učedník, kterého Ježíš miloval, řekl Petrovi: „To je Pán!“ Jakmile Šimon Petr uslyšel, že je to Pán, přehodil si plášť - byl totiž svlečen - a brodil se k němu vodou.
#21:8 Ostatní učedníci přijeli na lodi - nebyli daleko od břehu, jen asi dvě stě loket - a táhli za sebou síť s rybami.
#21:9 Když vstoupili na břeh, spatřili ohniště a na něm rybu a chléb.
#21:10 Ježíš jim řekl: „Přineste několik ryb z toho, co jste nalovili!“
#21:11 Šimon Petr šel a vytáhl na břeh síť plnou velkých ryb, bylo jich sto padesát tři; a ač jich bylo tolik, síť se neprotrhla.
#21:12 Ježíš jim řekl: „Pojďte jíst!“ A nikdo z učedníků se ho neodvážil zeptat: „Kdo jsi?“ Věděli, že je to Pán.
#21:13 Ježíš šel, vzal chléb a dával jim; stejně i rybu.
#21:14 To se již potřetí zjevil učedníkům po svém vzkříšení.
#21:15 Když pak pojedli, zeptal se Ježíš Šimona Petra: „Šimone, synu Janův, miluješ mne víc než ti zde?“ Odpověděl mu: „Ano, Pane, ty víš, že tě mám rád.“ Řekl mu: „Pas mé beránky.“
#21:16 Zeptal se ho podruhé: „Šimone, synu Janův, miluješ mne?“ Odpověděl: „Ano, Pane, ty víš, že tě mám rád.“ Řekl mu: „Buď pastýřem mých ovcí!“
#21:17 Zeptal se ho potřetí: „Šimone, synu Janův, máš mne rád?“ Petr se zarmoutil nad tím, že se ho potřetí zeptal, má-li ho rád. Odpověděl mu: „Pane, ty víš všecko, ty víš také, že tě mám rád.“ Ježíš mu řekl: „Pas mé ovce!
#21:18 Amen, amen, pravím tobě, když jsi byl mladší, sám ses přepásával a chodil jsi, kam jsi chtěl; ale až zestárneš, vztáhneš ruce a jiný tě přepáše a povede, kam nechceš.“
#21:19 To řekl, aby mu naznačil, jakou smrtí oslaví Boha. A po těch slovech dodal: „Následuj mne!“
#21:20 Petr se obrátil a spatřil, že za nimi jde učedník, kterého Ježíš miloval, ten, který byl při večeři po jeho boku a který se ho tehdy otázal: „Pane, kdo tě zrazuje?“
#21:21 Když ho Petr spatřil, řekl Ježíšovi: „Pane, co bude s ním?“
#21:22 Ježíš mu řekl: „Jestliže chci, aby tu zůstal, dokud nepřijde, není to tvá věc. Ty mne následuj!“
#21:23 Mezi bratřími se to slovo rozšířilo a říkalo se, že onen učedník nezemře. Ježíš však neřekl, že nezemře, nýbrž: „Jestliže chci, aby tu zůstal, dokud nepřijdu, není to tvá věc.“
#21:24 To je ten učedník, který vydává svědectví o těchto věcech a který je zapsal; a my víme, že jeho svědectví je pravdivé.
#21:25 Je ještě mnoho jiného, co Ježíš učinil; kdyby se mělo všechno dopodrobna vypsat, myslí, že by celý svět neměl dost místa pro knihy o tom napsané.  

\book{Acts}{Acts}
#1:1 První knihu, Theofile, jsem napsal o všem, co Ježíš činil a učil od samého počátku
#1:2 až do dne, kdy v Duchu svatém přikázal svým vyvoleným apoštolům, jak si mají počínat, a byl přijat k Bohu;
#1:3 jim také po svém umučení mnoha způsoby prokázal, že žije, po čtyřicet dní se jim dával spatřit a učil je o království Božím.
#1:4 Když s nimi byl u stolu, nařídil jim, aby neodcházeli z Jeruzaléma: „Čekejte, až se splní Otcovo zaslíbení, o němž jste ode mne slyšeli.
#1:5 Jan křtil vodou, vy však budete pokřtěni Duchem svatým, až uplyne těchto několik dní.“
#1:6 Ti, kteří byli s ním, se ho ptali: „Pane, už v tomto čase chceš obnovit království pro Izrael?“
#1:7 Řekl jim: „Není vaše věc znát čas a lhůtu, kterou si Otec ponechal ve své moci;
#1:8 ale dostanete sílu Ducha svatého, který na vás sestoupí, a budete mi svědky v Jeruzalémě a v celém Judsku, Samařsku a až na sám konec země.“
#1:9 Po těch slovech byl před jejich zraky vzat vzhůru a oblak jim ho zastřel.
#1:10 A když upřeně hleděli k nebi za ním, jak odchází, hle, stáli vedle nich dva muži v bílém rouchu
#1:11 a řekli: „Muži z Galileje, co tu stojíte a hledíte k nebi? Tento Ježíš, který byl od vás vzat do nebe, znovu přijde právě tak, jak jste ho viděli odcházet.“
#1:12 Potom se z hory, které se říká Olivová, vrátili do Jeruzaléma; není to daleko, jen asi kolik je dovoleno ujít v sobotu.
#1:13 Když přišli do města, vystoupili do horní místnosti domu, kde pobývali. Byli to Petr, Jan, Jakub, Ondřej, Filip a Tomáš, Bartoloměj a Matouš, Jakub Alfeův, Šimon Zélóta a Juda Jakubův.
#1:14 Ti všichni se svorně a vytrvale modlili spolu se ženami, s Marií, matkou Ježíšovou, a s jeho bratry.
#1:15 V těch dnech vstal Petr ve shromáždění bratří - bylo tam pohromadě asi sto dvacet lidí - a řekl:
#1:16 „Bratří, muselo se splnit slovo Písma, kde Duch svatý už ústy Davidovými mluvil o Jidášovi, o tom, který na Ježíše přivedl stráže,
#1:17 ačkoliv patřil do počtu nás Dvanácti a byl vyvolen ke stejné službě.
#1:18 Z odměny za svůj zlý čin si koupil pole, ale pak se střemhlav zřítil, jeho tělo se roztrhlo a všechny vnitřnosti vyhřezly.
#1:19 Všichni obyvatelé Jeruzaléma se o tom dověděli a začali tomu poli říkat ve svém jazyce Hakeldama, to znamená Krvavé pole.
#1:20 Neboť je psáno v knize Žalmů: ‚Jeho obydlí ať zpustne, ať není nikoho, kdo by v něm bydlil‘; a jinde je psáno: ‚Jeho pověření ať převezme jiný.‘
#1:21 Proto jeden z těch mužů, kteří s námi chodili po celý čas, kdy Pán Ježíš byl mezi námi,
#1:22 od křtu Janova až do dne, kdy byl od nás vzat, musí se spolu s námi stát svědkem jeho zmrtvýchvstání.“
#1:23 Vybrali tedy dva, Josefa jménem Barsabas, zvaného Justus, a Matěje;
#1:24 pak se modlili: „Ty, Pane, znáš srdce všech lidí; ukaž, koho z těch dvou sis vyvolil,
#1:25 aby převzal místo v této apoštolské službě, kterou Jidáš opustil a odešel tam, kam patří.“
#1:26 Potom jim dali losy a los padl na Matěje; tak byl připojen k jedenácti apoštolům. 
#2:1 Když nastal den letnic, byli všichni shromážděni na jednom místě.
#2:2 Náhle se strhl hukot z nebe, jako když se žene prudký vichr, a naplnil celý dům, kde byli.
#2:3 A ukázaly se jim jakoby ohnivé jazyky, rozdělily se a na každém z nich spočinul jeden;
#2:4 všichni byli naplněni Duchem svatým a začali ve vytržení mluvit jinými jazyky, jak jim Duch dával promlouvat.
#2:5 V Jeruzalémě byli zbožní židé ze všech národů na světě,
#2:6 a když se ozval ten zvuk, sešlo se jich mnoho a užasli, protože každý z nich je slyšel mluvit svou vlastní řečí.
#2:7 Byli ohromeni a divili se: „Což nejsou všichni, kteří tu mluví, z Galileje?
#2:8 Jak to, že je slyšíme každý ve své rodné řeči:
#2:9 Parthové, Médové a Elamité, obyvatelé Mezopotámie, Judeje a Kappadokie, Pontu a Asie,
#2:10 Frygie a Pamfylie, Egypta a krajů Libye u Kyrény a přistěhovalí Římané,
#2:11 židé i obrácení pohané, Kréťané i Arabové; všichni je slyšíme mluvit v našich jazycích o velikých skutcích Božích!“
#2:12 Žasli a v rozpacích říkali jeden druhému: „Co to má znamenat?“
#2:13 Ale jiní říkali s posměškem: „Jsou opilí!“
#2:14 Tu vystoupil Petr spolu s jedenácti, pozvedl hlas a oslovil je: „Muži judští a všichni, kdo bydlíte v Jeruzalémě, toto vám chci oznámit, poslouchejte mě pozorně:
#2:15 Tito lidé nejsou, jak se domníváte, opilí - vždyť je teprve devět hodin ráno.
#2:16 Ale děje se, co bylo řečeno ústy proroka Jóele:
#2:17 ‚A stane se v posledních dnech, praví Bůh, sešlu svého Ducha na všechny lidi, synové vaši a vaše dcery budou mluvit v prorockém vytržení, vaši mládenci budou mít vidění a vaši starci budou mít sny.
#2:18 I na své služebníky a na své služebnice v oněch dnech sešlu svého Ducha, a budou prorokovat.
#2:19 A učiním divy nahoře na nebi a znamení dole na zemi: krev a oheň a oblaka dýmu.
#2:20 Slunce se obrátí v temnotu a měsíc se změní v krev, než přijde den Páně, velký a slavný;
#2:21 a každý, kdo vzývá jméno Páně, bude zachráněn.‘
#2:22 Muži izraelští, slyšte tato slova: Ježíše Nazaretského Bůh potvrdil před vašimi zraky mocnými činy, divy a znameními, která mezi vámi skrze něho činil, jak sami víte.
#2:23 Bůh předem rozhodl, aby byl vydán, a vy jste ho rukou bezbožných přibili na kříž a zabili.
#2:24 Ale Bůh ho vzkřísil; vytrhl jej z bolestí smrti, a smrt ho nemohla udržet ve své moci.
#2:25 David o něm praví: ‚Viděl jsem Pána stále před sebou, je mi po pravici, abych nezakolísal;
#2:26 proto se mé srdce zaradovalo a jazyk můj se rozjásal, nadto i tělo mé odpočine v naději,
#2:27 neboť mě nezanecháš v říši smrti a nedopustíš, aby se tvůj Svatý rozpadl v prach.
#2:28 Dal jsi mi poznat cesty života a blízkost tvé tváře mne naplní radostí.‘
#2:29 Bratří, o praotci Davidovi vám mohu směle říci, že zemřel a byl pohřben; jeho hrob tu máme až dodnes.
#2:30 Byl to však prorok a věděl o přísaze, kterou se mu Bůh zavázal, že jeho potomka nastolí na jeho trůn;
#2:31 viděl do budoucnosti a mluvil tedy o vzkříšení Kristově, když řekl, že nezůstane v říši smrti a jeho tělo se nerozpadne v prach.
#2:32 Tohoto Ježíše Bůh vzkřísil a my všichni to můžeme dosvědčit.
#2:33 Byl vyvýšen na pravici Boží a přijal Ducha svatého, kterého Otec slíbil; nyní jej seslal na nás, jak to vidíte a slyšíte.
#2:34 David nevstoupil na nebe, ale sám říká: ‚Řekl Hospodin mému Pánu: Usedni po mé pravici,
#2:35 dokud ti nepoložím nepřátele pod nohy.‘
#2:36 Ať tedy všechen Izrael s jistotou ví, že toho Ježíše, kterého vy jste ukřižovali, učinil Bůh Pánem a Mesiášem.“
#2:37 Když to slyšeli, byli zasaženi v srdci a řekli Petrovi i ostatním apoštolům: „Co máme dělat, bratří?“
#2:38 Petr jim odpověděl: „Obraťte se a každý z vás ať přijme křest ve jménu Ježíše Krista na odpuštění svých hříchů, a dostanete dar Ducha svatého.
#2:39 Neboť to zaslíbení platí vám a vašim dětem i všem daleko široko, které si povolá Pán, náš Bůh.“
#2:40 A ještě mnoha jinými slovy je Petr zapřísahal a napomínal: „Zachraňte se z tohoto zvráceného pokolení!“
#2:41 Ti, kteří přijali jeho slovo, byli pokřtěni a přidalo se k nim toho dne na tři tisíce lidí.
#2:42 Vytrvale poslouchali učení apoštolů, byli spolu, lámali chléb a modlili se.
#2:43 Všech se zmocnila bázeň, neboť skrze apoštoly se stalo mnoho zázraků a znamení.
#2:44 Všichni, kteří uvěřili, byli pospolu a měli všechno společné.
#2:45 Prodávali svůj majetek a rozdělovali všem podle toho, jak kdo potřeboval.
#2:46 Každého dne pobývali svorně v chrámu, po domech lámali chléb a dělili se o jídlo s radostí a s upřímným srdcem.
#2:47 Chválili Boha a byli všemu lidu milí. A Pán denně přidával k jejich společenství ty, které povolával ke spáse. 
#3:1 Petr a Jan šli o třetí hodině do chrámu k odpolední modlitbě.
#3:2 Právě tam přinášeli nějakého člověka, chromého od narození; každý den ho posadili u chrámové brány, které se říká Krásná, aby prosil o almužnu ty, kdo tam vcházeli.
#3:3 Když viděl přicházet do chrámu Petra a Jana, prosil také je o almužnu.
#3:4 Petr spolu s Janem na něj upřeli zrak a řekli: „Pohleď na nás!“
#3:5 Obrátil se k nim a čekal, že od nich něco dostane.
#3:6 Petr však řekl: „Stříbro ani zlato nemám, ale co mám, to ti dám: ve jménu Ježíše Krista Nazaretského vstaň a choď!“
#3:7 Vzal ho za pravou ruku a pomáhal mu vstát; a vtom se chromému zpevnily klouby,
#3:8 vyskočil na nohy, vzpřímil se a začal chodit. Vešel s nimi do chrámu, chodil, skákal radostí a chválil Boha.
#3:9 A všichni ho viděli, jak chodí a chválí Boha.
#3:10 Když poznali, že je to ten, co sedal a žebral před chrámem u Krásné brány, žasli a byli u vytržení nad tím, co se stalo.
#3:11 Protože se držel Petra a Jana, všichni se k nim v úžasu sběhli do sloupoví, kterému se říká Šalomounovo.
#3:12 Když to Petr viděl, promluvil k lidu: „Muži izraelští, proč nad tím žasnete a proč hledíte na nás, jako bychom svou vlastní mocí nebo zbožností způsobili, že tento člověk chodí?
#3:13 Bůh Abrahamův, Izákův a Jákobův, Bůh našich otců oslavil svého služebníka Ježíše, kterého vy jste vydali a kterého jste se před Pilátem zřekli, když ho chtěl osvobodit.
#3:14 Svatého a spravedlivého jste se zřekli a vyprosili jste si propuštění vraha.
#3:15 Původce života jste zabili, Bůh ho však vzkřísil z mrtvých a my jsme toho svědky.
#3:16 A protože tento člověk, kterého tu vidíte a poznáváte, uvěřil v jeho jméno, moc Ježíšova mu dala sílu a zdraví - a víra, kterou jméno Ježíšovo v něm vzbudilo, úplně ho uzdravila před vašima očima.
#3:17 Vím ovšem, bratří, že jste jednali z nevědomosti, stejně jako vaši vůdcové.
#3:18 Bůh však tímto způsobem vyplnil, co předem ohlásil ústy všech proroků, že jeho Mesiáš bude trpět.
#3:19 Proto čiňte pokání a obraťte se, aby byly smazány vaše hříchy
#3:20 a přišel čas Hospodinův, čas odpočinutí, kdy pošle Ježíše, Mesiáše, kterého vám určil.
#3:21 On zůstane v nebi až do chvíle, kdy bude všechno nové, jak o tom Bůh od věků mluvil ústy svých svatých proroků.
#3:22 Mojžíš řekl: ‚Hospodin, náš Bůh, vám povolá proroka z vašich bratří, jako jsem já; toho budete poslouchat ve všem, co vám řekne.‘
#3:23 A ‚každý, kdo toho proroka neuposlechne, bude vyhlazen z mého lidu.‘
#3:24 Také všichni ostatní proroci, kolik jich jen od Samuele bylo, přinášeli zvěst právě o těchto dnech.
#3:25 Vás se týkají zaslíbení proroků i smlouva, kterou Bůh uzavřel s vašimi otci, když řekl Abrahamovi: ‚V tvém potomstvu budou požehnány všechny národy na zemi.‘
#3:26 Především pro vás povolal svého služebníka a poslal ho k vám, aby vám přinesl požehnání a odvrátil každého od jeho hříchů.“ 
#4:1 Když Petr a Jan ještě mluvili k lidu, přišli na ně kněží s velitelem chrámové stráže a saduceji,
#4:2 rozhořčeni, že učí lid a hlásají, že v Ježíši je vzkříšení z mrtvých.
#4:3 Násilím se jich chopili a vsadili je na noc do vězení, neboť už byl večer.
#4:4 Ale mnozí z těch, kteří slyšeli Boží slovo, uvěřili, takže jich bylo již na pět tisíc.
#4:5 Druhý den se shromáždili jeruzalémští představitelé židů, starší a znalci zákona,
#4:6 velekněz Annáš, Kaifáš, Jan a Alexandr a ostatní z velekněžského rodu,
#4:7 dali předvést Petra a Jana a začali je vyslýchat: „Jakou mocí a v jakém jménu jste to učinili?“
#4:8 Tu Petr, naplněn Duchem svatým, k nim promluvil: „Vůdcové lidu a starší,
#4:9 když nás dnes vyšetřujete pro dobrodiní, které jsme prokázali nemocnému člověku, a ptáte se, kdo ho uzdravil,
#4:10 vězte vy všichni i celý izraelský národ: Stalo se to ve jménu Ježíše Krista Nazaretského, kterého vy jste ukřižovali, ale Bůh ho vzkřísil z mrtvých. Mocí jeho jména stojí tento člověk před vámi zdráv.
#4:11 Ježíš je ten kámen, který jste vy stavitelé odmítli, ale on se stal kamenem úhelným.
#4:12 V nikom jiném není spásy; není pod nebem jiného jména, zjeveného lidem, jímž bychom mohli být spaseni.“
#4:13 Když viděli odvahu Petrovu i Janovu a shledali, že jsou to lidé neučení a prostí, žasli; poznávali, že jsou to ti, kteří bývali s Ježíšem.
#4:14 A když viděli, že ten uzdravený člověk tam stojí s nimi, neměli, co by na to řekli.
#4:15 Poručili jim, aby opustili zasedání; pak se mezi sebou radili:
#4:16 „Co s těmi lidmi uděláme? Bůh skrze ně způsobil zřejmý zázrak. Všichni, kdo bydlí v Jeruzalémě, to vědí, a my to nemůžeme popřít.
#4:17 Aby se to však příliš nerozneslo v lidu, pohrozíme jim, že už Ježíše nikomu nesmějí zvěstovat.“
#4:18 Zavolali je tedy a přikázali jim, aby jméno Ježíšovo vůbec nerozhlašovali a o něm neučili.
#4:19 Ale Petr a Jan jim odpověděli: „Posuďte sami, zda je před Bohem správné, abychom poslouchali vás, a ne jeho.
#4:20 Neboť o tom, co jsme viděli a slyšeli, nemůžeme mlčet.“
#4:21 A tak jim pohrozili a propustili je, protože nenašli nic, zač by je mohli potrestat; také měli obavy z lidu, neboť všichni chválili Boha za to, co se stalo.
#4:22 Tomu chromému, který byl zázračně uzdraven, bylo totiž už přes čtyřicet let.
#4:23 Když byli Petr a Jan propuštěni, vrátili se mezi své a oznámili, co jim řekli velekněží a starší.
#4:24 Když to bratří uslyšeli, pozdvihli jednomyslně hlas k Bohu a řekli: „Pane, který jsi učinil nebe i zemi i moře a všecko, co je v nich,
#4:25 ty jsi skrze Ducha svatého ústy našeho otce Davida, svého služebníka, řekl: ‚Proč zuří pohané hněvem a národy osnují marná spiknutí?
#4:26 Povstávají králové země a vladaři se srocují proti Hospodinu a jeho Mesiáši.‘
#4:27 Opravdu se srotili v tomto městě Herodes a Pontius Pilát spolu s pohany i s národem izraelským proti tvému svatému služebníku Ježíšovi, kterého jsi posvětil,
#4:28 a vykonali, co tvá ruka a tvá vůle předem určila.
#4:29 Pohleď tedy, Pane, na jejich hrozby a dej svým služebníkům, aby s odvahou a odhodlaně mluvili tvé slovo;
#4:30 a vztahuj svou ruku k uzdravování, čiň znamení a zázraky skrze jméno svého svatého služebníka Ježíše.“
#4:31 Když se pomodlili, otřáslo se místo, kde byli shromážděni, a všichni byli naplněni Duchem svatým a s odvahou mluvili slovo Boží.
#4:32 Všichni, kdo uvěřili, byli jedné mysli a jednoho srdce a nikdo neříkal o ničem, co měl, že je to jeho vlastní, nýbrž měli všechno společné.
#4:33 Boží moc provázela svědectví apoštolů o vzkříšení Pána Ježíše a na všech spočívala veliká milost.
#4:34 Nikdo mezi nimi netrpěl nouzi, neboť ti, kteří měli pole nebo domy, prodávali je, a peníze, které utržili,
#4:35 skládali apoštolům k nohám. Z toho se rozdávalo každému, jak potřeboval.
#4:36 Také Josef, kterého apoštolové nazvali Barnabáš - to znamená ‚syn útěchy‘ - levita původem z Kypru,
#4:37 měl pole, prodal je, peníze přinesl a položil před apoštoly. 
#5:1 Také nějaký muž, jménem Ananiáš, a jeho manželka Safira prodali svůj pozemek.
#5:2 Ananiáš si však s vědomím své ženy dal nějaké peníze stranou, zbytek přinesl a položil apoštolům k nohám.
#5:3 Ale Petr mu řekl: „Ananiáši, proč satan ovládl tvé srdce, že jsi lhal Duchu svatému a dal stranou část peněz za to pole?
#5:4 Bylo tvé a mohl sis je přece ponechat; a když jsi je prodal, mohl jsi s penězi naložit podle svého. Jak ses mohl odhodlat k tomuto činu? Nelhal jsi lidem, ale Bohu!“
#5:5 Když to Ananiáš uslyšel, skácel se a byl mrtev; a na všechny, kteří to slyšeli, padla velká bázeň.
#5:6 Mladší z bratří ho přikryli, vynesli a pohřbili.
#5:7 Asi po třech hodinách vstoupila jeho žena, netušíc, co se stalo.
#5:8 Petr se na ni obrátil: „Pověz mi, prodali jste to pole opravdu jen za tolik peněz?“ Ona řekla: „Ano, jen za tolik.“
#5:9 Petr jí řekl: „Proč jste se smluvili a tak pokoušeli Ducha Páně? Hle, za dveřmi je slyšet kroky těch, kteří pochovali tvého muže; ti odnesou i tebe.“
#5:10 A hned se skácela u jeho nohou a zemřela. Když ti mládenci vstoupili dovnitř, našli ji mrtvou. Vynesli ji a pohřbili k jejímu muži.
#5:11 A velká bázeň padla na celou církev i na všechny, kteří o tom slyšeli.
#5:12 Mezi lidem se rukama apoštolů dálo mnoho znamení a divů. Všichni se svorně scházeli v Šalomounově sloupoví
#5:13 a nikdo jiný se neodvažoval k nim přidružit, ale lid je chválil a ctil.
#5:14 A stále přibývalo mnoho mužů i žen, kteří uvěřili Pánu.
#5:15 Dokonce vynášeli nemocné i na ulici a kladli je na lehátka a na nosítka, aby na některého padl aspoň Petrův stín, až půjde kolem.
#5:16 Také z ostatních míst v okolí Jeruzaléma se scházelo množství lidí; přinášeli nemocné a sužované nečistými duchy a všichni byli uzdravováni.
#5:17 Ale velekněz a jeho stoupenci, totiž saducejská strana, byli naplněni závistí;
#5:18 chopili se apoštolů a vsadili je do městského vězení.
#5:19 Anděl Páně však v noci otevřel dveře vězení, vyvedl apoštoly ven a řekl:
#5:20 „Jděte znovu do chrámu a zvěstujte lidu ta slova života.“
#5:21 Oni poslechli, vešli na úsvitě do chrámu a učili. Jakmile se dostavil velekněz se svými lidmi, svolali veleradu a všechny starší izraelské a poslali pro apoštoly do vězení.
#5:22 Když tam stráže došly, vězení bylo prázdné. Vrátili se tedy a hlásili:
#5:23 „Dveře žaláře jsme našli pevně zavřené a stráže stály před nimi, ale když jsme otevřeli, nikoho jsme uvnitř nenašli.“
#5:24 Když ta slova uslyšel velitel chrámové stráže a velekněží, nedovedli si vysvětlit, co se to mohlo stát.
#5:25 Tu přišel nějaký člověk a oznámil jim: „Ti muži, které jste vsadili do vězení, jsou v chrámě a učí lid.“
#5:26 Velitel a stráž tedy pro ně došli a vedli je, ale bez násilí; báli se totiž, aby je lid nezačal kamenovat.
#5:27 Když je přivedli, postavili je před radu a velekněz je začal vyslýchat:
#5:28 „Důrazně jsme vám zakázali učit o tom člověku, a vy jste tím svým učením naplnili celý Jeruzalém; a na nás byste chtěli svalit odpovědnost za jeho krev!“
#5:29 Petr a apoštolové odpověděli: „Boha je třeba poslouchat, ne lidi.
#5:30 Bůh našich otců vzkřísil Ježíše, kterého vy jste pověsili na kříž a zabili;
#5:31 toho Bůh vyvýšil jako vůdce a spasitele a dal mu místo po své pravici, aby přinesl Izraeli pokání a odpuštění hříchů.
#5:32 My jsme svědkové toho všeho a s námi Duch svatý, kterého Bůh dal těm, kdo ho poslouchají.“
#5:33 Když to velekněží slyšeli, rozlítili se a chtěli apoštoly zabít.
#5:34 Tu vstal ve veleradě farizeus jménem Gamaliel, učitel zákona, kterého si vážil všechen lid; poručil, aby je na chvíli vyvedli ven,
#5:35 a řekl: „Dobře si rozmyslete, Izraelci, co s těmi lidmi chcete udělat.
#5:36 Před nedávnem povstal Theudas a tvrdil, že je Vyvolený; přidalo se k němu asi čtyři sta mužů. Když byl zabit, byli všichni jeho stoupenci rozprášeni a nakonec z toho nebylo nic.
#5:37 Po něm povstal ve dnech soupisu Judas Galilejský a strhl za sebou lid; také on zahynul a jeho stoupenci byli rozehnáni.
#5:38 Proto vám teď radím: Nechte tyto lidi a propusťte je. Pochází-li tento záměr a toto dílo z lidí, rozpadne se samo;
#5:39 pochází-li z Boha, nebudete moci ty lidi vyhubit - nechcete přece bojovat proti Bohu.“ Dali mu za pravdu;
#5:40 zavolali apoštoly, poručili je zbičovat, zakázali jim mluvit ve jménu Ježíšovu a pak je propustili.
#5:41 A oni odcházeli z velerady s radostnou myslí, že se jim dostalo té cti, aby nesli potupu pro jeho jméno.
#5:42 Dále učili den co den v chrámě i po domech a hlásali evangelium, že Ježíš je Mesiáš. 
#6:1 V té době, kdy učedníků stále přibývalo, začali si ti z nich, kteří vyrostli mezi Řeky, stěžovat na bratry z židovského prostředí, že se jejich vdovám nedává každodenně spravedlivý díl.
#6:2 A tak apoštolové svolali všechny učedníky a řekli: „Bohu se nebude líbit, jestliže my přestaneme kázat Boží slovo a budeme sloužit při stolech.
#6:3 Bratří, vyberte si proto mezi sebou sedm mužů, o nichž se ví, že jsou plni Ducha a moudrosti, a pověříme je touto službou.
#6:4 My pak budeme i nadále věnovat všechen svůj čas modlitbě a kázání slova.“
#6:5 Celé shromáždění s tímto návrhem rádo souhlasilo, a tak zvolili Štěpána, který byl plný víry a Ducha svatého, dále Filipa, Prochora, Nikánora, Timóna, Parména a Mikuláše z Antiochie, původem pohana, který přistoupil k židovství.
#6:6 Přivedli je před apoštoly, ti se pomodlili a vložili na ně ruce.
#6:7 Slovo Boží se šířilo a počet učedníků v Jeruzalémě velmi rostl. Také mnoho kněží přijalo víru.
#6:8 Štěpán byl obdařen Boží milostí a mocí a činil mezi lidem veliké divy a znamení.
#6:9 Tu proti němu vystoupili někteří židé, patřící k synagóze, zvané synagóga propuštěnců, a k synagóze Kyrénských a Alexandrijských, a společně se židy z Kilikie a z Asie se začali se Štěpánem přít.
#6:10 Nebyli však schopni čelit Duchu moudrosti, v jehož moci Štěpán mluvil.
#6:11 Navedli tedy několik mužů, aby prohlašovali: „My jsme slyšeli, jak mluví rouhavě proti Mojžíšovi a proti Bohu.“
#6:12 Tím pobouřili lid a starší se zákoníky; pak si pro Štěpána přišli, odvedli ho a postavili před radu.
#6:13 Přivedli křivé svědky a ti vypovídali: „Tenhle člověk znovu a znovu mluví proti tomuto svatému místu i proti Mojžíšovu zákonu.
#6:14 Na vlastní uši jsme slyšeli, jak řekl, že Ježíš Nazaretský zboří tento chrám a změní ustanovení, která nám dal Mojžíš.“
#6:15 Všichni, kteří v radě zasedali, pohlédli na Štěpána a viděli, že jeho tvář je jako tvář anděla. 
#7:1 Velekněz se Štěpána otázal: „Je tomu tak?“
#7:2 A Štěpán začal mluvit: „Bratří a otcové, vyslechněte mne: Bůh slávy se zjevil našemu praotci Abrahamovi ještě v Mezopotámii, než se usadil v Cháranu,
#7:3 a řekl mu: ‚Opusť svou zemi a své příbuzné a jdi do země, kterou ti ukážu.‘
#7:4 A tak vyšel Abraham z Chaldejské země a usadil se v Cháranu. Když jeho otec zemřel, vyzval jej Bůh, aby se odtud přestěhoval do této země, v které nyní žijete.
#7:5 Ale nedal mu z ní do vlastnictví ani píď, slíbil však, že ji dá natrvalo jemu i jeho potomkům, ačkoliv tehdy ještě Abraham neměl syna.
#7:6 A Bůh mu řekl: ‚Tvoji potomci se přestěhují do cizí země a budou tam bydlet; udělají z nich otroky a po čtyři sta let s nimi budou zle nakládat.
#7:7 Ale já budu soudit národ, který je zotročí‘, řekl Bůh, ‚a oni vyjdou svobodni a budou mi sloužit na tomto místě‘.
#7:8 Bůh uzavřel s Abrahamem smlouvu, jejímž znamením se stala obřízka. Když se podle Božího slibu Abrahamovi narodil Izák, osmého dne jej obřezal. Právě tak obřezal Izák Jákoba a Jákob svých dvanáct synů.
#7:9 Jákobovi synové žárlili na svého bratra Josefa a prodali ho do Egypta; ale Bůh byl s ním,
#7:10 vysvobodil ho ze všech jeho útrap a způsobil, že si ho farao, egyptský král, oblíbil pro jeho moudrost a svěřil mu správu nad Egyptem i nad celým královským domem.
#7:11 Potom nastal po celém Egyptě i Kanaánu hlad a veliká nouze, a naši praotcové neměli co jíst.
#7:12 Jákob se dověděl, že jsou v Egyptě zásoby obilí a dvakrát tam poslal své syny.
#7:13 Když tam přišli podruhé, dal se Josef svým bratřím poznat; tak se farao dověděl o Josefově původu.
#7:14 Pak si Josef vzal k sobě svého otce Jákoba a všechny své příbuzné; bylo jich sedmdesát pět.
#7:15 Tak se Jákob dostal do Egypta a tam zemřel on i naši praotcové.
#7:16 Jejich ostatky přenesli do Sichemu a uložili je do hrobky, kterou Abraham koupil od synů Emorových v Sichemu a zaplatil stříbrem.
#7:17 Když už byl blízko čas, kdy Bůh chtěl splnit to, co Abrahamovi slíbil, počet našeho lidu v Egyptě velmi vzrostl.
#7:18 Ale tu nastoupil v Egyptě jiný král, který už o Josefovi nic nevěděl.
#7:19 Ten jednal s naším národem nepřátelsky a přinutil naše praotce, aby odkládali děti, které se jim narodí, a nenechávali je naživu.
#7:20 V té době se narodil Mojžíš, kterého si Bůh vyvolil. Po tři měsíce o něj pečovali doma;
#7:21 a když ho museli odložit, ujala se ho faraónova dcera a starala se o něj jako o svého syna.
#7:22 Byl vychován ve vší egyptské moudrosti a byl mocný v slovech i činech.
#7:23 Když Mojžíš dosáhl věku čtyřiceti let, přišlo mu na mysl, aby se podíval, jak žijí jeho izraelští bratří.
#7:24 A tu uviděl, jak s jedním z nich nějaký Egypťan zle nakládá; přispěchal tomu ubožákovi na pomoc a pomstil ho tím, že Egypťana zabil.
#7:25 Myslel, že jeho bratří pochopí, že je chce Bůh skrze něho zachránit; ale oni to nepochopili.
#7:26 Druhého dne mezi ně zase přišel, právě když se dva z nich rvali; snažil se je smířit, a proto jim řekl: ‚Co to děláte? Jste přece bratří! Proč ubližujete jeden druhému?‘
#7:27 Ale ten, který svého bližního napadl, odstrčil Mojžíše a řekl: ‚Kdo tě nad námi ustanovil vládcem a soudcem?
#7:28 Snad mne nechceš také zabít jako včera toho Egypťana?‘
#7:29 Když to Mojžíš slyšel, utekl z Egypta a žil v zemi madiánské; tam se mu narodili dva synové.
#7:30 Když uplynulo čtyřicet let, zjevil se mu na poušti u hory Sínaj anděl v plameni hořícího keře.
#7:31 S údivem hleděl Mojžíš na to zjevení; šel blíž, aby se lépe podíval, a tu se ozval Hospodinův hlas:
#7:32 ‚Já jsem Bůh tvých otců, Bůh Abrahamův, Izákův a Jákobův.‘ Mojžíš se třásl strachem a neodvážil se vzhlédnout.
#7:33 Hospodin mu řekl: ‚Zuj si obuv, protože místo, na kterém stojíš, je země svatá.
#7:34 Dobře jsem viděl, jak můj lid v Egyptě těžce trpí, a slyšel jsem, jak sténá. Proto jsem sestoupil, abych je vysvobodil. Jdi tedy, posílám tě do Egypta.‘
#7:35 To je ten Mojžíš, kterého odmítli, když řekli: ‚Kdo tě nad námi ustanovil vládcem a soudcem?‘ Toho poslal Bůh jako vládce i vysvoboditele, když se mu prostřednictvím anděla zjevil v keři.
#7:36 To je ten Mojžíš, který je vyvedl z Egyptské země, který činil divy a znamení v Egyptě, při Rudém moři i na poušti po čtyřicet let.
#7:37 To je on, který řekl synům izraelským: ‚Bůh vám z vašich bratří povolá proroka, jako jsem já.‘
#7:38 To je ten, kdo při shromáždění lidu na poušti byl prostředníkem mezi andělem, který k němu mluvil na hoře Sínaj, a mezi našimi praotci. To on přijal pro vás slova života.
#7:39 Ale naši praotcové ho nechtěli za vůdce, zřekli se ho a zatoužili po Egyptu.
#7:40 Proto řekli Árónovi: ‚Udělej nám bohy, kteří by nás vedli. Vždyť nevíme, co se stalo s tím Mojžíšem, který nás vyvedl z Egypta.‘
#7:41 A tehdy si udělali sochu telete, začali této modle obětovat a před výtvorem vlastních rukou pořádali slavnosti.
#7:42 Proto se od nich Bůh odvrátil a dopustil, aby se klaněli nebeským mocnostem, jak se o tom píše v knize proroků: ‚Byl jsem to já, lide izraelský, komu jste po těch čtyřicet let na poušti obětovali a přinášeli dary?
#7:43 Byl to Moloch, jehož stánek jste s sebou nosili, a hvězdu boha Remfana, obrazy, které jste si udělali pro svou modloslužbu. Proto vás přestěhuji do vyhnanství, až do končin babylónských.‘
#7:44 Naši praotcové měli na poušti stánek svědectví; Bůh přikázal Mojžíšovi, aby jej udělal podle vzoru, který mu ukázal.
#7:45 Tento stánek odevzdali svým synům, a ti jej za Jozue vnesli do země pohanů, které Bůh před nimi zahnal. Tak tomu bylo až do časů Davidových.
#7:46 David nalezl u Boha milost a prosil, aby směl vyhledat místo, kde by přebýval Bůh Jákobův.
#7:47 Ale teprve Šalomoun vystavěl Bohu chrám.
#7:48 Avšak Nejvyšší nepřebývá v chrámech, vystavěných lidskýma rukama, jak praví prorok:
#7:49 ‚Mým trůnem je nebe a země podnoží mých nohou! Jaký chrám mi můžete vystavět, praví Hospodin, a je vůbec místo, kde bych mohl spočinout?
#7:50 Což to všechno nestvořila má ruka?‘
#7:51 Jste tvrdošíjní a máte pohanské srdce i uši! Nepřestáváte odporovat Duchu svatému, jako to dělali vaši otcové.
#7:52 Byl kdy nějaký prorok, aby ho vaši otcové nepronásledovali? Zabili ty, kteří předpověděli příchod Spravedlivého, a toho vy jste nyní zradili a zavraždili.
#7:53 Přijali jste Boží zákon z rukou andělů, ale sami jste jej nezachovali!“
#7:54 Když to členové rady slyšeli, začali na Štěpána v duchu zuřit a zlostí zatínali zuby.
#7:55 Ale on, plný Ducha svatého, pohleděl k nebi a uzřel Boží slávu i Ježíše, jak stojí po pravici Boží,
#7:56 a řekl: „Hle, vidím nebesa otevřená a Syna člověka, stojícího po pravici Boží.“
#7:57 Tu začali hrozně křičet a zacpávat si uši; všichni se na něho vrhli
#7:58 a hnali ho za město, aby ho kamenovali. Svědkové dali své pláště hlídat mládenci, který se jmenoval Saul.
#7:59 Když Štěpána kamenovali, on se modlil: „Pane Ježíši, přijmi mého ducha!“
#7:60 Pak klesl na kolena a zvolal mocným hlasem: „Pane, odpusť jim tento hřích!“ To řekl a zemřel. 
#8:1 Saul schvaloval, že Štěpána zabili. Tehdy začalo kruté pronásledování jeruzalémské církve; všichni kromě apoštolů se rozprchli po Judsku a Samařsku.
#8:2 Zbožní muži Štěpána pochovali a velmi nad ním truchlili.
#8:3 Saul se však snažil církev zničit: pátral dům od domu, zatýkal muže i ženy a dával je do žaláře.
#8:4 Ti, kteří se z Jeruzaléma rozprchli, začali kázat evangelium všude, kam přišli.
#8:5 Filip odešel do města Samaří a zvěstoval tam Krista.
#8:6 Všichni lidé byli zaujati Filipovými slovy, když je slyšeli, a když viděli znamení, která činil.
#8:7 Neboť z mnoha posedlých vycházeli s velikým křikem nečistí duchové a mnoho ochrnutých a chromých bylo uzdraveno.
#8:8 A tak nastala veliká radost v tom městě.
#8:9 Jeden muž, jménem Šimon, který tam žil, už dlouhou dobu svou magií uváděl v úžas samařský lid; říkal o sobě, že je v něm božská moc.
#8:10 Všichni - prostí i významní - mu dychtivě naslouchali a říkali si: „On je ta božská moc, která se nazývá Veliká.“
#8:11 Poslouchali ho ve všem proto, že na ně dlouhý čas působil svou magií.
#8:12 Ale když uvěřili Filipově zvěsti o Božím království a o Ježíši Kristu, dávali se pokřtít muži i ženy.
#8:13 Tu uvěřil i sám Šimon, dal se pokřtít, byl stále s Filipem a nevycházel z úžasu, když viděl, jak se tu dějí veliká znamení a mocné činy.
#8:14 Když apoštolové v Jeruzalémě uslyšeli, že v Samařsku přijali Boží slovo, poslali k nim Petra a Jana.
#8:15 Oni tam přišli a modlili se za ně, aby také jim byl dán Duch svatý,
#8:16 neboť ještě na nikoho z nich nesestoupil; byli jen pokřtěni ve jméno Pána Ježíše.
#8:17 Petr a Jan tedy na ně vložili ruce a oni přijali Ducha svatého.
#8:18 Když Šimon viděl, že ten, na koho apoštolové vloží ruce, dostává Ducha svatého, nabídl jim peníze a řekl:
#8:19 „Dejte i mně tu moc, aby Ducha svatého dostal každý, na koho vložím ruce.“
#8:20 Petr mu odpověděl: „Tvé peníze ať jsou zatraceny i s tebou: Myslil sis, že se Boží dar dá získat za peníze!
#8:21 Tato moc není pro tebe a nemůžeš mít na ní podíl, neboť tvé srdce není upřímné před Bohem.
#8:22 Odvrať se proto od této své ničemnosti a pros Boha; snad ti odpustí, co jsi zamýšlel.
#8:23 Vidím, že jsi pln hořké závisti a v zajetí nepravosti.“
#8:24 Šimon na to řekl: „Modlete se za mne k Bohu, aby mne nepostihlo nic z toho, o čem jste mluvili.“
#8:25 Apoštolové i potom vydávali svědectví a kázali slovo Páně. Pak se vraceli do Jeruzaléma a ještě cestou zvěstovali evangelium v mnoha samařských vesnicích.
#8:26 Anděl Páně řekl Filipovi: „Vydej se na jih k cestě, která vede z Jeruzaléma do Gázy.“ Ta cesta je opuštěná.
#8:27 Filip se vydal k té cestě a hle, právě přijížděl etiopský dvořan, správce všech pokladů kandaky, to jest etiopské královny. Ten vykonal pouť do Jeruzaléma
#8:28 a nyní se vracel na svém voze a četl proroka Izaiáše.
#8:29 Duch řekl Filipovi: „Běž k tomu vozu a jdi vedle něho!“
#8:30 Filip k němu přiběhl, a když uslyšel, že ten člověk čte proroka Izaiáše, zeptal se: „Rozumíš tomu, co čteš?“
#8:31 On odpověděl: „Jak bych mohl, když mi to nikdo nevyloží!“ A pozval Filipa, aby nastoupil a sedl si vedle něho.
#8:32 To místo Písma, které četl, znělo: ‚Jako ovce vedená na porážku, jako beránek, němý, když ho stříhají, ani on neotevřel ústa.
#8:33 Ponížil se, a proto byl soud nad ním zrušen; kdo bude moci vypravovat o jeho potomcích? Vždyť jeho život na této zemi byl ukončen.‘
#8:34 Dvořan se obrátil k Filipovi: „Vylož mi, prosím, o kom to prorok mluví - sám o sobě, či o někom jiném?“
#8:35 Tu Filip začal u toho slova Písma a zvěstoval mu Ježíše.
#8:36 Jak pokračovali v cestě, přijeli k místu, kde byla voda. Dvořan řekl: „Zde je voda. Co brání, abych byl pokřtěn?“
#8:37 ---
#8:38 Dal zastavit vůz a oba, Filip i dvořan, sestoupili do vody a Filip jej pokřtil.
#8:39 Když vystoupili z vody, Duch Páně se Filipa zmocnil a dvořan ho už neviděl, ale radoval se a jel dál svou cestou.
#8:40 Filip se pak ocitl v Azótu. Procházel všemi městy a přinášel jim radostnou zvěst, až se dostal do Cesareje. 
#9:1 Saul nepřestával vyhrožovat učedníkům Páně a chtěl je vyhladit. Šel proto k veleknězi
#9:2 a vyžádal si od něho doporučující listy pro synagógy v Damašku, aby tam mohl vyhledávat muže i ženy, kteří se hlásí k tomu směru, a přivést je v poutech do Jeruzaléma.
#9:3 Na cestě, když už byl blízko Damašku, zazářilo kolem něho náhle světlo z nebe.
#9:4 Padl na zem a uslyšel hlas: „Saule, Saule, proč mne pronásleduješ?“
#9:5 Saul řekl: „Kdo jsi, Pane?“ On odpověděl: „Já jsem Ježíš, kterého ty pronásleduješ.
#9:6 Vstaň, jdi do města a tam se dovíš, co máš dělat.“
#9:7 Muži, kteří ho doprovázeli, zůstali stát a nebyli schopni slova; slyšeli sice hlas, ale nespatřili nikoho.
#9:8 Saul vstal ze země, otevřel oči, ale nic neviděl. Museli ho vzít za ruce a dovést do Damašku.
#9:9 Po tři dny neviděl, nic nejedl a nepil.
#9:10 V Damašku žil jeden učedník, jménem Ananiáš. Toho Pán ve vidění zavolal: „Ananiáši!“ On odpověděl: „Zde jsem, Pane.“
#9:11 Pán mu řekl: „Jdi hned do ulice, která se jmenuje Přímá, a v domě Judově vyhledej Saula z Tarsu. Právě se modlí
#9:12 a dostalo se mu vidění, jak k němu vchází muž jménem Ananiáš a vkládá na něj ruce, aby opět viděl.“
#9:13 Ananiáš odpověděl: „Pane, mnoho lidí mi vyprávělo o tom člověku, kolik zla způsobil bratřím v Jeruzalémě.
#9:14 Také zde má od velekněží plnou moc zatknout každého, kdo vzývá tvé jméno.“
#9:15 Pán mu však řekl: „Jdi, neboť on je mým nástrojem, který jsem si zvolil, aby nesl mé jméno národům i králům a synům izraelským.
#9:16 Ukáži mu, co všechno musí podstoupit pro mé jméno.“
#9:17 Ananiáš šel, vstoupil do toho domu, vložil na Saula ruce a řekl: „Saule, můj bratře, posílá mě k tobě Pán - ten Ježíš, který se ti zjevil na tvé cestě; chce, abys opět viděl a byl naplněn Duchem svatým.“
#9:18 Tu jako by mu s očí spadly šupiny, zase viděl a hned se dal pokřtít.
#9:19 Pak přijal pokrm a síla se mu vrátila. S damašskými učedníky zůstal Saul několik dní
#9:20 a hned začal v synagógách kázat, že Ježíš je Syn Boží.
#9:21 Všichni, kdo ho slyšeli, žasli a říkali: „To je přece ten, který se snažil v Jeruzalémě vyhladit všechny vyznavače tohoto jména. I sem přišel jen proto, aby je v poutech odvedl k velekněžím.“
#9:22 Ale Saul působil čím dál tím mocněji a svými důkazy, že Ježíš je Mesiáš, přiváděl do úzkých damašské židy.
#9:23 Po nějaké době se židé uradili, že Saula zabijí,
#9:24 ale on se o jejich úkladech dověděl. Protože ve dne v noci hlídali i brány, aby ho mohli zahubit,
#9:25 spustili ho učedníci dolů z hradeb v koši po provazech.
#9:26 Když přišel do Jeruzaléma, chtěl se připojit k učedníkům; ale všichni se ho báli, protože nevěřili, že k nim patří.
#9:27 Tu se ho ujal Barnabáš, uvedl ho k apoštolům a vypravoval jim, jak Saul na cestě do Damašku uviděl Pána, uslyšel jeho hlas, a jak tam potom neohroženě kázal v Ježíšově jménu.
#9:28 Saul se nyní mohl v Jeruzalémě na všem podílet s apoštoly a všude směle mluvil ve jménu Páně.
#9:29 Kázal také řecky mluvícím židům a přel se s nimi, takže se ho pokoušeli zabít.
#9:30 Když se to bratří dověděli, doprovodili ho do Cesareje a poslali do Tarsu.
#9:31 A tak církev v celém Judsku, Galileji a Samaří měla klid, vnitřně i navenek rostla, žila v bázni Páně a vzrůstala počtem, protože ji Duch svatý posiloval.
#9:32 Když Petr procházel všechna ta místa, přišel také k bratřím, kteří žili v Lyddě.
#9:33 Tam se setkal s jedním člověkem, jménem Eneáš, který byl už osm let upoután na lůžko, poněvadž byl ochrnutý.
#9:34 Petr mu řekl: „Eneáši, Ježíš Kristus tě uzdravuje! Vstaň a ustel své lůžko!“ A Eneáš hned vstal.
#9:35 Všichni obyvatelé Lyddy a Sáronu, kteří to viděli, obrátili se k Pánu.
#9:36 V Joppe žila učednice jménem Tabita, řecky Dorkas. Konala mnoho dobrých skutků a štědře rozdávala almužny.
#9:37 Ale právě tehdy onemocněla a zemřela. Umyli ji a položili do horního pokoje.
#9:38 Poněvadž Lydda je blízko Joppe, dověděli se učedníci, že je tam Petr, a poslali k němu dva muže s naléhavou prosbou: „Přijď rychle k nám!“
#9:39 Petr se hned s nimi vydal na cestu. Když přišli do Joppe, zavedli jej do horního pokoje, kde ho s pláčem obklopily všechny vdovy a ukazovaly mu košile a pláště, které jim Tabita šila, dokud byla naživu.
#9:40 Petr poslal všechny z místnosti; pak poklekl, pomodlil se, obrátil se k mrtvé a řekl: „Tabito, vstaň!“ Ona otevřela oči, a když spatřila Petra, zvedla se na lůžku.
#9:41 Podal jí ruku a pomohl jí vstát. Pak všechny zavolal, i vdovy, a vrátil jim ji živou.
#9:42 Zpráva o tom se rozšířila po celém Joppe a mnoho lidí uvěřilo v Pána.
#9:43 Petr zůstal ještě delší čas v Joppe v domě Šimona koželuha. 
#10:1 V Cesareji žil nějaký muž jménem Kornélius, důstojník pluku zvaného Italský.
#10:2 Byl to člověk zbožný, s celou svou rodinou věřil v jediného Boha, byl velmi štědrý vůči židovskému lidu a pravidelně se modlil k Bohu.
#10:3 Ten měl kolem třetí hodiny odpoledne vidění, v němž jasně spatřil Božího anděla, jak k němu vchází a volá na něj: „Kornélie!“
#10:4 Pohlédl na něj a pln bázně řekl: „Co si přeješ, Pane?“ Anděl odpověděl: „Bůh přijal tvé modlitby a almužny a pamatuje na tebe.
#10:5 Vyprav hned posly do Joppe, ať sem přivedou Šimona, zvaného Petr.
#10:6 Bydlí u Šimona koželuha, který má dům u moře.“
#10:7 Když odešel ten anděl, zavolal si Kornélius ze svých lidí dva sluhy a jednoho zbožného vojáka ze své stráže,
#10:8 všechno jim vypravoval a pak je vyslal do Joppe.
#10:9 Druhého dne, právě když se blížili k městu, vyšel Petr za poledne na rovnou střechu domu, aby se modlil.
#10:10 Pak dostal hlad a chtěl se najíst. Zatímco mu připravovali jídlo, upadl do vytržení mysli:
#10:11 Vidí, jak se z otevřeného nebe cosi snáší; podobalo se to veliké plachtě, kterou spouštějí za čtyři cípy k zemi.
#10:12 Byly v ní všechny druhy živočichů: čtvernožci, plazi i ptáci.
#10:13 Tu k němu zazněl hlas: „Vstaň, Petře, zabíjej a jez!“
#10:14 Petr odpověděl: „To ne, Pane! Ještě nikdy jsem nejedl nic, co poskvrňuje a znečišťuje.“
#10:15 Ale hlas se ozval znovu: „Co Bůh prohlásil za čisté, nepokládej za nečisté.“
#10:16 To se opakovalo třikrát a zase to všechno bylo vyneseno vzhůru do nebe.
#10:17 Zatímco Petr úporně přemýšlel, co to jeho vidění může znamenat, podařilo se Kornéliovým poslům nalézt Šimonův dům. Zastavili se před vraty,
#10:18 zavolali a ptali se, zda tu bydlí Šimon, kterému říkají Petr.
#10:19 A Petr stále ještě přemýšlel o svém vidění, když mu Duch řekl: „Jsou tu tři muži a hledají tě;
#10:20 sejdi hned dolů a bez rozpaků s nimi jdi, neboť já jsem je poslal.“
#10:21 Petr tedy sešel dolů k těm mužům a řekl: „Já jsem ten, kterého hledáte. Proč jste za mnou přišli?“
#10:22 Oni odpověděli: „Posílá nás setník Kornélius, muž spravedlivý, který věří v jediného Boha a má dobrou pověst u všeho židovského lidu. Zjevil se mu anděl a rozkázal mu, aby tě pozval do svého domu a vyslechl, co mu máš říci.“
#10:23 Petr je zavedl dovnitř a nechal je u sebe přes noc. Hned druhého dne se s nimi Petr vydal na cestu a ještě několik bratří z Joppe ho doprovázelo.
#10:24 Nazítří přišli do Cesareje. Kornélius je očekával a spolu s ním jeho příbuzní a nejbližší přátelé.
#10:25 Když chtěl Petr vejít, vyšel mu Kornélius vstříc, padl na kolena a poklonil se mu.
#10:26 Ale Petr jej přinutil vstát a řekl: „Vstaň, vždyť i já jsem jen člověk.“
#10:27 Za rozhovoru vešli dovnitř a Petr shledal, že je tam shromážděno mnoho lidí.
#10:28 Promluvil k nim: „Dobře víte, že židům není dovoleno stýkat se s pohany a navštěvovat je. Mně však Bůh ukázal, abych si o žádném člověku nemyslel, že styk s ním poskvrňuje nebo znečišťuje.
#10:29 Proto jsem také bez váhání přišel, když jste pro mne poslali, a nyní se ptám, jaký jste k tomu měli důvod.“
#10:30 Kornélius odpověděl: „Jsou to právě tři dny, co jsem se v tuto chvíli modlil ve svém domě odpolední modlitbu, a náhle stál přede mnou muž v zářícím rouchu
#10:31 a řekl: ‚Kornélie, Bůh vyslyšel tvou modlitbu a ví o tvých dobrých skutcích.
#10:32 Vyprav posly do Joppe a povolej odtud Šimona, kterému říkají Petr. Bydlí v domě koželuha Šimona u moře.‘
#10:33 Hned jsem tedy pro tebe poslal a ty jsi ochotně přišel. Nyní jsme tu všichni shromážděni před Bohem a chceme slyšet vše, co ti Pán uložil.“
#10:34 A Petr se ujal slova: „Nyní skutečně vidím, že Bůh nikomu nestraní,
#10:35 ale v každém národě je mu milý ten, kdo v něho věří a činí, co je spravedlivé.
#10:36 To je ta zvěst, kterou Bůh poslal synům izraelským, když vyhlásil pokoj v Ježíši Kristu. On je Pánem všech.
#10:37 Dobře víte, co se dálo po celém Judsku: Začalo to v Galileji po křtu, který kázal Jan.
#10:38 Bůh obdařil Ježíše z Nazareta Duchem svatým a mocí, Ježíš procházel zemí, všem pomáhal a uzdravoval všechny, kteří byli v moci ďáblově, neboť Bůh byl s ním.
#10:39 A my jsme svědky všeho, co činil v zemi judské i v Jeruzalémě. Ale oni ho pověsili na kříž a zabili.
#10:40 Bůh jej však třetího dne vzkřísil a dal mu zjevit se -
#10:41 nikoli všemu lidu, nýbrž jen svědkům, které k tomu napřed vyvolil, totiž nám; my jsme s ním jedli a pili po jeho zmrtvýchvstání.
#10:42 A uložil nám, abychom kázali lidu a dosvědčovali, že je to on, koho Bůh ustanovil za soudce živých i mrtvých.
#10:43 Jemu všichni proroci vydávají svědectví, že pro jeho jméno budou odpuštěny hříchy každému, kdo v něho věří.“
#10:44 Ještě když Petr mluvil, sestoupil Duch svatý na všechny, kteří tu řeč slyšeli.
#10:45 Bratří židovského původu, kteří přišli s Petrem, žasli, že i pohanům byl dán dar Ducha svatého.
#10:46 Vždyť je slyšeli mluvit ve vytržení mysli a velebit Boha. Tu Petr prohlásil:
#10:47 „Kdo může zabránit, aby byli vodou pokřtěni ti, kteří přijali Ducha svatého jako my?“
#10:48 A dal pokyn, aby byli pokřtěni ve jménu Ježíše Krista. Potom jej pozvali, aby u nich zůstal několik dní. 
#11:1 O tom, že i pohané přijali slovo Boží, dověděli se apoštolové a bratří v Judsku.
#11:2 Když přišel Petr do Jeruzaléma, začali mu bratří židovského původu vyčítat:
#11:3 „Navštívil jsi neobřezané lidi a jedl s nimi!“
#11:4 A tak jim to Petr začal po pořádku vysvětlovat:
#11:5 „Byl jsem v městě Joppe a právě jsem se modlil, když jsem ve vytržení mysli měl vidění: Cosi se snáší dolů a podobá se to veliké plachtě, kterou spouštějí za čtyři cípy z nebe; zastavila se právě u mne.
#11:6 Když jsem se do ní pozorně podíval, uviděl jsem tam nejrůznější zvířata i divokou zvěř, plazy a ptáky.
#11:7 Uslyšel jsem hlas, který mi řekl: ‚Vstaň, Petře, zabíjej a jez!‘
#11:8 Odpověděl jsem: ‚To ne, Pane! Ještě nikdy nevešlo do mých úst nic, co poskvrňuje a znečišťuje.‘
#11:9 Ale hlas z nebe promluvil znovu: ‚Co Bůh prohlásil za čisté, nepokládej za nečisté.‘
#11:10 To se opakovalo třikrát a vše bylo opět vyzdviženo do nebe.
#11:11 A právě v tu chvíli se zastavili před domem, kde jsme bydleli, tři muži, poslaní ke mně z Cesareje.
#11:12 Duch mi řekl, abych bez rozpaků šel s nimi. Se mnou se vydalo na cestu i těchto šest bratří a všichni jsme vstoupili do Kornéliova domu.
#11:13 On nám vypravoval, jak se mu v jeho domě zjevil anděl a řekl mu: ‚Pošli někoho do Joppe a pozvi k sobě Šimona, kterému říkají Petr.
#11:14 Co on ti poví, přinese spásu tobě i tvé rodině.‘
#11:15 Když jsem k nim začal mluvit, sestoupil na ně Duch svatý, jako už na počátku sestoupil na nás.
#11:16 Tu jsem si vzpomněl na to, co řekl Pán: ‚Jan křtil vodou, ale vy budete pokřtěni Duchem svatým.‘
#11:17 Jestliže tedy jim Bůh dal stejný dar jako nám, když uvěřili v Pána Ježíše Krista, jak jsem já v tom mohl Bohu bránit?“ -
#11:18 Po těch slovech bratří už nic nenamítali, ale velebili Boha: „Tak i pohany povolal Bůh k pokání, aby dosáhli života!“
#11:19 Po smrti Štěpánově nastalo v Jeruzalémě pronásledování. Ti, kteří se odtud rozprchli, dostali se až do Fénicie, na Kypr a do Antiochie; slovo evangelia však zvěstovali jenom židům.
#11:20 Ale někteří z nich, původem z Kypru a z Kyrény, začali po svém příchodu do Antiochie zvěstovat Pána Ježíše také pohanům.
#11:21 Moc Boží byla s nimi, a veliké množství lidí uvěřilo a obrátilo se k Pánu.
#11:22 Zpráva o tom se dostala k sluchu církve v Jeruzalémě a bratří poslali do Antiochie Barnabáše.
#11:23 Když tam přišel a spatřil, co se z milosti Boží děje, měl radost a povzbuzoval všechny, aby ve svém rozhodnutí setrvali a zůstali Pánu věrni.
#11:24 Byl to muž dobrý, plný Ducha svatého a víry; a tak bylo mnoho lidí přivedeno k Pánu.
#11:25 Proto se Barnabáš odebral do Tarsu, aby vyhledal Saula.
#11:26 A když ho nalezl, vzal ho s sebou do Antiochie. Pracovali spolu v tamější církvi po celý rok a vyučovali velké množství lidí; a právě v Antiochii byli učedníci poprvé nazváni křesťany.
#11:27 V těch dnech přišli z Jeruzaléma do Antiochie proroci.
#11:28 Jeden z nich, jménem Agabus, veden Duchem předpověděl, že po celém světě nastane veliký hlad. To se také stalo za císaře Klaudia.
#11:29 Proto se učedníci rozhodli, že každý podle svých možností pomůže bratřím v Judsku.
#11:30 Učinili to a poslali sbírku po Barnabášovi a Saulovi jeruzalémským starším. 
#12:1 V té době král Herodes krutě zasáhl proti některým bratřím.
#12:2 Mečem dal popravit Jakuba, bratra Janova.
#12:3 Když viděl, že si tím získal židy, rozkázal zatknout také Petra. Byly právě velikonoce.
#12:4 Zmocnil se ho, dal ho zavřít do vězení a hlídat čtyřmi strážemi po čtyřech vojácích. Chtěl ho po velikonocích veřejně soudit.
#12:5 Petra tedy střežili ve vězení a církev se za něj stále modlila k Bohu.
#12:6 Noc předtím, kdy jej chtěl Herodes předvést na soud, spal Petr mezi dvěma vojáky, spoután dvěma řetězy, a stráže před dveřmi hlídaly vězení.
#12:7 Najednou u něho stál anděl Páně a v žaláři zazářilo světlo. Anděl udeřil Petra do boku, vzbudil ho a řekl: „Rychle! Vstaň!“ A s Petrových rukou spadly řetězy.
#12:8 Anděl mu řekl: „Opásej se a obuj se!“ Petr to udělal a anděl ho vyzval: „Vezmi si plášť a pojď za mnou.“
#12:9 Petr následoval tedy anděla ven z vězení, ale nebyl si jist, zda to všechno je skutečnost; myslel si, že má vidění.
#12:10 Prošli první stráží, pak druhou, a přišli k železné bráně, která vedla do města; ta se jim sama od sebe otevřela. Vyšli ven, prošli jednou ulicí a tu mu náhle anděl zmizel.
#12:11 Teprve nyní se Petr vzpamatoval a řekl: „Teď už vím, že Pán opravdu poslal svého anděla, vysvobodil mne z ruky Herodovy a uchránil od toho, co si přál židovský lid.“
#12:12 S tím vědomím šel do domu Marie, matky Jana, zvaného Marek. Tam se shromáždilo mnoho lidí a modlili se.
#12:13 Když Petr zatloukl na domovní dveře, přišla služka jménem Rodé, aby se zeptala, kdo to je.
#12:14 Tu poznala Petrův hlas, ale samou radostí hned neotevřela a běžela dovnitř oznámit, že Petr stojí přede dveřmi.
#12:15 Ale oni jí řekli: „Ty ses zbláznila!“ Ona však trvala na tom, že je to pravda. Řekli jí: „Tak to musí být jeho duch!“
#12:16 Ale Petr stále tloukl. Šli tedy otevřít a užasli, když ho spatřili.
#12:17 Pokynul jim rukou, aby je uklidnil, a vypravoval jim, jak ho Pán vyvedl z vězení; nakonec jim řekl: „Povězte to Jakubovi a ostatním bratřím.“ Pak odešel z města.
#12:18 Ráno se strhl mezi vojáky nemalý poplach, kam se Petr poděl.
#12:19 Herodes dal po něm pátrat, ale Petr nebyl k nalezení. Po výslechu tedy dal popravit stráže; potom opustil Judsko, odebral se do Cesareje a nějaký čas tam zůstal.
#12:20 Herodes byl tehdy velmi rozezlen na Tyrské a Sidonské. Jejich zástupci se však společně k němu dostavili; podplatili králova komorníka Blasta a požádali o smír, protože jejich země byla zásobována potravinami z Herodova království.
#12:21 Ve stanovený den zasedl Herodes v královském rouchu na trůn a pronesl k nim řeč;
#12:22 lid začal provolávat: „To mluví bůh, ne člověk!“
#12:23 A tu jej postihl anděl Páně za to, že si přivlastnil čest, patřící jen Bohu: zemřel rozežrán červy.
#12:24 Ale slovo Páně se šířit nepřestalo.
#12:25 Když Barnabáš a Saul splnili své poslání, vrátili se z Jeruzaléma do Antiochie a vzali s sebou Jana zvaného Marek. 
#13:1 V Antiochii byli v církvi proroci a učitelé: Barnabáš, Simeon zvaný Černý, Lucius z Kyrény, Manahem, který býval druhem tetrarchy Heroda, a Saul.
#13:2 Když konali bohoslužbu Pánu a postili se, řekl Duch svatý: „Oddělte mi Barnabáše a Saula k dílu, k němuž jsem je povolal.“
#13:3 A tak po modlitbách a postu na ně vložili ruce a vyslali je k dílu.
#13:4 Posláni tedy Duchem svatým, odešli Barnabáš a Saul do Seleukie a odtud se plavili na Kypr.
#13:5 Když dopluli do Salaminy, zvěstovali tu slovo Boží v židovských synagógách. Měli s sebou i Jana jako pomocníka.
#13:6 Pak prošli celým ostrovem až do Páfu. Tam se setkali s jakýmsi židovským kouzelníkem, který se vydával za proroka; říkali mu Barjezus.
#13:7 Byl u dvora místodržitele Sergia Paula, muže vzdělaného. Ten k sobě povolal Barnabáše a Saula, protože chtěl slyšet Boží slovo.
#13:8 Ale Elymas, ten kouzelník - tak se totiž vykládá jeho jméno - vystoupil proti nim a snažil se odvrátit místodržitele od víry.
#13:9 V tu chvíli Saul, kterému říkali také Pavel, byl naplněn Duchem svatým, upřel na Elymase zrak a řekl:
#13:10 „Ty svůdce všeho schopný, synu ďáblův, nepříteli Boží spravedlnosti, kdy už přestaneš podvracet přímé cesty Páně?
#13:11 Nyní na tebe dopadne Boží trest: Oslepneš a neuzříš sluneční světlo, dokud se nad tebou Bůh neslituje.“ Tu Elymase náhle obestřela mrákota a tma, tápal kolem sebe a hledal, kdo by ho vedl za ruku.
#13:12 Když místodržitel uviděl, co se stalo, uvěřil, pln údivu nad učením Páně.
#13:13 Z Páfu se Pavel se svými průvodci plavil do Perge v Pamfylii. Ale Jan se od nich oddělil a vrátil se do Jeruzaléma.
#13:14 Přes Perge šli dál do Pisidské Antiochie. Když nastala sobota, vešli do synagógy a posadili se.
#13:15 Po čtení ze Zákona a Proroků vybídli je představení synagógy: „Bratří, máte-li slovo povzbuzení, promluvte k lidu.“
#13:16 Tu povstal Pavel, pokynul rukou a řekl: „Muži izraelští a s nimi vy, kteří ctíte jediného Boha, slyšte mne!
#13:17 Bůh tohoto izraelského lidu si vyvolil naše praotce, učinil z nich za jejich pobytu v Egyptě silný národ a vyvedl je odtud svou velkou mocí.
#13:18 Celých čtyřicet let trpělivě snášel jejich chování na poušti,
#13:19 a když vyhladil sedm národů v kanaánské zemi, dal ji v úděl svému lidu.
#13:20 To trvalo asi čtyři sta padesát let. Potom jim dával soudce až do proroka Samuele.
#13:21 Pak chtěli krále, a Bůh jim dal Saula, syna Kíšova, z pokolení Benjamínova, který jim vládl čtyřicet let.
#13:22 Ale zbavil ho moci a povolal jim za krále Davida, o němž vydal svědectví: ‚V Davidovi, synu Isajovu, jsem nalezl muže, jakého jsem měl na mysli. On splní vše, co chci.‘
#13:23 A z Davidova potomstva dal Bůh Izraeli podle svého slibu Spasitele Ježíše.
#13:24 Před jeho vystoupením kázal Jan všemu izraelskému lidu, aby se odvrátili od svých hříchů a dali se pokřtít.
#13:25 Když Jan končil své poslání, řekl: ‚Já nejsem ten, za koho mě pokládáte. Za mnou přichází někdo, jemuž nejsem hoden rozvázat řemínek na jeho nohou.‘
#13:26 Bratří z rodu Abrahamova i vy, kteří s nimi ctíte jediného Boha, nám bylo posláno toto slovo spásy.
#13:27 Ale obyvatelé Jeruzaléma a jejich vůdcové Spasitele nepoznali, odsoudili ho, a tak naplnili slova proroků, která se čtou každou sobotu.
#13:28 Ačkoli na něm nenalezli žádný důvod pro trest smrti, vymohli si na Pilátovi, aby ho dal popravit.
#13:29 Když udělali všechno přesně tak, jak to bylo o něm v Písmu napsáno, sňali jej z kříže a položili do hrobu.
#13:30 Ale Bůh ho vzkřísil z mrtvých
#13:31 a on se po mnoho dní zjevoval těm, kteří s ním přišli z Galileje do Jeruzaléma; ti jsou nyní jeho svědky před lidem.
#13:32 My vám přinášíme radostnou zprávu, že slib, daný našim praotcům,
#13:33 splnil Bůh nám, jejich dětem, a vzkřísil Ježíše; vždyť je o něm psáno v druhém Žalmu: ‚Ty jsi můj Syn, já jsem tě dnes zplodil.‘
#13:34 To, že jej vzkřísí z mrtvých, takže se už nepromění v prach, slíbil těmito slovy: ‚Věrně vám splním zaslíbení, která jsem dal Davidovi.‘
#13:35 A pak na jiném místě říká: ‚Nedopustíš, aby se tvůj Svatý rozpadl v prach.‘
#13:36 David sloužil Bohu za svého života, potom podle jeho vůle zemřel, byl uložen k svým předkům a rozpadl se v prach.
#13:37 Ten však, kterého Bůh vzkřísil, se neobrátil v prach.
#13:38 Budiž vám tedy známo, bratří, že skrze něho se vám zvěstuje odpuštění všech hříchů, a to i těch, jichž vás nemohl zprostit Mojžíšův zákon.
#13:39 Ale v něm je ospravedlněn každý, kdo věří.
#13:40 Střezte se proto, aby vás nepostihlo, o čem se mluví v Prorocích:
#13:41 ‚Vy, kteří mnou pohrdáte, otevřte oči, zděste se a dejte se na útěk, protože já učiním ve vašich dnech něco, čemu nikdy neuvěříte, když vám to bude někdo vypravovat.‘“
#13:42 Když Pavel a Barnabáš vycházeli ze synagógy, všichni je prosili, aby k nim o tom všem znovu promluvili příští sobotu.
#13:43 Shromáždění se rozcházelo a mnoho židů i obrácených pohanů, kteří ctili jediného Boha, doprovázelo Pavla a Barnabáše; ti s nimi rozmlouvali a povzbuzovali je, aby se drželi Boží milosti.
#13:44 Příští sobotu přišlo skoro celé město slyšet Boží slovo.
#13:45 Když židé viděli tolik lidí, naplnila je závist a začali Pavlovým slovům odporovat a rouhat se.
#13:46 Ale Pavel a Barnabáš směle prohlásili: „Vám židům mělo být slovo Boží zvěstováno nejprve. Protože je odmítáte, a tak sami sebe odsuzujete k ztrátě věčného života, obracíme se k pohanům.
#13:47 Vždyť Pán nám přikázal: ‚Ustanovil jsem tě, abys byl světlem pohanům a nesl spásu až na sám konec země.‘
#13:48 Když to pohané uslyšeli, radovali se a velebili slovo Páně; ti pak, kteří byli vyvoleni k věčnému životu, uvěřili.
#13:49 A slovo Páně se šířilo po celé krajině.
#13:50 Ale židé pobouřili vznešené ženy, které také ctily jediného Boha, pobouřili i přední muže toho města, podnítili tím proti Pavlovi a Barnabášovi nepřátelství a vyhnali je ze svého kraje.
#13:51 Oni na svědectví proti nim setřásli prach s nohou a odešli do Ikonia.
#13:52 Ale učedníci byli naplnění radostí a Duchem svatým. 
#14:1 Totéž se stalo v Ikoniu: Pavel a Barnabáš vešli do židovské synagógy a mluvili tak mocně, že uvěřilo mnoho Židů i Řeků.
#14:2 Avšak ti židé, kteří neuvěřili, pobouřili pohany a vyvolali jejich nenávist proti bratřím.
#14:3 Přesto tam Pavel a Barnabáš dost dlouho zůstali a přes všechny překážky mluvili o Pánu; a Pán dosvědčoval svou milost tím, že jim dával moc konat znamení a zázraky.
#14:4 Obyvatelstvo města se rozdělilo: jedni byli při židech, druzí při apoštolech.
#14:5 Pohané i židé se svými představiteli se chystali apoštoly ztýrat a ukamenovat.
#14:6 Ti se o tom dověděli a uprchli do lykaonských měst Lystry a Derbe i okolí.
#14:7 Ani tu nepřestali kázat evangelium.
#14:8 V Lystře žil jeden člověk, který měl ochrnuté nohy; byl chromý od narození a nikdy nechodil.
#14:9 Ten poslouchal Pavlovo kázání. Pavel se na něho upřeně podíval, a když viděl, že věří v Boží pomoc,
#14:10 řekl mocným hlasem: „Postav se zpříma na nohy!“ A on vyskočil a chodil.
#14:11 Když zástupy viděly, co Pavel učinil, provolávaly lykaonsky: „To k nám sestoupili bohové v lidské podobě!“
#14:12 Barnabášovi začali říkat Zeus, Pavlovi pak Hermes, poněvadž to byl především on, kdo mluvil.
#14:13 Dokonce kněz Diova chrámu před hradbami dal přivést k bráně ověnčené býky a chtěl je s lidmi apoštolům obětovat.
#14:14 Když se to Barnabáš a Pavel doslechli, roztrhli svůj oděv, vběhli do zástupu mezi lidi
#14:15 a volali: „Co to děláte? Vždyť i my jsme smrtelní lidé jako vy. Zvěstujeme vám, abyste se od těchto marných věcí obrátili k živému Bohu, který učinil nebe, zemi, moře a všechno, co je v nich.
#14:16 Tento Bůh sice v minulosti nechával pohanské národy žít, jak chtěly,
#14:17 avšak nepřestal dosvědčovat sám sebe tím, že jim prokazoval dobro: dával vám s nebe déšť i úrodu v pravý čas, sytil vás pokrmem a naplňoval radostí.“
#14:18 Takovou řečí se jim jen s námahou podařilo zadržet zástupy, aby jim nezačaly obětovat.
#14:19 Tu se tam objevili Židé z Antiochie a Ikonia, strhli lidi na svou stranu a začali Pavla kamenovat. Když mysleli, že je mrtev, vyvlekli ho z města.
#14:20 Ale učedníci ho obstoupili a on vstal a vrátil se do města. Druhého dne pak odešel s Barnabášem do Derbe.
#14:21 I v tomto městě kázali evangelium a získali mnoho učedníků. Potom se vraceli přes Lystru, Ikonium a Pisidskou Antiochii.
#14:22 Všude tam posilovali učedníky a povzbuzovali je, aby vytrvali ve víře; říkali jim: „Musíme projít mnohým utrpením, než vejdeme do Božího království.“
#14:23 V každé té církvi ustanovili starší a v modlitbách a postech svěřili učedníky Pánu, v kterého uvěřili.
#14:24 Pak prošli Pisidií, dostali se do Pamfylie
#14:25 a tam kázali v městě Perge. Potom se odebrali do Attalie
#14:26 a odpluli do Antiochie v Sýrii, odkud před časem svěřeni Boží milosti vyšli k dílu, které právě teď dokončili.
#14:27 Po svém návratu shromáždili církev a vypravovali, co všechno skrze ně Bůh učinil a jak i pohanům otevřel dveře víry.
#14:28 A zůstali tam s učedníky delší dobu. 
#15:1 Tu přišli do Antiochie někteří lidé z Judska a začali bratry učit: „Nepřijmete-li obřízku, jak to předpisuje Mojžíšův zákon, nemůžete být spaseni.“
#15:2 Pavel a Barnabáš s tím nesouhlasili a dostali se s nimi do sporu; proto bylo rozhodnuto, aby ti dva a ještě někteří jiní z Antiochie šli do Jeruzaléma a předložili tuto otázku apoštolům a starším.
#15:3 Církev je tedy vyslala na cestu a oni šli přes Fénicii a Samařsko a všude k veliké radosti bratří vyprávěli, jak se i pohané obracejí k víře.
#15:4 Když přišli do Jeruzaléma, byli přijati církví, apoštoly a staršími a vypravovali jim, jak je Bůh ve všem vedl.
#15:5 Tu povstali někteří bratří z farizeů a prohlásili: „Pohané musí přijmout obřízku a musí se jim nařídit, aby zachovávali Mojžíšův zákon.“
#15:6 Apoštolové a starší se tedy sešli, aby celou tu věc uvážili.
#15:7 Když došlo k velké rozepři, povstal Petr a promluvil k nim: „Dobře víte, bratří, že si mě Bůh hned na začátku mezi vámi vyvolil, aby ode mne pohané uslyšeli slovo evangelia a uvěřili.
#15:8 A sám Bůh, jenž zná lidská srdce, se za ně postavil: Dal jim Ducha svatého tak jako nám
#15:9 a neučinil žádného rozdílu mezi námi a jimi, protože jejich srdce očistil vírou.
#15:10 Proč tedy nyní pokoušíte Boha a chcete vložit na učedníky břemeno, které nemohli unést ani naši otcové ani my!
#15:11 Věříme přece, že jsme stejně jako oni spaseni milostí Pána Ježíše.“
#15:12 A když Barnabáš a Pavel začali vypravovat, jaká znamení a divy činil Bůh skrze ně mezi pohany, všichni ve shromáždění zmlkli a poslouchali.
#15:13 Když domluvili, ujal se slova Jakub a řekl: „Bratří, slyšte mne!
#15:14 Šimon vypravoval, jak Bůh poprvé projevil pohanům svou milost a povolal si z nich svůj lid.
#15:15 S tím se shodují slova proroků, neboť je psáno:
#15:16 ‚Navrátím se zase a znovu postavím Davidův zbořený dům, z jeho trosek jej opět vystavím a zbuduji, jako byl dřív,
#15:17 aby také ostatní lidé hledali Pána, všechny národy, které jsem přijal za své. To praví Pán,
#15:18 který to oznámil už před věky.‘
#15:19 Proto já soudím, abychom nedělali potíže pohanům, kteří se obracejí k Bohu,
#15:20 ale jen jim napsali, aby se vyhýbali všemu, co přišlo do styku s pohanskou bohoslužbou, aby nežili ve smilstvu, aby nejedli maso zvířat, která nebyla zbavena krve, a aby nepožívali krev.
#15:21 Vždyť Mojžíš má odedávna po městech své kazatele, kteří čtou jeho Zákon v synagógách každou sobotu.“
#15:22 Tu se apoštolové a starší za souhlasu celé církve rozhodli, že ze svého středu vyvolí dva muže a pošlou je s Pavlem a Barnabášem do Antiochie. Byli to Juda, kterému říkali Barsabáš, a Silas, přední muži mezi bratřími.
#15:23 Po nich poslali tento dopis: „My, apoštolové a starší, vaši bratří, posíláme pozdrav vám, svým bratřím v Antiochii, Sýrii a Kilikii, kteří jste dříve byli pohany.
#15:24 Dověděli jsme se, že vás někteří lidé od nás zneklidnili a zmátli svými slovy, ačkoliv jsme jim k tomu nedali žádný pokyn.
#15:25 Proto jsme se jednomyslně rozhodli, že zvolíme dva muže a pošleme je k vám s našimi milými bratry Barnabášem a Pavlem,
#15:26 kteří ve službě našeho Pána Ježíše Krista nasadili svůj život.
#15:27 Posíláme k vám tedy Judu a Silase, kteří vám to vše sami potvrdí.
#15:28 Toto jest rozhodnutí Ducha svatého i naše: Nikdo ať vás nezatěžuje jinými povinnostmi než těmi, které jsou naprosto nutné:
#15:29 Zdržujte se všeho, co bylo obětováno modlám, také krve, pak masa zvířat, která nebyla zbavena krve, a konečně smilstva. Jestliže se toho všeho vyvarujete, budete jednat správně. Buďte zdrávi.“
#15:30 Pak se pověření bratří odebrali do Antiochie. Tam všechny shromáždili a odevzdali jim dopis.
#15:31 Když jej bratří přečetli, velmi se zaradovali, protože je uklidnil.
#15:32 Také Juda a Silas bratry velmi povzbudili a posílili svými slovy; vždyť i oni byli proroci.
#15:33 Zůstali tam nějakou dobu; pak je bratří s přáním pokoje propustili a oni se vrátili zpět.
#15:34 ---
#15:35 Pavel a Barnabáš zůstali ještě v Antiochii a s mnoha jinými tam učili a zvěstovali slovo Páně.
#15:36 Po nějaké době řekl Pavel Barnabášovi: „Navštivme opět naše bratry ve všech městech, kde jsme kázali slovo Páně, a podívejme se, jak se jim daří.“
#15:37 Barnabáš chtěl s sebou vzít také Jana Marka.
#15:38 Pavel však nepokládal za správné vzít ho s sebou, poněvadž je opustil v Pamfylii a v práci s nimi nepokračoval.
#15:39 Vznikla z toho taková neshoda, že se spolu rozešli: Barnabáš vzal s sebou Marka a plavil se na Kypr,
#15:40 Pavel si vybral za spolupracovníka Silase, a když ho bratří poručili milosti Páně, vydal se i on na cestu.
#15:41 Procházel Sýrií a Kilikií a všude posiloval církve. 
#16:1 Tak se Pavel dostal také do Derbe a do Lystry. Tam byl jeden učedník jménem Timoteus; jeho matka byla židovka, která uvěřila v Krista, ale jeho otec byl pohan.
#16:2 Bratří v Lystře a v Ikoniu o něm vydávali dobré svědectví,
#16:3 a Pavel ho chtěl vzít s sebou. Z ohledu na tamější židy jej dal obřezat: všichni totiž věděli, že jeho otec byl pohan.
#16:4 Ve všech městech, jimiž procházeli, předávali závazná ustanovení, na nichž se usnesli apoštolové a starší v Jeruzalémě.
#16:5 A tak se církve upevňovaly ve víře a počet bratří rostl každým dnem.
#16:6 Poněvadž jim Duch svatý zabránil zvěstovat Slovo v provincii Asii, procházeli Frygií a krajinou galatskou.
#16:7 Když přišli až k Mysii, pokoušeli se dostat do Bithynie, ale Duch Ježíšův jim to nedovolil.
#16:8 Prošli tedy Mysií a přišli k moři do Troady.
#16:9 Tam měl Pavel v noci vidění: Stanul před ním jakýsi Makedonec a velmi ho prosil: „Přeplav se do Makedonie a pomoz nám!“
#16:10 Po tomto Pavlově vidění jsme se bez váhání chystali na cestu do Makedonie, protože jsme usoudili, že nás volá Bůh, abychom tam kázali evangelium.
#16:11 Vypluli jsme tedy z Troady a plavili se přímo na ostrov Samothráké a druhého dne do makedonské Neapole.
#16:12 Odtud jsme šli do Filip, které jsou nejvýznamnějším městem té části Makedonie a římskou kolonií. V tomto městě jsme strávili několik dní.
#16:13 V sobotu jsme vyšli za bránu k řece, protože jsme se domnívali, že tam bude modlitebna; posadili jsme se a mluvili k ženám, které se tam sešly.
#16:14 Poslouchala nás i jedna žena jménem Lydie, obchodnice s purpurem z města Thyatir, která věřila v jediného Boha. Pán jí otevřel srdce, aby přijala, co Pavel zvěstoval.
#16:15 Když byla ona a všichni z jejího domu pokřtěni, obrátila se na nás s prosbou: „Jste-li přesvědčeni, že jsem uvěřila v Pána, vejděte do mého domu a buďte mými hosty„; a my jsme její naléhavé pozvání přijali.
#16:16 Když jsme šli jednou do modlitebny, potkala nás mladá otrokyně, která měla věšteckého ducha a předpovídáním budoucnosti přinášela svým pánům značný zisk.
#16:17 Chodila za Pavlem a za námi a stále volala: „Toto jsou služebníci nejvyššího Boha. Zvěstují vám cestu ke spáse.“
#16:18 A to dělala po mnoho dní. Pavlovi to bylo proti mysli, obrátil se proto na toho ducha a řekl: „Ve jménu Ježíše Krista ti přikazuji, abys z ní vyšel!“ A v tu chvíli ji ten zlý duch opustil.
#16:19 Když si její páni uvědomili, že tím přišli o svůj zisk, uchopili Pavla a Silase a vlekli je na náměstí před městskou správu.
#16:20 Tam je předvedli před soudce a řekli: „Tito lidé pobuřují naše město. Jsou to Židé
#16:21 a rozšiřují zvyky, které my jako Římané nemůžeme uznat, nebo se dokonce podle nich řídit.“
#16:22 Také dav se obrátil proti nim a soudcové z nich dali strhat šaty a rozkázali je zbičovat.
#16:23 Když je důkladně zbičovali, zavřeli je do vězení a žalářníkovi poručili, aby je dobře hlídal.
#16:24 On je podle toho rozkazu zavřel do nejbezpečnější cely a pro jistotu jim dal nohy do klády.
#16:25 Kolem půlnoci se Pavel a Silas modlili a zpěvem oslavovali Boha; ostatní vězňové je poslouchali.
#16:26 Tu náhle nastalo veliké zemětřesení a celé vězení se otřáslo až do základů. Rázem se otevřely všechny dveře a všem vězňům spadla pouta.
#16:27 Když se žalářník probudil a uviděl, že jsou všechny dveře vězení otevřené, vytasil meč a chtěl se zabít, protože myslel, že mu vězňové uprchli.
#16:28 Ale Pavel hlasitě vykřikl: „Nedělej to! Jsme tu všichni!“
#16:29 Tu žalářník rozkázal, aby mu přinesli světlo, vběhl dovnitř a pln strachu padl před Pavlem a Silasem na kolena.
#16:30 Pak je vyvedl ven a řekl: „Bohové a páni, co mám dělat, abych byl zachráněn?“
#16:31 Oni mu řekli: „Věř v Pána Ježíše, a budeš spasen ty i všichni, kdo jsou v tvém domě.“
#16:32 A začali jemu i všem v jeho domácnosti zvěstovat slovo Boží.
#16:33 Ještě v tu noční chvíli se jich ujal, očistil jim rány a hned se dal se všemi svými lidmi pokřtít.
#16:34 Pak je zavedl do svého domu, pozval je ke stolu a s celou rodinou se radoval, že uvěřili v Boha.
#16:35 Druhého dne ráno poslali soudcové ozbrojenou stráž s rozkazem: „Propusť ty muže!“
#16:36 Žalářník oznámil tento rozkaz Pavlovi: „Soudcové nařídili, abyste byli propuštěni. Od této chvíle jste svobodni a v pokoji odejděte.“
#16:37 Ale Pavel řekl: „Nás, římské občany, veřejně a bez soudu zbili a zavřeli do vězení. A teď se nás chtějí ve vší tichosti zbavit? To ne! Ať sem sami přijdou a propustí nás!“
#16:38 Stráž to oznámila soudcům. A ti, když uslyšeli, že Pavel a Silas jsou římští občané,
#16:39 ulekli se a přišli za nimi; omluvili se, vyvedli je z vězení a prosili, aby opustili město.
#16:40 Pavel a Silas vyšli z vězení a šli k Lydii. Tam se shledali s bratřími, povzbudili je a odešli z Filip. 
#17:1 Dali se cestou, která vede přes města Amfipolis a Apollonii a přišli do Tesaloniky, kde měli Židé synagógu.
#17:2 Pavel jako obvykle přišel do jejich shromáždění a po tři soboty k nim mluvil,
#17:3 vykládal Písmo a dokazoval, že Mesiáš musel trpět a vstát z mrtvých. „A ten Mesiáš“, řekl Pavel, „je Ježíš, kterého já vám zvěstuji.“
#17:4 Někteří z nich se tím dali přesvědčit a připojili se k Pavlovi a Silasovi, také velmi mnoho Řeků, kteří už ctili jediného Boha, a nemálo žen z významných rodin.
#17:5 To však naplnilo Židy hněvem a závistí. S pomocí několika ničemných lidí z ulice vyvolali srocení davu a tak pobouřili celé město. Pak napadli Jásonův dům a chtěli Pavla a Silase postavit před shromáždění.
#17:6 Když je nenalezli, vlekli Jásona a několik bratří k představeným města a křičeli: „Ti, kteří pobouřili celý svět, přišli i k nám, a Jáson je přijal do svého domu!
#17:7 Ti všichni porušují císařova nařízení, protože tvrdí, že pravým králem je Ježíš.“
#17:8 Tato slova poděsila všechen lid i představené města,
#17:9 ale když Jáson a ostatní zaplatili záruku, byli propuštěni.
#17:10 A hned té noci vypravili bratří Pavla a Silase do Beroje. Když tam přišli, odebrali se do židovské synagógy.
#17:11 Židé v Beroji byli přístupnější než v Tesalonice: Přijali evangelium s velikou dychtivostí a každý den zkoumali v Písmu, zdali je to tak, jak zvěstuje Pavel.
#17:12 A tak mnozí z nich uvěřili a s nimi nemálo Řeků, vznešených žen i mužů.
#17:13 Jakmile se však Židé v Tesalonice dověděli, že Pavel káže slovo Boží i v Beroji, vypravili se tam a začali podněcovat a pobuřovat lidi.
#17:14 Tu bratří hned vypravili Pavla z Beroje směrem k moři, ale Silas a Timoteus ve městě zůstali.
#17:15 Ti, kteří Pavla doprovázeli, šli s ním až do Athén a odtud se vrátili s jeho vzkazem, aby k němu Silas a Timoteus co nejdříve přišli.
#17:16 Zatím na ně Pavel v Athénách čekal; když shledal, kolik modlářství je v tom městě, velmi ho to znepokojovalo.
#17:17 Proto mluvil v synagóze se židy a s pohany, kteří uvěřili v jediného Boha; a každý den hovořil i na náměstí s lidmi, kteří tam právě byli.
#17:18 Rozmlouvali s ním i někteří epikurejští a stoičtí filosofové. Jedni se ptali: „Co nám to chce ten nedovzdělanec vykládat?“ Druzí říkali: „Zdá se, že nás chce získat pro cizí božstva.“ Tak soudili, protože Pavel kázal o Ježíšovi a o zmrtvýchvstání.
#17:19 Pak ho vzali s sebou, dovedli na Areopag a tam mu položili otázku: „Rádi bychom se dověděli, jaké je to tvé nové učení, které šíříš.
#17:20 Vždyť to, co nám vykládáš, zní velice podivně. Chceme se tedy dovědět, co to je.“
#17:21 Všichni Athéňané i cizinci, kteří tam pobývají, ničemu totiž nevěnují tolik času jako tomu, že vykládají a poslouchají něco nového.
#17:22 Pavel se tedy postavil doprostřed shromáždění na Areopagu a promluvil: „Athéňané, vidím, že jste v uctívání bohů velice horliví.
#17:23 Když jsem procházel vašimi posvátnými místy a prohlížel si je, nalezl jsem i oltář s nápisem: ‚Neznámému bohu‘. Koho takto uctíváte, a ještě neznáte, toho vám zvěstuji:
#17:24 Bůh, který učinil svět a všechno, co je v něm, ten je pánem nebe i země, a nebydlí v chrámech, které lidé vystavěli,
#17:25 ani si nedává od lidí sloužit, jako by byl na nich závislý; vždyť je to on sám, který všemu dává život, dech i všechno ostatní.
#17:26 On stvořil z jednoho člověka všechno lidstvo, aby přebývalo na povrchu země, určil pevná roční údobí i hranice lidských sídel.
#17:27 Bůh to učinil proto, aby jej lidé hledali, zda by se ho snad nějakým způsobem mohli dopátrat a tak jej nalézt, a přece není od nikoho z nás daleko.
#17:28 ‚Neboť v něm žijeme, pohybujeme se, jsme‘, jak to říkají i někteří z vašich básníků: ‚Vždyť jsme jeho děti.‘
#17:29 Jsme-li tedy Božími dětmi, nemůžeme si myslet, že božstvo se podobá něčemu, co bylo vyrobeno ze zlata, stříbra nebo z kamene lidskou zručností a důmyslem.
#17:30 Bůh však prominul lidem dobu, kdy to ještě nemohli pochopit, a nyní zvěstuje všem, ať jsou kdekoliv, aby této neznalosti litovali a obrátili se k němu.
#17:31 Neboť ustanovil den, v němž bude spravedlivě soudit celý svět skrze muže, kterého k tomu určil. Všem lidem o tom poskytl důkaz, když jej vzkřísil z mrtvých.“
#17:32 Jakmile uslyšeli o vzkříšení z mrtvých, jedni se mu začali smát a druzí řekli: „Rádi si tě poslechneme, ale až někdy jindy.“
#17:33 A tak Pavel od nich odešel.
#17:34 Někteří se však k němu připojili a uvěřili; mezi nimi byl i Dionysios z Areopagu, žena jménem Damaris a s nimi ještě jiní. 
#18:1 Potom odešel Pavel z Athén a přišel do Korintu.
#18:2 Tam se setkal s jedním Židem, který se jmenoval Akvila a pocházel z Pontu. Nedávno přišel se svou manželkou Priscillou z Itálie, protože císař Klaudius vydal rozkaz, aby všichni Židé opustili Řím. Pavel se s nimi sblížil,
#18:3 zůstal u nich a pracoval s nimi, poněvadž měli stejné řemeslo - dělali stany.
#18:4 Každou sobotu mluvil v synagóze a snažil se získat židy i pohany.
#18:5 A když přišli z Makedonie Silas a Timoteus, věnoval se Pavel zcela kázání a dokazoval židům, že Ježíš je zaslíbený Mesiáš.
#18:6 Židé se však proti Pavlovi postavili a rouhali se. Proto setřásl prach ze svého roucha a řekl: „Vy sami jste odpovědni za svou záhubu. Já jsem vůči vám bez viny a od této chvíle se obrátím k pohanům.“
#18:7 Odešel ze synagógy a působil v sousedním domě Tita Justa, pohana, který uvěřil v Hospodina.
#18:8 Představený synagógy Krispus a všichni, kteří byli v jeho domě, uvěřili Pánu; také mnozí z Korinťanů, kteří Pavla poslouchali, uvěřili a dali se pokřtít.
#18:9 Jedné noci měl Pavel vidění, v němž mu Pán řekl: „Neboj se! Mluv a nemlč,
#18:10 protože já jsem s tebou a nikdo ti neublíží. Mnozí v tomto městě patří k mému lidu.“
#18:11 A tak tam Pavel zůstal jeden a půl roku a učil je Božímu slovu.
#18:12 Když byl Gallio místodržitelem v Achaji, vystoupili Židé společně proti Pavlovi, přivedli ho na soud
#18:13 a takto ho obžalovali: „Tento člověk přemlouvá lidi, aby uctívali Boha v rozporu se zákonem.“
#18:14 Když se už Pavel chtěl hájit, řekl Gallio Židům: „Kdyby šlo o nějaký přečin nebo dokonce zločin, náležitě bych vás, Židé, vyslechl.
#18:15 Poněvadž se to však týká sporů o slova či nějaká jména a váš vlastní zákon, vyřiďte si to sami mezi sebou. Tím se já jako soudce zabývat nebudu.“
#18:16 A dal rozkaz, aby je vyvedli ze soudní síně.
#18:17 Tu se všichni chopili představeného synagógy Sosthena a bili ho přímo před zraky soudce; ale Gallio tomu nevěnoval pozornost.
#18:18 Pavel zůstal v Korintě ještě mnoho dní, pak se s bratřími rozloučil a odplul s Priscillou a jejím mužem Akvilou do Sýrie. Protože učinil slib, dal si v Kenchrejích ostříhat vlasy.
#18:19 Dostali se do Efezu, kde oba manželé už zůstali. Pavel tam šel do synagógy a hovořil se Židy.
#18:20 Ačkoli ho žádali, aby s nimi zůstal delší dobu, on jejich pozvání nepřijal, rozloučil se s nimi a řekl:
#18:21 „Já se k vám vrátím, bude-li Bůh chtít.“ Pak odplul z Efezu
#18:22 do Cesareje a odtud se vydal do Jeruzaléma, aby pozdravil církev. Potom odešel do Syrské Antiochie.
#18:23 Když tam pobyl nějaký čas, vydal se znovu na cestu. Procházel galatskou krajinou a Frygií a posiloval tam všechny učedníky.
#18:24 Mezitím přišel do Efezu Žid jménem Apollos, původem z Alexandrie, muž vzdělaný a výmluvný, který dovedl přesvědčivě vykládat svaté Písmo.
#18:25 Byl už poučen o cestě Páně, mluvil s velikým nadšením a učil přesně o Ježíšovi, ale znal jenom křest Janův.
#18:26 Začal neohroženě vystupovat v synagóze. Když ho uslyšeli Priscilla a Akvila, vzali ho k sobě a ještě důkladněji mu vyložili Boží cestu.
#18:27 Bratří ho také povzbudili v jeho úmyslu vypravit se do Achaje a napsali tamějším učedníkům doporučující dopis. Když tam Apollos přišel, z milosti Boží velmi prospěl těm, kteří uvěřili.
#18:28 Ve veřejných rozhovorech překonával židy přesvědčivými důkazy z Písem, že Mesiáš je Ježíš. 
#19:1 Zatímco byl Apollos v Korintě, prošel Pavel hornatým vnitrozemím a přišel do Efezu; tam se setkal s nějakými učedníky.
#19:2 Zeptal se jich: „Když jste uvěřili, přijali jste Ducha svatého?“ Odpověděli mu: „Vůbec jsme neslyšeli, že je seslán Duch svatý.“
#19:3 Pavel řekl: „Jakým křtem jste tedy byli pokřtěni?“ Oni řekli: „Křtem Janovým.“
#19:4 Tu jim Pavel prohlásil: „Jan křtil ty, kteří se odvrátili od svých hříchů, a vybízel lid, aby uvěřili v toho, který přijde po něm - v Ježíše.“
#19:5 Když to uslyšeli, dali se pokřtít ve jméno Pána Ježíše.
#19:6 Jakmile na ně Pavel vložil ruce, sestoupil na ně Duch svatý a oni mluvili v prorockém vytržení.
#19:7 Těch mužů bylo asi dvanáct.
#19:8 V synagóze pak Pavel neohroženě působil po tři měsíce; rozmlouval s lidmi a přesvědčoval je o království Božím.
#19:9 Protože však někteří tvrdošíjně trvali na svém a nedali se přesvědčit, ba dokonce cestu Páně přede všemi tupili, oddělil se Pavel od nich, odvedl s sebou i učedníky a začal denně mluvit v přednáškové síni filosofa jménem Tyrannos.
#19:10 To trvalo dva roky, takže všichni obyvatelé provincie Asie mohli slyšet slovo Páně, židé i pohané.
#19:11 Bůh konal skrze Pavla neobvyklé mocné činy.
#19:12 Lidé dokonce odnášeli k nemocným šátky a zástěry, kterých se dotkl, a zlí duchové je opouštěli.
#19:13 Také někteří židovští zaříkávači, kteří cestovali od města k městu, pokusili se užívat ke svému zaklínání jména Pána Ježíše. Nad těmi, kteří byli posedlí zlými duchy, říkali: „Zaklínáme vás Ježíšem, kterého káže Pavel.“
#19:14 Tak to dělalo sedm synů Skévy, prý židovského velekněze.
#19:15 Ale zlý duch jim řekl: „Ježíše znám a o Pavlovi vím. Ale kdo jste vy?“
#19:16 Tu člověk, v kterém byl ten zlý duch, se na ně vrhl, všechny je přemohl a tak je zřídil, že z toho domu utekli nazí a plní ran.
#19:17 To se rozhlásilo mezi všemi židy i pohany, kteří žili v Efezu; na všechny padla bázeň a jméno Pána Ježíše bylo ve velké úctě.
#19:18 Přicházeli i mnozí z těch, kteří uvěřili, a přede všemi vyznávali, že také oni dříve používali zaklínání.
#19:19 Nemálo pak těch, kteří se zabývali magií, přinesli své knihy a přede všemi je spálili. Jejich cena se odhadovala na padesát tisíc stříbrných.
#19:20 A tak mocí Páně rostlo a rozmáhalo se jeho slovo.
#19:21 Po těchto událostech se Pavel veden Duchem svatým rozhodl, že půjde přes Makedonii a Achaju do Jeruzaléma. Řekl: „Zůstanu tam nějaký čas, a potom musím také do Říma.“
#19:22 Poslal do Makedonie dva ze svých pomocníků, Timotea a Erasta, a sám se ještě nějaký čas zdržel v Asii.
#19:23 V té době došlo k velikému pobouření kvůli tomu učení.
#19:24 Nějaký Demetrios, zlatník a výrobce stříbrných napodobenin Artemidina chrámu, který poskytoval značný výdělek řemeslníkům,
#19:25 svolal je i ostatní, kteří dělali podobné věci, a řekl jim:
#19:26 „Mužové, víte, že z této práce máme blahobyt, a vidíte a slyšíte, že tenhle Pavel přemluvil a svedl mnoho lidí nejen z Efezu, nýbrž skoro z celé provincie. Říká, že bohové udělaní lidskýma rukama nejsou žádní bohové.
#19:27 Je nebezpečí, že nejen náš obor ztratí vážnost, nýbrž i chrám veliké bohyně Artemidy nebude považován za nic, a začne upadat sláva té, kterou uctívá naše provincie i celý svět.“
#19:28 Když to uslyšeli, velmi se rozzlobili a začali křičet: „Veliká je efezská Artemis!“
#19:29 Pobouření se rozšířilo na celé město. Lidé se hromadně hnali do divadla a vlekli s sebou Makedonce Gaia a Aristarcha, Pavlovy průvodce na cestách.
#19:30 Pavel chtěl jít do shromáždění lidu, ale učedníci mu v tom zabránili.
#19:31 Také někteří vysocí úředníci provincie, kteří mu byli nakloněni, mu vzkázali, aby se tam neodvažoval.
#19:32 Každý křičel něco jiného, neboť v shromáždění byl zmatek a většina nevěděla, proč se vůbec sešli.
#19:33 Židé postrčili dopředu Alexandra a zástup ho nechal vystoupit. Alexandr zamával rukou a chtěl pronést před lidem obhajobu.
#19:34 Když však poznali, že je to Žid, všichni jedním hlasem křičeli asi dvě hodiny: „Veliká je efezská Artemis!“
#19:35 Teprve městský tajemník uklidnil zástup a řekl: „Efezané, kterýpak člověk by nevěděl, že město Efez je strážcem chrámu velké Artemidy a jejího obrazu seslaného z nebe!
#19:36 Protože o tom nemůže být pochyb, musíte zachovat klid a nedělat nic ukvapeného.
#19:37 Přivedli jste tyto lidi, ale oni se nedopustili ani svatokrádeže, ani netupili naši bohyni.
#19:38 Chce-li si Demetrios a řemeslníci na někoho stěžovat, k tomu jsou soudní dny a místodržitelé. Tam ať se soudí.
#19:39 A žádáte-li ještě něco jiného, může se to vyřídit v řádném shromáždění.
#19:40 Vždyť je nebezpečí, že budeme kvůli dnešku obžalováni ze vzpoury. Není tu žádný důvod, kterým bychom mohli obhájit toto srocení.“ Po těch slovech rozpustil shromáždění. 
#20:1 Když ten zmatek ustal, svolal si Pavel učedníky a povzbudil je. Pak se s nimi rozloučil a vydal se na cestu do Makedonie.
#20:2 Prošel tamější krajiny, vytrvale povzbuzoval bratry slovem Božím a přišel do Řecka,
#20:3 kde strávil tři měsíce. Když se chystal vyplout do Sýrie, zosnovali proti němu Židé úklady, a proto se rozhodl vrátit se přes Makedonii.
#20:4 Doprovázel ho Sopatros, Pyrrhův syn z Beroje, Aristarchos a Sekundus z Tesaloniky, Gaius a Timoteus z Derbe a Tychikos a Trofimos z Asie.
#20:5 Ti šli napřed a čekali na nás v Troadě.
#20:6 My jsme po velikonocích vypluli z Filip a přijeli jsme k nim do Troady až za pět dní. Tam jsme zůstali týden.
#20:7 První den v týdnu jsme se sešli k lámání chleba a Pavel promluvil ke shromáždění. Protože chtěl na druhý den odcestovat, protáhl řeč až do půlnoci.
#20:8 Byli jsme shromážděni v horní místnosti, kde bylo mnoho lamp.
#20:9 Nějaký mladík jménem Eutychos seděl na okně, a protože Pavel mluvil dlouho, přemáhal ho spánek. Usnul a spadl z třetího poschodí a když ho zvedli, byl mrtvý.
#20:10 Pavel sešel dolů, sklonil se nad ním, objal ho a řekl: „Upokojte se, je v něm život.“
#20:11 Pak se vrátil nahoru, lámal a jedl chléb, dlouho do rána s nimi rozmlouval a potom odešel.
#20:12 Chlapce přivedli živého a to je velice povzbudilo.
#20:13 My jsme nastoupili na loď napřed a vypluli jsme směrem k Assu, kde se k nám měl připojit Pavel. Tak nám totiž nařídil a sám se rozhodl jít pěšky.
#20:14 Když se s námi v Assu sešel, vzali jsme ho na loď a dopluli do Mitylény.
#20:15 Odtud jsme pluli dál a na druhý den jsme se dostali do blízkosti Chia. Další den jsme připluli k Samu a příští den jsme dorazili do Milétu.
#20:16 Pavel se totiž rozhodl minout Efez a neztrácet čas v provincii Asii, neboť spěchal, aby byl pokud možno na den letnic v Jeruzalémě.
#20:17 Z Milétu poslal Pavel vzkaz do Efezu a zavolal si starší církve.
#20:18 Když k němu přišli, řekl jim: „Vy víte, jak jsem si u vás počínal celou dobu od prvního dne, kdy jsem přišel do Asie.
#20:19 Sloužil jsem Pánu s velkou pokorou, v slzách a zkouškách, které mě potkaly pro úklady židů.
#20:20 Víte, že jsem vám nezamlčel nic, co by vám bylo k prospěchu; všechno jsem vám řekl, když jsem vás učil ve shromáždění i v rodinách.
#20:21 Naléhal jsem na Židy i Řeky a vyzýval je, aby se obrátili k Bohu a uvěřili v našeho Pána, Ježíše Krista.
#20:22 Nyní jdu do Jeruzaléma, protože mě Duch nutí, a nevím, co mě tam potká.
#20:23 Vím jen tolik, že mi Duch svatý město od města ohlašuje, že na mne čekají pouta a utrpení.
#20:24 Ale já nepřikládám svému životu žádnou jinou cenu, než abych dokončil svůj běh a splnil úkol, který jsem dostal od Pána Ježíše: hlásat evangelium o Boží milosti.
#20:25 Nyní vím, že mě už neuvidí nikdo z vás, k nimž jsem na svých cestách přišel hlásat Boží království.
#20:26 Proto vám v tento den prohlašuji před Bohem, že mou vinou nikdo nezahyne,
#20:27 neboť jsem vám oznámil celou Boží vůli a nic jsem nezamlčel.
#20:28 Dávejte pozor na sebe i na celé stádo, ve kterém si vás Duch svatý ustanovil za strážce, abyste byli pastýři Boží církve, kterou si Bůh získal krví vlastního Syna.
#20:29 Vím, že po mém odchodu přijdou mezi vás draví vlci, kteří nebudou šetřit stádo.
#20:30 I mezi vámi samými povstanou lidé, kteří povedou scestné řeči, aby strhli učedníky na svou stranu.
#20:31 Buďte proto bdělí a pamatujte, že jsem se slzami v očích po tři roky ve dne v noci každému z vás neustále ukazoval cestu.
#20:32 Nyní vás svěřuji Bohu a slovu jeho milosti, které má moc vás proměnit a dát vám podíl mezi všemi, kdo jsou posvěceni.
#20:33 Od nikoho jsem nežádal stříbro, zlato ani oděv.
#20:34 Sami víte, že tyto mé ruce vydělávaly na všechno, co jsem potřeboval já i moji společníci.
#20:35 Tím vším jsem vám ukázal, že máme takto pracovat, pomáhat slabým a mít na paměti slova Pána Ježíše, který řekl: ‚Blaze tomu, kdo dává, ne tomu, kdo bere.‘“
#20:36 Po těch slovech si s nimi se všemi klekl a pomodlil se.
#20:37 Všichni se dali do hlasitého pláče, objímali Pavla a líbali ho,
#20:38 dojati nejvíce jeho slovy, že ho už nikdy neuvidí. Pak ho doprovodili k lodi. 
#21:1 Když nastal čas k odjezdu, rozloučili jsme se s nimi. Přímou cestou jsme dopluli na Kós, druhý den na Rodos a odtud jsme pluli do Patary.
#21:2 Tam jsme našli loď, která plula do Fénicie. Nasedli jsme na ni a vypluli.
#21:3 Přiblížili jsme se na dohled ke Kypru, ale nechali jsme ho vlevo a mířili k Sýrii. Přistáli jsme v Týru, kde měla loď vyložit náklad.
#21:4 Vyhledali jsme učedníky a zůstali jsme tam sedm dní. Ti z vnuknutí Ducha říkali Pavlovi, aby nechodil do Jeruzaléma.
#21:5 Když ty dny uplynuly, vydali jsme se zase na cestu. Všichni, i ženy a děti, nás vyprovázeli až za město. Na břehu jsme si klekli a pomodlili se.
#21:6 Pak jsme se rozloučili. My jsme nastoupili na loď a oni se vrátili domů.
#21:7 Potom jsme pokračovali v plavbě; z Týru jsme dorazili do Ptolemaidy. Tam jsme pozdravili bratry a zůstali u nich jeden den.
#21:8 Druhý den jsme se vydali na cestu a přišli do Cesareje. Tam jsme navštívili Filipa, kazatele evangelia a jednoho ze sedmi jáhnů, a ubytovali se u něho.
#21:9 Ten měl čtyři neprovdané dcery, prorokyně.
#21:10 Když jsme tam byli několik dní, přišel z Judska prorok, jménem Agabos.
#21:11 Přišel k nám, vzal Pavlův opasek, svázal si jím nohy i ruce a řekl: „Toto praví Duch svatý: Muže, kterému patří tento opasek, židé v Jeruzalémě takto svážou a vydají pohanům.“
#21:12 Když jsme to uslyšeli, prosili jsme my i tamější bratří Pavla, aby do Jeruzaléma nechodil.
#21:13 Ale on odpověděl: „Proč pláčete a působíte mi tím větší bolest? Vždyť já jsem připraven nejen nechat se svázat, nýbrž i zemřít v Jeruzalémě pro jméno Pána Ježíše!“
#21:14 Protože se nedal přemluvit, přestali jsme naléhat a řekli jsme: „Děj se vůle Páně!“
#21:15 Potom jsme se připravili na cestu a vydali se do Jeruzaléma.
#21:16 S námi šli i někteří učedníci z Cesareje a vedli nás k Mnasonovi z Kypru, jednomu z prvních učedníků. U něho jsme se měli ubytovat.
#21:17 Když jsme přišli do Jeruzaléma, přijali nás bratří s radostí.
#21:18 Nazítří se Pavel odebral i s námi k Jakubovi, kde se sešli všichni starší.
#21:19 Pavel je pozdravil a vyprávěl jim podrobně všechno, co Bůh jeho působením vykonal mezi pohany.
#21:20 Když to vyslechli, chválili Boha. Potom řekli Pavlovi: „Pohleď, bratře, kolik tisíc židů uvěřilo v Krista, a všichni jsou nadšenými zastánci Zákona.
#21:21 O tobě se doslechli, že prý učíš všechny židy, žijící mezi pohany, aby odpadli od Mojžíše: aby přestali obřezávat své syny a žít podle otcovských zvyků.
#21:22 Co teď? V každém případě se dovědí, žes přišel.
#21:23 Udělej tedy, co ti radíme: Jsou tu čtyři muži, kteří na sebe vzali slib. Připoj se k nim,
#21:24 podrob se s nimi očistnému obřadu a zaplať za ně, co je předepsáno; pak si budou moci dát ostříhat hlavu. Tak poznají všichni, že není ani trochu pravdy na tom, co se o tobě říká, a že naopak sám jsi věrný Zákonu a žiješ podle něho.
#21:25 Pokud jde o pohany, kteří přijali víru, těm jsme písemně oznámili své rozhodnutí, že nemají jíst pokrmy, obětované modlám, ani krev, ani maso zvířat nezbavených krve, a že se mají vyvarovat smilstva.“
#21:26 Nato Pavel vzal ty muže s sebou a na druhý den se s nimi dal zasvětit. Pak šel do chrámu a oznámil, kdy se skončí očistné dny a za každého z nich bude přinesena oběť.
#21:27 Když se těch sedm dní chýlilo ke konci, uviděli ho v chrámě židé z provincie Asie. Pobouřili celý dav, zmocnili se Pavla
#21:28 a křičeli: „Izraelci, pojďte sem! To je ten člověk, který všude všechny učí proti vyvolenému lidu, proti Zákonu a proti chrámu. A teď ještě přivedl do chrámu pohany a znesvětil toto svaté místo.“
#21:29 Viděli totiž předtím Trofima z Efezu s Pavlem ve městě a domnívali se, že ho Pavel přivedl do chrámu.
#21:30 V celém městě nastal rozruch a lid se začal sbíhat. Chytili Pavla a vlekli ho ven z chrámu. A hned byly zavřeny brány.
#21:31 Ve chvíli, kdy ho už chtěli zabít, došlo hlášení veliteli praporu, že celý Jeruzalém se bouří.
#21:32 Ten vzal okamžitě vojáky i důstojníky a běžel dolů k davu. Jakmile uviděli velitele a vojáky, přestali Pavla bít.
#21:33 Velitel přistoupil k Pavlovi, zatkl ho a rozkázal vojákům, aby ho spoutali dvěma řetězy. Potom vyšetřoval, kdo je a co udělal.
#21:34 V davu křičel každý něco jiného. Protože se velitel nemohl pro zmatek dovědět nic jistého, rozkázal odvést Pavla do pevnosti.
#21:35 Když se dostali ke schodišti, museli ho vojáci pro násilí davu nést,
#21:36 neboť celé množství lidu šlo za ním a křičeli: „Pryč s ním!“
#21:37 Než ho dovedli do pevnosti, řekl Pavel veliteli: „Smím ti něco říci?“ On se podivil: „Ty umíš řecky?
#21:38 Nejsi ten Egypťan, který nedávno podnítil ke vzpouře a vyvedl na poušť čtyři tisíce vzbouřenců?“
#21:39 Pavel odpověděl: „Já jsem Žid z Tarsu v Kilikii, občan ne bezvýznamného města. Prosím tě, dovol mi promluvit k lidu.“
#21:40 Velitel mu to dovolil. Pavel, jak stál na schodišti, dal lidu znamení rukou. Nastalo napjaté ticho a Pavel je oslovil hebrejsky: 
#22:1 „Bratří a otcové, vyslechněte, co vám chci nyní říci na svou obhajobu.“
#22:2 Když uslyšeli, že k nim mluví hebrejsky, úplně se uklidnili a Pavel pokračoval:
#22:3 „Já jsem Žid a narodil jsem se v Tarsu v Kilikii, ale vychován jsem byl zde v Jeruzalémě. V Gamalielově škole jsem byl přesně vyučen zákonu našich otců. Byl jsem právě tak plný horlivosti pro Boha, jako jste dnes vy všichni,
#22:4 a víru v Ježíše Krista jsem pronásledoval až na smrt, muže i ženy jsem dával spoutat a uvěznit,
#22:5 jak mi může dosvědčit velekněz a celý sbor starších. Od nich jsem dostal doporučující dopisy pro židovské souvěrce v Damašku a vydal jsem se tam, abych tamější stoupence nové víry přivedl v poutech do Jeruzaléma a dal je potrestat.
#22:6 Když jsem ještě byl na cestě a blížil se k Damašku, kolem poledne mě najednou obklopilo jasné světlo z nebe.
#22:7 Padl jsem na zem a uslyšel jsem hlas, který mi pravil: ‚Saule, Saule, proč mě pronásleduješ?‘
#22:8 Já jsem odpověděl: ‚Kdo jsi, Pane?‘ A on mi řekl: ‚Já jsem Ježíš Nazaretský, kterého ty pronásleduješ.‘
#22:9 Moji průvodci viděli sice světlo, ale neslyšeli hlas toho, kdo ke mně mluvil.
#22:10 Řekl jsem: ‚Co mám dělat, Pane?‘ A Pán mi řekl: ‚Vstaň a pokračuj v cestě do Damašku; tam ti bude řečeno všechno, co ti Bůh ukládá.‘
#22:11 Protože jsem byl oslepen jasem toho světla, museli mě moji druhové vzít za ruku a tak mě dovedli do Damašku.
#22:12 Jeden muž, zbožný podle Božího zákona, jménem Ananiáš, který měl dobrou pověst u všech místních židů,
#22:13 přišel za mnou, přistoupil ke mně a řekl: ‚Bratře Saule, otevři oči!‘ A já jsem v tu chvíli nabyl zraku.
#22:14 On mi řekl: ‚Bůh našich otců si tě vyvolil, abys poznal jeho vůli, spatřil jeho Spravedlivého a slyšel hlas z jeho úst.
#22:15 Budeš jeho svědkem před všemi lidmi a budeš mluvit o tom, co jsi viděl a slyšel.
#22:16 Nuže neváhej! Vstaň, vzývej jeho jméno a dej se pokřtít, abys byl obmyt ze svých hříchů.‘
#22:17 Když jsem se potom vrátil do Jeruzaléma a modlil se v chrámě, upadl jsem do vytržení mysli.
#22:18 Spatřil jsem Pána, jak mi říká: ‚Pospěš si a rychle odejdi z Jeruzaléma, protože nepřijmou tvé svědectví o mně.‘
#22:19 Já jsem odpověděl: ‚Pane, oni vědí, že jsem dával uvěznit a bičovat v synagógách ty, kdo v tebe věří.
#22:20 Když byla prolévána krev tvého svědka Štěpána, byl jsem při tom, schvaloval jsem to a hlídal jsem šaty těch, kdo ho kamenovali.‘
#22:21 Ale Pán mi řekl: ‚Jdi, neboť já tě chci poslat daleko k pohanům!‘“
#22:22 Poslouchali ho až do chvíle, kdy řekl tato slova; ale pak začali křičet: „Sprovoď ho ze světa! Nesmí zůstat na živu!“
#22:23 Protože křičeli, strhávali ze sebe šaty a házeli do vzduchu prach,
#22:24 rozkázal velitel vojákům, aby Pavla zavedli do pevnosti; dal jim pokyn, aby při výslechu použili bičování, neboť chtěl zjistit, proč proti němu tolik běsní.
#22:25 Když ho přivázali, řekl Pavel důstojníkovi, který měl službu: „Smíte bičovat římského občana, a to bez soudu?“
#22:26 Když to důstojník uslyšel, šel k veliteli a hlásil mu to. Řekl mu: „Co chceš dělat? Ten člověk je římský občan!“
#22:27 Velitel přišel k Pavlovi a ptal se ho: „Řekni mi, jsi římský občan?“ On odpověděl: „Ano.“
#22:28 Velitel mu na to řekl: „Já jsem získal toto občanství za veliké peníze.“ Ale Pavel prohlásil: „Já jsem se jako římský občan už narodil.“
#22:29 Ihned od něho odstoupili ti, co ho měli vyslýchat. A velitel dostal strach, když se dověděl, že Pavel je římský občan, a on ho dal spoutat.
#22:30 Na druhý den, aby bezpečně zjistil, z čeho židé Pavla obviňují, dal ho vyvést z vězení a nařídil, aby se shromáždili velekněží a celá rada. 
#23:1 Pak dal přivést Pavla a postavil ho před ně. Pavel upřel zrak na jejich shromáždění a řekl: „Bratří, až do dneška žiju se zcela dobrým svědomím před Bohem.“
#23:2 Tu nařídil velekněz Ananiáš těm, kdo stáli u Pavla, aby ho udeřili přes ústa.
#23:3 Na to se Pavel k němu obrátil a řekl: „Tebe bude bít Bůh, ty obílená stěno. Sedíš zde, abys mě soudil podle Zákona, a proti Zákonu rozkazuješ, aby mě bili?“
#23:4 Ti, co stáli u Pavla, řekli: „Troufáš si urážet Božího velekněze?“
#23:5 Pavel odpověděl: „Nevěděl jsem, bratří, že je to velekněz. Vím, že je psáno: ‚Nebudeš tupit vládce svého lidu.‘“
#23:6 Protože Pavel věděl, že jedna část rady patří k saduceům a druhá k farizeům, zvolal: „Bratří, já jsem farizeus, syn farizeův. Jsem souzen pro naději ve zmrtvýchvstání.“
#23:7 Když to řekl, vznikl spor mezi farizeji a saduceji a shromáždění se rozdvojilo.
#23:8 Saduceové totiž říkají, že není zmrtvýchvstání a že nejsou andělé a duchové, kdežto farizeové vyznávají obojí.
#23:9 Nastal velký křik a někteří zákoníci z farizejské strany vstali a začali namítat: „Nic zlého na tom člověku nenacházíme. Co když k němu mluvil duch nebo anděl?“
#23:10 Hádka byla stále prudší a velitel se bál, že Pavla rozsápou. Proto povolal vojenský oddíl, aby Pavla vyrval z jejich středu a zavedl ho do pevnosti.
#23:11 Následující noc stanul před Pavlem Pán a řekl: „Neztrácej odvahu! Jako jsi svědčil o mně v Jeruzalémě, tak musíš svědčit i v Římě.“
#23:12 Když nastal den, židé se spolčili a zapřisáhli, že nebudou jíst ani pít, dokud Pavla nezabijí.
#23:13 Bylo jich více než čtyřicet, kdo se takto spikli.
#23:14 Ti šli k velekněžím a starším a řekli: „Zapřisáhli jsme se, že nic nevezmeme do úst, dokud Pavla nezabijeme.
#23:15 Vy teď spolu s radou požádejte velitele, aby dal Pavla přivést před vás, pod záminkou, že chcete důkladněji vyšetřit jeho případ. A my jsme připraveni zabít ho dříve, než přijde na místo.“
#23:16 O těch úkladech se doslechl syn Pavlovy sestry. Šel do pevnosti a oznámil to Pavlovi.
#23:17 Pavel zavolal jednoho z důstojníků a řekl mu: „Zaveď tohoto mladíka k veliteli, má pro něho zprávu.“
#23:18 Ten ho vzal s sebou, dovedl k veliteli a řekl mu: „Zavolal mě vězeň Pavel a požádal mě, abych tohoto mladíka dovedl k tobě, protože ti chce něco sdělit.“
#23:19 Velitel ho vzal za ruku, odešel s ním stranou a zeptal se: „Co mi chceš oznámit?“
#23:20 On mu řekl: „Židé se domluvili, že tě požádají, abys dal zítra Pavla dovést před radu, která bude předstírat, že chce jeho případ důkladněji vyšetřit.
#23:21 Ale ty jim nevěř, neboť na něho číhá více než čtyřicet mužů, kteří se zavázali přísahou, že nebudou jíst ani pít, dokud ho nezabijí, a nyní jsou připraveni a čekají jen na tvé rozhodnutí.“
#23:22 Velitel mladíka propustil a přikázal mu: „Nikomu nevyzraď, žes mi to oznámil.“
#23:23 Pak zavolal dva ze svých důstojníků a dal jim rozkaz: „Připravte na devátou hodinu večer dvě stě pěšáků pro cestu do Cesareje, dále sedmdesát jezdců a dvě stě lehkooděnců.
#23:24 Ať přichystají mezky a Pavla bezpečně dopraví k místodržiteli Félixovi.“
#23:25 Současně napsal dopis tohoto znění:
#23:26 „Klaudios Lysias zdraví vznešeného místodržitele Félixe.
#23:27 Muže, kterého ti posílám, se zmocnili Židé a chtěli ho zabít. Když jsem se dověděl, že je to římský občan, zasáhl jsem s vojenským oddílem a vysvobodil ho.
#23:28 Chtěl jsem přesně zjistit, z čeho jej obviňují, a proto jsem ho dal přivést před jejich radu.
#23:29 Shledal jsem, že se žaloba týká sporných otázek jejich zákona, ale že nejde o žádný zločin, který by zasluhoval smrt nebo vězení.
#23:30 Protože jsem se dověděl, že proti tomuto muži chystají úklady, posílám ho ihned k tobě a také žalobcům jsem nařídil, aby ho žalovali před tvým soudem.“
#23:31 Vojáci tedy podle rozkazu vzali Pavla a během noci ho dopravili do Antipatridy.
#23:32 Na druhý den nechali jezdce pokračovat v cestě s Pavlem a sami se vrátili do pevnosti.
#23:33 Jezdci dojeli do Cesareje, odevzdali dopis místodržiteli a předali mu i Pavla.
#23:34 Místodržitel přečetl dopis a zeptal se Pavla, z které provincie pochází. Když se dověděl, že z Kilikie,
#23:35 řekl: „Vyslechnu tě, jakmile se dostaví tvoji žalobci.“ A dal rozkaz, aby byl hlídán v Herodově paláci. 
#24:1 Za pět dní přišel velekněz Ananiáš s několika staršími a s nějakým Tertullem právníkem, a dostavili se k místodržiteli, aby na Pavla podali žalobu. Když byl Pavel předvolán,
#24:2 Tertullus začal obžalobu: „Vznešený Félixi, tvou zásluhou se těšíme dokonalému míru a tvou moudrou péčí spěje všechno u tohoto národa k lepšímu.
#24:3 Vždy a všude to přijímáme s velikou vděčností.
#24:4 Ale abych tě už déle nezdržoval, prosím tě, abys byl tak laskav a krátce nás vyslechl.
#24:5 Shledali jsme, že tento člověk jako morová nákaza po celém světě vyvolává nepokoje mezi Židy a je hlavou nazorejské sekty.
#24:6 Pokusil se i znesvětit chrám a my jsme ho zadrželi.
#24:7 ---
#24:8 Až ho budeš vyslýchat, můžeš se sám přesvědčit o tom všem, co na něj žalujeme.“
#24:9 K jeho řeči se připojili i Židé a tvrdili, že je tomu tak.
#24:10 Když ho místodržitel vyzval k obhajobě, Pavel promluvil: „Vím, že už mnoho let vykonáváš úřad soudce v tomto národě, a proto s klidnou myslí přednáším svou obhajobu.
#24:11 Můžeš si zjistit, že tomu není víc než dvanáct dní, co jsem přišel do Jeruzaléma, abych se poklonil Bohu,
#24:12 a ani v chrámu, ani v synagógách, ani jinde ve městě mě nikdo nepřistihl, že bych se s někým přel anebo dokonce podněcoval vzpouru davu.
#24:13 A nemohou dokázat před tebou nic, pro co mě nyní žalují.
#24:14 Přiznávám se ti však k tomu, že podle směru, který oni označují za sektu, sloužím Bohu svých předků: Věřím všemu, co je napsáno v zákoně Mojžíšově a v prorockých knihách,
#24:15 a tak jako oni mám naději v Bohu, že jednou spravedliví i nespravedliví vstanou k soudu.
#24:16 Proto i já se vždy snažím zachovat neporušené svědomí před Bohem i lidmi.
#24:17 Po mnoha letech jsem se vrátil, abych svému národu odevzdal peněžitou podporu a přinesl oběti v chrámě.
#24:18 A při nich mě tam po očistném obřadu spatřili. Nedošlo při tom ani k shluku, ani k výtržnostem.
#24:19 Byli tam ovšem někteří Židé z provincie Asie a ti by měli být přítomni před tebou a přednést žalobu, mají-li něco proti mně.
#24:20 Anebo tito přítomní ať řeknou, jaký zločin na mně našli, když jsem stál před veleradou,
#24:21 kromě toho, že jsem mezi nimi zvolal: ‚Pro víru ve zmrtvýchvstání jsem dnes od vás souzen.‘“
#24:22 Pak Félix, ač dobře znal všecko o tom učení, odročil přelíčení a řekl: „Jakmile přijde velitel Lysias, rozhodnu ve vaší při.“
#24:23 Nařídil důstojníkovi, aby Pavla držel v mírné vazbě a nikomu z jeho přátel nebránil, kdyby mu chtěl posloužit.
#24:24 Po několika dnech přišel Félix se svou manželkou Drusillou, která byla židovka, dal si zavolat Pavla a poslouchal jeho výklad o víře v Krista Ježíše.
#24:25 Když však Pavel začal hovořit o spravedlnosti a zdrženlivosti a o budoucím soudu, pocítil Félix úzkost a řekl: „Pro dnešek můžeš jít, až budu mít čas, dám si tě zase zavolat.“
#24:26 Zároveň doufal, že dostane od Pavla peníze. Proto si ho také dal častěji zavolat a hovořil s ním.
#24:27 Po dvou letech se stal jeho nástupcem Porcius Festus. Ale Félix zanechal Pavla ve vazbě, protože se chtěl zavděčit Židům. 
#25:1 Za tři dny po svém příchodu do provincie odebral se Festus z Cesareje do Jeruzaléma.
#25:2 Tam se k němu dostavili velekněží a přední Židé se žalobou proti Pavlovi. Žádali ho,
#25:3 aby jim prokázal laskavost a dal Pavla přivést do Jeruzaléma. Ale byla to léčka, protože chtěli Pavla na cestě zabít.
#25:4 Festus jim odpověděl, že Pavel je ve vazbě v Cesareji a on sám že také brzo odcestuje.
#25:5 A řekl: „Ať tedy vaši zástupci tam jdou se mnou a podají na toho člověka žalobu, jestliže se něčeho dopustil.“
#25:6 Festus se mezi nimi nezdržel víc než osm nebo deset dní a vrátil se do Cesareje. Na druhý den zasedl k soudu a rozkázal přivést Pavla.
#25:7 Když ho přivedli, postavili se proti němu Židé, kteří přišli z Jeruzaléma, a vznášeli mnoho těžkých obvinění; nemohli je však nijak dokázat.
#25:8 Pavel se hájil takto: „Neprovinil jsem se ničím ani proti židovskému zákonu, ani proti chrámu, ani proti císaři.“
#25:9 Festus se chtěl zavděčit Židům a odpověděl Pavlovi: „Chceš jít do Jeruzaléma a tam být přede mnou souzen z těchto obvinění?“
#25:10 Ale Pavel řekl: „Stojím před císařským soudem, kde mám být souzen. Proti Židům jsem se v ničem neprovinil, jak i ty velmi dobře víš.
#25:11 Jestliže jsem vinen a spáchal jsem něco, za co si zasloužím smrt, nezdráhám se zemřít. Není-li však pravda, z čeho mě tito žalobci obviňují, nikdo mě jim nemůže vydat. Odvolávám se k císaři.“
#25:12 Nato Festus promluvil se svými rádci a pak odpověděl: „K císaři ses odvolal, k císaři půjdeš.“
#25:13 Uplynulo několik dní a do Cesareje přišli král Agrippa a Bereniké, aby pozdravili Festa.
#25:14 Protože se tam zdrželi nějakou dobu, předložil Festus králi Pavlův případ. Řekl: „Je zde muž, kterého Félix zanechal ve vazbě.
#25:15 Když jsem přišel do Jeruzaléma, obrátili se na mne velekněží a židovští starší se žalobou proti němu a žádali jeho odsouzení.
#25:16 Odpověděl jsem jim, že Římané nemají ve zvyku odsoudit člověka, dokud není postaven před žalobce a nedostane možnost hájit se proti jejich obvinění.
#25:17 Když sem přišli, bez odkladu jsem hned druhý den zasedl k soudu a rozkázal jsem přivést toho muže.
#25:18 Žalobci vystoupili proti němu, ale neobviňovali ho ze žádných zločinů, jak jsem očekával,
#25:19 nýbrž měli s ním jen nějaké spory o své náboženství a o nějakého zemřelého Ježíše, o němž Pavel tvrdil, že je živ.
#25:20 Protože se v těchto otázkách nevyznám, navrhoval jsem, zda by se nechtěl odebrat do Jeruzaléma a tam být v té věci souzen.
#25:21 Avšak Pavel podal odvolání a žádal, aby byl ponechán ve vazbě až do rozhodnutí císařova; proto jsem dal rozkaz, aby zůstal ve vězení, než ho pošlu k císaři.“
#25:22 Agrippa řekl Festovi: „Také já bych rád slyšel toho člověka.“ Festus odpověděl: „Zítra ho budeš slyšet.“
#25:23 Na druhý den tedy přišli Agrippa a Bereniké s velkou okázalostí a vstoupili spolu s vysokými důstojníky a předními muži města do přijímací síně. Na Festův rozkaz byl přiveden Pavel.
#25:24 Festus pak promluvil: „Králi Agrippo a všichni zde přítomní, vidíte toho muže, na něhož si u mne všichni Židé jak v Jeruzalémě, tak zde stěžovali a křičeli, že nesmí zůstat na živu.
#25:25 Já jsem zjistil, že nespáchal nic, zač by zasluhoval smrt. Protože on sám se odvolal k císaři, rozhodl jsem, že ho k němu pošlu.
#25:26 Nemám však nic určitého, co bych o něm císaři napsal. Proto jsem ho dal předvést před vás a zvláště před tebe, králi Agrippo, abyste ho vyslechli a já se dověděl, co mám napsat.
#25:27 Zdá se mi totiž nerozumné poslat vězně a neuvést zároveň obvinění vznesená proti němu.“ 
#26:1 Agrippa řekl Pavlovi: „Dovoluje se ti, aby ses hájil.“ Nato se Pavel ujal slova a začal obhajobu:
#26:2 „Pokládám se za šťastného, králi Agrippo, že se smím dnes před tebou hájit proti všemu, z čeho mě Židé obviňují,
#26:3 zvláště proto, že dobře znáš všecky židovské obyčeje i sporné otázky. Prosím tedy, abys mě trpělivě vyslechl:
#26:4 Můj život od mládí znají všichni Židé; vědí, jak jsem od počátku žil mezi svým lidem, a to i v Jeruzalémě.
#26:5 Znají mě už od dřívějška, a kdyby chtěli, mohou dosvědčit, že jsem žil podle nejpřísnějšího směru našeho náboženství jako farizeus.
#26:6 Nyní stojím před soudem, protože trvám na zaslíbení, které dal Bůh našim otcům.
#26:7 Dvanáct kmenů našeho národa doufá, že tohoto zaslíbení dosáhne, a slouží horlivě Bohu dnem i nocí. Pro tuto naději jsem, králi, obžalován od Židů.
#26:8 Což se vám zdá neuvěřitelné, že Bůh křísí mrtvé?
#26:9 Také já jsem se ovšem domníval, že musím všemožně bojovat proti jménu Ježíše Nazaretského.
#26:10 A to jsem v Jeruzalémě vskutku dělal. Když jsem k tomu dostal od velekněží plnou moc, dal jsem mnoho věřících uvěznit, a když měli být usmrceni, schvaloval jsem to.
#26:11 Po všech synagógách jsem je často dával trestat a nutil je, aby se rouhali. Jako smyslů zbavený jsem se chystal pronásledovat je i v cizích městech.
#26:12 A tak jsem se vypravil s plnou mocí a pověřením velekněží do Damašku.
#26:13 Když jsem byl na cestě, králi, spatřil jsem o poledni světlo z nebe, jasnější než slunce. Jeho záře obklopila mne i ty, kdo šli se mnou.
#26:14 Všichni jsme padli na zem a já jsem uslyšel hlas, který ke mně mluvil hebrejsky: ‚Saule, Saule, proč mě pronásleduješ? Marně se vzpínáš proti bodcům.‘
#26:15 Já jsem se zeptal: ‚Kdo jsi, Pane?‘ A Pán řekl: ‚Já jsem Ježíš, kterého ty pronásleduješ.
#26:16 Ale zvedni se, povstaň. Neboť proto jsem se ti zjevil, abych tě učinil svým služebníkem a svědkem toho, co jsi spatřil a co ti ještě dám poznat.
#26:17 Budu tě chránit před izraelským národem i před pohany, k nimž tě posílám,
#26:18 abys otevřel jejich oči a oni se obrátili od tmy ke světlu, od moci satanovy k Bohu, a vírou ve mne dosáhli odpuštění hříchů a podílu mezi posvěcenými.‘
#26:19 A tak jsem se, králi Agrippo, nemohl postavit proti nebeskému vidění,
#26:20 a kázal jsem nejprve obyvatelům Damašku a Jeruzaléma a po celé judské zemi, a pak i pohanům, aby se v pokání obrátili k Bohu a podle toho žili.
#26:21 Z toho důvodu se mne Židé zmocnili, když jsem byl v chrámě, a chtěli mě zabít.
#26:22 Ale Bůh mi pomáhal až do dnešního dne, a tak zde stojím jako svědek před velkými i malými. Neříkám nic než to, co předpověděli proroci i Mojžíš:
#26:23 Mesiáš má trpět, jako prvý vstane z mrtvých a bude zvěstovat světlo lidu izraelskému i pohanům.“
#26:24 Když se Pavel takto hájil, zvolal Festus. „Jsi blázen, Pavle, mnoho vědomostí tě připravuje o rozum!“
#26:25 Pavel odpověděl: „Nejsem blázen, vznešený Feste, nýbrž mluvím slova pravdivá a rozumná.
#26:26 Vždyť král to všechno zná, a proto také k němu mluvím otevřeně. Jsem přesvědčen, že mu nic z toho není neznámo. Vždyť se to neudálo někde v ústraní.
#26:27 Věříš, králi Agrippo, prorokům? Vím, že věříš.“
#26:28 Agrippa odpověděl Pavlovi: „Málem bys mě přesvědčil, abych se stal křesťanem.“
#26:29 A Pavel řekl: „Kéž by dal Bůh, aby ses nejen ty, nýbrž i všichni, kdo mě slyší, stali dříve nebo později tím, čím jsem já - kromě těchto pout.“
#26:30 Král povstal a s ním i místodržitel, Bereniké i všichni ostatní.
#26:31 Když odcházeli, hovořili mezi sebou: „Tento člověk nedělá nic, za co by zasluhoval smrt nebo vězení.“
#26:32 A Agrippa řekl Festovi: „Mohl být osvobozen, kdyby se nebyl odvolal k císaři.“ 
#27:1 Jakmile bylo rozhodnuto, že pojedeme po moři do Itálie, odevzdali Pavla a některé jiné vězně důstojníkovi, který se jmenoval Julius a byl od císařského praporu.
#27:2 Nastoupili jsme na loď z Adramytteia, která měla plout do přístavů v provincii Asii, a vypluli jsme. Byl s námi Makedoňan Aristarchos z Tesaloniky.
#27:3 Na druhý den jsme přistáli v Sidónu. Julius zacházel s Pavlem laskavě a dovolil mu, aby navštívil své přátele a přijal jejich pohostinství.
#27:4 Odtud jsme vypluli a plavili jsme se chráněni Kyprem, protože vítr vál proti nám.
#27:5 Propluli jsme mořem podél Kilikie a Pamfylie a přijeli jsme do Myry v Lykii.
#27:6 Tam našel důstojník loď z Alexandrie, která plula do Itálie, a nalodil nás na ni.
#27:7 Mnoho dní jsme pluli pomalu a stěží jsme se dostali do míst naproti Knidu. Protože nám vítr bránil, pluli jsme kolem Salmóny chráněni Krétou.
#27:8 S obtížemi jsme se plavili podél pobřeží, až jsme dopluli do místa, které se jmenovalo Dobré přístavy, nedaleko města Lasaia.
#27:9 Protože jsme ztratili mnoho dní a plavba byla nebezpečná, neboť již minul čas postu, Pavel je varoval:
#27:10 „Mužové, vidím, že plavba bude nejen spojena s nebezpečím a velkou škodou pro náklad a loď, nýbrž ohrozí i naše životy.“
#27:11 Ale důstojník věřil více kormidelníkovi a majiteli lodi než tomu, co říkal Pavel.
#27:12 Protože přístav nebyl vhodný k přezimování, většina se rozhodla plout odtud dále, dostat se - bude-li to možné - do Foiniku a tam zůstat přes zimu. Je to přístav na Krétě, otevřený k jihozápadu a severozápadu.
#27:13 Když začal vát slabý jižní vítr, domnívali se, že mohou provést svůj záměr. Zvedli kotvu a pluli těsně podél Kréty.
#27:14 Ale zanedlouho se přihnal z Kréty bouřlivý vítr od severovýchodu
#27:15 a opřel se do lodi tak, že ji nemohli ovládat. Nechali jsme se jím tedy unášet.
#27:16 Když jsme se dostali do závětří ostrůvku, který se jmenoval Kauda, museli jsme vynaložit veliké úsilí, abychom vytáhli záchranný člun na palubu.
#27:17 Pak lodníci zabezpečili loď tím, že ji převázali. Báli se, aby nenajeli na Syrtskou mělčinu, a tak stáhli plachty a nechali se unášet větrem.
#27:18 Protože jsme byli prudce zmítáni bouří, lodníci druhý den vyhazovali do moře náklad, aby lodi ulehčili;
#27:19 a třetí den vlastníma rukama hodili do moře lodní výstroj.
#27:20 Po mnoho dní se neukázalo ani slunce ani hvězdy. Prudká bouře nepřestávala a již jsme ztráceli všechnu naději, že se zachráníme.
#27:21 Když už nikdo neměl ani pomyšlení na jídlo, šel Pavel mezi lodníky a řekl jim: „Měli jste mě poslechnout a neopouštět Krétu, a mohli jste si ušetřit toto nebezpečí a škodu.
#27:22 Ale teď vás vyzývám, abyste neztráceli naději, neboť nikdo z vás nepřijde o život, jenom loď vezme za své.
#27:23 Dnes v noci ke mně přišel anděl od Boha, kterému patřím a kterému sloužím,
#27:24 a řekl mi: ‚Neboj se, Pavle, ty se před císaře dostaneš. A Bůh ti daroval všechny, kdo jsou s tebou na lodi.‘
#27:25 Buďte proto dobré mysli. Věřím Bohu, že tomu bude tak, jak mi oznámil.
#27:26 Máme se dostat k nějakému ostrovu.“
#27:27 Když jsme byli hnáni po Adriatickém moři již čtrnáctou noc, kolem půlnoci měli lodníci dojem, že je nablízku nějaká země.
#27:28 Spustili olovnici a naměřili hloubku dvacet sáhů. Kousek dále ji spustili znovu a naměřili patnáct sáhů.
#27:29 Báli se, abychom nenajeli na nějaká skaliska, a proto spustili ze zádi čtyři kotvy; toužebně čekali, až se rozední.
#27:30 Když se lodníci pokoušeli utéci z lodi a spustili záchranný člun na hladinu pod záminkou, že chtějí spustit kotvy také z přídi,
#27:31 řekl Pavel důstojníkovi a vojákům: „Nezůstanou-li oni na lodi, nemáte ani vy naději na záchranu.“
#27:32 Tu vojáci přesekli lana u člunu a nechali ho uplavat.
#27:33 Než se začalo rozednívat, vybízel Pavel všechny, aby pojedli. Pravil: „Už čtrnáct dní čekáte na záchranu, nic nejíte a jste o hladu.
#27:34 Proto vás vybízím, abyste pojedli. Je to potřebí k vaší záchraně. Neboť nikdo z vás nepřijde ani o vlas na hlavě.“
#27:35 Když to řekl, vzal chléb, vzdal přede všemi díky Bohu, lámal jej a začal jíst.
#27:36 Všichni nabyli dobré mysli a přijali pokrm.
#27:37 Bylo nás na lodi celkem dvě stě sedmdesát šest.
#27:38 Když se nasytili, vyhazovali obilí do moře, aby lodi odlehčili.
#27:39 Když však nastal den, nemohli poznat, která země je před nimi. Viděli jen nějaký záliv s plochým pobřežím a rozhodli se, že u něho s lodí přistanou, bude-li to možné.
#27:40 Odsekli kotvy a nechali je v moři a zároveň rozvázali provazy u kormidel. Pak nastavili přední plachtu větru a udržovali loď ve směru k břehu.
#27:41 Najeli však na mělčinu a uvízli s lodí. Příď se zabořila a nemohla se pohnout a záď se bortila pod náporem vlnobití.
#27:42 Vojáci chtěli vězně zabít, aby některý neutekl, až doplave na břeh.
#27:43 Ale důstojník chtěl zachránit Pavla, a proto jim v jejich úmyslu zabránil. Pak dal rozkaz, aby ti, kdo umějí plavat, první skočili do vody a plavali k zemi
#27:44 a ostatní aby se zachránili na prknech nebo troskách lodi. Tak se všichni dostali na břeh živi a zdrávi. 
#28:1 Když jsme se zachránili, dověděli jsme se, že se ten ostrov jmenuje Malta.
#28:2 Domorodci se k nám zachovali neobyčejně laskavě. Zapálili hranici dříví a všechny nás k ní pozvali, protože začalo pršet a bylo zima.
#28:3 Když Pavel nasbíral náruč chrastí a přiložil na oheň, zakousla se mu do ruky zmije, která prchala před žárem.
#28:4 Jakmile domorodci uviděli, že mu visí na ruce had, říkali si mezi sebou: „Ten člověk je určitě vrah. I když se zachránil z moře, bohyně odplaty nedovolila, aby zůstal na živu.“
#28:5 Ale Pavel setřásl hada do ohně a nic zlého se mu nestalo.
#28:6 Oni čekali, že oteče nebo že najednou padne mrtev. Když to však dlouho trvalo a viděli, že se s ním nic neděje, začali naopak říkat, že je to nějaký bůh.
#28:7 Blízko toho místa měl své statky správce ostrova Publius. Ten nás přijal a tři dni se o nás laskavě staral jako o své hosty.
#28:8 Publiův otec ležel s horečkou a úplavicí. Pavel přišel k němu, pomodlil se, vložil na něj ruce a uzdravil ho.
#28:9 A potom začali přicházet i jiní nemocní z ostrova a Pavel je uzdravoval.
#28:10 Štědře nás obdarovali, a když jsme měli odjíždět, přinesli nám, co jsme potřebovali na cestu.
#28:11 Po třech měsících jsme vypluli na alexandrijské lodi, která přezimovala na ostrově a měla jako znak Blížence.
#28:12 Dopluli jsme do Syrakus a zůstali tam tři dny.
#28:13 Odtud jsme pluli podél pobřeží a dorazili do Regia. Na druhý den začal vát jižní vítr, a tak jsme už za dva dny dokončili plavbu v Puteolech.
#28:14 Tam jsme se setkali s bratřími, a ti nás prosili, abychom u nich zůstali sedm dní.
#28:15 Odtud jsme šli do Říma. Tamější bratří dostali o nás zprávu a přišli nám naproti až ke Třem hospodám, někteří pak až k Appiovu tržišti. Když je Pavel spatřil, vzdal díky Bohu a nabyl odvahy.
#28:16 Když jsme přišli do Říma, dostal Pavel dovolení, že může bydlet v soukromém bytě s vojákem, který ho bude hlídat.
#28:17 Po třech dnech pozval k sobě významné Židy, a když se shromáždili, řekl jim: „Bratří, já jsem už v Jeruzalémě byl vydán jako vězeň do moci Římanů, ačkoliv jsem neudělal nic proti židovskému národu ani proti obyčejům našich předků.
#28:18 Římané mě vyslýchali a chtěli mě osvobodit, protože nejsem vinen ničím, zač bych zasluhoval smrt.
#28:19 Avšak Židé se proti tomu postavili, a tak mi nezbylo než odvolat se k císaři; ale ne proto, že bych chtěl žalovat na svůj národ.
#28:20 Z toho důvodu jsem vás pozval, abych vás mohl spatřit a mluvit s vámi. Vždyť jsem v těchto okovech pro to, v co Izrael doufá.“
#28:21 Oni mu odpověděli: „My jsme nedostali z Judska žádný dopis, ani nikdo z bratří nepřinesl o tobě žádnou zprávu, ani o tobě nikdo nemluvil nic špatného.
#28:22 Ale rádi bychom slyšeli, jaké je tvé smýšlení, neboť je nám známo, že všude jsou proti této sektě.“
#28:23 Určili mu den a přišli k němu ve velkém počtu. Od rána do večera k nim hovořil a vydával svědectví o Božím království. Přesvědčoval je o Ježíšovi důkazy z Mojžíšova zákona i z proroků.
#28:24 Jedni se jimi dali přesvědčit, druzí nechtěli uvěřit.
#28:25 Když odcházeli, rozděleni mezi sebou, řekl jim Pavel jen toto: „Duch svatý dobře pověděl vašim předkům ústy proroka Izaiáše:
#28:26 ‚Jdi k tomuto lidu a řekni mu: Budete stále poslouchat, a nepochopíte, ustavičně budete hledět, a neuvidíte.
#28:27 Neboť srdce tohoto lidu otupělo, ušima nedoslýchají a oči zavřeli, takže očima neuvidí a ušima neuslyší, srdcem nepochopí a neobrátí se - a já je neuzdravím.‘
#28:28 Vězte,“ dodal Pavel, „že tato Boží spása byla poslána pohanům. A oni ji přijmou.“
#28:29 ---
#28:30 Pavel zůstal celé dva roky v najatém bytě a přijímal všechny, kdo za ním přišli,
#28:31 zvěstoval Boží království a učil všemu o Pánu Ježíši Kristu bez bázně a bez překážek.  

\book{Romans}{Rom}
#1:1 Pavel, služebník Krista Ježíše, povolaný za apoštola, vyvolený ke zvěstování Božího evangelia,
#1:2 jež Bůh ústy svých proroků předem zaslíbil ve svatých Písmech,
#1:3 evangelia o jeho Synu, který tělem pocházel z rodu Davidova,
#1:4 ale Duchem svatým byl ve svém zmrtvýchvstání uveden do moci Božího Syna, evangelia o Ježíši Kristu, našem Pánu.
#1:5 Skrze něho jsme přijali milost apoštolského poslání, aby ke cti jeho jména uposlechly a uvěřily všecky národy;
#1:6 k nim patříte i vy, neboť jste byli povoláni Ježíšem Kristem.
#1:7 Všem vám v Římě, kdo jste Bohem milováni a povoláni ke svatosti: milost vám a pokoj od Boha Otce našeho a Pána Ježíše Krista.
#1:8 Nejprve vzdávám díky svému Bohu skrze Ježíše Krista za vás všechny, protože se po celém světě rozšiřuje zvěst o vaší víře.
#1:9 Bůh, jemuž z celé duše sloužím evangeliem o jeho Synu, je mi svědkem, jak na vás bez ustání pamatuji
#1:10 při všech svých modlitbách a prosím, zda by mi konečně jednou nebylo z vůle Boží dopřáno dostat se k vám.
#1:11 Toužím vás spatřit, abych se s vámi sdílel o některý duchovní dar a tak vás posílil,
#1:12 to jest abychom se spolu navzájem povzbudili vírou jak vaší, tak mou.
#1:13 Rád bych, abyste věděli, bratří, že jsem už často zamýšlel přijít k vám, abych i mezi vámi sklidil nějaké ovoce, tak jako mezi jinými národy; ale až dosud mi v tom bylo vždy zabráněno.
#1:14 Cítím se totiž dlužníkem Řeků i barbarů, vzdělaných i nevzdělaných.
#1:15 Odtud moje touha zvěstovat evangelium i vám, kteří jste v Římě.
#1:16 Nestydím se za evangelium: je to moc Boží ke spasení pro každého, kdo věří, předně pro Žida, ale také pro Řeka.
#1:17 Vždyť se v něm zjevuje Boží spravedlnost, která je přijímána vírou a vede k víře; stojí přece psáno: ‚Spravedlivý z víry bude živ.‘
#1:18 Boží hněv se zjevuje z nebe proti každé bezbožnosti a nepravosti lidí, kteří svou nepravostí potlačují pravdu.
#1:19 Vždyť to, co lze o Bohu poznat, je jim přístupné, Bůh jim to přece odhalil.
#1:20 Jeho věčnou moc a božství, které jsou neviditelné, lze totiž od stvoření světa vidět, když lidé přemýšlejí o jeho díle, takže nemají výmluvu.
#1:21 Poznali Boha, ale nevzdali mu čest jako Bohu ani mu nebyli vděčni, nýbrž jejich myšlení je zavedlo do marnosti a jejich scestná mysl se ocitla ve tmě.
#1:22 Tvrdí, že jsou moudří, ale upadli v bláznovství:
#1:23 zaměnili slávu nepomíjitelného Boha za zobrazení podoby pomíjitelného člověka, ano i ptáků a čtvernožců a plazů.
#1:24 Proto je Bůh nechal na pospas nečistým vášním jejich srdcí, takže zneuctívají svá vlastní těla;
#1:25 vyměnili Boží pravdu za lež a klanějí se a slouží tvorstvu místo Stvořiteli - on budiž veleben na věky! Amen.
#1:26 Proto je Bůh vydal v moc hanebných vášní. Jejich ženy zaměnily přirozený styk za nepřirozený
#1:27 a stejně i muži zanechali přirozeného styku s ženami a vzplanuli žádostí jeden k druhému, muži s muži provádějí hanebnosti a tak sami na sobě dostávají zaslouženou odplatu za svou scestnost.
#1:28 Protože si nedovedli vážit pravého poznání Boha, dal je Bůh na pospas jejich zvrácené mysli, aby dělali, co se nesluší.
#1:29 Jsou plni nepravosti, podlosti, lakoty, špatnosti, jsou samá závist, vražda, svár, lest, zlomyslnost, jsou donašeči,
#1:30 pomlouvači, Bohu odporní, zpupní, nadutí, chlubiví. Vymýšlejí zlé věci, neposlouchají rodiče,
#1:31 nemají rozum, nedovedou se s nikým snést, neznají lásku ani slitování.
#1:32 Vědí o spravedlivém rozhodnutí Božím, že ti, kteří tak jednají, jsou hodni smrti; a přece nejenže sami tak jednají, ale také jiným takové jednání schvalují. 
#2:1 Proto nemáš nic na svou omluvu, když vynášíš soud, ať jsi kdokoli. Tím, že soudíš druhého, odsuzuješ sám sebe. Neboť soudíš, ale činíš totéž.
#2:2 Víme přece, že Boží soud pravdivě postihuje ty, kteří tak jednají.
#2:3 Myslíš si snad, když soudíš ty, kdo takto jednají, a sám činíš totéž, že ty ujdeš soudu Božímu?
#2:4 Či snad pohrdáš bohatstvím jeho dobroty, shovívavosti a velkomyslnosti, a neuvědomuješ si, že dobrotivost Boží tě chce přivést k pokání?
#2:5 Svou tvrdostí a nekajícnou myslí si střádáš Boží hněv pro den hněvu, kdy se zjeví spravedlivý Boží soud.
#2:6 On ‚odplatí každému podle jeho skutků‘.
#2:7 Těm, kteří vytrvalostí v dobrém jednání hledají nepomíjející slávu a čest, dá život věčný.
#2:8 Ty však, kteří prosazují sebe, odpírají pravdě a podléhají nepravosti, očekává hněv a trest.
#2:9 Soužení a úzkost padne na každého, kdo působí zlo, předně na Žida, ale i na Řeka;
#2:10 avšak sláva, čest a pokoj čeká každého, kdo působí dobro, předně Žida, ale i Řeka.
#2:11 Bůh nikomu nestraní.
#2:12 Ti, kteří neměli zákon a hřešili, také bez zákona zahynou; ti, kteří znali zákon a hřešili, budou odsouzeni podle zákona.
#2:13 Před Bohem nejsou spravedliví ti, kdo zákon slyší; ospravedlněni budou, kdo zákon svými činy plní.
#2:14 Jestliže národy, které nemají zákon, samy od sebe činí to, co zákon žádá, pak jsou samy sobě zákonem, i když zákon nemají.
#2:15 Tím ukazují, že to, co zákon požaduje, mají napsáno ve svém srdci, jak dosvědčuje jejich svědomí, poněvadž jejich myšlenky je jednou obviňují, jednou hájí.
#2:16 Nastane den, kdy Bůh skrze Ježíše Krista bude soudit podle mého evangelia, co je v lidech skryto.
#2:17 Ty se tedy nazýváš židem, spoléháš na zákon, chlubíš se Bohem
#2:18 a tím, že znáš jeho vůli a vyučován zákonem dovedeš rozpoznat, na čem záleží.
#2:19 Myslíš si, že jsi vůdcem slepých, světlem těch, kteří jsou ve tmách,
#2:20 vychovatelem nevzdělaných, učitelem nedospělých, protože máš v zákoně ztělesnění všeho poznání a vší pravdy.
#2:21 Ty tedy poučuješ druhého, a sám sebe neučíš? Hlásáš, že se nemá krást, a sám kradeš?
#2:22 Říkáš, že se nesmí cizoložit, a sám cizoložíš? Ošklivíš si modly, a věci z jejich chrámů bereš?
#2:23 Zakládáš si na zákoně, a sám přestupováním zákona zneuctíváš Boha?
#2:24 ‚Jméno Boží je vaší vinou v posměchu mezi národy‘, jak stojí psáno.
#2:25 Obřízka má smysl, jestliže zachováváš zákon. Jestliže však zákon přestupuješ, je to, jako kdybys nebyl obřezán.
#2:26 Jestliže naopak ten, kdo není obřezán, zachovává požadavky zákona, není to, jako kdyby byl obřezán?
#2:27 Ten, kdo není obřezán, ale plní zákon, bude soudcem nad tebou, který s celou svou literou zákona a obřezaností zákon přestupuješ.
#2:28 Pravý žid není ten, kdo je jím navenek, a pravá obřízka není ta, která je zjevná na těle.
#2:29 Pravý žid je ten, kdo je jím uvnitř, s obřízkou srdce, která je působena Duchem, nikoli literou zákona. Ten dojde chvály ne od lidí, nýbrž od Boha. 
#3:1 Co má tedy žid navíc? A jaký je užitek obřízky?
#3:2 Veliký v každém ohledu! Předně ten, že židům byla svěřena Boží slova.
#3:3 A co když někteří byli nevěrní? Nezruší jejich nevěrnost věrnost Boží?
#3:4 Naprosto ne! Ať se ukáže, že Bůh je pravdivý, ale ‚každý člověk lhář‘, jak je psáno: ‚Aby ses ukázal spravedlivý ve svých slovech a zvítězil, přijdeš-li na soud.‘
#3:5 Jestliže však naše nepravost dává vyniknout spravedlnosti Boží, co k tomu řekneme? Není Bůh, po lidsku řečeno, nespravedlivý, když nás stíhá svým hněvem?
#3:6 Naprosto ne! Vždyť jak by potom Bůh mohl soudit svět?
#3:7 Jestliže však moje lež vyzdvihla Boží pravdu k jeho slávě, proč mám být ještě souzen jako hříšník?
#3:8 A nevede to potom k tomu, co nám někteří pomlouvači připisují, jako bychom říkali: „Čiňme zlo, aby přišlo dobro?“ Ty čeká spravedlivé odsouzení!
#3:9 Co tedy? Máme my židé nějakou přednost? Vůbec ne! Vždyť jsme už dříve ukázali, že všichni, židé i pohané, jsou pod mocí hříchu,
#3:10 jak je psáno: ‚Nikdo není spravedlivý, není ani jeden,
#3:11 nikdo není rozumný, není, kdo by hledal Boha;
#3:12 všichni se odchýlili, všichni propadli zvrácenosti, není, kdo by činil dobro, není ani jeden.
#3:13 Hrob otevřený je jejich hrdlo, svým jazykem mluví jen lest, hadí jed skrývají ve rtech,
#3:14 jejich ústa jsou samá kletba a hořkost,
#3:15 jejich nohy spěchají prolévat krev,
#3:16 zhouba a bída je na jejich cestách;
#3:17 nepoznali cestu pokoje
#3:18 a úctu před Bohem nemají.‘
#3:19 Víme, že co zákon říká, říká těm, kdo jsou pod zákonem, aby byla umlčena každá ústa a aby celý svět byl před Bohem usvědčen z viny.
#3:20 Vždyť ze skutků zákona ‚nebude před ním nikdo ospravedlněn‘, neboť ze zákona pochází poznání hříchu.
#3:21 Nyní však je zjevena Boží spravedlnost bez zákona, dosvědčovaná zákonem i proroky,
#3:22 Boží spravedlnost skrze víru v Ježíše Krista pro všecky, kdo věří. Není totiž rozdílu:
#3:23 všichni zhřešili a jsou daleko od Boží slávy;
#3:24 jsou ospravedlňováni zadarmo jeho milostí vykoupením v Kristu Ježíši.
#3:25 Jeho ustanovil Bůh, aby svou vlastní smrtí se stal smírnou obětí pro ty, kdo věří. Tak prokázal, že byl spravedlivý, když již dříve trpělivě promíjel hříchy.
#3:26 Svou spravedlnost prokázal i v nynějším čase, aby bylo zjevné, že je spravedlivý a ospravedlňuje toho, kdo žije z víry v Ježíše.
#3:27 Kde zůstala chlouba? Byla vyloučena! Jakým zákonem? Zákonem skutků? Nikoli, nýbrž zákonem víry.
#3:28 Jsme totiž přesvědčeni, že se člověk stává spravedlivým vírou bez skutků zákona.
#3:29 Je snad Bůh toliko Bohem židů? Což není též Bohem pohanů? Zajisté i pohanů!
#3:30 Vždyť je to jeden a týž Bůh, který obřezané ospravedlní z víry a neobřezané skrze víru.
#3:31 To tedy vírou rušíme zákon? Naprosto ne! Naopak, zákon potvrzujeme. 
#4:1 Co tedy řekneme o Abrahamovi, našem tělesném praotci? Čeho dosáhl?
#4:2 Kdyby Abraham dosáhl spravedlnosti svými skutky, měl by se čím chlubit - ale ne před Bohem!
#4:3 Co říká Písmo? ‚Uvěřil Abraham Bohu, a bylo mu to počítáno za spravedlnost.‘
#4:4 Kdo se vykazuje skutky, nedostává mzdu z milosti, nýbrž z povinnosti.
#4:5 Kdo se nevykazuje skutky, ale věří v toho, který dává spravedlnost bezbožnému, tomu se jeho víra počítá za spravedlnost.
#4:6 Vždyť i David prohlašuje za blahoslaveného člověka, jemuž Bůh připočítává spravedlnost bez skutků:
#4:7 ‚Blaze těm, jimž jsou odpuštěny nepravosti a jejich hříchy přikryty.
#4:8 Blaze tomu, jemuž Hospodin nepočítá hřích.‘
#4:9 Platí toto blahoslavenství jen pro ty, kdo jsou obřezáni, či také pro ty, kdo nejsou obřezáni? Čteme přece: ‚Abrahamovi byla víra počítána za spravedlnost.‘
#4:10 Kdy mu byla započtena? Byl už obřezán, nebo ještě nebyl? Nebylo to po obřízce, nýbrž ještě před ní.
#4:11 Znamení obřízky přijal jako pečeť spravedlnosti z víry, kterou měl ještě před obřízkou. Tak se stal otcem všech neobřezaných, kteří věří a jimž je spravedlnost připočtena,
#4:12 i otcem těch obřezaných, kteří nejsou jen obřezáni, nýbrž také jdou ve stopách víry našeho otce Abrahama - víry, kterou měl ještě před obřízkou.
#4:13 Zaslíbení, že dostane svět za dědictví, nebylo dáno Abrahamovi a jeho potomstvu na základě zákona, nýbrž na základě spravedlnosti z víry.
#4:14 Kdyby dědici byli ti, kteří stavějí na zákoně, byla by víra zbavena smyslu a zaslíbení zrušeno.
#4:15 Zákon s sebou nese Boží hněv: kde není zákon, není ani přestoupení zákona.
#4:16 Proto mluvíme o spravedlnosti z víry, aby bylo jasné, že je to spravedlnost z milosti. Tak zůstane v platnosti zaslíbení dané veškerému potomstvu Abrahamovu - nejen těm, kdo stavějí na zákoně, ale i těm, kdo následují Abrahama vírou. On je otcem nás všech,
#4:17 jak je psáno: ‚ustanovil jsem tě za otce mnohých národů‘. Je naším otcem před tváří toho, v nějž uvěřil, před Bohem, který dává život mrtvým a povolává v bytí to, co není.
#4:18 On uvěřil a měl naději, kde už naděje nebylo; tím se stal ‚otcem mnohých národů‘ podle slova: ‚tak četné bude tvé potomstvo‘.
#4:19 Neochabl ve víře, i když pomyslil na své již neplodné tělo - vždyť mu bylo asi sto let - i na to, že Sára již nemůže mít dítě;
#4:20 nepropadl pochybnosti o Božím zaslíbení, ale posílen vírou vzdal čest Bohu v pevné jistotě,
#4:21 že Bůh je mocen učinit, co zaslíbil.
#4:22 Proto mu to ‚bylo počítáno za spravedlnost‘.
#4:23 To, že mu to ‚bylo počítáno‘, nebylo napsáno jen kvůli němu,
#4:24 nýbrž také kvůli nám, jimž má být započteno, že věříme v toho, který vzkřísil z mrtvých Ježíše, našeho Pána,
#4:25 jenž byl vydán pro naše přestoupení a vzkříšen pro naše ospravedlnění. 
#5:1 Když jsme tedy ospravedlněni z víry, máme pokoj s Bohem skrze našeho Pána Ježíše Krista,
#5:2 neboť skrze něho jsme vírou získali přístup k této milosti. V ní stojíme a chlubíme se nadějí, že dosáhneme slávy Boží.
#5:3 A nejen to: chlubíme se i utrpením, vždyť víme, že z utrpení roste vytrvalost,
#5:4 z vytrvalosti osvědčenost a z osvědčenosti naděje.
#5:5 A naděje neklame, neboť Boží láska je vylita do našich srdcí skrze Ducha svatého, který nám byl dán.
#5:6 Když jsme ještě byli bezmocní, v čas, který Bůh určil, zemřel Kristus za bezbožné.
#5:7 Sotva kdo je hotov podstoupit smrt za spravedlivého člověka, i když za takového by se snad někdo odvážil nasadit život.
#5:8 Bůh však prokazuje svou lásku k nám tím, že Kristus za nás zemřel, když jsme ještě byli hříšní.
#5:9 Tím spíše nyní, když jsme byli ospravedlněni prolitím jeho krve, budeme skrze něho zachráněni od Božího hněvu.
#5:10 Jestliže jsme my, Boží nepřátelé, byli s Bohem smířeni smrtí jeho Syna, tím spíše nás smířené zachrání jeho život.
#5:11 A nejen to: chlubíme se dokonce Bohem skrze našeho Pána Ježíše Krista, který nás s ním smířil.
#5:12 Skrze jednoho člověka totiž vešel do světa hřích a skrze hřích smrt; a tak smrt zasáhla všechny, protože všichni zhřešili.
#5:13 Hřích byl ve světě už před zákonem, ač se hřích nezapočítává, pokud není zákon.
#5:14 Smrt však vládla od Adama až po Mojžíše i nad těmi, kdo hřešili jiným způsobem než Adam. On je protějšek toho, který měl přijít.
#5:15 S milostí tomu však není tak jako s proviněním. Proviněním toho jediného, totiž Adama, mnozí propadli smrti; oč spíše zahrnula mnohé Boží milost, milost darovaná v jediném člověku, Ježíši Kristu.
#5:16 A s darem milosti tomu není jako s následky toho, že jeden zhřešil. Soud nad jedním proviněním vedl k odsouzení, kdežto milost po mnohých proviněních vede k ospravedlnění.
#5:17 Jestliže proviněním Adamovým smrt se zmocnila vlády skrze jednoho člověka, tím spíše ti, kteří přijímají hojnost milosti a darované spravedlnosti, budou vládnout v životě věčném skrze jednoho jediného, Ježíše Krista.
#5:18 A tak tedy: Jako jediné provinění přineslo odsouzení všem, tak i jediný čin spravedlnosti přinesl všem ospravedlnění a život.
#5:19 Jako se neposlušností jednoho člověka mnozí stali hříšníky, tak zase poslušností jednoho jediného mnozí se stanou spravedlivými.
#5:20 K tomu navíc přistoupil zákon, aby se provinění rozmohlo. A kde se rozmohl hřích, tam se ještě mnohem více rozhojnila milost,
#5:21 aby tak jako vládl hřích a přinášel smrt, vládla ospravedlněním milost a přinášela věčný život skrze Ježíše Krista, našeho Pána. 
#6:1 Co tedy máme říci? Že máme dále žít v hříchu, aby se rozhojnila milost?
#6:2 Naprosto ne! Hříchu jsme přece zemřeli - jak bychom v něm mohli dále žít?
#6:3 Nevíte snad, že všichni, kteří jsme pokřtěni v Krista Ježíše, byli jsme pokřtěni v jeho smrt?
#6:4 Byli jsme tedy křtem spolu s ním pohřbeni ve smrt, abychom - jako Kristus byl vzkříšen z mrtvých slavnou mocí svého Otce - i my vstoupili na cestu nového života.
#6:5 Jestliže jsme s ním sjednoceni, protože máme účast na jeho smrti, jistě budeme mít účast i na jeho zmrtvýchvstání.
#6:6 Víme přece, že starý člověk v nás byl spolu s ním ukřižován, aby tělo ovládané hříchem bylo zbaveno moci a my už hříchu neotročili.
#6:7 Vždyť ten, kdo zemřel, je vysvobozen z moci hříchu.
#6:8 Jestliže jsme spolu s Kristem zemřeli, věříme, že spolu s ním budeme také žít.
#6:9 Vždyť víme, že Kristus, když byl vzkříšen z mrtvých, už neumírá, smrt nad ním už nepanuje.
#6:10 Když zemřel, zemřel hříchu jednou provždy, když nyní žije, žije Bohu.
#6:11 Tak i vy počítejte s tím, že jste mrtvi hříchu, ale živi Bohu v Kristu Ježíši.
#6:12 Nechť tedy hřích neovládá vaše smrtelné tělo, tak abyste poslouchali, čeho se mu zachce;
#6:13 ani nepropůjčujte hříchu své tělo za nástroj nepravosti, ale jako ti, kteří byli vyvedeni ze smrti do života, propůjčujte sami sebe a své tělo Bohu za nástroj spravedlnosti.
#6:14 Hřích nad vámi už nebude panovat; vždyť nejste pod zákonem, ale pod milostí.
#6:15 Co z toho plyne? Máme snad hřešit, protože nejsme pod zákonem, ale pod milostí? Naprosto ne!
#6:16 Víte přece, když se někomu zavazujete k poslušné službě, že se stáváte služebníky toho, koho posloucháte - buď otročíte hříchu, a to vede k smrti, nebo posloucháte Boha, a to vede k spravedlnosti.
#6:17 Díky Bohu za to, že jste sice byli služebníky hříchu, ale potom jste se ze srdce přiklonili k tomu učení, které vám bylo odevzdáno.
#6:18 A tak jste byli osvobozeni od hříchu a stali jste se služebníky spravedlnosti.
#6:19 Mluvím názorně z ohledu na vaši lidskou slabost: Jako jste se dříve propůjčovali k službě nečistotě a nepravosti k bezbožnému životu, tak se nyní dejte do služby spravedlnosti k posvěcení.
#6:20 Když jste byli služebníky hříchu, měli jste svobodu od spravedlnosti;
#6:21 jaký jste tehdy měli užitek z toho, zač se nyní stydíte? Konec toho všeho je přece smrt.
#6:22 Avšak nyní, když jste byli osvobozeni od hříchu a stali se služebníky Božími, máte z toho užitek, totiž posvěcení, a čeká vás život věčný.
#6:23 Mzdou hříchu je smrt, ale darem Boží milosti je život věčný v Kristu Ježíši, našem Pánu. 
#7:1 Což nevíte, bratří - vždyť mluvím k těm, kteří znají zákon - že zákon panuje nad člověkem, jen pokud je živ?
#7:2 Vdaná žena je zákonem vázána k žijícímu manželovi; když však muž zemře, je zproštěna zákona manželství.
#7:3 Pokud je tedy její muž naživu, bude prohlášena za cizoložnici, oddá-li se jinému muži. Jestliže však manžel zemře, je svobodná od zákona, takže nebude cizoložnicí, když se oddá jinému muži.
#7:4 Právě tak jste se i vy, bratří moji, stali mrtvými pro zákon skrze tělo Kristovo, abyste se oddali jinému, tomu, který byl vzkříšen z mrtvých, a tak nesli ovoce Bohu.
#7:5 Když jsme byli v moci hříchu, působily v nás vášně podněcované zákonem a nesly ovoce smrti.
#7:6 Nyní, když jsme zemřeli tomu, čím jsme byli spoutáni, byli jsme zproštěni zákona, takže sloužíme Bohu v novém životě Ducha, ne pod starou literou zákona.
#7:7 Co tedy máme říci? Že zákon je hříšný? Naprosto ne! Ale hřích bych byl nepoznal, kdyby nebylo zákona. Vždyť bych neznal žádostivost, kdyby zákon neřekl: ‚Nepožádáš!‘
#7:8 Hřích použil tohoto přikázání jako příležitosti, aby ve mně probudil všechnu žádostivost; bez zákona je totiž hřích mrtev.
#7:9 Já jsem kdysi žil bez zákona, když však přišel zákon, hřích ožil,
#7:10 a já jsem zemřel. Tak se ukázalo, že právě přikázání, které mi mělo dát život, přineslo mi smrt.
#7:11 Hřích použil přikázání jako příležitosti, aby mne oklamal a tak mě usmrtil.
#7:12 Zákon je tedy sám v sobě svatý a přikázání svaté, spravedlivé a dobré.
#7:13 Bylo snad to dobré příčinou mé smrti? Naprosto ne! Hřích však, aby se ukázal jako hřích, způsobil mi tím dobrým smrt; tak skrze přikázání ukázal hřích celou hloubku své hříšnosti.
#7:14 Víme, že zákon je svatý- já však jsem hříšný, hříchu zaprodán.
#7:15 Nepoznávám se ve svých skutcích; vždyť nedělám to, co chci, nýbrž to, co nenávidím.
#7:16 Jestliže však to, co dělám, je proti mé vůli, pak souhlasím se zákonem a uznávám, že je dobrý.
#7:17 Pak to vlastně nejsem já, kdo jedná špatně, ale hřích, který je ve mně.
#7:18 Vím totiž, že ve mně, to jest v mé lidské přirozenosti, nepřebývá dobro. Chtít dobro, to dokážu, ale vykonat už ne.
#7:19 Vždyť nečiním dobro, které chci, nýbrž zlo, které nechci.
#7:20 Jestliže však činím to, co nechci, nedělám to já, ale hřích, který ve mně přebývá.
#7:21 Objevuji tedy takový zákon: Když chci činit dobro, mám v dosahu jen zlo.
#7:22 Ve své nejvnitřnější bytosti s radostí souhlasím se zákonem Božím;
#7:23 když však mám jednat, pozoruji, že jiný zákon vede boj proti zákonu, kterému se podřizuje má mysl, a činí mě zajatcem zákona hříchu, kterému se podřizují mé údy.
#7:24 Jak ubohý jsem to člověk! Kdo mě vysvobodí z tohoto těla smrti?
#7:25 Jedině Bohu buď dík skrze Ježíše Krista, Pána našeho! - A tak tentýž já sloužím svou myslí zákonu Božímu, ale svým jednáním zákonu hříchu. 
#8:1 Nyní však není žádného odsouzení pro ty, kteří jsou v Kristu Ježíši,
#8:2 neboť zákon Ducha, který vede k životu v Kristu Ježíši, osvobodil tě od zákona hříchu a smrti.
#8:3 Bůh učinil to, co bylo zákonu nemožné pro lidskou slabost: Jako oběť za hřích poslal svého vlastního Syna v těle, jako má hříšný člověk, aby na lidském těle odsoudil hřích,
#8:4 a aby tak spravedlnost požadovaná zákonem byla naplněna v nás, kteří se neřídíme svou vůlí, nýbrž vůlí Ducha.
#8:5 Ti, kdo dělají jen to, co sami chtějí, tíhnou k tomu, co je tělesné; ale ti, kdo se dají vést Duchem, tíhnou k tomu, co je duchovní.
#8:6 Dát se vést sobectvím znamená smrt, dát se vést Duchem je život a pokoj.
#8:7 Soustředění na sebe je Bohu nepřátelské, neboť se nechce ani nemůže podřídit Božímu zákonu.
#8:8 Ti, kdo žijí jen z vlastních sil, nemohou se líbit Bohu.
#8:9 Vy však nejste živi ze své síly, ale z moci Ducha, jestliže ve vás Boží Duch přebývá. Kdo nemá Ducha Kristova, ten není jeho.
#8:10 Je-li však ve vás Kristus, pak vaše tělo sice podléhá smrti, protože jste zhřešili, ale Duch dává život, protože jste ospravedlněni.
#8:11 Jestliže ve vás přebývá Duch toho, který Ježíše vzkřísil z mrtvých, pak ten, kdo vzkřísil z mrtvých Krista Ježíše, obživí i vaše smrtelná těla Duchem, který ve vás přebývá.
#8:12 A tak, bratří, jsme dlužni, ale ne sami sobě, abychom museli žít podle své vůle.
#8:13 Vždyť žijete-li podle své vůle, spějete k smrti; jestliže však mocí Ducha usmrcujete hříšné činy, budete žít.
#8:14 Ti, kdo se dají vést Duchem Božím, jsou synové Boží.
#8:15 Nepřijali jste přece Ducha otroctví, abyste opět propadli strachu, nýbrž přijali jste Ducha synovství, v němž voláme: Abba, Otče!
#8:16 Tak Boží Duch dosvědčuje našemu duchu, že jsme Boží děti.
#8:17 A jsme-li děti, tedy i dědicové - dědicové Boží, spoludědicové Kristovi; trpíme-li spolu s ním, budeme spolu s ním účastni Boží slávy.
#8:18 Soudím totiž, že utrpení nynějšího času se nedají srovnat s budoucí slávou, která má být na nás zjevena.
#8:19 Celé tvorstvo toužebně vyhlíží a čeká, kdy se zjeví sláva Božích synů.
#8:20 Neboť tvorstvo bylo vydáno marnosti - ne vlastní vinou, nýbrž tím, kdo je marnosti vydal. Trvá však naděje,
#8:21 že i samo tvorstvo bude vysvobozeno z otroctví zániku a uvedeno do svobody a slávy dětí Božích.
#8:22 Víme přece, že veškeré tvorstvo až podnes společně sténá a pracuje k porodu.
#8:23 A nejen to: i my sami, kteří již máme Ducha jako příslib darů Božích, i my ve svém nitru sténáme, očekávajíce přijetí za syny, totiž vykoupení svého těla.
#8:24 Jsme spaseni v naději; naděje však, kterou je vidět, není už naděje. Kdo něco vidí, proč by v to ještě doufal?
#8:25 Ale doufáme-li v to, co nevidíme, trpělivě to očekáváme.
#8:26 Tak také Duch přichází na pomoc naší slabosti. Vždyť ani nevíme, jak a za co se modlit, ale sám Duch se za nás přimlouvá nevyslovitelným lkáním.
#8:27 Ten, který zkoumá srdce, ví, co je úmyslem Ducha; neboť Duch se přimlouvá za svaté podle Boží vůle.
#8:28 Víme, že všecko napomáhá k dobrému těm, kdo milují Boha, kdo jsou povoláni podle jeho rozhodnutí.
#8:29 Které předem vyhlédl, ty také předem určil, aby přijali podobu jeho Syna, tak aby byl prvorozený mezi mnoha bratřími;
#8:30 které předem určil, ty také povolal; které povolal, ty také ospravedlnil, a které ospravedlnil, ty také uvedl do své slávy.
#8:31 Co k tomu dodat? Je-li Bůh s námi, kdo proti nám?
#8:32 On neušetřil svého vlastního Syna, ale za nás za všecky jej vydal; jak by nám spolu s ním nedaroval všecko?
#8:33 Kdo vznese žalobu proti vyvoleným Božím? Vždyť Bůh ospravedlňuje!
#8:34 Kdo je odsoudí? Vždyť Kristus Ježíš, který zemřel a který byl vzkříšen, je na pravici Boží a přimlouvá se za nás!
#8:35 Kdo nás odloučí od lásky Kristovy? Snad soužení nebo úzkost, pronásledování nebo hlad, bída, nebezpečí nebo meč?
#8:36 Jak je psáno: „Denně jsme pro tebe vydáváni na smrt, jsme jako ovce určené na porážku.“
#8:37 Ale v tom ve všem slavně vítězíme mocí toho, který si nás zamiloval.
#8:38 Jsem jist, že ani smrt ani život, ani andělé ani mocnosti, ani přítomnost ani budoucnost, ani žádná moc,
#8:39 ani výšiny ani hlubiny, ani co jiného v celém tvorstvu nedokáže nás odloučit od lásky Boží, která je v Kristu Ježíši, našem Pánu. 
#9:1 Mluvím pravdu v Kristu, nelžu, a dosvědčuje mi to mé svědomí v Duchu svatém,
#9:2 že mám velký zármutek a neustálou bolest ve svém srdci.
#9:3 Přál bych si sám být proklet a odloučen od Krista Ježíše za své bratry, za lid, z něhož pocházím.
#9:4 Jsou to Izraelci, jim patří synovství i sláva i smlouvy s Bohem, jim je svěřen zákon i bohoslužba i zaslíbení,
#9:5 jejich jsou praotcové, z nich rodem pochází Kristus. Bůh, který je nade všemi, buď pochválen na věky, amen.
#9:6 Ne že by slovo Boží selhalo. Vždyť ne všichni, kteří jsou z Izraele, jsou Izrael,
#9:7 ani nejsou všichni dětmi Abrahamovými jen proto, že jsou jeho potomci, nýbrž ‚z Izáka bude povoláno tvé potomstvo‘, to jest:
#9:8 dětmi Božími nejsou tělesné děti, nýbrž za potomky se považují děti zaslíbené.
#9:9 Slovo zaslíbení zní takto: ‚V určený čas přijdu, a Sára bude mít syna.‘
#9:10 A nejen to: Také Rebeka měla obě děti z téhož muže, z našeho praotce Izáka;
#9:11 ještě se jí nenarodily a nemohly učinit nic dobrého ani zlého. Aby však zůstalo v platnosti Boží vyvolení, o kterém bylo předem rozhodnuto
#9:12 a které nezávisí na skutcích, nýbrž na tom, kdo povolává, bylo jí hned řečeno, že starší bude sloužit mladšímu.
#9:13 Neboť je psáno: ‚Jákoba jsem si zamiloval, ale Ezaua jsem odmítl.‘
#9:14 Co tedy řekneme? Je Bůh nespravedlivý? Naprosto ne!
#9:15 Mojžíšovi řekl: ‚Smiluji se, nad kým se smiluji, a slituji se, nad kým se slituji.‘
#9:16 Nezáleží tedy na tom, kdo chce, ani na tom, kdo se namáhá, ale na Bohu, který se smilovává.
#9:17 Písmo přece říká faraónovi: ‚Vyzdvihl jsem tě, abych na tobě ukázal svou moc a aby mé jméno bylo rozhlášeno po celé zemi.‘
#9:18 Smilovává se tedy, nad kým chce, a koho chce, činí zatvrzelým.
#9:19 Snad mi řekneš: „Proč nás tedy Bůh ještě kárá? Může se vůbec někdo vzepřít jeho vůli?“
#9:20 Člověče, co vlastně jsi, že odmlouváš Bohu? Řekne snad výtvor svému tvůrci: „Proč jsi mě udělal takto?“
#9:21 Nemá snad hrnčíř hlínu ve své moci, aby z téže hroudy udělal jednu nádobu ke vznešeným účelům a druhou ke všedním?
#9:22 Jestliže Bůh chtěl ukázat svůj hněv a zjevit svou moc, a proto s velkou shovívavostí snášel ty, kdo propadli jeho hněvu a byli určeni k záhubě,
#9:23 stejně chtěl ukázat bohatství své slávy na těch, nad nimiž se smiloval a které připravil k slávě -
#9:24 na nás, které povolal nejen ze židů, ale i z pohanských národů.
#9:25 Jak je psáno u Ozeáše: ‚Lid, který není můj, povolám za svůj lid a Nemilovanou nazvu Milovanou,
#9:26 a kde bylo řečeno: Vy nejste můj lid, tam budou nazváni syny Boha živého.‘
#9:27 A Izaiáš prohlašuje o Izraeli: ‚Kdyby bylo synů Izraele jako písku v moři, jen zbytek bude zachráněn,
#9:28 neboť Hospodin vykoná svůj soud na zemi rychle a úplně.‘
#9:29 A jak to Izaiáš předpověděl: ‚Kdyby nám Hospodin zástupů nenechal aspoň símě, bylo by to s námi jako se Sodomou, Gomoře bychom byli podobni.‘
#9:30 Co tedy nakonec řekneme? To, že pohanští národové, kteří neusilovali o spravedlnost, spravedlnosti dosáhli, a to spravedlnosti z víry;
#9:31 Izrael však, který usiloval o spravedlnost podle zákona, k cíli zákona nedospěl.
#9:32 Proč? Protože nevycházel z víry, nýbrž ze skutků. Narazili na kámen úrazu,
#9:33 jak je psáno: ‚Hle, kladu na Siónu kámen úrazu a skálu pohoršení, ale kdo v něho věří, nebude zahanben.‘ 
#10:1 Bratří, toužím z celého srdce a modlím se k Bohu, aby Izrael došel spásy.
#10:2 Vždyť jim mohu dosvědčit, že jsou plni horlivosti pro Boha, jenže bez pravého poznání.
#10:3 Nevědí, že spravedlnost je od Boha, a chtějí uplatnit svou vlastní; proto se spravedlnosti Boží nepodřídili.
#10:4 Vždyť Kristus je konec zákona, aby spravedlnosti došel každý, kdo věří.
#10:5 Mojžíš píše o spravedlnosti, založené na zákoně: ‚Člověk, který tak jedná, bude živ.‘
#10:6 Avšak spravedlnost založená na víře mluví takto: ‚Nezabývej se myšlenkou: kdo vystoupí na nebe?‘ - aby Krista přivedl dolů -
#10:7 ‚ani neříkej: kdo sestoupí do propasti?‘ - aby Krista vyvedl z říše mrtvých.
#10:8 Co však praví? ‚Blízko tebe je slovo, v tvých ústech a ve tvém srdci‘; je to slovo víry, které zvěstujeme.
#10:9 Vyznáš-li svými ústy Ježíše jako Pána a uvěříš-li ve svém srdci, že ho Bůh vzkřísil z mrtvých, budeš spasen.
#10:10 Srdcem věříme k spravedlnosti a ústy vyznáváme k spasení,
#10:11 neboť Písmo praví: ‚Kdo v něho věří, nebude zahanben.‘
#10:12 Není rozdílu mezi Židem a Řekem: Vždyť je jeden a týž Pán všech, štědrý ke všem, kdo ho vzývají, neboť
#10:13 ‚každý, kdo vzývá jméno Páně, bude spasen‘.
#10:14 Ale jak mohou vzývat toho, v něhož neuvěřili? A jak mohou uvěřit v toho, o kom neslyšeli? A jak mohou uslyšet, není-li tu nikdo, kdo by ho zvěstoval?
#10:15 A jak mohou zvěstovat, nejsou-li posláni? Je přece psáno: ‚Jak vítaný je příchod těch, kteří zvěstují dobré věci!‘
#10:16 Ale ne všichni přijali evangelium. Už Izaiáš říká: ‚Hospodine, kdo uvěřil naší zvěsti?‘
#10:17 Víra je tedy ze zvěstování a zvěstování z pověření Kristova.
#10:18 Ptám se však: To snad nikdy neslyšeli zvěst? Ovšemže slyšeli! ‚Po celé zemi se rozlehl jejich hlas, do nejzazších končin jejich slova.‘
#10:19 Ptám se tedy: Nepochopil Izrael, co mu bylo zvěstováno? Nepochopil; už u Mojžíše se přece říká: ‚Vzbudím ve vás žárlivost na národ, který není národem, proti národu pošetilému vás popudím k hněvu.‘
#10:20 A Izaiáš má odvahu říci: ‚Dal jsem se nalézti těm, kteří mě nehledali, dal jsem se poznat těm, kteří se po mně neptali.‘
#10:21 O Izraeli však říká: ‚Každý den jsem vztahoval ruce k lidu neposlušnému a vzpurnému.‘ 
#11:1 Chci tím říci, že Bůh zavrhl svůj lid? Naprosto ne! Vždyť i já jsem Izraelec, z potomstva Abrahamova, z pokolení Benjamínova.
#11:2 Bůh nezavrhl svůj lid, který si předem vyhlédl. Nevíte z Písma, jak si Eliáš Bohu naříká na Izrael?
#11:3 ‚Hospodine, proroky tvé pobili a oltáře tvé pobořili, já jediný jsem zůstal, a také mně ukládají o život!‘
#11:4 Jakou však dostal odpověď? ‚Zachoval jsem si sedm tisíc mužů, kteří nesklonili kolena před Baalem.‘
#11:5 A tak i nyní je tu zbytek lidu vyvolený z milosti.
#11:6 Když z milosti, tedy ne na základě skutků - jinak by milost nebyla milostí.
#11:7 Co z toho plyne? Izrael nedosáhl, oč usiluje. Dosáhli toho jen vyvolení z Izraele, ale ostatní zůstali zatvrzelí,
#11:8 jak je psáno: ‚Bůh otupil jejich mysl, dal jim oči, aby neviděli, uši, aby neslyšeli, až do dnešního dne.‘
#11:9 A David praví: ‚Ať se jim jejich stůl stane léčkou a pastí, kamenem úrazu a odplatou,
#11:10 ať se jim oči zatmějí, aby neviděli, a jejich šíje ať je navždy sehnutá.‘
#11:11 Chci tedy snad říci, že klopýtli tak, aby nadobro padli? Naprosto ne! Ale jejich selhání přineslo pohanům spásu, aby to vzbudilo žárlivost židů.
#11:12 Jestliže tedy jejich selháním svět získal a jejich úpadek obohatil pohany, co teprve, až se všichni obrátí?
#11:13 Vám z pohanských národů říkám, že právě já, apoštol pohanů, vidím slávu své služby v tom,
#11:14 abych vzbudil žárlivost svých pokrevních bratří a některé z nich přivedl ke spasení.
#11:15 Jestliže jejich zavržení znamenalo smíření světa s Bohem, co jiného bude znamenat jejich přijetí než vzkříšení mrtvých!
#11:16 Je-li první chléb zasvěcen, je svatý všechen chléb; je-li kořen svatý, jsou svaté i větve.
#11:17 Jestliže však některé větve byly vylomeny a ty, planá oliva, jsi byl naroubován na jejich místo a bereš sílu z kořene ušlechtilé olivy,
#11:18 nevynášej se nad ty větve! Začneš-li se vynášet, vzpomeň si, že ty neneseš kořen, nýbrž kořen nese tebe!
#11:19 Řekneš snad: Ty větve byly vylomeny, abych já byl naroubován.
#11:20 Dobře. Byly vylomeny pro svou nevěru, ty však stojíš vírou. Nepovyšuj se, ale boj se!
#11:21 Jestliže Bůh neušetřil přirozených větví, tím spíše neušetří tebe!
#11:22 Považ dobrotu i přísnost Boží: přísnost k těm, kteří odpadli, avšak dobrotu Boží k tobě, budeš-li se jeho dobroty držet. Jinak i ty budeš vyťat,
#11:23 oni však, nesetrvají-li v nevěře, budou naroubováni, neboť Bůh má moc naroubovat je znovu.
#11:24 Jestliže tys byl vyťat ze své plané olivy a proti přírodě naroubován na ušlechtilou olivu, tím spíše budou na svou vlastní olivu naroubováni ti, kteří k ní od přírody patří!
#11:25 Abyste nespoléhali na svou vlastní moudrost, chtěl bych, bratří, abyste věděli o tomto tajemství: Část Izraele propadla zatvrzení, avšak jen do té doby, pokud nevejde plný počet pohanů.
#11:26 Pak bude spasen všechen Izrael, jak je psáno: ‚Přijde ze Sióna vysvoboditel, odvrátí od Jákoba bezbožnost;
#11:27 to bude má smlouva s nimi, až sejmu jejich hříchy.‘
#11:28 Pokud jde o evangelium, stali se Božími nepřáteli, ale vám to přineslo prospěch; pokud jde o vyvolení, zůstávají Bohu milí pro své otce.
#11:29 Vždyť Boží dary a jeho povolání jsou neodvolatelná.
#11:30 Jako vy jste kdysi Boha neposlouchali, nyní však jste došli slitování pro jejich neposlušnost,
#11:31 tak i oni nyní upadli v neposlušnost, aby také došli slitování, jakého se dostalo vám.
#11:32 Bůh totiž všecky uzavřel pod neposlušnost, aby se nade všemi slitoval.
#11:33 Jak nesmírná je hloubka Božího bohatství, jeho moudrosti i vědění! Jak nevyzpytatelné jsou jeho soudy a nevystopovatelné jeho cesty!
#11:34 ‚Kdo poznal mysl Hospodinovu a kdo se stal jeho rádcem?‘
#11:35 ‚Kdo mu něco dal, aby mu to on musel vrátit?‘
#11:36 Vždyť z něho a skrze něho a pro něho je všecko! Jemu buď sláva na věky. Amen. 
#12:1 Vybízím vás, bratří, pro Boží milosrdenství, abyste sami sebe přinášeli jako živou, svatou, Bohu milou oběť; to ať je vaše pravá bohoslužba.
#12:2 A nepřizpůsobujte se tomuto věku, nýbrž proměňujte se obnovou své mysli, abyste mohli rozpoznat, co je vůle Boží, co je dobré, Bohu milé a dokonalé.
#12:3 Každému z vás říkám na základě milosti, která mi byla dána: Nesmýšlejte výš, než je komu určeno, ale smýšlejte o sobě střízlivě, podle toho, jakou míru víry udělil každému Bůh.
#12:4 Jako je v jednom těle mnoho údů a nemají všechny stejný úkol,
#12:5 tak i my, ač je nás mnoho, jsme jedno tělo v Kristu a jeden druhému sloužíme jako jednotlivé údy.
#12:6 Máme rozličné dary podle milosti, která byla dána každému z nás: Kdo má dar prorockého slova, ať ho užívá v souhlase s vírou.
#12:7 Kdo má dar služby, ať slouží. Kdo má dar učit, ať učí.
#12:8 Kdo dovede povzbuzovat, nechť povzbuzuje. Kdo rozdává, ať dává upřímně. Kdo stojí v čele, ať je horlivý. Kdo se stará o trpící, ať pomáhá s radostí.
#12:9 Láska nechť je bez přetvářky. Ošklivte si zlo, lněte k dobrému.
#12:10 Milujte se navzájem bratrskou láskou, v úctě dávejte přednost jeden druhému.
#12:11 V horlivosti neochabujte, buďte vroucího ducha, služte Pánu.
#12:12 Z naděje se radujte, v soužení buďte trpěliví, v modlitbách vytrvalí.
#12:13 Sdílejte se s bratřími v jejich nouzi, ochotně poskytujte pohostinství.
#12:14 Svolávejte dobro na ty, kteří vás pronásledují, dobro a ne zlo.
#12:15 Radujte se s radujícími, plačte s plačícími.
#12:16 Mějte porozumění jeden pro druhého. Nesmýšlejte vysoko, ale věnujte se všedním službám. Nespoléhejte na svou vlastní chytrost.
#12:17 Nikomu neodplácejte zlým za zlé. Vůči všem mějte na mysli jen dobré.
#12:18 Je-li možno, pokud to záleží na vás, žijte se všemi v pokoji.
#12:19 Nechtějte sami odplácet, milovaní, ale nechte místo pro Boží soud, neboť je psáno: ‚Mně patří pomsta, já odplatím, praví Pán.‘
#12:20 Ale také: ‚Jestliže má tvůj nepřítel hlad, nasyť ho, a má-li žízeň, dej mu pít; tím ho zahanbíš a přivedeš k lítosti.‘
#12:21 Nedej se přemoci zlem, ale přemáhej zlo dobrem. 
#13:1 Každý ať se podřizuje vládní moci, neboť není moci, leč od Boha. Ty, které jsou, jsou zřízeny od Boha,
#13:2 takže ten, kdo se staví proti vládnoucí moci, vzpírá se Božímu řádu. Kdo se takto vzpírá, přivolává na sebe soud.
#13:3 Vládcové nejsou přece hrozbou tomu, kdo jedná dobře, nýbrž tomu, kdo jedná zle. Chceš, aby ses nemusel bát vládnoucí moci? Jednej dobře, a dostane se ti od ní pochvaly.
#13:4 Vždyť je Božím služebníkem k tvému dobru. Jednáš-li však špatně, máš proč se bát, neboť nenese meč nadarmo; je Božím služebníkem, vykonavatelem trestu nad tím, kdo činí zlo.
#13:5 Proto je nutno podřizovat se, a to nejen z bázně před trestem, nýbrž i pro svědomí.
#13:6 Proto také platíte daň. Vládcové jsou v Boží službě, když se drží svých úkolů.
#13:7 Dávejte každému, co jste povinni: daň, komu daň; clo, komu clo; úctu, komu úctu; čest, komu čest.
#13:8 Nikomu nebuďte nic dlužni, než abyste se navzájem milovali, neboť ten, kdo miluje druhého, naplnil zákon.
#13:9 Vždyť přikázání ‚nezcizoložíš, nezabiješ, nepokradeš, nepožádáš‘ a kterákoli jiná jsou shrnuta v tomto slovu: ‚Milovati budeš bližního svého jako sebe samého.‘
#13:10 Láska neudělá bližnímu nic zlého. Je tedy láska naplněním zákona.
#13:11 Víte přece, co znamená tento čas: už nastala hodina, abyste procitli ze spánku; vždyť nyní je nám spása blíže, než byla tenkrát, když jsme uvěřili.
#13:12 Noc pokročila, den se přiblížil. Odložme proto skutky tmy a oblecme se ve zbroj světla.
#13:13 Žijme řádně jako za denního světla: ne v hýření a opilství, v nemravnosti a bezuzdnostech, ne ve sváru a závisti,
#13:14 nýbrž oblecte se v Pána Ježíše Krista a nevyhovujte svým sklonům, abyste nepropadali vášním. 
#14:1 Bratra ve víře slabšího přijímejte mezi sebe, ale nepřete se s ním o jeho názorech.
#14:2 Někdo třeba věří, že může jíst všechno, slabý však jí jen rostlinnou stravu.
#14:3 Ten, kdo jí všecko, nechť nezlehčuje toho, kdo nejí, a kdo nejí, nechť neodsuzuje toho, kdo jí; vždyť Bůh jej přijal za svého.
#14:4 Kdo jsi ty, že soudíš cizího služebníka? O tom, zda obstojí či ne, rozhoduje jeho vlastní pán. A on obstojí, neboť Pán má moc jej podepřít.
#14:5 Někdo rozlišuje dny, jinému je den jako den. Každý nechť má jistotu svého přesvědčení.
#14:6 Kdo zachovává určité dny, zachovává je Pánu. Kdo jí, dělá to Pánu ke cti, neboť děkuje Bohu; a kdo nejí, dělá to také Pánu ke cti, neboť i on děkuje Bohu.
#14:7 Nikdo z nás nežije sám sobě a nikdo sám sobě neumírá. Žijeme-li, žijeme Pánu,
#14:8 umíráme-li, umíráme Pánu. Ať žijeme, ať umíráme, patříme Pánu.
#14:9 Vždyť proto Kristus umřel i ožil, aby se stal Pánem i mrtvých i živých.
#14:10 Proč tedy, ty slabý, soudíš svého bratra? A ty, silný, proč zlehčuješ svého bratra? Všichni přece staneme před soudnou stolicí Boží.
#14:11 Neboť je psáno: ‚Jako že jsem živ, praví Hospodin, skloní se přede mnou každé koleno a každý jazyk vyzná, že jsem Bůh.‘
#14:12 Každý z nás tedy sám za sebe vydá počet Bohu.
#14:13 Nesuďme už tedy jeden druhého, ale raději posuďte, jak jednat, abyste nekladli bratru do cesty kámen úrazu a nepůsobili pohoršení.
#14:14 Vím a jsem přesvědčen v Pánu Ježíši, že nic není nečisté samo v sobě, ale tomu, kdo něco pokládá za nečisté, je to nečisté.
#14:15 Trápí-li se tvůj bratr pro to, co jíš, nežiješ už v lásce. Neuváděj tedy svým jídlem do záhuby toho, za nějž Kristus umřel!
#14:16 Nevydávej v potupu to dobré, co jste přijali.
#14:17 Vždyť království Boží není v tom, co jíte a pijete, nýbrž ve spravedlnosti, pokoji a radosti z Ducha svatého.
#14:18 Kdo takto slouží Kristu, je milý Bohu a lidé si ho váží.
#14:19 A tak usilujme o to, co slouží pokoji a společnému růstu.
#14:20 Nenič kvůli pokrmu Boží dílo! Ano, všecko je čisté, zlé však je, když někdo pohoršuje druhého tím, co jí.
#14:21 Je tedy dobré nejíst maso a nepít víno a nedělat nic, co je tvému bratru kamenem úrazu.
#14:22 Tvé přesvědčení ať zůstane mezi tebou a Bohem. Blaze tomu, kdo sám sebe neodsuzuje, když se pro něco rozhodl.
#14:23 Ten však, kdo pochybuje, byl by odsouzen, kdyby jedl, neboť by to nebylo z víry. A cokoli není z víry, je hřích. 
#15:1 My silní jsme povinni snášet slabosti slabých a nemít zalíbení sami v sobě.
#15:2 Každý z nás ať vychází vstříc bližnímu, aby to bylo k dobru společného růstu.
#15:3 Vždyť Kristus neměl zalíbení sám v sobě, nýbrž podle slov Písma: ‚Urážky těch, kdo tě tupí, padly na mne.‘
#15:4 Všecko, co je tam psáno, bylo napsáno k našemu poučení, abychom z trpělivosti a z povzbuzení, které nám dává Písmo, čerpali naději.
#15:5 Bůh trpělivosti a povzbuzení ať vám dá, abyste jedni i druzí stejně smýšleli po příkladu Krista Ježíše,
#15:6 a tak svorně jedněmi ústy slavili Boha a Otce našeho Pána Ježíše Krista.
#15:7 Proto přijímejte jeden druhého, tak jako Kristus k slávě Boží přijal vás.
#15:8 Chci říci: Kristus se stal služebníkem židů, aby ukázal Boží věrnost a potvrdil sliby, dané otcům,
#15:9 a pohanské národy aby slavily Boha za jeho slitování, jak je psáno: ‚Proto vzdám tobě chválu mezi národy a jménu tvému žalmy zpívati budu.‘
#15:10 A dále je řečeno: ‚Radujte se, pohané, spolu s jeho lidem.‘
#15:11 A opět: ‚Chvalte Hospodina všichni národové a vzdej mu chválu lid všech zemí.‘
#15:12 A Izaiáš k tomu říká: ‚Přijde potomek Isajův, povstane, aby vládl národům, v něj budou pohané doufat.‘
#15:13 Bůh naděje nechť vás naplní veškerou radostí a pokojem ve víře, aby se rozhojnila vaše naděje mocí Ducha svatého.
#15:14 Jsem přesvědčen také já o vás, bratří moji, že i vy jste plni dobroty, naplněni veškerým poznáním, takže sami můžete ukazovat cestu jeden druhému.
#15:15 V tomto dopise jsem se místy odvážil připomenout vám leccos ve jménu milosti, která mi byla dána od Boha,
#15:16 abych byl služebníkem Krista Ježíše mezi pohanskými národy. Konám tuto posvátnou službu kázáním Božího evangelia, abych pohany přinesl jako obětní dar milý Bohu, posvěcený Duchem svatým.
#15:17 To je má chlouba, kterou mám v Kristu Ježíši před Bohem.
#15:18 Neodvážil bych se totiž mluvit o něčem, co by nevykonal Kristus skrze mne, slovem i skutkem,
#15:19 v moci znamení a divů, v moci Ducha, aby pohané přijali evangelium. Tak jsem celý okruh od Jeruzaléma až po Illyrii naplnil Kristovým evangeliem.
#15:20 Zakládám si na tom, že kážu evangelium tam, kde o Kristu ještě neslyšeli; nechci stavět na cizím základu,
#15:21 ale jak je psáno: ‚Ti, jimž nebylo o něm zvěstováno, uvidí a ti, kteří neslyšeli, pochopí.‘
#15:22 To mi také mnohokrát zabránilo, abych k vám přišel.
#15:23 Nyní však už pro mne není žádné volné pole v těchto končinách; už mnoho let k vám toužím přijít,
#15:24 až se vydám do Hispanie. Doufám tedy, že se u vás zastavím a že mě vypravíte na další cestu, až aspoň trochu užiju radosti ze společenství s vámi.
#15:25 Zatím však se chystám do Jeruzaléma, abych přinesl pomoc tamějším bratřím.
#15:26 Makedonští a Achajští se totiž rozhodli vykonat sbírku ve prospěch chudých bratří v Jeruzalémě.
#15:27 Rozhodli se tak proto, že i oni jsou jejich dlužníky. Jestliže pohané dostali podíl na jejich duchovních darech, jsou zavázáni posloužit jim zase ve věcech hmotných.
#15:28 Až dokončím tento úkol a řádně jim odevzdám výtěžek sbírky, vydám se do Hispanie a zastavím se u vás.
#15:29 Jsem jist, že až k vám dojdu, přijdu s plností Kristova požehnání.
#15:30 Prosím vás, bratří, pro našeho Pána Ježíše Krista a pro lásku, která je z Ducha, pomáhejte mi v boji svými přímluvami u Boha,
#15:31 abych byl zachráněn před nevěřícími v Judsku a aby moje služba pro Jeruzalém byla tamějším bratřím vítaná.
#15:32 Pak budu moci z vůle Boží k vám přijít s radostí a najít mezi vámi chvíli odpočinku.
#15:33 Bůh pokoje buď se všemi vámi. Amen. 
#16:1 Doporučuji vám naši sestru Foibé, diakonku církve v Kenchrejích.
#16:2 Přijměte ji v Pánu, jak se sluší mezi věřícími, pomáhejte jí, kdyby vás v něčem potřebovala, neboť i ona byla pomocnicí mnohým i mně samému.
#16:3 Pozdravujte Prisku a Akvilu, mé spolupracovníky v díle Krista Ježíše,
#16:4 kteří pro mne nasadili život; jsem jim zavázán vděčností nejen já sám, ale i všechny církve z pohanských národů.
#16:5 Pozdravujte také shromáždění v jejich domě. Pozdravujte Epaineta, mně velmi drahého, který se jako první v provincii Asii oddal Kristu.
#16:6 Pozdravujte Marii - tolik se pro vás napracovala!
#16:7 Pozdravujte Andronika a Junia, původem židy jako já a kdysi spoluvězně, apoštoly, kteří se těší zvláštní vážnosti a uvěřili v Krista dříve než já.
#16:8 Pozdravujte Ampliata, kterého miluji v Pánu.
#16:9 Pozdravujte Urbana, mého spolupracovníka v díle Kristově, a Stachya, mně velmi drahého.
#16:10 Pozdravujte Apella, osvědčeného v Kristu. Pozdravujte ty, kteří jsou z domu Aristobulova.
#16:11 Pozdravujte mého krajana Herodiona. Pozdravujte ty z domu Narcisova, kteří se oddali Pánu.
#16:12 Pozdravujte Tryfainu a Tryfósu, které pracují v díle Páně. Pozdravujte Persidu, mně velmi drahou; horlivě pracovala v díle Páně.
#16:13 Pozdravujte Rufa, vyvoleného v Pánu, a jeho matku, která i mně byla matkou.
#16:14 Pozdravujte Asynkrita, Flegonta, Herma, Patrobia, Hermia i bratry, kteří jsou s nimi.
#16:15 Pozdravujte Filologa a Julii, Nerea a jeho sestru, Olympia a všecky věřící, kteří jsou s nimi.
#16:16 Pozdravte jedni druhé svatým políbením. Pozdravují vás všecky církve Kristovy.
#16:17 Prosím vás, bratří, abyste si dali pozor na ty, kdo působí roztržky a chtěli by vás svést od učení, které jste přijali. Vyhýbejte se jim!
#16:18 Takoví lidé totiž neslouží Kristu, našemu Pánu, nýbrž svému prospěchu a snaží se krásnými a pobožnými řečmi oklamat mysli bezelstných lidí.
#16:19 Všude se ví o vaší oddanosti evangeliu. Mám z vás proto radost a přeji si, abyste byli moudří v dobru a nezkušení ve zlu.
#16:20 Bůh pokoje brzo srazí satana pod vaše nohy. Milost Ježíše, našeho Pána, buď s vámi.
#16:21 Pozdravuje vás můj spolupracovník Timoteus i Lucius a Jáson a Sosipatros, moji krajané.
#16:22 Pozdravuji vás já Tercius, písař tohoto dopisu.
#16:23 Pozdravuje vás Gaius, který svůj dům otevřel mně i veškeré církvi. Pozdravuje vás Erastos, správce městské pokladny, a bratr Kvartus.
#16:24 ---
#16:25 Sláva tomu, který má moc upevnit vás ve víře podle mého evangelia a podle zvěsti Ježíše Krista: v ní je odhaleno tajemství, které od věčných časů nebylo vysloveno,
#16:26 nyní je však zjeveno prorockými Písmy a z příkazu věčného Boha stalo se známým mezi všemi národy, aby je poslušně přijali vírou.
#16:27 Jedinému moudrému Bohu buď skrze Ježíše Krista sláva na věky věků. Amen.  

\book{I Corinthians}{1Cor}
#1:1 Pavel, z vůle Boží povolaný apoštol Krista Ježíše, a bratr Sosthenes
#1:2 církvi Boží v Korintu, posvěceným v Kristu Ježíši, povolaným svatým, spolu se všemi, kteří vzývají jméno našeho Pána Ježíše Krista, ať jsou shromážděni kdekoliv, jinde či u nás:
#1:3 Milost vám a pokoj od Boha Otce našeho a Pána Ježíše Krista.
#1:4 Stále za vás Bohu děkuji pro milost Boží, která vám byla dána v Kristu Ježíši;
#1:5 on vás obohatil ve všem, v každém slovu i v každém poznání.
#1:6 Neboť svědectví o Kristu bylo mezi vámi potvrzeno,
#1:7 takže nejste pozadu v žádném daru milosti a čekáte, až se zjeví náš Pán Ježíš Kristus.
#1:8 On vám bude oporou až do konce, abyste v onen den našeho Pána Ježíše Krista nebyli obviněni.
#1:9 Věrný je Bůh, který vás povolal do společenství se svým Synem, naším Pánem Ježíšem Kristem.
#1:10 Prosím vás, bratří, pro jméno našeho Pána Ježíše Krista, abyste všichni byli svorni a neměli mezi sebou roztržky, nýbrž abyste dosáhli plné jednoty smýšlení i přesvědčení.
#1:11 Dověděl jsem se totiž o vás z domu Chloé, bratří, že jsou mezi vámi spory.
#1:12 Myslím tím to, že se mezi vámi říká: Já se hlásím k Pavlovi, já zase k Apollovi, já k Petrovi, já ke Kristu.
#1:13 Je snad Kristus rozdělen? Což byl Pavel za vás ukřižován? Nebo jste byli pokřtěni ve jméno Pavlovo?
#1:14 Děkuji Bohu, že jsem nikoho z vás nepokřtil kromě Krispa a Gaia;
#1:15 tak nemůže nikdo říci, že jste byli pokřtěni v moje jméno.
#1:16 Pokřtil jsem i rodinu Štěpánovu. Jinak už nevím, že bych byl ještě někoho pokřtil.
#1:17 Kristus mě totiž neposlal křtít, ale zvěstovat evangelium, ovšem ne moudrostí slov, aby Kristův kříž nepozbyl smyslu.
#1:18 Slovo o kříži je bláznovstvím těm, kdo jsou na cestě k záhubě; nám, kteří jdeme ke spáse, je mocí Boží.
#1:19 Je psáno: ‚Zahubím moudrost moudrých a rozumnost rozumných zavrhnu.‘
#1:20 Kde jsou učenci, kde znalci, kde řečníci tohoto věku? Neučinil Bůh moudrost světa bláznovstvím?
#1:21 Protože svět svou moudrostí nepoznal Boha v jeho moudrém díle, zalíbilo se Bohu spasit ty, kdo věří, bláznovskou zvěstí.
#1:22 Židé žádají zázračná znamení, Řekové vyhledávají moudrost,
#1:23 ale my kážeme Krista ukřižovaného. Pro Židy je to kámen úrazu, pro ostatní bláznovství,
#1:24 ale pro povolané, jak pro Židy, tak pro Řeky, je Kristus Boží moc a Boží moudrost.
#1:25 Neboť bláznovství Boží je moudřejší než lidé a slabost Boží je silnější než lidé.
#1:26 Pohleďte, bratří, koho si Bůh povolává: Není mezi vámi mnoho moudrých podle lidského soudu, ani mnoho mocných, ani mnoho urozených;
#1:27 ale co je světu bláznovstvím, to vyvolil Bůh, aby zahanbil moudré, a co je slabé, vyvolil Bůh, aby zahanbil silné;
#1:28 neurozené v očích světa a opovržené Bůh vyvolil, ano vyvolil to, co není, aby to, co jest, obrátil v nic -
#1:29 aby se tak žádný člověk nemohl vychloubat před Bohem.
#1:30 Vy však jste z Boží moci v Kristu Ježíši; on se nám stal moudrostí od Boha, spravedlností, posvěcením a vykoupením,
#1:31 jak je psáno: ‚Kdo se chlubí, ať se chlubí v Pánu‘. 
#2:1 Ani já, bratří, když jsem přišel k vám, nepřišel jsem vám hlásat Boží tajemství nadnesenými slovy nebo moudrostí.
#2:2 Rozhodl jsem se totiž, že mezi vámi nebudu znát nic než Ježíše Krista, a to Krista ukřižovaného.
#2:3 Přišel jsem k vám sláb, s velkou bázní a chvěním;
#2:4 má řeč a mé kázání se neopíraly o vemlouvavá slova lidské moudrosti, ale prokazovaly se Duchem a mocí,
#2:5 aby se tak vaše víra nezakládala na moudrosti lidské, ale na moci Boží.
#2:6 Moudrosti sice učíme, ale jen ty, kteří jsou dospělí ve víře - ne ovšem moudrosti tohoto věku či vládců tohoto věku, spějících k záhubě,
#2:7 nýbrž moudrosti Boží, skryté v tajemství, kterou Bůh od věčnosti určil pro naše oslavení.
#2:8 Tu moudrost nikdo z vládců tohoto věku nepoznal; neboť kdyby ji byli poznali, nebyli by ukřižovali Pána slávy.
#2:9 Ale jak je psáno: ‚Co oko nevidělo a ucho neslyšelo, co ani člověku na mysl nepřišlo, připravil Bůh těm, kdo ho milují.‘
#2:10 Nám však to Bůh zjevil skrze Ducha; Duch totiž zkoumá všechno, i hlubiny Boží.
#2:11 Neboť kdo z lidí zná, co je v člověku, než jeho vlastní duch? Právě tak nikdo nepoznal, co je v Bohu, než Duch Boží.
#2:12 My jsme však nepřijali ducha světa, ale Ducha, který je z Boha,
#2:13 abychom poznali, co nám Bůh daroval. O tom i mluvíme ne tak, jak nás naučila lidská moudrost, ale jak nás naučil Duch, a duchovní věci vykládáme slovy Ducha.
#2:14 Přirozený člověk nemůže přijmout věci Božího Ducha; jsou mu bláznovstvím a nemůže je chápat, protože se dají posoudit jen Duchem.
#2:15 Člověk obdařený Duchem je schopen posoudit všecko, ale sám nemůže být nikým správně posouzen.
#2:16 Vždyť ‚kdo poznal mysl Páně a kdo ho bude poučovat?‘ My však mysl Kristovu máme. 
#3:1 Já jsem k vám, bratří, nemohl mluvit jako k těm, kteří mají Ducha, nýbrž jako k těm, kteří myslí jen po lidsku, jako k nedospělým v Kristu.
#3:2 Mlékem jsem vás živil, pokrm jsem vám jíst nedával, protože jste jej nemohli snést - ani teď ještě nemůžete,
#3:3 neboť dosud patříte světu. Odkud je mezi vámi závist a svár, ne-li z toho, že patříte světu a žijete jako ostatní lidé?
#3:4 Když se jeden z vás hlásí k Pavlovi a druhý k Apollovi, neznamená to, že jste lidé světa?
#3:5 Kdo je vlastně Apollos? A kdo je Pavel? Služebníci, kteří vás přivedli k víře, každý tak, jak mu dal Pán.
#3:6 Já jsem zasadil, Apollos zaléval, ale Bůh dal vzrůst;
#3:7 a tak nic neznamená ten, kdo sází, ani kdo zalévá, nýbrž Bůh, který dává vzrůst.
#3:8 Kdo sází a kdo zalévá, patří k sobě, ale každý podle vlastní práce obdrží svou odměnu.
#3:9 Jsme spolupracovníci na Božím díle, a vy jste Boží pole, Boží stavba.
#3:10 Podle milosti Boží, která mi byla dána, jako rozumný stavitel jsem položil základ a druhý na něm staví. Každý ať dává pozor, jak na něm staví.
#3:11 Nikdo totiž nemůže položit jiný základ než ten, který už je položen, a to je Ježíš Kristus.
#3:12 Zda někdo na tomto základu staví ze zlata, stříbra, drahého kamení, či ze dřeva, trávy, slámy -
#3:13 dílo každého vyjde najevo. Ukáže je onen den, neboť se zjeví v ohni; a oheň vyzkouší, jaké je dílo každého člověka.
#3:14 Když jeho dílo vydrží, dostane odměnu.
#3:15 Když mu dílo shoří, utrpí škodu; sám bude sice zachráněn, ale projde ohněm.
#3:16 Nevíte, že jste Boží chrám a že Duch Boží ve vás přebývá?
#3:17 Kdo ničí chrám Boží, toho zničí Bůh; neboť Boží chrám je svatý, a ten chrám jste vy.
#3:18 Ať nikdo sám sebe neklame. Domnívá-li se někdo z vás, že je v tomto světě moudrý, ať se stane bláznem, aby se stal opravdu moudrým.
#3:19 Moudrost tohoto věku je bláznovstvím před Bohem, neboť je psáno: ‚Nachytá moudré na jejich vychytralost‘;
#3:20 a jinde: ‚Hospodin zná úmysly moudrých a ví, že jsou marné.‘
#3:21 A tak ať se nikdo nechlubí lidmi. Všechno je vaše,
#3:22 ať Pavel nebo Apollos nebo Petr, ať svět nebo život nebo smrt, přítomnost nebo budoucnost, všechno je vaše,
#3:23 vy však jste Kristovi a Kristus je Boží. 
#4:1 Proto ať nás všichni pokládají za služebníky Kristovy a správce Božích tajemství.
#4:2 Od správců se nežádá nic jiného, než aby byl každý shledán věrným.
#4:3 Mně tedy pramálo záleží na tom, soudíte-li mě vy nebo jakýkoliv lidský soud. Vždyť ani já nejsem soudcem sám nad sebou;
#4:4 ničeho si nejsem sice vědom, tím však ještě nejsem ospravedlněn, neboť mým soudcem je Pán.
#4:5 Nevyslovujte proto soudy předčasně, dokud Pán nepřijde. On vynese na světlo to, co je skryto ve tmě, a zjeví záměry srdcí; tehdy se člověku dostane chvály od Boha.
#4:6 Toto jsem, bratří, kvůli vám vztáhl na sebe a na Apolla, abyste se na nás naučili, co znamená ‚ne nad to, co je psáno‘, a nikdo se nepyšnil jedním z nás proti druhému.
#4:7 Kdo ti dal vyniknout? Máš něco, co bys nebyl dostal? A když jsi to dostal, proč se chlubíš, jako bys to nebyl dostal?
#4:8 Už jste nasyceni, už jste zbohatli, vešli jste do Božího království, a my ne! Kéž byste skutečně vešli do království, abychom i my kralovali spolu s vámi!
#4:9 Skoro se mi zdá, že nás apoštoly Bůh určil na poslední místo, jako vydané na smrt; stali jsme se podívanou světu, andělům i lidem.
#4:10 My jsme blázni pro Krista, vy ovšem jste v Kristu rozumní; my jsme slabí, vy silní; vy slavní, my beze cti.
#4:11 Až do této chvíle trpíme hladem a žízní a nemáme co na sebe, jsme biti, jsme bez domova,
#4:12 s námahou pracujeme svýma rukama. Jsme-li tupeni, žehnáme, pronásledováni neklesáme,
#4:13 když nám zlořečí, odpovídáme laskavě. Až dosud jsme vyděděnci světa, na které se všechno svaluje.
#4:14 Nepíši to proto, abych vás zahanbil, ale abych vás jako své milované děti napomenul.
#4:15 I kdybyste měli tisíce vychovatelů v Kristu, otců mnoho nemáte, neboť v Kristu Ježíši jsem vás já přivedl k životu skrze evangelium.
#4:16 Prosím vás: Jednejte podle mého příkladu!
#4:17 Právě proto jsem k vám poslal Timotea, svého v Pánu milovaného a věrného syna. On vám připomene můj způsob jednání v Kristu, jak učím v každé církvi.
#4:18 Někteří se začali povyšovat, protože prý už k vám nepřijdu;
#4:19 ale přijdu k vám brzy, bude-li Pán chtít, a učiním si úsudek o těch domýšlivcích ne podle toho, co mluví, ale podle toho, co dokážou.
#4:20 Království Boží nezáleží v slovech, nýbrž v moci.
#4:21 Vyberte si: Mám na vás přijít s holí, nebo s láskou a mírností? 
#5:1 Dokonce se proslýchá, že je mezi vámi případ smilstva, a to takový, jaký se nevyskytuje ani mezi pohany, že totiž kdosi žije s ženou svého otce.
#5:2 A vy jste přitom nadutí, místo abyste se raději zarmoutili; odstraňte ze svého středu toho, kdo to udělal.
#5:3 Neboť já, ač tělem vzdálen, duchem však přítomen, vyslovil jsem již soud nad tím, kdo se toho dopustil, jako bych byl s vámi -
#5:4 a to ve jménu Pána Ježíše Krista. Až se shromáždíte - já budu duchem s vámi a bude s námi i moc našeho Pána Ježíše -
#5:5 vydejte toho člověka satanu ke zkáze těla, aby duch mohl být zachráněn v den Páně.
#5:6 Vaše vychloubání není dobré. Nevíte, že ‚trocha kvasu všechno těsto prokvasí‘?
#5:7 Odstraňte starý kvas, abyste byli novým těstem, vždyť vám nastal čas nekvašených chlebů, neboť byl obětován náš velikonoční beránek, Kristus.
#5:8 Proto slavme velikonoce ne se starým kvasem, s kvasem zla a špatnosti, ale s nekvašeným chlebem upřímnosti a pravdy.
#5:9 Napsal jsem vám v listě, abyste neměli nic společného se smilníky;
#5:10 ale nemyslel jsem tím všecky smilníky tohoto světa nebo lakomce, lupiče a modláře, protože to byste museli ze světa utéci.
#5:11 Měl jsem však na mysli, abyste se nestýkali s tím, kdo si sice říká bratr, ale přitom je smilník nebo lakomec nebo modlář nebo utrhač nebo opilec nebo lupič; s takovým ani nejezte.
#5:12 Proč bych měl soudit ty, kdo jsou mimo nás? Nemáte soudit ty, kdo jsou z nás?
#5:13 Kdo jsou mimo nás, ty bude soudit Bůh. Odstraňte toho zlého ze svého středu! 
#6:1 Jak to, že se někdo z vás opovažuje, má-li spor s druhým, jít k pohanským soudcům místo k bratřím?
#6:2 Což nevíte, že Boží lid bude soudit svět? Jestliže budete soudit svět, nejste snad schopni rozsuzovat takové maličkosti?
#6:3 Nevíte, že budeme soudit anděly? Tím spíše věci všedního života!
#6:4 Máte-li spory o tyto všední záležitosti, proč se obracíte k těm, kdo nemají s církví nic společného?
#6:5 K vašemu zahanbení to říkám. Cožpak není mezi vámi nikdo rozumný, kdo by dovedl rozsoudit spor mezi bratřími?
#6:6 Ale bratr se soudí s bratrem, a to před nevěřícími!
#6:7 Již to je vaše prohra, že se vůbec mezi sebou soudíte. Proč raději netrpíte křivdu? Proč raději nenesete škodu?
#6:8 Vy však křivdíte a škodíte, a to bratřím!
#6:9 Což nevíte, že nespravedliví nebudou mít účast v Božím království? Nemylte se: Ani smilníci, ani modláři, ani cizoložníci, ani nemravní, ani zvrácení,
#6:10 ani zloději, ani lakomci, opilci, utrhači, lupiči nebudou mít účast v Božím království.
#6:11 A to jste někteří byli. Dali jste se však obmýt, byli jste posvěceni, byli jste ospravedlněni ve jménu Pána Ježíše Krista a Duchem našeho Boha.
#6:12 ‚Všechno je mi dovoleno‘ - ano, ale ne všechno prospívá. ‚Všechno je mi dovoleno‘ - ano, ale ničím se nedám zotročit.
#6:13 Jídlo je pro žaludek a žaludek pro jídlo; Bůh však jednou učiní konec obojímu. Tělo však není pro smilstvo, nýbrž pro Pána, a Pán pro tělo.
#6:14 Bůh, který vzkřísil Pána, vzkřísí svou mocí i nás.
#6:15 Nevíte, že vaše těla jsou údy Kristovými? Mám tedy z údů Kristových učinit údy nevěstky? Rozhodně ne!
#6:16 Což nevíte, že kdo se oddá nevěstce, je s ní jedno tělo? Je přece řečeno ‚budou ti dva jedno tělo‘.
#6:17 Kdo se oddá Pánu, je s ním jeden duch.
#6:18 Varujte se smilstva! Žádný jiný hřích, kterého by se člověk dopustil, netýká se jeho těla; kdo se však dopouští smilstva, hřeší proti vlastnímu tělu.
#6:19 Či snad nevíte, že vaše tělo je chrámem Ducha svatého, který ve vás přebývá a jejž máte od Boha? Nepatříte sami sobě!
#6:20 Bylo za vás zaplaceno výkupné. Proto svým tělem oslavujte Boha. 
#7:1 Pokud jde o to, co jste psali: Je pro muže lépe, když žije bez ženy.
#7:2 Abyste se však uvarovali smilstva, ať každý má svou ženu a každá svého muže.
#7:3 Muž ať prokazuje ženě, čím je jí povinen, a podobně i žena muži.
#7:4 Žena nemá své tělo pro sebe, ale pro svého muže. Podobně však ani muž nemá své tělo pro sebe, ale pro svou ženu.
#7:5 Neodpírejte se jeden druhému, leda se vzájemným souhlasem a jen na čas, abyste byli volni pro modlitbu. Potom zase buďte spolu, aby vás satan nepokoušel, když byste se nemohli ovládnout.
#7:6 To ovšem říkám jako ústupek, ne jako příkaz.
#7:7 Přál bych si totiž, aby všichni lidé byli jako já; ale každý má od Boha svůj vlastní dar, jeden tak, druhý jinak.
#7:8 Svobodným a vdovám pravím, že je pro ně lépe, když zůstanou tak jako já.
#7:9 Je-li jim zatěžko žít zdrženlivě, ať vstoupí v manželství, neboť je lepší žít v manželství než se trápit.
#7:10 Těm, kteří žijí v manželství, přikazuji - ne já, ale Pán - aby žena od muže neodcházela.
#7:11 A když už odejde, ať zůstane neprovdána nebo se s mužem smíří; a muž ať ženu neopouští.
#7:12 Ostatním pravím já, a ne už Pán: Má-li někdo z bratří ženu nevěřící a ona je ochotna s ním zůstat, ať ji neopouští.
#7:13 A má-li žena muže nevěřícího a on je ochoten s ní zůstat, ať ho neopouští.
#7:14 Nevěřící muž je totiž posvěcen manželstvím s věřící ženou a nevěřící žena manželstvím s věřícím mužem, jinak by vaše děti byly nečisté; jsou však přece svaté!
#7:15 Chce-li nevěřící odejít, ať odejde. Věřící nejsou v takových případech vázáni. Bůh nás povolal k pokoji.
#7:16 Víš snad, ženo, zda se ti podaří přivést muže ke spáse? Víš snad, muži, zda se ti podaří přivést ženu ke spáse?
#7:17 Každý ať žije v tom postavení, které měl od Pána, když ho povolal k víře. A tak to nařizuji v církvích všude.
#7:18 Jsi povolán jako žid? Nezatajuj svou obřízku. Jsi povolán jako pohan? Nedávej se obřezat.
#7:19 Nezáleží na tom, zda je někdo obřezán nebo není, ale na tom, zda zachovává Boží přikázání.
#7:20 Nikdo ať neopouští postavení, v němž ho Bůh povolal.
#7:21 Byl jsi povolán jako otrok? Netrap se tím. Ale kdyby ses mohl stát svobodným, raději toho použij.
#7:22 Koho Pán povolal jako otroka, má v Pánu svobodu. Koho povolal jako svobodného, je v poddanství Kristově.
#7:23 Bylo za vás zaplaceno výkupné, nebuďte otroky lidí!
#7:24 V čem byl kdo povolán, bratří, v tom ať před Bohem zůstane.
#7:25 Co se týká neprovdaných, nemám žádný rozkaz Páně, ale dávám jen radu jako ten, kdo je pro milosrdenství Páně hoden důvěry.
#7:26 Domnívám se, že vzhledem k tomu, co má přijít, je pro člověka nejlepší zůstat tak, jak je.
#7:27 Máš ženu? Nechtěj se s ní rozejít. Jsi bez ženy? Žádnou nehledej.
#7:28 Ale i když se oženíš, nezhřešíš. A vdá-li se dívka, nezhřeší. Dolehne však na ně tíseň tohoto času. Toho vás chci ušetřit.
#7:29 Chci říci, bratří, toto: Lhůta je krátká. Proto ti, kdo mají ženy, ať jsou, jako by je neměli,
#7:30 a kdo pláčí, jako by neplakali, a kdo jsou veselí, jako by nebyli, a kdo kupují, jako by nevlastnili,
#7:31 a kdo užívají věcí tohoto světa, jako by neužívali; neboť podoba tohoto světa pomíjí.
#7:32 Já bych však chtěl, abyste neměli starosti. Svobodný se stará o věci Páně, jak by se líbil Bohu,
#7:33 ale ženatý se stará o světské věci, jak by se zalíbil ženě, a je rozpolcen.
#7:34 Žena bez manžela nebo svobodná dívka se stará o věci Páně, aby byla svatá tělem i duchem. Provdaná se stará o světské věci, jak by se zalíbila muži.
#7:35 To vám říkám, abych vám pomohl, ne abych vás uvedl do nesnází, ale abyste žili důstojně a věrně lnuli k Pánu bez rozptylování.
#7:36 Domnívá-li se někdo, že jedná nečestně vůči své snoubence, která je už ve zralém věku, a že se patří, aby si ji vzal, ať udělá, co chce: nehřeší. Ať se vezmou.
#7:37 Ale kdo je v nitru pevný, nic ho nenutí, má v moci svou vlastní vůli a pevně se rozhodl, že se s ní neožení, jedná dobře.
#7:38 Takže kdo se ožení se svou snoubenkou, jedná dobře, ale kdo se neožení, udělá lépe.
#7:39 Žena je vázána zákonem, pokud žije její muž. Jestliže muž umře, je svobodná a může se vdát, za koho chce, ale jen v Kristu.
#7:40 Po mém soudu bude však pro ni lépe, zůstane-li tak; myslím, že i já mám Ducha Božího. 
#8:1 Pokud jde o maso obětované modlám, víme, že ‚všichni máme poznání‘. Poznání však vede k domýšlivosti, kdežto láska buduje.
#8:2 Jestliže si někdo myslí, že něco už plně poznal, ten ještě nepoznal tak, jak je třeba.
#8:3 Kdo však miluje Boha, je od něho poznán.
#8:4 Pokud tedy jde o to, zda se smí jíst maso obětované modlám, víme, že modly ani bohové tohoto světa nic nejsou a že jest jen jeden Bůh.
#8:5 I když jsou tak zvaní bohové na nebi či na zemi - jakože je mnoho takových bohů a pánů -
#8:6 my přece víme, že je jediný Bůh Otec, od něhož je všecko, a my jsme tu pro něho, a jediný Pán Ježíš Kristus, skrze něhož je všecko, i my jsme skrze něho.
#8:7 Ale všichni nemají toto poznání. Někteří jsou až podnes tak zvyklí na modly, že jedí toto maso jako oběti modlám; jejich svědomí je nejisté, a proto je poskvrněno.
#8:8 Pokrm nás Bohu nepřiblíží; nejíme-li obětované maso, nic neztrácíme, jíme-li, nic nezískáme.
#8:9 Dejte si pozor, aby se tato vaše svoboda nestala kamenem úrazu pro slabé.
#8:10 Když někdo tebe, který máš poznání, uvidí za stolem v pohanském chrámě, zda tím nepřivedeš svědomí toho slabého bratra k tomu, aby také jedl maso obětované modlám?
#8:11 Jenže tak bude ten slabý tvým poznáním uveden do záhuby - bratr, pro kterého Kristus zemřel!
#8:12 Když takto hřešíte proti bratřím a ubíjíte jejich slabé svědomí, hřešíte proti Kristu.
#8:13 A proto: je-li jídlo kamenem úrazu pro mého bratra, nechci už nikdy jíst maso, abych nepřivedl svého bratra k pádu. 
#9:1 Nejsem snad svoboden? Nejsem apoštol? Neviděl jsem Ježíše, našeho Pána? Nejste vy mým dílem v Pánu?
#9:2 I kdybych pro jiné apoštolem nebyl, pro vás jím jsem! Vždyť pečetí mého apoštolství jste vy sami tím, že jste uvěřili v Pána.
#9:3 Toto je má odpověď těm, kteří mne soudí:
#9:4 Což nemáme právo přijímat od vás jídlo a pití?
#9:5 Nemáme právo brát s sebou věřící ženu, tak jako ostatní apoštolové i bratři Páně i Petr?
#9:6 To snad jen já a Barnabáš jsme povinni vydělávat si na živobytí?
#9:7 Který voják slouží za vlastní peníze? Kdo vysadí vinici a nejí, co urodila? Kdo pase stádo a neživí se mlékem toho stáda?
#9:8 Jsou to snad jen mé lidské úvahy? Neříká to sám zákon?
#9:9 V Mojžíšově zákoně je přece psáno: ‚Nedáš náhubek dobytčeti, když mlátí obilí‘. Má tu Bůh na mysli dobytek?
#9:10 Není to řečeno spíše pro nás? Ano, pro nás to bylo napsáno, neboť ten, kdo orá a kdo mlátí, má pracovat s nadějí, že dostane svůj podíl.
#9:11 Když jsme vám zaseli duchovní setbu, bylo by to mnoho, kdybychom sklízeli vaši pozemskou úrodu?
#9:12 Mají-li na vás právo jiní, proč ne tím spíše my? Přesto jsme tohoto práva nepoužili, raději snášíme nedostatek, jen abychom nekladli žádnou překážku do cesty Kristovu evangeliu.
#9:13 Což nevíte, že ti, kteří slouží ve svatyni, dostávají ze svatyně jídlo, a ti, kteří přisluhují při oltáři, mají podíl na obětech?
#9:14 Tak i Pán ustanovil, aby ti, kteří zvěstují evangelium, měli z evangelia obživu.
#9:15 Já jsem však ničeho z toho nepoužil. Nepíšu o tom proto, abych se toho dožadoval. To bych raději umřel hladem, než aby mne někdo zbavil této chlouby!
#9:16 Nemohu se chlubit tím, že kážu evangelium; nemohu jinak, běda mně, kdybych nekázal.
#9:17 Kdybych to činil ze své vůle, mám nárok na odměnu; jestliže jsem byl povolán, plním svěřený úkol.
#9:18 Zač tedy mohu čekat odměnu? Za to, že přináším evangelium zadarmo, že jsem se vzdal práva, které mám jako kazatel evangelia.
#9:19 Jsem svoboden ode všech, ale učinil jsem se otrokem všech, abych mnohé získal.
#9:20 Židům jsem byl židem, abych získal židy. Těm, kteří jsou pod zákonem, byl jsem pod zákonem, abych získal ty, kteří jsou pod zákonem - i když sám pod zákonem nejsem.
#9:21 Těm, kteří jsou bez zákona, byl jsem bez zákona, abych získal ty, kteří jsou bez zákona - i když před Bohem nejsem bez zákona, neboť mým zákonem je Kristus.
#9:22 Těm, kdo jsou slabí, stal jsem se slabým, abych získal slabé. Všem jsem se stal vším, abych získal aspoň některé.
#9:23 Všecko to dělám pro evangelium, abych na něm měl podíl.
#9:24 Nevíte snad, že ti, kteří běží na závodní dráze, běží sice všichni, ale jen jeden dostane cenu? Běžte tak, abyste ji získali!
#9:25 Každý závodník se podrobuje všestranné kázni. Oni to podstupují pro pomíjitelný věnec, my však pro věnec nepomíjitelný.
#9:26 Já tedy běžím ne jako bez cíle; bojuji ne tak, jako bych dával rány do prázdna.
#9:27 Ranami nutím své tělo ke kázni, abych snad, když kážu jiným, sám neselhal. 
#10:1 Chtěl bych vám připomenout, bratří, že naši praotcové byli všichni pod oblakovým sloupem, všichni prošli mořem,
#10:2 všichni byli křtem v oblaku a moři spojeni s Mojžíšem,
#10:3 všichni jedli týž duchovní pokrm
#10:4 a pili týž duchovní nápoj; pili totiž z duchovní skály, která je doprovázela, a tou skálou byl Kristus.
#10:5 A přece se většina z nich Bohu nelíbila; vždyť ‚poušť byla poseta jejich těly‘.
#10:6 To vše se stalo nám na výstrahu, abychom nezatoužili po zlém jako oni.
#10:7 A také nebuďte modláři jako někteří z nich, jak je psáno: ‚Usadil se lid, aby jedl a pil, a potom povstali k tancům.‘
#10:8 Ani se neoddávejme smilstvu, jako někteří z nich, a padlo jich za jeden den třiadvacet tisíc.
#10:9 A také nechtějme zkoušet Pána, jako to dělali někteří z nich a hynuli od hadího uštknutí,
#10:10 ani nereptejte, jako někteří z nich, a byli zahubeni Zhoubcem.
#10:11 To, co se jim stalo, je výstražný obraz a bylo to napsáno k napomenutí nám, které zastihl přelom věků.
#10:12 A proto ten, kdo si myslí, že stojí, ať si dá pozor, aby nepadl.
#10:13 Nepotkala vás zkouška nad lidské síly. Bůh je věrný: nedopustí, abyste byli podrobeni zkoušce, kterou byste nemohli vydržet, nýbrž se zkouškou vám připraví i východisko a dá vám sílu, abyste mohli obstát.
#10:14 A proto, moji milovaní, utíkejte před modlářstvím.
#10:15 Mluvím k vám jako k rozumným lidem; posuďte sami, co říkám:
#10:16 Není kalich požehnání, za nějž děkujeme, účastí na krvi Kristově? A není chléb, který lámeme, účastí na těle Kristově?
#10:17 Protože je jeden chléb, jsme my mnozí jedno tělo, neboť všichni máme podíl na jednom chlebu.
#10:18 Pohleďte na Izraelský lid: Nespojuje ty, kteří jedí oběti, společenství oltáře?
#10:19 Co tím chci říci? Že pokrm obětovaný modlám něco znamená? Nebo že modla něco znamená?
#10:20 Nikoli, nýbrž že to, co pohané obětují, obětují démonům, a ne Bohu. Nechci, abyste vešli ve společenství s démony.
#10:21 Nemůžete pít kalich Páně i kalich démonů. Nemůžete mít účast na stolu Páně i na stolu démonů.
#10:22 Chceme snad popudit Pána k žárlivosti? Jsme snad silnější než on?
#10:23 ‚Všecko je dovoleno‘ - ano, ale ne všecko prospívá. ‚Všecko je dovoleno‘ - ano, ale ne všecko přispívá ke společnému růstu.
#10:24 Nikdo ať nemyslí sám na sebe, nýbrž ať má ohled na druhého!
#10:25 Jezte všechno, co se prodává v masných krámech, a pro své svědomí se nepotřebujete na nic ptát.
#10:26 ‚Hospodinova je země i její plnost.‘
#10:27 Pozve-li vás někdo z nevěřících a chcete tam jít, jezte všechno, co vám předloží, a pro své svědomí se na nic neptejte.
#10:28 Kdyby vám však někdo řekl: „Toto bylo obětováno božstvům“, nejezte kvůli tomu, kdo vás na to upozornil, a pro svědomí-
#10:29 nemyslím vaše svědomí, nýbrž svědomí toho druhého. Proč by moje svoboda měla být souzena cizím svědomím?
#10:30 Jestliže něco přijímám s vděčností, proč mám být odsuzován za to, zač vzdávám díky?
#10:31 Ať tedy jíte či pijete či cokoli jiného děláte, všecko čiňte k slávě Boží.
#10:32 Nebuďte kamenem úrazu ani Židům, ani Řekům, ani církvi Boží,
#10:33 já se také snažím všem vyjít vstříc. Nehledám svůj vlastní prospěch, nýbrž prospěch mnohých, aby byli spaseni. 
#11:1 Jednejte podle mého příkladu, jako já jednám podle příkladu Kristova.
#11:2 Chválím vás, že si mne stále připomínáte a držíte se tradice, kterou jste ode mne přijali.
#11:3 Rád bych, abyste si uvědomili, že hlavou každého muže je Kristus, hlavou ženy muž a hlavou Krista je Bůh.
#11:4 Každý muž, který se modlí nebo prorocky mluví s pokrytou hlavou, zneuctívá toho, kdo je mu hlavou,
#11:5 a každá žena, která se modlí nebo prorocky mluví s nezahalenou hlavou, zneuctívá toho, kdo je jí hlavou; je to jedno a totéž, jako kdyby byla ostříhaná.
#11:6 Jestliže si žena nezahaluje hlavu, ať se už také ostříhá. Je-li však pro ženu potupné dát se ostříhat nebo oholit, ať se zahaluje.
#11:7 Muž si nemá zahalovat hlavu, protože je obrazem a odleskem slávy Boží, kdežto žena je odleskem slávy mužovy.
#11:8 Vždyť muž není z ženy, nýbrž žena z muže.
#11:9 Muž přece nebyl stvořen pro ženu, ale žena pro muže.
#11:10 Proto má žena mít na hlavě znamení moci kvůli andělům.
#11:11 V Kristu ovšem není žena bez muže ani muž bez ženy,
#11:12 vždyť jako je žena z muže, tak i muž skrze ženu - všecko pak je z Boha.
#11:13 Posuďte to sami: Sluší se, aby se žena k Bohu modlila s nezahalenou hlavou?
#11:14 Cožpak vás sama příroda neučí, že pro muže jsou dlouhé vlasy hanbou,
#11:15 kdežto pro ženu ctí? Vlasy jsou jí totiž dány jako závoj.
#11:16 Chce-li někdo umíněně na tom trvat, tomu říkám: Není to obyčejem ani u nás, ani v ostatních církvích Božích.
#11:17 Když už vás napomínám, nemohu také pochválit, že se shromažďujete spíše ke škodě než k prospěchu.
#11:18 Předně slyším, že jsou mezi vámi roztržky, když se v církvi shromažďujete, a jsem nakloněn tomu věřit.
#11:19 Neboť musí mezi vámi být i různé skupiny, aby se ukázalo, kdo z vás se osvědčí.
#11:20 Když vy se však shromažďujete, není to už společenství večeře Páně:
#11:21 každý se dá hned do své večeře, a jeden má hlad, druhý se opije.
#11:22 Což nemáte své domácnosti, kde byste jedli a pili? Či snad pohrdáte církví Boží a chcete zahanbit ty, kteří nic nemají? Co vám mám říci? Mám vás snad pochválit? Za to vás nechválím!
#11:23 Já jsem přijal od Pána, co jsem vám také odevzdal: Pán Ježíš v tu noc, kdy byl zrazen, vzal chléb,
#11:24 vzdal díky, lámal jej a řekl: „Toto jest mé tělo, které se za vás vydává; to čiňte na mou památku.“
#11:25 Stejně vzal po večeři i kalich a řekl: „Tento kalich je nová smlouva, zpečetěná mou krví; to čiňte, kdykoli budete píti, na mou památku.“
#11:26 Kdykoli tedy jíte tento chléb a pijete tento kalich, zvěstujete smrt Páně, dokud on nepřijde.
#11:27 Kdo by tedy jedl tento chléb a pil kalich Páně nehodně, proviní se proti tělu a krvi Páně.
#11:28 Nechť každý sám sebe zkoumá, než tento chléb jí a z tohoto kalicha pije.
#11:29 Kdo jí a pije a nerozpoznává, že jde o tělo Páně, jí a pije sám sobě odsouzení.
#11:30 Proto je mezi vámi tolik slabých a nemocných a mnozí umírají.
#11:31 Kdybychom soudili sami sebe, nebyli bychom souzeni.
#11:32 Když nás však soudí Pán, je to k naší nápravě, abychom nebyli odsouzeni spolu se světem.
#11:33 A tak, bratří moji, když se shromažďujete k společnému stolu, čekejte jeden na druhého.
#11:34 Kdo má hlad, ať se nají doma, abyste se neshromažďovali k odsouzení. Ostatní věci zařídím, až přijdu. 
#12:1 Pokud jde o duchovní dary, bratří, nechtěl bych vás nechat v nevědomosti.
#12:2 Pamatujete se, že když jste byli pohané, táhlo vás to neodolatelně k němým modlám.
#12:3 Proto vám zdůrazňuji, že žádný, kdo mluví z Ducha Božího, neřekne: „Ježíš buď proklet“, a že nikdo nemůže říci: „Ježíš je Pán“, leč v Duchu svatém.
#12:4 Jsou rozdílná obdarování, ale tentýž Duch;
#12:5 rozdílné služby, ale tentýž Pán;
#12:6 a rozdílná působení moci, ale tentýž Bůh, který působí všecko ve všech.
#12:7 Každému je dán zvláštní projev Ducha ke společnému prospěchu.
#12:8 Jednomu je skrze Ducha dáno slovo moudrosti, druhému slovo poznání podle téhož Ducha,
#12:9 někomu zase víra v témž Duchu, někomu dar uzdravování v jednom a témž Duchu,
#12:10 někomu působení mocných činů, dalšímu zase proroctví, jinému rozlišování duchů, někomu dar mluvit ve vytržení, jinému dar vykládat, co to znamená.
#12:11 To všechno působí jeden a týž Duch, který uděluje každému zvláštní dar, jak sám chce.
#12:12 Tak jako tělo je jedno, ale má mnoho údů, a jako všecky údy těla jsou jedno tělo, ač je jich mnoho, tak je to i s Kristem.
#12:13 Neboť my všichni, ať Židé či Řekové, ať otroci či svobodní, byli jsme jedním Duchem pokřtěni v jedno tělo a všichni jsme byli napojeni týmž Duchem.
#12:14 Tělo není jeden úd, nýbrž mnoho údů.
#12:15 Kdyby řekla noha: „Protože nejsem ruka, nepatřím k tělu“, tím by ještě nepřestala být částí těla.
#12:16 A kdyby řeklo ucho „Protože nejsem oko, nepatřím k tělu,“ tím by ještě nepřestalo být částí těla.
#12:17 Kdyby celé tělo nebylo než oko, kde by byl sluch? A kdyby celé tělo nebylo než sluch, kde by byl čich?
#12:18 Ale Bůh dal tělu údy a každému z nich určil úkol, jak sám chtěl.
#12:19 Kdyby všechno bylo jen jedním údem, kam by se podělo tělo?
#12:20 Ve skutečnosti však je mnoho údů, ale jedno tělo.
#12:21 Oko nemůže říci ruce: „Nepotřebuji tě!“ Ani hlava nemůže říci nohám: „Nepotřebuji vás!“
#12:22 A právě ty údy těla, které se zdají méně významné, jsou nezbytné,
#12:23 a které pokládáme za méně čestné, těm prokazujeme zvláštní čest, a neslušné slušněji zahalujeme,
#12:24 jak to naše slušné údy nepotřebují. Bůh zařídil tělo tak, že přehlíženým údům dal hojnější čest,
#12:25 aby v těle nedošlo k roztržce, ale aby údy shodně pečovaly jeden o druhý.
#12:26 Trpí-li jeden úd, trpí spolu s ním všechny. A dochází-li slávy jeden úd, všechny se radují spolu s ním.
#12:27 Vy jste tělo Kristovo, a každý z vás je jedním z jeho údů.
#12:28 A v církvi ustanovil Bůh jedny za apoštoly, druhé za proroky, třetí za učitele; potom jsou mocné činy, pak dary uzdravování, služba potřebným, řízení církve, řeč ve vytržení.
#12:29 Jsou snad všichni apoštoly? Jsou všichni proroky? Jsou všichni učiteli? Mají všichni moc činit divy?
#12:30 Mají všichni dar uzdravovat? Mají všichni schopnost mluvit ve vytržení rozličnými jazyky? Dovedou je všichni vykládat?
#12:31 Usilujte o vyšší dary! A ukážu vám ještě mnohem vzácnější cestu: 
#13:1 Kdybych mluvil jazyky lidskými i andělskými, ale lásku bych neměl, jsem jenom dunící kov a zvučící zvon.
#13:2 Kdybych měl dar proroctví, rozuměl všem tajemstvím a obsáhl všecko poznání, ano kdybych měl tak velikou víru, že bych hory přenášel, ale lásku bych neměl, nic nejsem.
#13:3 A kdybych rozdal všecko, co mám, ano kdybych vydal sám sebe k upálení, ale lásku bych neměl, nic mi to neprospěje.
#13:4 Láska je trpělivá, laskavá, nezávidí, láska se nevychloubá a není domýšlivá.
#13:5 Láska nejedná nečestně, nehledá svůj prospěch, nedá se vydráždit, nepočítá křivdy.
#13:6 Nemá radost ze špatnosti, ale vždycky se raduje z pravdy.
#13:7 Ať se děje cokoliv, láska vydrží, láska věří, láska má naději, láska vytrvá.
#13:8 Láska nikdy nezanikne. Proroctví - to pomine; jazyky - ty ustanou; poznání - to bude překonáno.
#13:9 Vždyť naše poznání je jen částečné, i naše prorokování je jen částečné;
#13:10 až přijde plnost, tehdy to, co je částečné, bude překonáno.
#13:11 Dokud jsem byl dítě, mluvil jsem jako dítě, smýšlel jsem jako dítě, usuzoval jsem jako dítě; když jsem se stal mužem, překonal jsem to, co je dětinské.
#13:12 Nyní vidíme jako v zrcadle, jen v hádance, potom však uzříme tváří v tvář. Nyní poznávám částečně, ale potom poznám plně, jako Bůh zná mne.
#13:13 A tak zůstává víra, naděje, láska - ale největší z té trojice je láska. 
#14:1 Držte se lásky a usilujte o duchovní dary, nejvíce o dar prorocké řeči.
#14:2 Vždyť kdo ve vytržení mluví jazyky, nemluví k lidem, nýbrž k Bohu, a nikdo mu nerozumí. Je puzen Duchem, ale to, co říká, zůstává tajemstvím.
#14:3 Ten však, kdo má prorocký dar, mluví k lidem pro jejich duchovní užitek, napomenutí i povzbuzení.
#14:4 Kdo ve vytržení mluví jazyky, mluví k svému užitku, ale kdo mluví prorocky, mluví k užitku církve.
#14:5 Chtěl bych, abyste všichni mluvili jazyky, ale ještě více, abyste měli prorocký dar. Neboť ten, kdo mluví prorocky, znamená víc než ten, kdo mluví ve vytržení - ledaže by jeho řeč byla vykládána, aby z toho církev měla užitek.
#14:6 Vždyť kdybych k vám, bratří, přišel a mluvil jazyky, a nepřinesl vám žádné zjevení od Boha ani poznání ani prorocké slovo ani naučení - jaký prospěch byste z toho měli?
#14:7 Je to jako s hudebními nástroji, třeba s flétnou nebo kytarou: kdyby nevydávaly odlišné tóny, jak by bylo možno rozeznat nápěv, který se na nich hraje?
#14:8 A kdyby polnice vydala neurčitý zvuk, kdo by se připravoval k bitvě?
#14:9 Tak i vy: Jestliže ve vytržení nepromluvíte jasné slovo, jak se má poznat, co bylo řečeno? Budete mluvit jen do vzduchu!
#14:10 Na světě je mnoho různých řečí a každá má svá slova.
#14:11 Jestliže však neznám význam těch slov, budu pro mluvícího cizincem a on zase bude cizincem pro mne.
#14:12 Tak i vy: když tak horlivě usilujete o duchovní dary, snažte se, abyste měli hojnost těch, které slouží růstu celé církve.
#14:13 A proto ten, kdo mluví jazyky, nechť prosí, aby je dovedl také vykládat.
#14:14 Kdybych se modlil ve vytržení, modlil by se můj duch, ale má mysl by toho nebyla účastna.
#14:15 Co tedy? Budu se modlit ve vytržení ducha, ale budu se také modlit s vědomou myslí. Budu zpívat chvalozpěvy ve vytržení ducha, ale budu zpívat také s vědomou myslí.
#14:16 Kdybys děkoval Bohu ve vytržení, jak by ten, kdo do toho není zasvěcen, mohl říci Amen k tvému díkůvzdání, když nerozumí tomu, co říkáš?
#14:17 Ty sice dobře vzdáváš díky, ale druhý z toho nemá žádný užitek.
#14:18 Děkuji Bohu, že mám dar mluvit jazyky více než vy všichni,
#14:19 ve shromáždění však - abych poučil i druhé - raději řeknu pět slov srozumitelně než tisíce slov ve vytržení.
#14:20 Bratří, ve svém myšlení nebuďte jako děti. Ve zlém buďte jako nemluvňata, ale v myšlení buďte dospělí.
#14:21 V Zákoně je psáno: ‚Jinými jazyky a ústy cizozemců budu mluvit k tomuto lidu, ale ani tak mě nebudou poslouchat‘, praví Hospodin.
#14:22 Mluvení jazyky není tedy znamením k víře, nýbrž k nevěře, prorocká řeč však nevede k nevěře, nýbrž k víře.
#14:23 Kdyby se celá církev sešla ve shromáždění a všichni by mluvili ve vytržení, a přišli by tam lidé nezasvěcení a nevěřící, cožpak neřeknou, že blázníte?
#14:24 Budou-li všichni mluvit prorocky a přijde tam člověk nevěřící nebo nezasvěcený, bude vším, co slyší, souzen a usvědčován,
#14:25 vyjdou najevo věci skryté v jeho srdci, takže padne na kolena, pokoří se před Bohem a vyzná: „Vskutku je mezi vámi Bůh!“
#14:26 Co z toho plyne, bratří? Když se shromažďujete, jeden má žalm, druhý slovo naučení, jiný zjevení od Boha, ještě jiný promluví ve vytržení a další to vyloží. Všecko ať slouží společnému růstu.
#14:27 Pokud jde o mluvení jazyky, ať promluví dva nebo tři, jeden po druhém, a někdo ať vykládá.
#14:28 Kdyby neměli vykladače, ať ve shromáždění mlčí, každý ať mluví ve vytržení jen pro sebe a před Bohem.
#14:29 Z proroků ať promluví dva neb tři a ostatní ať to posuzují.
#14:30 Dostane-li se zjevení jinému ve shromáždění, nechť ten první umlkne.
#14:31 Jeden po druhém můžete všichni prorocky promluvit, aby všichni byli poučeni a všichni také povzbuzeni.
#14:32 Prorok přece ovládá svůj prorocký dar.
#14:33 Bůh není Bohem zmatku, nýbrž Bohem pokoje. Jako ve všech obcích Božího lidu,
#14:34 ženy nechť ve shromáždění mlčí. Nedovoluje se jim, aby mluvily; mají se podřizovat, jak to říká i Zákon.
#14:35 Chtějí-li se o něčem poučit, ať se doma zeptají svých mužů; ženě se nesluší mluvit ve shromáždění.
#14:36 Vyšlo snad slovo Boží od vás? Nebo přišlo jen k vám samotným?
#14:37 Pokládá-li se někdo za proroka nebo za člověka obdařeného Duchem, měl by poznat, že to, co vám píšu, je přikázání Páně.
#14:38 Kdo to neuznává, nedojde sám uznání.
#14:39 A tak, bratří moji, horlivě se snažte prorokovat a nebraňte mluvit jazyky.
#14:40 Všechno ať se děje slušně a spořádaně. 
#15:1 Chci vám připomenout, bratří, evangelium, které jsem vám zvěstoval, které jste přijali, které je základem, na němž stojíte,
#15:2 a skrze něž docházíte spásy, držíte-li se ho tak, jak jsem vám je zvěstoval - vždyť jste přece neuvěřili nadarmo.
#15:3 Odevzdal jsem vám především, co jsem sám přijal, že Kristus zemřel za naše hříchy podle Písem
#15:4 a byl pohřben; byl vzkříšen třetího dne podle Písem,
#15:5 ukázal se Petrovi, potom Dvanácti.
#15:6 Poté se ukázal více než pěti stům bratří najednou; většina z nich je posud na živu, někteří však již zesnuli.
#15:7 Pak se ukázal Jakubovi, potom všem apoštolům.
#15:8 Naposledy ze všech se jako nedochůdčeti ukázal i mně.
#15:9 Vždyť já jsem nejmenší z apoštolů a nejsem ani hoden jména apoštol, protože jsem pronásledoval církev Boží.
#15:10 Milostí Boží jsem to, co jsem, a milost, kterou mi prokázal, nebyla nadarmo; více než oni všichni jsem se napracoval - nikoli já, nýbrž milost Boží, která byla se mnou.
#15:11 Ať už tedy já, nebo oni - tak zvěstujeme a tak jste uvěřili.
#15:12 Když se tedy zvěstuje o Kristu, že byl vzkříšen, jak mohou někteří mezi vámi říkat, že není zmrtvýchvstání?
#15:13 Není-li zmrtvýchvstání, pak nebyl vzkříšen ani Kristus.
#15:14 A jestliže Kristus nebyl vzkříšen, pak je naše zvěst klamná, a klamná je i vaše víra,
#15:15 a my jsme odhaleni jako lživí svědkové o Bohu: dosvědčili jsme, že Bůh vzkřísil Krista, ale on jej nevzkřísil, není-li vzkříšení z mrtvých.
#15:16 Neboť není-li vzkříšení z mrtvých, nebyl vzkříšen ani Kristus.
#15:17 Nebyl-li však Kristus vzkříšen, je vaše víra marná, ještě jste ve svých hříších,
#15:18 a jsou ztraceni i ti, kteří zesnuli v Kristu.
#15:19 Máme-li naději v Kristu jen pro tento život, jsme nejubožejší ze všech lidí!
#15:20 Avšak Kristus byl vzkříšen jako první z těch, kdo zesnuli.
#15:21 A jako vešla do světa smrt skrze člověka, tak i zmrtvýchvstání:
#15:22 jako v Adamovi všichni umírají, tak v Kristu všichni dojdou života.
#15:23 Každý v daném pořadí: první vstal Kristus, potom při Kristově příchodu vstanou ti, kdo jsou jeho.
#15:24 Tu nastane konec, až Kristus zruší vládu všech mocností a sil a odevzdá království Bohu a Otci.
#15:25 Musí totiž kralovat, ‚dokud Bůh nepodmaní všechny nepřátele pod jeho nohy‘.
#15:26 Jako poslední nepřítel bude přemožena smrt,
#15:27 vždyť ‚pod nohy jeho podřídil všecko‘. Je-li řečeno, že je mu podřízeno všecko, je jasné, že s výjimkou toho, kdo mu všecko podřídil.
#15:28 Až mu bude podřízeno všecko, pak i sám Syn se podřídí tomu, kdo mu všecko podřídil, a tak bude Bůh všecko ve všem.
#15:29 Jaký by jinak mělo smysl to, že se někteří dávají křtít za mrtvé? Jestliže mrtví vůbec nevstanou, proč se za ně dávají křtít?
#15:30 A proč i my se v každou hodinu vydáváme v nebezpečí?
#15:31 Den ze dne hledím smrti do tváře - ujišťuji vás o tom, bratří, při všem, co pro mne znamenáte, v Kristu Ježíši, našem Pánu!
#15:32 Jestliže jsem v Efezu podstoupil zápas se šelmami jen z lidských pohnutek, co mi to prospěje? Jestliže mrtví nevstanou, pak ‚jezme a pijme, neboť zítra zemřeme‘.
#15:33 Neklamte se: ‚špatná společnost kazí dobré mravy‘.
#15:34 Vystřízlivějte, jak se sluší, a nehřešte. Někteří nemají ani ponětí o Bohu. Říkám to k vašemu zahanbení!
#15:35 Ale někdo snad řekne: „Jak vstanou mrtví? V jakém těle přijdou?“
#15:36 Jaká pošetilost! To, co zaséváš, nebude oživeno, jestliže neumře.
#15:37 A co zaséváš, není tělo, které vzejde, nýbrž holé zrno, ať už pšenice nebo nějaké jiné rostliny.
#15:38 Bůh však mu dává tělo, jak sám určil, každému semeni jeho zvláštní tělo.
#15:39 Není jedno tělo jako druhé, nýbrž jiné tělo mají lidé, jiné zvířata, jiné ptáci, jiné ryby.
#15:40 A jsou tělesa nebeská a tělesa pozemská, ale jiná je sláva nebeských a jiná pozemských.
#15:41 Jiná je záře slunce a jiná měsíce, a ještě jiná je záře hvězd, neboť hvězda od hvězdy se liší září.
#15:42 Tak je to i se zmrtvýchvstáním. Co je zaseto jako pomíjitelné, vstává jako nepomíjitelné.
#15:43 Co je zaseto v poníženosti, vstává v slávě. Co je zaseto v slabosti, vstává v moci.
#15:44 Zasévá se tělo přirozené, vstává tělo duchovní. Je-li tělo přirozené, je i tělo duchovní.
#15:45 Jak je psáno: ‚První člověk Adam se stal duší živou‘ - poslední Adam je však Duchem oživujícím.
#15:46 Nejprve tedy není tělo duchovní, nýbrž přirozené, pak teprve duchovní.
#15:47 První člověk byl z prachu země, druhý člověk z nebe.
#15:48 Jaký byl ten pozemský, takoví jsou i ostatní na zemi, a jaký je ten nebeský, takoví i ostatní v nebesích.
#15:49 A jako jsme nesli po dobu pozemského, tak poneseme i podobu nebeského.
#15:50 Chci říci to, bratří, že člověk, jak je, nemůže mít podíl na království Božím a pomíjitelné nemůže mít podíl na nepomíjitelném.
#15:51 Hle, odhalím vám tajemství: Ne všichni zemřeme, ale všichni budeme proměněni,
#15:52 naráz, v okamžiku, až se naposled ozve polnice. Až zazní, mrtví budou vzkříšeni k nepomíjitelnosti a my živí proměněni.
#15:53 Pomíjitelné tělo musí totiž obléci nepomíjitelnost a smrtelné nesmrtelnost.
#15:54 A když pomíjitelné obleče nepomíjitelnost a smrtelné nesmrtelnost, pak se naplní, co je psáno: ‚Smrt je pohlcena, Bůh zvítězil!
#15:55 Kde je, smrti, tvé vítězství? Kde je, smrti, tvá zbraň?‘
#15:56 Zbraní smrti je hřích a hřích má svou moc ze zákona.
#15:57 Chvála buď Bohu, který nám dává vítězství skrze našeho Pána Ježíše Krista!
#15:58 A tak, moji milovaní bratří, buďte pevní, nedejte se zviklat, buďte stále horlivější v díle Páně; vždyť víte, že vaše práce není v Pánu marná. 
#16:1 Pokud jde o sbírku pro církev v Jeruzalémě, dělejte to podle pokynů, které jsem dal církvím v Galacii.
#16:2 V první den týdne nechť každý z vás dá stranou, co může postrádat, aby sbírka nezačala teprve tehdy, až k vám přijdu.
#16:3 Až budu u vás, vyšlu ty, které doporučíte, s průvodními listy, aby donesli dar vaší vděčnosti do Jeruzaléma.
#16:4 Ukáže-li se vhodné, abych tam šel také já, půjdou se mnou.
#16:5 Přijdu k vám, až dokončím cestu Makedonií. Makedonií totiž jen projdu,
#16:6 ale u vás bych chtěl zůstat déle, snad i přes celou zimu, abyste vy mne potom vypravili na další cestu.
#16:7 Nerad bych se u vás jen zastavil; doufám, že s vámi budu moci zůstat nějaký čas, dovolí-li to Pán.
#16:8 V Efezu zůstanu až do letnic;
#16:9 otevřela se mi zde veliká a nadějná příležitost, ale také protivníků je mnoho.
#16:10 Přijde-li Timoteus, hleďte, aby byl mezi vámi bez obav; vždyť koná dílo Páně stejně jako já.
#16:11 Ať ho nikdo z vás nepodceňuje! Vypravte ho na cestu ke mně s bratrskou láskou, neboť na něho čekám spolu s jinými bratry.
#16:12 Pokud jde o bratra Apolla, velice jsem na něj naléhal, aby k vám šel spolu s jinými bratry, ale v žádném případě nechtěl jít už nyní; půjde však, jakmile se ukáže vhodná příležitost.
#16:13 Buďte bdělí, stůjte pevně ve víře, buďte stateční a silní!
#16:14 Všecko nechť se mezi vámi děje v lásce.
#16:15 O něco vás prosím, bratří. Víte o rodině Štěpánově, že první z celé Achaje uvěřili a dali se do služby bratřím.
#16:16 I vy se ochotně podřizujte takovým lidem a každému, kdo pracuje na společném díle.
#16:17 Mám radost, že přišli Štěpán, Fortunát a Achaikos, neboť mi nahradili vaši nepřítomnost.
#16:18 Uklidnili mé i vaše srdce. Važte si takových lidí!
#16:19 Pozdravují vás církve v Asii. Velice vás v Pánu pozdravuje Akvilas a Priska spolu s církví, která se shromažďuje v jejich domě.
#16:20 Pozdravují vás všichni bratří. Pozdravte se navzájem svatým políbením.
#16:21 Pozdrav mou, Pavlovou rukou.
#16:22 Kdo nemiluje Pána, ať je proklet! Maranatha!
#16:23 Milost Pána Ježíše buď s vámi.
#16:24 Moje láska je s vámi všemi v Kristu Ježíši.  

\book{II Corinthians}{2Cor}
#1:1 Pavel, z vůle Boží apoštol Krista Ježíše, a bratr Timoteus církvi Boží v Korintu i všem bratřím v celé Achaji:
#1:2 Milost vám a pokoj od Boha Otce našeho a Pána Ježíše Krista.
#1:3 Pochválen buď Bůh a Otec našeho Pána Ježíše Krista, Otec milosrdenství a Bůh veškeré útěchy!
#1:4 On nás potěšuje v každém soužení, abychom i my mohli těšit ty, kteří jsou v jakékoli tísni, tou útěchou, jaké se nám samým dostává od Boha.
#1:5 Jako na nás v hojnosti přicházejí utrpení Kristova, tak na nás skrze Krista přichází v hojnosti i útěcha.
#1:6 Máme-li soužení, je to k vašemu povzbuzení a spáse; docházíme-li útěchy, je to zase k vašemu povzbuzení; to vám dá sílu, abyste vydrželi stejné utrpení, v jakém jsme my.
#1:7 Máme pevnou naději a jsme si jisti, že jako jste účastni utrpení, tak budete účastni také útěchy.
#1:8 Chtěli bychom, bratří, abyste věděli o soužení, které nás potkalo v Asii. Dolehlo na nás nadmíru těžce, nad naši sílu, až jsme se dokonce vzdali naděje, že to přežijeme;
#1:9 už jsme se sami smířili s rozsudkem smrti - to proto, abychom nespoléhali na sebe, ale na Boha, který křísí mrtvé.
#1:10 On nás vysvobodil ze samého náručí smrti, a ještě vysvobodí; v něho jsme složili svou naději, že nás vždy znovu vysvobodí,
#1:11 když i vy nám budete nápomocni svými modlitbami. A tak, když nám mnozí vyprosili milost, budou za nás mnozí také děkovat.
#1:12 Toto je naše chlouba: Naše svědomí nám dosvědčuje, že jsme v tomto světě a zvláště vůči vám žili ve svatosti a ryzosti před Bohem a nespoléhali na světskou moudrost, nýbrž na milost Boží.
#1:13 Náš dopis nemá jiný smysl, než jak mu sami rozumíte, když ho čtete. Jednou, doufám, dokonale pochopíte,
#1:14 co jste snad aspoň zčásti již pochopili, že v den našeho Pána Ježíše budeme vaší chloubou, tak jako vy zase naší.
#1:15 V této důvěře jsem se už k vám dříve chystal, abyste zakusili novou milost.
#1:16 Zamýšlel jsem zastavit se u vás na cestě do Makedonie a opět se k vám odtamtud vrátit, abyste vy mě potom vypravili do Judska.
#1:17 Pojal jsem snad tento úmysl lehkovážně? Nebo se snad při svých úvahách dávám vést vlastními zájmy, takže by moje ‚ano, ano‘ znamenalo také ‚ne, ne‘?
#1:18 Bůh je svědek, že mé slovo k vám není zároveň ‚ano‘ i ‚ne‘!
#1:19 Vždyť Boží Syn Ježíš Kristus, kterého jsme u vás zvěstovali my - já a Silvanus a Timoteus - nebyl zároveň ‚ano‘ i ‚ne‘, nýbrž v něm jest jasné ‚Ano‘!
#1:20 Ke všem zaslíbením Božím, kolik jich jen jest, bylo v něm řečeno ‚Ano‘. A proto skrze něho zní i naše ‚Amen‘ k slávě Boží.
#1:21 Ten, kdo nás spolu s vámi staví na pevný základ v Kristu a kdo si nás posvětil, je Bůh.
#1:22 On nám také vtiskl svou pečeť a do srdce nám dal svého Ducha jako závdavek toho, co nám připravil.
#1:23 Dovolávám se za svědka samého Boha, že jen z ohledu na vás jsem dosud nepřišel do Korintu.
#1:24 Ne že bychom chtěli panovat nad vaší vírou, nýbrž chceme pomáhat vaší radosti - ve víře přece stojíte! 
#2:1 Rozhodl jsem se však, že vám nesmím znovu přinést zármutek.
#2:2 Kdybych já vás zarmoutil, kdo mě potěší, ne-li vy, které jsem zarmoutil?
#2:3 To vám píšu, abych, až přijdu, nebyl zarmoucen těmi, kteří by mě měli potěšit; spoléhám totiž na vás všecky, že moje radost bude radostí vás všech.
#2:4 Psal jsem ve veliké stísněnosti a se sevřeným srdcem, s mnohými slzami - ne proto, abyste byli zarmouceni, ale abyste poznali, jak veliká je má láska k vám.
#2:5 Jestliže někdo způsobil zármutek, nezpůsobil ho jen mně, nýbrž aspoň zčásti - nechci přehánět - vám všem.
#2:6 Stačí už to pokárání, kterého se mu dostalo od většiny z vás.
#2:7 Proto je nyní spíše třeba, abyste mu odpustili a potěšili ho, aby se takový člověk pod přívalem zármutku nezhroutil.
#2:8 Naléhavě vás prosím, abyste mu dokázali svou lásku.
#2:9 Proto jsem vám také psal, abych poznal, jak se osvědčíte, zda jste ve všem poslušní.
#2:10 Komukoli něco odpustíte, odpustím i já. A když já něco odpouštím - mám-li co odpouštět - činím to před tváří Kristovou kvůli vám,
#2:11 aby nás satan neobelstil; jeho úskočnost přece známe.
#2:12 Když jsem přišel do Troady zvěstovat evangelium Kristovo, našel jsem tam dveře otevřené pro dílo Páně,
#2:13 a přece jsem neměl stání, neboť jsem tam nezastihl bratra Tita; rozloučil jsem se s nimi a vydal se do Makedonie.
#2:14 Budiž vzdán dík Bohu, který nás stále vodí v triumfálním průvodu Kristově a všude skrze nás šíří vůni svého poznání.
#2:15 Jsme totiž jakoby vůní kadidla, jež Kristus obětuje Bohu; ta vůně proniká k těm, kteří docházejí spásy, i k těm, kteří spějí k zahynutí.
#2:16 Jedněm jsme smrtonosnou vůní vedoucí k záhubě, druhým vůní životodárnou vedoucí k životu. Ale kdo je k takovému poslání způsobilý?
#2:17 My nejsme jako mnozí, kteří kramaří s Božím slovem, nýbrž mluvíme upřímně, z Božího pověření a před tváří Boží v Kristu. 
#3:1 To zase začínáme sami sebe doporučovat? Či potřebujeme snad jako někdo doporučující listy k vám nebo od vás?
#3:2 Naším doporučujícím listem jste vy sami; je napsán na našem srdci, všichni jej znají a mohou číst.
#3:3 Je přece zjevné, že vy jste listem Kristovým, vzniklým z naší služby a napsaným ne inkoustem, nýbrž Duchem Boha živého, ne na kamenných deskách, nýbrž na živých deskách lidských srdcí.
#3:4 Odvažujeme se to říci, protože důvěřujeme v Boha skrze Krista.
#3:5 Ne že bychom mohli tuto způsobilost přičítat sami sobě na základě toho, co je v nás; naše způsobilost je od Boha,
#3:6 který nás učinil způsobilými sloužit nové smlouvě, jež není založena na liteře, nýbrž na Duchu. Litera zabíjí, ale Duch dává život.
#3:7 Jestliže smlouva literami vytesaná do kamene sloužila smrti, a přece byla nastolena s oslňující slávou, takže synové Izraele nemohli pohlédnout na tvář Mojžíšovu pro její pomíjivou zář -
#3:8 oč slavnější bude služba Ducha!
#3:9 Byla-li služba vedoucí k odsouzení slavná, oč ji převyšuje služba spravedlnosti!
#3:10 Ano, ona sláva vůbec nebyla slávou ve srovnání s touto slávou vše přesahující!
#3:11 Jestliže přišlo slavně to, co pomíjí, oč slavnější je to, co zůstává!
#3:12 Když tedy máme takovou naději, smíme vystupovat s plnou otevřeností a jistotou.
#3:13 Nepočínáme si jako Mojžíš, který zahaloval svou tvář závojem, aby synové Izraele nespatřili konec té pomíjející záře.
#3:14 Avšak jejich myšlení na tom ustrnulo. Až do dnešního dne zůstává onen závoj při čtení staré smlouvy a zůstává skryto, že je zrušen v Kristu.
#3:15 A tak až podnes, když se čte Mojžíš, leží na jejich srdci závoj.
#3:16 Avšak ‚když se obrátí k Pánu, je závoj odstraněn‘.
#3:17 Duch je tím Pánem, kde je Duch Páně, tam je svoboda.
#3:18 Na odhalené tváři nás všech se zrcadlí slavná zář Páně, a tak jsme proměňováni k jeho obrazu ve stále větší slávě - to vše mocí Ducha Páně. 
#4:1 A proto, když nám byla z Božího slitování svěřena tato služba, nepoddáváme se skleslosti.
#4:2 Nepotřebujeme skrývat nic nečestného, nepočínáme si lstivě ani nefalšujeme slovo Boží, nýbrž činíme pravdu zjevnou a tak se před tváří Boží doporučujeme svědomí všech lidí.
#4:3 Je-li přesto naše evangelium zahaleno, je zahaleno těm, kteří spějí k záhubě.
#4:4 Bůh tohoto světa oslepil jejich nevěřící mysl, aby jim nevzešlo světlo evangelia slávy Kristovy, slávy toho, který je obrazem Božím.
#4:5 Vždyť nezvěstujeme sami sebe, nýbrž Krista Ježíše jako Pána, a sebe jen jako vaše služebníky pro Ježíše.
#4:6 Neboť Bůh, který řekl ‚ze tmy ať zazáří světlo‘, osvítil naše srdce, aby nám dal poznat světlo své slávy ve tváři Kristově.
#4:7 Tento poklad máme však v hliněných nádobách, aby bylo patrno, že tato nesmírná moc je Boží a není z nás.
#4:8 Na všech stranách jsme tísněni, ale nejsme zahnáni do úzkých; jsme bezradni, ale nejsme v koncích;
#4:9 jsme pronásledováni, ale nejsme opuštěni; jsme sráženi k zemi, ale nejsme poraženi.
#4:10 Stále nosíme na sobě znamení Ježíšovy smrti, aby i život Ježíšův byl na nás zjeven.
#4:11 Vždyť my, pokud žijeme, jsme pro Ježíše stále vydáváni na smrt, aby byl na našem smrtelném těle zjeven i Ježíšův život.
#4:12 A tak na nás koná své dílo smrt, na vás však život.
#4:13 Ale máme ducha víry, o níž je psáno: ‚Uvěřil jsem, a proto jsem také promluvil‘ - i my věříme, a proto také mluvíme,
#4:14 vždyť víme, že ten, kdo vzkřísil Pána Ježíše, také nás s Ježíšem vzkřísí a postaví před svou tvář spolu s vámi.
#4:15 To všechno je kvůli vám, aby se milost ve mnohých hojně rozmáhala a tak přibývalo i díků k slávě Boží.
#4:16 A proto neklesáme na mysli: i když navenek hyneme, vnitřně se den ze dne obnovujeme.
#4:17 Toto krátké a lehké soužení působí přenesmírnou váhu věčné slávy
#4:18 nám, kteří nehledíme k viditelnému, nýbrž k neviditelnému. Viditelné je dočasné, neviditelné však věčné. 
#5:1 Víme přece, že bude-li stan našeho pozemského života stržen, čeká nás příbytek od Boha, věčný dům v nebesích, který nebyl zbudován rukama.
#5:2 Proto zde sténáme touhou, abychom byli oděni šatem nebeským.
#5:3 Vždyť jen když jej oblékneme, nebudeme shledáni nazí.
#5:4 Pokud jsme totiž v tomto stanu, sténáme pod těžkým břemenem, neboť nechceme, aby z nás bylo svlečeno naše pozemské tělo, nýbrž aby přes ně bylo oblečeno nebeské, aby to, co je smrtelné, bylo pohlceno životem.
#5:5 Ten, kdo nás k tomu připravil a dal nám již Ducha jako závdavek, je Bůh.
#5:6 Jsme tedy stále plni důvěry, neboť víme, že pokud jsme doma v tomto těle, nejsme doma u Pána -
#5:7 žijeme přece z víry, ne z toho, co vidíme.
#5:8 V této důvěře chceme raději odejít z těla a být už doma u Pána.
#5:9 Proto nám také nadevše záleží na tom, abychom se mu líbili, ať už odcházíme domů nebo zůstáváme v cizině.
#5:10 Vždyť se všichni musíme ukázat před soudným stolcem Kristovým, aby každý dostal odplatu za to, co činil ve svém životě, ať dobré či zlé.
#5:11 Protože známe tuto bázeň před Pánem, přesvědčujeme lidi. Bůh vidí do našeho srdce a doufám, že i vy ve svém svědomí do našeho srdce vidíte.
#5:12 Nechceme se vám znovu doporučovat, nýbrž dáváme vám příležitost, abyste se námi chlubili a měli se čím bránit lidem, kteří se chlubí tím, co je před očima, a ne tím, co je v srdci.
#5:13 Když jsme byli ve vytržení, bylo to mezi Bohem a námi, když jsme při smyslech, je to pro vás.
#5:14 Vždyť nás má ve své moci láska Kristova - nás, kteří jsme pochopili, že jeden zemřel za všecky, a že tedy všichni zemřeli;
#5:15 a za všechny zemřel proto, aby ti, kteří jsou naživu, nežili už sami sobě, nýbrž tomu, kdo za ně zemřel i vstal.
#5:16 A tak od nynějška už nikoho neposuzujeme podle lidských měřítek. Ačkoli jsme dříve viděli Krista po lidsku, nyní ho už takto neznáme.
#5:17 Kdo je v Kristu, je nové stvoření. Co je staré, pominulo, hle, je tu nové!
#5:18 To všecko je z Boha, který nás smířil sám se sebou skrze Krista a pověřil nás, abychom sloužili tomuto smíření.
#5:19 Neboť v Kristu Bůh usmířil svět se sebou. Nepočítá lidem jejich provinění a nám uložil zvěstovat toto smíření.
#5:20 Jsme tedy posly Kristovými, Bůh vám domlouvá našimi ústy; na místě Kristově vás prosíme: dejte se smířit s Bohem!
#5:21 Toho, který nepoznal hřích, kvůli nám ztotožnil s hříchem, abychom v něm dosáhli Boží spravedlnosti. 
#6:1 Jako spolupracovníci na tomto díle vás napomínáme, abyste milost Boží nepřijímali naprázdno,
#6:2 vždyť je psáno: ‚V čas příhodný jsem tě vyslyšel, v den spásy jsem ti přispěl na pomoc.‘ Hle, nyní je čas příhodný, nyní je den spásy!
#6:3 Nikomu nedáváme v ničem příležitost k pohoršení, aby tato služba nebyla uvedena v potupu,
#6:4 ale ve všem se prokazujeme jako Boží služebníci, v mnohé vytrvalosti, v souženích, tísni, úzkostech,
#6:5 pod ranami, v žalářích, nepokojích, vyčerpanosti, v bezesných nocích, v hladovění;
#6:6 prokazujeme se bezúhonností, poznáním, trpělivostí, dobrotivostí, Duchem svatým, nepředstíranou láskou,
#6:7 slovem pravdy, mocí Boží. Jsme vyzbrojeni spravedlností k útoku i k obraně,
#6:8 procházíme slávou i pohanou, zlou i dobrou pověstí; mají nás za svůdce, a přece mluvíme pravdu;
#6:9 jsme neznámí, a přece o nás všichni vědí; umíráme - a hle, jsme naživu; jsme týráni, a přece nejsme vydáni smrti;
#6:10 máme proč se rmoutit, a přece se stále radujeme; jsme chudí, a přece mnohé obohacujeme; nic nemáme, a přece nám patří vše.
#6:11 Nic jsem vám nezatajil, Korinťané, naše srdce se vám otevřelo.
#6:12 V našem srdci nemáte nedostatek místa, ale vy sami se nám uzavíráte.
#6:13 Na oplátku - mluvím k vám jako k svým dětem - udělejte nám i vy místo ve svém srdci!
#6:14 Nedejte se zapřáhnout do cizího jha spolu s nevěřícími! Co má společného spravedlnost s nepravostí? A jaké spolužití světla s temnotou?
#6:15 Jaký souzvuk Krista s Beliálem? Jaký podíl věřícího s nevěřícím?
#6:16 Jaké spojení chrámu Božího s modlami? My jsme přece chrám Boha živého. Jak řekl Bůh: ‚Budu přebývat a procházet se mezi nimi, budu jejich Bohem a oni budou mým lidem.‘
#6:17 A proto ‚vyjděte z jejich středu a oddělte se‘, praví Hospodina ‚ničeho nečistého se nedotýkejte, a já vás přijmu‘
#6:18 a ‚budu vám Otcem a vy budete mými syny a dcerami, praví Hospodin zástupů‘. 
#7:1 Když máme taková zaslíbení, moji nejmilejší, očisťme se od každé poskvrny těla i ducha a přiveďme k cíli své posvěcení v bázni Boží.
#7:2 Dejte nám místo ve svém srdci! Nikomu jsme neublížili, nikomu jsme neukřivdili, nikoho jsme neošidili.
#7:3 Neříkám to, abych vás odsuzoval, vždyť jsem vám už dříve řekl: Máte v mém srdci takové místo, že jsme spolu spjati ve smrti i v životě.
#7:4 Mám ve vás pevnou důvěru, jsem na vás opravdu hrdý; potěšili jste mě a naplnili nesmírnou radostí uprostřed všeho našeho soužení.
#7:5 Když jsme totiž přišli do Makedonie, nenašli jsme ve svých těžkostech žádnou úlevu, všude jen tíseň: navenek boje, uvnitř úzkosti.
#7:6 Ale Bůh, který těší sklíčené, potěšil i nás příchodem Titovým.
#7:7 A nejen jeho příchodem, ale i tím, jak jste vy ho potěšili. Vypravoval nám, jak velice po nás toužíte, jak jste plni lítosti a jak horlivě stojíte při mně; to mi způsobilo velikou radost.
#7:8 A jestliže jsem vás svým dopisem zarmoutil, už toho nelituji, i když jsem toho chvíli litoval. Vidím totiž, že vás ten dopis na čas zarmoutil,
#7:9 ale nyní se raduji, ne že jste se zarmoutili, ale že zármutek vás vedl k pokání. Byl to zármutek podle Boží vůle, a tak jsme vám nezpůsobili žádnou škodu.
#7:10 Zármutek podle Boží vůle působí pokání ke spáse, a toho není proč litovat, zármutek po způsobu světa však působí smrt.
#7:11 Pohleďte, k čemu vás vedl tento zármutek podle Boží vůle: jakou ve vás vzbudil opravdovost, jakou ochotu k omluvě, jaké znepokojení, jakou bázeň, jakou touhu, jakou horlivost, jakou snahu potrestat viníka! Tím vším jste prokázali, že jste se v té věci zachovali správně.
#7:12 A tak: Když jsem vám psal, nepsal jsem kvůli tomu, kdo se dopustil křivdy, ani kvůli tomu, komu bylo ukřivděno, nýbrž proto, aby se u vás před Bohem projevila vaše oddanost k nám.
#7:13 To je pro nás povzbuzením. Nadto však jsme se ještě mnohem více potěšili radostí Titovou, že jste vy všichni upokojili jeho mysl.
#7:14 Když jsem vás tedy před ním chválil, neudělali jste mi hanbu: stejně jako jsme vám ve všem mluvili pravdu, tak se také ukázalo, že naše chlouba před Titem byla oprávněná.
#7:15 Jste jeho srdci ještě bližší, když vzpomíná na poslušnost vás všech, jak jste ho přijali s uctivou pokorou.
#7:16 Mám radost, že se na vás mohu ve všem spolehnout. 
#8:1 Chceme vám povědět, bratří, jakou milost dal Bůh církvím v Makedonii.
#8:2 Tak se osvědčily v mnohém soužení, že z jejich nesmírné radosti a veliké chudoby vzešla jejich bohatá štědrost.
#8:3 Mohu jim dosvědčit, že dobrovolně dávali podle své největší možnosti, ano i nad možnost;
#8:4 s velkou naléhavostí nás prosili, aby se směli účastnit této pomoci pro bratry v Judsku.
#8:5 Překonali všechno naše očekávání: dali sami sebe předně Pánu a z vůle Boží také nám.
#8:6 A tak jsme vybídli Tita, který započal toto dílo lásky, aby je mezi vámi také dokončil.
#8:7 Jako jste ve všem bohatí, ve víře, v slovu, v poznání, v horlivosti i v lásce, kterou máte k nám, buďte bohatí i v tomto díle milosti.
#8:8 Neříkám to jako rozkaz, zmiňuji se však o horlivosti jiných, abych vyzkoušel opravdovost vaší lásky.
#8:9 Znáte přece štědrost našeho Pána Ježíše Krista: byl bohatý, ale pro vás se stal chudým, abyste vy jeho chudobou zbohatli.
#8:10 Řeknu vám, co si myslím: Sluší se, abyste nyní dokončili dílo, které jste už loni nejen začali, nýbrž také k němu dali podnět.
#8:11 Odkud vyšla ochota k pomoci, tam ať se pomoc také uskuteční. Každý ať dá podle toho, co má;
#8:12 vždyť je-li zde tato ochota, pak je dar před Bohem cenný podle toho, co kdo má, ne podle toho, co nemá.
#8:13 Nejde o to, aby se jiným ulehčilo a vy byli přetíženi, nýbrž abyste na tom byli stejně:
#8:14 váš přebytek pomůže nyní jejich nedostatku, aby zase jindy jejich přebytek přišel k dobru vám ve vašem nedostatku; tak nastane vyrovnání,
#8:15 jak je psáno: ‚Kdo měl mnoho, tomu nic nepřebylo, a kdo málo, neměl nedostatek.‘
#8:16 Buď Bohu dík, že dal Titovi do srdce stejně horlivý zájem o vás!
#8:17 Nejenže vyhověl mé výzvě, nýbrž ve své horlivosti sám od sebe se rozhodl k vám jít.
#8:18 Spolu s ním posílám bratra, který si službou evangeliu získal uznání ve všech církvích.
#8:19 A nejen to: byl také církvemi zvolen, aby nás doprovázel s výtěžkem této sbírky, jež se naší službou uskutečňuje k slávě Páně a jako projev naší ochoty.
#8:20 Chceme totiž zabránit tomu, aby nás při službě tomuto štědrému daru někdo nepodezíral.
#8:21 Myslíme na to, abychom byli bez výtky nejen před Bohem, nýbrž i před lidmi.
#8:22 S těmi dvěma posíláme dalšího bratra, o jehož horlivosti v každém směru jsme se mnohokrát přesvědčili; a nyní je obzvláště horlivý, protože vám velmi důvěřuje.
#8:23 Pokud jde o Tita, je to můj společník a spolupracovník v díle pro vás. Pokud jde o oba další bratry, jsou to vyslanci církví a sláva Kristova.
#8:24 Ukažte jim svou lásku a dokažte jim, že jsme na vás před ostatními církvemi právem hrdi. 
#9:1 O pomoci bratřím je vlastně zbytečné vám psát.
#9:2 Znám přece vaši ochotu, o níž s hrdostí makedonským bratřím říkám, že Achaia je už od loňska připravena. Vaše horlivost byla pobídkou pro mnohé z nich.
#9:3 Posílám však tyto bratry proto, aby se moje hrdost na vás v této věci neukázala prázdnou chloubou: chtěl bych, abyste byli připraveni tak, jak jsem jim o tom vyprávěl.
#9:4 Kdyby se mnou přišli makedonští a shledali, že nejste připraveni, bylo by to velkým zahanbením pro nás - natož pak pro vás! - protože jsme o vás mluvili s takovou jistotou.
#9:5 Proto jsem pokládal za nutné vyzvat tyto bratry, aby k vám šli napřed a připravili váš dávno slíbený dar, tak aby byl pohotově a byla to opravdu štědrost a ne lakota.
#9:6 Vždyť kdo skoupě rozsévá, bude také skoupě sklízet, a kdo štědře rozsévá, bude také štědře sklízet.
#9:7 Každý ať dává podle toho, jak se ve svém srdci předem rozhodl, ne s nechutí ani z donucení; vždyť ‚radostného dárce miluje Bůh‘.
#9:8 Bůh má moc zahrnout vás všemi dary své milosti, abyste vždycky měli dostatek všeho, co potřebujete, a ještě vám přebývalo pro každé dobré dílo,
#9:9 jak je psáno: ‚Rozdělil štědře, obdaroval nuzné, dobrota jeho trvá navěky.‘
#9:10 Ten, který ‚dává semeno k setbě i chléb k jídlu‘, dá vzrůst vaší setbě a rozmnoží ‚plody vaší spravedlnosti‘.
#9:11 Vším způsobem budete obohacováni, abyste mohli být velkoryse štědří; tak povzbudíme mnohé, aby vzdávali díky Bohu.
#9:12 Neboť služba této oběti nejen doplňuje, v čem mají bratří nedostatek, nýbrž také rozhojňuje díkůvzdání Bohu:
#9:13 Přesvědčeni touto vaší službou budou slavit Boha za to, jak jste se podřídili Kristovu evangeliu a jak štědře se projevuje vaše společenství s nimi i se všemi.
#9:14 Budou se za vás modlit a po vás toužit pro nesmírnou milost, kterou vám Bůh dal.
#9:15 Bohu budiž vzdán dík za jeho nevystižitelný dar! 
#10:1 Já Pavel vás napomínám tiše a mírně po způsobu Kristově - já, který tváří v tvář jsem prý mezi vámi pokorný, ale z dálky si na vás troufám.
#10:2 Prosím vás, nenuťte mne, až přijdu, jednat s tou troufalostí, na kterou mám podle svého přesvědčení právo, a to vůči těm, kteří nám připisují záměry jen lidské.
#10:3 Jsme ovšem jenom lidé, ale svůj zápas nevedeme po lidsku.
#10:4 Zbraně našeho boje nejsou světské, nýbrž mají od Boha sílu bořit hradby. Jimi boříme lidské výmysly
#10:5 a všecko, co se v pýše pozvedá proti poznání Boha. Uvádíme do poddanství každou mysl, aby byla poslušna Krista,
#10:6 a jsme připraveni potrestat každou neposlušnost, dokud vaše poslušnost nebude úplná.
#10:7 Hleďte na to, co máte před očima! Je-li si někdo jist, že je Kristův, nechť si uvědomí, že jako je on Kristův, tak jsme i my!
#10:8 I když se trochu víc pochlubím pravomocí, kterou mi Kristus dal - abych u vás stavěl, a ne bořil - nebudu zahanben.
#10:9 Nechtěl bych však, aby se zdálo, že vás chci svými listy zastrašovat.
#10:10 Říká se, že mé listy jsou závažné a mocné, ale osobní přítomnost slabá a řeč ubohá.
#10:11 Kdo tak mluví, ať si uvědomí: jak se projevujeme slovy svých dopisů, když jsme daleko od vás, tak se ukážeme svými činy, až budeme u vás.
#10:12 Neopovažujeme se zařadit mezi ty nebo srovnávat s těmi, kteří doporučují sami sebe: tím, že se měří jen podle sebe a srovnávají sami se sebou, ztrácejí soudnost.
#10:13 Nechceme si zakládat na díle přesahujícím hranice, které nám vyměřil Bůh, totiž abychom došli až k vám.
#10:14 Nechceme se rozpínat dál, než kam jsme sami došli - vždyť jsme ve své službě evangeliu Kristovu dospěli až k vám.
#10:15 Nezakládáme si na tom, co udělali jiní; máme však naději, že vaše víra bude růst a tak se hranice nám stanovená rozšíří,
#10:16 a že budu moci zvěstovat evangelium v krajích ležících za vámi a nebudu se chlubit tím, co je hotovo, kde už pracovali jiní.
#10:17 ‚Kdo se chlubí, ať se chlubí v Pánu.‘
#10:18 Ne ten, kdo doporučuje sám sebe, je osvědčený, nýbrž ten, koho doporučuje Pán. 
#11:1 Kéž byste ode mne snesli trochu nerozumu - ano, snášejte mne!
#11:2 Vždyť vás žárlivě střežím Boží žárlivostí; zasnoubil jsem vás jedinému muži, abych vás jako čistou pannu odevzdal Kristu.
#11:3 Obávám se však, aby to nebylo tak, jako když had ve své lstivosti oklamal Evu, aby totiž vaše mysl neztratila nevinnost a neodvrátila se od upřímné oddanosti Kristu.
#11:4 Když někdo přijde a zvěstuje jiného Ježíše, než jsme my zvěstovali, nebo vám nabízí jiného ducha, než jste dostali, nebo jiné evangelium, než jste přijali, klidně to snášíte!
#11:5 Mám však za to, že nejsem v ničem pozadu za těmi veleapoštoly!
#11:6 I když mi chybí výmluvnost, poznání mi nechybí; to jsme vám pokaždé a v každém ohledu ukázali.
#11:7 Nebo jsem se snad dopustil hříchu, když jsem sám sebe ponižoval, abyste vy byli vyvýšeni, že jsem vám totiž evangelium zvěstoval zadarmo?
#11:8 Jiné církve jsem okrádal, když jsem od nich přijímal podporu, abych mohl sloužit vám.
#11:9 Když jsem byl u vás a měl jsem nedostatek, nikomu jsem nebyl na obtíž, neboť čeho se mi nedostávalo, doplnili bratří, kteří přišli z Makedonie. Ve všem jsem jednal a budu jednat tak, abych pro vás nebyl břemenem.
#11:10 Jakože je při mně pravda Kristova, tato moje chlouba nebude v achajských končinách umlčena!
#11:11 Proč na tom trvám? Snad proto, že vás nemám v lásce? Bůh ví, že mám!
#11:12 Ale jak to dělám, budu dělat i nadále; nechci dát příležitost těm, kdo by se rádi chlubili, že si počínají jako my.
#11:13 Jsou to falešní apoštolové, nepoctiví dělníci, přestrojení za apoštoly Kristovy.
#11:14 A není divu, vždyť sám satan se převléká za anděla světla;
#11:15 není tedy nic překvapujícího na tom, že se jeho služebníci převlékají za služebníky spravedlnosti. Jejich konec bude jako jejich skutky!
#11:16 Opakuji: ať si nikdo nemyslí, že jsem tak nerozumný! Ale i kdyby si to někdo o mně myslel: přijměte mne i jako nerozumného, dovolte mi, abych se také já něčím maličko pochlubil.
#11:17 Co teď říkám, neříkám ve jménu Páně, nýbrž jako z nerozumu, v této roli vychloubače.
#11:18 Když se tak mnozí chlubí vnějšími věcmi, pochlubím se i já.
#11:19 Vy, kteří jste tak rozumní, rádi snášíte nerozumné.
#11:20 Ochotně snášíte, když vás někdo zotročuje, když vás někdo vyjídá, když vás obírá, když vás přezírá, když vás bije do tváře.
#11:21 Musím přiznat, že na něco takového jsem příliš slabý! Ale čeho se odváží někdo jiný - teď mluvím jako z nerozumu - toho se odvážím i já.
#11:22 Oni jsou Hebrejci? Já také! Jsou Izraelité? Já také! Jsou potomky Abrahamovými? Já také!
#11:23 Jsou služebníky Kristovými? Odpovím obzvlášť nerozumně: já tím víc! Namáhal jsem se usilovněji, ve vězení jsem byl vícekrát, ran jsem užil do sytosti, smrti jsem často hleděl do tváře.
#11:24 Od Židů jsem byl pětkrát odsouzen ke čtyřiceti ranám bez jedné,
#11:25 třikrát jsem byl trestán holí, jednou jsem byl kamenován, třikrát jsem s lodí ztroskotal, noc a den jsem jako trosečník strávil na širém moři.
#11:26 Častokrát jsem byl na cestách - v nebezpečí na řekách, v nebezpečí od lupičů, v nebezpečí od vlastního lidu, v nebezpečí od pohanů, v nebezpečí ve městech, v nebezpečí v pustinách, v nebezpečí na moři, v nebezpečí mezi falešnými bratřími,
#11:27 v námaze do úpadu, často v bezesných nocích, o hladu a žízni, v častých postech, v zimě a bez oděvu.
#11:28 A nadto ještě na mne denně doléhá starost o všechny církve.
#11:29 Je někdo sláb, abych já nebyl sláb spolu s ním? Propadá někdo pokušení, abych já se tím netrápil?
#11:30 Mám-li se chlubit, pochlubím se svou slabostí!
#11:31 Bůh a Otec Pána Ježíše, požehnaný na věky, ví, že nelžu.
#11:32 Místodržitel krále Arety dal hlídat brány města Damašku, aby se mne zmocnil,
#11:33 byl jsem však v koši spuštěn otvorem v hradbách, a tak jsem unikl jeho rukám. 
#12:1 Musím se pochlubit, i když to není k užitku; přicházím teď k viděním a zjevením Páně.
#12:2 Vím o člověku v Kristu, který byl před čtrnácti lety přenesen až do třetího nebe; zda to bylo v těle či mimo tělo, nevím - Bůh to ví.
#12:3 A vím o tomto člověku, že byl přenesen do ráje - zda v těle či mimo tělo, nevím, Bůh to ví -
#12:4 a uslyšel nevypravitelná slova, jež není člověku dovoleno vyslovit.
#12:5 Tím se budu chlubit, sám sebou se chlubit nebudu, leda svými slabostmi.
#12:6 I kdybych se chtěl chlubit, nebyl bych pošetilý, vždyť bych mluvil pravdu. Nechám toho však, aby si někdo o mně nemyslil víc, než co na mně vidí nebo ode mne slyší.
#12:7 A abych se nepovyšoval pro výjimečnost zjevení, jichž se mi dostalo, byl mi dán do těla osten, posel satanův, který mne sráží, abych se nepovyšoval.
#12:8 Kvůli tomu jsem třikrát volal k Pánu, aby mne toho zbavil,
#12:9 ale on mi řekl: „Stačí, když máš mou milost; vždyť v slabosti se projeví má síla.“ A tak se budu raději chlubit slabostmi, aby na mně spočinula moc Kristova.
#12:10 Proto rád přijímám slabost, urážky, útrapy, pronásledování a úzkosti pro Krista. Vždyť právě když jsem sláb, jsem silný.
#12:11 Propadl jsem nerozumu - ale k tomu jste mě donutili vy! Vždyť já bych měl být doporučován od vás! Nejsem přece v ničem pozadu za těmi veleapoštoly, i když nic nejsem.
#12:12 Znaky mého apoštolství se mezi vámi projevily s přesvědčivou vytrvalostí, v znameních, divech a mocných činech.
#12:13 V čem jste byli zkráceni ve srovnání s ostatními církvemi? Snad v tom, že jsem se vám nestal břemenem? Promiňte mi tuto křivdu!
#12:14 Jsem teď potřetí připraven jít k vám - a zase vám nebudu břemenem! Nestojím o váš majetek, nýbrž o vás. Děti přece nestřádají pro rodiče, nýbrž rodiče pro děti.
#12:15 Já velmi rád vynaložím všecko, ano vydám i sám sebe pro vaše duše. Když vás tak velice miluji, mám snad za to být méně milován?
#12:16 Na obtíž jsem vám tedy nebyl; ale co když jsem chytrák, který vás obelstil?
#12:17 Poslal jsem k vám snad někoho, skrze něhož jsem vás obral?
#12:18 Vybídl jsem Tita, aby vás navštívil, a spolu s ním jsem poslal ještě jednoho bratra. Obral vás snad Titus? Nejednali jsme v témž duchu a nešli ve stejných šlépějích?
#12:19 Už dávno si možná myslíte, že se chceme před vámi hájit. Nikoli, mluvíme v Kristu před tváří Boží, a to všecko, milí bratří, k vašemu růstu.
#12:20 Obávám se totiž, abych vás při svém příchodu neshledal, jakými bych vás mít nechtěl, a abych také já nebyl shledán, jakým vy mne mít nechcete - aby nepovstaly sváry, řevnivost, vášně, neurvalost, pomluvy, donašečství, nadutost, zmatky.
#12:21 Bojím se, až zase přijdu, aby mne Bůh před vámi nepokořil a abych se nemusel trápit nad mnohými z těch, kteří předtím žili v hříchu a dosud se v pokání neodvrátili od své nečistoty, necudnosti a bezuzdnosti. 
#13:1 Nyní se k vám chystám už potřetí - ‚ústy dvou nebo tří svědků bude potvrzena každá výpověď‘.
#13:2 Řekl jsem to už, když jsem byl u vás podruhé, a nyní to v dopise opakuji těm nekajícím hříšníkům i všem ostatním: až znovu přijdu, nebudu nikoho šetřit -
#13:3 žádáte-li důkaz toho, že skrze mne mluví Kristus, ten, který vůči vám není slabý, nýbrž je mezi vámi mocný.
#13:4 Zemřel sice na kříži v slabosti, ale z Boží moci je živ. I my jsme s ním slabí, ale pro vás budeme společně s ním žít z moci Boží.
#13:5 Sami sebe se ptejte, zda vskutku žijete z víry, sami sebe zkoumejte. Což nechápete, že Ježíš Kristus je mezi vámi? Ledaže jste před ním neobstáli!
#13:6 Doufám, že poznáte, že my jsme obstáli.
#13:7 Modlíme se k Bohu, abyste neudělali nic zlého - nejde nám o to, aby se ukázalo, jak my jsme obstáli, ale abyste vy činili to, co je správné, i kdybychom my neobstáli.
#13:8 Vždyť nic nezmůžeme proti pravdě, nýbrž jen ve jménu pravdy.
#13:9 Radujeme se, kdykoli vy jste silní, i když my jsme slabí. Za to se také modlíme - za vaši nápravu.
#13:10 Píšu vám to proto, abych, až přijdu, nemusel být přísný na základě pravomoci, kterou mi přece Pán dal k budování, nikoli k boření.
#13:11 Nakonec, bratří: žijte v radosti, napravujte své nedostatky, povzbuzujte se, buďte jednomyslní, pokojní, a Bůh lásky a pokoje bude s vámi. Pozdravte jedni druhé svatým políbením.
#13:12 Pozdravují vás všichni bratří.
#13:13 Milost našeho Pána Ježíše Krista a láska Boží a přítomnost Ducha svatého se všemi vámi.  

\book{Galatians}{Gal}
#1:1 Pavel, apoštol povolaný a pověřený nikoliv lidmi, ale Ježíšem Kristem a Bohem Otcem, který Ježíše vzkřísil z mrtvých,
#1:2 i všichni bratří, kteří jsou zde se mnou, církvím v Galacii:
#1:3 Milost vám a pokoj od Boha Otce našeho i Pána Ježíše Krista,
#1:4 který sám sebe vydal za naše hříchy, aby nás vysvobodil z nynějšího zlého věku podle vůle našeho Boha a Otce.
#1:5 Jemu buď sláva na věky věků. Amen.
#1:6 Divím se, že se od toho, který vás povolal milostí Kristovou, tak rychle odvracíte k jinému evangeliu.
#1:7 Jiné evangelium ovšem není; jsou jen někteří lidé, kteří vás zneklidňují a chtějí evangelium Kristovo obrátit v pravý opak.
#1:8 Ale i kdybychom my nebo sám anděl z nebe přišel hlásat jiné evangelium než to, které jsme vám zvěstovali, budiž proklet!
#1:9 Jak jsem právě řekl, a znovu to opakuji: Jestliže vám někdo hlásá jiné evangelium než to, které jste přijali, budiž proklet!
#1:10 Jde mi o přízeň u lidí, anebo u Boha? Snažím se zalíbit lidem? Kdybych se stále ještě chtěl líbit lidem, nebyl bych služebníkem Kristovým.
#1:11 Ujišťuji vás, bratří, že evangelium, které jste ode mne slyšeli, není z člověka.
#1:12 Vždyť já jsem je nepřevzal od žádného člověka ani se mu nenaučil od lidí, nýbrž zjevil mi je sám Ježíš Kristus.
#1:13 Slyšeli jste přece o tom, jak jsem si kdysi vedl, když jsem ještě byl oddán židovství, jak horlivě jsem pronásledoval církev Boží a snažil se ji vyhubit.
#1:14 Vynikal jsem ve věrnosti k židovství nad mnoho vrstevníků v našem lidu a nadmíru jsem horlil pro tradice našich otců.
#1:15 Ale ten, který mě vyvolil už v těle mé matky a povolal mě svou milostí, rozhodl se
#1:16 zjeviti mně svého Syna, abych radostnou zvěst o něm nesl všem národům. Tehdy jsem nešel o radu k žádnému člověku,
#1:17 ani jsem se nevypravil do Jeruzaléma k těm, kteří byli apoštoly dříve než já, nýbrž odešel jsem do Arábie a potom jsem se zase vrátil do Damašku.
#1:18 Teprve o tři léta později jsem se vydal do Jeruzaléma, abych se seznámil s Petrem, a zůstal jsem u něho dva týdny.
#1:19 Nikoho jiného z apoštolů jsem neviděl, jen Jakuba, bratra Páně.
#1:20 Před tváří Boží vás ujišťuji, že to, co vám píšu, není lež.
#1:21 Potom jsem odešel do končin Sýrie a Kilikie.
#1:22 V církvích Kristových v Judsku mne osobně neznali;
#1:23 jen slyšeli, že ten, který dříve pronásledoval církev, nyní zvěstuje víru, kterou předtím chtěl vyhubit;
#1:24 a děkovali za mne Bohu. 
#2:1 Potom jsem se po čtrnácti letech znovu vypravil do Jeruzaléma spolu s Barnabášem a vzal jsem s sebou i Tita.
#2:2 Šel jsem tam na Boží pokyn a těm, kteří jsou ve zvláštní vážnosti, jsem v soukromí předložil evangelium, které zvěstuji pohanům, aby snad moje nynější i dřívější úsilí nebylo nadarmo.
#2:3 Ale ani Titus, který tam byl se mnou a je Řek, nebyl přinucen, aby se dal obřezat,
#2:4 jak chtěli ti, kteří předstírali, že jsou bratří, a pokoutně se mezi nás vetřeli s úmyslem slídit po naší svobodě, kterou máme v Kristu, aby nás uvedli do otroctví zákona.
#2:5 Před těmi jsme však ani na okamžik necouvli a nepodrobili jsme se jim, aby vám byla zachována pravda evangelia.
#2:6 Od těch však, kteří se těšili zvláštní vážnosti - čím kdysi byli, na tom mi nic nezáleží, Bůh přece nikomu nestraní - ti tedy, kteří se těšili zvláštní vážnosti, mi nic dalšího neuložili;
#2:7 naopak nahlédli, že mně bylo svěřeno zvěstovat evangelium pohanům tak jako Petrovi židům.
#2:8 Vždyť ten, který dal Petrovi sílu k apoštolství mezi židy, dal ji také mně k službě mezi pohany.
#2:9 Když poznali milost, která mi byla dána - Jakub a Petr a Jan, kteří byli uznáváni za sloupy církve - podali mně a Barnabášovi pravici na stvrzení naší dohody, že my půjdeme mezi pohany a oni mezi židy.
#2:10 Jen žádali, abychom pamatovali na jejich chudé, a právě o to jsem vždy horlivě usiloval.
#2:11 Když pak Petr přišel do Antiochie, postavil jsem se otevřeně proti němu, protože byl zřejmě v neprávu.
#2:12 Nejprve jídal totiž společně s pohany; když však přišli někteří lidé z okolí Jakubova, začal couvat a oddělovat se, protože se bál zastánců obřízky.
#2:13 A spolu s ním se takto pokrytecky chovali i ostatní Židé, takže jejich pokrytectvím se dal strhnout i Barnabáš.
#2:14 Když jsem však viděl, že nejdou přímo za pravdou evangelia, řekl jsem Petrovi přede všemi: „Jestliže ty, který jsi Žid, nedodržuješ mezi námi židovský zákon, jak to, že nutíš pohany, aby ho dodržovali?“
#2:15 My jsme od narození Židé a ne ‚hříšní pohané‘,
#2:16 víme však, že člověk se nestává spravedlivým před Bohem na základě skutků přikázaných zákonem, nýbrž vírou v Krista Ježíše. I my jsme uvěřili v Ježíše Krista, abychom došli spravedlnosti z víry v Krista, a ne ze skutků zákona. Vždyť ze skutků zákona ‚nebude nikdo ospravedlněn‘.
#2:17 Jestliže hledáme ospravedlnění v Kristu a jsme tedy zřejmě i my hříšníci, je snad proto Kristus služebníkem hříchu? Naprosto ne!
#2:18 Jestliže chci znovu stavět, co jsem dříve zbořil, usvědčuji sám sebe jako viníka před zákonem.
#2:19 Já však, odsouzen zákonem, jsem mrtev pro zákon, abych živ byl pro Boha. Jsem ukřižován spolu s Kristem,
#2:20 nežiji už já, ale žije ve mně Kristus. A život, který zde nyní žiji, žiji ve víře v Syna Božího, který si mne zamiloval a vydal sebe samého za mne.
#2:21 Nepohrdám Boží milostí: Kdybychom mohli dosáhnout spravedlnosti skrze zákon, byla by Kristova smrt zbytečná. 
#3:1 Vy pošetilí Galatští, kdo vás to obloudil - vždyť vám byl tak jasně postaven před oči Ježíš Kristus ukřižovaný!
#3:2 Chtěl bych se vás zeptat jen na jedno: dal vám Bůh svého Ducha proto, že jste činili skutky zákona, nebo proto, že jste uvěřili zvěsti, kterou jste slyšeli?
#3:3 To jste tak pošetilí? Začali jste žít z Ducha Božího, a teď spoléháte sami na sebe?
#3:4 Tak veliké věci jste prožili nadarmo? A kdyby jen nadarmo!
#3:5 Ten, který vám udílí Ducha a působí mezi vámi mocné činy, činí tak proto, že plníte zákon, nebo proto, že jste slyšeli a uvěřili?
#3:6 Pohleďte na Abrahama: ‚uvěřil Bohu, a bylo mu to počítáno za spravedlnost.‘
#3:7 Pochopte tedy, že syny Abrahamovými jsou lidé víry.
#3:8 Protože se v Písmu předvídá, že Bůh na základě víry ospravedlní pohanské národy, dostal už Abraham zaslíbení: ‚V tobě dojdou požehnání všechny národy.‘
#3:9 A tak lidé víry docházejí požehnání spolu s věřícím Abrahamem.
#3:10 Ti však, kteří spoléhají na skutky zákona, jsou pod kletbou, neboť stojí psáno: ‚Proklet je každý, kdo nezůstává věren všemu, co je psáno v zákoně, a nečiní to.‘
#3:11 Je jasné, že nikdo není před Bohem ospravedlněn na základě zákona, neboť čteme: ‚Spravedlivý bude živ z víry.‘
#3:12 Zákon však nevychází z víry, nýbrž praví: ‚Kdo bude tyto věci činit, získá tím život.‘
#3:13 Ale Kristus nás vykoupil z kletby zákona tím, že za nás vzal prokletí na sebe, neboť je psáno: ‚Proklet je každý, kdo visí na dřevě‘.
#3:14 To proto, aby požehnání dané Abrahamovi dostaly v Ježíši Kristu i pohanské národy, abychom zaslíbeného Ducha přijali skrze víru.
#3:15 Bratří, znázorním to příkladem: ani lidskou závěť jednou pravoplatně potvrzenou nemůže nikdo zrušit nebo k ní něco přidat.
#3:16 Slib byl dán Abrahamovi a ‚jeho potomku‘; nemluví se o potomcích, nýbrž o potomku: je jím Kristus.
#3:17 Chci tím říci: Smlouvu, od Boha dávno pravoplatně potvrzenou, nemůže učinit neplatnou zákon, vydaný teprve po čtyřech stech třiceti letech, a tak zrušit slib.
#3:18 Kdyby totiž dědictví plynulo ze zákona, nebylo by založeno na slibu. Abrahamovi je však z milosti Bůh přiřkl svým slibem.
#3:19 Jak je to potom se zákonem? Byl přidán kvůli proviněním jen do doby, než přijde ten zaslíbený potomek; byl vyhlášen anděly a svěřen lidskému prostředníku.
#3:20 Prostředníka není potřebí tam, kde jedná jen jeden, a Bůh je jeden.
#3:21 Je tedy zákon proti Božím slibům? Naprosto ne! Kdyby tu byl zákon, který by mohl dát život, pak by vskutku spravedlnost byla ze zákona.
#3:22 Ale podle Písma je všechno v zajetí hříchu, aby se zaslíbení, dané víře v Ježíše Krista, splnilo těm, kdo věří.
#3:23 Dokud nepřišla víra, byli jsme zajatci, které zákon střežil pro chvíli, kdy víra měla být zjevena.
#3:24 Zákon byl tedy naším dozorcem až do příchodu Kristova, až do ospravedlnění z víry.
#3:25 Když však přišla víra, nemáme již nad sebou dozorce.
#3:26 Vy všichni jste přece skrze víru syny Božími v Kristu Ježíši.
#3:27 Neboť vy všichni, kteří jste byli pokřtěni v Krista, také jste Krista oblékli.
#3:28 Není už rozdíl mezi židem a pohanem, otrokem a svobodným, mužem a ženou.
#3:29 Vy všichni jste jedno v Kristu Ježíši. Jste-li Kristovi, jste potomstvo Abrahamovo a dědicové toho, co Bůh zaslíbil. 
#4:1 Chci říci: Pokud je dědic nezletilý, ničím se neliší od otroka, ač je pánem všeho.
#4:2 Je podřízen poručníkům a správcům až do doby, kterou otec předem stanovil.
#4:3 Tak i my, když jsme byli nedospělí, byli jsme otroky vesmírných mocí.
#4:4 Když se však naplnil stanovený čas, poslal Bůh svého Syna, narozeného z ženy, podrobeného zákonu,
#4:5 aby vykoupil ty, kteří jsou zákonu podrobeni, tak abychom byli přijati za syny.
#4:6 Protože jste synové, poslal Bůh do našich srdcí Ducha svého Syna, Ducha volajícího Abba, Otče.
#4:7 A tak už nejsi otrok, nýbrž syn, a když syn, tedy z moci Boží i dědic.
#4:8 Dříve jste však neznali Boha a byli jste otroky bohů, kteří ve skutečnosti bohy nejsou.
#4:9 Nyní jste však Boha poznali; lépe řečeno: byli jste od Boha poznáni. Jak to, že se zase navracíte k těm bezmocným a ubohým mocnostem a chcete se jim dát znovu do otroctví?
#4:10 Dodržujete ustanovení pro dny a měsíce, období a roky!
#4:11 Bojím se, aby úsilí, které jsem vám věnoval, nebylo nakonec nadarmo.
#4:12 Snažte se mi porozumět, bratří, jako já mám porozumění pro vás; prosím vás o to. Nic jste mi neublížili.
#4:13 Víte, že jsem byl nemocen, když jsem u vás poprvé zvěstoval evangelium.
#4:14 Vy jste se však ode mne neodvrátili s ošklivostí, ačkoli to pro vás bylo pokušením, ale přijali jste mne jako posla Božího, jako Krista Ježíše.
#4:15 Kam se podělo to vaše nadšení? Mohu vám dosvědčit, že kdyby to bylo možné, byli byste pro mne obětovali vlastní oči.
#4:16 Stal jsem se vaším nepřítelem tím, že vám říkám pravdu?
#4:17 Oni se o vás horlivě ucházejí, ale nemyslí to dobře; chtějí vás připravit o spásu a strhnout vás na svou stranu.
#4:18 Je správné horlit, ale pro dobrou věc a vždycky, nejen tehdy, když jsem u vás, moje děti.
#4:19 Znovu vás v bolestech rodím, dokud nebudete dotvořeni v podobu Kristovu.
#4:20 Jak bych teď chtěl být u vás a najít pro svou řeč pravý tón, vždyť si s vámi nevím rady!
#4:21 Odpovězte mi vy, kteří chcete být pod zákonem: Co slyšíte v zákoně?
#4:22 Čteme tam, že Abraham měl dva syny, jednoho z otrokyně a druhého ze ženy svobodné.
#4:23 Ten z otrokyně se narodil jen z vůle člověka, ten ze svobodné podle zaslíbení.
#4:24 Je to řečeno obrazně. Ty dvě ženy jsou dvě smlouvy, jedna z hory Sínaj, která rodí děti do otroctví; to je Hagar.
#4:25 Hagar znamená horu Sínaj v Arábii a odpovídá nynějšímu Jeruzalému, neboť žije v otroctví i se svými dětmi.
#4:26 Ale budoucí Jeruzalém je svobodný, a to je naše matka.
#4:27 Vždyť stojí psáno: ‚Raduj se, neplodná, která nerodíš, jásej a volej, která nemáš bolesti, neboť mnoho dětí bude mít osamělá, více než ta, která má muže.‘
#4:28 Vy, bratří, jste dětmi zaslíbení jako Izák.
#4:29 Ale jako tenkrát ten, který se narodil pouze z těla, pronásledoval toho, který se narodil z moci Ducha, tak je tomu i nyní.
#4:30 Co však říká Písmo? ‚Vyžeň otrokyni i jejího syna, neboť syn otrokyně nebude dědicem spolu se synem svobodné.‘
#4:31 A proto, bratří, nejsme syny otrokyně, nýbrž ženy svobodné. 
#5:1 Tu svobodu nám vydobyl Kristus. Stůjte proto pevně a nedejte si na sebe znovu vložit otrocké jho.
#5:2 Slyšte, co vám já Pavel říkám: Dáváte-li se obřezat, Kristus vám nic neprospěje.
#5:3 Znovu dosvědčuji každému, kdo se dá obřezat, že je zavázán zachovávat celý zákon.
#5:4 Odloučili jste se od Krista vy všichni, kteří chcete dojít ospravedlnění na základě zákona, pozbyli jste milosti.
#5:5 My však z moci Ducha a ve víře očekáváme spravedlnost, která je naší nadějí.
#5:6 V Kristu Ježíši nezáleží na tom, je-li někdo obřezán či ne; rozhodující je víra, která se uplatňuje láskou.
#5:7 Běželi jste dobře! Kdo vám zabránil, abyste se drželi pravdy?
#5:8 Jistě to nevyšlo od toho, který vás povolává.
#5:9 Málo kvasu celé těsto prokvasí.
#5:10 Ale já k vám v Pánu mám důvěru, že se neuchýlíte k jinému smýšlení. Ten však, kdo vás uvádí do zmatku, neujde soudu, ať je to kdokoli.
#5:11 Kdybych já, bratří, dosud kázal, že obřízka je nutná, proč bych byl vlastně pronásledován? Vždyť by tím bylo odstraněno pohoršení, jímž je kříž.
#5:12 Ti, kteří mezi vás kvůli obřízce vnášejí neklid, ať se rovnou vyklestí!
#5:13 Vy jste byli povoláni ke svobodě, bratří. Jen nemějte svobodu za příležitost k prosazování sebe, ale služte v lásce jedni druhým.
#5:14 Vždyť celý zákon je shrnut v jednom slově: Milovati budeš bližního svého jako sebe samého!
#5:15 Jestliže však jeden druhého koušete a požíráte, dejte si pozor, abyste se navzájem nezahubili.
#5:16 Chci říci: Žijte z moci Božího Ducha, a nepodlehnete tomu, k čemu vás táhne vaše přirozenost.
#5:17 Touhy lidské přirozenosti směřují proti Duchu Božímu, a Boží Duch proti nim. Jde tu o naprostý protiklad, takže děláte to, co dělat nechcete.
#5:18 Dáte-li se však vést Božím Duchem, nejste už pod zákonem.
#5:19 Skutky lidské svévole jsou zřejmé: necudnost, nečistota, bezuzdnost,
#5:20 modlářství, čarodějství, rozbroje, hádky, žárlivost, vášeň, podlost, rozpory, rozkoly,
#5:21 závist, opilství, nestřídmost a podobné věci. Řekl jsem už dříve a říkám znovu, že ti, kteří takové věci dělají, nebudou mít podíl na království Božím.
#5:22 Ovoce Božího Ducha však je láska, radost, pokoj, trpělivost, laskavost, dobrota, věrnost,
#5:23 tichost a sebeovládání. Proti tomu se zákon neobrací.
#5:24 Ti, kteří náležejí Kristu Ježíši, ukřižovali sami sebe se svými vášněmi a sklony.
#5:25 Jsme-li živi Božím Duchem, dejme se Duchem také řídit.
#5:26 Nehledejme prázdnou slávu, nebuďme jeden k druhému vyzývaví, nezáviďme jeden druhému. 
#6:1 Bratří, upadne-li někdo z vás do nějakého provinění, vy, kteří jste vedeni Božím Duchem, přivádějte ho na pravou cestu v duchu mírnosti a každý si dej pozor sám na sebe, abys také nepodlehl pokušení.
#6:2 Berte na sebe břemena jedni druhých, tak naplníte zákon Kristův.
#6:3 Myslí-li si někdo, že je něco, a přitom není nic, klame sám sebe.
#6:4 Každý ať zkoumá své vlastní jednání; pak bude mít čím se chlubit, bude-li hledět jen na sebe a nebude se porovnávat s druhými.
#6:5 Každý bude odpovídat sám za sebe.
#6:6 Kdo je vyučován v slovu, nechť se s vyučujícím dělí o všechno potřebné k životu.
#6:7 Neklamte se, Bohu se nikdo nebude posmívat. Co člověk zaseje, to také sklidí.
#6:8 Kdo zasévá pro své sobectví, sklidí zánik, kdo však zasévá pro Ducha, sklidí život věčný.
#6:9 V konání dobra neumdlévejme; neochabneme-li, budeme sklízet v ustanovený čas.
#6:10 A tak dokud je čas, čiňme dobře všem, nejvíce však těm, kteří patří do rodiny víry.
#6:11 Teď vám píši vlastní rukou; všimněte si velkého písma.
#6:12 Ti, kteří chtějí dobře vypadat před lidmi, nutí vás, abyste se dávali obřezat, jen aby nebyli pronásledováni pro kříž Krista Ježíše.
#6:13 Vždyť ani ti, kdo jsou obřezáni, zákon nezachovávají; chtějí, abyste se dali obřezat jen proto, aby se mohli pochlubit tím, co se stalo na vašem těle.
#6:14 Já však se zanic nechci chlubit ničím, leč křížem našeho Pána Ježíše Krista, jímž je pro mne svět ukřižován a já pro svět.
#6:15 Neboť nezáleží na obřezanosti ani neobřezanosti, nýbrž jen na novém stvoření.
#6:16 A všem, kdo se budou řídit tímto pravidlem, Izraeli Božímu, pokoj a slitování.
#6:17 Ať už mi nikdo nepůsobí těžkosti, vždyť já nosím na svém těle jizvy Ježíšovy.
#6:18 Milost našeho Pána Ježíše Krista buď s vámi, bratří. Amen.  

\book{Ephesians}{Eph}
#1:1 Pavel, z Boží vůle apoštol Ježíše Krista, bratřím věrným v Kristu Ježíši:
#1:2 Milost vám a pokoj od Boha Otce našeho a Pána Ježíše Krista.
#1:3 Pochválen buď Bůh a Otec našeho Pána Ježíše Krista, který nás v Kristu obdařil vším duchovním požehnáním nebeských darů;
#1:4 v něm nás již před stvořením světa vyvolil, abychom byli svatí a bez poskvrny před jeho tváří.
#1:5 Ve své lásce nás předem určil, abychom rozhodnutím jeho dobroty byli skrze Ježíše Krista přijati za syny
#1:6 a chválili slávu jeho milosti, kterou nám udělil ve svém Nejmilejším.
#1:7 V něm jsme vykoupeni jeho obětí a naše hříchy jsou nám odpuštěny pro přebohatou milost,
#1:8 kterou nás zahrnul ve vší moudrosti a prozíravosti,
#1:9 když nám dal poznat tajemství svého záměru, svého milostivého rozhodnutí, jímž si předsevzal,
#1:10 že podle svého plánu, až se naplní čas, přivede všechno na nebi i na zemi k jednotě v Kristu.
#1:11 On je ten, v němž se nám od Boha, jenž všechno působí rozhodnutím své vůle, dostalo podílu na předem daném poslání,
#1:12 abychom my, kteří jsme na Krista upnuli svou naději, stali se chválou jeho slávy.
#1:13 V něm byla i vám, když jste uslyšeli slovo pravdy, evangelium o svém spasení, a uvěřili mu, vtisknuta pečeť zaslíbeného Ducha svatého
#1:14 jako závdavek našeho dědictví na vykoupení těch, které si Bůh vydobyl k chvále své slávy.
#1:15 Proto i já, když jsem uslyšel o vaší víře v Pána Ježíše a lásce ke všem bratřím,
#1:16 nepřestávám za vás děkovat a stále na vás pamatuji ve svých modlitbách.
#1:17 Prosím, aby vám Bůh našeho Pána Ježíše Krista, Otec slávy, dal ducha moudrosti a zjevení, abyste ho poznali
#1:18 a osvíceným vnitřním zrakem viděli, k jaké naději vás povolal, jak bohaté a slavné je vaše dědictví v jeho svatém lidu
#1:19 a jak nesmírně veliký je ve své moci k nám, kteří věříme.
#1:20 Sílu svého mocného působení prokázal přece na Kristu: Vzkřísil ho z mrtvých a posadil po své pravici v nebesích,
#1:21 vysoko nad všechny vlády, mocnosti, síly i panstva, nad všechna jména, která jsou vzývána, jak v tomto věku, tak i v budoucím.
#1:22 ‚Všechno podrobil pod jeho nohy‘ a ustanovil jej svrchovanou hlavou církve,
#1:23 která je jeho tělem, plností toho, jenž přivádí k naplnění všechno, co jest. 
#2:1 I vy jste byli mrtvi pro své viny a hříchy,
#2:2 v nichž jste dříve žili podle běhu tohoto světa, poslušni vládce nadzemských mocí, ducha, působícího dosud v těch, kteří vzdorují Bohu.
#2:3 I my všichni jsme k nim kdysi patřili; žili jsme sklonům svého těla, dali jsme se vést svými sobeckými zájmy, a tím jsme nutně propadli Božímu soudu tak jako ostatní.
#2:4 Ale Bůh, bohatý v milosrdenství, z velké lásky, jíž si nás zamiloval,
#2:5 probudil nás k životu spolu s Kristem, když jsme byli mrtvi pro své hříchy. Milostí jste spaseni!
#2:6 Spolu s ním nás vzkřísil a spolu s ním uvedl na nebeský trůn v Kristu Ježíši,
#2:7 aby se nadcházejícím věkům prokázalo, jak nesmírné bohatství milosti je v jeho dobrotě k nám v Kristu Ježíši.
#2:8 Milostí tedy jste spaseni skrze víru.
#2:9 Spasení není z vás, je to Boží dar; není z vašich skutků, takže se nikdo nemůže chlubit.
#2:10 Jsme přece jeho dílo, v Kristu Ježíši stvořeni k tomu, abychom konali dobré skutky, které nám Bůh připravil.
#2:11 Pamatujte proto vy, kteří jste svým původem pohané a kterým ti, kdo jsou obřezaní na těle a lidskou rukou, říkají neobřezanci,
#2:12 že jste v té době opravdu byli bez Krista, odloučeni od společenství Izraele, bez účasti na smlouvách Božího zaslíbení, bez naděje a bez Boha na světě.
#2:13 Ale v Kristu Ježíši jste se nyní vy, kdysi vzdálení, stali blízkými pro Kristovu prolitou krev.
#2:14 V něm je náš mír, on dvojí spojil v jedno, když zbořil zeď, která rozděluje a působí svár. Svou obětí odstranil
#2:15 zákon ustanovení a předpisů, aby z těch dvou, z žida i pohana, stvořil jednoho nového člověka, a tak nastolil mír.
#2:16 Oba dva usmířil s Bohem v jednom těle, na kříži usmrtil jejich nepřátelství.
#2:17 Přišel a zvěstoval mír, mír vám, kteří jste dalecí, i těm, kteří jsou blízcí.
#2:18 A tak v něm smíme obojí, židé i pohané, v jednotě Ducha stanout před Otcem.
#2:19 Nejste již tedy cizinci a přistěhovalci, máte právo Božího lidu a patříte k Boží rodině.
#2:20 Jste stavbou, jejímž základem jsou apoštolové a proroci a úhelným kamenem sám Kristus Ježíš.
#2:21 V něm je celá stavba pevně spojena a roste v chrám, posvěcený v Pánu;
#2:22 v něm jste i vy společně budováni v duchovní příbytek Boží. 
#3:1 Proto jsem já, Pavel, vězněm Krista Ježíše pro vás pohany.
#3:2 Slyšeli jste přece o milosti, kterou mi Bůh podle svého plánu udělil kvůli vám:
#3:3 dal mi ve zjevení poznat tajemství, které jsem vám právě několika slovy vypsal.
#3:4 Z toho můžete vyčíst, že jsem porozuměl Kristovu tajemství,
#3:5 které v dřívějších pokoleních nebylo lidem známo, ale nyní je Duchem zjeveno jeho svatým apoštolům a prorokům:
#3:6 že pohané jsou spoludědicové, část společného těla, a mají v Kristu Ježíši podíl na zaslíbeních evangelia.
#3:7 Jeho služebníkem jsem se stal, když mě Bůh obdaroval svou milostí a působí ve mně svou mocí:
#3:8 mně, daleko nejmenšímu ze všech bratří, byla dána ta milost, abych pohanům zvěstoval nevystižitelné Kristovo bohatství
#3:9 a vynesl na světlo smysl tajemství od věků ukrytého v Bohu, jenž vše stvořil:
#3:10 Bůh chce, aby nebeským vládám a mocnostem bylo nyní skrze církev dáno poznat jeho mnohotvarou moudrost,
#3:11 podle odvěkého určení, které naplnil v Kristu Ježíši, našem Pánu.
#3:12 V něm smíme i my ve víře přistupovat k Bohu svobodně a s důvěrou.
#3:13 Proto prosím, abyste se nedali odradit tím, že pro vás musím trpět; vždyť se to obrátí k vaší slávě.
#3:14 Proto klekám na kolena před Otcem,
#3:15 od něhož pochází každý nebeský i pozemský rod, a prosím,
#3:16 aby se pro bohatství Boží slávy ve vás jeho Duchem posílil a upevnil ‚vnitřní člověk‘
#3:17 a aby Kristus skrze víru přebýval ve vašich srdcích; a tak abyste zakořeněni a zakotveni v lásce
#3:18 mohli spolu se všemi bratřími pochopit, co je skutečná šířka a délka, výška i hloubka:
#3:19 poznat Kristovu lásku, která přesahuje každé poznání, a dát se prostoupit vší plností Boží.
#3:20 Tomu pak, který působením své moci mezi námi může učinit neskonale víc, než zač prosíme a co si dovedeme představit,
#3:21 jemu samému buď sláva v církvi a v Kristu Ježíši po všecka pokolení na věky věků! Amen. 
#4:1 Proto vás já, vězeň kvůli Pánu, prosím, abyste tomu povolání, kterého se vám dostalo,
#4:2 dělali čest svým životem, vždy skromní, tiší a trpěliví. Snášejte se navzájem v lásce
#4:3 a usilovně hleďte zachovat jednotu Ducha, spojeni svazkem pokoje.
#4:4 Jedno tělo a jeden Duch, k jedné naději jste byli povoláni;
#4:5 jeden je Pán, jedna víra, jeden křest,
#4:6 jeden Bůh a Otec všech, který je nade všemi, skrze všechny působí a je ve všech.
#4:7 Každému z nás byla dána milost podle míry Kristova obdarování.
#4:8 Proto je řečeno: ‚Vystoupil vzhůru, zajal nepřátele, dal dary lidem.‘
#4:9 Co jiného znamená ‚vystoupil‘, než že předtím sestoupil dolů na zem?
#4:10 Ten, který sestoupil, je tedy tentýž, který také vystoupil nade všechna nebesa, aby naplnil všechno, co jest.
#4:11 A toto jsou jeho dary: jedny povolal za apoštoly, jiné za proroky, jiné za zvěstovatele evangelia, jiné za pastýře a učitele,
#4:12 aby své vyvolené dokonale připravil k dílu služby - k budování Kristova těla,
#4:13 až bychom všichni dosáhli jednoty víry a poznání Syna Božího, a tak dorostli zralého lidství, měřeno mírou Kristovy plnosti.
#4:14 Pak už nebudeme nedospělí, nebudeme zmítáni a unášeni závanem kdejakého učení - lidskou falší, chytráctvím a lstivým sváděním k bludu.
#4:15 Buďme pravdiví v lásce, ať ve všem dorůstáme v Krista. On je hlava,
#4:16 z něho roste celé tělo, pevně spojené klouby navzájem se podpírajícími, a buduje se v lásce podle toho, jak je každé části dáno.
#4:17 To vám říkám a dotvrzuji jménem Páně: nežijte tak, jako žijí pohané podle svých marných představ.
#4:18 Mají zatemnělou mysl a odcizili se Božímu životu pro svou nevědomost a zatvrzelé srdce.
#4:19 Otupěli, propadli bezuzdnosti a s chtivostí dělají hanebné věci.
#4:20 Vy jste se však u Krista takovým věcem neučili -
#4:21 pokud jste ovšem o něm slyšeli a byli v něm vyučeni podle pravdy, která je v Ježíši.
#4:22 Odložte dřívější způsob života, staré lidství, které hyne klamnými vášněmi,
#4:23 obnovte se duchovním smýšlením,
#4:24 oblecte nové lidství, stvořené k Božímu obrazu ve spravedlnosti a svatosti pravdy.
#4:25 Proto zanechte lži a ‚mluvte pravdu každý se svým bližním‘, vždyť jste údy téhož těla.
#4:26 ‚Hněváte-li se, nehřešte.‘ Nenechte nad svým hněvem zapadnout slunce
#4:27 a nedopřejte místa ďáblu.
#4:28 Kdo kradl, ať už nekrade, ale ať raději přiloží ruce k pořádné práci, aby se měl o co rozdělit s potřebnými.
#4:29 Z vašich úst ať nevyjde ani jedno špatné slovo, ale vždy jen dobré, které by pomohlo, kde je třeba, a tak posluchačům přineslo milost.
#4:30 A nezarmucujte svatého Ducha Božího, jehož pečeť nesete pro den vykoupení.
#4:31 Ať je vám vzdálena všechna tvrdost, zloba, hněv, křik, utrhání a s tím i každá špatnost;
#4:32 buďte k sobě navzájem laskaví, milosrdní, odpouštějte si navzájem, jako i Bůh v Kristu odpustil vám. 
#5:1 Jako milované děti následujte Božího příkladu
#5:2 a žijte v lásce, tak jako Kristus miloval nás a sám sebe dal za nás jako dar a oběť, jejíž vůně je Bohu milá.
#5:3 O smilstvu, jakékoliv nezřízenosti nebo chamtivosti ať se mezi vámi ani nemluví, jak se sluší na ty, kdo patří Bohu.
#5:4 Vést sprosté, hloupé a dvojsmyslné řeči se nepatří; vy máte vzdávat Bohu díky!
#5:5 Dobře si pamatujte, že žádný smilník, prostopášník ani lakomec, jehož bohem jsou peníze, nemá podíl v království Kristovu a Božím.
#5:6 Nenechte se od nikoho svést prázdnými slovy, aby vás nestihl Boží hněv jako ty, kdo ho neposlouchají.
#5:7 Proto s nimi nemějte nic společného.
#5:8 I vy jste kdysi byli tmou, ale nyní vás Pán učinil světlem.
#5:9 Žijte proto jako děti světla - ovocem světla je vždy dobrota, spravedlnost a pravda;
#5:10 zkoumejte, co se líbí Pánu.
#5:11 Nepodílejte se na neužitečných skutcích tmy, naopak je nazývejte pravým jménem.
#5:12 O tom, co oni dělají potají, je odporné i jen mluvit.
#5:13 Když se však ty věci správně pojmenují, je jasné, oč jde.
#5:14 A kde se rozjasní, tam je světlo. Proto je řečeno: Probuď se, kdo spíš, vstaň z mrtvých, a zazáří ti Kristus.
#5:15 Dávejte si dobrý pozor na to, jak žijete, abyste si nepočínali jako nemoudří, ale jako moudří;
#5:16 nepromarněte tento čas, neboť nastaly dny zlé.
#5:17 Proto nebuďte nerozumní, ale hleďte pochopit, co je vůle Páně.
#5:18 A neopíjejte se vínem, což je prostopášnost,
#5:19 ale plni Ducha zpívejte společně žalmy, chvalozpěvy a duchovní písně. Zpívejte Pánu, chvalte ho z celého srdce
#5:20 a vždycky za všecko vzdávejte díky Bohu a Otci ve jménu našeho Pána Ježíše Krista.
#5:21 V poddanosti Kristu se podřizujte jedni druhým:
#5:22 ženy svým mužům jako Pánu,
#5:23 protože muž je hlavou ženy, jako Kristus je hlavou církve, těla, které spasil.
#5:24 Ale jako církev je podřízena Kristu, tak ženy mají být ve všem podřízeny svým mužům.
#5:25 Muži, milujte své ženy, jako si Kristus zamiloval církev a sám se za ni obětoval,
#5:26 aby ji posvětil a očistil křtem vody a slovem;
#5:27 tak si on sám připravil církev slavnou, bez poskvrny, vrásky a čehokoli podobného, aby byla svatá a bezúhonná.
#5:28 Proto i muži mají milovat své ženy jako své vlastní tělo. Kdo miluje svou ženu, miluje sebe.
#5:29 Nikdo přece nemá v nenávisti své tělo, ale živí je a stará se o ně. Tak i Kristus pečuje o církev;
#5:30 vždyť jsme údy jeho těla.
#5:31 ‚Proto opustí muž otce i matku a připojí se k své manželce, a budou ti dva jedno tělo‘.
#5:32 Je to velké tajemství, které vztahuji na Krista a na církev.
#5:33 A tak i každý z vás bez výjimky ať miluje svou ženu jako sebe sama a žena ať má před mužem úctu. 
#6:1 Děti, poslouchejte své rodiče, protože to je spravedlivé před Bohem.
#6:2 ‚Cti otce svého i matku svou‘ je přece jediné přikázání, které má zaslíbení:
#6:3 ‚aby se ti dobře vedlo a abys byl dlouho živ na zemi.‘
#6:4 Otcové, nedrážděte své děti ke vzdoru, ale vychovávejte je v kázni a napomenutích našeho Pána.
#6:5 Otroci, poslouchejte své pozemské pány s uctivou pokorou a z upřímného přesvědčení jako Krista.
#6:6 Nejen naoko, abyste se zalíbili lidem, ale jako služebníci Kristovi, kteří rádi plní Boží vůli
#6:7 a lidem slouží ochotně, jako by sloužili Pánu.
#6:8 Víte, že Pán odmění každého, kdo něco dobrého učiní, ať je to otrok nebo svobodný.
#6:9 Vy páni, jednejte s otroky také tak a zanechte vyhrůžek. Víte přece, že jejich i váš Pán je v nebesích, a ten nikomu nestraní.
#6:10 A tak, bratří, svou sílu hledejte u Pána, v jeho veliké moci.
#6:11 Oblecte plnou Boží zbroj, abyste mohli odolat ďáblovým svodům.
#6:12 Nevedeme svůj boj proti lidským nepřátelům, ale proti mocnostem, silám a všemu, co ovládá tento věk tmy, proti nadzemským duchům zla.
#6:13 Proto vezměte na sebe plnou Boží zbroj abyste se mohli v den zlý postavit na odpor, všechno překonat a obstát.
#6:14 Stůjte tedy ‚opásáni kolem beder pravdou, obrněni pancířem spravedlnosti,
#6:15 obuti k pohotové službě evangeliu pokoje‘
#6:16 a vždycky se štítem víry, jímž byste uhasili všechny ohnivé střely toho Zlého.
#6:17 Přijměte také ‚přílbu spasení‘ a ‚meč Ducha, jímž je slovo Boží‘.
#6:18 V každý čas se v Duchu svatém modlete a proste, bděte na modlitbách a vytrvale se přimlouvejte za všechny bratry i za mne,
#6:19 aby mi bylo dáno pravé slovo, kdykoliv promluvím. Tak budu moci směle oznamovat tajemství evangelia,
#6:20 jehož jsem vyslancem i v okovech, a svobodně je zvěstovat, jak je mi uloženo.
#6:21 Chci, abyste i vy věděli, co je se mnou a co dělám. Všechno vám to vypoví Tychikus, milovaný bratr a věrný pomocník v díle našeho Pána.
#6:22 Poslal jsem ho k vám, abyste se dověděli, co je s námi, a aby povzbudil vaše srdce.
#6:23 Pokoj bratřím i láska a víra od Boha Otce a Pána Ježíše Krista.
#6:24 Milost všem, kdo nepomíjející láskou milují našeho Pána Ježíše Krista.  

\book{Philippians}{Phil}
#1:1 Pavel a Timoteus, služebníci Krista Ježíše, všem bratřím v Kristu Ježíši, kteří jsou ve Filipech, i biskupům a jáhnům:
#1:2 Milost vám a pokoj od Boha Otce našeho a Pána Ježíše Krista.
#1:3 Děkuji Bohu svému při každé vzpomínce na vás
#1:4 a v každé modlitbě za vás všechny s radostí prosím;
#1:5 jsem vděčen za vaši účast na díle evangelia od prvního dne až doposud
#1:6 a jsem si jist, že ten, který ve vás začal dobré dílo, dovede je až do dne Ježíše Krista.
#1:7 Vždyť právem tak smýšlím o vás všech, protože vás všechny mám v srdci jako spoluúčastníky milosti, i když jsem ve vězení, zodpovídám se před soudem a obhajuji evangelium.
#1:8 Bůh je mi svědkem, jak po vás všech vroucně toužím v Kristu Ježíši.
#1:9 A za to se modlím, aby se vaše láska ještě víc a více rozhojňovala a s ní i poznání a hluboká vnímavost;
#1:10 abyste rozpoznali, na čem záleží, a byli ryzí a bezúhonní pro den Kristův,
#1:11 plní ovoce spravedlnosti, které z moci Ježíše Krista roste k slávě a chvále Boží.
#1:12 Rád bych, bratří, abyste věděli, že to, co mě potkalo, je spíše k prospěchu evangelia,
#1:13 takže po celém soudu i všude jinde je známo, že jsem vězněn pro Krista,
#1:14 a mnohé bratry právě mé okovy povzbudily, aby se spolehli na Pána a s větší smělostí mluvili beze strachu slovo Boží.
#1:15 Někteří sice káží Krista také ze závisti a z řevnivosti, jiní však s dobrým úmyslem.
#1:16 Jedni z lásky, protože vědí, že jsem tu k obhajobě evangelia,
#1:17 druzí z touhy po uplatnění, ne z čistých pohnutek, a domnívají se, že mi v mém vězení způsobí bolest.
#1:18 Ale co na tom! Jen když se jakýmkoli způsobem, ať s postranními úmysly, ať upřímně zvěstuje Kristus; z toho se raduji a budu radovat.
#1:19 Neboť vím, že se mi vše obrátí k dobrému vaší modlitbou a přispěním Ducha Ježíše Krista.
#1:20 Toužebně očekávám a doufám, že v ničem nebudu zahanben, ale veřejně a směle jako vždycky i nyní na mně bude oslaven Kristus, ať životem, ať smrtí.
#1:21 Život, to je pro mne Kristus, a smrt je pro mne zisk.
#1:22 Mám-li žít v tomto těle, získám tím možnost další práce. Nevím tedy, co bych vyvolil,
#1:23 táhne mne to na obě strany: Toužím odejít a být s Kristem, což je jistě mnohem lepší;
#1:24 ale zůstat v tomto těle je zase potřebnější pro vás.
#1:25 Proto pevně spoléhám, že zůstanu a budu se všemi vámi k vašemu prospěchu a k radosti vaší víry,
#1:26 abyste se mohli mnou ještě více chlubit v Kristu Ježíši, když k vám opět přijdu.
#1:27 Jenom veďte život hodný Kristova evangelia, abych viděl, až přijdu, nebo nepřijdu-li, abych slyšel, že zakotveni v jednom Duchu vedete jednou myslí zápas ve víře v evangelium
#1:28 a v ničem se nestrachujete protivníků. Jim je to předzvěst zahynutí, vám však spasení, a to od Boha.
#1:29 Neboť vám je z milosti dáno netoliko v Krista věřit, ale pro něho i trpět;
#1:30 vedete týž zápas, jaký jste u mne viděli a o jakém nyní slyšíte. 
#2:1 Je-li možno povzbudit v Kristu, je-li možno posílit láskou, je-li jaké společenství Ducha, je-li jaký soucit a slitování:
#2:2 dovršte mou radost a buďte stejné mysli, mějte stejnou lásku, buďte jedné duše, jednoho smýšlení,
#2:3 v ničem se nedejte ovládat ctižádostí ani ješitností, nýbrž v pokoře pokládejte jeden druhého za přednějšího než sebe;
#2:4 každý ať má na mysli to, co slouží druhým, ne jen jemu.
#2:5 Nechť je mezi vámi takové smýšlení, jako v Kristu Ježíši:
#2:6 Způsobem bytí byl roven Bohu, a přece na své rovnosti nelpěl,
#2:7 nýbrž sám sebe zmařil, vzal na sebe způsob služebníka, stal se jedním z lidí. A v podobě člověka
#2:8 se ponížil, v poslušnosti podstoupil i smrt, a to smrt na kříži.
#2:9 Proto ho Bůh vyvýšil nade vše a dal mu jméno nad každé jméno,
#2:10 aby se před jménem Ježíšovým sklonilo každé koleno - na nebi, na zemi i pod zemí -
#2:11 a k slávě Boha Otce každý jazyk aby vyznával: Ježíš Kristus jest Pán.
#2:12 A tak, moji milí, jako jste vždycky byli poslušní - nikoli jen v mé přítomnosti, ale nyní mnohem více v mé nepřítomnosti - s bázní a chvěním uvádějte ve skutek své spasení.
#2:13 Neboť je to Bůh, který ve vás působí, že chcete i činíte, co se mu líbí.
#2:14 Všechno dělejte bez reptání a bez pochybování,
#2:15 abyste byli bezúhonní a ryzí, Boží děti bez poskvrny uprostřed pokolení pokřiveného a zvráceného. V něm sviťte jako hvězdy, které osvěcují svět,
#2:16 držte se slova života, abych se vámi mohl pochlubit v den Kristův, že jsem nadarmo neběžel ani se nadarmo nenamáhal.
#2:17 Ale i kdybych měl skropit krví oběť a službu, kterou Bohu přináším, totiž vaši víru, raduji a spoluraduji se s vámi se všemi;
#2:18 stejně tak se i vy radujte a spoluradujte se mnou.
#2:19 Mám naději v Pánu Ježíši, že vám brzo pošlu Timotea, abych i já byl dobré mysli, když se dovím, co je s vámi.
#2:20 Vždyť nemám nikoho, jako je on, kdo by se tak upřímně o vás staral;
#2:21 všichni si hledí jen svého, a ne toho, co je Krista Ježíše.
#2:22 O Timoteovi však dobře víte, jak se osvědčil, když se mnou jako syn s otcem sloužil evangeliu.
#2:23 Jeho tedy doufám pošlu, jakmile uvidím, co bude se mnou.
#2:24 Spoléhám pak na to v Pánu, že i sám brzo přijdu.
#2:25 Za nutné jsem však uznal poslat k vám Epafrodita, svého bratra, spolupracovníka i spolubojovníka, kterého jste vyslali, aby mi posloužil v tom, co jsem potřeboval.
#2:26 Stýskalo se mu totiž po všech vás a byl znepokojen, že jste se doslechli o jeho nemoci.
#2:27 A opravdu byl nemocen na smrt; ale Bůh se nad ním smiloval, a nejen nad ním, ale i nade mnou, abych neměl zármutek na zármutek.
#2:28 Proto jsem ho poslal co nejdříve k vám, abyste měli radost, že ho zase vidíte, a tak aby mi ubylo starostí.
#2:29 Přijměte ho tedy v Pánu se vší radostí a takových bratří si važte;
#2:30 neboť pro dílo Kristovo se přiblížil až k smrti a nasadil život, aby doplnil to, v čem vy jste mi posloužit nemohli. 
#3:1 A tak, bratří moji, radujte se v Pánu. Psáti vám stále totéž mně není zatěžko a vám to bude oporou.
#3:2 Dejte si pozor na ty psy, dejte si pozor na ty špatné dělníky, dejte si pozor na tu ‚rozřízku‘!
#3:3 Neboť pravá obřízka jsme my, kteří duchem sloužíme Bohu, chlubíme se Kristem Ježíšem a nedáme na vnější věci -
#3:4 ačkoli já bych měl proč na ně spoléhat. Zdá-li se někomu jinému, že může spoléhat na vnější věci, já tím víc:
#3:5 obřezán osmého dne, z rodu izraelského, z pokolení Benjamínova, Hebrej z Hebrejů; jde-li o zákon - farizeus;
#3:6 jde-li o horlivost - pronásledovatel církve; jde-li o spravedlnost podle zákona, byl jsem bez úhony.
#3:7 Ale cokoliv mi bylo ziskem, to jsem pro Krista odepsal jako ztrátu.
#3:8 A vůbec všecko pokládám za ztrátu, neboť to, že jsem poznal Ježíše, svého Pána, je mi nade všecko. Pro něho jsem všecko ostatní odepsal a pokládám to za nic,
#3:9 abych získal Krista a nalezen byl v něm nikoli s vlastní spravedlností, která je ze zákona, ale s tou, která je z víry v Krista - spravedlností z Boha založenou na víře,
#3:10 abych poznal jej a moc jeho vzkříšení i účast na jeho utrpení. Beru na sebe podobu jeho smrti,
#3:11 abych tak dosáhl zmrtvýchvstání.
#3:12 Nemyslím, že bych již byl u cíle anebo již dosáhl dokonalosti; běžím však, abych se jí zmocnil, protože mne se zmocnil Kristus Ježíš.
#3:13 Bratří, já nemám za to, že jsem již u cíle; jen to mohu říci: zapomínaje na to, co je za mnou, upřen k tomu, co je přede mnou,
#3:14 běžím k cíli, abych získal nebeskou cenu, jíž je Boží povolání v Kristu Ježíši.
#3:15 Kdo je dokonalý, ať smýšlí jako my; a jestliže v něčem smýšlíte jinak, i to vám Bůh objasní.
#3:16 Jen k čemu jsme již dospěli, toho se držme.
#3:17 Bratří, napodobujte mne. Hleďte na ty, kdo žijí podle našeho příkladu.
#3:18 Neboť mnozí, o nichž jsem vám často říkal a nyní to s pláčem opakuji, žijí jako nepřátelé Kristova kříže;
#3:19 jejich koncem je zahynutí, jejich bohem břicho a jejich chloubou to, zač by se měli stydět, neboť smýšlejí přízemně.
#3:20 My však máme občanství v nebesích, odkud očekáváme i Spasitele, Pána Ježíše Krista.
#3:21 On promění tělo naší poníženosti v podobu těla své slávy silou, kterou je mocen všecko si podmanit. 
#4:1 Moji bratří, které miluji a po nichž toužím, jste mou radostí a slávou; proto stůjte pevně v Pánu, milovaní.
#4:2 Euodii domlouvám i Syntyché domlouvám, aby byly zajedno v Pánu.
#4:3 Ano i tebe prosím, můj věrný druhu, ujmi se jich; vždyť vedly zápas za evangelium spolu se mnou i s Klementem a ostatními spolupracovníky, jejichž jména jsou v knize života.
#4:4 Radujte se v Pánu vždycky, znovu říkám, radujte se!
#4:5 Vaše mírnost ať je známa všem lidem. Pán je blízko.
#4:6 Netrapte se žádnou starostí, ale v každé modlitbě a prosbě děkujte a předkládejte své žádosti Bohu.
#4:7 A pokoj Boží, převyšující každé pomyšlení, bude střežit vaše srdce i mysl v Kristu Ježíši.
#4:8 Konečně, bratří, přemýšlejte o všem, co je pravdivé, čestné, spravedlivé, čisté, cokoli je hodné lásky, co má dobrou pověst, co se považuje za ctnost a co sklízí pochvalu.
#4:9 Čemu jste se u mne naučili, co jste přijali a uslyšeli i spatřili, to čiňte. A Bůh pokoje bude s vámi.
#4:10 Velmi jsem se v Pánu zaradoval, že již zase rozkvetla vaše péče o mne. Vím, vždycky jste na to mysleli, jen jste neměli příležitost.
#4:11 Ne že bych si naříkal na nedostatek; naučil jsem se být spokojen s tím, co mám.
#4:12 Dovedu trpět nouzi, dovedu mít hojnost. Ve všem a do všeho jsem zasvěcen: být syt i hladov, mít nadbytek i nedostatek.
#4:13 Všecko mohu v Kristu, který mi dává sílu.
#4:14 Avšak učinili jste dobře, když jste mi pomohli v mých těžkostech.
#4:15 I vy to víte, Filipští, že v počátcích evangelia, když jsem vyšel z Makedonie, ani jedna církev se nepodílela se mnou v příjmech a vydáních, jen vy sami;
#4:16 vždyť i do Tesaloniky víc než jednou jste mi poslali na mé potřeby.
#4:17 Ne že by mi šlo o dary; jde mi o to, aby rostl zisk, který se připisuje k vašemu dobru.
#4:18 Dostal jsem tedy všecko a mám hojnost; jsem plně opatřen, když jsem přijal od Epafrodita dary, které jste mi poslali - jsou vůní příjemnou, obětí vhodnou, Bohu milou.
#4:19 Můj Bůh vám dá všechno, co potřebujete, podle svého bohatství v slávě v Kristu Ježíši.
#4:20 Našemu Bohu a Otci sláva na věky věků. Amen.
#4:21 Pozdravujte všechny, kteří patří Kristu Ježíši. Pozdravují vás bratří, kteří jsou se mnou.
#4:22 Pozdravují vás i ostatní bratří, zvláště ti, kteří jsou z císařského domu.
#4:23 Milost Pána Ježíše Krista buď s vámi.  

\book{Colossians}{Col}
#1:1 Pavel, z Boží vůle apoštol Krista Ježíše, a bratr Timoteus
#1:2 Božímu lidu v Kolosách, věrným bratřím v Kristu: Milost vám a pokoj od Boha Otce našeho.
#1:3 Stále za vás v modlitbách děkujeme Bohu, Otci našeho Pána Ježíše Krista,
#1:4 neboť jsme slyšeli o vaší víře v Krista Ježíše a o vaší lásce, kterou máte ke všem bratřím
#1:5 pro naději zakotvenou v nebesích. Víte o ní, protože i k vám přišlo slovo pravdy, evangelium;
#1:6 tak jako na celém světě, i mezi vámi přináší ovoce a roste od toho dne, kdy jste uslyšeli o Boží milosti a přesvědčili se, že je pravdivá.
#1:7 Tak vás tomu učil Epafras, náš milovaný druh, jenž nás věrně zastupuje jako Kristův služebník.
#1:8 On nám také vyprávěl o lásce, kterou ve vás působí Boží Duch.
#1:9 Proto i my, ode dne, kdy jsme to uslyšeli, nepřestáváme za vás v modlitbách prosit, abyste plně, se vší moudrostí a duchovním pochopením poznali jeho vůli.
#1:10 Tak budete svým životem dělat Pánu čest a stále se mu líbit, ve všem ponesete ovoce dobrých skutků, budete růst v poznání Boha,
#1:11 a z moci jeho božské slávy nabudete síly k trpělivosti a radostné vytrvalosti;
#1:12 a budete děkovat Otci, který vás připravil k účasti na dědictví svatých ve světle.
#1:13 On nás vysvobodil z moci tmy a přenesl do království svého milovaného Syna.
#1:14 V něm máme vykoupení a odpuštění hříchů:
#1:15 On je obraz Boha neviditelného, prvorozený všeho stvoření,
#1:16 neboť v něm bylo stvořeno všechno na nebi i na zemi - svět viditelný i neviditelný; jak nebeské trůny, tak i panstva, vlády a mocnosti - a všechno je stvořeno skrze něho a pro něho.
#1:17 On předchází všechno, všechno v něm spočívá,
#1:18 on jest hlavou těla - totiž církve. On je počátek, prvorozený z mrtvých - takže je to on, jenž má prvenství ve všem.
#1:19 Plnost sama se rozhodla v něm přebývat,
#1:20 aby skrze něho a v něm bylo smířeno všechno, co jest, jak na zemi, tak v nebesích - protože smíření přinesla jeho oběť na kříži.
#1:21 I vás, kteří jste dříve byli odcizeni a nepřátelští Bohu svým smýšlením i zlými skutky,
#1:22 nyní s ním smířil, když ve svém pozemském těle podstoupil smrt, aby vás před Boží tvář přivedl svaté, neposkvrněné a bez úhony -
#1:23 pokud ovšem pevně zakotveni setrváte ve víře a nedáte se odtrhnout od naděje evangelia, jež jste slyšeli, jež bylo kázáno všemu stvoření pod nebem a jehož jsem se já, Pavel, stal služebníkem.
#1:24 Proto se raduji, že nyní trpím za vás a to, co zbývá do míry utrpení Kristových, doplňuji svým utrpením za jeho tělo, to jest církev.
#1:25 Stal jsem se jejím služebníkem, jak mi to uložil Bůh podle svého záměru, aby se na vás naplnilo Boží slovo,
#1:26 tajemství, které po věky a po celá pokolení bylo skryto, ale nyní je zjeveno jeho svatému lidu.
#1:27 Bůh jim chtěl dát poznat, jak bohatá je sláva jeho tajemství mezi pohany: Je to Kristus mezi vámi, v něm máte naději na Boží slávu.
#1:28 Jeho zvěstujeme, když se vší moudrostí napomínáme a učíme všechny lidi, abychom je mohli přivést před Boha jako dokonalé v Kristu.
#1:29 O to se snažím a zápasím tak, jak on ve mně působí svou silou. 
#2:1 Rád bych, abyste věděli, jak těžký zápas tu podstupuji pro vás, pro Laodikejské i pro ty, kteří mě ani osobně neznají.
#2:2 Chci, abyste povzbuzeni v srdci a spojeni láskou hluboce pochopili a plně poznali Boží tajemství, jímž je Kristus;
#2:3 v něm jsou skryty všechny poklady moudrosti a poznání.
#2:4 Říkám to proto, aby vás nikdo neoklamal líbivými řečmi.
#2:5 I když nejsem mezi vámi přítomen, přece jsem duchem s vámi a raduji se, když vidím vaši kázeň a pevnost vaší víry v Krista.
#2:6 Žijte v Kristu Ježíši, když jste ho přijali jako Pána.
#2:7 V něm zapusťte kořeny, na něm postavte základy, pevně se držte víry, jak jste v ní byli vyučeni, znovu a znovu vzdávejte díky.
#2:8 Dejte si pozor, aby vás někdo nesvedl prázdným a klamným filosofováním, založeným na lidských bájích, na vesmírných mocnostech, a ne na Kristu.
#2:9 V něm je přece vtělena všechna plnost božství;
#2:10 v něm jste i vy dosáhli plnosti. On je hlavou všech mocností a sil.
#2:11 V něm jste obřezáni obřízkou, která není udělána lidskou rukou; obřízka Kristova je odložením celého nevykoupeného těla.
#2:12 S Kristem jste byli ve křtu pohřbeni a spolu s ním také vzkříšeni vírou v Boha, jenž ho svou mocí vzkřísil z mrtvých.
#2:13 Když jste ještě byli mrtvi ve svých vinách a duchovně neobřezáni, probudil nás k životu spolu s ním a všechny viny nám odpustil.
#2:14 Vymazal dlužní úpis, jehož ustanovení svědčila proti nám, a zcela jej zrušil tím, že jej přibil na kříž.
#2:15 Tak odzbrojil a veřejně odhalil každou mocnost i sílu a slavil nad nimi vítězství.
#2:16 Nikdo tedy nemá právo odsuzovat vás za to, co jíte nebo pijete, nebo kvůli svátkům, kvůli novoluní nebo sobotám.
#2:17 To všechno je jen stín budoucích věcí, ale skutečnost je Kristus.
#2:18 Ať vám neupírá podíl na vykoupení nikdo, kdo si libuje v poníženosti a uctívání andělů, jak to v marné pýše své mysli viděl při svém zasvěcování.
#2:19 Takový člověk se nedrží hlavy, Krista, z níž celé tělo, pevně spojené klouby a šlachami, roste Božím růstem.
#2:20 Jestliže jste spolu s Kristem zemřeli mocnostem světa, proč si necháváte předpisovat:
#2:21 neber to do rukou, nejez, nedotýkej se - jako byste stále ještě byli v moci světa?
#2:22 Jsou to lidské předpisy a nauky o věcech, které se použitím ničí.
#2:23 Vydávají se za moudrost jako zvláštní projev zbožnosti, sebeponižování nebo tělesné umrtvování, ale nic neznamenají pro ovládání vášní. 
#3:1 Protože jste byli vzkříšeni s Kristem, hledejte to, co je nad vámi, kde Kristus sedí na pravici Boží.
#3:2 K tomu směřujte, a ne k pozemským věcem.
#3:3 Zemřeli jste a váš život je skryt spolu s Kristem v Bohu.
#3:4 Ale až se ukáže Kristus, váš život, tehdy i vy se s ním ukážete v slávě.
#3:5 Proto umrtvujte své pozemské sklony: smilstvo, necudnost, vášeň, zlou touhu a hrabivost, která je modloslužbou.
#3:6 Pro takové věci přichází Boží hněv.
#3:7 I vy jste dříve tak žili.
#3:8 Ale nyní odhoďte to všecko: zlobu, hněv, špatnost, rouhání, pomluvy z vašich úst.
#3:9 Neobelhávejte jeden druhého, svlecte se sebe starého člověka i s jeho skutky
#3:10 a oblecte nového, který dochází pravého poznání, když se obnovuje podle obrazu svého Stvořitele.
#3:11 Potom už není Řek a Žid, obřezaný a neobřezaný, barbar, divoch, otrok a svobodný - ale všechno a ve všech Kristus.
#3:12 Jako vyvolení Boží, svatí a milovaní, oblecte milosrdný soucit, dobrotu, skromnost, pokoru a trpělivost.
#3:13 Snášejte se navzájem a odpouštějte si, má-li kdo něco proti druhému. Jako Pán odpustil vám, odpouštějte i vy.
#3:14 Především však mějte lásku, která všechno spojuje k dokonalosti.
#3:15 A ve vašem srdci ať vládne mír Kristův, k němuž jste byli povoláni v jedno společné tělo. A buďte vděčni.
#3:16 Nechť ve vás přebývá slovo Kristovo v celém svém bohatství: se vší moudrostí se navzájem učte a napomínejte a s vděčností v srdci oslavujte Boha žalmy, chválami a zpěvem, jak vám dává Duch.
#3:17 Všechno, cokoli mluvíte nebo děláte, čiňte ve jménu Pána Ježíše a skrze něho děkujte Bohu Otci.
#3:18 Ženy, podřizujte se svým mužům, jak se sluší na ty, kdo patří Pánu.
#3:19 Muži, milujte své ženy a nechovejte se k nim drsně.
#3:20 Děti, poslouchejte ve všem své rodiče, protože se to líbí Pánu.
#3:21 Otcové, neponižujte své děti, aby nemalomyslněly.
#3:22 Otroci, poslouchejte ve všem své pozemské pány, nejen naoko, abyste se jim po lidsku zalíbili, nýbrž ze srdce, v bázni Páně.
#3:23 Cokoli děláte, dělejte upřímně, jako by to nebylo lidem, ale Pánu,
#3:24 s vědomím, že jako odměnu dostanete podíl na jeho království. Váš Pán je Kristus, jemu sloužíte.
#3:25 Kdo se dopouští křivdy, dostane za to odplatu. Náš Pán nikomu nestraní. 
#4:1 A vy, páni, dávejte otrokům, co jim spravedlivě patří. Pamatujte, že i vy máte Pána v nebi.
#4:2 V modlitbách buďte vytrvalí, bděte a děkujte Bohu.
#4:3 Modlete se také za nás, aby Bůh otevřel dveře našemu slovu, abych mohl zvěstovat tajemství Kristovo, pro něž jsem teď ve vězení,
#4:4 a tak je mohl rozhlásit; neboť k tomu jsem poslán.
#4:5 Jednejte moudře ve styku s okolním světem a využijte čas vám svěřený.
#4:6 Vaše slovo ať je vždy laskavé a určité; ať víte, jak ke komu promluvit.
#4:7 O tom, co je se mnou, vám všechno poví Tychikus, můj milovaný bratr a věrný pomocník, který se mnou slouží Pánu.
#4:8 Posílám ho k vám právě proto, abyste se dověděli, jak se nám vede, a aby povzbudil vaše srdce.
#4:9 Posílám s ním i Onezima, věrného a milovaného bratra, vašeho krajana. Oni vám povědí všechno, co se tu děje.
#4:10 Pozdravuje vás můj spoluvězeň Aristarchos a Barnabášův bratranec Marek, o němž jste již dostali pokyny. Až k vám přijde, přijměte ho.
#4:11 Také vás zdraví Ježíš zvaný Justus. Jsou to jediní židé, kteří s námi pracují pro království Boží a jsou také mou útěchou.
#4:12 Pozdravuje vás Epafras, váš krajan, služebník Krista Ježíše, který o vás stále zápasí modlitbami, abyste stáli pevně a věrně plnili Boží vůli.
#4:13 Mohu dosvědčit, že pro vás i pro Laodikejské a Hierapolské vynakládá mnoho námahy.
#4:14 Pozdravuje vás milovaný lékař Lukáš a Démas.
#4:15 Pozdravujte bratry v Laodikeji i Nymfu a církev, která se shromažďuje v jejím domě.
#4:16 Až tento list u vás přečtete, zařiďte, aby byl čten také v laodikejské církvi a abyste vy četli list Laodikejským.
#4:17 A Archippovi řekněte: Hleď, abys konal službu, kterou jsi přijal od Pána.
#4:18 Pozdrav mou, Pavlovou rukou. Pamatujte na to, že jsem v poutech. Milost s vámi.  

\book{I Thessalonians}{1Thess}
#1:1 Pavel, Silvanus a Timoteus tesalonické církvi v Bohu Otci a v Pánu Ježíši Kristu: Milost vám a pokoj.
#1:2 Stále vzdáváme díky Bohu za vás za všecky a ustavičně na vás pamatujeme ve svých modlitbách;
#1:3 před Bohem a Otcem naším si připomínáme vaši činnou víru, usilovnou lásku a vytrvalou naději v našeho Pána Ježíše Krista.
#1:4 Víme přece, bratří Bohem milovaní, že patříte k vyvoleným,
#1:5 neboť naše evangelium k vám nepřišlo pouze v slovech, ale v moci Ducha svatého a v přesvědčivé plnosti. Víte, jak jsme si kvůli vám počínali, když jsme byli u vás.
#1:6 A vy jste jednali jako my i Pán, když jste uprostřed mnohé tísně přijali slovo víry v radosti Ducha svatého.
#1:7 Tak jste se stali příkladem všem věřícím v Makedonii a v Acháji.
#1:8 Od vás pak se slovo Páně rozeznělo nejen po Makedonii a Acháji, ale o vaší víře v Boha se ví všude, takže není třeba, abychom o tom vůbec mluvili.
#1:9 Lidé sami vypravují, jak jste nás přijali a jak jste se obrátili od model k Bohu, abyste sloužili Bohu živému a skutečnému
#1:10 a očekávali z nebe jeho Syna, kterého vzkřísil z mrtvých, Ježíše, jenž nás vysvobozuje od přicházejícího hněvu. 
#2:1 Sami víte, bratří, že náš příchod k vám nebyl marný.
#2:2 Víte také, jak jsme předtím ve Filipech trpěli a byli pohaněni; a přece nám náš Bůh dal odvahu hlásat vám, přes mnohý těžký zápas, evangelium Boží.
#2:3 Naše poselství nepochází z omylu ani z nekalých úmyslů, ani vás nechceme podvést.
#2:4 Bůh nás uznal za hodné svěřit nám evangelium, a proto mluvíme tak, abychom se líbili ne lidem, ale Bohu, který zkoumá naše srdce.
#2:5 Nikdy, jak víte, jsme nesáhli k lichocení, ani jsme pod nějakou záminkou nebyli chtiví majetku - Bůh je svědek!
#2:6 Také jsme nehledali slávu u lidí, ani u vás, ani u jiných;
#2:7 ač jsme mohli jako Kristovi poslové dát najevo svou důležitost, byli jsme mezi vámi laskaví, jako když matka chová své děti.
#2:8 Tolik jsme po vás toužili, že jsme vám chtěli odevzdat nejen evangelium Boží, ale i svůj život. Tak jste se nám stali drahými!
#2:9 Jistě si, bratří, vzpomínáte na naše úsilí a námahu, jak jsme ve dne v noci pracovali, abychom nikomu z vás nebyli na obtíž, když jsme vám přinesli Boží evangelium.
#2:10 Vy i Bůh jste svědky, jak jsme se k vám věřícím zbožně, spravedlivě a bezúhonně chovali.
#2:11 Víte přece, že jsme každého z vás jako otec své děti
#2:12 napomínali, povzbuzovali a zapřísahali, abyste vedli život důstojný Boha, který vás povolal do slávy svého království.
#2:13 Proto i my děkujeme Bohu neustále, že jste od nás přijali slovo Boží zvěsti ne jako slovo lidské, ale jako slovo Boží, jímž skutečně jest. Vždyť také projevuje svou sílu ve vás, kteří věříte.
#2:14 Nesete podobný úděl jako církve Boží v Kristu Ježíši, které jsou v Judsku. Vytrpěli jste stejné věci od svých vlastních krajanů jako církve v Judsku od židů.
#2:15 Ti zabili i Pána Ježíše a proroky a také nás pronásledovali; nelíbí se Bohu a jsou v nepřátelství se všemi lidmi,
#2:16 když nám brání kázat pohanům cestu spásy. Tak jen dovršují míru svých hříchů. Už se však na nich ukazuje konečný hněv Boží.
#2:17 Bez vás, bratří, byli jsme jako sirotci, i když to bylo nakrátko a jen tělem, ne srdcem; tím usilovněji jsme vás toužili spatřit.
#2:18 Proto jsme vám chtěli přijít, já Pavel víc než jednou, ale satan nám v tom vždy zabránil.
#2:19 Vždyť kdo je naše naděje, radost a vavřín chlouby před naším Pánem Ježíšem Kristem při jeho příchodu, ne-li právě vy?
#2:20 Ano, vy jste naše sláva a radost. 
#3:1 Když jsme to již nemohli déle vydržet, rozhodli jsme se zůstat v Athénách sami
#3:2 a poslali jsme Timotea, našeho bratra a Božího spolupracovníka v evangeliu Kristově, aby vás utvrdil a povzbudil ve vaší víře,
#3:3 aby se nikdo nedal zviklat v těchto souženích; neboť sami víte, že to je náš úděl.
#3:4 Když jsme byli u vás, předpovídali jsme vám, že na nás přijdou soužení; a to se také, jak vidíte, stalo.
#3:5 A tak když jsem to již déle nemohl vydržet, poslal jsem Timotea, abych poznal vaši víru, zdali vás snad pokušitel nesvedl, takže by naše námaha vyšla naprázdno.
#3:6 Nyní však přišel Timoteus od vás a přinesl radostnou zprávu o vaší víře a lásce i o tom, jak na nás stále v dobrém vzpomínáte a toužíte nás spatřit stejně jako my vás.
#3:7 Tak jste nás, bratří, ve vší naší tísni a soužení potěšili svou vírou.
#3:8 Nyní jsme zase živi, když vy pevně stojíte v Pánu.
#3:9 Jakou chválou se můžeme Bohu odvděčit za všecku tu radost, kterou z vás máme před jeho tváří?
#3:10 Ve dne v noci vroucně prosíme, abychom vás mohli spatřit a doplnit, čeho se nedostává vaší víře.
#3:11 Sám náš Bůh a Otec i Ježíš, náš Pán, kéž připraví naši cestu k vám!
#3:12 Nechť Pán dá bohatě růst vaší vzájemné lásce i lásce ke všem, tak jako i my vás milujeme,
#3:13 ať posílí vaše srdce, abyste byli bezúhonní a svatí před Bohem, naším Otcem, až přijde náš Pán Ježíš se všemi svými svatými. 
#4:1 Konečně vás, bratří, prosíme a napomínáme v Pánu Ježíši, abyste vždy více prospívali v tom, co jste od nás přijali a co už také činíte: žijte tak, abyste se líbili Bohu.
#4:2 Vždyť víte, které příkazy jsme vám dali od Pána Ježíše.
#4:3 Neboť toto je vůle Boží, vaše posvěcení, abyste se zdržovali necudnosti
#4:4 a každý z vás aby uměl žít se svou vlastní ženou svatě a s úctou,
#4:5 ne ve vášnivé chtivosti jako pohané, kteří neznají Boha.
#4:6 Ať nikdo v této věci nevybočuje z mezí a neklame svého bratra, protože Pán ztrestá takové jednání, jak jsme vám už dříve řekli a dosvědčili.
#4:7 Vždyť Bůh nás nepovolal k nečistotě, nýbrž k posvěcení.
#4:8 Kdo tím pohrdá, nepohrdá člověkem, nýbrž Bohem, jenž vám dává svého svatého Ducha.
#4:9 O bratrské lásce není třeba, abych vám psal, neboť Bůh sám vás vyučil, jak se máte mít mezi sebou rádi.
#4:10 A takoví opravdu jste ke všem bratřím v celé Makedonii; jen vás prosíme, bratří, abyste v tom byli stále horlivější.
#4:11 Zakládejte si na tom, že budete žít pokojně, věnovat se své práci a získávat obživu vlastníma rukama, jak jsme vám již dříve uložili.
#4:12 Tak získáte úctu těch, kdo stojí mimo, a na nikoho nebudete odkázáni.
#4:13 Nechceme vás, bratří, nechat v nevědomosti o údělu těch, kdo zesnuli, abyste se nermoutili jako ti, kteří nemají naději.
#4:14 Věříme-li, že Ježíš zemřel a vstal z mrtvých, pak také víme, že Bůh ty, kdo zemřeli ve víře v Ježíše, přivede spolu s ním k životu.
#4:15 Toto vám říkáme podle slova Páně: My živí, kteří se dočkáme příchodu Páně, zesnulé nepředejdeme.
#4:16 Zazní povel, hlas archanděla a zvuk Boží polnice, sám Pán sestoupí z nebe, a ti, kdo zemřeli v Kristu, vstanou nejdříve;
#4:17 potom my živí, kteří se toho dočkáme, budeme spolu s nimi uchváceni v oblacích vzhůru vstříc Pánu. A pak už navždy budeme s Pánem.
#4:18 Těmito slovy se vzájemně potěšujte. 
#5:1 Není nutné, bratří, psát vám něco o době a hodině.
#5:2 Sami přece dobře víte, že den Páně přijde jako přichází zloděj v noci.
#5:3 Až budou říkat ‚je pokoj, nic nehrozí‘, tu je náhle přepadne zhouba jako bolest rodičku, a neuniknou.
#5:4 Vy však, bratří, nejste ve tmě, aby vás ten den mohl překvapit jako zloděj.
#5:5 Vy všichni jste synové světla a synové dne. Nepatříme noci ani temnotě.
#5:6 Nespěme tedy jako ostatní, nýbrž bděme a buďme střízliví.
#5:7 Ti, kdo spí, spí v noci, a kdo se opíjejí, opíjejí se v noci.
#5:8 My však, kteří patříme dni, buďme střízliví, oblecme si víru a lásku jako pancíř a naději na spásu jako přílbu.
#5:9 Vždyť Bůh nás neurčil k tomu, abychom propadli jeho hněvu, nýbrž abychom došli spásy skrze našeho Pána Ježíše Krista.
#5:10 On zemřel za nás, abychom my, ať živí či zemřelí, žili spolu s ním.
#5:11 Proto se navzájem povzbuzujte a buďte jeden druhému oporou, jak to již činíte.
#5:12 Žádáme vás, bratří, abyste uznávali ty, kteří mezi vámi pracují, jsou vašimi představenými v Kristu a napomínají vás.
#5:13 Velmi si jich važte a milujte je pro jejich dílo. Žijte mezi sebou v pokoji.
#5:14 Klademe vám na srdce, bratří, kárejte neukázněné, těšte malomyslné, ujímejte se slabých, se všemi mějte trpělivost.
#5:15 Hleďte, aby nikdo neoplácel zlým za zlé, ale vždycky usilujte o dobré mezi sebou a vůči všem.
#5:16 Stále se radujte,
#5:17 v modlitbách neustávejte.
#5:18 Za všech okolností děkujte, neboť to je vůle Boží v Kristu Ježíši pro vás.
#5:19 Plamen Ducha nezhášejte,
#5:20 prorockými dary nepohrdejte.
#5:21 Všecko zkoumejte, dobrého se držte;
#5:22 zlého se chraňte v každé podobě.
#5:23 Sám Bůh pokoje nechť vás cele posvětí a zachová vašeho ducha, duši i tělo bez úrazu a poskvrny do příchodu našeho Pána Ježíše Krista.
#5:24 Věrný je ten, který vás povolal; on to také učiní.
#5:25 Bratří, modlete se i vy za nás!
#5:26 Pozdravte všecky bratry svatým políbením.
#5:27 Zavazuji vás v Pánu, abyste tento list dali přečíst všem bratřím.
#5:28 Milost našeho Pána Ježíše Krista buď s vámi.  

\book{II Thessalonians}{2Thess}
#1:1 Pavel, Silvanus a Timoteus tesalonické církvi v Bohu, našem Otci, a v Pánu Ježíši Kristu.
#1:2 Milost vám a pokoj od Boha Otce a Pána Ježíše Krista.
#1:3 Musíme za vás, bratří, stále Bohu děkovat, jak se sluší, protože vaše víra mocně roste a vzájemná láska všech vás je stále větší.
#1:4 Proto jsme na vás hrdi v církvích Božích, neboť vaše víra je vytrvalá v každém pronásledování a útisku, které snášíte:
#1:5 to je předzvěst spravedlivého soudu Božího. Tak se stanete hodnými Božího království, pro něž trpíte.
#1:6 A je spravedlivé, že Bůh všem, kteří vás utiskují, odplatí útiskem,
#1:7 a vás utiskované spolu s námi vysvobodí, až se zjeví Pán Ježíš z nebe se svými mocnými anděly,
#1:8 aby v plameni ohně vykonal trest na těch, kteří neznají Boha, a na těch, kteří odpírají poslušnost evangeliu našeho Pána Ježíše.
#1:9 Jejich trestem bude věčná záhuba ‚daleko od Pána a slávy jeho moci‘,
#1:10 až v onen den přijde, aby byl oslaven svým lidem a veleben těmi, kdo uvěřili; také vy jste uvěřili našemu svědectví.
#1:11 Proto se stále za vás modlíme, aby vás náš Bůh učinil hodnými svého povolání a svou mocí přivedl k naplnění každé vaše dobré rozhodnutí a dílo víry.
#1:12 Tak bude oslaveno jméno našeho Pána Ježíše ve vás a vy v něm podle milosti našeho Boha a Pána Ježíše Krista. 
#2:1 Pokud jde o příchod Pána Ježíše, kolem něhož budeme shromážděni, prosíme vás, bratří,
#2:2 abyste se nedali snadno vyvést z rovnováhy nebo vylekat nějakým projevem ducha nebo řečí či listem, domněle pocházejícím od nás, jako by den Páně měl už nastat.
#2:3 Žádným způsobem se nedejte od nikoho oklamat, protože nenastane, dokud nedojde ke vzpouře proti Bohu a neobjeví se člověk nepravosti, Syn zatracení.
#2:4 Ten se postaví na odpor a ‚povýší se nade všecko, co má jméno Boží‘ nebo čemu se vzdává božská pocta. Dokonce ‚usedne v chrámu Božím‘ a bude se vydávat za Boha.
#2:5 Nevzpomínáte si, že jsem vám to říkal, ještě když jsem byl u vás?
#2:6 Víte přece, co zatím brání tomu, aby se ukázal dříve, než přijde jeho čas.
#2:7 Ta nepravost již působí, ale jen skrytě, dokud nebude odstraněn z cesty ten, kdo tomu brání.
#2:8 A pak se ukáže ten zlý, kterého Pán Ježíš ‚zabije dechem svých úst‘ a zničí svým slavným příchodem.
#2:9 Ten zlý přijde v moci satanově, bude konat kdejaký mocný čin, klamná znamení a zázraky
#2:10 a všemožnou nepravostí bude svádět ty, kdo jdou k záhubě, neboť nepřijali a nemilovali pravdu, která by je zachránila.
#2:11 Proto je Bůh vydává do moci klamu, aby uvěřili lži.
#2:12 Tak budou odsouzeni všichni, kdo neuvěřili pravdě, ale nalezli zalíbení v nepravosti.
#2:13 Stále musíme děkovat Bohu za vás, bratří milovaní od Pána, že vás Bůh jako první vyvolil ke spáse a posvětil Duchem a vírou v pravdu,
#2:14 abyste měli účast na slávě našeho Pána Ježíše Krista; k tomu vás povolal naším evangeliem.
#2:15 Nuže tedy, bratří, stůjte pevně a držte se toho učení, které jsme vám odevzdali, ať už slovem nebo dopisem.
#2:16 Sám pak náš Pán Ježíš Kristus a Bůh náš Otec, který si nás zamiloval a ze své milosti nám dal věčné potěšení a dobrou naději,
#2:17 nechť povzbudí vaše srdce a dá vám sílu ke každému dobrému činu i slovu. 
#3:1 A tak, bratří, modlete se za nás, aby se slovo Páně stále šířilo a bylo oslaveno jako u vás
#3:2 a abychom byli vysvobozeni od pochybných a zlých lidí. Neboť ne všichni věří.
#3:3 Ale Pán je věrný; on vás posílí a ochrání od zlého.
#3:4 Spoléháme na vás v Pánu, že konáte a budete konat, co přikazujeme.
#3:5 A Pán nechť řídí vaše srdce k Boží lásce a k trpělivosti Kristově.
#3:6 Přikazujeme vám, bratří, ve jménu Pána Ježíše Krista, abyste se stranili každého bratra, který vede zahálčivý život a nežije podle naučení, která jste od nás převzali.
#3:7 Sami přece víte, jak máte žít podle našeho příkladu. My jsme u vás nezaháleli
#3:8 ani jsme nikoho nevyjídali, ale ve dne v noci jsme namáhavě pracovali, abychom nikomu z vás nebyli na obtíž.
#3:9 Ne že bychom k tomu neměli právo, ale chtěli jsme vám sami sebe dát za příklad, abyste se jím řídili.
#3:10 Když jsme u vás byli, přikazovali jsme vám: Kdo nechce pracovat, ať nejí!
#3:11 Teď však slyšíme, že někteří mezi vámi vedou zahálčivý život, pořádně nepracují a pletou se do věcí, do kterých jim nic není.
#3:12 Takovým přikazujeme a vybízíme je ve jménu Pána Ježíše Krista, aby žili řádně a živili se vlastní prací.
#3:13 Vy však, bratří, neochabujte v dobrém jednání.
#3:14 Neuposlechne-li někdo těchto slov, která vám píšeme, dejte mu znát, že k vám nepatří; tím bude zahanben.
#3:15 Ale nejednejte s ním jako s nepřítelem, nýbrž varujte ho jako bratra.
#3:16 Sám Pán pokoje ať vám uděluje pokoj vždycky a ve všem. Pán se všemi vámi.
#3:17 Pozdrav je psán mou, Pavlovou rukou. Tím je ověřen každý můj dopis, tak vždy píšu.
#3:18 Milost Pána našeho Ježíše Krista s vámi se všemi.  

\book{I Timothy}{1Tim}
#1:1 Pavel, apoštol Krista Ježíše z pověření Boha, našeho Spasitele, a Krista Ježíše, naší naděje,
#1:2 Timoteovi, vlastnímu synu ve víře: Milost, slitování a pokoj od Boha Otce a Krista Ježíše, našeho Pána.
#1:3 Když jsem odcházel do Makedonie, žádal jsem tě, abys dále zůstal v Efezu a nikomu nedovolil učit odchylným naukám
#1:4 a zabývat se bájemi a nekonečnými rodokmeny, které vedou spíše k jalovému hloubání, než k účasti víry na Božím záměru.
#1:5 Cílem našeho vyučování je láska z čistého srdce, z dobrého svědomí a z upřímné víry.
#1:6 Od toho se někteří odchýlili a dali se na prázdné řeči.
#1:7 Chtějí být učiteli zákona a nechápou ani svá vlastní slova ani podstatu toho, o čem s takovou jistotou mluví.
#1:8 Víme, že zákon je dobrý, když ho někdo užívá správně
#1:9 a je si vědom, že zákon není určen pro spravedlivého, nýbrž pro lidi zlé a neposlušné, bezbožné a hříšníky, pro lidi bohaprázdné a světské, pro ty, kdo vztáhnou ruku proti otci a matce, pro vrahy,
#1:10 smilníky, zvrhlíky, únosce, lháře, křivopřísežníky, a co se ještě příčí zdravému učení
#1:11 podle evangelia slávy požehnaného Boha, které mi bylo svěřeno.
#1:12 Děkuji našemu Pánu, který mi dal sílu, Kristu Ježíši, že mě uznal za spolehlivého a určil ke své službě,
#1:13 ačkoli jsem byl předtím rouhač, pronásledovatel a násilník. A přece jsem došel slitování, protože jsem ve své nevěře nevěděl, co dělám.
#1:14 A milost našeho Pána se nadmíru rozhojnila, a s ní víra i láska v Kristu Ježíši.
#1:15 Věrohodné je to slovo a zaslouží si plného souhlasu: Kristus Ježíš přišel na svět, aby zachránil hříšníky. Já k nim patřím na prvním místě,
#1:16 avšak došel jsem slitování, aby Ježíš Kristus právě na mně ukázal všechnu svou shovívavost jako příklad pro ty, kteří v něho uvěří a tak dosáhnou věčného života.
#1:17 Králi věků, nepomíjejícímu, neviditelnému, jedinému Bohu buď čest a sláva na věky věků. Amen.
#1:18 To ti kladu na srdce, synu Timoteji, ve shodě s prorockými slovy, která byla o tobě pronesena, abys jimi povzbuzen bojoval dobrý boj
#1:19 a zachoval si víru i dobré svědomí, jímž někteří lidé pohrdli a tak ztroskotali ve víře.
#1:20 Patří k nim Hymenaios a Alexandr, které jsem vydal satanu, aby se odnaučili rouhat. 
#2:1 Na prvním místě žádám, aby se konaly prosby, modlitby, přímluvy, díkůvzdání za všechny lidi,
#2:2 za vládce a za všechny, kteří mají v rukou moc, abychom mohli žít tichým a klidným životem v opravdové zbožnosti a vážnosti.
#2:3 To je dobré a vítané u našeho Spasitele Boha, který chce,
#2:4 aby všichni lidé došli spásy a poznali pravdu.
#2:5 Je totiž jeden Bůh a jeden prostředník mezi Bohem a lidmi, člověk Kristus Ježíš,
#2:6 který dal sám sebe jako výkupné za všechny, jako svědectví v určený čas.
#2:7 Byl jsem ustanoven hlasatelem a apoštolem tohoto svědectví - mluvím pravdu a nelžu - učitelem pohanů ve víře a v pravdě.
#2:8 Chci tedy, aby se muži všude ve shromáždění modlili, pozvedajíce ruce v čistotě, bez hněvu a hádek.
#2:9 Rovněž ženy ať se oblékají slušně a zdobí se prostě a střízlivě, ne účesy a zlatem, perlami nebo drahými šaty,
#2:10 nýbrž dobrými skutky, jak se sluší na ženy, které se rozhodly pro zbožný život.
#2:11 Žena ať přijímá poučení mlčky s veškerou podřízeností.
#2:12 Učit ženě nedovoluji. Žena nemá mít moc nad mužem, nýbrž má se nechat vést.
#2:13 Vždyť první byl stvořen Adam a pak Eva.
#2:14 A nebyl to také Adam, kdo byl oklamán, ale žena byla oklamána a dopustila se přestoupení.
#2:15 Spasena bude jako matka, jestliže setrvá ve víře, lásce, svatosti a střízlivosti. 
#3:1 Věrohodné je to slovo: Kdo chce být biskupem, touží po krásném úkolu.
#3:2 Nuže, biskup má být bezúhonný, jen jednou ženatý, střídmý, rozvážný, řádný, pohostinný, schopný učit,
#3:3 ne pijan, ne rváč, nýbrž vlídný, smířlivý, nezištný.
#3:4 Má dobře vést svou rodinu a mít děti poslušné a počestné;
#3:5 nedovede-li někdo vést svou rodinu, jak se bude starat o Boží církev?
#3:6 Nemá být nově pokřtěný, aby nezpyšněl a nepropadl odsouzení ďáblovu.
#3:7 Musí mít také dobrou pověst u těch, kdo jsou mimo církev, aby neupadl do pomluv a ďáblových nástrah.
#3:8 Rovněž jáhnové mají být čestní, ne dvojací v řeči, ne oddaní vínu, ne ziskuchtiví.
#3:9 Mají uchovávat tajemství víry v čistém svědomí.
#3:10 Ať jsou nejdříve vyzkoušeni, a teprve potom, když jim nelze nic vytknout, ať konají svou službu.
#3:11 Právě tak ženy v této službě mají být čestné, ne pomlouvačné, střídmé, ve všem věrné.
#3:12 Jáhni ať jsou jen jednou ženatí, ať dobře vedou své děti a celou rodinu.
#3:13 Konají-li dobře svou službu, získávají si důstojné postavení a velkou jistotu ve víře v Krista Ježíše.
#3:14 Doufám, že k tobě brzo přijdu. Ale píši ti to
#3:15 pro případ, že bych se opozdil, abys věděl, jak je třeba si počínat v Božím domě, jímž je církev živého Boha, sloup a opora pravdy.
#3:16 Vpravdě veliké je tajemství zbožnosti: Byl zjeven v těle, ospravedlněn Duchem, viděn od andělů, hlásán národům, došel víry ve světě, byl přijat do slávy. 
#4:1 Duch výslovně praví, že v posledních dobách někteří odpadnou od víry a přidrží se těch, kteří svádějí démonskými naukami,
#4:2 jsou pokrytci, lháři a mají vypálen cejch na vlastním svědomí.
#4:3 Zakazují lidem ženit se a jíst pokrmy, které Bůh stvořil, aby je s děkováním požívali ti, kdo věří a kdo poznali pravdu.
#4:4 Neboť všechno, co Bůh stvořil, je dobré a nemá se zavrhovat nic, co se přijímá s díkůvzdáním.
#4:5 Vždyť je to posvěceno Božím slovem a modlitbou.
#4:6 Toto bratřím zdůrazňuj a budeš dobrým služebníkem Krista Ježíše, vychovaným slovy víry a pravého učení, které sis osvojil.
#4:7 Bezbožné a dětinské báje odmítej. Cvič se ve zbožnosti.
#4:8 Cvičení těla je užitečné pro málo věcí, avšak zbožnost je užitečná pro všechno a má zaslíbení pro život nynější i budoucí.
#4:9 Věrohodné je to slovo a zaslouží si plného souhlasu.
#4:10 Proto se namáháme a zápasíme, že máme naději v živém Bohu, který je Spasitel všech lidí, zvláště věřících.
#4:11 To přikazuj a tomu uč.
#4:12 Nikdo ať tebou nepohrdá proto, že jsi mladý; ale těm, kdo věří, buď vzorem v řeči, v chování, v lásce, ve víře, v čistotě.
#4:13 Než přijdu, ujmi se předčítání, kázání, vyučování.
#4:14 Nezanedbávej svůj dar, který ti byl dán podle prorockého pokynu, když na tebe starší vložili ruce.
#4:15 Na to mysli, v tom žij, aby tvůj pokrok byl všem patrný.
#4:16 Dávej pozor na své jednání i na své učení. Buď v tom vytrvalý. Tak posloužíš ke spasení nejen sobě, ale i svým posluchačům. 
#5:1 Proti staršímu člověku nevystupuj tvrdě, nýbrž domlouvej mu jako otci, mladším jako bratrům,
#5:2 starším ženám jako matkám, mladším jako sestrám, vždy s čistou myslí.
#5:3 Pečuj o vdovy, které jsou skutečně opuštěné.
#5:4 Má-li však některá vdova děti nebo vnuky, ti ať se učí mít péči především o své příbuzné a odplácet svým rodičům. To je totiž milé Bohu.
#5:5 Vdova, která je opravdu osamělá, doufá v Boha a oddává se vytrvale prosbám a modlitbám ve dne i v noci.
#5:6 Ta, která myslí jen na zábavu, je mrtvá, i když žije.
#5:7 To jim zdůrazňuj; ať se jim nedá nic vytknout.
#5:8 Kdo se nestará o své blízké a zvláště o členy rodiny, zapřel víru a je horší než nevěřící.
#5:9 Mezi zapsané vdovy smí být přijata žena ne mladší než šedesát let, jen jednou vdaná,
#5:10 známá dobrými skutky: jestliže vychovala děti, prokazovala věřícím pohostinnost a umývala jim nohy, pomáhala nešťastným a osvědčila se v každém dobrém díle.
#5:11 Mladší ženy nezapisuj mezi vdovy. Neboť jakmile je smyslnost odvrátí od Krista, chtějí se opět vdávat;
#5:12 tak propadají odsouzení, protože porušily slíbenou věrnost.
#5:13 Zároveň si navykají zahálet a chodit po návštěvách. A nejen zahálet, nýbrž i klevetit, plést se do cizích věcí a mluvit, co se nepatří.
#5:14 Chci tedy, aby se mladší vdovy vdávaly, měly děti, vedly domácnost a nedávaly protivníku příležitost k pomluvám.
#5:15 Některé totiž se již daly na satanovu cestu.
#5:16 Má-li některá věřící žena vdovy v příbuzenstvu, ať jim pomáhá, aby nebyla zatěžována církev, která má pomáhat osamělým vdovám.
#5:17 Starším, kteří svou službu konají dobře, ať se dostane dvojnásobné odměny, zvláště těm, kteří nesou břemeno kázání a vyučování.
#5:18 Neboť Písmo praví: ‚Nedáš náhubek dobytčeti, když mlátí obilí‘, a jinde: ‚Dělník si zaslouží svou mzdu‘.
#5:19 Stížnost proti starším nepřijímej, leda na základě výpovědi dvou nebo tří svědků.
#5:20 Ty, kteří hřeší, kárej přede všemi, aby se báli i ostatní.
#5:21 Zapřísahám tě před Bohem a Kristem Ježíšem a před vyvolenými anděly, abys takto postupoval bez předpojatosti a nikomu nestranil.
#5:22 Neustanovuj nikoho v církvi ukvapeně, abys neměl spoluvinu na cizím hříchu. Uchovávej se čistý.
#5:23 Nenuť se pít vodu, ale kvůli svému žaludku a kvůli svým častým nemocem mírně užívej vína.
#5:24 Hříchy některých lidí jsou zjevné, ještě než dojde k soudu; u jiných vyjdou najevo až na soudu.
#5:25 Právě tak jsou zjevné dobré skutky; a pokud ještě nejsou zjevné, nezůstanou skryty. 
#6:1 Všichni, kdo nesou jho otroctví, ať mají své pány v náležité úctě, aby Boží jméno ani naše učení neupadly do špatné pověsti.
#6:2 Ti, kdo mají věřící pány, ať k nim nemají menší úctu proto, že jsou jejich bratří, nýbrž ať jsou jim poddáni o to raději, že mohou sloužit věřícím a milovaným. Tomu všemu uč a to přikazuj.
#6:3 Jestliže někdo učí něco jiného a nedrží se zdravých slov našeho Pána Ježíše Krista a učení pravé zbožnosti,
#6:4 je nadutý, ničemu nerozumí a jen si libuje ve sporech a slovních potyčkách. Z toho vzniká závist, svár, urážky, podezřívání,
#6:5 ustavičné hádky lidí, kteří mají zvrácenou mysl a jsou zbaveni pravdy, a zbožnost pokládají za prostředek k obohacení.
#6:6 Zbožnost, která se spokojí s tím, co má, je už sama velké bohatství.
#6:7 Nic jsme si totiž na svět nepřinesli, a také si nic nemůžeme odnést.
#6:8 Máme-li jídlo a oděv, spokojme se s tím.
#6:9 Kdo chce být bohatý, upadá do osidel pokušení a do mnoha nerozumných a škodlivých tužeb, které strhují lidi do zkázy a záhuby,
#6:10 Kořenem všeho toho zla je láska k penězům. Z touhy po nich někteří lidé zbloudili z cesty víry a způsobili si mnoho trápení.
#6:11 Ty však se tomu jako Boží člověk vyhýbej! Usiluj o spravedlnost, zbožnost, víru, lásku, trpělivost, mírnost.
#6:12 Bojuj dobrý boj víry, abys dosáhl věčného života, k němuž jsi byl povolán a k němuž ses přihlásil dobrým vyznáním před mnoha svědky.
#6:13 Vyzývám tě před Bohem, který dává všemu život, a Kristem Ježíšem, který vydal svědectví svým dobrým vyznáním před Pontiem Pilátem,
#6:14 abys bez poskvrny a výtky plnil své poslání až do příchodu našeho Pána Ježíše Krista.
#6:15 Jeho příchod zjeví v určený čas požehnaný a jediný Vládce, Král králů a Pán pánů.
#6:16 On jediný je nesmrtelný a přebývá v nepřístupném světle; jeho nikdo z lidí neviděl a nemůže uvidět. Jemu patří čest a věčná moc. Amen.
#6:17 Těm, kteří jsou bohatí v tomto věku, přikazuj, ať nejsou pyšní a nedoufají v nejisté bohatství, nýbrž v Boha, který nás štědře opatřuje vším, co potřebujeme;
#6:18 napomínej je, ať konají dobro a jsou bohatí v dobrých skutcích, štědří, dobročinní,
#6:19 a tak ať si střádají dobrý základ pro budoucnost, aby obdrželi pravý život.
#6:20 Opatruj, co ti bylo svěřeno, Timoteji, vyhýbej se bezbožným řečem a protikladným naukám, které se lživě nazývají „poznání“.
#6:21 Kdo se k němu hlásí, zbloudil ve víře. Milost s vámi!  

\book{II Timothy}{2Tim}
#1:1 Pavel, z vůle Boží apoštol Krista Ježíše, poslaný hlásat zaslíbený život v Kristu Ježíši -
#1:2 Timoteovi, milovanému synu: Milost, milosrdenství a pokoj od Boha Otce a Krista Ježíše, našeho Pána.
#1:3 Děkuji za tebe Bohu, kterému sloužím s čistým svědomím jako moji předkové, když na tebe neustále myslím ve svých modlitbách ve dne v noci.
#1:4 Vzpomínám na tvé slzy a toužím tě spatřit, aby moje radost byla úplná.
#1:5 Připomínám si tvou upřímnou víru, kterou měla už tvá babička Lóis a tvá matka Euniké, a kterou máš, jak jsem přesvědčen, i ty.
#1:6 Proto ti kladu na srdce, abys rozněcoval oheň Božího daru, kterého se ti dostalo vzkládáním mých rukou.
#1:7 Neboť Bůh nám nedal ducha bázlivosti, nýbrž ducha síly, lásky a rozvahy.
#1:8 Nestyď se tedy vydávat svědectví o našem Pánu; ani za mne, jeho vězně, se nestyď, nýbrž snášej spolu se mnou všechno zlé pro evangelium. K tomu ti dá sílu Bůh,
#1:9 který nás spasil a povolal svatým povoláním ne pro naše skutky, nýbrž ze svého rozhodnutí a z milosti, kterou nám daroval v Kristu Ježíši před věčnými časy
#1:10 a nyní zjevil příchodem našeho Spasitele Ježíše Krista. On zlomil moc smrti a zjevil nepomíjející život v evangeliu.
#1:11 K jeho zvěstování jsem já byl ustanoven hlasatelem, apoštolem a učitelem.
#1:12 Proto také všechno snáším a nestydím se vydávat svědectví, neboť vím, komu jsem uvěřil. Jsem přesvědčen, že on má moc chránit, co mi svěřil, až do onoho dne.
#1:13 Měj za vzor zdravých slov to, co jsi slyšel ode mne ve víře a lásce, která nás spojuje v Kristu Ježíši.
#1:14 Svěřený poklad chraň mocí Ducha svatého, který v nás přebývá.
#1:15 Víš, že se ode mne odvrátili všichni v Asii, mezi nimi Fygelos a Hermogenés.
#1:16 Kéž Pán prokáže milosrdenství Oneziforově rodině. Několikrát mě potěšil a nestyděl se za mé řetězy;
#1:17 když přišel do Říma, usilovně mě hledal, až mě našel.
#1:18 Kéž mu Pán dá, aby u něho našel milosrdenství v onen den. A jaké služby vykonal v Efezu, víš nejlépe sám. 
#2:1 A ty, můj synu, buď silný milostí Krista Ježíše,
#2:2 a co jsi ode mne slyšel před mnoha svědky, svěř to věrným lidem, kteří budou schopni učit zase jiné.
#2:3 Snášej se mnou všechno zlé jako řádný voják Krista Ježíše.
#2:4 Kdo se dá na vojnu, nezaplétá se do záležitostí obyčejného života; chce obstát před tím, kdo mu velí.
#2:5 A kdo závodí, nedostane cenu, nezávodí-li podle pravidel.
#2:6 Také rolník musí nejprve těžce pracovat, než sklidí úrodu.
#2:7 Uvažuj o tom, co říkám. Pán ti dá, abys všechno pochopil.
#2:8 Pamatuj na Ježíše Krista vzkříšeného z mrtvých, původem z rodu Davidova; to je moje evangelium.
#2:9 Pro ně snáším utrpení a dokonce pouta jako zločinec. Ale Boží slovo není spoutáno.
#2:10 A tak všechno snáším pro vyvolené, aby i oni dosáhli spásy v Kristu Ježíši a věčné slávy.
#2:11 Věrohodné je to slovo: Jestliže jsme s ním zemřeli, budeme s ním i žít.
#2:12 Jestliže s ním vytrváme, budeme s ním i vládnout. Zapřeme-li ho, i on nás zapře.
#2:13 Jsme-li nevěrní, on zůstává věrný, neboť nemůže zapřít sám sebe.
#2:14 Toto připomínej a před tváří Boží naléhavě domlouvej bratřím, aby se nepřeli o slova. Není to k ničemu, leda k rozvrácení posluchačů.
#2:15 Usiluj o to, aby ses před Bohem osvědčil jako dělník, který se nemá zač stydět, protože správně zvěstuje slovo pravdy.
#2:16 Bezbožným a planým řečem se vyhýbej. Neboť takoví lidé půjdou stále dál ve své bezbožnosti
#2:17 a jejich učení se bude šířit jako rakovina. K nim patří Hymenaios a Filétos,
#2:18 kteří zbloudili z cesty pravdy, když říkají, že naše vzkříšení už nastalo; tak podvracejí víru některých bratří.
#2:19 Ovšem pevný Boží základ trvá a nese nápis ‚Pán zná ty, kdo jsou jeho‘ a ‚ať se odvrátí od nespravedlnosti každý, kdo vyznává jméno Páně‘.
#2:20 Ve velké domácnosti nejsou jen zlaté a stříbrné nádoby, nýbrž i dřevěné a hliněné, jedny pro cenné věci, druhé na odpadky.
#2:21 Kdo se od těch falešných nauk očistí, bude nástrojem vznešeným, posvěceným, užitečným pro hospodáře, připraveným ke každému dobrému dílu.
#2:22 Vyhýbej se mladické prudkosti, usiluj o spravedlnost, víru, lásku a pokoj s těmi, kdo vzývají Pána z čistého srdce.
#2:23 Nepouštěj se do hloupých sporů, v jakých si libují nepoučení lidé; víš, že vedou jen k hádkám.
#2:24 Služebník Kristův se nemá hádat, nýbrž má být laskavý ke všem, schopný učit a být trpělivý.
#2:25 Má vlídně poučovat odpůrce. Snad jim dá Bůh, že se obrátí, poznají pravdu
#2:26 a vzpamatují se z ďáblových nástrah, do kterých se dali polapit, když podlehli jeho vůli. 
#3:1 Věz, že v posledních dnech nastanou zlé časy.
#3:2 Lidé budou sobečtí, chamtiví, chvástaví, domýšliví, budou se rouhat, nebudou poslouchat rodiče, budou nevděční, bezbožní,
#3:3 bez lásky, nesmiřitelní, pomlouvační, nevázaní, hrubí, lhostejní k dobrému,
#3:4 zrádní, bezhlaví, nadutí, budou mít raději rozkoš než Boha,
#3:5 budou se tvářit jako zbožní, ale svým jednáním to budou popírat. Takových lidí se straň.
#3:6 Patří k nim ti, kdo vnikají do rodin, aby nalákali lehkověrné ženy plné hříchů, ovládané rozličnými touhami,
#3:7 které by se pořád chtěly učit, a nikdy nemohou přijít k poznání pravdy.
#3:8 Jako Jannés a Jambrés odporovali Mojžíšovi, tak i ti falešní učitelé odporují pravdě. Jsou to lidé se zvrácenou myslí, kteří ve víře selhali.
#3:9 Ale s jejich úspěchy už je konec. Jejich zaslepenost bude všem stejně zjevná jako oněch dvou.
#3:10 Ty však jsi sledoval mé učení, můj způsob života, mé úmysly, mou víru, shovívavost, lásku, trpělivost,
#3:11 pronásledování a útrapy, jaké mne stihly v Antiochii, v Ikoniu a v Lystře. Jaká pronásledování jsem přestál, a ze všech mě Pán vysvobodil!
#3:12 A všichni, kdo chtějí zbožně žít v Kristu Ježíši, zakusí pronásledování.
#3:13 Avšak se zlými lidmi a podvodníky to půjde stále k horšímu, neboť klamou jiné i sebe.
#3:14 Ty však setrvávej v tom, čemu ses naučil a o čem jsi přesvědčen. Víš, od koho ses tomu naučil.
#3:15 Od dětství znáš svatá Písma, která ti mohou dát moudrost ke spasení, a to vírou v Krista Ježíše.
#3:16 Veškeré Písmo pochází z Božího Ducha a je dobré k učení, k usvědčování, k nápravě, k výchově ve spravedlnosti,
#3:17 aby Boží člověk byl náležitě připraven ke každému dobrému činu. 
#4:1 Před Bohem a Kristem Ježíšem, který bude soudit živé i mrtvé, tě zapřísahám pro jeho příchod a jeho království:
#4:2 Hlásej slovo Boží, ať přijdeš vhod či nevhod, usvědčuj, domlouvej, napomínej v trpělivém vyučování.
#4:3 Neboť přijde doba, kdy lidé nesnesou zdravé učení, a podle svých choutek si seženou učitele, kteří by vyhověli jejich přáním.
#4:4 Odvrátí sluch od pravdy a přikloní se k bájím.
#4:5 Avšak ty buď ve všem střízlivý, snášej útrapy, konej dílo zvěstovatele evangelia a cele se věnuj své službě.
#4:6 Neboť já již budu obětován, přišel čas mého odchodu.
#4:7 Dobrý boj jsem bojoval, běh jsem dokončil, víru zachoval.
#4:8 Nyní je pro mne připraven vavřín spravedlnosti, který mi dá v onen den Pán, ten spravedlivý soudce. A nejen mně, nýbrž všem, kdo s láskou vyhlížejí jeho příchod.
#4:9 Pospěš si, abys přišel za mnou co nejdřív.
#4:10 Démas mě totiž opustil, protože více miloval tento svět, a odešel do Tesaloniky. Krescens odešel do Galacie, Titus do Dalmácie.
#4:11 Jediný Lukáš je se mnou. Marka vezmi s sebou, bude mi užitečný jako pomocník.
#4:12 Tychika jsem poslal do Efezu.
#4:13 Plášť, který jsem nechal v Troadě u Karpa, přines s sebou, též knihy a zvláště pergamen.
#4:14 Kovář Alexandr mi způsobil mnoho zlého. Odplatí mu Pán podle jeho činů.
#4:15 Také ty si dej před ním pozor. Velmi se totiž stavěl proti našim slovům.
#4:16 Při mé první obhajobě nikdo při mně nebyl, všichni mě opustili. Kéž jim to Bůh nepočítá!
#4:17 Pán však při mně stál a dal mi sílu, abych mohl dovršit zvěstování evangelia, a slyšeli je všichni pohané; a byl jsem vysvobozen ze lví tlamy.
#4:18 Pán mě vysvobodí ze všeho zlého a zachová pro své nebeské království. Jemu patří sláva na věky věků, amen.
#4:19 Pozdravuj Prisku a Akvilu a rodinu Oneziforovu.
#4:20 Erastos zůstal v Korintu. Trofima jsem nechal v Milétu, protože byl nemocen.
#4:21 Pospěš si, abys přišel, než nastane zima. Pozdravuje tě Eubulos, Pudens, Linus, Klaudia a všichni bratří.
#4:22 Pán buď s tvým duchem. Milost s vámi.  

\book{Titus}{Titus}
#1:1 Pavel, Boží služebník a apoštol Ježíše Krista, poslaný k tomu, aby Boží vyvolené přivedl k víře a k poznání pravdy našeho náboženství,
#1:2 aby měli naději na věčný život, jejž slíbil pravdomluvný Bůh před věky, a ve svůj čas
#1:3 zjevil své slovo v kázání, které mi bylo svěřeno z rozkazu našeho Spasitele Boha -
#1:4 Titovi, vlastnímu synu ve společné víře: Milost a pokoj od Boha Otce a Krista Ježíše, našeho Spasitele.
#1:5 Proto jsem tě ponechal na Krétě, abys uvedl do pořádku, co ještě zbývá, a ustanovil v jednotlivých městech starší, jak jsem ti nařídil.
#1:6 Mají to být lidé bezúhonní, jen jednou ženatí, mají mít věřící děti, kterým se nedá vytknout nevázanost a neposlušnost.
#1:7 Neboť biskup má být bezúhonný jako správce Božího domu. Nemá být nadutý, zlostný, pijan, rváč, ziskuchtivý.
#1:8 Má být pohostinný, dobrý, rozvážný, spravedlivý, zbožný, zdrženlivý,
#1:9 pevný ve slovech pravé nauky, aby byl schopen jak povzbuzovat ve zdravém učení, tak usvědčovat odpůrce.
#1:10 Je mnoho těch, kteří se nepodřizují, vedou prázdné řeči a svádějí lidi: jsou to hlavně ti, kteří lpí na obřízce.
#1:11 Těm je třeba zavřít ústa. Pro hanebný zisk učí tomu, co se nepatří, a rozvracejí tím celé rodiny.
#1:12 Jeden z nich, jejich vlastní prorok, řekl: ‚Kréťané jsou samí lháři, zlá zvířata, lenivá břicha.‘
#1:13 A je to pravdivé svědectví. Proto je přísně kárej, aby měli zdravou víru
#1:14 a nedrželi se židovských bájí a příkazů lidí, kteří se odvracejí od pravdy.
#1:15 Čistým je všechno čisté. Ale poskvrněným a nevěřícím nic není čisté. Jak jejich rozum, tak jejich svědomí jsou poskvrněny.
#1:16 Prohlašují, že znají Boha, avšak svým jednáním to popírají. Jsou odporní, neposlušní a neschopni jakéhokoli dobrého skutku. 
#2:1 Ty však mluv, co odpovídá zdravému učení.
#2:2 Starší muži ať jsou střídmí, vážní, rozumní, ať jsou zdraví ve víře, lásce a trpělivosti.
#2:3 Podobně starší ženy ať se chovají důstojně, ať nepomlouvají a nepropadají přílišnému pití vína. Ať vyučují mladší ženy v dobrém
#2:4 a vedou je k tomu, aby měly rády své muže a své děti,
#2:5 byly rozumné, cudné, staraly se o domácnost, byly laskavé a poslouchaly své muže, aby Boží slovo nebylo zneváženo.
#2:6 Rovněž mladší muže napomínej, ať jsou ve všem rozvážní,
#2:7 a sám jim dávej dobrý příklad. Tvé učení ať je nezkažené, důvěryhodné,
#2:8 ať je to zdravé a nepochybné slovo, aby protivník byl zahanben a nemohl o nás povědět nic špatného.
#2:9 Otroci ať ve všem poslouchají své pány. Ať jsou úslužní, neodmlouvají,
#2:10 ať nic nezpronevěří a jsou naprosto spolehliví, a tak ve všem dělají čest učení našeho Spasitele Boha.
#2:11 Ukázala se Boží milost, která přináší spásu všem lidem
#2:12 a vychovává nás k tomu, abychom se zřekli bezbožnosti a světských vášní, žili rozumně, spravedlivě a zbožně v tomto věku
#2:13 a očekávali blažené splnění naděje a příchod slávy velikého Boha a našeho Spasitele Ježíše Krista.
#2:14 On se za nás obětoval, aby nás vykoupil ze všeho hříchu a posvětil za svůj vlastní lid, horlivý v dobrých skutcích.
#2:15 Tak mluv, napomínej a přesvědčuj se vším důrazem. Nikdo ať tebou nepohrdá. 
#3:1 Připomínej bratřím, ať jsou podřízeni těm, kdo mají vládu a moc, a ať je poslouchají. Ať jsou vždy hotovi jednat dobře.
#3:2 Ať nikoho nepomlouvají, ať se nepřou, jsou mírní a vždycky se ke všem chovají vlídně.
#3:3 Vždyť i my jsme byli kdysi nerozumní, neposlušní, zbloudilí, byli jsme otroky všelijakých vášní a rozkoší, žili jsme ve zlobě a závisti, byli jsme hodni opovržení a navzájem jsme se nenáviděli.
#3:4 Ale ukázala se dobrota a láska našeho Spasitele Boha:
#3:5 On nás zachránil ne pro spravedlivé skutky, které my jsme konali, nýbrž ze svého slitování; zachránil nás obmytím, jímž jsme se znovu zrodili k novému životu skrze Ducha svatého.
#3:6 Bohatě na nás vylil svého Ducha skrze Ježíše Krista, našeho Spasitele,
#3:7 abychom ospravedlněni jeho milostí měli podíl na věčném životě, k němuž se upíná naše naděje.
#3:8 Tato slova jsou spolehlivá, a chci, abys tomu všemu neochvějně učil, tak aby ti, kdo uvěřili Bohu, snažili se vynikat dobrým jednáním. To je dobré a lidem prospěšné.
#3:9 Hloupým sporům o rodokmeny, rozbrojům a hádkám o Zákon se vyhýbej. Jsou zbytečné a k ničemu.
#3:10 Sektáře jednou nebo dvakrát napomeň a pak se ho zřekni;
#3:11 je jasné, že takový člověk je převrácený, hřeší, a tak sám nad sebou vynáší soud.
#3:12 Jakmile k tobě pošlu Artemu nebo Tychika, přijď hned ke mně do Nikopole. Rozhodl jsem se, že tam strávím zimu.
#3:13 Zákoníka Zénu a Apolla pečlivě vyprav na cestu, aby jim nic nechybělo.
#3:14 Ať se učí i naši lidé vynikat v dobrých skutcích, kde je jich naléhavě potřebí, aby nebyli neužiteční.
#3:15 Pozdravují tě všichni, kdo jsou se mnou. Vyřiď pozdrav našim přátelům ve víře. Milost s vámi se všemi.  

\book{Philemon}{Phlm}
#1:1 Pavel, vězeň Krista Ježíše, a bratr Timoteus Filemonovi, našemu milovanému spolupracovníku,
#1:2 sestře Apfii, Archippovi, našemu spolubojovníku, a církvi v tvém domě:
#1:3 Milost vám a pokoj od Boha Otce našeho a Pána Ježíše Krista.
#1:4 Děkuji svému Bohu a stále na tebe pamatuji ve svých modlitbách,
#1:5 když slyším o tvé víře v Pána Ježíše a o tvé lásce ke všem bratřím.
#1:6 Prosím za tebe, aby se tvá účast na společné víře projevila tím, že rozpoznáš, co dobrého můžeme učinit pro Krista.
#1:7 Tvá láska mi přinesla velkou radost a povzbuzení, protože jsi potěšil srdce věřících, bratře.
#1:8 Ačkoli bych ti v Kristu mohl směle nařídit, co máš udělat,
#1:9 pro lásku raději prosím, já Pavel, vyslanec a nyní i vězeň Krista Ježíše.
#1:10 Prosím tě za svého syna, kterému jsem dal život ve vězení, Onezima,
#1:11 který ti před časem způsobil škodu, ale nyní je tobě i mně k užitku.
#1:12 Posílám ti ho zpět, je mi drahý jako mé vlastní srdce.
#1:13 Chtěl jsem si ho ponechat u sebe, aby mi ve vězení, kde jsem pro evangelium, sloužil místo tebe,
#1:14 avšak bez tvého souhlasu jsem nechtěl nic udělat, aby tvá dobrota nebyla jakoby vynucená, nýbrž aby byla dobrovolná.
#1:15 Snad proto byl na čas od tebe odloučen, abys ho měl navěky -
#1:16 ne už jako otroka, nýbrž mnohem více než otroka: jako milovaného bratra. Když se jím stal mně, oč více jím bude tobě před lidmi i před Pánem.
#1:17 Jsme-li tedy spolu spojeni, přijmi ho k sobě jako mne.
#1:18 Jestliže ti způsobil nějakou škodu nebo je ti něco dlužen, připiš to na můj účet.
#1:19 Já Pavel píšu vlastní rukou: já to nahradím. Abych ti neřekl, že mi dlužíš i sám sebe.
#1:20 Ano, bratře, udělej mi tu radost v Pánu. Potěš mé srdce v Kristu.
#1:21 Píšu ti v důvěře ve tvou poslušnost a vím, že uděláš víc, než říkám.
#1:22 Zároveň mi také připrav ubytování, neboť doufám, že vám budu pro vaše modlitby darován.
#1:23 Pozdravuje tě Epafras, můj spoluvězeň v Kristu Ježíši,
#1:24 Marek, Aristarchos, Démas a Lukáš, moji spolupracovníci.
#1:25 Milost Pána Ježíše Krista buď s vámi.  

\book{Hebrews}{Heb}
#1:1 Mnohokrát a mnohými způsoby mluvíval Bůh k otcům ústy proroků;
#1:2 v tomto posledním čase k nám promluvil ve svém Synu, jehož ustanovil dědicem všeho a skrze něhož stvořil i věky.
#1:3 On, odlesk Boží slávy a výraz Boží podstaty, nese všecko svým mocným slovem. Když dokonal očištění od hříchů, usedl po pravici Božího majestátu na výsostech
#1:4 a stal se o to vznešenějším než andělé, oč je převyšuje jménem, které mu bylo dáno.
#1:5 Komu kdy z andělů Bůh řekl: ‚Ty jsi můj Syn, já jsem tě dnes zplodil!‘ A jinde se praví: ‚Já mu budu Otcem a on mi bude Synem.‘
#1:6 A když chce uvést Prvorozeného do světa, praví opět: ‚Ať se mu pokloní všichni andělé Boží!‘
#1:7 O andělích je řečeno: ‚Jeho andělé jsou vanutí větru a jeho služebníci plápolající oheň.‘
#1:8 O Synovi však: ‚Tvůj trůn, Bože, je na věky věků a žezlo práva je žezlem tvého království,
#1:9 Miluješ spravedlnost a nenávidíš nepravost, proto pomazal tě, Bože, Bůh tvůj olejem radosti nad všechny tvé druhy.‘
#1:10 A dále: ‚Ty, Pane, jsi na počátku založil zemi, i nebesa jsou dílem tvých rukou.
#1:11 Ona pominou, ty však zůstáváš; nebesa zvetšejí jako oděv,
#1:12 svineš je jako plášť a jako šat se změní, ty však jsi stále týž a tvá léta nikdy neustanou.‘
#1:13 Kterému z andělů kdy řekl: ‚Usedni po mé pravici, dokud ti nedám nepřátele za podnož tvého trůnu!‘
#1:14 Což není každý anděl jen duchem, vyslaným k službě těm, kdo mají dojít spasení? 
#2:1 Proto se tím více musíme držet toho, co jsme slyšeli, abychom nebyli strženi proudem.
#2:2 Jestliže už slovo zákona, které vyslovili andělé, bylo pevné a každý přestupek i každá neposlušnost došla spravedlivé odplaty,
#2:3 jak bychom mohli uniknout my, pohrdneme-li tak slavným spasením? První je zvěstoval sám Pán, a ti, kdo uslyšeli, dosvědčili toto spasení i nám;
#2:4 Bůh potvrzoval jejich svědectví znameními, divy i rozličnými projevy své moci a rozdílením Ducha svatého podle své vůle.
#2:5 Andělům Bůh také nepodřídil budoucí svět, o němž mluvíme,
#2:6 kdežto o Synu je na jednom místě řečeno: ‚Co je člověk, že ho máš, Bože, na mysli, a Syn člověka, že na něj hledíš?
#2:7 Jen nakrátko jsi ho postavil níž než anděly, pak jsi ho korunoval ctí a slávou,
#2:8 všecko jsi podrobil pod jeho nohy.‘ Když mu tedy podrobil všecko, znamená to, že nezůstalo nic, co by mu nebylo podmaněno. Nyní ovšem ještě nevidíme, že by mu vše bylo podmaněno.
#2:9 Ale vidíme toho, který byl nakrátko postaven níže než andělé, Ježíše, jak je pro utrpení smrti korunován ctí a slávou; neboť měl z milosti Boží zakusit smrt za všecky.
#2:10 Bylo přirozené, že Bůh, pro něhož je vše a skrze něhož je vše, přivedl mnoho synů k slávě, když skrze utrpení učinil dokonalým původce jejich spásy.
#2:11 A on, který posvěcuje, i ti, kdo jsou posvěcováni, jsou z téhož Otce. Proto se nestydí nazývat je svými bratry, když říká:
#2:12 ‚Budu zvěstovat tvé jméno svým bratřím, uprostřed shromáždění tě budu chválit.‘
#2:13 A jinde praví: ‚Také já svou důvěru složím v Boha.‘ A dále: ‚Hle, já a děti, které mi dal Bůh.‘
#2:14 Protože sourozence spojuje krev a tělo, i on se stal jedním z nich, aby svou smrtí zbavil moci toho, kdo smrtí vládne, totiž ďábla,
#2:15 a aby tak vysvobodil ty, kdo byli strachem před smrtí drženi po celý život v otroctví.
#2:16 Neujímá se přece andělů, ale ‚ujímá se potomků Abrahamových‘.
#2:17 Proto musil být ve všem jako jeho bratří, aby se stal veleknězem milosrdným a věrným v Boží službě a mohl tak smířit hříchy lidu.
#2:18 Protože sám prošel zkouškou utrpení, může pomoci těm, na které přicházejí zkoušky. 
#3:1 Proto, bratří, vy, kteří jste svatí a máte účast na nebeském povolání, hleďte na apoštola a velekněze našeho vyznání, Ježíše:
#3:2 byl věrný tomu, kdo jej ustanovil, jako i ‚Mojžíš byl věrný v celém Božím domě‘.
#3:3 Ježíš je však hoden větší slávy než Mojžíš, tak jako stavitel domu je víc nežli dům sám.
#3:4 Každý dům někdo staví; ten, kdo postavil vše, je Bůh.
#3:5 Mojžíš byl věrný v celém Božím domě, ale jen jako služebník, který měl dosvědčit to, co teprve bude vysloveno.
#3:6 Kristus však jako Syn je nad celým Božím domem. A tímto Božím domem jsme my, pokud si až do konce zachováme smělou jistotu a radostnou naději.
#3:7 Proto, jak říká Duch svatý: ‚Jestliže dnes uslyšíte jeho hlas,
#3:8 nezatvrzujte svá srdce ve vzdoru jako v den pokušení na poušti,
#3:9 kde si vaši otcové žádali důkazy a tak mě pokoušeli, ač viděli mé skutky
#3:10 po čtyřicet let. Proto jsem se na to pokolení rozhněval a řekl jsem: Jejich srdce stále bloudí, dodnes mé cesty nepoznali.
#3:11 Ve svém hněvu jsem přísahal: Nevejdou do mého odpočinutí!‘
#3:12 Dejte si pozor, bratří, aby někdo z vás neměl srdce zlé a nevěrné, takže by odpadl od živého Boha.
#3:13 Naopak, povzbuzujte se navzájem den co den, dokud ještě trvá ono ‚dnes‘, aby se nikdo z vás, oklamán hříchem, nezatvrdil.
#3:14 Vždyť máme účast na Kristu, jen když své počáteční předsevzetí zachováme pevné až do konce.
#3:15 Je řečeno: ‚Jestliže dnes uslyšíte jeho hlas, nezatvrzujte svá srdce ve vzdoru!‘
#3:16 Kdo slyšel a zatvrdil se? Což to nebyli všichni, kdo vyšli z Egypta pod Mojžíšovým vedením?
#3:17 A na koho se Bůh hněval po čtyřicet let? Zdali ne na ty, kdo zhřešili a jejichž těla padla na poušti?
#3:18 A komu přísahal, že nevejdou do jeho odpočinutí, ne-li těm, kdo se vzepřeli?
#3:19 Tak vidíme, že nemohli vejít pro svou nevěru. 
#4:1 Střezme se, aby o někom z vás neplatilo, že v čase, dokud zaslíbení trvá, promeškal vstup do Božího odpočinutí.
#4:2 I nám se přece dostalo zaslíbení jako těm na poušti. Ale zvěst, kterou slyšeli, jim neprospěla, když ji vírou nepřijali.
#4:3 Neboť do odpočinutí vcházíme jen my, kdo jsme uvěřili, jak bylo řečeno: ‚Přísahal jsem ve svém hněvu: Do mého odpočinutí nevejdou!‘ To řekl Bůh, ač jeho odpočinutí trvá od chvíle, kdy stvořil svět.
#4:4 O sedmém dni je přece v Písmu řečeno: ‚I odpočinul Bůh sedmého dne od všeho svého díla.‘
#4:5 Zde však čteme: ‚Do mého odpočinutí nevejdou.‘
#4:6 Trvá-li tedy možnost, aby někteří do odpočinutí vešli, a ti, kterým ta zvěst byla nejprve ohlášena, pro neposlušnost nevešli,
#4:7 určuje Bůh jiné ‚dnes‘. V Davidových žalmech, tedy po dlouhé době, jak už bylo pověděno, říká: ‚Jestliže dnes uslyšíte jeho hlas, nezatvrzujte svá srdce!‘
#4:8 Kdyby byl Jozue již uvedl lid do odpočinutí, nemluvil by Bůh o jiném, pozdějším dnu.
#4:9 Tak má Boží lid pravý sobotní odpočinek teprve před sebou.
#4:10 Neboť kdo vejde do Božího odpočinutí, odpočine od svého díla, tak jako Bůh odpočinul od svého.
#4:11 A tak usilujme, abychom vešli do toho odpočinutí a nikdo pro neposlušnost nepadl jako ti na poušti.
#4:12 Slovo Boží je živé, mocné a ostřejší než jakýkoli dvousečný meč; proniká až na rozhraní duše a ducha, kostí a morku, a rozsuzuje touhy i myšlenky srdce.
#4:13 Není tvora, který by se před ním mohl skrýt. Nahé a odhalené je všechno před očima toho, jemuž se budeme ze všeho odpovídat.
#4:14 Protože máme mocného velekněze, který vstoupil až před Boží tvář, Ježíše, Syna Božího, držme se toho, co vyznáváme.
#4:15 Nemáme přece velekněze, který není schopen mít soucit s našimi slabostmi; vždyť na sobě zakusil všechna pokušení jako my, ale nedopustil se hříchu.
#4:16 Přistupme tedy směle k trůnu milosti, abychom došli milosrdenství a nalezli milost a pomoc v pravý čas. 
#5:1 Každý velekněz, vybraný z lidí, bývá ustanoven jako zástupce lidí před Bohem, aby přinášel dary i oběti za hříchy.
#5:2 Má mít soucit s těmi, kdo chybují a bloudí, protože sám také podléhá slabosti.
#5:3 A proto je povinen přinášet oběti za hřích nejenom za lid, ale i sám za sebe.
#5:4 Hodnost velekněze si nikdo nemůže přisvojit sám, nýbrž povolává ho Bůh jako kdysi Árona.
#5:5 Tak ani Kristus si nepřisvojil slávu velekněze sám, ale dal mu ji ten, který řekl: ‚Ty jsi můj Syn, já jsem tě dnes zplodil.‘
#5:6 A na jiném místě říká: ‚Ty jsi kněz navěky podle řádu Melchisedechova.‘
#5:7 Ježíš za svého pozemského života přinesl s bolestným voláním a slzami oběť modliteb a úpěnlivých proseb Bohu, který ho mohl zachránit před smrtí; a Bůh ho pro jeho pokoru slyšel.
#5:8 Ačkoli to byl Boží Syn, naučil se poslušnosti z utrpení, jímž prošel,
#5:9 tak dosáhl dokonalosti a všem, kteří ho poslouchají, stal se původcem věčné spásy,
#5:10 když ho Bůh prohlásil veleknězem podle řádu Melchisedechova.
#5:11 O tom by bylo mnoho co mluvit, ale je těžké vám to vyložit, protože nejste ochotni slyšet.
#5:12 Za takovou dobu už byste měli být sami učiteli, a zatím opět potřebujete, aby vás někdo učil abecedě Boží řeči; potřebujete mléko, ne hutný pokrm.
#5:13 Každý, kdo potřebuje mléko, protože nepřivykl slovu spravedlnosti, je jako nemluvně.
#5:14 Hutný pokrm je pro vyspělé, pro ty, kdo mají cvičením své smysly vypěstovány tak, že rozeznají dobré od špatného. 
#6:1 Proto nezůstávejme již u počátečního učení o Kristu, ale směřujme k dospělosti. Nevracejme se k základním článkům o pokání z mrtvých skutků, o víře v Boha,
#6:2 k učení o křtu a vzkládání rukou, o vzkříšení z mrtvých a o posledním soudu.
#6:3 Budeme moci jít dále, když to Bůh dovolí.
#6:4 Kdo byli už jednou osvíceni a okusili nebeského daru, kdo se stali účastníky Ducha svatého
#6:5 a zakusili pravdivost Božího slova i moc budoucího věku,
#6:6 a pak odpadli, s těmi není možno znovu začínat a vést je k pokání, protože znovu křižují Božího Syna a uvádějí ho v posměch.
#6:7 Země, která často přijímá déšť a rodí užitečnou rostlinu těm, kdo ji obdělávají, má účast na Božím požehnání.
#6:8 Rodí-li však jen trní a bodláčí, je nepotřebná a blízká prokletí a nakonec bývá vypálena.
#6:9 I když takto mluvíme, jsme o vás přesvědčeni, milovaní, že jste na dobré cestě ke spáse.
#6:10 Bůh není nespravedlivý, a proto nezapomene, jak jste se činem své lásky k němu přiznali, když jste sloužili a ještě sloužíte bratřím.
#6:11 Toužíme jen, aby na každém z vás bylo vidět neutuchající horlivost až do konce, kdy se naplní naše naděje;
#6:12 proto neochabujte, ale vezměte si za vzor ty, kdo pro víru a trpělivost mají podíl na zaslíbení.
#6:13 Tak dal Bůh zaslíbení Abrahamovi. Poněvadž při nikom vyšším přísahat nemohl, přísahal při sobě samém:
#6:14 ‚Hojně ti požehnám a dám ti mnoho potomků.‘
#6:15 A protože byl Abraham trpělivý, dosáhl splnění Božího slibu.
#6:16 Lidé totiž přísahají při někom větším a přísaha je zárukou, kterou končí každý spor.
#6:17 Když Bůh chtěl účastníkům zaslíbení přesvědčivě prokázat nezměnitelnost svého rozhodnutí, potvrdil své zaslíbení ještě přísahou.
#6:18 A tak tyto dvě nezměnitelné věci, v nichž Bůh přece nemůže lhát, jsou mocným povzbuzením pro nás, kteří jsme nalezli útočiště v naději nám dané.
#6:19 V ní jsme bezpečně a pevně zakotveni, jí pronikáme až do nitra nebeské svatyně,
#6:20 kam jako první za nás vstoupil Ježíš, kněz na věky podle řádu Melchisedechova. 
#7:1 Tento Melchisedech, král Sálemu a kněz nejvyššího Boha, vyšel vstříc Abrahamovi, když se vracel po vítězství nad králi. Požehnal mu
#7:2 a Abraham mu z veškeré kořisti dal desátý díl. Jméno Melchisedech se vykládá jako král spravedlnosti, král Sálemu pak znamená král pokoje.
#7:3 Je bez otce, bez matky, bez předků, jeho dny nemají počátek a jeho život je bez konce. A tak podoben Synu Božímu zůstává knězem navždy.
#7:4 Hleďte, jak vznešený je ten, jemuž sám praotec Abraham dal jako desátek nejlepší část své kořisti.
#7:5 Levité, pověření kněžskou službou, mají podle zákona příkaz brát desátky od Božího lidu, to jest od svých bratří, ačkoli všichni pocházejí z Abrahamova rodu.
#7:6 Od Abrahama však dostal desátek ten, který nepocházel z rodu Levi, a Abraham, který měl zaslíbení, přijal od něho požehnání.
#7:7 A není sporu, že větší žehná menšímu.
#7:8 Levité dostávají desátky jako smrtelní lidé, Melchisedech však jako ten, o kom Písmo svědčí, že žije.
#7:9 Prostřednictvím Abrahamovým dal tak desátky i Levi, který sám desátky přijímá.
#7:10 Ještě se totiž nenarodil a byl v těle svého praotce Abrahama, když mu Melchisedech vyšel vstříc.
#7:11 Kdyby služba levitských kněží, která vedla lid k poslušnosti zákona, přinesla dokonalost, nač by ještě bylo třeba ustanovovat jiného kněze podle řádu Melchisedechova, a nezůstat při kněžství podle řádu Áronova?
#7:12 Avšak mění-li se kněžství, nutně nastává i změna zákona.
#7:13 A ten, na nějž se to slovo vztahuje, pocházel z jiného pokolení, z něhož nikdo nekonal službu u oltáře.
#7:14 Je dobře známo, že náš Pán pocházel z Judy, a u Mojžíše není zmínky o kněžích z tohoto pokolení.
#7:15 To vše je ještě zřejmější, když podobně jako Melchisedech je ustanoven jiný kněz
#7:16 ne podle zákona o tělesném původu, nýbrž na základě svého nepomíjejícího života,
#7:17 jak se o něm svědčí: ‚Ty jsi kněz navěky podle řádu Melchisedechova.‘
#7:18 Tím se ovšem ruší předchozí příkaz jako neúčinný a neužitečný -
#7:19 zákon totiž nic nepřivedl k dokonalosti - avšak na jeho místo přichází lepší naděje, jejíž mocí přistupujeme až k Bohu.
#7:20 A navíc se to nestalo bez přísahy. Levité se totiž stávali kněžími bez Boží přísahy.
#7:21 Ježíš však se jím stal s přísahou; vždyť mu bylo řečeno: ‚Hospodin přísahal a nebude toho litovat: Ty jsi kněz navěky.‘
#7:22 Proto se Ježíš stal ručitelem lepší smlouvy.
#7:23 A dále: Levitských kněží muselo být mnoho, protože umírali a nemohli sloužit trvale.
#7:24 Ježíšovo kněžství však nepřechází na jiného, neboť on zůstává navěky.
#7:25 Proto přináší dokonalé spasení těm, kdo skrze něho přistupují k Bohu; je stále živ a přimlouvá se za ně.
#7:26 To je ten velekněz, jakého jsme potřebovali: svatý, nevinný, neposkvrněný, oddělený od hříšníků a vyvýšený nad nebesa,
#7:27 který nemusí jako dřívější velekněží denně přinášet oběti napřed za vlastní hříchy a pak teprve za hříchy lidu. Ježíš to učinil jednou provždy, když obětoval sebe sama.
#7:28 Zákon totiž ustanovuje za velekněze lidi, podléhající slabosti, ale slovo přísahy, dané až po zákonu, ustanovuje Syna navěky dokonalého. 
#8:1 Z toho, co bylo řečeno, plyne: máme velekněze, který usedl po pravici Božího trůnu
#8:2 v nebesích jako služebník pravé svatyně a stánku, který zřídil sám Hospodin, a nikoli člověk.
#8:3 Každý velekněz bývá ustanoven k tomu, aby přinášel dary a oběti; proto musel i Ježíš nutně přinést oběť.
#8:4 Na zemi by nemohl být knězem, protože zde už byli kněží, kteří přinášeli dary podle zákona.
#8:5 Ti však sloužili ve svatyni, která je jen náznakem a stínem svatyně nebeské. Vždyť Bůh uložil Mojžíšovi, když měl zřídit stánek: ‚Hleď, ať uděláš vše podle vzoru, který ti byl ukázán na hoře.‘
#8:6 Avšak Ježíš dosáhl vznešenější služby, právě tak jako je prostředníkem vyšší smlouvy, založené na lepších zaslíbeních.
#8:7 Kdyby totiž ta první smlouva byla bez vady, nebylo by třeba připravovat druhou.
#8:8 Ale když Bůh kárá svůj lid, říká: ‚Hle, přicházejí dny, praví Hospodin, kdy s domem izraelským i s domem judským uzavřu smlouvu novou,
#8:9 ne jako byla ta smlouva, kterou jsem uzavřel s jejich otci v den, kdy jsem je vzal za ruku, abych je vyvedl ze země egyptské. Neboť oni nezůstali v mé smlouvě, a já jsem se jich zřekl, praví Hospodin.
#8:10 A toto je smlouva, kterou uzavřu s domem izraelským po oněch dnech, praví Hospodin: Dám své zákony do jejich mysli a napíšu jim je na srdce. Budu jim Bohem a oni budou mým lidem.
#8:11 Pak už nebude učit druh druha a bratr bratra a nebude vybízet: ‚Poznej Pána‘, protože mě budou znát všichni, od nejmenšího až do největšího.
#8:12 Slituji se nad jejich nepravostmi a na jejich hříchy už nevzpomenu.‘
#8:13 Když Bůh mluví o nové smlouvě, říká tím, že první je zastaralá. Co je zastaralé a vetché, blíží se zániku. 
#9:1 I ta první smlouva měla ovšem bohoslužebný řád i posvátné místo, ale pozemské.
#9:2 To byl stánek, v jehož přední části, zvané svatyně, byly svícny a stůl na předložené chleby.
#9:3 Za oponou byla druhá část stánku, nazývaná nejsvětější svatyně.
#9:4 Tam byl zlatý kadidlový oltář, truhla smlouvy, ze všech stran krytá zlatem, v ní pak byla zlatá nádoba s manou, Áronova hůl, která se kdysi zazelenala, a desky smlouvy.
#9:5 Nad ní byli cherubové Boží slávy, kteří zastiňovali slitovnici. O tom teď není třeba dopodrobna mluvit.
#9:6 Od té doby, kdy bylo všecko tak zařízeno, vcházeli stále do přední části stánku kněží a konali tam bohoslužbu.
#9:7 Do druhé části stánku vcházel jen jednou za rok sám velekněz, a to nikdy bez krve, kterou obětoval za sebe i za přestoupení lidu.
#9:8 Tím Duch svatý naznačuje, že ještě nebyla otevřena cesta do nejsvětější svatyně, pokud stála přední část stánku.
#9:9 To bylo podobenstvím pro nynější čas, neboť dary a oběti, které se tam přinášely, nemohly dokonale očistit svědomí toho, kdo je obětuje;
#9:10 jde jen o pokrmy, nápoje a různá omývání, tedy o vnější předpisy, platné jen do nového uspořádání.
#9:11 Ale když přišel Kristus, velekněz, který nám přináší skutečné dobro, neprošel stánkem zhotoveným rukama, to jest patřícím k tomuto světu, nýbrž stánkem větším a dokonalejším.
#9:12 A nevešel do svatyně s krví kozlů a telat, ale jednou provždy dal svou vlastní krev, a tak nám získal věčné vykoupení.
#9:13 Jestliže již pokropení krví kozlů a býků a popel z jalovice posvěcuje poskvrněné a zevně je očišťuje,
#9:14 čím více krev Kristova očistí naše svědomí od mrtvých skutků k službě živému Bohu! Vždyť on přinesl sebe sama jako neposkvrněnou oběť Bohu mocí Ducha, který nepomíjí.
#9:15 Proto je Kristus prostředníkem nové smlouvy, aby ti, kdo jsou od Boha povoláni, přijali věčné dědictví, které jim bylo zaslíbeno - neboť jeho smrt přinesla vykoupení z hříchů, spáchaných za první smlouvy.
#9:16 Při závěti se musí prokázat smrt toho, kdo ji ustanovil.
#9:17 Jen závěť zemřelých je totiž platná; nemá však platnost, dokud žije ten, kdo ji ustanovil.
#9:18 Proto ani první smlouva nebyla uzavřena bez vylití krve;
#9:19 když Mojžíš všemu lidu oznámil všecka přikázání podle zákona, vzal krev telat a kozlů, vodu, červenou vlnu s yzopem a pokropil knihu Zákona i všechen lid. A řekl jim:
#9:20 „Toto je krev smlouvy, kterou s vámi uzavřel Bůh.“
#9:21 Podobně pokropil krví i stánek a všecko bohoslužebné náčiní.
#9:22 Podle zákona se skoro vše očišťuje krví, a bez vylití krve není odpuštění.
#9:23 To, co je jen náznakem nebeských věcí, bylo nutno očišťovat takovým způsobem; nebeské věci samy však vyžadují vzácnějších obětí.
#9:24 Vždyť Kristus nevešel do svatyně, kterou lidské ruce udělaly jen jako napodobení té pravé, nýbrž vešel do samého nebe, aby se za nás postavil před Boží tváří.
#9:25 Není třeba, aby sám sebe obětoval vždy znovu, jako když velekněz rok co rok s cizí krví vchází do svatyně;
#9:26 jinak by musel trpět mnohokrát od založení světa. On se však zjevil jen jednou na konci věků, aby svou obětí sňal hřích.
#9:27 A jako každý člověk jen jednou umírá, a potom bude soud,
#9:28 tak i Kristus byl jen jednou obětován, aby na sebe vzal hříchy mnohých; po druhé se zjeví ne už kvůli hříchu, ale ke spáse těm, kdo ho očekávají. 
#10:1 V zákoně je pouze náznak budoucího dobra, ne sama jeho skutečnost. Proto stále stejné oběti, přinášené každoročně znovu a znovu, nemohou nikdy dokonale očistit ty, kdo s nimi přicházejí.
#10:2 Kdyby ti, kdo je přinášejí, neměli už vědomí hříchu, protože byli jednou provždy očištěni, dávno by byly tyto oběti přestaly.
#10:3 Ale těmito oběťmi se hříchy naopak každoročně připomínají,
#10:4 neboť krev býků a kozlů není s to hříchy odstranit.
#10:5 Proto Kristus říká, když přichází na svět: ‚Oběti ani dary jsi nechtěl, ale dal jsi mi tělo.
#10:6 V zápalné oběti ani v oběti za hřích, Bože, jsi nenašel zalíbení.
#10:7 Proto jsem řekl: Zde jsem, abych konal, Bože, tvou vůli, jak je o mně v tvé knize psáno.‘
#10:8 Předně říká: ‚Oběti ani dary, oběti zápalné ani oběti za hřích jsi nechtěl a nenašel jsi v nich zalíbení‘ - totiž v takových, jaké se obětují podle zákona.
#10:9 Potom však řekne: ‚Zde jsem, abych konal tvou vůli.‘ Tak ruší prvé, aby ustanovil druhé.
#10:10 Tou vůlí jsme posvěceni, neboť Ježíš Kristus jednou provždy obětoval své tělo.
#10:11 Každý kněz stojí a koná denně bohoslužbu, znovu a znovu přináší tytéž oběti, ale ty nikdy nemohou navždy zahladit hříchy.
#10:12 Kristus však přinesl za hříchy jedinou oběť, navěky usedl po pravici Boží
#10:13 a hledí vstříc tomu, ‚až mu budou nepřátelé dáni za podnož jeho trůnu‘.
#10:14 Tak jedinou obětí navždy přivedl k dokonalosti ty, které posvěcuje.
#10:15 Dosvědčuje nám to i Duch svatý, když říká:
#10:16 ‚Toto je smlouva, kterou s nimi uzavřu po oněch dnech, praví Pán; dám své zákony do jejich srdce a vepíšu jim je do mysli;
#10:17 na jejich hříchy a nepravosti už nikdy nevzpomenu.‘
#10:18 Tam, kde jsou hříchy odpuštěny, není už třeba přinášet za ně oběti.
#10:19 Protože Ježíš obětoval svou krev, smíme se, bratří, odvážit vejít do svatyně
#10:20 cestou novou a živou, kterou nám otevřel zrušením opony - to jest obětováním svého těla.
#10:21 Máme-li tedy tak velikého kněze nad celým Božím domem,
#10:22 přistupujme před Boha s opravdovým srdcem a v plné jistotě víry, se srdcem očištěným od zlého svědomí a s tělem obmytým čistou vodou.
#10:23 Držme se neotřesitelné naděje, kterou vyznáváme, protože ten, kdo nám dal zaslíbení, je věrný.
#10:24 Mějme zájem jeden o druhého a povzbuzujme se k lásce a k dobrým skutkům.
#10:25 Nezanedbávejte společná shromáždění, jak to někteří mají ve zvyku, ale napomínejte se tím více, čím více vidíte, že se blíží den Kristův.
#10:26 Jestliže svévolně hřešíme i po tom, když jsme už poznali pravdu, nemůžeme počítat s žádnou obětí za hříchy,
#10:27 ale jen s hrozným soudem a ‚žárem ohně, který stráví Boží odpůrce‘.
#10:28 Už ten, kdo pohrdne zákonem Mojžíšovým, nedojde slitování a propadá smrti na základě svědectví dvou nebo tří svědků.
#10:29 Pomyslete, oč hroznějšího trestu si zaslouží ten, kdo zneuctí Božího Syna a za nic nemá krev smlouvy, jíž byl posvěcen, a tak se vysmívá Duchu milosti.
#10:30 Vždyť víme, kdo řekl: ‚Já budu trestat, má je odplata.‘ A jinde: ‚Pán bude soudit svůj lid.‘
#10:31 Je hrozné upadnout do ruky živého Boha.
#10:32 Jen si vzpomeňte na dřívější dny! Sotva jste byli osvíceni, už jste museli podstoupit mnohý zápas s utrpením;
#10:33 někteří tím, že byli před očima všech uráženi a utiskováni, jiní tím, že stáli při postižených.
#10:34 Vždyť jste trpěli spolu s uvězněnými a s radostí jste snesli i to, že jste byli připraveni o majetek, neboť víte, že máte bohatství lepší a trvalé.
#10:35 Neztrácejte proto odvahu, neboť bude bohatě odměněna.
#10:36 Potřebujete však vytrvalost, abyste splnili Boží vůli a dosáhli toho, co bylo zaslíbeno.
#10:37 Vždyť už jen ‚docela krátký čas, a přijde ten, který má přijít, a neopozdí se.
#10:38 Avšak můj spravedlivý - říká Bůh - bude žít, protože uvěřil. Kdo by však odpadl, v tom nenajdu zalíbení.‘
#10:39 Ale my přece nepatříme k těm, kdo odpadají a zahynou, nýbrž k těm, kdo věří a dosáhnou života. 
#11:1 Věřit Bohu znamená spolehnout se na to, v co doufáme, a být si jist tím, co nevidíme.
#11:2 K takové víře předků se Bůh přiznal svým svědectvím.
#11:3 Ve víře chápeme, že Božím slovem byly založeny světy, takže to, na co hledíme, nevzniklo z viditelného.
#11:4 Ábel věřil, a proto přinesl Bohu lepší oběť než Kain a dostalo se mu svědectví, že je spravedlivý, když Bůh přijal jeho dary; protože věřil, ‚ještě mluví, ač zemřel‘.
#11:5 Henoch věřil, a proto nespatřil smrt, ale Bůh ho vzal k sobě. ‚Nebyl nalezen, protože ho Bůh přijal.‘ Ještě než ho přijal, dostalo se Henochovi svědectví, že v něm Bůh našel zalíbení.
#11:6 Bez víry však není možné zalíbit se Bohu. Kdo k němu přistupuje, musí věřit, že Bůh jest a že se odměňuje těm, kdo ho hledají.
#11:7 Noé věřil, a proto pokorně přijal, co mu Bůh oznámil a co ještě nebylo vidět, a připravil koráb k záchraně své rodiny. Svou vírou vynesl soud nad světem a získal podíl na spravedlnosti založené ve víře.
#11:8 Abraham věřil, a proto uposlechl, když byl povolán, aby šel do země, kterou měl dostat za úděl; a vydal se na cestu, ačkoli nevěděl, kam jde.
#11:9 Věřil, a proto žil v zemi zaslíbené jako cizinec, bydlil ve stanech s Izákem a Jákobem, pro které platilo totéž zaslíbení,
#11:10 a upínal naději k městu s pevnými základy, jehož stavitelem a tvůrcem je sám Bůh.
#11:11 Také Sára věřila, a proto přijala od Boha moc, aby se stala matkou, ačkoliv už překročila svůj čas; pevně věřila tomu, kdo jí dal zaslíbení.
#11:12 Tak z jednoho muže, a to už starce, vzešlo tolik potomků, ‚jako bezpočtu je hvězd na nebi a jako je písku na mořském břehu‘.
#11:13 Ve víře zemřeli ti všichni, i když se splnění slibů nedožili, nýbrž jen zdálky je zahlédli a pozdravili, vyznávajíce, že jsou na zemi jen cizinci a přistěhovalci.
#11:14 Tím dávají najevo, že po pravé vlasti teprve touží.
#11:15 Kdyby měli na mysli zemi, z níž vykročili, měli možnost se tam vrátit.
#11:16 Ale oni toužili po lepší vlasti, po vlasti nebeské. Proto sám Bůh se nestydí nazývat se jejich Bohem. Vždyť jim připravil své město.
#11:17 Abraham věřil, a proto šel obětovat Izáka, když byl podroben zkoušce. Svého jediného syna byl hotov obětovat, ačkoli se mu dostalo zaslíbení a bylo mu řečeno:
#11:18 ‚Z Izáka bude pocházet tvé potomstvo.‘
#11:19 Počítal s tím, že Bůh je mocen vzkřísit i mrtvé. Proto dostal Izáka zpět jako předobraz budoucího vzkříšení.
#11:20 Izák věřil, a proto Jákobovi a Ezauovi dal požehnání v pohledu do budoucna.
#11:21 Jákob věřil, a když umíral, požehnal oběma Josefovým synům a poklonil se přitom k vrcholu své berly.
#11:22 Josef věřil, a na sklonku svého života řekl, že synové izraelští vyjdou z Egypta, a nařídil, co se má stát s jeho kostmi.
#11:23 Mojžíšovi rodiče věřili, a proto svého syna tři měsíce po narození ukrývali; viděli, že je to vyvolené dítě, a nezalekli se královského rozkazu.
#11:24 Mojžíš věřil, a proto, když dospěl, odepřel nazývat se synem faraónovy dcery.
#11:25 Raději chtěl snášet příkoří s Božím lidem, než na čas žít příjemně v hříchu;
#11:26 a Kristovo pohanění pokládal za větší bohatství než všechny poklady Egypta, neboť upíral svou mysl k budoucí odplatě.
#11:27 Věřil, a proto vyšel z Egypta a nedal se zastrašit královým hněvem; zůstal pevný, jako by Neviditelného viděl.
#11:28 Věřil, a proto ustanovil hod beránka a dal pokropit dveře jeho krví, aby se zhoubce nemohl dotknout prvorozených.
#11:29 Izraelští věřili, a proto prošli Rudým mořem jako po suché zemi, ale když se o to pokusili Egypťané, pohltily je vlny.
#11:30 Izraelští věřili, a proto před nimi padly hradby Jericha, když je obcházeli po sedm dní.
#11:31 Nevěstka Rachab věřila, a proto přátelsky přijala vyzvědače a nezahynula s nevěřícími.
#11:32 Mám ještě pokračovat? Vždyť by mi nestačil čas, kdybych měl vypravovat o Gedeónovi, Barákovi, Samsonovi, Jeftovi, Davidovi, Samuelovi a prorocích,
#11:33 kteří svou vírou dobývali království, uskutečňovali Boží spravedlnost, dosáhli toho, co jim bylo zaslíbeno;
#11:34 zavírali tlamy lvům, krotili plameny ohně, unikali ostří meče, v slabosti nabývali síly, vedli si hrdinsky v boji, zaháněli na útěk vojska cizinců;
#11:35 ženám se jejich mrtví vraceli vzkříšení. Jiní byli mučeni a odmítli se zachránit, protože chtěli dosáhnout něčeho lepšího, totiž vzkříšení.
#11:36 Jiní zakusili výsměch a bičování, ba i okovy a žalář.
#11:37 Byli kamenováni, mučeni, řezáni pilou, umírali pod ostřím meče. Chodili v ovčích a kozích kůžích, trpěli nouzi, zakoušeli útisk a soužení.
#11:38 Svět jich nebyl hoden, bloudili po pouštích a horách, skrývali se v jeskyních a roklinách země.
#11:39 A ti všichni, ačkoliv osvědčili svou víru, nedočkali se splnění toho, co bylo zaslíbeno,
#11:40 neboť Bůh, který zamýšlel pro nás něco lepšího, nechtěl, aby dosáhli cíle bez nás. 
#12:1 Proto i my, obklopeni takovým zástupem svědků, odhoďme všecku přítěž i hřích, který se nás tak snadno přichytí, a vytrvejme v běhu, jak je nám uloženo,
#12:2 s pohledem upřeným na Ježíše, který vede naši víru od počátku až do cíle. Místo radosti, která se mu nabízela, podstoupil kříž, nedbaje na potupu; proto usedl po pravici Božího trůnu.
#12:3 Myslete na to, co všecko on musel snést od hříšníků, abyste neochabovali a neklesali na duchu.
#12:4 Ještě jste v zápase s hříchem nemuseli prolít svou krev.
#12:5 Což jste zapomněli na slova, jimiž vás Bůh povzbuzuje jako své syny: ‚Synu můj, podrobuj se kázni Páně a neklesej na mysli, když tě kárá.
#12:6 Koho Pán miluje, toho přísně vychovává, a trestá každého, koho přijímá za syna.‘
#12:7 Podvolujte se jeho výchově; Bůh s vámi jedná jako se svými syny. Byl by to vůbec syn, kdyby ho otec nevychovával?
#12:8 Jste-li bez takové výchovy, jaké se dostává všem synům, pak nejste synové, ale cizí děti.
#12:9 Naši tělesní otcové nás trestali, a přece jsme je měli v úctě; nemáme být mnohem víc poddáni tomu Otci, který dává Ducha a život?
#12:10 A to nás naši tělesní otcové vychovávali podle svého uvážení a jen pro krátký čas, kdežto nebeský Otec nás vychovává k vyššímu cíli, k podílu na své svatosti.
#12:11 Přísná výchova se ovšem v tu chvíli nikdy nezdá příjemná, nýbrž krušná, později však přináší ovoce pokoje a spravedlnost těm, kdo jí prošli.
#12:12 ‚Posilněte proto své zemdlené ruce i klesající kolena‘
#12:13 a ‚vykročte jistým krokem‘, aby to, co je chromé, docela nezchromlo, ale naopak se uzdravilo.
#12:14 Usilujte o pokoj se všemi a o svatost, bez níž nikdo nespatří Pána.
#12:15 Dbejte na to, ať nikdo nepromešká Boží milost; ať se nerozbují nějaký jedovatý kořen, který by nakazil mnohé.
#12:16 Ať nikdo není nevěrný a bezbožný jako Ezau, který prodal prvorozenství za jediný pokrm.
#12:17 Víte přece, že když se potom chtěl stát dědicem požehnání, byl odmítnut. Neměl už příležitost k nápravě, ačkoliv ji s pláčem hledal.
#12:18 Nestanuli jste jako Izrael před tím, co je hmatatelné: před ‚hořícím ohněm, temnotou, mrákotou, bouří,
#12:19 zvukem polnice a před hlasem Božích slov‘, při němž posluchači prosili, aby toho byli ušetřeni.
#12:20 Nemohli totiž snést slova výstrahy: ‚Dotkne-li se i jen zvíře té hory, bude ukamenováno.‘
#12:21 Pohled na to, co viděli, byl tak hrozný, že Mojžíš řekl: ‚Třesu se hrůzou a děsem.‘
#12:22 Vy stojíte před horou Siónem a městem Boha živého, nebeským Jeruzalémem, před nesčetným zástupem andělů
#12:23 a slavnostním shromážděním církve prvorozených, jejichž jména jsou zapsána v nebi, a před Bohem, soudcem všech, a před zesnulými spravedlivými, kteří již dosáhli cíle,
#12:24 a před Ježíšem, prostředníkem nové smlouvy, a před jeho krví, která nás očišťuje, neboť volá naléhavěji než krev Ábelova.
#12:25 Varujte se tedy odmítnout toho, kdo k vám mluví. Jestliže neunikli trestu ti, kdo odmítli tlumočníka Božích příkazů na zemi, tím spíše neunikneme my, odvrátíme-li se od toho, který mluví z nebe.
#12:26 Jeho hlas tehdy zatřásl zemí, nyní však slibuje: ‚Ještě jednou otřesu‘ nejen ‚zemí‘, nýbrž i ‚nebem‘.
#12:27 Těmito slovy naznačuje, že otřese vším stvořením a promění je, aby zůstalo jen to, co je neotřesitelné.
#12:28 Buďme vděčni za to, že dostáváme neotřesitelné království, a služme proto Bohu tak, jak se jemu líbí, s bázní a úctou.
#12:29 Vždyť ‚náš Bůh je oheň stravující‘. 
#13:1 Bratrská láska ať trvá;
#13:2 s láskou přijímejte i ty, kdo přicházejí odjinud - tak někteří, aniž to tušili, měli za hosty anděly.
#13:3 Pamatujte na vězně, jako byste byli uvězněni s nimi; pamatujte na ty, kdo trpí, vždyť i vás může potkat utrpení.
#13:4 Manželství ať mají všichni v úctě a manželé ať jsou si věrni, neboť neřestné a nevěrné bude soudit Bůh.
#13:5 Nedejte se vést láskou k penězům; buďte spokojeni s tím, co máte. Vždyť Bůh řekl: ‚Nikdy tě neopustím a nikdy se tě nezřeknu.‘
#13:6 Proto smíme říkat s důvěrou: ‚Pán při mně stojí, nebudu se bát. Co mi může udělat člověk?‘
#13:7 Mějte v paměti ty, kteří vás vedli a kázali vám slovo Boží. Myslete na to, jak dovršili svůj život, a následujte je ve víře!
#13:8 Ježíš Kristus je tentýž včera i dnes i na věky.
#13:9 Nedejte se strhnout všelijakými cizími naukami. Je dobré spolehnout se na milost a nikoli na předpisy o pokrmech; kdo je dodržoval, nic tím nezískal.
#13:10 Máme oltář, z něhož nemají právo jíst ti, kdo přinášejí oběti ve stánku.
#13:11 Vždyť těla zvířat, jejichž krev vnáší velekněz do svatyně jako oběť za hřích, spalují se za hradbami.
#13:12 Proto také Ježíš trpěl venku za branou, aby posvětil lid svou vlastní krví.
#13:13 Vyjděme tedy s ním za hradby, nesouce jeho potupu.
#13:14 Vždyť zde nemáme trvalý domov, nýbrž vyhlížíme město, které přijde.
#13:15 Přinášejme tedy skrze Ježíše stále oběť chvály Bohu; naše rty nechť vyznávají jeho jméno.
#13:16 Nezapomínejme také na dobročinnost a štědrost, takové oběti se Bohu líbí.
#13:17 Poslouchejte ty, kteří vás vedou, a podřizujte se jim, protože oni bdí nad vámi a budou se za vás zodpovídat. Kéž to mohou činit s radostí, a ne s nářkem; to by vám nebylo na prospěch.
#13:18 Modlete se za nás. Jsme sice jisti svým dobrým svědomím, neboť se snažíme jednat tak, aby nám nemohli nic vytknout;
#13:19 a přece vás důtklivě prosím, modlete se za mne, abych vám byl co nejdříve zase navrácen.
#13:20 A Bůh pokoje, který pro krev stvrzující věčnou smlouvu vyvedl z mrtvých velikého pastýře ovcí, našeho Pána Ježíše,
#13:21 nechť vás posílí ve všem dobrém, abyste plnili jeho vůli; on v nás působí to, co se mu líbí, skrze Ježíše Krista. Jemu buď sláva na věky věků! Amen.
#13:22 Prosím vás, bratří, přijměte toto slovo napomenutí; píšu vám jen krátce.
#13:23 Sděluji vám, že náš bratr Timoteus je na svobodě. Přijde-li sem brzo, navštívím vás s ním.
#13:24 Pozdravujte všecky své představené i všecky bratry. Pozdravují vás bratří z Itálie.
#13:25 Milost se všemi vámi!  

\book{James}{Jas}
#1:1 Jakub, služebník Boží a Pána Ježíše Krista, posílá pozdrav dvanácti pokolením v diaspoře.
#1:2 Mějte z toho jen radost, moji bratří, když na vás přicházejí rozličné zkoušky.
#1:3 Vždyť víte, že osvědčí-li se v nich vaše víra, povede to k vytrvalosti.
#1:4 A vytrvalost ať je dovršena skutkem, abyste byli dokonalí a neporušení, prosti všech nedostatků.
#1:5 Má-li kdo z vás nedostatek moudrosti, ať prosí Boha, který dává všem bez výhrad a bez výčitek, a bude mu dána.
#1:6 Nechť však prosí s důvěrou a nic nepochybuje. Kdo pochybuje, je podoben mořské vlně, hnané a zmítané vichřicí.
#1:7 Ať si takový člověk nemyslí, že od Pána něco dostane;
#1:8 je to muž rozpolcený, nestálý ve všem, co činí.
#1:9 Bratr v nízkém postavení ať s chloubou myslí na své vyvýšení
#1:10 a bohatý ať myslí na své ponížení - vždyť pomine jako květ trávy:
#1:11 vzejde slunce a svým žárem spálí trávu, květ opadne a jeho krása zajde. Tak i boháč se vším svým shonem vezme za své.
#1:12 Blahoslavený člověk, který obstojí ve zkoušce; když se osvědčí, dostane vavřín života, jejž Pán zaslíbil těm, kdo ho milují.
#1:13 Kdo prochází zkouškou, ať neříká, že ho pokouší Pán. Bůh nemůže být pokoušen ke zlému a sám také nikoho nepokouší.
#1:14 Každý, kdo je v pokušení, je sváděn a váben svou vlastní žádostivostí.
#1:15 Žádostivost pak počne a porodí hřích, a dokonaný hřích plodí smrt.
#1:16 Neklamte sami sebe, milovaní bratří!
#1:17 ‚Každý dobrý dar a každé dokonalé obdarování‘ je shůry, sestupuje od Otce nebeských světel. U něho není proměny ani střídání světla a stínu.
#1:18 Z jeho rozhodnutí jsme se znovu zrodili slovem pravdy, abychom byli jakoby první sklizní jeho stvoření.
#1:19 Pamatujte si, moji milovaní bratří: každý člověk ať je rychlý k naslouchání, ale pomalý k mluvení, pomalý k hněvu;
#1:20 vždyť lidským hněvem spravedlnost Boží neprosadíš.
#1:21 A proto odstraňte veškerou špínu a přemíru špatnosti a v tichosti přijměte zaseté slovo, které má moc spasit vaše duše.
#1:22 Podle slova však také jednejte, nebuďte jen posluchači - to byste klamali sami sebe!
#1:23 Vždyť kdo slovo jen slyší a nejedná podle něho, ten se podobá muži, který v zrcadle pozoruje svůj vzhled;
#1:24 podívá se na sebe, odejde a hned zapomene, jak vypadá.
#1:25 Kdo se však zahledí do dokonalého zákona svobody a vytrvá, takže není zapomnětlivý posluchač, nýbrž také jedná, ten bude blahoslavený pro své skutky.
#1:26 Domnívá-li se kdo, že je zbožný, a přitom nedrží na uzdě svůj jazyk, klame tím sám sebe a jeho zbožnost je marná.
#1:27 Pravá a čistá zbožnost před Bohem a Otcem znamená pamatovat na vdovy a sirotky v jejich soužení a chránit se před poskvrnou světa. 
#2:1 Bratří moji, jestliže věříte v Ježíše Krista, našeho Pána slávy, nesmíte dělat rozdíly mezi lidmi.
#2:2 Do vašeho shromáždění přijde třeba muž se zlatým prstenem a v nádherném oděvu. Přijde také chudák v ošumělých šatech.
#2:3 A vy věnujete svou pozornost tomu v nádherném oděvu a řeknete mu: „Posaď se na tomto čestném místě,“ kdežto chudému řeknete: „Ty postůj tamhle, nebo si sedni tady na zem.“
#2:4 Neděláte tím mezi sebou rozdíly a nestali se z vás soudci, kteří posuzují nesprávně?
#2:5 Poslyšte, moji milovaní bratří: Cožpak Bůh nevyvolil chudáky tohoto světa, aby byli bohatí ve víře a stali se dědici království, jež zaslíbil těm, kdo ho milují?
#2:6 Vy jste však ponížili chudého. Cožpak vás bohatí neutiskují? Nevláčejí vás před soudy?
#2:7 Nemluví právě oni s pohrdáním o slavném jménu, které bylo nad vámi vysloveno?
#2:8 Jestliže tedy zachováváte královský zákon, jak je napsán v Písmu: ‚Milovati budeš bližního svého jako sám sebe,‘ dobře činíte.
#2:9 Jestliže však někomu straníte, dopouštíte se hříchu a zákon vás usvědčuje z přestoupení.
#2:10 Kdo by totiž zachoval celý zákon, a jen v jednom přikázání klopýtl, provinil se proti všem.
#2:11 Vždyť ten, kdo řekl: ‚Nezcizoložíš,‘ řekl také: ‚Nezabiješ.‘ Jestliže necizoložíš, ale zabíjíš, přestupuješ zákon.
#2:12 Mluvte a jednejte jako ti, kteří mají být souzeni zákonem svobody.
#2:13 Na Božím soudu není milosrdenství pro toho, kdo neprokázal milosrdenství. Ale milosrdenství vítězí nad soudem.
#2:14 Co je platné, moji bratří, když někdo říká, že má víru, ale přitom nemá skutky? Může ho snad ta víra spasit?
#2:15 Kdyby některý bratr nebo sestra byli bez šatů a neměli jídlo ani na den,
#2:16 a někdo z vás by jim řekl: „Buďte s Bohem - ať vám není zima a nemáte hlad“, ale nedali byste jim, co potřebují pro své tělo, co by to bylo platné?
#2:17 Stejně tak i víra, není-li spojena se skutky, je sama o sobě mrtvá.
#2:18 Někdo však řekne: „Jeden má víru a druhý má skutky.“ Tomu odpovím: Ukaž mi tu svou víru bez skutků a já ti ukážu svou víru na skutcích.
#2:19 Ty věříš, že je jeden Bůh. To je správné. I démoni tomu věří, ale hrozí se toho.
#2:20 Neuznáš, ty nechápavý člověče, že víra bez skutků není k ničemu?
#2:21 Což nebyl náš otec Abraham ospravedlněn ze skutků, když položil na oltář svého syna Izáka?
#2:22 Nevidíš, že víra působila spolu s jeho skutky a že ve skutcích došla víra dokonalosti?
#2:23 Tak se naplnilo Písmo: ‚Uvěřil Abraham Bohu, a bylo mu to počítáno za spravedlnost‘ a byl nazván ‚přítelem Božím‘.
#2:24 Vidíte, že ze skutků je člověk ospravedlněn, a ne pouze z víry!
#2:25 Což nebyla i nevěstka Rachab podobně ospravedlněna ze skutků, když přijala posly a propustila je jinou cestou? -
#2:26 Jako je tělo bez ducha mrtvé, tak je mrtvá i víra bez skutků. 
#3:1 Nechtějte všichni učit druhé, moji bratří: vždyť víte, že my, kteří učíme, budeme souzeni s větší přísností.
#3:2 Všichni přece mnoho chybujeme. Kdo nechybuje slovem, je dokonalý muž a dovede držet na uzdě celé své tělo.
#3:3 Dáváme-li koňům do huby udidlo, aby nás poslouchali, můžeme tak řídit celé jejich tělo.
#3:4 Nebo si představte lodi: Jsou tak veliké a jsou hnány prudkými větry, ale malé kormidlo je řídí, kamkoli kormidelník chce.
#3:5 Tak i jazyk je malý úd, ale může se chlubit velkými věcmi. Považte, jak malý oheň může zapálit veliký les!
#3:6 I jazyk je oheň. Je to svět zla mezi našimi údy, poskvrňuje celé tělo a ničí celý náš život, sám podpalován pekelným plamenem.
#3:7 Všechny druhy zvířat i ptáků, plazů i mořských živočichů mohou být a jsou kroceny člověkem,
#3:8 ale jazyk neumí zkrotit nikdo z lidí. Je to zlo, které si nedá pokoj, plné smrtonosného jedu.
#3:9 Jím chválíme Pána a Otce, jím však také proklínáme lidi, kteří byli stvořeni k Boží podobě.
#3:10 Z týchž úst vychází žehnání i proklínání. Tak tomu být nemá, bratří moji.
#3:11 Což pramen z téhož zřídla vydává vodu sladkou i hořkou?
#3:12 Což může, bratří, fíkovník nést olivy nebo réva fíky? Právě tak nemůže slaný pramen dávat sladkou vodu.
#3:13 Kdo je mezi vámi moudrý a rozumný? Ať ukáže své skutky dobrým způsobem života, v tichosti, kterou dává moudrost.
#3:14 Máte-li však v srdci hořkou závist a svárlivost, nechlubte se moudrostí a nelžete proti pravdě.
#3:15 To přece není moudrost přicházející shůry, ale přízemní, živočišná, ďábelská.
#3:16 Vždyť kde je závist a svárlivost, tam je zmatek a kdejaká špatnost.
#3:17 Moudrost shůry je především čistá, dále mírumilovná, ohleduplná, ochotná dát se přesvědčit, plná slitování a dobrého ovoce, bez předsudků a bez přetvářky.
#3:18 Ovoce spravedlnosti sklidí u Boha ti, kdo rozsévají pokoj. 
#4:1 Odkud jsou mezi vámi boje a sváry? Nejsou to právě vášně, které vás vedou do bojů?
#4:2 Chcete mít, ale nemáte. Ubíjíte a nevražíte, ale ničeho nemůžete dosáhnout. Sváříte se a bojujete - a nic nemáte, protože neprosíte.
#4:3 Prosíte sice, ale nedostáváte, protože prosíte nedobře: jde vám o vaše vášně.
#4:4 Proradná stvoření! Což nevíte, že přátelství se světem je nepřátelství s Bohem? Kdo tedy chce být přítelem světa, stává se nepřítelem Božím.
#4:5 Či myslíte, že nadarmo je psáno: ‚Bůh žárlivě touží po duchu, kterého do nás vložil‘?
#4:6 Mocnější však je milost, kterou dává. Proto je řečeno: ‚Bůh se staví proti pyšným, ale pokorným dává milost.‘
#4:7 Podřiďte se tedy Bohu. Vzepřete se ďáblu a uteče od vás, přibližte se k Bohu a přiblíží se k vám.
#4:8 Umyjte si ruce, hříšníci, a očisťte svá srdce, lidé dvojí tváře!
#4:9 Bědujte, naříkejte a plačte! Váš smích ať se obrátí v pláč a vaše radost v žal.
#4:10 Pokořte se před Pánem, a on vás povýší.
#4:11 Bratří, nesnižujte jeden druhého. Kdo snižuje nebo odsuzuje bratra, snižuje a odsuzuje zákon. Jestliže však odsuzuješ zákon, neplníš zákon, nýbrž stavíš se nad něj jako soudce.
#4:12 Jeden je zákonodárce i soudce; on může zachránit i zahubit. Ale kdo jsi ty, že odsuzuješ bližního?
#4:13 A nyní vy, kteří říkáte: „Dnes nebo zítra půjdeme do toho a toho města, zůstaneme tam rok, budeme obchodovat a vydělávat“ -
#4:14 vy přece nevíte, co bude zítra! Co je váš život? Jste jako pára, která se na okamžik ukáže a potom zmizí!
#4:15 Raději byste měli říkat: „Bude-li Pán chtít, budeme naživu a uděláme to neb ono.“
#4:16 Vy se však vychloubáte a chvástáte.
#4:17 Každá taková chlouba je zlá. Kdo ví, co je činit dobré, a nečiní, má hřích. 
#5:1 A nyní, vy boháči, plačte a naříkejte nad pohromami, které na vás přicházejí.
#5:2 Vaše bohatství shnilo, vaše šatstvo je moly rozežráno,
#5:3 vaše zlato a stříbro zrezavělo, a ten rez bude svědčit proti vám a stráví vaše tělo jako oheň. Nashromáždili jste si poklady - pro konec dnů!
#5:4 Hle, mzda dělníků, kteří žali vaše pole, a vy jste jim ji upřeli, volá do nebes, a křik ženců pronikl ke sluchu Hospodina zástupů.
#5:5 Žili jste rozmařile a hýřili jste na zemi, vykrmili jste se - pro den porážky!
#5:6 Odsoudili jste a zahubili jste spravedlivého; a on se proti vám nestaví.
#5:7 Buďte tedy trpěliví, bratří, až do příchodu Páně. Pohleďte, jak rolník čeká trpělivě na drahocennou úrodu země, dokud se nedočká podzimního i jarního deště.
#5:8 I vy tedy trpělivě čekejte, posilněte svá srdce, vždyť příchod Páně je blízko.
#5:9 Nestěžujte si jeden na druhého, bratří, abyste nebyli odsouzeni. Hle, soudce stojí přede dveřmi!
#5:10 Za příklad trpělivosti v utrpení si, bratří, vezměte proroky, kteří mluvili ve jménu Páně.
#5:11 Hle, ‚blahoslavíme ty, kteří vytrvali‘. Slyšeli jste o vytrvalosti Jobově a víte, k čemu ho Pán nakonec přivedl. Vždyť ‚Pán je plný soucitu a slitování‘.
#5:12 Především nepřísahejte, bratří moji, ani při nebi ani při zemi ani při ničem jiném. Vaše ‚ano‘ ať je vždy ‚ano‘ a ‚ne‘ ať je ‚ne‘, abyste nepropadli soudu.
#5:13 Vede se někomu z vás zle? Ať se modlí! Je někdo dobré mysli? Ať zpívá Pánu!
#5:14 Je někdo z vás nemocen? Ať zavolá starší církve, ti ať se nad ním modlí a potírají ho olejem ve jménu Páně.
#5:15 Modlitba víry zachrání nemocného, Pán jej pozdvihne, a dopustil-li se hříchů, bude mu odpuštěno.
#5:16 Vyznávejte hříchy jeden druhému a modlete se jeden za druhého, abyste byli uzdraveni. Velkou moc má vroucí modlitba spravedlivého.
#5:17 Eliáš byl člověk jako my, a když se naléhavě modlil, aby nepršelo, nezapršelo v zemi po tři roky a šest měsíců.
#5:18 A opět se modlil, a nebe dalo déšť a země přinesla úrodu.
#5:19 Bratří moji, zbloudí-li kdo od pravdy a druhý ho přivede nazpět,
#5:20 vězte, že ten, kdo odvrátí hříšníka od bludné cesty, zachrání jeho duši od smrti a přikryje množství hříchů.  

\book{I Peter}{1Pet}
#1:1 Petr, apoštol Ježíše Krista, vyvoleným, kteří přebývají jako cizinci v diaspoře v Pontu, Galacii, Kappadokii, Asii a Bithynii
#1:2 a byli předem vyhlédnuti od Boha Otce a posvěceni Duchem, aby se poslušně odevzdali Ježíši Kristu a byli očištěni pokropením jeho krví: Milost vám a pokoj v hojnosti.
#1:3 Veleben buď Bůh a Otec Pána našeho Ježíše Krista, neboť nám ze svého velikého milosrdenství dal vzkříšením Ježíše Krista nově se narodit k živé naději.
#1:4 Dědictví nehynoucí, neposkvrněné a nevadnoucí je připraveno pro vás v nebesích
#1:5 a Boží moc vás skrze víru střeží ke spasení, které bude odhaleno v posledním čase.
#1:6 Z toho se radujte, i když snad máte ještě nakrátko projít zármutkem rozmanitých zkoušek,
#1:7 aby se pravost vaší víry, mnohem drahocennější než pomíjející zlato, jež přece též bývá zkoušeno ohněm, prokázala k vaší chvále, slávě a cti v den, kdy se zjeví Ježíš Kristus.
#1:8 Ač jste ho neviděli, milujete ho; ač ho ani nyní nevidíte, přec v něho věříte a jásáte nevýslovnou, vznešenou radostí,
#1:9 a tak docházíte cíle víry, spasení duší.
#1:10 Toto spasení hledali a po něm se ptali proroci, kteří prorokovali o milosti, která je vám připravena;
#1:11 zkoumali, na který čas a na jaké okolnosti ukazuje duch Kristův v nich přítomný, když předem svědčí o utrpeních, jež má Kristus vytrpět, i o veliké slávě, která potom přijde.
#1:12 Bylo jim zjeveno, že tím neslouží sami sobě, nýbrž vám; ti, kdo vám přinesli evangelium v moci Ducha svatého, seslaného z nebes, zvěstovali vám nyní toto spasení, které i andělé touží spatřit.
#1:13 Odhodlaně se tedy připravte ve své mysli, buďte střízliví a celou svou naději upněte k milosti, která k vám přichází ve zjevení Ježíše Krista.
#1:14 Jako poslušné děti nedejte se opanovat žádostmi, které vás ovládaly předtím, v době vaší nevědomosti;
#1:15 ale jako je svatý ten, který vás povolal, buďte i vy svatí v celém způsobu života.
#1:16 Vždyť je psáno: ‚Svatí buďte, neboť já jsem svatý.‘
#1:17 Jestliže ‚vzýváte jako Otce‘ toho, kdo nestranně soudí každého podle jeho činů, v bázni před ním žijte dny svého pozemského života.
#1:18 Víte přece, že jste z prázdnoty svého způsobu života, jak jste jej přejali od otců, nebyli vykoupeni pomíjitelnými věcmi, stříbrem nebo zlatem,
#1:19 nýbrž převzácnou krví Kristovou. On jako beránek bez vady a bez poskvrny
#1:20 byl k tomu předem vyhlédnut před stvořením světa a přišel kvůli vám na konci časů.
#1:21 Skrze něho věříte v Boha, který ho vzkřísil z mrtvých a dal mu slávu, takže se vaše víra i naděje upíná k Bohu.
#1:22 Když jste nyní přijali pravdu a tak očistili své duše k nepředstírané bratrské lásce, z upřímného srdce vytrvale se navzájem milujte.
#1:23 Vždyť jste se znovu narodili, nikoli z pomíjitelného semene, nýbrž z nepomíjitelného, skrze živé a věčné slovo Boží.
#1:24 Neboť ‚všichni lidé jako tráva a všechna jejich krása jako květ trávy. Uschne tráva, květ opadne, ale slovo Hospodinovo zůstává na věky‘ -
#1:25 to je to slovo, které vám bylo zvěstováno v evangeliu. 
#2:1 Odhoďte tedy všechnu špatnost, každou lest, přetvářku, závist, jakékoliv pomlouvání
#2:2 a jako novorozené děti mějte touhu jen po nefalšovaném duchovním mléku, abyste jím rostli ke spasení;
#2:3 vždyť jste ‚okusili, že Pán je dobrý!‘
#2:4 Přicházejte tedy k němu, kameni živému, jenž od lidí byl zavržen, ale před Bohem je ‚vyvolený a vzácný‘.
#2:5 I vy buďte živými kameny, z nichž se staví duchovní dům, abyste byli svatým kněžstvem a přinášeli duchovní oběti, milé Bohu pro Ježíše Krista.
#2:6 Neboť v Písmu stojí: ‚Hle, kladu na Siónu kámen vyvolený, úhelný, vzácný; kdo v něj věří, nebude zahanben.‘
#2:7 Vám, kteří věříte, je vzácný, ale nevěřícím je to ‚kámen, který stavitelé zavrhli; ten se stal kamenem úhelným‘,
#2:8 ale i ‚kamenem úrazu a skálou pádu‘. Oni přicházejí k pádu svým vzdorem proti slovu - k tomu také byli určeni.
#2:9 Vy však jste ‚rod vyvolený, královské kněžstvo, národ svatý, lid náležející Bohu‘, abyste hlásali mocné skutky toho, kdo vás povolal ze tmy do svého podivuhodného světla.
#2:10 Kdysi jste ‚vůbec nebyli lid‘, nyní však jste lid Boží; pro vás ‚nebylo slitování‘, ale nyní jste došli slitování.
#2:11 Milovaní, v tomto světě jste cizinci bez domovského práva. Prosím vás proto, zdržujte se sobeckých vášní, které vedou boj proti duši,
#2:12 a žijte vzorně mezi pohany; tak aby ti, kdo vás osočují jako zločince, prohlédli a za vaše dobré činy vzdali chválu Bohu ‚v den navštívení‘.
#2:13 Podřiďte se kvůli Pánu každému lidskému zřízení - ať už králi jako svrchovanému vládci,
#2:14 ať už místodržícím jako těm, které on posílá trestat zločince a odměňovat ty, kteří jednají dobře.
#2:15 Taková je přece Boží vůle, abyste dobrým jednáním umlčovali nevědomost nerozumných lidí.
#2:16 Jste svobodni, ale ne jako ti, jimž svoboda slouží za plášť nepravosti, nýbrž jako služebníci Boží.
#2:17 Ke všem lidem mějte úctu, bratrstvo milujte, Boha se bojte, krále ctěte.
#2:18 Služebníci, podřizujte se ve vší bázni pánům, nejen dobrým a mírným, nýbrž i tvrdým.
#2:19 V tom je totiž milost, když někdo pro svědomí odpovědné Bohu snáší bolest a trpí nevinně.
#2:20 Jaká však sláva, jestliže budete trpělivě snášet rány za to, že hřešíte? Ale budete-li trpělivě snášet soužení, ač jednáte dobře, to je milost před Bohem.
#2:21 K tomu jste přece byli povoláni; vždyť i Kristus trpěl za vás a zanechal vám tak příklad, abyste šli v jeho šlépějích.
#2:22 On ‚hříchu neučinil a v jeho ústech nebyla nalezena lest‘.
#2:23 Když mu spílali, neodplácel spíláním; když trpěl, nehrozil, ale vkládal vše do rukou toho, jenž soudí spravedlivě.
#2:24 On ‚na svém těle vzal naše hříchy‘ na kříž, abychom zemřeli hříchům a byli živi spravedlnosti.
#2:25 ‚Jeho rány vás uzdravily.‘ Vždyť jste ‚bloudili jako ovce‘, ale nyní jste byli obráceni k pastýři a strážci svých duší. 
#3:1 Stejně i vy ženy, podřizujte se svým mužům; i když se někteří z nich vzpírají Božímu slovu, můžete je beze slov získat svým jednáním,
#3:2 když uvidí váš čistý život v bázni Boží.
#3:3 Pro vás se nehodí vnější ozdoba - splétat si vlasy, ověšovat se zlatem, střídat oděvy -
#3:4 nýbrž to, co je skryto v srdci a co je nepomíjitelné: tichý a pokojný duch; to je před Bohem převzácné.
#3:5 Tak se kdysi zdobily svaté ženy, které doufaly v Boha. Podřizovaly se svým mužům,
#3:6 tak jako Sára poslouchala Abrahama a ‚nazvala jej pánem‘. Vy jste jejími dcerami, jednáte-li dobře a nedáte se ničím zastrašit.
#3:7 Stejně i muži: když žijete se svými ženami, mějte pro ně porozumění, že jsou slabší; a prokazujte jim úctu, protože jsou spolu s vámi dědičkami daru života. Tak vašim modlitbám nebude nic překážet.
#3:8 Nakonec pak: Všichni buďte jedné mysli, soucitní, plní bratrské lásky, milosrdní a pokorní,
#3:9 neodplácejte zlým za zlé ani urážkou za urážku, naopak žehnejte; vždyť jste byli povoláni k tomu, abyste se stali dědici požehnání.
#3:10 ‚Chceš-li milovat život a vidět dobré dny, zdržuj jazyk od zlého a rty od lstivých slov,
#3:11 odvrať se od zlého a čiň dobré, hledej pokoj a usiluj o něj.
#3:12 Vždyť oči Páně hledí na spravedlivé a jeho uši jsou otevřeny jejich prosbám, ale tvář Páně je proti těm, kteří činí zlo.‘
#3:13 Kdo vám ublíží, budete-li horlit pro dobro?
#3:14 Ale i kdybyste pro spravedlnost měli trpět, jste blahoslaveni. ‚Strach z nich ať vás neděsí ani nezviklá
#3:15 a Pán, Kristus, budiž svatý‘ ve vašich srdcích. Buďte vždy připraveni dát odpověď každému, kdo by vás vyslýchal o naději, kterou máte,
#3:16 ale čiňte to s tichostí a s uctivostí. Když jste vystaveni pomluvám, zachovávejte si dobré svědomí, aby ti, kteří hanobí váš dobrý způsob života v Kristu, byli zahanbeni.
#3:17 Je přece lépe, abyste trpěli za dobré jednání, bude-li to snad vůle Boží, než za zlé činy.
#3:18 Vždyť i Kristus dal svůj život jednou provždy za hříchy, spravedlivý za nespravedlivé, aby nás přivedl k Bohu. Byl usmrcen v těle, ale obživen Duchem.
#3:19 Tehdy také přišel vyhlásit zvěst duchům ve vězení,
#3:20 kteří neuposlechli kdysi ve dnech Noémových. Tenkrát Boží shovívavost vyčkávala s trestem, pokud se stavěl koráb, v němž bylo z vody zachráněno jenom osm lidí.
#3:21 To je předobraz křtu, který nyní zachraňuje vás. Nejde v něm zajisté o odstranění tělesné špíny, nýbrž o dobré svědomí, k němuž se před Bohem zavazujeme - na základě vzkříšení Ježíše Krista,
#3:22 jenž jest na pravici Boží, když vstoupil do nebe a podřídil si anděly a vlády a mocnosti. 
#4:1 Když tedy Kristus podstoupil tělesné utrpení, i vy se vyzbrojte stejnou myšlenkou: Ten, kdo trpěl v těle, skoncoval s hříchem.
#4:2 Proto i vy ve zbývajícím čase života nebuďte oddáni lidským vášním, ale vůli Boží.
#4:3 Dost dlouho už jste dělali to, v čem si libují pohané: žili jste v nevázanosti, vášních, v opilství, v hodech, pitkách a v hanebném modlářství.
#4:4 Když se již spolu s nimi nevrháte do téhož proudu prostopášnosti, dráždí je to a urážejí vás.
#4:5 Však vydají počet tomu, kdo je připraven soudit živé i mrtvé.
#4:6 Proto bylo evangelium zvěstováno i mrtvým, aby byli u Boha živi v Duchu, ačkoliv byli za svého života u lidí odsouzeni.
#4:7 Konec všech věcí je blízko. Žijte proto rozumně a střízlivě, abyste byli pohotoví k modlitbám.
#4:8 Především mějte vytrvalou lásku jedni k druhým; vždyť láska přikryje množství hříchů.
#4:9 Buďte jedni k druhým pohostinní a nestěžujte si na to!
#4:10 Každý ať slouží druhým tím darem milosti, který přijal; tak budete dobrými správci milosti Boží v její rozmanitosti.
#4:11 Kdo káže, ať zvěstuje slovo Boží. Kdo slouží, ať to činí ze síly, kterou dává Bůh - tak aby se všecko dělo k oslavě Boží skrze Ježíše Krista. Jemu buď sláva i moc na věky věků. Amen.
#4:12 Moji milovaní, nebuďte zmateni výhní zkoušky, která na vás přišla, jako by se s vámi dělo něco neobvyklého,
#4:13 ale radujte se, když máte podíl na Kristově utrpení, abyste se ještě více radovali, až se zjeví jeho sláva.
#4:14 Jestliže jste hanobeni pro jméno Kristovo, blaze vám, neboť na vás spočívá Duch slávy, Duch Boží.
#4:15 Ale ať nikdo z vás netrpí za vraždu, za krádež nebo jiný zlý čin anebo za intriky.
#4:16 Kdo však trpí za to, že je křesťan, ať se nestydí, ale slaví Boha, že smí nosit toto jméno.
#4:17 Přišel totiž čas, aby soud začal od domu Božího. Jestliže začíná od vás, jaký bude konec těch, kteří se Božímu evangeliu vzpírají?
#4:18 ‚Jestliže i spravedlivý bude stěží zachráněn, kde se ocitne bezbožný a hříšný?‘
#4:19 A tak ti, kteří trpí podle vůle Boží, ať svěří své duše věrnému Stvořiteli a činí dobré. 
#5:1 Starší mezi vámi napomínám, sám také starší, svědek utrpení Kristových i účastník slávy, která se má v budoucnu zjevit:
#5:2 Starejte se jako pastýři o Boží stádce u vás, ne z donucení, ale dobrovolně, jak to Bůh žádá, ne z nízké zištnosti, ale s horlivou ochotou,
#5:3 ne jako páni nad těmi, kdo jsou vám svěřeni, ale buďte jim příkladem.
#5:4 Když se pak ukáže nejvyšší pastýř, dostane se vám nevadnoucího vavřínu slávy.
#5:5 Stejně se i vy mladší podřizujte starším. Všichni se oblecte v pokoru jeden vůči druhému, neboť ‚Bůh se staví proti pyšným, ale pokorným dává milost‘.
#5:6 Pokořte se tedy pod mocnou ruku Boží, aby vás povýšil v ustanovený čas.
#5:7 Všechnu ‚svou starost vložte na něj‘, neboť mu na vás záleží.
#5:8 Buďte střízliví! Buďte bdělí! Váš protivník, ďábel, obchází jako ‚lev řvoucí‘ a hledá, koho by pohltil.
#5:9 Vzepřete se mu, zakotveni ve víře, a pamatujte, že vaši bratří všude ve světě procházejí týmž utrpením jako vy.
#5:10 A Bůh veškeré milosti, který vás povolal ke své věčné slávě v Kristu, po krátkém utrpení vás obnoví, utvrdí, posílí a postaví na pevný základ.
#5:11 Jemu náleží panství na věky věků! Amen.
#5:12 Prostřednictvím Silvána, kterého mám za věrného bratra, vám toto krátce píšu, abych vás povzbudil a dosvědčil vám, že taková je pravá milost Boží; v ní stůjte.
#5:13 Pozdravuje vás vaše spoluvyvolená, která je v Babylónu, a Marek, můj syn.
#5:14 Pozdravte jedni druhé políbením lásky. Pokoj všem vám, kteří jste v Kristu.  

\book{II Peter}{2Pet}
#1:1 Šimon Petr, služebník a apoštol Ježíše Krista, těm, kdo dosáhli stejně vzácné víry jako my díky spravedlnosti našeho Boha a Spasitele Ježíše Krista.
#1:2 Milost a pokoj ať se vám rozhojní poznáním Boha a Ježíše, našeho Pána.
#1:3 Všecko, čeho je třeba k zbožnému životu, darovala nám jeho božská moc, když jsme poznali toho, který nás povolal vlastní slávou a mocnými činy.
#1:4 Tím nám daroval vzácná a převeliká zaslíbení, abyste se tak stali účastnými božské přirozenosti a unikli zhoubě, do níž svět žene jeho zvrácená touha.
#1:5 Proto také vynaložte všecku snahu na to, abyste ke své víře připojili ctnost, k ctnosti poznání,
#1:6 k poznání zdrženlivost, ke zdrženlivosti trpělivost, k trpělivosti zbožnost,
#1:7 ke zbožnosti bratrskou náklonnost a k bratrské náklonnosti lásku.
#1:8 Máte-li tyto vlastnosti a rozhojňují-li se ve vás, nezůstanete v poznání našeho Pána Ježíše Krista nečinní a bez ovoce.
#1:9 Komu však scházejí, je slepý, krátkozraký a zapomněl na to, že byl očištěn od svých starých hříchů.
#1:10 Proto se, bratří, tím více snažte upevňovat své povolání a vyvolení. Budete-li to činit, nikdy neklopýtnete.
#1:11 Tak se vám široce otevře přístup do věčného království našeho Pána a Spasitele Ježíše Krista.
#1:12 Proto vám hodlám ty věci stále připomínat, ačkoliv o nich víte a jste utvrzeni v pravdě, kterou jste přijali.
#1:13 Ale považuji za správné probouzet vás napomínáním, pokud přebývám v tomto těle;
#1:14 vím totiž, že je budu muset brzo opustit, jak mi to dal poznat náš Pán Ježíš Kristus.
#1:15 Vynasnažím se, abyste si mohli tyto věci vždycky připomínat i po mém odchodu.
#1:16 Nedali jsme se vést vymyšlenými bájemi, ale zvěstovali jsme vám slavný příchod našeho Pána Ježíše Krista jako očití svědkové jeho velebnosti.
#1:17 On přijal od Boha Otce čest i slávu, když k němu ze svrchované slávy zazněl hlas: Toto jest můj milovaný Syn, v něm jsem nalezl zalíbení.
#1:18 A tento hlas, který vyšel z nebe, jsme my slyšeli, když jsme s ním byli na svaté hoře.
#1:19 Tím se nám potvrzuje prorocké slovo, a činíte dobře, že se ho držíte; je jako svíce, svítící v temném místě, dokud se nerozbřeskne den a jitřenka vám nevzejde v srdci.
#1:20 Toho si buďte především vědomi, že žádné proroctví v Písmu nevzniká z vlastního pochopení skutečnosti.
#1:21 Nikdy totiž nebylo vyřčeno proroctví z lidské vůle, nýbrž z popudu Ducha svatého mluvili lidé, poslaní od Boha. 
#2:1 V Božím lidu bývali ovšem i lživí proroci; tak i mezi vámi budou lživí učitelé, kteří budou záludně zavádět zhoubné nauky a budou popírat Panovníka, který je vykoupil. Tím na sebe uvedou náhlou zhoubu.
#2:2 A mnozí budou následovat jejich nezřízenost a cesta pravdy bude kvůli nim v opovržení.
#2:3 Ve své hrabivosti budou vám předkládat své výmysly, aby z vás těžili. Soud nad nimi je už připraven a jejich zhouba je blízká.
#2:4 Vždyť Bůh neušetřil ani anděly, kteří zhřešili, ale svrhl je do temné propasti podsvětí a dal je střežit, aby byli postaveni před soud.
#2:5 Ani starý svět neušetřil, nýbrž zachoval jen Noéma, kazatele spravedlnosti, spolu se sedmi jinými, když uvedl potopu na svět bezbožných.
#2:6 Také města Sodomu a Gomoru odsoudil k záhubě a obrátil v popel. Tím dal výstrahu budoucím bezbožníkům.
#2:7 Vysvobodil však spravedlivého Lota, sužovaného nezřízeným chováním těch zvrhlíků.
#2:8 Dokud ten spravedlivý přebýval mezi nimi, trápilo jeho spravedlivou duši, že musel den ze dne slyšet a vidět jejich nemravné skutky.
#2:9 Pán však dovede vytrhnout zbožné ze zkoušky, ale nespravedlivé uchovat pro trest v den soudu;
#2:10 a to především ty, kdo se svévolně ženou za poskvrňujícími vášněmi a pohrdají každou autoritou. Jsou to drzí opovážlivci; nechvějí se před nadpozemskými mocnostmi a rouhají se jim.
#2:11 Ani andělé, ačkoli jsou větší silou a mocí, nevynášejí nad těmi mocnostmi před Pánem zatracující soud.
#2:12 Ti lživí učitelé však jako nerozumná zvířata, určená od přírody za kořist a na porážku, rouhají se tomu, co neznají; zahynou ve své zkaženosti
#2:13 a dostanou odplatu za svou nepravost. Je jim požitkem vyhledávat i ve dne rozkoš; poskvrnění a hanební oddávají se prostopášnosti, když s vámi hodují.
#2:14 Oči mají plné cizoložství a nepřestávají hřešit, svádějí nepevné duše, srdce mají vycvičené v hrabivosti; jsou to synové prokletí.
#2:15 Opustili přímou cestu a zbloudili, když se dali na cestu Balaáma, syna Beorova, kterému se zalíbila mzda za nepravost,
#2:16 ale byl pokárán za své provinění. Němá oslice promluvila lidským hlasem, a tak zabránila prorokovu bláznovství.
#2:17 Takoví lidé jsou jako prameny bez vody, mračna hnaná bouří; je pro ně připravena nejčernější tma.
#2:18 Řeční prázdně a nabubřele a svádějí nezřízenými vášněmi ty, kdo se právě vymaňují ze života v klamu.
#2:19 Slibují jim svobodu, ale sami jsou služebníky záhuby. Co se člověka zmocní, tím je zotročen.
#2:20 Jestliže tedy ti, kdo poznáním Pána a Spasitele Ježíše Krista unikli poskvrnám světa, znovu se do nich zapletou a podlehnou jim, budou jejich konce horší než začátky.
#2:21 Bylo by pro ně lépe, kdyby vůbec nebyli poznali cestu spravedlnosti, než aby se po jejím poznání odvrátili od svatého přikázání, které jim bylo svěřeno.
#2:22 Přihodilo se jim to, co říká pravdivé přísloví: ‚Pes se vrátil k vlastnímu vývratku‘ a umytá svině se zase válí v bahništi. 
#3:1 To už je, milovaní, druhý dopis, který vám píšu. Tímto napomínáním chci probouzet vaše čisté smýšlení,
#3:2 abyste pamatovali na to, co předpověděli svatí proroci, i na to, co ustanovil Pán a Spasitel skrze vaše apoštoly.
#3:3 Především vám chci říci, že ke konci dnů přijdou posměvači, kteří žijí, jak se jim zachce,
#3:4 a budou se posmívat: „Kde je ten jeho zaslíbený příchod? Od té doby, co zesnuli otcové, všecko zůstává tak, jak to bylo od počátku stvoření.“
#3:5 Těm, kdo toto tvrdí, zůstává utajeno, že dávná nebesa i země byly vyvolány slovem Božím z vody a před vodou chráněny.
#3:6 Vodou byl také tehdejší svět zatopen a zahynul.
#3:7 Týmž slovem jsou udržována nynější nebesa a země, dokud nebudou zničena ohněm; Bůh je ponechal jen do dne soudu a záhuby bezbožných lidí.
#3:8 Ale tato jedna věc kéž vám nezůstane skryta, milovaní, že jeden den je u Pána jako tisíc let a ‚tisíc let jako jeden den‘.
#3:9 Pán neotálí splnit svá zaslíbení, jak si to někteří vykládají, nýbrž má s námi trpělivost, protože si nepřeje, aby někdo zahynul, ale chce, aby všichni dospěli k pokání.
#3:10 Den Páně přijde jako přichází zloděj. Tehdy nebesa s rachotem zaniknou, vesmír se žárem roztaví a země se všemi lidskými činy bude postavena před soud.
#3:11 Když tedy se toto vše rozplyne, jak svatě a zbožně musíte žít vy,
#3:12 kteří dychtivě očekáváte příchod Božího dne! V něm se nebesa roztaví v ohni a živly se rozpustí žárem.
#3:13 Podle jeho slibu čekáme nové nebe a novou zemi, ve kterých přebývá spravedlnost.
#3:14 Proto, milovaní, očekáváte-li takové věci, snažte se, abyste byli čistí a bez poskvrny a mohli ten den očekávat beze strachu před Božím soudem.
#3:15 A vězte, že ve své trpělivosti vám Pán poskytuje čas ke spáse, jak vám napsal i náš milý bratr Pavel podle moudrosti, která mu byla dána.
#3:16 Mluvil tak o tom ve všech svých listech. Některá místa jsou v nich těžko srozumitelná a neučení a neutvrzení lidé je překrucují, jako i ostatní Písmo, k vlastní záhubě.
#3:17 Ale vy, milovaní, protože to víte předem, střezte se, abyste nebyli oklamáni svodem těch neodpovědných lidí a neodpadli od vlastního pevného základu.
#3:18 Kéž rostete v milosti a v poznání našeho Pána a Spasitele Ježíše Krista. Jemu buď sláva nyní a až do dne věčnosti.  

\book{I John}{1John}
#1:1 Co bylo od počátku, co jsme slyšeli, co jsme na vlastní oči viděli, na co jsme hleděli a čeho se naše ruce dotýkaly, to zvěstujeme: Slovo života.
#1:2 Ten život byl zjeven, my jsme jej viděli, svědčíme o něm a zvěstujeme vám život věčný, který byl u Otce a nám byl zjeven.
#1:3 Co jsme viděli a slyšeli, zvěstujeme i vám, abyste se spolu s námi podíleli na společenství, které máme s Otcem a s jeho Synem Ježíšem Kristem.
#1:4 To píšeme, aby naše radost byla úplná.
#1:5 A toto je zvěst, kterou jsme od něho slyšeli a vám ji oznamujeme: že Bůh je světlo a není v něm nejmenší tmy.
#1:6 Říkáme-li, že s ním máme společenství, a přitom chodíme ve tmě, lžeme a nečiníme pravdu.
#1:7 Jestliže však chodíme v světle, jako on je v světle, máme společenství mezi sebou a krev Ježíše, jeho Syna, nás očišťuje od každého hříchu.
#1:8 Říkáme-li, že jsme bez hříchu, klameme sami sebe a pravda v nás není.
#1:9 Jestliže doznáváme své hříchy, on je tak věrný a spravedlivý, že nám hříchy odpouští a očišťuje nás od každé nepravosti.
#1:10 Říkáme-li, že jsme nezhřešili, děláme z něho lháře a jeho slovo v nás není. 
#2:1 Toto vám píšu, děti moje, abyste nehřešili. Avšak zhřeší-li kdo, máme u Otce přímluvce, Ježíše Krista spravedlivého.
#2:2 On je smírnou obětí za naše hříchy, a nejenom za naše, ale za hříchy celého světa.
#2:3 Podle toho víme, že jsme ho poznali, jestliže zachováváme jeho přikázání.
#2:4 Kdo říká: ‚Poznal jsem ho‘, a jeho přikázání nezachovává, je lhář a není v něm pravdy.
#2:5 Kdo však zachovává jeho slovo, vpravdě v něm láska Boží dosáhla svého cíle. Podle toho poznáváme, že v něm jsme.
#2:6 Kdo říká, že v něm zůstává, musí žít tak, jak žil on.
#2:7 Nepíšu vám, moji milí, nové přikázání, ale přikázání staré, které jste měli od počátku; staré přikázání je to slovo, které jste slyšeli.
#2:8 A přece vám píšu přikázání nové - vždyť se stalo skutečností v něm i ve vás, že tma ustupuje a pravé světlo již svítí.
#2:9 Kdo říká, že je v světle, a přitom nenávidí svého bratra, je dosud ve tmě.
#2:10 Kdo miluje svého bratra, zůstává ve světle a není nikomu kamenem úrazu.
#2:11 Kdo nenávidí svého bratra, je ve tmě a ve tmě chodí; neví, kam jde, neboť tma mu oslepila oči.
#2:12 Píšu vám, děti, že jsou vám odpuštěny hříchy pro jeho jméno.
#2:13 Píšu vám, otcové, že jste poznali toho, který je od počátku. Píšu vám, mládenci, že jste zvítězili nad Zlým.
#2:14 Napsal jsem vám, děti, že jste poznali Otce. Napsal jsem vám, otcové, že jste poznali toho, který jest od počátku. Napsal jsem vám, mládenci, že jste silní a slovo Boží ve vás zůstává, a tak jste zvítězili nad Zlým.
#2:15 Nemilujte svět ani to, co je ve světě. Miluje-li kdo svět, láska Otcova v něm není.
#2:16 Neboť všechno, co je ve světě, po čem dychtí člověk a co chtějí jeho oči a na čem si v životě zakládá, není z Otce, ale ze světa.
#2:17 A svět pomíjí i jeho chtivost; kdo však činí vůli Boží, zůstává na věky.
#2:18 Dítky, nastala poslední hodina; a jak jste slyšeli, že přijde antikrist, tak se nyní vyskytlo mnoho antikristů; podle toho víme, že nastala poslední hodina.
#2:19 Vyšli z nás, ale nebyli z nás. Kdyby byli z nás, byli by s námi zůstali. Ale nezůstali s námi, aby vyšlo najevo, že nepatří všichni k nám, kdo jsou s námi.
#2:20 Vy však máte zasvěcení od Svatého a znáte všechno.
#2:21 Nepsal jsem vám proto, že neznáte pravdu, ale protože ji znáte a víte, že žádná lež není z pravdy.
#2:22 Kdo je lhář, ne-li ten, kdo popírá, že Ježíš je Kristus? To je ten antikrist, který popírá Otce i Syna.
#2:23 Kdo popírá Syna, nemá ani Otce. Kdo vyznává Syna, má i Otce.
#2:24 Ať ve vás tedy zůstává to, co jste slyšeli od počátku. Zůstane-li ve vás, co jste slyšeli od počátku, zůstanete i vy v Synu i Otci.
#2:25 A to je zaslíbení, které on nám dal: život věčný.
#2:26 Toto jsem vám napsal o těch, kteří vás matou.
#2:27 Ale zasvěcení, které jste vy od něho přijali, zůstává ve vás, takže nepotřebujete, aby vás někdo učil; jeho zasvěcení vás učí všemu, a je pravé a není to žádná lež; jak vás vyučil, tak zůstávejte v něm.
#2:28 Nyní tedy, děti, zůstávejte v něm, abychom se nemuseli bát, až se ukáže, a nebyli jím zahanbeni při jeho příchodu.
#2:29 Víte-li, že on je spravedlivý, pochopte, že také každý, kdo činí spravedlnost, je z něho zrozen. 
#3:1 Hleďte, jak velikou lásku nám Otec daroval: byli jsme nazváni dětmi Božími, a jsme jimi. Proto jsme světu cizí, že nepoznal Boha.
#3:2 Milovaní, nyní jsme děti Boží; a ještě nevyšlo najevo, co budeme! Víme však, až se zjeví, že mu budeme podobni, protože ho spatříme takového, jaký jest.
#3:3 Každý, kdo má tuto naději v něho, usiluje být čistý, tak jako on je čistý.
#3:4 Každý, kdo se dopouští hříchu, jedná i proti zákonu Božímu, neboť hřích je porušení zákona.
#3:5 A víte, že Syn Boží se zjevil, aby hříchy sňal, a v něm žádný hřích není.
#3:6 Kdo v Synu zůstává, nehřeší; kdo hřeší, ten ho neviděl ani nepoznal.
#3:7 Dítky, ať vás nikdo neklame: Spravedlivý je ten, kdo činí spravedlnost - tak jako on je spravedlivý.
#3:8 Kdo však se dopouští hříchu, je z ďábla, protože ďábel od počátku hřeší. Proto se zjevil Syn Boží, aby zmařil činy ďáblovy.
#3:9 Kdo je narozen z Boha, nedopouští se hříchu, protože Boží símě v něm zůstává; ba ani nemůže hřešit, protože se narodil z Boha.
#3:10 Podle toho lze rozeznat děti Boží a děti ďáblovy: Není z Boha, kdokoliv nečiní spravedlnost a nemiluje svého bratra.
#3:11 Neboť to je zvěst, kterou jste slyšeli od počátku: abychom se navzájem milovali.
#3:12 Ne jako Kain, který byl z ďábla a zabil svého bratra. A proč ho zabil? Protože jeho vlastní skutky byly zlé, kdežto bratrovy spravedlivé.
#3:13 Nedivte se, bratří, když vás svět nenávidí.
#3:14 My víme, že jsme přešli ze smrti do života, protože milujeme své bratry. Kdo nemiluje, zůstává ve smrti.
#3:15 Kdokoliv nenávidí svého bratra, je vrah - a víte, že žádný vrah nemá podíl na věčném životě.
#3:16 Podle toho jsme poznali, co je láska, že on za nás položil život. A tak i my jsme povinni položit život za své bratry.
#3:17 Má-li někdo dostatek a vidí, že jeho bratr má nouzi, a bez soucitu se od něho odvrátí - jak v něm může zůstávat Boží láska?
#3:18 Dítky, nemilujme pouhým slovem, ale opravdovým činem.
#3:19 V tomto poznáme, že jsme z pravdy, a tak před ním upokojíme své srdce,
#3:20 ať nás srdce obviňuje z čehokoliv; neboť Bůh je větší než naše srdce a zná všecko!
#3:21 Moji milí, jestliže nás srdce neobviňuje, máme svobodný přístup k Bohu;
#3:22 oč bychom ho žádali, dostáváme od něho, protože zachováváme jeho přikázání a činíme, co se mu líbí.
#3:23 A to je jeho přikázání: věřit jménu jeho Syna Ježíše Krista a navzájem se milovat, jak nám přikázal.
#3:24 Kdo zachovává jeho přikázání, zůstává v Bohu a Bůh v něm; že v nás zůstává, poznáváme podle toho, že nám dal svého Ducha. 
#4:1 Milovaní, nevěřte každému vnuknutí, nýbrž zkoumejte duchy, zda jsou z Boha; neboť mnoho falešných proroků vyšlo do světa.
#4:2 Podle toho poznáte Ducha Božího: Každé vnuknutí, které vede k vyznání, že Ježíš Kristus přišel v těle, je z Boha;
#4:3 každé vnuknutí, které nevede k vyznání Ježíše, z Boha není. Naopak, je to duch antikristův, o němž jste slyšeli, že přijde, a který již nyní je na světě.
#4:4 Vy však jste z Boha, děti, a zvítězili jste nad falešnými proroky, protože ten, který je ve vás, je větší než ten, který je ve světě.
#4:5 Oni jsou ze světa; proto z nich mluví svět a svět je slyší.
#4:6 My jsme z Boha; kdo zná Boha, slyší nás, kdo není z Boha, neslyší nás. Podle toho rozeznáváme ducha pravdy a ducha klamu.
#4:7 Milovaní, milujme se navzájem, neboť láska je z Boha, a každý, kdo miluje, z Boha se narodil a Boha zná.
#4:8 Kdo nemiluje, nepoznal Boha, protože Bůh je láska.
#4:9 V tom se ukázala Boží láska k nám, že Bůh poslal na svět svého jediného Syna, abychom skrze něho měli život.
#4:10 V tom je láska: ne že my jsme si zamilovali Boha, ale že on si zamiloval nás a poslal svého Syna jako oběť smíření za naše hříchy.
#4:11 Milovaní, jestliže Bůh nás tak miloval, i my se máme navzájem milovat.
#4:12 Boha nikdy nikdo neviděl, ale jestliže se milujeme navzájem, Bůh v nás zůstává a jeho láska v nás dosáhla svého cíle.
#4:13 Že zůstáváme v něm a on v nás, poznáváme podle toho, že nám dal svého Ducha.
#4:14 A my jsme spatřili a dosvědčujeme, že Otec poslal Syna, aby byl Spasitelem světa.
#4:15 Kdo vyzná, že Ježíš je Syn Boží, v tom zůstává Bůh a on v Bohu.
#4:16 Také my jsme poznali lásku, kterou Bůh má k nám, a věříme v ni. Bůh je láska, a kdo zůstává v lásce, v Bohu zůstává a Bůh v něm.
#4:17 V tom jeho láska k nám dosáhla cíle, že máme plnou jistotu pro den soudu - neboť jaký je on, takoví jsme i my v tomto světě.
#4:18 Láska nezná strach; dokonalá láska strach zahání, vždyť strach působí muka, a kdo se bojí, nedošel dokonalosti v lásce.
#4:19 My milujeme, protože Bůh napřed miloval nás.
#4:20 Řekne-li někdo: „Já miluji Boha“, a přitom nenávidí svého bratra, je lhář. Kdo nemiluje svého bratra, kterého vidí, nemůže milovat Boha, kterého nevidí.
#4:21 A tak máme od něho toto přikázání: Kdo miluje Boha, ať miluje i svého bratra. 
#5:1 Každý, kdo věří, že Ježíš je Kristus, je zrozen z Boha; a kdo má rád otce, má rád i jeho dítě.
#5:2 Podle toho poznáváme, že milujeme Boží děti, když milujeme Boha a jeho přikázání zachováváme.
#5:3 V tom je totiž láska k Bohu, že zachováváme jeho přikázání; a jeho přikázání nejsou těžká,
#5:4 neboť kdo se narodil z Boha, přemáhá svět. A to vítězství, které přemohlo svět, je naše víra.
#5:5 Kdo jiný přemáhá svět, ne-li ten, kdo věří, že Ježíš je Syn Boží?
#5:6 To je ten, který přišel skrze vodu a krev: Ježíš Kristus. Ne pouze skrze vodu křtu, ale i skrze krev kříže; a Duch o tom vydává svědectví, neboť Duch jest pravda.
#5:7 Tři jsou, kteří vydávají svědectví
#5:8 - Duch, voda a krev - a ti tři jsou zajedno.
#5:9 Přijímáme-li svědectví lidí, oč větší je svědectví Boží; Boží svědectví je to, co pověděl o svém Synu.
#5:10 Kdo věří v Syna Božího, má to svědectví v sobě. Kdo nevěří Bohu, dělá z něho lháře, protože nevěří svědectví, které Bůh vydal o svém Synu.
#5:11 A to je to svědectví: Bůh nám dal věčný život, a ten život je v jeho Synu.
#5:12 Kdo má Syna, má život; kdo nemá Syna Božího, nemá život.
#5:13 Toto píšu vám, kteří věříte ve jméno Syna Božího, abyste věděli, že máte věčný život.
#5:14 Máme v něho pevnou důvěru, že nás slyší, kdykoliv o něco požádáme ve shodě s jeho vůlí.
#5:15 A víme-li, že nás slyší, kdykoliv o něco žádáme, pak také víme, že to, co máme, jsme dostali od něho.
#5:16 Vidí-li někdo, že jeho bratr se dopouští hříchu, který není k smrti, ať za něho prosí; a Bůh mu daruje život, jestliže nehřešil k smrti. Jest ovšem hřích, který je k smrti; o takovém neříkám, abyste za něj prosili.
#5:17 Každá nepravost je hřích, ale je i hřích, který není k smrti.
#5:18 Víme, že nikdo, kdo se narodil z Boha, nehřeší, ale Syn Boží jej chrání a Zlý se ho ani nedotkne.
#5:19 Víme, že jsme z Boha, kdežto celý svět je pod mocí Zlého.
#5:20 Víme, že Syn Boží přišel a dal nám schopnost rozeznávat, abychom poznali, kdo je pravý Bůh. A jsme v tom pravém Bohu, protože jsme v jeho Synu Ježíši Kristu. On je ten pravý Bůh a věčný život.
#5:21 Děti, varujte se modlářství.  

\book{II John}{2John}
#1:1 Já starší píšu vyvolené paní a jejím dětem, které opravdově miluji - a nejen já, nýbrž všichni, kdo poznali pravdu.
#1:2 Píšu kvůli pravdě, která v nás zůstává a bude s námi navěky:
#1:3 Bude s námi milost, milosrdenství a pokoj od Boha Otce i od Ježíše Krista, Syna Otcova, v pravdě a lásce.
#1:4 Velice jsem se zaradoval, když jsem mezi tvými dětmi našel takové, které žijí v pravdě, jak jsme dostali přikázání od Otce.
#1:5 A nyní tě prosím, paní, ne že bych ti psal nové přikázání, ale připomínám to, které máme od počátku: abychom milovali jedni druhé.
#1:6 A to je láska: žít podle Božích přikázání; to je to přikázání, o kterém jste od počátku slyšeli, že máte podle ní žít.
#1:7 Do světa vyšlo mnoho těch, kteří vás svádějí, neboť nevyznávají, že Ježíš Kristus přišel v těle; kdo takto učí, je svůdce a antikrist.
#1:8 Mějte se na pozoru, abyste nepřišli o to, na čem jste pracovali, ale abyste dostali plnou odměnu.
#1:9 Kdo zachází dál a nezůstává v učení Kristovu, nemá Boha; kdo zůstává v jeho učení, má i Otce i Syna.
#1:10 Přijde-li někdo k vám a nepřináší toto učení, nepřijímejte ho do domu a nevítejte ho;
#1:11 kdo ho vítá, má účast na jeho zlých skutcích.
#1:12 Ačkoliv bych měl ještě mnoho co psát, nemíním to svěřovat papíru a inkoustu; neboť doufám, že se k vám dostanu a budu s vámi mluvit tváří v tvář, aby naše radost byla úplná.
#1:13 Pozdravují tě děti tvé vyvolené sestry.  

\book{III John}{3John}
#1:1 Já starší milému Gáiovi, jehož miluji v pravdě.
#1:2 Modlím se za tebe, milovaný, aby se ti ve všem dobře dařilo a abys byl zdráv - tak jako se dobře daří tvé duši.
#1:3 Velice jsem se zaradoval, když přišli bratří a vydávali svědectví o tvé opravdovosti, o tom, že žiješ v pravdě.
#1:4 Nemám větší radost, než když slyším, že moje děti žijí v pravdě.
#1:5 Věrně jednáš, milovaný, v tom, co činíš pro bratry, a to pro ty, kteří přišli odjinud;
#1:6 oni vydali před církví svědectví o tvé lásce. Dobře učiníš, když je vypravíš na další cestu, jak se sluší před Bohem,
#1:7 neboť pro jméno Kristovo se vydali na cesty a od pohanů nic nepřijímají.
#1:8 Proto jsme povinni takových se ujímat, abychom měli podíl na práci pro pravdu.
#1:9 Něco jsem již o tom církvi psal; ale Diotrefés, který si osobuje právo být mezi nimi první, nás neuznává.
#1:10 Proto až přijdu, ukážu na jeho jednání, jak nás zlomyslně pomlouvá; a na tom nemá dost - ani bratry nepřijímá, a těm, kdo je chtějí přijmout, v tom brání a vylučuje je z církve.
#1:11 Můj milý, neřiď se podle zlého, ale podle dobrého. Kdo jedná dobře, je z Boha; kdo jedná špatně, Boha neviděl.
#1:12 Demétriovi všichni vydávají dobré svědectví, ano i sama pravda; i my mu vydáváme svědectví, a ty víš, že naše svědectví je pravé.
#1:13 Měl bych ti ještě mnoho co psát, ale nechci to svěřit inkoustu a peru;
#1:14 doufám totiž, že tě brzo uvidím, a budeme spolu mluvit tváří v tvář.
#1:15 Pokoj tobě. Přátelé tě pozdravují. Pozdravuj každého z přátel osobně!  

\book{Jude}{Jude}
#1:1 Juda, služebník Ježíše Krista, bratr Jakubův, těm, kdo jsou povoláni, milováni Bohem Otcem a zachováni pro Ježíše Krista:
#1:2 Milosrdenství, pokoj a láska ať se vám rozhojní!
#1:3 Milovaní, velmi jsem si přál psát vám o našem společném spasení, ale teď pokládám za nutné napomenout vás, abyste zápasili o víru, jednou provždy odevzdanou Božímu lidu.
#1:4 Vloudili se totiž mezi vás někteří bezbožní lidé, zapsaní už dávno k odsouzení, kteří zaměňují milost našeho Boha v nezřízenost a zapírají jediného vládce a našeho Pána Ježíše Krista.
#1:5 Chci vám však připomenout, třebaže to všechno již dávno víte, že Hospodin sice vysvobodil lid z egyptské země, potom však zahubil ty, kteří neuvěřili.
#1:6 Také anděly, kteří si nezachovali své vznešené postavení, ale opustili určené místo, drží ve věčných poutech v temnotě pro veliký den soudu.
#1:7 Podobně jako oni i Sodoma, Gomora a okolní města se oddaly smilstvu, propadly zvrhlosti, a jsou nám výstražným příkladem trestu věčného ohně.
#1:8 Podobně i tito blouznivci poskvrňují své tělo, žádnou autoritu neuznávají, nadpozemským mocnostem se rouhají.
#1:9 A přece sám archanděl Michael, když se přel s ďáblem o Mojžíšovo tělo, neosmělil se vynést nad ním zatracující soud, ale řekl: Potrestej tě Hospodin.
#1:10 Tito však se rouhají tomu, co neznají; a co pudem jako nerozumná zvířata znají, v tom propadají zhoubě.
#1:11 Běda jim, neboť se dali cestou Kainovou a jako Balaám se nechali svést úplatkem a jako Kore zahynuli pro svou vzpouru.
#1:12 Ti jsou úskalím vašeho bratrského stolování, když s vámi hodují bez bázně před Bohem a jsou jako pastýři, kteří pasou sami sebe. Jsou jako mraky bez deště, hnané větrem, podzimní stromy bez ovoce, dvakrát mrtvé a vykořeněné,
#1:13 divoké vlny mořské, vyvrhující své vlastní hanebnosti, bludné hvězdy, jimž navěky je připravena nejčernější tma.
#1:14 Prorokoval také o nich Henoch, sedmý od Adama: Hle, přichází Pán s desetitisíci svých svatých,
#1:15 aby vykonal soud nade všemi a usvědčil všechny bezbožné z jejich skutků, které kdy ve své bezbožnosti spáchali, i ze všech zpupných řečí, které ti hříšníci mluvili proti němu.
#1:16 Vzpouzejí se a odporují Božímu vedení a žijí si podle svých vášní, jejich ústa mluví nadutě a lichotí lidem pro svůj prospěch.
#1:17 Ale vy, milovaní, pamatujte na to, co předpověděli apoštolové Pána našeho Ježíše Krista,
#1:18 neboť vám říkali, že v posledním čase přijdou posměvači, žijící bezbožně podle svých vášní.
#1:19 To jsou ti původci roztržek; jsou pudoví a nemají Ducha Božího.
#1:20 Ale vy, milovaní, budujte svůj život na přesvaté víře, modlete se v Duchu svatém,
#1:21 uchovejte se v lásce Boží a očekávejte milosrdenství našeho Pána Ježíše Krista k věčnému životu.
#1:22 S těmi, kdo pochybují, mějte slitování;
#1:23 zachraňujte je z hořícího ohně. Mějte slitování i nad jinými, ale s obezřetností, ať se vám oškliví i jejich plášť, poskvrněný hříchem.
#1:24 Tomu pak, který má moc uchránit vás před pádem a postavit neposkvrněné a v radosti před tvář své slávy,
#1:25 jedinému Bohu, který nás spasil skrze Ježíše Krista, našeho Pána, buď sláva, velebnost, vláda i moc přede vším časem i nyní i po všecky věky. Amen.  

\book{Revelation of John}{Rev}
#1:1 Zjevení, které Bůh dal Ježíši Kristu, aby ukázal svým služebníkům, co se má brzo stát; naznačil to prostřednictvím anděla svému služebníku Janovi.
#1:2 Ten dosvědčil Boží slovo a svědectví Ježíše Krista, vše, co viděl.
#1:3 Blaze tomu, kdo předčítá slova tohoto proroctví, a blaze těm, kdo slyší a zachovávají, co je tu napsáno, neboť čas je blízko.
#1:4 Jan sedmi církvím v Asii: Milost vám a pokoj od toho, který jest a který byl a který přichází, i od sedmi duchů před jeho trůnem
#1:5 a od Ježíše Krista, věrného svědka, prvorozeného z mrtvých a vládce králů země. Jemu, jenž nás miluje a svou krví nás zprostil hříchů
#1:6 a učinil nás královským kněžstvem Boha, svého Otce - jemu sláva i moc navěky. Amen.
#1:7 Hle, přichází v oblacích! Uzří ho každé oko, i ti, kdo ho probodli, a budou kvůli němu naříkat všechna pokolení země. Tak jest, amen.
#1:8 Já jsem Alfa i Omega, praví Pán Bůh, ten, který jest a který byl a který přichází, Všemohoucí.
#1:9 Já Jan, váš bratr, který má s vámi účast na Ježíšově soužení i kralování a vytrvalosti, dostal jsem se pro slovo Boží a svědectví Ježíšovo na ostrov jménem Patmos.
#1:10 Ocitl jsem se ve vytržení ducha v den Páně, a uslyšel jsem za sebou mocný hlas jako zvuk polnice:
#1:11 „Co vidíš, napiš do knihy a pošli sedmi církvím: do Efezu, do Smyrny, do Pergama, do Thyatir, do Sard, do Filadelfie a do Laodikeje.“
#1:12 Obrátil jsem se, abych viděl, kdo se mnou mluví. A když jsem se obrátil, spatřil jsem sedm zlatých svícnů;
#1:13 uprostřed těch svícnů někdo jako Syn člověka, oděný řízou až na zem, a na prsou zlatý pás.
#1:14 Jeho hlava a vlasy bělostné jako sněhobílá vlna, jeho oči jako plamen ohně;
#1:15 jeho nohy podobné kovu přetavenému ve výhni a jeho hlas jako hukot příboje.
#1:16 V pravici držel sedm hvězd a z jeho úst vycházel ostrý dvousečný meč; jeho vzhled jako když slunce září v plné své síle.
#1:17 Když jsem ho spatřil, padl jsem k jeho nohám jako mrtvý; ale on vložil na mne svou pravici a řekl: „Neboj se. Já jsem první i poslední,
#1:18 ten živý; byl jsem mrtev - a hle, živ jsem na věky věků. Mám klíče od smrti i hrobu.
#1:19 Napiš tedy, co jsi viděl - to, co jest, i to, co se má stát potom.
#1:20 Tajemství těch sedmi hvězd, které jsi viděl v mé pravici, i těch sedmi zlatých svícnů: Sedm hvězd jsou andělé sedmi církví, a sedm svícnů je sedm církví.“ 
#2:1 Andělu církve v Efezu piš: Toto praví ten, který drží sedm hvězd ve své pravici, který se prochází mezi sedmi zlatými svícny:
#2:2 „Vím o tvých skutcích, o tvém úsilí i tvé vytrvalosti; vím, že nemůžeš snést ty, kdo jsou zlí, a vyzkoušel jsi ty, kdo se vydávají za apoštoly, ale nejsou, a shledal jsi, že jsou lháři.
#2:3 Máš vytrvalost a trpěl jsi pro mé jméno, a nepodlehls únavě.
#2:4 Ale to mám proti tobě, že už nemáš takovou lásku jako na počátku.
#2:5 Rozpomeň se, odkud jsi klesl, navrať se a jednej jako dřív. Ne-li, přijdu na tebe a pohnu tvým svícnem z jeho místa, jestliže se neobrátíš.
#2:6 To však máš k dobru, že nenávidíš skutky Nikolaitů stejně jako já.
#2:7 Kdo má uši, slyš, co Duch praví církvím: Tomu, kdo zvítězí, dám jíst ze stromu života v Božím ráji.“
#2:8 Andělu církve ve Smyrně piš: Toto praví ten první i poslední, který byl mrtev a je živ:
#2:9 „Vím o tvém soužení a tvé chudobě, ale jsi bohat; vím, jak tě urážejí ti, kdo si říkají židé, ale nejsou, nýbrž je to spolek satanův!
#2:10 Neboj se toho, co máš vytrpět. Hle, ďábel má některé z vás uvrhnout do vězení, abyste prošli zkouškou, a budete mít soužení po deset dní. Buď věrný až na smrt, a dám ti vítězný věnec života.
#2:11 Kdo má uši, slyš, co Duch praví církvím: Kdo zvítězí, tomu druhá smrt neublíží.“
#2:12 Andělu církve v Pergamu piš: Toto praví ten, který má ostrý dvousečný meč:
#2:13 „Vím, kde bydlíš: tam, kde je trůn satanův. Avšak pevně se držíš mého jména a nezapřel jsi víru ve mne ani ve dnech, kdy můj věrný svědek Antipas byl zabit mezi vámi, tam, kde bydlí satan.
#2:14 Jen to mám proti tobě, že u sebe máš zastánce učení Balaámova. Jako on učil Baláka svádět syny Izraele, aby se účastnili modlářských hostin a smilstva,
#2:15 tak i ty máš některé, kteří zastávají učení Nikolaitů.
#2:16 Proto se obrať! Ne-li, brzo k tobě přijdu a budu s nimi bojovat mečem svých úst.
#2:17 Kdo má uši, slyš, co Duch praví církvím: Tomu, kdo zvítězí, dám jíst ze skryté many; dám mu bílý kamének, a na tom kaménku je napsáno nové jméno, které nezná nikdo než ten, kdo je dostává.“
#2:18 Andělu církve v Thyatirech piš: Toto praví Syn Boží, jenž má oči jako planoucí oheň a nohy jako zářivý kov:
#2:19 „Vím o tvých skutcích, lásce a víře, službě a vytrvalosti; vím, že tvých skutků je čím dál více.
#2:20 Ale to mám proti tobě, že trpíš ženu Jezábel, která se vydává za prorokyni a svým učením svádí moje služebníky ke smilstvu a k účasti na modlářských hostinách.
#2:21 Dal jsem jí čas k pokání, ale ona se nechce odvrátit od svého smilstva.
#2:22 Hle, sešlu na ni nemoc a do velikého soužení uvrhnu ty, kdo s ní cizoloží, jestliže se od jejích činů neodvrátí;
#2:23 a její děti zahubím. Tu poznají všechny církve, že já vidím do nitra i srdce člověka, a každému z vás odplatím podle vašich činů.
#2:24 Vám ostatním v Thyatirech, kteří nepřijímáte toto učení, kteří jste nepoznali to, čemu říkají hlubiny satanovy, pravím: Nevkládám na vás jiné břemeno.
#2:25 Jenom se pevně držte toho, co máte, dokud nepřijdu.
#2:26 Kdo zvítězí a setrvá v mých skutcích až do konce, tomu dám moc nad národy:
#2:27 bude je pást železnou berlou, jako hliněné nádobí je bude rozbíjet
#2:28 - tak jako já jsem tu moc přijal od svého Otce. Tomu, kdo zvítězí, dám hvězdu jitřní.
#2:29 Kdo má uši, slyš, co Duch praví církvím.“ 
#3:1 Andělu církve v Sardách piš: Toto praví ten, který má sedmero duchů Božích a sedmero hvězd: „Vím o tvých skutcích - podle jména jsi živ, ale jsi mrtev.
#3:2 Probuď se a posilni to, co ještě zůstává a je už na vymření! Neboť shledávám, že tvým skutkům mnoho chybí před Bohem;
#3:3 rozpomeň se tedy, jak jsi mé slovo přijal a slyšel, zachovávej je a čiň pokání. Nebudeš-li bdít, přijdu tak, jako přichází zloděj, a nebudeš vědět, v kterou hodinu na tebe přijdu.
#3:4 Přece však máš v Sardách několik osob, které svůj šat nepotřísnily; ti budou chodit se mnou v bílém rouchu, protože jsou toho hodni.
#3:5 Kdo zvítězí, bude oděn bělostným rouchem a jeho jméno nevymažu z knihy života, nýbrž přiznám se k němu před svým Otcem a před jeho anděly.
#3:6 Kdo má uši, slyš, co Duch praví církvím.“
#3:7 Andělu církve ve Filadelfii piš: Toto prohlašuje ten Svatý a Pravý, který má klíč Davidův; když on otvírá, nikdo nezavře, a když on zavírá, nikdo neotevře:
#3:8 „Vím o tvých skutcích. Hle, otevřel jsem před tebou dveře, a nikdo je nemůže zavřít. Neboť ačkoli máš nepatrnou moc, zachoval jsi mé slovo a mé jméno jsi nezapřel.
#3:9 Hle, dávám do tvých rukou ty, kdo jsou ze synagógy satanovy; říkají si židé, a nejsou, ale lžou. Hle, způsobím, že přijdou a padnou ti k nohám; a poznají, že já jsem si tě zamiloval.
#3:10 Protože jsi zachoval mé slovo a vytrval, zachovám tě i já v hodině zkoušky, která přijde na celý svět a prověří obyvatele země.
#3:11 Přijdu brzy; drž se toho, co máš, aby tě nikdo nepřipravil o vavřín vítěze.
#3:12 Kdo zvítězí, toho učiním sloupem v chrámě svého Boha a chrám již neopustí; napíšu na něj jméno svého Boha a jméno jeho města, nového Jeruzaléma který sestupuje z nebe od mého Boha, i jméno své nové.
#3:13 Kdo má uši, slyš, co Duch praví církvím.“
#3:14 Andělu církve v Laodikeji piš: Toto praví ten, jehož jméno jest Amen, svědek věrný a pravý, počátek stvoření Božího:
#3:15 „Vím o tvých skutcích; nejsi studený ani horký. Kéž bys byl studený anebo horký!
#3:16 Ale že jsi vlažný, a nejsi horký ani studený, nesnesu tě v ústech.
#3:17 Vždyť říkáš: Jsem bohat, mám všecko a nic už nepotřebuji! A nevíš, že jsi ubohý, bědný a nuzný, slepý a nahý.
#3:18 Radím ti, abys u mne nakoupil zlata ohněm přečištěného, a tak zbohatl; a bílý šat, aby ses oblékl a nebylo vidět tvou nahotu; a mast k potření očí, abys prohlédl.
#3:19 Já kárám a trestám ty, které miluji; vzpamatuj se tedy a čiň pokání.
#3:20 Hle, stojím přede dveřmi a tluču; zaslechne-li kdo můj hlas a otevře mi, vejdu k němu a budu s ním večeřet a on se mnou.
#3:21 Kdo zvítězí, tomu dám usednout se mnou na trůn, tak jako já jsem zvítězil a usedl s Otcem na jeho trůn.
#3:22 Kdo má uši, slyš, co Duch praví církvím.“ 
#4:1 Potom jsem měl vidění: Hle, dveře do nebe otevřené, a ten hlas, který předtím ke mně mluvil a který zněl jako polnice, nyní řekl: „Pojď sem, a ukážu ti, co se má stát potom.“
#4:2 Ihned jsem se ocitl ve vytržení ducha: A hle, trůn v nebi, a na tom trůnu někdo,
#4:3 kdo byl na pohled jako jaspis a karneol; a kolem trůnu duha jako smaragdová.
#4:4 Okolo toho trůnu čtyřiadvacet jiných trůnů, a na nich sedělo čtyřiadvacet starců, oděných bělostným rouchem, na hlavách koruny ze zlata.
#4:5 Od trůnu šlehaly blesky a dunělo hromobití; před trůnem hořelo sedm světel - to je sedmero duchů Božích;
#4:6 a před trůnem moře jiskřící jako křišťál a uprostřed kolem trůnu čtyři živé bytosti plné očí zpředu i zezadu:
#4:7 První podobná lvu, druhá býku, třetí měla tvář člověka, čtvrtá byla podobná letícímu orlu.
#4:8 Všechny čtyři bytosti jedna jako druhá měly po šesti křídlech a plno očí hledících ven i dovnitř. A bez ustání dnem i nocí volají: „Svatý, svatý, svatý Hospodin, Bůh všemohoucí, ten, který byl a který jest a který přichází.“
#4:9 A kdykoli ty bytosti vzdají čest, slávu a díky tomu, který sedí na trůnu a je živ na věky věků,
#4:10 padá těch čtyřiadvacet starců na kolena před tím, který sedí na trůnu, a klanějí se tomu, který je živ na věky věků; pak položí své koruny před trůnem se slovy:
#4:11 „Jsi hoden, Pane a Bože náš, přijmout slávu, čest i moc, neboť ty jsi stvořil všechno a tvou vůlí všechno povstalo a jest.“ 
#5:1 A v pravici toho, který sedí na trůnu, spatřil jsem knihu úplně popsanou, zapečetěnou sedmi pečetěmi.
#5:2 Tu jsem uviděl mocného anděla, který vyhlásil velikým hlasem: „Kdo je hoden otevřít tu knihu a rozlomit její pečetě?“
#5:3 Ale nikdo na nebi ani na zemi ani pod zemí nemohl tu knihu otevřít a podívat se do ní.
#5:4 Velmi jsem plakal, že se nenašel nikdo, kdo by byl hoden tu knihu otevřít a podívat se do ní.
#5:5 Ale jeden ze starců mi řekl: „Neplač. Hle, zvítězil lev z pokolení Judova, potomek Davidův; on otevře tu knihu sedmkrát zapečetěnou.“
#5:6 Vtom jsem spatřil, že uprostřed mezi trůnem a těmi čtyřmi bytostmi a starci stojí Beránek, ten obětovaný; měl sedm rohů a sedm očí, což je sedmero duchů Božích vyslaných do celého světa.
#5:7 Přistoupil k tomu, který sedí na trůnu, a přijal knihu z jeho pravice.
#5:8 A když tu knihu uchopil, čtyři bytosti a čtyřiadvacet starců padlo na kolena před Beránkem; každý měl loutnu a zlatou nádobu naplněnou vůní kadidla, což jsou modlitby Božího lidu.
#5:9 A zpívali novou píseň: „Jsi hoden přijmout tu knihu a rozlomit její pečetě, protože jsi byl obětován, svou krví jsi Bohu vykoupil lidi ze všech kmenů, jazyků, národů a ras
#5:10 a učinil je královským kněžstvem našeho Boha; a ujmou se vlády nad zemí.“
#5:11 A viděl jsem, jak kolem trůnu a těch bytostí i starců stojí množství andělů - bylo jich na tisíce a na statisíce;
#5:12 slyšel jsem je mocným hlasem volat: „Hoden jest Beránek, ten obětovaný, přijmout moc, bohatství, moudrost, sílu, poctu, slávu i dobrořečení.“
#5:13 A všechno stvoření na nebi, na zemi, pod zemí i v moři, všecko, co v nich jest, slyšel jsem volat: „Tomu, jenž sedí na trůnu, i Beránkovi dobrořečení, čest, sláva i moc na věky věků!“
#5:14 A ty čtyři bytosti řekly: „Amen“; starci padli na kolena a klaněli se. 
#6:1 Tu jsem viděl, jak Beránek rozlomil první ze sedmi pečetí, a slyšel jsem, jak jedna z těch čtyř bytostí řekla hromovým hlasem: „Pojď!“
#6:2 A hle, bílý kůň, a na něm jezdec s lukem; byl mu dán věnec dobyvatele, aby vyjel a dobýval.
#6:3 Když Beránek rozlomil druhou pečeť, slyšel jsem, jak druhá z těch bytostí řekla: „Pojď!“
#6:4 A vyjel druhý kůň, ohnivý, a jeho jezdec obdržel moc odejmout zemi mír, aby se všichni navzájem vraždili; byl mu dán veliký meč.
#6:5 Když Beránek rozlomil třetí pečeť, slyšel jsem, jak třetí z těch bytostí řekla: „Pojď!“ A hle, kůň černý, a jezdec měl v ruce váhy.
#6:6 A z kruhu těch čtyř bytostí jsem slyšel hlas: „Za denní mzdu jen mírka pšenice, za denní mzdu tři mírky ječmene. Olej a víno však nech!“
#6:7 A když Beránek rozlomil čtvrtou pečeť, slyšel jsem hlas čtvrté bytosti: „Pojď!“
#6:8 A hle, kůň sinavý, a jméno jeho jezdce Smrt, a svět mrtvých zůstával za ním. Těm jezdcům byla dána moc, aby čtvrtinu země zhubili mečem, hladem, morem a dravými šelmami.
#6:9 Když Beránek rozlomil pátou pečeť, spatřil jsem pod oltářem ty, kdo byli zabiti pro slovo Boží a pro svědectví, které vydali.
#6:10 A křičeli velikým hlasem: „Kdy už, Pane svatý a věrný, vykonáš soud a za naši krev potrestáš ty, kdo bydlí na zemi?“
#6:11 Tu jim všem bylo dáno bílé roucho a bylo jim řečeno, aby měli strpení ještě krátký čas, dokud jejich počet nedoplní spoluslužebníci a bratří, kteří budou zabiti jako oni.
#6:12 A hle, když rozlomil šestou pečeť, nastalo veliké zemětřesení, slunce zčernalo jako smuteční šat, měsíc úplně zkrvavěl
#6:13 a nebeské hvězdy začaly padat na zem, jako když fík zmítaný vichrem shazuje své pozdní plody,
#6:14 nebesa zmizela, jako když se zavře kniha a žádná hora a žádný ostrov nezůstaly na svém místě.
#6:15 Králové země i velmoži a vojevůdci, boháči a mocní - jak otrok, tak svobodný, všichni prchali do hor, aby se ukryli v jeskyních a skalách,
#6:16 a volali k horám a skalám: „Padněte na nás a skryjte nás před tváří toho, který sedí na trůnu, a před hněvem Beránkovým!“
#6:17 Neboť přišel veliký den jeho hněvu; kdo bude moci obstát? 
#7:1 Potom jsem viděl, jak se čtyři andělé postavili do čtyř úhlů země a bránili všem čtyřem větrům, aby žádný z nich nevál na zemi ani na moře ani na jakékoli stromoví.
#7:2 A hle, jiný anděl vystupoval od východu slunce; v ruce držel pečetidlo živého Boha a mocným hlasem volal na ty čtyři anděly, jimž bylo dáno škodit zemi i moři:
#7:3 „Neškoďte zemi, moři ani stromoví, dokud neoznačíme služebníky našeho Boha na jejich čelech!“
#7:4 Pak jsem slyšel počet označených: sto čtyřiačtyřicet tisíc označených ze všech pokolení Izraele:
#7:5 z pokolení Juda dvanáct tisíc, z pokolení Rúben dvanáct tisíc, z pokolení Gád dvanáct tisíc,
#7:6 z pokolení Ašer dvanáct tisíc, z pokolení Neftalím dvanáct tisíc, z pokolení Manase dvanáct tisíc,
#7:7 z pokolení Šimeón dvanáct tisíc, z pokolení Levi dvanáct tisíc, z pokolení Isachar dvanáct tisíc,
#7:8 z pokolení Zabulón dvanáct tisíc, z pokolení Josef dvanáct tisíc, z pokolení Benjamín dvanáct tisíc označených.
#7:9 Potom jsem viděl, hle, tak veliký zástup, že by ho nikdo nedokázal sečíst, ze všech ras, kmenů, národů a jazyků, jak stojí před trůnem a před tváří Beránkovou, oblečeni v bílé roucho, palmové ratolesti v rukou.
#7:10 A volali velikým hlasem: „Díky Spasiteli, Bohu našemu, sedícímu na trůnu, a Beránkovi.“
#7:11 A všichni andělé se postavili kolem trůnu, kolem starců i těch čtyř bytostí a padli před trůnem tváří k zemi,
#7:12 klaněli se Bohu a volali: „Amen! Dobrořečení i sláva a moudrost, díky a čest i moc a síla Bohu našemu na věky věků. Amen!“
#7:13 Jeden z těch starců na mne promluvil: „Kdo jsou a odkud přišli ti v bílém rouchu?“
#7:14 Řekl jsem: „Pane můj, ty to víš!“ A on mi řekl: „To jsou ti, kteří přišli z velikého soužení a vyprali svá roucha a vybílili je v krvi Beránkově.
#7:15 Proto jsou před trůnem Božím a slouží mu v jeho chrámě dnem i nocí; a ten, který sedí na trůnu, bude jim záštitou.
#7:16 Již nebudou hladovět ani žíznit, ani slunce nebo jiný žár jim neublíží,
#7:17 neboť Beránek, který je před trůnem, je bude pást a povede je k pramenům vod života. A Bůh jim setře každou slzu s očí.“ 
#8:1 A když Beránek rozlomil pečeť sedmou, nastalo na nebi mlčení téměř na půl hodiny.
#8:2 Potom jsem viděl, jak sedmi andělům, stojícím před Bohem, bylo dáno sedm polnic.
#8:3 Jiný anděl předstoupil se zlatou kadidelnicí před oltář; bylo mu dáno množství kadidla, aby je s modlitbami všech posvěcených položil na zlatý oltář před trůnem.
#8:4 A vystoupil dým kadidla spolu s modlitbami posvěcených z ruky anděla před Boží tvář.
#8:5 Tu vzal anděl kadidelnici, nahrnul do ní oheň z oltáře a vrhl ji dolů na zem; a nastalo burácení, hřímání, blesky a zemětřesení.
#8:6 A sedm andělů s polnicemi se připravilo, aby začali troubit.
#8:7 Zatroubil první anděl: Nastalo krupobití a na zem začal padat oheň smíšený s krví. Třetina země, třetina stromoví a veškerá zeleň byla sežehnuta.
#8:8 Zatroubil druhý anděl; a jakoby mohutná hora hořící ohněm byla vržena do moře. Třetina moře se obrátila v krev
#8:9 a zahynula třetina mořských tvorů a byla zničena třetina lodí.
#8:10 Zatroubil třetí anděl, a zřítila se z nebe veliká hvězda hořící jako pochodeň, padla na třetinu řek a na prameny vod.
#8:11 Jméno té hvězdy je Pelyněk. Třetina vod se změnila v pelyněk a množství lidí umřelo z těch vod, protože byly otráveny.
#8:12 Zatroubil čtvrtý anděl, a byla zasažena třetina slunce, třetina měsíce a třetina hvězd, takže ze třetiny potemněly, a den i noc byly o třetinu temnější.
#8:13 A viděl jsem a slyšel, jak středem nebeské klenby letí orel a volá mocným hlasem: „Běda, běda, běda obyvatelům země, až zaznějí polnice tří andělů, kteří ještě netroubili!“ 
#9:1 Zatroubil pátý anděl. A viděl jsem, jak hvězdě, která spadla z nebe na zem, byl dán klíč od jícnu propasti;
#9:2 otevřela jícen propasti a vyvalil se dým jako z obrovské pece, a tím dýmem se zatmělo slunce i všechno ovzduší.
#9:3 Z dýmu se vyrojily kobylky na zem; byla jim dána moc, jakou mají pozemští škorpióni.
#9:4 Dostaly rozkaz neškodit trávě na zemi ani žádné zeleni ani stromoví, jenom lidem, kteří nejsou označeni Boží pečetí na čele.
#9:5 Ale nebyla jim dána moc, aby lidi zabíjely, nýbrž aby je po pět měsíců trýznily; byla to trýzeň, jako když škorpión bodne člověka.
#9:6 V ty dny budou lidé hledat smrt, ale nenajdou ji, budou si přát zemřít, ale smrt se jim vyhne.
#9:7 Ty kobylky vypadaly jako vyzbrojená válečná jízda; na hlavách měly něco jako zlaté věnce, tvář měly jako lidé,
#9:8 hřívu jako vlasy žen, ale zuby měly jako lvi.
#9:9 Měly jakoby železné pancíře a jejich křídla hřmotila, jako když množství spřežení se řítí do boje.
#9:10 A měly ocasy jako škorpióni a v nich žihadla, aby jimi trýznily lidi po pět měsíců.
#9:11 Nad sebou měly krále, anděla propasti, hebrejským jménem Abaddon - to znamená Hubitel.
#9:12 První ‚běda‘ pominulo, a hle, už jsou tu dvě další!
#9:13 Zatroubil šestý anděl. Uslyšel jsem jakýsi hlas od čtyř rohů zlatého oltáře, který je před Bohem.
#9:14 Ten hlas nařídil šestému andělu, držícímu polnici: „Rozvaž ty čtyři anděly, spoutané při veliké řece Eufratu!“
#9:15 Tu byli rozvázáni ti čtyři andělé, připravení na hodinu, den, měsíc a rok, kdy mají pobít třetinu lidí.
#9:16 A jejich jízdních oddílů bylo dvě stě miliónů - slyšel jsem jejich počet.
#9:17 A ve vidění jsem spatřil koně a na nich jezdce: pancíře měli ohnivé, rudě zářící a žhnoucí sírou; hlavy koňů byly jako hlavy lvů a z jejich tlam šel oheň, dým a síra.
#9:18 Touto trojí pohromou - ohněm, dýmem a sírou ze svých tlam - usmrtili třetinu lidí.
#9:19 Vražedná moc těch koňů je v jejich tlamách a v jejich ocasech; jejich ocasy jsou jako hadi a hlavy těch hadů přinášejí zhoubu.
#9:20 A přesto se ostatní lidé, kteří v těch pohromách nezahynuli, neodvrátili od výtvorů svých rukou; nepřestali se klanět démonům a modlám ze zlata, stříbra, mědi, kamene i dřeva, které jsou slepé, hluché a nemohou se pohybovat;
#9:21 neodvrátili se od svých vražd ani čarování, necudností ani krádeží. 
#10:1 Tu jsem viděl dalšího mocného anděla, jak sestupuje z nebe, zahalen v oblak. Nad jeho hlavou byla duha, jeho tvář byla jako slunce, nohy jako ohnivé sloupy;
#10:2 v ruce měl otevřenou knížečku. Pravou nohou se postavil na moře, levou na pevninu
#10:3 a vykřikl mocným hlasem, jako když zařve lev. Na jeho výkřik odpovědělo sedmero zahřmění.
#10:4 Jakmile doznělo to sedmero zahřmění, chtěl jsem to zapsat; ale uslyšel jsem hlas z nebe: „Zachovej v tajnosti, co se ozvalo v sedmeru zahřmění, a nic nepiš!“
#10:5 Potom anděl, kterého jsem viděl stát na moři i na zemi, pozvedl svou pravici k nebi
#10:6 a přísahou při tom, který je živ na věky věků, který stvořil nebesa, moře a všecko, co je v nich, potvrdil, že lhůta je u konce;
#10:7 ve dnech, kdy zazní polnice sedmého anděla, naplní se Boží tajemství, jak je Bůh oznámil svým služebníkům prorokům.
#10:8 Hlas, který jsem slyšel z nebe, opět ke mně promluvil: „Jdi, vezmi tu otevřenou knihu z ruky anděla, který stojí na moři i na zemi.“
#10:9 Přistoupil jsem tedy k tomu andělu a požádal ho, aby mi tu knihu dal. Řekl mi: „Vezmi ji a sněz; v žaludku ti zhořkne, ačkoli v tvých ústech bude sladká jako med.“
#10:10 Vzal jsem tu knihu z ruky anděla a snědl ji; v mých ústech byla sladká jako med, ale když jsem ji pozřel, zhořkla mi v žaludku.
#10:11 Tehdy jsem slyšel: „Je třeba, abys znovu prorokoval proti mnoha národům, kmenům, jazykům i králům.“ 
#11:1 Tu mi byla dána rákosová míra a anděl mi řekl: „Vstaň, změř Boží chrám i oltář a spočítej ty, kteří se tam klanějí.
#11:2 Ale vnější chrámový dvůr vynech a neměř, protože byl vydán pohanům; ti budou pustošit svaté město po dvaačtyřicet měsíců.
#11:3 A povolám své dva svědky, a oblečeni v smuteční šat budou prorokovat tisíc dvě stě šedesát dní.“
#11:4 To jsou ty dvě olivy a ty dva svícny, které stojí před Pánem země.
#11:5 A kdyby jim chtěl někdo ublížit, vyšlehne oheň z jejich úst a sežehne jejich nepřátele; takto zahyne každý, kdo by jim chtěl ublížit.
#11:6 Ti dva svědkové mají moc uzavřít nebesa, aby nebylo deště za dnů jejich prorokování, a mají moc proměnit vody v krev a sužovat zemi všemi možnými pohromami, kdykoliv budou chtít.
#11:7 Až ukončí své svědectví, vynoří se z propasti dravá šelma, svede s nimi bitvu, přemůže je a usmrtí.
#11:8 Jejich těla zůstanou ležet na náměstí toho velikého města, které se obrazně nazývá Sodoma a Egypt, kde byl také ukřižován jejich Pán.
#11:9 Lidé ze všech národů, čeledí, jazyků a kmenů budou hledět tři a půl dne na jejich mrtvá těla a nedovolí je pochovat.
#11:10 Obyvatelé země budou z toho mít radost, budou jásat a navzájem si posílat dary, protože tito dva proroci jim nedopřáli klidu.
#11:11 Ale po třech a půl dnech vstoupil do nich duch života přicházející od Boha, postavili se na nohy a hrůza padla na ty, kdo to viděli.
#11:12 Tu uslyšeli ti dva proroci mocný hlas z nebe: „Vstupte sem!“ A vstoupili do nebe v oblaku, a jejich nepřátelé na to hleděli.
#11:13 V tu hodinu nastalo veliké zemětřesení, desetina toho města se zřítila a v zemětřesení zahynulo sedm tisíc lidí. Ostatních se zmocnil strach a vzdali čest Bohu na nebesích.
#11:14 Druhé ‚běda‘ pominulo; hle, už je tu třetí!
#11:15 Zatroubil sedmý anděl. A ozvaly se mocné hlasy v nebi: „Vlády nad světem se ujal náš Pán a jeho Mesiáš; a bude kralovat na věky věků.“
#11:16 Čtyřiadvacet starců, sedících na svých trůnech před Bohem, padlo na kolena, klaněli se Bohu a volali:
#11:17 „Dobrořečíme tobě, Pane Bože všemohoucí, který jsi a kterýs byl, že ses chopil veliké moci, která ti náleží, a ujal ses vlády.
#11:18 Rozzuřily se národy, ale přišel hněv tvůj, čas, abys soudil mrtvé, odměnil své služebníky proroky a všechny, kdo jsou svatí a mají úctu ke tvému jménu, malé i velké; abys zahubil ty, kdo hubili zemi.“
#11:19 Tu se otevřel Boží chrám v nebesích, a bylo v něm vidět schránu jeho smlouvy; rozpoutaly se blesky a rachot hromu, zemětřesení a hrozné krupobití. 
#12:1 A ukázalo se veliké znamení na nebi: Žena oděná sluncem, s měsícem pod nohama a s korunou dvanácti hvězd kolem hlavy.
#12:2 Ta žena byla těhotná a křičela v bolestech, neboť přišla její hodina.
#12:3 Tu se ukázalo na nebi jiné znamení: Veliký ohnivý drak s deseti rohy a sedmi hlavami, a na každé hlavě měl královskou korunu.
#12:4 Ocasem smetl třetinu hvězd z nebe a svrhl je na zem. A drak se postavil před ženu, aby pohltil její dítě, jakmile se narodí.
#12:5 Ona porodila dítě, syna, který má železnou berlou pást všechny národy; ale dítě bylo přeneseno k Bohu a jeho trůnu.
#12:6 Žena pak uprchla na poušť, kde jí Bůh připravil útočiště, aby tam o ni bylo postaráno po tisíc dvě stě šedesát dní.
#12:7 A strhla se bitva na nebi: Michael a jeho andělé se utkali s drakem.
#12:8 Drak i jeho andělé bojovali, ale nezvítězili, a nebylo již pro ně místa v nebi.
#12:9 A veliký drak, ten dávný had, zvaný ďábel a satan, který sváděl celý svět, byl svržen na zem a s ním i jeho andělé.
#12:10 A slyšel jsem mocný hlas v nebi: „Nyní přišlo spasení, moc a království našeho Boha i vláda jeho Mesiáše; neboť byl svržen žalobce našich bratří, který je před Bohem osočoval dnem i nocí.
#12:11 Oni nad ním zvítězili pro krev Beránkovu a pro slovo svého svědectví. Nemilovali svůj život tak, aby se zalekli smrti.
#12:12 Proto jásejte, nebesa a všichni, kdo v nich přebýváte! Běda však zemi i moři, neboť sestoupil k vám ďábel, plný zlosti, protože ví, jak málo času mu zbývá.“
#12:13 Když drak viděl, že je svržen na zem, začal pronásledovat ženu, která porodila syna.
#12:14 Ale té ženě byla dána dvě mocná orlí křídla, aby mohla uletět na poušť do svého útočiště, kde ukryta před hadem byla zachována při životě rok a dva roky a polovinu roku.
#12:15 A had za ní vychrlil ze chřtánu proud vody jako řeku, aby ji smetl.
#12:16 Ale země přispěla ženě na pomoc, otevřela ústa a pohltila tu řeku, kterou drak vychrlil.
#12:17 Drak v hněvu vůči té ženě rozpoutal válku proti ostatnímu jejímu potomstvu, proti těm, kdo zachovávají přikázání Boží a drží se svědectví Ježíšova.
#12:18 A drak se postavil na břeh moře. 
#13:1 Tu jsem viděl, jak se z moře vynořila dravá šelma o deseti rozích a sedmi hlavách; na těch rozích deset královských korun a na hlavách jména urážející Boha.
#13:2 Ta šelma, kterou jsem viděl, byla jako levhart, její nohy jako tlapy medvěda a její tlama jako tlama lví. A drak jí dal svou sílu i trůn i velikou moc.
#13:3 Jedna z jejích hlav vypadala jako smrtelně raněná, ale ta rána se zahojila. A celá země v obdivu šla za tou šelmou;
#13:4 klekali před drakem, protože dal té šelmě svou moc, a klekali také před šelmou a volali: „Kdo se může rovnat té dravé šelmě, kdo se odváží s ní bojovat?“
#13:5 A bylo jí dáno, aby mluvila pyšně a rouhavě a měla moc po čtyřicet dva měsíce.
#13:6 A tak otevřela ústa a rouhala se Bohu, jeho jménu i jeho příbytku, všem, kdo přebývají v nebi.
#13:7 A bylo jí dáno, aby vedla válku proti svatým a aby nad nimi zvítězila. Dostala moc nad každým kmenem, národem, jazykem i rasou;
#13:8 budou před ní klekat všichni obyvatelé země, jejichž jména nejsou od stvoření světa zapsána v knize života, v knize toho zabitého Beránka.
#13:9 Kdo má uši, slyš!
#13:10 Kdo má jít do zajetí, půjde do zajetí. Kdo má zemřít mečem, musí mečem zemřít. Teď musí Boží lid osvědčit trpělivost a víru.
#13:11 Vtom jsem viděl jinou šelmu, jak vyvstala ze země: měla dva rohy jako beránek, ale mluvila jako drak.
#13:12 Z pověření té první šelmy vykonává veškerou její moc. Nutí zemi a její obyvatele, aby klekali před první šelmou, které se zahojila její smrtelná rána.
#13:13 A činí veliká znamení, dokonce i oheň z nebe nechá před zraky lidí sestoupit na zem.
#13:14 Bylo jí dáno dělat znamení ke cti první šelmy a svádět jimi obyvatele země; rozkazuje obyvatelům země, aby postavili sochu té šelmě, která byla smrtelně zraněna mečem, a přece zůstala naživu.
#13:15 Je jí dáno, aby do sochy té šelmy vdechla život, takže ta socha mluvila a vydala rozkaz, že zemřou všichni, kdo před ní nepokleknou.
#13:16 A nutí všechny, malé i veliké, bohaté i chudé, svobodné i otroky, aby měli na pravé ruce nebo na čele cejch,
#13:17 aby nemohl kupovat ani prodávat, kdo není označen jménem té šelmy nebo číslicí jejího jména.
#13:18 To je třeba pochopit: kdo má rozum, ať sečte číslice té šelmy. To číslo označuje člověka, a je to číslo šest set šedesát šest. 
#14:1 A viděl jsem, hle, Beránek stál na hoře Sión a s ním sto čtyřicet čtyři tisíce těch, kdo mají na čele napsáno jméno jeho i jméno jeho Otce.
#14:2 A slyšel jsem hlas z nebe jako hukot množství vod a jako zvuk mocného hřmění; a znělo to, jako když hudebníci rozezvučí své nástroje.
#14:3 Zpívali novou píseň před trůnem, před těmi čtyřmi bytostmi a před starci. Nikdo nebyl schopen naučit se té písni, než těch sto čtyřicet čtyři tisíce z obyvatel země, kteří byli vykoupeni.
#14:4 To jsou ti, kdo neporušili svou čistotu se ženami a zůstali panici. Ti následují Beránka, kamkoli jde. Ti jako první z lidstva byli vykoupeni Bohu a Beránkovi.
#14:5 Z jejich úst nikdo neuslyšel lež; jsou bez úhony.
#14:6 Tu jsem viděl jiného anděla, jak letí středem nebeské klenby, aby zvěstoval věčné evangelium obyvatelům země, každé rase, kmeni, jazyku i národu.
#14:7 Volal mocným hlasem: „Bojte se Boha a vzdejte mu čest, neboť nastala hodina jeho soudu; poklekněte před tím, kdo učinil nebe, zemi, moře i prameny vod.“
#14:8 Za ním letěl druhý anděl a volal: „Padl, padl veliký Babylón, který opojil všechny národy svým smilstvím a dal jim pít z poháru hněvu.“
#14:9 Za nimi letěl třetí anděl a volal mocným hlasem: „Kdo kleká před šelmou a před její sochou, kdo přijímá její cejch na čelo či na ruku,
#14:10 bude pít víno Božího rozhorlení, které Bůh nalévá neředěné do číše svého hněvu; a bude mučen ohněm a sírou před svatými anděly a před Beránkem.
#14:11 A jeho muka neuhasnou na věky věků a dnem ani nocí nedojde pokoje ten, kdo kleká před šelmou a jejím obrazem a nechal si vtisknout její jméno.
#14:12 Zde se ukáže vytrvalost svatých, kteří zachovávají Boží přikázání a věrnost Ježíši.“
#14:13 A slyšel jsem hlas z nebe: „Piš: Od této chvíle jsou blahoslaveni mrtví, kteří umírají v Pánu. Ano, praví Duch, ať odpočinou od svých prací, neboť jejich skutky jdou s nimi.“
#14:14 A viděl jsem, hle, bílý oblak, a na oblaku sedí někdo jako Syn člověka, na hlavě má korunu ze zlata a v ruce ostrý srp.
#14:15 Vtom další anděl vyšel z chrámu a mocným hlasem zavolal na toho, který seděl na oblaku: „Pošli svůj srp a začni žeň, protože nastala hodina žně a úroda země dozrála.“
#14:16 A ten, který seděl na oblaku, hodil svůj srp na zem, a země byla požata.
#14:17 A další anděl vyšel z nebeského chrámu a také on měl ostrý srp.
#14:18 A jiný anděl vyšel od oltáře a ten měl moc nad ohněm. Zavolal mocným hlasem na toho, který měl ostrý srp: „Pošli svůj ostrý srp a skliď hrozny na vinicích země, neboť víno dozrálo.“
#14:19 Tu hodil ten anděl svůj srp na zem a sklidil víno země a uvrhl je do velikého lisu Božího hněvu.
#14:20 A v lisu za městem byly hrozny stlačeny a vytékající krev dosahovala až po uzdy koní na vzdálenost sto šedesát mil. 
#15:1 A viděl jsem na nebi jiné znamení, veliké a podivuhodné: sedm andělů, kteří přinášejí sedm posledních pohrom - jimi se dovrší Boží hněv.
#15:2 Viděl jsem jakoby jiskřící moře, planoucí ohněm, a viděl jsem ty, kteří zvítězili nad dravou šelmou, nesklonili se před jejím obrazem a nenechali se označit číslicí jejího jména. Stáli na tom jiskřícím moři, měli Boží loutny
#15:3 a zpívali píseň Božího služebníka Mojžíše a píseň Beránkovu: „Veliké a podivuhodné jsou tvé činy, Pane Bože všemohoucí; spravedlivé a pravdivé jsou tvé cesty, Králi národů.
#15:4 Kdo by se nebál tebe, Pane, a nevzdal slávu tvému jménu, neboť ty jediný jsi Svatý; všechny národy přijdou a skloní se před tebou, neboť tvé spravedlivé soudy vyšly najevo.“
#15:5 Potom jsem viděl, jak se otevřela svatyně stánku svědectví v nebi a vyšlo sedm andělů, přinášejících sedm pohrom;
#15:6 byli oděni kněžským rouchem, čistým a zářícím, a kolem prsou měli zlaté pásy.
#15:7 Jedna ze čtyř bytostí před Božím trůnem podala těm sedmi andělům sedm zlatých nádob naplněných hněvem Boha, živého na věky věků.
#15:8 Svatyně byla naplněna oblakem slávy a moci Boží, takže nikdo nemohl vstoupit do svatyně, dokud se nedokoná sedmero pohrom, které přináší těch sedm andělů. 
#16:1 A slyšel jsem mocný hlas ze svatyně, jak praví sedmi andělům: „Jděte a vylejte těch sedm nádob Božího hněvu na zem!“
#16:2 První anděl šel a vylil svou nádobu na zem: a zlé, zhoubné vředy padly na lidi označené znamením dravé šelmy a klekající před jejím obrazem.
#16:3 Druhý vylil svou nádobu na moře: a změnilo se v krev jako krev zabitého a všechno živé v moři zahynulo.
#16:4 Třetí vylil svou nádobu na řeky a prameny vod: a změnily se v krev.
#16:5 Tu jsem slyšel, jak praví anděl, který má moc nad vodami: „Spravedlivý jsi, Bože, svatý, který jsi a kterýs byl, že jsi vynesl tento rozsudek:
#16:6 těm, kdo prolili krev svatých a proroků, dal jsi pít krev; stalo se jim po zásluze!“
#16:7 A od oltáře jsem slyšel hlas: „Ano, Pane Bože všemohoucí, pravé a spravedlivé jsou tvé soudy.“
#16:8 Čtvrtý anděl vylil svou nádobu na slunce: a byla mu dána moc spalovat lidi svou výhní.
#16:9 Lidé hynuli nesmírným žárem a proklínali Boha, který má moc nad takovými pohromami; ale neobrátili se, aby mu vzdali čest.
#16:10 Pátý anděl vylil svou nádobu na trůn šelmy: a v jejím království nastala tma,
#16:11 lidé se bolestí hryzali do rtů a trýzněni vředy rouhali se nebeskému Bohu, ale neodvrátili se od svých činů.
#16:12 Šestý anděl vylil svou nádobu na velikou řeku Eufrat: a její voda vyschla, aby byla připravena cesta králům od východu slunce.
#16:13 A hle, z úst draka i z úst dravé šelmy a z úst lživého proroka vystoupili tři nečistí duchové, podobní ropuchám.
#16:14 Jsou to duchové ďábelští, kteří činí zázračná znamení. Vyšli ke králům celého světa, aby je shromáždili k boji v rozhodující den všemohoucího Boha.
#16:15 „Hle, přicházím nečekaně jako zloděj! Blaze tomu, kdo bdí a střeží svůj šat, aby nechodil nahý a nebylo vidět jeho nahotu!“
#16:16 Shromáždili ty krále na místo, zvané hebrejsky Harmagedon.
#16:17 Sedmý anděl vylil svou nádobu do ovzduší a z chrámu od trůnu zazněl mocný hlas: „Stalo se!“
#16:18 A rozpoutaly se blesky, hřmění a burácení, a nastalo hrozné zemětřesení, jaké nebylo, co je člověk na zemi; tak silné bylo to zemětřesení.
#16:19 A to veliké město se roztrhlo na tři části a města národů se zřítila. Bůh se rozpomenul na veliký Babylón a dal mu pít z poháru vína svého trestajícího hněvu.
#16:20 Všechny ostrovy zmizely, po horách nezůstalo stopy,
#16:21 na lidi padaly z nebe kroupy těžké jako cent; a lidé proklínali Boha za pohromu krupobití, protože ta pohroma byla strašná. 
#17:1 Tu přišel jeden z těch sedmi andělů, kteří měli sedm nádob, a promluvil ke mně: „Pojď se mnou, ukážu ti soud nad velikou nevěstkou, usazenou nad vodami,
#17:2 se kterou se spustili králové světa a vínem jejího smilství se opíjeli obyvatelé země.“
#17:3 Anděl mě odvedl ve vytržení ducha na poušť. Tu jsem spatřil ženu sedící na dravé šelmě nachové barvy, plné rouhavých jmen, o sedmi hlavách a deseti rozích.
#17:4 Ta žena byla oděna purpurem a šarlatem a ozdobena zlatem, drahokamy a perlami; v ruce držela zlatý pohár, plný ohavností a nečistoty svého smilství,
#17:5 a na čele měla napsáno jméno - je v něm tajemství: „Babylón veliký, Matka všeho smilstva a všech ohavností na zemi.“
#17:6 Viděl jsem tu ženu, zpitou krví svatých a krví Ježíšových svědků. Velice jsem užasl, když jsem ji viděl.
#17:7 Ale anděl mi řekl: „Čemu se divíš? Já ti odhalím tajemství té ženy i té sedmihlavé a desetirohé šelmy, která ji nese.
#17:8 Ta dravá šelma, kterou jsi viděl, byla a není; vystoupí ještě z propasti, ale půjde do záhuby. A užasnou ti obyvatelé země, jejichž jméno není od založení světa zapsáno v knize života, až uvidí, že ta dravá šelma byla a není, a zase bude.
#17:9 Ať pochopí ten, komu je dána moudrost. Sedm hlav je sedm pahorků, na nichž ta žena sedí, a také sedm králů:
#17:10 pět jich padlo, jeden kraluje, jeden ještě nepřišel. Až přijde, bude smět zůstat jen nakrátko.
#17:11 A ta dravá šelma, která byla a není, je osmý král, a přece jeden z těch sedmi; jde však do záhuby.
#17:12 Deset rohů, které jsi viděl, je deset králů, kteří se ještě vlády neujali, ale v jedinou hodinu přijmou královskou moc spolu se šelmou.
#17:13 Budou zajedno ve svých úmyslech a svou sílu i moc dají té šelmě.
#17:14 Ti budou bojovat s Beránkem, ale Beránek je přemůže, protože je Pán pánů a Král králů; ti, kdo jsou s ním, jsou povolaní a vyvolení a věrní.“
#17:15 A řekl mi: „Vody, které jsi viděl, nad nimiž ta nevěstka sedí, to jsou národy, davy, rasy a jazyky.
#17:16 A těch deset rohů, které jsi viděl, i ta šelma pojmou nevěstku v nenávist, oberou ji o všecko až do naha a budou rvát její tělo a spálí ji ohněm.
#17:17 Neboť Bůh jim vložil do srdce, aby provedli jeho záměr, řídili se jedním úmyslem a odevzdali šelmě svou královskou moc, dokud se nedokoná, co Bůh řekl.
#17:18 Ta žena, kterou jsi viděl, je veliké město, panující nad králi země.“ 
#18:1 Potom jsem viděl jiného anděla, jak s velikou mocí sestupuje z nebe; a země byla ozářena jeho slávou.
#18:2 Zvolal mohutným hlasem: „Padl, padl veliký Babylón a stal se doupětem démonů, skrýší všech nečistých duchů a každého zlověstného a nenáviděného ptáka;
#18:3 neboť vínem Božího hněvu pro smilství té nevěstky byly opojeny všecky národy, králové světa s ní smilnili a bohatí kupci země bohatli z její rozmařilosti a přepychu.“
#18:4 A slyšel jsem jiný hlas z nebe: „Vyjdi, lide můj, z toho města, nemějte účast na jeho hříších, aby vás nestihly jeho pohromy.
#18:5 Neboť jeho hříchy se navršily až k nebi a Bůh nezapomněl na jeho viny.
#18:6 Odplaťte po zásluze té nevěstce, dvojnásob jí odplaťte za její činy! V poháru, který připravovala, připravte pro ni dvojnásob;
#18:7 kolik si užila slávy a hýření, tolik jí dejte teď trýzně a žalu. Protože si namlouvá: Trůním jako královna, vdovou nejsem a neokusím smutku -
#18:8 proto v jediném dni dopadnou na ni pohromy: smrt i žal a hlad, a bude zničena ohněm. Neboť mocný je Hospodin, Bůh, který ji odsoudil.“
#18:9 Pak budou naříkat nad ní a bědovat králové světa, kteří s ní smilnili a hýřili, až uvidí dým hořícího města;
#18:10 z hrůzy nad jeho zkázou neodváží se přiblížit a budou naříkat: „Běda, běda, ty veliký Babylóne, město tak mocné, jak v jedinou hodinu byl nad tebou vykonán soud!“
#18:11 A bohatí kupci země naříkají a bědují nad ním, protože už nikdo nekoupí jejich zboží:
#18:12 náklady zlata a stříbra, drahokamů a perel, kmentu a purpuru, hedvábí a šarlatu, cedrového dřeva a předmětů ze slonoviny a předmětů ze vzácných dřev, náklady mědi, železa a mramoru,
#18:13 skořice a indického koření, voňavek, mastí a vykuřovadel, vína a oleje, mouky a obilí, dobytčat a ovcí, koní a vozů a otroků a zajatců.
#18:14 Plody, které jsi dychtivě sklízelo, jsou pryč, všechen lesk a nádhera zašly a není po nich památky.
#18:15 Kupci, kteří s tím vším obchodovali a z toho města bohatli, z hrůzy nad jeho zkázou neodváží se přiblížit a budou plakat a naříkat:
#18:16 „Běda, běda, tak veliké město, které se oblékalo do kmentu, purpuru a šarlatu, které se zdobilo zlatem, drahokamy a perlami -
#18:17 a v jedinou hodinu je zničeno takové bohatství!“ A všichni velitelé lodí, dopravci zboží, námořníci a kdokoli se živí plavbou, neodváží se přiblížit
#18:18 a budou volat, až uvidí dým toho hořícího města: „Co se mohlo rovnat tomu velikému městu!“
#18:19 A sypou si prach na hlavu a v pláči a nářku křičí: „Běda, běda, tak veliké město, na jehož blahobytu zbohatli všichni majitelé námořních lodí - a v jedinou hodinu bylo zpustošeno!“
#18:20 Raduj se, nebe a svatí, apoštolové i proroci, protože Bůh vynesl nad tím městem rozsudek, jaký ono vyneslo nad vámi.
#18:21 A jeden silný anděl pozvedl balvan, těžký jako mlýnský kámen, vrhl jej do moře a zvolal: „Tak rázem bude svržen Babylón, to veliké město, a nebude po něm ani památky.
#18:22 Nikdy už v tobě nezazní hudba, neozve se harfa, píšťala ani trubka, nikdy už v tobě nebude řemeslník ani umělec, nikdo už v tobě neuslyší zvuk mlýnů,
#18:23 nikdy už v tobě nezasvitne světlo lampy, nikdo už v tobě nezaslechne hlas ženicha a nevěsty. Tvoji obchodníci vládli světem a tvé čarovné nápoje mámily celé národy;
#18:24 ve tvých zdech tekla krev proroků a svatých a všech zavražděných na zemi.“ 
#19:1 Potom jsem slyšel mocný hlas jakoby obrovského zástupu na nebi: „Haleluja! Spasení, sláva i moc patří Bohu našemu,
#19:2 protože pravé a spravedlivé jsou jeho soudy! Vždyť odsoudil tu velikou nevěstku, která přivedla zemi do záhuby svým smilstvím, spravedlivě ji potrestal za krev svých služebníků, která lpí na jejích rukou.“
#19:3 A znovu volali: „Haleluja! Na věky věků vystupuje dým z toho hořícího města.“
#19:4 Tehdy těch čtyřiadvacet starců spolu se čtyřmi bytostmi padlo na kolena; klaněli se před Bohem sedícím na trůnu a volali: „Amen, haleluja.“
#19:5 A od trůnu zazněl hlas: „Zpívejte Bohu našemu všichni jeho služebníci, kteří se ho bojíte, malí i velicí.“
#19:6 A slyšel jsem zpěv jakoby ohromného zástupu, jako hukot množství vod a jako dunění hromu: „Haleluja, ujal se vlády Pán Bůh náš všemohoucí.
#19:7 Radujme se a jásejme a vzdejme mu chválu; přišel den svatby Beránkovy, jeho choť se připravila
#19:8 a byl jí dán zářivě čistý kment, aby se jím oděla.“ Tím kmentem jsou spravedlivé skutky svatých.
#19:9 Tehdy mi řekl: „Piš: Blaze těm, kdo jsou pozváni na svatbu Beránkovu.“ A řekl mi: „Toto jsou pravá slova Boží.“
#19:10 Tu jsem padl na kolena k jeho nohám. Ale on mi řekl: „Střez se toho! Jsem jen služebník jako ty a tvoji bratří, kteří vydávají svědectví Ježíšovi. Před Bohem poklekni!“ Kdo vydává svědectví Ježíšovi, má ducha proroctví.
#19:11 A viděl jsem nebesa otevřená, a hle, bílý kůň, a na něm seděl ten, který má jméno Věrný a Pravý, neboť soudí a bojuje spravedlivě.
#19:12 Jeho oči plamen ohně a na hlavě množství královských korun; jeho jméno je napsáno a nezná je nikdo než on sám.
#19:13 Má na sobě plášť zbrocený krví a jeho jméno je Slovo Boží.
#19:14 Za ním nebeská vojska na bílých koních, oblečená do bělostného čistého kmentu.
#19:15 Z jeho úst vychází ostrý meč, aby jím pobíjel národy; bude je pást železnou berlou. On bude tlačit lis plný vína trestajícího hněvu Boha všemohoucího.
#19:16 Na plášti a na boku má napsáno jméno: Král králů a Pán pánů.
#19:17 A viděl jsem jednoho anděla, jak stojí na slunci a volá mocným hlasem na všechny ptáky letící středem nebes: „Pojďte, slétněte se k veliké hostině Boží!
#19:18 Budete se sytit těly králů a vojevůdců a bojovníků a koní i jezdců; těly všech, pánů i otroků, slabých i mocných.“
#19:19 A viděl jsem dravou šelmu a krále země i jejich vojska shromážděná, aby vytrhli do boje proti jezdci a proti jeho vojsku.
#19:20 Ale šelma byla zajata a s ní i falešný prorok, který k její cti činil zázračná znamení a svedl jimi ty, kdo přijali cejch šelmy a klekali před jejím obrazem. Za živa byli šelma a její prorok hozeni do ohnivého jezera hořícího sírou.
#19:21 Ostatní byli pobiti mečem vycházejícím z úst jezdce. A všichni ptáci se nasytili jejich těly. 
#20:1 Tu jsem viděl, jak z nebe sestupuje anděl, který má v ruce klíč od propasti a veliký řetěz.
#20:2 Zmocnil se draka, toho dávného hada, toho ďábla a satana,
#20:3 na tisíc let jej spoutal, uvrhl do propasti, uzamkl ji a zapečetil, aby již nemohl klamat národy, dokud se nedovrší těch tisíc let. Potom musí být ještě na krátký čas propuštěn.
#20:4 Viděl jsem trůny a na nich usedli ti, jimž byl svěřen soud. Spatřil jsem také ty, kdo byli sťati pro svědectví Ježíšovo a pro slovo Boží, protože nepoklekli před dravou šelmou ani jejím obrazem a nepřijali její znamení na čelo ani na ruku. Nyní povstali k životu a ujali se vlády s Kristem na tisíc let. -
#20:5 Ostatní mrtví však nepovstanou k životu, dokud se těch tisíc let nedovrší. -
#20:6 To je první vzkříšení. Blahoslavený a svatý, kdo má podíl na prvním vzkříšení! Nad těmi druhá smrt nemá moci, nýbrž Bůh a Kristus je učiní svými kněžími a budou s ním kralovat po tisíc let.
#20:7 Až se dovrší tisíc let, bude satan propuštěn ze svého žaláře
#20:8 a vyjde, aby oklamal národy ve všech čtyřech úhlech světa, Góga i Magóga. Shromáždí je k boji a bude jich jako písku v moři.
#20:9 Viděl jsem, jak vystoupili po celé šíři země a obklíčili tábor svatých a město, které miluje Bůh. Ale sestoupil oheň z nebe a pohltil je.
#20:10 Jejich svůdce ďábel byl uvržen do jezera, kde hoří síra a kde je již dravá šelma i falešný prorok. A budou trýzněni dnem i nocí na věky věků.
#20:11 A viděl jsem veliký bělostný trůn a toho, kdo na něm seděl; před jeho pohledem zmizela země i nebe a už pro ně nebylo místa.
#20:12 Viděl jsem mrtvé, mocné i prosté, jak stojí před trůnem, a byly otevřeny knihy. Ještě jedna kniha byla otevřena, kniha života. A mrtví byli souzeni podle svých činů zapsaných v těch knihách.
#20:13 Moře vydalo své mrtvé, i smrt a její říše vydaly své mrtvé, a všichni byli souzeni podle svých činů.
#20:14 Pak smrt i její říše byly uvrženy do hořícího jezera. To je druhá smrt: hořící jezero.
#20:15 A kdo nebyl zapsán v knize života, byl uvržen do hořícího jezera. 
#21:1 A viděl jsem nové nebe a novou zemi, neboť první nebe a první země pominuly a moře již vůbec nebylo.
#21:2 A viděl jsem od Boha z nebe sestupovat svaté město, nový Jeruzalém, krásný jako nevěsta ozdobená pro svého ženicha.
#21:3 A slyšel jsem veliký hlas od trůnu: „Hle, příbytek Boží uprostřed lidí, Bůh bude přebývat mezi nimi a oni budou jeho lid; on sám, jejich Bůh, bude s nimi,
#21:4 a setře jim každou slzu s očí. A smrti již nebude, ani žalu ani nářku ani bolesti už nebude - neboť co bylo, pominulo.“
#21:5 Ten, který seděl na trůnu, řekl: „Hle, všecko tvořím nové.“ A řekl: „Napiš: Tato slova jsou věrná a pravá.“
#21:6 A dodal: „Již se vyplnila. Já jsem Alfa i Omega, počátek i konec. Tomu, kdo žízní, dám napít zadarmo z pramene vody živé.
#21:7 Kdo zvítězí, dostane toto vše; já mu budu Bohem a on mi bude synem.
#21:8 Avšak zbabělci, nevěrní, nečistí, vrahové, cizoložníci, zaklínači, modláři a všichni lháři najdou svůj úděl v jezeře, kde hoří oheň a síra. To je ta druhá smrt.“
#21:9 A přistoupil jeden ze sedmi andělů, kteří měli těch sedm nádob a v nich připraveno sedm posledních pohrom, a řekl mi: „Pojď, ukážu ti nevěstu, choť Beránkovu.“
#21:10 Ve vytržení ducha mě vyvedl na velikou a vysokou horu a ukázal mi svaté město Jeruzalém, jak sestupuje z nebe od Boha,
#21:11 zářící Boží slávou; jeho jas jako nejdražší drahokam a jako průzračný křišťál.
#21:12 Město mělo mohutné a vysoké hradby, dvanáct bran střežených dvanácti anděly a na branách napsaná jména dvanácti pokolení synů Izraele.
#21:13 Tři brány byly na východ, tři brány na sever, tři brány na jih a tři brány na západ.
#21:14 A hradby města byly postaveny na dvanácti základních kamenech a na nich bylo dvanáct jmen dvanácti apoštolů Beránkových.
#21:15 Ten, který se mnou mluvil, měl zlatou míru, aby změřil město i jeho brány a hradby.
#21:16 Město je vystaveno do čtverce: jeho délka je stejná jako šířka. Změřil to město, a bylo to dvanáct tisíc měr. Jeho délka, šířka i výška jsou stejné.
#21:17 Změřil i hradbu, a bylo to sto čtyřicet čtyři loket lidskou mírou, kterou použil anděl.
#21:18 Hradby jsou postaveny z jaspisu a město je z ryzího zlata, zářícího jako křišťál.
#21:19 Základy hradeb toho města jsou samý drahokam: první základní kámen je jaspis, druhý safír, třetí chalcedon, čtvrtý smaragd,
#21:20 pátý sardonyx, šestý karneol, sedmý chrysolit, osmý beryl, devátý topas, desátý chrysopras, jedenáctý hyacint a dvanáctý ametyst.
#21:21 A dvanáct bran je z dvanácti perel, každá z jediné perly. A náměstí toho města je z ryzího zlata jako z průzračného křišťálu.
#21:22 Avšak chrám jsem v něm nespatřil: Jeho chrámem je Pán Bůh všemohoucí a Beránek.
#21:23 To město nepotřebuje ani slunce ani měsíc, aby mělo světlo: září nad ním sláva Boží a jeho světlem je Beránek.
#21:24 Národy budou žít v jeho světle; králové světa mu odevzdají svou slávu.
#21:25 Jeho brány zůstanou otevřeny, protože stále trvá den, a noci tam už nebude.
#21:26 V něm se shromáždí sláva i čest národů.
#21:27 A nevstoupí tam nic nesvatého ani ten, kdo se rouhá a lže, nýbrž jen ti, kdo jsou zapsáni v Beránkově knize života. 
#22:1 A ukázal mi řeku živé vody, čiré jako křišťál, která vyvěrala u trůnu Božího a Beránkova.
#22:2 Uprostřed města na náměstí, z obou stran řeky, bylo stromoví života nesoucí ovoce dvanáctkrát do roka; každý měsíc dozrává na něm ovoce a jeho listí má léčivou moc pro všechny národy.
#22:3 A nebude tam nic proklatého. Bude tam trůn Boží a Beránkův; jeho služebníci mu budou sloužit,
#22:4 budou hledět na jeho tvář a na čele ponesou jeho jméno.
#22:5 Noci tam již nebude a nebudou potřebovat světlo lampy ani světlo slunce, neboť Pán Bůh bude jejich světlem a budou s ním kralovat na věky věků.
#22:6 A řekl mi: „Tato slova jsou věrná a pravá; Pán, Bůh dávající Ducha prorokům, poslal svého anděla, aby oznámil svým služebníkům, co se má brzo stát:
#22:7 Hle, přijdu brzo. Blaze tomu, kdo se drží proroctví této knihy.“
#22:8 To jsem slyšel a viděl já, Jan. A když jsem to uslyšel a spatřil, padl jsem na kolena k nohám anděla, který mi to oznamoval.
#22:9 Ale on mi řekl: „Střez se toho! Jsem jen služebník jako ty a tvoji bratří proroci a ti, kdo se drží slov této knihy. Před Bohem poklekni!“
#22:10 A řekl mi: „Nezapečeťuj knihu se slovy tohoto proroctví: čas je blízko.
#22:11 Kdo křivdí, ať křivdí dál, kdo je pošpiněn, ať zůstane ve špíně - kdo je spravedlivý, ať zůstane spravedlivý, kdo je svatý, ať setrvá ve svatosti.
#22:12 Hle, přijdu brzo, a má odplata se mnou; odplatím každému podle toho, jak jednal.
#22:13 Já jsem Alfa i Omega, první i poslední, počátek i konec.
#22:14 Blaze těm, kdo si vyprali roucha, a tak mají přístup ke stromu života i do bran města.
#22:15 Venku zůstanou nečistí, zaklínači, smilníci, vrahové, modláři - každý, kdo si libuje ve lži.
#22:16 Já, Ježíš, posílám svého posla, aby vám to dosvědčil po všech církvích. Já jsem potomek z rodu Davidova, jasná hvězda jitřní.“
#22:17 A Duch i nevěsta praví: „Přijď!“ A kdokoli to slyší, ať řekne: „Přijď!“ Kdo žízní, ať přistoupí; kdo touží, ať zadarmo nabere vody života.
#22:18 Já dosvědčuji každému, kdo slyší slova proroctví této knihy: Kdo k nim něco přidá, tomu přidá Bůh ran popsaných v této knize.
#22:19 A jestliže kdo ubere ze slov knihy tohoto proroctví, tomu Bůh odejme podíl na stromu života a místo ve svatém městě, jak se o nich píše v této knize.
#22:20 Ten, od něhož je to svědectví, praví: „Ano, přijdu brzo.“ Amen, přijď, Pane Ježíši!
#22:21 Milost Pána Ježíše se všemi.


